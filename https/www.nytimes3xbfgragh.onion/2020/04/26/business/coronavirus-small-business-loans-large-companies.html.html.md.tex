Sections

SEARCH

\protect\hyperlink{site-content}{Skip to
content}\protect\hyperlink{site-index}{Skip to site index}

\href{https://www.nytimes3xbfgragh.onion/section/business}{Business}

\href{https://myaccount.nytimes3xbfgragh.onion/auth/login?response_type=cookie\&client_id=vi}{}

\href{https://www.nytimes3xbfgragh.onion/section/todayspaper}{Today's
Paper}

\href{/section/business}{Business}\textbar{}Large, Troubled Companies
Got Bailout Money in Small-Business Loan Program

\url{https://nyti.ms/3f389Q6}

\begin{itemize}
\item
\item
\item
\item
\item
\item
\end{itemize}

\hypertarget{the-coronavirus-outbreak}{%
\subsubsection{\texorpdfstring{\href{https://www.nytimes3xbfgragh.onion/news-event/coronavirus?name=styln-coronavirus-markets\&region=TOP_BANNER\&variant=undefined\&block=storyline_menu_recirc\&action=click\&pgtype=Article\&impression_id=0c3bba10-e0fd-11ea-bca3-17fca13c9abe}{The
Coronavirus
Outbreak}}{The Coronavirus Outbreak}}\label{the-coronavirus-outbreak}}

\begin{itemize}
\tightlist
\item
  live\href{https://www.nytimes3xbfgragh.onion/2020/08/17/world/coronavirus-covid.html?name=styln-coronavirus-markets\&region=TOP_BANNER\&variant=undefined\&block=storyline_menu_recirc\&action=click\&pgtype=Article\&impression_id=0c3bba11-e0fd-11ea-bca3-17fca13c9abe}{Latest
  Updates}
\item
  \href{https://www.nytimes3xbfgragh.onion/interactive/2020/us/coronavirus-us-cases.html?name=styln-coronavirus-markets\&region=TOP_BANNER\&variant=undefined\&block=storyline_menu_recirc\&action=click\&pgtype=Article\&impression_id=0c3bba12-e0fd-11ea-bca3-17fca13c9abe}{Maps
  and Cases}
\item
  \href{https://www.nytimes3xbfgragh.onion/interactive/2020/science/coronavirus-vaccine-tracker.html?name=styln-coronavirus-markets\&region=TOP_BANNER\&variant=undefined\&block=storyline_menu_recirc\&action=click\&pgtype=Article\&impression_id=0c3bba13-e0fd-11ea-bca3-17fca13c9abe}{Vaccine
  Tracker}
\item
  \href{https://www.nytimes3xbfgragh.onion/2020/08/17/us/k-12-schools-reopening.html?name=styln-coronavirus-markets\&region=TOP_BANNER\&variant=undefined\&block=storyline_menu_recirc\&action=click\&pgtype=Article\&impression_id=0c3bba14-e0fd-11ea-bca3-17fca13c9abe}{State
  of Play for K-12}
\item
  \href{https://www.nytimes3xbfgragh.onion/live/2020/08/17/business/stock-market-today-coronavirus?name=styln-coronavirus-markets\&region=TOP_BANNER\&variant=undefined\&block=storyline_menu_recirc\&action=click\&pgtype=Article\&impression_id=0c3bba15-e0fd-11ea-bca3-17fca13c9abe}{Markets
  \& Economy}
\end{itemize}

Advertisement

\protect\hyperlink{after-top}{Continue reading the main story}

Supported by

\protect\hyperlink{after-sponsor}{Continue reading the main story}

\hypertarget{large-troubled-companies-got-bailout-money-in-small-business-loan-program}{%
\section{Large, Troubled Companies Got Bailout Money in Small-Business
Loan
Program}\label{large-troubled-companies-got-bailout-money-in-small-business-loan-program}}

Companies with accounting problems or in trouble with the government
received millions in federal loans.

\includegraphics{https://static01.graylady3jvrrxbe.onion/images/2020/04/26/us/26virus-bailouts6/merlin_171347163_d1e72655-1a70-4df0-ae98-cf41f975c104-articleLarge.jpg?quality=75\&auto=webp\&disable=upscale}

By
\href{https://www.nytimes3xbfgragh.onion/by/jessica-silver-greenberg}{Jessica
Silver-Greenberg},
\href{https://www.nytimes3xbfgragh.onion/by/david-enrich}{David Enrich},
\href{https://www.nytimes3xbfgragh.onion/by/jesse-drucker}{Jesse
Drucker} and
\href{https://www.nytimes3xbfgragh.onion/by/stacy-cowley}{Stacy Cowley}

\begin{itemize}
\item
  Published April 26, 2020Updated May 13, 2020
\item
  \begin{itemize}
  \item
  \item
  \item
  \item
  \item
  \item
  \end{itemize}
\end{itemize}

A company in Georgia paid \$6.5 million to resolve a Justice Department
investigation --- and, two weeks later, received a \$10 million
federally backed loan to help it survive the coronavirus crisis.

Another company, AutoWeb, disclosed last week that it had paid its chief
executive \$1.7 million in 2019 --- a week after it
\href{https://investor.autoweb.com/news-releases/news-release-details/autoweb-secures-14-million-loan-through-cares-act-paycheck}{received
\$1.4 million} from the same loan program.

And Intellinetics, a software company in Ohio, got \$838,700 from the
government program --- and
then\href{https://www.sec.gov/Archives/edgar/data/1081745/000149315220007017/form8k.htm}{agreed},
the following week, to spend at least \$300,000 to purchase a rival
firm.

The
\href{https://www.nytimes3xbfgragh.onion/2020/03/26/business/economy/coronavirus-relief-bill.html}{vast
economic rescue package} that President Trump signed into law last month
included \$349 billion in low-interest
\href{https://www.nytimes3xbfgragh.onion/2020/05/13/business/paycheck-protection-program-small-business.html}{loans
for small businesses}. The so-called
\href{https://www.nytimes3xbfgragh.onion/2020/05/06/business/small-businesses-loans-ppp-coronavirus.html}{Paycheck
Protection Program} was supposed to help prevent small companies ---
generally those with fewer than 500 employees in the United States ---
from capsizing as the economy sinks into what looks like a severe
recession.

The loan program was meant for companies that could no longer finance
themselves through traditional means, like raising money in the markets
or borrowing from banks under existing credit lines. The law required
that the federal money --- which comes at a low 1 percent interest rate
and in some cases doesn't need to be paid back --- be spent on things
like payroll or rent.

But the program has been
\href{https://www.nytimes3xbfgragh.onion/2020/04/26/business/ppp-small-business-loans.html}{riddled
with problems}. Within days of its start, its money ran out, prompting
Congress to approve an additional \$310 billion in funding that will
open for applications on Monday. Countless small businesses were shut
out, even as a number of large companies received millions of dollars in
aid.

Some, including restaurant chains like Ruth's Chris and Shake Shack,
agreed
\href{https://www.nytimes3xbfgragh.onion/2020/04/20/business/shake-shack-returning-loan-ppp-coronavirus.html}{to
return their loans} after a public outcry. But dozens of large but
lower-profile companies with financial or legal problems have also
received large payouts under the program, according to an analysis of
the more than 200 publicly traded companies that have disclosed
receiving a total of more than \$750 million in bailout loans.

\includegraphics{https://static01.graylady3jvrrxbe.onion/images/2020/04/26/us/00virus-bailout-5/merlin_143581761_53016c5b-adca-4da3-8e06-f45a892da1a5-articleLarge.jpg?quality=75\&auto=webp\&disable=upscale}

Another dozen or so collected money even though they have recently
reported being able to raise large sums through private means. Several
others have recently showered top executives with seven-figure pay
packages.

The government isn't disclosing who receives aid, leaving it up to
individual companies to decide whether to disclose that they obtained
loans. That makes a full accounting of the loan program impossible.

\hypertarget{latest-updates-the-coronavirus-outbreak-and-the-economy}{%
\section{\texorpdfstring{\href{https://www.nytimes3xbfgragh.onion/live/2020/08/17/business/stock-market-today-coronavirus?action=click\&pgtype=Article\&state=default\&region=MAIN_CONTENT_1\&context=storylines_live_updates}{Latest
Updates: The Coronavirus Outbreak and the
Economy}}{Latest Updates: The Coronavirus Outbreak and the Economy}}\label{latest-updates-the-coronavirus-outbreak-and-the-economy}}

\href{https://www.nytimes3xbfgragh.onion/live/2020/08/17/business/stock-market-today-coronavirus?action=click\&pgtype=Article\&state=default\&region=MAIN_CONTENT_1\&context=storylines_live_updates\#robinhood-a-stock-trading-app-is-valued-at-11-2-billion-in-its-latest-funding-round}{7h
ago}

\href{https://www.nytimes3xbfgragh.onion/live/2020/08/17/business/stock-market-today-coronavirus?action=click\&pgtype=Article\&state=default\&region=MAIN_CONTENT_1\&context=storylines_live_updates\#robinhood-a-stock-trading-app-is-valued-at-11-2-billion-in-its-latest-funding-round}{Robinhood,
a stock trading app, is valued at \$11.2 billion in its latest funding
round.}

\href{https://www.nytimes3xbfgragh.onion/live/2020/08/17/business/stock-market-today-coronavirus?action=click\&pgtype=Article\&state=default\&region=MAIN_CONTENT_1\&context=storylines_live_updates\#new-unemployment-benefits-authorized-by-president-trump-wont-come-until-late-august}{10h
ago}

\href{https://www.nytimes3xbfgragh.onion/live/2020/08/17/business/stock-market-today-coronavirus?action=click\&pgtype=Article\&state=default\&region=MAIN_CONTENT_1\&context=storylines_live_updates\#new-unemployment-benefits-authorized-by-president-trump-wont-come-until-late-august}{New
unemployment benefits authorized by President Trump won't come until
late August.}

\href{https://www.nytimes3xbfgragh.onion/live/2020/08/17/business/stock-market-today-coronavirus?action=click\&pgtype=Article\&state=default\&region=MAIN_CONTENT_1\&context=storylines_live_updates\#heres-what-you-need-to-know-for-the-week-ahead}{15h
ago}

\href{https://www.nytimes3xbfgragh.onion/live/2020/08/17/business/stock-market-today-coronavirus?action=click\&pgtype=Article\&state=default\&region=MAIN_CONTENT_1\&context=storylines_live_updates\#heres-what-you-need-to-know-for-the-week-ahead}{Here's
what you need to know for the week ahead.}

\href{https://www.nytimes3xbfgragh.onion/live/2020/08/17/business/stock-market-today-coronavirus?action=click\&pgtype=Article\&state=default\&region=MAIN_CONTENT_1\&context=storylines_live_updates}{See
more updates}

More live coverage:
\href{https://www.nytimes3xbfgragh.onion/2020/08/17/world/coronavirus-covid.html?action=click\&pgtype=Article\&state=default\&region=MAIN_CONTENT_1\&context=storylines_live_updates}{Global}

``It's outrageous,'' said Amanda Ballantyne, the executive director of
Main Street Alliance, an advocacy group for small businesses. She added
that there were countless small business owners ``who have laid off all
their staff, are trying to file for unemployment and will go bankrupt
because of the problems with the way this Paycheck Protection Program
was designed.''

Applicants for loans do not need to provide evidence that they have been
harmed by the pandemic. They simply need to certify that ``current
economic uncertainty makes this loan request necessary'' to support
their operations.

Instead of having the Small Business Administration, which is
guaranteeing the loans, decide which companies get funding, the process
was essentially outsourced to banks. The banks collect fees for each
loan they make but don't have to monitor whether the recipients use the
money appropriately.

For small business owners shut out of the program, watching big
companies collect loans while their applications languish has been
infuriating.

``It has been beyond frustrating,'' said Diane Burgio, a single mother
who \href{http://dianeburgiodesign.com/}{runs a design business} in New
York City that employs four people. She was one of more than 280,000
applicants who sought, and did not get, a loan from JPMorgan Chase.

The New York Times identified roughly a dozen publicly traded companies
that had recently boasted about their access to ample capital --- and
then applied for and received millions of dollars in the federal loans.

Legacy Housing, a Texas company that manufactures premade homes,
announced on April 1 that it had access to
\href{https://investors.legacyhousingcorp.com/news-releases/news-release-details/legacy-housing-corporation-announces-new-credit-facility}{a
new \$25 million credit line}. Curtis D. Hodgson, Legacy's executive
chairman, told investors that he expected any damage from the
coronavirus to be short-lived. ``Our order book is still strong, and we
are well-positioned once the situation begins to normalize,'' he said.

Less than two weeks later, on April 10, the
company\href{https://www.sec.gov/Archives/edgar/data/1436208/000155837020003911/tmb-20200410x8k.htm}{announced}
that a local lender, Peoples Bank, had approved it for \$6.5 million
under the S.B.A. loan program.

In an interview on Sunday, Mr. Hodgson said that an inquiry from The
Times led the company to decide to give back the money it borrowed,
though he defended seeking the loan in the first place. ``Legacy is a
highly leveraged company without cash on hand,'' he said. ``Here was a
way to get a cash infusion.''

Escalade Sports, which makes things like table tennis tables and
basketball hoops, already had a \$50 million credit line from JPMorgan
Chase. The company's chief executive, Dave Fetherman, told investors
this month that the company, based in Evansville, Ind., had ``a strong
balance sheet'' and was seeing rising demand for its products, with so
many Americans cooped up in their homes.

Days earlier, Escalade got a \$5.6 million federally backed loan. A
spokesman for Escalade said the company ``fully met all required
conditions at the time we applied for the P.P.P. loan.''

Executives at some companies said applying for the loans made clear
business sense. The loans are essentially free money: They have
rock-bottom interest rates and can be forgiven if, among other things,
the borrower maintains the size of its work force. In some cases,
executives said, their bankers encouraged them to apply for the loans.

At least seven companies that received a total of \$45 million in loans
under the federal government's program have recently had serious scrapes
with the federal government.

Image

Treasury Secretary Steve Mnuchin at the White House last week with
President Trump. The Paycheck Protection Program was supposed to help
prevent small companies from capsizing as the economy sinks into what
looks like a severe recession.Credit...Doug Mills/The New York Times

MiMedx Group, a biopharmaceutical company in Marietta, Ga.,
\href{https://mimedx.gcs-web.com/news-releases/news-release-details/mimedx-announces-additional-access-capital-and-financing}{got
a \$10 million loan} on April 21. On April 6, the company had
\href{https://www.justice.gov/opa/pr/mimedx-group-inc-agrees-pay-65-million-resolve-false-claims-act-allegations-false-commercial}{agreed
to pay} the Justice Department \$6.5 million to resolve allegations that
it violated federal law by knowingly overcharging the Department of
Veterans Affairs for medical supplies.

MiMedx, which makes and sells human tissue grafts, also ran into
problems with the Securities and Exchange Commission. Last year, the
agency sued MiMedx, accusing the company of exaggerating its revenue to
investors over several years. MiMedx agreed to
\href{https://mimedx.gcs-web.com/news-releases/news-release-details/mimedx-announces-securities-and-exchange-commission-settlement}{settle
the case for \$1.5 million}, without admitting wrongdoing. Two of its
former top executives were indicted last year by federal prosecutors in
Manhattan on charges of accounting fraud.

A MiMedx spokeswoman, Hilary Dixon, said the company was trying to move
past its accounting scandal. ``We don't have the option of raising
capital in the public markets owing to our financial restatement
process,'' she said.

Another company, US Auto Parts Network, which received a \$4.1 million
loan through the program, has been in a heated dispute in recent years
with Customs and Border Protection. The agency has seized some of the
company's imported products, claiming they are counterfeit.

US Auto Parts Network didn't respond to requests for comment.

At least two companies that received federally backed loans have
previously borrowed heavily from their own executives or others close to
the firms --- meaning that the new loans could help the companies repay
their insiders.

Infinite Group, a cybersecurity firm in Pittsford, N.Y., had been
borrowing hundreds of thousands of dollars from its board members and
the brother of a top executive at annual interest rates as high as 7.5
percent. This month, Infinite
\href{https://www.sec.gov/Archives/edgar/data/884650/000165495420004141/igiform8-kpppnotewithupst.htm}{secured}
a nearly \$1 million federally backed loan whose 1 percent interest rate
could allow the company to dramatically lower its funding costs. Company
officials didn't respond to requests for comment.

Intellinetics, the company that announced that it was buying a rival
days after it received its emergency loan of \$838,700, borrowed nearly
\$400,000 last fall from two brothers who run a small New York brokerage
firm, Taglich Brothers. If the money isn't repaid by May 15,
Intellinetics will need to give the brothers stock in the company or
start paying a steep 12 percent interest rate. (Some of that debt has
already been converted into stock.)

``Securing the PPP funding gives us extra confidence and ability to
restart and hit the ground running,'' James F. DeSocio, the company's
chief executive,
\href{https://www.sec.gov/Archives/edgar/data/1081745/000149315220006705/ex99-1.htm}{said}
in a news release.

Infinite Group and Intellinetics have not said precisely how they intend
to use the loan proceeds.

Image

CPI Aerostructures, an aerospace manufacturer, got a \$4.8 million loan
through the federal program.Credit...Victor J. Blue/Bloomberg

A number of other companies have had serious accounting problems. The
chief financial officer of CPI Aerostructures, an aerospace manufacturer
that got
\href{http://www.cpiaero.com/cpi-aero-secures-\%244.8m-loan.html}{a
\$4.8 million loan},
\href{https://www.newsday.com/business/cpi-aerostructures-financial-reports-1.41833254}{resigned
in February} after the company disclosed major problems with how it
reported revenue.

And several firms have been paying their top executives millions of
dollars despite financial problems that predate the coronavirus crisis.

For example, AutoWeb's chief executive, Jared Rowe, got \$4.7 million in
total compensation over the past two years --- including \$1.7 million
in 2019 --- even as its stock price plummeted more than 70 percent. The
company declined to comment.

And Manning \& Napier, an investment firm in Fairport, N.Y., that has
about \$20 billion in assets under management,
\href{https://www.sec.gov/Archives/edgar/data/1524223/000152422320000015/mn201910-k_12312019.htm}{disclosed}
in March that its chief executive, Marc O. Mayer, earned nearly \$5
million last year. On April 19, the company was approved for \$6.7
million in the paycheck protection loans --- even as the company
\href{https://productionmn.blob.core.windows.net/media/Default/newsroom/2020/Manning\%20\%20Napier,\%20Inc.\%20Provides\%20Updates.pdf}{said}
it would pay out a quarterly dividend to its shareholders.

Last week, amid mounting public anger toward large recipients of the
rescue loans, Manning \&
Napier\href{https://www.sec.gov/Archives/edgar/data/1524223/000119312520118108/d885193dex991.htm}{said
it had decided} not to take the money.

While the federal loan program is supposed to help companies avoid
layoffs, some of the large recipients of loans have already dramatically
reduced their workforces --- and not always because of the coronavirus.

Harvard Bioscience, based in Holliston, Mass., has been trying since
last year to pacify an activist investor that is pressuring management
to boost the company's stock price. The company closed facilities in
North Carolina and Connecticut and said in February, before the
coronavirus upended the economy, that it was
\href{https://finance.yahoo.com/news/harvard-bioscience-announces-fourth-quarter-121510013.html}{laying
off about 10 percent}of its work force.

This month, Harvard Bioscience received a \$6.1 million loan through the
paycheck protection program. In
\href{https://www.sec.gov/Archives/edgar/data/1123494/000117184320002798/f8k_042220.htm}{a
securities filing} disclosing the loan, the company didn't say why it
sought the money or how it would use it. A spokesman didn't respond to
requests for comment.

A number of relatively large companies with connections to Mr. Trump
also received millions of dollars in loans.

Phunware, a data-collection company that
received\href{https://investors.phunware.com/news-events/press-releases/detail/96/phunware-receives-united-states-small-business}{a
\$2.9 million loan} this month, counts Mr. Trump's re-election campaign
and Fox News as two of its biggest clients.

Continental Materials, a heating and air conditioning and construction
material supplier based in Chicago, got
\href{https://www.sec.gov/Archives/edgar/data/24104/000110465920048601/tm2016372-1_8k.htm}{a
\$5.5 million loan}. The firm's chief executive, James Gidwitz, is a
major Trump donor, and his brother Ronald was appointed ambassador to
Brussels by Mr. Trump after serving as Illinois campaign finance
chairman for the 2016 Trump campaign.

It isn't clear whether political considerations helped Phunware and
Continental Materials get their loans approved. Neither company
responded to requests for comment.

William Rashbaum and Jeanna Smialek **** contributed reporting.

Advertisement

\protect\hyperlink{after-bottom}{Continue reading the main story}

\hypertarget{site-index}{%
\subsection{Site Index}\label{site-index}}

\hypertarget{site-information-navigation}{%
\subsection{Site Information
Navigation}\label{site-information-navigation}}

\begin{itemize}
\tightlist
\item
  \href{https://help.nytimes3xbfgragh.onion/hc/en-us/articles/115014792127-Copyright-notice}{©~2020~The
  New York Times Company}
\end{itemize}

\begin{itemize}
\tightlist
\item
  \href{https://www.nytco.com/}{NYTCo}
\item
  \href{https://help.nytimes3xbfgragh.onion/hc/en-us/articles/115015385887-Contact-Us}{Contact
  Us}
\item
  \href{https://www.nytco.com/careers/}{Work with us}
\item
  \href{https://nytmediakit.com/}{Advertise}
\item
  \href{http://www.tbrandstudio.com/}{T Brand Studio}
\item
  \href{https://www.nytimes3xbfgragh.onion/privacy/cookie-policy\#how-do-i-manage-trackers}{Your
  Ad Choices}
\item
  \href{https://www.nytimes3xbfgragh.onion/privacy}{Privacy}
\item
  \href{https://help.nytimes3xbfgragh.onion/hc/en-us/articles/115014893428-Terms-of-service}{Terms
  of Service}
\item
  \href{https://help.nytimes3xbfgragh.onion/hc/en-us/articles/115014893968-Terms-of-sale}{Terms
  of Sale}
\item
  \href{https://spiderbites.nytimes3xbfgragh.onion}{Site Map}
\item
  \href{https://help.nytimes3xbfgragh.onion/hc/en-us}{Help}
\item
  \href{https://www.nytimes3xbfgragh.onion/subscription?campaignId=37WXW}{Subscriptions}
\end{itemize}
