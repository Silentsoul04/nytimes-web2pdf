Sections

SEARCH

\protect\hyperlink{site-content}{Skip to
content}\protect\hyperlink{site-index}{Skip to site index}

\href{https://www.nytimes3xbfgragh.onion/section/obituaries}{Obituaries}

\href{https://myaccount.nytimes3xbfgragh.onion/auth/login?response_type=cookie\&client_id=vi}{}

\href{https://www.nytimes3xbfgragh.onion/section/todayspaper}{Today's
Paper}

\href{/section/obituaries}{Obituaries}\textbar{}Steve Dalkowski, Model
for Erratic Pitcher in `Bull Durham,' Dies at 80

\url{https://nyti.ms/2zqil4C}

\begin{itemize}
\item
\item
\item
\item
\item
\item
\end{itemize}

\href{https://www.nytimes3xbfgragh.onion/news-event/coronavirus?action=click\&pgtype=Article\&state=default\&region=TOP_BANNER\&context=storylines_menu}{The
Coronavirus Outbreak}

\begin{itemize}
\tightlist
\item
  live\href{https://www.nytimes3xbfgragh.onion/2020/08/04/world/coronavirus-covid-19.html?action=click\&pgtype=Article\&state=default\&region=TOP_BANNER\&context=storylines_menu}{Latest
  Updates}
\item
  \href{https://www.nytimes3xbfgragh.onion/interactive/2020/us/coronavirus-us-cases.html?action=click\&pgtype=Article\&state=default\&region=TOP_BANNER\&context=storylines_menu}{Maps
  and Cases}
\item
  \href{https://www.nytimes3xbfgragh.onion/interactive/2020/science/coronavirus-vaccine-tracker.html?action=click\&pgtype=Article\&state=default\&region=TOP_BANNER\&context=storylines_menu}{Vaccine
  Tracker}
\item
  \href{https://www.nytimes3xbfgragh.onion/2020/08/02/us/covid-college-reopening.html?action=click\&pgtype=Article\&state=default\&region=TOP_BANNER\&context=storylines_menu}{College
  Reopening}
\item
  \href{https://www.nytimes3xbfgragh.onion/live/2020/08/03/business/stock-market-today-coronavirus?action=click\&pgtype=Article\&state=default\&region=TOP_BANNER\&context=storylines_menu}{Economy}
\end{itemize}

Advertisement

\protect\hyperlink{after-top}{Continue reading the main story}

Supported by

\protect\hyperlink{after-sponsor}{Continue reading the main story}

THOSE We'VE LOST

\hypertarget{steve-dalkowski-model-for-erratic-pitcher-in-bull-durham-dies-at-80}{%
\section{Steve Dalkowski, Model for Erratic Pitcher in `Bull Durham,'
Dies at
80}\label{steve-dalkowski-model-for-erratic-pitcher-in-bull-durham-dies-at-80}}

Known for his extraordinary fastball, Dalkowski never made it to the
major leagues, walking batters almost as often as he struck them out.

\includegraphics{https://static01.graylady3jvrrxbe.onion/images/2020/04/27/obituaries/27dalkowski-obit1/merlin_171906978_e3b2c728-0117-4401-891c-99e2cdba4290-articleLarge.jpg?quality=75\&auto=webp\&disable=upscale}

By
\href{https://www.nytimes3xbfgragh.onion/by/richard-goldstein}{Richard
Goldstein}

\begin{itemize}
\item
  April 26, 2020
\item
  \begin{itemize}
  \item
  \item
  \item
  \item
  \item
  \item
  \end{itemize}
\end{itemize}

\emph{This obituary is part of a series about people who have died in
the coronavirus pandemic. Read about others}
\href{https://www.nytimes3xbfgragh.onion/series/people-who-have-died-of-the-coronavirus}{\emph{here}}\emph{.}

Steve Dalkowski, a left-handed pitcher with an unimposing physique,
spent nine seasons in the minor leagues without getting into a major
league game. But just about everyone in the baseball world of the 1950s
and early '60s seemed to have heard stories of his extraordinary gift.

Since radar guns had yet to arrive on the baseball scene, nobody could
measure Dalkowski's fastball under game conditions with precision. But
it became gospel that he could throw well in excess of 100 miles an
hour.

The Sporting News ran the headline ``Living Legend Released'' when his
career, spent mostly in the Baltimore Orioles' system, came to an end in
1966.

Ron Shelton, the director and screenwriter for the 1988 baseball movie
``Bull Durham,'' had been an infielder in the Orioles' organization for
several seasons after Dalkowski's departure and heard tales of his
pitching days. Shelton kept his legend alive when he loosely drew on
Dalkowski's exploits in creating the Bulls' hard-throwing pitcher Nuke
LaLoosh, portrayed by Tim Robbins.

\includegraphics{https://static01.graylady3jvrrxbe.onion/images/2020/04/27/obituaries/27dalkowski-virus-obit2/merlin_10278517_17bacea5-54d2-471f-b791-56023ba893e3-articleLarge.jpg?quality=75\&auto=webp\&disable=upscale}

But for all of the attention he received, Dalkowski's flaws on the ball
field and his troubles off it ruined what might have been a brilliant
career. He walked batters almost as often as he struck them out, and he
struggled with alcoholism.

He died on April 19 at 80 at a hospital in New Britain, Conn., after
spending his last 26 years at a nursing home in the city with
alcohol-induced dementia.

The cause was complications of the coronavirus, his sister, Patti Cain,
said.

Stephen Louis **** Dalkowski Jr. was born on June 3, 1939, in New
Britain. His father was a tool and die maker and his mother, Adele
Zaleski Dalkowski, worked in a ball bearing factory.

He became a star pitcher in high school and signed with the Orioles
after graduating in 1957, receiving a \$4,000 bonus.

Pitching for the Kingsport, Tenn., team of the Appalachian League in his
rookie season, Dalkowski hit a batter in the head, leaving him
unconscious. He struck out 121 batters and walked 129 in 62 innings with
a 1-8 record.

The following season, pitching for three teams, he struck out 203
batters and walked 207 in 104 innings. And so it went, though he showed
some improvement in his control while pitching for Baltimore's farm club
in Elmira, N.Y., for Earl Weaver, the future Hall of Fame manager of the
Orioles. Weaver tried to keep things simple for Dalkowski, telling him
to concentrate on taking something off his fastball in order to find
home plate.

Dalkowski's pitches, thrown from a 5-foot-11-inch, 175-pound frame, were
likely to arrive high or low rather than bearing in on a hitter or
straying wide of the plate. But the Yankees were taking no chances when
they faced him in a March 1963 exhibition night game with the Orioles
when Dalkowski was being considered for a call to the majors.

George Vecsey, who covered the game for Newsday and later became a
sports columnist for The New York Times, interviewed Dalkowski before
his induction in 2009 into the
\href{http://www.baseballreliquary.org/awards/shrine-of-the-eternals/}{Shrine
of the Eternals}, an alternate hall of fame in Pasadena, Calif. His
account recalled the moment when Roger Maris came to the plate.

``Maris was a legend for having hit a record 61 homers in 1961;
Dalkowski was a legend for being perhaps the fastest pitcher ever,'' he
wrote. ``When they met, Maris was theoretically standing near home plate
in Miami, but his fanny was more or less in the Bahamas.''

``Three straight pitches,'' Dalkowski remarked in that interview,
recalling his easy strikeout.

``After the game, the Yankees stars all yukked it up in the clubhouse,
imitating each other's bailout moves,'' Vecsey remembered.

But Dalkowski injured his elbow later in the game and never regained the
speed on his fastball. He also pitched briefly for the Pittsburgh
Pirates and the California Angels organizations in his last years in the
minors.

He struck out 1,324 batters and walked 1,236 in 956 innings with a
career record of 46-80, according to Baseball Reference.

Image

Dalkowski in 2009 at a residential care facility in New Britain, Conn.
``He had the equivalent of Michelangelo's gift but could never finish a
painting.''~Credit...George Ruhe for The New York Times

After leaving baseball, Dalkowski picked vegetables alongside migrant
workers in the San Joaquin Valley in California. But he never found a
steady job. His drinking continued, and he lost touch with his family.
When his sister learned of his whereabouts in 1994, she brought him back
to New Britain and placed him in the nursing home.

Dalkowski's first marriage, to Linda Moore, ended in divorce. His second
wife, Virginia Billingsley, died in 1994. He had no children, and his
sister was his only immediate survivor.

Shelton reflected on Dalkowski's life in an article for The Los Angeles
Times in 2009.

``It's the gift from the gods --- the arm, the power,'' he wrote. ``That
is what haunts us. He had it all and didn't know it. That's why Steve
Dalkowski stays in our minds. He had the equivalent of Michelangelo's
gift but could never finish a painting.''

\href{https://www.nytimes3xbfgragh.onion/interactive/2020/obituaries/people-died-coronavirus-obituaries.html?action=click\&pgtype=Article\&state=default\&region=BELOW_MAIN_CONTENT\&context=covid_obits_promo}{}

\hypertarget{those-weve-lost}{%
\section{Those We've Lost}\label{those-weve-lost}}

The coronavirus pandemic has taken an incalculable death toll. This
series is designed to put names and faces to the numbers.

Read more

\includegraphics{https://static01.graylady3jvrrxbe.onion/images/2020/07/30/obituaries/30Pedro/30Pedro-square640.jpg}

\hypertarget{bernaldina-josuxe9-pedro}{%
\section{Bernaldina José Pedro}\label{bernaldina-josuxe9-pedro}}

d. Boa Vista, Brazil

Leader among the Indigenous Macuxi

\includegraphics{https://static01.graylady3jvrrxbe.onion/images/2020/07/31/obituaries/31Swing/merlin_175167783_8913bc90-0d64-43f3-a655-1bb1bf1601c9-square640.jpg}

\hypertarget{john-eric-swing}{%
\section{John Eric Swing}\label{john-eric-swing}}

d. Fountain Valley, Calif.

Champion of Filipino-Americans

\includegraphics{https://static01.graylady3jvrrxbe.onion/images/2020/07/27/obituaries/27Victor/merlin_175001436_38b11f8e-227a-4e2c-9821-7618af9b2524-square640.jpg}

\hypertarget{victor-victor}{%
\section{Victor Victor}\label{victor-victor}}

d. Santo Domingo, Dominican Republic

Beloved musician of the Dominican Republic

\includegraphics{https://static01.graylady3jvrrxbe.onion/images/2020/07/31/obituaries/31Negron/merlin_175160169_516322ae-fd23-4969-b6b2-193ced371105-square640.jpg}

\hypertarget{dr-eddie-negruxf3n}{%
\section{Dr. Eddie Negrón}\label{dr-eddie-negruxf3n}}

d. Fort Walton Beach, Fla.

Internist on Florida's Emerald Coast

\includegraphics{https://static01.graylady3jvrrxbe.onion/images/2020/07/30/obituaries/30Dobson/merlin_175115928_f6b9271c-8f05-4fe1-a38a-5ca4a58f8935-square640.jpg}

\hypertarget{dobby-dobson}{%
\section{Dobby Dobson}\label{dobby-dobson}}

d. Coral Springs, Fla.

Jamaican singer and songwriter

\includegraphics{https://static01.graylady3jvrrxbe.onion/images/2020/08/01/obituaries/28Gonzalez/merlin_175002771_beb57888-3951-409a-ae13-03a94b2e962e-square640.jpg}

\hypertarget{waldemar-gonzalez}{%
\section{Waldemar Gonzalez}\label{waldemar-gonzalez}}

d. White Plains, N.Y.

Teacher and social worker

Advertisement

\protect\hyperlink{after-bottom}{Continue reading the main story}

\hypertarget{site-index}{%
\subsection{Site Index}\label{site-index}}

\hypertarget{site-information-navigation}{%
\subsection{Site Information
Navigation}\label{site-information-navigation}}

\begin{itemize}
\tightlist
\item
  \href{https://help.nytimes3xbfgragh.onion/hc/en-us/articles/115014792127-Copyright-notice}{©~2020~The
  New York Times Company}
\end{itemize}

\begin{itemize}
\tightlist
\item
  \href{https://www.nytco.com/}{NYTCo}
\item
  \href{https://help.nytimes3xbfgragh.onion/hc/en-us/articles/115015385887-Contact-Us}{Contact
  Us}
\item
  \href{https://www.nytco.com/careers/}{Work with us}
\item
  \href{https://nytmediakit.com/}{Advertise}
\item
  \href{http://www.tbrandstudio.com/}{T Brand Studio}
\item
  \href{https://www.nytimes3xbfgragh.onion/privacy/cookie-policy\#how-do-i-manage-trackers}{Your
  Ad Choices}
\item
  \href{https://www.nytimes3xbfgragh.onion/privacy}{Privacy}
\item
  \href{https://help.nytimes3xbfgragh.onion/hc/en-us/articles/115014893428-Terms-of-service}{Terms
  of Service}
\item
  \href{https://help.nytimes3xbfgragh.onion/hc/en-us/articles/115014893968-Terms-of-sale}{Terms
  of Sale}
\item
  \href{https://spiderbites.nytimes3xbfgragh.onion}{Site Map}
\item
  \href{https://help.nytimes3xbfgragh.onion/hc/en-us}{Help}
\item
  \href{https://www.nytimes3xbfgragh.onion/subscription?campaignId=37WXW}{Subscriptions}
\end{itemize}
