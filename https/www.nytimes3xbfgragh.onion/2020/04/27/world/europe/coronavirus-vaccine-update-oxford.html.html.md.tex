Sections

SEARCH

\protect\hyperlink{site-content}{Skip to
content}\protect\hyperlink{site-index}{Skip to site index}

\href{https://www.nytimes3xbfgragh.onion/section/world/europe}{Europe}

\href{https://myaccount.nytimes3xbfgragh.onion/auth/login?response_type=cookie\&client_id=vi}{}

\href{https://www.nytimes3xbfgragh.onion/section/todayspaper}{Today's
Paper}

\href{/section/world/europe}{Europe}\textbar{}In Race for a Coronavirus
Vaccine, an Oxford Group Leaps Ahead

\url{https://nyti.ms/3cWsEf9}

\begin{itemize}
\item
\item
\item
\item
\item
\item
\end{itemize}

\hypertarget{the-coronavirus-outbreak}{%
\subsubsection{\texorpdfstring{\href{https://www.nytimes3xbfgragh.onion/news-event/coronavirus?name=styln-coronavirus-national\&region=TOP_BANNER\&block=storyline_menu_recirc\&action=click\&pgtype=Article\&impression_id=1010b1b0-efb5-11ea-8c1d-cb4966d29e98\&variant=undefined}{The
Coronavirus
Outbreak}}{The Coronavirus Outbreak}}\label{the-coronavirus-outbreak}}

\begin{itemize}
\tightlist
\item
  live\href{https://www.nytimes3xbfgragh.onion/2020/09/05/world/coronavirus-covid.html?name=styln-coronavirus-national\&region=TOP_BANNER\&block=storyline_menu_recirc\&action=click\&pgtype=Article\&impression_id=1010d8c0-efb5-11ea-8c1d-cb4966d29e98\&variant=undefined}{Latest
  Updates}
\item
  \href{https://www.nytimes3xbfgragh.onion/interactive/2020/us/coronavirus-us-cases.html?name=styln-coronavirus-national\&region=TOP_BANNER\&block=storyline_menu_recirc\&action=click\&pgtype=Article\&impression_id=1010d8c1-efb5-11ea-8c1d-cb4966d29e98\&variant=undefined}{Maps
  and Cases}
\item
  \href{https://www.nytimes3xbfgragh.onion/interactive/2020/science/coronavirus-vaccine-tracker.html?name=styln-coronavirus-national\&region=TOP_BANNER\&block=storyline_menu_recirc\&action=click\&pgtype=Article\&impression_id=1010d8c2-efb5-11ea-8c1d-cb4966d29e98\&variant=undefined}{Vaccine
  Tracker}
\item
  \href{https://www.nytimes3xbfgragh.onion/2020/09/02/your-money/eviction-moratorium-covid.html?name=styln-coronavirus-national\&region=TOP_BANNER\&block=storyline_menu_recirc\&action=click\&pgtype=Article\&impression_id=1010d8c3-efb5-11ea-8c1d-cb4966d29e98\&variant=undefined}{Eviction
  Moratorium}
\item
  \href{https://www.nytimes3xbfgragh.onion/interactive/2020/09/02/magazine/food-insecurity-hunger-us.html?name=styln-coronavirus-national\&region=TOP_BANNER\&block=storyline_menu_recirc\&action=click\&pgtype=Article\&impression_id=1010d8c4-efb5-11ea-8c1d-cb4966d29e98\&variant=undefined}{American
  Hunger}
\end{itemize}

Advertisement

\protect\hyperlink{after-top}{Continue reading the main story}

Supported by

\protect\hyperlink{after-sponsor}{Continue reading the main story}

\hypertarget{in-race-for-a-coronavirus-vaccine-an-oxford-group-leaps-ahead}{%
\section{In Race for a Coronavirus Vaccine, an Oxford Group Leaps
Ahead}\label{in-race-for-a-coronavirus-vaccine-an-oxford-group-leaps-ahead}}

As scientists at the Jenner Institute prepare for mass clinical trials,
new tests show their vaccine to be effective in monkeys.

\includegraphics{https://static01.graylady3jvrrxbe.onion/images/2020/05/03/world/27virus-vaccine/merlin_171899610_d411bbb0-be14-4e31-89a9-28d47efd9cf6-articleLarge.jpg?quality=75\&auto=webp\&disable=upscale}

\href{https://www.nytimes3xbfgragh.onion/by/david-d-kirkpatrick}{\includegraphics{https://static01.graylady3jvrrxbe.onion/images/2018/10/15/multimedia/author-david-d-kirkpatrick/author-david-d-kirkpatrick-thumbLarge-v2.png}}

By
\href{https://www.nytimes3xbfgragh.onion/by/david-d-kirkpatrick}{David
D. Kirkpatrick}

\begin{itemize}
\item
  Published April 27, 2020Updated May 2, 2020
\item
  \begin{itemize}
  \item
  \item
  \item
  \item
  \item
  \item
  \end{itemize}
\end{itemize}

OXFORD, England --- In the worldwide race for a
\href{https://www.nytimes3xbfgragh.onion/2020/04/08/health/coronavirus-vaccines.html}{vaccine
to stop the coronavirus}, the laboratory sprinting fastest is at Oxford
University.

Most other teams have had to start with small clinical trials of a few
hundred participants to demonstrate safety. But scientists at the
university's \href{https://www.jenner.ac.uk/}{Jenner Institute} had a
head start on a vaccine, having proved in previous trials that similar
inoculations --- including one last year against an earlier coronavirus
--- were harmless to humans.

That has enabled them to leap ahead and schedule tests of their new
coronavirus
\href{https://www.nytimes3xbfgragh.onion/2020/05/15/us/politics/coronavirus-vaccine-timeline.html}{vaccine}
involving more than 6,000 people by the end of next month, hoping to
show not only that it is safe, but also that it works.

The Oxford scientists now say that with an emergency approval from
regulators, the first few million doses of their vaccine could be
available by September --- at least several months ahead of any of the
other announced efforts --- if it proves to be effective.

Now, they have received promising news suggesting that it might.

Scientists at the National Institutes of Health's
\href{https://www.niaid.nih.gov/about/rocky-mountain-overview}{Rocky
Mountain Laboratory} in Montana last month inoculated six rhesus macaque
monkeys with single doses of the Oxford vaccine. The animals were then
exposed to heavy quantities of the virus that is causing the pandemic
--- exposure that had consistently sickened other monkeys in the lab.
But more than 28 days later all six were healthy, said Vincent Munster,
the researcher who conducted the test.

``The rhesus macaque is pretty much the closest thing we have to
humans,'' Dr. Munster said, noting that scientists were still analyzing
the result. He said he expected to
\href{https://www.nytimes3xbfgragh.onion/2020/04/01/world/europe/coronavirus-science-research-cooperation.html}{share
it with other scientists} next week and then submit it to a
peer-reviewed journal.

\emph{{[}}\href{https://www.nytimes3xbfgragh.onion/interactive/2020/science/coronavirus-vaccine-tracker.html}{\emph{Follow
our Live Coronavirus Vaccine Tracker}}\emph{.{]}}

Immunity in monkeys is no guarantee that a vaccine will provide the same
degree of protection for humans. A Chinese company that recently started
a clinical trial with 144 participants,
\href{https://www.sciencemag.org/news/2020/04/covid-19-vaccine-protects-monkeys-new-coronavirus-chinese-biotech-reports}{SinoVac},
has also said that its vaccine was effective in rhesus macaques. But
with dozens of efforts now underway to find a vaccine, the monkey
results are the latest indication that Oxford's accelerated venture is
emerging as a bellwether.

``It is a very, very fast clinical program,'' said Emilio Emini, a
director of the vaccine program at the Bill and Melinda Gates
Foundation, which is providing financial support to many competing
efforts.

\includegraphics{https://static01.graylady3jvrrxbe.onion/images/2020/04/27/world/27vaccine2/merlin_171900807_ab1a20ca-e3d4-40f9-8b35-6dd047d03650-articleLarge.jpg?quality=75\&auto=webp\&disable=upscale}

Which potential vaccine will emerge from the scramble as the most
successful is impossible to know until clinical trial data becomes
available.

\hypertarget{latest-updates-the-coronavirus-outbreak}{%
\section{\texorpdfstring{\href{https://www.nytimes3xbfgragh.onion/2020/09/04/world/covid-19-coronavirus.html?action=click\&pgtype=Article\&state=default\&region=MAIN_CONTENT_1\&context=storylines_live_updates}{Latest
Updates: The Coronavirus
Outbreak}}{Latest Updates: The Coronavirus Outbreak}}\label{latest-updates-the-coronavirus-outbreak}}

Updated 2020-09-05T12:05:40.998Z

\begin{itemize}
\tightlist
\item
  \href{https://www.nytimes3xbfgragh.onion/2020/09/04/world/covid-19-coronavirus.html?action=click\&pgtype=Article\&state=default\&region=MAIN_CONTENT_1\&context=storylines_live_updates\#link-1654f6ad}{Research
  connects vaping to a higher chance of catching the virus --- and
  suffering its worst effects.}
\item
  \href{https://www.nytimes3xbfgragh.onion/2020/09/04/world/covid-19-coronavirus.html?action=click\&pgtype=Article\&state=default\&region=MAIN_CONTENT_1\&context=storylines_live_updates\#link-52e4198a}{Another
  college football game won't be played as planned.}
\item
  \href{https://www.nytimes3xbfgragh.onion/2020/09/04/world/covid-19-coronavirus.html?action=click\&pgtype=Article\&state=default\&region=MAIN_CONTENT_1\&context=storylines_live_updates\#link-181cef0}{Pharmaceutical
  companies plan a joint pledge on safety standards as they move
  vaccines to the marketplace.}
\end{itemize}

\href{https://www.nytimes3xbfgragh.onion/2020/09/04/world/covid-19-coronavirus.html?action=click\&pgtype=Article\&state=default\&region=MAIN_CONTENT_1\&context=storylines_live_updates}{See
more updates}

More live coverage:
\href{https://www.nytimes3xbfgragh.onion/live/2020/09/04/business/stock-market-today-coronavirus?action=click\&pgtype=Article\&state=default\&region=MAIN_CONTENT_1\&context=storylines_live_updates}{Markets}

More than one vaccine would be needed in any case, Dr. Emini argued.
Some may work more effectively than others in groups like children or
older people, or at different costs and dosages. Having more than one
variety of vaccine in production will also help avoid bottlenecks in
manufacturing, he said.

But as the first to reach such a relatively large scale, the Oxford
trial, even if it fails, will provide lessons about the nature of the
coronavirus and about the immune system's responses that can inform
governments, donors, drug companies and other scientists hunting for a
vaccine.

``This big U.K. study,'' Dr. Emini said, ``is actually going to
translate to learning a lot about some of the others as well.''

All of the others will face the same challenges, including obtaining
millions of dollars in funding, persuading regulators to approve human
tests, demonstrating a vaccine's safety and --- after all of that ---
proving its effectiveness in protecting people from the coronavirus.

Paradoxically, the growing success of efforts to contain the spread of
Covid-19, the disease caused by the virus, may present yet another
hurdle.

``We're the only people in the country who want the number of new
infections to stay up for another few weeks, so we can test our
vaccine,'' Prof. Adrian Hill, the Jenner Institute's director and one of
five researchers involved in the effort, said in an interview in a
laboratory building emptied by Britain's monthlong lockdown.

Ethics rules, as a general principle, forbid seeking to infect human
test participants with a serious disease. That means the only way to
prove that a vaccine works is to inoculate people in a place where the
virus is spreading naturally around them.

If social distancing measures or other factors continue to slow the rate
of new infections in Britain, he said, the trial might not be able to
show that the vaccine makes a difference: Participants who received a
placebo might not be infected any more frequently than those who have
been given the vaccine. The scientists would have to try again
elsewhere, a dilemma that every other vaccine effort will face as well.

Image

Social distancing at Oxford last week.Credit...Mary Turner for The New
York Times

The Jenner Institute's coronavirus efforts grew out of Professor Hill's
so-far unsuccessful pursuit of a vaccine against a different scourge,
malaria.

He developed a fascination with malaria and other tropical diseases as a
medical student in Dublin in the early 1980s, when he visited an uncle
who was a priest working in a hospital during the civil war in what is
now Zimbabwe.

``I came back wondering, `What do you see in these hospitals in England
and Ireland?''' Professor Hill said. ``They don't have any of these
diseases.''

The major drug companies typically see little profit in epidemics that
afflict mainly developing countries or run their course before a vaccine
can hit the market. So after training in tropical medicine and a
doctorate in molecular genetics, Professor Hill, 61, helped build
Oxford's institute into one of the largest academic centers dedicated to
nonprofit vaccine research, with its own pilot manufacturing facility
capable of producing a batch of up to 1,000 doses.

The institute's effort against the coronavirus uses a technology that
centers on altering the genetic code of a familiar virus. A classic
vaccine uses a weakened version of a virus to trigger an immune
response. But in the technology that the institute is using, a different
virus is modified first to neutralize its effects and then to make it
mimic the one scientists seek to stop --- in this case, the virus that
causes Covid-19. Injected into the body, the harmless impostor can
induce the immune system to fight and kill the targeted virus, providing
protection.

Professor Hill has worked with that technology for decades to try to
tweak a respiratory virus found in chimpanzees in order to elicit a
human immune response against malaria and other diseases. Over the last
20 years, the institute has conducted more than 70 clinical trials of
potential vaccines against the parasite that causes malaria. None have
yet yielded a successful inoculation.

In 2014, however, a vaccine based on the chimp virus that Professor Hill
had tested was manufactured in a large enough scale to provide a million
doses. That created a template for mass production of the coronavirus
vaccine, should it prove effective.

A longtime colleague, Prof. Sarah Gilbert, 58, modified the same
chimpanzee virus to make a vaccine against an earlier coronavirus, MERS.
After a clinical trial in Britain demonstrated its safety, another test
began in December in Saudi Arabia, where outbreaks of the deadly disease
are still common.

When she heard in January that Chinese scientists had identified the
genetic code of a mysterious virus in Wuhan, she thought she might have
a chance to prove the speed and versatility of their approach.

\href{https://www.nytimes3xbfgragh.onion/news-event/coronavirus?action=click\&pgtype=Article\&state=default\&region=MAIN_CONTENT_3\&context=storylines_faq}{}

\hypertarget{the-coronavirus-outbreak-}{%
\subsubsection{The Coronavirus Outbreak
›}\label{the-coronavirus-outbreak-}}

\hypertarget{frequently-asked-questions}{%
\paragraph{Frequently Asked
Questions}\label{frequently-asked-questions}}

Updated September 4, 2020

\begin{itemize}
\item ~
  \hypertarget{what-are-the-symptoms-of-coronavirus}{%
  \paragraph{What are the symptoms of
  coronavirus?}\label{what-are-the-symptoms-of-coronavirus}}

  \begin{itemize}
  \tightlist
  \item
    In the beginning, the coronavirus
    \href{https://www.nytimes3xbfgragh.onion/article/coronavirus-facts-history.html?action=click\&pgtype=Article\&state=default\&region=MAIN_CONTENT_3\&context=storylines_faq\#link-6817bab5}{seemed
    like it was primarily a respiratory illness}~--- many patients had
    fever and chills, were weak and tired, and coughed a lot, though
    some people don't show many symptoms at all. Those who seemed
    sickest had pneumonia or acute respiratory distress syndrome and
    received supplemental oxygen. By now, doctors have identified many
    more symptoms and syndromes. In April,
    \href{https://www.nytimes3xbfgragh.onion/2020/04/27/health/coronavirus-symptoms-cdc.html?action=click\&pgtype=Article\&state=default\&region=MAIN_CONTENT_3\&context=storylines_faq}{the
    C.D.C. added to the list of early signs}~sore throat, fever, chills
    and muscle aches. Gastrointestinal upset, such as diarrhea and
    nausea, has also been observed. Another telltale sign of infection
    may be a sudden, profound diminution of one's
    \href{https://www.nytimes3xbfgragh.onion/2020/03/22/health/coronavirus-symptoms-smell-taste.html?action=click\&pgtype=Article\&state=default\&region=MAIN_CONTENT_3\&context=storylines_faq}{sense
    of smell and taste.}~Teenagers and young adults in some cases have
    developed painful red and purple lesions on their fingers and toes
    --- nicknamed ``Covid toe'' --- but few other serious symptoms.
  \end{itemize}
\item ~
  \hypertarget{why-is-it-safer-to-spend-time-together-outside}{%
  \paragraph{Why is it safer to spend time together
  outside?}\label{why-is-it-safer-to-spend-time-together-outside}}

  \begin{itemize}
  \tightlist
  \item
    \href{https://www.nytimes3xbfgragh.onion/2020/05/15/us/coronavirus-what-to-do-outside.html?action=click\&pgtype=Article\&state=default\&region=MAIN_CONTENT_3\&context=storylines_faq}{Outdoor
    gatherings}~lower risk because wind disperses viral droplets, and
    sunlight can kill some of the virus. Open spaces prevent the virus
    from building up in concentrated amounts and being inhaled, which
    can happen when infected people exhale in a confined space for long
    stretches of time, said Dr. Julian W. Tang, a virologist at the
    University of Leicester.
  \end{itemize}
\item ~
  \hypertarget{why-does-standing-six-feet-away-from-others-help}{%
  \paragraph{Why does standing six feet away from others
  help?}\label{why-does-standing-six-feet-away-from-others-help}}

  \begin{itemize}
  \tightlist
  \item
    The coronavirus spreads primarily through droplets from your mouth
    and nose, especially when you cough or sneeze. The C.D.C., one of
    the organizations using that measure,
    \href{https://www.nytimes3xbfgragh.onion/2020/04/14/health/coronavirus-six-feet.html?action=click\&pgtype=Article\&state=default\&region=MAIN_CONTENT_3\&context=storylines_faq}{bases
    its recommendation of six feet}~on the idea that most large droplets
    that people expel when they cough or sneeze will fall to the ground
    within six feet. But six feet has never been a magic number that
    guarantees complete protection. Sneezes, for instance, can launch
    droplets a lot farther than six feet,
    \href{https://jamanetwork.com/journals/jama/fullarticle/2763852}{according
    to a recent study}. It's a rule of thumb: You should be safest
    standing six feet apart outside, especially when it's windy. But
    keep a mask on at all times, even when you think you're far enough
    apart.
  \end{itemize}
\item ~
  \hypertarget{i-have-antibodies-am-i-now-immune}{%
  \paragraph{I have antibodies. Am I now
  immune?}\label{i-have-antibodies-am-i-now-immune}}

  \begin{itemize}
  \tightlist
  \item
    As of right
    now,\href{https://www.nytimes3xbfgragh.onion/2020/07/22/health/covid-antibodies-herd-immunity.html?action=click\&pgtype=Article\&state=default\&region=MAIN_CONTENT_3\&context=storylines_faq}{~that
    seems likely, for at least several months.}~There have been
    frightening accounts of people suffering what seems to be a second
    bout of Covid-19. But experts say these patients may have a
    drawn-out course of infection, with the virus taking a slow toll
    weeks to months after initial exposure.~People infected with the
    coronavirus typically
    \href{https://www.nature.com/articles/s41586-020-2456-9}{produce}~immune
    molecules called antibodies, which are
    \href{https://www.nytimes3xbfgragh.onion/2020/05/07/health/coronavirus-antibody-prevalence.html?action=click\&pgtype=Article\&state=default\&region=MAIN_CONTENT_3\&context=storylines_faq}{protective
    proteins made in response to an
    infection}\href{https://www.nytimes3xbfgragh.onion/2020/05/07/health/coronavirus-antibody-prevalence.html?action=click\&pgtype=Article\&state=default\&region=MAIN_CONTENT_3\&context=storylines_faq}{.
    These antibodies may}~last in the body
    \href{https://www.nature.com/articles/s41591-020-0965-6}{only two to
    three months}, which may seem worrisome, but that's~perfectly normal
    after an acute infection subsides, said Dr. Michael Mina, an
    immunologist at Harvard University. It may be possible to get the
    coronavirus again, but it's highly unlikely that it would be
    possible in a short window of time from initial infection or make
    people sicker the second time.
  \end{itemize}
\item ~
  \hypertarget{what-are-my-rights-if-i-am-worried-about-going-back-to-work}{%
  \paragraph{What are my rights if I am worried about going back to
  work?}\label{what-are-my-rights-if-i-am-worried-about-going-back-to-work}}

  \begin{itemize}
  \tightlist
  \item
    Employers have to provide
    \href{https://www.osha.gov/SLTC/covid-19/standards.html}{a safe
    workplace}~with policies that protect everyone equally.
    \href{https://www.nytimes3xbfgragh.onion/article/coronavirus-money-unemployment.html?action=click\&pgtype=Article\&state=default\&region=MAIN_CONTENT_3\&context=storylines_faq}{And
    if one of your co-workers tests positive for the coronavirus, the
    C.D.C.}~has said that
    \href{https://www.cdc.gov/coronavirus/2019-ncov/community/guidance-business-response.html}{employers
    should tell their employees}~-\/- without giving you the sick
    employee's name -\/- that they may have been exposed to the virus.
  \end{itemize}
\end{itemize}

``We thought, `Well, should we have a go?''' she recalled. ```It'll be a
little lab project and we'll publish a paper.'''

It did not stay a ``little lab project'' for long.

Image

Professor Sarah Gilbert, a vaccinologist at the institute, has also
worked on developing a vaccine for MERS, an earlier
coronavirus.Credit...Mary Turner for The New York Times

As the pandemic exploded, grant money poured in. All other vaccines were
soon put into the freezer so that the institute's laboratory could focus
full-time on Covid-19. Then the lockdown forced everyone not working on
Covid-19 to stay home altogether.

``The whole world doesn't usually stand up and say, `How can we help? Do
you want some money?''' Professor Hill said.

``Vaccines are good for pandemics,'' he added, ``and pandemics are good
for vaccines.''

Other scientists involved in the project are working with a half dozen
drug manufacturing companies across Europe and Asia to prepare to churn
out billions of doses as quickly as possible if the vaccine is approved.
None have been granted exclusive marketing rights, and one is the giant
Serum Institute of India, the world's largest supplier of vaccines.

Donors are currently spending tens of millions of dollars to start the
manufacturing process at facilities in Britain and the Netherlands even
before the vaccine is proven to work, said Sandy Douglas, 37, a doctor
at Oxford overseeing vaccine production.

``There is no alternative,'' he said.

But the team has not yet reached an agreement with a North American
manufacturer, in part because the major pharmaceutical companies there
typically demand exclusive worldwide rights before investing in a
potential medicine.

``I personally don't believe that in a time of pandemic there should be
exclusive licenses,'' Professor Hill said. ``So we are asking a lot of
them. Nobody is going to make a lot of money off this.''

The Jenner Institute's vaccine effort is not the only one showing
promise. Two American companies,
\href{https://www.nytimes3xbfgragh.onion/2020/03/16/health/coronavirus-vaccine.html}{Moderna}
and
\href{https://www.nytimes3xbfgragh.onion/2020/01/28/health/coronavirus-vaccine.html}{Inovio},
have started small clinical trials with **** technologies involving
modified or otherwise manipulated genetic material. They are seeking
both to demonstrate their safety and to learn more about dosing and
other variables. Neither technology has ever produced a licensed drug or
been manufactured at scale.

A Chinese company, CanSino, has also started clinical trials in China
using a technology similar to the Oxford institute's, using a strain of
the same respiratory virus that is found in humans, not chimps. But
demonstrating the effectiveness of a vaccine in China may be difficult
because Covid-19 infections there have plummeted.

Armed with safety data from their human trials of similar vaccines for
Ebola, MERS and malaria, though, the scientists at Oxford's institute
persuaded British regulators to allow unusually accelerated trials while
the epidemic is still hot around them.

The institute last week began a Phase I clinical trial involving 1,100
people. Crucially, next month it will begin a combined Phase II and
Phase III trial involving another 5,000. Unlike any other vaccine
project now underway, that trial is designed to prove effectiveness as
well as safety.

The scientists would declare victory if as many as a dozen participants
who are given a placebo become sick with Covid-19 compared with only one
or two who receive the inoculation. ``Then we have a party and tell the
world,'' Professor Hill said. Everyone who had received only the placebo
would also be vaccinated immediately.

If too few participants are infected in Britain, the institute is
planning other trials where the coronavirus may still be spreading,
possibly in Africa or India.

``We'll have to chase the epidemic,'' Professor Hill said. ``If it is
still raging in certain states, it is not inconceivable we end up
testing in the United States in November.''

Carl Zimmer contributed reporting.

Advertisement

\protect\hyperlink{after-bottom}{Continue reading the main story}

\hypertarget{site-index}{%
\subsection{Site Index}\label{site-index}}

\hypertarget{site-information-navigation}{%
\subsection{Site Information
Navigation}\label{site-information-navigation}}

\begin{itemize}
\tightlist
\item
  \href{https://help.nytimes3xbfgragh.onion/hc/en-us/articles/115014792127-Copyright-notice}{©~2020~The
  New York Times Company}
\end{itemize}

\begin{itemize}
\tightlist
\item
  \href{https://www.nytco.com/}{NYTCo}
\item
  \href{https://help.nytimes3xbfgragh.onion/hc/en-us/articles/115015385887-Contact-Us}{Contact
  Us}
\item
  \href{https://www.nytco.com/careers/}{Work with us}
\item
  \href{https://nytmediakit.com/}{Advertise}
\item
  \href{http://www.tbrandstudio.com/}{T Brand Studio}
\item
  \href{https://www.nytimes3xbfgragh.onion/privacy/cookie-policy\#how-do-i-manage-trackers}{Your
  Ad Choices}
\item
  \href{https://www.nytimes3xbfgragh.onion/privacy}{Privacy}
\item
  \href{https://help.nytimes3xbfgragh.onion/hc/en-us/articles/115014893428-Terms-of-service}{Terms
  of Service}
\item
  \href{https://help.nytimes3xbfgragh.onion/hc/en-us/articles/115014893968-Terms-of-sale}{Terms
  of Sale}
\item
  \href{https://spiderbites.nytimes3xbfgragh.onion}{Site Map}
\item
  \href{https://help.nytimes3xbfgragh.onion/hc/en-us}{Help}
\item
  \href{https://www.nytimes3xbfgragh.onion/subscription?campaignId=37WXW}{Subscriptions}
\end{itemize}
