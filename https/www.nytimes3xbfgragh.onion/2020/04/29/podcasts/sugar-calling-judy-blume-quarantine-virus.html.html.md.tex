Sections

SEARCH

\protect\hyperlink{site-content}{Skip to
content}\protect\hyperlink{site-index}{Skip to site index}

\href{https://www.nytimes3xbfgragh.onion/spotlight/podcasts}{Podcasts}

\href{https://myaccount.nytimes3xbfgragh.onion/auth/login?response_type=cookie\&client_id=vi}{}

\href{https://www.nytimes3xbfgragh.onion/section/todayspaper}{Today's
Paper}

\href{/spotlight/podcasts}{Podcasts}\textbar{}`This Terrible Thing Is
Happening, but the World Goes On.'

\url{https://nyti.ms/2KLqFys}

\begin{itemize}
\item
\item
\item
\item
\item
\item
\end{itemize}

Advertisement

\protect\hyperlink{after-top}{Continue reading the main story}

transcript

Back to Sugar Calling

bars

0:00/0:00

-0:00

transcript

\hypertarget{this-terrible-thing-is-happening-but-the-world-goes-on}{%
\subsection{`This Terrible Thing Is Happening, but the World Goes
On.'}\label{this-terrible-thing-is-happening-but-the-world-goes-on}}

\hypertarget{hosted-by-cheryl-strayed-produced-by-kelly-prime-and-edited-by-sara-sarasohn-editorial-oversight-by-wendy-dorr}{%
\subsubsection{Hosted by Cheryl Strayed, produced by Kelly Prime and
edited by Sara Sarasohn. Editorial oversight by Wendy
Dorr.}\label{hosted-by-cheryl-strayed-produced-by-kelly-prime-and-edited-by-sara-sarasohn-editorial-oversight-by-wendy-dorr}}

\hypertarget{cheryl-strayed-talks-with-the-author-judy-blume-about-raising-teens-losing-religion-and-writing-it-all-down}{%
\paragraph{Cheryl Strayed talks with the author Judy Blume about raising
teens, losing religion and writing it all
down.}\label{cheryl-strayed-talks-with-the-author-judy-blume-about-raising-teens-losing-religion-and-writing-it-all-down}}

Wednesday, April 29th, 2020

\begin{itemize}
\item
  cheryl strayed\\
  Today, I'm going to call Judy Blume. She taught me a lot about being a
  teenager and a young woman in her books, ``Are You There God? It's Me,
  Margaret,'' ``Deenie,'' ``Tiger Eyes,'' ``Forever.'' Really, there's
  such a long list. I could go on and on. In fact, I distinctly remember
  the exact shelf that her books occupied in Chaska Middle School. I
  visited often, checked out every single book on that shelf at least
  one, sometimes two or three times. I can't think of anyone better to
  call when seeking wisdom at this really confusing moment in our lives,
  so I'm going to give her a call.
\item
  {[}music{]}
\item
  {[}dial tone{]}
\item
  judy blume\\
  Hello?
\item
  cheryl strayed\\
  Hi, is this Judy?
\item
  judy blume\\
  It is Judy.
\item
  cheryl strayed\\
  Hi, Judy. It's Cheryl Strayed.
\item
  judy blume\\
  Hi, Cheryl. How nice to talk to you.
\item
  cheryl strayed\\
  Oh. I mean, honestly, Judy, in all of the dreams of my life as a
  writer, I never, ever, ever thought I would get to talk to you. I have
  loved you for so long, since I was a little child.
\item
  judy blume\\
  Well, I'm so glad that I'm still here so we can do this.
\item
  cheryl strayed\\
  I'm so glad you're here too. We've made it this far, that I can
  finally reach you. So where are you?
\item
  judy blume\\
  Well, we live in Key West, which is a lovely place to live. It's
  summer all the time, and I'm a summer girl. And I'm here with my
  husband George, and we live in a old low-rise condo, and we're on the
  second floor. And so we're on the ocean with the birds, and it's
  feeding time in the afternoon, and we get to watch all the changes. I
  know it's good for me in normal time, so I'm assuming that it's good
  for me in these times too. I love the sea. I love everything about it.
  I like to go to sleep at night listening to it, so I'm lucky.
\item
  cheryl strayed\\
  So is it just the two of you? Have you been socially isolating for
  some time now?
\item
  judy blume\\
  Since March 12.
\item
  cheryl strayed\\
  Wow.
\item
  judy blume\\
  So yes, it's been a long time. The last day --- I think that may be
  the last day we went to the bookstore. I know it was on a Thursday.
  And it's so interesting, this role reversal with grown kids, and I've
  heard it from so many of my friends, but I know we're not the only
  ones, where they call --- they're worried. It's a very tender kind of
  worry, because I guess they think we're old. {[}LAUGHS{]} I guess we
  are old, but we don't see ourselves that way. We're 82. We think of
  ourselves as young, but we're not. And Larry called, and he said, you
  cannot be going to work at the bookstore with everybody coming into
  that store is a tourist from all over the place. You can't do it
  anymore. And we looked at each other, George and I, and we knew he was
  right. We knew he was right. And of course, the store closed. But we
  did stop. That was our last day. And that's been very hard, Cheryl.
\item
  cheryl strayed\\
  Oh. And so you own a bookstore, right? It's called Books \& Books in
  Key West.
\item
  judy blume\\
  Well, we started it. We don't own it. It's nonprofit. But we're the
  founders. And it's four years old, and it's our baby. And going to
  work four days a week in the bookstore is like this huge joy for me.
  It's my reward after 50 years of writing, which, for me, was enough.
  It's not enough for everybody. But for me, 50 years was enough ---
\item
  cheryl strayed\\
  Right.
\item
  {[}laughter{]}
\item
  judy blume\\
  --- I said, of being locked up in that little room, and here I'm right
  back. I'm right back living the way I lived when I was writing full
  time.
\item
  cheryl strayed\\
  Right. That's so interesting you say that, because the other writers
  I've talked to are all saying, well, you know, it's not as hard for me
  because, of course, as a writer, I have to socially isolate in order
  to get work done. And you're, I guess, retired as a writer, and then
  you were happily ---
\item
  judy blume\\
  No, no, no. No, never say retired. I've never retired. No, no, no.
\item
  cheryl strayed\\
  But you're just not writing right now. Is that what you mean?
\item
  judy blume\\
  I'm not writing right now. And I would have to be doing something. I'm
  a worker. I'm a worker bee. So having a bookstore to go to and run,
  and do all the things that one does in a bookstore, and I love them
  all. And I loved being allowed out after 50 years of not being allowed
  out. And now I'm back. But I know how to do this, because this is the
  way I lived. Except, except for the constant --- or I should say, the
  coming and going of anxiety.
\item
  cheryl strayed\\
  Mm-hm.
\item
  judy blume\\
  You know, it's just so different.
\item
  cheryl strayed\\
  Right. So what are you anxious about, Judy?
\item
  judy blume\\
  Oh. What am I anxious about?
\item
  cheryl strayed\\
  {[}LAUGHS{]} Let's make a list.
\item
  judy blume\\
  I'm anxious about what's happening around the world. I'm anxious about
  the virus. I'm anxious about what it's doing. It's just a very scary
  thing.
\item
  cheryl strayed\\
  Yeah.
\item
  judy blume\\
  And maybe it is that the older you are, the more afraid you are. Maybe
  we're not supposed to be afraid. I don't know. But I'm a mixture of
  the anxiety but the optimist. I'm scared one minute, and then it's
  like, oh, everything is going to be OK. Let's dance. That will help
  me.
\item
  cheryl strayed\\
  And how do you make that shift? How do you go from that anxiety that
  you feel to, let's dance, or to that sense of, it's going to be OK?
\item
  judy blume\\
  {[}LAUGHS{]}

  You know, I grew up with a wonderful father. I adored my father.
  Sadly, he died when he was 54, and I was just 21. And so we didn't get
  to know each other as adults. But he was a philosopher. He was the guy
  who everyone came to with their problems. As I became a teenager, of
  course, I didn't turn to him, because he was my father, and you know
  how that goes. You have to go someplace else.
\item
  cheryl strayed\\
  Right. Of course.
\item
  judy blume\\
  But in our family, he was the youngest of seven children. Nobody lived
  to be 60. All of his siblings. So there was always another death,
  another shiva, which is --- when a Jewish person dies, you call it
  sitting shiva, and people come and visit. That was going on all the
  time I was growing up. And he would always, when I was young, take me
  on his lap and say, life is for the living, and life goes on. Life
  will go on. And yeah, that's true. Life goes on.
\item
  cheryl strayed\\
  So you reach back for that sense of --- to me, when he's saying to
  you, life goes on even in the face of death, that to me is the
  ultimate lesson in essentially embracing that suffering exists, and
  that even in our joy, there is always a piece of sorrow.
\item
  judy blume\\
  Yes.
\item
  cheryl strayed\\
  I find that to be a calming thing. That's helped me shift from anxiety
  to calm, and sometimes all the way to let's dance, if you put that on
  the continuum.
\item
  judy blume\\
  {[}LAUGHS{]} Let's dance is because moving around, moving --- we go
  for morning walks. We have not given up our morning walk, thank
  goodness. We wear masks now, but we go for a two-mile walk every
  morning. And that is so, so important. And the physical activity for
  me --- George is much more mellow, much more relaxed and cavalier
  personality. And he doesn't worry, which is nice, but if --- there's
  always one --- maybe there are couples where no one worries. But the
  couples I know have one who's anxious and a worrier, and the other one
  who isn't. Two worriers in one family would be too much.
\item
  cheryl strayed\\
  I agree.
\item
  judy blume\\
  Right? But it's the one with all the imagination. He often will say to
  me, you have too much imagination. Well, if you're a fiction writer,
  you can't have too much imagination.
\item
  cheryl strayed\\
  That's right. It's come in handy for you, I would say, Judy, that
  imagination.
\item
  judy blume\\
  {[}LAUGHS{]}
\item
  cheryl strayed\\
  So many of those books that you've written, all of them, really, they
  always --- I think one of the most important things that you and your
  books have contributed is really teaching children and teenagers about
  how important it is to have that sense of honesty and integrity, and
  to be sensitive to our own needs and desires and thoughts, and those
  of others as well. How was it that you came to write these books that
  were so much about openness, and honesty, and honoring our emotional
  lives?
\item
  judy blume\\
  Yeah, but let's just --- I just have to tell you the truth here. I had
  no idea what I was doing. When I sat down to write books, I was in my
  very late 20s. Maybe I was 30 when ``Margaret'' was published. I had
  no idea what I was doing. And where does that stuff come from? I don't
  know. I don't understand it. I've given up trying to understand it.
  But it's there when I write. And it may not be there when I have a
  conversation with you. It's there somewhere deep inside, and it comes
  out when I'm writing. I just --- I have a couple of things I wanted to
  share with you, but --- and the other one is not from a book of mine,
  but this is so apropos of what we're saying. ``In the Unlikely
  Event,'' when I was writing that --- that's my last novel, and it's
  going to be my last novel. And it took five years, and it took a lot
  out of me. But it's all based on a series of tragedies that happened
  in my hometown when I was 14. And although it's not autobiographical
  in terms of the characters, it is very much, was a part of my life.
\item
  cheryl strayed\\
  Right. And it's Elizabeth, New Jersey, is that right?
\item
  judy blume\\
  Elizabeth, New Jersey, yeah. In, I can't remember how many days,
  50-something days, we had three major airlines crash.
\item
  cheryl strayed\\
  Into your town.
\item
  judy blume\\
  Yeah, into our town. Into Elizabeth. I did not see any of it. I
  remember it very clearly. And the only thing that surprises me now,
  that, as a fiction writer, I never wrote about it for 40 years, I
  think. But anyway, there was just this one tiny little line in it that
  I remember after I wrote it, I said to myself, where did that come
  from? And how did you know that? Because it just came. And it's the
  night after the first crash, when Miri, who is 15, is in bed with her
  single mom, Rusty, and neither one of them can sleep. And this is from
  Rusty's point of view.

  ``When Miri asked if she believed in God, what was she supposed to
  say? Of course I believe in God, she told her. But how could God let
  such a terrible thing happen? It's not God's job to decide what
  happens, Rusty said. It's his job to help you get through it. Oh, if
  only she really believed that.''

  And that makes me cry, because I don't know where that came from. I do
  remember after a series of tragedies in our family, my father, who was
  raised as an Orthodox Jew, but from the time he married my mother ---
  I mean, she wasn't, and so we weren't. We were Jews, but we weren't
  observant Jews in any way. But my father --- I think it was very, very
  hard for him. This was when his --- my 25-year-old cousin was
  diagnosed with terminal lung cancer, and she had a two-year-old baby.
  And on the night that the doctors told her parents this, my father's
  last surviving sibling, who was her father, died on the spot in the
  hospital. And then a few months later, Elsie, my cousin, died. And my
  father was sitting on the stairs, and I was sitting with him. And he
  said, you know, I just can't believe anymore. I just think I have lost
  my belief in God. It's not there anymore. I've lost it. And you know
  what? A few months later, my father died very quickly and suddenly,
  and I was with him. And all of this was tough. Very tough. And I
  wasn't writing then, of course. So you ask, how do you get through
  things? And after I became a writer, that's how I got through things.
\item
  cheryl strayed\\
  Yeah, writing became your healing.
\item
  judy blume\\
  Well, yeah, it did.
\item
  cheryl strayed\\
  But back there, when your father said, I don't believe in God, and
  then he died, you were a young woman. How old were you when he died?
  21?
\item
  judy blume\\
  21. I was about to get married, believe it or not. The invitations
  were out to the wedding. Way too young, but we did that then. You
  know.
\item
  cheryl strayed\\
  Mm-hm. Yeah. I did that too. Too young.
\item
  judy blume\\
  Too young.
\item
  cheryl strayed\\
  How did you find your way through that darkness?
\item
  judy blume\\
  That was very dark.

  That was very dark. I think I --- I don't know, because I had no one
  to talk to. My mother never spoke about that day. Never. My father was
  the nurturing parent, and the one you could go to, and the one who
  would help you through things. And my mother just couldn't do anything
  like that. And she didn't want us to show anything in public. And
  that's hard.
\item
  cheryl strayed\\
  It's incredibly, and I think too --- you know obviously, it sounds
  like there was this personality difference between your mother and
  father. But I would say that your mother's values around loss and
  grief --- saying, don't show it in public, let's not speak of it again
  --- is very much aligned with the times and the ways that we used to
  think about grief and loss. And I guess that's what I was trying to
  drive at when I was so struck by how brave you were as a writer to
  decide to break those codes of silence about subjects that are
  difficult. Grief, divorce, sexuality, masturbation, racism. All of
  these things that I was so drawn to, like many others, because you
  were speaking honestly with this deep emotional intelligence and
  compassion about things that we're many times told, do not speak of
  these things.

  Did it come alive on the page, and in your real life, you didn't ---
  did you obey your mother when she said, don't speak of it?
\item
  judy blume\\
  Well, I mean, what was I going to do? My mother wasn't going to talk
  to me. One time I went into her room, and she was crying. And I felt
  that, oh my god, I've invaded her privacy. And I felt terrible.
  Terrible. And so I left, instead of going and sitting down and crying.
  I knew that she just couldn't do that. And I felt, I think, kind of
  abandoned then. I felt --- and maybe that's where it came out, in the
  writing. Because it was there. It was there, inside me. And that was
  the way I let it out. But none of what you said just now was
  deliberate. When I look back, I think, oh, for a fearful person --- or
  I was a very fearful child, an anxious child with eczema and other
  anxiety-caused problems ---

  I was courageous in my writing. I was brave in my writing, if not in
  my life.
\item
  cheryl strayed\\
  Mm. And then it sounds like you became braver over time.
\item
  judy blume\\
  Yeah, I think so. I certainly hope so. Yeah.
\item
  cheryl strayed\\
  It's like ---
\item
  judy blume\\
  You know what, I'm a doer. I like to take action. So when I had breast
  cancer, I was told I had breast cancer. OK. OK. Let's get this thing
  done. Let's get this taken care of. And I do my research, and I call
  people, and then I did it. That was it. Let's just do this. And I
  think what's so difficult this time is, there's no action that I can
  take. I can't control this coronavirus all over the world. There's
  nothing I can do. I can vote, but there's really nothing much that I
  can do. Which is kind of why I'm talking to you. {[}LAUGHS{]} I feel
  like this is something I can do. This is something I can do. It may
  never help anybody, but I can get this out, and it's not just sitting
  here and doing nothing.
\item
  cheryl strayed\\
  I actually think you're doing a lot, Judy, in the form of all those
  books that you've written. Think of all the teenagers right now who
  are reading those words that you wrote so long ago, and feeling
  consoled by them during what is probably an incredibly hard time for
  them. I have two teenagers myself, and I know mine are challenged by
  having to be socially distant from their friends during this time when
  they really are wanting more and more independence. And my husband and
  I have gotten used to giving them independence. They get to roam the
  neighborhood or the city, and now suddenly, we're saying, nope. It's
  like you're little kids again. You have to stay with us at our side.
  If not at our side, it's certainly in our house. And that's been
  rather tricky. My joke is that my teenagers have been socially
  distancing from my husband and me for some time now.
\item
  judy blume\\
  Which is what teenagers are supposed to do.
\item
  cheryl strayed\\
  Exactly. Developmentally, that's their job. And now, what we have to
  say to them is, you have to be here with us all the time. And it's
  hard. And so I'm asking, do you have maybe some wisdom to share with
  them? Um, and also to share with the parents. How do we do a good job
  at this? How do we respect those boundaries while also obeying the
  rules of social isolation?
\item
  judy blume\\
  Oh, it's a very hard --- I mean, I wish I had answers for you. One
  thing I'm sharing with parents who have kids, teenage kids, is --- or
  I would say this directly to the kids. Acknowledge your feelings. And
  you don't have to discuss it with your parents either. Write it down.
  What I always did through the hardest times of my adult life --- I
  just got rid of all these books a year or two ago when I moved. I
  probably shouldn't have, but I did. I had notebooks. They weren't
  fancy, they were nothing, but they were whatever was around at the
  time. And through the hardest times --- I never wrote about the happy
  times, but through the hardest times, I would just write things down,
  not fancy. Just --- I'm a doodler. So even with doodles, I would write
  the word ``annoyed,'' and I would doodle all around it. Angry, sad,
  lost, lonely, wishing, missing. All the things that they may be
  feeling. Jot it down. If you can draw it, even better. I never could,
  but I decorated words with my doodles. Because someday when this ends,
  and it will end, then they will have this so that they can remember.
  And they never have to share it. I think that's the most important
  thing. It's theirs. It's private. It's off limits.
\item
  cheryl strayed\\
  Yeah. You know, I just want to pause and say, wait a minute, did you
  --- when you say you got rid of your notebooks, did you donate them to
  the library that's keeping your papers?
\item
  judy blume\\
  Oh, no. No, no, no ---
\item
  cheryl strayed\\
  Judy!
\item
  judy blume\\
  --- no, no, no. They ---
\item
  cheryl strayed\\
  Oh, come on.
\item
  judy blume\\
  No! No, this was stuff so private. I mean, my papers went, but these
  notebooks were shredded. Yes, under my supervision, shredded.
\item
  cheryl strayed\\
  Oh, Judy.
\item
  judy blume\\
  I read them, I saved them. Every year, I took them out. And then I
  thought, you know, I don't want to die with these notebooks out there.
  I don't want anyone I love --- my kids, my husband --- I don't want
  them to read this. Because this was for me.

  While some of it may have been about them, this release was for me.
\item
  cheryl strayed\\
  Right.
\item
  judy blume\\
  And then I didn't need it anymore. And I didn't feel that it belonged
  out there in the world either, because this was who I was at a moment
  in time. Just like this is who your kids are right now. And they may
  not need to do this, because again, they have friends that they're
  sharing with. I don't know. Are they sharing their most intimate, dark
  feelings? Do they want to do that? Then they do that for themselves. I
  never did that when I was growing up. I had a best friend. And we are
  still best friends, god knows how many years later.
\item
  cheryl strayed\\
  Oh, that's so nice.
\item
  judy blume\\
  12 to 82. It's a lot of years. And we talk about that now, how we
  loved each other, we enjoyed each other's company so much. Everything
  was perfect just being together. But we never really told each other
  what was deep, deep inside. Never. Never shared that.
\item
  cheryl strayed\\
  Why do you think you didn't?
\item
  judy blume\\
  Why. I don't know. Was it the times? Are kids more able to do that
  now?
\item
  cheryl strayed\\
  You know, I don't know. I had that same experience when I got back in
  touch with some of my high school friends, maybe a decade or so ago.
  And some of my best, best friends. And it was only then that we shared
  difficult things. That I shared that my father had physically abused
  my mother, for example. Or shared the economic hardship. I grew up in
  poverty, and how stressful that was for me, and how hard I worked as a
  teenager to conceal that. And all of that --- here we were, dear
  friends, and all of that, I was trying to make it invisible. And I
  think that for me, my theory is that it's just too painful. It's too
  painful, and so much about being a teenager, as you well know, that
  you've written about again and again, is just wanting everyone to
  think that you're normal. And we all have this bizarre idea of what
  normal is, and so we do everything we can to pretend to be that way.
  And what your books always reveal is, none of us are normal. We're all
  our original selves. And that's really the most beautiful face we can
  show the world, but it's really hard to do that when you're a
  teenager. And I think it's hard for my kids too. My son --- I think
  especially maybe in this case for boys, because I asked them that same
  question. I think my daughter does talk to her friends, at least to
  some degree, about her struggles. But my son has said, mom, we just
  don't operate that way. What he wants to do is go skateboarding with
  his friends. And they speak with a kind of body language that's not
  about words so much.
\item
  judy blume\\
  There's a little --- I mean, there is a male-female thing there too.
\item
  cheryl strayed\\
  Yeah, yeah.
\item
  judy blume\\
  I mean, that's like George. But --- oh, Cheryl, can I read you
  something that I absolutely adore?
\item
  cheryl strayed\\
  Yes, please do.
\item
  judy blume\\
  And this is not mine. This is something I just love. It's from Maira
  Kalman. Do you know who she is?
\item
  cheryl strayed\\
  I do.
\item
  judy blume\\
  She is my favorite Illustrator. And oh my gosh, I wish I could show
  you the glorious illustrations that are in this book. And the book is
  called ``Beloved Dog.'' And for some reason, I come to it again and
  again and again. And her words are simple, yet not simple. So this is
  it, OK?
\item
  cheryl strayed\\
  OK.
\item
  judy blume\\
  ``When I go out for a walk, there's so much I see that makes me happy
  to be alive --- breathing, not thinking, observing. I am grateful
  beyond measure to be a part of it all. There are people, of course,
  heroic and heartbreaking, going about their business in splendid
  fashion. There are trees, glorious and consoling, changing with the
  seasons, reminders that all things change and change again. There are
  flowers, birds, babies, buildings. I love all these things. But above
  all, I am besotted by dogs. They are constant reminders that life
  reveals the best of itself when we live fully in the moment and extend
  our unconditional love. And it is very true that the most tender,
  uncomplicated, most generous part of our being blossoms without any
  effort when it comes to the love of a dog.''
\item
  cheryl strayed\\
  Mm. Wow. That is so beautiful, Judy.
\item
  judy blume\\
  I have one fantasy left as a professional, as a writer. My one fantasy
  is that I would love to collaborate with Maira Kalman. I just think
  she is brilliant and wonderful. And now you have to ask me, do I have
  a dog?
\item
  cheryl strayed\\
  Do you have a dog, since you read that?
\item
  judy blume\\
  No.
\item
  cheryl strayed\\
  Why did you choose that passage?
\item
  judy blume\\
  I don't have a dog. And I love that in this book, she tells us that
  she was terrified of dogs because her mother and grandmother, I think,
  said, be careful, they'll lunge for your neck and bite your head off,
  something like that. And my mother was terrified of dogs, so I didn't
  grow up with a dog. But then my son got a dog. And I fell in love with
  this dog. I called Lukie my granddog. And I learned what it is about
  loving a dog. It was just so unconditional and satisfying, and very,
  very sad when Lukie died. She was 16. And I always thought that Larry
  would get another dog, but he never has. And of course, I watch them
  here in Key West, and I just love watching them. But I don't have my
  own dog.
\item
  cheryl strayed\\
  Well, why don't you get one, Judy?
\item
  judy blume\\
  Because George says, you have me, Judy. If you don't have me, you
  should definitely get a dog. But as long as you have me, you're OK.
  You don't need a dog.
\item
  cheryl strayed\\
  I'll just say --- I don't have a vote in this, so I can express my
  opinion. I have two dogs and a husband. And trust me, the things that
  the dogs give me are very different than the things my husband gives
  me.
\item
  judy blume\\
  Oh, I know. I know. But I agree with him that we are probably so
  different that we should not try to have a dog together. And you know,
  this is not just, of course, about the love of a dog. This also could
  be the love for a friend, or the love for a family member. It just
  seems to me to be a lovely piece. I just like opening this book and
  reading.
\item
  cheryl strayed\\
  Well, and to me, it seems to be about the pleasure we take in the
  small things. The way that we find beauty or wonder in every day. Are
  you finding wonder and beauty every day in the midst of these
  difficult times?
\item
  judy blume\\
  I think it's hard. I do love when we go out walking. I don't know if
  you know this about Key West, but chickens roam freely in Key West,
  and they're protected. And I don't always love the chickens, but this
  time of year, with their little babies walking behind them, that's
  wonder right there. You know, the world goes on. This terrible thing
  is happening, but the world goes on.
\item
  cheryl strayed\\
  There will always be little yellow baby birds, no matter what.
\item
  {[}laughter{]}
\item
  judy blume\\
  I guess so. I hope so.
\item
  cheryl strayed\\
  I do too. So Judy, it's been wonderful to speak to you, and thank you
  so much for sharing yourself with me.
\item
  judy blume\\
  Thank you. OK. Stay well.
\item
  cheryl strayed\\
  Bye bye, hon.
\item
  judy blume\\
  Bye.
\item
  cheryl strayed\\
  Bye.
\item
  {[}music{]}\\
  I'm Cheryl Strayed. This is ``Sugar Calling.'' Next week, Alice
  Walker.
\end{itemize}

\href{https://www.nytimes3xbfgragh.onion/column/sugar-calling}{\includegraphics{https://static01.graylady3jvrrxbe.onion/images/2020/04/29/podcasts/sugar-calling-album-art/sugar-calling-album-art-square320.jpg}Sugar
Calling}Subscribe:

\begin{itemize}
\tightlist
\item
  \href{https://itunes.apple.com/us/podcast/id1505881384}{Apple
  Podcasts}
\item
  \href{https://podcasts.google.com/?feed=aHR0cHM6Ly9yc3MuYXJ0MTkuY29tL3N1Z2FyLWNhbGxpbmc\&ved=0CAUQrrcFahcKEwjA8Kyn09voAhUAAAAAHQAAAAAQBQ}{Google
  Podcasts}
\end{itemize}

\hypertarget{this-terrible-thing-is-happening-but-the-world-goes-on-1}{%
\section{`This Terrible Thing Is Happening, but the World Goes
On.'}\label{this-terrible-thing-is-happening-but-the-world-goes-on-1}}

\hypertarget{cheryl-strayed-talks-with-the-author-judy-blume-about-raising-teens-losing-religion-and-writing-it-all-down-1}{%
\subsection{Cheryl Strayed talks with the author Judy Blume about
raising teens, losing religion and writing it all
down.}\label{cheryl-strayed-talks-with-the-author-judy-blume-about-raising-teens-losing-religion-and-writing-it-all-down-1}}

Hosted by Cheryl Strayed, produced by Kelly Prime and edited by Sara
Sarasohn. Editorial oversight by Wendy Dorr.

Transcript

transcript

Back to Sugar Calling

bars

0:00/0:00

-0:00

transcript

\hypertarget{this-terrible-thing-is-happening-but-the-world-goes-on-2}{%
\subsection{`This Terrible Thing Is Happening, but the World Goes
On.'}\label{this-terrible-thing-is-happening-but-the-world-goes-on-2}}

\hypertarget{hosted-by-cheryl-strayed-produced-by-kelly-prime-and-edited-by-sara-sarasohn-editorial-oversight-by-wendy-dorr-1}{%
\subsubsection{Hosted by Cheryl Strayed, produced by Kelly Prime and
edited by Sara Sarasohn. Editorial oversight by Wendy
Dorr.}\label{hosted-by-cheryl-strayed-produced-by-kelly-prime-and-edited-by-sara-sarasohn-editorial-oversight-by-wendy-dorr-1}}

\hypertarget{cheryl-strayed-talks-with-the-author-judy-blume-about-raising-teens-losing-religion-and-writing-it-all-down-2}{%
\paragraph{Cheryl Strayed talks with the author Judy Blume about raising
teens, losing religion and writing it all
down.}\label{cheryl-strayed-talks-with-the-author-judy-blume-about-raising-teens-losing-religion-and-writing-it-all-down-2}}

Wednesday, April 29th, 2020

\begin{itemize}
\item
  cheryl strayed\\
  Today, I'm going to call Judy Blume. She taught me a lot about being a
  teenager and a young woman in her books, ``Are You There God? It's Me,
  Margaret,'' ``Deenie,'' ``Tiger Eyes,'' ``Forever.'' Really, there's
  such a long list. I could go on and on. In fact, I distinctly remember
  the exact shelf that her books occupied in Chaska Middle School. I
  visited often, checked out every single book on that shelf at least
  one, sometimes two or three times. I can't think of anyone better to
  call when seeking wisdom at this really confusing moment in our lives,
  so I'm going to give her a call.
\item
  {[}music{]}
\item
  {[}dial tone{]}
\item
  judy blume\\
  Hello?
\item
  cheryl strayed\\
  Hi, is this Judy?
\item
  judy blume\\
  It is Judy.
\item
  cheryl strayed\\
  Hi, Judy. It's Cheryl Strayed.
\item
  judy blume\\
  Hi, Cheryl. How nice to talk to you.
\item
  cheryl strayed\\
  Oh. I mean, honestly, Judy, in all of the dreams of my life as a
  writer, I never, ever, ever thought I would get to talk to you. I have
  loved you for so long, since I was a little child.
\item
  judy blume\\
  Well, I'm so glad that I'm still here so we can do this.
\item
  cheryl strayed\\
  I'm so glad you're here too. We've made it this far, that I can
  finally reach you. So where are you?
\item
  judy blume\\
  Well, we live in Key West, which is a lovely place to live. It's
  summer all the time, and I'm a summer girl. And I'm here with my
  husband George, and we live in a old low-rise condo, and we're on the
  second floor. And so we're on the ocean with the birds, and it's
  feeding time in the afternoon, and we get to watch all the changes. I
  know it's good for me in normal time, so I'm assuming that it's good
  for me in these times too. I love the sea. I love everything about it.
  I like to go to sleep at night listening to it, so I'm lucky.
\item
  cheryl strayed\\
  So is it just the two of you? Have you been socially isolating for
  some time now?
\item
  judy blume\\
  Since March 12.
\item
  cheryl strayed\\
  Wow.
\item
  judy blume\\
  So yes, it's been a long time. The last day --- I think that may be
  the last day we went to the bookstore. I know it was on a Thursday.
  And it's so interesting, this role reversal with grown kids, and I've
  heard it from so many of my friends, but I know we're not the only
  ones, where they call --- they're worried. It's a very tender kind of
  worry, because I guess they think we're old. {[}LAUGHS{]} I guess we
  are old, but we don't see ourselves that way. We're 82. We think of
  ourselves as young, but we're not. And Larry called, and he said, you
  cannot be going to work at the bookstore with everybody coming into
  that store is a tourist from all over the place. You can't do it
  anymore. And we looked at each other, George and I, and we knew he was
  right. We knew he was right. And of course, the store closed. But we
  did stop. That was our last day. And that's been very hard, Cheryl.
\item
  cheryl strayed\\
  Oh. And so you own a bookstore, right? It's called Books \& Books in
  Key West.
\item
  judy blume\\
  Well, we started it. We don't own it. It's nonprofit. But we're the
  founders. And it's four years old, and it's our baby. And going to
  work four days a week in the bookstore is like this huge joy for me.
  It's my reward after 50 years of writing, which, for me, was enough.
  It's not enough for everybody. But for me, 50 years was enough ---
\item
  cheryl strayed\\
  Right.
\item
  {[}laughter{]}
\item
  judy blume\\
  --- I said, of being locked up in that little room, and here I'm right
  back. I'm right back living the way I lived when I was writing full
  time.
\item
  cheryl strayed\\
  Right. That's so interesting you say that, because the other writers
  I've talked to are all saying, well, you know, it's not as hard for me
  because, of course, as a writer, I have to socially isolate in order
  to get work done. And you're, I guess, retired as a writer, and then
  you were happily ---
\item
  judy blume\\
  No, no, no. No, never say retired. I've never retired. No, no, no.
\item
  cheryl strayed\\
  But you're just not writing right now. Is that what you mean?
\item
  judy blume\\
  I'm not writing right now. And I would have to be doing something. I'm
  a worker. I'm a worker bee. So having a bookstore to go to and run,
  and do all the things that one does in a bookstore, and I love them
  all. And I loved being allowed out after 50 years of not being allowed
  out. And now I'm back. But I know how to do this, because this is the
  way I lived. Except, except for the constant --- or I should say, the
  coming and going of anxiety.
\item
  cheryl strayed\\
  Mm-hm.
\item
  judy blume\\
  You know, it's just so different.
\item
  cheryl strayed\\
  Right. So what are you anxious about, Judy?
\item
  judy blume\\
  Oh. What am I anxious about?
\item
  cheryl strayed\\
  {[}LAUGHS{]} Let's make a list.
\item
  judy blume\\
  I'm anxious about what's happening around the world. I'm anxious about
  the virus. I'm anxious about what it's doing. It's just a very scary
  thing.
\item
  cheryl strayed\\
  Yeah.
\item
  judy blume\\
  And maybe it is that the older you are, the more afraid you are. Maybe
  we're not supposed to be afraid. I don't know. But I'm a mixture of
  the anxiety but the optimist. I'm scared one minute, and then it's
  like, oh, everything is going to be OK. Let's dance. That will help
  me.
\item
  cheryl strayed\\
  And how do you make that shift? How do you go from that anxiety that
  you feel to, let's dance, or to that sense of, it's going to be OK?
\item
  judy blume\\
  {[}LAUGHS{]}

  You know, I grew up with a wonderful father. I adored my father.
  Sadly, he died when he was 54, and I was just 21. And so we didn't get
  to know each other as adults. But he was a philosopher. He was the guy
  who everyone came to with their problems. As I became a teenager, of
  course, I didn't turn to him, because he was my father, and you know
  how that goes. You have to go someplace else.
\item
  cheryl strayed\\
  Right. Of course.
\item
  judy blume\\
  But in our family, he was the youngest of seven children. Nobody lived
  to be 60. All of his siblings. So there was always another death,
  another shiva, which is --- when a Jewish person dies, you call it
  sitting shiva, and people come and visit. That was going on all the
  time I was growing up. And he would always, when I was young, take me
  on his lap and say, life is for the living, and life goes on. Life
  will go on. And yeah, that's true. Life goes on.
\item
  cheryl strayed\\
  So you reach back for that sense of --- to me, when he's saying to
  you, life goes on even in the face of death, that to me is the
  ultimate lesson in essentially embracing that suffering exists, and
  that even in our joy, there is always a piece of sorrow.
\item
  judy blume\\
  Yes.
\item
  cheryl strayed\\
  I find that to be a calming thing. That's helped me shift from anxiety
  to calm, and sometimes all the way to let's dance, if you put that on
  the continuum.
\item
  judy blume\\
  {[}LAUGHS{]} Let's dance is because moving around, moving --- we go
  for morning walks. We have not given up our morning walk, thank
  goodness. We wear masks now, but we go for a two-mile walk every
  morning. And that is so, so important. And the physical activity for
  me --- George is much more mellow, much more relaxed and cavalier
  personality. And he doesn't worry, which is nice, but if --- there's
  always one --- maybe there are couples where no one worries. But the
  couples I know have one who's anxious and a worrier, and the other one
  who isn't. Two worriers in one family would be too much.
\item
  cheryl strayed\\
  I agree.
\item
  judy blume\\
  Right? But it's the one with all the imagination. He often will say to
  me, you have too much imagination. Well, if you're a fiction writer,
  you can't have too much imagination.
\item
  cheryl strayed\\
  That's right. It's come in handy for you, I would say, Judy, that
  imagination.
\item
  judy blume\\
  {[}LAUGHS{]}
\item
  cheryl strayed\\
  So many of those books that you've written, all of them, really, they
  always --- I think one of the most important things that you and your
  books have contributed is really teaching children and teenagers about
  how important it is to have that sense of honesty and integrity, and
  to be sensitive to our own needs and desires and thoughts, and those
  of others as well. How was it that you came to write these books that
  were so much about openness, and honesty, and honoring our emotional
  lives?
\item
  judy blume\\
  Yeah, but let's just --- I just have to tell you the truth here. I had
  no idea what I was doing. When I sat down to write books, I was in my
  very late 20s. Maybe I was 30 when ``Margaret'' was published. I had
  no idea what I was doing. And where does that stuff come from? I don't
  know. I don't understand it. I've given up trying to understand it.
  But it's there when I write. And it may not be there when I have a
  conversation with you. It's there somewhere deep inside, and it comes
  out when I'm writing. I just --- I have a couple of things I wanted to
  share with you, but --- and the other one is not from a book of mine,
  but this is so apropos of what we're saying. ``In the Unlikely
  Event,'' when I was writing that --- that's my last novel, and it's
  going to be my last novel. And it took five years, and it took a lot
  out of me. But it's all based on a series of tragedies that happened
  in my hometown when I was 14. And although it's not autobiographical
  in terms of the characters, it is very much, was a part of my life.
\item
  cheryl strayed\\
  Right. And it's Elizabeth, New Jersey, is that right?
\item
  judy blume\\
  Elizabeth, New Jersey, yeah. In, I can't remember how many days,
  50-something days, we had three major airlines crash.
\item
  cheryl strayed\\
  Into your town.
\item
  judy blume\\
  Yeah, into our town. Into Elizabeth. I did not see any of it. I
  remember it very clearly. And the only thing that surprises me now,
  that, as a fiction writer, I never wrote about it for 40 years, I
  think. But anyway, there was just this one tiny little line in it that
  I remember after I wrote it, I said to myself, where did that come
  from? And how did you know that? Because it just came. And it's the
  night after the first crash, when Miri, who is 15, is in bed with her
  single mom, Rusty, and neither one of them can sleep. And this is from
  Rusty's point of view.

  ``When Miri asked if she believed in God, what was she supposed to
  say? Of course I believe in God, she told her. But how could God let
  such a terrible thing happen? It's not God's job to decide what
  happens, Rusty said. It's his job to help you get through it. Oh, if
  only she really believed that.''

  And that makes me cry, because I don't know where that came from. I do
  remember after a series of tragedies in our family, my father, who was
  raised as an Orthodox Jew, but from the time he married my mother ---
  I mean, she wasn't, and so we weren't. We were Jews, but we weren't
  observant Jews in any way. But my father --- I think it was very, very
  hard for him. This was when his --- my 25-year-old cousin was
  diagnosed with terminal lung cancer, and she had a two-year-old baby.
  And on the night that the doctors told her parents this, my father's
  last surviving sibling, who was her father, died on the spot in the
  hospital. And then a few months later, Elsie, my cousin, died. And my
  father was sitting on the stairs, and I was sitting with him. And he
  said, you know, I just can't believe anymore. I just think I have lost
  my belief in God. It's not there anymore. I've lost it. And you know
  what? A few months later, my father died very quickly and suddenly,
  and I was with him. And all of this was tough. Very tough. And I
  wasn't writing then, of course. So you ask, how do you get through
  things? And after I became a writer, that's how I got through things.
\item
  cheryl strayed\\
  Yeah, writing became your healing.
\item
  judy blume\\
  Well, yeah, it did.
\item
  cheryl strayed\\
  But back there, when your father said, I don't believe in God, and
  then he died, you were a young woman. How old were you when he died?
  21?
\item
  judy blume\\
  21. I was about to get married, believe it or not. The invitations
  were out to the wedding. Way too young, but we did that then. You
  know.
\item
  cheryl strayed\\
  Mm-hm. Yeah. I did that too. Too young.
\item
  judy blume\\
  Too young.
\item
  cheryl strayed\\
  How did you find your way through that darkness?
\item
  judy blume\\
  That was very dark.

  That was very dark. I think I --- I don't know, because I had no one
  to talk to. My mother never spoke about that day. Never. My father was
  the nurturing parent, and the one you could go to, and the one who
  would help you through things. And my mother just couldn't do anything
  like that. And she didn't want us to show anything in public. And
  that's hard.
\item
  cheryl strayed\\
  It's incredibly, and I think too --- you know obviously, it sounds
  like there was this personality difference between your mother and
  father. But I would say that your mother's values around loss and
  grief --- saying, don't show it in public, let's not speak of it again
  --- is very much aligned with the times and the ways that we used to
  think about grief and loss. And I guess that's what I was trying to
  drive at when I was so struck by how brave you were as a writer to
  decide to break those codes of silence about subjects that are
  difficult. Grief, divorce, sexuality, masturbation, racism. All of
  these things that I was so drawn to, like many others, because you
  were speaking honestly with this deep emotional intelligence and
  compassion about things that we're many times told, do not speak of
  these things.

  Did it come alive on the page, and in your real life, you didn't ---
  did you obey your mother when she said, don't speak of it?
\item
  judy blume\\
  Well, I mean, what was I going to do? My mother wasn't going to talk
  to me. One time I went into her room, and she was crying. And I felt
  that, oh my god, I've invaded her privacy. And I felt terrible.
  Terrible. And so I left, instead of going and sitting down and crying.
  I knew that she just couldn't do that. And I felt, I think, kind of
  abandoned then. I felt --- and maybe that's where it came out, in the
  writing. Because it was there. It was there, inside me. And that was
  the way I let it out. But none of what you said just now was
  deliberate. When I look back, I think, oh, for a fearful person --- or
  I was a very fearful child, an anxious child with eczema and other
  anxiety-caused problems ---

  I was courageous in my writing. I was brave in my writing, if not in
  my life.
\item
  cheryl strayed\\
  Mm. And then it sounds like you became braver over time.
\item
  judy blume\\
  Yeah, I think so. I certainly hope so. Yeah.
\item
  cheryl strayed\\
  It's like ---
\item
  judy blume\\
  You know what, I'm a doer. I like to take action. So when I had breast
  cancer, I was told I had breast cancer. OK. OK. Let's get this thing
  done. Let's get this taken care of. And I do my research, and I call
  people, and then I did it. That was it. Let's just do this. And I
  think what's so difficult this time is, there's no action that I can
  take. I can't control this coronavirus all over the world. There's
  nothing I can do. I can vote, but there's really nothing much that I
  can do. Which is kind of why I'm talking to you. {[}LAUGHS{]} I feel
  like this is something I can do. This is something I can do. It may
  never help anybody, but I can get this out, and it's not just sitting
  here and doing nothing.
\item
  cheryl strayed\\
  I actually think you're doing a lot, Judy, in the form of all those
  books that you've written. Think of all the teenagers right now who
  are reading those words that you wrote so long ago, and feeling
  consoled by them during what is probably an incredibly hard time for
  them. I have two teenagers myself, and I know mine are challenged by
  having to be socially distant from their friends during this time when
  they really are wanting more and more independence. And my husband and
  I have gotten used to giving them independence. They get to roam the
  neighborhood or the city, and now suddenly, we're saying, nope. It's
  like you're little kids again. You have to stay with us at our side.
  If not at our side, it's certainly in our house. And that's been
  rather tricky. My joke is that my teenagers have been socially
  distancing from my husband and me for some time now.
\item
  judy blume\\
  Which is what teenagers are supposed to do.
\item
  cheryl strayed\\
  Exactly. Developmentally, that's their job. And now, what we have to
  say to them is, you have to be here with us all the time. And it's
  hard. And so I'm asking, do you have maybe some wisdom to share with
  them? Um, and also to share with the parents. How do we do a good job
  at this? How do we respect those boundaries while also obeying the
  rules of social isolation?
\item
  judy blume\\
  Oh, it's a very hard --- I mean, I wish I had answers for you. One
  thing I'm sharing with parents who have kids, teenage kids, is --- or
  I would say this directly to the kids. Acknowledge your feelings. And
  you don't have to discuss it with your parents either. Write it down.
  What I always did through the hardest times of my adult life --- I
  just got rid of all these books a year or two ago when I moved. I
  probably shouldn't have, but I did. I had notebooks. They weren't
  fancy, they were nothing, but they were whatever was around at the
  time. And through the hardest times --- I never wrote about the happy
  times, but through the hardest times, I would just write things down,
  not fancy. Just --- I'm a doodler. So even with doodles, I would write
  the word ``annoyed,'' and I would doodle all around it. Angry, sad,
  lost, lonely, wishing, missing. All the things that they may be
  feeling. Jot it down. If you can draw it, even better. I never could,
  but I decorated words with my doodles. Because someday when this ends,
  and it will end, then they will have this so that they can remember.
  And they never have to share it. I think that's the most important
  thing. It's theirs. It's private. It's off limits.
\item
  cheryl strayed\\
  Yeah. You know, I just want to pause and say, wait a minute, did you
  --- when you say you got rid of your notebooks, did you donate them to
  the library that's keeping your papers?
\item
  judy blume\\
  Oh, no. No, no, no ---
\item
  cheryl strayed\\
  Judy!
\item
  judy blume\\
  --- no, no, no. They ---
\item
  cheryl strayed\\
  Oh, come on.
\item
  judy blume\\
  No! No, this was stuff so private. I mean, my papers went, but these
  notebooks were shredded. Yes, under my supervision, shredded.
\item
  cheryl strayed\\
  Oh, Judy.
\item
  judy blume\\
  I read them, I saved them. Every year, I took them out. And then I
  thought, you know, I don't want to die with these notebooks out there.
  I don't want anyone I love --- my kids, my husband --- I don't want
  them to read this. Because this was for me.

  While some of it may have been about them, this release was for me.
\item
  cheryl strayed\\
  Right.
\item
  judy blume\\
  And then I didn't need it anymore. And I didn't feel that it belonged
  out there in the world either, because this was who I was at a moment
  in time. Just like this is who your kids are right now. And they may
  not need to do this, because again, they have friends that they're
  sharing with. I don't know. Are they sharing their most intimate, dark
  feelings? Do they want to do that? Then they do that for themselves. I
  never did that when I was growing up. I had a best friend. And we are
  still best friends, god knows how many years later.
\item
  cheryl strayed\\
  Oh, that's so nice.
\item
  judy blume\\
  12 to 82. It's a lot of years. And we talk about that now, how we
  loved each other, we enjoyed each other's company so much. Everything
  was perfect just being together. But we never really told each other
  what was deep, deep inside. Never. Never shared that.
\item
  cheryl strayed\\
  Why do you think you didn't?
\item
  judy blume\\
  Why. I don't know. Was it the times? Are kids more able to do that
  now?
\item
  cheryl strayed\\
  You know, I don't know. I had that same experience when I got back in
  touch with some of my high school friends, maybe a decade or so ago.
  And some of my best, best friends. And it was only then that we shared
  difficult things. That I shared that my father had physically abused
  my mother, for example. Or shared the economic hardship. I grew up in
  poverty, and how stressful that was for me, and how hard I worked as a
  teenager to conceal that. And all of that --- here we were, dear
  friends, and all of that, I was trying to make it invisible. And I
  think that for me, my theory is that it's just too painful. It's too
  painful, and so much about being a teenager, as you well know, that
  you've written about again and again, is just wanting everyone to
  think that you're normal. And we all have this bizarre idea of what
  normal is, and so we do everything we can to pretend to be that way.
  And what your books always reveal is, none of us are normal. We're all
  our original selves. And that's really the most beautiful face we can
  show the world, but it's really hard to do that when you're a
  teenager. And I think it's hard for my kids too. My son --- I think
  especially maybe in this case for boys, because I asked them that same
  question. I think my daughter does talk to her friends, at least to
  some degree, about her struggles. But my son has said, mom, we just
  don't operate that way. What he wants to do is go skateboarding with
  his friends. And they speak with a kind of body language that's not
  about words so much.
\item
  judy blume\\
  There's a little --- I mean, there is a male-female thing there too.
\item
  cheryl strayed\\
  Yeah, yeah.
\item
  judy blume\\
  I mean, that's like George. But --- oh, Cheryl, can I read you
  something that I absolutely adore?
\item
  cheryl strayed\\
  Yes, please do.
\item
  judy blume\\
  And this is not mine. This is something I just love. It's from Maira
  Kalman. Do you know who she is?
\item
  cheryl strayed\\
  I do.
\item
  judy blume\\
  She is my favorite Illustrator. And oh my gosh, I wish I could show
  you the glorious illustrations that are in this book. And the book is
  called ``Beloved Dog.'' And for some reason, I come to it again and
  again and again. And her words are simple, yet not simple. So this is
  it, OK?
\item
  cheryl strayed\\
  OK.
\item
  judy blume\\
  ``When I go out for a walk, there's so much I see that makes me happy
  to be alive --- breathing, not thinking, observing. I am grateful
  beyond measure to be a part of it all. There are people, of course,
  heroic and heartbreaking, going about their business in splendid
  fashion. There are trees, glorious and consoling, changing with the
  seasons, reminders that all things change and change again. There are
  flowers, birds, babies, buildings. I love all these things. But above
  all, I am besotted by dogs. They are constant reminders that life
  reveals the best of itself when we live fully in the moment and extend
  our unconditional love. And it is very true that the most tender,
  uncomplicated, most generous part of our being blossoms without any
  effort when it comes to the love of a dog.''
\item
  cheryl strayed\\
  Mm. Wow. That is so beautiful, Judy.
\item
  judy blume\\
  I have one fantasy left as a professional, as a writer. My one fantasy
  is that I would love to collaborate with Maira Kalman. I just think
  she is brilliant and wonderful. And now you have to ask me, do I have
  a dog?
\item
  cheryl strayed\\
  Do you have a dog, since you read that?
\item
  judy blume\\
  No.
\item
  cheryl strayed\\
  Why did you choose that passage?
\item
  judy blume\\
  I don't have a dog. And I love that in this book, she tells us that
  she was terrified of dogs because her mother and grandmother, I think,
  said, be careful, they'll lunge for your neck and bite your head off,
  something like that. And my mother was terrified of dogs, so I didn't
  grow up with a dog. But then my son got a dog. And I fell in love with
  this dog. I called Lukie my granddog. And I learned what it is about
  loving a dog. It was just so unconditional and satisfying, and very,
  very sad when Lukie died. She was 16. And I always thought that Larry
  would get another dog, but he never has. And of course, I watch them
  here in Key West, and I just love watching them. But I don't have my
  own dog.
\item
  cheryl strayed\\
  Well, why don't you get one, Judy?
\item
  judy blume\\
  Because George says, you have me, Judy. If you don't have me, you
  should definitely get a dog. But as long as you have me, you're OK.
  You don't need a dog.
\item
  cheryl strayed\\
  I'll just say --- I don't have a vote in this, so I can express my
  opinion. I have two dogs and a husband. And trust me, the things that
  the dogs give me are very different than the things my husband gives
  me.
\item
  judy blume\\
  Oh, I know. I know. But I agree with him that we are probably so
  different that we should not try to have a dog together. And you know,
  this is not just, of course, about the love of a dog. This also could
  be the love for a friend, or the love for a family member. It just
  seems to me to be a lovely piece. I just like opening this book and
  reading.
\item
  cheryl strayed\\
  Well, and to me, it seems to be about the pleasure we take in the
  small things. The way that we find beauty or wonder in every day. Are
  you finding wonder and beauty every day in the midst of these
  difficult times?
\item
  judy blume\\
  I think it's hard. I do love when we go out walking. I don't know if
  you know this about Key West, but chickens roam freely in Key West,
  and they're protected. And I don't always love the chickens, but this
  time of year, with their little babies walking behind them, that's
  wonder right there. You know, the world goes on. This terrible thing
  is happening, but the world goes on.
\item
  cheryl strayed\\
  There will always be little yellow baby birds, no matter what.
\item
  {[}laughter{]}
\item
  judy blume\\
  I guess so. I hope so.
\item
  cheryl strayed\\
  I do too. So Judy, it's been wonderful to speak to you, and thank you
  so much for sharing yourself with me.
\item
  judy blume\\
  Thank you. OK. Stay well.
\item
  cheryl strayed\\
  Bye bye, hon.
\item
  judy blume\\
  Bye.
\item
  cheryl strayed\\
  Bye.
\item
  {[}music{]}\\
  I'm Cheryl Strayed. This is ``Sugar Calling.'' Next week, Alice
  Walker.
\end{itemize}

Previous

More episodes ofSugar Calling

\href{https://www.nytimes3xbfgragh.onion/2020/05/20/podcasts/sugar-calling-joy-harjo-poetry-virus.html?action=click\&module=audio-series-bar\&region=header\&pgtype=Article}{\includegraphics{https://static01.graylady3jvrrxbe.onion/images/2020/05/22/podcasts/20sugar-hajo3/20sugar-hajo3-thumbLarge.jpg}}

May 20, 2020~~•~ 35:30`I Release You, Fear'

\href{https://www.nytimes3xbfgragh.onion/2020/05/13/podcasts/sugar-calling-billy-collins-poetry-virus.html?action=click\&module=audio-series-bar\&region=header\&pgtype=Article}{\includegraphics{https://static01.graylady3jvrrxbe.onion/images/2020/05/13/podcasts/13sugar-calling/13sugar-calling-thumbLarge.jpg}}

May 13, 2020`There's a Quiet All Over the World'

\href{https://www.nytimes3xbfgragh.onion/2020/05/06/podcasts/sugar-calling-alice-walker-quarantine-virus.html?action=click\&module=audio-series-bar\&region=header\&pgtype=Article}{\includegraphics{https://static01.graylady3jvrrxbe.onion/images/2020/05/06/podcasts/06sugarcalling/06sugarcalling-thumbLarge.jpg}}

May 6, 2020~~•~ 28:58`Whatever We Have, We Have to Work With It'

\href{https://www.nytimes3xbfgragh.onion/2020/04/29/podcasts/sugar-calling-judy-blume-quarantine-virus.html?action=click\&module=audio-series-bar\&region=header\&pgtype=Article}{\includegraphics{https://static01.graylady3jvrrxbe.onion/images/2020/04/29/podcasts/29sugarcalliing-blume-sub/29sugarcalliing-blume-sub-thumbLarge.jpg}}

April 29, 2020`This Terrible Thing Is Happening, but the World Goes On.'

\href{https://www.nytimes3xbfgragh.onion/2020/04/22/podcasts/sugar-calling-amy-tan-quarantine-virus.html?action=click\&module=audio-series-bar\&region=header\&pgtype=Article}{\includegraphics{https://static01.graylady3jvrrxbe.onion/images/2020/04/27/podcasts/22sugarcalling/22sugarcalling-thumbLarge.jpg}}

April 22, 2020~~•~ 39:19`You Don't Take Dictation. You Find the Truth.'

\href{https://www.nytimes3xbfgragh.onion/2020/04/15/podcasts/sugar-calling-pico-iyer-coronavirus.html?action=click\&module=audio-series-bar\&region=header\&pgtype=Article}{\includegraphics{https://static01.graylady3jvrrxbe.onion/images/2020/04/21/podcasts/15sugarcalling1/15sugarcalling1-thumbLarge.jpg}}

April 15, 2020~~•~ 35:45`Joyful Participation in a World of Sorrows'

\href{https://www.nytimes3xbfgragh.onion/2020/04/08/podcasts/sugar-calling-margaret-atwood-coronavirus.html?action=click\&module=audio-series-bar\&region=header\&pgtype=Article}{\includegraphics{https://static01.graylady3jvrrxbe.onion/images/2020/04/02/books/08sugarcalling1/08sugarcalling1-thumbLarge-v3.jpg}}

April 8, 2020~~•~ 34:32`Roll Up Your Sleeves, Girls'

\href{https://www.nytimes3xbfgragh.onion/2020/04/03/podcasts/sugar-calling-george-saunders-coronavirus.html?action=click\&module=audio-series-bar\&region=header\&pgtype=Article}{\includegraphics{https://static01.graylady3jvrrxbe.onion/images/2020/04/09/podcasts/03sugarcalling-image/merlin_171264408_4ac7fc67-d8cc-45b9-9ec6-bdd20672e694-thumbLarge.jpg}}

April 3, 2020~~•~ 41:16`Everything Is Always Keep Changing'

\href{https://www.nytimes3xbfgragh.onion/column/sugar-calling}{See All
Episodes ofSugar Calling}

Next

April 29, 2020

\begin{itemize}
\item
\item
\item
\item
\item
\item
\end{itemize}

\emph{\textbf{Listen and subscribe to our podcast from your mobile
device:}}
\textbf{\href{https://podcasts.apple.com/us/podcast/sugar-calling/id1505881384}{\emph{Via
Apple Podcasts}}} \emph{\textbf{\textbar{}}}
\textbf{\href{https://open.spotify.com/show/4U8hPiNGIBvTS9zLeiDCN7?si=gRyigD47SPWl-QWgNjgt2w}{\emph{Via
Spotify}}} \emph{\textbf{\textbar{}}}
\textbf{\href{https://www.stitcher.com/podcast/the-new-york-times/sugar-calling}{\emph{Via
Stitcher}}}

\hypertarget{im-scared-one-minute-and-then-its-like-oh-everything-is-gonna-be-ok-lets-dance-that-will-help}{%
\subsection{`I'm scared one minute and then it's like, oh, everything is
gonna be OK. Let's dance. That will
help.'}\label{im-scared-one-minute-and-then-its-like-oh-everything-is-gonna-be-ok-lets-dance-that-will-help}}

\emph{--- Author Judy Blume}

Today, Cheryl calls Judy Blume, the beloved young-adult novelist, at her
home in the Florida Keys. Judy tells Cheryl what it's like to run a book
store at 82 years old: ``It's my reward after 50 years of writing.''

Judy talks about journaling, and how it saved her through the hardest
times in her life. She passes on advice to teenagers living through the
pandemic: Acknowledge your feelings, and write them down. You don't have
to discuss it with your parents, either.

\includegraphics{https://static01.graylady3jvrrxbe.onion/images/2020/04/29/podcasts/29sugarcalliing-blume-sub/merlin_11422500_02787acd-ba7f-46fd-aa6a-8ab01e9de130-articleLarge.jpg?quality=75\&auto=webp\&disable=upscale}

\hypertarget{on-todays-episode}{%
\subsubsection{\texorpdfstring{\textbf{On today's
episode:}}{On today's episode:}}\label{on-todays-episode}}

\href{http://www.judyblume.com/}{Judy Blume} is the author of over 30
books, many of them written for young adults. Her novels taught
generations of teenage readers about
\href{https://www.nytimes3xbfgragh.onion/2015/06/02/books/what-judy-blumes-books-meant.html}{sex,
love and friendship}. For kids growing up before the internet,
``\href{https://www.nytimes3xbfgragh.onion/2015/05/24/magazine/judy-blume-knows-all-your-secrets.html}{there
was just Judy Blume},'' Susan Dominus wrote for The Times Magazine in
2015.

Judy Blume is the founder of \href{http://booksandbookskw.com/}{Books \&
Books} bookstore in Key West, Fla. Her latest and final novel,
``\href{https://www.nytimes3xbfgragh.onion/2015/05/31/books/review/judy-blumes-in-the-unlikely-event.html}{In
the Unlikely Event},'' was published in 2015.

\hypertarget{judy-blumes-quarantine-reading-list}{%
\subsubsection{\texorpdfstring{\textbf{Judy Blume's quarantine reading
list:}}{Judy Blume's quarantine reading list:}}\label{judy-blumes-quarantine-reading-list}}

\begin{itemize}
\item
  ``\href{https://www.amazon.com/Writers-Lovers-Lily-King/dp/0802148530}{Writers
  \& Lovers},'' Lily King
\item
  ``\href{https://www.amazon.com/Plot-Against-America-Philip-Roth/dp/1400079497}{The
  Plot Against America},'' Philip Roth
\item
  "\href{https://www.amazon.com/Little-Fires-Everywhere-Celeste-Ng/dp/0735224293}{Little
  Fires Everywhere},'' Celeste Ng
\item
  ``\href{https://www.amazon.com/Oksana-Behave-Novel-Maria-Kuznetsova/dp/0525511873}{Oksana,
  Behave!}'' Maria Kuznetsova
\item
  ``\href{https://www.amazon.com/Prep-Novel-Curtis-Sittenfeld/dp/081297235X}{Prep},''
  Curtis Sittenfeld
\end{itemize}

\begin{center}\rule{0.5\linewidth}{\linethickness}\end{center}

Cheryl Strayed is the author of ``Tiny Beautiful Things,'' ``Torch,''
``Brave Enough,'' and the New York Times best seller ``Wild.'' Her books
have been translated into more than 40 languages. She lives in Portland,
Ore.
\href{https://twitter.com/CherylStrayed?ref_src=twsrc\%5Egoogle\%7Ctwcamp\%5Eserp\%7Ctwgr\%5Eauthor}{@CherylStrayed}

``Sugar Calling'' is produced by Kelly Prime and edited by Sara
Sarasohn, with editorial oversight by Wendy Dorr. This episode was mixed
by Jamie Collazo. Our theme music is by Dan Powell.

Advertisement

\protect\hyperlink{after-bottom}{Continue reading the main story}

\hypertarget{site-index}{%
\subsection{Site Index}\label{site-index}}

\hypertarget{site-information-navigation}{%
\subsection{Site Information
Navigation}\label{site-information-navigation}}

\begin{itemize}
\tightlist
\item
  \href{https://help.nytimes3xbfgragh.onion/hc/en-us/articles/115014792127-Copyright-notice}{©~2020~The
  New York Times Company}
\end{itemize}

\begin{itemize}
\tightlist
\item
  \href{https://www.nytco.com/}{NYTCo}
\item
  \href{https://help.nytimes3xbfgragh.onion/hc/en-us/articles/115015385887-Contact-Us}{Contact
  Us}
\item
  \href{https://www.nytco.com/careers/}{Work with us}
\item
  \href{https://nytmediakit.com/}{Advertise}
\item
  \href{http://www.tbrandstudio.com/}{T Brand Studio}
\item
  \href{https://www.nytimes3xbfgragh.onion/privacy/cookie-policy\#how-do-i-manage-trackers}{Your
  Ad Choices}
\item
  \href{https://www.nytimes3xbfgragh.onion/privacy}{Privacy}
\item
  \href{https://help.nytimes3xbfgragh.onion/hc/en-us/articles/115014893428-Terms-of-service}{Terms
  of Service}
\item
  \href{https://help.nytimes3xbfgragh.onion/hc/en-us/articles/115014893968-Terms-of-sale}{Terms
  of Sale}
\item
  \href{https://spiderbites.nytimes3xbfgragh.onion}{Site Map}
\item
  \href{https://help.nytimes3xbfgragh.onion/hc/en-us}{Help}
\item
  \href{https://www.nytimes3xbfgragh.onion/subscription?campaignId=37WXW}{Subscriptions}
\end{itemize}
