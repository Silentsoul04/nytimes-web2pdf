Sections

SEARCH

\protect\hyperlink{site-content}{Skip to
content}\protect\hyperlink{site-index}{Skip to site index}

\href{https://www.nytimes3xbfgragh.onion/section/travel}{Travel}

\href{https://myaccount.nytimes3xbfgragh.onion/auth/login?response_type=cookie\&client_id=vi}{}

\href{https://www.nytimes3xbfgragh.onion/section/todayspaper}{Today's
Paper}

\href{/section/travel}{Travel}\textbar{}3 Baltic States Announced a
`Travel Bubble.' What Is It and Could It Work in the U.S.?

\url{https://nyti.ms/2KN9jRP}

\begin{itemize}
\item
\item
\item
\item
\item
\item
\end{itemize}

\hypertarget{the-coronavirus-outbreak}{%
\subsubsection{\texorpdfstring{\href{https://www.nytimes3xbfgragh.onion/news-event/coronavirus?name=styln-coronavirus-national\&region=TOP_BANNER\&variant=undefined\&block=storyline_menu_recirc\&action=click\&pgtype=Article\&impression_id=d5b12900-e38e-11ea-bbe8-bf70f691015a}{The
Coronavirus
Outbreak}}{The Coronavirus Outbreak}}\label{the-coronavirus-outbreak}}

\begin{itemize}
\tightlist
\item
  live\href{https://www.nytimes3xbfgragh.onion/2020/08/20/world/coronavirus-covid.html?name=styln-coronavirus-national\&region=TOP_BANNER\&variant=undefined\&block=storyline_menu_recirc\&action=click\&pgtype=Article\&impression_id=d5b12901-e38e-11ea-bbe8-bf70f691015a}{Latest
  Updates}
\item
  \href{https://www.nytimes3xbfgragh.onion/interactive/2020/us/coronavirus-us-cases.html?name=styln-coronavirus-national\&region=TOP_BANNER\&variant=undefined\&block=storyline_menu_recirc\&action=click\&pgtype=Article\&impression_id=d5b12902-e38e-11ea-bbe8-bf70f691015a}{Maps
  and Cases}
\item
  \href{https://www.nytimes3xbfgragh.onion/interactive/2020/science/coronavirus-vaccine-tracker.html?name=styln-coronavirus-national\&region=TOP_BANNER\&variant=undefined\&block=storyline_menu_recirc\&action=click\&pgtype=Article\&impression_id=d5b12903-e38e-11ea-bbe8-bf70f691015a}{Vaccine
  Tracker}
\item
  \href{https://www.nytimes3xbfgragh.onion/2020/08/19/us/colleges-closing-covid.html?name=styln-coronavirus-national\&region=TOP_BANNER\&variant=undefined\&block=storyline_menu_recirc\&action=click\&pgtype=Article\&impression_id=d5b12904-e38e-11ea-bbe8-bf70f691015a}{Colleges
  Closing}
\item
  \href{https://www.nytimes3xbfgragh.onion/live/2020/08/20/business/stock-market-today-coronavirus?name=styln-coronavirus-national\&region=TOP_BANNER\&variant=undefined\&block=storyline_menu_recirc\&action=click\&pgtype=Article\&impression_id=d5b12905-e38e-11ea-bbe8-bf70f691015a}{Economy}
\end{itemize}

Advertisement

\protect\hyperlink{after-top}{Continue reading the main story}

Supported by

\protect\hyperlink{after-sponsor}{Continue reading the main story}

\hypertarget{3-baltic-states-announced-a-travel-bubble-what-is-it-and-could-it-work-in-the-us}{%
\section{3 Baltic States Announced a `Travel Bubble.' What Is It and
Could It Work in the
U.S.?}\label{3-baltic-states-announced-a-travel-bubble-what-is-it-and-could-it-work-in-the-us}}

Estonia, Latvia and Lithuania announced a plan to allow for internal
travel, and Australia and New Zealand are likely to follow suit. But in
other parts of the world, establishing such a bubble would be tricky.

\includegraphics{https://static01.graylady3jvrrxbe.onion/images/2020/05/29/travel/29travelbubbles-baltic/29travelbubbles-baltic-articleLarge.jpg?quality=75\&auto=webp\&disable=upscale}

\href{https://www.nytimes3xbfgragh.onion/by/tariro-mzezewa}{\includegraphics{https://static01.graylady3jvrrxbe.onion/images/2018/08/24/opinion/tariro-headshot/tariro-headshot-thumbLarge-v2.png}}

By \href{https://www.nytimes3xbfgragh.onion/by/tariro-mzezewa}{Tariro
Mzezewa}

\begin{itemize}
\item
  Published April 29, 2020Updated Aug. 6, 2020
\item
  \begin{itemize}
  \item
  \item
  \item
  \item
  \item
  \item
  \end{itemize}
\end{itemize}

Estonia, Latvia and Lithuania plan to open their internal borders for
citizens to travel between the three countries beginning on May 15, a
decision that creates the first ``travel bubble'' among European Union
countries as coronavirus-related travel restrictions begin to ease.

The announcement, made by the prime ministers of the three Baltic
nations on Wednesday, follows a statement made in late April by the
prime ministers of Australia and New Zealand, who
\href{https://www.theguardian.com/world/2020/may/05/trans-tasman-travel-bubble-to-allow-flights-as-soon-as-lockdowns-ease-morrison-and-ardern-agree}{agreed
this week to establish a similar bubble} when flights between the
countries can be safely arranged. Both Australia and New Zealand,
compared to other nations, have made significant progress in containing
the coronavirus outbreak within their borders.

People traveling to the Baltic corridor from other countries will
continue to be required to go into isolation for 14 days after entry.

``It's a big step towards life as normal,'' wrote Jüri Ratas, the prime
minister of Estonia,
\href{https://twitter.com/ratasjuri/status/1257960686489161733}{in a
Twitter message}.

Regional bubbles --- among a group of states, for example --- could be a
way to get people traveling again and help revive businesses as the
world readjusts to life with the coronavirus. Earlier in the pandemic,
the governors of California, Washington, Oregon, Nevada and Colorado
said that they will coordinate the reopening of their states to limit
the spread of Covid-19. A travel bubble among those states might make it
easier for residents to travel and work across their borders, though
spokesmen for the governors of California, Oregon and Washington said
that they have not yet discussed creating such a bubble.

\hypertarget{are-travel-bubbles-legal}{%
\subsection{Are travel bubbles legal?}\label{are-travel-bubbles-legal}}

Travel bubbles imposed by nations --- where strict immigration rules can
dictate who enters and exits --- are one thing. But would they be legal
within the United States? Maybe not.

``If any states created a bubble to keep out residents of disfavored
states, the action would be open to challenge that the bubble violated
U.S. Constitutional principles of equal treatment,'' said Cam Winton, a
lawyer with the firm Dorsey \& Whitney, who has been advising people
dealing with state-imposed coronavirus rules. ``Those principles are
enshrined in the Privileges and Immunities Clause, the Equal Protection
Clause and the so-called Dormant Commerce Clause, which prohibits states
from putting undue burdens on interstate commerce.''

Under its police power, a state can deploy nondiscriminatory public
health measures like taking the temperature of everyone entering the
state or requiring them to go into quarantine, as long as it imposes
that rule on every person, regardless of where the person is from.

Mr. Winton added that neighboring states could enter into agreements for
coordination of checkpoints, sharing data and other anti-coronavirus
efforts, but those initiatives would all be subject to the constraints
on government power conferred by constitutional rights.

\hypertarget{latest-updates-the-coronavirus-outbreak}{%
\section{\texorpdfstring{\href{https://www.nytimes3xbfgragh.onion/2020/08/20/world/coronavirus-covid.html?action=click\&pgtype=Article\&state=default\&region=MAIN_CONTENT_1\&context=storylines_live_updates}{Latest
Updates: The Coronavirus
Outbreak}}{Latest Updates: The Coronavirus Outbreak}}\label{latest-updates-the-coronavirus-outbreak}}

Updated 2020-08-21T09:06:19.820Z

\begin{itemize}
\tightlist
\item
  \href{https://www.nytimes3xbfgragh.onion/2020/08/20/world/coronavirus-covid.html?action=click\&pgtype=Article\&state=default\&region=MAIN_CONTENT_1\&context=storylines_live_updates\#link-68774d88}{Shutdowns,
  warnings and scoldings follow alarming incidents on college campuses.}
\item
  \href{https://www.nytimes3xbfgragh.onion/2020/08/20/world/coronavirus-covid.html?action=click\&pgtype=Article\&state=default\&region=MAIN_CONTENT_1\&context=storylines_live_updates\#link-26b58724}{Biden
  knocks Trump's pandemic response, and outlines a national strategy.}
\item
  \href{https://www.nytimes3xbfgragh.onion/2020/08/20/world/coronavirus-covid.html?action=click\&pgtype=Article\&state=default\&region=MAIN_CONTENT_1\&context=storylines_live_updates\#link-4e542da3}{U.S.
  health agencies announce moves to confront the flu season and
  plummeting child vaccination rates.}
\end{itemize}

\href{https://www.nytimes3xbfgragh.onion/2020/08/20/world/coronavirus-covid.html?action=click\&pgtype=Article\&state=default\&region=MAIN_CONTENT_1\&context=storylines_live_updates}{See
more updates}

More live coverage:
\href{https://www.nytimes3xbfgragh.onion/live/2020/08/20/business/stock-market-today-coronavirus?action=click\&pgtype=Article\&state=default\&region=MAIN_CONTENT_1\&context=storylines_live_updates}{Markets}

In the United States, measures created by some municipalities in recent
months have created de facto bubbles, but the porousness of these varies
from place to place and would be difficult, if not impossible, to make
official without infringing upon the rights of states and citizens,
experts said.

\includegraphics{https://static01.graylady3jvrrxbe.onion/images/2020/04/29/travel/29travelbubbles-keys/merlin_171748053_bdd2efd1-91c5-4b2c-800f-ab8fb889ec10-articleLarge.jpg?quality=75\&auto=webp\&disable=upscale}

\hypertarget{the-key-west-example}{%
\subsection{The Key West example}\label{the-key-west-example}}

In Florida, the Keys have become their own bubble. Since March 22,
Monroe County, which includes the islands, has had two checkpoints at
the top of the Florida Keys Overseas Highway --- one at mile marker
112.5 on the 18-mile stretch of U.S. 1 and the other on County Road 905.
Only people who can prove that they live or own property or work in the
Keys have been allowed onto the chain of islands that is about 60 miles
away from Miami.

When Key West, the seat of Monroe County, began reopening its beaches
and parks on Monday for locals, people from neighboring Miami-Dade
County tried to drive to the beach but were turned away at the
checkpoints.

``Historically we've had day trippers come then leave, but Miami-Dade
County is a hot spot with a lion's share of cases, so what we're trying
to do is only allow essential services into the Keys and say that the
beach is open only for locals to get out a few hours a day to get sun,''
said Teri Johnston, the mayor of Key West. ``Our residents want to know
that we aren't relaxing orders in a way that would bring visitors to our
island until it is safe for that.''

Officials in Key West have been in conversation with business owners,
realtors and the administrators of attractions like museums about a
phased reopening of the city, in which everything would reopen to locals
first, said Alyson Crean, a public information officer.

``Everyone agrees that opening up to visitors is the last phase,'' Ms.
Crean said.

Image

In Colorado, Estes Park, the town that serves as a gateway to Rocky
Mountain National Park, is allowing hotels to reopen.~Credit...Christina
Kiffney for The New York Times

\hypertarget{where-does-one-bubble-end-and-the-other-begin}{%
\subsection{Where does one bubble end and the other
begin?}\label{where-does-one-bubble-end-and-the-other-begin}}

For places without the natural geography to help create a bubble, things
get murkier. Under Colorado's new ``Safer at Home'' policy, people have
been ordered by Gov. Jared Polis to limit recreational travel to no more
than 10 miles from their homes, and to try to keep essential travel
within their county, if possible. Theoretically, those orders would
create bubbles that would prevent visitors from traveling to the state's
more isolated areas.

\href{https://www.nytimes3xbfgragh.onion/news-event/coronavirus?action=click\&pgtype=Article\&state=default\&region=MAIN_CONTENT_3\&context=storylines_faq}{}

\hypertarget{the-coronavirus-outbreak-}{%
\subsubsection{The Coronavirus Outbreak
›}\label{the-coronavirus-outbreak-}}

\hypertarget{frequently-asked-questions}{%
\paragraph{Frequently Asked
Questions}\label{frequently-asked-questions}}

Updated August 17, 2020

\begin{itemize}
\item ~
  \hypertarget{why-does-standing-six-feet-away-from-others-help}{%
  \paragraph{Why does standing six feet away from others
  help?}\label{why-does-standing-six-feet-away-from-others-help}}

  \begin{itemize}
  \tightlist
  \item
    The coronavirus spreads primarily through droplets from your mouth
    and nose, especially when you cough or sneeze. The C.D.C., one of
    the organizations using that measure,
    \href{https://www.nytimes3xbfgragh.onion/2020/04/14/health/coronavirus-six-feet.html?action=click\&pgtype=Article\&state=default\&region=MAIN_CONTENT_3\&context=storylines_faq}{bases
    its recommendation of six feet} on the idea that most large droplets
    that people expel when they cough or sneeze will fall to the ground
    within six feet. But six feet has never been a magic number that
    guarantees complete protection. Sneezes, for instance, can launch
    droplets a lot farther than six feet,
    \href{https://jamanetwork.com/journals/jama/fullarticle/2763852}{according
    to a recent study}. It's a rule of thumb: You should be safest
    standing six feet apart outside, especially when it's windy. But
    keep a mask on at all times, even when you think you're far enough
    apart.
  \end{itemize}
\item ~
  \hypertarget{i-have-antibodies-am-i-now-immune}{%
  \paragraph{I have antibodies. Am I now
  immune?}\label{i-have-antibodies-am-i-now-immune}}

  \begin{itemize}
  \tightlist
  \item
    As of right
    now,\href{https://www.nytimes3xbfgragh.onion/2020/07/22/health/covid-antibodies-herd-immunity.html?action=click\&pgtype=Article\&state=default\&region=MAIN_CONTENT_3\&context=storylines_faq}{that
    seems likely, for at least several months.} There have been
    frightening accounts of people suffering what seems to be a second
    bout of Covid-19. But experts say these patients may have a
    drawn-out course of infection, with the virus taking a slow toll
    weeks to months after initial exposure. People infected with the
    coronavirus typically
    \href{https://www.nature.com/articles/s41586-020-2456-9}{produce}
    immune molecules called antibodies, which are
    \href{https://www.nytimes3xbfgragh.onion/2020/05/07/health/coronavirus-antibody-prevalence.html?action=click\&pgtype=Article\&state=default\&region=MAIN_CONTENT_3\&context=storylines_faq}{protective
    proteins made in response to an
    infection}\href{https://www.nytimes3xbfgragh.onion/2020/05/07/health/coronavirus-antibody-prevalence.html?action=click\&pgtype=Article\&state=default\&region=MAIN_CONTENT_3\&context=storylines_faq}{.
    These antibodies may} last in the body
    \href{https://www.nature.com/articles/s41591-020-0965-6}{only two to
    three months}, which may seem worrisome, but that's perfectly normal
    after an acute infection subsides, said Dr. Michael Mina, an
    immunologist at Harvard University. It may be possible to get the
    coronavirus again, but it's highly unlikely that it would be
    possible in a short window of time from initial infection or make
    people sicker the second time.
  \end{itemize}
\item ~
  \hypertarget{im-a-small-business-owner-can-i-get-relief}{%
  \paragraph{I'm a small-business owner. Can I get
  relief?}\label{im-a-small-business-owner-can-i-get-relief}}

  \begin{itemize}
  \tightlist
  \item
    The
    \href{https://www.nytimes3xbfgragh.onion/article/small-business-loans-stimulus-grants-freelancers-coronavirus.html?action=click\&pgtype=Article\&state=default\&region=MAIN_CONTENT_3\&context=storylines_faq}{stimulus
    bills enacted in March} offer help for the millions of American
    small businesses. Those eligible for aid are businesses and
    nonprofit organizations with fewer than 500 workers, including sole
    proprietorships, independent contractors and freelancers. Some
    larger companies in some industries are also eligible. The help
    being offered, which is being managed by the Small Business
    Administration, includes the Paycheck Protection Program and the
    Economic Injury Disaster Loan program. But lots of folks have
    \href{https://www.nytimes3xbfgragh.onion/interactive/2020/05/07/business/small-business-loans-coronavirus.html?action=click\&pgtype=Article\&state=default\&region=MAIN_CONTENT_3\&context=storylines_faq}{not
    yet seen payouts.} Even those who have received help are confused:
    The rules are draconian, and some are stuck sitting on
    \href{https://www.nytimes3xbfgragh.onion/2020/05/02/business/economy/loans-coronavirus-small-business.html?action=click\&pgtype=Article\&state=default\&region=MAIN_CONTENT_3\&context=storylines_faq}{money
    they don't know how to use.} Many small-business owners are getting
    less than they expected or
    \href{https://www.nytimes3xbfgragh.onion/2020/06/10/business/Small-business-loans-ppp.html?action=click\&pgtype=Article\&state=default\&region=MAIN_CONTENT_3\&context=storylines_faq}{not
    hearing anything at all.}
  \end{itemize}
\item ~
  \hypertarget{what-are-my-rights-if-i-am-worried-about-going-back-to-work}{%
  \paragraph{What are my rights if I am worried about going back to
  work?}\label{what-are-my-rights-if-i-am-worried-about-going-back-to-work}}

  \begin{itemize}
  \tightlist
  \item
    Employers have to provide
    \href{https://www.osha.gov/SLTC/covid-19/standards.html}{a safe
    workplace} with policies that protect everyone equally.
    \href{https://www.nytimes3xbfgragh.onion/article/coronavirus-money-unemployment.html?action=click\&pgtype=Article\&state=default\&region=MAIN_CONTENT_3\&context=storylines_faq}{And
    if one of your co-workers tests positive for the coronavirus, the
    C.D.C.} has said that
    \href{https://www.cdc.gov/coronavirus/2019-ncov/community/guidance-business-response.html}{employers
    should tell their employees} -\/- without giving you the sick
    employee's name -\/- that they may have been exposed to the virus.
  \end{itemize}
\item ~
  \hypertarget{what-is-school-going-to-look-like-in-september}{%
  \paragraph{What is school going to look like in
  September?}\label{what-is-school-going-to-look-like-in-september}}

  \begin{itemize}
  \tightlist
  \item
    It is unlikely that many schools will return to a normal schedule
    this fall, requiring the grind of
    \href{https://www.nytimes3xbfgragh.onion/2020/06/05/us/coronavirus-education-lost-learning.html?action=click\&pgtype=Article\&state=default\&region=MAIN_CONTENT_3\&context=storylines_faq}{online
    learning},
    \href{https://www.nytimes3xbfgragh.onion/2020/05/29/us/coronavirus-child-care-centers.html?action=click\&pgtype=Article\&state=default\&region=MAIN_CONTENT_3\&context=storylines_faq}{makeshift
    child care} and
    \href{https://www.nytimes3xbfgragh.onion/2020/06/03/business/economy/coronavirus-working-women.html?action=click\&pgtype=Article\&state=default\&region=MAIN_CONTENT_3\&context=storylines_faq}{stunted
    workdays} to continue. California's two largest public school
    districts --- Los Angeles and San Diego --- said on July 13, that
    \href{https://www.nytimes3xbfgragh.onion/2020/07/13/us/lausd-san-diego-school-reopening.html?action=click\&pgtype=Article\&state=default\&region=MAIN_CONTENT_3\&context=storylines_faq}{instruction
    will be remote-only in the fall}, citing concerns that surging
    coronavirus infections in their areas pose too dire a risk for
    students and teachers. Together, the two districts enroll some
    825,000 students. They are the largest in the country so far to
    abandon plans for even a partial physical return to classrooms when
    they reopen in August. For other districts, the solution won't be an
    all-or-nothing approach.
    \href{https://bioethics.jhu.edu/research-and-outreach/projects/eschool-initiative/school-policy-tracker/}{Many
    systems}, including the nation's largest, New York City, are
    devising
    \href{https://www.nytimes3xbfgragh.onion/2020/06/26/us/coronavirus-schools-reopen-fall.html?action=click\&pgtype=Article\&state=default\&region=MAIN_CONTENT_3\&context=storylines_faq}{hybrid
    plans} that involve spending some days in classrooms and other days
    online. There's no national policy on this yet, so check with your
    municipal school system regularly to see what is happening in your
    community.
  \end{itemize}
\end{itemize}

However, Estes Park, a tourist town in Larimer County that is the
gateway to Rocky Mountain National Park and 30 miles from the nearest
city, recently began allowing its hotels and short-term rentals to
reopen with social distancing policies in place.

Kate Rusch, a spokeswoman for the Town of Estes Park, said that
essential travelers and people traveling within Larimer County might
need accommodations and that the town wanted to be ready for when travel
restrictions were loosened further.

But some saw the move as an open invitation to visit. Hattie Serrano, a
local resident,
\href{https://twitter.com/themadhattiee/status/1253779577400430594}{wrote
in a Tweet} that hundreds of people had been visiting even during the
shelter-in-place order. ``Telling people to stay home and then opening
hotels at the same time is contradictory to one another. If the hotels
are open they will come. They've come despite them being closed.
Reopening them is a terrible mistake.''

Others, like Stephen Rapp, objected to staying within the 10-mile
bubble.
\href{https://twitter.com/StephenRapp3/status/1253786973002702849}{He
tweeted} that it would be ``unconstitutional'' to prevent him from
staying in his second home in Estes where he normally spends six months
of the year and pays taxes. He asked: ``I'm an Estes Park property
owner, so am I an outsider or not a local?''

\hypertarget{even-islands-can-struggle-with-bubbles}{%
\subsection{Even islands can struggle with
bubbles}\label{even-islands-can-struggle-with-bubbles}}

In South Carolina, the city of Folly Beach, located on Folly Island,
just south of Charleston, found itself in conflict with the state's
governor when it set up checkpoints on the road leading to the island
and banned short-term rentals in March. ``By creating the checkpoint, we
had essentially created a bubble for our island,'' said the town's
mayor, Tim Goodwin.

But under an order from the state's attorney general, officials were
forced to lift the restrictions. Mr. Goodwin said people immediately
began flooding in. ``The day that we did lift the checkpoint we went
from an average of 2,000 cars to an average of 10,000 cars,'' he said.

On Peaks Island, an island in Casco Bay, Maine, that is 15 minutes away
from downtown Portland and is only accessible by ferry, residents
\href{https://www.wmtw.com/article/islanders-ask-people-only-make-essential-trips-for-now/32010258}{recently
asked city leaders} to encourage people to only take essential ferry
trips to the island out of fear that the guests could increase the
number of cases already recorded. Some locals worry that people will
flood back to the island over Memorial Day weekend.

``It's a very conflicted situation with community needing the business
that comes with visitors, but also needing to prioritize everyone's
health,'' said Randy Schaeffer, chair of the Peaks Island Council, which
is not a legislative body but an advisory group for the island. ``People
might want something like a bubble, but we don't have the legal power to
do that and it is highly unlikely that we would be able to keep people
from coming here.''

``We aren't North Korea,'' he added.

\begin{center}\rule{0.5\linewidth}{\linethickness}\end{center}

Karen Schwartz contributed reporting.

\emph{\textbf{Follow New York Times Travel}}
\emph{on}\href{https://www.instagram.com/nytimestravel/}{\emph{Instagram}}\emph{,}\href{https://twitter.com/nytimestravel}{\emph{Twitter}}
\emph{and}\href{https://www.facebookcorewwwi.onion/nytimestravel/}{\emph{Facebook}}\emph{.
And}\href{https://www.nytimes3xbfgragh.onion/newsletters/traveldispatch}{\emph{sign
up for our weekly Travel Dispatch newsletter}} \emph{to receive expert
tips on traveling smarter and inspiration for your next vacation.}

Advertisement

\protect\hyperlink{after-bottom}{Continue reading the main story}

\hypertarget{site-index}{%
\subsection{Site Index}\label{site-index}}

\hypertarget{site-information-navigation}{%
\subsection{Site Information
Navigation}\label{site-information-navigation}}

\begin{itemize}
\tightlist
\item
  \href{https://help.nytimes3xbfgragh.onion/hc/en-us/articles/115014792127-Copyright-notice}{©~2020~The
  New York Times Company}
\end{itemize}

\begin{itemize}
\tightlist
\item
  \href{https://www.nytco.com/}{NYTCo}
\item
  \href{https://help.nytimes3xbfgragh.onion/hc/en-us/articles/115015385887-Contact-Us}{Contact
  Us}
\item
  \href{https://www.nytco.com/careers/}{Work with us}
\item
  \href{https://nytmediakit.com/}{Advertise}
\item
  \href{http://www.tbrandstudio.com/}{T Brand Studio}
\item
  \href{https://www.nytimes3xbfgragh.onion/privacy/cookie-policy\#how-do-i-manage-trackers}{Your
  Ad Choices}
\item
  \href{https://www.nytimes3xbfgragh.onion/privacy}{Privacy}
\item
  \href{https://help.nytimes3xbfgragh.onion/hc/en-us/articles/115014893428-Terms-of-service}{Terms
  of Service}
\item
  \href{https://help.nytimes3xbfgragh.onion/hc/en-us/articles/115014893968-Terms-of-sale}{Terms
  of Sale}
\item
  \href{https://spiderbites.nytimes3xbfgragh.onion}{Site Map}
\item
  \href{https://help.nytimes3xbfgragh.onion/hc/en-us}{Help}
\item
  \href{https://www.nytimes3xbfgragh.onion/subscription?campaignId=37WXW}{Subscriptions}
\end{itemize}
