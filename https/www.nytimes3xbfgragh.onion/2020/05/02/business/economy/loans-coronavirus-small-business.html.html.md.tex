Sections

SEARCH

\protect\hyperlink{site-content}{Skip to
content}\protect\hyperlink{site-index}{Skip to site index}

\href{https://www.nytimes3xbfgragh.onion/section/business/economy}{Economy}

\href{https://myaccount.nytimes3xbfgragh.onion/auth/login?response_type=cookie\&client_id=vi}{}

\href{https://www.nytimes3xbfgragh.onion/section/todayspaper}{Today's
Paper}

\href{/section/business/economy}{Economy}\textbar{}Some Small Businesses
That Got Aid Fear the Rules Too Much to Spend It

\url{https://nyti.ms/3d9WPQs}

\begin{itemize}
\item
\item
\item
\item
\item
\item
\end{itemize}

\hypertarget{the-coronavirus-outbreak}{%
\subsubsection{\texorpdfstring{\href{https://www.nytimes3xbfgragh.onion/news-event/coronavirus?name=styln-coronavirus-markets\&region=TOP_BANNER\&variant=undefined\&block=storyline_menu_recirc\&action=click\&pgtype=Article\&impression_id=3c771800-e107-11ea-8db3-f56620f69f1a}{The
Coronavirus
Outbreak}}{The Coronavirus Outbreak}}\label{the-coronavirus-outbreak}}

\begin{itemize}
\tightlist
\item
  live\href{https://www.nytimes3xbfgragh.onion/2020/08/17/world/coronavirus-covid.html?name=styln-coronavirus-markets\&region=TOP_BANNER\&variant=undefined\&block=storyline_menu_recirc\&action=click\&pgtype=Article\&impression_id=3c771801-e107-11ea-8db3-f56620f69f1a}{Latest
  Updates}
\item
  \href{https://www.nytimes3xbfgragh.onion/interactive/2020/us/coronavirus-us-cases.html?name=styln-coronavirus-markets\&region=TOP_BANNER\&variant=undefined\&block=storyline_menu_recirc\&action=click\&pgtype=Article\&impression_id=3c771802-e107-11ea-8db3-f56620f69f1a}{Maps
  and Cases}
\item
  \href{https://www.nytimes3xbfgragh.onion/interactive/2020/science/coronavirus-vaccine-tracker.html?name=styln-coronavirus-markets\&region=TOP_BANNER\&variant=undefined\&block=storyline_menu_recirc\&action=click\&pgtype=Article\&impression_id=3c771803-e107-11ea-8db3-f56620f69f1a}{Vaccine
  Tracker}
\item
  \href{https://www.nytimes3xbfgragh.onion/2020/08/17/us/k-12-schools-reopening.html?name=styln-coronavirus-markets\&region=TOP_BANNER\&variant=undefined\&block=storyline_menu_recirc\&action=click\&pgtype=Article\&impression_id=3c771804-e107-11ea-8db3-f56620f69f1a}{State
  of Play for K-12}
\item
  \href{https://www.nytimes3xbfgragh.onion/live/2020/08/17/business/stock-market-today-coronavirus?name=styln-coronavirus-markets\&region=TOP_BANNER\&variant=undefined\&block=storyline_menu_recirc\&action=click\&pgtype=Article\&impression_id=3c771805-e107-11ea-8db3-f56620f69f1a}{Markets
  \& Economy}
\end{itemize}

Advertisement

\protect\hyperlink{after-top}{Continue reading the main story}

Supported by

\protect\hyperlink{after-sponsor}{Continue reading the main story}

\hypertarget{some-small-businesses-that-got-aid-fear-the-rules-too-much-to-spend-it}{%
\section{Some Small Businesses That Got Aid Fear the Rules Too Much to
Spend
It}\label{some-small-businesses-that-got-aid-fear-the-rules-too-much-to-spend-it}}

Requirements for using federal coronavirus loans are complicated and
confusing for owners. ``It's chaos,'' one lawyer said.

\includegraphics{https://static01.graylady3jvrrxbe.onion/images/2020/05/01/business/oakImage-1588372598096/oakImage-1588372598096-articleLarge.jpg?quality=75\&auto=webp\&disable=upscale}

\href{https://www.nytimes3xbfgragh.onion/by/stacy-cowley}{\includegraphics{https://static01.graylady3jvrrxbe.onion/images/2018/10/03/multimedia/author-stacy-cowley/author-stacy-cowley-thumbLarge.png}}\href{https://www.nytimes3xbfgragh.onion/by/emily-flitter}{\includegraphics{https://static01.graylady3jvrrxbe.onion/images/2019/06/19/reader-center/author-emily-flitter/author-emily-flitter-thumbLarge.png}}\href{https://www.nytimes3xbfgragh.onion/by/david-enrich}{\includegraphics{https://static01.graylady3jvrrxbe.onion/images/2020/05/12/reader-center/author-david-enrich/author-david-enrich-thumbLarge.png}}

By \href{https://www.nytimes3xbfgragh.onion/by/stacy-cowley}{Stacy
Cowley},
\href{https://www.nytimes3xbfgragh.onion/by/emily-flitter}{Emily
Flitter} and
\href{https://www.nytimes3xbfgragh.onion/by/david-enrich}{David Enrich}

\begin{itemize}
\item
  Published May 2, 2020Updated June 30, 2020
\item
  \begin{itemize}
  \item
  \item
  \item
  \item
  \item
  \item
  \end{itemize}
\end{itemize}

When a \$192,000
\href{https://www.nytimes3xbfgragh.onion/2020/05/13/business/paycheck-protection-program-small-business.html}{loan}from
the federal government's
\href{https://www.nytimes3xbfgragh.onion/2020/05/13/business/paycheck-protection-program-small-business.html}{small-business
aid program} arrived in his bank account last month, George Evageliou,
the founder of a custom woodworking company, felt like one of the lucky
ones.

Under the program's rules, Mr. Evageliou has eight weeks from the day he
received the cash to spend it. But nearly three weeks after the clock
started on April 14, he hasn't used a penny.

His quandary? If Mr. Evageliou wants his loan to be forgiven, he must
spend three-quarters of it paying the 16 workers he laid off from
\href{http://www.urbanhomecraft.com/about}{Urban Homecraft}, his
Brooklyn business, in late March. But bringing his workers back now,
when they can't work in their fabrication shop or install woodwork in
clients' homes, won't help his business. And if New York City remains
shut when his eight weeks are up in mid-June, Mr. Evageliou would have
to lay off his employees again --- something he wants to spare them.

The government has ``made this so hard to use,'' he said. ``It starts to
feel like a lose-lose situation.''

The \$660 billion
\href{https://www.nytimes3xbfgragh.onion/2020/06/30/us/politics/ppp-extension.html}{Paycheck
Protection Program} was meant to extend a lifeline to
\href{https://www.nytimes3xbfgragh.onion/2020/07/13/business/small-businesses-coronavirus.html}{small
businesses battered by the pandemic}, allowing them to keep employees on
the payroll. But it has been dogged by problems. Countless small
businesses couldn't get money, and hundreds of millions of dollars
instead flowed
\href{https://www.nytimes3xbfgragh.onion/2020/04/26/business/coronavirus-small-business-loans-large-companies.html}{to
publicly traded companies}.

Now many of the small businesses that did get loans are sitting on the
money, unsure about whether and how to spend it. That's compromising the
effectiveness of a program meant to help stabilize the country's reeling
economy.

Some owners don't see the point of hiring back workers when business is
so slow. Others chafe at having to use the money within eight weeks,
when they would like to keep the financial cushion for longer. And many
of the owners are confused about whether they have any flexibility. They
would rather use the cash to retool their operations for an altered
world or buy protective equipment for workers, but the rules require
them to spend it on specific expenses, like payroll.

Owners also say they are afraid of running afoul of the program's rules,
which are
\href{https://www.rklcpa.com/ppp-loan-forgiveness-known-unknown-action-items/}{complicated,
ambiguous} and still evolving. Accountants, lawyers and lenders are
struggling to understand the nuances and offering clients tentative
guidance.

``It's chaos,'' said Howard M. Berkower, a New York
\href{https://www.mccarter.com/people/howard-m-berkower/}{lawyer} who
advises corporate clients. ``It's impossible for businesses to have any
degree of comfort that they're following the rules when the rules are
still being written.''

The \$2 trillion CARES Act, which created the program, specifies that
small businesses --- generally those with fewer than 500 employees ---
can use the loan money to pay employees, but also for rent, utilities or
interest payments. The loans will be forgiven if they are spent on those
expenses within eight weeks and the business keeps paying the same
number of employees, at the same rate, as it did before the pandemic.

\hypertarget{latest-updates-the-coronavirus-outbreak-and-the-economy}{%
\section{\texorpdfstring{\href{https://www.nytimes3xbfgragh.onion/live/2020/08/17/business/stock-market-today-coronavirus?action=click\&pgtype=Article\&state=default\&region=MAIN_CONTENT_1\&context=storylines_live_updates}{Latest
Updates: The Coronavirus Outbreak and the
Economy}}{Latest Updates: The Coronavirus Outbreak and the Economy}}\label{latest-updates-the-coronavirus-outbreak-and-the-economy}}

\href{https://www.nytimes3xbfgragh.onion/live/2020/08/17/business/stock-market-today-coronavirus?action=click\&pgtype=Article\&state=default\&region=MAIN_CONTENT_1\&context=storylines_live_updates\#robinhood-a-stock-trading-app-is-valued-at-11-2-billion-in-its-latest-funding-round}{8h
ago}

\href{https://www.nytimes3xbfgragh.onion/live/2020/08/17/business/stock-market-today-coronavirus?action=click\&pgtype=Article\&state=default\&region=MAIN_CONTENT_1\&context=storylines_live_updates\#robinhood-a-stock-trading-app-is-valued-at-11-2-billion-in-its-latest-funding-round}{Robinhood,
a stock trading app, is valued at \$11.2 billion in its latest funding
round.}

\href{https://www.nytimes3xbfgragh.onion/live/2020/08/17/business/stock-market-today-coronavirus?action=click\&pgtype=Article\&state=default\&region=MAIN_CONTENT_1\&context=storylines_live_updates\#new-unemployment-benefits-authorized-by-president-trump-wont-come-until-late-august}{11h
ago}

\href{https://www.nytimes3xbfgragh.onion/live/2020/08/17/business/stock-market-today-coronavirus?action=click\&pgtype=Article\&state=default\&region=MAIN_CONTENT_1\&context=storylines_live_updates\#new-unemployment-benefits-authorized-by-president-trump-wont-come-until-late-august}{New
unemployment benefits authorized by President Trump won't come until
late August.}

\href{https://www.nytimes3xbfgragh.onion/live/2020/08/17/business/stock-market-today-coronavirus?action=click\&pgtype=Article\&state=default\&region=MAIN_CONTENT_1\&context=storylines_live_updates\#heres-what-you-need-to-know-for-the-week-ahead}{16h
ago}

\href{https://www.nytimes3xbfgragh.onion/live/2020/08/17/business/stock-market-today-coronavirus?action=click\&pgtype=Article\&state=default\&region=MAIN_CONTENT_1\&context=storylines_live_updates\#heres-what-you-need-to-know-for-the-week-ahead}{Here's
what you need to know for the week ahead.}

\href{https://www.nytimes3xbfgragh.onion/live/2020/08/17/business/stock-market-today-coronavirus?action=click\&pgtype=Article\&state=default\&region=MAIN_CONTENT_1\&context=storylines_live_updates}{See
more updates}

More live coverage:
\href{https://www.nytimes3xbfgragh.onion/2020/08/17/world/coronavirus-covid.html?action=click\&pgtype=Article\&state=default\&region=MAIN_CONTENT_1\&context=storylines_live_updates}{Global}

The Treasury Department and the Small Business Administration, which is
running the program, added a restriction: For a loan to be forgivable,
businesses have to spend at least 75 percent of it on payroll.
Otherwise, the rules say, the borrower will pay interest of 1 percent on
any portion of the loan that is not forgiven.

But what's unclear is what happens if borrowers keep all the money as a
loan to be used later or if they must spend the entire sum within eight
weeks, with an economic turnaround still months away.

Take Jodi Burns, the owner of
\href{https://www.blazingfreshdonuts.com/}{Blazing Fresh Donuts} in
Guilford, Conn. Ms. Burns could use the loan she got --- an amount under
\$50,000 --- to hire back her eight employees, but she would be paying
most of them to stay home, since the bakery is open only 12 hours a week
these days. She would prefer to hold on to the cash beyond eight weeks;
her hope is that it becomes a low-interest loan she can use for payroll
and rent when her shop is open longer.

Ms. Burns doesn't know whether she can do that. She has called her local
S.B.A. office, small-business advisory organizations, a law firm and her
lender to ask for guidance, but no one has given her any assurances.
Moreover, having
\href{https://home.treasury.gov/system/files/136/PPP-Borrower-Application-Form-Fillable.pdf}{signed
documents} requiring her to use the funds for purposes allowed under the
paycheck program's rules, Ms. Burns is nervous about misusing them.

``I don't accidentally want to commit bank fraud,'' she said.

\includegraphics{https://static01.graylady3jvrrxbe.onion/images/2020/05/03/business/03virus-sba3/merlin_172084545_9159176c-0beb-4eb4-ab2f-cf6b6817fe19-articleLarge.jpg?quality=75\&auto=webp\&disable=upscale}

Many lawyers are telling small-business owners that they think the loans
can be used broadly, although no one is certain. Some bankers are
reasoning that since the aid program is based on existing S.B.A.
programs that are more flexible, the pandemic loans will be, too.

``As long as they're using the funding for the operating expenses of the
business, our interpretation --- and we think it's clear --- is yes, you
can use it as effectively a working capital loan,'' said John Asbury,
the chief executive of Atlantic Union Bankshares, a community lender in
Richmond, Va.

But officials at Treasury and the S.B.A. won't confirm that
interpretation. Asked repeatedly if companies can simply hold on to the
money for now because paying employees doesn't make sense to them, an
S.B.A. spokeswoman would say only that the funds must be used for
purposes ``consistent with the Paycheck Protection Program.''

Ryan Hurst, a partner at RKL, an accounting and advisory firm, said the
program had been put together hastily and remained murky on critical
issues. ``Every day I'm sitting at my computer, hitting refresh multiple
times a day, hoping we'll get more guidance from Treasury and the
S.B.A.,'' he said.

Since the S.B.A. has not provided lenders with customized application
forms, many banks are using a generic document with provisions that do
not apply to the paycheck program.

Dutchess Maye, the owner of
\href{https://www.educonsultingfirm.com/}{eduConsulting Firm}, an
educational services provider in Raleigh, N.C., received a contract from
her bank that made no mention of having her \$20,000 loan forgiven.

Ms. Maye, who plans to use the money for payroll, balked at signing a
legal document that didn't seem to describe the forgivable loan she
thought she was getting. Her business has no debt, and the idea of
incurring any --- especially
\href{https://www.nytimes3xbfgragh.onion/2020/04/29/business/economy/us-gdp.html}{as
the economy is nose-diving} --- spooked her.

``I felt it was predatory,'' she said.

She called her lender, which assured her that the loan would be eligible
for forgiveness, but the representative she spoke with told her that the
bank had no idea yet what the process would be. In the end, reluctant to
risk missing out on badly needed aid, she signed. But Ms. Maye plans to
set \$20,000 from her savings aside for a few months as a reserve.

``I had to have a backup plan in order to take the money, in case I have
to pay it back,'' she said.

Coyote Ugly, an international chain of honky-tonk bars made famous by
the 2000 movie of the same name, is sitting on its loan money. The
company's American bars have been closed since mid-March. Bartenders and
security staff were laid off immediately, but the bars' managers were
kept on.

Through a small Louisiana bank, nine of the company's bars in the United
States applied for loans ``because they were there,'' said Jeff Wiseman,
Coyote Ugly's general counsel. At the time, executives figured the
economy might reopen before the loans came due, in which case the money
could be used for payroll and overhead like rent.

The bars' loan applications --- ranging from \$40,000 to \$120,000 ---
were approved in mid-April. By then it had become clear that Coyote Ugly
would not be serving customers for a long time. Some locations might
never reopen.

On April 18, Liliana Lovell, the company's founder and chief executive,
told managers that most of them were being furloughed. Some were furious
to be let go just as the company was granted the federal loans.

Ms. Lovell and Mr. Wiseman acknowledged those grievances, but said
Coyote Ugly hadn't had much choice. They didn't see the point in paying
managers to sit around in empty bars, and in any case the funds would be
exhausted within a couple of pay cycles. Their understanding was that if
Coyote Ugly used most of the money for purposes other than payroll, like
buying personal protective gear or cleaning supplies, the company would
have to repay the loans with interest, further weakening its precarious
finances.

And so the hundreds of thousands of dollars remain deposited in Coyote
Ugly's bank accounts, unused.

``It's important for us to sit and wait,'' Ms. Lovell wrote in an email
on Thursday to the laid-off managers.

Even borrowers who are happy with their aid see it as a temporary fix.

Erik Anderson is a co-owner of a string of high-end hair salons for men,
\href{https://www.scissorsscotch.com/\#scissors-scotch}{Scissors and
Scotch}, which has locations in several Midwestern cities. He and his
partners, along with their franchisees, all got relief money and used it
to pay employees, rent and utilities at their stores while they remained
shuttered.

Now, some of the states where Scissors and Scotch has locations are
slowly reopening. But fewer stylists can work in the salons at once, and
fewer customers will be allowed in. Everyone has to wear a mask. The
salons' aid money will help supplement their stylists' earnings, since
few, if any, of them will be able to work full 35-hour weeks.

Mr. Anderson's understanding is that he is not allowed to use money from
the small-business program for work like reconfiguring his spaces, he
said. He hopes more aid will be coming if he needs it --- or his company
may not survive.

When the loans run out, Mr. Anderson asked, ``what are we supposed to do
then?''

Advertisement

\protect\hyperlink{after-bottom}{Continue reading the main story}

\hypertarget{site-index}{%
\subsection{Site Index}\label{site-index}}

\hypertarget{site-information-navigation}{%
\subsection{Site Information
Navigation}\label{site-information-navigation}}

\begin{itemize}
\tightlist
\item
  \href{https://help.nytimes3xbfgragh.onion/hc/en-us/articles/115014792127-Copyright-notice}{©~2020~The
  New York Times Company}
\end{itemize}

\begin{itemize}
\tightlist
\item
  \href{https://www.nytco.com/}{NYTCo}
\item
  \href{https://help.nytimes3xbfgragh.onion/hc/en-us/articles/115015385887-Contact-Us}{Contact
  Us}
\item
  \href{https://www.nytco.com/careers/}{Work with us}
\item
  \href{https://nytmediakit.com/}{Advertise}
\item
  \href{http://www.tbrandstudio.com/}{T Brand Studio}
\item
  \href{https://www.nytimes3xbfgragh.onion/privacy/cookie-policy\#how-do-i-manage-trackers}{Your
  Ad Choices}
\item
  \href{https://www.nytimes3xbfgragh.onion/privacy}{Privacy}
\item
  \href{https://help.nytimes3xbfgragh.onion/hc/en-us/articles/115014893428-Terms-of-service}{Terms
  of Service}
\item
  \href{https://help.nytimes3xbfgragh.onion/hc/en-us/articles/115014893968-Terms-of-sale}{Terms
  of Sale}
\item
  \href{https://spiderbites.nytimes3xbfgragh.onion}{Site Map}
\item
  \href{https://help.nytimes3xbfgragh.onion/hc/en-us}{Help}
\item
  \href{https://www.nytimes3xbfgragh.onion/subscription?campaignId=37WXW}{Subscriptions}
\end{itemize}
