Sections

SEARCH

\protect\hyperlink{site-content}{Skip to
content}\protect\hyperlink{site-index}{Skip to site index}

\href{https://www.nytimes3xbfgragh.onion/section/business}{Business}

\href{https://myaccount.nytimes3xbfgragh.onion/auth/login?response_type=cookie\&client_id=vi}{}

\href{https://www.nytimes3xbfgragh.onion/section/todayspaper}{Today's
Paper}

\href{/section/business}{Business}\textbar{}J. Crew Files for Bankruptcy
in Virus's First Big Retail Casualty

\url{https://nyti.ms/3fg0ePr}

\begin{itemize}
\item
\item
\item
\item
\item
\item
\end{itemize}

\href{https://www.nytimes3xbfgragh.onion/news-event/coronavirus?action=click\&pgtype=Article\&state=default\&region=TOP_BANNER\&context=storylines_menu}{The
Coronavirus Outbreak}

\begin{itemize}
\tightlist
\item
  live\href{https://www.nytimes3xbfgragh.onion/2020/08/03/world/coronavirus-covid-19.html?action=click\&pgtype=Article\&state=default\&region=TOP_BANNER\&context=storylines_menu}{Latest
  Updates}
\item
  \href{https://www.nytimes3xbfgragh.onion/interactive/2020/us/coronavirus-us-cases.html?action=click\&pgtype=Article\&state=default\&region=TOP_BANNER\&context=storylines_menu}{Maps
  and Cases}
\item
  \href{https://www.nytimes3xbfgragh.onion/interactive/2020/science/coronavirus-vaccine-tracker.html?action=click\&pgtype=Article\&state=default\&region=TOP_BANNER\&context=storylines_menu}{Vaccine
  Tracker}
\item
  \href{https://www.nytimes3xbfgragh.onion/2020/08/02/us/covid-college-reopening.html?action=click\&pgtype=Article\&state=default\&region=TOP_BANNER\&context=storylines_menu}{College
  Reopening}
\item
  \href{https://www.nytimes3xbfgragh.onion/live/2020/08/03/business/stock-market-today-coronavirus?action=click\&pgtype=Article\&state=default\&region=TOP_BANNER\&context=storylines_menu}{Economy}
\end{itemize}

Advertisement

\protect\hyperlink{after-top}{Continue reading the main story}

Supported by

\protect\hyperlink{after-sponsor}{Continue reading the main story}

\hypertarget{j-crew-files-for-bankruptcy-in-viruss-first-big-retail-casualty}{%
\section{J. Crew Files for Bankruptcy in Virus's First Big Retail
Casualty}\label{j-crew-files-for-bankruptcy-in-viruss-first-big-retail-casualty}}

It was struggling before the coronavirus pandemic, but J. Crew is
unlikely to be the last retailer to fall.

\includegraphics{https://static01.graylady3jvrrxbe.onion/images/2020/06/03/business/03virus-jcrew-swap/03virus-jcrew-swap-articleLarge.jpg?quality=75\&auto=webp\&disable=upscale}

\href{https://www.nytimes3xbfgragh.onion/by/vanessa-friedman}{\includegraphics{https://static01.graylady3jvrrxbe.onion/images/2018/06/12/multimedia/vanessa-friedman/vanessa-friedman-thumbLarge.png}}\href{https://www.nytimes3xbfgragh.onion/by/sapna-maheshwari}{\includegraphics{https://static01.graylady3jvrrxbe.onion/images/2018/02/20/multimedia/author-sapna-maheshwari/author-sapna-maheshwari-thumbLarge.jpg}}\href{https://www.nytimes3xbfgragh.onion/by/michael-j-de-la-merced}{\includegraphics{https://static01.graylady3jvrrxbe.onion/images/2019/10/07/reader-center/author-michael-j-de-la-merced/author-michael-j-de-la-merced-thumbLarge.png}}

By \href{https://www.nytimes3xbfgragh.onion/by/vanessa-friedman}{Vanessa
Friedman},
\href{https://www.nytimes3xbfgragh.onion/by/sapna-maheshwari}{Sapna
Maheshwari} and
\href{https://www.nytimes3xbfgragh.onion/by/michael-j-de-la-merced}{Michael
J. de la Merced}

\begin{itemize}
\item
  Published May 3, 2020Updated May 14, 2020
\item
  \begin{itemize}
  \item
  \item
  \item
  \item
  \item
  \item
  \end{itemize}
\end{itemize}

\href{https://www.nytimes3xbfgragh.onion/2020/05/14/business/coronavirus-retail-bankruptcies-private-equity.html}{J.
Crew}, the mass-market clothing company whose preppy-with-a-twist
products were worn by Michelle Obama and appeared at New York Fashion
Week, filed for bankruptcy protection on Monday. It is the first major
retailer to fall during the coronavirus pandemic, though other big
industry names including Neiman Marcus and J.C. Penney are also
struggling with the toll of mass shutdowns.

J. Crew announced that its parent company, Chinos Holdings, had
\href{https://jcrewgrouprestructuring.com/our-announcement/}{filed for
Chapter 11 protection} in federal bankruptcy court for the Eastern
District of Virginia. As part of its financial reorganization plan, it
will hand over control to top creditors, including the hedge fund
Anchorage Capital, by converting \$1.65 billion of its debt into equity.
The company, which has secured a \$400 million debtor-in-possession
loan, also plans to hold onto its Madewell brand, which it had
considered spinning off into a public company.

J. Crew added that its online business would continue to operate
normally throughout its restructuring, and that it planned to reopen its
J. Crew and Madewell stores once lockdowns are lifted. Gift cards and
returns and exchanges would not be affected, it said
\href{https://jcrewgrouprestructuring.com/message-to-our-customers/}{in
a message} to customers, adding that it planned to serve shoppers ``for
years to come.''

``This agreement with our lenders represents a critical milestone in the
ongoing process to transform our business,'' Jan Singer, J. Crew's chief
executive, said in a statement.

The company had been in negotiations with lenders on how to handle its
debt for weeks and made the decision after its board conferred Sunday
evening, according to two people with knowledge of the situation, who
spoke on the condition of anonymity because discussions were
confidential.

The pandemic has been disastrous for the already weakened retail
industry. In March, sales of clothing and accessories fell by
\href{https://www.nytimes3xbfgragh.onion/2020/04/15/business/economy/coronavirus-retail-sales.html?searchResultPosition=1}{more
than half}. The numbers for April are expected to be worse, because many
stores were open for at least some of March (e-commerce, a relatively
small contributor to total sales for most store chains, is not enough to
make up for the closures).

Retailers have furloughed employees, slashed executive salaries and
hoarded cash in a desperate attempt to survive until the shutdowns are
lifted. And there is widespread acknowledgment that J. Crew is unlikely
to be the only retailer to face the brink.

\hypertarget{latest-updates-economy}{%
\section{\texorpdfstring{\href{https://www.nytimes3xbfgragh.onion/live/2020/08/03/business/stock-market-today-coronavirus?action=click\&pgtype=Article\&state=default\&region=MAIN_CONTENT_1\&context=storylines_live_updates}{Latest
Updates:
Economy}}{Latest Updates: Economy}}\label{latest-updates-economy}}

\href{https://www.nytimes3xbfgragh.onion/live/2020/08/03/business/stock-market-today-coronavirus?action=click\&pgtype=Article\&state=default\&region=MAIN_CONTENT_1\&context=storylines_live_updates\#the-chicago-fed-president-says-its-up-to-congress-to-save-the-economy}{10h
ago}

\href{https://www.nytimes3xbfgragh.onion/live/2020/08/03/business/stock-market-today-coronavirus?action=click\&pgtype=Article\&state=default\&region=MAIN_CONTENT_1\&context=storylines_live_updates\#the-chicago-fed-president-says-its-up-to-congress-to-save-the-economy}{The
Chicago Fed president says it's up to Congress to save the economy.}

\href{https://www.nytimes3xbfgragh.onion/live/2020/08/03/business/stock-market-today-coronavirus?action=click\&pgtype=Article\&state=default\&region=MAIN_CONTENT_1\&context=storylines_live_updates\#faa-says-boeing-has-effectively-mitigated-defects-in-the-737-max}{11h
ago}

\href{https://www.nytimes3xbfgragh.onion/live/2020/08/03/business/stock-market-today-coronavirus?action=click\&pgtype=Article\&state=default\&region=MAIN_CONTENT_1\&context=storylines_live_updates\#faa-says-boeing-has-effectively-mitigated-defects-in-the-737-max}{F.A.A.
says Boeing has `effectively mitigated' defects in the 737 Max.}

\href{https://www.nytimes3xbfgragh.onion/live/2020/08/03/business/stock-market-today-coronavirus?action=click\&pgtype=Article\&state=default\&region=MAIN_CONTENT_1\&context=storylines_live_updates\#small-businesses-got-emergency-loans-but-not-what-they-expected}{13h
ago}

\href{https://www.nytimes3xbfgragh.onion/live/2020/08/03/business/stock-market-today-coronavirus?action=click\&pgtype=Article\&state=default\&region=MAIN_CONTENT_1\&context=storylines_live_updates\#small-businesses-got-emergency-loans-but-not-what-they-expected}{Small
businesses got emergency loans, but not what they expected.}

\href{https://www.nytimes3xbfgragh.onion/live/2020/08/03/business/stock-market-today-coronavirus?action=click\&pgtype=Article\&state=default\&region=MAIN_CONTENT_1\&context=storylines_live_updates}{See
more updates}

More live coverage:
\href{https://www.nytimes3xbfgragh.onion/2020/08/03/world/coronavirus-covid-19.html?action=click\&pgtype=Article\&state=default\&region=MAIN_CONTENT_1\&context=storylines_live_updates}{Global}

J. Crew was carrying a
\href{https://investors.jcrew.com/static-files/7e60f08e-e8c7-454b-8b7c-d78ec412efcc}{debt
burden of \$1.7 billion} based on
\href{https://dealbook.nytimes3xbfgragh.onion/2011/03/01/j-crew-shareholders-approve-buyout/}{a
leveraged buyout} in 2011 by two private equity firms ---~TPG Capital
and Leonard Green \& Partners ---~even before the coronavirus brought
clothing sales to a near-halt in its 181 stores, 140 Madewells and 170
outlets. It had also struggled to adapt to changing consumer tastes.

But it seemed to be making strides in recent months toward a more viable
future. The company
\href{https://www.nytimes3xbfgragh.onion/2020/01/28/business/j-crew-jan-singer.html}{named
Ms. Singer} its new chief executive in January and was planning an
\href{https://www.nytimes3xbfgragh.onion/2019/09/17/business/madewell-ipo-j-crew.html}{initial
public offering of Madewell} this spring in order to pay down some of
the debt and rehabilitate the J. Crew brand.

The coronavirus pandemic scuttled those plans and eventually toppled the
company.

J. Crew started life in 1947 as a family-run low-priced clothing line
for women called Popular Club Plan, and in 1983 it was renamed and
reinvented as a catalog company selling turtleneck tops and crew neck
sweaters in ``Preppy Handbook'' shades. It made the leap to household
name and 21st century fashion fairy tale in October 2008 when Mrs.
Obama, whose husband was then the Democratic candidate for president,
\href{https://www.nytimes3xbfgragh.onion/2008/11/17/business/media/17crew.html}{appeared}
on ``The Tonight Show With Jay Leno.'' This was just days after it had
been revealed that Sarah Palin, the Republican candidate for vice
president, had been given a costly wardrobe makeover. ``I want to ask
you about your wardrobe,'' Mr. Leno said to Mrs. Obama. ``I'm guessing
about 60 grand? Sixty, 70 thousand for that outfit?''

``Actually, this is a J. Crew ensemble,'' Mrs. Obama replied, referring
to her \$148 yellow pencil skirt, \$148 yellow and brown print tank top
and \$118 matching yellow cardigan. ``Ladies, we know J. Crew. You can
get some good stuff online!''

It was a priceless marketing moment. After that, everyone knew J. Crew,
which seemed to embody the high/low mix-and-match trend of the moment.

The company was purchased by TPG in 1997 in a leveraged buyout from the
\href{https://www.nytimes3xbfgragh.onion/2017/10/17/obituaries/arthur-cinader-who-started-j-crew-avatar-of-preppy-style-dies-at-90.html}{founding
Cinader} family, and was taken public in 2003 --- only to be reacquired
for approximately \$3 billion by TPG and Leonard Green \& Partners
nearly a decade ago.

Its creative director, Jenna Lyons, who first joined as part of the
design team in 1990, became a boldface name, known for her black-rimmed
glasses, gangly frame and love of sequins and camouflage. Newspaper
reports crowed about the comeback of the company's chief executive,
Millard S. Drexler, who had previously led Gap Inc. for years. Mr.
Drexler, who goes by Mickey, became famous for riding his bicycle around
the office and checking in with store associates via speakerphone.

In 2011 J. Crew became the first mass-market accessible brand to breach
the high fashion parapet and present at New York Fashion Week.
\href{https://www.vogue.com/article/and-at-long-last-its-showtime-jcrews-runway-debut-at-new-york-fashion-week}{Vogue}
crowned the brand ``a significant voice in the conversation on American
style.'' As the face of the brand, Ms. Lyons attended the Met Gala and,
in 2014, played a role on the HBO show ``Girls.''

In 2017, however, after two years of falling sales,
\href{https://www.nytimes3xbfgragh.onion/2017/04/04/fashion/j-crew-jenna-lyons.html}{Ms.
Lyons left the company}. J. Crew, the criticism went, had gone
\emph{too} fashion, falling into the trap of prizing quirk over quality
and pricing itself out of practicality. It had diminished its own brand
with a
\href{https://www.buzzfeednews.com/article/sapna/introducing-jcrew-mercantile-the-factory-outlet-that-isnt}{heavy
push} into outlet merchandise. (There are now nearly as many full-price
J.Crew stores as factory stores.) And it had never focused enough on
e-commerce. Madewell, its younger, simpler --- ``more authentic'' ---
sister brand, acquired by Mr. Drexler in 2006, was the company's new
shining star. Indeed, after Ms. Lyons left, Madewell's designer, Somsack
Sikhounmuong, who had switched over to
\href{https://www.nytimes3xbfgragh.onion/2015/06/12/fashion/somsack-sikhounmuong-of-madewell-becomes-womens-wear-designer-at-j-crew.html}{J.
Crew in 2015}, took the top creative spot. Much was made of a return to
core values.

It was too little, too late. For a fashion brand to thrive it must be
either needed or wanted. J. Crew, sitting somewhere in the netherland of
style and price, was neither. A few months after Ms. Lyons's departure,
\href{https://www.nytimes3xbfgragh.onion/2017/06/05/business/jcrew-mickey-drexler.html}{Mr.
Drexler stepped down} and Mr. Sikhounmuong left two months later,
starting a round robin of executives and designers. That served
ultimately to confuse rather than clarify the identity of the company
and its strategy. Jan Singer, formerly of Nike and Victoria's Secret,
\href{https://www.nytimes3xbfgragh.onion/2020/01/28/business/j-crew-jan-singer.html}{was
named} J. Crew's newest leader in January.

Madewell, which filed for an I.P.O. in the fall, was expected to go
public this spring while J. Crew remained private, but those plans were
ultimately scrapped in March as the stock market spiraled, adding a new
wave of pressure and question marks to J. Crew's future.

J. Crew clearly had plenty of problems even without its massive debt
load. Ongoing interest payments and looming maturities only added to the
company's challenges, and in the eyes of some, left it ill-equipped to
adjust to a new shopping environment.

As the retail landscape has shifted, ``these businesses have faced a
huge investment need,'' said Raya Sokolyanska, a senior analyst at
Moody's. ``Having a burdened balance sheet certainly greatly diminishes
their chances of doing this successfully.''

That's even more challenging for fashion retail, she said, which is
``notoriously fickle.''

Now the question is whether the upheaval of the retail industry ---
which predates the pandemic, with the
\href{https://www.nytimes3xbfgragh.onion/2019/11/01/business/barneys-bankruptcy-authentic-brands.html}{collapse
of Barneys New York} late last year --- will continue.

``The companies going into bankruptcy, for the most part, were companies
that were struggling before Covid --- we have not seen true Covid-only
bankruptcies,'' said James Van Horn, a partner at the law firm Barnes \&
Thornburg and a specialist in retail bankruptcy.

However, he added, ``depending on how the current situation continues,
that may change.''

For instance, Brooks Brothers, another quintessential American shopping
institution, is already facing questions about its future.

``In the ordinary course of business, Brooks Brothers consistently
explores various strategic options to position the company for growth
and success, in partnership with its financial advisers at P.J.
Solomon,'' a spokesman said, in response to question about a potential
sale.

Advertisement

\protect\hyperlink{after-bottom}{Continue reading the main story}

\hypertarget{site-index}{%
\subsection{Site Index}\label{site-index}}

\hypertarget{site-information-navigation}{%
\subsection{Site Information
Navigation}\label{site-information-navigation}}

\begin{itemize}
\tightlist
\item
  \href{https://help.nytimes3xbfgragh.onion/hc/en-us/articles/115014792127-Copyright-notice}{©~2020~The
  New York Times Company}
\end{itemize}

\begin{itemize}
\tightlist
\item
  \href{https://www.nytco.com/}{NYTCo}
\item
  \href{https://help.nytimes3xbfgragh.onion/hc/en-us/articles/115015385887-Contact-Us}{Contact
  Us}
\item
  \href{https://www.nytco.com/careers/}{Work with us}
\item
  \href{https://nytmediakit.com/}{Advertise}
\item
  \href{http://www.tbrandstudio.com/}{T Brand Studio}
\item
  \href{https://www.nytimes3xbfgragh.onion/privacy/cookie-policy\#how-do-i-manage-trackers}{Your
  Ad Choices}
\item
  \href{https://www.nytimes3xbfgragh.onion/privacy}{Privacy}
\item
  \href{https://help.nytimes3xbfgragh.onion/hc/en-us/articles/115014893428-Terms-of-service}{Terms
  of Service}
\item
  \href{https://help.nytimes3xbfgragh.onion/hc/en-us/articles/115014893968-Terms-of-sale}{Terms
  of Sale}
\item
  \href{https://spiderbites.nytimes3xbfgragh.onion}{Site Map}
\item
  \href{https://help.nytimes3xbfgragh.onion/hc/en-us}{Help}
\item
  \href{https://www.nytimes3xbfgragh.onion/subscription?campaignId=37WXW}{Subscriptions}
\end{itemize}
