Sections

SEARCH

\protect\hyperlink{site-content}{Skip to
content}\protect\hyperlink{site-index}{Skip to site index}

\href{https://www.nytimes3xbfgragh.onion/section/health}{Health}

\href{https://myaccount.nytimes3xbfgragh.onion/auth/login?response_type=cookie\&client_id=vi}{}

\href{https://www.nytimes3xbfgragh.onion/section/todayspaper}{Today's
Paper}

\href{/section/health}{Health}\textbar{}A Coronavirus Vaccine Project
Takes a Page From Gene Therapy

\url{https://nyti.ms/2z8sTVZ}

\begin{itemize}
\item
\item
\item
\item
\item
\end{itemize}

\hypertarget{the-coronavirus-outbreak}{%
\subsubsection{\texorpdfstring{\href{https://www.nytimes3xbfgragh.onion/news-event/coronavirus?name=styln-coronavirus-national\&region=TOP_BANNER\&block=storyline_menu_recirc\&action=click\&pgtype=Article\&impression_id=075be580-efba-11ea-9ddb-53a6a90d6799\&variant=undefined}{The
Coronavirus
Outbreak}}{The Coronavirus Outbreak}}\label{the-coronavirus-outbreak}}

\begin{itemize}
\tightlist
\item
  live\href{https://www.nytimes3xbfgragh.onion/2020/09/05/world/coronavirus-covid.html?name=styln-coronavirus-national\&region=TOP_BANNER\&block=storyline_menu_recirc\&action=click\&pgtype=Article\&impression_id=075c0c90-efba-11ea-9ddb-53a6a90d6799\&variant=undefined}{Latest
  Updates}
\item
  \href{https://www.nytimes3xbfgragh.onion/interactive/2020/us/coronavirus-us-cases.html?name=styln-coronavirus-national\&region=TOP_BANNER\&block=storyline_menu_recirc\&action=click\&pgtype=Article\&impression_id=075c0c91-efba-11ea-9ddb-53a6a90d6799\&variant=undefined}{Maps
  and Cases}
\item
  \href{https://www.nytimes3xbfgragh.onion/interactive/2020/science/coronavirus-vaccine-tracker.html?name=styln-coronavirus-national\&region=TOP_BANNER\&block=storyline_menu_recirc\&action=click\&pgtype=Article\&impression_id=075c0c92-efba-11ea-9ddb-53a6a90d6799\&variant=undefined}{Vaccine
  Tracker}
\item
  \href{https://www.nytimes3xbfgragh.onion/2020/09/02/your-money/eviction-moratorium-covid.html?name=styln-coronavirus-national\&region=TOP_BANNER\&block=storyline_menu_recirc\&action=click\&pgtype=Article\&impression_id=075c0c93-efba-11ea-9ddb-53a6a90d6799\&variant=undefined}{Eviction
  Moratorium}
\item
  \href{https://www.nytimes3xbfgragh.onion/interactive/2020/09/02/magazine/food-insecurity-hunger-us.html?name=styln-coronavirus-national\&region=TOP_BANNER\&block=storyline_menu_recirc\&action=click\&pgtype=Article\&impression_id=075c0c94-efba-11ea-9ddb-53a6a90d6799\&variant=undefined}{American
  Hunger}
\end{itemize}

Advertisement

\protect\hyperlink{after-top}{Continue reading the main story}

Supported by

\protect\hyperlink{after-sponsor}{Continue reading the main story}

\hypertarget{a-coronavirus-vaccine-project-takes-a-page-from-gene-therapy}{%
\section{A Coronavirus Vaccine Project Takes a Page From Gene
Therapy}\label{a-coronavirus-vaccine-project-takes-a-page-from-gene-therapy}}

The technique aims to make a person's cells churn out proteins that will
stimulate the body to fight the coronavirus.

\includegraphics{https://static01.graylady3jvrrxbe.onion/images/2020/05/04/science/04virus-genetherapy01/merlin_172185261_12649b52-1ede-40c1-800e-dd49a24c8a41-articleLarge.jpg?quality=75\&auto=webp\&disable=upscale}

By \href{https://www.nytimes3xbfgragh.onion/by/denise-grady}{Denise
Grady}

\begin{itemize}
\item
  Published May 4, 2020Updated May 7, 2020
\item
  \begin{itemize}
  \item
  \item
  \item
  \item
  \item
  \end{itemize}
\end{itemize}

Researchers at two Harvard-affiliated hospitals are adapting a proven
form of gene therapy to develop a
\href{https://www.nytimes3xbfgragh.onion/2020/05/07/health/coronavirus-vaccine-moderna.html}{coronavirus
vaccine}, which they expect to test in people later this year, they
announced on Monday.

Their work employs a method already used in
\href{https://www.nytimes3xbfgragh.onion/2020/05/06/health/coronavirus-mutation-transmission.html}{gene}
therapy for two inherited diseases, including a form of blindness: It
uses a harmless virus as a vector, or carrier, to bring DNA into the
patient's cells. In this case, the
\href{https://www.nytimes3xbfgragh.onion/2020/07/04/health/coronavirus-neanderthals.html}{DNA}
should instruct the cells to make a coronavirus protein that would
stimulate the immune system to fight off future infections.

So far, the team has studied its vaccine candidates only in mice. Tests
for safety and potency in monkeys should begin within a month or so at
another academic center, the researchers said. But two of seven
promising versions are already being manufactured for studies in humans.

The research is one of at
least\href{https://www.nytimes3xbfgragh.onion/2020/05/02/us/politics/vaccines-coronavirus-research.html}{90
vaccine projects speeding ahead} around the world in desperate efforts
that hold the best and probably only hope of stopping or at least
slowing the pandemic.

``We are presenting a different angle from everybody else,'' Dr. Luk H.
Vandenberghe, director of the Grousbeck Gene Therapy Center at
Massachusetts Eye and Ear, said in an interview.

Their angle has several advantages, he said. One is that the type of
vector, an adeno-associated virus, or AAV, is a harmless virus that is
already used in two approved forms of gene therapy and has been tested
in many patients and found to be safe.

Another plus is that the technique requires very small amounts of the
vector and DNA to produce immunity, so the yield of doses would be high,
Dr. Vandenberghe said. Small amounts are enough because the vector is
very good at getting into cells, which the DNA it carries transforms
into factories that crank out the protein needed to put the immune
system on alert against the coronavirus, he added.

\hypertarget{latest-updates-the-coronavirus-outbreak}{%
\section{\texorpdfstring{\href{https://www.nytimes3xbfgragh.onion/2020/09/04/world/covid-19-coronavirus.html?action=click\&pgtype=Article\&state=default\&region=MAIN_CONTENT_1\&context=storylines_live_updates}{Latest
Updates: The Coronavirus
Outbreak}}{Latest Updates: The Coronavirus Outbreak}}\label{latest-updates-the-coronavirus-outbreak}}

Updated 2020-09-05T12:05:40.998Z

\begin{itemize}
\tightlist
\item
  \href{https://www.nytimes3xbfgragh.onion/2020/09/04/world/covid-19-coronavirus.html?action=click\&pgtype=Article\&state=default\&region=MAIN_CONTENT_1\&context=storylines_live_updates\#link-1654f6ad}{Research
  connects vaping to a higher chance of catching the virus --- and
  suffering its worst effects.}
\item
  \href{https://www.nytimes3xbfgragh.onion/2020/09/04/world/covid-19-coronavirus.html?action=click\&pgtype=Article\&state=default\&region=MAIN_CONTENT_1\&context=storylines_live_updates\#link-52e4198a}{Another
  college football game won't be played as planned.}
\item
  \href{https://www.nytimes3xbfgragh.onion/2020/09/04/world/covid-19-coronavirus.html?action=click\&pgtype=Article\&state=default\&region=MAIN_CONTENT_1\&context=storylines_live_updates\#link-181cef0}{Pharmaceutical
  companies plan a joint pledge on safety standards as they move
  vaccines to the marketplace.}
\end{itemize}

\href{https://www.nytimes3xbfgragh.onion/2020/09/04/world/covid-19-coronavirus.html?action=click\&pgtype=Article\&state=default\&region=MAIN_CONTENT_1\&context=storylines_live_updates}{See
more updates}

More live coverage:
\href{https://www.nytimes3xbfgragh.onion/live/2020/09/04/business/stock-market-today-coronavirus?action=click\&pgtype=Article\&state=default\&region=MAIN_CONTENT_1\&context=storylines_live_updates}{Markets}

In addition, many drug and biotech companies, large and small, already
produce adeno-associated virus and could easily switch to producing the
form needed for the vaccine. New facilities would not have to be built
from scratch. That means production could quickly be scaled up to help
meet the huge and urgent global need for vaccine.

``We are leaning on an established industry,'' Dr. Vandenberghe said.
``AAV as a class has had investments in the dozens of billions of
dollars. There's capacity out there that can be leveraged if we are
lucky and successful.''

The approved gene therapies that use adeno-associated viruses are
Luxturna, for a form of hereditary blindness, and
\href{https://www.nytimes3xbfgragh.onion/2019/05/24/health/zolgensma-gene-therapy-drug.html}{Zolgensma},
for a lethal nerve disease, spinal muscular atrophy. The price of
Luxturna is \$425,000 per eye. It is made and sold in the United States
by Spark Therapeutics, and sold overseas by Novartis. Zolgensma, made by
Novartis, is given to young children as a one-time injection, at a price
of \$2.1 million.

At this early stage, Dr. Vandenberghe estimates the manufacturing cost
per dose of vaccine to be from \$2.50 to \$250.

The research has the financial backing of donors, led by Wyc Grousbeck,
lead owner of the Boston Celtics, his wife, Emilia Fazzalari, the chief
executive of Cinco Spirits Group, and members of the Grousbeck family.

Several other vaccine projects involve viral vectors, but no others use
adeno-associated viruses.

Hundreds of AAVs infect humans and other mammals. Researchers in gene
therapy turned to them as a safer alternative after
\href{https://www.sciencehistory.org/distillations/the-death-of-jesse-gelsinger-20-years-later}{a
patient died in a study in 1999} from a severe immune reaction to a
different type of viral vector.

For gene therapy, scientists sought adeno-associated viruses that would
act as a sort of stealth carrier for DNA and not set off any
inflammatory or immune response. They also tweaked the viruses
genetically to prevent them from replicating inside human cells.

Working at the University of Pennsylvania about 15 years ago, Dr.
Vandenberghe created a hybrid AAV from two versions found in monkeys,
for possible use in gene therapy. It was no good: It provoked an immune
response.

\href{https://www.nytimes3xbfgragh.onion/news-event/coronavirus?action=click\&pgtype=Article\&state=default\&region=MAIN_CONTENT_3\&context=storylines_faq}{}

\hypertarget{the-coronavirus-outbreak-}{%
\subsubsection{The Coronavirus Outbreak
›}\label{the-coronavirus-outbreak-}}

\hypertarget{frequently-asked-questions}{%
\paragraph{Frequently Asked
Questions}\label{frequently-asked-questions}}

Updated September 4, 2020

\begin{itemize}
\item ~
  \hypertarget{what-are-the-symptoms-of-coronavirus}{%
  \paragraph{What are the symptoms of
  coronavirus?}\label{what-are-the-symptoms-of-coronavirus}}

  \begin{itemize}
  \tightlist
  \item
    In the beginning, the coronavirus
    \href{https://www.nytimes3xbfgragh.onion/article/coronavirus-facts-history.html?action=click\&pgtype=Article\&state=default\&region=MAIN_CONTENT_3\&context=storylines_faq\#link-6817bab5}{seemed
    like it was primarily a respiratory illness}~--- many patients had
    fever and chills, were weak and tired, and coughed a lot, though
    some people don't show many symptoms at all. Those who seemed
    sickest had pneumonia or acute respiratory distress syndrome and
    received supplemental oxygen. By now, doctors have identified many
    more symptoms and syndromes. In April,
    \href{https://www.nytimes3xbfgragh.onion/2020/04/27/health/coronavirus-symptoms-cdc.html?action=click\&pgtype=Article\&state=default\&region=MAIN_CONTENT_3\&context=storylines_faq}{the
    C.D.C. added to the list of early signs}~sore throat, fever, chills
    and muscle aches. Gastrointestinal upset, such as diarrhea and
    nausea, has also been observed. Another telltale sign of infection
    may be a sudden, profound diminution of one's
    \href{https://www.nytimes3xbfgragh.onion/2020/03/22/health/coronavirus-symptoms-smell-taste.html?action=click\&pgtype=Article\&state=default\&region=MAIN_CONTENT_3\&context=storylines_faq}{sense
    of smell and taste.}~Teenagers and young adults in some cases have
    developed painful red and purple lesions on their fingers and toes
    --- nicknamed ``Covid toe'' --- but few other serious symptoms.
  \end{itemize}
\item ~
  \hypertarget{why-is-it-safer-to-spend-time-together-outside}{%
  \paragraph{Why is it safer to spend time together
  outside?}\label{why-is-it-safer-to-spend-time-together-outside}}

  \begin{itemize}
  \tightlist
  \item
    \href{https://www.nytimes3xbfgragh.onion/2020/05/15/us/coronavirus-what-to-do-outside.html?action=click\&pgtype=Article\&state=default\&region=MAIN_CONTENT_3\&context=storylines_faq}{Outdoor
    gatherings}~lower risk because wind disperses viral droplets, and
    sunlight can kill some of the virus. Open spaces prevent the virus
    from building up in concentrated amounts and being inhaled, which
    can happen when infected people exhale in a confined space for long
    stretches of time, said Dr. Julian W. Tang, a virologist at the
    University of Leicester.
  \end{itemize}
\item ~
  \hypertarget{why-does-standing-six-feet-away-from-others-help}{%
  \paragraph{Why does standing six feet away from others
  help?}\label{why-does-standing-six-feet-away-from-others-help}}

  \begin{itemize}
  \tightlist
  \item
    The coronavirus spreads primarily through droplets from your mouth
    and nose, especially when you cough or sneeze. The C.D.C., one of
    the organizations using that measure,
    \href{https://www.nytimes3xbfgragh.onion/2020/04/14/health/coronavirus-six-feet.html?action=click\&pgtype=Article\&state=default\&region=MAIN_CONTENT_3\&context=storylines_faq}{bases
    its recommendation of six feet}~on the idea that most large droplets
    that people expel when they cough or sneeze will fall to the ground
    within six feet. But six feet has never been a magic number that
    guarantees complete protection. Sneezes, for instance, can launch
    droplets a lot farther than six feet,
    \href{https://jamanetwork.com/journals/jama/fullarticle/2763852}{according
    to a recent study}. It's a rule of thumb: You should be safest
    standing six feet apart outside, especially when it's windy. But
    keep a mask on at all times, even when you think you're far enough
    apart.
  \end{itemize}
\item ~
  \hypertarget{i-have-antibodies-am-i-now-immune}{%
  \paragraph{I have antibodies. Am I now
  immune?}\label{i-have-antibodies-am-i-now-immune}}

  \begin{itemize}
  \tightlist
  \item
    As of right
    now,\href{https://www.nytimes3xbfgragh.onion/2020/07/22/health/covid-antibodies-herd-immunity.html?action=click\&pgtype=Article\&state=default\&region=MAIN_CONTENT_3\&context=storylines_faq}{~that
    seems likely, for at least several months.}~There have been
    frightening accounts of people suffering what seems to be a second
    bout of Covid-19. But experts say these patients may have a
    drawn-out course of infection, with the virus taking a slow toll
    weeks to months after initial exposure.~People infected with the
    coronavirus typically
    \href{https://www.nature.com/articles/s41586-020-2456-9}{produce}~immune
    molecules called antibodies, which are
    \href{https://www.nytimes3xbfgragh.onion/2020/05/07/health/coronavirus-antibody-prevalence.html?action=click\&pgtype=Article\&state=default\&region=MAIN_CONTENT_3\&context=storylines_faq}{protective
    proteins made in response to an
    infection}\href{https://www.nytimes3xbfgragh.onion/2020/05/07/health/coronavirus-antibody-prevalence.html?action=click\&pgtype=Article\&state=default\&region=MAIN_CONTENT_3\&context=storylines_faq}{.
    These antibodies may}~last in the body
    \href{https://www.nature.com/articles/s41591-020-0965-6}{only two to
    three months}, which may seem worrisome, but that's~perfectly normal
    after an acute infection subsides, said Dr. Michael Mina, an
    immunologist at Harvard University. It may be possible to get the
    coronavirus again, but it's highly unlikely that it would be
    possible in a short window of time from initial infection or make
    people sicker the second time.
  \end{itemize}
\item ~
  \hypertarget{what-are-my-rights-if-i-am-worried-about-going-back-to-work}{%
  \paragraph{What are my rights if I am worried about going back to
  work?}\label{what-are-my-rights-if-i-am-worried-about-going-back-to-work}}

  \begin{itemize}
  \tightlist
  \item
    Employers have to provide
    \href{https://www.osha.gov/SLTC/covid-19/standards.html}{a safe
    workplace}~with policies that protect everyone equally.
    \href{https://www.nytimes3xbfgragh.onion/article/coronavirus-money-unemployment.html?action=click\&pgtype=Article\&state=default\&region=MAIN_CONTENT_3\&context=storylines_faq}{And
    if one of your co-workers tests positive for the coronavirus, the
    C.D.C.}~has said that
    \href{https://www.cdc.gov/coronavirus/2019-ncov/community/guidance-business-response.html}{employers
    should tell their employees}~-\/- without giving you the sick
    employee's name -\/- that they may have been exposed to the virus.
  \end{itemize}
\end{itemize}

But that makes it a good candidate for a vaccine, because such a
reaction can help stoke the immune system to fight the coronavirus. Many
vaccines contain substances called adjuvants to help rev up the response
to the vaccine itself. For this vaccine, the hybrid AAV may act not as
only a vector, but also as an adjuvant.

Dr. Mason Freeman, director and founder of the translational research
center at the Massachusetts General Hospital, who is planning the human
tests of the vaccine, said the researchers were hoping for just enough
immune response to the AAV and not too much. But the hybrid vector has
never been used in humans before, so there is still much to learn.

Like other vaccine projects, this one is focusing on the so-called spike
on the coronavirus, which it uses to grab onto cells and invade them. In
theory, if the immune system can be trained to make antibodies to block
the spike, the virus will not be able to establish an infection.

The DNA being carried by the vector holds directions for making a
protein portion of the spike. Because it is not known which section of
the spike is likely to set off the best immune response, researchers
test DNA for different parts.

One potential problem that every vaccine project will be on the lookout
for is disease enhancement: the possibility that a vaccine, instead of
preventing infection, could actually make the disease worse. Dr.
Vandenberghe said his team was working with three other laboratories
that would conduct tests to address this question.

Disease enhancement has become a particular concern because all the
vaccine projects are trying to move so much faster than usual that there
is a fear the problem could go undetected, Dr. Freeman said. It is
challenging to design studies to find it, particularly with a new
disease that is not well understood.

The two scientists said the many research groups forging ahead with
vaccine projects were racing not against one another, but against the
coronavirus.

``We need multiple shots at goal,'' Dr. Vandenberghe said. ``The level
of unknowns of what we have to try to achieve here is too high, and the
level of urgency is equally high.''

Advertisement

\protect\hyperlink{after-bottom}{Continue reading the main story}

\hypertarget{site-index}{%
\subsection{Site Index}\label{site-index}}

\hypertarget{site-information-navigation}{%
\subsection{Site Information
Navigation}\label{site-information-navigation}}

\begin{itemize}
\tightlist
\item
  \href{https://help.nytimes3xbfgragh.onion/hc/en-us/articles/115014792127-Copyright-notice}{©~2020~The
  New York Times Company}
\end{itemize}

\begin{itemize}
\tightlist
\item
  \href{https://www.nytco.com/}{NYTCo}
\item
  \href{https://help.nytimes3xbfgragh.onion/hc/en-us/articles/115015385887-Contact-Us}{Contact
  Us}
\item
  \href{https://www.nytco.com/careers/}{Work with us}
\item
  \href{https://nytmediakit.com/}{Advertise}
\item
  \href{http://www.tbrandstudio.com/}{T Brand Studio}
\item
  \href{https://www.nytimes3xbfgragh.onion/privacy/cookie-policy\#how-do-i-manage-trackers}{Your
  Ad Choices}
\item
  \href{https://www.nytimes3xbfgragh.onion/privacy}{Privacy}
\item
  \href{https://help.nytimes3xbfgragh.onion/hc/en-us/articles/115014893428-Terms-of-service}{Terms
  of Service}
\item
  \href{https://help.nytimes3xbfgragh.onion/hc/en-us/articles/115014893968-Terms-of-sale}{Terms
  of Sale}
\item
  \href{https://spiderbites.nytimes3xbfgragh.onion}{Site Map}
\item
  \href{https://help.nytimes3xbfgragh.onion/hc/en-us}{Help}
\item
  \href{https://www.nytimes3xbfgragh.onion/subscription?campaignId=37WXW}{Subscriptions}
\end{itemize}
