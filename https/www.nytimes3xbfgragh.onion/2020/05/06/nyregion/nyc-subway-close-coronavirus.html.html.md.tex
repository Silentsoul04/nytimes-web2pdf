Sections

SEARCH

\protect\hyperlink{site-content}{Skip to
content}\protect\hyperlink{site-index}{Skip to site index}

\href{/section/nyregion}{New York}\textbar{}Subway Shutdown: New York
Closes System for First Time in 115 Years

\url{https://nyti.ms/2SIaqGH}

\begin{itemize}
\item
\item
\item
\item
\item
\item
\end{itemize}

\hypertarget{the-coronavirus-outbreak}{%
\subsubsection{\texorpdfstring{\href{https://www.nytimes3xbfgragh.onion/news-event/coronavirus?name=styln-coronavirus-national\&region=TOP_BANNER\&variant=undefined\&block=storyline_menu_recirc\&action=click\&pgtype=Article\&impression_id=6dfa89b0-e3ab-11ea-b9dd-61427990e3d0}{The
Coronavirus
Outbreak}}{The Coronavirus Outbreak}}\label{the-coronavirus-outbreak}}

\begin{itemize}
\tightlist
\item
  live\href{https://www.nytimes3xbfgragh.onion/2020/08/21/world/covid-19-coronavirus.html?name=styln-coronavirus-national\&region=TOP_BANNER\&variant=undefined\&block=storyline_menu_recirc\&action=click\&pgtype=Article\&impression_id=6dfa89b1-e3ab-11ea-b9dd-61427990e3d0}{Latest
  Updates}
\item
  \href{https://www.nytimes3xbfgragh.onion/interactive/2020/us/coronavirus-us-cases.html?name=styln-coronavirus-national\&region=TOP_BANNER\&variant=undefined\&block=storyline_menu_recirc\&action=click\&pgtype=Article\&impression_id=6dfa89b2-e3ab-11ea-b9dd-61427990e3d0}{Maps
  and Cases}
\item
  \href{https://www.nytimes3xbfgragh.onion/interactive/2020/science/coronavirus-vaccine-tracker.html?name=styln-coronavirus-national\&region=TOP_BANNER\&variant=undefined\&block=storyline_menu_recirc\&action=click\&pgtype=Article\&impression_id=6dfa89b3-e3ab-11ea-b9dd-61427990e3d0}{Vaccine
  Tracker}
\item
  \href{https://www.nytimes3xbfgragh.onion/2020/08/19/us/colleges-closing-covid.html?name=styln-coronavirus-national\&region=TOP_BANNER\&variant=undefined\&block=storyline_menu_recirc\&action=click\&pgtype=Article\&impression_id=6dfab0c0-e3ab-11ea-b9dd-61427990e3d0}{Colleges
  Closing}
\item
  \href{https://www.nytimes3xbfgragh.onion/live/2020/08/21/business/stock-market-today-coronavirus?name=styln-coronavirus-national\&region=TOP_BANNER\&variant=undefined\&block=storyline_menu_recirc\&action=click\&pgtype=Article\&impression_id=6dfab0c1-e3ab-11ea-b9dd-61427990e3d0}{Economy}
\end{itemize}

\includegraphics{https://static01.graylady3jvrrxbe.onion/images/2020/05/06/nyregion/06nyvirus-subway1/06nyvirus-subway1-articleLarge-v2.jpg?quality=75\&auto=webp\&disable=upscale}

\hypertarget{subway-shutdown-new-york-closes-system-for-first-time-in-115-years}{%
\section{Subway Shutdown: New York Closes System for First Time in 115
Years}\label{subway-shutdown-new-york-closes-system-for-first-time-in-115-years}}

On Wednesday morning, the system, a 24-hour mainstay, closed for
overnight disinfection.

The entrance to the Wakefield-241 Street station early
Wednesday.Credit...Hilary Swift for The New York Times

Supported by

\protect\hyperlink{after-sponsor}{Continue reading the main story}

\href{https://www.nytimes3xbfgragh.onion/by/christina-goldbaum}{\includegraphics{https://static01.graylady3jvrrxbe.onion/images/2019/11/22/reader-center/author-christina-goldbaum/author-christina-goldbaum-thumbLarge.png}}

By
\href{https://www.nytimes3xbfgragh.onion/by/christina-goldbaum}{Christina
Goldbaum}

\begin{itemize}
\item
  May 6, 2020
\item
  \begin{itemize}
  \item
  \item
  \item
  \item
  \item
  \item
  \end{itemize}
\end{itemize}

At the Stillwell Avenue station in Coney Island, Brooklyn, over a dozen
police officers waited on the platform early Wednesday for trains on the
F line to arrive. As one train pulled in at 12:43 a.m., transit agency
outreach workers and social workers walked onto the cars, cajoling
homeless riders to leave their temporary shelter.

Overhead an announcement echoed through the vast station: ``Last stop on
arriving train. No passengers, please.''

It was a moment that New York City's subway had never before
experienced: the first planned overnight shutdown since the system
opened 115 years ago.

But with the city still in the grip of the coronavirus pandemic, the
subway will remain closed from 1 a.m. to 5 a.m. for the foreseeable
future to provide more time to thoroughly disinfect trains, stations and
equipment.

The shutdown reflects the enormous challenge officials face in ensuring
that the
\href{https://www.nytimes3xbfgragh.onion/2020/06/08/nyregion/mta-subway-riding-health-coronavirus.html}{subway
is safe enough to lure back leery riders}, even as the system reels from
a
\href{https://www.nytimes3xbfgragh.onion/2020/05/07/us/new-york-city-coronavirus-outbreak.html}{deadly
outbreak that has ravaged New York} and crippled its public transit
network.

No other American city in normal times relies as much on public transit
and restoring the subway is essential to New York's recovery.

Since March, the virus has sent
\href{https://www.nytimes3xbfgragh.onion/2020/03/24/nyregion/coronavirus-nyc-mta-cuts-.html}{ridership
plummeting}by more than 90 percent,
\href{https://www.nytimes3xbfgragh.onion/2020/05/05/nyregion/bus-drivers-new-york-coronavirus.html}{killed
at least 109 transit workers}, starved the authority of
\href{https://www.nytimes3xbfgragh.onion/2020/04/20/nyregion/nyc-mta-subway-coronavirus.html}{its
usual revenue streams} and prompted an influx of
\href{https://www.nytimes3xbfgragh.onion/2020/05/02/nyregion/coronavirus-nyc-subway-homeless.html}{homeless
people} seeking refuge on mostly empty trains.

\includegraphics{https://static01.graylady3jvrrxbe.onion/images/2020/05/07/nyregion/07nyvirus-subway-2/merlin_172225674_a86abefc-4715-42e6-9baf-6379c77e9e43-articleLarge.jpg?quality=75\&auto=webp\&disable=upscale}

``We're in an unprecedented moment in the history of our city,'' Patrick
J. Foye, the chairman of the Metropolitan Transportation Authority, said
at a news conference late Tuesday night. ``The reason we're taking this
extraordinary, unprecedented action is to protect the safety and public
health of our customers and our employees.''

Still, the closing leaves an indelible mark on a city long defined by
its round-the-clock hustle and unending energy.

\hypertarget{latest-updates-the-coronavirus-outbreak}{%
\section{\texorpdfstring{\href{https://www.nytimes3xbfgragh.onion/2020/08/21/world/covid-19-coronavirus.html?action=click\&pgtype=Article\&state=default\&region=MAIN_CONTENT_1\&context=storylines_live_updates}{Latest
Updates: The Coronavirus
Outbreak}}{Latest Updates: The Coronavirus Outbreak}}\label{latest-updates-the-coronavirus-outbreak}}

Updated 2020-08-21T12:38:27.712Z

\begin{itemize}
\tightlist
\item
  \href{https://www.nytimes3xbfgragh.onion/2020/08/21/world/covid-19-coronavirus.html?action=click\&pgtype=Article\&state=default\&region=MAIN_CONTENT_1\&context=storylines_live_updates\#link-6a60a19d}{`Be
  adults': Universities in the U.S. are warning students about
  gatherings as they return to campus.}
\item
  \href{https://www.nytimes3xbfgragh.onion/2020/08/21/world/covid-19-coronavirus.html?action=click\&pgtype=Article\&state=default\&region=MAIN_CONTENT_1\&context=storylines_live_updates\#link-324af071}{As
  he accepts the Democratic nomination, Biden knocks Trump's pandemic
  response.}
\item
  \href{https://www.nytimes3xbfgragh.onion/2020/08/21/world/covid-19-coronavirus.html?action=click\&pgtype=Article\&state=default\&region=MAIN_CONTENT_1\&context=storylines_live_updates\#link-191d44be}{South
  Korea threatens to detain people who obstruct virus-control efforts.}
\end{itemize}

\href{https://www.nytimes3xbfgragh.onion/2020/08/21/world/covid-19-coronavirus.html?action=click\&pgtype=Article\&state=default\&region=MAIN_CONTENT_1\&context=storylines_live_updates}{See
more updates}

More live coverage:
\href{https://www.nytimes3xbfgragh.onion/live/2020/08/21/business/stock-market-today-coronavirus?action=click\&pgtype=Article\&state=default\&region=MAIN_CONTENT_1\&context=storylines_live_updates}{Markets}

Since the 1880s, New Yorkers have ridden public transit at all hours of
the day and night on street trolleys, moving underground in 1904 to the
city's first subway line connecting Lower Manhattan and Harlem.

At that time, the 24-hour service carried crowds of workers from
manufacturing plants and port docks who, even in the wee hours, proved
lucrative to the private companies that once ran the system.

In the decades that followed, the constant movement of people shaped the
growing metropolis. The steady pulse of the city's underground arteries
fueled New York's economic expansion and enabled the lifestyle that
sparked its clichéd reputation as an insomniac city.

But in the wake of a catastrophic virus, even this mainstay of New York
life has not been spared. Early Wednesday, as transit workers locked
turnstiles and police officers taped off station entrances, the
outbreak's lasting blow to the city was immortalized for the millions of
New Yorkers who have trudged onto a late-night train.

Image

Cleaners wearing Tyvek suits, shoe coverings, face masks and surgical
gloves sanitized a No. 2 train.Credit...Hilary Swift for The New York
Times

Across the system, the complicated task of shutting the system played
out in nearly every station: Often confused early-morning riders
endeavored to find other ways of getting to work. Police officers and
social workers attempted to coax sometimes agitated homeless people off
the subway. And crews of cleaners clamored to thoroughly disinfect the
rolling stock and stations before the system reopened.

Just before 1 a.m. at the Wakefield-241st Street station in the Bronx,
outreach workers from the city's Department of Homeless Services tried
to persuade homeless people in the station to leave for a shelter.

One man in a hooded sweatshirt refused help, pleading with them to let
him stay.

``You can't do this to me,'' the man cried. ``I want to get back on the
train.''

``Please get me back on the train,'' he said, his voice breaking. A
police officer directed him instead to the Bx42 bus stop on the street
below.

Image

Police officers and a nurse led homeless people out of a train station
in the Bronx.Credit...Hilary Swift for The New York Times

His experience echoed concerns from advocates that the subway closure
will leave an already vulnerable population even more exposed.

``Without the offer of a safe, private room, most people are going to be
displaced into the streets, where they are going to be more vulnerable
than they were on the subway,'' said Josh Dean, executive director of
Human.nyc, a policy organization that focuses on homelessness.

Around the same time, a homeless rider at the Stillwell Avenue station
in Coney Island took up the offer from outreach workers to be steered to
a shelter.

The rider, Stephen Bell, 33, said he had lost his job as an
environmental researcher a week ago and soon afterward his home in
Queens. Three weeks ago, he also tested positive for Covid-19.

With just a blanket tucked under his arm, Mr. Bell followed social
workers out of the station.

``They were really polite about it,'' he said of the outreach workers,
adding, ``Staying out all night on the subway is much worse'' than the
coronavirus.

As social workers cleared stragglers from platforms and cleaners flooded
into cars, the man charged with orchestrating the shutdown, Hugo Zamora,
stared intently at a live map of each train from his perch at the
agency's Rail Control Center in Manhattan.

There, a digital screen the width of a football field shows brightly lit
rectangles depicting trains in real time as they prattle along a map of
the tracks across the city. Dispatchers, who control the movement of
trains, called out radio dispatches to crews between sharp clatters of
static.

Image

Hugo Zamora in the Rail Control Center as the subway was shutting
down.Credit...Kholood Eid for The New York Times

Mr. Zamora, a transit veteran who has been nicknamed the Shutdown Czar
by some colleagues, explained that the sheer scope of the system makes
bringing it offline a complicated process.

\href{https://www.nytimes3xbfgragh.onion/news-event/coronavirus?action=click\&pgtype=Article\&state=default\&region=MAIN_CONTENT_3\&context=storylines_faq}{}

\hypertarget{the-coronavirus-outbreak-}{%
\subsubsection{The Coronavirus Outbreak
›}\label{the-coronavirus-outbreak-}}

\hypertarget{frequently-asked-questions}{%
\paragraph{Frequently Asked
Questions}\label{frequently-asked-questions}}

Updated August 17, 2020

\begin{itemize}
\item ~
  \hypertarget{why-does-standing-six-feet-away-from-others-help}{%
  \paragraph{Why does standing six feet away from others
  help?}\label{why-does-standing-six-feet-away-from-others-help}}

  \begin{itemize}
  \tightlist
  \item
    The coronavirus spreads primarily through droplets from your mouth
    and nose, especially when you cough or sneeze. The C.D.C., one of
    the organizations using that measure,
    \href{https://www.nytimes3xbfgragh.onion/2020/04/14/health/coronavirus-six-feet.html?action=click\&pgtype=Article\&state=default\&region=MAIN_CONTENT_3\&context=storylines_faq}{bases
    its recommendation of six feet} on the idea that most large droplets
    that people expel when they cough or sneeze will fall to the ground
    within six feet. But six feet has never been a magic number that
    guarantees complete protection. Sneezes, for instance, can launch
    droplets a lot farther than six feet,
    \href{https://jamanetwork.com/journals/jama/fullarticle/2763852}{according
    to a recent study}. It's a rule of thumb: You should be safest
    standing six feet apart outside, especially when it's windy. But
    keep a mask on at all times, even when you think you're far enough
    apart.
  \end{itemize}
\item ~
  \hypertarget{i-have-antibodies-am-i-now-immune}{%
  \paragraph{I have antibodies. Am I now
  immune?}\label{i-have-antibodies-am-i-now-immune}}

  \begin{itemize}
  \tightlist
  \item
    As of right
    now,\href{https://www.nytimes3xbfgragh.onion/2020/07/22/health/covid-antibodies-herd-immunity.html?action=click\&pgtype=Article\&state=default\&region=MAIN_CONTENT_3\&context=storylines_faq}{that
    seems likely, for at least several months.} There have been
    frightening accounts of people suffering what seems to be a second
    bout of Covid-19. But experts say these patients may have a
    drawn-out course of infection, with the virus taking a slow toll
    weeks to months after initial exposure. People infected with the
    coronavirus typically
    \href{https://www.nature.com/articles/s41586-020-2456-9}{produce}
    immune molecules called antibodies, which are
    \href{https://www.nytimes3xbfgragh.onion/2020/05/07/health/coronavirus-antibody-prevalence.html?action=click\&pgtype=Article\&state=default\&region=MAIN_CONTENT_3\&context=storylines_faq}{protective
    proteins made in response to an
    infection}\href{https://www.nytimes3xbfgragh.onion/2020/05/07/health/coronavirus-antibody-prevalence.html?action=click\&pgtype=Article\&state=default\&region=MAIN_CONTENT_3\&context=storylines_faq}{.
    These antibodies may} last in the body
    \href{https://www.nature.com/articles/s41591-020-0965-6}{only two to
    three months}, which may seem worrisome, but that's perfectly normal
    after an acute infection subsides, said Dr. Michael Mina, an
    immunologist at Harvard University. It may be possible to get the
    coronavirus again, but it's highly unlikely that it would be
    possible in a short window of time from initial infection or make
    people sicker the second time.
  \end{itemize}
\item ~
  \hypertarget{im-a-small-business-owner-can-i-get-relief}{%
  \paragraph{I'm a small-business owner. Can I get
  relief?}\label{im-a-small-business-owner-can-i-get-relief}}

  \begin{itemize}
  \tightlist
  \item
    The
    \href{https://www.nytimes3xbfgragh.onion/article/small-business-loans-stimulus-grants-freelancers-coronavirus.html?action=click\&pgtype=Article\&state=default\&region=MAIN_CONTENT_3\&context=storylines_faq}{stimulus
    bills enacted in March} offer help for the millions of American
    small businesses. Those eligible for aid are businesses and
    nonprofit organizations with fewer than 500 workers, including sole
    proprietorships, independent contractors and freelancers. Some
    larger companies in some industries are also eligible. The help
    being offered, which is being managed by the Small Business
    Administration, includes the Paycheck Protection Program and the
    Economic Injury Disaster Loan program. But lots of folks have
    \href{https://www.nytimes3xbfgragh.onion/interactive/2020/05/07/business/small-business-loans-coronavirus.html?action=click\&pgtype=Article\&state=default\&region=MAIN_CONTENT_3\&context=storylines_faq}{not
    yet seen payouts.} Even those who have received help are confused:
    The rules are draconian, and some are stuck sitting on
    \href{https://www.nytimes3xbfgragh.onion/2020/05/02/business/economy/loans-coronavirus-small-business.html?action=click\&pgtype=Article\&state=default\&region=MAIN_CONTENT_3\&context=storylines_faq}{money
    they don't know how to use.} Many small-business owners are getting
    less than they expected or
    \href{https://www.nytimes3xbfgragh.onion/2020/06/10/business/Small-business-loans-ppp.html?action=click\&pgtype=Article\&state=default\&region=MAIN_CONTENT_3\&context=storylines_faq}{not
    hearing anything at all.}
  \end{itemize}
\item ~
  \hypertarget{what-are-my-rights-if-i-am-worried-about-going-back-to-work}{%
  \paragraph{What are my rights if I am worried about going back to
  work?}\label{what-are-my-rights-if-i-am-worried-about-going-back-to-work}}

  \begin{itemize}
  \tightlist
  \item
    Employers have to provide
    \href{https://www.osha.gov/SLTC/covid-19/standards.html}{a safe
    workplace} with policies that protect everyone equally.
    \href{https://www.nytimes3xbfgragh.onion/article/coronavirus-money-unemployment.html?action=click\&pgtype=Article\&state=default\&region=MAIN_CONTENT_3\&context=storylines_faq}{And
    if one of your co-workers tests positive for the coronavirus, the
    C.D.C.} has said that
    \href{https://www.cdc.gov/coronavirus/2019-ncov/community/guidance-business-response.html}{employers
    should tell their employees} -\/- without giving you the sick
    employee's name -\/- that they may have been exposed to the virus.
  \end{itemize}
\item ~
  \hypertarget{what-is-school-going-to-look-like-in-september}{%
  \paragraph{What is school going to look like in
  September?}\label{what-is-school-going-to-look-like-in-september}}

  \begin{itemize}
  \tightlist
  \item
    It is unlikely that many schools will return to a normal schedule
    this fall, requiring the grind of
    \href{https://www.nytimes3xbfgragh.onion/2020/06/05/us/coronavirus-education-lost-learning.html?action=click\&pgtype=Article\&state=default\&region=MAIN_CONTENT_3\&context=storylines_faq}{online
    learning},
    \href{https://www.nytimes3xbfgragh.onion/2020/05/29/us/coronavirus-child-care-centers.html?action=click\&pgtype=Article\&state=default\&region=MAIN_CONTENT_3\&context=storylines_faq}{makeshift
    child care} and
    \href{https://www.nytimes3xbfgragh.onion/2020/06/03/business/economy/coronavirus-working-women.html?action=click\&pgtype=Article\&state=default\&region=MAIN_CONTENT_3\&context=storylines_faq}{stunted
    workdays} to continue. California's two largest public school
    districts --- Los Angeles and San Diego --- said on July 13, that
    \href{https://www.nytimes3xbfgragh.onion/2020/07/13/us/lausd-san-diego-school-reopening.html?action=click\&pgtype=Article\&state=default\&region=MAIN_CONTENT_3\&context=storylines_faq}{instruction
    will be remote-only in the fall}, citing concerns that surging
    coronavirus infections in their areas pose too dire a risk for
    students and teachers. Together, the two districts enroll some
    825,000 students. They are the largest in the country so far to
    abandon plans for even a partial physical return to classrooms when
    they reopen in August. For other districts, the solution won't be an
    all-or-nothing approach.
    \href{https://bioethics.jhu.edu/research-and-outreach/projects/eschool-initiative/school-policy-tracker/}{Many
    systems}, including the nation's largest, New York City, are
    devising
    \href{https://www.nytimes3xbfgragh.onion/2020/06/26/us/coronavirus-schools-reopen-fall.html?action=click\&pgtype=Article\&state=default\&region=MAIN_CONTENT_3\&context=storylines_faq}{hybrid
    plans} that involve spending some days in classrooms and other days
    online. There's no national policy on this yet, so check with your
    municipal school system regularly to see what is happening in your
    community.
  \end{itemize}
\end{itemize}

For instance, on the system's longest lines, the final trains of the
night were departing from the first station around 12:19 a.m. and would
not reach the end of the line until 2:09 a.m.

On Wednesday, the final passenger train to leave the system --- a
southbound No. 2 train --- arrived at its last station at 2:12 a.m.,
four minutes late.

``Not bad, not bad, yes it's a little tardy,'' Mr. Zamora said.

By then, riders who were not already on the subway were trying to find
other ways to get to work. Many turned to the bus network, which has
been bolstered to accommodate the roughly 11,000 riders who have relied
on late night subway service in recent weeks.

Image

Outside the closed Flatbush Avenue station in Brooklyn, Earla George
waited for a ride to her job as a certified nursing assistant at Staten
Island Hospital.Credit...Todd Heisler/The New York Times

The transit agency has added 1,168 nightly bus trips --- a 76 percent
increase from the usual nightly bus schedules --- and put 344 additional
buses on the road, officials said. Buses were expected to arrive at
stops every 15 or 30 minutes, similar to recent train schedules.

William Rodriguez, a lab tech at NewYork-Presbyterian/The Allen Hospital
in northern Manhattan, was waiting to catch a northbound M15 bus from
the Lower East Side when the bus went right by his stop.

He was saved, he said, by a police officer who saw what happened and
gave him a ride in his cruiser. They overtook the bus and Mr. Rodriguez
was able to board at another stop.

``It was that or go back home,'' said Mr. Rodriguez, who normally takes
two trains to get to work. ``I never use the bus, always the trains,''
he added.

Around the same time, Yussef Said was waiting for a southbound M15 bus
from his Upper East Side neighborhood. He had started working a 6 a.m.
shift at an Amazon warehouse on Staten Island a month ago and had
counted on catching the No. 6 train to the Staten Island Ferry --- a
plan that was upended by the subway shutdown.

Image

Yussef Said on the M15 bus during his commute to an Amazon supply
warehouse on Staten Island.Credit...Hilary Swift for The New York Times

On Wednesday morning, his goal was to make a 4:30 a.m. ferry.

``I have to catch anything I can,'' Mr. Said, 40, said. ``I want to
work.''

By the time he reached the ferry, the subway system was beginning to
come back to life. The crews of cleaners in Tyvek suits, shoe coverings,
face masks and surgical gloves had swept through trains as they rolled
into stations, mopping cars with cleanser before sending them to the
train yards for more intense disinfection.

At 5 a.m. caution tape still hung across entrances to the Jay
Street-Metro Tech subway station in Brooklyn.

Allan Harris, 61, a mason from Red Hook, Brooklyn, was en route to a 7
a.m. construction job in Far Rockaway, Queens. The shutdown had already
forced him to take a bus to reach the station.

``I catch the 61, took it to come here,'' he said referring to bus line,
``but I need to move from here now, you know?'' As he stood behind the
tape, the sound of a train could be heard and with that, Mr. Allan
maneuvered over the tape and was on his way.

Image

Riders stepped over a chain at the Court Square station in
Queens.Credit...Victor J. Blue for The New York Times

Victor Blue, Jonah Markowitz, Azi Paybarah, Sean Piccoli and Nate
Schweber contributed reporting.

Advertisement

\protect\hyperlink{after-bottom}{Continue reading the main story}

\hypertarget{site-index}{%
\subsection{Site Index}\label{site-index}}

\hypertarget{site-information-navigation}{%
\subsection{Site Information
Navigation}\label{site-information-navigation}}

\begin{itemize}
\tightlist
\item
  \href{https://help.nytimes3xbfgragh.onion/hc/en-us/articles/115014792127-Copyright-notice}{©~2020~The
  New York Times Company}
\end{itemize}

\begin{itemize}
\tightlist
\item
  \href{https://www.nytco.com/}{NYTCo}
\item
  \href{https://help.nytimes3xbfgragh.onion/hc/en-us/articles/115015385887-Contact-Us}{Contact
  Us}
\item
  \href{https://www.nytco.com/careers/}{Work with us}
\item
  \href{https://nytmediakit.com/}{Advertise}
\item
  \href{http://www.tbrandstudio.com/}{T Brand Studio}
\item
  \href{https://www.nytimes3xbfgragh.onion/privacy/cookie-policy\#how-do-i-manage-trackers}{Your
  Ad Choices}
\item
  \href{https://www.nytimes3xbfgragh.onion/privacy}{Privacy}
\item
  \href{https://help.nytimes3xbfgragh.onion/hc/en-us/articles/115014893428-Terms-of-service}{Terms
  of Service}
\item
  \href{https://help.nytimes3xbfgragh.onion/hc/en-us/articles/115014893968-Terms-of-sale}{Terms
  of Sale}
\item
  \href{https://spiderbites.nytimes3xbfgragh.onion}{Site Map}
\item
  \href{https://help.nytimes3xbfgragh.onion/hc/en-us}{Help}
\item
  \href{https://www.nytimes3xbfgragh.onion/subscription?campaignId=37WXW}{Subscriptions}
\end{itemize}
