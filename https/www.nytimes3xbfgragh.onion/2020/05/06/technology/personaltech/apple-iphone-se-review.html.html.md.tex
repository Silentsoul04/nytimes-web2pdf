Sections

SEARCH

\protect\hyperlink{site-content}{Skip to
content}\protect\hyperlink{site-index}{Skip to site index}

\href{https://www.nytimes3xbfgragh.onion/section/technology/personaltech}{Personal
Tech}

\href{https://myaccount.nytimes3xbfgragh.onion/auth/login?response_type=cookie\&client_id=vi}{}

\href{https://www.nytimes3xbfgragh.onion/section/todayspaper}{Today's
Paper}

\href{/section/technology/personaltech}{Personal Tech}\textbar{}Apple
iPhone SE Review: A Superb Smartphone for a Humble Price

\href{https://nyti.ms/3bgTu0F}{https://nyti.ms/3bgTu0F}

\begin{itemize}
\item
\item
\item
\item
\item
\end{itemize}

Advertisement

\protect\hyperlink{after-top}{Continue reading the main story}

Supported by

\protect\hyperlink{after-sponsor}{Continue reading the main story}

tech fix

\hypertarget{apple-iphone-se-review-a-superb-smartphone-for-a-humble-price}{%
\section{Apple iPhone SE Review: A Superb Smartphone for a Humble
Price}\label{apple-iphone-se-review-a-superb-smartphone-for-a-humble-price}}

For \$399, this smartphone hits the high notes: speedy, a great camera
and a nice screen. Took long enough, didn't it?

\includegraphics{https://static01.graylady3jvrrxbe.onion/images/2020/05/07/business/06techfix/06techfix-articleLarge.jpg?quality=75\&auto=webp\&disable=upscale}

\href{https://www.nytimes3xbfgragh.onion/by/brian-x-chen}{\includegraphics{https://static01.graylady3jvrrxbe.onion/images/2018/02/16/multimedia/author-brian-x-chen/author-brian-x-chen-thumbLarge.jpg}}

By \href{https://www.nytimes3xbfgragh.onion/by/brian-x-chen}{Brian X.
Chen}

\begin{itemize}
\item
  May 6, 2020
\item
  \begin{itemize}
  \item
  \item
  \item
  \item
  \item
  \end{itemize}
\end{itemize}

\href{https://www.nytimes3xbfgragh.onion/es/2020/05/08/espanol/ciencia-y-tecnologia/iphone-SE.html}{Leer
en español}

Apple's
\href{https://www.nytimes3xbfgragh.onion/2020/04/15/technology/personaltech/apple-new-iphone-se.html}{recently
introduced \$399 iPhone SE} marks a turning point in consumer
technology. It's a smartphone that delivers all the tech that we care
about, without making us pay through the nose for it.

Many of us have been waiting for this moment. Long ago, the technologies
driving TVs and personal computers became so commonplace that good
television sets and PCs became affordable for the masses. The ubiquitous
smartphone, we presumed, would follow.

Instead, as the smartphone matured over the last decade and a bit, the
opposite happened. The
\href{https://www.nytimes3xbfgragh.onion/article/iphone-costs.html}{price}
tag for the
\href{https://www.nytimes3xbfgragh.onion/article/iphone-costs.html}{iPhone},
the most popular handset in the world, reached heights that were
previously unimaginable. Last year's new iPhones peaked at \$1,449,
compared with \$599 in 2007. Yet
\href{https://www.nytimes3xbfgragh.onion/2018/02/28/technology/personaltech/cheaper-smartphone.html}{budget
phones} ranging from \$200 to \$400 had major shortcomings like lousy
cameras and slow chips.

The new iPhone SE's lack of compromise is what makes it remarkable.
Apple took all the best parts from its expensive iPhones --- including a
fast computing processor and an excellent camera --- and squeezed them
into the shell of an older iPhone with a home button and smaller screen.
At the same time, it managed to include useful features that were
previously exclusive to fancy new phones, like water resistance,
wireless charging and so-called portrait photos.

That means state-of-the-art smartphone technology has finally come down
to a modest price. It's about time.

After testing the new SE for a few weeks, I can confidently say that
this device is ideal for many people --- especially for those who think
about buying tech only when they feel they have to.

Justin Adler, 33, is one of them. He owned the first SE, which debuted
in 2016, for years, subjecting himself to mockery from his techie
colleagues in San Francisco who had much nicer phones. He recently
bought the latest SE.

``I just never wanted to shell out \$1,000 to replace something that was
working perfectly fine,'' he said. ``I was the exact core audience of,
if you haven't upgraded your phone we're going to give you the cheapest
bait as possible.''

This new iPhone may suit you as well. Here's what you need to know.

\hypertarget{zooming-in-on-the-camera}{%
\subsection{Zooming in on the camera}\label{zooming-in-on-the-camera}}

With the SE, Apple took the computing processor of
\href{https://www.nytimes3xbfgragh.onion/2019/09/17/technology/personaltech/iphone-11-review.html}{the
\$699 iPhone 11} --- the fastest on the market --- and stuck it in a
body that is practically the same as the iPhone 8 from 2017. In the
process, the company also made significant improvements to the SE's
camera, which has a single lens but now relies on software and
artificial intelligence to make photos look much better.

In past iterations of high-end iPhones, the devices had two camera
lenses that worked together to produce the popular effect known as
portrait mode, which sharpens a subject while gently blurring the
background.

But in the
\href{https://www.nytimes3xbfgragh.onion/2018/10/23/technology/personaltech/apple-iphone-xr-review.html}{iPhone
XR from 2018}, Apple used machine learning, which involved computers
analyzing images to recognize people and properly sharpen them while
blurring the background, to produce portrait shots of people using a
single lens.

Apple has applied a similar A.I.-assisted approach to the SE, making
portrait photos of people possible with a single lens on the cheaper
device. In my tests, portrait shots looked excellent --- on a par with
similar photos taken with the iPhone 11. This is significant for a
device of this price.
(\href{https://www.nytimes3xbfgragh.onion/2019/05/07/technology/personaltech/pixel-3a.html}{Google's
\$399 Pixel 3A}, my favorite Android phone, is also exceptional for its
ability to take portrait photos, though that phone is much slower.)

Of course, Apple had to cut costs somewhere, so the cheaper iPhone SE's
camera lacks some frills seen in the iPhone 11 and \$999 iPhone 11 Pro.

Specifically, Apple limited its machine-assisted image processing
specifically to human subjects, meaning I couldn't take artsy photos of
my dogs. The camera also lacks the so-called ultra-wide-angle lens for
taking shots with a broader field of view, as well as night mode, a
feature to take better photos in the dark.

Image

A daylight photo taken with an iPhone SE.Credit...Brian X. Chen

Image

A daylight photo taken with an iPhone 11.Credit...Brian X. Chen

But if you are a casual photographer, you could probably live without
those whiz-bang features and be happy saving lots of money by choosing
the SE.

For what it's worth, the shots I took in daylight of my corgi, Max, on
the SE looked just as good as similar shots with the iPhone 11 and other
phones on the market, like Google's \$399 Pixel 3A. They came out crisp
with natural-looking colors and nice shadow detail.

\hypertarget{lets-talk-about-that-screen-and-home-button}{%
\subsection{Let's talk about that screen and home
button}\label{lets-talk-about-that-screen-and-home-button}}

The other feature that makes the SE cheaper is the first thing you will
look at: the screen.

At 4.7 inches, the display is smaller than the jumbo 6.1-inch screen on
the iPhone 11. But that may be a benefit. The 4.7-inch size is better
suited to one-hand use, so it's easy to use your thumb to reach from the
home button to each corner of the screen. Also important, the phone's
smaller body fits more comfortably in a pocket.

The SE's second big cost saver is the use of a home button for unlocking
the device, rather than the face scanners seen on modern smartphones.
The iPhone fingerprint scanners have always worked quickly and reliably,
and so does this one.

It might be nice to have a face scanner, but many of us will probably be
happy without the feature if it means saving some money. For comparison,
I have given big minus points to \$1,000 phones, like the
\href{https://www.nytimes3xbfgragh.onion/2019/02/27/technology/personaltech/samsung-galaxy-s10-review.html}{Samsung
Galaxy S10}, for a lousy fingerprint reader and a subpar face scanner.

\hypertarget{putting-the-iphone-ses-value-in-perspective}{%
\subsection{Putting the iPhone SE's value in
perspective}\label{putting-the-iphone-ses-value-in-perspective}}

To fully understand the SE's value, it's important to note that this
isn't Apple's first rodeo with a cheaper iPhone. Those past models were
not as compelling when compared with their pricier counterparts. To wit:

\begin{itemize}
\item
  The \$549
  \href{https://www.nytimes3xbfgragh.onion/2013/09/11/technology/apple-shows-off-2-new-iphones-one-a-lower-cost-model.html}{iPhone
  5C}, introduced in 2013, came in colorful plastic and felt cheaper and
  slower than the iPhone 5S, the model with a sleek aluminum body that
  cost \$100 more.
\item
  The
  \$399\href{https://www.nytimes3xbfgragh.onion/2016/03/24/technology/personaltech/who-will-like-the-new-smaller-iphone-se.html}{iPhone
  SE from 2016} had the design of older iPhones, with the same computing
  power as newer ones at the time. Yet that SE had a notably inferior
  camera and dimmer screen than its \$649, bigger-screen counterpart at
  the time, the
  \href{https://www.nytimes3xbfgragh.onion/2015/09/24/technology/personaltech/testing-iphone-6s-3d-touch-and-live-photos-features.html?mtrref=www.nytimes3xbfgragh.onion\&gwh=C46D33577D43FB58BDB34B5FA60A3C9E\&gwt=pay\&_r=0}{iPhone
  6S}.
\end{itemize}

This time, the new SE's trade-offs seem trivial. No face scanner,
shooting photos in the dark or humongous screen? Those are minor
inconveniences when you are paying 40 percent less than for an iPhone
11.

I will note one big downside: The SE has significantly shorter battery
life than the iPhone 11.

After a day of shooting photos and juggling work tasks, the SE battery
needed to be replenished by dinnertime. With the more expensive phone, I
had more than 25 percent battery by bedtime. So people who work long
hours and rely on their phones would probably be happier with a high-end
one.

\hypertarget{so-who-is-the-expected-buyer}{%
\subsection{So who is the expected
buyer?}\label{so-who-is-the-expected-buyer}}

I'm not the target demographic for the SE since I'm willing to pay for
fancier features. I splurged on the
\href{https://www.nytimes3xbfgragh.onion/2018/09/18/technology/personaltech/iphone-xs-max-review.html}{\$999
iPhone XS} two years ago because I loved taking portrait photos of food
and my dogs.

But there was something unique about the announcement of the SE that I
found striking in a pandemic that has dampened most people's enthusiasm
for buying nice things.

When new iPhones are unveiled, I usually get questions from friends who
work in tech and are giddy about shiny new gadgets. With the SE, I got
text messages from people who hate talking about tech: a friend who is
in her 60s and a self-proclaimed Luddite, and a family member who is an
environmentalist. Both are using iPhones that are at least five years
old, and both were relieved that their next phone won't cost more than a
month's mortgage.

``It's a slow burner that's going to help Apple upgrade their base,''
Carolina Milanesi, a consumer tech analyst for the research firm
Creative Strategies, said of the SE. That includes people who hold on to
phones for three to five years and those with hand-me-downs, she said.

Come to think of it, that's most people I know.

Advertisement

\protect\hyperlink{after-bottom}{Continue reading the main story}

\hypertarget{site-index}{%
\subsection{Site Index}\label{site-index}}

\hypertarget{site-information-navigation}{%
\subsection{Site Information
Navigation}\label{site-information-navigation}}

\begin{itemize}
\tightlist
\item
  \href{https://help.nytimes3xbfgragh.onion/hc/en-us/articles/115014792127-Copyright-notice}{©~2020~The
  New York Times Company}
\end{itemize}

\begin{itemize}
\tightlist
\item
  \href{https://www.nytco.com/}{NYTCo}
\item
  \href{https://help.nytimes3xbfgragh.onion/hc/en-us/articles/115015385887-Contact-Us}{Contact
  Us}
\item
  \href{https://www.nytco.com/careers/}{Work with us}
\item
  \href{https://nytmediakit.com/}{Advertise}
\item
  \href{http://www.tbrandstudio.com/}{T Brand Studio}
\item
  \href{https://www.nytimes3xbfgragh.onion/privacy/cookie-policy\#how-do-i-manage-trackers}{Your
  Ad Choices}
\item
  \href{https://www.nytimes3xbfgragh.onion/privacy}{Privacy}
\item
  \href{https://help.nytimes3xbfgragh.onion/hc/en-us/articles/115014893428-Terms-of-service}{Terms
  of Service}
\item
  \href{https://help.nytimes3xbfgragh.onion/hc/en-us/articles/115014893968-Terms-of-sale}{Terms
  of Sale}
\item
  \href{https://spiderbites.nytimes3xbfgragh.onion}{Site Map}
\item
  \href{https://help.nytimes3xbfgragh.onion/hc/en-us}{Help}
\item
  \href{https://www.nytimes3xbfgragh.onion/subscription?campaignId=37WXW}{Subscriptions}
\end{itemize}
