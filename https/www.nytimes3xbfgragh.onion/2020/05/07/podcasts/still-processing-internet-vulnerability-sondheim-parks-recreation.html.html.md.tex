Sections

SEARCH

\protect\hyperlink{site-content}{Skip to
content}\protect\hyperlink{site-index}{Skip to site index}

\href{https://www.nytimes3xbfgragh.onion/spotlight/podcasts}{Podcasts}

\href{https://myaccount.nytimes3xbfgragh.onion/auth/login?response_type=cookie\&client_id=vi}{}

\href{https://www.nytimes3xbfgragh.onion/section/todayspaper}{Today's
Paper}

\href{/spotlight/podcasts}{Podcasts}\textbar{}Does This Phone Make Me
Look Human?

\url{https://nyti.ms/3cd7olH}

\begin{itemize}
\item
\item
\item
\item
\item
\end{itemize}

Advertisement

\protect\hyperlink{after-top}{Continue reading the main story}

transcript

Back to Still Processing

bars

0:00/0:00

-0:00

transcript

\hypertarget{does-this-phone-make-me-look-human}{%
\subsection{Does This Phone Make Me Look
Human?}\label{does-this-phone-make-me-look-human}}

\hypertarget{hosted-by-wesley-morris-and-jenna-wortham-produced-by-hans-buetow}{%
\subsubsection{Hosted by Wesley Morris and Jenna Wortham. Produced by
Hans
Buetow.}\label{hosted-by-wesley-morris-and-jenna-wortham-produced-by-hans-buetow}}

\hypertarget{the-internet-is-bringing-us-closer-together--but-will-the-intimacy-last}{%
\paragraph{The internet is bringing us closer together --- but will the
intimacy
last?}\label{the-internet-is-bringing-us-closer-together--but-will-the-intimacy-last}}

Thursday, May 7th, 2020

\begin{itemize}
\item
  {[}music{]}
\item
  jenna wortham\\
  I'm Jenna Wortham.
\item
  wesley morris\\
  I'm Wesley Morris. We're two culture writers at The New York Times.
\item
  jenna wortham\\
  This is ``Still Processing.''

  So last weekend, beautiful weather in Brooklyn. Just crisp, sunny, the
  sweetest breeze. And ---
\item
  wesley morris\\
  You just want to stick a straw in it.
\item
  jenna wortham\\
  You just want to drink it up. So I got my straw, and I decided to go
  for a bike ride. And I had a mask on, of course. I had all the proper
  gear. I went with my best friend, Kimberly, and we rode around
  Brooklyn. It just --- it feels great after being stuck in the house,
  to be able to navigate your neighborhood and see things you haven't
  seen before. As lovely as the ride was, it was really hard to see that
  a lot of folks were not covering their faces, were not even attempting
  to socially distance. And you know, when you're trying to do the right
  thing, it's really hard to see a lot of people who really don't care
  about doing the right thing. And what does that mean? Also, how the
  decisions you're making are going to impact me and my people. And when
  I got home, I just had all these pent-up emotions. And you know,
  moving is medicine. Dancing is medicine. Shaking is medicine. So I
  remembered that there's this queer collective called BUFU, which is By
  Us For Us. Very cute, very P.O.C. I remembered that they were having a
  rave, a 24-hour rave that was supposed to last all weekend. And they
  had a really stellar lineup ---
\item
  wesley morris\\
  What?
\item
  jenna wortham\\
  --- of D.Js. Yes. I mean, it was the function. The fun-s-ion. It was a
  Zoom call basically. It was a private Zoom call. You had to know the
  code to get into the room. And they have had a party before, but I
  didn't go because I wasn't really up for it.
\item
  wesley morris\\
  But you know you like to move it, move it.
\item
  jenna wortham\\
  I love to move it, move it. And again, I had all this energy, all this
  adrenaline from the bike ride. I was like, I got this. I decided to
  prop the computer up in my kitchen, because the lighting was great in
  there and I wanted to make some food.
\item
  wesley morris\\
  {[}LAUGHS{]}
\item
  jenna wortham\\
  And so I logged in, and just like the first thing I see is somebody's
  ass, just like shaking onscreen. Green silk, bright sun. And I was
  like, I'm home. This is where I need to be. It's a Zoom call, right?
  We've all been on those at this point for work at some point. But a
  Zoom party is totally different, because the host is a D.J. And all
  the little windows are occupied by people partying, whether it's in
  their backyard, whether it's in their bedroom. I even saw somebody who
  had a stripper pole installed in their living room. I need to find out
  who that is so I can go to that house when all this is over. It was
  the best of both worlds for me. I got to chop my greens. I got to make
  a little stir fry. And I got to twerk in my kitchen. I mean, it was
  all wrapped up in one. And it was great, because all the kind of anger
  and the frustration that I felt at the world at people not being
  properly covered up melted away, and it was replaced by this vibrant
  beauty of just seeing people trying to do the right thing, of trying
  to stay safe, keep themselves healthy. And I felt all the emotions
  inside of me kind of bubbling up into my eyes, and they wanted to come
  out. And I was like on the verge of tears. I kept, like, backing it up
  into my fridge and then dropping it low, kind of letting it wiggle on
  the camera. But I was crying into the pickles, the homemade pickles
  and kombucha, because I just didn't want --- I didn't want to have to,
  like, explain what was happening with my face. But I don't know,
  Wesley, you know me. I'm a pretty private person. I keep a lot of my
  private life to myself. I don't post pictures of my home. But in this
  call, I mean, I was like --- at one point I was stir-frying some
  broccoli, and I had a knee on the countertop just bouncing along,
  because the Megan Thee Stallion ``Savage'' remix with Beyoncé came on.
  And I was like a hundred percent, yes, this deserves more. This
  deserves more than just me stirring a pot, you know?
\item
  wesley morris\\
  You were on a camera literally putting your foot in it.
\item
  jenna wortham\\
  I was really putting my foot in it. And I continued to do so for the
  rest of the day. It was a moment of connection that I didn't know that
  I needed. These were not my immediate circle. These were not my
  closest friends. These were people whose faces I would normally see at
  the local coffee shop or I would see at a reading or I might see at a
  screening. And I haven't really thought about most of them, to be
  honest with you. I've been too consumed with the day-to-day of my own
  life and checking in on you and the people in my world. It was really
  affirming, and it really helped me reinforce my own commitment to
  being part of this broader public health moment, my own desires to do
  the right thing. And I just --- that is what community looks like
  right now. It's actually not seeing friends of friends of friends of
  friends, former co-workers, former lovers, lovers of their lovers out
  in the world. It looks like being alone together through a screen. And
  I really didn't know that that's something that I was missing. I
  didn't even know that that was a possibility. I didn't even know that
  was on offer in this way.
\item
  wesley morris\\
  No, because you go out.
\item
  jenna wortham\\
  Because I'm out, honey. I'm in these streets, more than I need to be.
\item
  wesley morris\\
  More than you need to be sometimes. I've seen it. I've seen it.
\item
  jenna wortham\\
  {[}LAUGHS{]} So that's my experience, Wesley, which is --- I feel like
  we're going to have different experiences of this. And so I'm curious
  for you, are you missing anything now that every interaction,
  everything is becoming facilitated by the videophone?
\item
  wesley morris\\
  Ugh. Uh, yeah! I mean, I miss it a lot. I've lived in New York for
  five years, six years? And this whole time, I've been going to the
  theater almost --- maybe once a week, sometimes two or three times a
  week. Sometimes four times a week.
\item
  jenna wortham\\
  You do.
\item
  wesley morris\\
  And I love going to the theater. It is a luxury and a privilege, and I
  can't believe I get to do it, even when the show is the worst. And I
  have seen some bad shows!
\item
  jenna wortham\\
  You've taken me to some of those bad shows!
\item
  wesley morris\\
  I've taken several people to some of those bad shows, and I apologize
  to everybody. But I really, really miss going to the theater. I miss
  sitting in those uncomfortable seats. I miss the randomness of who you
  might be seated next to. I mean, it could be some woman from
  Parsippany, N.J. It could be the vice president. You just don't know.
  And then nobody knows what's going to happen once the show starts. Is
  somebody's phone going to ring? You don't know! I miss all of the
  serendipity and randomness of live theater. And lo and behold, a
  miracle happened. Early last week, Stephen Sondheim --- it's his 90th
  birthday year. Stephen Sondheim is one of America's great artists,
  period. He is the author of numerous musicals that we love and
  appreciate. ``Into the Woods'' is his. ``Sweeney Todd'' is his. He is
  partially responsible for the greatness that is ``West Side Story.''
  He is the definition of the American musical theater to many, many of
  us. And somebody had the great idea that instead of coming out before
  the curtain of a revival of one of his musicals that should be on
  Broadway right now and saying, ``Happy birthday, Stephen Sondheim. We
  love you.'' They got everybody who's anybody in musical theater to
  come and sing individual tributes to Stephen Sondheim from their
  living rooms, their backyards, their bathrooms. The event itself was
  called Take Me to the World. And it's just everybody comes out to
  sing. Lin-Manuel Miranda is there. Christine Baranski, Meryl Streep,
  Audra McDonald, Mandy Patinkin, Bernadette Peters, Patti LuPone, Lea
  Salonga. And doing one of my favorite Sondheim songs --- one of the
  best Sondheim songs --- ``Someone in a Tree'' from ``Pacific
  Overtures,'' Ann Harada, Austin Ku, Kelvin Moon Loh, and Thom Sesma.
  And the thing that I loved the most, and didn't realize I would love
  most about it, is something that you can't get from live theater, that
  you can only get in this particular moment, which is when all the
  musicians start the overture for this two-and-a-half hour event.
\item
  music --- ``someone in a tree"\\
  {[}OVERTURE{]}
\end{itemize}

wesley morris

It's edited so that all the musicians are in their respective homes
playing their instruments, and you just get to see like this drummer's
house. You get to see the fingernail polish on some of the musicians. I
got teary just looking at all this stuff. The French horn lady got a
closeup! The French horn lady, when you go to see a Broadway musical,
you can't even see her because she's under the stage.

jenna wortham

{[}LAUGHS{]}

wesley morris

The idea that you're --- they're paid tribute to for like the opening 10
minutes of this thing? It just really, really moved me.

\begin{itemize}
\tightlist
\item
  music --- ``someone in a tree"\\
  {[}OVERTURE{]}
\end{itemize}

jenna wortham

So the night the Sondheim tribute's happening, I was unaware, but I
started seeing this screenshot of Christine Baranski, Audra McDonald,
and of course the queen, Meryl Streep, floating through my Twitter
stream. And I'm like, what are these ladies up to?

\begin{itemize}
\tightlist
\item
  archived recording (singing)\\
  Here's to the ladies who lunch. Everybody laugh.
\end{itemize}

jenna wortham

And it was so funny to see like Christine Baranski, Meryl Streep, and
Audra McDonald pretty gritty looking with their drinks. It's not clear
if they're drinking, but they're all miming that they're drinking
alcohol. Christine Baranski was definitely drinking wine. And they're
all just like --- nobody's hair is done. Nobody really seems to have a
ton of makeup on. Everybody's in their favorite hotel bathrobe that
they've stolen. {[}LAUGHS{]} Probably. And they're all kind of just
commiserating. And I was like, wait, that looks like my Friday night
happy hour Zoom. Like, that is what we do. It made me care a lot more
about those famous people and their lives and their inner lives than I
normally ever do --- other than Meryl Streep, who really is just, you
know, for me the GOAT in so many ways for so many people. It was really
evocative.

\begin{itemize}
\item
  archived recording (singing)\\
  Another reason not to move. Another vodka stinger.

  I'll drink to that.
\end{itemize}

jenna wortham

And I also can't stop thinking about the bookcase that was behind Meryl
Streep, because when reporters have to phone in for live broadcasts or
they want to appear on news shows, a lot of them are sitting in front of
their, like, beautifully arranged, highly decorated shelves full of
highly lauded and esteemed authors, like these mountainous towers of
books behind them. I guess it makes you seem ---

wesley morris

Yes, much has been made of this.

jenna wortham

Much has been made about this. Our colleague Amanda Hess has written
about it. I guess it makes you seem more studious. But Meryl's
bookshelves. Empty. I was like, what kind of larder is this? There's
nothing on these shelves. But there was a hand-carved --- well, it
looked hand-carved --- there was one lone duck or egret, I don't know
what kind of bird it was, just perched on the shelf behind her. And I
was like, you know, Meryl? I didn't think I could love you more, but now
I'm head over heels in love with you because you just left this crooked
souvenir from a Cape Cod trip that you probably don't even remember just
chilling on your bookshelves. She didn't dress it up with any fancy
cutlery or dishware or crystals. Who knows what she has? She probably
has gems. She probably has gems she could have put on that bookshelf.
No, she left the duck. And it just felt so human. Which for most
celebrities, that's usually the last thing they want to appear. They
don't want to turn the human dial up too much. They only want to give
you as much human as they think you can handle. Whereas in this special,
the dial was broken. It was just cranked all the way up.

wesley morris

The other amazing thing about that moment with Audra McDonald and
Christine Baranski and Meryl Streep was when they cut to a wide
three-shot where it became basically a Zoom call with music. On each
person's screen they say, like, C.B.`s phone for Christine Baranski.

jenna wortham

Yes, yes, yes! So-and-so's iPad.

wesley morris

{[}LAUGHS{]}

I just, I love the jankiness or the averageness of --- I mean, of
course. These are human beings! Like, what else is Meryl Streep's screen
supposed to say besides M.S.`s phone? It wasn't going to say, like,
three-time Oscar winner, most nominations in the history of the Academy
Awards's phone. No! It's going to say M.S., because that's all she is,
is M.S. She's Meryl Streep. Just a person who wears a bathrobe and walks
around and breathes air just like everybody else --- with a mask on,
hopefully. And I am often astonished that like, why don't I give myself
a better name than Wesley Morris when I'm on these calls? Why am I not
like Frenchie Laroucho, something like that?

jenna wortham

I've been changing mine up. But I'll tell you the risk with that, OK? My
rave name was Jenny Deluxxx with three X's. And then I went into a
meetup with my sound healers and I was like, Oh, just a moment,
everybody. Everyone was like, What have you been up to? And I was like,
(WHISPERING) Don't worry about it. Don't worry about it. So you know,
there's upsides and there's downsides. Anyway.

wesley morris

So I'll just stay --- if Meryl Streep can just be M.S.`s iPhone, if she
can just be M.S., I'll just be Wesley Morris. That's good enough for me.

jenna wortham

You know, Wesley, I also noticed the little device tags in the corner of
everybody's screens. And I just thought, wow, you know, we're really all
operating under a very similar set of circumstances right now. And I
want to be very clear, I'm not saying that Covid-19 is a great
equalizer. Absolutely not. This virus is hitting marginalized
communities and black communities, native communities, indigenous
communities, Latinx communities, way harder than any other community.
And so in that sense, we're not all dealing with this in the same way.
But I do think that something else has happened with people who are now
comporting their lives to the rectangular screens that they live on.
There is something that's like a great flattening that's happening.
There's a context collapse, right? Because we're all forced pretty much
to use the same several pieces of technology, whether it's your
smartphone, an iPod, your laptop, Zoom, Google Hangouts, FaceTime.
Everything is basically being funneled through those portals, which
means that it all ends up looking the same. Now, this is something
that's been happening for a much longer time, obviously, on social media
on things like TikTok, Instagram, YouTube. I'm thinking a lot about the
Don't Rush Challenge on TikTok, Wesley, which I feel like you've
probably seen but maybe didn't know it was called that. But it's like
you and your crew, whatever that means to you, basically what happens
--- one person starts. And they look down at their clothes and they're
kind of like, ugh. They throw something at the camera, and then the next
shot is them looking really fine. And then they pass the baton to the
next person, who you know --- and these things are stitched digitally
together. And so what ends up happening is you get kind of the same
effect that you had with Meryl, Audra, and Christine. But it just has
been happening in a much broader scale on TikTok for much longer.

wesley morris

Yeah, that just sounds like TikTok. It sounds like TikTok magic.

jenna wortham

It's just TikTok magic. But I guess to say, it's really interesting to
see basically anyone who's trying to make cultural content for the web
reduced to the exact same mechanisms and formats, which is ---

wesley morris

Yes, yes, yes!

jenna wortham

{[}LAUGHS{]} It's actually a really remarkable thing, because it's very
ephemeral. It's limited. People who have made their life's work social
media, that invitation into personal and private space has always been
part of the deal, because that's all you had to work with. And so that
kind of intimacy was par for the course. And it was great. I love seeing
kids' closets open in the background of their YouTube videos and all
their clothes and toys spilling out. I mean, it's one of my favorite
things about social media. Or that kids in Houston all live in those
same kind of cul-de-sacs and those same tract houses with those huge
double-car wide driveways that they dance in. I mean, there's something
about seeing those elements of similarity and those moments of sameness
that make me feel deeply connected to whatever a kid in Ohio is doing.
And I think seeing that transferal happening to pretty much anyone who
wants to be present right now in this moment under quarantine, they have
to, to some degree, allow us in, allow us to see them being vulnerable.
And sometimes it's just as simple as a messed up hair --- I mean, Meryl
Streep's hair was so messed up and I will never love her more because of
that. Like, I just ---

wesley morris

I was gonna say, you sound concerned!

jenna wortham

No, I --- it's not concern. It's admiration, because it takes a lot to
let people see you disheveled. It takes a lot to let people see the
messy bits, you know? Even if they're a little bit curated, that's a
real moment of tenderness, and trust, honestly, that people aren't going
to be like, Gross, she looks disgusting. But just to be like, Those
pillow creases on my face, that's humanity. That's humanity that you're
looking at.

wesley morris

It's life.

jenna wortham

It's life! It's real frickin' life.

wesley morris

And what I will say is, when you make this great point about how we're
all TikTokkers and YouTubers now, one of my favorite things ---

jenna wortham

We are.

wesley morris

--- about the Sondheim tribute was like right almost in the middle, was
a little tiny bit by Randy Rainbow. He is one of the most brilliant
people you are ever going to see do anything, and he lives on YouTube.

jenna wortham

OK.

wesley morris

And essentially, what he does is he takes some kind of fake background,
lots of editing, and his beautiful voice, and he does showtune satires
where he takes classic songs and he rejiggers them to comment on the
moment. So in the middle of the Stephen Sondheim tribute are like two
minutes of Randy Rainbow doing ``By the Sea'' from ``Sweeney Todd.''

\begin{itemize}
\tightlist
\item
  archived recording (randy rainbow)\\
  But a seaside wedding could be devised, me rumpled bedding
  legitimized.
\end{itemize}

wesley morris

This is where Mandy Patinkin and Bernadette Peters, they've all come to
Randy Rainbow's world.

\begin{itemize}
\tightlist
\item
  archived recording (randy rainbow)\\
  Oh, I can just see us now in our bathing dresses, you respecting
  social distancing guidelines from six feet away and me in a facemask I
  made from my own underwear.
\end{itemize}

wesley morris

He might not make sense on Broadway. But he makes sense in this world.
And he is the person who seems most natural in it, more natural than
Patti LuPone, more natural than Meryl Streep and Christine Baranski and
Audra McDonald. He knows how to use this space to bring a song to life.

\begin{itemize}
\tightlist
\item
  archived recording (randy rainbow)\\
  Happy birthday, Mr. Sondheim. I'm available.
\end{itemize}

jenna wortham

Honestly though, Wesley, it's going to be really, really, really
interesting to see if our movies and our TV shows also start to catch up
to this new reality and start to look like the way we're living now. And
we're actually already seeing it a little bit. For example, that episode
of ``Parks and Rec,'' the special reunion episode that you reminded me
to watch. It aired last week. And I'm so glad we watched it. So let's
take a break, and we'll come back and we will get into it.

{[}music{]}

wesley morris

When it started in 2009, ``Parks and Recreation'' was this show about a
fictional town called Pawnee, Ind., and the cast of characters who
worked there, centered around one Leslie Knope. And she, when the show
starts, is the deputy director of the Parks and Recreation Department of
Pawnee. And it was an office comedy, essentially. And I think the thing
that bugged me about ``Parks and Recreation'' when it was actually on
NBC was that it seemed tonally and, to the degree that it was
ideological at all, redundant.

jenna wortham

Hmm.

wesley morris

I felt like that show during the Obama era just felt like it was riding
the coattails of the mood of the country in a way. I watched it for most
of its run. I've seen almost every one of its 125 episodes. I didn't
love it the way other people who watched this show religiously loved
this show. Did you watch the show?

jenna wortham

I did. I watched it off and on. But I was really interested when they
announced last week there would be a comeback special where they would
reunite after, I don't know, five or so years of being off the air. And
what would that look like? Obviously, it's something that wants to speak
to this moment. But like, what the heck does that look like for a
beloved TV show property? And you and I both watched it. We watched it
together, we were texting during it a little bit. I have to admit that
my initial response was kind of skepticism, because when the episode
opens, Leslie is kind of starting a phone tree, and she's imploring
every member of her community or the cast of the show to reach out to
someone else. And nobody wants to do it. And the way the show is
structured is they're doing that split screen to let you know that
people are having video calls with one another.

\begin{itemize}
\item
  archived recording 1\\
  Who am I calling?
\item
  archived recording 2\\
  Gary.
\item
  archived recording 3\\
  {[}LAUGHS{]} I'm not calling Garry. {[}DING{]} What up, what up?
\item
  archived recording 4\\
  Tommy! Oh, damn.
\end{itemize}

jenna wortham

But they all are doing it. And it's mostly for her. It's for her
well-being. And you get to revisit with all the characters from that
show, including Retta, Aziz Ansari. And then later on you see Aubrey
Plaza, Chris Pratt, Nick Offerman, and then Richard Jones. I mean, the
whole cast.

\begin{itemize}
\item
  archived recording 1\\
  It's great to see you too, Leslie.
\item
  archived recording 2\\
  So what's new, Mayor Gergich? How's Pawnee doing?
\item
  archived recording 3\\
  Well, not bad. But I will tell you, some people really fought me when
  I had to cancel the annual Pawnee Popsicle Lick `n Pass.
\item
  archived recording 4\\
  Very weird tradition. Why did we ever do that?
\item
  archived recording 5\\
  Hold on a second. I got some frosting in my ear thing. {[}POP{]} OK.
\end{itemize}

jenna wortham

It really warmed my heart. I don't know. Because it felt like they were
trying to do something for us. It felt like they understood --- they
being the cast of the show --- understood how important the show was to
its core audience. And they --- it wasn't just a reunion for the sake of
a reunion, for the sake of like dragging out old tired jokes or old
running memes or themes. They were actually trying to do a public
service, which was the very meta thing about a show about public
service, more or less. There were a lot of mentions about mental health,
which I really appreciated. You know, there's Donna, who's played by
Retta. She used her time onscreen to talk about her man who's a teacher,
and how hard it is for all the teachers right now who are wrangling
these kids who are also having a hard time stuck at home. It felt in
service to something bigger than just their celebrity. And that's --- I
have not seen that on TV in a very long time.

wesley morris

Yeah, yeah. I agree. And I think that the warmth that you felt is the
warmth that I felt. I didn't understand why this show couldn't make up
its mind during the Obama era about what ---

jenna wortham

What it was?

wesley morris

--- its tone was going to be, right. Like, is it making fun of our
enthusiasm for that administration and for that moment in American
psyche and American history and American optimism? But coming back for
30 minutes in 2020 under these circumstances? I mean, it --- it not only
warmed my heart. It kind of broke my heart a little bit, because all of
that earnestness makes sense right now. I need somebody to remind me
with comedy the government is good and can be good for people in a
moment where the government doesn't make any sense. Our national
government is incoherent. It is telling us to do two things at the same
time --- sometimes four things at the same time, depending on who you're
listening to.

jenna wortham

Yeah, yeah.

wesley morris

And I don't really necessarily know that I need somebody to make fun of
the Trump administration. So a show like ``Parks and Recreation'' can be
revealed in this half an hour incarnation as having always believed in
the power of government and in the civic duty of public service. But
also in people, and people connecting to people and people checking on
people and people coming together for each other. And we have back, for
this one night only, a show that actually does believe in government, a
show that believes in the people.

\begin{itemize}
\item
  archived recording 1\\
  {[}VIDEO CALL RING{]} Oh, OK. Hold on. I'm getting another call.
\item
  archived recording 2\\
  Oh, merge it. Trust me.
\item
  archived recording 3\\
  Hey! {[}SEVERAL PEOPLE TALKING{]}
\end{itemize}

jenna wortham

Well, by the end of the episode, the entire cast of ``Parks and Rec''
has gathered to talk to their boss, Leslie, who's played by Amy Poehler,
and they want to be there for her. She's doing all this work to check in
on them, and then by the turn of the end the episode, they decide to
show up for her. And they sing a song. And the whole thing is funny and
hilarious. But I was also like, this is what my emotional interactions
look like right now. I've been on a lot of Zoom birthday calls. I've
been on many surprise Zoom birthday parties. And that's what it looks
like, a grid of your favorite faces. Everyone's just compressed to this
little four-by-four box and just trying to figure out how to be there
for someone. I don't know, that part --- it's funny, because the cynic
in me would normally loathe something like that in any other regular
type of TV or movie property unless it was hilarious, right? But because
we're living in this moment where that is actually what real life looks
like --- and what their life looks like. I mean, they're filming this
show remotely, so that's what it has to look like --- was just a
disappearing of the fourth wall that I desperately needed.

wesley morris

Yeah. And then there's that other dimension that also makes it
surprisingly moving, which is not dissimilar from the Sondheim show. But
it's different slightly here because we're watching people we know to be
playing characters, but they're also in the actors' houses. So there is
this like two-ply relatability and this double flattening, where the
flattening is occurring in the actors' world and the characters' world,
and it directly mirrors almost --- I mean, to the extent that our
screens are a mirror, it actually literally mirrors our lives. And so
you know, you get to see Nick Offerman's, what I guess is his actual
garage, and maybe Chris Pratt's actual garage?

jenna wortham

Potentially.

wesley morris

Potentially.

jenna wortham

That was definitely Rashida Jones' couch. That was definitely her couch
that she was leaning up against.

wesley morris

Retta's closet. You could see Retta's shoes.

jenna wortham

Definitely Retta's shoes in her closet. Aziz Ansari decided to show
nothing about his own life. He had a fake background of a beach, which I
also respect. I love a Zoom background, because sometimes I'm just like,
don't worry about where I am at my house. You don't need to know. You
don't need to know. I also really liked how you could see the edges. You
can see where the plastic is starting to peel back from the sofa. And
it's just --- it renders as very honest and very earnest. And I think
that has a lot of currency and a lot of capital right now.

wesley morris

Yeah. It is the struggle of the moment. People talk about quote,
``showing up,'' unquote. I don't even know what that means anymore. I
feel like the real question is, who are you going to be when you get
there? And what are you going to let yourself be open to feeling and
sharing?

jenna wortham

Yeah. I also can't stop thinking about all of these accidental moments
of vulnerability that have been happening online. And I'm thinking very,
very, very specifically about this music duel between Babyface and Teddy
Riley that's part of this bigger versus challenge that's happening that
pits two legends in music against each other, where they just play
records back and forth for an hour or two online. There was T-Pain and
Lil Jon. Erykah Badu and Gil Scott have one coming up next week. I mean,
this is a thing ---

wesley morris

I cannot wait.

jenna wortham

--- that's going on. I cannot wait for that one either. Legendary. But
you know, it's hard to go live. I mean, the moment you press the button
live on Instagram, you are literally broadcasting live and it's
recording.

\begin{itemize}
\item
  archived recording (teddy riley)\\
  Hey, everybody. Everybody, everybody.
\item
  archived recording\\
  Look for Kenny, and then accept him.
\item
  archived recording (teddy riley)\\
  Yeah, I'll look for him.
\end{itemize}

jenna wortham

But bless Teddy Riley's heart. He could not figure out what to do. So
much starts happening right away. You have to be on top of it. You've
got to pin a comment. You've got to add the person to the chat that you
want to add. I mean, meanwhile, the entire internet can see your face
and can see what you're doing. It is stressful! Bless Teddy Riley's
heart. He's just talking to the comments. He's like, Oh, hey, man. How
you doing? Oh, hey. There you go. What's up?

\begin{itemize}
\item
  archived recording (teddy riley)\\
  I don't see Babyface.
\item
  archived recording\\
  Not yet. He's coming.
\end{itemize}

jenna wortham

Meanwhile, Babyface keeps typing in the chat, I'm here. Hey, man. Let me
in. I'm here. And it was just so funny. And then someone comes to the
room and is like ---

\begin{itemize}
\tightlist
\item
  archived recording\\
  He says it's better if we all go to his page right now.
\end{itemize}

jenna wortham

We're going to switch this. We're actually gonna just let Babyface run
this, because Teddy cannot figure it out.

wesley morris

{[}LAUGHS{]}

jenna wortham

And then he goes, So do I log out? And the guy's like, just hit the
button. He's like, OK, gang. We're gonna go over to Babyface's house
now. It was just so dad moment.

\begin{itemize}
\item
  archived recording (teddy riley)\\
  Well, can they still go live on my page?
\item
  archived recording\\
  Tell them to go to Facepage.
\end{itemize}

jenna wortham

Listen. In that moment, though, I felt that. It's stressful! It's really
hard. He was doing the best he can. And I felt for him. But it just ---
listen, here's the thing. He recovered so gracefully, which is not an
easy thing to do. And I've really been meditating on that, because the
whole internet was making fun of him, myself included. I mean, I was
laughing so hard I was crying in my bed. But I've been thinking a lot
about the actual power and strength it takes to look like you don't know
what you're doing. And Brené Brown is an author and a vulnerability
researcher who I really, really, really respect and admire. She has a
great Ted Talk that we'll link to in the show notes. I have her book
``Daring Greatly'' on my Kindle, and I reread it all the time. But her
entire philosophy is trying to understand what it means to be vulnerable
and what it means to open yourself up for love, and why so many of us
don't do it. She's done research for decades into this. And she says
that one of the number-one barriers between people for actual connection
is shame. The fear that if you look stupid --- using air quotes --- if
you look dumb or you reveal too much of yourself, then people are going
to think it's gross or they're no longer going to respect you. And that
keeps us from actually trying to really let ourselves be seen. And I
don't know, I can't stop thinking about that in terms of what people ---
and I don't mean just celebrities, but what even you and I are doing
right now. We are recording this using a video call to communicate so we
can see each other. And we're not coming into the office place as
groomed as we normally are. We're in our messy homes with whatever is
going on behind us. Your hair is growing out. I'm not wearing makeup.
We're really just showing the F up. And I think one of the biggest
transformations in terms of human expression online over the last couple
of years has been a migration away from honesty and reality and realness
towards performance and presentation. You know, it's really rare to see
somebody on Instagram Live without a full face of makeup. It's really
rare to see a post that isn't perfectly curated or adjusted, the
contrast and brightness adjusted just so so that you can't see even the
hint of an acne scar. And I think a lot of that has to do with the way
reality TV has gotten a lot more polished. I mean, I think it has a lot
to do with the Kardashians. I think it has a lot to do with the fact
that you can make a lot of money off of social media, so it started to
trend towards a type of professionalism. But right now, all of that
feels really fake. And I think people are just like, Yes, we actually
want reality to look like reality again. And I think it just feels
really strange and bizarro to look at someone with a full face of makeup
and a perfectly painted lip, even if it is, you know, Rhianna lip paint,
which we love. Who is that lip for? Like, where are you going and how
long is it going to take you to take it off? It just doesn't feel as
authentic anymore.

wesley morris

Well, I think what we've noticed in this moment is that people have been
trying to sell us stuff, and we don't either have money to buy anything
or we don't have any place to shop. And so you gotta sell me something
else. You gotta sell me your heart. You gotta sell me some feelings. You
have to sell me some vulnerability. And don't sell it to me. Just
present it. I'm not trying to buy. I'm just open, and I'm ready for what
you have to give me. This is sort of --- this is the thing that really
got me about that ``Parks and Recreation'' episode, by the way, which is
basically that the moment at the end where, of all people, it's Ron ---
curmudgeonly weird-facial-hair-having Ron --- I think Ron looks right at
the camera and says, I miss you guys.

jenna wortham

Mm-hmm. And his eyes are red-rimmed and watery. And that's the part that
just sent chills up my spine, because I'm like, yeah, I miss lots of
things. I miss lots of people. And I believe that you miss lots of
things and lots of people too. And I guess that's what we've been
talking about too, is like, you know, a pretty subtle and subconscious
reorganization of priorities and principles, which is really exciting. I
mean, it is a subtle shift, but it has tremendous potential in it.

wesley morris

What's going to happen at the end of this? Like, Chris Pratt is going to
go back to another ``Jurassic Park'' movie. Meryl Streep's going to go
back to the Oscars.

jenna wortham

Yeah.

wesley morris

What are we going to do? Where are we going back to?

jenna wortham

{[}LAUGHS{]}

wesley morris

Are we all just going to go back to what, to how things were before, you
know, January 2020?

jenna wortham

I mean, you're not going to like my answer. I think the truth is a
little uncomfortable in that we probably do resume business as usual.
But I don't think that people come out of this experience unaltered. I
don't think they come out of this the same. I think things might look
the same. But I think that there have been tweaks into fundamental
understandings of how the world works, of what it looks like to be in
community, who you can rely on, how important communication is. And it
will be subtle, and I think it will be discreet. But I think that's how
change happens. I do think people carry it with them, and they pass it
down. And this moment is so big that we really haven't even begun to
reckon with it or process it. And I don't think we will for a really
long time. But it will linger. It will linger. I have no doubt about
that. That is where my eternal optimist comes forward. I do believe we
come out of this changed.

{[}music{]}

wesley morris

That's our show. Jenna, I hate to tell you this. I mean, I hate to be
reminded of it. But next week is going to be our last episode for a
minute.

jenna wortham

I know.

wesley morris

And who knows if we're going to be back in the studio when we come out.
But we'll figure that out. We'll figure --- we'll know where we're going
to be next week, right? Everything will just make more sense in a week.

jenna wortham

Right here. Yeah. We'll still be here in our living rooms.

wesley morris

``Still Processing'' is a product of The New York Times. And it was
recorded, once again, in our living rooms.

jenna wortham

It is produced by Hans Buetow.

wesley morris

Our editors of Sarah Sarasohn, Sasha Weiss, Wendy Dorr, and Lisa Tobin.

jenna wortham

Our engineer is Jake Gorski.

wesley morris

And our theme music is by Kindness. It's called ``World Restart'' from
the album ``Otherness.''

jenna wortham

And you can find all of our episodes and various fun things at
NYTimes.com/stillprocessing.

wesley morris

So thank you for listening, everybody. Wear a mask. Wash your hands. Do
a dance.

jenna wortham

We'll see you next week.

\href{https://www.nytimes3xbfgragh.onion/column/still-processing-podcast}{\includegraphics{https://static01.graylady3jvrrxbe.onion/images/2019/09/15/podcasts/still-processing-album-art-2/still-processing-album-art-2-square320.jpg}Still
Processing}Subscribe:

\begin{itemize}
\tightlist
\item
  \href{https://itunes.apple.com/us/podcast/id1151436460}{Apple
  Podcasts}
\item
  \href{https://www.google.com/podcasts?feed=aHR0cHM6Ly9yc3MuYXJ0MTkuY29tL255dC1zdGlsbC1wcm9jZXNzaW5n}{Google
  Podcasts}
\end{itemize}

\hypertarget{does-this-phone-make-me-look-human-1}{%
\section{Does This Phone Make Me Look
Human?}\label{does-this-phone-make-me-look-human-1}}

\hypertarget{the-internet-is-bringing-us-closer-together--but-will-the-intimacy-last-1}{%
\subsection{The internet is bringing us closer together --- but will the
intimacy
last?}\label{the-internet-is-bringing-us-closer-together--but-will-the-intimacy-last-1}}

Hosted by Wesley Morris and Jenna Wortham. Produced by Hans Buetow.

Transcript

transcript

Back to Still Processing

bars

0:00/0:00

-0:00

transcript

\hypertarget{does-this-phone-make-me-look-human-2}{%
\subsection{Does This Phone Make Me Look
Human?}\label{does-this-phone-make-me-look-human-2}}

\hypertarget{hosted-by-wesley-morris-and-jenna-wortham-produced-by-hans-buetow-1}{%
\subsubsection{Hosted by Wesley Morris and Jenna Wortham. Produced by
Hans
Buetow.}\label{hosted-by-wesley-morris-and-jenna-wortham-produced-by-hans-buetow-1}}

\hypertarget{the-internet-is-bringing-us-closer-together--but-will-the-intimacy-last-2}{%
\paragraph{The internet is bringing us closer together --- but will the
intimacy
last?}\label{the-internet-is-bringing-us-closer-together--but-will-the-intimacy-last-2}}

Thursday, May 7th, 2020

\begin{itemize}
\item
  {[}music{]}
\item
  jenna wortham\\
  I'm Jenna Wortham.
\item
  wesley morris\\
  I'm Wesley Morris. We're two culture writers at The New York Times.
\item
  jenna wortham\\
  This is ``Still Processing.''

  So last weekend, beautiful weather in Brooklyn. Just crisp, sunny, the
  sweetest breeze. And ---
\item
  wesley morris\\
  You just want to stick a straw in it.
\item
  jenna wortham\\
  You just want to drink it up. So I got my straw, and I decided to go
  for a bike ride. And I had a mask on, of course. I had all the proper
  gear. I went with my best friend, Kimberly, and we rode around
  Brooklyn. It just --- it feels great after being stuck in the house,
  to be able to navigate your neighborhood and see things you haven't
  seen before. As lovely as the ride was, it was really hard to see that
  a lot of folks were not covering their faces, were not even attempting
  to socially distance. And you know, when you're trying to do the right
  thing, it's really hard to see a lot of people who really don't care
  about doing the right thing. And what does that mean? Also, how the
  decisions you're making are going to impact me and my people. And when
  I got home, I just had all these pent-up emotions. And you know,
  moving is medicine. Dancing is medicine. Shaking is medicine. So I
  remembered that there's this queer collective called BUFU, which is By
  Us For Us. Very cute, very P.O.C. I remembered that they were having a
  rave, a 24-hour rave that was supposed to last all weekend. And they
  had a really stellar lineup ---
\item
  wesley morris\\
  What?
\item
  jenna wortham\\
  --- of D.Js. Yes. I mean, it was the function. The fun-s-ion. It was a
  Zoom call basically. It was a private Zoom call. You had to know the
  code to get into the room. And they have had a party before, but I
  didn't go because I wasn't really up for it.
\item
  wesley morris\\
  But you know you like to move it, move it.
\item
  jenna wortham\\
  I love to move it, move it. And again, I had all this energy, all this
  adrenaline from the bike ride. I was like, I got this. I decided to
  prop the computer up in my kitchen, because the lighting was great in
  there and I wanted to make some food.
\item
  wesley morris\\
  {[}LAUGHS{]}
\item
  jenna wortham\\
  And so I logged in, and just like the first thing I see is somebody's
  ass, just like shaking onscreen. Green silk, bright sun. And I was
  like, I'm home. This is where I need to be. It's a Zoom call, right?
  We've all been on those at this point for work at some point. But a
  Zoom party is totally different, because the host is a D.J. And all
  the little windows are occupied by people partying, whether it's in
  their backyard, whether it's in their bedroom. I even saw somebody who
  had a stripper pole installed in their living room. I need to find out
  who that is so I can go to that house when all this is over. It was
  the best of both worlds for me. I got to chop my greens. I got to make
  a little stir fry. And I got to twerk in my kitchen. I mean, it was
  all wrapped up in one. And it was great, because all the kind of anger
  and the frustration that I felt at the world at people not being
  properly covered up melted away, and it was replaced by this vibrant
  beauty of just seeing people trying to do the right thing, of trying
  to stay safe, keep themselves healthy. And I felt all the emotions
  inside of me kind of bubbling up into my eyes, and they wanted to come
  out. And I was like on the verge of tears. I kept, like, backing it up
  into my fridge and then dropping it low, kind of letting it wiggle on
  the camera. But I was crying into the pickles, the homemade pickles
  and kombucha, because I just didn't want --- I didn't want to have to,
  like, explain what was happening with my face. But I don't know,
  Wesley, you know me. I'm a pretty private person. I keep a lot of my
  private life to myself. I don't post pictures of my home. But in this
  call, I mean, I was like --- at one point I was stir-frying some
  broccoli, and I had a knee on the countertop just bouncing along,
  because the Megan Thee Stallion ``Savage'' remix with Beyoncé came on.
  And I was like a hundred percent, yes, this deserves more. This
  deserves more than just me stirring a pot, you know?
\item
  wesley morris\\
  You were on a camera literally putting your foot in it.
\item
  jenna wortham\\
  I was really putting my foot in it. And I continued to do so for the
  rest of the day. It was a moment of connection that I didn't know that
  I needed. These were not my immediate circle. These were not my
  closest friends. These were people whose faces I would normally see at
  the local coffee shop or I would see at a reading or I might see at a
  screening. And I haven't really thought about most of them, to be
  honest with you. I've been too consumed with the day-to-day of my own
  life and checking in on you and the people in my world. It was really
  affirming, and it really helped me reinforce my own commitment to
  being part of this broader public health moment, my own desires to do
  the right thing. And I just --- that is what community looks like
  right now. It's actually not seeing friends of friends of friends of
  friends, former co-workers, former lovers, lovers of their lovers out
  in the world. It looks like being alone together through a screen. And
  I really didn't know that that's something that I was missing. I
  didn't even know that that was a possibility. I didn't even know that
  was on offer in this way.
\item
  wesley morris\\
  No, because you go out.
\item
  jenna wortham\\
  Because I'm out, honey. I'm in these streets, more than I need to be.
\item
  wesley morris\\
  More than you need to be sometimes. I've seen it. I've seen it.
\item
  jenna wortham\\
  {[}LAUGHS{]} So that's my experience, Wesley, which is --- I feel like
  we're going to have different experiences of this. And so I'm curious
  for you, are you missing anything now that every interaction,
  everything is becoming facilitated by the videophone?
\item
  wesley morris\\
  Ugh. Uh, yeah! I mean, I miss it a lot. I've lived in New York for
  five years, six years? And this whole time, I've been going to the
  theater almost --- maybe once a week, sometimes two or three times a
  week. Sometimes four times a week.
\item
  jenna wortham\\
  You do.
\item
  wesley morris\\
  And I love going to the theater. It is a luxury and a privilege, and I
  can't believe I get to do it, even when the show is the worst. And I
  have seen some bad shows!
\item
  jenna wortham\\
  You've taken me to some of those bad shows!
\item
  wesley morris\\
  I've taken several people to some of those bad shows, and I apologize
  to everybody. But I really, really miss going to the theater. I miss
  sitting in those uncomfortable seats. I miss the randomness of who you
  might be seated next to. I mean, it could be some woman from
  Parsippany, N.J. It could be the vice president. You just don't know.
  And then nobody knows what's going to happen once the show starts. Is
  somebody's phone going to ring? You don't know! I miss all of the
  serendipity and randomness of live theater. And lo and behold, a
  miracle happened. Early last week, Stephen Sondheim --- it's his 90th
  birthday year. Stephen Sondheim is one of America's great artists,
  period. He is the author of numerous musicals that we love and
  appreciate. ``Into the Woods'' is his. ``Sweeney Todd'' is his. He is
  partially responsible for the greatness that is ``West Side Story.''
  He is the definition of the American musical theater to many, many of
  us. And somebody had the great idea that instead of coming out before
  the curtain of a revival of one of his musicals that should be on
  Broadway right now and saying, ``Happy birthday, Stephen Sondheim. We
  love you.'' They got everybody who's anybody in musical theater to
  come and sing individual tributes to Stephen Sondheim from their
  living rooms, their backyards, their bathrooms. The event itself was
  called Take Me to the World. And it's just everybody comes out to
  sing. Lin-Manuel Miranda is there. Christine Baranski, Meryl Streep,
  Audra McDonald, Mandy Patinkin, Bernadette Peters, Patti LuPone, Lea
  Salonga. And doing one of my favorite Sondheim songs --- one of the
  best Sondheim songs --- ``Someone in a Tree'' from ``Pacific
  Overtures,'' Ann Harada, Austin Ku, Kelvin Moon Loh, and Thom Sesma.
  And the thing that I loved the most, and didn't realize I would love
  most about it, is something that you can't get from live theater, that
  you can only get in this particular moment, which is when all the
  musicians start the overture for this two-and-a-half hour event.
\item
  music --- ``someone in a tree"\\
  {[}OVERTURE{]}
\end{itemize}

wesley morris

It's edited so that all the musicians are in their respective homes
playing their instruments, and you just get to see like this drummer's
house. You get to see the fingernail polish on some of the musicians. I
got teary just looking at all this stuff. The French horn lady got a
closeup! The French horn lady, when you go to see a Broadway musical,
you can't even see her because she's under the stage.

jenna wortham

{[}LAUGHS{]}

wesley morris

The idea that you're --- they're paid tribute to for like the opening 10
minutes of this thing? It just really, really moved me.

\begin{itemize}
\tightlist
\item
  music --- ``someone in a tree"\\
  {[}OVERTURE{]}
\end{itemize}

jenna wortham

So the night the Sondheim tribute's happening, I was unaware, but I
started seeing this screenshot of Christine Baranski, Audra McDonald,
and of course the queen, Meryl Streep, floating through my Twitter
stream. And I'm like, what are these ladies up to?

\begin{itemize}
\tightlist
\item
  archived recording (singing)\\
  Here's to the ladies who lunch. Everybody laugh.
\end{itemize}

jenna wortham

And it was so funny to see like Christine Baranski, Meryl Streep, and
Audra McDonald pretty gritty looking with their drinks. It's not clear
if they're drinking, but they're all miming that they're drinking
alcohol. Christine Baranski was definitely drinking wine. And they're
all just like --- nobody's hair is done. Nobody really seems to have a
ton of makeup on. Everybody's in their favorite hotel bathrobe that
they've stolen. {[}LAUGHS{]} Probably. And they're all kind of just
commiserating. And I was like, wait, that looks like my Friday night
happy hour Zoom. Like, that is what we do. It made me care a lot more
about those famous people and their lives and their inner lives than I
normally ever do --- other than Meryl Streep, who really is just, you
know, for me the GOAT in so many ways for so many people. It was really
evocative.

\begin{itemize}
\item
  archived recording (singing)\\
  Another reason not to move. Another vodka stinger.

  I'll drink to that.
\end{itemize}

jenna wortham

And I also can't stop thinking about the bookcase that was behind Meryl
Streep, because when reporters have to phone in for live broadcasts or
they want to appear on news shows, a lot of them are sitting in front of
their, like, beautifully arranged, highly decorated shelves full of
highly lauded and esteemed authors, like these mountainous towers of
books behind them. I guess it makes you seem ---

wesley morris

Yes, much has been made of this.

jenna wortham

Much has been made about this. Our colleague Amanda Hess has written
about it. I guess it makes you seem more studious. But Meryl's
bookshelves. Empty. I was like, what kind of larder is this? There's
nothing on these shelves. But there was a hand-carved --- well, it
looked hand-carved --- there was one lone duck or egret, I don't know
what kind of bird it was, just perched on the shelf behind her. And I
was like, you know, Meryl? I didn't think I could love you more, but now
I'm head over heels in love with you because you just left this crooked
souvenir from a Cape Cod trip that you probably don't even remember just
chilling on your bookshelves. She didn't dress it up with any fancy
cutlery or dishware or crystals. Who knows what she has? She probably
has gems. She probably has gems she could have put on that bookshelf.
No, she left the duck. And it just felt so human. Which for most
celebrities, that's usually the last thing they want to appear. They
don't want to turn the human dial up too much. They only want to give
you as much human as they think you can handle. Whereas in this special,
the dial was broken. It was just cranked all the way up.

wesley morris

The other amazing thing about that moment with Audra McDonald and
Christine Baranski and Meryl Streep was when they cut to a wide
three-shot where it became basically a Zoom call with music. On each
person's screen they say, like, C.B.`s phone for Christine Baranski.

jenna wortham

Yes, yes, yes! So-and-so's iPad.

wesley morris

{[}LAUGHS{]}

I just, I love the jankiness or the averageness of --- I mean, of
course. These are human beings! Like, what else is Meryl Streep's screen
supposed to say besides M.S.`s phone? It wasn't going to say, like,
three-time Oscar winner, most nominations in the history of the Academy
Awards's phone. No! It's going to say M.S., because that's all she is,
is M.S. She's Meryl Streep. Just a person who wears a bathrobe and walks
around and breathes air just like everybody else --- with a mask on,
hopefully. And I am often astonished that like, why don't I give myself
a better name than Wesley Morris when I'm on these calls? Why am I not
like Frenchie Laroucho, something like that?

jenna wortham

I've been changing mine up. But I'll tell you the risk with that, OK? My
rave name was Jenny Deluxxx with three X's. And then I went into a
meetup with my sound healers and I was like, Oh, just a moment,
everybody. Everyone was like, What have you been up to? And I was like,
(WHISPERING) Don't worry about it. Don't worry about it. So you know,
there's upsides and there's downsides. Anyway.

wesley morris

So I'll just stay --- if Meryl Streep can just be M.S.`s iPhone, if she
can just be M.S., I'll just be Wesley Morris. That's good enough for me.

jenna wortham

You know, Wesley, I also noticed the little device tags in the corner of
everybody's screens. And I just thought, wow, you know, we're really all
operating under a very similar set of circumstances right now. And I
want to be very clear, I'm not saying that Covid-19 is a great
equalizer. Absolutely not. This virus is hitting marginalized
communities and black communities, native communities, indigenous
communities, Latinx communities, way harder than any other community.
And so in that sense, we're not all dealing with this in the same way.
But I do think that something else has happened with people who are now
comporting their lives to the rectangular screens that they live on.
There is something that's like a great flattening that's happening.
There's a context collapse, right? Because we're all forced pretty much
to use the same several pieces of technology, whether it's your
smartphone, an iPod, your laptop, Zoom, Google Hangouts, FaceTime.
Everything is basically being funneled through those portals, which
means that it all ends up looking the same. Now, this is something
that's been happening for a much longer time, obviously, on social media
on things like TikTok, Instagram, YouTube. I'm thinking a lot about the
Don't Rush Challenge on TikTok, Wesley, which I feel like you've
probably seen but maybe didn't know it was called that. But it's like
you and your crew, whatever that means to you, basically what happens
--- one person starts. And they look down at their clothes and they're
kind of like, ugh. They throw something at the camera, and then the next
shot is them looking really fine. And then they pass the baton to the
next person, who you know --- and these things are stitched digitally
together. And so what ends up happening is you get kind of the same
effect that you had with Meryl, Audra, and Christine. But it just has
been happening in a much broader scale on TikTok for much longer.

wesley morris

Yeah, that just sounds like TikTok. It sounds like TikTok magic.

jenna wortham

It's just TikTok magic. But I guess to say, it's really interesting to
see basically anyone who's trying to make cultural content for the web
reduced to the exact same mechanisms and formats, which is ---

wesley morris

Yes, yes, yes!

jenna wortham

{[}LAUGHS{]} It's actually a really remarkable thing, because it's very
ephemeral. It's limited. People who have made their life's work social
media, that invitation into personal and private space has always been
part of the deal, because that's all you had to work with. And so that
kind of intimacy was par for the course. And it was great. I love seeing
kids' closets open in the background of their YouTube videos and all
their clothes and toys spilling out. I mean, it's one of my favorite
things about social media. Or that kids in Houston all live in those
same kind of cul-de-sacs and those same tract houses with those huge
double-car wide driveways that they dance in. I mean, there's something
about seeing those elements of similarity and those moments of sameness
that make me feel deeply connected to whatever a kid in Ohio is doing.
And I think seeing that transferal happening to pretty much anyone who
wants to be present right now in this moment under quarantine, they have
to, to some degree, allow us in, allow us to see them being vulnerable.
And sometimes it's just as simple as a messed up hair --- I mean, Meryl
Streep's hair was so messed up and I will never love her more because of
that. Like, I just ---

wesley morris

I was gonna say, you sound concerned!

jenna wortham

No, I --- it's not concern. It's admiration, because it takes a lot to
let people see you disheveled. It takes a lot to let people see the
messy bits, you know? Even if they're a little bit curated, that's a
real moment of tenderness, and trust, honestly, that people aren't going
to be like, Gross, she looks disgusting. But just to be like, Those
pillow creases on my face, that's humanity. That's humanity that you're
looking at.

wesley morris

It's life.

jenna wortham

It's life! It's real frickin' life.

wesley morris

And what I will say is, when you make this great point about how we're
all TikTokkers and YouTubers now, one of my favorite things ---

jenna wortham

We are.

wesley morris

--- about the Sondheim tribute was like right almost in the middle, was
a little tiny bit by Randy Rainbow. He is one of the most brilliant
people you are ever going to see do anything, and he lives on YouTube.

jenna wortham

OK.

wesley morris

And essentially, what he does is he takes some kind of fake background,
lots of editing, and his beautiful voice, and he does showtune satires
where he takes classic songs and he rejiggers them to comment on the
moment. So in the middle of the Stephen Sondheim tribute are like two
minutes of Randy Rainbow doing ``By the Sea'' from ``Sweeney Todd.''

\begin{itemize}
\tightlist
\item
  archived recording (randy rainbow)\\
  But a seaside wedding could be devised, me rumpled bedding
  legitimized.
\end{itemize}

wesley morris

This is where Mandy Patinkin and Bernadette Peters, they've all come to
Randy Rainbow's world.

\begin{itemize}
\tightlist
\item
  archived recording (randy rainbow)\\
  Oh, I can just see us now in our bathing dresses, you respecting
  social distancing guidelines from six feet away and me in a facemask I
  made from my own underwear.
\end{itemize}

wesley morris

He might not make sense on Broadway. But he makes sense in this world.
And he is the person who seems most natural in it, more natural than
Patti LuPone, more natural than Meryl Streep and Christine Baranski and
Audra McDonald. He knows how to use this space to bring a song to life.

\begin{itemize}
\tightlist
\item
  archived recording (randy rainbow)\\
  Happy birthday, Mr. Sondheim. I'm available.
\end{itemize}

jenna wortham

Honestly though, Wesley, it's going to be really, really, really
interesting to see if our movies and our TV shows also start to catch up
to this new reality and start to look like the way we're living now. And
we're actually already seeing it a little bit. For example, that episode
of ``Parks and Rec,'' the special reunion episode that you reminded me
to watch. It aired last week. And I'm so glad we watched it. So let's
take a break, and we'll come back and we will get into it.

{[}music{]}

wesley morris

When it started in 2009, ``Parks and Recreation'' was this show about a
fictional town called Pawnee, Ind., and the cast of characters who
worked there, centered around one Leslie Knope. And she, when the show
starts, is the deputy director of the Parks and Recreation Department of
Pawnee. And it was an office comedy, essentially. And I think the thing
that bugged me about ``Parks and Recreation'' when it was actually on
NBC was that it seemed tonally and, to the degree that it was
ideological at all, redundant.

jenna wortham

Hmm.

wesley morris

I felt like that show during the Obama era just felt like it was riding
the coattails of the mood of the country in a way. I watched it for most
of its run. I've seen almost every one of its 125 episodes. I didn't
love it the way other people who watched this show religiously loved
this show. Did you watch the show?

jenna wortham

I did. I watched it off and on. But I was really interested when they
announced last week there would be a comeback special where they would
reunite after, I don't know, five or so years of being off the air. And
what would that look like? Obviously, it's something that wants to speak
to this moment. But like, what the heck does that look like for a
beloved TV show property? And you and I both watched it. We watched it
together, we were texting during it a little bit. I have to admit that
my initial response was kind of skepticism, because when the episode
opens, Leslie is kind of starting a phone tree, and she's imploring
every member of her community or the cast of the show to reach out to
someone else. And nobody wants to do it. And the way the show is
structured is they're doing that split screen to let you know that
people are having video calls with one another.

\begin{itemize}
\item
  archived recording 1\\
  Who am I calling?
\item
  archived recording 2\\
  Gary.
\item
  archived recording 3\\
  {[}LAUGHS{]} I'm not calling Garry. {[}DING{]} What up, what up?
\item
  archived recording 4\\
  Tommy! Oh, damn.
\end{itemize}

jenna wortham

But they all are doing it. And it's mostly for her. It's for her
well-being. And you get to revisit with all the characters from that
show, including Retta, Aziz Ansari. And then later on you see Aubrey
Plaza, Chris Pratt, Nick Offerman, and then Richard Jones. I mean, the
whole cast.

\begin{itemize}
\item
  archived recording 1\\
  It's great to see you too, Leslie.
\item
  archived recording 2\\
  So what's new, Mayor Gergich? How's Pawnee doing?
\item
  archived recording 3\\
  Well, not bad. But I will tell you, some people really fought me when
  I had to cancel the annual Pawnee Popsicle Lick `n Pass.
\item
  archived recording 4\\
  Very weird tradition. Why did we ever do that?
\item
  archived recording 5\\
  Hold on a second. I got some frosting in my ear thing. {[}POP{]} OK.
\end{itemize}

jenna wortham

It really warmed my heart. I don't know. Because it felt like they were
trying to do something for us. It felt like they understood --- they
being the cast of the show --- understood how important the show was to
its core audience. And they --- it wasn't just a reunion for the sake of
a reunion, for the sake of like dragging out old tired jokes or old
running memes or themes. They were actually trying to do a public
service, which was the very meta thing about a show about public
service, more or less. There were a lot of mentions about mental health,
which I really appreciated. You know, there's Donna, who's played by
Retta. She used her time onscreen to talk about her man who's a teacher,
and how hard it is for all the teachers right now who are wrangling
these kids who are also having a hard time stuck at home. It felt in
service to something bigger than just their celebrity. And that's --- I
have not seen that on TV in a very long time.

wesley morris

Yeah, yeah. I agree. And I think that the warmth that you felt is the
warmth that I felt. I didn't understand why this show couldn't make up
its mind during the Obama era about what ---

jenna wortham

What it was?

wesley morris

--- its tone was going to be, right. Like, is it making fun of our
enthusiasm for that administration and for that moment in American
psyche and American history and American optimism? But coming back for
30 minutes in 2020 under these circumstances? I mean, it --- it not only
warmed my heart. It kind of broke my heart a little bit, because all of
that earnestness makes sense right now. I need somebody to remind me
with comedy the government is good and can be good for people in a
moment where the government doesn't make any sense. Our national
government is incoherent. It is telling us to do two things at the same
time --- sometimes four things at the same time, depending on who you're
listening to.

jenna wortham

Yeah, yeah.

wesley morris

And I don't really necessarily know that I need somebody to make fun of
the Trump administration. So a show like ``Parks and Recreation'' can be
revealed in this half an hour incarnation as having always believed in
the power of government and in the civic duty of public service. But
also in people, and people connecting to people and people checking on
people and people coming together for each other. And we have back, for
this one night only, a show that actually does believe in government, a
show that believes in the people.

\begin{itemize}
\item
  archived recording 1\\
  {[}VIDEO CALL RING{]} Oh, OK. Hold on. I'm getting another call.
\item
  archived recording 2\\
  Oh, merge it. Trust me.
\item
  archived recording 3\\
  Hey! {[}SEVERAL PEOPLE TALKING{]}
\end{itemize}

jenna wortham

Well, by the end of the episode, the entire cast of ``Parks and Rec''
has gathered to talk to their boss, Leslie, who's played by Amy Poehler,
and they want to be there for her. She's doing all this work to check in
on them, and then by the turn of the end the episode, they decide to
show up for her. And they sing a song. And the whole thing is funny and
hilarious. But I was also like, this is what my emotional interactions
look like right now. I've been on a lot of Zoom birthday calls. I've
been on many surprise Zoom birthday parties. And that's what it looks
like, a grid of your favorite faces. Everyone's just compressed to this
little four-by-four box and just trying to figure out how to be there
for someone. I don't know, that part --- it's funny, because the cynic
in me would normally loathe something like that in any other regular
type of TV or movie property unless it was hilarious, right? But because
we're living in this moment where that is actually what real life looks
like --- and what their life looks like. I mean, they're filming this
show remotely, so that's what it has to look like --- was just a
disappearing of the fourth wall that I desperately needed.

wesley morris

Yeah. And then there's that other dimension that also makes it
surprisingly moving, which is not dissimilar from the Sondheim show. But
it's different slightly here because we're watching people we know to be
playing characters, but they're also in the actors' houses. So there is
this like two-ply relatability and this double flattening, where the
flattening is occurring in the actors' world and the characters' world,
and it directly mirrors almost --- I mean, to the extent that our
screens are a mirror, it actually literally mirrors our lives. And so
you know, you get to see Nick Offerman's, what I guess is his actual
garage, and maybe Chris Pratt's actual garage?

jenna wortham

Potentially.

wesley morris

Potentially.

jenna wortham

That was definitely Rashida Jones' couch. That was definitely her couch
that she was leaning up against.

wesley morris

Retta's closet. You could see Retta's shoes.

jenna wortham

Definitely Retta's shoes in her closet. Aziz Ansari decided to show
nothing about his own life. He had a fake background of a beach, which I
also respect. I love a Zoom background, because sometimes I'm just like,
don't worry about where I am at my house. You don't need to know. You
don't need to know. I also really liked how you could see the edges. You
can see where the plastic is starting to peel back from the sofa. And
it's just --- it renders as very honest and very earnest. And I think
that has a lot of currency and a lot of capital right now.

wesley morris

Yeah. It is the struggle of the moment. People talk about quote,
``showing up,'' unquote. I don't even know what that means anymore. I
feel like the real question is, who are you going to be when you get
there? And what are you going to let yourself be open to feeling and
sharing?

jenna wortham

Yeah. I also can't stop thinking about all of these accidental moments
of vulnerability that have been happening online. And I'm thinking very,
very, very specifically about this music duel between Babyface and Teddy
Riley that's part of this bigger versus challenge that's happening that
pits two legends in music against each other, where they just play
records back and forth for an hour or two online. There was T-Pain and
Lil Jon. Erykah Badu and Gil Scott have one coming up next week. I mean,
this is a thing ---

wesley morris

I cannot wait.

jenna wortham

--- that's going on. I cannot wait for that one either. Legendary. But
you know, it's hard to go live. I mean, the moment you press the button
live on Instagram, you are literally broadcasting live and it's
recording.

\begin{itemize}
\item
  archived recording (teddy riley)\\
  Hey, everybody. Everybody, everybody.
\item
  archived recording\\
  Look for Kenny, and then accept him.
\item
  archived recording (teddy riley)\\
  Yeah, I'll look for him.
\end{itemize}

jenna wortham

But bless Teddy Riley's heart. He could not figure out what to do. So
much starts happening right away. You have to be on top of it. You've
got to pin a comment. You've got to add the person to the chat that you
want to add. I mean, meanwhile, the entire internet can see your face
and can see what you're doing. It is stressful! Bless Teddy Riley's
heart. He's just talking to the comments. He's like, Oh, hey, man. How
you doing? Oh, hey. There you go. What's up?

\begin{itemize}
\item
  archived recording (teddy riley)\\
  I don't see Babyface.
\item
  archived recording\\
  Not yet. He's coming.
\end{itemize}

jenna wortham

Meanwhile, Babyface keeps typing in the chat, I'm here. Hey, man. Let me
in. I'm here. And it was just so funny. And then someone comes to the
room and is like ---

\begin{itemize}
\tightlist
\item
  archived recording\\
  He says it's better if we all go to his page right now.
\end{itemize}

jenna wortham

We're going to switch this. We're actually gonna just let Babyface run
this, because Teddy cannot figure it out.

wesley morris

{[}LAUGHS{]}

jenna wortham

And then he goes, So do I log out? And the guy's like, just hit the
button. He's like, OK, gang. We're gonna go over to Babyface's house
now. It was just so dad moment.

\begin{itemize}
\item
  archived recording (teddy riley)\\
  Well, can they still go live on my page?
\item
  archived recording\\
  Tell them to go to Facepage.
\end{itemize}

jenna wortham

Listen. In that moment, though, I felt that. It's stressful! It's really
hard. He was doing the best he can. And I felt for him. But it just ---
listen, here's the thing. He recovered so gracefully, which is not an
easy thing to do. And I've really been meditating on that, because the
whole internet was making fun of him, myself included. I mean, I was
laughing so hard I was crying in my bed. But I've been thinking a lot
about the actual power and strength it takes to look like you don't know
what you're doing. And Brené Brown is an author and a vulnerability
researcher who I really, really, really respect and admire. She has a
great Ted Talk that we'll link to in the show notes. I have her book
``Daring Greatly'' on my Kindle, and I reread it all the time. But her
entire philosophy is trying to understand what it means to be vulnerable
and what it means to open yourself up for love, and why so many of us
don't do it. She's done research for decades into this. And she says
that one of the number-one barriers between people for actual connection
is shame. The fear that if you look stupid --- using air quotes --- if
you look dumb or you reveal too much of yourself, then people are going
to think it's gross or they're no longer going to respect you. And that
keeps us from actually trying to really let ourselves be seen. And I
don't know, I can't stop thinking about that in terms of what people ---
and I don't mean just celebrities, but what even you and I are doing
right now. We are recording this using a video call to communicate so we
can see each other. And we're not coming into the office place as
groomed as we normally are. We're in our messy homes with whatever is
going on behind us. Your hair is growing out. I'm not wearing makeup.
We're really just showing the F up. And I think one of the biggest
transformations in terms of human expression online over the last couple
of years has been a migration away from honesty and reality and realness
towards performance and presentation. You know, it's really rare to see
somebody on Instagram Live without a full face of makeup. It's really
rare to see a post that isn't perfectly curated or adjusted, the
contrast and brightness adjusted just so so that you can't see even the
hint of an acne scar. And I think a lot of that has to do with the way
reality TV has gotten a lot more polished. I mean, I think it has a lot
to do with the Kardashians. I think it has a lot to do with the fact
that you can make a lot of money off of social media, so it started to
trend towards a type of professionalism. But right now, all of that
feels really fake. And I think people are just like, Yes, we actually
want reality to look like reality again. And I think it just feels
really strange and bizarro to look at someone with a full face of makeup
and a perfectly painted lip, even if it is, you know, Rhianna lip paint,
which we love. Who is that lip for? Like, where are you going and how
long is it going to take you to take it off? It just doesn't feel as
authentic anymore.

wesley morris

Well, I think what we've noticed in this moment is that people have been
trying to sell us stuff, and we don't either have money to buy anything
or we don't have any place to shop. And so you gotta sell me something
else. You gotta sell me your heart. You gotta sell me some feelings. You
have to sell me some vulnerability. And don't sell it to me. Just
present it. I'm not trying to buy. I'm just open, and I'm ready for what
you have to give me. This is sort of --- this is the thing that really
got me about that ``Parks and Recreation'' episode, by the way, which is
basically that the moment at the end where, of all people, it's Ron ---
curmudgeonly weird-facial-hair-having Ron --- I think Ron looks right at
the camera and says, I miss you guys.

jenna wortham

Mm-hmm. And his eyes are red-rimmed and watery. And that's the part that
just sent chills up my spine, because I'm like, yeah, I miss lots of
things. I miss lots of people. And I believe that you miss lots of
things and lots of people too. And I guess that's what we've been
talking about too, is like, you know, a pretty subtle and subconscious
reorganization of priorities and principles, which is really exciting. I
mean, it is a subtle shift, but it has tremendous potential in it.

wesley morris

What's going to happen at the end of this? Like, Chris Pratt is going to
go back to another ``Jurassic Park'' movie. Meryl Streep's going to go
back to the Oscars.

jenna wortham

Yeah.

wesley morris

What are we going to do? Where are we going back to?

jenna wortham

{[}LAUGHS{]}

wesley morris

Are we all just going to go back to what, to how things were before, you
know, January 2020?

jenna wortham

I mean, you're not going to like my answer. I think the truth is a
little uncomfortable in that we probably do resume business as usual.
But I don't think that people come out of this experience unaltered. I
don't think they come out of this the same. I think things might look
the same. But I think that there have been tweaks into fundamental
understandings of how the world works, of what it looks like to be in
community, who you can rely on, how important communication is. And it
will be subtle, and I think it will be discreet. But I think that's how
change happens. I do think people carry it with them, and they pass it
down. And this moment is so big that we really haven't even begun to
reckon with it or process it. And I don't think we will for a really
long time. But it will linger. It will linger. I have no doubt about
that. That is where my eternal optimist comes forward. I do believe we
come out of this changed.

{[}music{]}

wesley morris

That's our show. Jenna, I hate to tell you this. I mean, I hate to be
reminded of it. But next week is going to be our last episode for a
minute.

jenna wortham

I know.

wesley morris

And who knows if we're going to be back in the studio when we come out.
But we'll figure that out. We'll figure --- we'll know where we're going
to be next week, right? Everything will just make more sense in a week.

jenna wortham

Right here. Yeah. We'll still be here in our living rooms.

wesley morris

``Still Processing'' is a product of The New York Times. And it was
recorded, once again, in our living rooms.

jenna wortham

It is produced by Hans Buetow.

wesley morris

Our editors of Sarah Sarasohn, Sasha Weiss, Wendy Dorr, and Lisa Tobin.

jenna wortham

Our engineer is Jake Gorski.

wesley morris

And our theme music is by Kindness. It's called ``World Restart'' from
the album ``Otherness.''

jenna wortham

And you can find all of our episodes and various fun things at
NYTimes.com/stillprocessing.

wesley morris

So thank you for listening, everybody. Wear a mask. Wash your hands. Do
a dance.

jenna wortham

We'll see you next week.

Previous

More episodes ofStill Processing

\href{https://www.nytimes3xbfgragh.onion/2020/07/23/podcasts/hamilton-ziwe-discomfort.html?action=click\&module=audio-series-bar\&region=header\&pgtype=Article}{\includegraphics{https://static01.graylady3jvrrxbe.onion/images/2020/07/23/multimedia/23stillprocessing-pix/23stillprocessing-pix-thumbLarge.jpg}}

July 23, 2020~~•~ 38:10Ziwe May Destroy Hamilton

\href{https://www.nytimes3xbfgragh.onion/2020/07/16/podcasts/reparations-for-aunt-jemima.html?action=click\&module=audio-series-bar\&region=header\&pgtype=Article}{\includegraphics{https://static01.graylady3jvrrxbe.onion/images/2020/07/18/multimedia/16stillprocessing-pix/16stillprocessing-pix-thumbLarge.jpg}}

July 16, 2020~~•~ 35:35Reparations for Aunt Jemima!

\href{https://www.nytimes3xbfgragh.onion/2020/07/09/podcasts/still-processing-black-lives-matter.html?action=click\&module=audio-series-bar\&region=header\&pgtype=Article}{\includegraphics{https://static01.graylady3jvrrxbe.onion/images/2020/07/12/podcasts/09stillprocessing-image/xx-stillprocessing-thumbLarge.jpg}}

July 9, 2020~~•~ 26:29So Y'all Finally Get It

\href{https://www.nytimes3xbfgragh.onion/2020/05/14/podcasts/still-processing-westworld-hollywood-utopia-dystopia.html?action=click\&module=audio-series-bar\&region=header\&pgtype=Article}{\includegraphics{https://static01.graylady3jvrrxbe.onion/images/2020/05/16/podcasts/14stillprocessing-image/14stillprocessing-image-thumbLarge-v2.jpg}}

May 14, 2020New Loop, America

\href{https://www.nytimes3xbfgragh.onion/2020/05/07/podcasts/still-processing-internet-vulnerability-sondheim-parks-recreation.html?action=click\&module=audio-series-bar\&region=header\&pgtype=Article}{\includegraphics{https://static01.graylady3jvrrxbe.onion/images/2020/04/28/pageoneplus/28sondheimjp-sp/28sondheimjp-sp-thumbLarge-v4.jpg}}

May 7, 2020Does This Phone Make Me Look Human?

\href{https://www.nytimes3xbfgragh.onion/2020/04/30/podcasts/still-processing-fiona-apple-fetch-bolt-cutters.html?action=click\&module=audio-series-bar\&region=header\&pgtype=Article}{\includegraphics{https://static01.graylady3jvrrxbe.onion/images/2020/05/03/multimedia/30stillpro-image/30stillpro-image-thumbLarge.jpg}}

May 1, 2020Fiona Ex Machina

\href{https://www.nytimes3xbfgragh.onion/2020/04/23/podcasts/still-processing-halle-berry-sharon-stone-catwoman-quarantine.html?action=click\&module=audio-series-bar\&region=header\&pgtype=Article}{\includegraphics{https://static01.graylady3jvrrxbe.onion/images/2020/04/25/arts/23stillprocessing/23stillprocessing-thumbLarge-v3.jpg}}

April 23, 2020Halle Berry? Hallelujah.

\href{https://www.nytimes3xbfgragh.onion/2020/04/16/podcasts/still-processing-AIDS-survive-coronavirus.html?action=click\&module=audio-series-bar\&region=header\&pgtype=Article}{\includegraphics{https://static01.graylady3jvrrxbe.onion/images/2020/04/20/us/16stillprocessing/16stillprocessing-thumbLarge-v3.jpg}}

April 16, 2020How to Learn From a Plague

\href{https://www.nytimes3xbfgragh.onion/2020/04/09/podcasts/still-processing-tiger-king.html?action=click\&module=audio-series-bar\&region=header\&pgtype=Article}{\includegraphics{https://static01.graylady3jvrrxbe.onion/images/2020/04/11/podcasts/09stillprocessing-image2/09stillprocessing-image2-thumbLarge-v2.jpg}}

April 9, 2020~~•~ 39:49Frosted Flakes

\href{https://www.nytimes3xbfgragh.onion/2020/04/02/podcasts/high-fidelity-zoe-kravitz.html?action=click\&module=audio-series-bar\&region=header\&pgtype=Article}{\includegraphics{https://static01.graylady3jvrrxbe.onion/images/2020/04/05/arts/02still-processing-highfidelity/13highfidelity-thumbLarge.jpg}}

April 2, 2020~~•~ 40:55Delicious Vinyl

\href{https://www.nytimes3xbfgragh.onion/2020/03/26/podcasts/still-processing-quarantine.html?action=click\&module=audio-series-bar\&region=header\&pgtype=Article}{\includegraphics{https://static01.graylady3jvrrxbe.onion/images/2020/03/29/podcasts/26stillprocessing1/26stillprocessing1-thumbLarge.jpg}}

March 26, 2020~~•~ 30:47A Pod From Both Our Houses

\href{https://www.nytimes3xbfgragh.onion/2019/11/07/podcasts/still-processing-parasite-watchmen-bong-joon-ho.html?action=click\&module=audio-series-bar\&region=header\&pgtype=Article}{\includegraphics{https://static01.graylady3jvrrxbe.onion/images/2019/11/08/arts/07stilpr-parasite/00parasite-1-thumbLarge.jpg}}

November 7, 2019Wake

\href{https://www.nytimes3xbfgragh.onion/column/still-processing-podcast}{See
All Episodes ofStill Processing}

Next

Published May 7, 2020Updated May 12, 2020

\begin{itemize}
\item
\item
\item
\item
\item
\end{itemize}

By \href{https://www.nytimes3xbfgragh.onion/by/wesley-morris}{Wesley
Morris} and
\href{https://www.nytimes3xbfgragh.onion/by/jenna-wortham}{Jenna
Wortham}

Covid-19 isn't ``the great equalizer'' --- except when it comes to
making us need our devices more than ever. Screens have revealed
superstars as civilians, and turned sitcom grouches into teddy bears.
Basically: We're ready to be more open with one another.

\includegraphics{https://static01.graylady3jvrrxbe.onion/images/2020/04/28/pageoneplus/28sondheimjp-sp/28sondheimjp-sp-articleLarge-v6.jpg?quality=75\&auto=webp\&disable=upscale}

Discussed this week:

\begin{itemize}
\item
  ``\href{https://youtu.be/A92wZIvEUAw}{Take Me to the World: A Sondheim
  90th Birthday Celebration}'' (Broadway.com)
\item
  ``\href{https://www.nytimes3xbfgragh.onion/2020/05/01/arts/quarantine-bookcase-coronavirus.html}{The
  `Credibility Bookcase' Is the Quarantine's Hottest Accessory}''
  (Amanda Hess, The New York Times)
\item
  Randy Rainbow's
  \href{https://www.youtube.com/channel/UC07F26kHKkpW_qqvXzEGALA}{YouTube
  Channel}
\item
  ``\href{https://www.youtube.com/watch?v=1Vl_oUMyGYU}{A Parks and
  Recreation Special}'' (NBC, 2020)
\item
  ``\href{https://www.ted.com/talks/brene_brown_the_power_of_vulnerability?language=en}{The
  Power of Vulnerability}'' (Brené Brown, TEDx)
\item
  ``\href{https://brenebrown.com/books-audio/}{Daring Greatly: How the
  Courage to Be Vulnerable Transforms the Way We Live, Love, Parent, and
  Lead}'' (Brené Brown, 2012)
\item
  The
  ``\href{https://brenebrown.com/podcast/introducing-unlocking-us/}{Unlocking
  Us}'' Podcast (Brené Brown)
\item
  \href{https://www.youtube.com/watch?v=Z9xRjR8iz9c}{Teddy Riley Verzuz
  Babyface}
\end{itemize}

``Still Processing'' is produced by Hans Buetow and edited by Sara
Sarasohn and Sasha Weiss, with editorial oversight from Wendy Dorr and
Lisa Tobin. Our engineer is Jake Gorski. Our theme music is by Kindness.
It's called ``World Restart,'' from the album ``Otherness.''

Advertisement

\protect\hyperlink{after-bottom}{Continue reading the main story}

\hypertarget{site-index}{%
\subsection{Site Index}\label{site-index}}

\hypertarget{site-information-navigation}{%
\subsection{Site Information
Navigation}\label{site-information-navigation}}

\begin{itemize}
\tightlist
\item
  \href{https://help.nytimes3xbfgragh.onion/hc/en-us/articles/115014792127-Copyright-notice}{©~2020~The
  New York Times Company}
\end{itemize}

\begin{itemize}
\tightlist
\item
  \href{https://www.nytco.com/}{NYTCo}
\item
  \href{https://help.nytimes3xbfgragh.onion/hc/en-us/articles/115015385887-Contact-Us}{Contact
  Us}
\item
  \href{https://www.nytco.com/careers/}{Work with us}
\item
  \href{https://nytmediakit.com/}{Advertise}
\item
  \href{http://www.tbrandstudio.com/}{T Brand Studio}
\item
  \href{https://www.nytimes3xbfgragh.onion/privacy/cookie-policy\#how-do-i-manage-trackers}{Your
  Ad Choices}
\item
  \href{https://www.nytimes3xbfgragh.onion/privacy}{Privacy}
\item
  \href{https://help.nytimes3xbfgragh.onion/hc/en-us/articles/115014893428-Terms-of-service}{Terms
  of Service}
\item
  \href{https://help.nytimes3xbfgragh.onion/hc/en-us/articles/115014893968-Terms-of-sale}{Terms
  of Sale}
\item
  \href{https://spiderbites.nytimes3xbfgragh.onion}{Site Map}
\item
  \href{https://help.nytimes3xbfgragh.onion/hc/en-us}{Help}
\item
  \href{https://www.nytimes3xbfgragh.onion/subscription?campaignId=37WXW}{Subscriptions}
\end{itemize}
