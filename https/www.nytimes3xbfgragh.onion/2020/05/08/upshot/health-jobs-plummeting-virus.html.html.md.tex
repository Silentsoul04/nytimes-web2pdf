Sections

SEARCH

\protect\hyperlink{site-content}{Skip to
content}\protect\hyperlink{site-index}{Skip to site index}

\href{https://myaccount.nytimes3xbfgragh.onion/auth/login?response_type=cookie\&client_id=vi}{}

\href{https://www.nytimes3xbfgragh.onion/section/todayspaper}{Today's
Paper}

\href{/section/upshot}{The Upshot}\textbar{}Why 1.4 Million Health Jobs
Have Been Lost During a Huge Health Crisis

\url{https://nyti.ms/2YJ0r7M}

\begin{itemize}
\item
\item
\item
\item
\item
\item
\end{itemize}

\href{https://www.nytimes3xbfgragh.onion/news-event/coronavirus?action=click\&pgtype=Article\&state=default\&region=TOP_BANNER\&context=storylines_menu}{The
Coronavirus Outbreak}

\begin{itemize}
\tightlist
\item
  live\href{https://www.nytimes3xbfgragh.onion/2020/08/01/world/coronavirus-covid-19.html?action=click\&pgtype=Article\&state=default\&region=TOP_BANNER\&context=storylines_menu}{Latest
  Updates}
\item
  \href{https://www.nytimes3xbfgragh.onion/interactive/2020/us/coronavirus-us-cases.html?action=click\&pgtype=Article\&state=default\&region=TOP_BANNER\&context=storylines_menu}{Maps
  and Cases}
\item
  \href{https://www.nytimes3xbfgragh.onion/interactive/2020/science/coronavirus-vaccine-tracker.html?action=click\&pgtype=Article\&state=default\&region=TOP_BANNER\&context=storylines_menu}{Vaccine
  Tracker}
\item
  \href{https://www.nytimes3xbfgragh.onion/interactive/2020/07/29/us/schools-reopening-coronavirus.html?action=click\&pgtype=Article\&state=default\&region=TOP_BANNER\&context=storylines_menu}{What
  School May Look Like}
\item
  \href{https://www.nytimes3xbfgragh.onion/live/2020/07/31/business/stock-market-today-coronavirus?action=click\&pgtype=Article\&state=default\&region=TOP_BANNER\&context=storylines_menu}{Economy}
\end{itemize}

Advertisement

\protect\hyperlink{after-top}{Continue reading the main story}

Upshot

Supported by

\protect\hyperlink{after-sponsor}{Continue reading the main story}

\hypertarget{why-14-million-health-jobs-have-been-lost-during-a-huge-health-crisis}{%
\section{Why 1.4 Million Health Jobs Have Been Lost During a Huge Health
Crisis}\label{why-14-million-health-jobs-have-been-lost-during-a-huge-health-crisis}}

The industry used to be recession-proof. Not anymore. The postponement
of elective procedures has crushed the bottom line.

\href{https://www.nytimes3xbfgragh.onion/by/margot-sanger-katz}{\includegraphics{https://static01.graylady3jvrrxbe.onion/images/2019/12/13/reader-center/author-margot-sanger-katz/author-margot-sanger-katz-thumbLarge.png}}

By
\href{https://www.nytimes3xbfgragh.onion/by/margot-sanger-katz}{Margot
Sanger-Katz}

\begin{itemize}
\item
  Published May 8, 2020Updated May 10, 2020
\item
  \begin{itemize}
  \item
  \item
  \item
  \item
  \item
  \item
  \end{itemize}
\end{itemize}

For more than half a century, in good economic times and bad, health
care jobs in the United States just kept increasing. Economists and
health analysts thought of them as nearly recession-proof: a buffer
against the business cycle.

But like so many other patterns, the coronavirus pandemic has broken
this relationship. With the virus and its fallout deterring Americans
from using the health system, job losses started in March and
accelerated to 1.4 million last month.

``This is a disruption unlike any we've seen in decades,'' said Ani
Turner, the co-director of sustainable health spending strategies at the
Altarum Institute, which tracks trends in health care spending and
employment. Ms. Turner recently wrote an
\href{https://altarum.org/news/health-care-sheds-43000-jobs-march-biggest-loss-least-three-decades}{essay}
titled ``Health Sector Won't Be the Backstop in This Downturn.''

A sudden drop in health spending and
\href{https://www.nytimes3xbfgragh.onion/2020/04/03/us/politics/coronavirus-health-care-workers-layoffs.html}{employment
amid a pandemic} might seem like a paradox. But it reflects how the
health industry tends to make its money: Treating patients for a deadly
illness is far less profitable than offering them elective surgeries.
When the federal government
\href{https://www.cms.gov/files/document/covid-elective-surgery-recommendations.pdf}{asked}
hospitals to stop such procedures to free up capacity, that changed
their economics profoundly.

\hypertarget{latest-updates-global-coronavirus-outbreak}{%
\section{\texorpdfstring{\href{https://www.nytimes3xbfgragh.onion/2020/08/01/world/coronavirus-covid-19.html?action=click\&pgtype=Article\&state=default\&region=MAIN_CONTENT_1\&context=storylines_live_updates}{Latest
Updates: Global Coronavirus
Outbreak}}{Latest Updates: Global Coronavirus Outbreak}}\label{latest-updates-global-coronavirus-outbreak}}

Updated 2020-08-01T19:54:00.494Z

\begin{itemize}
\tightlist
\item
  \href{https://www.nytimes3xbfgragh.onion/2020/08/01/world/coronavirus-covid-19.html?action=click\&pgtype=Article\&state=default\&region=MAIN_CONTENT_1\&context=storylines_live_updates\#link-3ac56579}{Top
  officials work to break impasse over jobless benefit.}
\item
  \href{https://www.nytimes3xbfgragh.onion/2020/08/01/world/coronavirus-covid-19.html?action=click\&pgtype=Article\&state=default\&region=MAIN_CONTENT_1\&context=storylines_live_updates\#link-8796723}{The
  virus picks up dangerous speed in the Midwest, and in areas that had
  seen success.}
\item
  \href{https://www.nytimes3xbfgragh.onion/2020/08/01/world/coronavirus-covid-19.html?action=click\&pgtype=Article\&state=default\&region=MAIN_CONTENT_1\&context=storylines_live_updates\#link-25930521}{Thousands
  in Berlin protest Germany's coronavirus measures.}
\end{itemize}

\href{https://www.nytimes3xbfgragh.onion/2020/08/01/world/coronavirus-covid-19.html?action=click\&pgtype=Article\&state=default\&region=MAIN_CONTENT_1\&context=storylines_live_updates}{See
more updates}

More live coverage:
\href{https://www.nytimes3xbfgragh.onion/live/2020/07/31/business/stock-market-today-coronavirus?action=click\&pgtype=Article\&state=default\&region=MAIN_CONTENT_1\&context=storylines_live_updates}{Markets}

And even if various governments across the nation hadn't then ordered
such a pause, many patients would have probably avoided doctors' offices
and hospitals anyway, to lessen the risk of contracting the coronavirus.
Independent medical practices have seen
\href{https://www.nytimes3xbfgragh.onion/2020/05/05/health/coronavirus-primary-care-doctor.html}{huge
reductions} in their business, as some patients connect with doctors
virtually, while many others patient visits have simply vanished.

In previous recessions, the health industry has not taken such a hit.
Because most Americans have health insurance, health services are more
insulated from the business cycle than other kinds of spending. The
biggest users of the system are older Americans. Besides being likelier
to have health problems, they also tend to have comprehensive insurance
coverage through Medicare and a stable source of income from Social
Security. The Medicaid program, which is structured to allow people to
enroll when their incomes fall, tends to offer access to health care for
the poor, even if their jobs disappear.

And, of course, many of the problems that send people to the doctor ---
heart disease, appendicitis, cancer or the flu --- do not go away during
a struggling economy. As a result, the doctors and nurses and medical
assistants and billing clerks who work in health care are usually
protected from an economic downturn.

But lately Medicare beneficiaries, the age group at highest risk of
serious disease or death from the coronavirus, have been particularly
spooked from seeking medical care. And emergency rooms have reported
\href{https://www.nytimes3xbfgragh.onion/2020/04/06/well/live/coronavirus-doctors-hospitals-emergency-care-heart-attack-stroke.html}{shocking
declines} in visits for what doctors had always thought of as
life-threatening emergencies. A
\href{http://www.onlinejacc.org/content/accj/early/2020/04/07/j.jacc.2020.04.011.full.pdf}{recent
survey} of nine major U.S. hospitals showed visits for a common but
serious kind of heart attack have fallen by nearly 40 percent. Those
declines were seen even in places without major coronavirus outbreaks.

The industry still seems somewhat protected: Health care jobs have
\href{https://twitter.com/Altarum_SHSS/status/1258745215382413312}{fallen
by less} than jobs in the rest of the economy. But in the Great
Recession, as jobs of nearly every kind plummeted, health jobs kept
growing at a good clip. In the eyes of many economists, it was health
care that led the economic recovery, by providing a powerful and
reliable jobs engine. All those new health workers helped strengthen
their local economies.

And in many places hit hard by the recession, the local hospital became
a dominant employer, supporting entire communities. There were some
small reductions in the use of health services in the following years;
those led to
\href{https://www.nytimes3xbfgragh.onion/2014/08/28/upshot/medicare-not-such-a-budget-buster-anymore.html}{a
slowdown} in what had been seen as the inexorable growth of health
spending. Those changes, which earned the nickname
``\href{https://www.advisory.com/daily-briefing/blog/2014/03/health-care-slowth-continues}{slowth}''
among health care analysts, were
\href{https://www.nytimes3xbfgragh.onion/2014/12/04/upshot/good-news-inside-the-health-spending-numbers.html}{unusual
enough} to be
\href{https://www.nytimes3xbfgragh.onion/2013/01/08/us/health-spending-growth-stays-low-for-third-straight-year.html}{major
news}. Health care still grew, however, just more slowly. About 60
percent of health care spending goes to labor, so all the growth, almost
by definition, meant new jobs.

That growth was seen as good news in the short term for the economy. But
health care's ever-growing share of the national economy has
\href{http://www.sangerkatz.com/uploads/1/5/0/0/15005400/cover_story-the_false_promise.pdf}{not
always been seen} in a positive light. The rising costs of the Medicare
and Medicaid programs have meant that health care has come to represent
a larger share of the federal budget, limiting capacity for other
government investments. Strain under the growing costs has caused many
employers to switch from older forms of insurance, which were
comprehensive, to plans with high deductibles for workers,
\href{https://www.nytimes3xbfgragh.onion/2019/09/25/health/employer-health-insurance-cost.html}{shifting
the financial burden} onto individuals and families, and leaving them
vulnerable to big bills.

This downturn is clearly different, and the enormous reductions in the
health work force mean the recovery may be different, too. Some of the
lost jobs in health care are likely to come back later. Cancer patients
who postponed chemotherapy, or people who canceled their hip
replacements, will eventually want that care. But other changes may be
permanent.

\href{https://www.nytimes3xbfgragh.onion/news-event/coronavirus?action=click\&pgtype=Article\&state=default\&region=MAIN_CONTENT_3\&context=storylines_faq}{}

\hypertarget{the-coronavirus-outbreak-}{%
\subsubsection{The Coronavirus Outbreak
›}\label{the-coronavirus-outbreak-}}

\hypertarget{frequently-asked-questions}{%
\paragraph{Frequently Asked
Questions}\label{frequently-asked-questions}}

Updated July 27, 2020

\begin{itemize}
\item ~
  \hypertarget{should-i-refinance-my-mortgage}{%
  \paragraph{Should I refinance my
  mortgage?}\label{should-i-refinance-my-mortgage}}

  \begin{itemize}
  \tightlist
  \item
    \href{https://www.nytimes3xbfgragh.onion/article/coronavirus-money-unemployment.html?action=click\&pgtype=Article\&state=default\&region=MAIN_CONTENT_3\&context=storylines_faq}{It
    could be a good idea,} because mortgage rates have
    \href{https://www.nytimes3xbfgragh.onion/2020/07/16/business/mortgage-rates-below-3-percent.html?action=click\&pgtype=Article\&state=default\&region=MAIN_CONTENT_3\&context=storylines_faq}{never
    been lower.} Refinancing requests have pushed mortgage applications
    to some of the highest levels since 2008, so be prepared to get in
    line. But defaults are also up, so if you're thinking about buying a
    home, be aware that some lenders have tightened their standards.
  \end{itemize}
\item ~
  \hypertarget{what-is-school-going-to-look-like-in-september}{%
  \paragraph{What is school going to look like in
  September?}\label{what-is-school-going-to-look-like-in-september}}

  \begin{itemize}
  \tightlist
  \item
    It is unlikely that many schools will return to a normal schedule
    this fall, requiring the grind of
    \href{https://www.nytimes3xbfgragh.onion/2020/06/05/us/coronavirus-education-lost-learning.html?action=click\&pgtype=Article\&state=default\&region=MAIN_CONTENT_3\&context=storylines_faq}{online
    learning},
    \href{https://www.nytimes3xbfgragh.onion/2020/05/29/us/coronavirus-child-care-centers.html?action=click\&pgtype=Article\&state=default\&region=MAIN_CONTENT_3\&context=storylines_faq}{makeshift
    child care} and
    \href{https://www.nytimes3xbfgragh.onion/2020/06/03/business/economy/coronavirus-working-women.html?action=click\&pgtype=Article\&state=default\&region=MAIN_CONTENT_3\&context=storylines_faq}{stunted
    workdays} to continue. California's two largest public school
    districts --- Los Angeles and San Diego --- said on July 13, that
    \href{https://www.nytimes3xbfgragh.onion/2020/07/13/us/lausd-san-diego-school-reopening.html?action=click\&pgtype=Article\&state=default\&region=MAIN_CONTENT_3\&context=storylines_faq}{instruction
    will be remote-only in the fall}, citing concerns that surging
    coronavirus infections in their areas pose too dire a risk for
    students and teachers. Together, the two districts enroll some
    825,000 students. They are the largest in the country so far to
    abandon plans for even a partial physical return to classrooms when
    they reopen in August. For other districts, the solution won't be an
    all-or-nothing approach.
    \href{https://bioethics.jhu.edu/research-and-outreach/projects/eschool-initiative/school-policy-tracker/}{Many
    systems}, including the nation's largest, New York City, are
    devising
    \href{https://www.nytimes3xbfgragh.onion/2020/06/26/us/coronavirus-schools-reopen-fall.html?action=click\&pgtype=Article\&state=default\&region=MAIN_CONTENT_3\&context=storylines_faq}{hybrid
    plans} that involve spending some days in classrooms and other days
    online. There's no national policy on this yet, so check with your
    municipal school system regularly to see what is happening in your
    community.
  \end{itemize}
\item ~
  \hypertarget{is-the-coronavirus-airborne}{%
  \paragraph{Is the coronavirus
  airborne?}\label{is-the-coronavirus-airborne}}

  \begin{itemize}
  \tightlist
  \item
    The coronavirus
    \href{https://www.nytimes3xbfgragh.onion/2020/07/04/health/239-experts-with-one-big-claim-the-coronavirus-is-airborne.html?action=click\&pgtype=Article\&state=default\&region=MAIN_CONTENT_3\&context=storylines_faq}{can
    stay aloft for hours in tiny droplets in stagnant air}, infecting
    people as they inhale, mounting scientific evidence suggests. This
    risk is highest in crowded indoor spaces with poor ventilation, and
    may help explain super-spreading events reported in meatpacking
    plants, churches and restaurants.
    \href{https://www.nytimes3xbfgragh.onion/2020/07/06/health/coronavirus-airborne-aerosols.html?action=click\&pgtype=Article\&state=default\&region=MAIN_CONTENT_3\&context=storylines_faq}{It's
    unclear how often the virus is spread} via these tiny droplets, or
    aerosols, compared with larger droplets that are expelled when a
    sick person coughs or sneezes, or transmitted through contact with
    contaminated surfaces, said Linsey Marr, an aerosol expert at
    Virginia Tech. Aerosols are released even when a person without
    symptoms exhales, talks or sings, according to Dr. Marr and more
    than 200 other experts, who
    \href{https://academic.oup.com/cid/article/doi/10.1093/cid/ciaa939/5867798}{have
    outlined the evidence in an open letter to the World Health
    Organization}.
  \end{itemize}
\item ~
  \hypertarget{what-are-the-symptoms-of-coronavirus}{%
  \paragraph{What are the symptoms of
  coronavirus?}\label{what-are-the-symptoms-of-coronavirus}}

  \begin{itemize}
  \tightlist
  \item
    Common symptoms
    \href{https://www.nytimes3xbfgragh.onion/article/symptoms-coronavirus.html?action=click\&pgtype=Article\&state=default\&region=MAIN_CONTENT_3\&context=storylines_faq}{include
    fever, a dry cough, fatigue and difficulty breathing or shortness of
    breath.} Some of these symptoms overlap with those of the flu,
    making detection difficult, but runny noses and stuffy sinuses are
    less common.
    \href{https://www.nytimes3xbfgragh.onion/2020/04/27/health/coronavirus-symptoms-cdc.html?action=click\&pgtype=Article\&state=default\&region=MAIN_CONTENT_3\&context=storylines_faq}{The
    C.D.C. has also} added chills, muscle pain, sore throat, headache
    and a new loss of the sense of taste or smell as symptoms to look
    out for. Most people fall ill five to seven days after exposure, but
    symptoms may appear in as few as two days or as many as 14 days.
  \end{itemize}
\item ~
  \hypertarget{does-asymptomatic-transmission-of-covid-19-happen}{%
  \paragraph{Does asymptomatic transmission of Covid-19
  happen?}\label{does-asymptomatic-transmission-of-covid-19-happen}}

  \begin{itemize}
  \tightlist
  \item
    So far, the evidence seems to show it does. A widely cited
    \href{https://www.nature.com/articles/s41591-020-0869-5}{paper}
    published in April suggests that people are most infectious about
    two days before the onset of coronavirus symptoms and estimated that
    44 percent of new infections were a result of transmission from
    people who were not yet showing symptoms. Recently, a top expert at
    the World Health Organization stated that transmission of the
    coronavirus by people who did not have symptoms was ``very rare,''
    \href{https://www.nytimes3xbfgragh.onion/2020/06/09/world/coronavirus-updates.html?action=click\&pgtype=Article\&state=default\&region=MAIN_CONTENT_3\&context=storylines_faq\#link-1f302e21}{but
    she later walked back that statement.}
  \end{itemize}
\end{itemize}

Over just a few weeks, there has been a shift to
\href{https://www.nytimes3xbfgragh.onion/2020/03/11/health/telemedicine-coronavirus.html}{telemedicine
visits}, in which patients can talk with their doctors by video,
telephone or even email. Many health systems had long avoided setting up
an infrastructure for them. But the new virtual visits have many
advantages: Patients can avoid travel and waiting rooms; caregivers can
easily participate; and several doctors can talk to the same patient at
once. There are
\href{https://www.nytimes3xbfgragh.onion/2016/09/08/upshot/release-your-medical-records-first-you-must-collect-them.html}{no
clipboards} full of personal information to fill out by hand, and no
person needed to retype the answers into a computer system before such
visits.

Other changes are more invisible. Hospitals have been forced to find
small ways to digitize processes and share records that used to involve
labor and bureaucracy. Bob Kocher, a partner at the venture capital firm
Venrock, has been acting as an adviser to Gov. Gavin Newsom of
California through the crisis. He said he had seen hospitals adapt
quickly to begin sharing their daily bed counts with one another and the
state, for example, a task that used to involve hours of phone calls and
faxes. Data about laboratory testing --- how much is being done, how
many patients have been found to have coronavirus --- has been similarly
digitized in real time.

Are those sorts of on-the-fly tweaks enough to have an appreciable
effect on long-term health care employment? Mr. Kocher thinks so,
describing the recent cuts in health care administration as a ``silver
lining'' of the crisis, while acknowledging the short-term pain of the
job losses.

``I will passionately say that's a good thing, because health care costs
in America are high, and most of the health care cost in America is
labor,'' he said.

Another possibility is that the financial shock hitting smaller and less
capitalized hospitals and physician practices could lead to a wave of
consolidation, as bigger competitors gobble them up for a discount.
Those kinds of mergers have typically led to rising health costs, since
bigger systems can demand higher prices from private health insurers.
That could make the legacy of this shock a more expensive health system
instead of a cheaper one.

``We have a great opportunity ahead of us, but I don't see us making any
progress,'' said Amitabh Chandra, a health economist at Harvard, who
once was a
\href{https://jamanetwork.com/journals/jama/article-abstract/2678994}{co-author
of an essay} describing health care as ``recession-free,'' and not in a
good way. ``I think we'll go back to business as usual.''

Either way, health care is unlikely be the economic stabilizer it has
been in the past. The loss of industry jobs on top of the major losses
in other sectors are likely to make the recession deeper, and the
recovery slower.

Advertisement

\protect\hyperlink{after-bottom}{Continue reading the main story}

\hypertarget{site-index}{%
\subsection{Site Index}\label{site-index}}

\hypertarget{site-information-navigation}{%
\subsection{Site Information
Navigation}\label{site-information-navigation}}

\begin{itemize}
\tightlist
\item
  \href{https://help.nytimes3xbfgragh.onion/hc/en-us/articles/115014792127-Copyright-notice}{©~2020~The
  New York Times Company}
\end{itemize}

\begin{itemize}
\tightlist
\item
  \href{https://www.nytco.com/}{NYTCo}
\item
  \href{https://help.nytimes3xbfgragh.onion/hc/en-us/articles/115015385887-Contact-Us}{Contact
  Us}
\item
  \href{https://www.nytco.com/careers/}{Work with us}
\item
  \href{https://nytmediakit.com/}{Advertise}
\item
  \href{http://www.tbrandstudio.com/}{T Brand Studio}
\item
  \href{https://www.nytimes3xbfgragh.onion/privacy/cookie-policy\#how-do-i-manage-trackers}{Your
  Ad Choices}
\item
  \href{https://www.nytimes3xbfgragh.onion/privacy}{Privacy}
\item
  \href{https://help.nytimes3xbfgragh.onion/hc/en-us/articles/115014893428-Terms-of-service}{Terms
  of Service}
\item
  \href{https://help.nytimes3xbfgragh.onion/hc/en-us/articles/115014893968-Terms-of-sale}{Terms
  of Sale}
\item
  \href{https://spiderbites.nytimes3xbfgragh.onion}{Site Map}
\item
  \href{https://help.nytimes3xbfgragh.onion/hc/en-us}{Help}
\item
  \href{https://www.nytimes3xbfgragh.onion/subscription?campaignId=37WXW}{Subscriptions}
\end{itemize}
