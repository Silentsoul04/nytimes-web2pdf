Sections

SEARCH

\protect\hyperlink{site-content}{Skip to
content}\protect\hyperlink{site-index}{Skip to site index}

\href{https://www.nytimes3xbfgragh.onion/section/world/asia}{Asia
Pacific}

\href{https://myaccount.nytimes3xbfgragh.onion/auth/login?response_type=cookie\&client_id=vi}{}

\href{https://www.nytimes3xbfgragh.onion/section/todayspaper}{Today's
Paper}

\href{/section/world/asia}{Asia Pacific}\textbar{}As South Korea Eases
Limits, Virus Cluster Prompts Seoul to Close Bars

\url{https://nyti.ms/2LfXgMY}

\begin{itemize}
\item
\item
\item
\item
\item
\end{itemize}

\hypertarget{the-coronavirus-outbreak}{%
\subsubsection{\texorpdfstring{\href{https://www.nytimes3xbfgragh.onion/news-event/coronavirus?name=styln-coronavirus-national\&region=TOP_BANNER\&variant=undefined\&block=storyline_menu_recirc\&action=click\&pgtype=Article\&impression_id=5c719a90-e396-11ea-8c52-fd2265771867}{The
Coronavirus
Outbreak}}{The Coronavirus Outbreak}}\label{the-coronavirus-outbreak}}

\begin{itemize}
\tightlist
\item
  live\href{https://www.nytimes3xbfgragh.onion/2020/08/21/world/covid-19-coronavirus.html?name=styln-coronavirus-national\&region=TOP_BANNER\&variant=undefined\&block=storyline_menu_recirc\&action=click\&pgtype=Article\&impression_id=5c719a91-e396-11ea-8c52-fd2265771867}{Latest
  Updates}
\item
  \href{https://www.nytimes3xbfgragh.onion/interactive/2020/us/coronavirus-us-cases.html?name=styln-coronavirus-national\&region=TOP_BANNER\&variant=undefined\&block=storyline_menu_recirc\&action=click\&pgtype=Article\&impression_id=5c719a92-e396-11ea-8c52-fd2265771867}{Maps
  and Cases}
\item
  \href{https://www.nytimes3xbfgragh.onion/interactive/2020/science/coronavirus-vaccine-tracker.html?name=styln-coronavirus-national\&region=TOP_BANNER\&variant=undefined\&block=storyline_menu_recirc\&action=click\&pgtype=Article\&impression_id=5c71c1a0-e396-11ea-8c52-fd2265771867}{Vaccine
  Tracker}
\item
  \href{https://www.nytimes3xbfgragh.onion/2020/08/19/us/colleges-closing-covid.html?name=styln-coronavirus-national\&region=TOP_BANNER\&variant=undefined\&block=storyline_menu_recirc\&action=click\&pgtype=Article\&impression_id=5c71c1a1-e396-11ea-8c52-fd2265771867}{Colleges
  Closing}
\item
  \href{https://www.nytimes3xbfgragh.onion/live/2020/08/20/business/stock-market-today-coronavirus?name=styln-coronavirus-national\&region=TOP_BANNER\&variant=undefined\&block=storyline_menu_recirc\&action=click\&pgtype=Article\&impression_id=5c71c1a2-e396-11ea-8c52-fd2265771867}{Economy}
\end{itemize}

Advertisement

\protect\hyperlink{after-top}{Continue reading the main story}

Supported by

\protect\hyperlink{after-sponsor}{Continue reading the main story}

\hypertarget{as-south-korea-eases-limits-virus-cluster-prompts-seoul-to-close-bars}{%
\section{As South Korea Eases Limits, Virus Cluster Prompts Seoul to
Close
Bars}\label{as-south-korea-eases-limits-virus-cluster-prompts-seoul-to-close-bars}}

With no vaccine in sight, the country is urging people to reclaim more
of their daily lives, while bracing for new waves that are all but
inevitable.

\includegraphics{https://static01.graylady3jvrrxbe.onion/images/2020/05/10/world/10skorea-secondwave/merlin_172261203_89ee440c-e710-49a5-aadb-299d8a0dba97-articleLarge.jpg?quality=75\&auto=webp\&disable=upscale}

\href{https://www.nytimes3xbfgragh.onion/by/choe-sang-hun}{\includegraphics{https://static01.graylady3jvrrxbe.onion/images/2018/07/18/multimedia/author-choe-sang-hun/author-choe-sang-hun-thumbLarge.png}}

By \href{https://www.nytimes3xbfgragh.onion/by/choe-sang-hun}{Choe
Sang-Hun}

\begin{itemize}
\item
  Published May 9, 2020Updated May 15, 2020
\item
  \begin{itemize}
  \item
  \item
  \item
  \item
  \item
  \end{itemize}
\end{itemize}

Go out, socialize and have fun, South Korea's government told its
people, declaring the start of ``a new daily life with Covid-19'' ---
while keeping a vigilant eye out for any sign of backsliding, any need
for restrictions to snap back into place.

It didn't take long.

On Saturday, just the fourth day of the new phase, the mayor of Seoul
ordered all the capital's bars and nightclubs shut down indefinitely
after the discovery of a cluster of dozens of coronavirus infections.

South Korea initially attacked the pandemic with such success that it
became a model cited worldwide,
\href{https://www.nytimes3xbfgragh.onion/2020/03/23/world/asia/coronavirus-south-korea-flatten-curve.html}{all
but halting a large outbreak} without choking off nearly as much of its
economy
\href{https://www.nytimes3xbfgragh.onion/2020/04/29/business/economy/us-gdp.html}{as
other nations have}. Now it is attempting something just as difficult:
moving gradually, safely closer to something resembling everyday life.

Government officials, health workers and much of the public know full
well that until there is a vaccine, relaxing restrictions will lead to
more infections, and possibly more deaths. The trick will be to do it
without allowing the contagion to come roaring back.

\includegraphics{https://static01.graylady3jvrrxbe.onion/images/2020/05/08/world/xxskorea-secondwave-2/merlin_172261206_3424bc4b-0e98-469b-9fe5-768b47510fca-articleLarge.jpg?quality=75\&auto=webp\&disable=upscale}

Other nations, eager to reopen but fearful of the consequences, will be
watching closely to see what happens in South Korea.

``A second wave is inevitable,'' said Son Young-rae, a senior
epidemiological strategist at the government's Central Disaster
Management Headquarters. ``But we are running a constant monitoring and
screening system throughout our society so that we can prevent it from
exploding rapidly into hundreds or thousands of cases like the one we
had in the past.''

\hypertarget{latest-updates-the-coronavirus-outbreak}{%
\section{\texorpdfstring{\href{https://www.nytimes3xbfgragh.onion/2020/08/21/world/covid-19-coronavirus.html?action=click\&pgtype=Article\&state=default\&region=MAIN_CONTENT_1\&context=storylines_live_updates}{Latest
Updates: The Coronavirus
Outbreak}}{Latest Updates: The Coronavirus Outbreak}}\label{latest-updates-the-coronavirus-outbreak}}

Updated 2020-08-21T09:57:24.778Z

\begin{itemize}
\tightlist
\item
  \href{https://www.nytimes3xbfgragh.onion/2020/08/21/world/covid-19-coronavirus.html?action=click\&pgtype=Article\&state=default\&region=MAIN_CONTENT_1\&context=storylines_live_updates\#link-4690b6aa}{Shutdowns,
  warnings and scoldings follow gatherings on college campuses.}
\item
  \href{https://www.nytimes3xbfgragh.onion/2020/08/21/world/covid-19-coronavirus.html?action=click\&pgtype=Article\&state=default\&region=MAIN_CONTENT_1\&context=storylines_live_updates\#link-324af071}{As
  he accepts the Democratic nomination, Biden knocks Trump's pandemic
  response.}
\item
  \href{https://www.nytimes3xbfgragh.onion/2020/08/21/world/covid-19-coronavirus.html?action=click\&pgtype=Article\&state=default\&region=MAIN_CONTENT_1\&context=storylines_live_updates\#link-35890b73}{Hundreds
  of doctors in Kenya go on strike over their pay and protective gear.}
\end{itemize}

\href{https://www.nytimes3xbfgragh.onion/2020/08/21/world/covid-19-coronavirus.html?action=click\&pgtype=Article\&state=default\&region=MAIN_CONTENT_1\&context=storylines_live_updates}{See
more updates}

More live coverage:
\href{https://www.nytimes3xbfgragh.onion/live/2020/08/20/business/stock-market-today-coronavirus?action=click\&pgtype=Article\&state=default\&region=MAIN_CONTENT_1\&context=storylines_live_updates}{Markets}

``We hope to slow the spread and keep the size down to small, sporadic
outbreaks, hopefully of 20 to 30 cases, that come and go,'' he said,
``so that we can handle them while the people go on with their daily
lives.''

South Korea has had nearly 11,000 confirmed cases of the virus and
reported 256 deaths. But it has
\href{https://www.nytimes3xbfgragh.onion/interactive/2020/03/19/world/coronavirus-flatten-the-curve-countries.html?searchResultPosition=360}{slowed
the spread}from several hundred new infections recorded daily in late
February and early March, to around 10 per day in recent weeks.

The country adopted a massive, multipronged approach, including
aggressive testing and contact-tracing, near-universal use of masks,
social distancing, and localized clampdowns on hot spots. It was aided
by a high degree of public cooperation.

Now it is counting on the same tools to prevent a resurgence, creating a
new strategy on the fly.

Image

In Suwon's professional baseball park, games resumed this week, but with
no spectators allowed in the stands.Credit...Woohae Cho for The New York
Times

``We can't sustain our society with our daily life and economic
activities standing still​,'' said Health Minister Park Neung-hoo.​
``But unfortunately, we could not find a precedent for what we are
trying to do​. More likely, our experience, with its trials and errors,
will serve as a reference for other nations​ down the road.''

After a 29-year-old man tested positive for the virus on Wednesday,
epidemiologists quickly learned that he had visited three nightclubs in
Itaewon, a popular nightlife district in Seoul, on May 2. By Saturday
evening, they said they were tracking down 7,200 people who had visited
five Itaewon nightclubs where the virus might have been spread.

So far, 27 cases have been found among the club-goers and people who had
close contact with them, Kwon Jun-wok, a senior disease-control
official, said during a news briefing on Saturday.

The
\href{https://www.nytimes3xbfgragh.onion/2020/07/09/world/asia/seoul-mayor-missing.html}{mayor,
Park Won-soon}, cited a higher figure, saying that at least 40
infections had been linked to the nightclubs. As he closed the clubs, he
scolded patrons who had failed to practice safeguards like wearing
masks, accusing them of putting the entire nation's health at risk.

``Just because of a few people's carelessness, all our efforts so far
can go to waste,'' he said.

Under the newly eased policy that went into effect on Wednesday, the
government is urging people to reclaim pieces of their daily lives, and
gathering places like schools, museums, libraries, stadiums and concert
halls are expected to reopen in the coming weeks.

Image

Employees during the Korea Baseball Organization League in
Suwon.Credit...Woohae Cho for The New York Times

If it weren't for the ubiquitous masks, South Korean cities these days
would look almost as they did before the virus. Subways have filled up
with commuters. Long lines have started forming on sidewalks in Seoul,
not to buy masks but to get seats in favored restaurants.

​The government estimates that the medical system can ​comfortably
​control Covid-19 if there are fewer than 50 new cases per day, and
epidemiologists can trace the source of infection at least 95 percent of
the time --- milestones the country passed last month.

\href{https://www.nytimes3xbfgragh.onion/news-event/coronavirus?action=click\&pgtype=Article\&state=default\&region=MAIN_CONTENT_3\&context=storylines_faq}{}

\hypertarget{the-coronavirus-outbreak-}{%
\subsubsection{The Coronavirus Outbreak
›}\label{the-coronavirus-outbreak-}}

\hypertarget{frequently-asked-questions}{%
\paragraph{Frequently Asked
Questions}\label{frequently-asked-questions}}

Updated August 17, 2020

\begin{itemize}
\item ~
  \hypertarget{why-does-standing-six-feet-away-from-others-help}{%
  \paragraph{Why does standing six feet away from others
  help?}\label{why-does-standing-six-feet-away-from-others-help}}

  \begin{itemize}
  \tightlist
  \item
    The coronavirus spreads primarily through droplets from your mouth
    and nose, especially when you cough or sneeze. The C.D.C., one of
    the organizations using that measure,
    \href{https://www.nytimes3xbfgragh.onion/2020/04/14/health/coronavirus-six-feet.html?action=click\&pgtype=Article\&state=default\&region=MAIN_CONTENT_3\&context=storylines_faq}{bases
    its recommendation of six feet} on the idea that most large droplets
    that people expel when they cough or sneeze will fall to the ground
    within six feet. But six feet has never been a magic number that
    guarantees complete protection. Sneezes, for instance, can launch
    droplets a lot farther than six feet,
    \href{https://jamanetwork.com/journals/jama/fullarticle/2763852}{according
    to a recent study}. It's a rule of thumb: You should be safest
    standing six feet apart outside, especially when it's windy. But
    keep a mask on at all times, even when you think you're far enough
    apart.
  \end{itemize}
\item ~
  \hypertarget{i-have-antibodies-am-i-now-immune}{%
  \paragraph{I have antibodies. Am I now
  immune?}\label{i-have-antibodies-am-i-now-immune}}

  \begin{itemize}
  \tightlist
  \item
    As of right
    now,\href{https://www.nytimes3xbfgragh.onion/2020/07/22/health/covid-antibodies-herd-immunity.html?action=click\&pgtype=Article\&state=default\&region=MAIN_CONTENT_3\&context=storylines_faq}{that
    seems likely, for at least several months.} There have been
    frightening accounts of people suffering what seems to be a second
    bout of Covid-19. But experts say these patients may have a
    drawn-out course of infection, with the virus taking a slow toll
    weeks to months after initial exposure. People infected with the
    coronavirus typically
    \href{https://www.nature.com/articles/s41586-020-2456-9}{produce}
    immune molecules called antibodies, which are
    \href{https://www.nytimes3xbfgragh.onion/2020/05/07/health/coronavirus-antibody-prevalence.html?action=click\&pgtype=Article\&state=default\&region=MAIN_CONTENT_3\&context=storylines_faq}{protective
    proteins made in response to an
    infection}\href{https://www.nytimes3xbfgragh.onion/2020/05/07/health/coronavirus-antibody-prevalence.html?action=click\&pgtype=Article\&state=default\&region=MAIN_CONTENT_3\&context=storylines_faq}{.
    These antibodies may} last in the body
    \href{https://www.nature.com/articles/s41591-020-0965-6}{only two to
    three months}, which may seem worrisome, but that's perfectly normal
    after an acute infection subsides, said Dr. Michael Mina, an
    immunologist at Harvard University. It may be possible to get the
    coronavirus again, but it's highly unlikely that it would be
    possible in a short window of time from initial infection or make
    people sicker the second time.
  \end{itemize}
\item ~
  \hypertarget{im-a-small-business-owner-can-i-get-relief}{%
  \paragraph{I'm a small-business owner. Can I get
  relief?}\label{im-a-small-business-owner-can-i-get-relief}}

  \begin{itemize}
  \tightlist
  \item
    The
    \href{https://www.nytimes3xbfgragh.onion/article/small-business-loans-stimulus-grants-freelancers-coronavirus.html?action=click\&pgtype=Article\&state=default\&region=MAIN_CONTENT_3\&context=storylines_faq}{stimulus
    bills enacted in March} offer help for the millions of American
    small businesses. Those eligible for aid are businesses and
    nonprofit organizations with fewer than 500 workers, including sole
    proprietorships, independent contractors and freelancers. Some
    larger companies in some industries are also eligible. The help
    being offered, which is being managed by the Small Business
    Administration, includes the Paycheck Protection Program and the
    Economic Injury Disaster Loan program. But lots of folks have
    \href{https://www.nytimes3xbfgragh.onion/interactive/2020/05/07/business/small-business-loans-coronavirus.html?action=click\&pgtype=Article\&state=default\&region=MAIN_CONTENT_3\&context=storylines_faq}{not
    yet seen payouts.} Even those who have received help are confused:
    The rules are draconian, and some are stuck sitting on
    \href{https://www.nytimes3xbfgragh.onion/2020/05/02/business/economy/loans-coronavirus-small-business.html?action=click\&pgtype=Article\&state=default\&region=MAIN_CONTENT_3\&context=storylines_faq}{money
    they don't know how to use.} Many small-business owners are getting
    less than they expected or
    \href{https://www.nytimes3xbfgragh.onion/2020/06/10/business/Small-business-loans-ppp.html?action=click\&pgtype=Article\&state=default\&region=MAIN_CONTENT_3\&context=storylines_faq}{not
    hearing anything at all.}
  \end{itemize}
\item ~
  \hypertarget{what-are-my-rights-if-i-am-worried-about-going-back-to-work}{%
  \paragraph{What are my rights if I am worried about going back to
  work?}\label{what-are-my-rights-if-i-am-worried-about-going-back-to-work}}

  \begin{itemize}
  \tightlist
  \item
    Employers have to provide
    \href{https://www.osha.gov/SLTC/covid-19/standards.html}{a safe
    workplace} with policies that protect everyone equally.
    \href{https://www.nytimes3xbfgragh.onion/article/coronavirus-money-unemployment.html?action=click\&pgtype=Article\&state=default\&region=MAIN_CONTENT_3\&context=storylines_faq}{And
    if one of your co-workers tests positive for the coronavirus, the
    C.D.C.} has said that
    \href{https://www.cdc.gov/coronavirus/2019-ncov/community/guidance-business-response.html}{employers
    should tell their employees} -\/- without giving you the sick
    employee's name -\/- that they may have been exposed to the virus.
  \end{itemize}
\item ~
  \hypertarget{what-is-school-going-to-look-like-in-september}{%
  \paragraph{What is school going to look like in
  September?}\label{what-is-school-going-to-look-like-in-september}}

  \begin{itemize}
  \tightlist
  \item
    It is unlikely that many schools will return to a normal schedule
    this fall, requiring the grind of
    \href{https://www.nytimes3xbfgragh.onion/2020/06/05/us/coronavirus-education-lost-learning.html?action=click\&pgtype=Article\&state=default\&region=MAIN_CONTENT_3\&context=storylines_faq}{online
    learning},
    \href{https://www.nytimes3xbfgragh.onion/2020/05/29/us/coronavirus-child-care-centers.html?action=click\&pgtype=Article\&state=default\&region=MAIN_CONTENT_3\&context=storylines_faq}{makeshift
    child care} and
    \href{https://www.nytimes3xbfgragh.onion/2020/06/03/business/economy/coronavirus-working-women.html?action=click\&pgtype=Article\&state=default\&region=MAIN_CONTENT_3\&context=storylines_faq}{stunted
    workdays} to continue. California's two largest public school
    districts --- Los Angeles and San Diego --- said on July 13, that
    \href{https://www.nytimes3xbfgragh.onion/2020/07/13/us/lausd-san-diego-school-reopening.html?action=click\&pgtype=Article\&state=default\&region=MAIN_CONTENT_3\&context=storylines_faq}{instruction
    will be remote-only in the fall}, citing concerns that surging
    coronavirus infections in their areas pose too dire a risk for
    students and teachers. Together, the two districts enroll some
    825,000 students. They are the largest in the country so far to
    abandon plans for even a partial physical return to classrooms when
    they reopen in August. For other districts, the solution won't be an
    all-or-nothing approach.
    \href{https://bioethics.jhu.edu/research-and-outreach/projects/eschool-initiative/school-policy-tracker/}{Many
    systems}, including the nation's largest, New York City, are
    devising
    \href{https://www.nytimes3xbfgragh.onion/2020/06/26/us/coronavirus-schools-reopen-fall.html?action=click\&pgtype=Article\&state=default\&region=MAIN_CONTENT_3\&context=storylines_faq}{hybrid
    plans} that involve spending some days in classrooms and other days
    online. There's no national policy on this yet, so check with your
    municipal school system regularly to see what is happening in your
    community.
  \end{itemize}
\end{itemize}

It also gained confidence when 30 million people participated in
\href{https://www.nytimes3xbfgragh.onion/2020/04/15/world/asia/south-korea-election.html}{parliamentary
elections} on April 15 without triggering a​ new outbreak.

But things are far from normal. Nightclubs and bathhouses take the
temperatures of everyone who enters. Students wear masks in class and
are not allowed to play contact sports. At Suwon Hi-Tech High School in
Suwon, a city south of Seoul, every student's temperature is checked
four times a day.

South Korea's baseball league
\href{https://www.nytimes3xbfgragh.onion/2020/05/07/sports/coronavirus-korea-baseball-opening.html}{started
its delayed season}on Tuesday and
\href{https://www.nytimes3xbfgragh.onion/2020/05/08/sports/soccer-south-korea.html}{the
soccer league got underway} on Friday --- with spitting prohibited and
no fans in the stands. Some stadiums played recorded fan noise, while
coaches and players on the sidelines wore masks.

``Complacency is the biggest risk,'' said Jung Eun-kyeong, head of the
Korea Centers for Disease Control and Prevention.

​South Korea still finds occasional patients whose origin of infection
​cannot be established. Ms. Jung said, ``this means that the virus that
has infected these people is still out there in the community.''

Image

The government will also gradually reopen public facilities like
schools, museums, libraries, stadiums and concert halls in coming
weeks.Credit...Woohae Cho for The New York Times

A government task force of ​economists and sociologists, as well as
infectious-disease experts, drafted a 68-page ``guidebook for distancing
in daily life.'' It outlined measures like installing partitions at
cafeteria and dining-hall tables, keeping masks on in church and having
visitors to weddings, funerals, karaoke bars, nightclubs and
internet-game parlors write down their names and telephone numbers so
they can be traced later.

It calls for workers with even minor potential symptoms of Covid-19 to
call in sick for a few days --- a tall order in a culture where
reporting for work even when sick is considered a virtue.

The draft was posted online in mid-April ​for public feedback. One
change made with citizens' suggestion: keeping every other seat empty in
movie theaters.​

``There is no going back to the life we had before Covid-19,'' said Kim
Gang-lip, a senior policy coordinator at Central Disaster Management
Headquarters. ``Instead, we ​are creating a new set of social norms and
culture."

Advertisement

\protect\hyperlink{after-bottom}{Continue reading the main story}

\hypertarget{site-index}{%
\subsection{Site Index}\label{site-index}}

\hypertarget{site-information-navigation}{%
\subsection{Site Information
Navigation}\label{site-information-navigation}}

\begin{itemize}
\tightlist
\item
  \href{https://help.nytimes3xbfgragh.onion/hc/en-us/articles/115014792127-Copyright-notice}{©~2020~The
  New York Times Company}
\end{itemize}

\begin{itemize}
\tightlist
\item
  \href{https://www.nytco.com/}{NYTCo}
\item
  \href{https://help.nytimes3xbfgragh.onion/hc/en-us/articles/115015385887-Contact-Us}{Contact
  Us}
\item
  \href{https://www.nytco.com/careers/}{Work with us}
\item
  \href{https://nytmediakit.com/}{Advertise}
\item
  \href{http://www.tbrandstudio.com/}{T Brand Studio}
\item
  \href{https://www.nytimes3xbfgragh.onion/privacy/cookie-policy\#how-do-i-manage-trackers}{Your
  Ad Choices}
\item
  \href{https://www.nytimes3xbfgragh.onion/privacy}{Privacy}
\item
  \href{https://help.nytimes3xbfgragh.onion/hc/en-us/articles/115014893428-Terms-of-service}{Terms
  of Service}
\item
  \href{https://help.nytimes3xbfgragh.onion/hc/en-us/articles/115014893968-Terms-of-sale}{Terms
  of Sale}
\item
  \href{https://spiderbites.nytimes3xbfgragh.onion}{Site Map}
\item
  \href{https://help.nytimes3xbfgragh.onion/hc/en-us}{Help}
\item
  \href{https://www.nytimes3xbfgragh.onion/subscription?campaignId=37WXW}{Subscriptions}
\end{itemize}
