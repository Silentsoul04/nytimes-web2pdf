Sections

SEARCH

\protect\hyperlink{site-content}{Skip to
content}\protect\hyperlink{site-index}{Skip to site index}

\href{https://myaccount.nytimes3xbfgragh.onion/auth/login?response_type=cookie\&client_id=vi}{}

\href{https://www.nytimes3xbfgragh.onion/section/todayspaper}{Today's
Paper}

\href{/section/opinion}{Opinion}\textbar{}Amid Covid-19, a Call for
M.D.s to Mail the Abortion Pill

\url{https://nyti.ms/3cDWrtH}

\begin{itemize}
\item
\item
\item
\item
\item
\item
\end{itemize}

Advertisement

\protect\hyperlink{after-top}{Continue reading the main story}

\href{/section/opinion}{Opinion}

Supported by

\protect\hyperlink{after-sponsor}{Continue reading the main story}

Fixes

\hypertarget{amid-covid-19-a-call-for-mds-to-mail-the-abortion-pill}{%
\section{Amid Covid-19, a Call for M.D.s to Mail the Abortion
Pill}\label{amid-covid-19-a-call-for-mds-to-mail-the-abortion-pill}}

For decades, the consensus has been that F.D.A. regulations require that
the abortion pill be obtained in a clinic. But that's changing.

By Patrick Adams

Mr. Adams is a journalist in Atlanta.

\begin{itemize}
\item
  May 12, 2020
\item
  \begin{itemize}
  \item
  \item
  \item
  \item
  \item
  \item
  \end{itemize}
\end{itemize}

\includegraphics{https://static01.graylady3jvrrxbe.onion/images/2020/05/12/opinion/12FixesAdams/12FixesAdams-articleLarge.jpg?quality=75\&auto=webp\&disable=upscale}

Last fall, months before America's first outbreak of the coronavirus,
Francine Coeytaux and Elisa Wells, co-founders of the
\href{https://www.nytimes3xbfgragh.onion/2020/06/29/us/abortion-supreme-court.html}{abortion}
rights advocacy group Plan C, were reaching out to doctors with a
question they said was urgent:

``Would you be willing to mail the `abortion pills' to women in their
homes?''

For millions of women across America, abortion access was
\href{https://www.nytimes3xbfgragh.onion/interactive/2019/05/31/us/abortion-clinics-map.html}{already
severely limited} --- the result of
\href{https://www.nytimes3xbfgragh.onion/2019/05/29/us/louisiana-abortion-heartbeat-bill.html}{restrictive
new laws} that have forced dozens of clinics to close their doors. Now,
with the spread of Covid-19, some states have classified abortions as
``nonessential,'' putting access to the procedure
\href{https://www.nytimes3xbfgragh.onion/2020/03/23/us/coronavirus-texas-ohio-abortion.html}{even
further out of reach}.

But the pandemic has also shone a spotlight on what's known as
``medication abortion,'' or the use of pills to terminate an early
pregnancy. And Ms. Coeytaux and Ms. Wells say that has only broadened
support for their efforts to make the medicines available by mail.

The medicines are mifepristone and misoprostol. Mifepristone blocks the
effects of progesterone, a hormone without which the lining of the
uterus begins to break down, while misoprostol, to be taken 24 to 48
hours later, induces contractions of the uterus that expel its contents.
Both drugs are approved by the Food and Drug Administration for use up
to
\href{https://www.nytimes3xbfgragh.onion/2016/03/31/health/abortion-pill-mifeprex-ru-486-fda.html}{10
weeks into pregnancy}.

With its approval in 2000, mifepristone promised to substantially expand
access to abortion care in the United States. (Misoprostol, which had
been developed as anti-ulcer therapy, was already
\href{https://www.nytimes3xbfgragh.onion/2016/06/28/opinion/from-uruguay-a-model-for-making-abortion-safer.html}{in
use}.) Suddenly, what had previously required a surgical procedure could
be done safely and effectively
\href{https://www.guttmacher.org/journals/psrh/2004/01/convincing-new-providers-offer-medical-abortion-what-will-it-take}{anywhere
a woman chose}.

But approval came with stringent restrictions on mifepristone's
distribution.

Regulated under what's called a ``risk evaluation and mitigation
strategy,'' or R.E.M.S., mifepristone can be dispensed only in clinics,
medical offices and hospitals; only by, or under the supervision of, a
doctor certified to prescribe the drug; and only to patients who have
signed an F.D.A.-approved patient agreement.

Abortion rights advocates have long held that there is
\href{https://www.vice.com/en_us/article/vb5vzd/fda-abortion-pill-regulations-controversy}{no
medical justification} for applying that regulation to mifepristone, the
safety of which has been
\href{https://www.nejm.org/doi/full/10.1056/NEJMsb1612526}{well
established}, and that the decision to do so was politically motivated.
And for years,
\href{https://www.aafp.org/dam/AAFP/documents/advocacy/prevention/women/LT-FDA-MifepristoneREMS-062019.pdf}{leading}
\href{https://policysearch.ama-assn.org/policyfinder/detail/mifepristone?uri=\%2FAMADoc\%2FHOD.xml-H-100.948.xml}{medical
societies} and
\href{https://www.nejm.org/doi/full/10.1056/NEJMp1908305}{reproductive}
\href{https://emaaproject.org/}{health} experts have petitioned the
F.D.A. to loosen the restrictions or lift them altogether.

Now, as stay-at-home orders further impede access to clinic-based
abortions, prominent voices --- including those of
\href{https://ag.ny.gov/sites/default/files/final_ag_letter_hhs_medication_abortion_2020.pdf}{attorneys
general from 21 states}, a former
\href{https://thehill.com/opinion/healthcare/494914-the-uk-allows-home-use-of-the-abortion-pill-the-us-should-do-the-same}{F.D.A.
official}, and The Times
\href{https://www.nytimes3xbfgragh.onion/2020/03/26/opinion/abortion-law-coronavirus.html}{editorial
board} --- have joined calls for the F.D.A. to act.

But what if the consensus understanding of the regulation is wrong? What
if it doesn't actually prohibit doctors from mailing mifepristone?

It's a question at the heart of the debate around abortion access and
one that has not been considered by a court. And for that reason, Ms.
Wells and Ms. Coeytaux feel they are on firm legal ground in assembling
what they envision as a national network of new providers --- those who,
like them, embrace a broader interpretation of the rules.

``Most U.S. providers have taken the R.E.M.S. to mean that mifepristone
cannot be mailed,'' Ms. Wells said. ``We disagree. We think providers
have clear latitude to `dispense' the drug from their offices and then
ship it to patients, and we're hearing from more and more of our
colleagues who see it the same way.''

``I strongly believe that the R.E.M.S. does not limit the ability to
send mifepristone by mail,'' said Dr. Mitchell Creinin, a veteran
medical researcher and professor of obstetrics and gynecology at the
University of California, Davis, who is not affiliated with Plan C.

``It says you have to dispense the drug to the patient in the clinic,''
Dr. Creinin added. ``But the act of distribution --- by mail, by
overnight delivery, whatever --- is different than dispensing.''

In early March, Plan C called on doctors across the country to take the
first step for mailing mifepristone: registering with
\href{https://www.nytimes3xbfgragh.onion/2000/09/30/us/abortion-pill-distributor-energized-by-new-mission.html}{Danco
Laboratories} or \href{https://genbiopro.com/prescribing/}{GenBioPro},
the only F.D.A.-approved manufacturers.

Already, Ms. Wells said, dozens of doctors have responded. ``And we know
of several who have started shipping the pills or are planning to
soon,'' she added.

A family physician, who spoke on condition of anonymity out of fear of
harassment, said she had never provided abortion care when a friend
introduced her to Plan C in early April. After confirming that her
malpractice insurance covered abortion by medication and reviewing the
guidelines for managing patients, she enrolled in GenBioPro's
\href{https://genbiopro.com/ordering/}{provider program} and filled out
a prescriber agreement attesting to the R.E.M.S.

``It says all of the things you would want it to say, like that there
has to be a doctor-patient relationship, that you have to educate the
patient and obtain consent, and that you have to log and label the drug
correctly,'' she said. ``But how you deliver it to the patient --- by
hand? By mail? By throwing it over a fence? It says nothing about
that.''

The paperwork took her less than a day, she said. ``And I thought, oh my
God, why didn't I do this a long time ago? Honestly, I'm a little
ashamed that I didn't.''

When they started Plan C in 2016, Ms. Wells and Ms. Coeytaux --- who in
the late 1990s were instrumental in making
\href{https://www.whijournal.com/article/S1049-3867(00)00072-4/fulltext}{emergency
contraception available over the counter} --- set out to raise awareness
about self-managed abortion through a grass-roots approach. They held
meetings
\href{https://www.nytimes3xbfgragh.onion/2017/04/27/opinion/spreading-plan-c-to-end-pregnancy.html?_r=0}{in
their homes}, trained groups of millennial ``ambassadors,'' and put out
\href{http://cdn.cnn.com/cnn/2018/images/10/22/plancreportcardnew.pdf}{a
report card} ranking the various vendors offering pills online.

While Plan C was getting the word out, the nonprofit research group
\href{https://gynuity.org/}{Gynuity Health Projects} was gathering
evidence for advocacy efforts aimed at removing the regulation. In 2016,
the group launched a direct-to-patient telemedicine abortion service as
part of a
\href{https://www.nytimes3xbfgragh.onion/2016/11/11/health/abortion-study-mail.html}{study
in four states}, with pills shipped to women in their homes.

Gynuity secured special permission from the F.D.A. under the Obama
administration to mail mifepristone and its program, called TelAbortion,
has been allowed to
\href{https://www.nytimes3xbfgragh.onion/2020/04/28/health/telabortion-abortion-telemedicine.html}{continue
ever since}. Over the past year, TelAbortion added eight new states and
now operates in a total of 13. But in terms of the number of patients
served, the program's impact has been modest: It has shipped just 841
packages containing abortion pills over a four-year period.

One reason is that the F.D.A. approved the study on the condition that
all patients arrange an in-person visit for an ultrasound to gauge their
gestational age. A routine ultrasound is
\href{https://www.guttmacher.org/state-policy/explore/requirements-ultrasound}{not
considered medically} necessary for a first-trimester abortion;
\href{https://www.ncbi.nlm.nih.gov/pubmed/21091926}{research shows} that
\href{https://www.ncbi.nlm.nih.gov/pubmed/25152258}{most women} can
accurately recall their last menstrual period; and
\href{https://www.ncbi.nlm.nih.gov/pmc/articles/PMC7161512/\#b0090}{new
medical protocols} are now changing the standard of care. But barring
exceptions made for the pandemic, the TelAbortion study can only serve
patients who are able to come in for a test.By comparison, the website
\href{https://aidaccess.org/}{Aid Access}, which does not require an
ultrasound, was serving almost 800 patients a month before the pandemic
began.

Founded in 2018 by the Dutch physician and activist
\href{https://www.nytimes3xbfgragh.onion/2001/08/26/magazine/the-pro-choice-extremist.html}{Rebecca
Gomperts}, Aid Access offers women in America who are less than nine
weeks pregnant the ability, for the first time, to obtain abortion pills
by mail with a prescription from a licensed physician. After writing the
prescription, Dr. Gomperts provides patients with instructions for how
to request the pills from an exporter in India.

Within its first year, Aid Access was contacted by more than 21,000
women in America, as requests for consultations flooded in from across
the country. The F.D.A. took notice. Last year,
\href{https://www.fda.gov/inspections-compliance-enforcement-and-criminal-investigations/warning-letters/aidaccessorg-575658-03082019}{the
agency accused} the organization of violating federal law and ordered it
to stop distributing the drugs in the United States. Dr. Gomperts
responded by
\href{https://www.npr.org/2019/09/09/758871490/european-doctor-who-prescribes-abortion-pills-to-u-s-women-online-sues-fda}{suing
the F.D.A.}, claiming federal officials had seized medications
prescribed to patients through her website.

Amid the pandemic, demand for Dr. Gomperts' services has surged. Since
late March, she says, ``approximately 3,000 women living in the U.S.
have requested my help.'' For a time, Dr. Gomperts had to suspend
operations; after India closed its airspace to international traffic,
shipments from the pharmacy she works with were halted. But with Plan
C's network, she was able to find American providers registered with
GenBioPro and ready to mail.

``This is why I went to med school,'' said a doctor in upstate New York
who is mailing the abortion drugs on behalf of Aid Access and also spoke
on condition of anonymity. ``If there's one thing I hope we can learn
from life under Covid, it's to trust and empower women to take care of
themselves, and this is one way of doing that.''

Patrick Adams is a journalist in Atlanta and an occasional contributor
to Fixes.

\emph{To receive email alerts for Fixes columns, sign up}
\href{http://eepurl.com/ABIxL}{\emph{here.}}

\emph{The Times is committed to publishing}
\href{https://www.nytimes3xbfgragh.onion/2019/01/31/opinion/letters/letters-to-editor-new-york-times-women.html}{\emph{a
diversity of letters}} \emph{to the editor. We'd like to hear what you
think about this or any of our articles. Here are some}
\href{https://help.nytimes3xbfgragh.onion/hc/en-us/articles/115014925288-How-to-submit-a-letter-to-the-editor}{\emph{tips}}\emph{.
And here's our email:}
\href{mailto:letters@NYTimes.com}{\emph{letters@NYTimes.com}}\emph{.}

\emph{Follow The New York Times Opinion section on}
\href{https://www.facebookcorewwwi.onion/nytopinion}{\emph{Facebook}}\emph{,}
\href{http://twitter.com/NYTOpinion}{\emph{Twitter (@NYTopinion)}}
\emph{and}
\href{https://www.instagram.com/nytopinion/}{\emph{Instagram}}\emph{.}

Advertisement

\protect\hyperlink{after-bottom}{Continue reading the main story}

\hypertarget{site-index}{%
\subsection{Site Index}\label{site-index}}

\hypertarget{site-information-navigation}{%
\subsection{Site Information
Navigation}\label{site-information-navigation}}

\begin{itemize}
\tightlist
\item
  \href{https://help.nytimes3xbfgragh.onion/hc/en-us/articles/115014792127-Copyright-notice}{©~2020~The
  New York Times Company}
\end{itemize}

\begin{itemize}
\tightlist
\item
  \href{https://www.nytco.com/}{NYTCo}
\item
  \href{https://help.nytimes3xbfgragh.onion/hc/en-us/articles/115015385887-Contact-Us}{Contact
  Us}
\item
  \href{https://www.nytco.com/careers/}{Work with us}
\item
  \href{https://nytmediakit.com/}{Advertise}
\item
  \href{http://www.tbrandstudio.com/}{T Brand Studio}
\item
  \href{https://www.nytimes3xbfgragh.onion/privacy/cookie-policy\#how-do-i-manage-trackers}{Your
  Ad Choices}
\item
  \href{https://www.nytimes3xbfgragh.onion/privacy}{Privacy}
\item
  \href{https://help.nytimes3xbfgragh.onion/hc/en-us/articles/115014893428-Terms-of-service}{Terms
  of Service}
\item
  \href{https://help.nytimes3xbfgragh.onion/hc/en-us/articles/115014893968-Terms-of-sale}{Terms
  of Sale}
\item
  \href{https://spiderbites.nytimes3xbfgragh.onion}{Site Map}
\item
  \href{https://help.nytimes3xbfgragh.onion/hc/en-us}{Help}
\item
  \href{https://www.nytimes3xbfgragh.onion/subscription?campaignId=37WXW}{Subscriptions}
\end{itemize}
