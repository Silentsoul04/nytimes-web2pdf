Sections

SEARCH

\protect\hyperlink{site-content}{Skip to
content}\protect\hyperlink{site-index}{Skip to site index}

\href{https://www.nytimes3xbfgragh.onion/spotlight/podcasts}{Podcasts}

\href{https://myaccount.nytimes3xbfgragh.onion/auth/login?response_type=cookie\&client_id=vi}{}

\href{https://www.nytimes3xbfgragh.onion/section/todayspaper}{Today's
Paper}

\href{/spotlight/podcasts}{Podcasts}\textbar{}`There's a Quiet All Over
the World'

\url{https://nyti.ms/3bumeD4}

\begin{itemize}
\item
\item
\item
\item
\item
\item
\end{itemize}

Advertisement

\protect\hyperlink{after-top}{Continue reading the main story}

transcript

Back to Sugar Calling

bars

0:00/0:00

-0:00

transcript

\hypertarget{theres-a-quiet-all-over-the-world}{%
\subsection{`There's a Quiet All Over the
World'}\label{theres-a-quiet-all-over-the-world}}

\hypertarget{hosted-by-cheryl-strayed-produced-by-kelly-prime-and-edited-by-sara-sarasohn-editorial-oversight-by-wendy-dorr}{%
\subsubsection{Hosted by Cheryl Strayed, produced by Kelly Prime and
edited by Sara Sarasohn. Editorial oversight by Wendy
Dorr.}\label{hosted-by-cheryl-strayed-produced-by-kelly-prime-and-edited-by-sara-sarasohn-editorial-oversight-by-wendy-dorr}}

\hypertarget{cheryl-strayed-talks-with-the-poet-billy-collins-about-memorization-picture-language-and-the-power-of-collective-silence}{%
\paragraph{Cheryl Strayed talks with the poet Billy Collins about
memorization, ``picture language'' and the power of collective
silence.}\label{cheryl-strayed-talks-with-the-poet-billy-collins-about-memorization-picture-language-and-the-power-of-collective-silence}}

Wednesday, May 13th, 2020

\begin{itemize}
\item
  cheryl strayed\\
  I'm going to call the poet Billy Collins. I love his poetry so much,
  and I know I'm not alone in that. He's that rare thing in America, a
  bestselling poet. He was the U.S. Poet Laureate from 2001 to 2003 as
  well. He said that poetry is the home of ambiguity, ambivalence, and
  uncertainty, and I think that wraps up quite succinctly what a lot of
  us are living in right now. I thought it would be a great time to talk
  to Billy Collins.
\item
  {[}music{]}
\item
  {[}phone rings{]}
\item
  billy collins\\
  Oh, Cheryl.
\item
  cheryl strayed\\
  Billy, it's so nice to talk to you!
\item
  billy collins\\
  I sort of knew it was you.
\item
  cheryl strayed\\
  You knew it was me?
\item
  billy collins\\
  Yeah. I can't say I wasn't expecting your call.
\item
  cheryl strayed\\
  You're just hanging out waiting for Cheryl to call.
\item
  billy collins\\
  Every afternoon.
\item
  cheryl strayed\\
  Where am I calling you, Billy Collins?
\item
  billy collins\\
  I'm in my study in our house in a little town in Florida called Winter
  Park. The streetside wall is all windows, so I got a good view of
  tropical or subtropical vegetation. I mean, there's like pygmy palms
  and, I think, two queen palms. And there's this red bougainvillea.
\item
  cheryl strayed\\
  Nice.
\item
  billy collins\\
  And very few cars are going up and down that street these days. More
  common to see families bicycling, or dad taking his son down to the
  lake to fish, and stuff like that. It's like going back to the `50s.
\item
  cheryl strayed\\
  Are you just alone in the house with Suzannah, just the two of you?
\item
  billy collins\\
  Well, there are two cats, Francis, a boy, and Audrey, a girl.
\item
  cheryl strayed\\
  OK, so two couples, Francis and Audrey, and Suzannah and Billy.
\item
  billy collins\\
  Yes. Well, Suzannah and I get along a lot better than Francis and
  Audrey. Thank god.
\item
  cheryl strayed\\
  Perhaps because you guys are still newlyweds, right? You just got
  married last summer.
\item
  billy collins\\
  Well, it's true. We're still --- not to get too sentimental, but we
  still celebrate our monthly anniversary on the 21st of the month.
\item
  cheryl strayed\\
  Oh, that's so sweet.
\item
  billy collins\\
  I think that we're losing track of that.
\item
  cheryl strayed\\
  Oh. I'd love that. So what do you do on your monthly anniversary? Like
  what have you done historically?
\item
  billy collins\\
  Well, we used to go out to dinner, but now Suzannah just cooks another
  dinner. We've been very, very careful, very assiduous about going out
  only when necessary, and wash your hands all the time. And I usually
  say, if you say two Hail Marys, that's about 20 seconds of
  handwashing. And there could be an added benefit. You never know.
\item
  cheryl strayed\\
  Right.
\item
  billy collins\\
  So I'm 79 years old. And it sounds insane to say that, but it's the
  truth, and I'm pretty sure if I caught it, it would kill me. And
  Suzannah shares that view. So she's pretty willful about keeping me in
  the house.
\item
  cheryl strayed\\
  Uh-huh. Do you even go out for walks?
\item
  billy collins\\
  I go out for a walk almost every day. Not quite as far as you, but
  it's 2 miles. And it's very clarifying, though. I've always liked
  walking, and there's a Latin expression, solvitur ambulando, which
  means it is solved by walking. You probably know that.
\item
  cheryl strayed\\
  I do, I do. As you know, my memoir ``Wild'' is all about walking. I
  walked 1,100 miles on the Pacific Crest Trail. A lot is solved by
  walking.
\item
  billy collins\\
  I would say after a thousand miles, you've got yourself together.
\item
  cheryl strayed\\
  I'm sad to report that the walk goes on. The journey continues even
  after that. But I do think that that's one of the things that you see
  over and over, is that power of a sort of walking meditation, where
  the classic journey one takes by foot, that is always enlightening.
  And I think we see that even on a micro scale on those 2-mile walks.
\item
  billy collins\\
  Well, I'm not as adventurous as you are, but it's the same walk. But
  it's never the same walk. It's like putting yourself and putting your
  foot in the same river twice. And, often, just when I get back to the
  house, I might start a poem just about what I saw on the walk, which
  is --- sometimes that goes nowhere, but other times it develops into
  something. So I find it very clarifying and --- I mean, the English
  romantic poets, particularly Wordsworth, the way we would watch
  television, they would go for walks. And Wordsworth said that he
  composed a number of his poems on this walk, which would kind of give
  him the rhythm of the poem maybe. I don't know. If you walked
  iambically, like thu-dud thu-dud, you'd probably be skipping. But I
  guess just the pacing provides a kind of steady rhythm. Did you listen
  --- let me ask you a question. Did you listen to music on your walk?
\item
  cheryl strayed\\
  No. But the reason I didn't listen to music on my walk is it was 1995,
  and so I would have had to carry a Walkman, and cassette tapes, and so
  forth.
\item
  billy collins\\
  Oh, right.
\item
  cheryl strayed\\
  But, Billy, what was so utterly fascinating to me about my experience
  with music on the Pacific Crest Trail is I could not keep music out of
  my mind. And I call it in ``Wild'' like --- that my mind was like the
  transistor radio station in my head, because I kept having songs come
  to me. And, many times, it would be songs I couldn't --- I didn't want
  to have in my head, like jingles for, like, gum, or things like this
  from my youth. Sometimes Bob Dylan songs that I couldn't remember all
  the lyrics. And so I was constantly singing in my head, and the thing
  I missed, I would say, almost as much as food. I really ached for
  music in a way that surprised me. I didn't realize how deep that was.
\item
  billy collins\\
  Well, that's your little internal jukebox that got activated.
\item
  cheryl strayed\\
  Yeah.
\item
  billy collins\\
  Yeah. I wrote a poem about getting a bad song stuck in your head,
  because they seem to be stickier than most. And I titled the poem
  ``More Than a Woman,'' but I say to other people that you can call it
  any song. You can call it ``Dancing Queen'' or any song.
\item
  cheryl strayed\\
  That's a great song.
\item
  billy collins\\
  And sometimes just mentioning the title is enough to activate the song
  in other people's heads, so you have to be careful.
\item
  cheryl strayed\\
  Well, have you --- there's this documentary that I'm forgetting the
  title of it now, but it's about people who have really deep, profound
  dementia, or are at the very end of their lives, and they can't really
  communicate even any more. But they asked their kids, like, well, what
  were their favorite songs when they were young? And they put
  headphones on these people who are so inside themselves, and suddenly
  they would come awake, and they would even sometimes be able to sing
  along with the songs. I think it's an ancient kind of knowing.
\item
  billy collins\\
  Yeah.
\item
  cheryl strayed\\
  And I know you do that with --- I was reading about your mother, who
  would recite poetry to you. I wonder if you have that relationship
  with poetry because you were hearing it so young.
\item
  billy collins\\
  Well, I think I got --- I mean, who knows in these things. If you look
  back on your life and try to make sense of it, you're looking through
  a kind of distorted rear view mirror. But I do have early memories of
  my being a toddler, or advanced toddler stage, and my mother would say
  lines of poetry that she memorized as a school girl in rural Ontario,
  Canada. And if it was very cold out, she'd have a line from
  Shakespeare about the coldness of the weather and not as sharp as
  man's ingratitude to man, or something like that. And when I was a
  kid, I remember hearing my mother talking in two different ways. One
  was when she was regular talking, and one when she was reciting
  poetry. And I didn't know it was poetry. I'd never heard the word. As
  I said, I was sort of a pre-literate toddler. But I knew that the
  poetry sounded nicer.
\item
  cheryl strayed\\
  Mmm, yeah.
\item
  billy collins\\
  And if we believe in really early suggestions on our consciousness,
  that might account from the fact that's where I am today.
\item
  cheryl strayed\\
  Right. That's --- so where did your mom get that? Was she asked as a
  student to memorize these poems?
\item
  billy collins\\
  Well, she and my father were both born in 1901. So when she was a
  schoolgirl, this would be 1907 or `14, or something in there,
  memorization was not only an acceptable way of teaching poetry, it was
  --- I believe it was the only way.
\item
  cheryl strayed\\
  Right.
\item
  billy collins\\
  I mean, they weren't --- this was 100 miles or so north of Toronto.
  They weren't analyzing poetry up there in the schoolhouse. They were
  having kids memorize it. And I try to get my students to memorize a
  little bit of poetry, and it's interesting because their reaction is
  they groan. But once they get a sonnet in their head, or even a little
  poem by Emily Dickinson, they line up outside my office ready to
  recite it for me.
\item
  cheryl strayed\\
  That's so interesting. No. I think I'm absolutely fascinated by this,
  because when I started school in the mid `70s, this idea was that
  memorization was this rote learning that didn't at all feed the
  spirit, and it was sort of oppressive.
\item
  billy collins\\
  Very old fashioned, yeah.
\item
  cheryl strayed\\
  Yeah, yeah. And yet, I wish that people had made me memorize poems.
  Because it is kind of like those jingles. They become part of you, and
  you can call upon them when you need them then, right? They're with
  you.
\item
  billy collins\\
  You could have carried these poems up the Pacific Crest Trail.
\item
  cheryl strayed\\
  Yeah.
\item
  billy collins\\
  You know, it's interesting that one of the most rewarding instances I
  ever had as a teacher, it was on a New York subway. There was a guy, a
  grown guy, across on the other side of the aisle there, or whatever.
  And he was looking at me. And, of course, in this New York subway,
  avoid eye contact is the main rule. Anyway, he got up and approached
  me, and he sat down. And over the roar of the subway, he said,
  Professor Collins, I was your student. And he was then a doctor, an
  oncologist, working at Mount Sinai Hospital. And he said you made us
  memorize a poem, and I want to say it for you.
\item
  cheryl strayed\\
  Oh, wow.
\item
  billy collins\\
  Again, over the roar of the subway, he recited this Emily Dickinson
  poem. And it was just so amazing that I'd done something.
\item
  cheryl strayed\\
  Right. It's so measurable, right? That guy has Emily Dickinson in him
  forever because of you.
\item
  billy collins\\
  It's stored there. And I also think, if they don't remember anything
  I've said, they'll remember 10 lines of Shakespeare or whatever.
\item
  cheryl strayed\\
  Yeah. Well, and what's interesting to ponder about that is that we
  are, in some sort of brute way, just forcing ourselves to remember
  this arrangement of words. And I'm curious, will this doctor, 50 years
  from now on his deathbed, will he be able to say that poem because it
  is part of him. And it reminds me, when I was hiking on the Pacific
  Crest Trail, I realized that I remembered the names of flowers that
  --- my mom was a wonderful gardener and really knew everything about
  plants, and she would always teach me the names of wildflowers. And
  there was a point on the trail where I stepped into this meadow and I
  saw all these flowers, and I realized I knew their names. And I said
  them to myself in my mind as I walked past them, and I felt the
  presence of my mother so strongly. And it was a kind of poem embedded
  in me. It was a memorization that I didn't know I had received until
  then.
\item
  billy collins\\
  Well, and I also think that you really internalize this series of
  lines and bring it into you for yourself. And it's a part of you then.
  And once the student realizes the power of that --- they're now
  walking around with these lines in them, and they can summon them up
  at any moment --- they find that pretty thrilling, actually.
\item
  cheryl strayed\\
  It's a powerful thing. And I think even in cases where you don't
  necessarily memorize the lines, that poems can stay with you in
  another way. That you remember them, or you remember the way that they
  made you feel. And one of your poems that did that for me, Bill, is
  ``The Names.'' You were asked by the librarian of Congress to write it
  for the first anniversary of 9/11. And the reason it stuck with me is
  it made real to me this tragedy that was this grand scale --- many,
  many deaths that seemed kind of abstract until I heard you read that
  poem. And it wasn't then just numbers, it was names. And I thought of
  it now because we are hearing numbers every day. The number of people
  who die of Covid-19 climbs day by day. I would love for you to read
  ``The Names'' to me.
\item
  billy collins\\
  Well, yeah, I'd love to. It's called ``The Names,'' and then, in
  parenthesis, it just says, ``For the victims of Sept. 11 and their
  survivors.''

  ``Yesterday, I lay awake in the palm of the night. A fine rain stole
  in, unhelped by any breeze. And when I saw the silver glaze on the
  windows, I started with A, with Ackerman, as it happened, then Baxter
  and Calabro, Davis and Eberling, names falling into place as droplets
  fell through the dark. Names printed on the ceiling of the night.
  Names slipping around a watery bend. Twenty-six willows on the banks
  of a stream.

  In the morning, I walked out barefoot among thousands of flowers,
  heavy with dew, like the eyes of tears, and each had a name. Fiore
  inscribed on a yellow petal, then Gonzalez and Hon, Ishikawa and
  Jenkins. Names written in the air and stitched into the cloth of the
  day. A name under a photograph taped to a mailbox. Monogram on a torn
  shirt. I see you spelled out on storefront windows and on the bright
  unfurled awnings of this city. I say the names as I turn a corner ---
  Kelly and Lee, Medina, Nardella, and O'Connor.

  When I peer into the woods, I see a thick tangle where letters are
  hidden, as in a puzzle concocted for children. Parker and Quigley in
  the twigs of an ash. Rizzo, Schubert, Torres, and Upton, secrets in
  the boughs of an ancient maple. Names written in the pale sky. Names
  rising in the updraft amid buildings. Silent names in stone, or cried
  out behind a door. Names blown over the earth and out to sea.

  In the evening --- weakening light, the last swallows, a boy on a lake
  lifts his oars, a woman by a window puts a match to a candle, and then
  the names are outlined on the rose clouds --- Vanacore and Wallace.
  Let X, if it can, stand for the ones unfound. Then Young and Ziminsky,
  the final jolt of Z. Names etched on the head of a pin. One name
  spanning a bridge, another undergoing a tunnel. A blue name needled
  into the skin.

  Names of citizens, workers, mothers and fathers, the bright-eyed
  daughter, the quick son. Alphabet of names in green rows in a field.
  Names in the small tracks of birds. Names lifted from a hat or
  balanced on the tip of the tongue. Names wheeled into the dim
  warehouse of memory. So many names, there is barely room on the walls
  of the heart.''
\item
  cheryl strayed\\
  What a powerful poem. What a powerful piece of work.
\item
  billy collins\\
  Yeah. And it was pretty nerve-racking really in front of a joint
  session of Congress, I can tell you.
\item
  cheryl strayed\\
  So you're called to this joint session, and are other people doing
  other things?
\item
  billy collins\\
  Well, yes, lots of other people. It was one of the first times in
  history that both houses of Congress met outside of Washington. They
  met in New York, where it took place. And there were many speeches.
  And I must say something, not in praise of my poem but to make a point
  about poetry, that when I started reading the poem, and it starts
  lying there in the middle of the night with the rain against the
  windows, et cetera, many people were looking up at me the way a Cocker
  Spaniel would cock his head, like, what?
\item
  cheryl strayed\\
  Uh-huh.
\item
  billy collins\\
  It was not political language. It was not using generalities. It was
  not using the recycled language, including a history, heroism,
  tragedy, the common words that kept popping up in these talks. It was
  imagery. It was picture language, as Emerson calls it. It was the
  language of the world, of rain, and windows, and reality. And it
  demonstrated the difference between poetic language and, for lack of a
  better word, other language.
\item
  cheryl strayed\\
  And I think that that's --- I mean, you've just expressed why, in my
  mind, poetry is so necessary, poetry and really all literature. But
  poetry, in particular, to me it speaks to the heart. It speaks beyond
  the intellect, beyond information.
\item
  billy collins\\
  Oh, I think poetry takes us inward, right? Poetry privileges our
  subjectivity. A poem tends to start out in the world and describing
  something, but it usually ends up going down and in inside us. But it
  works in images. I mean, a poem about the virus might be an image of a
  face mask on a curb, a discarded face mask. Just that one thing might
  be enough to tell the whole story, in a sense, just that one image,
  instead of a larger but probably less effective rhetoric.
\item
  cheryl strayed\\
  Right. That's what art does. And I think too, just the ordinary
  stories, whenever --- my favorite parts of all of Ken Burns's
  documentaries are those, the journals that people kept, that weren't
  saying, here is the objective news of the day. It's rather here's what
  happened on my farm on this day. Here's what I saw. Here's what I
  witnessed. Here's what I felt, and feared, and loved, and treasured.
  And those are the kinds of things that poetry shows us and offers us.
\item
  billy collins\\
  And those letters, I mean, I can't believe how articulate everybody
  was in the middle of the 19th century. The letters are --- not only
  the penmanship is beautiful, but the letters are so articulate, at
  least the ones that we're exposed to.
\item
  cheryl strayed\\
  Well, those are the ones that made the cut.
\item
  billy collins\\
  Yeah, that's right. That's the cream of the crop.
\item
  cheryl strayed\\
  Do you keep a journal?
\item
  billy collins\\
  I don't keep a --- I have in the past. I don't --- I'm not very good
  at it. But I keep notebooks, and I always keep a little notebook in my
  pocket, and a pen. And thinking about our talk today, I have a page
  here of things to talk about or points we might --- so anything can go
  into these journals. They're kind of catch-alls, but --- and poems
  too.
\item
  cheryl strayed\\
  What have you written down that we haven't talked about yet? Let's
  talk about it.
\item
  {[}laughter{]}
\item
  billy collins\\
  That's good. Well, I wrote down the name of a poem that's very short
  that I'd love to read before we're done by --- not by me.
\item
  cheryl strayed\\
  Please read it to me. Please read it to me, yes.
\item
  billy collins\\
  May I? Well, it's --- I guess it's a sonnet. It's 14 lines. It's by a
  poet I don't know well, an Irish poet, John O'Donnell. And it's about
  the virus, and it's called ``When.'' And the interesting thing about
  it is that he's slightly ambivalent about the whole thing, about it
  ending. And Eavan Boland, another Irish poet who just died on Monday
  ---
\item
  cheryl strayed\\
  Right.
\item
  billy collins\\
  --- she said, ``Poetry begins where certitude leaves off.'' Like
  poetry deals with ambivalence and ambiguity. And if you're certain
  about a thing, you don't need to write a poem about it. But you'll see
  the word, about halfway through, the word ``although,'' and that's
  when he --- the tone of the poem changes. So I'll read it if you like.
  It's titled ``When,'' by John O'Donnell.

  ``And when this ends we will emerge, shyly and then all at once,
  dazed, long-haired as we embrace loved ones the shadow spared, and
  weep for those it gathered in its shroud. A kind of rapture, this
  longed-for laying on of hands, high cries as we nuzzle, leaning in to
  kiss, and whisper that now things will be different.

  Although a time will come when we'll forget the curve's approaching
  wave, the hiss and sigh of ventilators, the crowded, makeshift
  morgues. A time when we may even miss the old-world arm's-length
  courtesy, small kindnesses left on doorsteps, the drifting, idle days,
  and nights when we flung open all the windows to arias in the
  darkness, our voices reaching out, holding each other till this
  passes.''

  So interesting that he sees the distancing as being a courtesy, and
  the quietness and the kindness of things left on doorsteps, and the
  idleness and the drifting. This really talks about the horror of
  people lost, but he says that --- it's like people talk about. There's
  a quiet all over the world. The whole globe has quieted down. There
  are no sports. There's no concerts. There's no people cheering. A
  friend of mine said you can hear a birdsong in New York City.
\item
  cheryl strayed\\
  Right.
\item
  billy collins\\
  And another --- in The Times last week, this guy writing from Paris
  said you could hear ducks quacking on the Seine. I have a friend, if
  one of us says something, the other one often says, and other poems.
  And I'm sure that would get it --- ducks quacking on the Seine, and
  other poems.
\item
  {[}laughter{]}
\item
  cheryl strayed\\
  That's your next book, Billy.
\item
  {[}laughter{]}
\item
  billy collins\\
  It could be. But that's, I mean, amazing, like being back in the `50s.
  It's quieter. It's slower. And, also, nature is healing herself. Fish
  are coming back into these grounds where they were, where there's too
  much pollution. Maybe we should take one month off a year and just let
  the Earth lick its wounds for a month before we continue to ravish it.
\item
  cheryl strayed\\
  Yeah, I thought of that. In that poem, when he says although, he's
  saying, but we're going to also forget all this, right?
\item
  billy collins\\
  Yeah, yeah. We'll forget the curves and all that. But we might even
  miss --- we'll forget, and we might even be nostalgic about the time
  when we stayed indoors with our spouses or whatever, or maybe a
  friend, sheltering in place. I was on the phone with my friend in New
  York City last night about --- well, I know it's about 2 minutes to 7.
  He said, well, at 7 o'clock, I have to get my pot and my spoon and
  bang it out the window. Everybody does that at 7 o'clock in New York.
\item
  cheryl strayed\\
  Mh-hm. In Portland too.
\item
  billy collins\\
  Yeah, for the responders. And maybe people will miss that. I guess the
  big question is, will this --- I mean, people banter around the word
  globalization, and global this, and global that. And that's primarily
  an economic term, I think. It meant the globalization of business. But
  this is something that's happening all over the world, and maybe this
  will give people a sense of being Earthlings, in the way that the blue
  marble did. Remember that first picture of Earth? That was the first
  time man had ever seen the planet.
\item
  cheryl strayed\\
  Mm-hm, mm-hm.
\item
  billy collins\\
  And there was some sense of optimism --- it became the symbol of Earth
  Day --- that a global event like that, or the globe made visible,
  might tie people together. I'm not too optimistic about that, but it's
  going to be very interesting to see what happens. I mean, maybe
  nothing will happen, and we'll just go back to being the human beings
  we are. But do you want to hear another poem by another poet?
\item
  cheryl strayed\\
  Sure, sure.
\item
  billy collins\\
  This is a poem by Wislawa Szymborska, the Polish poet. And it's a poem
  about the war, and what happens after the war, and then what happens
  when people forget about the war.
\item
  cheryl strayed\\
  And it's about World War II?
\item
  billy collins\\
  It's about World War II, yeah. But I think it applies today. Anyway,
  it's not long. It's called ``The End and the Beginning,'' by Wislawa
  Szymborska.

  ``After every war, someone has to pick up. Things won't pick
  themselves up, after all. Someone has to shove the rubble to the
  roadsides so the carts loaded with corpses can get by. Someone has to
  trudge through the sludge and ashes, through the sofa springs and
  shards of glass, the bloody rags. Someone has to lug the post to prop
  the wall. Someone has to glaze the windows, set the door in its frame.
  No sound bites, no photo opportunities, and it takes years. All the
  cameras have gone to other wars.

  The bridges need to be rebuilt. The railroad stations too.
  Shirtsleeves will be rolled up to shreds. Someone, broom in hand,
  still remembers how it was. Someone else listens, nodding his
  unshattered head. But others are bound to be bustling nearby who will
  find all that quite boring. From time to time, someone still must dig
  up a rusted argument from underneath a bush and haul it off to the
  dump.

  Those who knew what this was all about must make way for those who
  know little, and less than that, and at last nothing, less than
  nothing. Someone has to lie there in the grass that covers up the
  causes and effects with a corn stalk in his teeth, gawking at
  clouds.''
\item
  cheryl strayed\\
  Wow. I love that poem. What a great poem. That poem so aptly describes
  not just things like a world war and its aftermath, and the way that
  time changes it. It's even --- like, I was just thinking of the most
  intimate thing, like a broken heart. Like it's so painful at first,
  and then it gets slowly cleaned up, and then it slowly disappears. And
  then it's like that love was never even in existence.
\item
  billy collins\\
  It becomes history.
\item
  cheryl strayed\\
  It's history.
\item
  billy collins\\
  And there's a test on it.
\item
  cheryl strayed\\
  And there's no great sorrow. I mean, I think that, obviously, about
  big things, like a war or a virus, and so many of us are revisiting
  things like --- I've never heard so much conversation about the
  Spanish influenza epidemic as I have in the last month or two. And
  it's because we're saying, Oh, this thing that we all had the luxury
  of not remembering, or not really caring much about, suddenly might be
  the thing that teaches us about now.
\item
  billy collins\\
  In her poem, after all the destruction, to end with that figure lying
  on the grass with a corn stalk in his teeth, just looking up at
  clouds, that little personal individual moment of being alone. And
  Robert Frost said anybody can start a poem, but it takes a real poet
  to end one. And that's such a beautiful ending, gawking at clouds.
\item
  cheryl strayed\\
  Yeah. Is there anything that I didn't ask you that you want to talk
  about?
\item
  billy collins\\
  I don't know. I mean, I did spend one day in this monastery in Big
  Sur, the Camaldolese order of the Benedictines. They were, I think,
  like a 12th century breakaway from the Benedictines. And someone said
  they make the Benedictines look like Hells Angels, because they're a
  deeply reclusive order. None of them speak. There are about 30 of them
  there. And when they have meals, there are readings from the Bible,
  from whatever. But I had heard that my poems were being read. So when
  I was in California, I made a visit to the monastery and spent one day
  in one of these rooms. It's kind of a cell or a dorm room. And I had
  this experience of silence I'd never quite had before. I wrote a poem
  about what a noisy person I'd been all my life. And, gradually, after
  like an hour or so, I was aware that there are 30 other men right
  around me in these other buildings or rooms who have devoted their
  life to silence. They're professionals, professional quiet people. And
  I felt like this little droplet of my silence just dropping into this
  river of silence they were creating. And communal silence, I realized,
  is such a powerful thing. And when I think about the quieter world
  that the virus has put us into, it just reminded me of that deep
  experience of silence.
\item
  cheryl strayed\\
  What about it felt profound to you? I mean, can you identify what that
  particular silence does for you or for us?
\item
  billy collins\\
  Well, it's very cleansing, I think. I was very comfortable with it,
  knowing, as I said, the presence of these professional men around me
  keeping their silence so as to be less ensnared by the ways of the
  world. And when I was at the monastery in Big Sur, they referred very
  reverentially to two of the monks who had achieved, achieved what they
  call permanent reclusion. There are two monks who never see anybody.
  They're not only silent. They never talk during the year. There's no
  break, and they never go to communal meals. They're like prisoners of
  their silence. And that just blew my mind. They got in the cell and
  locked the door, and they loved it. They wouldn't have it any other
  way.
\item
  cheryl strayed\\
  It's astounding to me, really.

  My first reaction when I was hearing you talk about this place --- and
  I know this place, and I have to say, what you've contributed most
  powerfully to the world is the opposite of silence. It is that we find
  some truth in the words you share with us. And yet there is something
  about poetry and art that is about the deepest silence, right? That
  it's not about isolation. It's not about distance. It's about
  connection, I think, when silence is most profound and powerful.
\item
  billy collins\\
  Mm-hm. I think another way to connect them is to say that the --- a
  poem takes up a certain amount of space on a page. It doesn't fill the
  page. Most people would recognize a poem as being a poem by seeing
  that it just occupies this discrete place on the poem --- on the page,
  rather. And I get my students to think of the white space around the
  poem as silence, and so that the poem is an interruption of silence.
  And so it takes silence into a place. I knew a nun --- well, she was a
  crazy high school friend of mine, but she became a nun, and she was
  --- ended up in a little convent in Rome on the Appian Way there, and
  they were a reclusive order of nuns. And we had corresponded, but when
  she corresponded to me in letters, she would only talk about the love
  of God and all that. She never said what she was doing or anything, or
  if she was planting corn or something. And when I was in Italy, I
  called her --- I called the convent, and I said, Susan --- that was
  her old name --- I said, I'm coming into Rome, and can we take you out
  to lunch? Do you go out? And she said, Cheryl, she said, Fortunately,
  I can't.
\item
  {[}laughter{]}
\item
  cheryl strayed\\
  Really?
\item
  billy collins\\
  Isn't that great?
\item
  cheryl strayed\\
  So she'd ---
\item
  billy collins\\
  No irony --- yes. No irony at all. Fortunately, Billy, I can't.
\item
  cheryl strayed\\
  Wow. I mean, that's a really kind of wonderful way to think about now,
  isn't it? Because ---
\item
  billy collins\\
  Yeah.
\item
  cheryl strayed\\
  --- we who can't go out, to turn that into a blessing rather than a
  curse is something, I think.
\item
  billy collins\\
  Yes, yeah. I think it's the difference between people who will rush
  back into their jobs and can't wait to get back to normal who are
  enjoying, on some level, this quiet time, this traveling back in time,
  in a way, to where you can hear --- you can hear ducks quacking on the
  Seine, for god's sake.
\item
  {[}laughter{]}
\item
  cheryl strayed\\
  Well, it has been such a delight to speak to you.
\item
  billy collins\\
  Well, great. I've been looking forward to this a lot. So great to talk
  to you too.
\item
  cheryl strayed\\
  Bye, Billy.
\item
  billy collins\\
  Bye bye. Take care.
\item
  cheryl strayed\\
  I'm Cheryl Strayed. This is ``Sugar Calling.'' Next week, Joy Harjo.
\end{itemize}

\href{https://www.nytimes3xbfgragh.onion/column/sugar-calling}{\includegraphics{https://static01.graylady3jvrrxbe.onion/images/2020/04/29/podcasts/sugar-calling-album-art/sugar-calling-album-art-square320.jpg}Sugar
Calling}Subscribe:

\begin{itemize}
\tightlist
\item
  \href{https://itunes.apple.com/us/podcast/id1505881384}{Apple
  Podcasts}
\item
  \href{https://podcasts.google.com/?feed=aHR0cHM6Ly9yc3MuYXJ0MTkuY29tL3N1Z2FyLWNhbGxpbmc\&ved=0CAUQrrcFahcKEwjA8Kyn09voAhUAAAAAHQAAAAAQBQ}{Google
  Podcasts}
\end{itemize}

\hypertarget{theres-a-quiet-all-over-the-world-1}{%
\section{`There's a Quiet All Over the
World'}\label{theres-a-quiet-all-over-the-world-1}}

\hypertarget{cheryl-strayed-talks-with-the-poet-billy-collins-about-memorization-picture-language-and-the-power-of-collective-silence-1}{%
\subsection{Cheryl Strayed talks with the poet Billy Collins about
memorization, ``picture language'' and the power of collective
silence.}\label{cheryl-strayed-talks-with-the-poet-billy-collins-about-memorization-picture-language-and-the-power-of-collective-silence-1}}

Hosted by Cheryl Strayed, produced by Kelly Prime and edited by Sara
Sarasohn. Editorial oversight by Wendy Dorr.

Transcript

transcript

Back to Sugar Calling

bars

0:00/0:00

-0:00

transcript

\hypertarget{theres-a-quiet-all-over-the-world-2}{%
\subsection{`There's a Quiet All Over the
World'}\label{theres-a-quiet-all-over-the-world-2}}

\hypertarget{hosted-by-cheryl-strayed-produced-by-kelly-prime-and-edited-by-sara-sarasohn-editorial-oversight-by-wendy-dorr-1}{%
\subsubsection{Hosted by Cheryl Strayed, produced by Kelly Prime and
edited by Sara Sarasohn. Editorial oversight by Wendy
Dorr.}\label{hosted-by-cheryl-strayed-produced-by-kelly-prime-and-edited-by-sara-sarasohn-editorial-oversight-by-wendy-dorr-1}}

\hypertarget{cheryl-strayed-talks-with-the-poet-billy-collins-about-memorization-picture-language-and-the-power-of-collective-silence-2}{%
\paragraph{Cheryl Strayed talks with the poet Billy Collins about
memorization, ``picture language'' and the power of collective
silence.}\label{cheryl-strayed-talks-with-the-poet-billy-collins-about-memorization-picture-language-and-the-power-of-collective-silence-2}}

Wednesday, May 13th, 2020

\begin{itemize}
\item
  cheryl strayed\\
  I'm going to call the poet Billy Collins. I love his poetry so much,
  and I know I'm not alone in that. He's that rare thing in America, a
  bestselling poet. He was the U.S. Poet Laureate from 2001 to 2003 as
  well. He said that poetry is the home of ambiguity, ambivalence, and
  uncertainty, and I think that wraps up quite succinctly what a lot of
  us are living in right now. I thought it would be a great time to talk
  to Billy Collins.
\item
  {[}music{]}
\item
  {[}phone rings{]}
\item
  billy collins\\
  Oh, Cheryl.
\item
  cheryl strayed\\
  Billy, it's so nice to talk to you!
\item
  billy collins\\
  I sort of knew it was you.
\item
  cheryl strayed\\
  You knew it was me?
\item
  billy collins\\
  Yeah. I can't say I wasn't expecting your call.
\item
  cheryl strayed\\
  You're just hanging out waiting for Cheryl to call.
\item
  billy collins\\
  Every afternoon.
\item
  cheryl strayed\\
  Where am I calling you, Billy Collins?
\item
  billy collins\\
  I'm in my study in our house in a little town in Florida called Winter
  Park. The streetside wall is all windows, so I got a good view of
  tropical or subtropical vegetation. I mean, there's like pygmy palms
  and, I think, two queen palms. And there's this red bougainvillea.
\item
  cheryl strayed\\
  Nice.
\item
  billy collins\\
  And very few cars are going up and down that street these days. More
  common to see families bicycling, or dad taking his son down to the
  lake to fish, and stuff like that. It's like going back to the `50s.
\item
  cheryl strayed\\
  Are you just alone in the house with Suzannah, just the two of you?
\item
  billy collins\\
  Well, there are two cats, Francis, a boy, and Audrey, a girl.
\item
  cheryl strayed\\
  OK, so two couples, Francis and Audrey, and Suzannah and Billy.
\item
  billy collins\\
  Yes. Well, Suzannah and I get along a lot better than Francis and
  Audrey. Thank god.
\item
  cheryl strayed\\
  Perhaps because you guys are still newlyweds, right? You just got
  married last summer.
\item
  billy collins\\
  Well, it's true. We're still --- not to get too sentimental, but we
  still celebrate our monthly anniversary on the 21st of the month.
\item
  cheryl strayed\\
  Oh, that's so sweet.
\item
  billy collins\\
  I think that we're losing track of that.
\item
  cheryl strayed\\
  Oh. I'd love that. So what do you do on your monthly anniversary? Like
  what have you done historically?
\item
  billy collins\\
  Well, we used to go out to dinner, but now Suzannah just cooks another
  dinner. We've been very, very careful, very assiduous about going out
  only when necessary, and wash your hands all the time. And I usually
  say, if you say two Hail Marys, that's about 20 seconds of
  handwashing. And there could be an added benefit. You never know.
\item
  cheryl strayed\\
  Right.
\item
  billy collins\\
  So I'm 79 years old. And it sounds insane to say that, but it's the
  truth, and I'm pretty sure if I caught it, it would kill me. And
  Suzannah shares that view. So she's pretty willful about keeping me in
  the house.
\item
  cheryl strayed\\
  Uh-huh. Do you even go out for walks?
\item
  billy collins\\
  I go out for a walk almost every day. Not quite as far as you, but
  it's 2 miles. And it's very clarifying, though. I've always liked
  walking, and there's a Latin expression, solvitur ambulando, which
  means it is solved by walking. You probably know that.
\item
  cheryl strayed\\
  I do, I do. As you know, my memoir ``Wild'' is all about walking. I
  walked 1,100 miles on the Pacific Crest Trail. A lot is solved by
  walking.
\item
  billy collins\\
  I would say after a thousand miles, you've got yourself together.
\item
  cheryl strayed\\
  I'm sad to report that the walk goes on. The journey continues even
  after that. But I do think that that's one of the things that you see
  over and over, is that power of a sort of walking meditation, where
  the classic journey one takes by foot, that is always enlightening.
  And I think we see that even on a micro scale on those 2-mile walks.
\item
  billy collins\\
  Well, I'm not as adventurous as you are, but it's the same walk. But
  it's never the same walk. It's like putting yourself and putting your
  foot in the same river twice. And, often, just when I get back to the
  house, I might start a poem just about what I saw on the walk, which
  is --- sometimes that goes nowhere, but other times it develops into
  something. So I find it very clarifying and --- I mean, the English
  romantic poets, particularly Wordsworth, the way we would watch
  television, they would go for walks. And Wordsworth said that he
  composed a number of his poems on this walk, which would kind of give
  him the rhythm of the poem maybe. I don't know. If you walked
  iambically, like thu-dud thu-dud, you'd probably be skipping. But I
  guess just the pacing provides a kind of steady rhythm. Did you listen
  --- let me ask you a question. Did you listen to music on your walk?
\item
  cheryl strayed\\
  No. But the reason I didn't listen to music on my walk is it was 1995,
  and so I would have had to carry a Walkman, and cassette tapes, and so
  forth.
\item
  billy collins\\
  Oh, right.
\item
  cheryl strayed\\
  But, Billy, what was so utterly fascinating to me about my experience
  with music on the Pacific Crest Trail is I could not keep music out of
  my mind. And I call it in ``Wild'' like --- that my mind was like the
  transistor radio station in my head, because I kept having songs come
  to me. And, many times, it would be songs I couldn't --- I didn't want
  to have in my head, like jingles for, like, gum, or things like this
  from my youth. Sometimes Bob Dylan songs that I couldn't remember all
  the lyrics. And so I was constantly singing in my head, and the thing
  I missed, I would say, almost as much as food. I really ached for
  music in a way that surprised me. I didn't realize how deep that was.
\item
  billy collins\\
  Well, that's your little internal jukebox that got activated.
\item
  cheryl strayed\\
  Yeah.
\item
  billy collins\\
  Yeah. I wrote a poem about getting a bad song stuck in your head,
  because they seem to be stickier than most. And I titled the poem
  ``More Than a Woman,'' but I say to other people that you can call it
  any song. You can call it ``Dancing Queen'' or any song.
\item
  cheryl strayed\\
  That's a great song.
\item
  billy collins\\
  And sometimes just mentioning the title is enough to activate the song
  in other people's heads, so you have to be careful.
\item
  cheryl strayed\\
  Well, have you --- there's this documentary that I'm forgetting the
  title of it now, but it's about people who have really deep, profound
  dementia, or are at the very end of their lives, and they can't really
  communicate even any more. But they asked their kids, like, well, what
  were their favorite songs when they were young? And they put
  headphones on these people who are so inside themselves, and suddenly
  they would come awake, and they would even sometimes be able to sing
  along with the songs. I think it's an ancient kind of knowing.
\item
  billy collins\\
  Yeah.
\item
  cheryl strayed\\
  And I know you do that with --- I was reading about your mother, who
  would recite poetry to you. I wonder if you have that relationship
  with poetry because you were hearing it so young.
\item
  billy collins\\
  Well, I think I got --- I mean, who knows in these things. If you look
  back on your life and try to make sense of it, you're looking through
  a kind of distorted rear view mirror. But I do have early memories of
  my being a toddler, or advanced toddler stage, and my mother would say
  lines of poetry that she memorized as a school girl in rural Ontario,
  Canada. And if it was very cold out, she'd have a line from
  Shakespeare about the coldness of the weather and not as sharp as
  man's ingratitude to man, or something like that. And when I was a
  kid, I remember hearing my mother talking in two different ways. One
  was when she was regular talking, and one when she was reciting
  poetry. And I didn't know it was poetry. I'd never heard the word. As
  I said, I was sort of a pre-literate toddler. But I knew that the
  poetry sounded nicer.
\item
  cheryl strayed\\
  Mmm, yeah.
\item
  billy collins\\
  And if we believe in really early suggestions on our consciousness,
  that might account from the fact that's where I am today.
\item
  cheryl strayed\\
  Right. That's --- so where did your mom get that? Was she asked as a
  student to memorize these poems?
\item
  billy collins\\
  Well, she and my father were both born in 1901. So when she was a
  schoolgirl, this would be 1907 or `14, or something in there,
  memorization was not only an acceptable way of teaching poetry, it was
  --- I believe it was the only way.
\item
  cheryl strayed\\
  Right.
\item
  billy collins\\
  I mean, they weren't --- this was 100 miles or so north of Toronto.
  They weren't analyzing poetry up there in the schoolhouse. They were
  having kids memorize it. And I try to get my students to memorize a
  little bit of poetry, and it's interesting because their reaction is
  they groan. But once they get a sonnet in their head, or even a little
  poem by Emily Dickinson, they line up outside my office ready to
  recite it for me.
\item
  cheryl strayed\\
  That's so interesting. No. I think I'm absolutely fascinated by this,
  because when I started school in the mid `70s, this idea was that
  memorization was this rote learning that didn't at all feed the
  spirit, and it was sort of oppressive.
\item
  billy collins\\
  Very old fashioned, yeah.
\item
  cheryl strayed\\
  Yeah, yeah. And yet, I wish that people had made me memorize poems.
  Because it is kind of like those jingles. They become part of you, and
  you can call upon them when you need them then, right? They're with
  you.
\item
  billy collins\\
  You could have carried these poems up the Pacific Crest Trail.
\item
  cheryl strayed\\
  Yeah.
\item
  billy collins\\
  You know, it's interesting that one of the most rewarding instances I
  ever had as a teacher, it was on a New York subway. There was a guy, a
  grown guy, across on the other side of the aisle there, or whatever.
  And he was looking at me. And, of course, in this New York subway,
  avoid eye contact is the main rule. Anyway, he got up and approached
  me, and he sat down. And over the roar of the subway, he said,
  Professor Collins, I was your student. And he was then a doctor, an
  oncologist, working at Mount Sinai Hospital. And he said you made us
  memorize a poem, and I want to say it for you.
\item
  cheryl strayed\\
  Oh, wow.
\item
  billy collins\\
  Again, over the roar of the subway, he recited this Emily Dickinson
  poem. And it was just so amazing that I'd done something.
\item
  cheryl strayed\\
  Right. It's so measurable, right? That guy has Emily Dickinson in him
  forever because of you.
\item
  billy collins\\
  It's stored there. And I also think, if they don't remember anything
  I've said, they'll remember 10 lines of Shakespeare or whatever.
\item
  cheryl strayed\\
  Yeah. Well, and what's interesting to ponder about that is that we
  are, in some sort of brute way, just forcing ourselves to remember
  this arrangement of words. And I'm curious, will this doctor, 50 years
  from now on his deathbed, will he be able to say that poem because it
  is part of him. And it reminds me, when I was hiking on the Pacific
  Crest Trail, I realized that I remembered the names of flowers that
  --- my mom was a wonderful gardener and really knew everything about
  plants, and she would always teach me the names of wildflowers. And
  there was a point on the trail where I stepped into this meadow and I
  saw all these flowers, and I realized I knew their names. And I said
  them to myself in my mind as I walked past them, and I felt the
  presence of my mother so strongly. And it was a kind of poem embedded
  in me. It was a memorization that I didn't know I had received until
  then.
\item
  billy collins\\
  Well, and I also think that you really internalize this series of
  lines and bring it into you for yourself. And it's a part of you then.
  And once the student realizes the power of that --- they're now
  walking around with these lines in them, and they can summon them up
  at any moment --- they find that pretty thrilling, actually.
\item
  cheryl strayed\\
  It's a powerful thing. And I think even in cases where you don't
  necessarily memorize the lines, that poems can stay with you in
  another way. That you remember them, or you remember the way that they
  made you feel. And one of your poems that did that for me, Bill, is
  ``The Names.'' You were asked by the librarian of Congress to write it
  for the first anniversary of 9/11. And the reason it stuck with me is
  it made real to me this tragedy that was this grand scale --- many,
  many deaths that seemed kind of abstract until I heard you read that
  poem. And it wasn't then just numbers, it was names. And I thought of
  it now because we are hearing numbers every day. The number of people
  who die of Covid-19 climbs day by day. I would love for you to read
  ``The Names'' to me.
\item
  billy collins\\
  Well, yeah, I'd love to. It's called ``The Names,'' and then, in
  parenthesis, it just says, ``For the victims of Sept. 11 and their
  survivors.''

  ``Yesterday, I lay awake in the palm of the night. A fine rain stole
  in, unhelped by any breeze. And when I saw the silver glaze on the
  windows, I started with A, with Ackerman, as it happened, then Baxter
  and Calabro, Davis and Eberling, names falling into place as droplets
  fell through the dark. Names printed on the ceiling of the night.
  Names slipping around a watery bend. Twenty-six willows on the banks
  of a stream.

  In the morning, I walked out barefoot among thousands of flowers,
  heavy with dew, like the eyes of tears, and each had a name. Fiore
  inscribed on a yellow petal, then Gonzalez and Hon, Ishikawa and
  Jenkins. Names written in the air and stitched into the cloth of the
  day. A name under a photograph taped to a mailbox. Monogram on a torn
  shirt. I see you spelled out on storefront windows and on the bright
  unfurled awnings of this city. I say the names as I turn a corner ---
  Kelly and Lee, Medina, Nardella, and O'Connor.

  When I peer into the woods, I see a thick tangle where letters are
  hidden, as in a puzzle concocted for children. Parker and Quigley in
  the twigs of an ash. Rizzo, Schubert, Torres, and Upton, secrets in
  the boughs of an ancient maple. Names written in the pale sky. Names
  rising in the updraft amid buildings. Silent names in stone, or cried
  out behind a door. Names blown over the earth and out to sea.

  In the evening --- weakening light, the last swallows, a boy on a lake
  lifts his oars, a woman by a window puts a match to a candle, and then
  the names are outlined on the rose clouds --- Vanacore and Wallace.
  Let X, if it can, stand for the ones unfound. Then Young and Ziminsky,
  the final jolt of Z. Names etched on the head of a pin. One name
  spanning a bridge, another undergoing a tunnel. A blue name needled
  into the skin.

  Names of citizens, workers, mothers and fathers, the bright-eyed
  daughter, the quick son. Alphabet of names in green rows in a field.
  Names in the small tracks of birds. Names lifted from a hat or
  balanced on the tip of the tongue. Names wheeled into the dim
  warehouse of memory. So many names, there is barely room on the walls
  of the heart.''
\item
  cheryl strayed\\
  What a powerful poem. What a powerful piece of work.
\item
  billy collins\\
  Yeah. And it was pretty nerve-racking really in front of a joint
  session of Congress, I can tell you.
\item
  cheryl strayed\\
  So you're called to this joint session, and are other people doing
  other things?
\item
  billy collins\\
  Well, yes, lots of other people. It was one of the first times in
  history that both houses of Congress met outside of Washington. They
  met in New York, where it took place. And there were many speeches.
  And I must say something, not in praise of my poem but to make a point
  about poetry, that when I started reading the poem, and it starts
  lying there in the middle of the night with the rain against the
  windows, et cetera, many people were looking up at me the way a Cocker
  Spaniel would cock his head, like, what?
\item
  cheryl strayed\\
  Uh-huh.
\item
  billy collins\\
  It was not political language. It was not using generalities. It was
  not using the recycled language, including a history, heroism,
  tragedy, the common words that kept popping up in these talks. It was
  imagery. It was picture language, as Emerson calls it. It was the
  language of the world, of rain, and windows, and reality. And it
  demonstrated the difference between poetic language and, for lack of a
  better word, other language.
\item
  cheryl strayed\\
  And I think that that's --- I mean, you've just expressed why, in my
  mind, poetry is so necessary, poetry and really all literature. But
  poetry, in particular, to me it speaks to the heart. It speaks beyond
  the intellect, beyond information.
\item
  billy collins\\
  Oh, I think poetry takes us inward, right? Poetry privileges our
  subjectivity. A poem tends to start out in the world and describing
  something, but it usually ends up going down and in inside us. But it
  works in images. I mean, a poem about the virus might be an image of a
  face mask on a curb, a discarded face mask. Just that one thing might
  be enough to tell the whole story, in a sense, just that one image,
  instead of a larger but probably less effective rhetoric.
\item
  cheryl strayed\\
  Right. That's what art does. And I think too, just the ordinary
  stories, whenever --- my favorite parts of all of Ken Burns's
  documentaries are those, the journals that people kept, that weren't
  saying, here is the objective news of the day. It's rather here's what
  happened on my farm on this day. Here's what I saw. Here's what I
  witnessed. Here's what I felt, and feared, and loved, and treasured.
  And those are the kinds of things that poetry shows us and offers us.
\item
  billy collins\\
  And those letters, I mean, I can't believe how articulate everybody
  was in the middle of the 19th century. The letters are --- not only
  the penmanship is beautiful, but the letters are so articulate, at
  least the ones that we're exposed to.
\item
  cheryl strayed\\
  Well, those are the ones that made the cut.
\item
  billy collins\\
  Yeah, that's right. That's the cream of the crop.
\item
  cheryl strayed\\
  Do you keep a journal?
\item
  billy collins\\
  I don't keep a --- I have in the past. I don't --- I'm not very good
  at it. But I keep notebooks, and I always keep a little notebook in my
  pocket, and a pen. And thinking about our talk today, I have a page
  here of things to talk about or points we might --- so anything can go
  into these journals. They're kind of catch-alls, but --- and poems
  too.
\item
  cheryl strayed\\
  What have you written down that we haven't talked about yet? Let's
  talk about it.
\item
  {[}laughter{]}
\item
  billy collins\\
  That's good. Well, I wrote down the name of a poem that's very short
  that I'd love to read before we're done by --- not by me.
\item
  cheryl strayed\\
  Please read it to me. Please read it to me, yes.
\item
  billy collins\\
  May I? Well, it's --- I guess it's a sonnet. It's 14 lines. It's by a
  poet I don't know well, an Irish poet, John O'Donnell. And it's about
  the virus, and it's called ``When.'' And the interesting thing about
  it is that he's slightly ambivalent about the whole thing, about it
  ending. And Eavan Boland, another Irish poet who just died on Monday
  ---
\item
  cheryl strayed\\
  Right.
\item
  billy collins\\
  --- she said, ``Poetry begins where certitude leaves off.'' Like
  poetry deals with ambivalence and ambiguity. And if you're certain
  about a thing, you don't need to write a poem about it. But you'll see
  the word, about halfway through, the word ``although,'' and that's
  when he --- the tone of the poem changes. So I'll read it if you like.
  It's titled ``When,'' by John O'Donnell.

  ``And when this ends we will emerge, shyly and then all at once,
  dazed, long-haired as we embrace loved ones the shadow spared, and
  weep for those it gathered in its shroud. A kind of rapture, this
  longed-for laying on of hands, high cries as we nuzzle, leaning in to
  kiss, and whisper that now things will be different.

  Although a time will come when we'll forget the curve's approaching
  wave, the hiss and sigh of ventilators, the crowded, makeshift
  morgues. A time when we may even miss the old-world arm's-length
  courtesy, small kindnesses left on doorsteps, the drifting, idle days,
  and nights when we flung open all the windows to arias in the
  darkness, our voices reaching out, holding each other till this
  passes.''

  So interesting that he sees the distancing as being a courtesy, and
  the quietness and the kindness of things left on doorsteps, and the
  idleness and the drifting. This really talks about the horror of
  people lost, but he says that --- it's like people talk about. There's
  a quiet all over the world. The whole globe has quieted down. There
  are no sports. There's no concerts. There's no people cheering. A
  friend of mine said you can hear a birdsong in New York City.
\item
  cheryl strayed\\
  Right.
\item
  billy collins\\
  And another --- in The Times last week, this guy writing from Paris
  said you could hear ducks quacking on the Seine. I have a friend, if
  one of us says something, the other one often says, and other poems.
  And I'm sure that would get it --- ducks quacking on the Seine, and
  other poems.
\item
  {[}laughter{]}
\item
  cheryl strayed\\
  That's your next book, Billy.
\item
  {[}laughter{]}
\item
  billy collins\\
  It could be. But that's, I mean, amazing, like being back in the `50s.
  It's quieter. It's slower. And, also, nature is healing herself. Fish
  are coming back into these grounds where they were, where there's too
  much pollution. Maybe we should take one month off a year and just let
  the Earth lick its wounds for a month before we continue to ravish it.
\item
  cheryl strayed\\
  Yeah, I thought of that. In that poem, when he says although, he's
  saying, but we're going to also forget all this, right?
\item
  billy collins\\
  Yeah, yeah. We'll forget the curves and all that. But we might even
  miss --- we'll forget, and we might even be nostalgic about the time
  when we stayed indoors with our spouses or whatever, or maybe a
  friend, sheltering in place. I was on the phone with my friend in New
  York City last night about --- well, I know it's about 2 minutes to 7.
  He said, well, at 7 o'clock, I have to get my pot and my spoon and
  bang it out the window. Everybody does that at 7 o'clock in New York.
\item
  cheryl strayed\\
  Mh-hm. In Portland too.
\item
  billy collins\\
  Yeah, for the responders. And maybe people will miss that. I guess the
  big question is, will this --- I mean, people banter around the word
  globalization, and global this, and global that. And that's primarily
  an economic term, I think. It meant the globalization of business. But
  this is something that's happening all over the world, and maybe this
  will give people a sense of being Earthlings, in the way that the blue
  marble did. Remember that first picture of Earth? That was the first
  time man had ever seen the planet.
\item
  cheryl strayed\\
  Mm-hm, mm-hm.
\item
  billy collins\\
  And there was some sense of optimism --- it became the symbol of Earth
  Day --- that a global event like that, or the globe made visible,
  might tie people together. I'm not too optimistic about that, but it's
  going to be very interesting to see what happens. I mean, maybe
  nothing will happen, and we'll just go back to being the human beings
  we are. But do you want to hear another poem by another poet?
\item
  cheryl strayed\\
  Sure, sure.
\item
  billy collins\\
  This is a poem by Wislawa Szymborska, the Polish poet. And it's a poem
  about the war, and what happens after the war, and then what happens
  when people forget about the war.
\item
  cheryl strayed\\
  And it's about World War II?
\item
  billy collins\\
  It's about World War II, yeah. But I think it applies today. Anyway,
  it's not long. It's called ``The End and the Beginning,'' by Wislawa
  Szymborska.

  ``After every war, someone has to pick up. Things won't pick
  themselves up, after all. Someone has to shove the rubble to the
  roadsides so the carts loaded with corpses can get by. Someone has to
  trudge through the sludge and ashes, through the sofa springs and
  shards of glass, the bloody rags. Someone has to lug the post to prop
  the wall. Someone has to glaze the windows, set the door in its frame.
  No sound bites, no photo opportunities, and it takes years. All the
  cameras have gone to other wars.

  The bridges need to be rebuilt. The railroad stations too.
  Shirtsleeves will be rolled up to shreds. Someone, broom in hand,
  still remembers how it was. Someone else listens, nodding his
  unshattered head. But others are bound to be bustling nearby who will
  find all that quite boring. From time to time, someone still must dig
  up a rusted argument from underneath a bush and haul it off to the
  dump.

  Those who knew what this was all about must make way for those who
  know little, and less than that, and at last nothing, less than
  nothing. Someone has to lie there in the grass that covers up the
  causes and effects with a corn stalk in his teeth, gawking at
  clouds.''
\item
  cheryl strayed\\
  Wow. I love that poem. What a great poem. That poem so aptly describes
  not just things like a world war and its aftermath, and the way that
  time changes it. It's even --- like, I was just thinking of the most
  intimate thing, like a broken heart. Like it's so painful at first,
  and then it gets slowly cleaned up, and then it slowly disappears. And
  then it's like that love was never even in existence.
\item
  billy collins\\
  It becomes history.
\item
  cheryl strayed\\
  It's history.
\item
  billy collins\\
  And there's a test on it.
\item
  cheryl strayed\\
  And there's no great sorrow. I mean, I think that, obviously, about
  big things, like a war or a virus, and so many of us are revisiting
  things like --- I've never heard so much conversation about the
  Spanish influenza epidemic as I have in the last month or two. And
  it's because we're saying, Oh, this thing that we all had the luxury
  of not remembering, or not really caring much about, suddenly might be
  the thing that teaches us about now.
\item
  billy collins\\
  In her poem, after all the destruction, to end with that figure lying
  on the grass with a corn stalk in his teeth, just looking up at
  clouds, that little personal individual moment of being alone. And
  Robert Frost said anybody can start a poem, but it takes a real poet
  to end one. And that's such a beautiful ending, gawking at clouds.
\item
  cheryl strayed\\
  Yeah. Is there anything that I didn't ask you that you want to talk
  about?
\item
  billy collins\\
  I don't know. I mean, I did spend one day in this monastery in Big
  Sur, the Camaldolese order of the Benedictines. They were, I think,
  like a 12th century breakaway from the Benedictines. And someone said
  they make the Benedictines look like Hells Angels, because they're a
  deeply reclusive order. None of them speak. There are about 30 of them
  there. And when they have meals, there are readings from the Bible,
  from whatever. But I had heard that my poems were being read. So when
  I was in California, I made a visit to the monastery and spent one day
  in one of these rooms. It's kind of a cell or a dorm room. And I had
  this experience of silence I'd never quite had before. I wrote a poem
  about what a noisy person I'd been all my life. And, gradually, after
  like an hour or so, I was aware that there are 30 other men right
  around me in these other buildings or rooms who have devoted their
  life to silence. They're professionals, professional quiet people. And
  I felt like this little droplet of my silence just dropping into this
  river of silence they were creating. And communal silence, I realized,
  is such a powerful thing. And when I think about the quieter world
  that the virus has put us into, it just reminded me of that deep
  experience of silence.
\item
  cheryl strayed\\
  What about it felt profound to you? I mean, can you identify what that
  particular silence does for you or for us?
\item
  billy collins\\
  Well, it's very cleansing, I think. I was very comfortable with it,
  knowing, as I said, the presence of these professional men around me
  keeping their silence so as to be less ensnared by the ways of the
  world. And when I was at the monastery in Big Sur, they referred very
  reverentially to two of the monks who had achieved, achieved what they
  call permanent reclusion. There are two monks who never see anybody.
  They're not only silent. They never talk during the year. There's no
  break, and they never go to communal meals. They're like prisoners of
  their silence. And that just blew my mind. They got in the cell and
  locked the door, and they loved it. They wouldn't have it any other
  way.
\item
  cheryl strayed\\
  It's astounding to me, really.

  My first reaction when I was hearing you talk about this place --- and
  I know this place, and I have to say, what you've contributed most
  powerfully to the world is the opposite of silence. It is that we find
  some truth in the words you share with us. And yet there is something
  about poetry and art that is about the deepest silence, right? That
  it's not about isolation. It's not about distance. It's about
  connection, I think, when silence is most profound and powerful.
\item
  billy collins\\
  Mm-hm. I think another way to connect them is to say that the --- a
  poem takes up a certain amount of space on a page. It doesn't fill the
  page. Most people would recognize a poem as being a poem by seeing
  that it just occupies this discrete place on the poem --- on the page,
  rather. And I get my students to think of the white space around the
  poem as silence, and so that the poem is an interruption of silence.
  And so it takes silence into a place. I knew a nun --- well, she was a
  crazy high school friend of mine, but she became a nun, and she was
  --- ended up in a little convent in Rome on the Appian Way there, and
  they were a reclusive order of nuns. And we had corresponded, but when
  she corresponded to me in letters, she would only talk about the love
  of God and all that. She never said what she was doing or anything, or
  if she was planting corn or something. And when I was in Italy, I
  called her --- I called the convent, and I said, Susan --- that was
  her old name --- I said, I'm coming into Rome, and can we take you out
  to lunch? Do you go out? And she said, Cheryl, she said, Fortunately,
  I can't.
\item
  {[}laughter{]}
\item
  cheryl strayed\\
  Really?
\item
  billy collins\\
  Isn't that great?
\item
  cheryl strayed\\
  So she'd ---
\item
  billy collins\\
  No irony --- yes. No irony at all. Fortunately, Billy, I can't.
\item
  cheryl strayed\\
  Wow. I mean, that's a really kind of wonderful way to think about now,
  isn't it? Because ---
\item
  billy collins\\
  Yeah.
\item
  cheryl strayed\\
  --- we who can't go out, to turn that into a blessing rather than a
  curse is something, I think.
\item
  billy collins\\
  Yes, yeah. I think it's the difference between people who will rush
  back into their jobs and can't wait to get back to normal who are
  enjoying, on some level, this quiet time, this traveling back in time,
  in a way, to where you can hear --- you can hear ducks quacking on the
  Seine, for god's sake.
\item
  {[}laughter{]}
\item
  cheryl strayed\\
  Well, it has been such a delight to speak to you.
\item
  billy collins\\
  Well, great. I've been looking forward to this a lot. So great to talk
  to you too.
\item
  cheryl strayed\\
  Bye, Billy.
\item
  billy collins\\
  Bye bye. Take care.
\item
  cheryl strayed\\
  I'm Cheryl Strayed. This is ``Sugar Calling.'' Next week, Joy Harjo.
\end{itemize}

Previous

More episodes ofSugar Calling

\href{https://www.nytimes3xbfgragh.onion/2020/05/20/podcasts/sugar-calling-joy-harjo-poetry-virus.html?action=click\&module=audio-series-bar\&region=header\&pgtype=Article}{\includegraphics{https://static01.graylady3jvrrxbe.onion/images/2020/05/22/podcasts/20sugar-hajo3/20sugar-hajo3-thumbLarge.jpg}}

May 20, 2020~~•~ 35:30`I Release You, Fear'

\href{https://www.nytimes3xbfgragh.onion/2020/05/13/podcasts/sugar-calling-billy-collins-poetry-virus.html?action=click\&module=audio-series-bar\&region=header\&pgtype=Article}{\includegraphics{https://static01.graylady3jvrrxbe.onion/images/2020/05/13/podcasts/13sugar-calling/13sugar-calling-thumbLarge.jpg}}

May 13, 2020`There's a Quiet All Over the World'

\href{https://www.nytimes3xbfgragh.onion/2020/05/06/podcasts/sugar-calling-alice-walker-quarantine-virus.html?action=click\&module=audio-series-bar\&region=header\&pgtype=Article}{\includegraphics{https://static01.graylady3jvrrxbe.onion/images/2020/05/06/podcasts/06sugarcalling/06sugarcalling-thumbLarge.jpg}}

May 6, 2020~~•~ 28:58`Whatever We Have, We Have to Work With It'

\href{https://www.nytimes3xbfgragh.onion/2020/04/29/podcasts/sugar-calling-judy-blume-quarantine-virus.html?action=click\&module=audio-series-bar\&region=header\&pgtype=Article}{\includegraphics{https://static01.graylady3jvrrxbe.onion/images/2020/04/29/podcasts/29sugarcalliing-blume-sub/29sugarcalliing-blume-sub-thumbLarge.jpg}}

April 29, 2020`This Terrible Thing Is Happening, but the World Goes On.'

\href{https://www.nytimes3xbfgragh.onion/2020/04/22/podcasts/sugar-calling-amy-tan-quarantine-virus.html?action=click\&module=audio-series-bar\&region=header\&pgtype=Article}{\includegraphics{https://static01.graylady3jvrrxbe.onion/images/2020/04/27/podcasts/22sugarcalling/22sugarcalling-thumbLarge.jpg}}

April 22, 2020~~•~ 39:19`You Don't Take Dictation. You Find the Truth.'

\href{https://www.nytimes3xbfgragh.onion/2020/04/15/podcasts/sugar-calling-pico-iyer-coronavirus.html?action=click\&module=audio-series-bar\&region=header\&pgtype=Article}{\includegraphics{https://static01.graylady3jvrrxbe.onion/images/2020/04/21/podcasts/15sugarcalling1/15sugarcalling1-thumbLarge.jpg}}

April 15, 2020~~•~ 35:45`Joyful Participation in a World of Sorrows'

\href{https://www.nytimes3xbfgragh.onion/2020/04/08/podcasts/sugar-calling-margaret-atwood-coronavirus.html?action=click\&module=audio-series-bar\&region=header\&pgtype=Article}{\includegraphics{https://static01.graylady3jvrrxbe.onion/images/2020/04/02/books/08sugarcalling1/08sugarcalling1-thumbLarge-v3.jpg}}

April 8, 2020~~•~ 34:32`Roll Up Your Sleeves, Girls'

\href{https://www.nytimes3xbfgragh.onion/2020/04/03/podcasts/sugar-calling-george-saunders-coronavirus.html?action=click\&module=audio-series-bar\&region=header\&pgtype=Article}{\includegraphics{https://static01.graylady3jvrrxbe.onion/images/2020/04/09/podcasts/03sugarcalling-image/merlin_171264408_4ac7fc67-d8cc-45b9-9ec6-bdd20672e694-thumbLarge.jpg}}

April 3, 2020~~•~ 41:16`Everything Is Always Keep Changing'

\href{https://www.nytimes3xbfgragh.onion/column/sugar-calling}{See All
Episodes ofSugar Calling}

Next

May 13, 2020

\begin{itemize}
\item
\item
\item
\item
\item
\item
\end{itemize}

\emph{\textbf{Listen and subscribe to our podcast from your mobile
device:}}
\textbf{\href{https://podcasts.apple.com/us/podcast/sugar-calling/id1505881384}{\emph{Via
Apple Podcasts}}} \emph{\textbf{\textbar{}}}
\textbf{\href{https://open.spotify.com/show/4U8hPiNGIBvTS9zLeiDCN7?si=gRyigD47SPWl-QWgNjgt2w}{\emph{Via
Spotify}}} \emph{\textbf{\textbar{}}}
\textbf{\href{https://www.stitcher.com/podcast/the-new-york-times/sugar-calling}{\emph{Via
Stitcher}}}

\hypertarget{a-poem-about-the-virus-might-be-an-image-of-just-a-face-mask-on-a-curb-a-discarded-face-mask-just-that-one-thing-might-be-enough-to-tell-the-whole-story}{%
\subsection{``A poem about the virus might be an image of just a face
mask on a curb, a discarded face mask. Just that one thing might be
enough to tell the whole
story.''}\label{a-poem-about-the-virus-might-be-an-image-of-just-a-face-mask-on-a-curb-a-discarded-face-mask-just-that-one-thing-might-be-enough-to-tell-the-whole-story}}

\emph{--- The poet Billy Collins}

Today, Cheryl calls Billy Collins, a former poet laureate of the United
States, at his home in Florida. Billy shares the poem he wrote for the
first anniversary of 9/11,
``\href{https://www.nytimes3xbfgragh.onion/2002/09/06/opinion/the-names.html}{The
Names},'' and his experience reading it before a rare joint session of
Congress in New York.

Billy and Cheryl reflect on the quietness --- and slowness --- that has
befallen the world, and they discuss the unexpected comfort in
memorizing poetry: ``You really internalize this series of lines and
bring it into yourself. And it's a part of you then,'' Billy tells
Cheryl.

\includegraphics{https://static01.graylady3jvrrxbe.onion/images/2020/05/13/podcasts/13sugar-calling/13sugar-calling-articleLarge.jpg?quality=75\&auto=webp\&disable=upscale}

\hypertarget{on-todays-episode}{%
\subsubsection{\texorpdfstring{\textbf{On today's
episode:}}{On today's episode:}}\label{on-todays-episode}}

\href{https://www.poetryfoundation.org/poets/billy-collins}{Billy
Collins} is
\href{https://www.nytimes3xbfgragh.onion/2004/10/07/books/poetry-starts-to-wear-100-million-crown.html}{one
of America's best-known poets}. He has written 15 books of poetry,
including ``Sailing Alone Around the Room,''
\href{https://www.nytimes3xbfgragh.onion/2001/11/18/nyregion/citypeople-the-selling-of-billy-collins.html}{a
poetry megaseller}. Collins
\href{https://www.nytimes3xbfgragh.onion/2013/11/17/opinion/sunday/billy-collins.html}{served
as poet laureate of the United States from 2001 to 2003}, and poet
laureate of New York from 2004 to 2006. His latest book,
``\href{https://www.penguinrandomhouse.com/books/547113/whale-day-by-billy-collins/}{Whale
Day},'' is set to be published in Sept. 2020.

\hypertarget{billy-collinss-quarantine-reading-list}{%
\subsubsection{\texorpdfstring{\textbf{Billy Collins's quarantine
reading
list:}}{Billy Collins's quarantine reading list:}}\label{billy-collinss-quarantine-reading-list}}

\begin{itemize}
\item
  ``\href{https://www.indiebound.org/book/9781566893428}{Collected
  Poems},'' Ron Padgett
\item
  ``\href{https://www.indiebound.org/book/9780547928289?aff=NPR}{New and
  Selected Poems},'' Charles Simic
\item
  ``\href{https://www.indiebound.org/book/9780940450158}{Emerson: Essays
  and Lectures},'' Ralph Waldo Emerson
\item
  ``\href{https://www.indiebound.org/book/9780385539302}{The Body: A
  Guide for Occupants},'' Bill Bryson
\item
  ``\href{https://www.indiebound.org/book/9780062963673}{The Dutch
  House},'' Ann Patchett
\item
  ``\href{https://www.indiebound.org/book/9780824521158}{Simplicity},''
  Richard Rohr
\end{itemize}

\begin{center}\rule{0.5\linewidth}{\linethickness}\end{center}

Cheryl Strayed is the author of ``Tiny Beautiful Things,'' ``Torch,''
``Brave Enough,'' and the New York Times best seller ``Wild.'' Her books
have been translated into more than 40 languages. She lives in Portland,
Ore.
\href{https://twitter.com/CherylStrayed?ref_src=twsrc\%5Egoogle\%7Ctwcamp\%5Eserp\%7Ctwgr\%5Eauthor}{@CherylStrayed}

``Sugar Calling'' is produced by Kelly Prime and edited by Sara
Sarasohn, with editorial oversight by Wendy Dorr. This episode was mixed
by Jamie Collazo. Our theme music is by Dan Powell.

Advertisement

\protect\hyperlink{after-bottom}{Continue reading the main story}

\hypertarget{site-index}{%
\subsection{Site Index}\label{site-index}}

\hypertarget{site-information-navigation}{%
\subsection{Site Information
Navigation}\label{site-information-navigation}}

\begin{itemize}
\tightlist
\item
  \href{https://help.nytimes3xbfgragh.onion/hc/en-us/articles/115014792127-Copyright-notice}{©~2020~The
  New York Times Company}
\end{itemize}

\begin{itemize}
\tightlist
\item
  \href{https://www.nytco.com/}{NYTCo}
\item
  \href{https://help.nytimes3xbfgragh.onion/hc/en-us/articles/115015385887-Contact-Us}{Contact
  Us}
\item
  \href{https://www.nytco.com/careers/}{Work with us}
\item
  \href{https://nytmediakit.com/}{Advertise}
\item
  \href{http://www.tbrandstudio.com/}{T Brand Studio}
\item
  \href{https://www.nytimes3xbfgragh.onion/privacy/cookie-policy\#how-do-i-manage-trackers}{Your
  Ad Choices}
\item
  \href{https://www.nytimes3xbfgragh.onion/privacy}{Privacy}
\item
  \href{https://help.nytimes3xbfgragh.onion/hc/en-us/articles/115014893428-Terms-of-service}{Terms
  of Service}
\item
  \href{https://help.nytimes3xbfgragh.onion/hc/en-us/articles/115014893968-Terms-of-sale}{Terms
  of Sale}
\item
  \href{https://spiderbites.nytimes3xbfgragh.onion}{Site Map}
\item
  \href{https://help.nytimes3xbfgragh.onion/hc/en-us}{Help}
\item
  \href{https://www.nytimes3xbfgragh.onion/subscription?campaignId=37WXW}{Subscriptions}
\end{itemize}
