Sections

SEARCH

\protect\hyperlink{site-content}{Skip to
content}\protect\hyperlink{site-index}{Skip to site index}

\href{https://www.nytimes3xbfgragh.onion/section/nyregion}{New York}

\href{https://myaccount.nytimes3xbfgragh.onion/auth/login?response_type=cookie\&client_id=vi}{}

\href{https://www.nytimes3xbfgragh.onion/section/todayspaper}{Today's
Paper}

\href{/section/nyregion}{New York}\textbar{}He Saw `No Proof' Closures
Would Curb Virus. Now He Has De Blasio's Trust.

\url{https://nyti.ms/2yPCrFM}

\begin{itemize}
\item
\item
\item
\item
\item
\end{itemize}

\href{https://www.nytimes3xbfgragh.onion/news-event/coronavirus?action=click\&pgtype=Article\&state=default\&region=TOP_BANNER\&context=storylines_menu}{The
Coronavirus Outbreak}

\begin{itemize}
\tightlist
\item
  live\href{https://www.nytimes3xbfgragh.onion/2020/08/04/world/coronavirus-cases.html?action=click\&pgtype=Article\&state=default\&region=TOP_BANNER\&context=storylines_menu}{Latest
  Updates}
\item
  \href{https://www.nytimes3xbfgragh.onion/interactive/2020/us/coronavirus-us-cases.html?action=click\&pgtype=Article\&state=default\&region=TOP_BANNER\&context=storylines_menu}{Maps
  and Cases}
\item
  \href{https://www.nytimes3xbfgragh.onion/interactive/2020/science/coronavirus-vaccine-tracker.html?action=click\&pgtype=Article\&state=default\&region=TOP_BANNER\&context=storylines_menu}{Vaccine
  Tracker}
\item
  \href{https://www.nytimes3xbfgragh.onion/2020/08/02/us/covid-college-reopening.html?action=click\&pgtype=Article\&state=default\&region=TOP_BANNER\&context=storylines_menu}{College
  Reopening}
\item
  \href{https://www.nytimes3xbfgragh.onion/live/2020/08/04/business/stock-market-today-coronavirus?action=click\&pgtype=Article\&state=default\&region=TOP_BANNER\&context=storylines_menu}{Economy}
\end{itemize}

Advertisement

\protect\hyperlink{after-top}{Continue reading the main story}

Supported by

\protect\hyperlink{after-sponsor}{Continue reading the main story}

\hypertarget{he-saw-no-proof-closures-would-curb-virus-now-he-has-de-blasios-trust}{%
\section{He Saw `No Proof' Closures Would Curb Virus. Now He Has De
Blasio's
Trust.}\label{he-saw-no-proof-closures-would-curb-virus-now-he-has-de-blasios-trust}}

The head of New York City's public hospitals pushed to keep the city
open in early March. Now the mayor has put him in charge of contact
tracing, deepening a rift with the Health Department.

\includegraphics{https://static01.graylady3jvrrxbe.onion/images/2020/05/15/nyregion/00NYVIRUS-HHC1/00NYVIRUS-HHC1-articleLarge.jpg?quality=75\&auto=webp\&disable=upscale}

\href{https://www.nytimes3xbfgragh.onion/by/william-k-rashbaum}{\includegraphics{https://static01.graylady3jvrrxbe.onion/images/2018/06/13/multimedia/author-william-k-rashbaum/author-william-k-rashbaum-thumbLarge.jpg}}\href{https://www.nytimes3xbfgragh.onion/by/j-david-goodman}{\includegraphics{https://static01.graylady3jvrrxbe.onion/images/2018/07/18/nyregion/author-j-david-goodman/author-j-david-goodman-thumbLarge.png}}\href{https://www.nytimes3xbfgragh.onion/by/jeffery-c-mays}{\includegraphics{https://static01.graylady3jvrrxbe.onion/images/2018/07/18/multimedia/author-jeffery-c-mays/author-jeffery-c-mays-thumbLarge.png}}\href{https://www.nytimes3xbfgragh.onion/by/joseph-goldstein}{\includegraphics{https://static01.graylady3jvrrxbe.onion/images/2018/07/16/multimedia/author-joseph-goldstein/author-joseph-goldstein-thumbLarge.png}}

By
\href{https://www.nytimes3xbfgragh.onion/by/william-k-rashbaum}{William
K. Rashbaum},
\href{https://www.nytimes3xbfgragh.onion/by/j-david-goodman}{J. David
Goodman},
\href{https://www.nytimes3xbfgragh.onion/by/jeffery-c-mays}{Jeffery C.
Mays} and
\href{https://www.nytimes3xbfgragh.onion/by/joseph-goldstein}{Joseph
Goldstein}

\begin{itemize}
\item
  May 14, 2020
\item
  \begin{itemize}
  \item
  \item
  \item
  \item
  \item
  \end{itemize}
\end{itemize}

As Mayor Bill de Blasio was resisting calls in March to cancel large
gatherings and slow the spread of the coronavirus in New York City, he
found behind-the-scenes support from a trusted voice: the head of his
public hospital system, Dr. Mitchell Katz.

There was ``no proof that closures will help stop the spread,'' Dr. Katz
wrote in an email to the mayor's closest aides. He believed that banning
large events would hurt the economy and sow fear. ``If it is not safe to
go to a conference, why is it safe to go to the hospital or ride in the
subway?'' he wrote. And, he said, many New Yorkers were going to get
infected anyway.

``We have to accept that unless a vaccine is rapidly developed, large
numbers of people will get infected,'' he wrote. ``The good thing is
greater than 99 percent will recover without harm. Once people recover
they will have immunity. The immunity will protect the herd.''

For Mr. de Blasio, the arguments in Dr. Katz's March 10 email, obtained
by The New York Times, appeared to hold sway over the calls for greater
restrictions on daily life from top Health Department officials, who
were alarmed by public health surveillance data pointing toward a
looming outbreak.

The mayor did not order major closures, including of schools and
restaurants, until almost a week after the email ---
\href{https://www.nytimes3xbfgragh.onion/2020/04/08/nyregion/new-york-coronavirus-response-delays.html}{a
delay that epidemiologists say allowed the virus to spread}.

\includegraphics{https://static01.graylady3jvrrxbe.onion/images/2020/05/13/nyregion/00nyvirus-hhc2/merlin_170320284_ec945e83-055f-42e7-b857-79c768940c29-articleLarge.jpg?quality=75\&auto=webp\&disable=upscale}

Now, as the crisis in New York City enters the next stage, Mr. de
Blasio, Dr. Katz and Health Department officials are once again
navigating a nasty public fissure.

Health experts fear that their rift may threaten the city's ability to
limit the spread of the disease once the city --- which has seen more
than 20,000 people die of the virus --- begins to reopen.

The mayor last week shocked the Health Department by taking away its
authority to oversee contact tracing,
\href{https://www.nytimes3xbfgragh.onion/2020/05/07/nyregion/coronavirus-contact-tracing-nyc.html}{giving
the job} to Health and Hospitals, the agency overseen by Dr. Katz. It is
a monumental task: The city must build and run an army of some 2,500
people to track and trace the close contacts of every infected person.

The mayor's decision to shift the responsibility to the public hospital
system illustrated how Mr. de Blasio's faith in Dr. Katz's leadership
abilities took precedence over the experience and knowledge of his
public health officials, who have clashed with the mayor over a variety
of issues.

Epidemiologists, former public health officials and the city comptroller
criticized the move, pointing out that the Health Department has for
decades expertly performed contact tracing for diseases like
tuberculosis and H.I.V., and had been preparing for two weeks to run the
expanded tracing operation for Covid-19, the disease caused by the
coronavirus.

On Tuesday, the city comptroller, Scott M. Stringer, requested documents
from City Hall as part of a formal investigation into the city's
response to the pandemic, including its handling of public health
recommendations. On Friday, the City Council will hold a hearing on the
mayor's decision to give Health and Hospitals control of the contact
tracing effort.

Even the person who ran the city's hospital system before Dr. Katz
thought the move was a mistake.

``It is a head-scratcher. I can't figure out the rationale, and I don't
think it's worth the risks,'' said Stanley Brezenoff,
\href{https://www.nytimes3xbfgragh.onion/2016/11/08/nyregion/chief-of-new-yorks-struggling-public-hospital-system-is-resigning.html}{who
was chosen} by Mr. de Blasio in 2016 to temporarily lead Health and
Hospitals. ``Just because they both have `health' in the name doesn't
mean they're in the same business.''

\hypertarget{latest-updates-global-coronavirus-outbreak}{%
\section{\texorpdfstring{\href{https://www.nytimes3xbfgragh.onion/2020/08/04/world/coronavirus-cases.html?action=click\&pgtype=Article\&state=default\&region=MAIN_CONTENT_1\&context=storylines_live_updates}{Latest
Updates: Global Coronavirus
Outbreak}}{Latest Updates: Global Coronavirus Outbreak}}\label{latest-updates-global-coronavirus-outbreak}}

Updated 2020-08-04T19:23:08.893Z

\begin{itemize}
\tightlist
\item
  \href{https://www.nytimes3xbfgragh.onion/2020/08/04/world/coronavirus-cases.html?action=click\&pgtype=Article\&state=default\&region=MAIN_CONTENT_1\&context=storylines_live_updates\#link-4825b93}{Public
  and private schools in Maryland and elsewhere are divided over
  in-person instruction.}
\item
  \href{https://www.nytimes3xbfgragh.onion/2020/08/04/world/coronavirus-cases.html?action=click\&pgtype=Article\&state=default\&region=MAIN_CONTENT_1\&context=storylines_live_updates\#link-4d1eafa8}{N.Y.C.'s
  health commissioner resigns after clashing with the mayor over the
  virus.}
\item
  \href{https://www.nytimes3xbfgragh.onion/2020/08/04/world/coronavirus-cases.html?action=click\&pgtype=Article\&state=default\&region=MAIN_CONTENT_1\&context=storylines_live_updates\#link-6b644638}{`Long
  days, long nights': Washington prepares for a prolonged fight over
  virus relief.}
\end{itemize}

\href{https://www.nytimes3xbfgragh.onion/2020/08/04/world/coronavirus-cases.html?action=click\&pgtype=Article\&state=default\&region=MAIN_CONTENT_1\&context=storylines_live_updates}{See
more updates}

More live coverage:
\href{https://www.nytimes3xbfgragh.onion/live/2020/08/04/business/stock-market-today-coronavirus?action=click\&pgtype=Article\&state=default\&region=MAIN_CONTENT_1\&context=storylines_live_updates}{Markets}

``I'm second to none in my admiration for Mitch's clinical prowess,''
Mr. Brezenoff added, ``but this is a job for the Health Department.''

Dr. Katz, who is well regarded in the hospitals field, declined to
comment for this article. His defenders said it was important to view
his March 10 email about keeping the city open within the context of
that time. Public health experts were wrestling with a highly unusual
series of factors, and there was a wide range of opinions about how to
respond.

The mayor's press secretary, Freddi Goldstein, when asked about the
March 10 email, said that the mayor had sought advice from many experts,
and that a range of individuals including Dr. Katz and the health
commissioner, Dr. Oxiris Barbot, ``gave the best advice based on what
they knew at that time.''

Ms. Goldstein said the decision to place its new contact tracing corps
under the control of Health and Hospitals was rooted in the need for a
``single, streamlined entity'' to manage the program, along with the
city's diagnostic testing efforts and oversight of care for infected
patients isolated in hotels.

``While the academic and expert knowledge that the Health Department
brings are essential pieces,'' Ms. Goldstein said in an email, ``the
ability to rely on the organizational infrastructure that H+H brings is
also essential.''

The mayor, she said, wanted the city to ``capitalize on both.''

Mr. de Blasio, Dr. Katz and senior city officials have insisted that
Health and Hospitals --- a quasi-private organization controlled by the
government --- was chosen largely for practical reasons. Because of its
structure, it could hire people and award contracts more quickly than
the Health Department, a city agency that generally must follow city
procurement policies.

But during the coronavirus crisis, the hiring and contracts had already
been taken over by a foundation, the Fund for Public Health in New York,
which works with the Health Department; the city has also streamlined
Covid-19-related spending by suspending typical procurement
requirements.

Ms. Goldstein contended that Health and Hospitals had built-in
advantages, including access to supply chains, clinical surge staffing
contracts, telemedicine and reference lab contracts. It made sense to
give the tracing program to the hospitals, she added, so that the major
components of the city's future efforts to control the virus ---
testing, tracing, isolation --- could be run under one roof.

An effective contact tracing program must be up and running before New
Yorkers can begin to safely emerge from their lockdown, public health
officials said. Getting such a system ready is a daunting challenge ---
even without any stumbles or political infighting.

Image

With the number of new coronavirus cases decreasing in New York City, an
effective contact tracing system will be necessary to help prevent a new
outbreak as the city opens.Credit...Desiree Rios for The New York Times

But since taking office in 2014, Mr. de Blasio has found himself
routinely at odds with his own public health officials, differing on
issues including a proposed ban on horse carriages in Central Park and
an outbreak of Legionnaires' disease in the Bronx.

Current and former administration officials said the conflict stemmed
from Mr. de Blasio's apparent distrust of experts and his
dissatisfaction with public health recommendations, which are often
based on scientific analysis of imperfect and at times incomplete
information.

The mayor prizes certainty, decisiveness and directness, aides and
former officials said. Dr. Katz, who came to New York after holding top
public health positions in San Francisco and Los Angeles, often
presented Mr. de Blasio with information in the way the mayor favored.

``He understands the importance of making clear, definitive decisions
for the top elected official he's working for,'' Eric Phillips, the
mayor's former press secretary who now works in crisis management for
the public relations firm Edelman, said of Dr. Katz. ``That's why the
mayor trusts his judgment. Dr. Katz is not afraid to take a complicated
subject, give his opinion and make a recommendation.''

That dynamic emerged again in the early days of the coronavirus
outbreak. By the second week of March, the city's public health warning
system --- known as
\href{https://a816-health.nyc.gov/hdi/epiquery/visualizations?PageType=ps\&PopulationSource=Syndromic}{the
syndromic surveillance system} --- began strongly signaling the spread
of a flulike illness, officials said.

City Hall wanted to see firm numbers of positive test results before
ordering closures. Publicly, as well as in guidance to other agencies as
late as March 9, the Health Department was not recommending the closure
of events, like the city's half marathon, according to an email shared
with The Times.

But inside the department, staff members were deeply concerned that
their warnings were not getting through. Some were ready to walk out in
protest; others were threatening to quit.

In his March 10 email to top city officials, Dr. Katz made the case that
keeping the city open was the best approach at the time.

``Canceling large gatherings gives people the wrong impression of this
illness,'' he wrote. ``Many of the events are being canceled anyway, and
fewer people are going out. However, it is very different when the
government starts telling people to do this.''

He wrote that Italy ``is having a terrible problem that I do not believe
we will have,'' and ended the message by arguing that shutting down
events could create fear among some with mental health issues.

``If even a few people with serious mental illness become more isolated
or fearful due to messaging, we could have more permanent harm than we
currently have with Covid-19,'' he wrote in the email, which was sent to
three deputy mayors, top health officials and the budget director.

Ms. Goldstein said Dr. Katz stood by his concern over the harm caused by
isolation to those with mental illness and by his comments on herd
immunity.

\href{https://www.nytimes3xbfgragh.onion/news-event/coronavirus?action=click\&pgtype=Article\&state=default\&region=MAIN_CONTENT_3\&context=storylines_faq}{}

\hypertarget{the-coronavirus-outbreak-}{%
\subsubsection{The Coronavirus Outbreak
›}\label{the-coronavirus-outbreak-}}

\hypertarget{frequently-asked-questions}{%
\paragraph{Frequently Asked
Questions}\label{frequently-asked-questions}}

Updated August 4, 2020

\begin{itemize}
\item ~
  \hypertarget{i-have-antibodies-am-i-now-immune}{%
  \paragraph{I have antibodies. Am I now
  immune?}\label{i-have-antibodies-am-i-now-immune}}

  \begin{itemize}
  \tightlist
  \item
    As of right
    now,\href{https://www.nytimes3xbfgragh.onion/2020/07/22/health/covid-antibodies-herd-immunity.html?action=click\&pgtype=Article\&state=default\&region=MAIN_CONTENT_3\&context=storylines_faq}{that
    seems likely, for at least several months.} There have been
    frightening accounts of people suffering what seems to be a second
    bout of Covid-19. But experts say these patients may have a
    drawn-out course of infection, with the virus taking a slow toll
    weeks to months after initial exposure. People infected with the
    coronavirus typically
    \href{https://www.nature.com/articles/s41586-020-2456-9}{produce}
    immune molecules called antibodies, which are
    \href{https://www.nytimes3xbfgragh.onion/2020/05/07/health/coronavirus-antibody-prevalence.html?action=click\&pgtype=Article\&state=default\&region=MAIN_CONTENT_3\&context=storylines_faq}{protective
    proteins made in response to an
    infection}\href{https://www.nytimes3xbfgragh.onion/2020/05/07/health/coronavirus-antibody-prevalence.html?action=click\&pgtype=Article\&state=default\&region=MAIN_CONTENT_3\&context=storylines_faq}{.
    These antibodies may} last in the body
    \href{https://www.nature.com/articles/s41591-020-0965-6}{only two to
    three months}, which may seem worrisome, but that's perfectly normal
    after an acute infection subsides, said Dr. Michael Mina, an
    immunologist at Harvard University. It may be possible to get the
    coronavirus again, but it's highly unlikely that it would be
    possible in a short window of time from initial infection or make
    people sicker the second time.
  \end{itemize}
\item ~
  \hypertarget{im-a-small-business-owner-can-i-get-relief}{%
  \paragraph{I'm a small-business owner. Can I get
  relief?}\label{im-a-small-business-owner-can-i-get-relief}}

  \begin{itemize}
  \tightlist
  \item
    The
    \href{https://www.nytimes3xbfgragh.onion/article/small-business-loans-stimulus-grants-freelancers-coronavirus.html?action=click\&pgtype=Article\&state=default\&region=MAIN_CONTENT_3\&context=storylines_faq}{stimulus
    bills enacted in March} offer help for the millions of American
    small businesses. Those eligible for aid are businesses and
    nonprofit organizations with fewer than 500 workers, including sole
    proprietorships, independent contractors and freelancers. Some
    larger companies in some industries are also eligible. The help
    being offered, which is being managed by the Small Business
    Administration, includes the Paycheck Protection Program and the
    Economic Injury Disaster Loan program. But lots of folks have
    \href{https://www.nytimes3xbfgragh.onion/interactive/2020/05/07/business/small-business-loans-coronavirus.html?action=click\&pgtype=Article\&state=default\&region=MAIN_CONTENT_3\&context=storylines_faq}{not
    yet seen payouts.} Even those who have received help are confused:
    The rules are draconian, and some are stuck sitting on
    \href{https://www.nytimes3xbfgragh.onion/2020/05/02/business/economy/loans-coronavirus-small-business.html?action=click\&pgtype=Article\&state=default\&region=MAIN_CONTENT_3\&context=storylines_faq}{money
    they don't know how to use.} Many small-business owners are getting
    less than they expected or
    \href{https://www.nytimes3xbfgragh.onion/2020/06/10/business/Small-business-loans-ppp.html?action=click\&pgtype=Article\&state=default\&region=MAIN_CONTENT_3\&context=storylines_faq}{not
    hearing anything at all.}
  \end{itemize}
\item ~
  \hypertarget{what-are-my-rights-if-i-am-worried-about-going-back-to-work}{%
  \paragraph{What are my rights if I am worried about going back to
  work?}\label{what-are-my-rights-if-i-am-worried-about-going-back-to-work}}

  \begin{itemize}
  \tightlist
  \item
    Employers have to provide
    \href{https://www.osha.gov/SLTC/covid-19/standards.html}{a safe
    workplace} with policies that protect everyone equally.
    \href{https://www.nytimes3xbfgragh.onion/article/coronavirus-money-unemployment.html?action=click\&pgtype=Article\&state=default\&region=MAIN_CONTENT_3\&context=storylines_faq}{And
    if one of your co-workers tests positive for the coronavirus, the
    C.D.C.} has said that
    \href{https://www.cdc.gov/coronavirus/2019-ncov/community/guidance-business-response.html}{employers
    should tell their employees} -\/- without giving you the sick
    employee's name -\/- that they may have been exposed to the virus.
  \end{itemize}
\item ~
  \hypertarget{should-i-refinance-my-mortgage}{%
  \paragraph{Should I refinance my
  mortgage?}\label{should-i-refinance-my-mortgage}}

  \begin{itemize}
  \tightlist
  \item
    \href{https://www.nytimes3xbfgragh.onion/article/coronavirus-money-unemployment.html?action=click\&pgtype=Article\&state=default\&region=MAIN_CONTENT_3\&context=storylines_faq}{It
    could be a good idea,} because mortgage rates have
    \href{https://www.nytimes3xbfgragh.onion/2020/07/16/business/mortgage-rates-below-3-percent.html?action=click\&pgtype=Article\&state=default\&region=MAIN_CONTENT_3\&context=storylines_faq}{never
    been lower.} Refinancing requests have pushed mortgage applications
    to some of the highest levels since 2008, so be prepared to get in
    line. But defaults are also up, so if you're thinking about buying a
    home, be aware that some lenders have tightened their standards.
  \end{itemize}
\item ~
  \hypertarget{what-is-school-going-to-look-like-in-september}{%
  \paragraph{What is school going to look like in
  September?}\label{what-is-school-going-to-look-like-in-september}}

  \begin{itemize}
  \tightlist
  \item
    It is unlikely that many schools will return to a normal schedule
    this fall, requiring the grind of
    \href{https://www.nytimes3xbfgragh.onion/2020/06/05/us/coronavirus-education-lost-learning.html?action=click\&pgtype=Article\&state=default\&region=MAIN_CONTENT_3\&context=storylines_faq}{online
    learning},
    \href{https://www.nytimes3xbfgragh.onion/2020/05/29/us/coronavirus-child-care-centers.html?action=click\&pgtype=Article\&state=default\&region=MAIN_CONTENT_3\&context=storylines_faq}{makeshift
    child care} and
    \href{https://www.nytimes3xbfgragh.onion/2020/06/03/business/economy/coronavirus-working-women.html?action=click\&pgtype=Article\&state=default\&region=MAIN_CONTENT_3\&context=storylines_faq}{stunted
    workdays} to continue. California's two largest public school
    districts --- Los Angeles and San Diego --- said on July 13, that
    \href{https://www.nytimes3xbfgragh.onion/2020/07/13/us/lausd-san-diego-school-reopening.html?action=click\&pgtype=Article\&state=default\&region=MAIN_CONTENT_3\&context=storylines_faq}{instruction
    will be remote-only in the fall}, citing concerns that surging
    coronavirus infections in their areas pose too dire a risk for
    students and teachers. Together, the two districts enroll some
    825,000 students. They are the largest in the country so far to
    abandon plans for even a partial physical return to classrooms when
    they reopen in August. For other districts, the solution won't be an
    all-or-nothing approach.
    \href{https://bioethics.jhu.edu/research-and-outreach/projects/eschool-initiative/school-policy-tracker/}{Many
    systems}, including the nation's largest, New York City, are
    devising
    \href{https://www.nytimes3xbfgragh.onion/2020/06/26/us/coronavirus-schools-reopen-fall.html?action=click\&pgtype=Article\&state=default\&region=MAIN_CONTENT_3\&context=storylines_faq}{hybrid
    plans} that involve spending some days in classrooms and other days
    online. There's no national policy on this yet, so check with your
    municipal school system regularly to see what is happening in your
    community.
  \end{itemize}
\end{itemize}

She added that if there was anything to be second-guessed, it would be
the shifting guidance from the Centers for Disease Control. ``The C.D.C.
repeatedly misled us and greatly hampered our ability to catch and
respond to this disease,'' she said.

Part of Dr. Katz's reasoning in March for opposing closures,
particularly of city schools, was that it would lead to health care
workers not showing up to work --- a concern shared by leaders of New
York's private hospitals. Kenneth E. Raske, the president of the Greater
New York Hospital Association, said large gatherings were a ``corollary
issue'' to the question of closing schools.

Michael Dowling, the chief executive of Northwell Health, the state's
largest health care provider, said, ``Mitch was asking questions that a
lot of people were asking at that point.''

But Mr. Dowling also recalled that on March 10, the day that Dr. Katz
sent the email, Northwell decided to cancel all of its in-person
executive meetings, after a top hospital leader involved in the pandemic
response tested positive for Covid-19.

That same day, Dr. Katz appeared with the mayor at a news conference at
Bellevue Hospital, one of 11 public hospitals in the city's system. In
spite of signs that the virus's presence was growing, Dr. Katz gave an
optimistic report, saying the system was ready to handle a surge in
patients.

Image

Dr. Katz and leaders of private hospitals initially did not want city
schools to close because they feared that it might force health care
workers to miss work.Credit...Sarah Blesener for The New York Times

His confidence --- and his arguments in favor of keeping the city open
--- increased his stature with the mayor, said two people with knowledge
of City Hall discussions.

His influence grew further during that period in March, as the mayor
sought an expansion of testing and the city began opening testing sites
at public hospitals. The Health Department objected at the time, saying
that outpatient testing sites would draw those with the disease into
contact with those who were not yet infected.

Mr. de Blasio's decision to take the contact tracing program away from
Health Department leadership may have been the most public example of
their fractured relationship, but their conflicts stretch back to the
mayor's first term.

Aides to the mayor, looking to help him deliver a campaign promise to
ban horse-drawn carriages from Central Park, sought assistance from the
Health Department, said Daniel Kass, a former deputy health
commissioner.

``City Hall tried to enlist the Health Department to support the
argument that the stables were unfit, and that horses were injured at a
high rate on the streets in New York,'' Mr. Kass said. ``The Health
Department had no data to support those arguments.''

The rift widened during the Legionnaires' disease outbreak in 2015 in
the Bronx that killed 12 people. The Health Department had identified
five buildings in the Bronx as the source of the disease, but Mr. de
Blasio wanted the department to test cooling towers broadly across the
city.

That struck officials as illogical, but Mr. de Blasio was insistent. In
one gathering of roughly 50 people at Lincoln Hospital in the Bronx, the
mayor appeared eager for other city agencies to step in, and he looked
to his buildings commissioner for possible assistance.

``Do you need helicopters?'' Mr. de Blasio asked, according to a person
who was present at the meeting. The dumbfounded buildings commissioner
said he did not. Mr. de Blasio upbraided him, and then stormed out of
the meeting.

``The mayor was elected to protect and advocate for New Yorkers ---
that's what he did,'' Ms. Goldstein said of the Legionnaires' episode.

Another flare-up emerged during the coronavirus outbreak, when the mayor
interceded in an argument in late March between Dr. Barbot and Terence
Monahan, the highest-ranking uniformed member of the Police Department.
Chief Monahan had demanded that the Health Department relinquish
hundreds of thousands of protective masks for the police force to use.

The mayor sided with Chief Monahan; the roughly seven-week-old
confrontation,
\href{https://nypost.com/2020/05/13/nyc-health-commissioner-wouldnt-supply-nypd-with-masks/}{reported
by The New York Post} late Wednesday, led Dr. Barbot to apologize,
according to her spokesman.

An official at the Health Department said Dr. Barbot's remarks came
after Police Department officials had shown up at a secure Health
Department warehouse and tried to commandeer 500,000 N-95 masks that
were earmarked for hospitals. Chief Monahan said in an interview that
the city's Office of Emergency Management had given the go-ahead to pick
up 250,000 masks; when police showed up to the warehouse, they were told
they were to receive only 50,000 masks.

During the confrontation, Dr. Barbot told Chief Monahan that she did not
``give two rats' asses about your cops,'' according to The Post. The
alleged remark drew vitriolic criticism from police unions, including
the Sergeants Benevolent Association,
\href{https://twitter.com/SBANYPD/status/1260730557190848512}{which
referred} to Dr. Barbot as a ``bitch'' who had ``blood on her hands'' in
a tweet.

After the mayor's office intervened, the police were given 250,000
masks.

Mr. de Blasio said on Thursday that he was not previously aware of the
heated argument, and that he planned to discuss it with Dr. Barbot. ``If
what was reported was accurate, the commissioner needs to apologize to
the men and women of the N.Y.P.D.,'' he said.

The episode was emblematic of the growing rift between City Hall and the
Health Department, a state of affairs that Denis Nash, a professor at
the C.U.N.Y. School of Public Health who served as director of
H.I.V./AIDS surveillance at the Health Department, said was ``incredibly
tragic.''

``It seems to be getting worse when we're in the middle of a public
health crisis,'' he said. ``I worry that a rift like that could cost
many lives in New York.''

Ashley Southall contributed reporting.

Advertisement

\protect\hyperlink{after-bottom}{Continue reading the main story}

\hypertarget{site-index}{%
\subsection{Site Index}\label{site-index}}

\hypertarget{site-information-navigation}{%
\subsection{Site Information
Navigation}\label{site-information-navigation}}

\begin{itemize}
\tightlist
\item
  \href{https://help.nytimes3xbfgragh.onion/hc/en-us/articles/115014792127-Copyright-notice}{©~2020~The
  New York Times Company}
\end{itemize}

\begin{itemize}
\tightlist
\item
  \href{https://www.nytco.com/}{NYTCo}
\item
  \href{https://help.nytimes3xbfgragh.onion/hc/en-us/articles/115015385887-Contact-Us}{Contact
  Us}
\item
  \href{https://www.nytco.com/careers/}{Work with us}
\item
  \href{https://nytmediakit.com/}{Advertise}
\item
  \href{http://www.tbrandstudio.com/}{T Brand Studio}
\item
  \href{https://www.nytimes3xbfgragh.onion/privacy/cookie-policy\#how-do-i-manage-trackers}{Your
  Ad Choices}
\item
  \href{https://www.nytimes3xbfgragh.onion/privacy}{Privacy}
\item
  \href{https://help.nytimes3xbfgragh.onion/hc/en-us/articles/115014893428-Terms-of-service}{Terms
  of Service}
\item
  \href{https://help.nytimes3xbfgragh.onion/hc/en-us/articles/115014893968-Terms-of-sale}{Terms
  of Sale}
\item
  \href{https://spiderbites.nytimes3xbfgragh.onion}{Site Map}
\item
  \href{https://help.nytimes3xbfgragh.onion/hc/en-us}{Help}
\item
  \href{https://www.nytimes3xbfgragh.onion/subscription?campaignId=37WXW}{Subscriptions}
\end{itemize}
