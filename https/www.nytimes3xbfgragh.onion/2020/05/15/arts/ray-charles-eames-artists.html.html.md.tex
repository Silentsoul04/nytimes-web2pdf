\href{/section/arts}{Arts}\textbar{}Ray Eames, Out of Her Husband's
Shadow

\url{https://nyti.ms/2WyXvsX}

\begin{itemize}
\item
\item
\item
\item
\item
\item
\end{itemize}

\includegraphics{https://static01.graylady3jvrrxbe.onion/images/2020/05/17/lens/00-mrs-eames-04/00-mrs-eames-04-articleLarge.jpg?quality=75\&auto=webp\&disable=upscale}

Sections

\protect\hyperlink{site-content}{Skip to
content}\protect\hyperlink{site-index}{Skip to site index}

The Great Read

\hypertarget{ray-eames-out-of-her-husbands-shadow}{%
\section{Ray Eames, Out of Her Husband's
Shadow}\label{ray-eames-out-of-her-husbands-shadow}}

The designer let her larger-than-life husband Charles do the talking.
But the style symbolized by their shared name was a joint creation.

The designer Ray Eames around 1950.Credit...Peter Stackpole/The LIFE
Picture Collection, via Getty Images

Supported by

\protect\hyperlink{after-sponsor}{Continue reading the main story}

\href{https://www.nytimes3xbfgragh.onion/by/jennifer-schuessler}{\includegraphics{https://static01.graylady3jvrrxbe.onion/images/2018/02/16/multimedia/author-jennifer-schuessler/author-jennifer-schuessler-thumbLarge-v2.png}}

By
\href{https://www.nytimes3xbfgragh.onion/by/jennifer-schuessler}{Jennifer
Schuessler}

\begin{itemize}
\item
  May 15, 2020
\item
  \begin{itemize}
  \item
  \item
  \item
  \item
  \item
  \item
  \end{itemize}
\end{itemize}

\emph{The Mrs. Files looks at history through a contemporary lens to see
what the honorific ``Mrs.'' means to women and their identity.}

America had its first introduction to the artist and designer Ray Eames
in 1956, and it was brief. NBC's ``Home'' show was featuring the debut
of the now-famous Eames lounge chair, and the host, Arlene Francis,
opened with Charles Eames --- who was ``almost a household word,'' she
told the audience, thanks to a series of earlier chairs made from molded
plastic and plywood.

But the name, like the chair, wasn't his alone. ``Almost always when
there's a successful man, there's a very interesting and able woman
behind him,'' Francis said, before calling ``Mrs. Eames'' to the stage
to say a few halting words.

Francis asked the couple to explain how the Eames design process worked.
``Ray, shall we let Charles do it, or do you want to help?'' she asked,
barely pausing. ``No? See, as I told you, she's behind the man, but
terribly important.'' In a clip
\href{https://www.youtube.com/watch?v=z_X6RsN-HFw}{available on
YouTube}, as Charles starts talking and the broadcast cuts to a close-up
of an undulating plywood seat, you can see Ray's feet in a corner of the
screen, backing away and then disappearing.

It's not exactly right to say that Ray was sidelined, let alone
invisible, in the intensely masculine world of midcentury design. The
steady stream of imagery of the couple --- sometimes riding on a
motorcycle, sometimes frolicking in homemade animal masks, sometimes
posing in matched clothing --- was as much a part of their work as their
chairs, their dozens of experimental movies, or their famous house on
the Pacific Ocean, whose light-filled space filled with exquisitely
curated assemblages of colorful objects has become a beau ideal of
humanized modernism.

But
\href{https://www.nytimes3xbfgragh.onion/1988/08/23/obituaries/ray-eames-72-a-founder-of-firm-that-designed-the-eames-chair.html}{since
her death in 1988}, Ray Eames has been moving back to the center of the
picture. Or rather, the world has been discovering that she was there
all along --- not just as the beaming helpmeet with sparkling eyes and a
Dorothy-from-Oz dress sense, but as an artistic force in and of herself.

She was born Bernice Kaiser in 1911 in Sacramento, where her father for
a time ran a vaudeville theater. ``Ray'' was a childhood nickname, and
it stuck (though it would sometimes cause its own confusions: as
recently as 2006, an article in The New York Times Magazine credited the
classic film ``Powers of Ten'' to ``the Eames brothers'').

She moved to New York City in the 1930s, where she studied with the
painter Hans Hofmann and was active with American Abstract Artists, an
activist group that picketed galleries that refused to show
nonrepresentational art. In 1940, she enrolled at the Cranbrook Academy
of Art outside Detroit, where she met Charles, at the time the head of
the department of industrial design.

Image

The Eameses designing the Aluminum Group series of furniture in
1958.Credit...Eames Office LLC. All Rights Reserved.

Image

Charles and Ray in the house they designed and built in
1949.Credit...Eames Office LLC. All Rights Reserved.

He was tall and angular, a charismatic former track star and football
captain whose looks would later draw comparison to Henry Fonda. He was
also married, with a young child. After a semester, Ray left, but they
exchanged smitten letters.

``We must see each other soon,'' he wrote to her in March 1941. ``This
business of becoming dream people in each other's minds is no good.''
Soon after, Charles divorced his wife, and he and Ray married and moved
to California.

Articles and documentaries about the Eameses lean on Charles's gnomic
statements of design philosophy: ``The best for the most for the
least.'' ``Never delegate understanding.'' ``Never let the blood show.''
And then there was this one: ``Whatever I can do, Ray can do better.''

It's the kind of tribute that can come off as lip service. But in the
decades since their deaths, design critics have increasingly challenged
the gendered division of labor --- architecture and form-giving for him;
color, whimsy and ``décor'' for her --- presumed in many accounts of
their work, and of modernism itself.

\includegraphics{https://static01.graylady3jvrrxbe.onion/images/2020/05/17/lens/00-mrs-eames-02/00-mrs-eames-02-articleLarge.jpg?quality=75\&auto=webp\&disable=upscale}

The birth of the famous molded Eames chairs might be traced to the
winning design Charles and Eero Saarinen submitted to a Museum of Modern
Art competition in 1940. But it wasn't until later that Charles and Ray,
working in their first Los Angeles apartment on a cobbled-together
machine nicknamed
\href{https://www.eamesoffice.com/blog/the-kazam-machine/}{the Kazam!},
began figuring out how to make the compound curves of their now classic
LCW chair work for mass production, with Ray mostly working the machine
while Charles earned a paycheck as a set designer at MGM Studios.

And Ray wasn't merely working out the kinks. The sculptural curves
themselves --- and the rest of the sprawling Eames output that followed
--- were influenced by Ray's early immersion in abstract art, the design
historian Pat Kirkham wrote in her 1995 book ``Charles and Ray Eames:
Designers of the Twentieth Century.''

``To the extent that the Eames style of the 1940s and 1950s was
`progressive' and `avant-garde,''' Kirkham wrote, ``Ray was certainly as
responsible as Charles.''

This was a point frequently underlined by Charles, who spoke about the
design process in terms of ``we,'' ``us'' and ``ours'' --- not that the
world seemed to really hear it. Until the 1970s, the titles of museum
exhibitions about their work tended to omit her name. In a 1973 profile
in The New York Times titled ``Casual Giant of Design,'' Charles Eames
describes their relationship as ``an equal and total alliance.'' But the
article is described in the paper's index as being about Charles and his
``wife and assistant.''

Image

The Eameses' entry in the card catalogue of The New York Times's photo
archive.

Image

A profile of Charles Eames, which appeared in The Times on April 24,
1973.

Image

Ray at home in California around 1950.Credit...Peter Stackpole/The LIFE
Picture Collection, via Getty Images

One 1969 television program described Ray as ``sitting like a delicious
dumpling in a doll's dress.'' And Ray, at least in front of the cameras,
seemed willing to play the part of supportive, mostly silent wife.

``It isn't the same when I talk,'' she
\href{https://www.washingtonpost.com/archive/lifestyle/1977/07/24/the-eames-legend-of-spare-simplicity/dabc414e-583b-406d-8e45-a1fd3a9828d0/}{told
an interviewer} in 1977. ``Charles has a way of putting things which
people pay attention to.''

A creative woman seemingly content to let her husband speak for her and
their work together may feel uneasy amid recent feminist reclamation
projects aimed at bringing other midcentury artistic women --- like
Ray's friend Lee Krasner, who was married to Jackson Pollock --- out of
their husbands' shadows (while sometimes also trying to topple the men
from the pedestal of male genius).

And Ray, who died 10 years to the day after her husband, left behind few
public statements about how she felt, say, in 1985 when she appeared at
the World Design Conference to accept his posthumous award as the
world's most influential designer, for work they did together. If she
ever seethed at her husband's higher profile or felt marginalized (as
some colleagues in the Eames studio have suggested), she did not show
it.

Kirkham, who interviewed her in the years before her death, says that
Ray wasn't concerned with teasing out what part of the creativity
symbolized by the name ``Eames'' was hers.

``She got annoyed when people would say, `Ray designed those stools,'
because it meant she wasn't involved in other stuff, when it was all
joint work,'' she said in an interview. ``She just kept coming back to
that. It was all joint work.''

Advertisement

\protect\hyperlink{after-bottom}{Continue reading the main story}

\hypertarget{site-index}{%
\subsection{Site Index}\label{site-index}}

\hypertarget{site-information-navigation}{%
\subsection{Site Information
Navigation}\label{site-information-navigation}}

\begin{itemize}
\tightlist
\item
  \href{https://help.nytimes3xbfgragh.onion/hc/en-us/articles/115014792127-Copyright-notice}{©~2020~The
  New York Times Company}
\end{itemize}

\begin{itemize}
\tightlist
\item
  \href{https://www.nytco.com/}{NYTCo}
\item
  \href{https://help.nytimes3xbfgragh.onion/hc/en-us/articles/115015385887-Contact-Us}{Contact
  Us}
\item
  \href{https://www.nytco.com/careers/}{Work with us}
\item
  \href{https://nytmediakit.com/}{Advertise}
\item
  \href{http://www.tbrandstudio.com/}{T Brand Studio}
\item
  \href{https://www.nytimes3xbfgragh.onion/privacy/cookie-policy\#how-do-i-manage-trackers}{Your
  Ad Choices}
\item
  \href{https://www.nytimes3xbfgragh.onion/privacy}{Privacy}
\item
  \href{https://help.nytimes3xbfgragh.onion/hc/en-us/articles/115014893428-Terms-of-service}{Terms
  of Service}
\item
  \href{https://help.nytimes3xbfgragh.onion/hc/en-us/articles/115014893968-Terms-of-sale}{Terms
  of Sale}
\item
  \href{https://spiderbites.nytimes3xbfgragh.onion}{Site Map}
\item
  \href{https://help.nytimes3xbfgragh.onion/hc/en-us}{Help}
\item
  \href{https://www.nytimes3xbfgragh.onion/subscription?campaignId=37WXW}{Subscriptions}
\end{itemize}
