Sections

SEARCH

\protect\hyperlink{site-content}{Skip to
content}\protect\hyperlink{site-index}{Skip to site index}

\href{https://www.nytimes3xbfgragh.onion/section/books/review}{Book
Review}

\href{https://myaccount.nytimes3xbfgragh.onion/auth/login?response_type=cookie\&client_id=vi}{}

\href{https://www.nytimes3xbfgragh.onion/section/todayspaper}{Today's
Paper}

\href{/section/books/review}{Book Review}\textbar{}Rape, Torture,
Murder, Beheadings --- All in a Day's Work for Marilyn Stasio

\begin{itemize}
\item
\item
\item
\item
\item
\end{itemize}

Advertisement

\protect\hyperlink{after-top}{Continue reading the main story}

Supported by

\protect\hyperlink{after-sponsor}{Continue reading the main story}

\href{/column/crime}{Crime}

\hypertarget{rape-torture-murder-beheadings--all-in-a-days-work-for-marilyn-stasio}{%
\section{Rape, Torture, Murder, Beheadings --- All in a Day's Work for
Marilyn
Stasio}\label{rape-torture-murder-beheadings--all-in-a-days-work-for-marilyn-stasio}}

\includegraphics{https://static01.graylady3jvrrxbe.onion/images/2020/05/17/books/review/17Crime/17Crime-articleLarge.jpg?quality=75\&auto=webp\&disable=upscale}

By Marilyn Stasio

\begin{itemize}
\item
  May 15, 2020
\item
  \begin{itemize}
  \item
  \item
  \item
  \item
  \item
  \end{itemize}
\end{itemize}

The Danish crime writer Jussi Adler-Olsen doesn't fool around. If he
wants to creep you out, he knows exactly how to do it. Take this passage
from his latest Department Q novel, \textbf{VICTIM 2117 (Dutton, 480
pp., \$28),} in which the villain torments a man named Assad by sending
him a message about his abducted wife, Marwa. ``It said that he made
sure Marwa lost our third child,'' Assad discloses, ``and that he raped
Marwa and my daughters every day, and that every time they gave birth,
he immediately killed the child.'' Has a certain sadistic ring to it,
doesn't it? That's how we know we are indeed reading Adler-Olsen, whose
style might best be described as literary brutalism.

Fans of this eccentric series of police procedurals might recognize
Assad as the sweet guy who keeps the Copenhagen Police Department
supplied with scrumptious snacks and robust Arab coffee. Here, we
discover his real name (Zaid al-Asadi), learn about his extensive
military background in Iraq and Afghanistan, and become acquainted with
his personal history. Aside from the fact that he doesn't resemble the
person we met in previous books, this Assad is an interesting character,
embroiled in an international terrorist plot.

A second story line involves a suicidal photojournalist named Joan who
finds a reason to live when he forges a bond with a dead woman, a
refugee who has washed up on the Cyprus shore. (``The clear, open gaze
of her eyes hit him. \emph{Why did this happen?} the eyes asked.
\ldots{} `I'll find out,' he said, and closed her eyes. `I promise.''')
Although less elaborately plotted than the central narrative, this
simpler story covers a similar theme about the plight of refugees and
does it with more heart.

Yet another plot thread deals with a computer gamer named Alexander, who
is coming up on his 2,117th win, at which point he intends to cut off a
bunch of human heads with a samurai sword. A total cuckoo, Alex is more
suited to the author's macabre gifts than either of those good guys,
Assad and Joan. While it's nice to see the tender side of Adler-Olsen, a
nut job with a samurai sword brings out his true talent.

♦

C. J. Box's straight-shooter hero, Joe Pickett, is back in Saddlestring,
Wyo., protecting his patch of wilderness from evil men and ugly beasts.
\textbf{LONG RANGE (Putnam, 368 pp., \$28)} finds the game warden called
home from the Teton wilderness, where he and a posse of other lawmen had
trekked to recover the body of an elk-hunting guide who had been mauled
by a grizzly bear. Seems that back in Twelve Sleep County, someone took
a long-range shot at a local judge, accidentally hitting his wife
instead.

The judge suspects a drive-by, but we know the sniper is a sharpshooter
whose expertise is described with impressive precision and
heart-gripping suspense. That's enough to make a suspect of Joe's good
friend Nate Romanowski, a skilled marksman and master falconer who named
his baby Kestrel. That's a red herring for sure (Nate always seems to be
a suspect in this series), but Box makes up for the gaffe with good
characters, an extra-good story and great scenes of life and death in
the wilderness.

♦

If I were a 14-year-old girl (the target audience of this Y.A. mystery),
I would feel cheated by Emily Arsenault's \textbf{ALL THE PRETTY THINGS
(Delacorte, 352 pp., paper, \$17.99).} The premise is so enticing: A
murder at a New Hampshire amusement park gets solved by the teenage
daughter of the park's owner. Ivy is a smart, resourceful girl, more of
a grown-up than her crude, rather stupid father, who bought Fabuland
without knowing a thing about how to run it. When Ethan Lavoie, who
worked in the maintenance department, fell to his death from a train
trestle, Ivy's best friend, Morgan, climbed to the top of the Ferris
wheel and wouldn't come down. That stunt put Morgan in the psych ward
and left Ivy to figure things out by herself. Ivy is one smart kid, and
Fabuland has some cool rides, like the Laser Coaster, which is painted
neon green and looks like ``an exotic snake zipping along the grass.''
But there isn't anything particularly mysterious in this mystery, except
why someone doesn't bump off Ivy's abominable father.

♦

Playing favorites? Who, me? When it comes to the short stories in Don
Winslow's collection, \textbf{BROKEN (Morrow, 352 pp., \$27.99),} I
plead guilty as charged. Actually, it's a tossup between the title
story, an operatically violent tale of brotherly revenge, and ``The San
Diego Zoo,'' a comic caper about a fugitive chimpanzee armed with a gun
and plenty of simian attitude. ``I know something's broken in me,''
admits Jimmy McNabb, a New Orleans street cop who goes off the rails
when his younger brother, Danny, is tortured and murdered. ``You're an
angry man, Jimmy,'' his mother tells him. ``You were an angry boy.
\ldots{} You hate for the sake of hating.'' After the violence in the
title story, the humor of ``The San Diego Zoo'' comes as a welcome
palate cleanser. Frankly, I never knew that Winslow, a writer noted for
the emotive intensity of his storytelling, had it in him. But one
priceless scene --- of a little guy in a gold lamé jockstrap and dog
collar being led around on a leash by a great big guy wearing a Superman
costume (``complete with cape'') --- pretty much did it for me.

Advertisement

\protect\hyperlink{after-bottom}{Continue reading the main story}

\hypertarget{site-index}{%
\subsection{Site Index}\label{site-index}}

\hypertarget{site-information-navigation}{%
\subsection{Site Information
Navigation}\label{site-information-navigation}}

\begin{itemize}
\tightlist
\item
  \href{https://help.nytimes3xbfgragh.onion/hc/en-us/articles/115014792127-Copyright-notice}{©~2020~The
  New York Times Company}
\end{itemize}

\begin{itemize}
\tightlist
\item
  \href{https://www.nytco.com/}{NYTCo}
\item
  \href{https://help.nytimes3xbfgragh.onion/hc/en-us/articles/115015385887-Contact-Us}{Contact
  Us}
\item
  \href{https://www.nytco.com/careers/}{Work with us}
\item
  \href{https://nytmediakit.com/}{Advertise}
\item
  \href{http://www.tbrandstudio.com/}{T Brand Studio}
\item
  \href{https://www.nytimes3xbfgragh.onion/privacy/cookie-policy\#how-do-i-manage-trackers}{Your
  Ad Choices}
\item
  \href{https://www.nytimes3xbfgragh.onion/privacy}{Privacy}
\item
  \href{https://help.nytimes3xbfgragh.onion/hc/en-us/articles/115014893428-Terms-of-service}{Terms
  of Service}
\item
  \href{https://help.nytimes3xbfgragh.onion/hc/en-us/articles/115014893968-Terms-of-sale}{Terms
  of Sale}
\item
  \href{https://spiderbites.nytimes3xbfgragh.onion}{Site Map}
\item
  \href{https://help.nytimes3xbfgragh.onion/hc/en-us}{Help}
\item
  \href{https://www.nytimes3xbfgragh.onion/subscription?campaignId=37WXW}{Subscriptions}
\end{itemize}
