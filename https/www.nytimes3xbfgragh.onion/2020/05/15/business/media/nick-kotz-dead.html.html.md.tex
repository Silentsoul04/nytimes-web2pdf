Sections

SEARCH

\protect\hyperlink{site-content}{Skip to
content}\protect\hyperlink{site-index}{Skip to site index}

\href{https://www.nytimes3xbfgragh.onion/section/business/media}{Media}

\href{https://myaccount.nytimes3xbfgragh.onion/auth/login?response_type=cookie\&client_id=vi}{}

\href{https://www.nytimes3xbfgragh.onion/section/todayspaper}{Today's
Paper}

\href{/section/business/media}{Media}\textbar{}Nick Kotz, Crusading
Journalist and Author, Dies at 87

\url{https://nyti.ms/3dRY6vR}

\begin{itemize}
\item
\item
\item
\item
\item
\end{itemize}

Advertisement

\protect\hyperlink{after-top}{Continue reading the main story}

Supported by

\protect\hyperlink{after-sponsor}{Continue reading the main story}

\hypertarget{nick-kotz-crusading-journalist-and-author-dies-at-87}{%
\section{Nick Kotz, Crusading Journalist and Author, Dies at
87}\label{nick-kotz-crusading-journalist-and-author-dies-at-87}}

He won a Pulitzer Prize for exposing unsafe conditions in meatpacking
plants. He also wrote about hunger in America and the politics of the
B-1 bomber.

\includegraphics{https://static01.graylady3jvrrxbe.onion/images/2020/05/17/obituaries/17kotz-obit1/merlin_172498053_88d3adf0-e724-476e-a27f-860391fa95bb-articleLarge.jpg?quality=75\&auto=webp\&disable=upscale}

\href{https://www.nytimes3xbfgragh.onion/by/sam-roberts}{\includegraphics{https://static01.graylady3jvrrxbe.onion/images/2018/02/20/multimedia/author-sam-roberts/author-sam-roberts-thumbLarge.jpg}}

By \href{https://www.nytimes3xbfgragh.onion/by/sam-roberts}{Sam Roberts}

\begin{itemize}
\item
  May 15, 2020
\item
  \begin{itemize}
  \item
  \item
  \item
  \item
  \item
  \end{itemize}
\end{itemize}

Nick Kotz, a Pulitzer Prize-winning journalist and author who exposed
health hazards in the nation's slaughterhouses, the gamut of hunger in
America and the politics behind the Pentagon's B-1 bomber, died on April
26 in Broad Run, Va. He was 87.

His wife, Mary Lynn Kotz, an author, said he died in an accident on his
cattle farm after he had mistakenly left his 2006 Mercedes in neutral as
he tried to retrieve a package from the back seat. The car struck him as
it rolled backward.

Mr. Kotz was a Washington correspondent for
\href{https://www.desmoinesregister.com/}{The Des Moines Register} and
its sister paper The Minneapolis Tribune when he wrote a series of
articles in the mid-1960s on the unsanitary and unsafe conditions in
meatpacking plants He found that many plants were not subject to federal
inspection because they were not engaged in interstate commerce.

The series brought him the Pulitzer Prize for national reporting in
1968. In their citation, the Pulitzer judges said that Mr. Kotz's
articles had ``helped insure the passage of the Federal Wholesome Meat
Act of 1967,'' which extended federal standards to all manufacturers.

\includegraphics{https://static01.graylady3jvrrxbe.onion/images/2020/05/20/obituaries/17kotz-obit3/merlin_172390785_7b845913-54b1-4f61-bf6d-09888ea7d6ef-articleLarge.jpg?quality=75\&auto=webp\&disable=upscale}

His series evoked
\href{https://www.crf-usa.org/bill-of-rights-in-action/bria-24-1-b-upton-sinclairs-the-jungle-muckraking-the-meat-packing-industry.html}{Upton
Sinclair's 1906 novel ``The Jungle,''} which dramatized horrific
conditions among immigrant workers in Chicago's stockyards and
abattoirs. When President Lyndon B. Johnson signed the legislation in
1967, he was joined at the White House by Mr. Kotz and Mr. Sinclair, who
was 89 at the time.
\href{https://www.nytimes3xbfgragh.onion/1968/11/26/archives/upton-sinclair-author-dead-crusader-for-social-justice-90-90-books.html}{(He
died the following year.)}

When the Pulitzer Prize was announced, the consumer advocate Ralph
Nader, who had collaborated in publicizing Mr. Kotz's findings, said Mr.
Kotz's articles were ``a classic performance of objectivity, timeliness,
stamina and thorough coverage'' that demonstrated ``how investigative
journalism can break through the elaborate obstructions to information
flow on the part of both government and industry.''

Mr. Kotz was a national investigative reporter for The Washington Post
from 1970 to 1973 covering civil rights and organized labor. He later
contributed to The New York Times Magazine and other publications.

His books include ``Let Them Eat Promises: The Politics of Hunger in
America'' (1971); ``A Passion for Equality: George A. Wiley and the
Movement'' (1977), which he wrote with his wife; ``Wild Blue Yonder:
Money, Politics, and the B-1 Bomber'' (1988); and ``Judgment Days:
Lyndon Baines Johnson, Martin Luther King Jr., and the Laws that Changed
America'' (2006).

Reviewing ``Let Them Eat Promises'' in The New York Times, the
\href{https://www.nytimes3xbfgragh.onion/1970/01/15/archives/books-of-the-times-seven-million-hungry-children.html}{critic
John Leonard} wrote that Mr. Kotz ``paints an appalling picture of
political persiflage, bureaucratic ineptitude and moral obtuseness.''

Image

In addition to the Pulitzer, Mr. Kotz won the Robert F. Kennedy Award
for Excellence in Journalism, the National Magazine Award for public
service and the Sigma Delta Chi Award for Washington correspondence. At
his death he was completing a memoir about his writing career.

Nick Kotz was born Nathan Kallison Lasser on Sept. 16, 1932, in San
Antonio to Benjamin and Tibe (Kallison) Lasser. His father handled
advertising for the family farm supply business. After his parents
divorced when he was an infant, he was brought up by his mother and
maternal grandparents. His mother later headed a real estate company.

In 1945, she married Dr. Jacob Kotz, and the family lived in Washington,
where Nick, as he was known, graduated from the private St. Albans
School.

After graduating from Dartmouth in 1955 with a degree in history and
international relations, he was awarded a James B. Reynolds Scholarship
to the London School of Economics. When a friend recommended that he
take a class in contemporary American literature there, he decided to
become a writer.

Mr. Kotz served as a lieutenant in the Marines in Japan before he was
hired as a reporter by The Des Moines Register in 1958. He had chosen
The Register from a list of midsize newspapers recommended by his
mentor,
\href{https://www.nytimes3xbfgragh.onion/1984/06/25/us/some-observations-from-a-ringside-seat.html}{D.B.
Hardeman}, an assistant to Sam Rayburn, the Texas Democrat who was
speaker of the House.

While working in Des Moines, he encountered a fellow journalist, Mary
Lynn Booth, at a party as she was preparing to leave for a magazine job
in New York. After they met, she decided to remain in Des Moines. They
married in 1960.

In addition to his wife, Mr. Kotz is survived by a son, Jack, and a
grandson.

In his latest book, ``The Harness Maker's Dream: Nathan Kallison and the
Rise of South Texas'' (2013), Mr. Kotz wrote about his grandfather, a
Jewish refugee who fled Ukraine in 1890 and built a ranch and the
largest farm supply business in the American Southwest.

``As a veteran journalist and historian, I only now have become fully
aware that the most important history of our country is not found in the
grand events of wars and presidencies,'' Mr. Kotz wrote, ``but rather in
the everyday lives of our citizens, how they worked hard to support
their families; how they coped with hardships, discrimination and human
tragedy; and how they contributed to their own communities and nation.''

Advertisement

\protect\hyperlink{after-bottom}{Continue reading the main story}

\hypertarget{site-index}{%
\subsection{Site Index}\label{site-index}}

\hypertarget{site-information-navigation}{%
\subsection{Site Information
Navigation}\label{site-information-navigation}}

\begin{itemize}
\tightlist
\item
  \href{https://help.nytimes3xbfgragh.onion/hc/en-us/articles/115014792127-Copyright-notice}{©~2020~The
  New York Times Company}
\end{itemize}

\begin{itemize}
\tightlist
\item
  \href{https://www.nytco.com/}{NYTCo}
\item
  \href{https://help.nytimes3xbfgragh.onion/hc/en-us/articles/115015385887-Contact-Us}{Contact
  Us}
\item
  \href{https://www.nytco.com/careers/}{Work with us}
\item
  \href{https://nytmediakit.com/}{Advertise}
\item
  \href{http://www.tbrandstudio.com/}{T Brand Studio}
\item
  \href{https://www.nytimes3xbfgragh.onion/privacy/cookie-policy\#how-do-i-manage-trackers}{Your
  Ad Choices}
\item
  \href{https://www.nytimes3xbfgragh.onion/privacy}{Privacy}
\item
  \href{https://help.nytimes3xbfgragh.onion/hc/en-us/articles/115014893428-Terms-of-service}{Terms
  of Service}
\item
  \href{https://help.nytimes3xbfgragh.onion/hc/en-us/articles/115014893968-Terms-of-sale}{Terms
  of Sale}
\item
  \href{https://spiderbites.nytimes3xbfgragh.onion}{Site Map}
\item
  \href{https://help.nytimes3xbfgragh.onion/hc/en-us}{Help}
\item
  \href{https://www.nytimes3xbfgragh.onion/subscription?campaignId=37WXW}{Subscriptions}
\end{itemize}
