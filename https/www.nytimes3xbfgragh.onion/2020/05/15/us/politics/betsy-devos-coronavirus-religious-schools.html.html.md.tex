Sections

SEARCH

\protect\hyperlink{site-content}{Skip to
content}\protect\hyperlink{site-index}{Skip to site index}

\href{https://www.nytimes3xbfgragh.onion/section/politics}{Politics}

\href{https://myaccount.nytimes3xbfgragh.onion/auth/login?response_type=cookie\&client_id=vi}{}

\href{https://www.nytimes3xbfgragh.onion/section/todayspaper}{Today's
Paper}

\href{/section/politics}{Politics}\textbar{}DeVos Funnels Coronavirus
Relief Funds to Favored Private and Religious Schools

\url{https://nyti.ms/363HmyI}

\begin{itemize}
\item
\item
\item
\item
\item
\item
\end{itemize}

\href{https://www.nytimes3xbfgragh.onion/news-event/coronavirus?action=click\&pgtype=Article\&state=default\&region=TOP_BANNER\&context=storylines_menu}{The
Coronavirus Outbreak}

\begin{itemize}
\tightlist
\item
  live\href{https://www.nytimes3xbfgragh.onion/2020/08/04/world/coronavirus-cases.html?action=click\&pgtype=Article\&state=default\&region=TOP_BANNER\&context=storylines_menu}{Latest
  Updates}
\item
  \href{https://www.nytimes3xbfgragh.onion/interactive/2020/us/coronavirus-us-cases.html?action=click\&pgtype=Article\&state=default\&region=TOP_BANNER\&context=storylines_menu}{Maps
  and Cases}
\item
  \href{https://www.nytimes3xbfgragh.onion/interactive/2020/science/coronavirus-vaccine-tracker.html?action=click\&pgtype=Article\&state=default\&region=TOP_BANNER\&context=storylines_menu}{Vaccine
  Tracker}
\item
  \href{https://www.nytimes3xbfgragh.onion/2020/08/02/us/covid-college-reopening.html?action=click\&pgtype=Article\&state=default\&region=TOP_BANNER\&context=storylines_menu}{College
  Reopening}
\item
  \href{https://www.nytimes3xbfgragh.onion/live/2020/08/04/business/stock-market-today-coronavirus?action=click\&pgtype=Article\&state=default\&region=TOP_BANNER\&context=storylines_menu}{Economy}
\end{itemize}

Advertisement

\protect\hyperlink{after-top}{Continue reading the main story}

Supported by

\protect\hyperlink{after-sponsor}{Continue reading the main story}

\hypertarget{devos-funnels-coronavirus-relief-funds-to-favored-private-and-religious-schools}{%
\section{DeVos Funnels Coronavirus Relief Funds to Favored Private and
Religious
Schools}\label{devos-funnels-coronavirus-relief-funds-to-favored-private-and-religious-schools}}

Education Secretary Betsy DeVos, using discretion written into the
coronavirus stabilization law, is using millions of dollars to pursue
long-sought policy goals that Congress has blocked.

\includegraphics{https://static01.graylady3jvrrxbe.onion/images/2020/05/12/us/politics/12dc-virus-devos/12dc-virus-devos-articleLarge-v2.jpg?quality=75\&auto=webp\&disable=upscale}

\href{https://nytimes3xbfgragh.onion/by/erica-l-green}{\includegraphics{https://static01.graylady3jvrrxbe.onion/images/2018/06/14/multimedia/author-erica-l-green/author-erica-l-green-thumbLarge-v2.png}}

By \href{https://nytimes3xbfgragh.onion/by/erica-l-green}{Erica L.
Green}

\begin{itemize}
\item
  May 15, 2020
\item
  \begin{itemize}
  \item
  \item
  \item
  \item
  \item
  \item
  \end{itemize}
\end{itemize}

WASHINGTON --- Education Secretary
\href{https://www.nytimes3xbfgragh.onion/2020/05/20/us/politics/trump-betsy-devos-student-debt-forgiveness.html}{Betsy
DeVos} is using the \$2 trillion coronavirus stabilization law to throw
a lifeline to education sectors she has long championed, directing
millions of federal dollars intended primarily for public schools and
colleges to private and religious schools.

\href{https://www.nytimes3xbfgragh.onion/2020/03/27/us/politics/coronavirus-house-voting.html}{The
Coronavirus Aid, Relief and Economic Security Act}, signed in late
March, included \$30 billion for education institutions turned upside
down by the pandemic shutdowns, about \$14 billion for higher education,
\$13.5 billion to elementary and secondary schools, and the rest for
state governments.

Ms. DeVos has used \$180 million of those dollars to encourage states to
create ``microgrants'' that parents of elementary and secondary school
students can use to pay for educational services, including private
school tuition. She has directed school districts to share millions of
dollars designated for low-income students with wealthy private schools.

And she has nearly depleted the 2.5 percent of higher education funding,
about \$350 million, set aside for struggling colleges to bolster small
colleges --- many of them private, religious or on the margins of higher
education --- regardless of need. The Wright Graduate University for the
Realization of Human Potential, a private college in Wisconsin that has
a \href{http://www.wrightinstitutecult.com}{website debunking claims
that it is a cult}, was allocated about \$495,000. All of the colleges
could apply for the funds or reject them, and Wright officials said the
school did not claim the funds.

Bergin University of Canine Studies in California said its \$472,850
allocation was a ``godsend.''

``I think we are one of the most important educational institutions out
there right now,'' said its founder, Bonnie Bergin, who is credited with
inventing the service dog.

On the Senate floor this week, Senator Chuck Schumer, Democrat of New
York and the minority leader, accused Ms. DeVos of ``exploiting
congressional relief efforts.'' He said she had been ``using a portion
of that funding not to help states or localities cope with the crisis,
but to augment her push for voucherlike programs, a prior initiative
that has nothing to do with Covid-19.''

House Democrats included language in
\href{https://docs.house.gov/billsthisweek/20200511/BILLS-116hr6800ih.pdf}{a
stimulus bill set for a vote on Friday} that would limit Ms. DeVos's
ability to use about \$58 billion in additional education relief for
K-12 school districts for private schools. Congress has largely rejected
Ms. DeVos's proposals to create programs that resemble private school
vouchers, and public education groups say Ms. DeVos is abusing
discretion granted to her under the emergency legislation to achieve a
long-held agenda.

``And it only took a pandemic,'' said Sasha Pudelski, the advocacy
director at the AASA, the School Superintendents Association.

The Education Department called the accusation ``absurd.'' But in a
statement, the department said that every student and teacher had been
affected by the pandemic. ``The current disruption to our education
system has reaffirmed what Secretary DeVos has been saying for years: We
need to rethink education for all students, of every age, no matter the
type of school setting,'' it said.

Ms. DeVos has long held that taxpayer funds should be available for
private school tuition, giving parents the chance to escape failing
public schools and public education competition to drive improvement.

\hypertarget{latest-updates-global-coronavirus-outbreak}{%
\section{\texorpdfstring{\href{https://www.nytimes3xbfgragh.onion/2020/08/04/world/coronavirus-cases.html?action=click\&pgtype=Article\&state=default\&region=MAIN_CONTENT_1\&context=storylines_live_updates}{Latest
Updates: Global Coronavirus
Outbreak}}{Latest Updates: Global Coronavirus Outbreak}}\label{latest-updates-global-coronavirus-outbreak}}

Updated 2020-08-04T21:34:02.738Z

\begin{itemize}
\tightlist
\item
  \href{https://www.nytimes3xbfgragh.onion/2020/08/04/world/coronavirus-cases.html?action=click\&pgtype=Article\&state=default\&region=MAIN_CONTENT_1\&context=storylines_live_updates\#link-2daa96b5}{As
  talks drag on, McConnell signals openness to jobless aid extension
  that Republicans have opposed.}
\item
  \href{https://www.nytimes3xbfgragh.onion/2020/08/04/world/coronavirus-cases.html?action=click\&pgtype=Article\&state=default\&region=MAIN_CONTENT_1\&context=storylines_live_updates\#link-1228a480}{Novavax
  sees encouraging results from two studies of its experimental
  vaccine.}
\item
  \href{https://www.nytimes3xbfgragh.onion/2020/08/04/world/coronavirus-cases.html?action=click\&pgtype=Article\&state=default\&region=MAIN_CONTENT_1\&context=storylines_live_updates\#link-4825b93}{Public
  and private schools in Maryland and elsewhere are divided over
  in-person instruction.}
\end{itemize}

\href{https://www.nytimes3xbfgragh.onion/2020/08/04/world/coronavirus-cases.html?action=click\&pgtype=Article\&state=default\&region=MAIN_CONTENT_1\&context=storylines_live_updates}{See
more updates}

More live coverage:
\href{https://www.nytimes3xbfgragh.onion/live/2020/08/04/business/stock-market-today-coronavirus?action=click\&pgtype=Article\&state=default\&region=MAIN_CONTENT_1\&context=storylines_live_updates}{Markets}

A spokesman for Republican members of the House Education Committee
defended Ms. DeVos's actions: ``While there are likely multiple ways the
secretary could have interpreted this broadly written law, the language
the appropriators wrote gave her the flexibility to implement it as she
has done.''

The most contentious move is
\href{https://www.chalkbeat.org/2020/5/5/21248179/equitable-services-coronavirus-private-schools}{guidance
that directs school districts to increase the share of dollars} they
spend on students in private schools. Under federal education law,
school districts are required to use funding they receive for their
poorest students to provide ``equitable services,'' such as tutoring and
transportation for low-income students attending private schools in
their districts. But the department said districts should use their
emergency funding, which was doled out based on student poverty rates,
to support all students attending private schools in their districts,
regardless of income.

Her guidance comes as elementary and secondary education groups lobby
Congress for billions of additional dollars to lift students out of the
educational crisis caused by the pandemic. In big cities, which serve
the most vulnerable students, district leaders are projecting
\href{https://www.politico.com/states/new-york/albany/story/2020/05/12/carranza-city-is-facing-most-horrific-budget-this-school-system-has-ever-seen-1283528}{budget
shortfalls of up to 25 percent} because of collapsing tax revenues, said
the Council of the Great City Schools, which represents 76 of the
nation's large urban districts. Its member districts said they could be
forced to lay off 275,000 teachers.

In New York City, Chancellor Richard A. Carranza
\href{https://ny.chalkbeat.org/2020/5/12/21256562/nyc-education-department-state-budget-cuts}{told
City Council members on Tuesday} that the school district was facing
``the most horrific budget'' it had ever seen.

The federal Education Department said if school districts were to count
only poor students, ``they would be placing nonpublic school students
and teachers at a disadvantage that Congress did not intend.''

``It's sad, but unsurprising, that some would put their own financial
interests ahead of the needs of all students and teachers,'' the
department said.

Educators are pleading with the department to revise or rescind the
guidance. In Montana, school officials estimate that compliance would
shift more than \$1.5 million to private and home schools, up from about
\$206,469 that the schools are due under current law. In Louisiana,
private schools would receive at least 267 percent more funding, and at
least 77 percent of the relief allocation for Orleans Parish would be
redirected, according to
\href{https://ccsso.org/sites/default/files/2020-05/DeVosESLetter050520.pdf}{a
letter state that education chiefs} sent to Ms. DeVos. The Newark Public
Schools in New Jersey
\href{https://edlawcenter.org/assets/files/pdfs/School\%20Funding/ELC_Letter_to_Governor_Murphy_re.pdf}{would
lose \$800,000 in federal relief funds} to private schools, David G.
Sciarra, the executive director of the Education Law Center, said in a
letter to the governor of New Jersey asking him to reject the guidance.

Pennsylvania's education secretary, Pedro A. Rivera, protested to the
department that under the guidance, 53 percent more money would flow
``from most disadvantaged to more advantaged students'' in urban
districts like Philadelphia, while rural districts like Northeast
Bradford would see a 932 percent increase.

``School districts can --- and should --- ignore this guidance, which
flouts what Congress intended to do with the CARES Act: support students
who need it the most,'' said Randi Weingarten, the president of the
American Federation of Teachers, and Daniel A. Domenech, the executive
director of AASA.

Indiana has announced it would not enforce the guidance.
\href{https://www.doe.in.gov/sites/default/files/grants/final-language-equitable-share-cares-51220.pdf}{In
a memo}, its superintendent of public instruction, Jennifer McCormick, a
Republican, said the state ``ensures that the funds are distributed
according to congressional intent and a plain reading of the law.''

``I will not play political agenda games with COVID relief funds,'' she
said on Twitter.

Private school educators say that they have always been included in
emergency relief funding, including for Hurricanes Katrina and Sandy,
and this situation should be no different.

\href{https://www.nytimes3xbfgragh.onion/news-event/coronavirus?action=click\&pgtype=Article\&state=default\&region=MAIN_CONTENT_3\&context=storylines_faq}{}

\hypertarget{the-coronavirus-outbreak-}{%
\subsubsection{The Coronavirus Outbreak
›}\label{the-coronavirus-outbreak-}}

\hypertarget{frequently-asked-questions}{%
\paragraph{Frequently Asked
Questions}\label{frequently-asked-questions}}

Updated August 4, 2020

\begin{itemize}
\item ~
  \hypertarget{i-have-antibodies-am-i-now-immune}{%
  \paragraph{I have antibodies. Am I now
  immune?}\label{i-have-antibodies-am-i-now-immune}}

  \begin{itemize}
  \tightlist
  \item
    As of right
    now,\href{https://www.nytimes3xbfgragh.onion/2020/07/22/health/covid-antibodies-herd-immunity.html?action=click\&pgtype=Article\&state=default\&region=MAIN_CONTENT_3\&context=storylines_faq}{that
    seems likely, for at least several months.} There have been
    frightening accounts of people suffering what seems to be a second
    bout of Covid-19. But experts say these patients may have a
    drawn-out course of infection, with the virus taking a slow toll
    weeks to months after initial exposure. People infected with the
    coronavirus typically
    \href{https://www.nature.com/articles/s41586-020-2456-9}{produce}
    immune molecules called antibodies, which are
    \href{https://www.nytimes3xbfgragh.onion/2020/05/07/health/coronavirus-antibody-prevalence.html?action=click\&pgtype=Article\&state=default\&region=MAIN_CONTENT_3\&context=storylines_faq}{protective
    proteins made in response to an
    infection}\href{https://www.nytimes3xbfgragh.onion/2020/05/07/health/coronavirus-antibody-prevalence.html?action=click\&pgtype=Article\&state=default\&region=MAIN_CONTENT_3\&context=storylines_faq}{.
    These antibodies may} last in the body
    \href{https://www.nature.com/articles/s41591-020-0965-6}{only two to
    three months}, which may seem worrisome, but that's perfectly normal
    after an acute infection subsides, said Dr. Michael Mina, an
    immunologist at Harvard University. It may be possible to get the
    coronavirus again, but it's highly unlikely that it would be
    possible in a short window of time from initial infection or make
    people sicker the second time.
  \end{itemize}
\item ~
  \hypertarget{im-a-small-business-owner-can-i-get-relief}{%
  \paragraph{I'm a small-business owner. Can I get
  relief?}\label{im-a-small-business-owner-can-i-get-relief}}

  \begin{itemize}
  \tightlist
  \item
    The
    \href{https://www.nytimes3xbfgragh.onion/article/small-business-loans-stimulus-grants-freelancers-coronavirus.html?action=click\&pgtype=Article\&state=default\&region=MAIN_CONTENT_3\&context=storylines_faq}{stimulus
    bills enacted in March} offer help for the millions of American
    small businesses. Those eligible for aid are businesses and
    nonprofit organizations with fewer than 500 workers, including sole
    proprietorships, independent contractors and freelancers. Some
    larger companies in some industries are also eligible. The help
    being offered, which is being managed by the Small Business
    Administration, includes the Paycheck Protection Program and the
    Economic Injury Disaster Loan program. But lots of folks have
    \href{https://www.nytimes3xbfgragh.onion/interactive/2020/05/07/business/small-business-loans-coronavirus.html?action=click\&pgtype=Article\&state=default\&region=MAIN_CONTENT_3\&context=storylines_faq}{not
    yet seen payouts.} Even those who have received help are confused:
    The rules are draconian, and some are stuck sitting on
    \href{https://www.nytimes3xbfgragh.onion/2020/05/02/business/economy/loans-coronavirus-small-business.html?action=click\&pgtype=Article\&state=default\&region=MAIN_CONTENT_3\&context=storylines_faq}{money
    they don't know how to use.} Many small-business owners are getting
    less than they expected or
    \href{https://www.nytimes3xbfgragh.onion/2020/06/10/business/Small-business-loans-ppp.html?action=click\&pgtype=Article\&state=default\&region=MAIN_CONTENT_3\&context=storylines_faq}{not
    hearing anything at all.}
  \end{itemize}
\item ~
  \hypertarget{what-are-my-rights-if-i-am-worried-about-going-back-to-work}{%
  \paragraph{What are my rights if I am worried about going back to
  work?}\label{what-are-my-rights-if-i-am-worried-about-going-back-to-work}}

  \begin{itemize}
  \tightlist
  \item
    Employers have to provide
    \href{https://www.osha.gov/SLTC/covid-19/standards.html}{a safe
    workplace} with policies that protect everyone equally.
    \href{https://www.nytimes3xbfgragh.onion/article/coronavirus-money-unemployment.html?action=click\&pgtype=Article\&state=default\&region=MAIN_CONTENT_3\&context=storylines_faq}{And
    if one of your co-workers tests positive for the coronavirus, the
    C.D.C.} has said that
    \href{https://www.cdc.gov/coronavirus/2019-ncov/community/guidance-business-response.html}{employers
    should tell their employees} -\/- without giving you the sick
    employee's name -\/- that they may have been exposed to the virus.
  \end{itemize}
\item ~
  \hypertarget{should-i-refinance-my-mortgage}{%
  \paragraph{Should I refinance my
  mortgage?}\label{should-i-refinance-my-mortgage}}

  \begin{itemize}
  \tightlist
  \item
    \href{https://www.nytimes3xbfgragh.onion/article/coronavirus-money-unemployment.html?action=click\&pgtype=Article\&state=default\&region=MAIN_CONTENT_3\&context=storylines_faq}{It
    could be a good idea,} because mortgage rates have
    \href{https://www.nytimes3xbfgragh.onion/2020/07/16/business/mortgage-rates-below-3-percent.html?action=click\&pgtype=Article\&state=default\&region=MAIN_CONTENT_3\&context=storylines_faq}{never
    been lower.} Refinancing requests have pushed mortgage applications
    to some of the highest levels since 2008, so be prepared to get in
    line. But defaults are also up, so if you're thinking about buying a
    home, be aware that some lenders have tightened their standards.
  \end{itemize}
\item ~
  \hypertarget{what-is-school-going-to-look-like-in-september}{%
  \paragraph{What is school going to look like in
  September?}\label{what-is-school-going-to-look-like-in-september}}

  \begin{itemize}
  \tightlist
  \item
    It is unlikely that many schools will return to a normal schedule
    this fall, requiring the grind of
    \href{https://www.nytimes3xbfgragh.onion/2020/06/05/us/coronavirus-education-lost-learning.html?action=click\&pgtype=Article\&state=default\&region=MAIN_CONTENT_3\&context=storylines_faq}{online
    learning},
    \href{https://www.nytimes3xbfgragh.onion/2020/05/29/us/coronavirus-child-care-centers.html?action=click\&pgtype=Article\&state=default\&region=MAIN_CONTENT_3\&context=storylines_faq}{makeshift
    child care} and
    \href{https://www.nytimes3xbfgragh.onion/2020/06/03/business/economy/coronavirus-working-women.html?action=click\&pgtype=Article\&state=default\&region=MAIN_CONTENT_3\&context=storylines_faq}{stunted
    workdays} to continue. California's two largest public school
    districts --- Los Angeles and San Diego --- said on July 13, that
    \href{https://www.nytimes3xbfgragh.onion/2020/07/13/us/lausd-san-diego-school-reopening.html?action=click\&pgtype=Article\&state=default\&region=MAIN_CONTENT_3\&context=storylines_faq}{instruction
    will be remote-only in the fall}, citing concerns that surging
    coronavirus infections in their areas pose too dire a risk for
    students and teachers. Together, the two districts enroll some
    825,000 students. They are the largest in the country so far to
    abandon plans for even a partial physical return to classrooms when
    they reopen in August. For other districts, the solution won't be an
    all-or-nothing approach.
    \href{https://bioethics.jhu.edu/research-and-outreach/projects/eschool-initiative/school-policy-tracker/}{Many
    systems}, including the nation's largest, New York City, are
    devising
    \href{https://www.nytimes3xbfgragh.onion/2020/06/26/us/coronavirus-schools-reopen-fall.html?action=click\&pgtype=Article\&state=default\&region=MAIN_CONTENT_3\&context=storylines_faq}{hybrid
    plans} that involve spending some days in classrooms and other days
    online. There's no national policy on this yet, so check with your
    municipal school system regularly to see what is happening in your
    community.
  \end{itemize}
\end{itemize}

Sister Dale McDonald, the director of public policy and educational
research at the National Catholic Educational Association, said many of
its schools would need to be cleaned, and their staffing would need to
be shored up. At least 100 member schools are at risk for not reopening
at all.

``In an emergency, kids shouldn't have to prove they're poor to get what
they need to continue their education,'' Sister McDonald said.

A competition announced by Ms. DeVos in which states can vie for tens of
millions of dollars either to create statewide virtual schools or offer
``microgrants'' is also drawing fire for mirroring voucher programs that
help parents pay for services outside the public school system. The
program also stands to
\href{https://www.the74million.org/article/betsy-devos-trumps-edsec-pick-promoted-virtual-schools-despite-dismal-results/}{benefit
virtual education companies} that Ms. DeVos has personally invested in.

Representative Robert C. Scott of Virginia, the chairman of the House
education committee, said the competition's point system was weighted in
favor of rural areas and voucher-friendly states, rather than those most
affected by the coronavirus.

``This program design is indistinguishable from a standard voucher
scheme and is the latest attempt by this department to promote
privatization initiatives against both the wishes of the American
people, and the intent of Congress,'' he wrote to Ms. DeVos.

The microgrant program has been cheered by champions of school choice.

``They are smart to take advantage of the lag and lack of disciplined
delivery of education,'' said Jeanne Allen, the chief executive of the
Center for Education Reform. ``We don't have any choice but to make
parents and families the unit of education right now.''

Trish Stevens, who has a special-needs daughter, said a program in
Arizona that was much like the microgrant proposal had been ``life
changing'' for her child, who is supposed to have \$150-an-hour speech
therapy and \$250-an-hour tutors.

``It's like the Wild West of education right now,'' she said, ``and
we're all just trying to figure it out.''

Ms. DeVos is also under fire from college educators
\href{https://www.insidehighered.com/news/2020/05/07/small-colleges-get-millions-while-other-colleges-struggle\#.XrQZ_cKvTlA.twitter}{for
disbursing millions}of dollars to hundreds of small colleges that may
not need it. The coronavirus relief law set aside about \$350 million
for schools that demonstrated ``significant unmet needs related to
expenses associated with coronavirus.'' The department was supposed to
prioritize schools that did not receive at least \$500,000 from other
categories of higher education funding. Instead, Ms. DeVos used the
money to ensure that small schools received \$500,000 each.

That meant outsize per-pupil allocations at
\href{https://www2.ed.gov/about/offices/list/ope/allocationsfipse.pdf}{several
private schools and religious institutions} with as few as 50 students
while some public community colleges received as little as \$500 a
student.

Ben Miller, the vice president for postsecondary education at the
liberal Center for American Progress, said the allocations came as large
public colleges were ``rationing,'' and community colleges ``starve.''

Aaron D. Profitt, the vice president for academic affairs at God's Bible
School and College in Ohio, said the school did not plan to claim its
allocation because it was getting by on small donations. Ms. DeVos had
criticized elite colleges that received stimulus funding they did not
apply for and had urged schools to reject money they did not need.

``Of course, when you get a letter from the Department of Education
giving you money, you start thinking about all the good things you can
do,'' Mr. Profitt said. ``But when I read the CARES Act, the intention
was not to do all the good things you could do but try to meet needs. We
are trying to cooperate with the law as written.''

Advertisement

\protect\hyperlink{after-bottom}{Continue reading the main story}

\hypertarget{site-index}{%
\subsection{Site Index}\label{site-index}}

\hypertarget{site-information-navigation}{%
\subsection{Site Information
Navigation}\label{site-information-navigation}}

\begin{itemize}
\tightlist
\item
  \href{https://help.nytimes3xbfgragh.onion/hc/en-us/articles/115014792127-Copyright-notice}{©~2020~The
  New York Times Company}
\end{itemize}

\begin{itemize}
\tightlist
\item
  \href{https://www.nytco.com/}{NYTCo}
\item
  \href{https://help.nytimes3xbfgragh.onion/hc/en-us/articles/115015385887-Contact-Us}{Contact
  Us}
\item
  \href{https://www.nytco.com/careers/}{Work with us}
\item
  \href{https://nytmediakit.com/}{Advertise}
\item
  \href{http://www.tbrandstudio.com/}{T Brand Studio}
\item
  \href{https://www.nytimes3xbfgragh.onion/privacy/cookie-policy\#how-do-i-manage-trackers}{Your
  Ad Choices}
\item
  \href{https://www.nytimes3xbfgragh.onion/privacy}{Privacy}
\item
  \href{https://help.nytimes3xbfgragh.onion/hc/en-us/articles/115014893428-Terms-of-service}{Terms
  of Service}
\item
  \href{https://help.nytimes3xbfgragh.onion/hc/en-us/articles/115014893968-Terms-of-sale}{Terms
  of Sale}
\item
  \href{https://spiderbites.nytimes3xbfgragh.onion}{Site Map}
\item
  \href{https://help.nytimes3xbfgragh.onion/hc/en-us}{Help}
\item
  \href{https://www.nytimes3xbfgragh.onion/subscription?campaignId=37WXW}{Subscriptions}
\end{itemize}
