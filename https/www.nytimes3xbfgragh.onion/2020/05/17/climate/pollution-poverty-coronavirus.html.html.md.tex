Sections

SEARCH

\protect\hyperlink{site-content}{Skip to
content}\protect\hyperlink{site-index}{Skip to site index}

\href{/section/climate}{Climate}\textbar{}In the Shadows of America's
Smokestacks, Virus Is One More Deadly Risk

\url{https://nyti.ms/2WBIOW0}

\begin{itemize}
\item
\item
\item
\item
\item
\end{itemize}

\hypertarget{climate-and-environment}{%
\subsubsection{\texorpdfstring{\href{https://www.nytimes3xbfgragh.onion/section/climate?name=styln-climate\&region=TOP_BANNER\&variant=undefined\&block=storyline_menu_recirc\&action=click\&pgtype=Article\&impression_id=01570e60-e39b-11ea-9f3a-f5bd08dc668f}{Climate
and
Environment}}{Climate and Environment}}\label{climate-and-environment}}

\begin{itemize}
\tightlist
\item
  \href{https://www.nytimes3xbfgragh.onion/2020/08/17/climate/alaska-oil-drilling-anwr.html?name=styln-climate\&region=TOP_BANNER\&variant=undefined\&block=storyline_menu_recirc\&action=click\&pgtype=Article\&impression_id=01570e61-e39b-11ea-9f3a-f5bd08dc668f}{Arctic
  Refuge}
\item
  \href{https://www.nytimes3xbfgragh.onion/interactive/2020/climate/trump-environment-rollbacks.html?name=styln-climate\&region=TOP_BANNER\&variant=undefined\&block=storyline_menu_recirc\&action=click\&pgtype=Article\&impression_id=01570e62-e39b-11ea-9f3a-f5bd08dc668f}{Trump's
  Changes}
\item
  \href{https://www.nytimes3xbfgragh.onion/interactive/2020/04/19/climate/climate-crash-course-1.html?name=styln-climate\&region=TOP_BANNER\&variant=undefined\&block=storyline_menu_recirc\&action=click\&pgtype=Article\&impression_id=01573570-e39b-11ea-9f3a-f5bd08dc668f}{Climate
  101}
\item
  \href{https://www.nytimes3xbfgragh.onion/interactive/2018/08/30/climate/how-much-hotter-is-your-hometown.html?name=styln-climate\&region=TOP_BANNER\&variant=undefined\&block=storyline_menu_recirc\&action=click\&pgtype=Article\&impression_id=01573571-e39b-11ea-9f3a-f5bd08dc668f}{Is
  Your Hometown Hotter?}
\end{itemize}

\includegraphics{https://static01.graylady3jvrrxbe.onion/images/2020/05/17/climate/00CLI-VIRUS-POLLUTION-main/00CLI-VIRUS-POLLUTION1-articleLarge.jpg?quality=75\&auto=webp\&disable=upscale}

\hypertarget{in-the-shadows-of-americas-smokestacks-virus-is-one-more-deadly-risk}{%
\section{In the Shadows of America's Smokestacks, Virus Is One More
Deadly
Risk}\label{in-the-shadows-of-americas-smokestacks-virus-is-one-more-deadly-risk}}

Factories along the Rouge River in southwest Detroit.Credit...Emily Rose
Bennett for The New York Times

Supported by

\protect\hyperlink{after-sponsor}{Continue reading the main story}

\href{https://www.nytimes3xbfgragh.onion/by/hiroko-tabuchi}{\includegraphics{https://static01.graylady3jvrrxbe.onion/images/2018/02/20/multimedia/author-hiroko-tabuchi/author-hiroko-tabuchi-thumbLarge.jpg}}

By \href{https://www.nytimes3xbfgragh.onion/by/hiroko-tabuchi}{Hiroko
Tabuchi}

\begin{itemize}
\item
  Published May 17, 2020Updated May 19, 2020
\item
  \begin{itemize}
  \item
  \item
  \item
  \item
  \item
  \end{itemize}
\end{itemize}

This isn't the first time Vicki Dobbins's town has been forced to
shelter in place.

Last year, the Marathon Petroleum refinery that looms over her
neighborhood near Detroit emitted a pungent gas, causing nausea and
dizziness among neighbors and prompting health officials to warn people
to stay inside. When a stay-at-home advisory returned in March, this
time for the coronavirus, ``it was just devastating,'' Ms. Dobbins said.

Ms. Dobbins, who is 76, later contracted Covid-19, and spent two weeks
on oxygen in intensive care. Now she has a question. ``Do the polluters
in our area make us more susceptible to asthma, bronchitis, heart
failure, cancers?'' she asked. ``Is the virus just going to be one of
the ones added to that list?''

Nationwide, low-income communities of color like hers, River Rouge,
Mich., are exposed to significantly higher levels of pollution,
\href{https://ajph.aphapublications.org/doi/abs/10.2105/AJPH.2017.304297?journalCode=ajph}{studies
have found}, and also see higher levels of lung disease and other
ailments. Now, scientists are racing to understand if long-term exposure
to air pollution plays a role in the coronavirus crisis, particularly
since minorities are disproportionately dying.

The science is preliminary --- the virus, being so new, remains poorly
understood --- though researchers are finding reason to look closely.
People with two conditions tied to air pollution,
\href{https://www.resmedjournal.com/article/S0954-6111(20)30081-0/fulltext}{inflammatory
lung disease} and
\href{https://www.medrxiv.org/content/10.1101/2020.03.25.20043133v1}{coronary
heart disease}, face a higher risk for severe Covid-19, preliminary
research has shown. Last month, work
\href{https://www.nytimes3xbfgragh.onion/2020/04/07/climate/air-pollution-coronavirus-covid.html}{by
Harvard specialists} found that coronavirus patients in areas with
historically heavy air pollution are more likely to die than patients
elsewhere.

And while it's impossible to say with certainty that any one person was
made more vulnerable to the virus because of pollution, earlier studies
of other respiratory diseases
\href{https://www.ncbi.nlm.nih.gov/pubmed/14629774}{have established
that} long-term exposure to air pollution
\href{https://www.ncbi.nlm.nih.gov/pubmed/28966029}{increases the risk
of those illness}.

``The system has allowed, basically, low-income people and people of
color to have to breathe the pollution,'' said Dr. Abdul El-Sayed, an
epidemiologist and Detroit's former health director.

The tensions are playing out in minority communities across the country
that live with industrial air pollution and the health risks that come
with it. A neighborhood in Houston, Texas, for instance, that is home
not only to factories making plastics materials used in medical masks,
but also incinerators that burn medical waste. A community outside San
Francisco near the state's largest refinery but far from most hospitals.

And the county where Ms. Dobbins lives, which has seen more Covid-19
deaths than almost any other outside of New York state.

Michigan:

\hypertarget{a-national-hot-spot}{%
\subsection{A National Hot Spot}\label{a-national-hot-spot}}

\includegraphics{https://static01.graylady3jvrrxbe.onion/images/2020/05/17/climate/00CLI-VIRUS-POLLUTION-jump4/merlin_171906717_1a74bd6b-a852-4692-b1c7-75a667198755-articleLarge.jpg?quality=75\&auto=webp\&disable=upscale}

Ms. Dobbins lives in one of Michigan's most polluted ZIP codes.

There's a refinery, two power stations, a steel mill and a sewage
treatment plant within a five-mile radius. The area's levels of ozone, a
gas that has been linked to lung disease and other ailments, frequently
exceed federal limits.

Her county has seen 2,192 deaths so far, putting it in the same league
as much larger Cook County, Ill., which is home to Chicago and 2,589
recorded deaths. In Michigan, African-Americans have accounted for more
than 40 percent of deaths, even though they make up only 15 percent of
the population.

A substitute teacher, Ms. Dobbins had already struggled to breathe since
developing asthma after moving back to the neighborhood 20 years ago to
care for her mother. She got used to carrying around her inhaler and
also sheltering in place, as she did a year ago during the warning about
the Marathon refinery.

Back then, the Detroit Health Department advised people to stay indoors
with the windows closed, saying that the odors could cause ``symptoms
such as nausea, vomiting, headaches, or difficulty breathing'' among
people sensitive to the smells. A flare failure had released hydrogen
sulfide, sulfur dioxide and other compounds,
\href{https://www.michigan.gov/mienvironment/0,9349,7-385-90161-504479--,00.html}{Marathon
told regulators.} In a statement, a company spokesman, Jamal T. Kheiry,
said the plant operator ``did not detect any emission levels of
concern'' in the episode.

Image

Vicki Dobbins has a question: Have decades of industrial air pollution
in her area made people more susceptible to Covid-19?Credit...Emily Rose
Bennett for The New York Times

Image

The area around River Rouge, on the southwest edge of Detroit, is among
the state's most polluted.Credit...Emily Rose Bennett for The New York
Times

Today, amid the pandemic, Marathon has urged state regulators to suspend
environmental monitoring rules, partly so its staff will not have to
work and risk infection. On April 2, Timothy J. Peterkoski, Marathon's
environmental director, wrote to regulators that some ``sampling,
testing, record-keeping and reporting activities may need to be
deferred.''

When the virus lockdown came in March, the entire state sheltered at
home, including Ms. Dobbins. Only then did she discover that she had
already caught the coronavirus, most likely at a birthday party a few
weeks earlier in Detroit, where she danced the night away with friends,
she said.

\href{https://www.nytimes3xbfgragh.onion/section/climate?action=click\&pgtype=Article\&state=default\&region=MAIN_CONTENT_1\&context=storylines_keepup}{}

\hypertarget{climate-and-environment-}{%
\subsubsection{Climate and Environment
›}\label{climate-and-environment-}}

\hypertarget{keep-up-on-the-latest-climate-news}{%
\paragraph{Keep Up on the Latest Climate
News}\label{keep-up-on-the-latest-climate-news}}

Updated Aug. 18, 2020

Here's what you need to know this week:

\begin{itemize}
\item
  \begin{itemize}
  \tightlist
  \item
    Five automakers
    \href{https://www.nytimes3xbfgragh.onion/2020/08/17/climate/california-automakers-pollution.html?action=click\&pgtype=Article\&state=default\&region=MAIN_CONTENT_1\&context=storylines_keepup}{sealed
    a binding agreement} with California to follow the state's stricter
    tailpipe emissions rules.
  \item
    The Trump
    administration\href{https://www.nytimes3xbfgragh.onion/2020/08/13/climate/trump-methane.html?action=click\&pgtype=Article\&state=default\&region=MAIN_CONTENT_1\&context=storylines_keepup}{eliminated
    a major methane rule}, even as leaks are worsening, in a decision
    that researchers warned ignored science.
  \item
    Climate change leaders said
    \href{https://www.nytimes3xbfgragh.onion/2020/08/12/climate/kamala-harris-environmental-justice.html?action=click\&pgtype=Article\&state=default\&region=MAIN_CONTENT_1\&context=storylines_keepup}{the
    vice-presidential choice of Kamala Harris} signaled that Democrats
    will have a focus on environmental justice.
  \end{itemize}
\end{itemize}

She was hospitalized with pneumonia, then transferred to intensive care.
Then, her kidneys failed. She was among the lucky and today is
recovering at home. Nevertheless, Ms. Dobbins now struggles to take the
20 steps to and from her bathroom. ``I'm so winded I can't breathe,''
she said, pausing to catch her breath while speaking on the phone.

And she struggled to describe the industrial smells that plague her
neighborhood: ``I smell it every day. I live it every day.''

Texas

\hypertarget{making-plastic-burning-plastic}{%
\subsection{Making Plastic, Burning
Plastic}\label{making-plastic-burning-plastic}}

Image

The Port Arthur skyline.Credit...Brandon Thibodeaux for The New York
Times

The neighborhoods surrounding the Houston Ship Channel, a bustling
petrochemical hub of refineries and oil tankers, produce the raw
materials vital to some of the most highly sought-after products in the
nation right now: masks, plastic gowns and other medical equipment.

And when that gear is discarded, residents fear that some of it is
coming back to be incinerated in the
\href{https://www.tceq.texas.gov/assets/public/permitting/waste/msw/medical-waste-facilities-active.pdf}{five
medical waste facilities} in and around Houston, Port Arthur and
surrounding counties.

It is the kind of one-two industrial punch that has contributed to air
pollution for decades around the neighborhoods' sizable African-American
and Hispanic populations. The American Lung Association
\href{http://www.stateoftheair.org/city-rankings/msas/houston-the-woodlands-tx.html\#ozone}{ranks
Houston among the nation's most polluted} cities.

Today, Harris County, which includes metropolitan Houston, has reported
more than 9,000 coronavirus cases. Minority groups have accounted for
about two-thirds of early Covid-19 deaths in the city, despite making up
only 22 percent of the population.

``Hospitals need the masks, the gloves,'' said Yvette Arellano, a
community organizer in Houston's polluted neighborhoods. But the irony,
she said, is that communities like this ``are breathing in the toxins
that industry says is necessary for the safety of other people.''

Image

Yvette Arellano on the bank of Buffalo Bayou in industrial east
Houston.Credit...Brandon Thibodeaux for The New York Times

Despite the economic shutdown, petrochemical companies around Houston
have kept operating because they are essential for the production of
masks and protective equipment.
\href{https://www.newschool.edu/pressroom/pressreleases/2019/incinerators.htm}{Research
has shown that} most waste incinerators in the United States are in
lower-income communities of color, and medical waste, when burned, can
release dioxins and other compounds.

Denae W. King, an expert in environmental health at Texas Southern
University, said more research was needed to pin down if and precisely
how air pollution might make communities more vulnerable. But
particulate matter, which can lodge deep in the lungs and cause
inflammation, adds risk, she said. ``If your lungs have already been
exposed, you already have underlying issues related to inflammation, and
then you're diagnosed with Covid-19, that just exacerbates the problems
that already exist.''

Ms. Arellano says she suspects, but can't be sure, that her own mother
caught coronavirus. She had the dry cough, a headache and muscle pain.
But health officials said her mother wouldn't qualify for a test without
proof that she had run a sustained fever, a tricky ask for Americans
with no family physician.

Her mother never got tested. But because of her cough, she hasn't been
able to keep working as a grocery cashier.

California

\hypertarget{pollution-nearby-hospitals-far-away}{%
\subsection{Pollution Nearby, Hospitals Far
Away}\label{pollution-nearby-hospitals-far-away}}

Image

Vandee Lakthanasuk, left, with her father, Siengther Lakthanasuk, at
home in Richmond, Calif.Credit...Preston Gannaway for The New York Times

Siengther Lakthanasuk fought for the Americans against communist forces
for 15 years in Indochina, and waited another 16 years at a refugee camp
in Thailand before landing in 1991 in Richmond, Calif., in Contra Costa
County, just a few minutes from a Chevron refinery that is the state's
largest polluter.

Mr. Lakthanasuk's neighborhood, a community of people hailing from his
native Laos, is also affected by other industrial pollution, including
coal trains headed to port.

While Richmond doesn't lack for industrial infrastructure, it does fall
short in health care options. The only public hospital serving the city
of 110,000 shut its doors in 2015.

Hunkered down at home, Mr. Lakthanasuk worries what might happen if his
daughters bring the virus into the house from their work. ``When I was
in the war, we could hear the enemy, we could hear the guns shooting,
and we could protect ourselves,'' Mr. Lakthanasuk said, speaking through
a translator. But ``you cannot see the coronavirus.''

Contra Costa County has recorded 1,089 coronavirus cases, and its
fatality rate has climbed to nearly 3 percent, almost twice that of
wealthier San Francisco a short drive away. That disparity underscores
regional inequalities even as California has been praised for its early
virus intervention --- including the nation's first shelter-in-place
orders in six counties, including Contra Costa.

Many local families are like Mr. Lakthanasuk's, intergenerational
households with service industry jobs that are either risky or have
disappeared. Mr. Lakthanasuk's wife lost her job at a nearby casino
shuttered by the pandemic. His two adult daughters work at nearby
grocery stores, essential workers both to the community and for their
income.

Image

Siengther Lakthanasuk at home in Richmond, Calif.Credit...Preston
Gannaway for The New York Times

Image

The Chevron refinery in Richmond.Credit...Preston Gannaway for The New
York Times

Image

Members of Mr. Lakthanasuk's family cleaned their shoes after coming
home from work at a grocery store.Credit...Preston Gannaway for The New
York Times

John Gioa, who serves on the county's board of supervisors and the
state's Air Resources Board, said a new medical station in Richmond with
250 beds, housed in a former Ford Motor factory, would bring much-needed
care. And testing had been greatly expanded, he said, with at least
three locations in the city.

``I'm still concerned about the future,'' he said. ``Lower-income
communities and those impacted by air pollution are at greater risk, and
we need to be prepared.''

In 2012, a fire at the Chevron facility sent more than 10,000 people to
seek treatment for respiratory difficulties, including Mr. Lakthanasuk
and his family. ``We were packed in the hospital,'' Mr. Lakthanasuk
said, recalling the fire. ``But that's not the only time. There's been a
lot of incidents.''

Last year at the refinery, episodes of flaring, the intentional burning
of hydrogen, sent black smoke across the neighborhood, prompting an
investigation by air-quality officials. Chevron said the flaring was
related to the start-up of a more efficient hydrogen processing unit,
now complete.

A spokesman for Chevron, Braden Reddall, said the company monitored 10
chemical compounds at three fence-line locations at the refinery and
tracked additional compounds at three other locations in the community.
As of **** mid-May, none of those readings exceed health limits.

Mr. Lakthanasuk's health concerns and financial strains are now starting
to reach well beyond his boxy, red-and-yellow home in Richmond. His
extended family back in Laos is now feeling the effects, too.

They are rice farmers, he said, and their crop last year was hit hard by
drought, which local officials
\href{https://en.vietnamplus.vn/lao-ministry-blames-recent-droughts-on-climate-change/165553.vnp}{have
linked to climate change}. Despite fears for his own safety, Mr.
Lakthanasuk recently ventured out to wire his nephew an emergency
payment of \$300 so he could buy food.

In his many years spent fighting communists in Indochina, Mr.
Lakthanasuk said, he is proud of having never been captured. The virus,
though, has given him a small taste. ``Right now, home is worse than
being a prisoner,'' he said.

Advertisement

\protect\hyperlink{after-bottom}{Continue reading the main story}

\hypertarget{site-index}{%
\subsection{Site Index}\label{site-index}}

\hypertarget{site-information-navigation}{%
\subsection{Site Information
Navigation}\label{site-information-navigation}}

\begin{itemize}
\tightlist
\item
  \href{https://help.nytimes3xbfgragh.onion/hc/en-us/articles/115014792127-Copyright-notice}{©~2020~The
  New York Times Company}
\end{itemize}

\begin{itemize}
\tightlist
\item
  \href{https://www.nytco.com/}{NYTCo}
\item
  \href{https://help.nytimes3xbfgragh.onion/hc/en-us/articles/115015385887-Contact-Us}{Contact
  Us}
\item
  \href{https://www.nytco.com/careers/}{Work with us}
\item
  \href{https://nytmediakit.com/}{Advertise}
\item
  \href{http://www.tbrandstudio.com/}{T Brand Studio}
\item
  \href{https://www.nytimes3xbfgragh.onion/privacy/cookie-policy\#how-do-i-manage-trackers}{Your
  Ad Choices}
\item
  \href{https://www.nytimes3xbfgragh.onion/privacy}{Privacy}
\item
  \href{https://help.nytimes3xbfgragh.onion/hc/en-us/articles/115014893428-Terms-of-service}{Terms
  of Service}
\item
  \href{https://help.nytimes3xbfgragh.onion/hc/en-us/articles/115014893968-Terms-of-sale}{Terms
  of Sale}
\item
  \href{https://spiderbites.nytimes3xbfgragh.onion}{Site Map}
\item
  \href{https://help.nytimes3xbfgragh.onion/hc/en-us}{Help}
\item
  \href{https://www.nytimes3xbfgragh.onion/subscription?campaignId=37WXW}{Subscriptions}
\end{itemize}
