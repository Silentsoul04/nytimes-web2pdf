Sections

SEARCH

\protect\hyperlink{site-content}{Skip to
content}\protect\hyperlink{site-index}{Skip to site index}

\href{https://myaccount.nytimes3xbfgragh.onion/auth/login?response_type=cookie\&client_id=vi}{}

\href{https://www.nytimes3xbfgragh.onion/section/todayspaper}{Today's
Paper}

\href{/section/opinion}{Opinion}\textbar{}America's Killer Lawns

\url{https://nyti.ms/2Th9Ujk}

\begin{itemize}
\item
\item
\item
\item
\item
\end{itemize}

Advertisement

\protect\hyperlink{after-top}{Continue reading the main story}

\href{/section/opinion}{Opinion}

Supported by

\protect\hyperlink{after-sponsor}{Continue reading the main story}

\hypertarget{americas-killer-lawns}{%
\section{America's Killer Lawns}\label{americas-killer-lawns}}

Homeowners use up 10 times more pesticide per acre than farmers do. But
we can change what we do in our own yards.

\href{https://www.nytimes3xbfgragh.onion/by/margaret-renkl}{\includegraphics{https://static01.graylady3jvrrxbe.onion/images/2017/04/08/opinion/margaret-renkl/margaret-renkl-thumbLarge-v2.png}}

By \href{https://www.nytimes3xbfgragh.onion/by/margaret-renkl}{Margaret
Renkl}

Contributing Opinion Writer

\begin{itemize}
\item
  May 18, 2020
\item
  \begin{itemize}
  \item
  \item
  \item
  \item
  \item
  \end{itemize}
\end{itemize}

\includegraphics{https://static01.graylady3jvrrxbe.onion/images/2020/05/18/opinion/18reenklWeb/merlin_172588689_d56fee67-435b-44bf-91a8-254ba28fab2d-articleLarge.jpg?quality=75\&auto=webp\&disable=upscale}

NASHVILLE --- One day last fall, deep in the middle of a devastating
drought, I was walking the dog when a van bearing the logo of a
mosquito-control company blew past me and parked in front of a
neighbor's house. The whole vehicle stank of chemicals, even going 40
miles an hour.

The man who emerged from the truck donned a massive backpack carrying a
tank full of insecticide and proceeded to spray every bush and plant in
the yard. Then he got in his truck, drove two doors down, and sprayed
that yard, too, before continuing his route all around the block.

Here's the most heartbreaking thing about the whole episode: He was
spraying for mosquitoes that didn't even exist:
\href{https://www.climate.gov/news-features/event-tracker/flash-drought-engulfs-us-southeast-september-2019}{Last
year's extreme drought} ended mosquito-breeding season long before the
first freeze. Nevertheless, the mosquito vans arrived every three weeks,
right on schedule, drenching the yards with poison for no reason but the
schedule itself.

And spraying for mosquitoes isn't the half of it, as any walk through
the lawn-care department of a big-box store will attest. People want the
outdoors to work like an extension of their homes --- fashionable, tidy,
predictable. Above all, comfortable. So weedy yards filled with tiny
wildflowers get bulldozed end to end and replaced with sod cared for by
homeowners spraying from a bottle marked ``backyard bug control'' or by
lawn services that leave behind tiny signs warning, ``Lawn care
application; keep off the grass.''

If only songbirds could read.

Most people don't seem to know that in this context ``application'' and
``control'' are simply euphemisms for ``poison.'' A friend once
mentioned to me that she'd love to put up a nest box for bluebirds, and
I offered to help her choose a good box and a safe spot for it in her
yard, explaining that she would also need to tell her yard service to
stop spraying. ``I had no idea those guys were spraying,'' she said.

To enjoy a lush green lawn or to sit on your patio without being eaten
alive by mosquitoes doesn't seem like too much to ask unless you
actually \emph{know} that insecticides designed to kill mosquitoes will
also kill every other kind of insect: earthworms and caterpillars,
spiders and mites, honeybees and butterflies, native bees and lightning
bugs. Unless you actually \emph{know} that herbicides also kill insects
when they ingest the poisoned plants.

The global insect die-off is so precipitous that, if the trend
continues,
\href{https://www.plantbasednews.org/news/insects-facing-extinction-within-100-years}{there
will be no insects left a hundred years from now}. That's a problem for
more than the bugs themselves: Insects are responsible for pollinating
roughly 75 percent of all flowering plants, including
\href{https://www.nationalgeographic.com/animals/2019/02/why-insect-populations-are-plummeting-and-why-it-matters/}{one-third
of the human world's food supply}.

They form the basis of much of the animal world's food supply, as well.
When we poison the bugs and the weeds, we are also
\href{https://www.instagram.com/p/CAEKJq8BNPr/}{poisoning the turtles}
and tree frogs, the bats and screech owls, the songbirds and skinks.

``If insect species losses cannot be halted, this will have catastrophic
consequences for both the planet's ecosystems and for the survival of
mankind,'' Francisco Sánchez-Bayo of the University of Sydney,
Australia,
\href{https://www.theguardian.com/environment/2019/feb/10/plummeting-insect-numbers-threaten-collapse-of-nature}{told
The Guardian last year}.

Lawn chemicals are not, by themselves, the cause of
\href{https://www.nytimes3xbfgragh.onion/2018/11/27/magazine/insect-apocalypse.html}{the
insect apocalypse}, of course. Heat waves can
\href{https://www.theguardian.com/environment/2018/nov/13/heatwaves-wipe-out-male-insect-fertility-beetles-study}{render
male insects sterile}; loss of habitat can cause precipitous population
declines; agricultural pesticides kill land insects and, by way of
runoff into the nation's waterways, aquatic insects, as well.

As individuals,
\href{https://www.nytimes3xbfgragh.onion/2020/01/13/opinion/earth-environmentalism.html?searchResultPosition=1}{we
can help to slow such trends}, but we don't have the power to reverse
them. Changing the way we think about our own yards is the only thing we
have complete control over. And since homeowners use
\href{https://www.fws.gov/dpps/visualmedia/printingandpublishing/publications/2003_HomeownersGuidetoProtectingFrogs.pdf}{up
10 times more pesticide per acre} than farmers do, changing the way we
think about our yards can make a huge difference to our fellow
creatures.

It can make a huge difference to our own health, too: As the Garden Club
of America notes in its \href{http://tghyp.com/468-revision-v1/}{Great
Healthy Yard Project}, synthetic pesticides are endocrine disrupters
linked to an array of human health problems, including autism, A.D.H.D.,
diabetes and cancer. So many people have invested so completely in the
chemical control of the outdoors that every subdivision in this country
might as well be declared a Superfund site.

Changing our relationship to our yards is simple: Just don't spray. Let
the tiny wildflowers take root within the grass. Use an oscillating fan
to keep the mosquitoes away. Tug the weeds out of the flower bed with
your own hands and feel
\href{https://www.medicalnewstoday.com/articles/66840\#_blank}{the
benefit of a natural antidepressant} at the same time. Trust the natural
world to perform its own insect control, and watch the songbirds and the
tree frogs and the box turtles and the friendly garter snakes return to
their homes among us.

Because butterflies and bluebirds don't respect property lines, our best
hope is to make this simple change a community effort. For 25 years, my
husband and I have been trying to create a wildlife sanctuary of this
half-acre lot, planting native flowers for the bees and the butterflies,
\href{https://www.nytimes3xbfgragh.onion/2018/02/10/opinion/sunday/let-your-winter-garden-go-wild.html}{leaving
the garden messy} as a safe place for overwintering insects.

Despite our best efforts, our yard is being visibly changed anyway.
Fewer birds. Fewer insects. Fewer everything. Half an acre, it turns
out, is not enough to sustain wildlife unless the other half-acre lots
are nature-friendly, too.

It's spring now, and nearly every day I get a flier in the mail
advertising a yard service or a mosquito-control company. I will never
poison this yard, but I save the fliers anyway, as a reminder of what
we're up against. I keep them next to an eastern swallowtail butterfly
that my 91-year-old father-in-law found dead on the sidewalk. He saved
it for me because he knows how many flowers I've planted over the years
to feed the pollinators.

I keep that poor dead butterfly, even though it breaks my heart, because
I know what it cost my father-in-law to bring it to me. How he had to
lock the brakes on his walker, hold onto one of the handles and stoop on
arthritic knees to get to the ground. How gently he had to pick up the
butterfly to keep from crumbling its wings into powder. How carefully he
set it in the basket of the walker to protect it.

My father-in-law didn't know that the time for protection had passed.
The butterfly he found is perfect, unbattered by age or struggle. It was
healthy and strong until someone sprayed for mosquitoes, or weeds, and
killed it, too.

Margaret Renkl is a contributing opinion writer who covers flora, fauna,
politics and culture in the American South. She is the author of
``\href{https://milkweed.org/book/late-migrations}{Late Migrations: A
Natural History of Love and Loss}.''

\emph{The Times is committed to publishing}
\href{https://www.nytimes3xbfgragh.onion/2019/01/31/opinion/letters/letters-to-editor-new-york-times-women.html}{\emph{a
diversity of letters}} \emph{to the editor. We'd like to hear what you
think about this or any of our articles. Here are some}
\href{https://help.nytimes3xbfgragh.onion/hc/en-us/articles/115014925288-How-to-submit-a-letter-to-the-editor}{\emph{tips}}\emph{.
And here's our email:}
\href{mailto:letters@NYTimes.com}{\emph{letters@NYTimes.com}}\emph{.}

\emph{Follow The New York Times Opinion section on}
\href{https://www.facebookcorewwwi.onion/nytopinion}{\emph{Facebook}}\emph{,}
\href{http://twitter.com/NYTOpinion}{\emph{Twitter (@NYTopinion)}}
\emph{and}
\href{https://www.instagram.com/nytopinion/}{\emph{Instagram}}\emph{.}

Advertisement

\protect\hyperlink{after-bottom}{Continue reading the main story}

\hypertarget{site-index}{%
\subsection{Site Index}\label{site-index}}

\hypertarget{site-information-navigation}{%
\subsection{Site Information
Navigation}\label{site-information-navigation}}

\begin{itemize}
\tightlist
\item
  \href{https://help.nytimes3xbfgragh.onion/hc/en-us/articles/115014792127-Copyright-notice}{©~2020~The
  New York Times Company}
\end{itemize}

\begin{itemize}
\tightlist
\item
  \href{https://www.nytco.com/}{NYTCo}
\item
  \href{https://help.nytimes3xbfgragh.onion/hc/en-us/articles/115015385887-Contact-Us}{Contact
  Us}
\item
  \href{https://www.nytco.com/careers/}{Work with us}
\item
  \href{https://nytmediakit.com/}{Advertise}
\item
  \href{http://www.tbrandstudio.com/}{T Brand Studio}
\item
  \href{https://www.nytimes3xbfgragh.onion/privacy/cookie-policy\#how-do-i-manage-trackers}{Your
  Ad Choices}
\item
  \href{https://www.nytimes3xbfgragh.onion/privacy}{Privacy}
\item
  \href{https://help.nytimes3xbfgragh.onion/hc/en-us/articles/115014893428-Terms-of-service}{Terms
  of Service}
\item
  \href{https://help.nytimes3xbfgragh.onion/hc/en-us/articles/115014893968-Terms-of-sale}{Terms
  of Sale}
\item
  \href{https://spiderbites.nytimes3xbfgragh.onion}{Site Map}
\item
  \href{https://help.nytimes3xbfgragh.onion/hc/en-us}{Help}
\item
  \href{https://www.nytimes3xbfgragh.onion/subscription?campaignId=37WXW}{Subscriptions}
\end{itemize}
