Sections

SEARCH

\protect\hyperlink{site-content}{Skip to
content}\protect\hyperlink{site-index}{Skip to site index}

\href{https://www.nytimes3xbfgragh.onion/section/us}{U.S.}

\href{https://myaccount.nytimes3xbfgragh.onion/auth/login?response_type=cookie\&client_id=vi}{}

\href{https://www.nytimes3xbfgragh.onion/section/todayspaper}{Today's
Paper}

\href{/section/us}{U.S.}\textbar{}Freed by Court Ruling, Republicans
Step Up Effort to Patrol Voting

\url{https://nyti.ms/2zJmX6j}

\begin{itemize}
\item
\item
\item
\item
\item
\item
\end{itemize}

\begin{itemize}
\item
  \href{https://www.nytimes3xbfgragh.onion/interactive/2020/08/04/us/elections/results-arizona-kansas-michigan-missouri-primaries.html?action=click\&pgtype=Article\&state=default\&region=TOP_BANNER\&context=storylines_menu}{Latest
  Results}
\item
  \href{https://www.nytimes3xbfgragh.onion/article/biden-vice-president-2020.html?action=click\&pgtype=Article\&state=default\&region=TOP_BANNER\&context=storylines_menu}{Biden's
  V.P. Search}
\item
  \href{https://www.nytimes3xbfgragh.onion/interactive/2020/07/24/us/politics/trump-biden-campaign-donors.html?action=click\&pgtype=Article\&state=default\&region=TOP_BANNER\&context=storylines_menu}{Map
  of Donations}
\item
  \href{https://www.nytimes3xbfgragh.onion/interactive/2020/us/elections/delegate-count-primary-results.html?action=click\&pgtype=Article\&state=default\&region=TOP_BANNER\&context=storylines_menu}{Delegate
  Count}
\item
  \href{https://www.nytimes3xbfgragh.onion/interactive/2019/us/politics/2020-presidential-candidates.html?action=click\&pgtype=Article\&state=default\&region=TOP_BANNER\&context=storylines_menu}{The
  Candidates}
\item
  \href{https://www.nytimes3xbfgragh.onion/newsletters/politics?action=click\&pgtype=Article\&state=default\&region=TOP_BANNER\&context=storylines_menu}{Politics
  Newsletter}
\end{itemize}

Advertisement

\protect\hyperlink{after-top}{Continue reading the main story}

Supported by

\protect\hyperlink{after-sponsor}{Continue reading the main story}

\hypertarget{freed-by-court-ruling-republicans-step-up-effort-to-patrol-voting}{%
\section{Freed by Court Ruling, Republicans Step Up Effort to Patrol
Voting}\label{freed-by-court-ruling-republicans-step-up-effort-to-patrol-voting}}

Officials seek to recruit 50,000 poll watchers and spend millions to
fight voter fraud. Democrats say the real goal is to stop them from
voting.

\includegraphics{https://static01.graylady3jvrrxbe.onion/images/2020/05/08/us/00VOTING-ca/merlin_169921596_763afaf1-bd90-47b6-9956-cb82367fbed7-articleLarge.jpg?quality=75\&auto=webp\&disable=upscale}

By \href{https://www.nytimes3xbfgragh.onion/by/michael-wines}{Michael
Wines}

\begin{itemize}
\item
  May 18, 2020
\item
  \begin{itemize}
  \item
  \item
  \item
  \item
  \item
  \item
  \end{itemize}
\end{itemize}

WASHINGTON --- Six months before a presidential election in which
turnout could matter more than persuasion, the Republican Party, the
Trump campaign and conservative activists are mounting an aggressive
national effort to shape who gets to vote in November --- and whose
ballots are counted.

Its premise is that a Republican victory in November is imperiled by
widespread voter fraud, a baseless charge embraced by President Trump
\href{https://www.brennancenter.org/sites/default/files/analysis/Briefing_Memo_Debunking_Voter_Fraud_Myth.pdf}{but
repeatedly debunked by research.} Democrats and voting rights advocates
say the driving factor is politics, not fraud --- especially since
\href{https://www.nytimes3xbfgragh.onion/2020/06/27/us/politics/trump-biden-protests-polling.html}{Mr.
Trump's} narrow win in 2016 underscored the potentially crucial value of
depressing turnout by Democrats, particularly minorities.

The Republican program, which has gained steam in recent weeks,
envisions recruiting up to 50,000 volunteers in 15 key states to monitor
polling places and challenge ballots and voters deemed suspicious. That
is part of a \$20 million plan that also allots millions to challenge
lawsuits by Democrats and
\href{https://www.nytimes3xbfgragh.onion/2020/07/18/us/supreme-court-voting-rights.html}{voting-rights}
advocates seeking to loosen state restrictions on balloting. The party
and its allies also intend to use advertising,
\href{https://protectthevote.com/?utm_medium=email\&utm_source=pu_48\&utm_campaign=20200508_123842_\&utm_content=\&_ga=2.58845101.124784289.1588967538-1349174632.1588967538}{the
internet} and Mr. Trump's command of the airwaves to cast Democrats as
agents of election theft.

\emph{{[}Read more about}
\href{https://www.nytimes3xbfgragh.onion/article/mail-in-vote-fraud-ballot.html}{\emph{fake
ballots, mail-in voting and voter fraud}}\emph{.{]}}

The efforts are bolstered by a 2018 federal court ruling that for the
first time in nearly four decades allows the national Republican Party
to mount campaigns against purported voter fraud without court approval.
The court ban on Republican Party voter-fraud operations
\href{https://www.theatlantic.com/politics/archive/2018/01/the-gop-just-received-another-tool-for-suppressing-votes/550052/}{was
imposed in 1982, and then modified in 1986 and again in 1990,} each time
after courts found instances of Republicans intimidating or working to
exclude minority voters in the name of preventing fraud. The party was
found to have violated it yet again in 2004.

The 2018 ruling merely ``allows the R.N.C. to play by the same rules as
Democrats,'' a spokeswoman for the Republican National Committee, Mandi
Merritt, said in a statement.

``Now the R.N.C. can work more closely with state parties and campaigns
to do what we do best --- ensure that more people vote through our
unmatched field program,'' the statement said.

Democrats will deploy their own army of poll watchers, seeking both to
maximize Democratic turnout and contest Republican practices they
believe improperly challenge or deter voters. One allied group seeking
to counter the Republican effort, Fair Fight, plans to have its own
representatives in the same swing states Republicans have targeted.

The Republican program escalates a focus on limiting who can vote that
became a juggernaut after the Supreme Court dismantled the Voting Rights
Act in 2013. It also reflects an enduring tension in American life in
which the voting rights of minorities --- whether granted in 1870 by the
15th Amendment or nearly a century later by the Voting Rights Act of
1965 --- seldom seem free from challenge.

Besides the national party and Mr. Trump's campaign strategists,
conservative advocacy groups are joining lawsuits, recruiting poll
monitors and mounting media campaigns of their own. Leading them is a
new and well-funded organization, the
\href{https://www.honestelections.org/}{Honest Elections Project},
formed by
\href{https://www.washingtonpost.com/graphics/2019/investigations/leonard-leo-federalists-society-courts/}{Leonard
Leo}, a prolific fund-raiser, advocate of a conservative judiciary and
confidant of Mr. Trump.

Republicans will have an Election Day operations program ``that probably
no other presidential campaign has had before,'' Josh Helton, a
Republican consultant, said
\href{https://cpac.conservative.org/protecting-the-ballot-box-defeating-the-lefts-voter-fraud-machine/}{at
a meeting} of the Conservative Political Action Committee in March.
``It's going to be all hands on deck.''

\hypertarget{latest-updates-2020-election}{%
\section{\texorpdfstring{\href{https://www.nytimes3xbfgragh.onion/2020/08/04/us/elections/primary-election-michigan-arizona-kansas.html?action=click\&pgtype=Article\&state=default\&region=MAIN_CONTENT_1\&context=storylines_live_updates}{Latest
Updates: 2020
Election}}{Latest Updates: 2020 Election}}\label{latest-updates-2020-election}}

Updated 2020-08-05T03:23:56.561Z

\begin{itemize}
\tightlist
\item
  \href{https://www.nytimes3xbfgragh.onion/2020/08/04/us/elections/primary-election-michigan-arizona-kansas.html?action=click\&pgtype=Article\&state=default\&region=MAIN_CONTENT_1\&context=storylines_live_updates\#link-3924dd44}{Two
  G.O.P. Senate primaries offer --- what else? --- a test of loyalty to
  Trump.}
\item
  \href{https://www.nytimes3xbfgragh.onion/2020/08/04/us/elections/primary-election-michigan-arizona-kansas.html?action=click\&pgtype=Article\&state=default\&region=MAIN_CONTENT_1\&context=storylines_live_updates\#link-62a8e06b}{The
  military-style uniforms of federal agents who responded to the unrest
  in Portland will be replaced.}
\item
  \href{https://www.nytimes3xbfgragh.onion/2020/08/04/us/elections/primary-election-michigan-arizona-kansas.html?action=click\&pgtype=Article\&state=default\&region=MAIN_CONTENT_1\&context=storylines_live_updates\#link-32b39e33}{President
  Trump is suddenly a big supporter of mail-in voting --- in Florida.}
\end{itemize}

\href{https://www.nytimes3xbfgragh.onion/2020/08/04/us/elections/primary-election-michigan-arizona-kansas.html?action=click\&pgtype=Article\&state=default\&region=MAIN_CONTENT_1\&context=storylines_live_updates}{See
more updates}

In battleground states, that extends even to comparatively quiet places
like Fond du Lac County, an eastern Wisconsin outpost of about 100,000
people and 1,200 farms midway between Green Bay and Milwaukee.

``I think the big push is going to be for poll observers'' in November's
general election, the Republican Party county chairman, Rohn Bishop,
said this month. ``No harm in making sure.'' Indeed, he said that
training sessions for election monitors were already in the works.

Democrats who have been tracking the effort say the goal is not to limit
fraud, but to make the supposed threat of election theft the tentpole of
a coordinated campaign by Republicans and their allies to limit the
number of Democratic ballots counted in November.

``This is a burn-it-down strategy, a strategy to win at all costs,''
said Lauren Groh-Wargo, the senior adviser at Fair Fight, the voting
rights group founded by Stacey Abrams, the former Democratic candidate
for governor of Georgia. ``They see this as central to victory.''

\href{https://fairfight.com/fair-fight-2020/}{Fair Fight} claims that
the groups' combined spending on lawsuits, election monitoring and
spreading allegations of cheating will far exceed the \$20 million
announced to date. That message, blasted out, in particular by Mr.
Trump, has stirred concerns that the Republican fraud drumbeat could lay
the groundwork for Mr. Trump and his supporters
\href{https://www.washingtonpost.com/opinions/2020/05/14/we-need-prepare-possibility-trump-rejecting-election-results/}{to
reject the election results} should he lose.

The Covid-19 pandemic has raised the stakes further, leading Democrats
and voting rights advocates to call for expanded voting by mail and Mr.
Trump and some Republicans to claim with little evidence that it would
invite fraud.

Some skeptics say the voting wars are partly political Kabuki, acted out
to rally supporters in both parties and raise funds for advocacy groups.
But in a presidential election where social distancing has muffled
campaigning and few voters remain on the fence, turnout has taken on
outsize importance. And neither side disputes that November's vote, as
in 2016, could turn on a relative handful of ballots in key states.

Neither the Trump campaign nor the Republican National Committee
responded to requests for interviews, although the committee provided a
summary of its work and policies. In essence, Republicans say Democratic
efforts to relax voting restrictions are partisan moves that demand a
firm response, and that Republican countermeasures reflect standard
political mobilizing.

Others say the Republican focus on vanishingly rare cases of fraud
targets a politically useful phantom.

``It's utter nonsense.
\href{https://www.brennancenter.org/sites/default/files/legal-work/Briefing_Memo_Debunking_Voter_Fraud_Myth.pdf}{This
has been shown over and over,''} said Kenneth R. Mayer, an elections
expert at the University of Wisconsin-Madison. ``The continued
insistence that there are material levels of intentional voter fraud is
itself a form of fraud.''

But political strategists insisted at the conservative committee
conference in March that ballot fakery was a major concern. ``In some of
these areas where there's no Republican presence whatsoever, then
they're going to cheat, and they're going to cheat early and they're
going to cheat often,'' Mr. Helton said at the March conference. At
polling places, he said, ``just having a presence of some sort is a
deterrent for probably 80 percent of the bad behavior.''

Being present at the polls is not unusual; in fact, both parties monitor
polls. Monitors check whether poll workers follow the rules and can
complain to election supervisors or summon party lawyers if differences
are not resolved.

They also can challenge voters' right to cast a ballot --- if, for
instance, a voter lacks a required ID card. That can force voters to
cast provisional ballots that are not counted unless they prove their
eligibility.

But Democrats say the Republican focus on monitors and repeated
allegations of fraud are part of a coordinated strategy to depress
turnout, especially by minorities, by fueling anxieties among voters
already suspicious of the authorities.

``They don't need to keep millions of people away'' from the polls, Ms.
Groh-Wargo said. ``Challenge a couple of voters here, a couple there,
and it all aggregates up. They realize they're going to win or lose this
thing at the margins.''

Among other things, Democrats cite Mr. Trump's repeated demands that law
enforcement officers patrol the polls and the recent creation of
voter-fraud task forces by Republicans in four state governments, at
least in part at the national party's urging.

They also point to a meeting in February attended by conservative
political luminaries and at least one national Republican Party
official, sponsored by the Center for National Policy, a group of
conservative power brokers. The topic was voter fraud and ``ballot
security'' operations, particularly in inner cities and areas with
Native American populations, according to The Intercept, which
\href{https://theintercept.com/2020/04/11/republican-poll-watchers-vote-by-mail-voter-fraud/?ref=hvper.com}{published
excerpts from a recording} of the meeting.

One group represented at that meeting, Texas-based
\href{https://truethevote.org/}{True the Vote}, is recruiting military
veterans to become poll monitors. The group, an offshoot of a Houston
Tea Party branch, was scrutinized by local prosecutors after its first
poll-monitoring effort in 2012
\href{https://www.theatlantic.com/magazine/archive/2012/10/the-ballot-cops/309085/}{sparked
complaints of voter intimidation}.

The group's founder, Catherine Engelbrecht, told the gathering that
Democrats could inundate the polls with phony votes. ``The swarming
tactics of a radicalized socialist mind-set,'' she warned, ``is a
dangerous thing to behold.'' The group did not respond to a request for
comment.

History also offers reason for Democrats' concern. The court order
vacated in 2018 involved repeated efforts to depress Democratic turnout.
In the first instance, the party recruited off-duty police officers
wearing ``National Ballot Security Task Force'' armbands to monitor
polling places in black and Latino neighborhoods in New Jersey. A
Democratic lawsuit claimed the officers hectored poll workers and voters
and stopped volunteers from helping voters cast ballots.

At the Conservative Political Action Committee conference, Justin Clark,
a Trump campaign senior adviser overseeing Election Day operations,
argued that the court order had handed a decades-long edge to Democrats.

``We were really operating with one hand behind our back,'' he said.

Speaking to Wisconsin Republicans in November, Mr. Clark said the
party's expanded poll-monitoring plans were accelerated by defeats last
November in governor's races in Kentucky and Louisiana.

The party has named three regional directors of Election Day operations,
is hiring directors in 15 key states and will beef up the paid staffs
that recruit and work with volunteers. Wisconsin, for example, is to
receive 100 operatives, compared with 62 in 2016.

One aim, he said, is to expand poll monitoring beyond the usual big-city
Democratic strongholds. Mr. Clark, in remarks
\href{https://www.youtube.com/watch?v=am0egba-KNQ}{that were posted
online} by the Democratic opposition group
\href{https://americanbridgepac.org}{American Bridge}, cited a county
where he said Mr. Trump won by 14,000 votes in 2016. ``But maybe he
should have won by 17,000,'' he said. ``Their cheating doesn't just
happen when you lose a county.''

In addition to the \$20 million raised by the party for legal battles
over election rules, conservative advocacy groups have joined the legal
war, filing lawsuits and briefs in states such as New Mexico, Minnesota
and Nevada. The Honest Elections Project, which surfaced only this
spring, already has joined legal battles over voting in six states and
has spent \$250,000 on advertising opposing voting by mail.

Honest Elections officials did not respond to a request for an
interview. But an account in the online publication Axios in January
detailed plans by Mr. Leo, its creator,
\href{https://www.axios.com/leonard-leo-crc-advisors-federalist-society-50d4d844-19a3-4eab-af2b-7b74f1617d1c.html?utm_source=twitter\&utm_medium=twsocialshare\&utm_campaign=organic}{``to
funnel tens of millions of dollars} into conservative fights''
nationwide.

Republicans said the goal of their litigation effort was ``to ensure the
integrity of the 2020 election'' and rebuff Democratic attempts ``to sue
their way to victory in 2020.'' But
\href{https://www.perkinscoie.com/en/professionals/marc-e-elias.html}{Marc
Elias, a Washington lawyer} who represents Democrats in many of the
suits that Republicans are contesting, said every Republican court
filing had sought to add or keep limits on voting rather than remove
them.

``I go to bed sleeping pretty well, thinking I'm fighting for everybody
to be able to vote,'' he said. ``When was the last time a party said it
would spend \$20 million to make voting harder?''

\hypertarget{our-2020-election-guide}{%
\section{Our 2020 Election Guide}\label{our-2020-election-guide}}

Updated Aug. 4, 2020

\begin{itemize}
\item
  \begin{center}\rule{0.5\linewidth}{\linethickness}\end{center}

  \hypertarget{the-latest}{%
  \subsection{The Latest}\label{the-latest}}

  \begin{itemize}
  \tightlist
  \item
    Kris Kobach, a polarizing figure in Kansas politics,
    \href{https://www.nytimes3xbfgragh.onion/2020/08/04/us/politics/kobach-tlaib.html?action=click\&pgtype=Article\&state=default\&region=BELOW_MAIN_CONTENT\&context=storylines_guide}{lost
    the Senate primary there}, relieving G.O.P. officials who feared he
    could jeopardize a safe seat.
  \end{itemize}
\item
  \begin{center}\rule{0.5\linewidth}{\linethickness}\end{center}

  \hypertarget{bidens-vp-search}{%
  \subsection{Biden's V.P. Search}\label{bidens-vp-search}}

  \begin{itemize}
  \tightlist
  \item
    \href{https://www.nytimes3xbfgragh.onion/article/biden-vice-president-2020.html?action=click\&pgtype=Article\&state=default\&region=BELOW_MAIN_CONTENT\&context=storylines_guide}{Here
    are 13 women} who have been under consideration to be Joe Biden's
    running mate, and why each might be chosen --- and might not be.
  \end{itemize}
\item
  \begin{center}\rule{0.5\linewidth}{\linethickness}\end{center}

  \hypertarget{keep-up-with-our-coverage}{%
  \subsection{Keep Up With Our
  Coverage}\label{keep-up-with-our-coverage}}

  \begin{itemize}
  \tightlist
  \item
    Get an
    \href{https://www.nytimes3xbfgragh.onion/newsletters/politics?action=click\&pgtype=Article\&state=default\&region=BELOW_MAIN_CONTENT\&context=storylines_guide}{email}
    recapping the day's news
  \end{itemize}

  \begin{itemize}
  \tightlist
  \item
    Download our mobile app on
    \href{https://apps.apple.com/us/app/nytimes/id284862083?ls=1\&mat_click_id=5c79ae7455014fd1bd66b5610c05b8f2-20191112-16948\&referrer=mat_click_id\%3D5c79ae7455014fd1bd66b5610c05b8f2-20191112-16948\%26link_click_id\%3D722930677036718082}{iOS}
    and
    \href{http://a.localytics.com/android?id=com.nytimes.android\&referrer=utm_source\%3Dother_nyt_mobile_web\%26utm_medium\%3DWeb\%2520page\%26utm_term\%3DGeneral\%2520Mobile\%2520Page\%26utm_campaign\%3DNYT\%2520Mobile\%2520General\%2520Page}{Android}
    and turn on Breaking News and Politics alerts
  \end{itemize}
\end{itemize}

Advertisement

\protect\hyperlink{after-bottom}{Continue reading the main story}

\hypertarget{site-index}{%
\subsection{Site Index}\label{site-index}}

\hypertarget{site-information-navigation}{%
\subsection{Site Information
Navigation}\label{site-information-navigation}}

\begin{itemize}
\tightlist
\item
  \href{https://help.nytimes3xbfgragh.onion/hc/en-us/articles/115014792127-Copyright-notice}{©~2020~The
  New York Times Company}
\end{itemize}

\begin{itemize}
\tightlist
\item
  \href{https://www.nytco.com/}{NYTCo}
\item
  \href{https://help.nytimes3xbfgragh.onion/hc/en-us/articles/115015385887-Contact-Us}{Contact
  Us}
\item
  \href{https://www.nytco.com/careers/}{Work with us}
\item
  \href{https://nytmediakit.com/}{Advertise}
\item
  \href{http://www.tbrandstudio.com/}{T Brand Studio}
\item
  \href{https://www.nytimes3xbfgragh.onion/privacy/cookie-policy\#how-do-i-manage-trackers}{Your
  Ad Choices}
\item
  \href{https://www.nytimes3xbfgragh.onion/privacy}{Privacy}
\item
  \href{https://help.nytimes3xbfgragh.onion/hc/en-us/articles/115014893428-Terms-of-service}{Terms
  of Service}
\item
  \href{https://help.nytimes3xbfgragh.onion/hc/en-us/articles/115014893968-Terms-of-sale}{Terms
  of Sale}
\item
  \href{https://spiderbites.nytimes3xbfgragh.onion}{Site Map}
\item
  \href{https://help.nytimes3xbfgragh.onion/hc/en-us}{Help}
\item
  \href{https://www.nytimes3xbfgragh.onion/subscription?campaignId=37WXW}{Subscriptions}
\end{itemize}
