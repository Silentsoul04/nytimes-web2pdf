Sections

SEARCH

\protect\hyperlink{site-content}{Skip to
content}\protect\hyperlink{site-index}{Skip to site index}

\href{https://www.nytimes3xbfgragh.onion/section/world/asia}{Asia
Pacific}

\href{https://myaccount.nytimes3xbfgragh.onion/auth/login?response_type=cookie\&client_id=vi}{}

\href{https://www.nytimes3xbfgragh.onion/section/todayspaper}{Today's
Paper}

\href{/section/world/asia}{Asia Pacific}\textbar{}Amid Pandemic, Finding
Normalcy in the Abnormal

\url{https://nyti.ms/3e48MHR}

\begin{itemize}
\item
\item
\item
\item
\item
\item
\end{itemize}

\hypertarget{the-coronavirus-outbreak}{%
\subsubsection{\texorpdfstring{\href{https://www.nytimes3xbfgragh.onion/news-event/coronavirus?name=styln-coronavirus-national\&region=TOP_BANNER\&variant=undefined\&block=storyline_menu_recirc\&action=click\&pgtype=Article\&impression_id=01b0b860-e3a1-11ea-8b08-6ddcf202df60}{The
Coronavirus
Outbreak}}{The Coronavirus Outbreak}}\label{the-coronavirus-outbreak}}

\begin{itemize}
\tightlist
\item
  live\href{https://www.nytimes3xbfgragh.onion/2020/08/21/world/covid-19-coronavirus.html?name=styln-coronavirus-national\&region=TOP_BANNER\&variant=undefined\&block=storyline_menu_recirc\&action=click\&pgtype=Article\&impression_id=01b0b861-e3a1-11ea-8b08-6ddcf202df60}{Latest
  Updates}
\item
  \href{https://www.nytimes3xbfgragh.onion/interactive/2020/us/coronavirus-us-cases.html?name=styln-coronavirus-national\&region=TOP_BANNER\&variant=undefined\&block=storyline_menu_recirc\&action=click\&pgtype=Article\&impression_id=01b0df70-e3a1-11ea-8b08-6ddcf202df60}{Maps
  and Cases}
\item
  \href{https://www.nytimes3xbfgragh.onion/interactive/2020/science/coronavirus-vaccine-tracker.html?name=styln-coronavirus-national\&region=TOP_BANNER\&variant=undefined\&block=storyline_menu_recirc\&action=click\&pgtype=Article\&impression_id=01b0df71-e3a1-11ea-8b08-6ddcf202df60}{Vaccine
  Tracker}
\item
  \href{https://www.nytimes3xbfgragh.onion/2020/08/19/us/colleges-closing-covid.html?name=styln-coronavirus-national\&region=TOP_BANNER\&variant=undefined\&block=storyline_menu_recirc\&action=click\&pgtype=Article\&impression_id=01b0df72-e3a1-11ea-8b08-6ddcf202df60}{Colleges
  Closing}
\item
  \href{https://www.nytimes3xbfgragh.onion/live/2020/08/20/business/stock-market-today-coronavirus?name=styln-coronavirus-national\&region=TOP_BANNER\&variant=undefined\&block=storyline_menu_recirc\&action=click\&pgtype=Article\&impression_id=01b0df73-e3a1-11ea-8b08-6ddcf202df60}{Economy}
\end{itemize}

Advertisement

\protect\hyperlink{after-top}{Continue reading the main story}

Supported by

\protect\hyperlink{after-sponsor}{Continue reading the main story}

Hong Kong Dispatch

\hypertarget{amid-pandemic-finding-normalcy-in-the-abnormal}{%
\section{Amid Pandemic, Finding Normalcy in the
Abnormal}\label{amid-pandemic-finding-normalcy-in-the-abnormal}}

In a city that is no stranger to outbreaks, life has a way of going on.

\includegraphics{https://static01.graylady3jvrrxbe.onion/images/2020/05/11/world/00virus-hk-dispatch-1/merlin_172336776_b8038a9b-0402-4c36-95d7-ef7b3d8ccf3c-articleLarge.jpg?quality=75\&auto=webp\&disable=upscale}

By \href{https://www.nytimes3xbfgragh.onion/by/vivian-wang}{Vivian Wang}

Photographs by Lam Yik Fei

\begin{itemize}
\item
  May 19, 2020
\item
  \begin{itemize}
  \item
  \item
  \item
  \item
  \item
  \item
  \end{itemize}
\end{itemize}

\href{https://cn.nytimes3xbfgragh.onion/china/20200520/coronavirus-hong-kong/}{阅读简体中文版}\href{https://cn.nytimes3xbfgragh.onion/china/20200520/coronavirus-hong-kong/zh-hant/}{閱讀繁體中文版}

HONG KONG --- ~Two blocks from my apartment on the western edge of Hong
Kong Island, a Starbucks has been transformed into what looks like a
construction zone, or maybe a strange art installation.

An armchair near the window was cordoned off for a time with masking
tape, and more strips stretched over and around other chairs nearby,
taut like tightropes over their neighboring tabletops. Rectangles of
white cardboard are clipped to the sides of tables, which now look more
like office cubicles than places to gather with friends.

But if the customers are fazed by the oddness of their surroundings,
they don't show it.

On a recent Tuesday night, a young couple huddled at one of the
tape-free tables, laughing at something on the girl's phone. A man
hunched over his laptop, seemingly oblivious to the silos shielding him
from his fellow patrons.

\includegraphics{https://static01.graylady3jvrrxbe.onion/images/2020/05/11/world/00virus-hk-dispatch-2/merlin_172301760_e24b0ab9-4e59-4ca2-a6d5-ff2a86268e10-articleLarge.jpg?quality=75\&auto=webp\&disable=upscale}

\href{https://www.nytimes3xbfgragh.onion/2020/07/20/world/asia/hong-kong-coronavirus.html}{Hong
Kong} was one of the first places outside mainland China to be
\href{https://www.scmp.com/news/hong-kong/health-environment/article/3047193/china-coronavirus-first-case-confirmed-hong-kong}{hit
by the coronavirus}, and immediately the landscape of the city changed.

There were temperature checks at every public building, and signs in
elevators telling you how often the buttons were sanitized. A pharmacy
chain handed out fistfuls of stickers with every purchase, featuring the
chain's mascot --- a winking orange cat --- and a reminder: ``Wash your
hands! Rub your hands! 20 seconds, Thx.''

Everywhere, there were reminders that these were not normal times.

Four months later, those signs are still around. But the city is humming
back to life ---~not really in spite of those omnipresent reminders so
much as alongside them.

\hypertarget{latest-updates-the-coronavirus-outbreak}{%
\section{\texorpdfstring{\href{https://www.nytimes3xbfgragh.onion/2020/08/21/world/covid-19-coronavirus.html?action=click\&pgtype=Article\&state=default\&region=MAIN_CONTENT_1\&context=storylines_live_updates}{Latest
Updates: The Coronavirus
Outbreak}}{Latest Updates: The Coronavirus Outbreak}}\label{latest-updates-the-coronavirus-outbreak}}

Updated 2020-08-21T11:05:09.310Z

\begin{itemize}
\tightlist
\item
  \href{https://www.nytimes3xbfgragh.onion/2020/08/21/world/covid-19-coronavirus.html?action=click\&pgtype=Article\&state=default\&region=MAIN_CONTENT_1\&context=storylines_live_updates\#link-4690b6aa}{Shutdowns,
  warnings and scoldings follow gatherings on college campuses.}
\item
  \href{https://www.nytimes3xbfgragh.onion/2020/08/21/world/covid-19-coronavirus.html?action=click\&pgtype=Article\&state=default\&region=MAIN_CONTENT_1\&context=storylines_live_updates\#link-324af071}{As
  he accepts the Democratic nomination, Biden knocks Trump's pandemic
  response.}
\item
  \href{https://www.nytimes3xbfgragh.onion/2020/08/21/world/covid-19-coronavirus.html?action=click\&pgtype=Article\&state=default\&region=MAIN_CONTENT_1\&context=storylines_live_updates\#link-35890b73}{Hundreds
  of doctors in Kenya go on strike over their pay and protective gear.}
\end{itemize}

\href{https://www.nytimes3xbfgragh.onion/2020/08/21/world/covid-19-coronavirus.html?action=click\&pgtype=Article\&state=default\&region=MAIN_CONTENT_1\&context=storylines_live_updates}{See
more updates}

More live coverage:
\href{https://www.nytimes3xbfgragh.onion/live/2020/08/20/business/stock-market-today-coronavirus?action=click\&pgtype=Article\&state=default\&region=MAIN_CONTENT_1\&context=storylines_live_updates}{Markets}

The attendance for morning tai chi in the park behind my apartment has
grown from a few elderly ladies in face masks to dozens. The crowds
strolling along Victoria Harbor have grown denser, children giggling
behind the plastic visors their parents force on them. Many cha chaan
tengs --- the hole-in-the-wall Cantonese diners that serve up milk tea,
egg tarts and beef chow fun --- still offer discounts for takeout, but
the tables inside are beginning to fill up, too.

Image

Sun Yat-sen Park, alongside Hong Kong Harbor, is a magnet for exercise
enthusiasts.

Most directly, this is a response to the good news of recent weeks.

Hong Kong has recorded just three locally transmitted cases in the last
30 days. Only four people are reported to have died of Covid-19 since
the outbreak began. The government has loosened social-distancing
restrictions, allowing civil servants to go back to work and restaurants
to return to full capacity, instead of half.

But that's not the only reason the virus no longer seems to rule every
facet of life here. While fear and anxiety linger, Hong Kongers seem
particularly adept at living with those emotions --- maybe not embracing
this strange new reality, but not recoiling from it either.

That unflappability has struck me sharply during my time here.

I moved to Hong Kong from New York City three months ago. Before
boarding the plane, I had never worn a face mask. Many of my
conversations with friends back home revolve around how long it will
take for things to go back to the way they were before. Like,
\emph{really} like before --- not just without stay-at-home orders and
shuttered businesses, but also without masks and the words ``social
distancing.''

Image

A gravestone for a nurse who died during the SARS epidemic in 2003 after
having contracted the virus from a patient.~

In Hong Kong, ``real" life doesn't seem so mutually exclusive with our
present one. That's in no small part because the city has been through
this before.

Before moving, I read up on Hong Kong's last battle with an epidemic:
SARS. I knew it had been scarred by the disease, which barreled across
the city in 2003 and killed almost 300 people. But I didn't realize how
deeply that experience had embedded itself in the city's psyche until I
arrived.

Face masks are not uncommon even in outbreak-free times. And Hong
Kongers easily remember to press elevator buttons with their keys rather
than their fingertips because they have been doing so for years.

So when the coronavirus hit, people simply took what they had already
been doing and escalated it --- sometimes
\href{https://www.statnews.com/2020/03/26/coronavirus-hong-kong-resurgenece-holds-lesson-defeating-it-demands-persistence/}{even
before the government told them} to.

\href{https://www.nytimes3xbfgragh.onion/news-event/coronavirus?action=click\&pgtype=Article\&state=default\&region=MAIN_CONTENT_3\&context=storylines_faq}{}

\hypertarget{the-coronavirus-outbreak-}{%
\subsubsection{The Coronavirus Outbreak
›}\label{the-coronavirus-outbreak-}}

\hypertarget{frequently-asked-questions}{%
\paragraph{Frequently Asked
Questions}\label{frequently-asked-questions}}

Updated August 17, 2020

\begin{itemize}
\item ~
  \hypertarget{why-does-standing-six-feet-away-from-others-help}{%
  \paragraph{Why does standing six feet away from others
  help?}\label{why-does-standing-six-feet-away-from-others-help}}

  \begin{itemize}
  \tightlist
  \item
    The coronavirus spreads primarily through droplets from your mouth
    and nose, especially when you cough or sneeze. The C.D.C., one of
    the organizations using that measure,
    \href{https://www.nytimes3xbfgragh.onion/2020/04/14/health/coronavirus-six-feet.html?action=click\&pgtype=Article\&state=default\&region=MAIN_CONTENT_3\&context=storylines_faq}{bases
    its recommendation of six feet} on the idea that most large droplets
    that people expel when they cough or sneeze will fall to the ground
    within six feet. But six feet has never been a magic number that
    guarantees complete protection. Sneezes, for instance, can launch
    droplets a lot farther than six feet,
    \href{https://jamanetwork.com/journals/jama/fullarticle/2763852}{according
    to a recent study}. It's a rule of thumb: You should be safest
    standing six feet apart outside, especially when it's windy. But
    keep a mask on at all times, even when you think you're far enough
    apart.
  \end{itemize}
\item ~
  \hypertarget{i-have-antibodies-am-i-now-immune}{%
  \paragraph{I have antibodies. Am I now
  immune?}\label{i-have-antibodies-am-i-now-immune}}

  \begin{itemize}
  \tightlist
  \item
    As of right
    now,\href{https://www.nytimes3xbfgragh.onion/2020/07/22/health/covid-antibodies-herd-immunity.html?action=click\&pgtype=Article\&state=default\&region=MAIN_CONTENT_3\&context=storylines_faq}{that
    seems likely, for at least several months.} There have been
    frightening accounts of people suffering what seems to be a second
    bout of Covid-19. But experts say these patients may have a
    drawn-out course of infection, with the virus taking a slow toll
    weeks to months after initial exposure. People infected with the
    coronavirus typically
    \href{https://www.nature.com/articles/s41586-020-2456-9}{produce}
    immune molecules called antibodies, which are
    \href{https://www.nytimes3xbfgragh.onion/2020/05/07/health/coronavirus-antibody-prevalence.html?action=click\&pgtype=Article\&state=default\&region=MAIN_CONTENT_3\&context=storylines_faq}{protective
    proteins made in response to an
    infection}\href{https://www.nytimes3xbfgragh.onion/2020/05/07/health/coronavirus-antibody-prevalence.html?action=click\&pgtype=Article\&state=default\&region=MAIN_CONTENT_3\&context=storylines_faq}{.
    These antibodies may} last in the body
    \href{https://www.nature.com/articles/s41591-020-0965-6}{only two to
    three months}, which may seem worrisome, but that's perfectly normal
    after an acute infection subsides, said Dr. Michael Mina, an
    immunologist at Harvard University. It may be possible to get the
    coronavirus again, but it's highly unlikely that it would be
    possible in a short window of time from initial infection or make
    people sicker the second time.
  \end{itemize}
\item ~
  \hypertarget{im-a-small-business-owner-can-i-get-relief}{%
  \paragraph{I'm a small-business owner. Can I get
  relief?}\label{im-a-small-business-owner-can-i-get-relief}}

  \begin{itemize}
  \tightlist
  \item
    The
    \href{https://www.nytimes3xbfgragh.onion/article/small-business-loans-stimulus-grants-freelancers-coronavirus.html?action=click\&pgtype=Article\&state=default\&region=MAIN_CONTENT_3\&context=storylines_faq}{stimulus
    bills enacted in March} offer help for the millions of American
    small businesses. Those eligible for aid are businesses and
    nonprofit organizations with fewer than 500 workers, including sole
    proprietorships, independent contractors and freelancers. Some
    larger companies in some industries are also eligible. The help
    being offered, which is being managed by the Small Business
    Administration, includes the Paycheck Protection Program and the
    Economic Injury Disaster Loan program. But lots of folks have
    \href{https://www.nytimes3xbfgragh.onion/interactive/2020/05/07/business/small-business-loans-coronavirus.html?action=click\&pgtype=Article\&state=default\&region=MAIN_CONTENT_3\&context=storylines_faq}{not
    yet seen payouts.} Even those who have received help are confused:
    The rules are draconian, and some are stuck sitting on
    \href{https://www.nytimes3xbfgragh.onion/2020/05/02/business/economy/loans-coronavirus-small-business.html?action=click\&pgtype=Article\&state=default\&region=MAIN_CONTENT_3\&context=storylines_faq}{money
    they don't know how to use.} Many small-business owners are getting
    less than they expected or
    \href{https://www.nytimes3xbfgragh.onion/2020/06/10/business/Small-business-loans-ppp.html?action=click\&pgtype=Article\&state=default\&region=MAIN_CONTENT_3\&context=storylines_faq}{not
    hearing anything at all.}
  \end{itemize}
\item ~
  \hypertarget{what-are-my-rights-if-i-am-worried-about-going-back-to-work}{%
  \paragraph{What are my rights if I am worried about going back to
  work?}\label{what-are-my-rights-if-i-am-worried-about-going-back-to-work}}

  \begin{itemize}
  \tightlist
  \item
    Employers have to provide
    \href{https://www.osha.gov/SLTC/covid-19/standards.html}{a safe
    workplace} with policies that protect everyone equally.
    \href{https://www.nytimes3xbfgragh.onion/article/coronavirus-money-unemployment.html?action=click\&pgtype=Article\&state=default\&region=MAIN_CONTENT_3\&context=storylines_faq}{And
    if one of your co-workers tests positive for the coronavirus, the
    C.D.C.} has said that
    \href{https://www.cdc.gov/coronavirus/2019-ncov/community/guidance-business-response.html}{employers
    should tell their employees} -\/- without giving you the sick
    employee's name -\/- that they may have been exposed to the virus.
  \end{itemize}
\item ~
  \hypertarget{what-is-school-going-to-look-like-in-september}{%
  \paragraph{What is school going to look like in
  September?}\label{what-is-school-going-to-look-like-in-september}}

  \begin{itemize}
  \tightlist
  \item
    It is unlikely that many schools will return to a normal schedule
    this fall, requiring the grind of
    \href{https://www.nytimes3xbfgragh.onion/2020/06/05/us/coronavirus-education-lost-learning.html?action=click\&pgtype=Article\&state=default\&region=MAIN_CONTENT_3\&context=storylines_faq}{online
    learning},
    \href{https://www.nytimes3xbfgragh.onion/2020/05/29/us/coronavirus-child-care-centers.html?action=click\&pgtype=Article\&state=default\&region=MAIN_CONTENT_3\&context=storylines_faq}{makeshift
    child care} and
    \href{https://www.nytimes3xbfgragh.onion/2020/06/03/business/economy/coronavirus-working-women.html?action=click\&pgtype=Article\&state=default\&region=MAIN_CONTENT_3\&context=storylines_faq}{stunted
    workdays} to continue. California's two largest public school
    districts --- Los Angeles and San Diego --- said on July 13, that
    \href{https://www.nytimes3xbfgragh.onion/2020/07/13/us/lausd-san-diego-school-reopening.html?action=click\&pgtype=Article\&state=default\&region=MAIN_CONTENT_3\&context=storylines_faq}{instruction
    will be remote-only in the fall}, citing concerns that surging
    coronavirus infections in their areas pose too dire a risk for
    students and teachers. Together, the two districts enroll some
    825,000 students. They are the largest in the country so far to
    abandon plans for even a partial physical return to classrooms when
    they reopen in August. For other districts, the solution won't be an
    all-or-nothing approach.
    \href{https://bioethics.jhu.edu/research-and-outreach/projects/eschool-initiative/school-policy-tracker/}{Many
    systems}, including the nation's largest, New York City, are
    devising
    \href{https://www.nytimes3xbfgragh.onion/2020/06/26/us/coronavirus-schools-reopen-fall.html?action=click\&pgtype=Article\&state=default\&region=MAIN_CONTENT_3\&context=storylines_faq}{hybrid
    plans} that involve spending some days in classrooms and other days
    online. There's no national policy on this yet, so check with your
    municipal school system regularly to see what is happening in your
    community.
  \end{itemize}
\end{itemize}

They stayed home when they could, and donned masks when they couldn't. I
met many new people in my first weeks in the city; none of them shook my
hand. The first time I rode the subway after arriving, I could see down
the entire length of the train, a forest of red subway poles with not a
single person in sight.

Paradoxically, this longstanding vigilance and self-restraint have
helped Hong Kong preserve some of itself, perhaps more successfully than
other cities that have tried to cling to their pre-virus selves.
Museums, schools and gyms closed here, but restaurants never had to, nor
did hair salons or retail stores.

Image

A family picnic at Sun Yat-sen Park on Hong Kong Island this month.

Still, in many ways it's an illusion. In a city as dense as Hong Kong,
even life at half-volume looks vibrant.

It's the balance sheets that reveal the truth.

Hong Kong is deep in recession, having just recorded its
\href{https://www.scmp.com/news/hong-kong/hong-kong-economy/article/3082769/coronavirus-hong-kongs-economy-slumps-89-cent}{biggest-ever
economic contraction}. Restaurants may have stayed open, but
they're\href{https://www.scmp.com/news/hong-kong/health-environment/article/3083160/coronavirus-earnings-hong-kongs-food-and-beverage}{floundering}.
One woman
\href{https://www.nytimes3xbfgragh.onion/2020/03/04/world/coronavirus-schools-closed.html}{I
interviewed in March} had quit her job as a security guard to watch her
children after schools were closed; they still have not reopened.

And, like everyone else in the world, Hong Kongers are tired.

Sometimes social distancing gets lip service. At a noodle shop two weeks
ago, after the available tables filled up, a waitress deposited a
stranger at the table where I was already sitting with two friends.
``It's OK, four people,'' she said, referring to the government's
four-person limit on public gatherings. (It has since been raised to
eight.)

And in Hong Kong, anyway, even the end of the outbreak may not bring
true normalcy. Already, the
\href{https://www.nytimes3xbfgragh.onion/news-event/hong-kong-protests}{pro-democracy
protests} that roiled the city for most of last year have begun
rekindling, as people feel more comfortable gathering en masse. Many
believe they will soon come roaring back.

So I may not know what real normal life ---~like, \emph{reall}y real ---
looks like in Hong Kong for a while. But, increasingly, it feels as if
that's not so important, anyway.

Image

A roped-off playground in Hong Kong.

Advertisement

\protect\hyperlink{after-bottom}{Continue reading the main story}

\hypertarget{site-index}{%
\subsection{Site Index}\label{site-index}}

\hypertarget{site-information-navigation}{%
\subsection{Site Information
Navigation}\label{site-information-navigation}}

\begin{itemize}
\tightlist
\item
  \href{https://help.nytimes3xbfgragh.onion/hc/en-us/articles/115014792127-Copyright-notice}{©~2020~The
  New York Times Company}
\end{itemize}

\begin{itemize}
\tightlist
\item
  \href{https://www.nytco.com/}{NYTCo}
\item
  \href{https://help.nytimes3xbfgragh.onion/hc/en-us/articles/115015385887-Contact-Us}{Contact
  Us}
\item
  \href{https://www.nytco.com/careers/}{Work with us}
\item
  \href{https://nytmediakit.com/}{Advertise}
\item
  \href{http://www.tbrandstudio.com/}{T Brand Studio}
\item
  \href{https://www.nytimes3xbfgragh.onion/privacy/cookie-policy\#how-do-i-manage-trackers}{Your
  Ad Choices}
\item
  \href{https://www.nytimes3xbfgragh.onion/privacy}{Privacy}
\item
  \href{https://help.nytimes3xbfgragh.onion/hc/en-us/articles/115014893428-Terms-of-service}{Terms
  of Service}
\item
  \href{https://help.nytimes3xbfgragh.onion/hc/en-us/articles/115014893968-Terms-of-sale}{Terms
  of Sale}
\item
  \href{https://spiderbites.nytimes3xbfgragh.onion}{Site Map}
\item
  \href{https://help.nytimes3xbfgragh.onion/hc/en-us}{Help}
\item
  \href{https://www.nytimes3xbfgragh.onion/subscription?campaignId=37WXW}{Subscriptions}
\end{itemize}
