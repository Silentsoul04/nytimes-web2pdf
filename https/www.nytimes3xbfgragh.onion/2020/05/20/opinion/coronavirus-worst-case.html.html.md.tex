Sections

SEARCH

\protect\hyperlink{site-content}{Skip to
content}\protect\hyperlink{site-index}{Skip to site index}

\href{https://myaccount.nytimes3xbfgragh.onion/auth/login?response_type=cookie\&client_id=vi}{}

\href{https://www.nytimes3xbfgragh.onion/section/todayspaper}{Today's
Paper}

\href{/section/opinion}{Opinion}\textbar{}The Worst Is Yet to Come

\url{https://nyti.ms/2Zn7pQi}

\begin{itemize}
\item
\item
\item
\item
\item
\item
\end{itemize}

Advertisement

\protect\hyperlink{after-top}{Continue reading the main story}

\href{/section/opinion}{Opinion}

Supported by

\protect\hyperlink{after-sponsor}{Continue reading the main story}

\hypertarget{the-worst-is-yet-to-come}{%
\section{The Worst Is Yet to Come}\label{the-worst-is-yet-to-come}}

The coronavirus and our disastrous national response to it has smashed
optimists like me in the head.

\href{https://www.nytimes3xbfgragh.onion/by/farhad-manjoo}{\includegraphics{https://static01.graylady3jvrrxbe.onion/images/2019/01/08/opinion/farhad-manjoo-opinion/farhad-manjoo-opinion-thumbLarge.png}}

By \href{https://www.nytimes3xbfgragh.onion/by/farhad-manjoo}{Farhad
Manjoo}

Opinion Columnist

\begin{itemize}
\item
  May 20, 2020
\item
  \begin{itemize}
  \item
  \item
  \item
  \item
  \item
  \item
  \end{itemize}
\end{itemize}

\includegraphics{https://static01.graylady3jvrrxbe.onion/images/2020/05/20/opinion/20manjooWeb/merlin_172280343_12e5bd8d-9741-4589-8746-31bb2e175080-articleLarge.jpg?quality=75\&auto=webp\&disable=upscale}

\href{https://cn.nytimes3xbfgragh.onion/opinion/20200521/coronavirus-worst-case/}{阅读简体中文版}\href{https://cn.nytimes3xbfgragh.onion/opinion/20200521/coronavirus-worst-case/zh-hant/}{閱讀繁體中文版}

For as long as I can remember, I have identified as an optimist. Like a
seedling reaching toward the golden sun, I'm innately tuned to seek out
the bright side.

Of course, in recent years this confidence has grown tougher to
maintain. The industry I've long covered, technology, has
\href{https://www.businessinsider.com/employees-lament-the-open-culture-that-made-google-famous-2020-5}{lost
its rebel edge}, and
\href{https://www.nytimes3xbfgragh.onion/2017/10/18/technology/frightful-five-start-ups.html}{grown
monopolistic} and power hungry. The economy at large echoed these
trends, leaving
\href{https://www.nytimes3xbfgragh.onion/2020/04/09/opinion/sunday/inequality-coronavirus.html}{all
but the wealthiest out in the cold}. All the while the entire planet
veered toward uninhabitability.

And yet, for much of the last year, I remained an optimist. A
re-energized Democratic Party
\href{https://www.nytimes3xbfgragh.onion/2019/06/05/opinion/elizabeth-warren-america.html}{looked
poised to push for grand solutions to big problems}, from health care to
education to climate change. There was finally some talk about reining
in monopolies and creating a fairer economy. Things weren't looking
good, exactly, but if you squinted hard, you could just make out a
sunnier future.

\begin{quote}
Hi! I'm chatting live about my
\href{https://twitter.com/nytopinion?ref_src=twsrc\%5Etfw}{@nytopinion}
column on how I've gone from optimist to pessimist. Also Trump, Twitter,
Facebook and whatever else. Join me!\url{https://t.co/A15JhDex4P}
\url{https://t.co/uTPWyvySil}

--- Farhad Manjoo (@fmanjoo)
\href{https://twitter.com/fmanjoo/status/1266414229713637376?ref_src=twsrc\%5Etfw}{May
29, 2020}
\end{quote}

Now all that seems lost. The coronavirus and our disastrous national
response to it has smashed optimists like me in the head. If there is a
silver lining, we'll have to work hard to find it.

To do that, we should spend more time considering the real possibility
that every problem we face will get much worse than we ever imagined.
The coronavirus is like a heat-seeking missile designed to frustrate
progress in almost every corner of society, from politics to the economy
to the environment.

The only way to avoid the worst fate might be to dwell on it. To
forestall doom, it's time to go full doomer.

Why so glum? It is not just that nearly 92,000 Americans are dead and
tens of millions are unemployed. It's not just that our federal
government has been asleep, with Congress unable or unwilling to push a
disaster-response bill on anything like the scale this crisis demands,
and an inept president unable to muster much greater sympathy than,
``\href{https://twitter.com/atrupar/status/1262769806639587329?s=21}{It's
too bad}.'' It's not only that
\href{https://www.nytimes3xbfgragh.onion/2020/05/19/us/coronavirus-updates.html\#link-1fe21236}{global
cooperation is in tatters when we need it most}.

It is all these things and something more fundamental: a startling lack
of leadership on identifying the worst consequences of this crisis and
marshaling a united front against them. Indeed, division and chaos might
now be the permanent order of the day.

In a book published more than a decade ago,
\href{https://www.nytimes3xbfgragh.onion/2016/11/03/technology/how-the-internet-is-loosening-our-grip-on-the-truth.html}{I
argued} that the internet might lead to a choose-your-own-facts world in
which different segments of society believe in different versions of
reality. The Trump era, and now the coronavirus, has confirmed this grim
prediction.

That's because the pandemic actually has created different political
realities. The coronavirus has hit dense, racially diverse Democratic
urban strongholds like New York
\href{https://www.npr.org/2020/04/12/832455226/what-coronavirus-exposes-about-americas-political-divide}{much
harder} than sparsely populated rural areas, which lean strongly to the
G.O.P. That divergent impact --- with help from the president and his
acolytes --- is feeding a dangerous partisan split about the nature of
the virus itself.

Consider the
\href{https://www.vox.com/2020/5/13/21257181/coronavirus-masks-trump-republicans-culture-war}{emerging
culture war about wearing masks} or about
\href{https://www.nytimes3xbfgragh.onion/2020/04/22/business/media/virus-fox-news-hydroxychloroquine.html}{whether
to take certain unproven therapies}. Look at the protests over whether
it's safe to reopen. Now play these divisions forward. As The Times's
Kevin Roose wrote last week, when a vaccine does emerge, what if many
Americans, fed on anti-vax rumors, simply
\href{https://www.nytimes3xbfgragh.onion/2020/05/13/technology/coronavirus-vaccine-disinformation.html}{refuse
to take it}?

The virus's economic effects will only create further inequality and
division. Google, Facebook, Amazon and other behemoths will not only
survive, they
\href{https://www.nytimes3xbfgragh.onion/2020/03/23/technology/coronavirus-facebook-amazon-youtube.html}{look
poised to emerge stronger than ever}. Most of their competition --- not
just small businesses but many of America's physical retailers and their
millions of employees --- could be decimated.

Worst of all, it's possible that the pain of this crisis might not fully
register in broad economic indicators , especially if, as
\href{https://www.nytimes3xbfgragh.onion/2019/09/18/opinion/obama-2008-financial-crisis.html}{happened
after the 2008 recession}, we see a long, slow recovery that benefits
mainly the wealthy. There are already signs that this is happening:
Thousands died, millions lost their jobs, but stock indexes are
rebounding.

The economic impacts feed into the political ones: The virus-induced
recession could further destroy the news industry and dramatically
reduce the number of working journalists in the country, our last
defense against misinformation.

Even worse, the virus is making a hash of emerging solutions to
entrenched problems.
\href{https://www.nytimes3xbfgragh.onion/2020/02/14/books/review/golden-gates-housing-conor-dougherty.html}{As
The Times's Conor Dougherty chronicled in ``Golden Gates,''} his recent
book on America's housing crisis, activists have lately been finding
success in pushing to build more housing in restrictive regions like the
San Francisco Bay Area. The virus may put such reforms on ice.
\href{https://qz.com/1824243/coronavirus-has-killed-off-public-transportation-across-the-world/}{And
consider the grim future of public transportation} after the pandemic:
Will people just get back in their cars, driving everywhere they go?

I called a few economists, activists and historians to discuss my
growing alarm about the future. Many were less pessimistic than I am;
some suggested that the virus could prompt much-needed action. The most
instructive example is the Great Depression. In the 1930s, after years
of inaction, reformers who came into office with Franklin D. Roosevelt
were able to push through laws that improved American life for good.

Matt Stoller, an antimonopoly scholar at the American Economic Liberties
Project, a think tank, agreed that this crisis could be the jolt we need
to fix American institutions. But he also noted that the United States
has failed to make the best of our most recent national calamities. The
9/11 attacks pushed us into needless quagmires in the Middle East. The
2008 recession deepened inequality.

Let us not squander another crisis. We need to take a long, hard look at
all the ways the pandemic can push this little planet of ours to further
ruin ---~and then work like crazy, together, to stave off the coming
hell.

\hypertarget{office-hours-with-farhad-manjoo}{%
\subsection{Office Hours With Farhad
Manjoo}\label{office-hours-with-farhad-manjoo}}

\emph{Farhad wants to}
\href{https://www.nytimes3xbfgragh.onion/2019/05/16/opinion/farhad-office-hours.html?module=inline}{\emph{chat
with readers on the phone}}\emph{. If you're interested in talking to a
New York Times columnist about anything that's on your mind, please fill
out this form. Farhad will select a few readers to call.}

\emph{The Times is committed to publishing}
\href{https://www.nytimes3xbfgragh.onion/2019/01/31/opinion/letters/letters-to-editor-new-york-times-women.html}{\emph{a
diversity of letters}} \emph{to the editor. We'd like to hear what you
think about this or any of our articles. Here are some}
\href{https://help.nytimes3xbfgragh.onion/hc/en-us/articles/115014925288-How-to-submit-a-letter-to-the-editor}{\emph{tips}}\emph{.
And here's our email:}
\href{mailto:letters@NYTimes.com}{\emph{letters@NYTimes.com}}\emph{.}

\emph{Follow The New York Times Opinion section on}
\href{https://www.facebookcorewwwi.onion/nytopinion}{\emph{Facebook}}\emph{,}
\href{http://twitter.com/NYTOpinion}{\emph{Twitter (@NYTopinion)}}
\emph{and}
\href{https://www.instagram.com/nytopinion/}{\emph{Instagram}}\emph{.}

Advertisement

\protect\hyperlink{after-bottom}{Continue reading the main story}

\hypertarget{site-index}{%
\subsection{Site Index}\label{site-index}}

\hypertarget{site-information-navigation}{%
\subsection{Site Information
Navigation}\label{site-information-navigation}}

\begin{itemize}
\tightlist
\item
  \href{https://help.nytimes3xbfgragh.onion/hc/en-us/articles/115014792127-Copyright-notice}{©~2020~The
  New York Times Company}
\end{itemize}

\begin{itemize}
\tightlist
\item
  \href{https://www.nytco.com/}{NYTCo}
\item
  \href{https://help.nytimes3xbfgragh.onion/hc/en-us/articles/115015385887-Contact-Us}{Contact
  Us}
\item
  \href{https://www.nytco.com/careers/}{Work with us}
\item
  \href{https://nytmediakit.com/}{Advertise}
\item
  \href{http://www.tbrandstudio.com/}{T Brand Studio}
\item
  \href{https://www.nytimes3xbfgragh.onion/privacy/cookie-policy\#how-do-i-manage-trackers}{Your
  Ad Choices}
\item
  \href{https://www.nytimes3xbfgragh.onion/privacy}{Privacy}
\item
  \href{https://help.nytimes3xbfgragh.onion/hc/en-us/articles/115014893428-Terms-of-service}{Terms
  of Service}
\item
  \href{https://help.nytimes3xbfgragh.onion/hc/en-us/articles/115014893968-Terms-of-sale}{Terms
  of Sale}
\item
  \href{https://spiderbites.nytimes3xbfgragh.onion}{Site Map}
\item
  \href{https://help.nytimes3xbfgragh.onion/hc/en-us}{Help}
\item
  \href{https://www.nytimes3xbfgragh.onion/subscription?campaignId=37WXW}{Subscriptions}
\end{itemize}
