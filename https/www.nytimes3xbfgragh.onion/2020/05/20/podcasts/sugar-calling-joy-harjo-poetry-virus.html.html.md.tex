Sections

SEARCH

\protect\hyperlink{site-content}{Skip to
content}\protect\hyperlink{site-index}{Skip to site index}

\href{https://www.nytimes3xbfgragh.onion/spotlight/podcasts}{Podcasts}

\href{https://myaccount.nytimes3xbfgragh.onion/auth/login?response_type=cookie\&client_id=vi}{}

\href{https://www.nytimes3xbfgragh.onion/section/todayspaper}{Today's
Paper}

\href{/spotlight/podcasts}{Podcasts}\textbar{}`I Release You, Fear'

\url{https://nyti.ms/3geHalh}

\begin{itemize}
\item
\item
\item
\item
\item
\item
\end{itemize}

Advertisement

\protect\hyperlink{after-top}{Continue reading the main story}

transcript

Back to Sugar Calling

bars

0:00/35:30

-35:30

transcript

\hypertarget{i-release-you-fear}{%
\subsection{`I Release You, Fear'}\label{i-release-you-fear}}

\hypertarget{hosted-by-cheryl-strayed-produced-by-kelly-prime-and-edited-by-sara-sarasohn-editorial-oversight-by-wendy-dorr}{%
\subsubsection{Hosted by Cheryl Strayed, produced by Kelly Prime and
edited by Sara Sarasohn. Editorial oversight by Wendy
Dorr.}\label{hosted-by-cheryl-strayed-produced-by-kelly-prime-and-edited-by-sara-sarasohn-editorial-oversight-by-wendy-dorr}}

\hypertarget{cheryl-strayed-talks-with-the-poet-joy-harjo-about-beauty-prophecies-and-listening-to-your-spiritual-council}{%
\paragraph{Cheryl Strayed talks with the poet Joy Harjo about beauty,
prophecies and listening to your spiritual
council.}\label{cheryl-strayed-talks-with-the-poet-joy-harjo-about-beauty-prophecies-and-listening-to-your-spiritual-council}}

Wednesday, May 20th, 2020

\begin{itemize}
\item
  cheryl strayed\\
  I'm going to call Joy Harjo. Joy is a poet, musician, playwright, and
  author. She's a member of the Muskogee Creek Nation and our current
  U.S. Poet Laureate. I first discovered Joy in my early 20s when I was
  an undergraduate at the University of Minnesota. And I was in this
  little bookstore near campus, and I saw this book called ``She Had
  Some Horses.'' I was drawn to it because I grew up having horses too.
  That book was one of Joy's beautiful poetry collections. When I read
  it, I knew that I had found a writer I would follow anywhere, and I
  have through her many books of poems and also her beautiful memoir,
  ``Crazy Brave.'' She offers us insight and truth that feels like she's
  working in this place that's sort of beyond knowing or beyond
  explanation, and yet it rings the deepest bell of truth within me. I'm
  going to give Joy a call.
\item
  {[}music playing{]}
\item
  {[}dial tone{]}
\item
  joy harjo\\
  Hi, Cheryl.
\item
  cheryl strayed\\
  Hi, Joy.
\item
  joy harjo\\
  How are you doing?
\item
  cheryl strayed\\
  I'm good. It's so nice to talk to you. Where are you right now?
\item
  joy harjo\\
  I'm in Tulsa, Oklahoma.
\item
  cheryl strayed\\
  Uh-huh.
\item
  joy harjo\\
  I have a Tulsa Artist Fellowship, and they give you housing. So my
  husband and I are living in a two-bedroom apartment here in the Arts
  District downtown. And we have land south of here, so he takes off
  every day there because we eventually want to build a house. And I'm
  holed up here. And I know how to be here. I mean, that's --- we
  writers --- it's the space I've always occupied primarily.
\item
  cheryl strayed\\
  Yeah, yeah. I mean, we need to be alone to write after all. But I have
  to say, Joy, now that it's been a couple months, it's getting old for
  me. I miss seeing people. I don't always want to be socially distant.
  In fact, I was remembering a day when we were not so socially distant.
  The last time I saw you in Portland, we were at a writer's conference.
\item
  joy harjo\\
  Yes, I remember that.
\item
  cheryl strayed\\
  And I had this clear image of whispering with you. We were at the sort
  of back of the room when somebody was giving a talk, and I realized I
  missed being up close like that with friends and people who weren't
  family members.
\item
  joy harjo\\
  Yes, it's starting to wear a little bit. Because as much as I'm a
  person that requires lots of solitude, I also am very social. And I
  miss that. I perform a lot. I go out a lot. And this online stuff just
  doesn't quite do the trick. I did a reading. I'm not saying yes to a
  lot of these, but I had one organization out of three months worth of
  performances agreed to let me do a livestream. And so I said sure. And
  I did a reading. I did a whole reading, sitting at my little table.
  And I guess they had several hundred people on. So I asked my spirit.
  I said, OK, I'm about to do this. I always ask for help and that what
  comes through needs to, et cetera. And I said, please help me feel and
  get a sense of the people. Because we all, any of us that get up there
  and do this, we feel the people and feel the energy and what needs to
  be said. And so I asked that. And so I started to feel them, but it's
  not the same.
\item
  cheryl strayed\\
  No.
\item
  joy harjo\\
  It's nothing like being with living, breathing bodies, everyone
  together, listening together, and being together. It's just not the
  same thing.
\item
  cheryl strayed\\
  No, no. Like you, I do a lot of events where there are a lot of bodies
  gathered together. And often, I think the work that both of us do ---
  and you and I tend to write about matters of the heart, and the
  spirit, and the soul --- there's that kind of healing aspect. And I
  think that especially, there is this something that you feel, like the
  magic, or the vibration, or whatever it is in the room. And I'm
  wondering, what is the future of that gathering? Like, what is the
  long-term fallout for us not being able to gather together and to feel
  that spirit that lives within each of us?
\item
  joy harjo\\
  I've been wondering about that too because I'm getting the sense that
  this is going to go on for a while. Oklahoma, of course, being a red
  state, it's pretty much open. And the deaths keep rising. So I wonder
  about that. I had a lot of gigs rescheduled to the fall, but I don't
  know that that will happen because we don't really know. And there's a
  lot of people here who are defiant, as there are in a lot of Southern
  states. And it's just going to get worse.
\item
  cheryl strayed\\
  Yeah, that's my fear too. So what's that been like? Even though the
  state decided to open up, what are you doing?
\item
  joy harjo\\
  I'm pretty much staying the way it is. I had to go out to Home Depot
  because a granddaughter sent me a birthday present of a lemon tree, so
  I had to go get potting soil. But I was all decked out with my face
  mask and everything. We'll make forays out to get groceries, and then
  I do a lot of online ordering, but mostly just staying in.
\item
  cheryl strayed\\
  So it was just your birthday. Your granddaughter gave you a lemon
  tree. How old are you?
\item
  joy harjo\\
  69.
\item
  cheryl strayed\\
  69, the last year of your 60s. So with this pandemic and all we hear
  about you essentially being in an at-risk age group, are you afraid
  for your safety and health?
\item
  joy harjo\\
  Well, at this point, I mean, of course. I am part of the demographic
  that is most vulnerable. I've also nearly died twice with pneumonia.
  But I figure my saxophone, playing saxophone has been my lung
  protector. And natives are one of the demographics that suffer the
  most from this. It's really taken a huge toll and a lot of deaths in
  native communities. So I get concerned, but I don't want to live in
  fear. I figure that my time is my time. And when you get to my age,
  almost every day you get a notice that somebody you know has passed.
  So there's the reality of that. We don't live forever. And at this
  point, you are looking back. I mean, I'm still making a life. I still
  feel like I haven't written my best work yet. I'm still in it. And at
  this point, it's a different kind of space. I love this kind of space,
  actually.
\item
  cheryl strayed\\
  This space in your life, you mean, where you feel like you have so
  much wisdom behind you and you're still moving into your best work?
\item
  joy harjo\\
  Well, I don't know about wisdom. At this point, you better have gained
  some wisdom or you're ---
\item
  {[}laughs{]}\\
  --- or something is very wrong. But I remember Meridel Le Sueur ---
\item
  cheryl strayed\\
  Yes.
\item
  joy harjo\\
  --- the incredible activist. And she was one of my mentors.
\item
  cheryl strayed\\
  Wow, was she, Joy? I just have to tell you, I grew up in Minnesota.
  And I met Meridel when I was just really a baby writer. And I was so
  honored to meet her in Minneapolis at that time. And I just love her.
  So I'm excited to hear that she was one of your mentors.
\item
  joy harjo\\
  Yes. And I met her when I was a baby writer too. In fact, my tiny,
  little chapbook had just come out. I think she asked me for it. She'd
  heard about it. But she became quite a mentor. But I remember her
  calling me in the `80s. She'd call me, and she was just ecstatic. And
  she says, I'm dying. And I said, no! I was so upset, but she was
  thrilled because she loved the process. She was aware that she was in
  the process of dying. And she probably died about eight years after
  that call and died with her pen in her hand. And she was working on
  three different novels.
\item
  cheryl strayed\\
  Wow. I think that's a really beautiful way to think about dying. And
  it's one that at least sort of mainstream American culture, I think,
  entirely rejects, that we're supposed to say no, no, no, no. You're
  not dying. You're not dying. You've got all this time left. And how
  interesting that she put her arms around it and said, I'm in the last
  era of my life. It seems like such a gracious and courageous way to
  accept our mortality.
\item
  joy harjo\\
  So I think about this, and sometimes I think people are taken before
  their time. And sometimes I think some of them are left way too long.
  I shouldn't say that.
\item
  {[}laughter{]}\\
  But this whole epidemic, I think it's --- I don't know. It feels so
  man-made to me. It feels like it's born of greed. And yet, each person
  has such an intricate and incredible life as stories and incredible
  stories. And to lose them before they're ripe or before their time is
  tragic.
\item
  cheryl strayed\\
  It is. And that can be --- we can lose somebody who's 80, and it's
  before their time, because of this virus. I want to --- so we're
  talking about death and the end of things, but I want to back up to
  birth. Where were you born, and when, and can you just tell me in
  brief the story of your life?
\item
  {[}laughter{]}
\item
  joy harjo\\
  That's funny. In brief --- I was born here in Tulsa, Oklahoma, which
  is the Creek Nation. It's part of the Muskogee Creek Nation. We're
  living on Muskogee Creek lands. And actually, I was dying as I was
  being born. I was kept alive on a ventilator, on a breathing
  apparatus. And my mother always told this story about how they said,
  well, we're going to take her off now. You can't keep --- she had to
  basically say yes to pull the plug.
\item
  cheryl strayed\\
  Wow.
\item
  joy harjo\\
  But I kept breathing. I know that I was --- I was arguing with who I
  call the council. And I was arguing, I don't know if I want to go
  there.
\item
  cheryl strayed\\
  You weren't saying I want to live. You were saying, I don't know. I
  don't know if I want to live.
\item
  joy harjo\\
  Yeah, I think I was --- you can't. So I came into this world at a
  certain time and place, and I've come to understand that generations
  come in together. We all have a purpose. So I came of age in the `60s.
  And then I left Oklahoma. I went one year of high school here, and
  then I went to Indian boarding school, which saved my life. Yes, it
  was a B.I.E. school, but not like a lot of them. It was an experiment
  in Indian education. It was in the `60s, the late `60s. It was in
  Santa Fe, all the arts and hippies going on there and the communes
  north of there. We had the best native artists teaching us. And for
  the first time, I was in a circle of relatives, a circle of other
  students like me who were natives, who had been through a lot of the
  same stories. And we were making art.
\item
  cheryl strayed\\
  And you were actually getting an education that centered native
  history and Native Americans rather than white America?
\item
  joy harjo\\
  For the most part. Now, what happened at that school too is that it
  was still a Bureau of Indian Affairs school, and we still had a lot of
  the military nomenclature and so on. For the most part, it was a
  wonderful education, and we did have Indian studies. And we didn't
  have that in Oklahoma. Oklahoma has one of the largest numbers of
  tribal groups and people. And there was, I think, one day where they
  were mentioned in Oklahoma state history, and that's about it.
\item
  cheryl strayed\\
  So that changed your life. Was it just a year at that high school, or
  did you graduate?
\item
  joy harjo\\
  I did graduate. It was just a year, but it was a good thing they put
  me ahead a year because I had enough credits. So it's a good thing I
  got to graduate basically as a junior because I wound up pregnant, a
  teenage mother. And at the same time, I went on the road that June
  after graduation. We were one of the first all-native drama and dance
  troupes up into the Pacific Northwest. And no one knew I was pregnant,
  and I came back to Oklahoma. Yeah.
\item
  cheryl strayed\\
  So how many children do you have?
\item
  joy harjo\\
  That's a good question.
\item
  {[}laughter{]}\\
  I have two I gave birth to. I have a stepdaughter I've had since the
  beginning. She's my son, who is my oldest child, that's his
  half-sister. She's older than him. And then I have a stepson, who is
  my daughter's half-brother. And then I have, actually, another child
  that I was his guardian for a while --- my daughter's boyfriend, the
  father of her first child, who I took guardianship of at one point.
  And then I have five stepchildren to my husband.
\item
  cheryl strayed\\
  OK, so now I know why you answered that way, because I've lost count
  too.
\item
  {[}laughter{]}
\item
  joy harjo\\
  I do too.
\item
  cheryl strayed\\
  So you have a bunch of children, and you're also a grandmother and a
  great-grandmother. Is that right?
\item
  joy harjo\\
  Yes, that's right.
\item
  cheryl strayed\\
  How old were you when you became a grandmother?
\item
  joy harjo\\
  I was in my 30s when Krista was born. And it's interesting that you
  notice that at different points along the way, certain things become
  unlocked. And it was then that I started writing stories, vignettes,
  and so on, was when she was born. It was almost like when my
  daughter's milk came in, the need to tell stories opened up.
\item
  cheryl strayed\\
  And when you began writing, how did you --- I don't want to use the
  word ``decide'' because I already know the answer, that we don't
  really decide necessarily to become writers. But what was your
  experience of this call, or this ache, or yearning to write?
\item
  joy harjo\\
  It surprised me because I didn't grow up --- probably everybody else
  you talk to who's a writer will say they were compelled. I read. I
  read extensively --- fiction, nonfiction, poetry. But I didn't see
  myself necessarily as a poet. But then poetry emerged with native
  rights movements. I went to a lot of political meetings, and protests,
  and things and rarely heard the voices of women. I didn't think of any
  of that consciously. I just started writing poetry out of the lives of
  native women and going on in the area of what we saw. And a lot of
  people think poets only write from their own experience, but that's
  not necessarily so. We research. And that's how poetry started for me.
  And then it just took hold. I mean, there was no way I could say no. I
  began to think of it like a calling. I knew what it was like to say no
  to the council.

  It surprised me. And you can try to run away from it, and you can try
  to duck out, but you suffer.
\item
  cheryl strayed\\
  You mentioned, so the council said, Joy Harjo, you have to enter this
  world. The council said, Joy Harjo, you have to be a poet. Can you
  explain to me in more detail what's the council? What does that mean
  to you?
\item
  joy harjo\\
  Well, it's kind of common sense. We come in. We have a family. And I
  think that there's also in the spiritual realm, we have guardians that
  look over us. We're part of a whole system that doesn't just end at
  birth or end at death. It's those caretakers or the guardians. And you
  can always say yes or no, and nobody's trying to force --- it's not
  about forcing you to do anything. It's about what you --- it's how you
  grow your spirit in what you were given to do.
\item
  cheryl strayed\\
  So you were given to live and write poetry. And now here you are, the
  U.S. Poet Laureate. Tell the story of how you found out and what
  meaning you've made of having this really, I think, important job.
\item
  joy harjo\\
  Well, it is quite an interesting story, I mean, how I got here.
  Because I think as a native woman poet --- I've seen this over and
  over. I think that often in a group of literati, we're often
  disregarded, or we're ghettoized. There's something about hearing the
  voice of a contemporary native that says subconsciously, oh, oh, no.
  They're still here. This is stolen land. We're going to have to deal
  with history. We don't want to deal with --- let's not go there.

  When I got the call --- and I remember --- I know Rob Casper at the
  Poetry and Literature Center at the Library of Congress because he's
  out and about, and wonderful guy, and runs the National Book Festival.
  And I had a new book coming out ``An American Sunrise.'' So I figured
  he was calling about that, but I didn't even hear him say hello. He
  says, you're on speakerphone, and we have the head librarian, Carla
  Hayden, who wants to speak with you. And then she asked if I would be
  the 23rd U.S. Poet Laureate.
\item
  cheryl strayed\\
  Wow.
\item
  joy harjo\\
  Felt like lightning. It did. It literally felt like lightning moving
  through my system. And the first thing that went through my mind was
  responsibility.
\item
  {[}laughs{]}
\item
  cheryl strayed\\
  Yeah.
\item
  joy harjo\\
  Actually thinking I can't even represent native poets. I can't even
  represent Muskogee Creek poets. There are so many of us. And you think
  of American poets and all of American poetry. But what an honor. I
  mean, it's meant a tremendous amount to natives.
\item
  cheryl strayed\\
  Of course, yes. So I think you have a poem to read me, since we're
  talking about the power of poetry and your poetry in particular. I'd
  love to hear something.
\item
  joy harjo\\
  Yeah. This is one of the earliest poems I wrote. And I've begun to
  think that a lot of these poems have come to me because they're coming
  through me. And then I have to do my part. I have to bring out my
  hammer and nails, and build a place for them to live. So this one came
  when I desperately needed it. It's called ``I Give You Back.'' And
  it's helpful, I think, during this time because it's to get rid of
  fear. And we're in a pandemic, something we've never been in before,
  in a time like the times we're in now. And what does that mean? And
  what's going to happen to us? So this poem is to get rid of fear. I
  think it comes out of the tribal tradition of writing poems to be
  useful to go out into the world --- OK, poem you have work to do. And
  you have to go out and help people not be afraid.
\item
  cheryl strayed\\
  Yeah. I love a working poem. I think that's why those of us who love
  poetry turn to it in that way. That's what people mean when they say
  that poem saved my life or that book saved my life, right?
\item
  joy harjo\\
  Yes. And this poem --- I don't know that it's taught so much now, but
  there was a period when I got a note or a letter from somebody so
  frequently saying, this poem, I carry it with me. It saved my life.
  And it probably saved my life too. It's called ``I Give You Back,'' or
  it's come to be known as the fear poem. ``I release you, my beautiful
  and terrible fear. I release you. You are my beloved and hated twin,
  but now I don't know you as myself. I release you with all the pain I
  would know at the death of my children. You are not my blood anymore.
  I give you back to the soldiers who burned down my home, beheaded my
  children, raped and sodomized my brothers and sisters. I give you back
  to those who stole the food from our plates when we were starving. I
  release you, fear, because you hold these things in front of me, and I
  was born with eyes that can never close. I release you. I release you.
  I release you. I release you. I am not afraid to be angry. I am not
  afraid to rejoice. I am not afraid to be black. I am not afraid to be
  white. I am not afraid to be hungry. I am not afraid to be full. I am
  not afraid to be hated. I am not afraid to be loved, to be loved, to
  be loved. Fear, oh, you have choked me, but I gave you the leash. You
  have gutted me, but I gave you the knife. You have devoured me, but I
  laid myself across the fire. I take myself back, fear. You are not my
  shadow any longer. I won't hold you in my hands. You can't live in my
  eyes, my ears, my voice, my belly, or in my heart, my heart, my heart,
  my heart, my heart. But come here, fear. I am alive. And you are so
  afraid of dying.
\item
  cheryl strayed\\
  Joy, that is such a powerful poem, so many powerful things about it.
  But I think for me the most powerful is, ``You have choked me, but I
  gave you the leash. You have gutted me, but I gave you the knife. You
  have devoured me, but I laid myself across the fire.''

  I think that that's such a fascinating and true turn, that when we're
  ruled by fear, or hate, or any of those --- anger, in some ways we're
  allowing ourselves to be. We're giving ourselves to that. And you turn
  it around in this poem, right?
\item
  joy harjo\\
  Right. And there are so many sources of planting fear going on right
  now.
\item
  cheryl strayed\\
  Yeah. Yeah. It's such an empowering poem, I mean, even the title, ``I
  Give You Back.'' We have the power to turn away from those things in
  ourselves. Do you remember what exactly provoked you to write this
  poem, what situation you were in?
\item
  joy harjo\\
  Well, man, I was in a lot of situations.
\item
  {[}laughter{]}
\item
  cheryl strayed\\
  Weren't we all?
\item
  joy harjo\\
  Yeah. I was good at bad situations.
\item
  cheryl strayed\\
  I love that.
\item
  joy harjo\\
  I've been trying to write about it because there was a period where I
  was in a lot of stress. I was trying to graduate from school. I was
  dealing with a partner who I had kicked out, who would come and break
  in the house and would say he was going to kill me, and calling
  police. And yet, I was getting straight As at school and raising two
  children.
\item
  cheryl strayed\\
  This was when you were a college student at the University of New
  Mexico?
\item
  joy harjo\\
  Yes. And then what started happening is I would lay there to go to
  sleep. And I would close my eyes, and there would be these demonic
  figures reaching out to get me. And I would yell and get myself up.
  And then I would keep the lights on. Just keeps them away because they
  don't like light. And then my spirit --- we all have guardians ---
  told me, says, speed up your energy, because light is a high energy.
  It has a fast speed to it, faster speed than something evil, which has
  a density to it. And the last time I saw them was when that spirit
  told me, speed it up. Speed up your energy. And then I saw them fall
  away.
\item
  cheryl strayed\\
  Wow.
\item
  joy harjo\\
  So this poem came out of that time.
\item
  cheryl strayed\\
  So you're writing a memoir right now, and you're reflecting in
  particular on the teachers you've had throughout your life, people who
  inspired you, but also people who taught you lessons that were
  painful. I'm wondering if you can tell me a bit about the kinds of
  teachers you're writing about in this memoir you're working on.
\item
  joy harjo\\
  Yeah. So I'm writing about the stepfather who was --- I tried to
  understand him when he passed. And my mother told me --- because he
  was being very violent to his children. And he would do things so that
  people wouldn't see. He was very twisted in his behavior. And I
  remember going to my mother and saying, please, please, we're fine by
  ourselves, after I saw him beat my sister, a toddler, and hold her up
  and beat her. And she was a baby. And she told me I was her confidant.
  She said, he will kill all of us. He has threatened he will kill me
  and all of us if we leave. I asked my sister, because I was writing
  this memoir. I called her up and said, did you know this? And she
  said, no. She said she knew that he had threatened to kill her. In
  fact, when I came back one night he had made her play Russian roulette
  in front of them. She didn't know that he threatened to kill all of
  us, but I went through from being nine years old through leaving for
  Indian school knowing that he would kill all of us. So I work on this.
  I don't know that I came up with an answer, but I tried to understand.
  And this is raw. I'll read it. It's the last paragraph of this
  particular story. ``I did not ask for him and do not want his story
  here with mine, even now. Perhaps he was one of my greatest teachers.
  Because of him, I learned to find myself in the spiritual world. In
  that realm there lived an immense house of imagination through which I
  could walk and immerse myself without fear. In that house, I met and
  spoke with my ancestors who had gone on, but came back when we needed
  their assistance. I found the ability to construct dreams with many
  kinds of materials. I saw the future. I saw the past. I battled
  monsters and sat with them at the table to hear their stories.
  Everyone has a story.''
\item
  cheryl strayed\\
  Joy, that's so powerful. And I think, gosh --- I mean, I have a father
  like the stepfather you describe. And I've had to come to those terms.
  What you put so beautifully and powerfully in that last paragraph that
  when you're pushed in the direction of that kind of sorrow, and fear,
  and rage, and pain, that ultimately those people do force us to find a
  way that we wouldn't have otherwise. And it's a complicated thing,
  isn't it, in some ways to say thank you. Thank you for that horror.
  Thank you for that difficult lesson. And yet to do it is, I think, the
  most empowering act of all. It's in some ways what you're saying in
  your poem.
\item
  joy harjo\\
  Yes. You wouldn't have your work. You wouldn't have ``Wild,'' or your
  novel, or anything else without that.
\item
  cheryl strayed\\
  That's really true.
\item
  joy harjo\\
  Uh-huh.
\item
  cheryl strayed\\
  Yeah.
\item
  joy harjo\\
  I wish I could remember his name. Because I've been around a lot of
  really wise people, and I always like to sit in those circles of wise
  native people that would come together. And they predicted these
  times. But I remember this man saying that a teacher loves the most,
  they test the most. Sometimes I laughed and said, well, I guess native
  people are the most beloved.
\item
  {[}laughter{]}\\
  And I remember saying that once at a school, a native school. And some
  kid wrote me after that and said, that changed my life.
\item
  cheryl strayed\\
  Oh, wow. Yeah.
\item
  joy harjo\\
  Because we can all get bent by feeling like we are being tested
  because someone is after us or someone hates us. But sometimes it is
  because we're the most beloved. And then I wonder about this country
  and about this earth, because this earth is a planetary being. And
  we're being tested right now.
\item
  cheryl strayed\\
  We are.
\item
  joy harjo\\
  Yes, we're being tested. And I think, OK, so what are the parameters
  of the test, and how is this going to play out, and how do we as
  artists and poets subsist in this world pandemic storytelling so that
  what is nourishing in the culture survives? Because we have been going
  on for a long time with a lot of false stories. You think about music
  created only for money, or stories created to make money, or drugs
  created to make money. That whole paradigm, it's not taking care of
  people or the earth.
\item
  cheryl strayed\\
  This pandemic has certainly laid bare that truth that was already
  apparent to many, but I think now we see it more clearly. You said
  earlier that wise native elders had predicted these times. Would you
  talk to me about that?
\item
  joy harjo\\
  Yes. It's been in the prophecies. They all agreed that these times
  were coming, the times that there would be droughts, not enough food,
  that there were going to be storms, powerful storms, pandemics, and so
  on unless people changed their ways. And that's really the council. I
  mean, if you think of the council --- the council is always here.
  There's an earth council. I mean, they're the ones that uphold those
  universal truths that are true for everyone. It's about compassion, or
  what they call it in Muskogee --- we call it {[}NON-ENGLISH{]}. It's
  about understanding everyone as a beloved person, essentially.
\item
  cheryl strayed\\
  That's the way. It's the way forward, it seems, if we can get there.
  If we were ever there, I don't know. But maybe that's where we can
  think about going.
\item
  joy harjo\\
  I think we have moments of being there. We have all had moments in our
  lives of being there when it comes together, and we feel --- we know
  what that feels like. And it's often in a moment of compassion. I
  remember being really poor as a young mother. And it was so cool that
  people would share.
\item
  cheryl strayed\\
  Yeah.
\item
  joy harjo\\
  And there was that sense of sharing and compassion that moves through
  then at those moments or when you're looking out the window. You're
  cleaning the kitchen or washing dishes even, or rinsing them and
  looking out the window, and that light hits at that angle for dusk
  when dusk happens, and that beautiful glow in the light. And there's
  often a moment in those times where you know that you feel what that
  kind of world is like. Or when a child --- you hold a newborn child,
  and they're still fresh. They are fresh from the other world, and they
  carry that knowledge, and they still remember beauty.
\item
  cheryl strayed\\
  Yeah. Well, it seems to me you never forgot it, Joy.
\item
  joy harjo\\
  Well, it motivates me. I haven't been given up on yet.
\item
  {[}laughter{]}
\item
  cheryl strayed\\
  Me either. I'm with you in the beauty club. I try to see it every day.
\item
  joy harjo\\
  Yes.
\item
  cheryl strayed\\
  So I cannot tell you just how much heart it gives me to know that you
  are out there across the land in Oklahoma, writing that next book and
  being the beautiful truth teller, and powerhouse writer and U.S. Poet
  Laureate that you are.
\item
  joy harjo\\
  Well, thank you. I loved getting a chance to visit with you. And I do
  miss --- I always remember us getting together and having a moment in
  that wonderful, beautiful, incredible sea of human beings there to
  celebrate writing. And I look forward to being able to do that again.
\item
  cheryl strayed\\
  Me too. We will meet again, Joy. And we will whisper up close with
  each other as we did that day just a year ago. Be well.
\item
  joy harjo\\
  OK, you too.
\item
  cheryl strayed\\
  Bye.
\item
  joy harjo\\
  Bye.
\item
  {[}music playing{]}
\item
  cheryl strayed\\
  I'm Cheryl Strayed. This is ``Sugar Calling.'' When I conceived
  ``Sugar Calling,'' I thought of it as only a pop-up podcast. It would
  be about eight episodes long, and by then we'd be back to normal life.
  But here we are, and life is getting ever more not normal. I've had
  such a warm response to the podcast. And I've frankly had so much fun
  doing it that I've decided to go on --- not right now, but sometime
  down the line I'll be back to make more calls and to share them with
  you. Until then, stay safe. Do good. Be well. I love you guys. Bye.
\item
  {[}music playing{]}
\end{itemize}

\href{https://www.nytimes3xbfgragh.onion/column/sugar-calling}{\includegraphics{https://static01.graylady3jvrrxbe.onion/images/2020/04/29/podcasts/sugar-calling-album-art/sugar-calling-album-art-square320.jpg}Sugar
Calling}Subscribe:

\begin{itemize}
\tightlist
\item
  \href{https://itunes.apple.com/us/podcast/id1505881384}{Apple
  Podcasts}
\item
  \href{https://podcasts.google.com/?feed=aHR0cHM6Ly9yc3MuYXJ0MTkuY29tL3N1Z2FyLWNhbGxpbmc\&ved=0CAUQrrcFahcKEwjA8Kyn09voAhUAAAAAHQAAAAAQBQ}{Google
  Podcasts}
\end{itemize}

\hypertarget{i-release-you-fear-1}{%
\section{`I Release You, Fear'}\label{i-release-you-fear-1}}

\hypertarget{cheryl-strayed-talks-with-the-poet-joy-harjo-about-beauty-prophecies-and-listening-to-your-spiritual-council-1}{%
\subsection{Cheryl Strayed talks with the poet Joy Harjo about beauty,
prophecies and listening to your spiritual
council.}\label{cheryl-strayed-talks-with-the-poet-joy-harjo-about-beauty-prophecies-and-listening-to-your-spiritual-council-1}}

Hosted by Cheryl Strayed, produced by Kelly Prime and edited by Sara
Sarasohn. Editorial oversight by Wendy Dorr.

Transcript

transcript

Back to Sugar Calling

bars

0:00/35:30

-0:00

transcript

\hypertarget{i-release-you-fear-2}{%
\subsection{`I Release You, Fear'}\label{i-release-you-fear-2}}

\hypertarget{hosted-by-cheryl-strayed-produced-by-kelly-prime-and-edited-by-sara-sarasohn-editorial-oversight-by-wendy-dorr-1}{%
\subsubsection{Hosted by Cheryl Strayed, produced by Kelly Prime and
edited by Sara Sarasohn. Editorial oversight by Wendy
Dorr.}\label{hosted-by-cheryl-strayed-produced-by-kelly-prime-and-edited-by-sara-sarasohn-editorial-oversight-by-wendy-dorr-1}}

\hypertarget{cheryl-strayed-talks-with-the-poet-joy-harjo-about-beauty-prophecies-and-listening-to-your-spiritual-council-2}{%
\paragraph{Cheryl Strayed talks with the poet Joy Harjo about beauty,
prophecies and listening to your spiritual
council.}\label{cheryl-strayed-talks-with-the-poet-joy-harjo-about-beauty-prophecies-and-listening-to-your-spiritual-council-2}}

Wednesday, May 20th, 2020

\begin{itemize}
\item
  cheryl strayed\\
  I'm going to call Joy Harjo. Joy is a poet, musician, playwright, and
  author. She's a member of the Muskogee Creek Nation and our current
  U.S. Poet Laureate. I first discovered Joy in my early 20s when I was
  an undergraduate at the University of Minnesota. And I was in this
  little bookstore near campus, and I saw this book called ``She Had
  Some Horses.'' I was drawn to it because I grew up having horses too.
  That book was one of Joy's beautiful poetry collections. When I read
  it, I knew that I had found a writer I would follow anywhere, and I
  have through her many books of poems and also her beautiful memoir,
  ``Crazy Brave.'' She offers us insight and truth that feels like she's
  working in this place that's sort of beyond knowing or beyond
  explanation, and yet it rings the deepest bell of truth within me. I'm
  going to give Joy a call.
\item
  {[}music playing{]}
\item
  {[}dial tone{]}
\item
  joy harjo\\
  Hi, Cheryl.
\item
  cheryl strayed\\
  Hi, Joy.
\item
  joy harjo\\
  How are you doing?
\item
  cheryl strayed\\
  I'm good. It's so nice to talk to you. Where are you right now?
\item
  joy harjo\\
  I'm in Tulsa, Oklahoma.
\item
  cheryl strayed\\
  Uh-huh.
\item
  joy harjo\\
  I have a Tulsa Artist Fellowship, and they give you housing. So my
  husband and I are living in a two-bedroom apartment here in the Arts
  District downtown. And we have land south of here, so he takes off
  every day there because we eventually want to build a house. And I'm
  holed up here. And I know how to be here. I mean, that's --- we
  writers --- it's the space I've always occupied primarily.
\item
  cheryl strayed\\
  Yeah, yeah. I mean, we need to be alone to write after all. But I have
  to say, Joy, now that it's been a couple months, it's getting old for
  me. I miss seeing people. I don't always want to be socially distant.
  In fact, I was remembering a day when we were not so socially distant.
  The last time I saw you in Portland, we were at a writer's conference.
\item
  joy harjo\\
  Yes, I remember that.
\item
  cheryl strayed\\
  And I had this clear image of whispering with you. We were at the sort
  of back of the room when somebody was giving a talk, and I realized I
  missed being up close like that with friends and people who weren't
  family members.
\item
  joy harjo\\
  Yes, it's starting to wear a little bit. Because as much as I'm a
  person that requires lots of solitude, I also am very social. And I
  miss that. I perform a lot. I go out a lot. And this online stuff just
  doesn't quite do the trick. I did a reading. I'm not saying yes to a
  lot of these, but I had one organization out of three months worth of
  performances agreed to let me do a livestream. And so I said sure. And
  I did a reading. I did a whole reading, sitting at my little table.
  And I guess they had several hundred people on. So I asked my spirit.
  I said, OK, I'm about to do this. I always ask for help and that what
  comes through needs to, et cetera. And I said, please help me feel and
  get a sense of the people. Because we all, any of us that get up there
  and do this, we feel the people and feel the energy and what needs to
  be said. And so I asked that. And so I started to feel them, but it's
  not the same.
\item
  cheryl strayed\\
  No.
\item
  joy harjo\\
  It's nothing like being with living, breathing bodies, everyone
  together, listening together, and being together. It's just not the
  same thing.
\item
  cheryl strayed\\
  No, no. Like you, I do a lot of events where there are a lot of bodies
  gathered together. And often, I think the work that both of us do ---
  and you and I tend to write about matters of the heart, and the
  spirit, and the soul --- there's that kind of healing aspect. And I
  think that especially, there is this something that you feel, like the
  magic, or the vibration, or whatever it is in the room. And I'm
  wondering, what is the future of that gathering? Like, what is the
  long-term fallout for us not being able to gather together and to feel
  that spirit that lives within each of us?
\item
  joy harjo\\
  I've been wondering about that too because I'm getting the sense that
  this is going to go on for a while. Oklahoma, of course, being a red
  state, it's pretty much open. And the deaths keep rising. So I wonder
  about that. I had a lot of gigs rescheduled to the fall, but I don't
  know that that will happen because we don't really know. And there's a
  lot of people here who are defiant, as there are in a lot of Southern
  states. And it's just going to get worse.
\item
  cheryl strayed\\
  Yeah, that's my fear too. So what's that been like? Even though the
  state decided to open up, what are you doing?
\item
  joy harjo\\
  I'm pretty much staying the way it is. I had to go out to Home Depot
  because a granddaughter sent me a birthday present of a lemon tree, so
  I had to go get potting soil. But I was all decked out with my face
  mask and everything. We'll make forays out to get groceries, and then
  I do a lot of online ordering, but mostly just staying in.
\item
  cheryl strayed\\
  So it was just your birthday. Your granddaughter gave you a lemon
  tree. How old are you?
\item
  joy harjo\\
  69.
\item
  cheryl strayed\\
  69, the last year of your 60s. So with this pandemic and all we hear
  about you essentially being in an at-risk age group, are you afraid
  for your safety and health?
\item
  joy harjo\\
  Well, at this point, I mean, of course. I am part of the demographic
  that is most vulnerable. I've also nearly died twice with pneumonia.
  But I figure my saxophone, playing saxophone has been my lung
  protector. And natives are one of the demographics that suffer the
  most from this. It's really taken a huge toll and a lot of deaths in
  native communities. So I get concerned, but I don't want to live in
  fear. I figure that my time is my time. And when you get to my age,
  almost every day you get a notice that somebody you know has passed.
  So there's the reality of that. We don't live forever. And at this
  point, you are looking back. I mean, I'm still making a life. I still
  feel like I haven't written my best work yet. I'm still in it. And at
  this point, it's a different kind of space. I love this kind of space,
  actually.
\item
  cheryl strayed\\
  This space in your life, you mean, where you feel like you have so
  much wisdom behind you and you're still moving into your best work?
\item
  joy harjo\\
  Well, I don't know about wisdom. At this point, you better have gained
  some wisdom or you're ---
\item
  {[}laughs{]}\\
  --- or something is very wrong. But I remember Meridel Le Sueur ---
\item
  cheryl strayed\\
  Yes.
\item
  joy harjo\\
  --- the incredible activist. And she was one of my mentors.
\item
  cheryl strayed\\
  Wow, was she, Joy? I just have to tell you, I grew up in Minnesota.
  And I met Meridel when I was just really a baby writer. And I was so
  honored to meet her in Minneapolis at that time. And I just love her.
  So I'm excited to hear that she was one of your mentors.
\item
  joy harjo\\
  Yes. And I met her when I was a baby writer too. In fact, my tiny,
  little chapbook had just come out. I think she asked me for it. She'd
  heard about it. But she became quite a mentor. But I remember her
  calling me in the `80s. She'd call me, and she was just ecstatic. And
  she says, I'm dying. And I said, no! I was so upset, but she was
  thrilled because she loved the process. She was aware that she was in
  the process of dying. And she probably died about eight years after
  that call and died with her pen in her hand. And she was working on
  three different novels.
\item
  cheryl strayed\\
  Wow. I think that's a really beautiful way to think about dying. And
  it's one that at least sort of mainstream American culture, I think,
  entirely rejects, that we're supposed to say no, no, no, no. You're
  not dying. You're not dying. You've got all this time left. And how
  interesting that she put her arms around it and said, I'm in the last
  era of my life. It seems like such a gracious and courageous way to
  accept our mortality.
\item
  joy harjo\\
  So I think about this, and sometimes I think people are taken before
  their time. And sometimes I think some of them are left way too long.
  I shouldn't say that.
\item
  {[}laughter{]}\\
  But this whole epidemic, I think it's --- I don't know. It feels so
  man-made to me. It feels like it's born of greed. And yet, each person
  has such an intricate and incredible life as stories and incredible
  stories. And to lose them before they're ripe or before their time is
  tragic.
\item
  cheryl strayed\\
  It is. And that can be --- we can lose somebody who's 80, and it's
  before their time, because of this virus. I want to --- so we're
  talking about death and the end of things, but I want to back up to
  birth. Where were you born, and when, and can you just tell me in
  brief the story of your life?
\item
  {[}laughter{]}
\item
  joy harjo\\
  That's funny. In brief --- I was born here in Tulsa, Oklahoma, which
  is the Creek Nation. It's part of the Muskogee Creek Nation. We're
  living on Muskogee Creek lands. And actually, I was dying as I was
  being born. I was kept alive on a ventilator, on a breathing
  apparatus. And my mother always told this story about how they said,
  well, we're going to take her off now. You can't keep --- she had to
  basically say yes to pull the plug.
\item
  cheryl strayed\\
  Wow.
\item
  joy harjo\\
  But I kept breathing. I know that I was --- I was arguing with who I
  call the council. And I was arguing, I don't know if I want to go
  there.
\item
  cheryl strayed\\
  You weren't saying I want to live. You were saying, I don't know. I
  don't know if I want to live.
\item
  joy harjo\\
  Yeah, I think I was --- you can't. So I came into this world at a
  certain time and place, and I've come to understand that generations
  come in together. We all have a purpose. So I came of age in the `60s.
  And then I left Oklahoma. I went one year of high school here, and
  then I went to Indian boarding school, which saved my life. Yes, it
  was a B.I.E. school, but not like a lot of them. It was an experiment
  in Indian education. It was in the `60s, the late `60s. It was in
  Santa Fe, all the arts and hippies going on there and the communes
  north of there. We had the best native artists teaching us. And for
  the first time, I was in a circle of relatives, a circle of other
  students like me who were natives, who had been through a lot of the
  same stories. And we were making art.
\item
  cheryl strayed\\
  And you were actually getting an education that centered native
  history and Native Americans rather than white America?
\item
  joy harjo\\
  For the most part. Now, what happened at that school too is that it
  was still a Bureau of Indian Affairs school, and we still had a lot of
  the military nomenclature and so on. For the most part, it was a
  wonderful education, and we did have Indian studies. And we didn't
  have that in Oklahoma. Oklahoma has one of the largest numbers of
  tribal groups and people. And there was, I think, one day where they
  were mentioned in Oklahoma state history, and that's about it.
\item
  cheryl strayed\\
  So that changed your life. Was it just a year at that high school, or
  did you graduate?
\item
  joy harjo\\
  I did graduate. It was just a year, but it was a good thing they put
  me ahead a year because I had enough credits. So it's a good thing I
  got to graduate basically as a junior because I wound up pregnant, a
  teenage mother. And at the same time, I went on the road that June
  after graduation. We were one of the first all-native drama and dance
  troupes up into the Pacific Northwest. And no one knew I was pregnant,
  and I came back to Oklahoma. Yeah.
\item
  cheryl strayed\\
  So how many children do you have?
\item
  joy harjo\\
  That's a good question.
\item
  {[}laughter{]}\\
  I have two I gave birth to. I have a stepdaughter I've had since the
  beginning. She's my son, who is my oldest child, that's his
  half-sister. She's older than him. And then I have a stepson, who is
  my daughter's half-brother. And then I have, actually, another child
  that I was his guardian for a while --- my daughter's boyfriend, the
  father of her first child, who I took guardianship of at one point.
  And then I have five stepchildren to my husband.
\item
  cheryl strayed\\
  OK, so now I know why you answered that way, because I've lost count
  too.
\item
  {[}laughter{]}
\item
  joy harjo\\
  I do too.
\item
  cheryl strayed\\
  So you have a bunch of children, and you're also a grandmother and a
  great-grandmother. Is that right?
\item
  joy harjo\\
  Yes, that's right.
\item
  cheryl strayed\\
  How old were you when you became a grandmother?
\item
  joy harjo\\
  I was in my 30s when Krista was born. And it's interesting that you
  notice that at different points along the way, certain things become
  unlocked. And it was then that I started writing stories, vignettes,
  and so on, was when she was born. It was almost like when my
  daughter's milk came in, the need to tell stories opened up.
\item
  cheryl strayed\\
  And when you began writing, how did you --- I don't want to use the
  word ``decide'' because I already know the answer, that we don't
  really decide necessarily to become writers. But what was your
  experience of this call, or this ache, or yearning to write?
\item
  joy harjo\\
  It surprised me because I didn't grow up --- probably everybody else
  you talk to who's a writer will say they were compelled. I read. I
  read extensively --- fiction, nonfiction, poetry. But I didn't see
  myself necessarily as a poet. But then poetry emerged with native
  rights movements. I went to a lot of political meetings, and protests,
  and things and rarely heard the voices of women. I didn't think of any
  of that consciously. I just started writing poetry out of the lives of
  native women and going on in the area of what we saw. And a lot of
  people think poets only write from their own experience, but that's
  not necessarily so. We research. And that's how poetry started for me.
  And then it just took hold. I mean, there was no way I could say no. I
  began to think of it like a calling. I knew what it was like to say no
  to the council.

  It surprised me. And you can try to run away from it, and you can try
  to duck out, but you suffer.
\item
  cheryl strayed\\
  You mentioned, so the council said, Joy Harjo, you have to enter this
  world. The council said, Joy Harjo, you have to be a poet. Can you
  explain to me in more detail what's the council? What does that mean
  to you?
\item
  joy harjo\\
  Well, it's kind of common sense. We come in. We have a family. And I
  think that there's also in the spiritual realm, we have guardians that
  look over us. We're part of a whole system that doesn't just end at
  birth or end at death. It's those caretakers or the guardians. And you
  can always say yes or no, and nobody's trying to force --- it's not
  about forcing you to do anything. It's about what you --- it's how you
  grow your spirit in what you were given to do.
\item
  cheryl strayed\\
  So you were given to live and write poetry. And now here you are, the
  U.S. Poet Laureate. Tell the story of how you found out and what
  meaning you've made of having this really, I think, important job.
\item
  joy harjo\\
  Well, it is quite an interesting story, I mean, how I got here.
  Because I think as a native woman poet --- I've seen this over and
  over. I think that often in a group of literati, we're often
  disregarded, or we're ghettoized. There's something about hearing the
  voice of a contemporary native that says subconsciously, oh, oh, no.
  They're still here. This is stolen land. We're going to have to deal
  with history. We don't want to deal with --- let's not go there.

  When I got the call --- and I remember --- I know Rob Casper at the
  Poetry and Literature Center at the Library of Congress because he's
  out and about, and wonderful guy, and runs the National Book Festival.
  And I had a new book coming out ``An American Sunrise.'' So I figured
  he was calling about that, but I didn't even hear him say hello. He
  says, you're on speakerphone, and we have the head librarian, Carla
  Hayden, who wants to speak with you. And then she asked if I would be
  the 23rd U.S. Poet Laureate.
\item
  cheryl strayed\\
  Wow.
\item
  joy harjo\\
  Felt like lightning. It did. It literally felt like lightning moving
  through my system. And the first thing that went through my mind was
  responsibility.
\item
  {[}laughs{]}
\item
  cheryl strayed\\
  Yeah.
\item
  joy harjo\\
  Actually thinking I can't even represent native poets. I can't even
  represent Muskogee Creek poets. There are so many of us. And you think
  of American poets and all of American poetry. But what an honor. I
  mean, it's meant a tremendous amount to natives.
\item
  cheryl strayed\\
  Of course, yes. So I think you have a poem to read me, since we're
  talking about the power of poetry and your poetry in particular. I'd
  love to hear something.
\item
  joy harjo\\
  Yeah. This is one of the earliest poems I wrote. And I've begun to
  think that a lot of these poems have come to me because they're coming
  through me. And then I have to do my part. I have to bring out my
  hammer and nails, and build a place for them to live. So this one came
  when I desperately needed it. It's called ``I Give You Back.'' And
  it's helpful, I think, during this time because it's to get rid of
  fear. And we're in a pandemic, something we've never been in before,
  in a time like the times we're in now. And what does that mean? And
  what's going to happen to us? So this poem is to get rid of fear. I
  think it comes out of the tribal tradition of writing poems to be
  useful to go out into the world --- OK, poem you have work to do. And
  you have to go out and help people not be afraid.
\item
  cheryl strayed\\
  Yeah. I love a working poem. I think that's why those of us who love
  poetry turn to it in that way. That's what people mean when they say
  that poem saved my life or that book saved my life, right?
\item
  joy harjo\\
  Yes. And this poem --- I don't know that it's taught so much now, but
  there was a period when I got a note or a letter from somebody so
  frequently saying, this poem, I carry it with me. It saved my life.
  And it probably saved my life too. It's called ``I Give You Back,'' or
  it's come to be known as the fear poem. ``I release you, my beautiful
  and terrible fear. I release you. You are my beloved and hated twin,
  but now I don't know you as myself. I release you with all the pain I
  would know at the death of my children. You are not my blood anymore.
  I give you back to the soldiers who burned down my home, beheaded my
  children, raped and sodomized my brothers and sisters. I give you back
  to those who stole the food from our plates when we were starving. I
  release you, fear, because you hold these things in front of me, and I
  was born with eyes that can never close. I release you. I release you.
  I release you. I release you. I am not afraid to be angry. I am not
  afraid to rejoice. I am not afraid to be black. I am not afraid to be
  white. I am not afraid to be hungry. I am not afraid to be full. I am
  not afraid to be hated. I am not afraid to be loved, to be loved, to
  be loved. Fear, oh, you have choked me, but I gave you the leash. You
  have gutted me, but I gave you the knife. You have devoured me, but I
  laid myself across the fire. I take myself back, fear. You are not my
  shadow any longer. I won't hold you in my hands. You can't live in my
  eyes, my ears, my voice, my belly, or in my heart, my heart, my heart,
  my heart, my heart. But come here, fear. I am alive. And you are so
  afraid of dying.
\item
  cheryl strayed\\
  Joy, that is such a powerful poem, so many powerful things about it.
  But I think for me the most powerful is, ``You have choked me, but I
  gave you the leash. You have gutted me, but I gave you the knife. You
  have devoured me, but I laid myself across the fire.''

  I think that that's such a fascinating and true turn, that when we're
  ruled by fear, or hate, or any of those --- anger, in some ways we're
  allowing ourselves to be. We're giving ourselves to that. And you turn
  it around in this poem, right?
\item
  joy harjo\\
  Right. And there are so many sources of planting fear going on right
  now.
\item
  cheryl strayed\\
  Yeah. Yeah. It's such an empowering poem, I mean, even the title, ``I
  Give You Back.'' We have the power to turn away from those things in
  ourselves. Do you remember what exactly provoked you to write this
  poem, what situation you were in?
\item
  joy harjo\\
  Well, man, I was in a lot of situations.
\item
  {[}laughter{]}
\item
  cheryl strayed\\
  Weren't we all?
\item
  joy harjo\\
  Yeah. I was good at bad situations.
\item
  cheryl strayed\\
  I love that.
\item
  joy harjo\\
  I've been trying to write about it because there was a period where I
  was in a lot of stress. I was trying to graduate from school. I was
  dealing with a partner who I had kicked out, who would come and break
  in the house and would say he was going to kill me, and calling
  police. And yet, I was getting straight As at school and raising two
  children.
\item
  cheryl strayed\\
  This was when you were a college student at the University of New
  Mexico?
\item
  joy harjo\\
  Yes. And then what started happening is I would lay there to go to
  sleep. And I would close my eyes, and there would be these demonic
  figures reaching out to get me. And I would yell and get myself up.
  And then I would keep the lights on. Just keeps them away because they
  don't like light. And then my spirit --- we all have guardians ---
  told me, says, speed up your energy, because light is a high energy.
  It has a fast speed to it, faster speed than something evil, which has
  a density to it. And the last time I saw them was when that spirit
  told me, speed it up. Speed up your energy. And then I saw them fall
  away.
\item
  cheryl strayed\\
  Wow.
\item
  joy harjo\\
  So this poem came out of that time.
\item
  cheryl strayed\\
  So you're writing a memoir right now, and you're reflecting in
  particular on the teachers you've had throughout your life, people who
  inspired you, but also people who taught you lessons that were
  painful. I'm wondering if you can tell me a bit about the kinds of
  teachers you're writing about in this memoir you're working on.
\item
  joy harjo\\
  Yeah. So I'm writing about the stepfather who was --- I tried to
  understand him when he passed. And my mother told me --- because he
  was being very violent to his children. And he would do things so that
  people wouldn't see. He was very twisted in his behavior. And I
  remember going to my mother and saying, please, please, we're fine by
  ourselves, after I saw him beat my sister, a toddler, and hold her up
  and beat her. And she was a baby. And she told me I was her confidant.
  She said, he will kill all of us. He has threatened he will kill me
  and all of us if we leave. I asked my sister, because I was writing
  this memoir. I called her up and said, did you know this? And she
  said, no. She said she knew that he had threatened to kill her. In
  fact, when I came back one night he had made her play Russian roulette
  in front of them. She didn't know that he threatened to kill all of
  us, but I went through from being nine years old through leaving for
  Indian school knowing that he would kill all of us. So I work on this.
  I don't know that I came up with an answer, but I tried to understand.
  And this is raw. I'll read it. It's the last paragraph of this
  particular story. ``I did not ask for him and do not want his story
  here with mine, even now. Perhaps he was one of my greatest teachers.
  Because of him, I learned to find myself in the spiritual world. In
  that realm there lived an immense house of imagination through which I
  could walk and immerse myself without fear. In that house, I met and
  spoke with my ancestors who had gone on, but came back when we needed
  their assistance. I found the ability to construct dreams with many
  kinds of materials. I saw the future. I saw the past. I battled
  monsters and sat with them at the table to hear their stories.
  Everyone has a story.''
\item
  cheryl strayed\\
  Joy, that's so powerful. And I think, gosh --- I mean, I have a father
  like the stepfather you describe. And I've had to come to those terms.
  What you put so beautifully and powerfully in that last paragraph that
  when you're pushed in the direction of that kind of sorrow, and fear,
  and rage, and pain, that ultimately those people do force us to find a
  way that we wouldn't have otherwise. And it's a complicated thing,
  isn't it, in some ways to say thank you. Thank you for that horror.
  Thank you for that difficult lesson. And yet to do it is, I think, the
  most empowering act of all. It's in some ways what you're saying in
  your poem.
\item
  joy harjo\\
  Yes. You wouldn't have your work. You wouldn't have ``Wild,'' or your
  novel, or anything else without that.
\item
  cheryl strayed\\
  That's really true.
\item
  joy harjo\\
  Uh-huh.
\item
  cheryl strayed\\
  Yeah.
\item
  joy harjo\\
  I wish I could remember his name. Because I've been around a lot of
  really wise people, and I always like to sit in those circles of wise
  native people that would come together. And they predicted these
  times. But I remember this man saying that a teacher loves the most,
  they test the most. Sometimes I laughed and said, well, I guess native
  people are the most beloved.
\item
  {[}laughter{]}\\
  And I remember saying that once at a school, a native school. And some
  kid wrote me after that and said, that changed my life.
\item
  cheryl strayed\\
  Oh, wow. Yeah.
\item
  joy harjo\\
  Because we can all get bent by feeling like we are being tested
  because someone is after us or someone hates us. But sometimes it is
  because we're the most beloved. And then I wonder about this country
  and about this earth, because this earth is a planetary being. And
  we're being tested right now.
\item
  cheryl strayed\\
  We are.
\item
  joy harjo\\
  Yes, we're being tested. And I think, OK, so what are the parameters
  of the test, and how is this going to play out, and how do we as
  artists and poets subsist in this world pandemic storytelling so that
  what is nourishing in the culture survives? Because we have been going
  on for a long time with a lot of false stories. You think about music
  created only for money, or stories created to make money, or drugs
  created to make money. That whole paradigm, it's not taking care of
  people or the earth.
\item
  cheryl strayed\\
  This pandemic has certainly laid bare that truth that was already
  apparent to many, but I think now we see it more clearly. You said
  earlier that wise native elders had predicted these times. Would you
  talk to me about that?
\item
  joy harjo\\
  Yes. It's been in the prophecies. They all agreed that these times
  were coming, the times that there would be droughts, not enough food,
  that there were going to be storms, powerful storms, pandemics, and so
  on unless people changed their ways. And that's really the council. I
  mean, if you think of the council --- the council is always here.
  There's an earth council. I mean, they're the ones that uphold those
  universal truths that are true for everyone. It's about compassion, or
  what they call it in Muskogee --- we call it {[}NON-ENGLISH{]}. It's
  about understanding everyone as a beloved person, essentially.
\item
  cheryl strayed\\
  That's the way. It's the way forward, it seems, if we can get there.
  If we were ever there, I don't know. But maybe that's where we can
  think about going.
\item
  joy harjo\\
  I think we have moments of being there. We have all had moments in our
  lives of being there when it comes together, and we feel --- we know
  what that feels like. And it's often in a moment of compassion. I
  remember being really poor as a young mother. And it was so cool that
  people would share.
\item
  cheryl strayed\\
  Yeah.
\item
  joy harjo\\
  And there was that sense of sharing and compassion that moves through
  then at those moments or when you're looking out the window. You're
  cleaning the kitchen or washing dishes even, or rinsing them and
  looking out the window, and that light hits at that angle for dusk
  when dusk happens, and that beautiful glow in the light. And there's
  often a moment in those times where you know that you feel what that
  kind of world is like. Or when a child --- you hold a newborn child,
  and they're still fresh. They are fresh from the other world, and they
  carry that knowledge, and they still remember beauty.
\item
  cheryl strayed\\
  Yeah. Well, it seems to me you never forgot it, Joy.
\item
  joy harjo\\
  Well, it motivates me. I haven't been given up on yet.
\item
  {[}laughter{]}
\item
  cheryl strayed\\
  Me either. I'm with you in the beauty club. I try to see it every day.
\item
  joy harjo\\
  Yes.
\item
  cheryl strayed\\
  So I cannot tell you just how much heart it gives me to know that you
  are out there across the land in Oklahoma, writing that next book and
  being the beautiful truth teller, and powerhouse writer and U.S. Poet
  Laureate that you are.
\item
  joy harjo\\
  Well, thank you. I loved getting a chance to visit with you. And I do
  miss --- I always remember us getting together and having a moment in
  that wonderful, beautiful, incredible sea of human beings there to
  celebrate writing. And I look forward to being able to do that again.
\item
  cheryl strayed\\
  Me too. We will meet again, Joy. And we will whisper up close with
  each other as we did that day just a year ago. Be well.
\item
  joy harjo\\
  OK, you too.
\item
  cheryl strayed\\
  Bye.
\item
  joy harjo\\
  Bye.
\item
  {[}music playing{]}
\item
  cheryl strayed\\
  I'm Cheryl Strayed. This is ``Sugar Calling.'' When I conceived
  ``Sugar Calling,'' I thought of it as only a pop-up podcast. It would
  be about eight episodes long, and by then we'd be back to normal life.
  But here we are, and life is getting ever more not normal. I've had
  such a warm response to the podcast. And I've frankly had so much fun
  doing it that I've decided to go on --- not right now, but sometime
  down the line I'll be back to make more calls and to share them with
  you. Until then, stay safe. Do good. Be well. I love you guys. Bye.
\item
  {[}music playing{]}
\end{itemize}

Previous

More episodes ofSugar Calling

\href{https://www.nytimes3xbfgragh.onion/2020/05/20/podcasts/sugar-calling-joy-harjo-poetry-virus.html?action=click\&module=audio-series-bar\&region=header\&pgtype=Article}{\includegraphics{https://static01.graylady3jvrrxbe.onion/images/2020/05/22/podcasts/20sugar-hajo3/20sugar-hajo3-thumbLarge.jpg}}

May 20, 2020~~•~ 35:30`I Release You, Fear'

\href{https://www.nytimes3xbfgragh.onion/2020/05/13/podcasts/sugar-calling-billy-collins-poetry-virus.html?action=click\&module=audio-series-bar\&region=header\&pgtype=Article}{\includegraphics{https://static01.graylady3jvrrxbe.onion/images/2020/05/13/podcasts/13sugar-calling/13sugar-calling-thumbLarge.jpg}}

May 13, 2020`There's a Quiet All Over the World'

\href{https://www.nytimes3xbfgragh.onion/2020/05/06/podcasts/sugar-calling-alice-walker-quarantine-virus.html?action=click\&module=audio-series-bar\&region=header\&pgtype=Article}{\includegraphics{https://static01.graylady3jvrrxbe.onion/images/2020/05/06/podcasts/06sugarcalling/06sugarcalling-thumbLarge.jpg}}

May 6, 2020~~•~ 28:58`Whatever We Have, We Have to Work With It'

\href{https://www.nytimes3xbfgragh.onion/2020/04/29/podcasts/sugar-calling-judy-blume-quarantine-virus.html?action=click\&module=audio-series-bar\&region=header\&pgtype=Article}{\includegraphics{https://static01.graylady3jvrrxbe.onion/images/2020/04/29/podcasts/29sugarcalliing-blume-sub/29sugarcalliing-blume-sub-thumbLarge.jpg}}

April 29, 2020`This Terrible Thing Is Happening, but the World Goes On.'

\href{https://www.nytimes3xbfgragh.onion/2020/04/22/podcasts/sugar-calling-amy-tan-quarantine-virus.html?action=click\&module=audio-series-bar\&region=header\&pgtype=Article}{\includegraphics{https://static01.graylady3jvrrxbe.onion/images/2020/04/27/podcasts/22sugarcalling/22sugarcalling-thumbLarge.jpg}}

April 22, 2020~~•~ 39:19`You Don't Take Dictation. You Find the Truth.'

\href{https://www.nytimes3xbfgragh.onion/2020/04/15/podcasts/sugar-calling-pico-iyer-coronavirus.html?action=click\&module=audio-series-bar\&region=header\&pgtype=Article}{\includegraphics{https://static01.graylady3jvrrxbe.onion/images/2020/04/21/podcasts/15sugarcalling1/15sugarcalling1-thumbLarge.jpg}}

April 15, 2020~~•~ 35:45`Joyful Participation in a World of Sorrows'

\href{https://www.nytimes3xbfgragh.onion/2020/04/08/podcasts/sugar-calling-margaret-atwood-coronavirus.html?action=click\&module=audio-series-bar\&region=header\&pgtype=Article}{\includegraphics{https://static01.graylady3jvrrxbe.onion/images/2020/04/02/books/08sugarcalling1/08sugarcalling1-thumbLarge-v3.jpg}}

April 8, 2020~~•~ 34:32`Roll Up Your Sleeves, Girls'

\href{https://www.nytimes3xbfgragh.onion/2020/04/03/podcasts/sugar-calling-george-saunders-coronavirus.html?action=click\&module=audio-series-bar\&region=header\&pgtype=Article}{\includegraphics{https://static01.graylady3jvrrxbe.onion/images/2020/04/09/podcasts/03sugarcalling-image/merlin_171264408_4ac7fc67-d8cc-45b9-9ec6-bdd20672e694-thumbLarge.jpg}}

April 3, 2020~~•~ 41:16`Everything Is Always Keep Changing'

\href{https://www.nytimes3xbfgragh.onion/column/sugar-calling}{See All
Episodes ofSugar Calling}

Next

May 20, 2020

\begin{itemize}
\item
\item
\item
\item
\item
\item
\end{itemize}

\emph{\textbf{Listen and subscribe to our podcast from your mobile
device:}}
\textbf{\href{https://podcasts.apple.com/us/podcast/sugar-calling/id1505881384}{\emph{Via
Apple Podcasts}}} \emph{\textbf{\textbar{}}}
\textbf{\href{https://open.spotify.com/show/4U8hPiNGIBvTS9zLeiDCN7?si=gRyigD47SPWl-QWgNjgt2w}{\emph{Via
Spotify}}} \emph{\textbf{\textbar{}}}
\textbf{\href{https://www.stitcher.com/podcast/the-new-york-times/sugar-calling}{\emph{Via
Stitcher}}}

\hypertarget{we-have-guardians-that-look-over-us-were-part-of-a-whole-system-that-doesnt-just-end-at-birth-or-end-at-death}{%
\subsection{`We have guardians that look over us. We're part of a whole
system that doesn't just end at birth or end at
death.'}\label{we-have-guardians-that-look-over-us-were-part-of-a-whole-system-that-doesnt-just-end-at-birth-or-end-at-death}}

\emph{--- Joy Harjo, the poet laureate of the United States}

Today, Cheryl calls the poet, playwright and musician Joy Harjo at her
apartment in Tulsa, Okla., the city where Joy was born. Joy opens up
about leaving home to attend a Bureau of Indian Education boarding
school in the 1960s. "For the first time, I was in a circle of
relatives,'' Joy says, ``a circle of other students like me who were
natives, who had been through a lot of the same stories.''

Joy then tells the story of being named the 23rd poet laureate of the
United States: ``It literally felt like lightning moving through my
system.'' She and Cheryl also discuss why it takes a good ``working
poem'' to fight fear.

\includegraphics{https://static01.graylady3jvrrxbe.onion/images/2020/05/22/podcasts/20sugar-hajo3/merlin_156628830_b7d4172a-d0ab-42fd-ad2e-f457f44dee06-articleLarge.jpg?quality=75\&auto=webp\&disable=upscale}

\hypertarget{on-todays-episode}{%
\subsubsection{\texorpdfstring{\textbf{On today's
episode:}}{On today's episode:}}\label{on-todays-episode}}

\href{https://www.joyharjo.com/}{Joy Harjo} is a poet, musician and
playwright of the Muscogee Creek Nation. She is the author of
\href{https://www.nytimes3xbfgragh.onion/2019/06/19/books/joy-harjo-poet-laureate.html?searchResultPosition=3}{eight
books of poetry, a memoir and two books for young audiences}. In June
2019, Joy was named the 23rd poet laureate of the United States. In her
second term, which starts in September, she'll focus on a project called
``\href{https://www.nytimes3xbfgragh.onion/2020/04/30/books/joy-harjo-poet-laureate-second-term.html?searchResultPosition=1}{Living
Nations, Living Words: A Map of First Peoples Poetry},'' a digital
interactive tool highlighting the work and locations of contemporary
Native poets.

Image

In the episode, Joy explains that playing the saxophone has been her
``lung protector.''Credit...Karen Kuehn

\hypertarget{joy-harjos-pandemic-reading-list}{%
\subsubsection{\texorpdfstring{\textbf{Joy Harjo's pandemic reading
list:}}{Joy Harjo's pandemic reading list:}}\label{joy-harjos-pandemic-reading-list}}

\begin{itemize}
\tightlist
\item
  ``\href{https://milkweed.org/book/thrown-in-the-throat}{Thrown in the
  Throat},'' Benjamin Garcia
\end{itemize}

\begin{itemize}
\item
  ``\href{https://www.coppercanyonpress.org/books/deluge-by-leila-chatti/}{Deluge},''
  Leila Chatti
\item
  ``\href{https://www.indiebound.org/book/9781646140138}{Apple (Skin to
  the Core)},'' Eric Gansworth
\item
  ``\href{https://www.harpercollins.com/9780062671189/the-night-watchman/}{The
  Night Watchman},'' Louise Erdrich
\item
  ``\href{https://www.graywolfpress.org/books/postcolonial-love-poem}{Postcolonial
  Love Poem: Poems},'' Natalie Diaz
\end{itemize}

\begin{center}\rule{0.5\linewidth}{\linethickness}\end{center}

Cheryl Strayed is the author of ``Tiny Beautiful Things,'' ``Torch,''
``Brave Enough,'' and The New York Times best seller ``Wild.'' Her books
have been translated into more than 40 languages. She lives in Portland,
Ore.
\href{https://twitter.com/CherylStrayed?ref_src=twsrc\%5Egoogle\%7Ctwcamp\%5Eserp\%7Ctwgr\%5Eauthor}{@CherylStrayed}

``Sugar Calling'' is produced by Kelly Prime and edited by Sara
Sarasohn, with editorial oversight by Wendy Dorr. This episode was mixed
by Jamie Collazo. Our theme music is by Dan Powell.

Advertisement

\protect\hyperlink{after-bottom}{Continue reading the main story}

\hypertarget{site-index}{%
\subsection{Site Index}\label{site-index}}

\hypertarget{site-information-navigation}{%
\subsection{Site Information
Navigation}\label{site-information-navigation}}

\begin{itemize}
\tightlist
\item
  \href{https://help.nytimes3xbfgragh.onion/hc/en-us/articles/115014792127-Copyright-notice}{©~2020~The
  New York Times Company}
\end{itemize}

\begin{itemize}
\tightlist
\item
  \href{https://www.nytco.com/}{NYTCo}
\item
  \href{https://help.nytimes3xbfgragh.onion/hc/en-us/articles/115015385887-Contact-Us}{Contact
  Us}
\item
  \href{https://www.nytco.com/careers/}{Work with us}
\item
  \href{https://nytmediakit.com/}{Advertise}
\item
  \href{http://www.tbrandstudio.com/}{T Brand Studio}
\item
  \href{https://www.nytimes3xbfgragh.onion/privacy/cookie-policy\#how-do-i-manage-trackers}{Your
  Ad Choices}
\item
  \href{https://www.nytimes3xbfgragh.onion/privacy}{Privacy}
\item
  \href{https://help.nytimes3xbfgragh.onion/hc/en-us/articles/115014893428-Terms-of-service}{Terms
  of Service}
\item
  \href{https://help.nytimes3xbfgragh.onion/hc/en-us/articles/115014893968-Terms-of-sale}{Terms
  of Sale}
\item
  \href{https://spiderbites.nytimes3xbfgragh.onion}{Site Map}
\item
  \href{https://help.nytimes3xbfgragh.onion/hc/en-us}{Help}
\item
  \href{https://www.nytimes3xbfgragh.onion/subscription?campaignId=37WXW}{Subscriptions}
\end{itemize}
