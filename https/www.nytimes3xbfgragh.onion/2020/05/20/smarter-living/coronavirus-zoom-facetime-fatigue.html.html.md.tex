Sections

SEARCH

\protect\hyperlink{site-content}{Skip to
content}\protect\hyperlink{site-index}{Skip to site index}

\href{https://www.nytimes3xbfgragh.onion/section/smarter-living}{Smarter
Living}

\href{https://myaccount.nytimes3xbfgragh.onion/auth/login?response_type=cookie\&client_id=vi}{}

\href{https://www.nytimes3xbfgragh.onion/section/todayspaper}{Today's
Paper}

\href{/section/smarter-living}{Smarter Living}\textbar{}Zoom Fatigue:
How to Politely Decline a Call During Quarantine

\url{https://nyti.ms/3cPhHMT}

\begin{itemize}
\item
\item
\item
\item
\item
\end{itemize}

\href{https://www.nytimes3xbfgragh.onion/spotlight/at-home?action=click\&pgtype=Article\&state=default\&region=TOP_BANNER\&context=at_home_menu}{At
Home}

\begin{itemize}
\tightlist
\item
  \href{https://www.nytimes3xbfgragh.onion/2020/07/28/books/time-for-a-literary-road-trip.html?action=click\&pgtype=Article\&state=default\&region=TOP_BANNER\&context=at_home_menu}{Take:
  A Literary Road Trip}
\item
  \href{https://www.nytimes3xbfgragh.onion/2020/07/29/magazine/bored-with-your-home-cooking-some-smoky-eggplant-will-fix-that.html?action=click\&pgtype=Article\&state=default\&region=TOP_BANNER\&context=at_home_menu}{Cook:
  Smoky Eggplant}
\item
  \href{https://www.nytimes3xbfgragh.onion/2020/07/27/travel/moose-michigan-isle-royale.html?action=click\&pgtype=Article\&state=default\&region=TOP_BANNER\&context=at_home_menu}{Look
  Out: For Moose}
\item
  \href{https://www.nytimes3xbfgragh.onion/interactive/2020/at-home/even-more-reporters-editors-diaries-lists-recommendations.html?action=click\&pgtype=Article\&state=default\&region=TOP_BANNER\&context=at_home_menu}{Explore:
  Reporters' Obsessions}
\end{itemize}

Advertisement

\protect\hyperlink{after-top}{Continue reading the main story}

Supported by

\protect\hyperlink{after-sponsor}{Continue reading the main story}

\hypertarget{zoom-fatigue-how-to-politely-decline-a-call-during-quarantine}{%
\section{Zoom Fatigue: How to Politely Decline a Call During
Quarantine}\label{zoom-fatigue-how-to-politely-decline-a-call-during-quarantine}}

The normal boundaries that once dictated social etiquette have
essentially dissolved. So how do you disconnect?

By Kathleen Walsh

\includegraphics{https://static01.graylady3jvrrxbe.onion/images/2020/05/11/smarter-living/00sl-zoom-decline/00sl-zoom-decline-superJumbo.jpg}

May 20, 2020

Much of the **** United States is still on lockdown, so your loved ones
are aware that you, like them, are probably sprawled on the couch in
elastic waistband pants Googling how to create a sourdough starter,
making it hard to find guilt-free excuses for declining a FaceTime call
or virtual happy hour invite.

Quarantine has essentially dissolved the normal boundaries that once
dictated social etiquette. Before the pandemic, you were unlikely to be
surprised by a morning call from a relative or to schedule back-to-back
hangouts with different groups of friends. But we're operating on
airport rules now, where cocktail hour is a construct and dress code is
whatever's comfy.

Initially, pondering months of isolation from friends and family, people
needed to feel connected. Now that quarantine is just how we will live
for the foreseeable future, how do you disconnect? The
\href{https://www.nytimes3xbfgragh.onion/2020/06/11/technology/zoom-china-tiananmen-square.html}{Zoom}
fatigue has set in, and it's time to learn to deal with it.

``I think at the beginning it was easy to overestimate'' how much
socialization people assumed they needed or could handle, said
\href{https://kathleensmith.net/}{Kathleen Smith}, a therapist who
focuses on anxiety.

Introverts, who rely on solitude to mentally recharge, may have
initially adjusted more easily to life on lockdown. Several weeks in,
they're navigating between a sense of obligation to the people who lean
on them and their own need for a little space --- in extraordinary
circumstances, no less, when communal solidarity feels like a duty.

``In real life, you have plans for the evening with the same people and
can focus on them,'' said Marissa Karlins, who is quarantined in Queens,
New York, with her boyfriend. ``But with this, there's pressure to talk
to everyone because you worry that they're lonely. I don't want to turn
them down.''

Without access to our usual coping tools, everyone ``is going to try to
rely on each other and that makes sense and that's normal,'' said
Melissa Martin, a clinical counselor in Chicago. ``But we need to be
able to be in a place where we have the emotional energy and capacity.''

So, decline the call and follow up with a text. In general, experts
agree it's better to be honest, more or less, stopping short of, ``I
just don't want to talk to you.''

``I don't believe in beating around the bush,''
said\href{https://dawnaltman.com/}{Dawn Altman}, a psychotherapist.
Maybe your phone was off, or you were reading or watching TV. Just say
that. Even in the midst of a pandemic you're entitled to your own space,
including mentally.

``In many cases we all are on top of each other now,'' Ms. Altman said,
whereas before the pandemic, parents, for example, might have had ``time
to just take a breath'' when their children were at school. ``So I think
it's absolutely imperative that we set some boundaries.'' This applies
even if you live alone, she added.

If you need some breathing room, Ms. Altman advises turning your phone
off, or designating certain times of the day to stay disconnected. You
can later explain to your caller, ``I did miss your call, I'm sorry, but
I've delegated these two hours today to be device-free.''

Ms. Martin agrees, adding, ``There doesn't have to be a huge
explanation.'' You can say honestly, ``I'm kind of stressed out and want
some time to just watch Netflix or read a book, so I'll call you
later.'' From an emotional-health standpoint, she cautioned, ``Struggles
can kind of get bigger when we're hiding them or we're feeling ashamed
that we don't have the emotional energy to talk to someone.''

Being forthright about why you're unavailable is not only better for you
psychologically, but is also more understandable than a weak excuse,
which can eventually make those who are reaching out to you feel as if
you're blowing them off.

``Just speak honestly, because how can anybody be offended by that?''
Ms. Altman said. ``And if they are, that's sad for them.''

You can also take your time before responding, to think of how you want
to respond, give yourself some space or decide if you want to respond at
all.

``Even if they're asking that you talk right then, you don't necessarily
owe them a response,'' Dr. Smith said. ``So, just give yourself
permission to take some time to think about it. If you need to sleep on
it, sleep on it.''

What you need emotionally has probably changed since before quarantine,
so take stock.

Reanna Kawatha, who is quarantining with family in Detroit, said she's
feeling more fatigued by socializing than she normally is.

``I'm typically pretty extroverted but the past few weeks have been
exhausting,'' she said. ``I'm working full time and living with my
family, so it hasn't been any less busy for me since quarantine
started.''

Ms. Kawatha added that she's noticed an uptick in calls from old friends
trying to reconnect. ``Calls with friends I haven't spoken to in a while
can be stressful because I know they'll go on for at least an hour,
sometimes two or three,'' she said, ``and it's hard to make time for
that in the evenings after work.''

Differences in environment or temperament between you and your friends
can also feel more stark right now. ``Sometimes conversations with
friends about the situation with Covid-19 can make me uneasy, because
everyone has such polarizing opinions about it and it can be
uncomfortable to talk about our day-to-day as a result,'' Ms. Kawatha
said.

Many in her orbit are dealing with lost income or difficult family
dynamics, she said, and these conversations can become emotionally
taxing because ``there's not much you can do to help.''

``We all have such limited control over the situation and it's so hard
watching everyone struggle,'' Ms. Kawatha said. So if she knows a
conversation is going to be an emotional slog, she cancels.

If you do back out of a social obligation, Ms. Altman suggests a
well-timed pivot. She said her own friends had been scheduling
near-constant virtual cocktail hours, which she doesn't really want to
do. So when they ask why she wasn't at the latest trivia night or happy
hour or so on, she sidesteps: ``Oh, I had something else planned that
day,'' or, ``My phone was off during that time.'' It's the truth
``without saying that really isn't my gig,'' she added.

Even so --- if we're really being honest --- sometimes it is just easier
to say you are overbooked, even if you aren't.

``My knee-jerk is to just claim multiple
\href{https://www.nytimes3xbfgragh.onion/2020/06/11/technology/zoom-china-tiananmen-square.html}{Zoom}
hangs. It doesn't ruffle feathers, and everyone has generally been
chill,'' said Joe Tower, a freelancer in Colorado who is quarantining
with family. ``It's actually salad days for placaters like me. If you're
the type who can't say no and prone to double-booking, no one will blame
you because in quarantine everyone can relate to trying to O.D. on
contact.''

Such fibs are especially tempting when pleading ``mental health'' feels
like oversharing, or when you're pretty sure the ensuing conversation is
going to be more exhausting than it's worth.

You are not a mind reader, so don't assume you know how the other person
is going to react. ``When we're more stressed and more anxious, like
right now, we tend to imagine the worst of people without a lot of
evidence,'' Dr. Smith said. Still, experts acknowledge that, in some
cases, a little white lie isn't necessarily a bad thing.

If you need a delicate exit from a draining conversation, or you're
suddenly getting calls from a long-lost acquaintance, it's OK. Go ahead
and pretend your delivery order just arrived.

Advertisement

\protect\hyperlink{after-bottom}{Continue reading the main story}

\hypertarget{site-index}{%
\subsection{Site Index}\label{site-index}}

\hypertarget{site-information-navigation}{%
\subsection{Site Information
Navigation}\label{site-information-navigation}}

\begin{itemize}
\tightlist
\item
  \href{https://help.nytimes3xbfgragh.onion/hc/en-us/articles/115014792127-Copyright-notice}{©~2020~The
  New York Times Company}
\end{itemize}

\begin{itemize}
\tightlist
\item
  \href{https://www.nytco.com/}{NYTCo}
\item
  \href{https://help.nytimes3xbfgragh.onion/hc/en-us/articles/115015385887-Contact-Us}{Contact
  Us}
\item
  \href{https://www.nytco.com/careers/}{Work with us}
\item
  \href{https://nytmediakit.com/}{Advertise}
\item
  \href{http://www.tbrandstudio.com/}{T Brand Studio}
\item
  \href{https://www.nytimes3xbfgragh.onion/privacy/cookie-policy\#how-do-i-manage-trackers}{Your
  Ad Choices}
\item
  \href{https://www.nytimes3xbfgragh.onion/privacy}{Privacy}
\item
  \href{https://help.nytimes3xbfgragh.onion/hc/en-us/articles/115014893428-Terms-of-service}{Terms
  of Service}
\item
  \href{https://help.nytimes3xbfgragh.onion/hc/en-us/articles/115014893968-Terms-of-sale}{Terms
  of Sale}
\item
  \href{https://spiderbites.nytimes3xbfgragh.onion}{Site Map}
\item
  \href{https://help.nytimes3xbfgragh.onion/hc/en-us}{Help}
\item
  \href{https://www.nytimes3xbfgragh.onion/subscription?campaignId=37WXW}{Subscriptions}
\end{itemize}
