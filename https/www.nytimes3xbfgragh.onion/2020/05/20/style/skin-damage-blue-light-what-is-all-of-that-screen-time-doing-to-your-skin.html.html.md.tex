Sections

SEARCH

\protect\hyperlink{site-content}{Skip to
content}\protect\hyperlink{site-index}{Skip to site index}

\href{https://www.nytimes3xbfgragh.onion/section/style}{Style}

\href{https://myaccount.nytimes3xbfgragh.onion/auth/login?response_type=cookie\&client_id=vi}{}

\href{https://www.nytimes3xbfgragh.onion/section/todayspaper}{Today's
Paper}

\href{/section/style}{Style}\textbar{}What Is All of That Screen Time
Doing to Your Skin?

\href{https://nyti.ms/2Xk4W6K}{https://nyti.ms/2Xk4W6K}

\begin{itemize}
\item
\item
\item
\item
\item
\end{itemize}

\href{https://www.nytimes3xbfgragh.onion/spotlight/at-home?action=click\&pgtype=Article\&state=default\&region=TOP_BANNER\&context=at_home_menu}{At
Home}

\begin{itemize}
\tightlist
\item
  \href{https://www.nytimes3xbfgragh.onion/2020/08/03/well/family/the-benefits-of-talking-to-strangers.html?action=click\&pgtype=Article\&state=default\&region=TOP_BANNER\&context=at_home_menu}{Talk:
  To Strangers}
\item
  \href{https://www.nytimes3xbfgragh.onion/2020/08/01/at-home/coronavirus-make-pizza-on-a-grill.html?action=click\&pgtype=Article\&state=default\&region=TOP_BANNER\&context=at_home_menu}{Make:
  Grilled Pizza}
\item
  \href{https://www.nytimes3xbfgragh.onion/2020/07/31/arts/television/goldbergs-abc-stream.html?action=click\&pgtype=Article\&state=default\&region=TOP_BANNER\&context=at_home_menu}{Watch:
  'The Goldbergs'}
\item
  \href{https://www.nytimes3xbfgragh.onion/interactive/2020/at-home/even-more-reporters-editors-diaries-lists-recommendations.html?action=click\&pgtype=Article\&state=default\&region=TOP_BANNER\&context=at_home_menu}{Explore:
  Reporters' Google Docs}
\end{itemize}

Advertisement

\protect\hyperlink{after-top}{Continue reading the main story}

Supported by

\protect\hyperlink{after-sponsor}{Continue reading the main story}

Skin Deep

\hypertarget{what-is-all-of-that-screen-time-doing-to-your-skin}{%
\section{What Is All of That Screen Time Doing to Your
Skin?}\label{what-is-all-of-that-screen-time-doing-to-your-skin}}

We checked in with experts to learn about the risks of indoor light and
how we can protect ourselves.

\includegraphics{https://static01.graylady3jvrrxbe.onion/images/2020/05/19/fashion/19SKIN-BLUELIGHT/19SKIN-BLUELIGHT-articleLarge.jpg?quality=75\&auto=webp\&disable=upscale}

\href{https://www.nytimes3xbfgragh.onion/by/crystal-martin}{\includegraphics{https://static01.graylady3jvrrxbe.onion/images/2019/03/01/multimedia/author-crystal-martin/author-crystal-martin-thumbLarge.png}}

By \href{https://www.nytimes3xbfgragh.onion/by/crystal-martin}{Crystal
Martin}

\begin{itemize}
\item
  May 20, 2020
\item
  \begin{itemize}
  \item
  \item
  \item
  \item
  \item
  \end{itemize}
\end{itemize}

\href{https://www.nytimes3xbfgragh.onion/es/2020/05/26/espanol/estilos-de-vida/efectos-luz-pantalla-piel.html}{Leer
en español}

You've probably heard more about the perils of blue light lately because
our lives are more likely to be lived indoors and online. Our laptops,
phones, tablets, TVs and even LED light bulbs are all sources of blue
light. And now that we're tethered to those devices, are we getting
drenched? Should we be more worried about damage to our skin?

Here's what we know: Compared with the well-understood dangers of
ultraviolet light (skin aging and cancer), science isn't settled on the
effects of indoor sources of blue light on skin. It can cause
hyperpigmentation and premature aging, but the rest --- what dose of it
causes trouble, for instance --- was debated well before we were
confined to our homes.

Here, we've checked in with some blue light and skin experts to help us
understand the real risks.

\hypertarget{what-is-blue-light}{%
\subsection{What is blue light?}\label{what-is-blue-light}}

When we think about the harmful effects of light, we're usually thinking
ultraviolet light (UV), which is invisible. But we can see blue light.
You may perceive it as a cool-toned white light (as with an LED light
bulb), or you may not be aware of much blue at all. That's because your
indoor light sources are emitting varying wavelengths that combine to
create the colors you perceive.

Though the effects of blue light on the skin are yet to be fully
understood, the light is an important health concern because of other
risks. ``Blue light damages the retina and reduces your excretion of
melatonin, so it interrupts your sleep cycle,'' said Michelle Henry, a
dermatologist in New York.

Proximity is, of course, a factor when thinking about the danger.
``You'll get less blue light from your TV than from your computer
because it's farther away,'' Dr. Henry said. ``And more light from your
phone than your computer because your phone is so close to your face.''

\hypertarget{how-does-blue-light-damage-my-skin}{%
\subsection{How does blue light damage my
skin?}\label{how-does-blue-light-damage-my-skin}}

While ultraviolet light damages cells' DNA directly, blue light destroys
collagen through oxidative stress. A chemical in skin called flavin
absorbs blue light. The reaction that takes place during that absorption
produces unstable oxygen molecules (free radicals) that damage the skin.

``They go in and basically poke holes in your collagen,'' Dr. Henry
said.

Exposure to blue light is more problematic for skin of color. In
\href{https://www.sciencedirect.com/science/article/pii/S0022202X15349307}{a
2010 study published in the Journal of Investigative Dermatology}, it
was shown to cause hyperpigmentation in medium to dark skin, while
leaving lighter skin relatively unaffected.

The medical community categorizes skin color based on how it reacts to
UV light. Type 1 is the lightest color with the most UV sensitivity.
``This would be Nicole Kidman and Conan O'Brien,'' said Mathew M. Avram,
the director of the Massachusetts General Hospital Dermatology Laser and
Cosmetic Center in Boston. The scale goes up to Type 6, which is the
darkest and least likely to burn.

In the 2010 study, Type 2 skin was exposed to blue light but didn't
develop pigmentation. Skin of color darkened, and that darkness
persisted for a couple of weeks.

``There is something about the pigmentation in Types 4, 5 and 6 that
reacts differently than in patients with fair skin,'' Dr. Avram said.
``There should be more large-scale studies looking at this because
pigmentation is one of the biggest patient concerns and the one where
treatment creates less patient satisfaction.''

\hypertarget{but-isnt-blue-light-used-to-treat-acne}{%
\subsection{But isn't blue light used to treat
acne?}\label{but-isnt-blue-light-used-to-treat-acne}}

Yes, blue light lamps treat acne and precancerous lesions. ``It damages
the skin, but on the other hand it can treat acne,'' Dr. Avram said.
``It can help your mood and memory as well. So it's more complicated
than just saying `good' or `bad.'''

\hypertarget{how-can-i-prevent-skin-damage}{%
\subsection{How can I prevent skin
damage?}\label{how-can-i-prevent-skin-damage}}

The simplest intervention is to limit the amount of blue light emitted
from your devices. Apple products have ``night shift'' that creates a
warmer screen tone. Swap out your standard LED bulbs for versions that
emit less blue light.

Mineral sunscreens with iron oxides are the gold standard in blue light
protection. Iron oxides have been
\href{https://www.ncbi.nlm.nih.gov/pmc/articles/PMC6718061/}{shown to be
more protective} against visible light than zinc oxide and titanium
dioxide alone.

``A good cheat for this is any tinted sunscreen, which usually has iron
oxide,'' Dr. Henry said.
\href{https://skinbetter.com/products/sunbetter-tone-smart-spf-68-sunscreen-compact/}{Skinbetter
Science Sunbetter Tone Smart SPF 68 Sunscreen Compact,} \$55, is one
such mineral sunblock. The formula combines zinc oxide, titanium dioxide
and iron oxide, and it blends smoothly, even on brown skin.

Topical antioxidants should help tame the free radicals blue light
creates, but again, the science isn't fully formed.

``I cannot recommend antioxidants from a purely scientific
perspective,'' said Alexander Wolf, a senior assistant professor at
Nippon Medical School in Tokyo and an expert in how light and oxidative
stress cause premature aging. ``But there are certainly a lot of
experiments that show antioxidants work well in cultured cells. Vitamin
C enters the cells directly, and if you do some oxidative damage to the
cells, the vitamin C or some antioxidant definitely helps.''

``But a dish with some cells is not skin,'' Dr. Wolf added.

As long as you're clear that antioxidants haven't been proven to work on
blue light, but would likely work, they are a good substitute for
sunscreen if you feel weird about sitting at home with a face full of
minerals. It's likely that antioxidants will also minimize the damage of
a blue LED light device used at home to treat acne. (A mineral sunscreen
would block the blue light and stop its bacteria-killing action.)

As far as antioxidants go, vitamin C is a good choice because the
molecule is actually small enough to penetrate the skin.
\href{https://gethyperskin.com/products/hyper-clear}{Hyper Skin Hyper
Clear Brightening Clearing Vitamin C Serum}, \$36, contains 15 percent
vitamin C paired with vitamin E, and the two ingredients boost each
other's potential to fight free radicals.

The buzz around blue light has led to new lines like Goodhabit. Its
\href{https://goodhabitskin.com/products/glow-potion-oil-serum}{Rescue
Me Glow Potion Oil Serum}, \$80, combines marine-sourced proteins with
exopolysaccharides --- that is, polymers secreted by microorganisms that
create a protective barrier over the skin. The polymers act like a
sunscreen that blocks blue light (rather than neutralizing free radicals
like an antioxidant).

Though alpha-lipoic acid is not touted for its blue light protective
qualities, Dr. Wolf has studied its effect on oxidative stress (in mouse
skin) and thinks it is promising for human skin.

``It works differently than an antioxidant,'' he said. ``It activates
the natural defenses of the skin cell by tricking the skin cell to
think, `Oh, there is oxidative stress.' The cell turns up its own
defense mechanisms. I think that's a much more elegant way to defend
yourself.''

\href{https://www.perriconemd.com/products/face-finishing-firming-moisturizer-51090023}{Perricone
MD High Potency Classics: Face Finishing \& Firming Moisturizer,}\$69,
contains both vitamin C and alpha-lipoic acid.

One important fact is often left out of the blue light conversation: The
sun is by far our most abundant source of blue light.

``Brightness is not something the human eye is good at gauging because
the pupil adjusts,'' Dr. Wolf said. ``You may think your tablet or
smartphone is bright, but as far as the amount of light reaching your
skin, it is very weak, especially compared to the sun.''

All things considered, then, your blue light exposure may well be down
when compared to your pre-pandemic life for the simple reason that
you're spending more time indoors.

Advertisement

\protect\hyperlink{after-bottom}{Continue reading the main story}

\hypertarget{site-index}{%
\subsection{Site Index}\label{site-index}}

\hypertarget{site-information-navigation}{%
\subsection{Site Information
Navigation}\label{site-information-navigation}}

\begin{itemize}
\tightlist
\item
  \href{https://help.nytimes3xbfgragh.onion/hc/en-us/articles/115014792127-Copyright-notice}{©~2020~The
  New York Times Company}
\end{itemize}

\begin{itemize}
\tightlist
\item
  \href{https://www.nytco.com/}{NYTCo}
\item
  \href{https://help.nytimes3xbfgragh.onion/hc/en-us/articles/115015385887-Contact-Us}{Contact
  Us}
\item
  \href{https://www.nytco.com/careers/}{Work with us}
\item
  \href{https://nytmediakit.com/}{Advertise}
\item
  \href{http://www.tbrandstudio.com/}{T Brand Studio}
\item
  \href{https://www.nytimes3xbfgragh.onion/privacy/cookie-policy\#how-do-i-manage-trackers}{Your
  Ad Choices}
\item
  \href{https://www.nytimes3xbfgragh.onion/privacy}{Privacy}
\item
  \href{https://help.nytimes3xbfgragh.onion/hc/en-us/articles/115014893428-Terms-of-service}{Terms
  of Service}
\item
  \href{https://help.nytimes3xbfgragh.onion/hc/en-us/articles/115014893968-Terms-of-sale}{Terms
  of Sale}
\item
  \href{https://spiderbites.nytimes3xbfgragh.onion}{Site Map}
\item
  \href{https://help.nytimes3xbfgragh.onion/hc/en-us}{Help}
\item
  \href{https://www.nytimes3xbfgragh.onion/subscription?campaignId=37WXW}{Subscriptions}
\end{itemize}
