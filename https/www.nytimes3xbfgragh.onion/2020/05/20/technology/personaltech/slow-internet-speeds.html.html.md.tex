Sections

SEARCH

\protect\hyperlink{site-content}{Skip to
content}\protect\hyperlink{site-index}{Skip to site index}

\href{https://www.nytimes3xbfgragh.onion/section/technology/personaltech}{Personal
Tech}

\href{https://myaccount.nytimes3xbfgragh.onion/auth/login?response_type=cookie\&client_id=vi}{}

\href{https://www.nytimes3xbfgragh.onion/section/todayspaper}{Today's
Paper}

\href{/section/technology/personaltech}{Personal
Tech}\textbar{}Everything You Need to Know About Slow Internet Speeds

\href{https://nyti.ms/2WKViuB}{https://nyti.ms/2WKViuB}

\begin{itemize}
\item
\item
\item
\item
\item
\end{itemize}

\href{https://www.nytimes3xbfgragh.onion/spotlight/at-home?action=click\&pgtype=Article\&state=default\&region=TOP_BANNER\&context=at_home_menu}{At
Home}

\begin{itemize}
\tightlist
\item
  \href{https://www.nytimes3xbfgragh.onion/2020/07/28/books/time-for-a-literary-road-trip.html?action=click\&pgtype=Article\&state=default\&region=TOP_BANNER\&context=at_home_menu}{Take:
  A Literary Road Trip}
\item
  \href{https://www.nytimes3xbfgragh.onion/2020/07/29/magazine/bored-with-your-home-cooking-some-smoky-eggplant-will-fix-that.html?action=click\&pgtype=Article\&state=default\&region=TOP_BANNER\&context=at_home_menu}{Cook:
  Smoky Eggplant}
\item
  \href{https://www.nytimes3xbfgragh.onion/2020/07/27/travel/moose-michigan-isle-royale.html?action=click\&pgtype=Article\&state=default\&region=TOP_BANNER\&context=at_home_menu}{Look
  Out: For Moose}
\item
  \href{https://www.nytimes3xbfgragh.onion/interactive/2020/at-home/even-more-reporters-editors-diaries-lists-recommendations.html?action=click\&pgtype=Article\&state=default\&region=TOP_BANNER\&context=at_home_menu}{Explore:
  Reporters' Obsessions}
\end{itemize}

Advertisement

\protect\hyperlink{after-top}{Continue reading the main story}

Supported by

\protect\hyperlink{after-sponsor}{Continue reading the main story}

tech fix

\hypertarget{everything-you-need-to-know-about-slow-internet-speeds}{%
\section{Everything You Need to Know About Slow Internet
Speeds}\label{everything-you-need-to-know-about-slow-internet-speeds}}

Our crummy connections are the biggest tech headache in the pandemic.
Here's a comprehensive guide to what to do about them.

\includegraphics{https://static01.graylady3jvrrxbe.onion/images/2020/05/20/business/20Techfix-illo/20Techfix-illo-articleLarge.gif?quality=75\&auto=webp\&disable=upscale}

\href{https://www.nytimes3xbfgragh.onion/by/brian-x-chen}{\includegraphics{https://static01.graylady3jvrrxbe.onion/images/2018/02/16/multimedia/author-brian-x-chen/author-brian-x-chen-thumbLarge.jpg}}

By \href{https://www.nytimes3xbfgragh.onion/by/brian-x-chen}{Brian X.
Chen}

\begin{itemize}
\item
  May 20, 2020
\item
  \begin{itemize}
  \item
  \item
  \item
  \item
  \item
  \end{itemize}
\end{itemize}

Restricted to our homes for months now, many of us have been putting up
with a persistent annoyance: a lousy internet connection.

When we are working, a video call with colleagues becomes pixelated,
with delayed audio. When we are relaxing, movies and video games take
ages to download. In the worst cases, the connection drops altogether.

As people have hunkered down to contain the spread of the coronavirus,
\href{https://www.nytimes3xbfgragh.onion/2020/03/26/business/coronavirus-internet-traffic-speed.html}{average
internet speeds all over the world have slowed}. Some broadband
providers are feeling crushed by the heavy traffic. And dated internet
equipment can create a bottleneck for our speeds.

Even the most tech savvy are affected. Keerti Melkote, the founder of
Aruba Networks, a division of Hewlett Packard Enterprise that offers
Wi-Fi products for businesses, said that in recent weeks, his DSL
service from AT\&T had dropped periodically. He waited several days for
a technician to arrive and is now contemplating subscribing to Comcast
for a second internet connection.

``I had three or four days of calls, and I had to go find a particular
spot in my house where I had better coverage,'' Mr. Melkote said.

At the beginning of the pandemic, my internet also became unbearably
slow and suffered several outages. So I asked experts to explain what's
causing our internet problems --- and the different remedies.

\hypertarget{first-diagnose-the-problem}{%
\subsection{First, diagnose the
problem.}\label{first-diagnose-the-problem}}

What's causing your slow speeds --- your internet provider or your
equipment at home? Here's a method to figuring that out.

\begin{itemize}
\item
  Download an internet speed test app on your phone, like Speedtest by
  Ookla (free for
  \href{https://apps.apple.com/us/app/speedtest-by-ookla/id300704847}{iPhones}
  and
  \href{https://play.google.com/store/apps/details?id=org.zwanoo.android.speedtest\&hl=en_US}{Android
  phones}).
\item
  Stand near your router and use the app to run a speed test.
\item
  Move to a room farther away from the router and run the speed test
  again.
\item
  Compare the results.
\end{itemize}

Less than 15 megabits a second is pretty slow. Speeds of about 25
megabits a second are sufficient for streaming high-definition video;
more than 40 megabits a second is ideal for streaming lots of video and
playing video games.

If the speed test results were fast near your Wi-Fi router but slow
farther away, the problem is probably your router, said Sanjay Noronha,
the product lead of Google's Nest Wifi internet router. If speeds were
slow in both test locations, the issue is probably your internet
provider.

\hypertarget{if-its-your-router-heres-what-to-do}{%
\subsection{If it's your router, here's what to
do.}\label{if-its-your-router-heres-what-to-do}}

If you have pinpointed that the problem is your router, the bad news is
that you may have to buy new equipment. The good news is that there are
many approaches to improving your Wi-Fi connection.

Start by asking yourself these questions:

\begin{itemize}
\item
  \textbf{How old is my router?} If it's more than five years old, you
  should definitely replace it. In 2015, the
  \href{https://www.fcc.gov/document/5-ghz-u-nii-ro}{Federal
  Communications Commission removed restrictions} that had limited the
  wireless transmission power of Wi-Fi routers, allowing new routers to
  be 20 times more powerful than they were before. Upgrading to a newer
  router will probably be one of your most life-changing tech purchases.
\item
  \textbf{Where is my router placed?} Ideally, your router should be in
  a central location in your home so that the signal covers as many
  rooms as possible. In addition, your router should be out in the open,
  like on top of a shelf, not hidden inside cabinets or under a desk, to
  beam a clear signal. You should also avoid placing the router near
  objects and materials that cause interference, like large fish tanks
  and metal.
\item
  \textbf{How big is my home?} If you have a home with multiple stories
  and lots of rooms, and your Wi-Fi is weak in some areas, the best
  solution is to buy a so-called mesh network system. It's a system of
  multiple Wi-Fi access points, including a main router and satellite
  hubs, that lets you connect multiple wireless access points together
  to blanket your home with a strong internet connection.

  My favorite mesh systems are
  \href{https://store.google.com/us/config/nest_wifi}{Google Wifi}
  and\href{https://www.amazon.com/Introducing-eero-mesh-WiFi-system-3-pack-/dp/B07WMLPSRL}{Amazon's
  Eero}, which start at \$99 for a single router and can be bundled with
  additional access points. In general,
  \href{https://www.nytimes3xbfgragh.onion/2017/04/26/technology/personaltech/mesh-network-vs-router.html}{I
  recommend mesh systems} even for smaller homes, because they are fast
  and very easy to install.
\item
  \textbf{Are my other devices slowing down my connection?} Gadgets with
  slower internet technology can slow down speeds for all your other
  devices.

  For example, the iPhone 5 from 2012 uses an older-generation Wi-Fi
  standard. Newer iPhones, from 2014 and later, use a faster wireless
  standard.

  Let's say you own a new iPhone and your teenager owns the iPhone 5. If
  your teenager begins downloading a video on the iPhone 5 and then you
  start downloading something on your iPhone, the older phone will take
  longer to finish before the signal frees up for your phone to download
  at maximum speed.

  As a remedy, many modern Wi-Fi routers offer settings that can give
  specific devices
  \href{https://support.google.com/wifi/answer/6246483?hl=en}{a priority
  for faster speeds}. Consult your router's instruction manual for the
  steps. In this hypothetical example, you would want to give your new
  iPhone top priority and move your teenager's old iPhone to the bottom.
\item
  \textbf{Are my neighbors slowing down my connection?} In apartment
  buildings crowded with gadgets, the devices' signals are fighting for
  room on the same radio channels. You can see what radio channels your
  neighbors' devices are using with scanning apps like
  \href{https://play.google.com/store/apps/details?id=abdelrahman.wifianalyzerpro\&hl=en_US}{WiFi
  Analyzer}. Then consult your router's instruction manual for steps on
  picking a clearer radio channel.

  This step is tedious, and many modern routers automatically choose the
  clearest radio channel for you. In general, replacing an outdated
  router is the most practical solution.
\end{itemize}

\hypertarget{if-its-your-service-provider-theres-not-much-to-do}{%
\subsection{If it's your service provider, there's not much to
do.}\label{if-its-your-service-provider-theres-not-much-to-do}}

If you have determined that your internet provider's service is the root
of the issue, your only option is to call your internet service provider
and ask for help.

When you call, ask a support agent these questions:

\begin{itemize}
\item
  \textbf{Why are my speeds slow?} Occasionally a support agent can
  analyze your internet performance and make changes to speed up your
  connection. This rarely happens, and more often a technician will need
  to pay a visit.
\item
  \textbf{Does my modem need to be replaced?} The modem, which is the
  box that connects your home to the internet provider's service, also
  can become outdated and occasionally needs to be replaced. If the
  support agent confirms the modem is old, you can schedule an
  appointment for a technician to install a new one.

  Or you can buy your own modem and call the internet provider to
  activate it. Wirecutter, our sister publication that tests products,
  recommends
  \href{https://thewirecutter.com/reviews/best-cable-modem/}{modems from
  Motorola and Netgear}, which cost about \$80 to \$90.
\item
  \textbf{Can I buy faster speeds?} Your provider may offer packages
  with more bandwidth meant for higher-quality video streaming and
  faster downloads. Ask about your options.
\end{itemize}

As a last resort, you can turn to backups. Many modern phones come with
a hot spot feature, which turns the device's cellular connection into a
miniature Wi-Fi network. (Apple and Google list steps on their websites
on how to use the hot spot feature on
\href{https://support.apple.com/en-us/HT204023}{iPhones} and
\href{https://support.google.com/android/answer/9059108?hl=en}{Androids}.)

Whatever you do, be patient. In these trying times, everything takes
longer.

As for me, I confirmed my slow speeds were related to my internet
provider, Monkeybrains. I called to report the issue, and after more
than a month, a technician replaced the antenna on our roof. Now my
speeds are even faster than before the pandemic, so it was well worth
the wait.

Advertisement

\protect\hyperlink{after-bottom}{Continue reading the main story}

\hypertarget{site-index}{%
\subsection{Site Index}\label{site-index}}

\hypertarget{site-information-navigation}{%
\subsection{Site Information
Navigation}\label{site-information-navigation}}

\begin{itemize}
\tightlist
\item
  \href{https://help.nytimes3xbfgragh.onion/hc/en-us/articles/115014792127-Copyright-notice}{©~2020~The
  New York Times Company}
\end{itemize}

\begin{itemize}
\tightlist
\item
  \href{https://www.nytco.com/}{NYTCo}
\item
  \href{https://help.nytimes3xbfgragh.onion/hc/en-us/articles/115015385887-Contact-Us}{Contact
  Us}
\item
  \href{https://www.nytco.com/careers/}{Work with us}
\item
  \href{https://nytmediakit.com/}{Advertise}
\item
  \href{http://www.tbrandstudio.com/}{T Brand Studio}
\item
  \href{https://www.nytimes3xbfgragh.onion/privacy/cookie-policy\#how-do-i-manage-trackers}{Your
  Ad Choices}
\item
  \href{https://www.nytimes3xbfgragh.onion/privacy}{Privacy}
\item
  \href{https://help.nytimes3xbfgragh.onion/hc/en-us/articles/115014893428-Terms-of-service}{Terms
  of Service}
\item
  \href{https://help.nytimes3xbfgragh.onion/hc/en-us/articles/115014893968-Terms-of-sale}{Terms
  of Sale}
\item
  \href{https://spiderbites.nytimes3xbfgragh.onion}{Site Map}
\item
  \href{https://help.nytimes3xbfgragh.onion/hc/en-us}{Help}
\item
  \href{https://www.nytimes3xbfgragh.onion/subscription?campaignId=37WXW}{Subscriptions}
\end{itemize}
