Sections

SEARCH

\protect\hyperlink{site-content}{Skip to
content}\protect\hyperlink{site-index}{Skip to site index}

\href{https://www.nytimes3xbfgragh.onion/section/health}{Health}

\href{https://myaccount.nytimes3xbfgragh.onion/auth/login?response_type=cookie\&client_id=vi}{}

\href{https://www.nytimes3xbfgragh.onion/section/todayspaper}{Today's
Paper}

\href{/section/health}{Health}\textbar{}How to Improve and Protect
Nursing Homes From Outbreaks

\url{https://nyti.ms/2TxGnC9}

\begin{itemize}
\item
\item
\item
\item
\item
\item
\end{itemize}

\href{https://www.nytimes3xbfgragh.onion/news-event/coronavirus?action=click\&pgtype=Article\&state=default\&region=TOP_BANNER\&context=storylines_menu}{The
Coronavirus Outbreak}

\begin{itemize}
\tightlist
\item
  live\href{https://www.nytimes3xbfgragh.onion/2020/08/01/world/coronavirus-covid-19.html?action=click\&pgtype=Article\&state=default\&region=TOP_BANNER\&context=storylines_menu}{Latest
  Updates}
\item
  \href{https://www.nytimes3xbfgragh.onion/interactive/2020/us/coronavirus-us-cases.html?action=click\&pgtype=Article\&state=default\&region=TOP_BANNER\&context=storylines_menu}{Maps
  and Cases}
\item
  \href{https://www.nytimes3xbfgragh.onion/interactive/2020/science/coronavirus-vaccine-tracker.html?action=click\&pgtype=Article\&state=default\&region=TOP_BANNER\&context=storylines_menu}{Vaccine
  Tracker}
\item
  \href{https://www.nytimes3xbfgragh.onion/interactive/2020/07/29/us/schools-reopening-coronavirus.html?action=click\&pgtype=Article\&state=default\&region=TOP_BANNER\&context=storylines_menu}{What
  School May Look Like}
\item
  \href{https://www.nytimes3xbfgragh.onion/live/2020/07/31/business/stock-market-today-coronavirus?action=click\&pgtype=Article\&state=default\&region=TOP_BANNER\&context=storylines_menu}{Economy}
\end{itemize}

Advertisement

\protect\hyperlink{after-top}{Continue reading the main story}

Supported by

\protect\hyperlink{after-sponsor}{Continue reading the main story}

the new old age

\hypertarget{how-to-improve-and-protect-nursing-homes-from-outbreaks}{%
\section{How to Improve and Protect Nursing Homes From
Outbreaks}\label{how-to-improve-and-protect-nursing-homes-from-outbreaks}}

More than a third of America's Covid-19 deaths can be traced back to
these facilities. Experts suggest several ways to make them safer.

\includegraphics{https://static01.graylady3jvrrxbe.onion/images/2020/05/26/science/00SCI-SPAN-VIRUS-HOMES3/merlin_172683525_81033965-7eed-46ec-ac8b-49d520ef1ddb-articleLarge.jpg?quality=75\&auto=webp\&disable=upscale}

By \href{https://www.nytimes3xbfgragh.onion/by/paula-span}{Paula Span}

\begin{itemize}
\item
  Published May 22, 2020Updated July 23, 2020
\item
  \begin{itemize}
  \item
  \item
  \item
  \item
  \item
  \item
  \end{itemize}
\end{itemize}

The doctors, researchers and advocates who have been paying close
attention for years are appalled at the way the
\href{https://www.nytimes3xbfgragh.onion/2020/07/23/nyregion/nursing-homes-deaths-cuomo.html}{coronavirus
has devastated the nation's nursing homes} --- but they're not shocked.

``Every geriatrician knew what was coming,'' said Dr. Mike Wasserman, a
geriatrician and president of the California Association of Long Term
Care Medicine.

Robyn Grant, the director of public policy and advocacy for the National
Consumer Voice for Quality Long-Term Care: ``The sheer numbers are
horrifying. The underlying factors that have contributed are no
surprise; they've been issues of concern for a long time.''

\href{https://www.nytimes3xbfgragh.onion/interactive/2020/05/09/us/coronavirus-cases-nursing-homes-us.html}{A
New York Times analysis} as of May 21 showed that
\href{https://www.nytimes3xbfgragh.onion/interactive/2020/05/09/us/coronavirus-cases-nursing-homes-us.html}{more
than 34,000 deaths} --- 37 percent of the nation's fatalities from
Covid-19 --- occurred among residents and staff in long-term care
facilities. In 15 states, long-term care accounted for more than half of
all Covid-19 deaths.

Because states report cases in varying ways, and some report few numbers
at all, ``all of this could be undercounted,'' said David Grabowski, a
health care policy researcher at Harvard Medical School, noting that
testing remains inadequate.

Not until mid-April did the federal Center for Medicare and Medicaid
Services announce
\href{https://www.cms.gov/files/document/qso-20-26-nh.pdf}{a reporting
system} to track Covid-19 in nursing homes and funnel the data to the
Centers for Disease Control and Prevention.

But because nursing home care receives scant public attention even in
better times, advocates like Ms. Grant see an opportunity, however
grimly won. ``People have been horrified by what's happening, and that's
shining a light on the changes we need to see,'' she said.

It's not hard to understand why the virus has streaked through nursing
homes like ``fire through dry grass,'' as Gov. Andrew Cuomo of New York
put it. Dr. Philip Sloane, in a recent editorial in The Journal of the
American Medical Directors Association,
\href{https://www.jamda.com/article/S1525-8610(20)30350-9/fulltext}{compared
them to cruise ships and prisons} as incubators for disease.

``All three have large numbers of people in relatively small spaces, so
it's hard to do isolation,'' said Dr. Sloane, a geriatrician who
co-directs the Program on Aging, Disability and Long-Term Care at the
University of North Carolina. ``They have congregant meals prepared in
central kitchens, staff that have a lot of personal contacts with
residents. They have activities that bring a lot of people together.''

Nursing home residents, of course, are frailer and sicker than cruisers
and inmates. ``These nursing homes are yesterday's hospitals,'' minus
the on-site medical staff, Dr. Wasserman said.

\hypertarget{latest-updates-global-coronavirus-outbreak}{%
\section{\texorpdfstring{\href{https://www.nytimes3xbfgragh.onion/2020/08/01/world/coronavirus-covid-19.html?action=click\&pgtype=Article\&state=default\&region=MAIN_CONTENT_1\&context=storylines_live_updates}{Latest
Updates: Global Coronavirus
Outbreak}}{Latest Updates: Global Coronavirus Outbreak}}\label{latest-updates-global-coronavirus-outbreak}}

Updated 2020-08-01T19:54:00.494Z

\begin{itemize}
\tightlist
\item
  \href{https://www.nytimes3xbfgragh.onion/2020/08/01/world/coronavirus-covid-19.html?action=click\&pgtype=Article\&state=default\&region=MAIN_CONTENT_1\&context=storylines_live_updates\#link-3ac56579}{Top
  officials work to break impasse over jobless benefit.}
\item
  \href{https://www.nytimes3xbfgragh.onion/2020/08/01/world/coronavirus-covid-19.html?action=click\&pgtype=Article\&state=default\&region=MAIN_CONTENT_1\&context=storylines_live_updates\#link-8796723}{The
  virus picks up dangerous speed in the Midwest, and in areas that had
  seen success.}
\item
  \href{https://www.nytimes3xbfgragh.onion/2020/08/01/world/coronavirus-covid-19.html?action=click\&pgtype=Article\&state=default\&region=MAIN_CONTENT_1\&context=storylines_live_updates\#link-25930521}{Thousands
  in Berlin protest Germany's coronavirus measures.}
\end{itemize}

\href{https://www.nytimes3xbfgragh.onion/2020/08/01/world/coronavirus-covid-19.html?action=click\&pgtype=Article\&state=default\&region=MAIN_CONTENT_1\&context=storylines_live_updates}{See
more updates}

More live coverage:
\href{https://www.nytimes3xbfgragh.onion/live/2020/07/31/business/stock-market-today-coronavirus?action=click\&pgtype=Article\&state=default\&region=MAIN_CONTENT_1\&context=storylines_live_updates}{Markets}

What needs to change?

Over the coming weeks,
\href{https://onlinelibrary.wiley.com/doi/full/10.1111/jgs.16477}{experts
on nursing homes} say, the top priority should be to greatly expand
rapid testing and tracing for residents and staff, as some states have
begun to require, and to acquire sufficient protective equipment.

``We've basically locked nursing homes down, yet Covid is still
spreading because we don't know who has it and we don't have the P.P.E.
to protect the staff,'' Dr. Grabowski said.

For this pandemic and beyond, researchers and advocates suggest several
broad ideas for improvement.

\includegraphics{https://static01.graylady3jvrrxbe.onion/images/2020/05/20/science/00SCI-SPAN-VIRUS-HOMES1/00SCI-SPAN-VIRUS-HOMES1-articleLarge.jpg?quality=75\&auto=webp\&disable=upscale}

\hypertarget{increase-infection-control}{%
\subsubsection{\texorpdfstring{\textbf{Increase infection
control}}{Increase infection control}}\label{increase-infection-control}}

Even before the coronavirus arrived in nursing homes, they had a poor
record of preventing contagion.

``Facilities know it's a problem, yet it's remained one of the top
violations in the country'' in federally mandated inspections, Ms. Grant
said. ``These problems are cited year after year.''

In 2016, Medicare began requiring each facility to employ an ``infection
preventionist'' to oversee policies and train workers. But that is often
a part-time position. ``The person in charge of infection control always
has another job,'' Dr. Sloane said. ``That person also doesn't have much
clout.''

Last year, Medicare proposed relaxing that rule, so that the
preventionist no longer needed to be an employee, but must log
``sufficient hours,'' which Ms. Grant called ``part of the deregulatory
policy of this administration.'' She thinks the pandemic has instead
spotlighted the need for mandatory, full-time infection preventionists.

\hypertarget{change-designs}{%
\subsubsection{\texorpdfstring{\textbf{Change
designs}}{Change designs}}\label{change-designs}}

American nursing homes have, on average, about 100 beds, in rooms
flanking long corridors, with staff moving from one to another.
Residents typically share a small room and bathroom --- an arrangement
that many dislike, and one that provides excellent conditions for viral
transmission. Assisted living complexes appear to have fared somewhat
better during the pandemic, partly because individual apartments make
isolation easier.

``It's time to really focus on private rooms in nursing homes,'' said
Karl Pillemer, a gerontologist and researcher at Cornell University. In
\href{https://www.nytimes3xbfgragh.onion/2017/12/22/health/green-houses-nursing-homes.html}{the
Green House model}, for example, a dozen residents live in private rooms
with homelike common spaces and assigned staff who know them well. This
approach has gained ground very slowly, with 268 homes, of more than
15,000 nursing homes nationwide.

But the Green House Project reports that as of May 21, in 245 homes with
2,653 residents, only nine have had Covid-19 cases, resulting in six
deaths. With several small buildings on a campus instead of one large
one, administrators could also more easily quarantine infected
residents, Dr. Sloane pointed out.

\href{https://www.nytimes3xbfgragh.onion/news-event/coronavirus?action=click\&pgtype=Article\&state=default\&region=MAIN_CONTENT_3\&context=storylines_faq}{}

\hypertarget{the-coronavirus-outbreak-}{%
\subsubsection{The Coronavirus Outbreak
›}\label{the-coronavirus-outbreak-}}

\hypertarget{frequently-asked-questions}{%
\paragraph{Frequently Asked
Questions}\label{frequently-asked-questions}}

Updated July 27, 2020

\begin{itemize}
\item ~
  \hypertarget{should-i-refinance-my-mortgage}{%
  \paragraph{Should I refinance my
  mortgage?}\label{should-i-refinance-my-mortgage}}

  \begin{itemize}
  \tightlist
  \item
    \href{https://www.nytimes3xbfgragh.onion/article/coronavirus-money-unemployment.html?action=click\&pgtype=Article\&state=default\&region=MAIN_CONTENT_3\&context=storylines_faq}{It
    could be a good idea,} because mortgage rates have
    \href{https://www.nytimes3xbfgragh.onion/2020/07/16/business/mortgage-rates-below-3-percent.html?action=click\&pgtype=Article\&state=default\&region=MAIN_CONTENT_3\&context=storylines_faq}{never
    been lower.} Refinancing requests have pushed mortgage applications
    to some of the highest levels since 2008, so be prepared to get in
    line. But defaults are also up, so if you're thinking about buying a
    home, be aware that some lenders have tightened their standards.
  \end{itemize}
\item ~
  \hypertarget{what-is-school-going-to-look-like-in-september}{%
  \paragraph{What is school going to look like in
  September?}\label{what-is-school-going-to-look-like-in-september}}

  \begin{itemize}
  \tightlist
  \item
    It is unlikely that many schools will return to a normal schedule
    this fall, requiring the grind of
    \href{https://www.nytimes3xbfgragh.onion/2020/06/05/us/coronavirus-education-lost-learning.html?action=click\&pgtype=Article\&state=default\&region=MAIN_CONTENT_3\&context=storylines_faq}{online
    learning},
    \href{https://www.nytimes3xbfgragh.onion/2020/05/29/us/coronavirus-child-care-centers.html?action=click\&pgtype=Article\&state=default\&region=MAIN_CONTENT_3\&context=storylines_faq}{makeshift
    child care} and
    \href{https://www.nytimes3xbfgragh.onion/2020/06/03/business/economy/coronavirus-working-women.html?action=click\&pgtype=Article\&state=default\&region=MAIN_CONTENT_3\&context=storylines_faq}{stunted
    workdays} to continue. California's two largest public school
    districts --- Los Angeles and San Diego --- said on July 13, that
    \href{https://www.nytimes3xbfgragh.onion/2020/07/13/us/lausd-san-diego-school-reopening.html?action=click\&pgtype=Article\&state=default\&region=MAIN_CONTENT_3\&context=storylines_faq}{instruction
    will be remote-only in the fall}, citing concerns that surging
    coronavirus infections in their areas pose too dire a risk for
    students and teachers. Together, the two districts enroll some
    825,000 students. They are the largest in the country so far to
    abandon plans for even a partial physical return to classrooms when
    they reopen in August. For other districts, the solution won't be an
    all-or-nothing approach.
    \href{https://bioethics.jhu.edu/research-and-outreach/projects/eschool-initiative/school-policy-tracker/}{Many
    systems}, including the nation's largest, New York City, are
    devising
    \href{https://www.nytimes3xbfgragh.onion/2020/06/26/us/coronavirus-schools-reopen-fall.html?action=click\&pgtype=Article\&state=default\&region=MAIN_CONTENT_3\&context=storylines_faq}{hybrid
    plans} that involve spending some days in classrooms and other days
    online. There's no national policy on this yet, so check with your
    municipal school system regularly to see what is happening in your
    community.
  \end{itemize}
\item ~
  \hypertarget{is-the-coronavirus-airborne}{%
  \paragraph{Is the coronavirus
  airborne?}\label{is-the-coronavirus-airborne}}

  \begin{itemize}
  \tightlist
  \item
    The coronavirus
    \href{https://www.nytimes3xbfgragh.onion/2020/07/04/health/239-experts-with-one-big-claim-the-coronavirus-is-airborne.html?action=click\&pgtype=Article\&state=default\&region=MAIN_CONTENT_3\&context=storylines_faq}{can
    stay aloft for hours in tiny droplets in stagnant air}, infecting
    people as they inhale, mounting scientific evidence suggests. This
    risk is highest in crowded indoor spaces with poor ventilation, and
    may help explain super-spreading events reported in meatpacking
    plants, churches and restaurants.
    \href{https://www.nytimes3xbfgragh.onion/2020/07/06/health/coronavirus-airborne-aerosols.html?action=click\&pgtype=Article\&state=default\&region=MAIN_CONTENT_3\&context=storylines_faq}{It's
    unclear how often the virus is spread} via these tiny droplets, or
    aerosols, compared with larger droplets that are expelled when a
    sick person coughs or sneezes, or transmitted through contact with
    contaminated surfaces, said Linsey Marr, an aerosol expert at
    Virginia Tech. Aerosols are released even when a person without
    symptoms exhales, talks or sings, according to Dr. Marr and more
    than 200 other experts, who
    \href{https://academic.oup.com/cid/article/doi/10.1093/cid/ciaa939/5867798}{have
    outlined the evidence in an open letter to the World Health
    Organization}.
  \end{itemize}
\item ~
  \hypertarget{what-are-the-symptoms-of-coronavirus}{%
  \paragraph{What are the symptoms of
  coronavirus?}\label{what-are-the-symptoms-of-coronavirus}}

  \begin{itemize}
  \tightlist
  \item
    Common symptoms
    \href{https://www.nytimes3xbfgragh.onion/article/symptoms-coronavirus.html?action=click\&pgtype=Article\&state=default\&region=MAIN_CONTENT_3\&context=storylines_faq}{include
    fever, a dry cough, fatigue and difficulty breathing or shortness of
    breath.} Some of these symptoms overlap with those of the flu,
    making detection difficult, but runny noses and stuffy sinuses are
    less common.
    \href{https://www.nytimes3xbfgragh.onion/2020/04/27/health/coronavirus-symptoms-cdc.html?action=click\&pgtype=Article\&state=default\&region=MAIN_CONTENT_3\&context=storylines_faq}{The
    C.D.C. has also} added chills, muscle pain, sore throat, headache
    and a new loss of the sense of taste or smell as symptoms to look
    out for. Most people fall ill five to seven days after exposure, but
    symptoms may appear in as few as two days or as many as 14 days.
  \end{itemize}
\item ~
  \hypertarget{does-asymptomatic-transmission-of-covid-19-happen}{%
  \paragraph{Does asymptomatic transmission of Covid-19
  happen?}\label{does-asymptomatic-transmission-of-covid-19-happen}}

  \begin{itemize}
  \tightlist
  \item
    So far, the evidence seems to show it does. A widely cited
    \href{https://www.nature.com/articles/s41591-020-0869-5}{paper}
    published in April suggests that people are most infectious about
    two days before the onset of coronavirus symptoms and estimated that
    44 percent of new infections were a result of transmission from
    people who were not yet showing symptoms. Recently, a top expert at
    the World Health Organization stated that transmission of the
    coronavirus by people who did not have symptoms was ``very rare,''
    \href{https://www.nytimes3xbfgragh.onion/2020/06/09/world/coronavirus-updates.html?action=click\&pgtype=Article\&state=default\&region=MAIN_CONTENT_3\&context=storylines_faq\#link-1f302e21}{but
    she later walked back that statement.}
  \end{itemize}
\end{itemize}

Although new nursing homes offer private rooms, very few are being
built. But renovation can create similar small households within older
nursing homes, said Martin Siefering, a principal architect who
co-directs the senior living practice at Perkins Eastman.

At the New Jewish Home's campus in Mamaroneck, N.Y., for instance, the
firm converted 59 of 300 beds into five small house communities. It also
retrofitted nonprofit nursing homes in Tulsa, Okla., and Ocean City,
N.J., to create smaller households.

\hypertarget{raise-workers-pay}{%
\subsubsection{\texorpdfstring{\textbf{Raise workers'
pay}}{Raise workers' pay}}\label{raise-workers-pay}}

A common way that the coronavirus enters nursing homes is through the
employees, inadvertently. ``It's staff bringing this in and spreading
the virus to residents,'' Dr. Grabowski said.

He pointed out that the aides who provide hands-on care are poorly paid
(median hourly wage last year: \$13.38,
\href{https://phinational.org/resource/u-s-nursing-assistants-employed-in-nursing-homes-key-facts-2019/}{according
to PHI}), so ``they're often piecing together multiple part-time jobs,''
possibly spreading infections not only within but between facilities.

``Our hospital workers are held up as heroes, and they are,'' Dr.
Grabowski said. ``Nursing home workers are, too. And they're making
minimum wage.''

Higher wages with hazard pay, health coverage and paid sick leave so
that workers can stay home when they are ill could reduce both rampant
staff turnover and viral transmission. A few long-term care
administrators are experimenting with having staff live on campus during
the crisis.

\hypertarget{reconsider-visitor-policies}{%
\subsubsection{\texorpdfstring{\textbf{Reconsider visitor
policies}}{Reconsider visitor policies}}\label{reconsider-visitor-policies}}

It made sense to bar outsiders during the height of the pandemic, when
knowledge of symptoms and transmissions was even more incomplete than
now. But for long-term residents, isolation carries its own perils.

``We already knew families were providing care, but the extent of it has
been eye-opening,'' Ms. Grant said. ``They tell us, `I help my mom eat.'
`I'm the one that helps her get enough fluids.'''

Some geriatricians have called on nursing homes to designate a relative
or friend to undergo regular testing and learn the proper use of
protective equipment, then be allowed access. Medicare has just issued
general
\href{https://www.cms.gov/files/document/nursing-home-reopening-recommendations-state-and-local-officials.pdf}{guidance
for states} hoping to slowly reopen facilities.

``Older people's voices are missing from this discussion,'' Dr. Pillemer
said. ``They may want to make the decision to see family members, at
their own risk.''

Making nursing homes better and safer serves not only a humanitarian
purpose, he added. Governors issued stay-at-home orders to prevent the
coronavirus from overwhelming health care systems, particularly
hospitals. Nursing home residents were, disproportionately, the patients
filling those intensive-care units.

``It's not the nail salons --- these deaths are in long-term care,'' Dr.
Pillemer said. ``Stopping the virus in long-term care, which is fully
possible, is the key to reopening the country.''

Advertisement

\protect\hyperlink{after-bottom}{Continue reading the main story}

\hypertarget{site-index}{%
\subsection{Site Index}\label{site-index}}

\hypertarget{site-information-navigation}{%
\subsection{Site Information
Navigation}\label{site-information-navigation}}

\begin{itemize}
\tightlist
\item
  \href{https://help.nytimes3xbfgragh.onion/hc/en-us/articles/115014792127-Copyright-notice}{©~2020~The
  New York Times Company}
\end{itemize}

\begin{itemize}
\tightlist
\item
  \href{https://www.nytco.com/}{NYTCo}
\item
  \href{https://help.nytimes3xbfgragh.onion/hc/en-us/articles/115015385887-Contact-Us}{Contact
  Us}
\item
  \href{https://www.nytco.com/careers/}{Work with us}
\item
  \href{https://nytmediakit.com/}{Advertise}
\item
  \href{http://www.tbrandstudio.com/}{T Brand Studio}
\item
  \href{https://www.nytimes3xbfgragh.onion/privacy/cookie-policy\#how-do-i-manage-trackers}{Your
  Ad Choices}
\item
  \href{https://www.nytimes3xbfgragh.onion/privacy}{Privacy}
\item
  \href{https://help.nytimes3xbfgragh.onion/hc/en-us/articles/115014893428-Terms-of-service}{Terms
  of Service}
\item
  \href{https://help.nytimes3xbfgragh.onion/hc/en-us/articles/115014893968-Terms-of-sale}{Terms
  of Sale}
\item
  \href{https://spiderbites.nytimes3xbfgragh.onion}{Site Map}
\item
  \href{https://help.nytimes3xbfgragh.onion/hc/en-us}{Help}
\item
  \href{https://www.nytimes3xbfgragh.onion/subscription?campaignId=37WXW}{Subscriptions}
\end{itemize}
