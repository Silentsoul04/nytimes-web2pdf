Sections

SEARCH

\protect\hyperlink{site-content}{Skip to
content}\protect\hyperlink{site-index}{Skip to site index}

\href{https://myaccount.nytimes3xbfgragh.onion/auth/login?response_type=cookie\&client_id=vi}{}

\href{https://www.nytimes3xbfgragh.onion/section/todayspaper}{Today's
Paper}

\href{/section/opinion}{Opinion}\textbar{}Michael Flynn and the
Presumption of Guilt

\url{https://nyti.ms/3bXHZeO}

\begin{itemize}
\item
\item
\item
\item
\item
\item
\end{itemize}

Advertisement

\protect\hyperlink{after-top}{Continue reading the main story}

\href{/section/opinion}{Opinion}

Supported by

\protect\hyperlink{after-sponsor}{Continue reading the main story}

\hypertarget{michael-flynn-and-the-presumption-of-guilt}{%
\section{Michael Flynn and the Presumption of
Guilt}\label{michael-flynn-and-the-presumption-of-guilt}}

A distrust of prosecutorial power should not be abandoned.

\href{https://www.nytimes3xbfgragh.onion/by/bret-stephens}{\includegraphics{https://static01.graylady3jvrrxbe.onion/images/2017/08/27/insider/bretstephens/bretstephens-thumbLarge-v6.png}}

By \href{https://www.nytimes3xbfgragh.onion/by/bret-stephens}{Bret
Stephens}

Opinion Columnist

\begin{itemize}
\item
  May 22, 2020
\item
  \begin{itemize}
  \item
  \item
  \item
  \item
  \item
  \item
  \end{itemize}
\end{itemize}

\includegraphics{https://static01.graylady3jvrrxbe.onion/images/2020/05/22/opinion/22stephens1/merlin_172757931_70dffd81-4c9c-40f4-b5ab-328e628ebc15-articleLarge.jpg?quality=75\&auto=webp\&disable=upscale}

Of all the low moments in the 2016 Republican presidential convention
--- there were many ---
\href{https://www.nytimes3xbfgragh.onion/2020/05/29/us/politics/flynn-russian-ambassador-transcripts.html}{Michael
Flynn's} speech ranks high.

``Damn right, exactly right,'' the fired, retired three-star general
said in answer to audience chants of ``lock her up.'' ``And you know why
we're saying that? We're saying that because if I, a guy who knows this
business, if I did a tenth, a tenth of what {[}Hillary Clinton{]} did, I
would be in jail today.''

This was said by a man who, as a former head of the Defense Intelligence
Agency and top foreign policy adviser to Donald Trump,
\href{https://www.nbcnews.com/news/us-news/russians-paid-mike-flynn-45k-moscow-speech-documents-show-n734506}{had
already taken \$45,000 from the K.G.B. regime} in Moscow and
\href{https://www.nytimes3xbfgragh.onion/2017/03/10/us/politics/michael-flynn-turkey.html?_r=0}{would
later take \$530,000} from the Islamist regime in Ankara as an
unregistered foreign agent. If Flynn had been prosecuted, judged and
sentenced according to his own moral arithmetic, he'd be behind bars
today.

Fortunately he isn't, because sleazy behavior isn't the same as criminal
conduct.

By now, every thoughtful observer should have learned two things from
the experience of the Trump administration.

The first is that few things in politics are as despicable as efforts to
use the power of the state to criminalize a political opponent. That's
why Trump's efforts to bully a foreign ally into digging up dirt on his
domestic opponent was so reprehensible. That's why Trump deserved his
impeachment.

The second is that civil liberties matter, never more so than when a
government seeks to prosecute the people it dislikes by hiding
\href{https://thehill.com/regulation/court-battles/494620-doj-gives-flynns-attorneys-documents-uncovered-in-review}{exculpatory
evidence}, using deceptive methods, relying on outdated laws or
threatening them with financial or familial ruin.

Yet as Bloomberg columnist Eli Lake lays out in painstaking detail in
\href{https://www.commentarymagazine.com/articles/eli-lake/michael-flynn-gets-railroaded-by-the-fbi/}{an
essay for Commentary magazine,} this is what happened to Flynn. This has
been obscured by the fact he twice pleaded guilty to a crime he likely
did not commit, as part of an investigation into a conspiracy that the
Mueller investigation could not prove. It's obscured because some of the
\href{https://www.theguardian.com/us-news/2017/mar/31/michael-flynn-new-evidence-spy-chiefs-had-concerns-about-russian-ties}{sensational
reporting about him} that once appeared solid later turned out to be
dubious.

And it's obscured because the administration's inveterate critics have a
hard time conceding that the theory in which they were politically and
emotionally invested --- that the Trump campaign had colluded with
Russia to steal the election --- was built on a shaky foundation.

What, after all, is the supposed case against Flynn? There's the claim
that he undermined American interests by urging the Russian ambassador
not to expel American diplomats in Russia. But that was a request in the
\emph{service} of American interests, not against them. There's the
argument that the call violated the 1799 Logan Act. But nobody has been
convicted under that act and calls between incoming administration
officials and foreign diplomats are hardly unprecedented.

There is his alleged lying to F.B.I. agents about his conversation with
the Russian ambassador. But the bureau's record of the interview shows
the agents thought Flynn ``did not give any indicators of deception,''
according to the Justice Department's
\href{https://int.graylady3jvrrxbe.onion/data/documenthelper/6936-michael-flynn-motion-to-dismiss/fa06f5e13a0ec71843b6/optimized/full.pdf}{motion
to dismiss the case}. There is the suggestion that Flynn exposed himself
to Russian blackmail by supposedly lying to Mike Pence about the call
with the ambassador. But as Lake astutely points out, ``Perhaps it was
Pence who lied, because he was asked a question he found difficult to
answer on national television.''

Of course, there is Flynn's guilty plea. But Flynn --- like so many
defendants in the U.S. justice system--- pleaded guilty to avoid having
a different charge, in his case the foreign-registration issue in the
Turkish matter, thrown at him and his son. What that has to do with an
investigation into Russian meddling in the American election is a
question to ponder.

Against all this, consider the behavior of the F.B.I. In December, the
Justice Department's independent inspector general noted that the bureau
\href{https://www.nytimes3xbfgragh.onion/interactive/2019/12/09/us/politics/fbi-ig-report-document.html}{repeatedly
misled} the Foreign Intelligence Surveillance court in its investigation
of Russian collusion.

As for Flynn, the F.B.I. discouraged him from having counsel present for
the interview. It did not alert him that he was a target of a secret
investigation. It withheld the transcript of his call, meaning that any
discrepancy between Flynn's memory and the transcript could be termed a
lie rather than simple misremembering. It did not ask him direct
questions about his conversations with Pence, though the rationale for
the interview was to clear up supposed discrepancies between the record
of his call and Pence's televised comments. It later withheld from his
counsel a key memo detailing the F.B.I.'s previous attempts to find
evidence that he was a Russian asset, which had come up empty-handed.

What this amounts to, Lake writes, is ``not only an injustice against
Flynn but an assault on the peaceful transition of presidential power.
The F.B.I.'s job is not to entangle the new president's
national-security adviser in a spurious investigation.''

Liberals used to have a healthy distrust of prosecutorial power, just as
they had a healthy belief in the presumption of innocence. In
t\href{https://www.nytimes3xbfgragh.onion/2020/05/01/opinion/joe-biden-tara-reade.html}{he
Tara Reade story,} they've been reminded of the political folly of
abandoning the second belief. It may not be long before they learn a
similar lesson about the folly of abandoning the first.

\emph{The Times is committed to publishing}
\href{https://www.nytimes3xbfgragh.onion/2019/01/31/opinion/letters/letters-to-editor-new-york-times-women.html}{\emph{a
diversity of letters}} \emph{to the editor. We'd like to hear what you
think about this or any of our articles. Here are some}
\href{https://help.nytimes3xbfgragh.onion/hc/en-us/articles/115014925288-How-to-submit-a-letter-to-the-editor}{\emph{tips}}\emph{.
And here's our email:}
\href{mailto:letters@NYTimes.com}{\emph{letters@NYTimes.com}}\emph{.}

\emph{Follow The New York Times Opinion section on}
\href{https://www.facebookcorewwwi.onion/nytopinion}{\emph{Facebook}}\emph{,}
\href{http://twitter.com/NYTOpinion}{\emph{Twitter (@NYTopinion)}}
\emph{and}
\href{https://www.instagram.com/nytopinion/}{\emph{Instagram}}\emph{.}

Advertisement

\protect\hyperlink{after-bottom}{Continue reading the main story}

\hypertarget{site-index}{%
\subsection{Site Index}\label{site-index}}

\hypertarget{site-information-navigation}{%
\subsection{Site Information
Navigation}\label{site-information-navigation}}

\begin{itemize}
\tightlist
\item
  \href{https://help.nytimes3xbfgragh.onion/hc/en-us/articles/115014792127-Copyright-notice}{©~2020~The
  New York Times Company}
\end{itemize}

\begin{itemize}
\tightlist
\item
  \href{https://www.nytco.com/}{NYTCo}
\item
  \href{https://help.nytimes3xbfgragh.onion/hc/en-us/articles/115015385887-Contact-Us}{Contact
  Us}
\item
  \href{https://www.nytco.com/careers/}{Work with us}
\item
  \href{https://nytmediakit.com/}{Advertise}
\item
  \href{http://www.tbrandstudio.com/}{T Brand Studio}
\item
  \href{https://www.nytimes3xbfgragh.onion/privacy/cookie-policy\#how-do-i-manage-trackers}{Your
  Ad Choices}
\item
  \href{https://www.nytimes3xbfgragh.onion/privacy}{Privacy}
\item
  \href{https://help.nytimes3xbfgragh.onion/hc/en-us/articles/115014893428-Terms-of-service}{Terms
  of Service}
\item
  \href{https://help.nytimes3xbfgragh.onion/hc/en-us/articles/115014893968-Terms-of-sale}{Terms
  of Sale}
\item
  \href{https://spiderbites.nytimes3xbfgragh.onion}{Site Map}
\item
  \href{https://help.nytimes3xbfgragh.onion/hc/en-us}{Help}
\item
  \href{https://www.nytimes3xbfgragh.onion/subscription?campaignId=37WXW}{Subscriptions}
\end{itemize}
