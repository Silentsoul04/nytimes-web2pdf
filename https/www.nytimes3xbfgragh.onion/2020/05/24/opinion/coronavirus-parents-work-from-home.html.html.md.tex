Sections

SEARCH

\protect\hyperlink{site-content}{Skip to
content}\protect\hyperlink{site-index}{Skip to site index}

\href{https://myaccount.nytimes3xbfgragh.onion/auth/login?response_type=cookie\&client_id=vi}{}

\href{https://www.nytimes3xbfgragh.onion/section/todayspaper}{Today's
Paper}

\href{/section/opinion}{Opinion}\textbar{}Camp Is Canceled. Three More
Months of Family Time. Help.

\url{https://nyti.ms/2TDI5lw}

\begin{itemize}
\item
\item
\item
\item
\item
\item
\end{itemize}

Advertisement

\protect\hyperlink{after-top}{Continue reading the main story}

\href{/section/opinion}{Opinion}

Supported by

\protect\hyperlink{after-sponsor}{Continue reading the main story}

\hypertarget{camp-is-canceled-three-more-months-of-family-time-help}{%
\section{Camp Is Canceled. Three More Months of Family Time.
Help.}\label{camp-is-canceled-three-more-months-of-family-time-help}}

We're homemakers, stay-at-home parents and paid workers. All at the same
time.

\href{https://www.nytimes3xbfgragh.onion/by/jennifer-senior}{\includegraphics{https://static01.graylady3jvrrxbe.onion/images/2018/10/26/opinion/jennifer-senior/jennifer-senior-thumbLarge.png}}

By \href{https://www.nytimes3xbfgragh.onion/by/jennifer-senior}{Jennifer
Senior}

Opinion columnist

\begin{itemize}
\item
  May 24, 2020
\item
  \begin{itemize}
  \item
  \item
  \item
  \item
  \item
  \item
  \end{itemize}
\end{itemize}

\includegraphics{https://static01.graylady3jvrrxbe.onion/images/2020/05/24/opinion/24Senior/24Senior-articleLarge.jpg?quality=75\&auto=webp\&disable=upscale}

Summer is upon us, and many summer camps across the Northeast --- and
other parts of the country ---
\href{https://www.nytimes3xbfgragh.onion/2020/05/22/nyregion/summer-camp-coronavirus.html?searchResultPosition=1}{have
been canceled}. We all know what this means: three more months of family
togetherness! 270 more meals! 540 if you have teenagers! All without the
nominal structure of online school! Who's excited?

I know. It's a formula for enough tantrums to split the atom.

Because I'm a mother, and because I once wrote
\href{https://www.nytimes3xbfgragh.onion/2014/02/02/books/review/all-joy-and-no-fun-by-jennifer-senior.html}{a
book about modern parenthood}, I've spent a lot of time these days
trying to diagnose why it is, exactly, that the nerves of so many
parents have been torn to ribbons in the age of quarantine. I'm talking
about the lucky ones, the ones who still have jobs and do them from
home. Here's my best stab:

\begin{itemize}
\item
  \textbf{1.} Quarantine parenting is marked by a dire absence of flow,
  which is more essential to our well-being at this moment than we ever
  knew, and
\item
  \textbf{2.} We're living with the household requirements of the 1960s
  but the work and parenting expectations of 2020, which is a rotten
  combination, especially for mothers, and
\item
  \textbf{3.} Both of the above are probably related.
\end{itemize}

Proceeding in order: ``Flow'' is that heavenly state of total absorption
in a project. Your sense of time vanishes; it's just you and the task at
hand, whether it's painting or sinking shots through a basketball hoop.

It turns out that flow is critical to our well-being during this strange
time of self-exile. A few weeks ago I spoke to Kate Sweeny, a professor
of psychology at the University of California, Riverside, who recently
collaborated on a survey of 5,115 people under quarantine in China. To
her surprise, the people who best tolerated their confinement were not
the most mindful or optimistic; they were the ones who'd found the most
flow. She suspected it was why Americans have spent the last two months
baking bread and doing puzzles. ``They're intuitively seeking out flow
activities,'' she said.

Flow, unfortunately, is rare in family life. The father of flow
research, Mihaly Csikszentmihalyi, told me so point-blank when I wrote
my book. When kids are small, their developing brains actually conspire
\emph{against} flow, because they're wired to sweep in as much stimuli
as possible, rather than to focus; even when they're older, they're
still churning windmills of need.

And that's during the best of times. Now, not only are we looking after
our children, an inherently non-flow activity, and not only are we
supervising their schoolwork and recreational pursuits --- two things we
used to outsource --- but we're \emph{working}.

You need a stretch of continuous, unmolested time to do good work.
Instead, your day is a torrent of interruptions, endlessly divided and
subdivided, a Zeno's paradox of infinite tasks. There's no flow at all.

Now add to this blurry slurry the other half of my theory: We're both
1960s parents and 2020s parents all at once, a nightmare mash-up in the
space-time continuum, brought to you by a wormhole from hell.

Without school lunches and cartons of takeout at dinner, most of us,
both men and women, are doing more cooking --- and therefore more
cleaning --- than we've ever done in our lives. The home has become the
renewed locus of attention, just as it was when Betty Friedan wrote
``The Feminine Mystique'' in 1963.

The trouble is, it's 2020.
\href{https://www.bls.gov/news.release/pdf/famee.pdf}{Over 70 percent}
of all mothers now participate in the work force. And as soon as women
entered the work force,
\href{https://www.pewsocialtrends.org/2013/03/14/chapter-5-americans-time-at-paid-work-housework-child-care-1965-to-2011/}{the
first thing that went to seed} was their homes. Instead, we compensate
for our domestic delinquency by actively, intensively parenting our
children, spending \href{https://nymag.com/news/features/67024/}{more
time} with them now than we did in 1965.

But in quarantine, we're doing all of it. We're homemakers. We're
stay-at-home parents. We're paid workers. All at the same bloody time.
But there \emph{isn't} time for all three, only time to feel like we're
\emph{failing} at all three, sometimes simultaneously, devoting
low-quality or insufficient attention to each role. We're all making
choices about where to cut corners.

I cut corners on cleaning. But also, if I were to be honest, on
intensive mothering. It's basically a return to the laissez-faire
parenting of the 1960s in my house. Fortnite has become my favorite
child-care provider. It's just fine, really --- my son is
self-regulating, so I can always get him to stop, and Fortnite gives him
a chance to talk to friends he dearly misses. But I still can't shake
the dull sense, unique to our era, that this is simply wrong. Even
though I know Betty Draper would never have harbored this kind of guilt.
She'd have shooed Sally out of the kitchen and enjoyed a cigarette.

Recently, a grateful employee
\href{https://twitter.com/slavetothehat/status/1259978637266366465}{tweeted
out a memo} from the Canadian federal government, which told its workers
not to hold themselves to pre-pandemic standards during this time. ``You
are not `working from home','' it said. ``You are `at your home, during
a crisis, trying to work.' ''

It was such a generous distinction. It should be extended to raising
kids. We are not really ``parenting,'' in whatever sense that usually
means to us. We are managing parenthood during a pandemic. They are not
the same. And whatever we're doing? It's good enough.

\emph{The Times is committed to publishing}
\href{https://www.nytimes3xbfgragh.onion/2019/01/31/opinion/letters/letters-to-editor-new-york-times-women.html}{\emph{a
diversity of letters}} \emph{to the editor. We'd like to hear what you
think about this or any of our articles. Here are some}
\href{https://help.nytimes3xbfgragh.onion/hc/en-us/articles/115014925288-How-to-submit-a-letter-to-the-editor}{\emph{tips}}\emph{.
And here's our email:}
\href{mailto:letters@NYTimes.com}{\emph{letters@NYTimes.com}}\emph{.}

\emph{Follow The New York Times Opinion section on}
\href{https://www.facebookcorewwwi.onion/nytopinion}{\emph{Facebook}}\emph{,}
\href{http://twitter.com/NYTOpinion}{\emph{Twitter (@NYTopinion)}}
\emph{and}
\href{https://www.instagram.com/nytopinion/}{\emph{Instagram}}\emph{.}

Advertisement

\protect\hyperlink{after-bottom}{Continue reading the main story}

\hypertarget{site-index}{%
\subsection{Site Index}\label{site-index}}

\hypertarget{site-information-navigation}{%
\subsection{Site Information
Navigation}\label{site-information-navigation}}

\begin{itemize}
\tightlist
\item
  \href{https://help.nytimes3xbfgragh.onion/hc/en-us/articles/115014792127-Copyright-notice}{©~2020~The
  New York Times Company}
\end{itemize}

\begin{itemize}
\tightlist
\item
  \href{https://www.nytco.com/}{NYTCo}
\item
  \href{https://help.nytimes3xbfgragh.onion/hc/en-us/articles/115015385887-Contact-Us}{Contact
  Us}
\item
  \href{https://www.nytco.com/careers/}{Work with us}
\item
  \href{https://nytmediakit.com/}{Advertise}
\item
  \href{http://www.tbrandstudio.com/}{T Brand Studio}
\item
  \href{https://www.nytimes3xbfgragh.onion/privacy/cookie-policy\#how-do-i-manage-trackers}{Your
  Ad Choices}
\item
  \href{https://www.nytimes3xbfgragh.onion/privacy}{Privacy}
\item
  \href{https://help.nytimes3xbfgragh.onion/hc/en-us/articles/115014893428-Terms-of-service}{Terms
  of Service}
\item
  \href{https://help.nytimes3xbfgragh.onion/hc/en-us/articles/115014893968-Terms-of-sale}{Terms
  of Sale}
\item
  \href{https://spiderbites.nytimes3xbfgragh.onion}{Site Map}
\item
  \href{https://help.nytimes3xbfgragh.onion/hc/en-us}{Help}
\item
  \href{https://www.nytimes3xbfgragh.onion/subscription?campaignId=37WXW}{Subscriptions}
\end{itemize}
