Sections

SEARCH

\protect\hyperlink{site-content}{Skip to
content}\protect\hyperlink{site-index}{Skip to site index}

\href{https://www.nytimes3xbfgragh.onion/section/world/asia}{Asia
Pacific}

\href{https://myaccount.nytimes3xbfgragh.onion/auth/login?response_type=cookie\&client_id=vi}{}

\href{https://www.nytimes3xbfgragh.onion/section/todayspaper}{Today's
Paper}

\href{/section/world/asia}{Asia Pacific}\textbar{}Why China's Move to
Rein In Hong Kong Is Just the Start

\url{https://nyti.ms/36trzcJ}

\begin{itemize}
\item
\item
\item
\item
\item
\end{itemize}

Advertisement

\protect\hyperlink{after-top}{Continue reading the main story}

Supported by

\protect\hyperlink{after-sponsor}{Continue reading the main story}

News analysis

\hypertarget{why-chinas-move-to-rein-in-hong-kong-is-just-the-start}{%
\section{Why China's Move to Rein In Hong Kong Is Just the
Start}\label{why-chinas-move-to-rein-in-hong-kong-is-just-the-start}}

Xi Jinping's China, emboldened by its handling of the coronavirus
pandemic, no longer seems constrained by the threat of international
rebuke.

\includegraphics{https://static01.graylady3jvrrxbe.onion/images/2020/05/24/world/24china01/merlin_172703271_d3ff6544-6b4c-4ad2-9cde-222d7dd89f02-articleLarge.jpg?quality=75\&auto=webp\&disable=upscale}

\href{https://www.nytimes3xbfgragh.onion/by/steven-lee-myers}{\includegraphics{https://static01.graylady3jvrrxbe.onion/images/2018/10/15/multimedia/author-steven-lee-myers/author-steven-lee-myers-thumbLarge.png}}

By \href{https://www.nytimes3xbfgragh.onion/by/steven-lee-myers}{Steven
Lee Myers}

\begin{itemize}
\item
  Published May 24, 2020Updated May 28, 2020
\item
  \begin{itemize}
  \item
  \item
  \item
  \item
  \item
  \end{itemize}
\end{itemize}

\href{https://cn.nytimes3xbfgragh.onion/china/20200525/china-hong-kong-taiwan/}{阅读简体中文版}\href{https://cn.nytimes3xbfgragh.onion/china/20200525/china-hong-kong-taiwan/zh-hant}{閱讀繁體中文版}

China's move to
\href{https://www.nytimes3xbfgragh.onion/2020/05/23/world/asia/hong-kong-china.html}{strip
away another layer of Hong Kong's autonomy} was not a rash impulse. It
was a deliberate act, months in the making. It took into account the
risks of international umbrage and reached the reasonable assumption
that there would not be a significant geopolitical price to pay.

As a provocative move, it is just the latest.

With the world
\href{https://www.nytimes3xbfgragh.onion/2020/05/22/business/china-hong-kong-national-security.html}{distracted
by the pandemic's devastating toll}, China has
\href{https://www.nytimes3xbfgragh.onion/2020/05/21/us/politics/trump-china-hong-kong.html}{taken
a series of aggressive actions} in recent weeks to flex its economic,
diplomatic and military muscle across the region.

China's Coast Guard rammed and sank a fishing boat in disputed waters
off Vietnam, and its ships
\href{https://www.nytimes3xbfgragh.onion/2020/04/21/world/asia/coronavirus-south-china-sea-warships.html}{swarmed
an offshore oil} rig operated by Malaysia. Beijing denounced the
\href{https://www.nytimes3xbfgragh.onion/2016/05/21/world/asia/taiwan-president-tsai-ing-wen.html}{second
inauguration of Taiwan's president, Tsai Ing-wen}, and pointedly dropped
the word peaceful from its annual call for unification with the island
democracy.

Chinese troops squared off again last week with India's along their
contentious border in the Himalayas.

All are longstanding tensions, but the decision to impose new national
security laws on Hong Kong, bypassing the semiautonomous territory's own
legislative process, shows what can happen with an unbridled China, no
longer restrained by the fear of international rebuke.

``There was this idea before about China being cautious and trying to
cultivate its soft power around the world,'' said Jean-Pierre Cabestan,
a professor at Hong Kong Baptist University and the author of ``China
Tomorrow: Democracy or Dictatorship?'' ``Those times are gone with Xi
Jinping.''

Mr. Xi, who in seven years in power has pursued a ``great rejuvenation''
of the Chinese state, has
\href{https://www.nytimes3xbfgragh.onion/2020/05/20/world/asia/coronavirus-china-xi-jinping.html}{emerged
from the pandemic newly emboldened}, seizing on nationalistic themes to
deflect from the government's early failures in stopping the
coronavirus's spread.

He still faces enormous economic and diplomatic challenges. New
\href{https://www.nytimes3xbfgragh.onion/2020/05/24/world/asia/hong-kong-protest-coronavirus-china.html}{protests
erupted in Hong Kong on Sunday}, and resistance to greater control by
Beijing could threaten the territory's role as a financial center.

Officials and state media outlets have
\href{https://www.nytimes3xbfgragh.onion/2020/05/23/world/asia/china-hong-kong-propaganda.html}{lashed
out} at the United States and other countries, accusing them of
supporting ``separatists'' and ``terrorists'' in an effort to weaken the
power of the Communist Party.

On the defensive over their handling of the virus, President Trump and
his aides have sought to blame China for the pandemic's toll in the
United States. The criticism, by all appearances, has done little to
moderate Mr. Xi's actions. It may even have emboldened them, as Chinese
officials point to the failures in the United States and other countries
as evidence of the Communist Party's better model of governance.

The Trump administration has, in turn, intensified its actions against
China, imposing restrictions on trade and technology, praising Ms.
Tsai's inauguration and even marking the 25th anniversary of the
disappearance of the 11th Panchen Lama, the second-highest figure in
Tibetan Buddhism.

\includegraphics{https://static01.graylady3jvrrxbe.onion/images/2020/05/24/world/24china03/merlin_162694701_12f2a23f-a5b3-4cfe-8c02-2b91fd8a7e21-articleLarge.jpg?quality=75\&auto=webp\&disable=upscale}

``The United States, in fact, is pouring oil on the fire, barrel by
barrel,'' Tian Feilong, a professor of law at Beihang University in
Beijing, said in a telephone interview. ``The central government is
therefore actually just safeguarding its own most basic national
security interests.''

China's top diplomat, Wang Yi, said on Sunday that the two countries
could still work together to promote global peace and stability, but he
denounced those in the United States who seek American hegemony.

``It's time for the United States to give up its wishful thinking of
changing China,'' Mr. Wang said, accusing American officials of having a
Cold War mentality.

Mr. Xi's move against Hong Kong has nonviolent echoes of President
Vladimir V. Putin's forceful seizure of Crimea from Ukraine in 2014,
which was a violation of international law and of Russia's
\href{https://www.nytimes3xbfgragh.onion/2015/02/21/world/europe/britain-europe-ukraine-house-of-lords-report.html}{previous
diplomatic commitments}. The annexation made Mr. Putin an international
pariah for a while, but Russia still remains firmly in control of
Crimea.

While Mr. Xi is using legislation rather than military force in a
territory already under Chinese rule, it is nonetheless a brash move by
an autocratic leader willing to risk international condemnation to
resist what he views as foreign encroachment on his country's security.

``The Communist Party doesn't care anymore about the reactions, because
it's about survival, the stability of the one-party system, avoiding the
fate of the Soviet Union,'' Mr. Cabestan said. ``Hong Kong is being
perceived more and more as a base of surveillance, as a factor in the
destabilization of the Chinese state.''

Image

Protesting new security legislation in Hong Kong on Sunday.Credit...Lam
Yik Fei for The New York Times

The challenges facing Mr. Xi come at a time when China's major rivals,
the United States above all, are in disarray, giving Mr. Xi more room to
maneuver.

Britain, which is a signatory to the 1984 treaty that promised Hong Kong
--- its former colony --- basic freedoms until 2047, issued
\href{https://www.gov.uk/government/news/joint-statement-from-the-uk-australia-and-canada-on-hong-kong}{a
statement} with Australia and Canada saying that they were ``deeply
concerned.''
\href{https://www.nytimes3xbfgragh.onion/2020/05/22/world/asia/trump-pompeo-china-hong-kong.html}{Senior
Trump administration officials also denounced Mr. Xi's gambit}, warning
that they could
\href{https://www.nytimes3xbfgragh.onion/2020/05/28/business/hong-kong-special-status-explained.html}{reconsider
the territory's special trade privileges} or impose other sanctions.
President Trump, whose few comments about Hong Kong have been
inconsistent, said little.

For those who support
\href{https://www.nytimes3xbfgragh.onion/2020/05/23/world/asia/hong-kong-china.html}{Hong
Kong's unique status as Asia's commercial and cultural crossroads},
warnings no longer suffice in the face of determined pressure from
Beijing.

Victoria Hui, a political scientist at the University of Notre Dame and
author of a book on the 2014 Hong Kong protests known as
\href{https://www.nytimes3xbfgragh.onion/2019/04/23/world/asia/hong-kong-umbrella-movement.html}{the
Umbrella Movement}, said the international community had often spoken
out against China's steady accretion of power over the territory but had
exacted no real punishment.

That has been the case for the most egregious violations of basic rights
in Hong Kong in recent years, including extrajudicial kidnappings,
excessive use of force by the police last year and the arrests of
\href{https://www.nytimes3xbfgragh.onion/2020/04/18/world/asia/hong-kong-arrests.html}{leading
democratic leaders} a month ago.

``The international pushback has been so weak,'' Ms. Hui said. ``Beijing
is daring foreign governments to continue to issue words but take no
actions.''

China's tactics under Mr. Xi today contrast those of his immediate
predecessors, who prioritized China's reforms and opening over
confrontation with its neighbors or the broader world. ``Hide our
strength, bide our time'' was Deng Xiaoping's adage a generation ago.

When Taiwan was moving to hold its first presidential elections in 1996,
China conducted intimidating missile tests in the Taiwan Strait. It was
forced to back down when President Bill Clinton ordered American
aircraft carriers to the waters in a show of military support for the
island's defense.

Mr. Xi has steadily built up China's air and naval power, making a
similar move by the United States today much riskier. Chinese forces
routinely menace the island, as its first operational aircraft carrier
did last month, forcing Taiwan's military to scramble jets and ships.
\href{https://www.youtube.com/watch?v=PUj-vrQTT84}{The seventh similar
incident} this year, it signaled China's determination to block Taiwan
from formally establishing its independence.

Image

A photo released by the state-owned Xinhua news agency showing the
commissioning ceremony of China's second aircraft carrier in
December.Credit...Li Gang/Xinhua, via Associated Press

For Beijing's leaders, China's sovereignty over Hong Kong is as
emotionally charged.

Under the Basic Law, the mini-constitution that governs the territory,
Hong Kong is obliged to adopt rules ``to prohibit any act of treason,
secession, sedition and subversion'' against the Chinese government.
When the city's legislature tried to do so in 2003, Beijing retreated in
the face of huge street protests.

``China was in a very different place globally,'' said Rana Mitter, the
director of the University of Oxford China Center. ``China's economy was
growing in 2003, but it wasn't the second biggest economy in the world
and quite the economic behemoth it is today.''

There is also a more subtle difference that the pandemic has
accentuated. Beijing spent years deflecting criticism of its system by
saying that China was not yet ready for more democratic freedoms,
effectively leaving open the possibility for greater liberalization of
the political system, as many inside and outside the country hoped.

China, Mr. Mitter said, is now a ``state which no longer apologizes for
being authoritarian.''

On Friday, Mr. Xi told delegates at
\href{https://www.nytimes3xbfgragh.onion/2020/05/21/business/economy/coronavirus-china-economy.html}{the
annual session of the legislature, the National People's Congress}, that
the country's system was the ``the broadest, most genuine, and most
effective democracy to safeguard the fundamental interests of the
people.''

Such confidence has allowed Mr. Xi to brush aside international concerns
about China's behavior at home and abroad: the absence of government
transparency and accountability, the countless arrests of those who
express dissent, the
\href{https://www.nytimes3xbfgragh.onion/interactive/2019/11/16/world/asia/china-xinjiang-documents.html}{mass
detention of more than one million Uighurs and other Muslims} in the
western province of Xinjiang.

It has also emboldened China in ways that create the possibility of
armed conflict.

On the remote border with India, Chinese forces have twice in the last
month clashed with Indian troops, prompting both sides to send in
re-enforcements. India has accused China of blocking patrols on its side
on the Line of Control, the unofficial border.

China has also stepped up its efforts to dominate the South China Sea
despite the territorial claims of countries like Vietnam, Malaysia and
the Philippines.

Image

A photo of a U.S. Navy plane's monitor showing Scarborough Shoal,
formerly administered by the Philippines but now controlled by
China.Credit...Adam Dean for The New York Times

In April, it created
\href{https://amti.csis.org/fishing-while-the-water-is-muddy-chinas-newly-announced-administrative-districts-in-the-south-china-sea/}{two
new administrative districts} to govern the islands it controls in the
Paracel and Spratly chains. China's Navy
\href{https://www.globaltimes.cn/content/1188873.shtml}{also said} that
it had succeeded in growing cabbage and other vegetables in the sand of
Woody Island, helping to feed the growing number of troops stationed
there.

``Chinese aggression is not always just rhetorical,'' Alice G. Wells, an
assistant U.S. Secretary of State, said in a telephone briefing in
Washington last week.

``So whether it's in the South China Sea or whether it's along the
border with India,'' she said, ``we continue to see provocations and
disturbing behavior by China that raises questions about how China seeks
to use its growing power.''

Claire Fu contributed research.

Advertisement

\protect\hyperlink{after-bottom}{Continue reading the main story}

\hypertarget{site-index}{%
\subsection{Site Index}\label{site-index}}

\hypertarget{site-information-navigation}{%
\subsection{Site Information
Navigation}\label{site-information-navigation}}

\begin{itemize}
\tightlist
\item
  \href{https://help.nytimes3xbfgragh.onion/hc/en-us/articles/115014792127-Copyright-notice}{©~2020~The
  New York Times Company}
\end{itemize}

\begin{itemize}
\tightlist
\item
  \href{https://www.nytco.com/}{NYTCo}
\item
  \href{https://help.nytimes3xbfgragh.onion/hc/en-us/articles/115015385887-Contact-Us}{Contact
  Us}
\item
  \href{https://www.nytco.com/careers/}{Work with us}
\item
  \href{https://nytmediakit.com/}{Advertise}
\item
  \href{http://www.tbrandstudio.com/}{T Brand Studio}
\item
  \href{https://www.nytimes3xbfgragh.onion/privacy/cookie-policy\#how-do-i-manage-trackers}{Your
  Ad Choices}
\item
  \href{https://www.nytimes3xbfgragh.onion/privacy}{Privacy}
\item
  \href{https://help.nytimes3xbfgragh.onion/hc/en-us/articles/115014893428-Terms-of-service}{Terms
  of Service}
\item
  \href{https://help.nytimes3xbfgragh.onion/hc/en-us/articles/115014893968-Terms-of-sale}{Terms
  of Sale}
\item
  \href{https://spiderbites.nytimes3xbfgragh.onion}{Site Map}
\item
  \href{https://help.nytimes3xbfgragh.onion/hc/en-us}{Help}
\item
  \href{https://www.nytimes3xbfgragh.onion/subscription?campaignId=37WXW}{Subscriptions}
\end{itemize}
