Sections

SEARCH

\protect\hyperlink{site-content}{Skip to
content}\protect\hyperlink{site-index}{Skip to site index}

\href{https://myaccount.nytimes3xbfgragh.onion/auth/login?response_type=cookie\&client_id=vi}{}

\href{https://www.nytimes3xbfgragh.onion/section/todayspaper}{Today's
Paper}

\href{/section/opinion}{Opinion}\textbar{}New York City Doesn't Have to
Suffer This Summer

\url{https://nyti.ms/3c2d6Gb}

\begin{itemize}
\item
\item
\item
\item
\item
\end{itemize}

Advertisement

\protect\hyperlink{after-top}{Continue reading the main story}

\href{/section/opinion}{Opinion}

Supported by

\protect\hyperlink{after-sponsor}{Continue reading the main story}

\hypertarget{new-york-city-doesnt-have-to-suffer-this-summer}{%
\section{New York City Doesn't Have to Suffer This
Summer}\label{new-york-city-doesnt-have-to-suffer-this-summer}}

Officials shouldn't let the coronavirus end a long history of helping
people stay cool.

By Kara Murphy Schlichting and Adrian Benepe

\emph{Ms. Schlichting is an assistant professor of history at Queens
College. Mr. Benepe, senior vice president at the Trust for Public Land,
was the New York City Parks Commissioner from 2002 to 2012.}

\begin{itemize}
\item
  May 25, 2020
\item
  \begin{itemize}
  \item
  \item
  \item
  \item
  \item
  \end{itemize}
\end{itemize}

\includegraphics{https://static01.graylady3jvrrxbe.onion/images/2020/05/25/opinion/25schlichting1/merlin_158221335_9fafad3d-fa46-4c24-bbd3-b870c005a478-articleLarge.jpg?quality=75\&auto=webp\&disable=upscale}

New Yorkers are entering the summer season with no access to the
traditional cooling infrastructure --- the cold water found in
playground sprinklers, public pools, and beaches --- which Mayor Bill de
Blasio has declared off-limits because of concerns about the coronavirus
and the budget. Millions of residents will find themselves on an
unpleasant --- and potentially deadly --- trip back in time to the turn
of the 20th century, when summers were always ``too darn hot.''

New York bakes under the summer sun; the average high temperature in
July and August is
\href{https://www.weather.gov/okx/CentralParkHistorical}{above 80
degrees}. By the late 19th century, Lower Manhattan created a
microclimate effect wherein its densest neighborhoods maintained its
highest temperatures, what we call today the urban heat island. Before
the widespread adoption of air conditioning in the mid-20th century, New
Yorkers could do little to regulate heat and humidity, and most could
not afford to leave.

Water, whether from the city's hydrants, at beaches, the East River (for
the hardy), public baths, or, by the 1930s, pools, was the only way to
beat the heat. The city's response to heat tended to be reactive, rather
than systematic or proactive --- hydrants opened during a period of
extreme weather. New Yorkers faced the challenges of summer heat daily,
at the private level, and suffered.

The painter George Bellows immortalized East River swimmers in his 1907
painting
``\href{https://www.nga.gov/collection/art-object-page.134485.html}{Forty-Two
Kids,}'' capturing boys cavorting on a dilapidated wharf. Young swimmers
--- mostly boys from sweltering slums --- braving the fierce tidal
currents of New York Harbor were symbols of summer in the city in the
early 20th century.

But among those symbols were the environmental inequalities of summer
heat and the challenge of safely cooling off --- and swimmers died by
the hundreds. By the early 1930s, official records chronicled more than
500 drowning deaths every year in city waters.

As one New York Times Times journalist
\href{https://timesmachine.nytimes3xbfgragh.onion/timesmachine/1925/06/07/98830729.html?pageNumber=1}{observed}
during a June 1925 heat wave, ``the only relief'' in the city ``was in
water, wherever it could be found.'' In 1933, two policemen tried to
quell illegal hydrant use on the Upper East Side but
\href{https://timesmachine.nytimes3xbfgragh.onion/timesmachine/1933/06/10/119445643.html?pageNumber=3}{were
rebuffed --- and drenched} --- by children determined to play in the
water. In the infamous heat wave of 1896, the city commissioner of
public works instructed employees to open hydrants to lower the
temperature of the pavement. In tenement districts, where interior
temperatures of stuffy, cramped apartments reached an unbearable 120
degrees, desperate mothers held overheated children in hydrant streams.

People bathing in water meant for street cleaning symbolized the
discomfort, health dangers, and inequalities of summer in the city.

The city \href{https://www.nycgovparks.org/about/history/pools}{opened}
its first free seasonal bath on the East River in 1870, but industrial
waste and sewage
\href{https://babel.hathitrust.org/cgi/pt?id=coo1.ark:/13960/t1jh43s08\&view=1up\&seq=8}{so
polluted} New York Harbor that by 1910 health officials suggested a ban
on harbor swimming. By 1929, five years before Robert Moses began his
pool-building spree as parks commissioner, there were
but\href{https://www.nycgovparks.org/about/history/pools}{five public
pools}, and only one outside Manhattan, in the city. Lacking options,
New Yorkers sought relief at regional beaches. In the 1920s, summer heat
forced the government to act. Gov. Al Smith announced the creation of
the Long Island State Park Commission beach plan during a heat wave in
July 1925, after crowds from the city overwhelmed Nassau County beaches
and roads. The disadvantages of New York's muggy summer climate had
brought into stark relief the importance of large, accessible beaches.

Robert Moses, who had earlier overseen the creation of the massive Jones
Beach complex for Governor Smith, oversaw the creation of the city's ---
and the nation's --- largest public cooling infrastructure during the
Great Depression. Using Works Progress Administration funds --- the
equivalent of stimulus funding --- he built a network of palatial public
pools capable of holding thousands of bathers at a time. He also built
\href{https://www.nytimes3xbfgragh.onion/1981/07/30/obituaries/robert-moses-master-builder-is-dead-at-92.html}{658
playgrounds}, most with sprinklers and many with large wading pools;
expanded and improved the city's beach system; and added the first
official municipal lifeguard corps. Within five years, drownings had
been
\href{https://nypost.com/2020/04/17/1940-city-parks-dept-report-links-decrease-in-drownings-with-opening-pools/}{cut
by more than 30 percent}, and today there are generally fewer than 20
every year.

\includegraphics{https://static01.graylady3jvrrxbe.onion/images/2020/05/25/opinion/25schlichting2/merlin_140858724_f311498a-eb4c-41fc-84d5-ec9a40b0c815-articleLarge.jpg?quality=75\&auto=webp\&disable=upscale}

Cooling infrastructure remains essential. In early 2019, NASA scientists
announced that the last five years have been the five hottest recorded
globally. Excessive heat is a leading cause of weather-related deaths in
the United States. Climate change is making heat waves more frequent and
longer in duration. The urban heat island magnifies these environmental
challenges.

With access to the safe, regulated and guarded cooling infrastructure
completely shut off, it's likely that New Yorkers, especially those
living in poverty without access to summer homes or trips to faraway
beaches, will turn again to the desperate measures of the 1920s ---
opening hydrants and swimming at unguarded beaches and in dangerous
tidal rivers.

It's also unfortunately likely that some will die from heat stroke or
drowning, and that many open fire hydrants will reduce water pressure,
threatening firefighters' ability to put out fires.

So how to avoid the potential of a long, hot and deadly summer? It won't
be easy, but public health
\href{https://thehill.com/policy/healthcare/496483-evidence-mounts-that-outside-is-safer-when-it-comes-to-covid-19}{experts
have advised} that it is possible to open beaches and pools if
precautions are taken. The city should set up a multi-agency task force
to make plans for modified openings of pools and beaches, redeploying
hundreds of parks employees from shuttered indoor recreation centers to
manage crowds and social distancing. By definition, public pools are
very large tanks of heavily chlorinated water
(\href{https://www.cdc.gov/coronavirus/2019-ncov/community/parks-rec/park-administrators.html}{which
the C.D.C. says kills the virus}), with existing secured entrances, and
the ability to easily monitor and reduce the number of users. Public
changing rooms can be closed, and outdoor showers installed to minimize
the risk of virus transmission outside of the pools.

The city can use the well-established practice of ticketed but free
major concerts in Central Park, where fans apply online via a lottery
system for free tickets, to allow people to reserve spots on carefully
managed beaches. Crowds at Orchard Beach in the Bronx can be reduced by
limiting capacity in the large parking lot.

The city can use newly created maps to prioritize neighborhoods
\href{https://www.arcgis.com/apps/webappviewer/index.html?id=4d082c62efb44e56b105366fb92335b3\&extent=-8287910.233\%2C4941135.2385\%2C-8185178.8669\%2C4998463.0097\%2C102100}{that
lack park access} and need open streets created for recreation and bike
commuting. Parks that lack trees should be equipped with canopies to
provide shade. And if playgrounds and their sprinklers cannot be safely
opened, the city can revive another
\href{http://nycma.lunaimaging.com/luna/servlet/detail/RECORDSPHOTOUNITARC~9~9~29533~104331:mrc_160?qvq=q:croton\%20surf\&mi=0\&trs=2}{artifact
from 1920,} very large showers mounted on light poles that can cool
scores of overheated New Yorkers at a time, perhaps on the mayor's newly
instituted open streets.

Kara Murphy Schlichting is an assistant professor of history at Queens
College and the author of ``New York Recentered: Building the Metropolis
from the Shore.'' Adrian Benepe, senior vice president at the Trust for
Public Land, was the New York City Parks Commissioner from 2002 to 2012.

\emph{The Times is committed to publishing}
\href{https://www.nytimes3xbfgragh.onion/2019/01/31/opinion/letters/letters-to-editor-new-york-times-women.html}{\emph{a
diversity of letters}} \emph{to the editor. We'd like to hear what you
think about this or any of our articles. Here are some}
\href{https://help.nytimes3xbfgragh.onion/hc/en-us/articles/115014925288-How-to-submit-a-letter-to-the-editor}{\emph{tips}}\emph{.
And here's our email:}
\href{mailto:letters@NYTimes.com}{\emph{letters@NYTimes.com}}\emph{.}

\emph{Follow The New York Times Opinion section on}
\href{https://www.facebookcorewwwi.onion/nytopinion}{\emph{Facebook}}\emph{,}
\href{http://twitter.com/NYTOpinion}{\emph{Twitter (@NYTopinion)}}
\emph{and}
\href{https://www.instagram.com/nytopinion/}{\emph{Instagram}}\emph{.}

Advertisement

\protect\hyperlink{after-bottom}{Continue reading the main story}

\hypertarget{site-index}{%
\subsection{Site Index}\label{site-index}}

\hypertarget{site-information-navigation}{%
\subsection{Site Information
Navigation}\label{site-information-navigation}}

\begin{itemize}
\tightlist
\item
  \href{https://help.nytimes3xbfgragh.onion/hc/en-us/articles/115014792127-Copyright-notice}{©~2020~The
  New York Times Company}
\end{itemize}

\begin{itemize}
\tightlist
\item
  \href{https://www.nytco.com/}{NYTCo}
\item
  \href{https://help.nytimes3xbfgragh.onion/hc/en-us/articles/115015385887-Contact-Us}{Contact
  Us}
\item
  \href{https://www.nytco.com/careers/}{Work with us}
\item
  \href{https://nytmediakit.com/}{Advertise}
\item
  \href{http://www.tbrandstudio.com/}{T Brand Studio}
\item
  \href{https://www.nytimes3xbfgragh.onion/privacy/cookie-policy\#how-do-i-manage-trackers}{Your
  Ad Choices}
\item
  \href{https://www.nytimes3xbfgragh.onion/privacy}{Privacy}
\item
  \href{https://help.nytimes3xbfgragh.onion/hc/en-us/articles/115014893428-Terms-of-service}{Terms
  of Service}
\item
  \href{https://help.nytimes3xbfgragh.onion/hc/en-us/articles/115014893968-Terms-of-sale}{Terms
  of Sale}
\item
  \href{https://spiderbites.nytimes3xbfgragh.onion}{Site Map}
\item
  \href{https://help.nytimes3xbfgragh.onion/hc/en-us}{Help}
\item
  \href{https://www.nytimes3xbfgragh.onion/subscription?campaignId=37WXW}{Subscriptions}
\end{itemize}
