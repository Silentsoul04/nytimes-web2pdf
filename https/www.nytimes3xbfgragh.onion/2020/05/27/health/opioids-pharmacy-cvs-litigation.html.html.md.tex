Sections

SEARCH

\protect\hyperlink{site-content}{Skip to
content}\protect\hyperlink{site-index}{Skip to site index}

\href{https://www.nytimes3xbfgragh.onion/section/health}{Health}

\href{https://myaccount.nytimes3xbfgragh.onion/auth/login?response_type=cookie\&client_id=vi}{}

\href{https://www.nytimes3xbfgragh.onion/section/todayspaper}{Today's
Paper}

\href{/section/health}{Health}\textbar{}Big Pharmacy Chains Also Fed the
Opioid Epidemic, Court Filing Says

\url{https://nyti.ms/2ZGhurJ}

\begin{itemize}
\item
\item
\item
\item
\item
\item
\end{itemize}

Advertisement

\protect\hyperlink{after-top}{Continue reading the main story}

Supported by

\protect\hyperlink{after-sponsor}{Continue reading the main story}

\hypertarget{big-pharmacy-chains-also-fed-the-opioid-epidemic-court-filing-says}{%
\section{Big Pharmacy Chains Also Fed the Opioid Epidemic, Court Filing
Says}\label{big-pharmacy-chains-also-fed-the-opioid-epidemic-court-filing-says}}

New details emerge in a lawsuit asserting that chains including CVS,
Rite Aid and Walgreens sold millions of pills in small towns but rarely
flagged suspicious orders to authorities.

\includegraphics{https://static01.graylady3jvrrxbe.onion/images/2020/05/27/science/27OPIOIDS/27OPIOIDS-articleLarge.jpg?quality=75\&auto=webp\&disable=upscale}

By \href{https://www.nytimes3xbfgragh.onion/by/jan-hoffman}{Jan Hoffman}

\begin{itemize}
\item
  May 27, 2020
\item
  \begin{itemize}
  \item
  \item
  \item
  \item
  \item
  \item
  \end{itemize}
\end{itemize}

Through years of lawsuits and rising public anger over the opioid
epidemic, the big American pharmacy retailers have largely eluded
scrutiny. But
\href{https://www.lakecountyohio.gov/Portals/4/3306-2\%20Ex_\%20B\%20to\%20CT3\%20Plfs\%27\%20Notice\%20of\%20Redacted\%20Exhibits\%20to\%20Mtn\%20for\%20Leave\%20\%28Lake\%20Proposed\%20Supp_\%29.pdf}{a
new court filing}Wednesday morning asserts that pharmacies including
CVS, Rite Aid, Walgreens and Giant Eagle as well as those operated by
Walmart were as complicit in perpetuating the crisis as the
\href{https://www.nytimes3xbfgragh.onion/2019/10/21/health/opioid-settlement.html}{manufacturers
and distributors} of the addictive drugs.

The retailers sold
\href{https://www.nytimes3xbfgragh.onion/2019/07/19/health/opioids-trial-addiction-drugstores.html}{millions
of pills in tiny communities}, offered bonuses for high-volume
pharmacists and even worked directly with drug manufacturers to promote
opioids as safe and effective, according to the complaint filed in
federal court in Cleveland by two Ohio counties.

Specifically, the complaint lays out evidence that:

\begin{itemize}
\item
  CVS worked with
  \href{https://www.nytimes3xbfgragh.onion/2020/04/13/upshot/opioids-oxycontin-purdue-pharma.html}{Purdue
  Pharma, the maker of OxyContin}, to offer promotional seminars on pain
  management to its pharmacists so they could reassure patients and
  doctors about the safety of the drug.
\item
  In partnership with Endo Pharmaceuticals, CVS sent letters to patients
  encouraging them to maintain prescriptions of Opana, a potent opioid
  so prone to abuse that in 2017 the Food and Drug Administration
  ordered its extended-release formulation
  \href{https://www.nytimes3xbfgragh.onion/2017/06/08/health/fda-opioid-opana-er.html}{removed
  from the market.}
\item
  From 2006 through 2014, the Rite Aid in
  \href{https://www.cleveland.com/metro/2017/02/fentanyl_overdose_deaths_soar.html}{Painesville,
  Ohio}, a town with a population of 19,524, sold over 4.2 million doses
  of oxycodone and hydrocodone. The national retailer offered bonuses to
  stores with the highest productivity.
\item
  Walgreens' contract with
  \href{https://www.nytimes3xbfgragh.onion/2019/04/22/health/opioids-lawsuits-distributors.html}{the
  drug distributor AmerisourceBergen} specified that Walgreens be
  allowed to police its own orders, without oversight from the
  distributor. Similar conditions were struck by CVS with its
  distributor, Cardinal Health.
\end{itemize}

Most of the companies did not respond to a request for comment. CVS
emailed a statement that said, ``Opioids are made and marketed by drug
manufacturers, not pharmacists. Pharmacists dispense opioid
prescriptions written by a licensed physician for a legitimate medical
need.''

The other companies have made similar arguments in the past.

Walgreens referred to an earlier statement responding to the litigation,
noting ``Prescriptions are written by doctors based on their medical
training, experience and clinical judgment, and when a patient presents
a prescription that gives no reason to question its legitimacy, the
pharmacist is obligated to fill the prescription exactly as written.''

Federal law requires manufacturers, drug retailers and suppliers to
report suspiciously high orders to the federal Drug Enforcement
Administration. But despite being repeatedly fined by the D.E.A. for
failing to do so, the chains continued to sell outsize quantities of
opioids, the complaint contends, only rarely sounding alarms, a charge
also made against the drug distributors in numerous other lawsuits.

Walmart devised a workaround to that reporting requirement, the
complaint says. In mid-2012, it fixed a hard limit on opioid quantities
it would distribute to its stores, foreclosing the need for its
pharmacists to report excessive orders. Yet Walmart simply allowed its
stores to make up the difference by buying the remainder of their large
opioid orders from other distributors.

Until now, the focus of thousands of lawsuits across the country related
to the opioid health crisis has largely been on drug manufacturers and
distributors. A handful of those cases have settled. Representative
cases, called bellwethers, selected by
\href{https://www.nytimes3xbfgragh.onion/2018/03/05/health/opioid-crisis-judge-lawsuits.html}{Judge
Dan A. Polster in Cleveland} from thousands of similar federal lawsuits
to test both sides' arguments, are moving through early stages in
Chicago and West Virginia.

Cases brought by New York State and two New York counties are awaiting a
joint trial date; originally
\href{https://ag.ny.gov/press-release/2020/out-abundance-caution-opioid-trial-delayed-due-coronavirus}{set
to begin March 20}, their trial was postponed because of the pandemic
lockdown.

But relatively few cases against the retail pharmacy chains have
advanced. Like most of the lawsuits in the sprawling national
litigation, those cases are on hold, pending the outcome of the
bellwethers. Judge Polster recently gave bellwether status to retail
pharmacy cases brought by San Francisco and the Cherokee Nation. Those
lawsuits will now proceed in the plaintiffs' local federal courts.

The first case to advance against retailers, brought by Cuyahoga and
Summit Counties in Ohio, is scheduled for November 2020. But those
counties are only suing the chains in their capacity as distributors of
opioids to their own drugstores.

In contrast, the complaint filed on Wednesday was a major salvo in a
more far-reaching trial, scheduled for next May. Lake and Trumbull
Counties in Ohio are suing the chains on two fronts: as distributors to
their own pharmacies, and as dispensers, whom the counties say
intentionally fed customers' appetite for opioids.

The 209-page complaint was filed under seal earlier this month. Barring
objections by the defendants, Judge Polster allowed the complaint to be
filed publicly, which the lawyers did on Wednesday. As the lawyers note,
its claims are based on documents and interviews already completed for
the federal opioid litigation. But lawyers anticipate obtaining
considerably more information through the discovery process.

According to federal data presented in the complaint, from 2006 through
2014, the retail pharmacy chains ran 31 pharmacies in Lake County, which
has a population of 220,000, and sold nearly 64 million doses of
oxycodone and hydrocodone --- or 290 pills for every man, woman and
child.

During that same period, the numbers in Trumbull County were even more
extreme. A combined 28 pharmacies sold nearly 68 million doses to a
population of 209,837, or more than 322 pills for every man, woman and
child.

A major thrust of the complaint is that the chains dragged their feet in
setting up monitoring protocols at regional distribution centers,
enforced those programs anemically at best, kept raising thresholds for
quantities of pills that might otherwise trigger a flagged report,
rewarded pharmacists for churning volume rapidly and, in some instances,
pointedly ordered them never to refuse a doctor's prescription.

The complaint also says that supervisors ignored store pharmacists who
warned about pill mill doctors, including
\href{https://www.cleveland.com/court-justice/2015/02/akron_doctor_who_illegally_pre.html}{those
who were ultimately convicted.}

CVS, for example, did not have a companywide policy for reporting
suspicious orders until 2010, posting it the day after a D.E.A. audit
raised concerns, the complaint says. But CVS didn't report any orders as
suspicious until February 2012, when the epidemic was already in sharp
ascent. Through November 2013, CVS reported only seven such orders from
across the entire country. None were from Ohio.

Three Walmart pharmacies in Lake County sold 6.4 million opioid pills
from 2006 through 2014. But Walmart, the complaint says, did not file
any suspicious order reports from those stores between 2007 and 2014,
the period for which the plaintiffs have such data.

The complaint also details the chains' full-throated participation in
\href{https://www.washingtonpost.com/graphics/2017/investigations/dea-drug-industry-congress/}{trade
groups that lobbied Congress to pass the 2016 Marino bill}, which
curtailed the D.E.A.'s ability to immediately suspend the registration
of a manufacturer, distributor or pharmacy. CVS, along with Cardinal
Health, a distributor, and Teva, a manufacturer of generic opioids,
worked through a lobbying group called the Alliance to Prevent the Abuse
of Medicines, which often cloaked individual corporate identity.

CVS repeatedly joined with manufacturers to train its pharmacists about
their drugs. In the complaint, the lawyers write, ``CVS was so eager to
ally itself with Purdue that it solicited Purdue for its participation
in co-hosting Continuing Education programs for health care providers
and pharmacists regarding training on diversion of prescription
opioids.''

In an undated letter outlining such a venture, CVS directors wrote that
to address the challenges of a community pharmacy practice, programs
should include topics ``often associated with Purdue's products.''

They offered as an example:

``How to communicate effectively with patients and physicians about
appropriate pain management therapy, and how to resolve potential
conflict with a drug `seeker.'''

Advertisement

\protect\hyperlink{after-bottom}{Continue reading the main story}

\hypertarget{site-index}{%
\subsection{Site Index}\label{site-index}}

\hypertarget{site-information-navigation}{%
\subsection{Site Information
Navigation}\label{site-information-navigation}}

\begin{itemize}
\tightlist
\item
  \href{https://help.nytimes3xbfgragh.onion/hc/en-us/articles/115014792127-Copyright-notice}{©~2020~The
  New York Times Company}
\end{itemize}

\begin{itemize}
\tightlist
\item
  \href{https://www.nytco.com/}{NYTCo}
\item
  \href{https://help.nytimes3xbfgragh.onion/hc/en-us/articles/115015385887-Contact-Us}{Contact
  Us}
\item
  \href{https://www.nytco.com/careers/}{Work with us}
\item
  \href{https://nytmediakit.com/}{Advertise}
\item
  \href{http://www.tbrandstudio.com/}{T Brand Studio}
\item
  \href{https://www.nytimes3xbfgragh.onion/privacy/cookie-policy\#how-do-i-manage-trackers}{Your
  Ad Choices}
\item
  \href{https://www.nytimes3xbfgragh.onion/privacy}{Privacy}
\item
  \href{https://help.nytimes3xbfgragh.onion/hc/en-us/articles/115014893428-Terms-of-service}{Terms
  of Service}
\item
  \href{https://help.nytimes3xbfgragh.onion/hc/en-us/articles/115014893968-Terms-of-sale}{Terms
  of Sale}
\item
  \href{https://spiderbites.nytimes3xbfgragh.onion}{Site Map}
\item
  \href{https://help.nytimes3xbfgragh.onion/hc/en-us}{Help}
\item
  \href{https://www.nytimes3xbfgragh.onion/subscription?campaignId=37WXW}{Subscriptions}
\end{itemize}
