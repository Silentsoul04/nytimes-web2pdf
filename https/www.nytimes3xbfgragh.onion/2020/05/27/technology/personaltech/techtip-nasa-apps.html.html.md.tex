Sections

SEARCH

\protect\hyperlink{site-content}{Skip to
content}\protect\hyperlink{site-index}{Skip to site index}

\href{https://www.nytimes3xbfgragh.onion/section/technology/personaltech}{Personal
Tech}

\href{https://myaccount.nytimes3xbfgragh.onion/auth/login?response_type=cookie\&client_id=vi}{}

\href{https://www.nytimes3xbfgragh.onion/section/todayspaper}{Today's
Paper}

\href{/section/technology/personaltech}{Personal Tech}\textbar{}Space
Out and Explore the Universe Without Leaving Home

\href{https://nyti.ms/3eoItvX}{https://nyti.ms/3eoItvX}

\begin{itemize}
\item
\item
\item
\item
\item
\end{itemize}

Advertisement

\protect\hyperlink{after-top}{Continue reading the main story}

Supported by

\protect\hyperlink{after-sponsor}{Continue reading the main story}

Tech Tip

\hypertarget{space-out-and-explore-the-universe-without-leaving-home}{%
\section{Space Out and Explore the Universe Without Leaving
Home}\label{space-out-and-explore-the-universe-without-leaving-home}}

If NASA's new 2020 missions have inspired an interest in science and
celestial objects, these apps and sites can open a whole new batch of
worlds.

\includegraphics{https://static01.graylady3jvrrxbe.onion/images/2020/05/27/technology/personaltech/27TECHTIP_TOP/27TECHTIP_TOP-articleLarge.jpg?quality=75\&auto=webp\&disable=upscale}

\href{https://www.nytimes3xbfgragh.onion/by/j-d-biersdorfer}{\includegraphics{https://static01.graylady3jvrrxbe.onion/images/2018/06/14/multimedia/author-j-d-biersdorfer/author-j-d-biersdorfer-thumbLarge.png}}

By \href{https://www.nytimes3xbfgragh.onion/by/j-d-biersdorfer}{J. D.
Biersdorfer}

\begin{itemize}
\item
  Published May 27, 2020Updated May 28, 2020
\item
  \begin{itemize}
  \item
  \item
  \item
  \item
  \item
  \end{itemize}
\end{itemize}

After almost nine years of
\href{https://www.nytimes3xbfgragh.onion/2020/04/17/science/spacex-nasa-crew-dragon.html}{astronauts
hitching rides} off the planet, Americans are finally traveling on their
own rockets again with
\href{https://www.nytimes3xbfgragh.onion/2020/05/22/science/spacex-nasa-launch.html}{this
week's NASA-SpaceX launch} of crew to the
\href{https://www.nasa.gov/mission_pages/station/main/index.html}{International
Space Station}. And with
\href{https://www.nasa.gov/perseverance}{Perseverance}, another NASA
rover vehicle, headed to Mars this summer, 2020 is shaping up to be a
busy year on the final frontier.

If you find yourself with a new (or renewed) interest in America's space
program or general astronomy, you're in luck --- the web is full of free
and inexpensive learning resources to feed your mind. You can even find
a few science projects to keep the family busy. Here's a guide.

\hypertarget{tour-nasagov}{%
\subsection{Tour NASA.gov}\label{tour-nasagov}}

The National Aeronautics and Space Administration
\href{https://www.nasa.gov}{main website} is a great place to discover
the \href{https://www.nasa.gov/topics/history/index.html}{past},
\href{https://www.nasa.gov/news/releases/latest/index.html}{present} and
\href{https://www.nasa.gov/launchschedule/}{future} of the country's
role in cosmic exploration. Along with history articles and multimedia,
the live
\href{https://www.nasa.gov/multimedia/nasatv/index.html\#public}{NASA
TV} video stream is on the site, as are links to the agency's expansive
presence across \href{https://www.nasa.gov/socialmedia}{social-media
channels} (including \href{https://giphy.com/nasa}{Giphy},
\href{https://soundcloud.com/nasa}{SoundCloud} and
\href{https://www.twitch.tv/nasa/}{Twitch}).

\includegraphics{https://static01.graylady3jvrrxbe.onion/images/2020/05/27/technology/personaltech/27TECHTIP_01A/27TECHTIP_01A-articleLarge.jpg?quality=75\&auto=webp\&disable=upscale}

Visit the downloads section of the site for links to
\href{https://www.nasa.gov/connect/apps.html}{NASA's mobile apps} or
free \href{https://www.nasa.gov/connect/ebooks/index.html}{e-books}
including the
\href{https://www.nasa.gov/connect/ebooks/nasa-systems-engineering-handbook}{``NASA
Systems Engineering Handbook.''} There's also
\href{https://www.nasa.gov/connect/sounds/index.html}{an audio library}
with clips you can use as ringtones or computer alerts, like the
foreboding
\href{https://www.nasa.gov/mp3/574928main_houston_problem.mp3}{``Houston,
we've had a problem''} warning from the nationally nerve-racking
\href{https://www.nasa.gov/feature/50-years-ago-houston-we-ve-had-a-problem}{Apollo
13} mission in 1970.

NASA maintains many
\href{https://www.nasa.gov/audience/foreducators/Alpha_index.html}{other
websites}, as does its \href{https://www.jpl.nasa.gov/}{Jet Propulsion
Laboratory} at the \href{https://www.caltech.edu/}{California Institute
of Technology}, which hosts sites for the
\href{https://mars.nasa.gov/}{Mars projects}, for
\href{https://solarsystem.nasa.gov/}{solar-system exploration} and even
the old 1970s \href{https://voyager.jpl.nasa.gov/}{Voyager interstellar
probes}.

Image

NASA produces several additional websites related to its work, including
this information hub for Mars-exploration missions like this summer's
Perseverance rover launch.

\hypertarget{mark-your-calendar}{%
\subsection{Mark Your Calendar}\label{mark-your-calendar}}

Want to know when to look up for a
\href{https://www.amsmeteors.org/meteor-showers/meteor-shower-calendar/}{meteor
shower} or the next photo-worthy
\href{https://solarsystem.nasa.gov/news/1220/the-next-full-moon-is-a-supermoon-flower-moon/}{supermoon}?
Bookmark or
\href{https://www.nytimes3xbfgragh.onion/2014/12/26/technology/personaltech/adding-the-sun-moon-and-stars-to-google-calendar.html}{subscribe}
to an astronomy calendar. For example, TimeandDate.com hosts a
\href{https://www.timeanddate.com/astronomy/sights-to-see.html}{Calendar
of Cosmic Events} that notes eclipses, equinoxes, solstices and more.

Image

Subscribing to one of the many astronomy calendars available online puts
events like moon phases and meteor showers on your schedule.

To get major astronomical events to appear automatically on your online
calendar, search for a subscription option, like the
\href{http://cantonbecker.com/astronomy-calendar/}{Moon Phase \&
Astronomy Calendar}. And the Science desk of The New York Times
maintains
\href{https://www.nytimes3xbfgragh.onion/interactive/2020/science/2020-astronomy-space-calendar.html}{its
own astronomy and space calendar} that works with Google Calendar and
the iOS Calendar app.

\hypertarget{see-the-stars}{%
\subsection{See the Stars}\label{see-the-stars}}

When you're ready to explore the cosmos on your own, use your
smartphone; just search for ``astronomy'' in your app store. Many
popular apps combine augmented reality, 3-D models and your location
information to provide a customized guided tour of the sky.

The \$3 \href{https://starwalk.space/en}{Star Walk 2} for Android and
iOS is beautifully designed, and the app's news section alerts you to
upcoming astronomical events. The \$2 Star Rover for
\href{https://play.google.com/store/apps/details?id=net.eefan.star\&hl=en_US}{Android}
and
\href{https://apps.apple.com/us/app/star-rover-hd-night-sky-map/id437217048}{iOS}
also provides a pocket planetarium experience, as does the free SkyView
Lite for
\href{https://play.google.com/store/apps/details?id=com.t11.skyviewfree\&hl=en_US}{Android}
and \href{https://apps.apple.com/mw/app/skyview-lite/id413936865}{iOS}.

Image

Like many space-exploration apps, SkySafari for Android and iOS uses
your device's location information to show you the skies above, along
with other astronomical information.

\href{https://skysafariastronomy.com/}{SkySafari} has apps for Android
and iOS on several enthusiast levels (paid versions range from about \$3
to \$40), and the more expensive editions include mobile control for
compatible home telescopes. And if you don't have your own telescope,
peek at the images by the \href{https://hubblesite.org/}{Hubble Space
Telescope} on its official website.

\hypertarget{virtually-visit-a-museum}{%
\subsection{Virtually Visit a Museum}\label{virtually-visit-a-museum}}

Although many institutions have been affected by the Covid-19 pandemic,
their online exhibits are up. For example, the Smithsonian's
\href{https://airandspace.si.edu}{National Air and Space Museum} website
has \href{https://airandspace.si.edu/exhibitions/online}{several
displays to browse} --- as well as
\href{https://artsandculture.google.com/partner/smithsonian-national-air-and-space-museum}{other
content} in the free Google
\href{https://artsandculture.google.com/}{Arts \& Culture} app for
\href{https://play.google.com/store/apps/details?id=com.google.android.apps.cultural}{Android}
and
\href{https://apps.apple.com/us/app/google-arts-culture/id1050970557}{iOS}.

Image

The Google Arts and Culture app provides a virtual peek into the Steven
F. Udvar-Hazy Center of the Smithsonian's National Air and Space Museum,
as well as scores of other historical and cultural institutions around
the world.

Google's app also lets you
\href{https://artsandculture.google.com/partner/intrepid-sea-air-and-space-museum}{take
a tour} of \href{https://www.intrepidmuseum.org/}{Intrepid Sea, Air and
Space Museum}, and the complex's own site
\href{https://www.intrepidmuseum.org/digital-resources}{hosts virtual
learning events}. And if your apartment is feeling cramped, the site for
the \href{https://www.amnh.org/research/hayden-planetarium}{Hayden
Planetarium} at the American Museum of Natural History has a free
educational version of its
\href{https://www.amnh.org/research/hayden-planetarium/digital-universe}{Digital
Universe} 3-D atlas as a hefty download for desktop systems.

\hypertarget{find-family-learning-activities}{%
\subsection{Find Family Learning
Activities}\label{find-family-learning-activities}}

Colorful planets and rolling rovers have a certain appeal for children
and can help foster an early interest in science, technology engineering
and mathematics. Chicago's Museum of Science and Industry offers a
\href{https://www.msichicago.org/science-at-home/hands-on-science/}{Science
at Home} collection of hands-on projects for children, like designing
\href{https://www.msichicago.org/science-at-home/hands-on-science/parachute-design/}{a
parachute} or building a
\href{https://www.msichicago.org/science-at-home/hands-on-science/stomp-rocket/}{``stomp
rocket.''}

The California Science Center has several projects as part of its
\href{https://californiasciencecenter.org/stuck-at-home-science}{Stuck
at Home Science} activity series that teaches material using household
supplies.

Image

NASA's educational content for students includes games, science projects
and free downloads, like print-ready bookmarks of the agency's famous
mathematicians.

NASA's website has a huge area devoted to
\href{https://www.nasa.gov/stem}{STEM education} projects and
\href{https://www.nasa.gov/education/materials/}{resources for students}
of all ages to use or download --- like
\href{https://www.nasa.gov/stem-ed-resources/from-hidden-figures-to-modern-figures-bookmark-and-minibiography.html}{free
bookmarks} with biographies of
\href{https://www.nasa.gov/content/katherine-johnson-biography}{Katherine
Johnson} and the math wizards of
\href{https://family.20thcenturystudios.com/movies/hidden-figures}{``Hidden
Figures''} fame. Lessons start at the kindergarten level to teach
fundamental concepts in physics and engineering, like
\href{https://www.nasa.gov/stem-ed-resources/building-with-pasta.html}{a
pasta project} that has students ``use uncooked spaghetti to build the
tallest free-standing structure that will support a marshmallow for at
least 15 seconds.'' And if that project leads to culinary questions,
search the NASA site for its
\href{https://www.nasa.gov/audience/foreducators/stem-on-station/ditl_eating}{``Eating
in Space''} videos.

Advertisement

\protect\hyperlink{after-bottom}{Continue reading the main story}

\hypertarget{site-index}{%
\subsection{Site Index}\label{site-index}}

\hypertarget{site-information-navigation}{%
\subsection{Site Information
Navigation}\label{site-information-navigation}}

\begin{itemize}
\tightlist
\item
  \href{https://help.nytimes3xbfgragh.onion/hc/en-us/articles/115014792127-Copyright-notice}{©~2020~The
  New York Times Company}
\end{itemize}

\begin{itemize}
\tightlist
\item
  \href{https://www.nytco.com/}{NYTCo}
\item
  \href{https://help.nytimes3xbfgragh.onion/hc/en-us/articles/115015385887-Contact-Us}{Contact
  Us}
\item
  \href{https://www.nytco.com/careers/}{Work with us}
\item
  \href{https://nytmediakit.com/}{Advertise}
\item
  \href{http://www.tbrandstudio.com/}{T Brand Studio}
\item
  \href{https://www.nytimes3xbfgragh.onion/privacy/cookie-policy\#how-do-i-manage-trackers}{Your
  Ad Choices}
\item
  \href{https://www.nytimes3xbfgragh.onion/privacy}{Privacy}
\item
  \href{https://help.nytimes3xbfgragh.onion/hc/en-us/articles/115014893428-Terms-of-service}{Terms
  of Service}
\item
  \href{https://help.nytimes3xbfgragh.onion/hc/en-us/articles/115014893968-Terms-of-sale}{Terms
  of Sale}
\item
  \href{https://spiderbites.nytimes3xbfgragh.onion}{Site Map}
\item
  \href{https://help.nytimes3xbfgragh.onion/hc/en-us}{Help}
\item
  \href{https://www.nytimes3xbfgragh.onion/subscription?campaignId=37WXW}{Subscriptions}
\end{itemize}
