Sections

SEARCH

\protect\hyperlink{site-content}{Skip to
content}\protect\hyperlink{site-index}{Skip to site index}

\href{https://www.nytimes3xbfgragh.onion/section/business}{Business}

\href{https://myaccount.nytimes3xbfgragh.onion/auth/login?response_type=cookie\&client_id=vi}{}

\href{https://www.nytimes3xbfgragh.onion/section/todayspaper}{Today's
Paper}

\href{/section/business}{Business}\textbar{}How Automated Background
Checks Freeze Out Renters

\url{https://nyti.ms/3evZJQ9}

\begin{itemize}
\item
\item
\item
\item
\item
\item
\end{itemize}

Advertisement

\protect\hyperlink{after-top}{Continue reading the main story}

Supported by

\protect\hyperlink{after-sponsor}{Continue reading the main story}

\hypertarget{how-automated-background-checks-freeze-out-renters}{%
\section{How Automated Background Checks Freeze Out
Renters}\label{how-automated-background-checks-freeze-out-renters}}

Algorithms that scan everything from terror watch lists to eviction
records spit out flawed tenant screening reports. And almost nobody is
watching.

\includegraphics{https://static01.graylady3jvrrxbe.onion/images/2020/05/31/business/00markup-renters1/00markup-renters1-articleLarge.jpg?quality=75\&auto=webp\&disable=upscale}

By Lauren Kirchner and
\href{https://www.nytimes3xbfgragh.onion/by/matthew-goldstein}{Matthew
Goldstein}

\begin{itemize}
\item
  May 28, 2020
\item
  \begin{itemize}
  \item
  \item
  \item
  \item
  \item
  \item
  \end{itemize}
\end{itemize}

\emph{This article was reported and written in collaboration with}
\href{https://themarkup.org/locked-out/2020/05/28/access-denied-faulty-automated-background-checks-freeze-out-renters}{\emph{The
Markup}}\emph{, a nonprofit newsroom investigating technology's effects
on society.}

Burglary and domestic assault in Minnesota. Selling meth and jumping
bail in Kentucky. Driving without insurance in Arkansas. Disorderly
conduct. Theft. Lying to a police officer. Unspecified ``crimes.'' Too
many narcotics charges to count.

That's what the landlord for an apartment in St. Helens, Ore., saw when
he ran a background check for Samantha Johnson, a prospective tenant, in
2018.

But none of the charges were hers.

The growing data economy and the
\href{https://www.pewresearch.org/fact-tank/2017/07/19/more-u-s-households-are-renting-than-at-any-point-in-50-years/}{rise
of American rentership} since the 2008 financial crisis have fueled a
rapid expansion of the tenant screening industry,
\href{https://www.ibisworld.com/united-states/market-research-reports/background-check-services-industry/}{now
valued at \$1 billion}. The companies produce cheap and fast --- but not
necessarily accurate --- reports for
\href{https://www.mysmartmove.com/SmartMove/blog/landlord-rental-market-survey-insights-infographic.page}{an
estimated nine out of 10 landlords} across the country.

The automated background check for Ms. Johnson cast a wide net, looking
for negative information from criminal databases even in states where
she had never lived and pulling in records for women whose middle names,
races and dates of birth didn't match her own. It combined criminal
records from five other women: four Samantha Johnsons and a woman who
had used the name as an alias, even though the screening report said she
was an ``active inmate'' in a Kentucky jail at the time.

``You can totally tell we're not the same person at all,'' said Ms.
Johnson, who eventually got the apartment after she convinced the
landlord she wasn't a criminal.

It was not the first time she had been the victim of incorrect automated
screening reports. It wouldn't be the last, either.

\hypertarget{false-reports-of-crime-with-no-human-review}{%
\subsection{False Reports of Crime, With No Human
Review}\label{false-reports-of-crime-with-no-human-review}}

The reports can be created in a few seconds, using searches based on
partial names or incomplete dates of birth. Tenants generally have no
choice but to submit to the screenings and typically pay an application
fee for the privilege. Automated reports are usually delivered to
landlords without a human ever glancing at the results to see if they
contain obvious mistakes, according to court records and interviews.

A review of hundreds of federal lawsuits filed against screening
companies over the past 10 years shows how hasty, sloppy matches can
lead to reports that wrongly label people deadbeats, criminals or sex
offenders. Among those who say they were wrongly maligned:

\begin{itemize}
\tightlist
\item
  Davone Jackson, who was denied low-income housing in Tennessee after
  the screening company RealPage reported that he had twice been
  convicted of trafficking in heroin in Kentucky and was on Wisconsin's
  sex offender registry. In fact, those records belonged to an Eric
  Jackson and a James Jackson. After the denial, Davone Jackson said, he
  and his 9-year-old daughter were forced to live in a small motel room
  for nearly a year.
\end{itemize}

\begin{itemize}
\tightlist
\item
  Glenn Patrick Thompson Sr. and Glenn Patrick Thompson Jr., who said
  they had been left homeless near Seattle after a tenant screening
  company called On-Site, which is now part of RealPage, told two
  different landlords that the father and son had been previously
  evicted. In fact, the eviction was for a Patricia Thompson, who was
  not related to them.
\end{itemize}

\begin{itemize}
\tightlist
\item
  William Hall Jr., who lost out on a duplex in his small town in
  Georgia after TransUnion Rental Screening Solutions said he had
  sexually abused a minor. The criminal record belonged to a William
  Hall who was 30 years older and possibly dead. Mr. Hall said the
  landlord had stopped returning his telephone calls after receiving the
  incorrect report.
\end{itemize}

Mr. Hall's suit is pending; the others were settled for undisclosed
sums.

The screening process happens so quickly and the competition for
apartments can be so fierce that prospective renters don't always know
why they were turned down, much less whether an incorrect background
report was the cause.

Some screening companies don't even provide the underlying records to
landlords, instead producing a color-coded ``risk'' score or a thumbs-up
or thumbs-down lease recommendation.

\includegraphics{https://static01.graylady3jvrrxbe.onion/images/2020/05/31/business/28markup-renters-Johnsongraoh2/oakImage-1590695634907-articleLarge.png?quality=75\&auto=webp\&disable=upscale}

Screening company employees have stated in lawsuits that they err on the
side of including any possible match, rather than excluding possible
errors. The owner of one screening company criticized his industry,
saying his peers can do better.

``We can figure out how to match a record,'' said Matt Visser, chief
executive of Victig Screening Solutions, whose Utah company sells 10,000
to 20,000 tenant and employment screening reports a month. He said his
company verified negative findings. ``It requires a human element,'' he
said.

``When we are performing any of these reports, it is a fairly monumental
moment in someone's life,'' he added. ``You just have to give a crap.''

Large background firms, including RealPage, CoreLogic and TransUnion,
declined interview requests for this article. They referred specific
questions to a trade group, the Consumer Data Industry Association.

Noting the millions of tenant background reports produced each year, the
group denied that any systemic problems existed and accused consumer
lawyers of being myopic.

``If I sat in a cardiologist's office all day, all I would see is people
with heart problems,'' said Eric Ellman, the association's senior vice
president for public policy and legal affairs. He acknowledged that it
hadn't developed any standards for screening accuracy but said the
companies had their own policies.

In responses to lawsuits, tenant screening companies say renters dispute
fewer than 1 percent of reports. But it's impossible to know the actual
error rate because tenants may not always know to complain.

With about half of the nation's 43 million rentals
\href{https://www.cbre.us/research-and-reports/US-Multifamily-Research-Brief---Apartment-Turnover-Rate-Continues-to-Fall-July-2019}{turning
over every year}, even an error rate of 1 percent could upend the lives
of hundreds of thousands of people.

Regulators have taken action against a few companies for slipshod tenant
screenings; the Fair Credit Reporting Act requires background screeners
to ``follow reasonable procedures to assure maximum possible accuracy.''
But rejected tenants continue to complain about the same careless
practices by companies that regulators had called out, interviews and
federal lawsuits show.

Ms. Johnson has lost count of the times she has been turned down for
housing or work because of incorrect background reports. Since 2016, she
has sued six tenant screening companies for incorrect reports. All of
them have settled.

When she moved out of the apartment in St. Helens and applied to rent a
house, the background report was 112 pages long. By then Ms. Johnson
knew the drill: Find out where the report came from, call the screening
company, fax in a copy of her ID and start the dispute process.

``I've tried to figure out if there's something I can do, to stop that
from happening,'' she said. ``But I don't think that there really is,
because there's just so many background companies out there --- and
they're not doing their jobs.''

\hypertarget{lax-rules-and-wild-cards}{%
\subsection{Lax Rules and Wild Cards}\label{lax-rules-and-wild-cards}}

Tenant screening was once confined to a simple credit check with the
three major credit bureaus and a few phone calls to references, but it
was revolutionized by the advent of cheap or even free, easily available
electronic court records. These include criminal records from across the
country, sex-offender registries, terrorism watch lists and housing
court records.

Easy access to the troves of data has also made it possible for anyone
with a computer to become a background screener: About 2,000 companies
offer the service, but
\href{https://files.consumerfinance.gov/f/documents/201909_cfpb_market-snapshot-background-screening_report.pdf}{that's
only an estimate}. Tenant screeners don't have to register with any
government agency.

People can complain about faulty background reports to the Federal Trade
Commission or the Consumer Financial Protection Bureau --- or sue. But
regulators have not limited tenant screening as much as other kinds of
background checks.

Regulators forced credit bureaus to follow standards for matching
records to a person, and the kinds of records the bureaus can legally
report are limited. Rules for employment screening, which some of the
tenant-screening firms provide, require employers to share the negative
report with a rejected applicant.

None of those restrictions applies to tenant screening.

Federal law requires landlords only to tell tenants if they were turned
down because of a negative report and who produced it. Under the Fair
Credit Reporting Act, screening companies have 30 days to respond to
tenants' requests for corrections. By then, a landlord may have given
the apartment away.

``It's just crazy that you can't get immediate results,'' said Andrew
Guzzo, a consumer lawyer who has filed more than 100 federal lawsuits
against background screeners and other consumer reporting agencies.
``There's not many worse imaginable consumer financial services related
impacts that you could have, more than an inaccurate tenant screening
that costs you the ability to rent an apartment.''

Tenants can't get ahead of the problem by checking their background
reports in advance because too many companies provide the service.

``You can't just go to one place and request your free annual report,''
said Ariel Nelson, author of a 52-page National Consumer Law Center
\href{https://www.nclc.org/issues/rpt-broken-records-redux.html}{report}
on screening errors.

A handful of cities have begun to regulate tenant screening by limiting
a landlord's ability to reject an applicant for old criminal convictions
or evictions.

Courts have also begun to take notice; a panel of federal judges
recently consolidated seven lawsuits about errors in TransUnion's tenant
screening reports.

Screening companies put the onus for accuracy on landlords, telling them
right in the reports that they should double-check them.

Steven Schachtman, a longtime Minneapolis landlord and property manager
who oversees about 10,000 rentals and uses screening firms, said it was
difficult for a landlord to check the reports to ensure they matched
their applicants.

``That is why we are hiring them,'' he said. ``I assume they have
matched everything up.''

One CoreLogic employee said during a deposition for a federal lawsuit
that she considered a background report ``accurate'' if it correctly
reported what was in public records.

A common method that screening companies use to increase hits, court
records show, is a so-called wild-card search, which gathers different
names that start with the same few letters.

Terrence Enright's experience, described in his federal lawsuit against
National Tenant Network, shows how it works. When he applied for an
apartment in Chicago in 2014, the company searched for ``Enright, Ter*''
and ``Terrence, Enr*.'' But the company also searched for misspelled
versions, including ``Enwright, Ter*'' and found a match: an eviction
for a Teri Enwright in California, one of what Mr. Enright said were
three evictions mistakenly attributed to him, which resulted in his
being denied the apartment. His lawsuit was settled out of court.

Screening companies could use more careful methods to reduce incorrect
reports, such as excluding eviction records for addresses that don't
appear on a person's credit report. Or reporting only records that match
full names and other data, such as complete dates of birth.

But they often don't, tenants and consumer attorneys say, and the errors
can have an outsize effect on people with common names --- particularly
members of minority groups, which
\href{https://www.census.gov/library/stories/2017/08/what-is-in-a-name.html}{tend
to have fewer unique last names}. For example, more than 12 million
Latinos nationwide share just 26 surnames, according to the census.

Image

Marco Fernandez is suing RentGrow, another screening firm, after it
included information in his report about another person named Mario
Fernandez Santana. Mr. Fernandez lives in Maryland works for an elite
military cybercommand strike force and has a top-secret security
clearance. The other man is on a federal watch list for suspected
terrorists or drug traffickers, lives in Mexico and has a different date
of birth, according to the lawsuit.

``These matching algorithms treat Hispanic names just like a
mix-and-match,'' said Mr. Fernandez's attorney, E. Michelle Drake.

RentGrow declined to explain why the company included the other man's
record in Mr. Fernandez's report, but said in a statement that it had
``promptly resolved all concerns in his favor.'' Mr. Fernandez was able
to get into the apartment he wanted.

While disputing the report, however, Mr. Fernandez said he discovered
that RentGrow had incorrectly reported the same federal watch list
information to a landlord four years earlier, according to the lawsuit.
He said he was not informed at the time.

Hector Hernandez took CoreLogic to court after the company mixed him up
with an accused drug smuggler, Hector Hernandez-Garcia, causing him, his
wife and their newborn son to be temporarily homeless in the Washington,
D.C., suburbs.

``I kept telling them: `You got the wrong guy. I'm telling you, that's
not me,''' Mr. Hernandez, who works for a pest control company, said in
an interview. CoreLogic settled the lawsuit.

In a deposition for another federal lawsuit, a CoreLogic employee said
independently verifying reports like these before sending them out
``would be an overwhelming task.''

It's certainly more expensive: Court filings in a federal lawsuit show
that RealPage pays one data broker 22 cents for each criminal record by
buying data in bulk. The company typically charges landlords \$12 per
report. If there's a dispute, the data broker would charge RealPage \$7
to hand-check a record, according to the contract, shrinking RealPage's
profits --- but not eliminating them.

That lawsuit claims RealPage produced 11,000 inaccurate renter
background reports between 2014 and 2019 using ``abbreviated'' criminal
records, which are cheaper than a full record check, bought from an
affiliate of
\href{https://www.backgroundchecks.com/}{Backgroundchecks.com}. The
records don't include details of the resolution or complete dates of
birth, so are more likely to lead to incorrect reports.

In a statement, RealPage said that ``the screening that gave rise to
this case occurred years ago'' and that the company ``has long since
made changes to its criminal matching logic that would prevent this
record from returning for the plaintiff in that case.'' It would not
elaborate on those changes.

Federal regulators have fined both RealPage and Backgroundchecks.com.

Backgroundchecks.com and an affiliated company were fined by the
Consumer Financial Protection Bureau in 2015 over employment screenings.
Among other things, the
\href{https://www.consumerfinance.gov/about-us/newsroom/cfpb-takes-action-against-two-of-the-largest-employment-background-screening-report-providers-for-serious-inaccuracies/}{agency
said the company} had failed to look for patterns in customer disputes
that would identify root causes of inaccuracies, such as court
jurisdictions with unreliable data, error-prone procedures or employees
who are particularly sloppy.

In 2018, the Federal Trade Commission fined RealPage \$3 million for
using wild-card searches; the company did not have to admit wrongdoing.
The company brings in about \$48 million annually from tenant background
screenings, according to court filings.

Despite the federal action, RealPage continues to employ a name-matching
program that tolerates a significant amount of imprecision, according to
federal lawsuits.

For instance, according to a lawsuit by Leon Howard, RealPage included
criminal records for a Lonnie Howard and eviction records for a Linnea
Howard in his background report when he applied for a rental in Georgia
in 2019. Lonnie Howard's date of birth was different than Leon Howard's,
and the report included a photo of the offender, according to the
lawsuit, which RealPage settled for an undisclosed sum.

\hypertarget{vindicated-but-still-waiting}{%
\subsection{Vindicated but Still
Waiting}\label{vindicated-but-still-waiting}}

Image

Sandra Smith spent a year and a half trying to get a denial overturned
after another person's information was sent to the public housing
authority in Jacksonville, Fla.Credit...Jon M. Fletcher

Most hurt are people living on the edge, especially those who made it to
the top of long wait lists for a housing voucher or a public housing
unit only to lose out because of faulty background checks.

Sandra Smith waited more than a year for a spot in public housing in her
hometown, Jacksonville, Fla. After a divorce, she had been staying with
friends and her mother. Unemployed while recovering from health
problems, Ms. Smith was relieved she'd finally be able to move back out
of her teenage bedroom.

``I'm 55,'' she said. ``Having my own space is something I haven't had
in a long time, and I felt like I was ready for that.''

But the housing authority turned her down after a background check
reported a 2013 eviction for a different Sandra Smith.

``Everything was a runaround,'' she said. ``No one had any more anything
to show me --- why they were so convinced that this was me.''

Agency officials said she had been denied because of the screening,
which is their policy.

``It would've taken them two to three minutes to investigate this,''
said Adam Thoresen, a Jacksonville Area Legal Aid lawyer whom Ms. Smith
turned to for help.

The signature on the lease agreement, which was attached to the online
eviction records, wasn't the same as his client's, he said. The lease
listed multiple children --- but his client has only one child, an adult
in his 30s, and the names didn't match.

``This is really simple stuff,'' Mr. Thoresen said.

It took Ms. Smith a year and a half to get her denial overturned. She is
still dealing with the paperwork and hasn't gotten an apartment.

Maddy Varner and Surya Mattu contributed reporting.

Advertisement

\protect\hyperlink{after-bottom}{Continue reading the main story}

\hypertarget{site-index}{%
\subsection{Site Index}\label{site-index}}

\hypertarget{site-information-navigation}{%
\subsection{Site Information
Navigation}\label{site-information-navigation}}

\begin{itemize}
\tightlist
\item
  \href{https://help.nytimes3xbfgragh.onion/hc/en-us/articles/115014792127-Copyright-notice}{©~2020~The
  New York Times Company}
\end{itemize}

\begin{itemize}
\tightlist
\item
  \href{https://www.nytco.com/}{NYTCo}
\item
  \href{https://help.nytimes3xbfgragh.onion/hc/en-us/articles/115015385887-Contact-Us}{Contact
  Us}
\item
  \href{https://www.nytco.com/careers/}{Work with us}
\item
  \href{https://nytmediakit.com/}{Advertise}
\item
  \href{http://www.tbrandstudio.com/}{T Brand Studio}
\item
  \href{https://www.nytimes3xbfgragh.onion/privacy/cookie-policy\#how-do-i-manage-trackers}{Your
  Ad Choices}
\item
  \href{https://www.nytimes3xbfgragh.onion/privacy}{Privacy}
\item
  \href{https://help.nytimes3xbfgragh.onion/hc/en-us/articles/115014893428-Terms-of-service}{Terms
  of Service}
\item
  \href{https://help.nytimes3xbfgragh.onion/hc/en-us/articles/115014893968-Terms-of-sale}{Terms
  of Sale}
\item
  \href{https://spiderbites.nytimes3xbfgragh.onion}{Site Map}
\item
  \href{https://help.nytimes3xbfgragh.onion/hc/en-us}{Help}
\item
  \href{https://www.nytimes3xbfgragh.onion/subscription?campaignId=37WXW}{Subscriptions}
\end{itemize}
