Sections

SEARCH

\protect\hyperlink{site-content}{Skip to
content}\protect\hyperlink{site-index}{Skip to site index}

\href{https://myaccount.nytimes3xbfgragh.onion/auth/login?response_type=cookie\&client_id=vi}{}

\href{https://www.nytimes3xbfgragh.onion/section/todayspaper}{Today's
Paper}

\href{/section/opinion}{Opinion}\textbar{}In Africa, a Drive to End
Malnutrition Meets Covid-19

\url{https://nyti.ms/2Xa0l7Y}

\begin{itemize}
\item
\item
\item
\item
\item
\item
\end{itemize}

Advertisement

\protect\hyperlink{after-top}{Continue reading the main story}

\href{/section/opinion}{Opinion}

Supported by

\protect\hyperlink{after-sponsor}{Continue reading the main story}

fIXES

\hypertarget{in-africa-a-drive-to-end-malnutrition-meets-covid-19}{%
\section{In Africa, a Drive to End Malnutrition Meets
Covid-19}\label{in-africa-a-drive-to-end-malnutrition-meets-covid-19}}

A way has been found to enrich the unfortified flour that Tanzanians eat
as a staple. But the pandemic is getting in the way.

\includegraphics{https://static01.graylady3jvrrxbe.onion/images/2019/02/13/opinion/tina-rosenberg/tina-rosenberg-thumbLarge-v2.png}

By Tina Rosenberg

Ms. Rosenberg is a co-founder of the
\href{http://solutionsjournalism.org}{Solutions Journalism Networ}k,
which supports rigorous reporting about responses to social problems.

\begin{itemize}
\item
  May 28, 2020
\item
  \begin{itemize}
  \item
  \item
  \item
  \item
  \item
  \item
  \end{itemize}
\end{itemize}

\includegraphics{https://static01.graylady3jvrrxbe.onion/images/2020/05/28/opinion/28FIXESRosenberg3/28FIXESRosenberg3-articleLarge.jpg?quality=75\&auto=webp\&disable=upscale}

When I began reporting this article, before the pandemic, it was about
an ingenious solution to a huge but hard-to-see problem: In many poor
countries, most people get their grain from mom-and-pop local mills. But
these mills don't fortify their flour with the basic nutrients that
children (and others) need.

That solution, which is important and hopeful, is still part of this
story. But it has also become a tale of how Covid-19 kills in more than
one way, including the terrible complexity of decisions of whether to
reopen economies whose workers are desperate for an income.

The case for letting people go back to work isn't just about individual
liberty. The public health argument for speeding up recovery is that
poverty too sickens and kills. A childhood in poverty can mean a
lifetime of suffering.

Wealthy countries can create a strong social safety net. France and
Germany, for example, are replacing people's lost incomes. It's costly,
but less costly than a recession. That could be the answer in America
--- if we choose it.

Outside of wealthy countries, you and your family must play the role of
safety net, which means you go to work. ``Poor people will prefer the
lottery of infection over the certainty of starvation,'' Alex De Waal
and Paul Richards
\href{https://www.bbc.com/news/world-africa-52268320}{wrote} sadly in an
article for BBC News.

So here's the hopeful story: Every rich country fortifies food. Our
diets lack certain nutrients, so governments require manufacturers to
add them to certain foods. (Milk doesn't actually contain vitamin D
until it's fortified. Froot Loops don't supply vitamin C without help.)

If Americans need fortified foods, so much more do people who don't have
the luxury of a varied diet. Most Tanzanians eat cornmeal mush, or
ugali, every day. Many eat very little else. Ugali is filling. But it's
not nutritious.

A third of Tanzanian children are deficient in iron and vitamin A (which
prevents blindness). Many also lack zinc, vitamin B12 and iodine,
causing damage to their immune systems and cognitive development. Women
lack folate, a deficiency that can lead to neural tube defects like
spina bifida in their children at birth. Only two-thirds of Tanzanian
children grow to a normal height. And 130 children die of malnutrition
every day; countless more are damaged for life.

At a cost of 25 cents per person per year, fortifying food is by far the
cheapest way to improve health. Better nutrition also increases economic
productivity. Every dollar a country spends on fortification will reap
\$30
in\href{https://www.copenhagenconsensus.com/publication/third-copenhagen-consensus-hunger-and-malnutrition-assessment-hoddinott-rosegrant-torero}{economic
benefits paid back.}

Since 2011, Tanzania has required all maize mills to add a powdered
mixture of iron, zinc, vitamin B12 and folate to their flour.

Yet according to the Global Fortification Data Exchange, the
\href{https://fortificationdata.org/country-fortification-dashboard/?alpha3_code=TZA\&lang=en}{amount
of maize flour} fortified is \ldots{} close to zero. The equipment and
nutrient mix must be imported, and millers know there are no
consequences for failing to fortify.

\includegraphics{https://static01.graylady3jvrrxbe.onion/images/2020/05/28/opinion/28FIXESRosenberg5/28FIXESRosenberg5-articleLarge.jpg?quality=75\&auto=webp\&disable=upscale}

One problem is that the government only has only about 55 people to
monitor all food and medicine issues nationwide. ``Prioritizing
fortification is a big challenge when the benefits are invisible,'' said
Penjani Mkambula, the global program leader for fortification at the
Geneva-based \href{http://www.gainhealth.org/}{Global Alliance for
Improved Nutrition}. ``If somebody eats unfortified food, it's still
food. This is a hidden problem.''

The situation is even worse in the mom-and-pop village mills that
produce 87 percent of all maize flour.

``It was just too much money,'' said Philipo Kulwa, the chief operating
officer of Lina Millers in Dar es Salaam, a medium-size mill by
Tanzanian standards. He said that in cities, many customers are aware of
the benefits of fortification. ``Sometimes people call and ask if our
flour is fortified,'' he said. But Tanzania is overwhelmingly rural.
``There, people have no idea about fortified foods,'' he said. ``They
just consider the price.''

At the end of 2018, Mr. Kulwa began working with
\href{http://projecthealthychildren.com/small-scale-fortification/}{Sanku},
a nongovernmental organization. Sanku started as part of
\href{http://www.projecthealthychildren.org/}{Project Healthy Children},
which promotes large-scale fortification. ``We were working with
governments at the policy level,'' said Felix Brooks-church, an American
based in Dar es Salaam who co-founded Sanku (with Dave Dodson, a
Stanford University lecturer, who is also a former Republican candidate
in Wyoming for the U.S. Senate). ``But 10 years in, we realized we were
leaving out those arguably most at risk.'' In 2013, they created Sanku
to work on small-scale fortification.

Other organizations have tried to help small mills fortify flour. That
involved scooping the right amount of nutrients into flour by hand.
``They gave up,'' Mr. Brooks-church said. ``Small-scale fortification
got a reputation as a waste of time.''

But Sanku invented new technology. It worked with Stanford to develop
what it now calls a dosifier --- a machine that mixes the right amount
of nutrients into the flour.

Various organizations working in Africa now use Sanku's dosifier. The
World Food Program employs it to fortify flour in refugee camps in Kenya
and Tanzania, feeding several hundred thousand children.

Image

Parcels of empty flour bags bundled with a nutrient mix, in Sanku's
warehouse in Dar es Salaam for delivery to local maize
millers.Credit...Malicky Boaz/Sanku

Sanku itself works directly with millers in Tanzania. Mr. Kulwa said
that Sanku helped him get a grant for the dosifier. It supplies the
nutrient mix at no cost. Sanku also trained him, monitors the equipment
via a cellular link and comes back when there's a problem.

Sanku also tackled a second challenge: a business model. ``The cost of
concentrated nutrients is not huge for small millers, but it's still
material,'' Mr. Brooks-church said. ``They couldn't afford it or pass it
on to consumers --- a mother in a village couldn't afford to pay more
for a fortified product.''

Sanku's answer was to bundle the nutrient mix with something every
miller needs --- flour sacks. Mr. Kulwa buys sacks from Sanku at market
prices. Because Sanku buys in bulk, it can make enough on the sacks to
throw in the vitamin-mineral mix.

Sanku tries to recruit millers by asking them to help their neighbors.
``The first thing we say is, Do you want to be a health champion in your
community?'' Mr. Brooks-church said.

Of course, the miller says. The next question is invariably, What's it
going to cost me?

Mr. Brooks-church tries to persuade them that they will make money.
``Food fortification is a really hard concept to sell,'' he said. ``But
everybody knows what quality is.'' Most small mills also sell flour to
the public. Mr. Brooks-church tells millers that Sanku will help bring
visible improvements to their mills and flour. Normal bags are simple
gunny sacks with the mill's logo. Sanku's bags are shiny, with big pink
stripes. ``They have the logo of Tanzania's food agency,'' Mr.
Brooks-church said. ``They look clean. We holistically try to make their
business better.''

Image

A vendor delivering fortified flour to shops in Dar es
Salaam.Credit...Malicky Boaz/Sanku

The government logo is particularly important, said James Flock, the
head of the U.S. Agency for International Development's Nafaka (cereals)
program in Tanzania: ``The trust that small-scale farmers hold in
government information is powerful.''

Nafaka aims to help farmers and millers professionalize, including
adding fortification. It also creates markets for their improved flour.
Between November and January, Nafaka tried sending regular texts to
farmers (most of them female) about fortified flour and where to buy it.
Recipients' purchase of fortified flour went from 5 percent to 30
percent.

Another way Nafaka creates markets is by connecting millers to the
government's school lunch programs. It worked with 37 millers to provide
fortified flour to nearly 100 schools.

Until now, that is. Covid-19 has closed schools, and with them, a
regular source of nutrients for many children.

Maize flour will be the last food Tanzanian families buy when they can
buy nothing else. So fortification is more needed than ever. But it's
risky. A miller who carried the virus could become a super-spreader.
Sanku has given all its millers health kits containing masks, gloves,
alcohol rub and cleaner.

Except for schools, Tanzania is largely open. People need to work today
to eat today, and so in the markets it is business as usual. Buses are
jammed. Many churches are full --- the president of Tanzania, John
Magufuli, has told people that prayer can vanquish the disease.

Tanzania,
\href{https://www.nytimes3xbfgragh.onion/2020/04/18/world/africa/africa-coronavirus-ventilators.html}{like
most countries in Africa}, is not equipped for the consequences. Many
people have no water or soap. In hospitals, oxygen is in short supply.
There are virtually no ventilators.

Tanzanians can, however, wear masks. So Sanku has hired an army of
workers to cut and sew its flour sacks into masks --- 10,000 so far. It
has enough sacks to make millions of masks. Those polypropylene sacks
embody the terrible dilemma of Covid-19: Should they be made into masks?
Or hold fortified flour? Should they fight a virus? Or fight
malnutrition? It is an impossible choice. But it's like the ones that
billions of people must make.

Tina Rosenberg won a Pulitzer Prize for her book
``\href{http://www.randomhouse.com/catalog/display.pperl?isbn=9780679744993}{The
Haunted Land:} Facing Europe's Ghosts After Communism.'' She is a former
editorial writer for The Times and the author, most recently, of
``\href{http://books.wwnorton.com/books/Join-the-Club}{Join the Club:}
How Peer Pressure Can Transform the World'' and the World War II spy
story e-book
\href{https://www.goodreads.com/book/show/16124470-d-for-deception}{``D
for Deception.''}

\emph{To receive email alerts for Fixes columns, sign up}
\href{http://eepurl.com/ABIxL}{\emph{here.}}

\emph{The Times is committed to publishing}
\href{https://www.nytimes3xbfgragh.onion/2019/01/31/opinion/letters/letters-to-editor-new-york-times-women.html}{\emph{a
diversity of letters}} \emph{to the editor. We'd like to hear what you
think about this or any of our articles. Here are some}
\href{https://help.nytimes3xbfgragh.onion/hc/en-us/articles/115014925288-How-to-submit-a-letter-to-the-editor}{\emph{tips}}\emph{.
And here's our email:}
\href{mailto:letters@NYTimes.com}{\emph{letters@NYTimes.com}}\emph{.}

\emph{Follow The New York Times Opinion section on}
\href{https://www.facebookcorewwwi.onion/nytopinion}{\emph{Facebook}}\emph{,}
\href{http://twitter.com/NYTOpinion}{\emph{Twitter (@NYTopinion)}}
\emph{and}
\href{https://www.instagram.com/nytopinion/}{\emph{Instagram}}\emph{.}

Advertisement

\protect\hyperlink{after-bottom}{Continue reading the main story}

\hypertarget{site-index}{%
\subsection{Site Index}\label{site-index}}

\hypertarget{site-information-navigation}{%
\subsection{Site Information
Navigation}\label{site-information-navigation}}

\begin{itemize}
\tightlist
\item
  \href{https://help.nytimes3xbfgragh.onion/hc/en-us/articles/115014792127-Copyright-notice}{©~2020~The
  New York Times Company}
\end{itemize}

\begin{itemize}
\tightlist
\item
  \href{https://www.nytco.com/}{NYTCo}
\item
  \href{https://help.nytimes3xbfgragh.onion/hc/en-us/articles/115015385887-Contact-Us}{Contact
  Us}
\item
  \href{https://www.nytco.com/careers/}{Work with us}
\item
  \href{https://nytmediakit.com/}{Advertise}
\item
  \href{http://www.tbrandstudio.com/}{T Brand Studio}
\item
  \href{https://www.nytimes3xbfgragh.onion/privacy/cookie-policy\#how-do-i-manage-trackers}{Your
  Ad Choices}
\item
  \href{https://www.nytimes3xbfgragh.onion/privacy}{Privacy}
\item
  \href{https://help.nytimes3xbfgragh.onion/hc/en-us/articles/115014893428-Terms-of-service}{Terms
  of Service}
\item
  \href{https://help.nytimes3xbfgragh.onion/hc/en-us/articles/115014893968-Terms-of-sale}{Terms
  of Sale}
\item
  \href{https://spiderbites.nytimes3xbfgragh.onion}{Site Map}
\item
  \href{https://help.nytimes3xbfgragh.onion/hc/en-us}{Help}
\item
  \href{https://www.nytimes3xbfgragh.onion/subscription?campaignId=37WXW}{Subscriptions}
\end{itemize}
