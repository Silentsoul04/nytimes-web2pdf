Sections

SEARCH

\protect\hyperlink{site-content}{Skip to
content}\protect\hyperlink{site-index}{Skip to site index}

\href{https://myaccount.nytimes3xbfgragh.onion/auth/login?response_type=cookie\&client_id=vi}{}

\href{https://www.nytimes3xbfgragh.onion/section/todayspaper}{Today's
Paper}

\href{/section/opinion}{Opinion}\textbar{}If We Had a Real Leader

\url{https://nyti.ms/2ZO3Nae}

\begin{itemize}
\item
\item
\item
\item
\item
\item
\end{itemize}

Advertisement

\protect\hyperlink{after-top}{Continue reading the main story}

\href{/section/opinion}{Opinion}

Supported by

\protect\hyperlink{after-sponsor}{Continue reading the main story}

\hypertarget{if-we-had-a-real-leader}{%
\section{If We Had a Real Leader}\label{if-we-had-a-real-leader}}

Imagining Covid under a normal president.

\href{https://www.nytimes3xbfgragh.onion/by/david-brooks}{\includegraphics{https://static01.graylady3jvrrxbe.onion/images/2018/04/03/opinion/david-brooks/david-brooks-thumbLarge-v2.png}}

By \href{https://www.nytimes3xbfgragh.onion/by/david-brooks}{David
Brooks}

Opinion Columnist

\begin{itemize}
\item
  May 28, 2020
\item
  \begin{itemize}
  \item
  \item
  \item
  \item
  \item
  \item
  \end{itemize}
\end{itemize}

\includegraphics{https://static01.graylady3jvrrxbe.onion/images/2020/05/28/opinion/28brooksWeb/merlin_166270017_ff3065a8-98d1-4e55-9aa3-5a3e0c8e4f0a-articleLarge.jpg?quality=75\&auto=webp\&disable=upscale}

This week I had a conversation that left a mark. It was with Mary Louise
Kelly and E.J. Dionne on NPR's ``All Things Considered,'' and it was
about
\href{https://www.npr.org/2020/05/27/863422824/how-presidents-lead-in-times-of-national-mourning}{how
past presidents had handled moments of national mourning} --- Lincoln
after Gettysburg, Reagan after the Challenger explosion and Obama after
the Sandy Hook school shootings.

The conversation left me wondering what America's experience of the
pandemic would be like if we had a real leader in the White House.

If we had a real leader, he would have realized that tragedies like
100,000 Covid-19 deaths touch something deeper than politics: They touch
our shared vulnerability and our profound and natural sympathy for one
another.

In such moments, a real leader steps outside of his political role and
reveals himself uncloaked and humbled, as someone who can draw on his
own pains and simply be present with others as one sufferer among a
common sea of sufferers.

If we had a real leader, she would speak of the dead not as a faceless
mass but as individual persons, each seen in unique dignity. Such a
leader would draw on the common sources of our civilization, the stores
of wisdom that bring collective strength in hard times.

Lincoln went back to the old biblical cadences to comfort a nation.
After the church shooting in Charleston, Barack Obama went to ``Amazing
Grace,'' the old abolitionist anthem that has wafted down through the
long history of African-American suffering and redemption.

In his impromptu remarks right after the assassination of Martin Luther
King, Robert Kennedy recalled the slaying of his own brother and quoted
Aeschylus: ``In our sleep, pain which cannot forget falls drop by drop
upon the heart until, in our own despair, against our will, comes wisdom
through the awful grace of God.''

If we had a real leader, he would be bracingly honest about how bad
things are, like Churchill after the fall of Europe. He would have
stored in his upbringing the understanding that hard times are the
making of character, a revelation of character and a test of character.
He would offer up the reality that to be an American is both a gift and
a task. Every generation faces its own apocalypse, and, of course, we
will live up to our moment just as our ancestors did theirs.

If we had a real leader, she would remind us of our common covenants and
our common purposes. America is a diverse country joined more by a
common future than by common pasts. In times of hardships real leaders
re-articulate the purpose of America, why we endure these hardships and
what good we will make out of them.

After the Challenger explosion, Reagan reminded us that we are a nation
of explorers and that the explorations at the frontiers of science would
go on, thanks in part to those who ``slipped the surly bonds of earth to
touch the face of God.''

At Gettysburg, Lincoln crisply described why the fallen had sacrificed
their lives --- to show that a nation ``dedicated to the proposition
that all men are created equal'' can long endure and also to bring about
``a new birth of freedom'' for all the world.

Of course, right now we don't have a real leader. We have Donald Trump,
a man who can't fathom empathy or express empathy, who can't laugh or
cry, love or be loved --- a damaged narcissist who is unable to see the
true existence of other human beings except insofar as they are good or
bad for himself.

But it's too easy to offload all blame on Trump. Trump's problem is not
only that he's emotionally damaged; it is that he is unlettered. He has
no literary, spiritual or historical resources to draw upon in a crisis.

All the leaders I have quoted above were educated under a curriculum
that put character formation at the absolute center of education. They
were trained by people who assumed that life would throw up hard and
unexpected tests, and it was the job of a school, as one headmaster put
it, to produce young people who would be ``acceptable at a dance,
invaluable in a shipwreck.''

Think of the generations of religious and civic missionaries, like
Frances Perkins, who flowed out of Mount Holyoke. Think of all the
Morehouse Men and Spelman Women. Think of all the young students, in
schools everywhere, assigned Plutarch and Thucydides, Isaiah and
Frederick Douglass --- the great lessons from the past on how to lead,
endure, triumph or fail. Only the great books stay in the mind for
decades and serve as storehouses of wisdom when hard times come.

Right now, science and the humanities should be in lock step: science
producing vaccines, with the humanities stocking leaders and citizens
with the capacities of resilience, care and collaboration until they
come. But, instead, the humanities are in crisis at the exact moment
history is revealing how vital moral formation really is.

One of the lessons of this crisis is that help isn't coming from some
centralized place at the top of society. If you want real leadership,
look around you.

\emph{The Times is committed to publishing}
\href{https://www.nytimes3xbfgragh.onion/2019/01/31/opinion/letters/letters-to-editor-new-york-times-women.html}{\emph{a
diversity of letters}} \emph{to the editor. We'd like to hear what you
think about this or any of our articles. Here are some}
\href{https://help.nytimes3xbfgragh.onion/hc/en-us/articles/115014925288-How-to-submit-a-letter-to-the-editor}{\emph{tips}}\emph{.
And here's our email:}
\href{mailto:letters@NYTimes.com}{\emph{letters@NYTimes.com}}\emph{.}

\emph{Follow The New York Times Opinion section on}
\href{https://www.facebookcorewwwi.onion/nytopinion}{\emph{Facebook}}\emph{,}
\href{http://twitter.com/NYTOpinion}{\emph{Twitter (@NYTopinion)}}
\emph{and}
\href{https://www.instagram.com/nytopinion/}{\emph{Instagram}}\emph{.}

Advertisement

\protect\hyperlink{after-bottom}{Continue reading the main story}

\hypertarget{site-index}{%
\subsection{Site Index}\label{site-index}}

\hypertarget{site-information-navigation}{%
\subsection{Site Information
Navigation}\label{site-information-navigation}}

\begin{itemize}
\tightlist
\item
  \href{https://help.nytimes3xbfgragh.onion/hc/en-us/articles/115014792127-Copyright-notice}{©~2020~The
  New York Times Company}
\end{itemize}

\begin{itemize}
\tightlist
\item
  \href{https://www.nytco.com/}{NYTCo}
\item
  \href{https://help.nytimes3xbfgragh.onion/hc/en-us/articles/115015385887-Contact-Us}{Contact
  Us}
\item
  \href{https://www.nytco.com/careers/}{Work with us}
\item
  \href{https://nytmediakit.com/}{Advertise}
\item
  \href{http://www.tbrandstudio.com/}{T Brand Studio}
\item
  \href{https://www.nytimes3xbfgragh.onion/privacy/cookie-policy\#how-do-i-manage-trackers}{Your
  Ad Choices}
\item
  \href{https://www.nytimes3xbfgragh.onion/privacy}{Privacy}
\item
  \href{https://help.nytimes3xbfgragh.onion/hc/en-us/articles/115014893428-Terms-of-service}{Terms
  of Service}
\item
  \href{https://help.nytimes3xbfgragh.onion/hc/en-us/articles/115014893968-Terms-of-sale}{Terms
  of Sale}
\item
  \href{https://spiderbites.nytimes3xbfgragh.onion}{Site Map}
\item
  \href{https://help.nytimes3xbfgragh.onion/hc/en-us}{Help}
\item
  \href{https://www.nytimes3xbfgragh.onion/subscription?campaignId=37WXW}{Subscriptions}
\end{itemize}
