Sections

SEARCH

\protect\hyperlink{site-content}{Skip to
content}\protect\hyperlink{site-index}{Skip to site index}

\href{https://www.nytimes3xbfgragh.onion/section/technology}{Technology}

\href{https://myaccount.nytimes3xbfgragh.onion/auth/login?response_type=cookie\&client_id=vi}{}

\href{https://www.nytimes3xbfgragh.onion/section/todayspaper}{Today's
Paper}

\href{/section/technology}{Technology}\textbar{}A.C.L.U. Accuses
Clearview AI of Privacy `Nightmare Scenario'

\url{https://nyti.ms/2TNET6P}

\begin{itemize}
\item
\item
\item
\item
\item
\end{itemize}

Advertisement

\protect\hyperlink{after-top}{Continue reading the main story}

Supported by

\protect\hyperlink{after-sponsor}{Continue reading the main story}

\hypertarget{aclu-accuses-clearview-ai-of-privacy-nightmare-scenario}{%
\section{A.C.L.U. Accuses Clearview AI of Privacy `Nightmare
Scenario'}\label{aclu-accuses-clearview-ai-of-privacy-nightmare-scenario}}

The facial recognition start-up violated the privacy of Illinois
residents by collecting their images without their consent, the civil
liberties group says in a new lawsuit.

\includegraphics{https://static01.graylady3jvrrxbe.onion/images/2020/05/29/business/28clearview-print/28clearview-articleLarge-v2.jpg?quality=75\&auto=webp\&disable=upscale}

By \href{https://www.nytimes3xbfgragh.onion/by/davey-alba}{Davey Alba}

\begin{itemize}
\item
  Published May 28, 2020Updated June 3, 2020
\item
  \begin{itemize}
  \item
  \item
  \item
  \item
  \item
  \end{itemize}
\end{itemize}

The American Civil Liberties Union on Thursday sued the facial
recognition start-up Clearview AI, which claims to have helped hundreds
of law enforcement agencies use online photos to solve crimes, accusing
the company of ``unlawful, privacy-destroying surveillance activities.''

In a suit filed in Illinois, the
\href{https://www.nytimes3xbfgragh.onion/2020/06/03/business/aclu-sues-police-minneapolis.html}{A.C.L.U.}
said that Clearview violated a state law that forbids companies from
using a resident's fingerprints or face scans without consent. Under the
law, residents have the right to sue companies for up to \$5,000 per
privacy violation.

``The bottom line is that, if left unchecked, Clearview's product is
going to end privacy as we know it,'' said Nathan Freed Wessler, a
lawyer at the
\href{https://www.nytimes3xbfgragh.onion/2020/06/03/business/aclu-sues-police-minneapolis.html}{A.C.L.U.},
``and we're taking the company to court to prevent that from
happening.''

The suit, filed in the Circuit Court of Cook County, adds to the growing
backlash against Clearview since January, when
\href{https://www.nytimes3xbfgragh.onion/2020/01/18/technology/clearview-privacy-facial-recognition.html}{The
New York Times reported} that the company had amassed a database of more
than three billion photos across the internet, including from Facebook,
YouTube, Twitter and Venmo. This trove of photos enables anyone with the
Clearview app to match a person to their online photos and find links
back to the sites where the images originated.

People in New York and Vermont have also filed suits in against the
company in recent months, and the state attorneys general of Vermont and
New Jersey have ordered Clearview to stop collecting residents' photos.

According to the A.C.L.U.
\href{https://www.aclu.org/legal-document/aclu-v-clearview-ai-complaint}{suit},
``Clearview has set out to do what many companies have intentionally
avoided out of ethical concerns: create a mass database of billions of
face prints of people, including millions of Illinoisans, entirely
unbeknownst to those people, and offer paid access to that database to
private and governmental actors worldwide.''

The company's business model, the complaint said, ``appears to embody
the nightmare scenario'' of a ``private company capturing untold
quantities of biometric data for purposes of surveillance and tracking
without notice to the individuals affected, much less their consent.''

``Clearview AI is a search engine that uses only publicly available
images accessible on the internet,'' Tor Ekeland, a lawyer for
Clearview, said in a statement. ``It is absurd that the A.C.L.U. wants
to censor which search engines people can use to access public
information on the internet. The First Amendment forbids this.''

Mr. Wessler of the A.C.L.U. said the First Amendment ``does not shield
Clearview's unlawful conducts.''

``Our lawsuit does not challenge Clearview's scraping of images off of
social media platforms,'' he said. ``It challenges the secret,
nonconsensual and unlawful capture of individuals' biometric identifiers
from those images. Capturing a face print is conduct, not speech.''

The Illinois suit was prepared by the A.C.L.U., the A.C.L.U. of Illinois
and the law firm Edelson PC, which has specialized in class action suits
against technology companies for privacy violations. The firm was
involved in a suit that ended in a
\href{https://www.nytimes3xbfgragh.onion/2020/01/29/technology/facebook-privacy-lawsuit-earnings.html}{\$550
million settlement with Facebook for the tech giant's use of facial
recognition technology in Illinois}.

Other organizations that have signed on to the legal action include the
Chicago Alliance Against Sexual Exploitation, the Sex Workers Outreach
Project and the Illinois State Public Interest Research Group.

The A.C.L.U. said the lawsuit would compel a facial recognition company
to answer to groups representing sexual assault survivors, undocumented
immigrants and other vulnerable communities uniquely harmed by
surveillance.

There is a growing understanding among researchers that facial
recognition systems are worse at accurately identifying the faces of
people of color. Last December, the federal government released a study,
one of the largest of its kind, that found that most commercial facial
recognition systems exhibited bias,
\href{https://www.nytimes3xbfgragh.onion/2019/07/08/us/detroit-facial-recognition-cameras.htmlhttps://www.nytimes3xbfgragh.onion/2019/12/19/technology/facial-recognition-bias.html}{falsely
identifying African-American and Asian faces} 10 to 100 times more than
Caucasian faces.

Mallory Littlejohn from the Chicago Alliance Against Sexual
Exploitation, a Chicago-based nonprofit, said, ``We can change our names
and addresses to shield our whereabouts and identities from stalkers and
abusive partners, but we can't change our faces.''

Clearview, she said, ``put survivors in constant fear of being tracked
by those who seek to harm them, and are a threat to our security, safety
and well-being.''

Advertisement

\protect\hyperlink{after-bottom}{Continue reading the main story}

\hypertarget{site-index}{%
\subsection{Site Index}\label{site-index}}

\hypertarget{site-information-navigation}{%
\subsection{Site Information
Navigation}\label{site-information-navigation}}

\begin{itemize}
\tightlist
\item
  \href{https://help.nytimes3xbfgragh.onion/hc/en-us/articles/115014792127-Copyright-notice}{©~2020~The
  New York Times Company}
\end{itemize}

\begin{itemize}
\tightlist
\item
  \href{https://www.nytco.com/}{NYTCo}
\item
  \href{https://help.nytimes3xbfgragh.onion/hc/en-us/articles/115015385887-Contact-Us}{Contact
  Us}
\item
  \href{https://www.nytco.com/careers/}{Work with us}
\item
  \href{https://nytmediakit.com/}{Advertise}
\item
  \href{http://www.tbrandstudio.com/}{T Brand Studio}
\item
  \href{https://www.nytimes3xbfgragh.onion/privacy/cookie-policy\#how-do-i-manage-trackers}{Your
  Ad Choices}
\item
  \href{https://www.nytimes3xbfgragh.onion/privacy}{Privacy}
\item
  \href{https://help.nytimes3xbfgragh.onion/hc/en-us/articles/115014893428-Terms-of-service}{Terms
  of Service}
\item
  \href{https://help.nytimes3xbfgragh.onion/hc/en-us/articles/115014893968-Terms-of-sale}{Terms
  of Sale}
\item
  \href{https://spiderbites.nytimes3xbfgragh.onion}{Site Map}
\item
  \href{https://help.nytimes3xbfgragh.onion/hc/en-us}{Help}
\item
  \href{https://www.nytimes3xbfgragh.onion/subscription?campaignId=37WXW}{Subscriptions}
\end{itemize}
