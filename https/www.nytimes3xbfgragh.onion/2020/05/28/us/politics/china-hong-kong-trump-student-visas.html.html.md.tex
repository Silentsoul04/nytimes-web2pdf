Sections

SEARCH

\protect\hyperlink{site-content}{Skip to
content}\protect\hyperlink{site-index}{Skip to site index}

\href{https://www.nytimes3xbfgragh.onion/section/politics}{Politics}

\href{https://myaccount.nytimes3xbfgragh.onion/auth/login?response_type=cookie\&client_id=vi}{}

\href{https://www.nytimes3xbfgragh.onion/section/todayspaper}{Today's
Paper}

\href{/section/politics}{Politics}\textbar{}U.S. to Expel Chinese
Graduate Students With Ties to China's Military Schools

\href{https://nyti.ms/2BaJxW3}{https://nyti.ms/2BaJxW3}

\begin{itemize}
\item
\item
\item
\item
\item
\item
\end{itemize}

Advertisement

\protect\hyperlink{after-top}{Continue reading the main story}

Supported by

\protect\hyperlink{after-sponsor}{Continue reading the main story}

\hypertarget{us-to-expel-chinese-graduate-students-with-ties-to-chinas-military-schools}{%
\section{U.S. to Expel Chinese Graduate Students With Ties to China's
Military
Schools}\label{us-to-expel-chinese-graduate-students-with-ties-to-chinas-military-schools}}

The move is the latest in the Trump administration's efforts to impose
limits on Chinese students. But university officials say the government
is paranoid, and that the United States will lose out.

\includegraphics{https://static01.graylady3jvrrxbe.onion/images/2020/05/28/us/politics/28dc-trump-china/merlin_172874568_08e7913a-684b-4b5b-8d3d-8479a5ebc2aa-articleLarge.jpg?quality=75\&auto=webp\&disable=upscale}

\href{https://www.nytimes3xbfgragh.onion/by/edward-wong}{\includegraphics{https://static01.graylady3jvrrxbe.onion/images/2018/09/24/multimedia/author-edward-wong/author-edward-wong-thumbLarge-v5.png}}\href{https://www.nytimes3xbfgragh.onion/by/julian-e-barnes}{\includegraphics{https://static01.graylady3jvrrxbe.onion/images/2019/12/13/reader-center/author-julian-barnes/author-julian-barnes-thumbLarge.png}}

By \href{https://www.nytimes3xbfgragh.onion/by/edward-wong}{Edward Wong}
and \href{https://www.nytimes3xbfgragh.onion/by/julian-e-barnes}{Julian
E. Barnes}

\begin{itemize}
\item
  May 28, 2020
\item
  \begin{itemize}
  \item
  \item
  \item
  \item
  \item
  \item
  \end{itemize}
\end{itemize}

\href{https://cn.nytimes3xbfgragh.onion/usa/20200529/china-hong-kong-trump-student-visas/}{阅读简体中文版}\href{https://cn.nytimes3xbfgragh.onion/usa/20200529/china-hong-kong-trump-student-visas/zh-hant/}{閱讀繁體中文版}

WASHINGTON --- The Trump administration plans to cancel the visas of
thousands of
\href{https://www.nytimes3xbfgragh.onion/2017/12/12/opinion/chinese-students-mental-health.html}{Chinese
graduate students} and researchers in the United States who have direct
ties to universities affiliated with the People's Liberation Army,
according to American officials with knowledge of the discussions.

The plan would be the first designed to bar the access of a category of
Chinese students, who, over all, form the single largest foreign student
population in the United States.

It portends possible further educational restrictions, and the Chinese
government could retaliate by imposing its own visa or educational bans
on Americans. The two nations have already engaged in rounds of
retribution over policies involving trade, technology and media access,
and relations are at their
\href{https://www.nytimes3xbfgragh.onion/2020/03/22/us/politics/coronavirus-us-china.html}{worst
point in decades}.

American officials
\href{https://www.nytimes3xbfgragh.onion/2020/05/27/us/politics/china-hong-kong-pompeo-trade.html}{are
discussing ways to punish China} for its passage of a new national
security law intended to enable crackdowns in Hong Kong, but the plans
to cancel student visas were under consideration before the crisis over
the law, which was announced last week by Chinese officials. Secretary
of State Mike Pompeo discussed the visa plans with President Trump on
Tuesday in a White House meeting.

American universities are expected to push back against the
administration's move. While international educational exchange is
prized for its intellectual value, many schools also rely on full
tuition payments from foreign students to help cover costs, especially
the large group of students from China.

Administrators and teachers have been briefed in recent years by the
F.B.I. and the Justice Department on
\href{https://www.nytimes3xbfgragh.onion/2019/11/04/health/china-nih-scientists.html}{potential
national security threats} posed by Chinese students, especially ones
working in the sciences. But the university employees are wary of a
possible new
\href{https://www.nytimes3xbfgragh.onion/2019/07/20/us/politics/china-red-scare-washington.html}{``red
scare''} that targets students of a specific national background and
that could contribute to
\href{https://www.nytimes3xbfgragh.onion/2020/03/23/us/chinese-coronavirus-racist-attacks.html}{anti-Asian
racism}.

Many of them argue that they have effective security protocols in place,
and that having Chinese students be exposed to the liberalizing effects
of Western institutions outweighs the risks. Moreover, they say, the
Chinese students are experts in their subject fields and bolster
American research efforts.

Chinese students and researchers say growing scrutiny from the American
government and new official limits on visas would
\href{https://www.nytimes3xbfgragh.onion/2018/07/25/us/politics/visa-restrictions-chinese-students.html}{create
biases against them}, including when they apply for jobs or grants.

The visa cancellation could affect at least 3,000 students, according to
some official estimates. That is a tiny percentage of the approximately
360,000 Chinese students in the United States. But some of those
affected might be working on important research projects.

The move is certain to ignite public debate. Officials acknowledged
there was no direct evidence that pointed to
\href{https://theintercept.com/2020/02/02/fbi-chinese-scientists-surveillance/}{wrongdoing
by the students} who are about to lose their visas. Instead, suspicions
by American officials center on the Chinese universities at which the
students trained as undergraduates.

``In China, much more of society is government-controlled or
government-affiliated,'' said
\href{https://queenseagle.com/all/queens-college-new-president-frank-wu}{Frank
Wu}, a law professor who is the incoming president of Queens College.
``You can't function there or have partners from there if you aren't
comfortable with how the system is set up.''

``Targeting only some potential professors, scholars, students and
visitors from China is a lower level of stereotyping than banning all,''
he added. ``But it is still selective, based on national origin.''

The State Department and the National Security Council both declined to
comment.

American officials who defend the visa cancellation said the ties to the
Chinese military at those schools go far deeper than mere campus
recruiting. Instead, in many cases, the Chinese government plays a role
in selecting which students from the schools with ties to the military
can study abroad, one official said. In some cases, students who are
allowed to go overseas are expected to collect information as a
condition of having their tuition paid, the official said, declining to
reveal specific intelligence on the matter.

Officials did not provide the list of affected schools, but the People's
Liberation Army has ties to military institutions and defense research
schools,
\href{https://www.aspi.org.au/report/china-defence-universities-tracker}{as
well as to seven more traditional universities}, many of them
prestigious colleges in China with well-funded science and technology
programs.

The F.B.I. and the Justice Department have long viewed the
military-affiliated schools as a particular problem, believing military
officials train some of the graduates in basic espionage techniques and
compel them to gather and transmit information to Chinese officers.

While some government officials emphasize the intelligence threat posed
by students from military-affiliated universities, others see those
Chinese citizens as potential recruits for American spy agencies.
Preventing the students from coming to the United States may make it
more difficult for the agencies to recruit assets inside the Chinese
military.

After completing their graduate work, some students land jobs at
prominent technology companies in the United States. That has made some
current and former American officials wary that the employees could
engage in industrial espionage.

Last year, Senator Richard M. Burr, Republican of North Carolina, who
was then the chairman of the Intelligence Committee,
\href{https://www.nytimes3xbfgragh.onion/2019/02/06/us/politics/richard-burr-china-huawei-5g.html}{predicted
the administration would cut the number of visas} going to Chinese
students, citing the threat of technology theft.

Senator Marco Rubio, Republican of Florida, who now leads the committee,
has sent letters to universities in his state warning about ties to the
Chinese government.

Mr. Rubio has been pushing schools to cut relations with China's
\href{https://www.nytimes3xbfgragh.onion/2020/02/06/us/chinas-lavish-funds-lured-us-scientists-what-did-it-get-in-return.html}{Thousand
Talents program}, which has provided
\href{https://www.nytimes3xbfgragh.onion/2020/02/06/us/chinas-lavish-funds-lured-us-scientists-what-did-it-get-in-return.html}{funding
for American researchers} --- including Charles M. Lieber, the chairman
of Harvard University's chemistry and chemical biology department, who
was
\href{https://www.nytimes3xbfgragh.onion/2020/01/28/us/charles-lieber-harvard.html}{arrested}
by the F.B.I. in January on charges of concealing his financial
relationship with the Chinese government.

Asked about the Trump administration's move to cancel the visas of some
Chinese students studying in the United States, Mr. Rubio said he
supported ``a targeted approach'' to make it more difficult for the
Chinese Communist Party to exploit the openness of American schools to
advance their own military and intelligence abilities.

``The Chinese government too often entraps its own people into service''
to the Communist Party and its objectives ``in exchange for an education
in the U.S.,'' Mr. Rubio said, adding that ``higher education
institutions in America need to be fully aware of this
counterintelligence threat.''

Other Republican lawmakers
\href{https://www.cotton.senate.gov/?p=press_release\&id=1371}{proposed
legislation} on Wednesday to bar any Chinese citizen from getting a visa
for graduate or postgraduate study in science or technology.

Trump administration officials have discussed restricting Chinese
student visas over the past three years, current and former officials
said.

In 2018, the State Department
\href{https://www.nytimes3xbfgragh.onion/2018/07/25/us/politics/visa-restrictions-chinese-students.html}{began
limiting} the length of visas to one year, with an option for renewal,
for Chinese graduate students working in fields deemed sensitive. Two
officials said targeting graduates of the military-linked schools
gathered steam after the
\href{https://www.fbi.gov/wanted/counterintelligence/yanqing-ye}{F.B.I.
announced in January} that it was seeking a Boston University student
who had hidden her affiliation with the People's Liberation Army when
applying for a visa.

F.B.I. officials said the student, Yanqing Ye, had studied at the
National University of Defense Technology in China and was commissioned
as a lieutenant before enrolling in Boston University's department of
physics, chemistry and biomedical engineering from October 2017 to April
2019.

While in Boston, Lieutenant Ye continued to get assignments from the
Chinese military, including ``conducting research, assessing United
States military websites and sending United States documents and
information to China,'' according to the F.B.I. wanted poster.

The
\href{https://www.justice.gov/opa/pr/harvard-university-professor-and-two-chinese-nationals-charged-three-separate-china-related}{Justice
Department charged} Lieutenant Ye, who is believed to be in China, with
acting as a foreign agent, visa fraud and false statements.

The vigorous interagency debate over the move to cancel visas has lasted
about six months, with science and technology officials generally
opposing the action and national security officials supporting it.

The Australian Strategic Policy Institute, a think tank, has researched
the Chinese military-affiliated universities, and that has influenced
thinking in the American government. A 2018 report called
\href{https://www.aspi.org.au/report/picking-flowers-making-honey}{``Picking
Flowers, Making Honey''} said China was sending students from those
universities to Western universities to try to build up its own military
technology.

The study suggested that the graduates were targeting the so-called Five
Eyes countries that share intelligence: the United States, Canada,
Britain, New Zealand and Australia. In many cases, the report said,
students hid their military affiliations while seeking work in fields
with defense applications, like hypersonics.

Under the current Chinese government, Beijing has aggressively tried to
combine military and civilian work on important technology, said
American officials and outside researchers. That often includes tapping
the expertise of civilian companies and universities.

``To some degree, U.S. concerns are driven by the assessment that
Chinese companies and universities seem unlikely to refuse outright or
could be compelled to work with the military, whereas their American
counterparts often appear more resistant to working on military
research,'' Elsa B. Kania, an adjunct senior fellow at the Center for a
New American Security,
\href{https://www.cnas.org/publications/commentary/in-military-civil-fusion-china-is-learning-lessons-from-the-united-states-and-starting-to-innovate}{wrote
in a report} last August.

``It is also striking at the same time that some of China's leading
technology companies appear to be less directly engaged in supporting
defense initiatives than might be expected relative to their American
counterparts,'' she added.

United States officials said the fusion policy also entailed sending
military-trained students to American universities to try to gain access
to technological know-how that would be valuable to China and its
defense industry.

The Chinese military has strong ties to a number of schools with an
overt military bent, according to the Australian think tank.

Less obvious to the casual observer are the more traditional
universities with longstanding ties to the military.

According to
\href{https://www.aspi.org.au/report/china-defence-universities-tracker}{the
policy institute} and American officials, those are Northwestern
Polytechnical University, Harbin Engineering University, Beijing
Institute of Technology, Harbin Institute of Technology, Beihang
University, Nanjing University of Aeronautics and Astronautics, and
Nanjing University of Science and Technology.

Keith Bradsher contributed reporting from Beijing.

Advertisement

\protect\hyperlink{after-bottom}{Continue reading the main story}

\hypertarget{site-index}{%
\subsection{Site Index}\label{site-index}}

\hypertarget{site-information-navigation}{%
\subsection{Site Information
Navigation}\label{site-information-navigation}}

\begin{itemize}
\tightlist
\item
  \href{https://help.nytimes3xbfgragh.onion/hc/en-us/articles/115014792127-Copyright-notice}{©~2020~The
  New York Times Company}
\end{itemize}

\begin{itemize}
\tightlist
\item
  \href{https://www.nytco.com/}{NYTCo}
\item
  \href{https://help.nytimes3xbfgragh.onion/hc/en-us/articles/115015385887-Contact-Us}{Contact
  Us}
\item
  \href{https://www.nytco.com/careers/}{Work with us}
\item
  \href{https://nytmediakit.com/}{Advertise}
\item
  \href{http://www.tbrandstudio.com/}{T Brand Studio}
\item
  \href{https://www.nytimes3xbfgragh.onion/privacy/cookie-policy\#how-do-i-manage-trackers}{Your
  Ad Choices}
\item
  \href{https://www.nytimes3xbfgragh.onion/privacy}{Privacy}
\item
  \href{https://help.nytimes3xbfgragh.onion/hc/en-us/articles/115014893428-Terms-of-service}{Terms
  of Service}
\item
  \href{https://help.nytimes3xbfgragh.onion/hc/en-us/articles/115014893968-Terms-of-sale}{Terms
  of Sale}
\item
  \href{https://spiderbites.nytimes3xbfgragh.onion}{Site Map}
\item
  \href{https://help.nytimes3xbfgragh.onion/hc/en-us}{Help}
\item
  \href{https://www.nytimes3xbfgragh.onion/subscription?campaignId=37WXW}{Subscriptions}
\end{itemize}
