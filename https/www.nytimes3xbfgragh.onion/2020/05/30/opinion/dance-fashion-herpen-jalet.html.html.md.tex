Sections

SEARCH

\protect\hyperlink{site-content}{Skip to
content}\protect\hyperlink{site-index}{Skip to site index}

\href{https://myaccount.nytimes3xbfgragh.onion/auth/login?response_type=cookie\&client_id=vi}{}

\href{https://www.nytimes3xbfgragh.onion/section/todayspaper}{Today's
Paper}

\href{/section/opinion}{Opinion}\textbar{}Where Dance and Fashion
Collide

\url{https://nyti.ms/2BhucTG}

\begin{itemize}
\item
\item
\item
\item
\item
\end{itemize}

Advertisement

\protect\hyperlink{after-top}{Continue reading the main story}

\href{/section/opinion}{Opinion}

Supported by

\protect\hyperlink{after-sponsor}{Continue reading the main story}

THE big ideas: why does art matter?

\hypertarget{where-dance-and-fashion-collide}{%
\section{Where Dance and Fashion
Collide}\label{where-dance-and-fashion-collide}}

A conversation between the designer Iris van Herpen and the
choreographer Damien Jalet on the creative process.

By Iris van Herpen and Damien Jalet

Ms. van Herpen is a designer. Mr. Jalet is a choreographer.

\begin{itemize}
\item
  May 30, 2020
\item
  \begin{itemize}
  \item
  \item
  \item
  \item
  \item
  \end{itemize}
\end{itemize}

\includegraphics{https://static01.graylady3jvrrxbe.onion/images/2020/05/30/multimedia/30bigideas-jalet/30bigideas-jalet-articleLarge.jpg?quality=75\&auto=webp\&disable=upscale}

\emph{This conversation is part of}
\href{https://www.nytimes3xbfgragh.onion/spotlight/the-big-ideas}{\emph{The
Big Ideas}}\emph{, a special section of The Times's philosophy series,}
\href{https://www.nytimes3xbfgragh.onion/column/the-stone?action=click\&module=RelatedLinks\&pgtype=Article}{\emph{The
Stone}}\emph{, in which more than a dozen artists, writers and thinkers
answer the question, ``Why does art matter?'' The entire series can be
found}
\href{https://www.nytimes3xbfgragh.onion/spotlight/the-big-ideas}{\emph{here}}\emph{.}

\begin{center}\rule{0.5\linewidth}{\linethickness}\end{center}

\emph{Iris van Herpen and Damien Jalet are known for pushing boundaries
in their respective fields. Ms. van Herpen, a Dutch fashion designer,
fuses traditional craftsmanship with modern technology to create
sculptural forms. She has dressed Beyoncé, Björk and other performers,
and has designed costumes for the Paris Opera and the New York City
Ballet. Mr. Jalet, a Belgian-French choreographer, often experiments
with perspectives and blends visual art into his presentations. He has
collaborated with Thom Yorke, Madonna, Paul Thomas Anderson and many
others.}

\emph{In early May, we asked Ms. van Herpen and Mr. Jalet to discuss the
creative process, the importance of originality and why art matters in
our modern times. This is an edited transcript of their conversation.
--- The Editors}

\textbf{Damien Jalet:} I look at your work --- we've worked
together\href{https://www.operaballet.be/nl/programma/2017-2018/pelleas-et-melisande}{once},
and I hope we will again --- and I am struck by how much you think like
a dancer, like a choreographer. You indeed started as a dancer at a
young age. How did that shape you as a designer --- your view of the
body, how you relate to anatomy? Did it give you a perspective to start
with?

\textbf{Iris van Herpen:} Yes, I come from dance; I wanted to become a
dancer, and I think those years are still shaping my fascination for
movement in my work today. Like you, I think of the body as a sculpture.
A lot of designers have a woman in mind, with a certain look and
identity, but I don't really think that way. I have the human anatomy as
my muse, so it's a bit more abstract. The way I start a collection, I
really see the body as a blank canvas. And when I danced I learned a lot
about the transformative strength of my own body, and also about the
symbiotic power between our mind and body. It's the transformative power
of dance that really attracted me, as we all are creating the concept of
being us.

In my work, in every piece I make, I'm looking for the movement and the
aliveness that dance can express. Dance inspired me to look at fashion
from a transformable perspective --- my design process is sort of
translating a piece of dance, a three-dimensional choreography of
micromovement, into a garment.

\includegraphics{https://static01.graylady3jvrrxbe.onion/images/2020/05/30/opinion/30bigideas-jalet06/30bigideas-jalet06-articleLarge.jpg?quality=75\&auto=webp\&disable=upscale}

\textbf{D.J.:} Do you use your own body as you create a dress to
understand how the flow of the dress should work?

\textbf{I.V.H.:} Absolutely, the process is always through draping; I
don't really make drawings. I drape on a mannequin, and it's really my
hands and the material that are doing a little dance. When I start, I
really don't know the outcome, and that openness is really important
because I really don't want the work to become too organized or too
controlled. I need a moment of chaos, and the moment of chaos is really
the draping moment.

After draping, I put the material on and I start testing it myself, and
I start interacting with it and seeing the way it moves. And then often
that influences my next motion in the draping process.

Fashion and art are never only a reflection of who we are --- they are
about seeking who we want to be, and who we want to become. My role in
this search is about making fashion more collaborative, shaping fashion
more intelligently, and about empowering women and fashion in the fields
of science, art, architecture, biology and engineering.

\textbf{D.J.:} You seem to always have this high level of pressure on
you every year, or twice a year, to present a show. You have all these
people working with you. There are also all the technical elements you
use, and you don't know how those will turn out at the beginning. When
you start a collection, it seems you are just guided by intuition, which
is a very hard thing to grasp. I guess it's a bit like walking in the
dark and just guiding people, guiding everyone who is working with you
toward one goal that you can't clearly see yet, but you somehow feel.

\textbf{I.V.H.:} But that's similar to you and your dancers, right? It's
a conscious and an unconscious parallel system, I would say. You seem to
play with the borders of the conscious and unconscious, at least that's
what I feel when looking at your work.

It makes me think about my collection about
\href{https://www.irisvanherpen.com/haute-couture/lucid}{lucid
dreaming}. At a moment in my life when I had a lot of lucid dreams I
started realizing how much came from those --- from that unconscious
place. And I started to use that blurry border in my mind to design
while I was dreaming. I still think back on that collection a lot
because it brought me closer to the question of why I create.

Which makes me want to ask you: Why do you create? Is it for yourself or
for others? What do you look for?

\textbf{D.J.:} The whole parallel between dreaming and creating is
really interesting. When I'm creating, I see very little, because it's
really about trying to understand what my deeper intuition is telling
me. That state when you start falling asleep, or when you are slowly
waking up, is a very interesting moment because you are between two
worlds in a way, half-unconscious or half-conscious. Like daydreaming,
the lightest form of an altered state of consciousness.

I think if you push this further, really to the extreme, you go into a
trance, to a place where you are completely unconscious yet awake, which
is actually a technique that humans have been using for very long time.
I believe we started dancing to get to this place.

I got this notion kind of confirmed while observing rituals where trance
is very present, in places like Indonesia. For me these unconscious
places are basically where all mythology originates. What's unconscious
is somehow the same as what's invisible, what can't be grasped, because
the moment you become conscious, it's gone. But sometimes there is this
little bridge --- and it comes to us through using intuition. When I
have dancers work around certain ideas, using improvisation and
restrictions, somehow things happen that even they had never thought of.
That's always the most beautiful thing, when you just are exploring
something and expand your consciousness through that experience.

Image

A scene from Opera Vlaanderen's production of ``Pelléas et Mélisande''
in 2018 in Antwerp, Belgium. The opera was co-directed by Damien Jalet
and Sidi Larbi Cherkaoui. Iris van Herpen designed the
costumes.Credit...Annemie Augustijns

I think it's through sharing your work that the work is born. I think of
birth as always very intimate, but also violent --- there's this mixture
of power and vulnerability. When you create your art, you explore the
little corners of yourself. You put some of your most intimate thoughts
in it. And then when you share it, when you release that, it can be both
exhilarating and traumatic, because suddenly you have to face how the
world reacts to it. You can be naturally creative, but the experience of
premiering something is quite challenging.

\textbf{I.V.H.:} So true. When I started, I never thought it would be
this personal. I think that's something I discovered along the way, that
there is no way you can create when you don't go down deeply there.

That also raises the question whether we create our works or the other
way around, and I actually dare say it's sort of equal with me. I really
think my work and my processes have created me as much as I have created
them.

Throughout the years, looking at my work, I really think I've become a
different person because of my work, because when you have to create
constantly, the process pushes you constantly as well, to improve and to
search beyond yourself for inspiration. I really see that process as a
constant ebb and flow of taking and giving. As much as I create, I think
the world is re-creating me. That feedback loop is challenging, but also
very electric and addictive at the same time.

\textbf{D.J.:} I also think that as an artist sometimes you have to have
a disassociation between you and the art. But I've found it nearly
impossible to do so because we are bonded to it.

For example, I could speak very clearly about one time when I used my
art to deal with a very traumatic event --- being three meters away from
a gunman during the attacks in Paris, and managing to escape while he
was shooting. I remember being so shaken, so obsessed, that I even
considered no longer creating. That was my first instinct. But I had
this work scheduled, so I said, ``Why don't I confront this in this
creation and find a way to articulate with my medium something that I
cannot articulate with words and explore that?''

Image

Iris van HerpenCredit...Luigi \& Iango

Image

Damien JaletCredit...Mats Bäcker

So I decided to focus on certain images and certain principles --- one
of them being, if you don't move, you die. And there was this idea of a
tunnel turning on itself, threatening the dancers in it, rolling forward
and backward, and the people were just stuck in it, and getting expelled
from it. I realized when I finished the work that this tunnel was a kind
of passage, one perhaps for me to pass through. I understood what this
piece meant. I had a very strong emotional reaction to it because I
really had let my unconscious push me to activate certain ideas.

It was something that really helped me to let go of the trauma --- of
almost dying, of seeing people die. That's the cathartic power of what
we do, sometimes we are able to transform something that is pitch dark
into something else through intuition.

When you start being creative it's, as you say, very addictive. Moments
of realization are what you are hunting for, moments of discovery that
can transform your perception as a creator, and also as a spectator.

\textbf{I.V.H.:} Precisely. It's that important and infinite hunt for
all the layers of life around us and all the invisible forces that shape
our world --- from the microscopic to the macroscopic. Our senses
usually just tune into that first layer around us, even though there are
so many other realities happening at the same time.

\textbf{D.J.:} I was wondering if there was a work of art that shaped
your perspective on life.

\textbf{I.V.H.:} There are so many works of art that have shaped me.
It's hard to really pick one, but if I have to I would say the sculpture
``\href{https://borghese.gallery/collection/sculpture/apollo-and-daphne.html}{Apollo
and Daphne}'' by Bernini. I remember seeing it for the first time, that
transformation of a woman that grows back into nature, that sort of
radiates the circle of life so beautifully. It shows that nature is
taking and giving back life to earth constantly, captured in a moment,
that moment of metamorphosis. And also the craftsmanship is beyond
belief. That has really shaped the way I see us relating to the planet
--- we will always remain part of it, no matter how far we manage to
think ourselves away from it.

\textbf{D.J.:} I can see that. It seems that through your metamorphic
designs, you want to express how much we share with nonhuman forms of
life. Is this a way to shift away from a mainly anthropocentric point of
view? Like, the first works of art found in caves were about depicting
animals, and man was represented in a humble and somehow clumsy,
unfinished way. This humility has vanished through history, yet today
there's a focus in art on deconstructing the idea of humans being above
nature, or disconnected from it, which for sure is a motive that
animates my own work. Do you feel somehow connected to that approach as
a creator?

\textbf{I.V.H.:} Very much. I look at the forces behind the forms in
nature through biomimicry; these endless mysteries within nature create
a huge influence on my work. A lot of the three-dimensional patterns I
create echo the rhythms of life, and the geometric patterns unveil the
mathematical logic hidden inside nature.

Image

``Fashion and art are never only a reflection of who we are,'' says the
Dutch fashion designer Iris van Herpen, ``they are about seeking who we
want to be, and who we want to become.''Credit...Valerio Mezzanotti for
The New York Times

Coming back to your previous question a bit, if I were to name a
different artwork that influenced me, I think it would be the Folkloric
Dance Company, which doesn't exist anymore, unfortunately. As a kid I
didn't travel the world; holidays were spent in my own country. But my
family did go see the Folkloric Dance Company a lot in Amsterdam. This
group of dancers, they would go and study a certain culture or community
somewhere far away. They would study and learn their rituals, and their
dancing, and their instruments, and their music and costumes. And they
would put everything together in a performance when they returned. When
I was little, that was really my way of traveling the world. It was
through dance, and through music, and the clothes. It left such a huge
impression on me, to know that these cultures or communities were out
there, and to be able to come really close to them in a way. It made me
realize at the time how big the planet was, and it took me out of my own
bubble of the Netherlands.

\textbf{D.J.:} The art itself was the escape?

\textbf{I.V.H.:} It was freedom, I guess. I think art can really
sometimes give you a feeling of freedom, that you can be anywhere at any
place.

\textbf{D.J.:} There was this one question that I wanted to ask that I
felt was important.
\href{https://variety.com/2020/film/reviews/kubrick-by-kubrick-review-1234601551/}{Stanley
Kubrick said} that some of the artistic failures of the 20th century
came from an obsession with total originality, and that innovation
didn't happen through abandoning the classical art form of your own
discipline.

It seems to me that you might agree with that, because there's a sense
of craftsmanship that you bring to fashion design. What you do is haute
couture, but it comes from a certain tradition of tailoring. Does the
question ``Is what I'm doing original?'' ever come up in your creative
process.

\textbf{I.V.H.:} Yeah, that's a nice question. I very much agree with
what you said. It's really the history of couture that is guiding me in
my work. Nothing comes out of nothing, so the craftsmanship that we
master we can attribute to a long evolution of craftsmanship and
innovation combined throughout so many centuries. And we are looking at
that constantly. So in that sense, I don't believe at all in
originality. But at the same time we are combining it with technologies
of today and newer techniques --- like 3D printing and injection molding
and laser cutting. Without the knowledge of the traditional
craftsmanship, we would not be able to integrate these new techniques at
all. So they really need each other.

I don't think I could create anything beautiful or balanced if I did not
know both the traditional techniques, as well as some of the newer tools
that we are discovering today. To me it really feels important to
continue this evolution of craftsmanship.

But I don't believe pure originality exists. I think we're all
programmed and shaped by our upbringings, and by our culture, and all
the art that we've seen in our lives. So while in my process I'm
definitely looking for a uniqueness in my form, and the femininity, and
in the identity I create, I can only shape that from what has been done
before me.

\emph{\textbf{Now in print}}*:
``\emph{\href{http://bitly.com/1MW2kN3}{\emph{Modern Ethics in 77
Arguments}}},'' and ``\emph{\href{http://bitly.com/1MW2kN3}{\emph{The
Stone Reader: Modern Philosophy in 133 Arguments}}},'' with essays from
the series, edited by Peter Catapano and Simon Critchley, published by
Liveright Books.*

\emph{The Times is committed to publishing}
\href{https://www.nytimes3xbfgragh.onion/2019/01/31/opinion/letters/letters-to-editor-new-york-times-women.html}{\emph{a
diversity of letters}} \emph{to the editor. We'd like to hear what you
think about this or any of our articles. Here are some}
\href{https://help.nytimes3xbfgragh.onion/hc/en-us/articles/115014925288-How-to-submit-a-letter-to-the-editor}{\emph{tips}}\emph{.
And here's our email:}
\href{mailto:letters@NYTimes.com}{\emph{letters@NYTimes.com}}\emph{.}

\emph{Follow The New York Times Opinion section on}
\href{https://www.facebookcorewwwi.onion/nytopinion}{\emph{Facebook}}\emph{,}
\href{http://twitter.com/NYTOpinion}{\emph{Twitter (@NYTopinion)}}
\emph{and}
\href{https://www.instagram.com/nytopinion/}{\emph{Instagram}}\emph{.}

Advertisement

\protect\hyperlink{after-bottom}{Continue reading the main story}

\hypertarget{site-index}{%
\subsection{Site Index}\label{site-index}}

\hypertarget{site-information-navigation}{%
\subsection{Site Information
Navigation}\label{site-information-navigation}}

\begin{itemize}
\tightlist
\item
  \href{https://help.nytimes3xbfgragh.onion/hc/en-us/articles/115014792127-Copyright-notice}{©~2020~The
  New York Times Company}
\end{itemize}

\begin{itemize}
\tightlist
\item
  \href{https://www.nytco.com/}{NYTCo}
\item
  \href{https://help.nytimes3xbfgragh.onion/hc/en-us/articles/115015385887-Contact-Us}{Contact
  Us}
\item
  \href{https://www.nytco.com/careers/}{Work with us}
\item
  \href{https://nytmediakit.com/}{Advertise}
\item
  \href{http://www.tbrandstudio.com/}{T Brand Studio}
\item
  \href{https://www.nytimes3xbfgragh.onion/privacy/cookie-policy\#how-do-i-manage-trackers}{Your
  Ad Choices}
\item
  \href{https://www.nytimes3xbfgragh.onion/privacy}{Privacy}
\item
  \href{https://help.nytimes3xbfgragh.onion/hc/en-us/articles/115014893428-Terms-of-service}{Terms
  of Service}
\item
  \href{https://help.nytimes3xbfgragh.onion/hc/en-us/articles/115014893968-Terms-of-sale}{Terms
  of Sale}
\item
  \href{https://spiderbites.nytimes3xbfgragh.onion}{Site Map}
\item
  \href{https://help.nytimes3xbfgragh.onion/hc/en-us}{Help}
\item
  \href{https://www.nytimes3xbfgragh.onion/subscription?campaignId=37WXW}{Subscriptions}
\end{itemize}
