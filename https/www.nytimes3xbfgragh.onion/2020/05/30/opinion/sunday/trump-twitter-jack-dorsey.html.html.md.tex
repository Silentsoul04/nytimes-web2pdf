Sections

SEARCH

\protect\hyperlink{site-content}{Skip to
content}\protect\hyperlink{site-index}{Skip to site index}

\href{https://www.nytimes3xbfgragh.onion/section/opinion/sunday}{Sunday
Review}

\href{https://myaccount.nytimes3xbfgragh.onion/auth/login?response_type=cookie\&client_id=vi}{}

\href{https://www.nytimes3xbfgragh.onion/section/todayspaper}{Today's
Paper}

\href{/section/opinion/sunday}{Sunday Review}\textbar{}Think Outside the
Box, Jack

\url{https://nyti.ms/3gCEQV4}

\begin{itemize}
\item
\item
\item
\item
\item
\item
\end{itemize}

Advertisement

\protect\hyperlink{after-top}{Continue reading the main story}

\href{/section/opinion}{Opinion}

Supported by

\protect\hyperlink{after-sponsor}{Continue reading the main story}

\hypertarget{think-outside-the-box-jack}{%
\section{Think Outside the Box, Jack}\label{think-outside-the-box-jack}}

Trump, Twitter and the society-crushing pursuit of monetized rage.

\href{https://www.nytimes3xbfgragh.onion/by/maureen-dowd}{\includegraphics{https://static01.graylady3jvrrxbe.onion/images/2018/04/02/opinion/maureen-dowd/maureen-dowd-thumbLarge.png}}

By \href{https://www.nytimes3xbfgragh.onion/by/maureen-dowd}{Maureen
Dowd}

Opinion Columnist

\begin{itemize}
\item
  May 30, 2020
\item
  \begin{itemize}
  \item
  \item
  \item
  \item
  \item
  \item
  \end{itemize}
\end{itemize}

\includegraphics{https://static01.graylady3jvrrxbe.onion/images/2020/05/31/opinion/sunday/31Dowd/31Dowd-articleLarge.jpg?quality=75\&auto=webp\&disable=upscale}

WASHINGTON --- C'mon, @Jack. You can do it.

\href{https://twitter.com/jack/status/630260438647898112?lang=en}{Throw
on some Kendrick Lamar}and get your head in the right space. Pour
yourself a big old glass of
\href{https://www.businessinsider.com/why-jack-dorsey-drinks-salt-juice-every-morning-2019-5}{salt
juice}. Draw an ice bath and fire up the cryotherapy pod and
\href{https://www.nytimes3xbfgragh.onion/2019/05/02/fashion/jack-dorsey-influencer.html}{the
infrared sauna}. Then just pull the plug on him. You know you want to.

You could answer the existential question of whether
\href{https://www.nytimes3xbfgragh.onion/2020/06/23/technology/trump-twitter-label-seattle.html}{@realDonaldTrump}
even exists if he doesn't exist on
\href{https://www.nytimes3xbfgragh.onion/2020/06/23/technology/trump-twitter-label-seattle.html}{Twitter}.
\emph{I tweet, therefore I am.} Dorsey meets Descartes.

All it would take is one sweet click to force the greatest troll in the
history of the internet to meet his maker. Maybe he just disappears in
an orange cloud of smoke, screaming, ``I'm melllllllting.''

Do Trump --- and the world --- a favor and send him back into the void
whence he came. And then go have
\href{https://twitter.com/jack/status/1071575088695140353?s=20}{some
fun}: Meditate and fast for days on end!

Our country is going through biological, economic and societal
convulsions. We can't trust the powerful forces in this nation to tell
us the truth or do the right thing. In fact, not only can we not trust
them. We have every reason to believe they're gunning for us.

In Washington, the Trump administration's deception about the virus was
lethal. On Wall Street and in Silicon Valley, the fat cats who carved up
the country, drained us dry and left us with no safety net
\href{https://www.cnn.com/2020/04/28/perspectives/inequality-coronavirus-billionaires/index.html}{profiteered}
off the virus. In Minneapolis, the barbaric death of George Floyd after
a police officer knelt on him for almost nine minutes showed yet again
that black Americans have everything to fear from some who are charged
with protecting them.

As if that weren't enough, from the slough of our despond, we have to
watch Donald Trump duke it out with the lords of the cloud in a contest
to see who can destroy our democracy faster.

I wish I could go along with those who say this dark period of American
life will ultimately make us nicer and simpler and more contemplative.
How can that happen when the whole culture has been re-engineered to put
us at each other's throats?

Trump constantly torques up the tribal friction and cruelty, even as
Twitter and Facebook refine their systems to ratchet up rage. It is
amazing that a septuagenarian became the greatest exploiter of social
media. Trump and Twitter were a match made in hell.

The Wall Street Journal had a
\href{https://www.wsj.com/articles/facebook-knows-it-encourages-division-top-executives-nixed-solutions-11590507499}{chilling
report} a few days ago that Facebook's own research in 2018 revealed
that ``our algorithms exploit the human brain's attraction to
divisiveness. If left unchecked,'' Facebook would feed users ``more and
more divisive content in an effort to gain user attention \& increase
time on the platform.''

Mark Zuckerberg shelved the research.

Why not just let all the bots trying to undermine our elections and
spreading false information about the coronavirus and right-wing
conspiracy theories and smear campaigns run amok? Sure, we're weakening
our society, but the weird, infantile maniacs running Silicon Valley
must be allowed to rake in more billions and finish their mission of
creating a giant cyberorganism of people, one huge and lucrative ball of
rage.

``The shareholders of Facebook decided, `If you can increase my stock
tenfold, we can put up with a lot of rage and hate,''' says Scott
Galloway, professor of marketing at New York University's Stern School
of Business.

``These platforms have very dangerous profit motives. When you monetize
rage at such an exponential rate, it's bad for the world. These guys
don't look left or right; they just look down. They're willing to
promote white nationalism if there's money in it. The rise of social
media will be seen as directly correlating to the decline of Western
civilization.''

Dorsey, who has more leeway because his stock isn't as valuable as
Facebook's, made some mild moves against the president who has been
spewing lies and inciting violence on Twitter for years. He added
footnotes clarifying false Trump tweets about mail-in ballots and put a
warning label on the president's tweet about the Minneapolis riots that
\href{https://www.nytimes3xbfgragh.onion/aponline/2020/05/29/us/ap-us-trump-minneapolis-death-quote-.html}{echo
the language} of a Miami police chief in 1967 and segregationist George
Wallace: ``When the looting starts, the shooting starts.''

``Jack is really sincerely trying to find something to make it better,''
said one friend of the Twitter chief's. ``He's like somebody trapped in
a maze, going down every hallway and turning every corner.''

Zuckerberg, on the other hand, went on Fox to report that he was happy
to continue enabling the Emperor of Chaos, noting that he did not think
Facebook should be ``the arbiter of truth of everything that people say
online.''

It was a sickening display that made even some loyal Facebook staffers
queasy. As
\href{https://www.theverge.com/2020/5/29/21275044/facebook-trump-tweets-employee-reaction-criticism}{The
Verge's Casey Newton reported}, some employees objected to the company's
rationale in internal posts.

``I have to say I am finding the contortions we have to go through
incredibly hard to stomach,'' one wrote. ``All this points to a very
high risk of a violent escalation and civil unrest in November and if we
fail the test case here, history will not judge us kindly.''

Trump, furious that Dorsey would attempt to rein him in on the very
platform that catapulted him into the White House, immediately decided
to try to rein in Dorsey.

He
\href{https://www.nytimes3xbfgragh.onion/2020/05/28/us/politics/trump-order-social-media.html}{signed
an executive order} that might strip liability protection from social
media sites, which would mean they would have to more assiduously police
false and defamatory posts. Now that social media sites are behemoths,
Galloway thinks that the removal of the Communications Decency Act makes
a lot of sense even if the president is trying to do it for the wrong
reasons.

Trump does not seem to realize, however, that he's removing his own
protection. He huffs and puffs about freedom of speech when he really
wants the freedom to be vile. ``It's the mother of all
cutting-off-your-nose-to-spite-your-face moves,'' says Galloway.

The president wants to say things on Twitter that he will not be allowed
to say if he exerts this control over Twitter. In a sense, it's Trump
versus his own brain. If Twitter can be sued for what people say on it,
how can Trump continue to torment? Wouldn't thousands of his own tweets
have to be deleted?

``He'd be the equivalent of a slippery floor at a store that sells
equipment for hip replacements,'' says Galloway, who also posits that,
in our hyper-politicized world, this will turn Twitter into a Democratic
site and Facebook into a Republican one.

Nancy Pelosi, whose district encompasses Twitter, said that it did
little good for Dorsey to put up a few fact-checks while letting Trump's
rants about murder and other ``misrepresentations'' stay up.

``Facebook, all of them, they are all about making money,'' the speaker
said. ``Their business model is to make money at the expense of the
truth and the facts.'' She crisply concluded that ``all they want is to
not pay taxes; they got their tax break in 2017'' and ``they don't want
to be regulated, so they pander to the White House.''

C'mon, Jack. Make @realDonaldTrump melt to help end our meltdown.

\emph{The Times is committed to publishing}
\href{https://www.nytimes3xbfgragh.onion/2019/01/31/opinion/letters/letters-to-editor-new-york-times-women.html}{\emph{a
diversity of letters}} \emph{to the editor. We'd like to hear what you
think about this or any of our articles. Here are some}
\href{https://help.nytimes3xbfgragh.onion/hc/en-us/articles/115014925288-How-to-submit-a-letter-to-the-editor}{\emph{tips}}\emph{.
And here's our email:}
\href{mailto:letters@NYTimes.com}{\emph{letters@NYTimes.com}}\emph{.}

\emph{Follow The New York Times Opinion section on}
\href{https://www.facebookcorewwwi.onion/nytopinion}{\emph{Facebook}}\emph{,}
\href{http://twitter.com/NYTOpinion}{\emph{Twitter (@NYTopinion)}}
\emph{and}
\href{https://www.instagram.com/nytopinion/}{\emph{Instagram}}\emph{.}

Advertisement

\protect\hyperlink{after-bottom}{Continue reading the main story}

\hypertarget{site-index}{%
\subsection{Site Index}\label{site-index}}

\hypertarget{site-information-navigation}{%
\subsection{Site Information
Navigation}\label{site-information-navigation}}

\begin{itemize}
\tightlist
\item
  \href{https://help.nytimes3xbfgragh.onion/hc/en-us/articles/115014792127-Copyright-notice}{©~2020~The
  New York Times Company}
\end{itemize}

\begin{itemize}
\tightlist
\item
  \href{https://www.nytco.com/}{NYTCo}
\item
  \href{https://help.nytimes3xbfgragh.onion/hc/en-us/articles/115015385887-Contact-Us}{Contact
  Us}
\item
  \href{https://www.nytco.com/careers/}{Work with us}
\item
  \href{https://nytmediakit.com/}{Advertise}
\item
  \href{http://www.tbrandstudio.com/}{T Brand Studio}
\item
  \href{https://www.nytimes3xbfgragh.onion/privacy/cookie-policy\#how-do-i-manage-trackers}{Your
  Ad Choices}
\item
  \href{https://www.nytimes3xbfgragh.onion/privacy}{Privacy}
\item
  \href{https://help.nytimes3xbfgragh.onion/hc/en-us/articles/115014893428-Terms-of-service}{Terms
  of Service}
\item
  \href{https://help.nytimes3xbfgragh.onion/hc/en-us/articles/115014893968-Terms-of-sale}{Terms
  of Sale}
\item
  \href{https://spiderbites.nytimes3xbfgragh.onion}{Site Map}
\item
  \href{https://help.nytimes3xbfgragh.onion/hc/en-us}{Help}
\item
  \href{https://www.nytimes3xbfgragh.onion/subscription?campaignId=37WXW}{Subscriptions}
\end{itemize}
