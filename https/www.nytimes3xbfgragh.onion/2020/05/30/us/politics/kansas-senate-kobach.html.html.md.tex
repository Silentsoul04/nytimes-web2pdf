Sections

SEARCH

\protect\hyperlink{site-content}{Skip to
content}\protect\hyperlink{site-index}{Skip to site index}

\href{https://www.nytimes3xbfgragh.onion/section/politics}{Politics}

\href{https://myaccount.nytimes3xbfgragh.onion/auth/login?response_type=cookie\&client_id=vi}{}

\href{https://www.nytimes3xbfgragh.onion/section/todayspaper}{Today's
Paper}

\href{/section/politics}{Politics}\textbar{}Kris Kobach Is Back, and a
Kansas Senate Seat May Be Up for Grabs

\href{https://nyti.ms/3ciINLe}{https://nyti.ms/3ciINLe}

\begin{itemize}
\item
\item
\item
\item
\item
\end{itemize}

\begin{itemize}
\item
  \href{https://www.nytimes3xbfgragh.onion/2020/08/04/us/elections/primary-election-michigan-arizona-kansas.html?action=click\&pgtype=Article\&state=default\&region=TOP_BANNER\&context=storylines_menu}{Election
  Updates}
\item
  \href{https://www.nytimes3xbfgragh.onion/article/biden-vice-president-2020.html?action=click\&pgtype=Article\&state=default\&region=TOP_BANNER\&context=storylines_menu}{Biden's
  V.P. Search}
\item
  \href{https://www.nytimes3xbfgragh.onion/interactive/2020/07/24/us/politics/trump-biden-campaign-donors.html?action=click\&pgtype=Article\&state=default\&region=TOP_BANNER\&context=storylines_menu}{Map
  of Donations}
\item
  \href{https://www.nytimes3xbfgragh.onion/interactive/2020/us/elections/delegate-count-primary-results.html?action=click\&pgtype=Article\&state=default\&region=TOP_BANNER\&context=storylines_menu}{Delegate
  Count}
\item
  \href{https://www.nytimes3xbfgragh.onion/interactive/2019/us/politics/2020-presidential-candidates.html?action=click\&pgtype=Article\&state=default\&region=TOP_BANNER\&context=storylines_menu}{The
  Candidates}
\item
  \href{https://www.nytimes3xbfgragh.onion/newsletters/politics?action=click\&pgtype=Article\&state=default\&region=TOP_BANNER\&context=storylines_menu}{Politics
  Newsletter}
\end{itemize}

Advertisement

\protect\hyperlink{after-top}{Continue reading the main story}

Supported by

\protect\hyperlink{after-sponsor}{Continue reading the main story}

\hypertarget{kris-kobach-is-back-and-a-kansas-senate-seat-may-be-up-for-grabs}{%
\section{Kris Kobach Is Back, and a Kansas Senate Seat May Be Up for
Grabs}\label{kris-kobach-is-back-and-a-kansas-senate-seat-may-be-up-for-grabs}}

Mr. Kobach's polarizing presence in a race in a reliably Republican
stronghold has party leaders worried that Kansas will become a
high-stakes battleground for control of the Senate.

\includegraphics{https://static01.graylady3jvrrxbe.onion/images/2020/05/31/us/politics/31kansas-senate-print-top/merlin_146462643_55871dff-3987-4a1f-a337-a12146cbeb5b-articleLarge.jpg?quality=75\&auto=webp\&disable=upscale}

\href{https://www.nytimes3xbfgragh.onion/by/katie-glueck}{\includegraphics{https://static01.graylady3jvrrxbe.onion/images/2020/01/29/reader-center/author-katie-glueck/author-katie-glueck-thumbLarge.png}}

By \href{https://www.nytimes3xbfgragh.onion/by/katie-glueck}{Katie
Glueck}

\begin{itemize}
\item
  Published May 30, 2020Updated Aug. 4, 2020, 6:27 a.m. ET
\item
  \begin{itemize}
  \item
  \item
  \item
  \item
  \item
  \end{itemize}
\end{itemize}

LEAWOOD, Kan. --- It has been 88 years since Kansas last sent a Democrat
to the United States Senate --- one of only three
\href{http://www.senate.gov/states/KS/senators.htm}{Democrats ever} to
represent the state in the upper chamber.

But this year, in a
\href{https://www.nytimes3xbfgragh.onion/2020/04/25/us/politics/trump-election-briefings.html}{challenging
and unstable political environment} for Republicans, party leaders are
growing fearful that this reliably Republican stronghold will instead
become an expensive, high-stakes battleground that could determine the
balance of power in Washington.

Ahead of the
\href{https://www.nytimes3xbfgragh.onion/2020/08/04/us/elections/primary-election-michigan-arizona-kansas.html}{August
primary}, Democrats have largely rallied around Barbara Bollier, a
retired anesthesiologist who was
\href{https://www.kansas.com/news/politics-government/article222990385.html}{until
recently a Republican}. Republicans, meanwhile, are locked in an
intraparty competition that has all of the trappings of a full-out
brawl: attack ads, bitter recriminations between the candidates and a
party chair who tried to intervene and sparked backlash. At the center
of the fireworks is
\href{https://www.nytimes3xbfgragh.onion/2020/08/03/us/politics/kris-kobach-kansas-senate-primary.html}{Kris
W. Kobach}, a hard-line Trump supporter who has been an incendiary
presence in Kansas politics for years.

Mr. Kobach is seeking the party's nomination after
\href{https://www.nytimes3xbfgragh.onion/2018/11/06/us/laura-kelly-wins-kansas-governors-race.html}{losing
a bid for governor} two years ago, a defeat that embittered many
Republicans who felt he hand-delivered the state's top office to the
opposition and who now worry he will do it again with the Senate seat.

``If we're going to beat ourselves up in a primary with so many
candidates spending Kansas Republican donor money, coming out with
battle scars, it just makes ourselves more vulnerable,'' said Mike
Kuckelman, the chairman of the Kansas Republican Party.

Beyond the local anxiety, a race that might normally pass under the
radar nationally is instead drawing the attention of G.O.P. leaders as
they try to maintain control of the Senate, an effort that has become
increasingly challenging amid rising unemployment and faltering approval
of
\href{https://www.nytimes3xbfgragh.onion/interactive/2020/us/elections/donald-trump.html}{President
Trump}'s handling of the coronavirus crisis. **** With vulnerable
incumbents in states like Colorado, Maine and Arizona, the prospect of
having to devote time and money to retain a seat in deep-red Kansas is
an unwelcome headache that infuriates many Republicans.

The race --- to replace the
\href{https://www.nytimes3xbfgragh.onion/2019/01/04/us/politics/roberts-kansas-senate.html}{retiring
Republican Pat Roberts} --- may also test whether there are limits to a
message centered on fealty to Mr. Trump, even in a strongly conservative
state.

Ms. Bollier, a moderate Democratic state senator,
\href{http://www.kslegislature.org/li/m/pdf/district_maps/district_map_s_007.pdf}{represents}
parts of this well-manicured Kansas City suburb. It is in a
congressional district that
\href{https://www.dailykos.com/stories/2012/11/19/1163009/-Daily-Kos-Elections-presidential-results-by-congressional-district-for-the-2012-2008-elections}{voted
for} Mitt Romney for president by around 10 percentage points in 2012,
for Hillary Clinton by one point in 2016, and for Representative Sharice
Davids, a Democrat, by about
\href{https://www.politico.com/election-results/2018/kansas/}{10
percentage points} in 2018. It's the story of many districts that
delivered Democrats the House of Representatives that year.

Ms. Bollier still faces enormous difficulties in a state Mr. Trump is
expected to win handily. Her support for abortion rights is
disqualifying for many religious Kansans. Some business-oriented
Republicans dislike their party's tone on matters like immigration but
oppose a Democratic-controlled Senate. And national Democrats consider a
number of other seats to be far more competitive, and bigger priorities
for investing resources.

But instead of defining Ms. Bollier to their advantage early in the
contest, Republicans have largely had to focus on their own messy
primary.

\hypertarget{latest-updates-2020-election}{%
\section{\texorpdfstring{\href{https://www.nytimes3xbfgragh.onion/2020/08/04/us/elections/primary-election-michigan-arizona-kansas.html?action=click\&pgtype=Article\&state=default\&region=MAIN_CONTENT_1\&context=storylines_live_updates}{Latest
Updates: 2020
Election}}{Latest Updates: 2020 Election}}\label{latest-updates-2020-election}}

Updated 2020-08-04T19:32:25.304Z

\begin{itemize}
\tightlist
\item
  \href{https://www.nytimes3xbfgragh.onion/2020/08/04/us/elections/primary-election-michigan-arizona-kansas.html?action=click\&pgtype=Article\&state=default\&region=MAIN_CONTENT_1\&context=storylines_live_updates\#link-3924dd44}{Two
  G.O.P. Senate primaries offer --- what else? --- a test of loyalty to
  Trump.}
\item
  \href{https://www.nytimes3xbfgragh.onion/2020/08/04/us/elections/primary-election-michigan-arizona-kansas.html?action=click\&pgtype=Article\&state=default\&region=MAIN_CONTENT_1\&context=storylines_live_updates\#link-32b39e33}{President
  Trump is suddenly a big supporter of mail-in voting --- in Florida.}
\item
  \href{https://www.nytimes3xbfgragh.onion/2020/08/04/us/elections/primary-election-michigan-arizona-kansas.html?action=click\&pgtype=Article\&state=default\&region=MAIN_CONTENT_1\&context=storylines_live_updates\#link-6d019753}{Election
  experts warn Congress about widespread disenfranchisement of voters of
  color in November.}
\end{itemize}

\href{https://www.nytimes3xbfgragh.onion/2020/08/04/us/elections/primary-election-michigan-arizona-kansas.html?action=click\&pgtype=Article\&state=default\&region=MAIN_CONTENT_1\&context=storylines_live_updates}{See
more updates}

The biggest source of Republican anxiety is Mr. Kobach, the former
Kansas secretary of state
\href{https://www.nytimes3xbfgragh.onion/2018/11/06/us/laura-kelly-wins-kansas-governors-race.html}{who
lost the 2018 governor's race} to Laura Kelly, a Democrat, despite an
\href{https://www.nytimes3xbfgragh.onion/2018/08/06/us/politics/kris-kobach-trump-kansas.html}{endorsement
from Mr. Trump}. He is admired by supporters and reviled by detractors
for his severe views on immigration and voting rights --- a law
requiring proof of citizenship for voter registration, pressed by Mr.
Kobach, was recently
\href{https://www.nbcnews.com/politics/politics-news/appeals-court-blow-kobach-strikes-down-kansas-law-requiring-proof-n1195511}{rejected
by} a federal appeals court.

``He had no ability to raise the money, to consolidate the different
wings of his party, and to turn out any votes --- besides that, he did
great,'' said Scott W. Reed, the senior political strategist at the U.S.
Chamber of Commerce, referring to Mr. Kobach's 2018 performance. ``It
appears Kobach would be a stretch to win a statewide election and would
cause a lot of resources to be diverted.''

\includegraphics{https://static01.graylady3jvrrxbe.onion/images/2020/05/21/us/politics/00kansas-senate/00kansas-senate-articleLarge.jpg?quality=75\&auto=webp\&disable=upscale}

A number of prominent Republicans --- especially Mitch McConnell, the
Senate majority leader ---
\href{https://www.nytimes3xbfgragh.onion/2020/01/06/us/politics/mike-pompeo-senate-kansas.html}{had
hoped} that Secretary of State Mike Pompeo, a former Kansas congressman,
would enter the race, something he has resisted. Once the June 1 filing
deadline passes and the field is finalized, a number of organizations
including the U.S. Chamber of Commerce plan to assess potential
endorsements and spending in the race.

Representative Roger Marshall of the ``Big First'' District --- the
sprawling rural area that has launched the careers of statewide leaders
including former Senator Bob Dole --- is seeking to emerge as the
consensus choice, and there are increasing signs that key party leaders
and groups are inclined to coalesce around him.

Mr. Marshall is armed with an
\href{https://www.kansascity.com/news/politics-government/article239180118.html}{endorsement}
from Mr. Dole, who remains beloved by many Kansans, and support from a
growing list of significant organizations, including the Kansas Farm
Bureau. And anti-Kobach Republicans see him as a more traditional if
deeply conservative candidate who doesn't inspire the same visceral
reactions Mr. Kobach does.

``There are lots of people that have been concerned about this,'' said
former Gov. Jeff Colyer, a Republican whom Mr. Kobach
\href{https://www.nytimes3xbfgragh.onion/2018/08/14/us/politics/kansas-kobach-colyer.html}{narrowly
defeated} in the 2018 gubernatorial primary. ``I'm supporting
Congressman Marshall because he's the one who can most readily, stably
defend the seat.''

Mr. Marshall, an obstetrician-gynecologist, is emphasizing his strong
opposition to abortion rights, his work on agricultural matters and his
support of Mr. Trump.

``If we Kansans send the wrong person to the general election, it could
be competitive,'' Mr. Marshall said in an interview. ``I just don't
think Kansans will make that mistake twice.''

Asked about engaging moderate Republicans who are uncomfortable with Mr.
Trump, he replied, ``Well, goodness, that wouldn't be very many.'' But,
he said, ``you compare and contrast. Do you want the left's radical
socialist agenda, or President Trump's great economy and national
security?''

A recent poll conducted by the firm Public Opinion Strategies --- for
Mr. Marshall's campaign --- showed him with a primary lead over Mr.
Kobach, a reversal of March numbers that had shown Mr. Kobach, who is
well known in the state and boasts a base of devoted supporters, ahead.

But some national Republicans are unenthusiastic about Mr. Marshall, and
underwhelmed by
\href{https://www.kansascity.com/news/politics-government/article242025641.html}{his
fund-raising}. The political arm of Club for Growth, a conservative
outside group, is planning an ad campaign of around \$2 million that
will cast Mr. Marshall as aligned with the ``Mitt Romney wing of the
party,'' and will criticize his record on matters like big spending,
said David McIntosh, the group's president. Mr. Marshall has countered
by saying he votes with Mr. Trump.

The group has not yet endorsed Mr. Kobach, whom Mr. McIntosh called a
``strong conservative,'' saying earlier this month that the organization
was reserving judgment to see what kind of campaign Mr. Kobach would
assemble.

Several other Republicans are running. In a controversial move, Mr.
Kuckelman, the state chairman, asked all of them to drop out except for
Mr. Marshall and Mr. Kobach --- either by sending letters, as
\href{https://www.kansascity.com/news/politics-government/article242232001.html}{The
Kansas City Star reported}, or in other conversations, Mr. Kuckelman
said. It struck some observers as an effort to thwart Mr. Kobach by
thinning the field, and as a sign of Republican anxiety about the seat.

On Thursday, State Senate President Susan Wagle said she would not file
for the race, citing conversations with national party leaders and a
belief that ``a divisive primary will only benefit the campaign of
Barbara Bollier.''

Mr. Kuckelman said he was focused on limiting infighting. But he
acknowledged that he had heard concerns about Mr. Kobach, including
about his potentially negative effect on Republican candidates in other
races, though he said he had no data to prove that.

``I have absolutely seen those comments and I understand them and I will
not be without a plan,'' he said.

Image

Mr. Kobach at a rally with President Trump in Topeka in 2018. ``I think
people understand that I will be supporting the president and carrying
the ball for him in the Senate,'' he said recently.~Credit...Scott
Olson/Getty Images

In an interview, Mr. Kobach dismissed those who said his 2018
performance was predictive of his 2020 general election viability; he
pointed out that Kansas had elected several Democratic governors in
recent years. He learned new tactics from his defeat, he said, and
suggested that the Kansas electorate that shows up for Mr. Trump will
embrace Republicans down the ballot, too.

``One of the key elements of the 2020 general election will be President
Trump versus Joe Biden --- where does the Senate candidate stand on
those issues?'' he said. ``People know my long record of working with
the president, and I think people understand that I will be supporting
the president and carrying the ball for him in the Senate.''

Mr. Kobach said that he discusses immigration and ``election security
issues'' with the president and that they had spoken this month. He
declined to say whether they had discussed an endorsement (Mr. Marshall
has also been in touch with Mr. Trump this year and hopes for his
endorsement). The White House and the Trump campaign had no comment.

In contrast, Ms. Bollier sought distance from her national party during
an interview from her home in Mission Hills, Kan. She declined to say
whom she had supported in the presidential primary, or whether it would
be helpful for
\href{https://www.nytimes3xbfgragh.onion/interactive/2020/us/elections/joe-biden.html}{Joseph
R. Biden Jr.} to campaign with her in the general election, emphasizing
her interest in appearing with locals and noting the governor's
endorsement.

``Most people are in more toward the middle,'' said Ms. Bollier, who, in
some of her fund-raising appeals, is already running against Mr. Kobach.
``Most people want to see our elected officials working together.''

Kansas Republicans argue that moderate Mission Hills has little in
common, culturally or politically, with the vast majority of the state.

``Her views might work in that very small area where she was a Kansas
senator from,'' Mr. Kuckelman said. ``She cannot win a statewide
election running when she's exposed as having the liberal views she
does.''

But some are more worried, for now, about Republicans' statewide
prospects.

``It's something everybody's talking about: This has emerged as a
political hot spot,'' said Mr. Reed of the Chamber of Commerce. ``It's
one that didn't need to be a hot spot. It should be a layup for a
Republican.''

Nick Corasaniti contributed reporting.

\hypertarget{our-2020-election-guide}{%
\section{Our 2020 Election Guide}\label{our-2020-election-guide}}

Updated Aug. 4, 2020

\begin{itemize}
\item
  \begin{center}\rule{0.5\linewidth}{\linethickness}\end{center}

  \hypertarget{the-latest}{%
  \subsection{The Latest}\label{the-latest}}

  \begin{itemize}
  \tightlist
  \item
    Five states are holding primary elections Tuesday, with voters in
    Arizona, Kansas, Michigan, Missouri and Washington State choosing
    nominees for Congress and local offices.
    \href{https://www.nytimes3xbfgragh.onion/2020/08/04/us/elections/primary-election-michigan-arizona-kansas.html?action=click\&pgtype=Article\&state=default\&region=BELOW_MAIN_CONTENT\&context=storylines_guide}{Follow
    live election updates here.}
  \end{itemize}
\item
  \begin{center}\rule{0.5\linewidth}{\linethickness}\end{center}

  \hypertarget{bidens-vp-search}{%
  \subsection{Biden's V.P. Search}\label{bidens-vp-search}}

  \begin{itemize}
  \tightlist
  \item
    \href{https://www.nytimes3xbfgragh.onion/article/biden-vice-president-2020.html?action=click\&pgtype=Article\&state=default\&region=BELOW_MAIN_CONTENT\&context=storylines_guide}{Here
    are 13 women} who have been under consideration to be Joe Biden's
    running mate, and why each might be chosen --- and might not be.
  \end{itemize}
\item
  \begin{center}\rule{0.5\linewidth}{\linethickness}\end{center}

  \hypertarget{keep-up-with-our-coverage}{%
  \subsection{Keep Up With Our
  Coverage}\label{keep-up-with-our-coverage}}

  \begin{itemize}
  \tightlist
  \item
    Get an
    \href{https://www.nytimes3xbfgragh.onion/newsletters/politics?action=click\&pgtype=Article\&state=default\&region=BELOW_MAIN_CONTENT\&context=storylines_guide}{email}
    recapping the day's news
  \end{itemize}

  \begin{itemize}
  \tightlist
  \item
    Download our mobile app on
    \href{https://apps.apple.com/us/app/nytimes/id284862083?ls=1\&mat_click_id=5c79ae7455014fd1bd66b5610c05b8f2-20191112-16948\&referrer=mat_click_id\%3D5c79ae7455014fd1bd66b5610c05b8f2-20191112-16948\%26link_click_id\%3D722930677036718082}{iOS}
    and
    \href{http://a.localytics.com/android?id=com.nytimes.android\&referrer=utm_source\%3Dother_nyt_mobile_web\%26utm_medium\%3DWeb\%2520page\%26utm_term\%3DGeneral\%2520Mobile\%2520Page\%26utm_campaign\%3DNYT\%2520Mobile\%2520General\%2520Page}{Android}
    and turn on Breaking News and Politics alerts
  \end{itemize}
\end{itemize}

Advertisement

\protect\hyperlink{after-bottom}{Continue reading the main story}

\hypertarget{site-index}{%
\subsection{Site Index}\label{site-index}}

\hypertarget{site-information-navigation}{%
\subsection{Site Information
Navigation}\label{site-information-navigation}}

\begin{itemize}
\tightlist
\item
  \href{https://help.nytimes3xbfgragh.onion/hc/en-us/articles/115014792127-Copyright-notice}{©~2020~The
  New York Times Company}
\end{itemize}

\begin{itemize}
\tightlist
\item
  \href{https://www.nytco.com/}{NYTCo}
\item
  \href{https://help.nytimes3xbfgragh.onion/hc/en-us/articles/115015385887-Contact-Us}{Contact
  Us}
\item
  \href{https://www.nytco.com/careers/}{Work with us}
\item
  \href{https://nytmediakit.com/}{Advertise}
\item
  \href{http://www.tbrandstudio.com/}{T Brand Studio}
\item
  \href{https://www.nytimes3xbfgragh.onion/privacy/cookie-policy\#how-do-i-manage-trackers}{Your
  Ad Choices}
\item
  \href{https://www.nytimes3xbfgragh.onion/privacy}{Privacy}
\item
  \href{https://help.nytimes3xbfgragh.onion/hc/en-us/articles/115014893428-Terms-of-service}{Terms
  of Service}
\item
  \href{https://help.nytimes3xbfgragh.onion/hc/en-us/articles/115014893968-Terms-of-sale}{Terms
  of Sale}
\item
  \href{https://spiderbites.nytimes3xbfgragh.onion}{Site Map}
\item
  \href{https://help.nytimes3xbfgragh.onion/hc/en-us}{Help}
\item
  \href{https://www.nytimes3xbfgragh.onion/subscription?campaignId=37WXW}{Subscriptions}
\end{itemize}
