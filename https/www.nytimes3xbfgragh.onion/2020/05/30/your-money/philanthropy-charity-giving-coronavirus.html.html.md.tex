Sections

SEARCH

\protect\hyperlink{site-content}{Skip to
content}\protect\hyperlink{site-index}{Skip to site index}

\href{https://www.nytimes3xbfgragh.onion/section/your-money}{Your Money}

\href{https://myaccount.nytimes3xbfgragh.onion/auth/login?response_type=cookie\&client_id=vi}{}

\href{https://www.nytimes3xbfgragh.onion/section/todayspaper}{Today's
Paper}

\href{/section/your-money}{Your Money}\textbar{}How to Get Your Money to
Those Who Need It More Than You

\url{https://nyti.ms/2TT4N99}

\begin{itemize}
\item
\item
\item
\item
\item
\end{itemize}

\hypertarget{the-coronavirus-outbreak}{%
\subsubsection{\texorpdfstring{\href{https://www.nytimes3xbfgragh.onion/news-event/coronavirus?name=styln-coronavirus-national\&region=TOP_BANNER\&variant=undefined\&block=storyline_menu_recirc\&action=click\&pgtype=Article\&impression_id=31013260-e3b1-11ea-881f-496d10ec9bb8}{The
Coronavirus
Outbreak}}{The Coronavirus Outbreak}}\label{the-coronavirus-outbreak}}

\begin{itemize}
\tightlist
\item
  live\href{https://www.nytimes3xbfgragh.onion/2020/08/21/world/covid-19-coronavirus.html?name=styln-coronavirus-national\&region=TOP_BANNER\&variant=undefined\&block=storyline_menu_recirc\&action=click\&pgtype=Article\&impression_id=31015970-e3b1-11ea-881f-496d10ec9bb8}{Latest
  Updates}
\item
  \href{https://www.nytimes3xbfgragh.onion/interactive/2020/us/coronavirus-us-cases.html?name=styln-coronavirus-national\&region=TOP_BANNER\&variant=undefined\&block=storyline_menu_recirc\&action=click\&pgtype=Article\&impression_id=31015971-e3b1-11ea-881f-496d10ec9bb8}{Maps
  and Cases}
\item
  \href{https://www.nytimes3xbfgragh.onion/interactive/2020/science/coronavirus-vaccine-tracker.html?name=styln-coronavirus-national\&region=TOP_BANNER\&variant=undefined\&block=storyline_menu_recirc\&action=click\&pgtype=Article\&impression_id=31015972-e3b1-11ea-881f-496d10ec9bb8}{Vaccine
  Tracker}
\item
  \href{https://www.nytimes3xbfgragh.onion/2020/08/19/us/colleges-closing-covid.html?name=styln-coronavirus-national\&region=TOP_BANNER\&variant=undefined\&block=storyline_menu_recirc\&action=click\&pgtype=Article\&impression_id=31018080-e3b1-11ea-881f-496d10ec9bb8}{Colleges
  Closing}
\item
  \href{https://www.nytimes3xbfgragh.onion/live/2020/08/21/business/stock-market-today-coronavirus?name=styln-coronavirus-national\&region=TOP_BANNER\&variant=undefined\&block=storyline_menu_recirc\&action=click\&pgtype=Article\&impression_id=31018081-e3b1-11ea-881f-496d10ec9bb8}{Economy}
\end{itemize}

Advertisement

\protect\hyperlink{after-top}{Continue reading the main story}

Supported by

\protect\hyperlink{after-sponsor}{Continue reading the main story}

Your Money

\hypertarget{how-to-get-your-money-to-those-who-need-it-more-than-you}{%
\section{How to Get Your Money to Those Who Need It More Than
You}\label{how-to-get-your-money-to-those-who-need-it-more-than-you}}

Your instinct may be to help right now, in cash. These services bridge
the gap between Venmoing someone from Twitter and a check to a big
charity that could steer that money anywhere.

\includegraphics{https://static01.graylady3jvrrxbe.onion/images/2020/05/30/business/29money/29money-articleLarge.jpg?quality=75\&auto=webp\&disable=upscale}

\href{https://www.nytimes3xbfgragh.onion/by/ron-lieber}{\includegraphics{https://static01.graylady3jvrrxbe.onion/images/2018/10/22/multimedia/author-ron-lieber/author-ron-lieber-thumbLarge.png}}

By \href{https://www.nytimes3xbfgragh.onion/by/ron-lieber}{Ron Lieber}

\begin{itemize}
\item
  May 30, 2020
\item
  \begin{itemize}
  \item
  \item
  \item
  \item
  \item
  \end{itemize}
\end{itemize}

As unemployment claims pass
\href{https://www.nytimes3xbfgragh.onion/2020/05/28/business/economy/coronavirus-unemployment-claims.html}{40
million} and the anxious people who file them
\href{https://www.nytimes3xbfgragh.onion/2020/05/28/business/economy/coronavirus-stimulus-unemployment.html}{grow
more desperate}, an altruistic instinct has emerged among those who are
more financially secure.

But the sheer breadth of the pain is almost overwhelming, and the
appeals are everywhere. And the impulse is to help --- now --- when
confronted with a personal plea.

So what is the very best way for people with more money than they need
to quickly hand it over to those in need, so they can use it for food,
shelter and other necessities?

It isn't easy to find a satisfying answer. Sites and services like
GoFundMe can connect donors with real people, but they may lack vetting
of recipients, their back stories or their plans. They also may not make
it possible to be identified or anonymous, depending on your preference
as a giver or a beneficiary. Donors with large amounts to give may want
to use tax deductions to increase what they can afford to donate, but
may not be able to get them through one-off cash transfers.

The elusiveness of perfect solutions has inspired a variety of social
entrepreneurs to pursue various forms of direct giving. If you've sent
money via \href{https://www.donorschoose.org/about}{DonorsChoose} to
help a teacher pay for a classroom project, you get the basic idea: Give
a little money, know exactly where it's going, have some sense of who's
getting it and have someone between you and the recipient to provide at
least some verification. The goal is to provide a charitable option that
lives somewhere between handing money over to a large, faceless
nonprofit group and just
\href{https://www.nytimes3xbfgragh.onion/2020/03/16/business/coronavirus-bills-charity.html}{Venmoing
people} who say on Twitter that they need help.

Now that the pandemic has left millions more people with acute needs,
two existing organizations and one new entrant are offering some of the
most satisfying ways of providing few-strings-attached assistance.
\href{https://www.modestneeds.org/}{Modest Needs Foundation} and
\href{https://www.givedirectly.org/}{GiveDirectly}, both nonprofit
organizations, are using years of experience to pay people's bills or
hand them money to pay for things themselves. And the
\href{https://www.1kproject.org/}{1K Project} is facilitating money
transfers, although without the tax deductions the other two can offer
donors.

First, a few words for the wary. Cash transfers have been the subject of
a fair bit of study, particularly on transfers outside the United
States. \href{https://www.givewell.org/about}{GiveWell}, an extremely
particular, research-driven nonprofit group, evaluates how much good
other charitable organizations do. Its
\href{https://www.givewell.org/international/technical/programs/cash-transfers}{extended
take} speaks to a few findings: Recipients tend not to spend more on
``temptation goods'' like tobacco and alcohol but do increase what they
lay out for more and better food. Many people also invest in
infrastructure upgrades, like replacing thatch roofs with iron.

\hypertarget{latest-updates-the-coronavirus-outbreak}{%
\section{\texorpdfstring{\href{https://www.nytimes3xbfgragh.onion/2020/08/21/world/covid-19-coronavirus.html?action=click\&pgtype=Article\&state=default\&region=MAIN_CONTENT_1\&context=storylines_live_updates}{Latest
Updates: The Coronavirus
Outbreak}}{Latest Updates: The Coronavirus Outbreak}}\label{latest-updates-the-coronavirus-outbreak}}

Updated 2020-08-21T13:16:36.823Z

\begin{itemize}
\tightlist
\item
  \href{https://www.nytimes3xbfgragh.onion/2020/08/21/world/covid-19-coronavirus.html?action=click\&pgtype=Article\&state=default\&region=MAIN_CONTENT_1\&context=storylines_live_updates\#link-6a60a19d}{`Be
  adults': Universities in the U.S. are warning students about
  gatherings as they return to campus.}
\item
  \href{https://www.nytimes3xbfgragh.onion/2020/08/21/world/covid-19-coronavirus.html?action=click\&pgtype=Article\&state=default\&region=MAIN_CONTENT_1\&context=storylines_live_updates\#link-324af071}{As
  he accepts the Democratic nomination, Biden knocks Trump's pandemic
  response.}
\item
  \href{https://www.nytimes3xbfgragh.onion/2020/08/21/world/covid-19-coronavirus.html?action=click\&pgtype=Article\&state=default\&region=MAIN_CONTENT_1\&context=storylines_live_updates\#link-191d44be}{South
  Korea threatens to detain people who obstruct virus-control efforts.}
\end{itemize}

\href{https://www.nytimes3xbfgragh.onion/2020/08/21/world/covid-19-coronavirus.html?action=click\&pgtype=Article\&state=default\&region=MAIN_CONTENT_1\&context=storylines_live_updates}{See
more updates}

More live coverage:
\href{https://www.nytimes3xbfgragh.onion/live/2020/08/21/business/stock-market-today-coronavirus?action=click\&pgtype=Article\&state=default\&region=MAIN_CONTENT_1\&context=storylines_live_updates}{Markets}

GiveWell recommends
\href{https://www.givewell.org/charities/top-charities}{a handful} of
charities, including GiveDirectly, which began its coronavirus efforts
in March with the same basic goals it set for its work outside of the
United States: Give \$1,000 to poor people and let them decide what to
do with it.

GiveDirectly partnered with
\href{https://www.joinpropel.com/covid-19}{Propel}, a company that helps
recipients of SNAP (the Supplemental Nutrition Assistance Program once
known as food stamps) manage their benefits. They receive a message at
random offering the money, a bit like a lottery that they don't have to
enter.

It is not actually legal for a nonprofit group to help you transfer
money directly to specific individuals if you want the amount to be a
tax deductible donation, though GiveDirectly does
\href{https://www.givedirectly.org/covid-19/us/recipient-stories/}{post
videos} from grateful recipients.
\href{https://www.givedirectly.org/team/}{Michael Faye}, a co-founder
and the president, who has a Ph.D in economics, said in an interview
that the organization wouldn't want to allow that anyway --- research
indicates that donors might choose people based on physical appearance
rather than pure need.

Donors can, however, direct their money to
\href{https://community.givedirectly.org/}{specific hard-hit cities}. So
far, GiveDirectly has sent \$1,000 each to about 82,000 people. Its goal
is to reach 100,000, though it is likely to continue the efforts if
donations keep coming in.

Modest Needs, a nonprofit organization, operates on a smaller scale,
with a slightly different model. Recipients need to find their way to
the group and apply for help paying particular bills. It requires
documentation of the need and pays bills directly, without giving money
to the applicant.

Keith Taylor, a former humanities professor, started the organization
with the intent to help people cover a out-of-the-blue emergency
expenses. In practice, he said, people often turn to Modest Needs after
they've paid for the emergency and can no longer cover food or shelter
or that month's car payment.

Modest Needs posts
\href{https://www.modestneeds.org/features/ledger/index.asp}{requests
for help} on its website, but by the time they are there, the
organization has verified that they are legitimate. All that's left is
for donors to give. In the last two months, requests for help have more
than quadrupled: 539 people or families are currently waiting for
\$842,000 in donations.

More than half of the people who have received help in the past are now
among the group's donors. Some of them give as little as \$2 per month,
but those that do give regularly have often been doing it month-in,
month-out since they received their own assistance.

\href{https://www.nytimes3xbfgragh.onion/news-event/coronavirus?action=click\&pgtype=Article\&state=default\&region=MAIN_CONTENT_3\&context=storylines_faq}{}

\hypertarget{the-coronavirus-outbreak-}{%
\subsubsection{The Coronavirus Outbreak
›}\label{the-coronavirus-outbreak-}}

\hypertarget{frequently-asked-questions}{%
\paragraph{Frequently Asked
Questions}\label{frequently-asked-questions}}

Updated August 17, 2020

\begin{itemize}
\item ~
  \hypertarget{why-does-standing-six-feet-away-from-others-help}{%
  \paragraph{Why does standing six feet away from others
  help?}\label{why-does-standing-six-feet-away-from-others-help}}

  \begin{itemize}
  \tightlist
  \item
    The coronavirus spreads primarily through droplets from your mouth
    and nose, especially when you cough or sneeze. The C.D.C., one of
    the organizations using that measure,
    \href{https://www.nytimes3xbfgragh.onion/2020/04/14/health/coronavirus-six-feet.html?action=click\&pgtype=Article\&state=default\&region=MAIN_CONTENT_3\&context=storylines_faq}{bases
    its recommendation of six feet} on the idea that most large droplets
    that people expel when they cough or sneeze will fall to the ground
    within six feet. But six feet has never been a magic number that
    guarantees complete protection. Sneezes, for instance, can launch
    droplets a lot farther than six feet,
    \href{https://jamanetwork.com/journals/jama/fullarticle/2763852}{according
    to a recent study}. It's a rule of thumb: You should be safest
    standing six feet apart outside, especially when it's windy. But
    keep a mask on at all times, even when you think you're far enough
    apart.
  \end{itemize}
\item ~
  \hypertarget{i-have-antibodies-am-i-now-immune}{%
  \paragraph{I have antibodies. Am I now
  immune?}\label{i-have-antibodies-am-i-now-immune}}

  \begin{itemize}
  \tightlist
  \item
    As of right
    now,\href{https://www.nytimes3xbfgragh.onion/2020/07/22/health/covid-antibodies-herd-immunity.html?action=click\&pgtype=Article\&state=default\&region=MAIN_CONTENT_3\&context=storylines_faq}{that
    seems likely, for at least several months.} There have been
    frightening accounts of people suffering what seems to be a second
    bout of Covid-19. But experts say these patients may have a
    drawn-out course of infection, with the virus taking a slow toll
    weeks to months after initial exposure. People infected with the
    coronavirus typically
    \href{https://www.nature.com/articles/s41586-020-2456-9}{produce}
    immune molecules called antibodies, which are
    \href{https://www.nytimes3xbfgragh.onion/2020/05/07/health/coronavirus-antibody-prevalence.html?action=click\&pgtype=Article\&state=default\&region=MAIN_CONTENT_3\&context=storylines_faq}{protective
    proteins made in response to an
    infection}\href{https://www.nytimes3xbfgragh.onion/2020/05/07/health/coronavirus-antibody-prevalence.html?action=click\&pgtype=Article\&state=default\&region=MAIN_CONTENT_3\&context=storylines_faq}{.
    These antibodies may} last in the body
    \href{https://www.nature.com/articles/s41591-020-0965-6}{only two to
    three months}, which may seem worrisome, but that's perfectly normal
    after an acute infection subsides, said Dr. Michael Mina, an
    immunologist at Harvard University. It may be possible to get the
    coronavirus again, but it's highly unlikely that it would be
    possible in a short window of time from initial infection or make
    people sicker the second time.
  \end{itemize}
\item ~
  \hypertarget{im-a-small-business-owner-can-i-get-relief}{%
  \paragraph{I'm a small-business owner. Can I get
  relief?}\label{im-a-small-business-owner-can-i-get-relief}}

  \begin{itemize}
  \tightlist
  \item
    The
    \href{https://www.nytimes3xbfgragh.onion/article/small-business-loans-stimulus-grants-freelancers-coronavirus.html?action=click\&pgtype=Article\&state=default\&region=MAIN_CONTENT_3\&context=storylines_faq}{stimulus
    bills enacted in March} offer help for the millions of American
    small businesses. Those eligible for aid are businesses and
    nonprofit organizations with fewer than 500 workers, including sole
    proprietorships, independent contractors and freelancers. Some
    larger companies in some industries are also eligible. The help
    being offered, which is being managed by the Small Business
    Administration, includes the Paycheck Protection Program and the
    Economic Injury Disaster Loan program. But lots of folks have
    \href{https://www.nytimes3xbfgragh.onion/interactive/2020/05/07/business/small-business-loans-coronavirus.html?action=click\&pgtype=Article\&state=default\&region=MAIN_CONTENT_3\&context=storylines_faq}{not
    yet seen payouts.} Even those who have received help are confused:
    The rules are draconian, and some are stuck sitting on
    \href{https://www.nytimes3xbfgragh.onion/2020/05/02/business/economy/loans-coronavirus-small-business.html?action=click\&pgtype=Article\&state=default\&region=MAIN_CONTENT_3\&context=storylines_faq}{money
    they don't know how to use.} Many small-business owners are getting
    less than they expected or
    \href{https://www.nytimes3xbfgragh.onion/2020/06/10/business/Small-business-loans-ppp.html?action=click\&pgtype=Article\&state=default\&region=MAIN_CONTENT_3\&context=storylines_faq}{not
    hearing anything at all.}
  \end{itemize}
\item ~
  \hypertarget{what-are-my-rights-if-i-am-worried-about-going-back-to-work}{%
  \paragraph{What are my rights if I am worried about going back to
  work?}\label{what-are-my-rights-if-i-am-worried-about-going-back-to-work}}

  \begin{itemize}
  \tightlist
  \item
    Employers have to provide
    \href{https://www.osha.gov/SLTC/covid-19/standards.html}{a safe
    workplace} with policies that protect everyone equally.
    \href{https://www.nytimes3xbfgragh.onion/article/coronavirus-money-unemployment.html?action=click\&pgtype=Article\&state=default\&region=MAIN_CONTENT_3\&context=storylines_faq}{And
    if one of your co-workers tests positive for the coronavirus, the
    C.D.C.} has said that
    \href{https://www.cdc.gov/coronavirus/2019-ncov/community/guidance-business-response.html}{employers
    should tell their employees} -\/- without giving you the sick
    employee's name -\/- that they may have been exposed to the virus.
  \end{itemize}
\item ~
  \hypertarget{what-is-school-going-to-look-like-in-september}{%
  \paragraph{What is school going to look like in
  September?}\label{what-is-school-going-to-look-like-in-september}}

  \begin{itemize}
  \tightlist
  \item
    It is unlikely that many schools will return to a normal schedule
    this fall, requiring the grind of
    \href{https://www.nytimes3xbfgragh.onion/2020/06/05/us/coronavirus-education-lost-learning.html?action=click\&pgtype=Article\&state=default\&region=MAIN_CONTENT_3\&context=storylines_faq}{online
    learning},
    \href{https://www.nytimes3xbfgragh.onion/2020/05/29/us/coronavirus-child-care-centers.html?action=click\&pgtype=Article\&state=default\&region=MAIN_CONTENT_3\&context=storylines_faq}{makeshift
    child care} and
    \href{https://www.nytimes3xbfgragh.onion/2020/06/03/business/economy/coronavirus-working-women.html?action=click\&pgtype=Article\&state=default\&region=MAIN_CONTENT_3\&context=storylines_faq}{stunted
    workdays} to continue. California's two largest public school
    districts --- Los Angeles and San Diego --- said on July 13, that
    \href{https://www.nytimes3xbfgragh.onion/2020/07/13/us/lausd-san-diego-school-reopening.html?action=click\&pgtype=Article\&state=default\&region=MAIN_CONTENT_3\&context=storylines_faq}{instruction
    will be remote-only in the fall}, citing concerns that surging
    coronavirus infections in their areas pose too dire a risk for
    students and teachers. Together, the two districts enroll some
    825,000 students. They are the largest in the country so far to
    abandon plans for even a partial physical return to classrooms when
    they reopen in August. For other districts, the solution won't be an
    all-or-nothing approach.
    \href{https://bioethics.jhu.edu/research-and-outreach/projects/eschool-initiative/school-policy-tracker/}{Many
    systems}, including the nation's largest, New York City, are
    devising
    \href{https://www.nytimes3xbfgragh.onion/2020/06/26/us/coronavirus-schools-reopen-fall.html?action=click\&pgtype=Article\&state=default\&region=MAIN_CONTENT_3\&context=storylines_faq}{hybrid
    plans} that involve spending some days in classrooms and other days
    online. There's no national policy on this yet, so check with your
    municipal school system regularly to see what is happening in your
    community.
  \end{itemize}
\end{itemize}

The \href{https://www.1kproject.org/}{1K Project} sprang from the minds
of two venture capitalists,
\href{https://www.firstrowpartners.vc/}{Minda Brusse} and
\href{https://www.2048.vc/}{Alex Iskold}. In conversation and on their
site, they speak of quick-scaling, self-organizing systems, complex
networks and a sort of human blockchain.

In practice, they aim to find needy people through a rapidly expanding
network of \href{https://airtable.com/shrbR4m4Gkbh3vadJ}{trusted
referral sources}, vet them with the help of volunteers and pair donors
of \$1,000 per month for three months with recipients. (Neither reveal
their full names to one another. Donors and recipients at GiveDirectly
and Modest Needs rarely if ever do, either.)

The 1K Project is not a nonprofit organization, so people who itemize
their deductions won't get a tax break for helping a family through its
service. It uses private GoFundMe pages to transfer funds to recipients,
so 1K is never in possession of donor funds itself. (1K charges no user
fees of its own, though GoFundMe levies payment processing fees.
GiveDirectly and Modest Needs are also fee-free, though as with any
nonprofit group, a small portion of donations may help cover some
administrative expenses.)

So far, it's helped about 250 recipients, mostly families with
dependents. Many referrals come from despondent small-business owners
trying to help their employees who have lost their jobs or had their
income severely cut back.

We should never stop reminding ourselves that these sorts of makeshift
stopgaps are often
\href{https://www.nytimes3xbfgragh.onion/2020/03/28/business/nordic-way-economic-rescue-virus.html}{not
necessary elsewhere} at the scale they are in the United States, where
people must
\href{https://www.nytimes3xbfgragh.onion/interactive/2020/05/26/magazine/coronavirus-san-antonio-unemployment-jobs.html}{queue
up in cars} at food banks and ping websites hundreds of times or
\href{https://www.nytimes3xbfgragh.onion/2020/04/04/nyregion/coronavirus-ny-unemployment-benefits.html}{find
a fax machine} just to register for assistance.

My conversation with Ms. Brusse and Mr. Iskold felt marked by a sort of
low-grade hum of urgency with a side of frustration over the quantity
and speed of the available help in the United States. So I had to ask:
In addition to trying to help, are they also trying to make some kind of
a point about politics or nonprofit groups or the world as it is --- or
as we might wish it to be?

They demurred. ``We find ourselves in this unique moment in time where
help isn't reaching people fast enough, and we want to be a bridge to
when other support and employment is available,'' Ms. Brusse said.

``The truth about this,'' Mr. Iskold added, ``is that this whole thing
is just a massive heartbreak.''

Advertisement

\protect\hyperlink{after-bottom}{Continue reading the main story}

\hypertarget{site-index}{%
\subsection{Site Index}\label{site-index}}

\hypertarget{site-information-navigation}{%
\subsection{Site Information
Navigation}\label{site-information-navigation}}

\begin{itemize}
\tightlist
\item
  \href{https://help.nytimes3xbfgragh.onion/hc/en-us/articles/115014792127-Copyright-notice}{©~2020~The
  New York Times Company}
\end{itemize}

\begin{itemize}
\tightlist
\item
  \href{https://www.nytco.com/}{NYTCo}
\item
  \href{https://help.nytimes3xbfgragh.onion/hc/en-us/articles/115015385887-Contact-Us}{Contact
  Us}
\item
  \href{https://www.nytco.com/careers/}{Work with us}
\item
  \href{https://nytmediakit.com/}{Advertise}
\item
  \href{http://www.tbrandstudio.com/}{T Brand Studio}
\item
  \href{https://www.nytimes3xbfgragh.onion/privacy/cookie-policy\#how-do-i-manage-trackers}{Your
  Ad Choices}
\item
  \href{https://www.nytimes3xbfgragh.onion/privacy}{Privacy}
\item
  \href{https://help.nytimes3xbfgragh.onion/hc/en-us/articles/115014893428-Terms-of-service}{Terms
  of Service}
\item
  \href{https://help.nytimes3xbfgragh.onion/hc/en-us/articles/115014893968-Terms-of-sale}{Terms
  of Sale}
\item
  \href{https://spiderbites.nytimes3xbfgragh.onion}{Site Map}
\item
  \href{https://help.nytimes3xbfgragh.onion/hc/en-us}{Help}
\item
  \href{https://www.nytimes3xbfgragh.onion/subscription?campaignId=37WXW}{Subscriptions}
\end{itemize}
