Sections

SEARCH

\protect\hyperlink{site-content}{Skip to
content}\protect\hyperlink{site-index}{Skip to site index}

\href{https://www.nytimes3xbfgragh.onion/section/business}{Business}

\href{https://myaccount.nytimes3xbfgragh.onion/auth/login?response_type=cookie\&client_id=vi}{}

\href{https://www.nytimes3xbfgragh.onion/section/todayspaper}{Today's
Paper}

\href{/section/business}{Business}\textbar{}In Hong Kong, China
Threatens Businesses and Workers

\url{https://nyti.ms/2ZT3JWA}

\begin{itemize}
\item
\item
\item
\item
\item
\end{itemize}

Advertisement

\protect\hyperlink{after-top}{Continue reading the main story}

Supported by

\protect\hyperlink{after-sponsor}{Continue reading the main story}

\hypertarget{in-hong-kong-china-threatens-businesses-and-workers}{%
\section{In Hong Kong, China Threatens Businesses and
Workers}\label{in-hong-kong-china-threatens-businesses-and-workers}}

Beijing is using fear and pressure to drum up support for its
increasingly hard-line stance in the Asian financial capital,
threatening its status as a global business center.

\includegraphics{https://static01.graylady3jvrrxbe.onion/images/2020/01/30/business/00hkfear-01/merlin_165615234_52e4b8fc-ac81-43b9-ae30-8463a884fac4-articleLarge.jpg?quality=75\&auto=webp\&disable=upscale}

\href{https://www.nytimes3xbfgragh.onion/by/alexandra-stevenson}{\includegraphics{https://static01.graylady3jvrrxbe.onion/images/2018/02/20/multimedia/author-alexandra-stevenson/author-alexandra-stevenson-thumbLarge.jpg}}

By
\href{https://www.nytimes3xbfgragh.onion/by/alexandra-stevenson}{Alexandra
Stevenson}

\begin{itemize}
\item
  Published May 31, 2020Updated June 1, 2020
\item
  \begin{itemize}
  \item
  \item
  \item
  \item
  \item
  \end{itemize}
\end{itemize}

\href{https://cn.nytimes3xbfgragh.onion/business/20200601/hong-kong-china-business/}{阅读简体中文版}\href{https://cn.nytimes3xbfgragh.onion/business/20200601/hong-kong-china-business/zh-hant/}{閱讀繁體中文版}

HONG KONG ---
\href{https://www.nytimes3xbfgragh.onion/2020/06/02/world/asia/china-george-floyd.html}{China}
and its allies are using threats and pressure to get business to back
Beijing's increasingly hard-line stance toward
\href{https://www.nytimes3xbfgragh.onion/2020/06/01/world/asia/Hong-kong-Tiananmen-vigil-banned.html}{Hong
Kong}, leading companies to muzzle or intimidate workers who speak out
in protest.

Leung Chun-ying, Hong Kong's former top leader, on Friday called for a
boycott of HSBC, the London bank, because it had not publicly backed
Beijing's push to enact a new national security law covering the
territory. ``Neither China nor Hong Kong owes HSBC anything,'' he wrote
\href{https://www.facebookcorewwwi.onion/leung.cy.108/posts/1176081609420981}{in
a Facebook post}. ``HSBC's businesses in China can be replaced overnight
by banks from China and from other countries.''

Days earlier, a union representing financial workers filed complaints
with
\href{https://www.nytimes3xbfgragh.onion/2020/06/01/world/asia/Hong-kong-Tiananmen-vigil-banned.html}{Hong
Kong} financial regulators alleging that two Chinese banks had pressured
their employees to sign a petition supporting the law. ``Such behavior
by a supervisor to compel employees to take political sides could be
considered abusive,'' the union wrote in letters to local officials.

Lawyers, bankers, professors and other professionals interviewed by The
New York Times described a growing culture of fear in offices across the
city. Employees face pressure to support pro-Beijing candidates in local
elections and echo the Chinese government's official line. Those who
speak out can be punished or even forced out.

China and the United States are clashing over the future of Hong Kong,
and global businesses are
\href{https://www.nytimes3xbfgragh.onion/2020/05/28/business/hong-kong-special-status-explained.html}{caught
in the middle}. President Trump on Friday said he would
\href{https://www.nytimes3xbfgragh.onion/2020/05/29/us/politics/trump-hong-kong-china-WHO.html?action=click\&module=Top\%20Stories\&pgtype=Homepage}{begin
rolling back the special trade and financial privileges} that the United
States extends to Hong Kong after Chinese leaders pushed through the
plan to enact the national security law, which critics fear will curtail
the city's independent judicial system and civil liberties.

Hong Kong's success as a global financial hub stems from its status as a
bridge between China's economic miracle and the rest of the world. Now
that balance is looking increasingly precarious.

Protests
\href{https://www.nytimes3xbfgragh.onion/2019/06/09/world/asia/hong-kong-extradition-protest.html}{erupted
last year} after Hong Kong's unpopular Beijing-backed government
\href{https://www.nytimes3xbfgragh.onion/2019/06/07/world/asia/hong-kong-china-extradition-protest.html}{tried
to give Chinese authorities more say} in the city's affairs. As it
\href{https://www.nytimes3xbfgragh.onion/2019/06/20/business/hong-kong-business-extradition.html}{has
pressured business} to take its side, China has used access to its vast
market as an incentive to toe the Communist Party line.

\includegraphics{https://static01.graylady3jvrrxbe.onion/images/2020/06/01/business/00JPhk-fear2-print/merlin_167620722_94931d10-3153-4573-9427-a26524456ee3-articleLarge.jpg?quality=75\&auto=webp\&disable=upscale}

``We've seen a rapid deterioration in free expression in Hong Kong since
the anti-government protests began,'' said Jason Ng, a former lawyer for
BNP Paribas, the French bank.

Cathay Pacific, the Hong Kong-based airline, drew headlines late last
year when it
\href{https://www.nytimes3xbfgragh.onion/2019/09/11/business/cathay-pacific-hong-kong-protests.html}{fired
employees for voicing views} that angered Chinese authorities. Four of
the world's biggest accounting firms
\href{https://www.nytimes3xbfgragh.onion/2019/08/23/world/asia/hong-kong-protests-accountants.html}{condemned
the Hong Kong protests} and distanced themselves from employees who
supported them.

Mr. Ng was punished by his former employer for writing his political
views on his Facebook page, using the phrase ``monkey see, monkey do''
to complain about pro-Chinese demonstrators. The comments, which were
later taken down, were heavily criticized in China's state media and on
the Chinese internet. BNP
\href{https://www.bnpparibas.com.hk/en/2019/09/13/statement-from-bnp-paribas-regarding-incident-involving-our-employee/}{apologized}
and pledged to take immediate action. Mr. Ng then left the bank.

``There is this awful environment now,'' said Mr. Ng, who has
co-authored a book about the pressure in Hong Kong called ``Unfree
Speech.'' ``The whole banking industry, at least Chinese-funded banks,
they face quite a lot of pressure from China.''

Something similar happened to Ka-chung Law, a high-profile economist at
Bank of Communications, a state-backed Chinese bank. For two decades,
Mr. Law said he never felt any topic was off limits.

Last summer, as violence flared, Mr. Law was told not to talk about the
role that the political chaos was having on the local economy. It was a
difficult proposition. He could see it was having a direct impact.

Image

Ka-chung Law was told not to talk about the role that the political
chaos was having on the local economy.Credit...Lam Yik Fei for The New
York Times

Then in early October, Mr. Law said, he emailed an article to his team
that was critical of China and discussed ways in which the United States
could punish Beijing economically. One of his bosses called him in.

The bank distanced itself from the article. Mr. Law's note had come from
his work email, therefore implicating the bank. ``That day I was told,
`This is your view,''' he said. ``I was not the author of the article,
but I didn't want to argue.''

Mr. Law said he was told to resign. He did. ``I don't want to stay in
that kind of environment,'' he said, ``and I don't think I deserve to
stay in the position if I keep my mouth shut.'' The bank declined to
comment.

The silencing of views different from Beijing's on the protests can be
both subtle and overt.

Gios Choong works for a Chinese state-backed company doing quarantine
checks and quality control inspection at the Hong Kong border. When he
first started out more than two decades ago, most of his colleagues were
Hong Kongers, and the atmosphere was more open, he said. But in recent
years, resentment built up as Hong Kong employees like himself were
replaced with mainlanders.

These days, when conversation at work turns to the protests, managers
label them as riots. Mr. Choong, who is a supporter of the pro-democracy
protests, said he found it alienating.

``My boss said to me, `Why do they go out?''' referring to the
protesters. ```You eat from China. Your food is from China. The water
comes from China. So why?'''

On the Friday before Hong Kong held district council elections in
November, Mr. Choong's manager approached him with a request. Vote No.
2, he was told. That was the number for the pro-Beijing candidate in his
district. He voted for the pro-democracy candidate instead. The
pro-democracy camp swept the election.

Increasingly, multinationals have found themselves in Beijing's
censorship cross hairs. The
\href{https://www.nytimes3xbfgragh.onion/2019/10/07/sports/basketball/nba-china-hong-kong.html}{N.B.A.
was thrust} into the harsh spotlight last year after the general manager
of the Houston Rockets wrote a message on Twitter in support of the Hong
Kong protesters. State media acted swiftly in retaliation, canceling the
broadcast of preseason games.

\href{https://www.nytimes3xbfgragh.onion/2019/08/12/fashion/china-donatella-versace-t-shirt.html}{Coach,
Givenchy and Versace} have also been forced to apologize for selling
clothes with designs that suggested Hong Kong was separate from China.

Image

Ming-tak Ng, a professor at Hong Kong Baptist University, was asked to
stop teaching at Chinese campuses.Credit...Lam Yik Fei for The New York
Times

Ming-tak Ng, a professor at Hong Kong Baptist University, has witnessed
firsthand the fury of ordinary Chinese citizens.

Until August, many of his weekends were devoted to teaching part-time
M.B.A. students in the mainland. Then he was photographed at a protest
with Jimmy Lai, the owner of a media group who is critical of China.

When his students saw it, they wrote to university officials to complain
about Mr. Ng's participation, requesting in a letter that the university
delete ``any information about him during the process of our study and
in our graduation thesis'' and threatening to boycott events where Mr.
Ng was in attendance.

After discussing the situation with the university, Mr. Ng agreed to
stop teaching at the Chinese campuses. ``I don't blame them,'' Mr. Ng
said. ``In China, everyone is under a tightly controlled system. I
appreciate that they did this to protect themselves politically.''

Christina Wu, a spokeswoman for the university, confirmed Mr. Ng's
change of schedule but said it was done ``purely based on academic
considerations.'' She said the university did not delete any information
about Mr. Ng.

This week, as Beijing pushed on with plans to implement its national
security law in Hong Kong, pro-Beijing groups fanned out across the city
in search of support. According to some local workers, their bosses
helped in the effort.

Managers at Chiyu Banking Corporation, a local bank owned by Xiamen
International Bank, sent a WhatsApp message to employees asking them to
sign a petition, according to a complaint filed by the Hong Kong
Financial Industry Employees General Union. Once they had done so, the
complaint said, they were told to screenshot their signature and share
it.

Similar instructions were sent to employees at Wing Lung Bank, according
to the union. Workers at other banks said they had received similar
messages, said Ka-wing Kwok, the union's chairman, but the union was
unable to verify them.

Chiyu Banking and Wing Lung Bank did not respond to requests for
comment. Hong Kong regulators declined to comment.

``Such behavior caused a chilling effect among employees,'' the union
wrote in letters to the Hong Kong authorities.

``Employees could not help worrying that if they do not obey the
instructions of their superiors, they might either be singled out by the
company or their personal work performance evaluation would be affected
in the future.''

Cao Li contributed research.

Advertisement

\protect\hyperlink{after-bottom}{Continue reading the main story}

\hypertarget{site-index}{%
\subsection{Site Index}\label{site-index}}

\hypertarget{site-information-navigation}{%
\subsection{Site Information
Navigation}\label{site-information-navigation}}

\begin{itemize}
\tightlist
\item
  \href{https://help.nytimes3xbfgragh.onion/hc/en-us/articles/115014792127-Copyright-notice}{©~2020~The
  New York Times Company}
\end{itemize}

\begin{itemize}
\tightlist
\item
  \href{https://www.nytco.com/}{NYTCo}
\item
  \href{https://help.nytimes3xbfgragh.onion/hc/en-us/articles/115015385887-Contact-Us}{Contact
  Us}
\item
  \href{https://www.nytco.com/careers/}{Work with us}
\item
  \href{https://nytmediakit.com/}{Advertise}
\item
  \href{http://www.tbrandstudio.com/}{T Brand Studio}
\item
  \href{https://www.nytimes3xbfgragh.onion/privacy/cookie-policy\#how-do-i-manage-trackers}{Your
  Ad Choices}
\item
  \href{https://www.nytimes3xbfgragh.onion/privacy}{Privacy}
\item
  \href{https://help.nytimes3xbfgragh.onion/hc/en-us/articles/115014893428-Terms-of-service}{Terms
  of Service}
\item
  \href{https://help.nytimes3xbfgragh.onion/hc/en-us/articles/115014893968-Terms-of-sale}{Terms
  of Sale}
\item
  \href{https://spiderbites.nytimes3xbfgragh.onion}{Site Map}
\item
  \href{https://help.nytimes3xbfgragh.onion/hc/en-us}{Help}
\item
  \href{https://www.nytimes3xbfgragh.onion/subscription?campaignId=37WXW}{Subscriptions}
\end{itemize}
