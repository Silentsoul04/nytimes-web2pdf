Sections

SEARCH

\protect\hyperlink{site-content}{Skip to
content}\protect\hyperlink{site-index}{Skip to site index}

\href{https://myaccount.nytimes3xbfgragh.onion/auth/login?response_type=cookie\&client_id=vi}{}

\href{https://www.nytimes3xbfgragh.onion/section/todayspaper}{Today's
Paper}

\href{/section/opinion}{Opinion}\textbar{}What Trump and Toxic Cops Have
in Common

\url{https://nyti.ms/2ZV96oc}

\begin{itemize}
\item
\item
\item
\item
\item
\item
\end{itemize}

Advertisement

\protect\hyperlink{after-top}{Continue reading the main story}

\href{/section/opinion}{Opinion}

Supported by

\protect\hyperlink{after-sponsor}{Continue reading the main story}

\hypertarget{what-trump-and-toxic-cops-have-in-common}{%
\section{What Trump and Toxic Cops Have in
Common}\label{what-trump-and-toxic-cops-have-in-common}}

It's us versus them.

\href{https://www.nytimes3xbfgragh.onion/by/jennifer-senior}{\includegraphics{https://static01.graylady3jvrrxbe.onion/images/2018/10/26/opinion/jennifer-senior/jennifer-senior-thumbLarge.png}}

By \href{https://www.nytimes3xbfgragh.onion/by/jennifer-senior}{Jennifer
Senior}

Opinion columnist

\begin{itemize}
\item
  May 31, 2020
\item
  \begin{itemize}
  \item
  \item
  \item
  \item
  \item
  \item
  \end{itemize}
\end{itemize}

\includegraphics{https://static01.graylady3jvrrxbe.onion/images/2020/05/31/opinion/31Senior-sub/merlin_173015139_fef4c7e9-6d66-4789-a52a-1a9ddb6d84dc-articleLarge.jpg?quality=75\&auto=webp\&disable=upscale}

\hypertarget{listen-to-this-op-ed}{%
\subsubsection{Listen to This Op-Ed}\label{listen-to-this-op-ed}}

Audio Recording by Audm

\emph{To hear more audio stories from publishers like The New York
Times,
download}\href{https://www.audm.com/?utm_source=nytmag\&utm_medium=embed\&utm_campaign=left_behind_draper}{**}\href{https://www.audm.com/?utm_source=nytopinion\&utm_medium=embed\&utm_campaign=trump_toxic_cops}{\emph{Audm
for iPhone or Android}}\emph{.}

In his
\href{https://www.nytimes3xbfgragh.onion/2017/01/20/us/politics/trump-resurrects-dark-definition-of-america-first-vision.html}{first
Inaugural Address}, and hopefully his last,
\href{https://www.nytimes3xbfgragh.onion/2020/06/12/us/politics/trump-police-chokeholds.html}{Donald
Trump} talked about American carnage. He got it this week. What we
couldn't have known in January 2017 is that he wasn't here to save us
from this carnage, but to perpetuate it; that incitement wasn't just a
feature of his campaign, but of his governance. When historians look
back at the Trump era, they may very well say his presidency was
encapsulated by this moment, when a sadistic cop knelt on the neck of an
African-American man for almost nine minutes in plain view and the
streets exploded in rage.

Derek Chauvin was by no means the first cop to gratuitously brutalize
and lynch an African-American. But he embodied something essential about
Trumpism: It's us versus them. That's the poison ethos at the heart of
\href{https://www.nytimes3xbfgragh.onion/2020/06/12/us/politics/trump-police-chokeholds.html}{police}
brutality, and it's the septic core of our 45th president's philosophy.
Neither a toxic cop nor Donald Trump sees himself as a servant of
\emph{all} the people they've sworn to protect. They are solely servants
of their own. Everyone else is the enemy.

From the beginning, the police have received a lot of perverse messages
from Trump, encouraging them to embrace the bitter angels of their
nature. Three summers ago, he gave a speech on Long Island, disparaging
officers who cradled the heads of suspects as they tucked them into
their squad cars:
``\href{https://www.washingtonpost.com/news/post-nation/wp/2017/07/28/trump-tells-police-not-to-worry-about-injuring-suspects-during-arrests/}{You
can take the hand away,}OK?'' (A bank of police officers, seated behind
him, started to laugh and cheer.)

One of Trump's most revealing tweets since the rioting began was a boast
about the prowess of the Secret Service --- and to threaten to sic
``\href{https://www.nytimes3xbfgragh.onion/2020/05/30/us/politics/trump-threatens-protesters-dogs-weapons.html?smtyp=cur\&smid=tw-nytimes}{the
most vicious dogs, and most ominous weapons}'' on the crowds outside the
White House if things intensified. He's Bull Connor with a comb-over. Or
Walter E. Headley, Miami's former police chief, who in 1967 said,
``\href{https://www.nytimes3xbfgragh.onion/2020/05/29/us/looting-starts-shooting-starts.html}{When
the looting starts, the shooting starts,}'' a phrase that reappeared in
a Trump tweet on Friday.

And this is the point, is it not? Trump, who made his political bones by
peddling apocrypha about our first African-American president's country
of origin, thrives on racial divisions. Us-them. Conflict zones are his
comfort zone, perfect for firing up his base.

But the pressures of this historic moment proved to be too much. We
can't see the African-Americans who are dying in disproportionate
numbers inside our hospitals, but
\href{https://www.youtube.com/watch?v=61QNgIVj9mI\&bpctr=1590953361}{we
can see George Floyd}, an African-American, cruelly singled out for
asphyxiation in the street. His death in police custody is a potent
symbol of the obscene inequality and racial hostility of this moment,
with the police officer as Trump's smirking and pitiless proxy.
African-Americans --- and many whites too --- were so enraged that they
poured out into the streets to protest, even in the midst of a pandemic,
even though African-Americans are \emph{most at risk} in this pandemic.

A month from now, it's quite likely many will end up in hospitals, once
again in disproportionate numbers. It's too awful to contemplate.

And once again, there's a leadership vacuum in response to the chaos,
just as there is with Covid-19. It's every state for itself, with Trump
trolling the most liberal leaders for their supposed failures to contain
the unrest.

How these protests devolved into violence across the country will be the
subject of analysis for years to come. For now, what has riveted me is
that somehow, in spite of the dystopian horror unfolding in front of us,
in spite of execrable responses from some of the largest police forces
in the country
(\href{https://twitter.com/samdolnick/status/1266893561481568263}{including
New York City's}), we're nonetheless hearing talk of America as a
perfectible place --- of the arc still bending. It's been more than
three years since we've heard that tune.

Yet there was Joe Biden, the presumptive Democratic presidential
nominee, issuing a
\href{https://deadline.com/2020/05/george-floyd-protests-joe-biden-statement-violence-curfews-donald-trump-silent-1202947533/}{wee-hours
statement} that asked Americans not to ignore their pain, but to use it
``to compel our nation across this turbulent threshold into the next
phase of progress, inclusion, and opportunity.'' There was
\href{https://www.youtube.com/watch?v=rapfJYfPU38}{Killer Mike}, the
rapper from Atlanta, reminding his fellow citizens, ``Atlanta's not
perfect, but we're a lot better than we ever were, and a lot better than
cities are.''

Conservatives will focus on the pleas for law and order in their
messages. But what I hear is a repudiation of Trumpian nihilism --- a
rejection of the tyranny of the perpetual ``anxious present'' that Masha
Gessen describes in her forthcoming book
``\href{https://www.washingtonpost.com/outlook/2020/05/28/how-fight-back-against-trumpism-with-moral-aspiration/}{Surviving
Autocracy}.'' They're instead speaking with what Gessen calls ``moral
ambition,'' inviting fellow citizens to build a more expansive country,
rather than an us-versus-them one. Their messages weren't, ``Don't
destroy your community,'' so much as, ``There's still a community left
for you to join. Come and make it better.''

And so, along with terrifying imagery of fire and fury, we also saw
images over the weekend of police officers and protesters in solidarity.
The bonds were sometimes fragile, only to later disappear. But they
happened.
\href{https://twitter.com/Goodable/status/1266933266545881088}{In Flint,
Mich}. In
\href{https://abcnews.go.com/US/wireStory/police-praised-joining-front-line-floyd-protests-70983707}{Camden,
N.J.} In
\href{https://cbs12.com/news/local/police-take-a-knee-in-solidarity-prayer-with-protesters-in-coral-gables}{Coral
Gables, Fla}. In
\href{https://twitter.com/hknightsf/status/1266850494527365120}{Santa
Cruz, Calif.} In
\href{https://www.forbes.com/sites/lisettevoytko/2020/05/31/in-some-cities-police-officers-joined-protesters-marching-against-brutality/\#76c983145edb}{Ferguson,
Mo}. In Kansas City, Mo., where two cops, one white, held aloft a sign
saying
``\href{https://twitter.com/JasonKander/status/1266735093500051457/photo/1}{End
Police Brutality}.''

Or listen to the chief of police in Atlanta, Erika Shields, tell an
anxious protester,
``\href{https://twitter.com/search?q=chief\%20of\%20police\%20atlanta\%20\&src=typed_query}{I
hear you.}'' When Trump met with those who'd lost loved ones in the
Parkland shooting, he needed an empathy cheat sheet that contained those
very words;
\href{https://www.nytimes3xbfgragh.onion/2018/02/22/us/politics/trump-talking-points.html}{it
was item No. 5}. For her, they simply spilled out, as naturally as rain.

\emph{The Times is committed to publishing}
\href{https://www.nytimes3xbfgragh.onion/2019/01/31/opinion/letters/letters-to-editor-new-york-times-women.html}{\emph{a
diversity of letters}} \emph{to the editor. We'd like to hear what you
think about this or any of our articles. Here are some}
\href{https://help.nytimes3xbfgragh.onion/hc/en-us/articles/115014925288-How-to-submit-a-letter-to-the-editor}{\emph{tips}}\emph{.
And here's our email:}
\href{mailto:letters@NYTimes.com}{\emph{letters@NYTimes.com}}\emph{.}

\emph{Follow The New York Times Opinion section on}
\href{https://www.facebookcorewwwi.onion/nytopinion}{\emph{Facebook}}\emph{,}
\href{http://twitter.com/NYTOpinion}{\emph{Twitter (@NYTopinion)}}
\emph{and}
\href{https://www.instagram.com/nytopinion/}{\emph{Instagram}}\emph{.}

Advertisement

\protect\hyperlink{after-bottom}{Continue reading the main story}

\hypertarget{site-index}{%
\subsection{Site Index}\label{site-index}}

\hypertarget{site-information-navigation}{%
\subsection{Site Information
Navigation}\label{site-information-navigation}}

\begin{itemize}
\tightlist
\item
  \href{https://help.nytimes3xbfgragh.onion/hc/en-us/articles/115014792127-Copyright-notice}{©~2020~The
  New York Times Company}
\end{itemize}

\begin{itemize}
\tightlist
\item
  \href{https://www.nytco.com/}{NYTCo}
\item
  \href{https://help.nytimes3xbfgragh.onion/hc/en-us/articles/115015385887-Contact-Us}{Contact
  Us}
\item
  \href{https://www.nytco.com/careers/}{Work with us}
\item
  \href{https://nytmediakit.com/}{Advertise}
\item
  \href{http://www.tbrandstudio.com/}{T Brand Studio}
\item
  \href{https://www.nytimes3xbfgragh.onion/privacy/cookie-policy\#how-do-i-manage-trackers}{Your
  Ad Choices}
\item
  \href{https://www.nytimes3xbfgragh.onion/privacy}{Privacy}
\item
  \href{https://help.nytimes3xbfgragh.onion/hc/en-us/articles/115014893428-Terms-of-service}{Terms
  of Service}
\item
  \href{https://help.nytimes3xbfgragh.onion/hc/en-us/articles/115014893968-Terms-of-sale}{Terms
  of Sale}
\item
  \href{https://spiderbites.nytimes3xbfgragh.onion}{Site Map}
\item
  \href{https://help.nytimes3xbfgragh.onion/hc/en-us}{Help}
\item
  \href{https://www.nytimes3xbfgragh.onion/subscription?campaignId=37WXW}{Subscriptions}
\end{itemize}
