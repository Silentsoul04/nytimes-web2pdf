Sections

SEARCH

\protect\hyperlink{site-content}{Skip to
content}\protect\hyperlink{site-index}{Skip to site index}

\href{https://www.nytimes3xbfgragh.onion/section/health}{Health}

\href{https://myaccount.nytimes3xbfgragh.onion/auth/login?response_type=cookie\&client_id=vi}{}

\href{https://www.nytimes3xbfgragh.onion/section/todayspaper}{Today's
Paper}

\href{/section/health}{Health}\textbar{}After 6 Months, Important
Mysteries About Coronavirus Endure

\url{https://nyti.ms/2MhiXwP}

\begin{itemize}
\item
\item
\item
\item
\item
\item
\end{itemize}

\hypertarget{the-coronavirus-outbreak}{%
\subsubsection{\texorpdfstring{\href{https://www.nytimes3xbfgragh.onion/news-event/coronavirus?name=styln-coronavirus-national\&region=TOP_BANNER\&variant=undefined\&block=storyline_menu_recirc\&action=click\&pgtype=Article\&impression_id=dfebfb20-e398-11ea-88e1-ed0daa705d1f}{The
Coronavirus
Outbreak}}{The Coronavirus Outbreak}}\label{the-coronavirus-outbreak}}

\begin{itemize}
\tightlist
\item
  live\href{https://www.nytimes3xbfgragh.onion/2020/08/21/world/covid-19-coronavirus.html?name=styln-coronavirus-national\&region=TOP_BANNER\&variant=undefined\&block=storyline_menu_recirc\&action=click\&pgtype=Article\&impression_id=dfebfb21-e398-11ea-88e1-ed0daa705d1f}{Latest
  Updates}
\item
  \href{https://www.nytimes3xbfgragh.onion/interactive/2020/us/coronavirus-us-cases.html?name=styln-coronavirus-national\&region=TOP_BANNER\&variant=undefined\&block=storyline_menu_recirc\&action=click\&pgtype=Article\&impression_id=dfebfb22-e398-11ea-88e1-ed0daa705d1f}{Maps
  and Cases}
\item
  \href{https://www.nytimes3xbfgragh.onion/interactive/2020/science/coronavirus-vaccine-tracker.html?name=styln-coronavirus-national\&region=TOP_BANNER\&variant=undefined\&block=storyline_menu_recirc\&action=click\&pgtype=Article\&impression_id=dfec2230-e398-11ea-88e1-ed0daa705d1f}{Vaccine
  Tracker}
\item
  \href{https://www.nytimes3xbfgragh.onion/2020/08/19/us/colleges-closing-covid.html?name=styln-coronavirus-national\&region=TOP_BANNER\&variant=undefined\&block=storyline_menu_recirc\&action=click\&pgtype=Article\&impression_id=dfec2231-e398-11ea-88e1-ed0daa705d1f}{Colleges
  Closing}
\item
  \href{https://www.nytimes3xbfgragh.onion/live/2020/08/20/business/stock-market-today-coronavirus?name=styln-coronavirus-national\&region=TOP_BANNER\&variant=undefined\&block=storyline_menu_recirc\&action=click\&pgtype=Article\&impression_id=dfec2232-e398-11ea-88e1-ed0daa705d1f}{Economy}
\end{itemize}

Advertisement

\protect\hyperlink{after-top}{Continue reading the main story}

Supported by

\protect\hyperlink{after-sponsor}{Continue reading the main story}

\hypertarget{after-6-months-important-mysteries-about-coronavirus-endure}{%
\section{After 6 Months, Important Mysteries About Coronavirus
Endure}\label{after-6-months-important-mysteries-about-coronavirus-endure}}

Times journalists summarize some of the most critical things that
scientists and public health officials have yet to understand.

\includegraphics{https://static01.graylady3jvrrxbe.onion/images/2020/06/02/science/02SCI-CV-DONTKNOW-promo/02SCI-CV-DONTKNOW-promo-threeByTwoMediumAt2X.jpg}

By The New York Times

\begin{itemize}
\item
  June 1, 2020
\item
  \begin{itemize}
  \item
  \item
  \item
  \item
  \item
  \item
  \end{itemize}
\end{itemize}

\href{https://www.nytimes3xbfgragh.onion/es/2020/06/03/espanol/misterios-coronavirus.html}{Leer
en español}

\emph{{[}}\href{https://www.nytimes3xbfgragh.onion/2020/06/02/world/live-coronavirus.html}{\emph{Follow
our live coronavirus pandemic updates}}\emph{.{]}}

In the time since the world's scientists and public health officials
first became widely aware of the new coronavirus in January, they've had
\href{https://www.nytimes3xbfgragh.onion/2020/06/02/health/coronavirus-profile-covid.html}{six
months to learn about it}.

They've reached
\href{https://www.nytimes3xbfgragh.onion/article/coronavirus-facts-history.html}{many
conclusions about the virus and the illness it causes}, from the
importance of wearing masks to contain it, to the unusual range of
symptoms it provokes.

But there are major gaps in scientific knowledge about the virus. In the
half year that journalists of the health and science desk of The Times
have been reporting on Sars-CoV-2, we have identified some of the
enduring uncertainties. How scientists resolve some of these mysteries
will shape our future with the
\href{https://www.nytimes3xbfgragh.onion/2020/06/11/health/coronavirus-lung-transplant.html}{coronavirus}.

\hypertarget{here-are-some-of-the-things-we-dont-know-yet}{%
\subsubsection{Here are some of the things we don't know
yet:}\label{here-are-some-of-the-things-we-dont-know-yet}}

\begin{itemize}
\tightlist
\item
  \protect\hyperlink{link-3972962}{How many people have been infected.}
\item
  \protect\hyperlink{link-187ea3c}{The amount of virus it takes to make
  you sick.}
\item
  \protect\hyperlink{link-6983205a}{Why some people get so much sicker
  than others.}
\item
  \protect\hyperlink{link-7d05573}{The role of children in spreading the
  virus.}
\item
  \protect\hyperlink{link-1d483e4d}{When or where the new coronavirus
  started spreading.}
\item
  \protect\hyperlink{link-605f4b38}{How long you'll be immune after
  infection.}
\end{itemize}

\hypertarget{how-many-people-have-been-infected}{%
\subsection{How many people have been
infected.}\label{how-many-people-have-been-infected}}

By Donald G. McNeil Jr.

\includegraphics{https://static01.graylady3jvrrxbe.onion/images/2020/06/02/science/02SCI-CV-DONTKNOW2/merlin_172974849_28df5417-973e-4a11-8777-84f24fb2d1d3-articleLarge.jpg?quality=75\&auto=webp\&disable=upscale}

One of the epidemic's great unknowns is how many Americans have been
infected so far.

Only
\href{https://www.cdc.gov/coronavirus/2019-ncov/cases-updates/testing-in-us.html}{about
1.9 million Americans} had tested positive as of May 28, according to
the Centers for Disease Control and Prevention --- or perhaps only
\href{https://coronavirus.jhu.edu/map.html}{about 1.7 million} according
to the Johns Hopkins Center for Health Security, which produces a
frequently cited map of world cases.

Statisticians believe the actual number of cases is higher, but it is
not clear by how much.

They have more confidence in the accuracy of data produced
by\href{https://www.worldometers.info/coronavirus/country/us/}{states
that do the most testing}. On that score, New York is a leader; it has
tested about 9.6 percent of its population, which is almost double the
national average. (Rhode Island tested 13 percent of its people, beating
New York, but its population is small.)

New York has also done tens of thousands of antibody tests for
surveillance purposes. On May 22,
\href{https://www.governor.ny.gov/news/amid-ongoing-covid-19-pandemic-governor-cuomo-announces-results-states-antibody-testing-survey}{Gov.
Andrew Cuomo announced} that, according to these tests, about a fifth of
all New York City residents have had the virus, as had about 27 percent
of thousands of black and Hispanic New Yorkers who were tested at their
local churches.

If that same pattern were assumed to hold in other hard-hit big cities
like Chicago, New Orleans, Detroit and Miami, it is likely that
somewhere between 3 and 4 million urban Americans were infected.

Even if that figure was doubled or tripled --- which is probably
generous --- to account for Americans infected all around the country
before the lockdowns, that would still be only 9 to 12 million people.

That is in line with the low end of estimates made in mid-April, before
antibody testing
began;\href{https://www.nytimes3xbfgragh.onion/2020/04/18/health/coronavirus-america-future.html}{different
modelers predicted then that between 3 and 10 percent} of the country
was infected.

In any case, since this is a country of 330 million, a mere 10 million
or even 20 million infected would be a drop in the bucket. As the
country comes out of lockdown, the vast majority of Americans remain
vulnerable to the virus.

\hypertarget{latest-updates-the-coronavirus-outbreak}{%
\section{\texorpdfstring{\href{https://www.nytimes3xbfgragh.onion/2020/08/21/world/covid-19-coronavirus.html?action=click\&pgtype=Article\&state=default\&region=MAIN_CONTENT_1\&context=storylines_live_updates}{Latest
Updates: The Coronavirus
Outbreak}}{Latest Updates: The Coronavirus Outbreak}}\label{latest-updates-the-coronavirus-outbreak}}

Updated 2020-08-21T10:13:38.790Z

\begin{itemize}
\tightlist
\item
  \href{https://www.nytimes3xbfgragh.onion/2020/08/21/world/covid-19-coronavirus.html?action=click\&pgtype=Article\&state=default\&region=MAIN_CONTENT_1\&context=storylines_live_updates\#link-4690b6aa}{Shutdowns,
  warnings and scoldings follow gatherings on college campuses.}
\item
  \href{https://www.nytimes3xbfgragh.onion/2020/08/21/world/covid-19-coronavirus.html?action=click\&pgtype=Article\&state=default\&region=MAIN_CONTENT_1\&context=storylines_live_updates\#link-324af071}{As
  he accepts the Democratic nomination, Biden knocks Trump's pandemic
  response.}
\item
  \href{https://www.nytimes3xbfgragh.onion/2020/08/21/world/covid-19-coronavirus.html?action=click\&pgtype=Article\&state=default\&region=MAIN_CONTENT_1\&context=storylines_live_updates\#link-35890b73}{Hundreds
  of doctors in Kenya go on strike over their pay and protective gear.}
\end{itemize}

\href{https://www.nytimes3xbfgragh.onion/2020/08/21/world/covid-19-coronavirus.html?action=click\&pgtype=Article\&state=default\&region=MAIN_CONTENT_1\&context=storylines_live_updates}{See
more updates}

More live coverage:
\href{https://www.nytimes3xbfgragh.onion/live/2020/08/20/business/stock-market-today-coronavirus?action=click\&pgtype=Article\&state=default\&region=MAIN_CONTENT_1\&context=storylines_live_updates}{Markets}

Nor will the country be able to spot all of those who get it. The
C.D.C.'s ``best
estimate,''\href{https://www.cdc.gov/coronavirus/2019-ncov/hcp/planning-scenarios.html}{according
to the pandemic scenarios} it issued May 22, is that about 35 percent of
those infected have no symptoms.

Currently, about 21,000 Americans are getting infected each day. If a
third of them show no symptoms, almost 210,000 ``silent spreaders'' are
created each month.

It seems impossible to imagine that any contact-tracing program, no
matter how large, will be able to keep up with that.

\textbf{\emph{{[}}\href{http://on.fb.me/1paTQ1h}{\emph{Like the Science
Times page on Facebook.}}} ****** \emph{\textbar{} Sign up for the}
\textbf{\href{http://nyti.ms/1MbHaRU}{\emph{Science Times
newsletter.}}\emph{{]}}}

\hypertarget{the-amount-of-virus-it-takes-to-make-you-sick}{%
\subsection{The amount of virus it takes to make you
sick.}\label{the-amount-of-virus-it-takes-to-make-you-sick}}

By Apoorva Mandavilli

Here's what we can say for sure about the minimum number of viral
particles it takes to seed a coronavirus infection: The number is
somewhere between one and one million.

If you push scientists harder, they might offer a slightly smaller range
of a few hundred to a few thousand --- and some might even bravely throw
out an estimate in the high hundreds, based on what they have learned
about the behavior of the coronavirus that caused the 2002-03 SARS
epidemic.

But the bottom line is that no one will know for sure until more
research is completed.

``It's very unsatisfying to tell people, `Oh we don't know,''' said
Angela Rasmussen, a virologist at Columbia University in New York.
``People do have a lot of questions about this.''

It's difficult to say anything definitive about the coronavirus because
it behaves like an influenza virus in how easily it is transmitted, and
in that people can pass it on even when they don't seem sick. But its
structure, origin in bats and overall symptoms are similar to those of
its cousin coronaviruses.

So scientists cannot say whether touching a surface with a smidgen of
virus or breathing air with a few droplets exhaled by a sick person will
make you ill. But it is safe to say exposure to more of the coronavirus
is more likely to cause infection, and also to lead to more severe
symptoms. That's why it's important to avoid crowded indoor spaces, wear
masks and wash your hands. Each of those steps can decrease your chances
of being exposed to large amounts of virus.

\hypertarget{why-some-people-get-so-much-sicker-than-others}{%
\subsection{Why some people get so much sicker than
others.}\label{why-some-people-get-so-much-sicker-than-others}}

By Roni Caryn Rabin

Covid-19 is a mercurial disease. While some people experience only mild,
fleeting symptoms, others are knocked over with a severe flulike illness
that can last several weeks. A minority of patients develop
life-threatening complications. Death can ensue.

Why do some people sail through the illness, and others develop the
severe inflammation and lung damage that are hallmarks of the disease?
It is one of the great mysteries of Covid-19.

Experts say the patient's immune response to the viral infection
determines the severity of the illness.
\href{https://www.nytimes3xbfgragh.onion/2020/04/01/health/coronavirus-cytokine-storm-immune-system.html}{If
the immune system goes into overdrive}, it can trigger a cascade of
harmful effects, injuring the lungs and other organs.

Immune function declines with age, and
\href{https://www.nytimes3xbfgragh.onion/2020/03/14/health/coronavirus-elderly-protection.html}{elderly
people with Covid-19 are among the most vulnerable to poor outcomes}, as
are those with chronic health conditions like high blood pressure,
diabetes and cardiovascular disease.
\href{https://www.nytimes3xbfgragh.onion/2020/04/16/health/coronavirus-obesity-higher-risk.html}{Obesity,
which affects 4 in 10 American adults}, also appears to exacerbate the
illness.

\href{https://www.nytimes3xbfgragh.onion/2020/04/07/health/coronavirus-new-york-men.html}{Men
are at greater risk for critical illness and death}, a sex disparity
that may be explained by women's more robust immune systems, scientists
say.

Generally speaking, patients get sicker faster if they are exposed to a
large dose of the virus when they are first infected, said Dr. William
Schaffner, an infectious disease specialist at Vanderbilt University.

Much scientific inquiry has focused on the role of a receptor called
angiotensin-converting enzyme 2, or ACE2, which is the entry way for
coronaviruses into the cells. The receptor is found on the outer
surfaces of cells in the lungs, blood vessels, intestines and other
organs, as well as in the back of the throat and high up in the nasal
passage.

When the pandemic started, there was concern that people taking blood
pressure medications like ACE inhibitors could be at greater risk from
the coronavirus, but so far studies have not found that to be the case,
and doctors are urging patients to continue their medications.

Though the SARS-CoV-2 virus attaches to the receptor in order to
penetrate cells, ACE2 also helps regulate blood pressure and
inflammation. Some scientists have suggested that children may be less
susceptible to infection with Covid-19 because they have fewer of these
receptors. ACE2 is also regulated differently in men and women,
according to scientists who study sex differences in medicine, and men
tend to develop hypertension, or high blood pressure, at younger ages
than women. But much is still unknown.

``ACE2 can play two very critical roles, getting the virus into the
cell, but also modulating some of the damage that takes place in the
blood vessels and the lungs,'' said Dr. Ankit B. Patel, a nephrologist
at Brigham and Women's Hospital in Boston. ``So it's a double-edged
sword in a sense, and that's made the whole story all the more
complicated.''

\hypertarget{the-role-of-children-in-spreading-the-virus}{%
\subsection{The role of children in spreading the
virus.}\label{the-role-of-children-in-spreading-the-virus}}

By Pam Belluck

Image

Credit...Jens Mortensen for The New York Times

There are many crucial unresolved questions about children and Covid-19.
Finding the answers is not only important for them and their families,
but for society at large, as communities plan to reopen schools, day
care centers, playgrounds and other places children frequent.

\href{https://www.nytimes3xbfgragh.onion/news-event/coronavirus?action=click\&pgtype=Article\&state=default\&region=MAIN_CONTENT_3\&context=storylines_faq}{}

\hypertarget{the-coronavirus-outbreak-}{%
\subsubsection{The Coronavirus Outbreak
›}\label{the-coronavirus-outbreak-}}

\hypertarget{frequently-asked-questions}{%
\paragraph{Frequently Asked
Questions}\label{frequently-asked-questions}}

Updated August 17, 2020

\begin{itemize}
\item ~
  \hypertarget{why-does-standing-six-feet-away-from-others-help}{%
  \paragraph{Why does standing six feet away from others
  help?}\label{why-does-standing-six-feet-away-from-others-help}}

  \begin{itemize}
  \tightlist
  \item
    The coronavirus spreads primarily through droplets from your mouth
    and nose, especially when you cough or sneeze. The C.D.C., one of
    the organizations using that measure,
    \href{https://www.nytimes3xbfgragh.onion/2020/04/14/health/coronavirus-six-feet.html?action=click\&pgtype=Article\&state=default\&region=MAIN_CONTENT_3\&context=storylines_faq}{bases
    its recommendation of six feet} on the idea that most large droplets
    that people expel when they cough or sneeze will fall to the ground
    within six feet. But six feet has never been a magic number that
    guarantees complete protection. Sneezes, for instance, can launch
    droplets a lot farther than six feet,
    \href{https://jamanetwork.com/journals/jama/fullarticle/2763852}{according
    to a recent study}. It's a rule of thumb: You should be safest
    standing six feet apart outside, especially when it's windy. But
    keep a mask on at all times, even when you think you're far enough
    apart.
  \end{itemize}
\item ~
  \hypertarget{i-have-antibodies-am-i-now-immune}{%
  \paragraph{I have antibodies. Am I now
  immune?}\label{i-have-antibodies-am-i-now-immune}}

  \begin{itemize}
  \tightlist
  \item
    As of right
    now,\href{https://www.nytimes3xbfgragh.onion/2020/07/22/health/covid-antibodies-herd-immunity.html?action=click\&pgtype=Article\&state=default\&region=MAIN_CONTENT_3\&context=storylines_faq}{that
    seems likely, for at least several months.} There have been
    frightening accounts of people suffering what seems to be a second
    bout of Covid-19. But experts say these patients may have a
    drawn-out course of infection, with the virus taking a slow toll
    weeks to months after initial exposure. People infected with the
    coronavirus typically
    \href{https://www.nature.com/articles/s41586-020-2456-9}{produce}
    immune molecules called antibodies, which are
    \href{https://www.nytimes3xbfgragh.onion/2020/05/07/health/coronavirus-antibody-prevalence.html?action=click\&pgtype=Article\&state=default\&region=MAIN_CONTENT_3\&context=storylines_faq}{protective
    proteins made in response to an
    infection}\href{https://www.nytimes3xbfgragh.onion/2020/05/07/health/coronavirus-antibody-prevalence.html?action=click\&pgtype=Article\&state=default\&region=MAIN_CONTENT_3\&context=storylines_faq}{.
    These antibodies may} last in the body
    \href{https://www.nature.com/articles/s41591-020-0965-6}{only two to
    three months}, which may seem worrisome, but that's perfectly normal
    after an acute infection subsides, said Dr. Michael Mina, an
    immunologist at Harvard University. It may be possible to get the
    coronavirus again, but it's highly unlikely that it would be
    possible in a short window of time from initial infection or make
    people sicker the second time.
  \end{itemize}
\item ~
  \hypertarget{im-a-small-business-owner-can-i-get-relief}{%
  \paragraph{I'm a small-business owner. Can I get
  relief?}\label{im-a-small-business-owner-can-i-get-relief}}

  \begin{itemize}
  \tightlist
  \item
    The
    \href{https://www.nytimes3xbfgragh.onion/article/small-business-loans-stimulus-grants-freelancers-coronavirus.html?action=click\&pgtype=Article\&state=default\&region=MAIN_CONTENT_3\&context=storylines_faq}{stimulus
    bills enacted in March} offer help for the millions of American
    small businesses. Those eligible for aid are businesses and
    nonprofit organizations with fewer than 500 workers, including sole
    proprietorships, independent contractors and freelancers. Some
    larger companies in some industries are also eligible. The help
    being offered, which is being managed by the Small Business
    Administration, includes the Paycheck Protection Program and the
    Economic Injury Disaster Loan program. But lots of folks have
    \href{https://www.nytimes3xbfgragh.onion/interactive/2020/05/07/business/small-business-loans-coronavirus.html?action=click\&pgtype=Article\&state=default\&region=MAIN_CONTENT_3\&context=storylines_faq}{not
    yet seen payouts.} Even those who have received help are confused:
    The rules are draconian, and some are stuck sitting on
    \href{https://www.nytimes3xbfgragh.onion/2020/05/02/business/economy/loans-coronavirus-small-business.html?action=click\&pgtype=Article\&state=default\&region=MAIN_CONTENT_3\&context=storylines_faq}{money
    they don't know how to use.} Many small-business owners are getting
    less than they expected or
    \href{https://www.nytimes3xbfgragh.onion/2020/06/10/business/Small-business-loans-ppp.html?action=click\&pgtype=Article\&state=default\&region=MAIN_CONTENT_3\&context=storylines_faq}{not
    hearing anything at all.}
  \end{itemize}
\item ~
  \hypertarget{what-are-my-rights-if-i-am-worried-about-going-back-to-work}{%
  \paragraph{What are my rights if I am worried about going back to
  work?}\label{what-are-my-rights-if-i-am-worried-about-going-back-to-work}}

  \begin{itemize}
  \tightlist
  \item
    Employers have to provide
    \href{https://www.osha.gov/SLTC/covid-19/standards.html}{a safe
    workplace} with policies that protect everyone equally.
    \href{https://www.nytimes3xbfgragh.onion/article/coronavirus-money-unemployment.html?action=click\&pgtype=Article\&state=default\&region=MAIN_CONTENT_3\&context=storylines_faq}{And
    if one of your co-workers tests positive for the coronavirus, the
    C.D.C.} has said that
    \href{https://www.cdc.gov/coronavirus/2019-ncov/community/guidance-business-response.html}{employers
    should tell their employees} -\/- without giving you the sick
    employee's name -\/- that they may have been exposed to the virus.
  \end{itemize}
\item ~
  \hypertarget{what-is-school-going-to-look-like-in-september}{%
  \paragraph{What is school going to look like in
  September?}\label{what-is-school-going-to-look-like-in-september}}

  \begin{itemize}
  \tightlist
  \item
    It is unlikely that many schools will return to a normal schedule
    this fall, requiring the grind of
    \href{https://www.nytimes3xbfgragh.onion/2020/06/05/us/coronavirus-education-lost-learning.html?action=click\&pgtype=Article\&state=default\&region=MAIN_CONTENT_3\&context=storylines_faq}{online
    learning},
    \href{https://www.nytimes3xbfgragh.onion/2020/05/29/us/coronavirus-child-care-centers.html?action=click\&pgtype=Article\&state=default\&region=MAIN_CONTENT_3\&context=storylines_faq}{makeshift
    child care} and
    \href{https://www.nytimes3xbfgragh.onion/2020/06/03/business/economy/coronavirus-working-women.html?action=click\&pgtype=Article\&state=default\&region=MAIN_CONTENT_3\&context=storylines_faq}{stunted
    workdays} to continue. California's two largest public school
    districts --- Los Angeles and San Diego --- said on July 13, that
    \href{https://www.nytimes3xbfgragh.onion/2020/07/13/us/lausd-san-diego-school-reopening.html?action=click\&pgtype=Article\&state=default\&region=MAIN_CONTENT_3\&context=storylines_faq}{instruction
    will be remote-only in the fall}, citing concerns that surging
    coronavirus infections in their areas pose too dire a risk for
    students and teachers. Together, the two districts enroll some
    825,000 students. They are the largest in the country so far to
    abandon plans for even a partial physical return to classrooms when
    they reopen in August. For other districts, the solution won't be an
    all-or-nothing approach.
    \href{https://bioethics.jhu.edu/research-and-outreach/projects/eschool-initiative/school-policy-tracker/}{Many
    systems}, including the nation's largest, New York City, are
    devising
    \href{https://www.nytimes3xbfgragh.onion/2020/06/26/us/coronavirus-schools-reopen-fall.html?action=click\&pgtype=Article\&state=default\&region=MAIN_CONTENT_3\&context=storylines_faq}{hybrid
    plans} that involve spending some days in classrooms and other days
    online. There's no national policy on this yet, so check with your
    municipal school system regularly to see what is happening in your
    community.
  \end{itemize}
\end{itemize}

One puzzle is what role children play in spreading the virus. They seem
\href{https://www.nytimes3xbfgragh.onion/2020/04/06/health/coronavirus-children-us.html}{less
likely to become seriously ill than adults}, making up about
\href{https://www.cdc.gov/coronavirus/2019-ncov/hcp/pediatric-hcp.html}{2
percent of confirmed American coronavirus cases}. There are
\href{https://www.nature.com/articles/d41586-020-01354-0}{different
theories} about whether that is because children are less likely to
become infected to begin with, or whether the virus infects them just as
easily but mostly causes few or no symptoms.

Either way, a growing body of evidence suggests that infected
\href{https://www.nytimes3xbfgragh.onion/2020/05/05/health/coronavirus-children-transmission-school.html}{children
can transmit the virus}, possibly as easily as adults. And
\href{https://science.sciencemag.org/content/early/2020/05/04/science.abb8001.full}{one
recent study} suggests that when children attend school, they come in
contact with three times as many people as average adults do, providing
more opportunities for children to become infected and infect others.

Although far fewer children than adults have experienced severe
symptoms,
\href{https://www.nytimes3xbfgragh.onion/2020/05/11/health/coronavirus-children-icu.html}{some
children have become devastatingly ill} and there have been at least 20
\href{https://data.cdc.gov/NCHS/Provisional-COVID-19-Death-Counts-by-Sex-Age-and-S/9bhg-hcku}{deaths
of children from Covid-19 in the United States} and elsewhere. Reports
from hospitals suggest that the children most vulnerable to the
respiratory failure adults develop are those who already have a serious
medical condition. Some studies also suggest that
\href{https://www.nytimes3xbfgragh.onion/2020/04/06/health/coronavirus-children-us.html}{infants}
and
\href{https://www.nytimes3xbfgragh.onion/2020/03/17/health/coronavirus-childen.html}{preschoolers}
may be more vulnerable than older children.

But a small number of other children, including teenagers, who did not
have any symptoms when they were first infected, have developed
\href{https://www.nytimes3xbfgragh.onion/2020/05/17/health/coronavirus-multisystem-fnflammatory-syndrome-children-teenagers.html\#commentsContainer}{a
newly-identified inflammatory syndrome that can cause serious heart
problems}. The syndrome, which seems to occur weeks after infection and
to result from a revved-up immune response to the virus, has been
reported in Europe and throughout the United States and has caused
several deaths. Doctors are urgently trying to understand what causes
the syndrome, why it afflicts some children and not others, and how to
best treat or prevent it.

\hypertarget{when-or-where-the-new-coronavirus-started-spreading}{%
\subsection{When or where the new coronavirus started
spreading.}\label{when-or-where-the-new-coronavirus-started-spreading}}

By Benedict Carey

Image

Credit...Jens Mortensen for The New York Times

The notion of
\href{https://www.nytimes3xbfgragh.onion/2016/10/27/health/hiv-patient-zero-genetic-analysis.html}{a
single patient zero} is both theatrical and real: In any new epidemic,
some unlucky soul seeds the first infection, several links of which are
fated to seed chains of their own and spark a viral Big Bang.

By analyzing the genetic material of people who test positive,
scientists can trace the lineage of each virus back to a common
ancestor, and often to an individual carrier. The first confirmed
coronavirus case in the United States was a man who landed at
Seattle-Tacoma Airport on Jan. 15, from China. Other introductions came
in February, and scientists are now closing in on who, exactly, sparked
the outbreak in Washington State.

New York confirmed its first case on March 1, and by that time there
were already thousands of infected people walking around, for a week or
more. Scientists have found genetic signatures on the viruses studied so
far that link them to Europe, likely brought in by some of the millions
of people arriving in New York in February, and it is likely that there
were multiple introductions that spread widely: patients zero, plural.

The first infected arrivals in a community are not necessarily the ones
who light the fuse.
\href{https://www.biorxiv.org/content/10.1101/2020.05.21.109322v1.full.pdf}{In
a report published last week}, genetic scientists argued that infected
people were among both Americans and Europeans in January, but that most
of those viruses fizzled out. And French doctors
\href{https://www.sciencedirect.com/science/article/pii/S0924857920301643}{recently
reported that a respiratory sample} from a man hospitalized near Paris,
in late December, tested positive. That virus, too, likely died out.
France's outbreak did not start until many weeks later.

The world's patient zero, in China, began infecting others in the late
fall of last year, the evidence thus far suggests.
\href{https://www.thelancet.com/journals/lancet/article/PIIS0140-6736(20)30183-5/fulltext?fbclid=IwAR18JG3cpHvWPVa9ZOZ9ZZnUS1hVjQJ0A2KcTcRAp0z9I12TbIv8rUC4sv0}{An
analysis of the first 41 confirmed cases,} all in people who had visited
the same seafood market in Wuhan, indicates that the first hospital
admission was on Dec. 16, 2019. The patient first noticed symptoms on
Dec. 1, so the infection dates back earlier. Several scientists have
estimated that the first outbreak began in late or mid-November, and
\href{http://virological.org/t/phylodynamic-analysis-176-genomes-6-mar-2020/356}{have
inferred a probable common viral ancestor}, though additional virus
samples could change the picture.

The level of detective work required to find the actual patient zero
might be steeper than it appears.
\href{https://news.tulane.edu/pr/study-coronavirus-pandemic-sparked-nature-not-bioengineering}{At
least one genetic scientist} has argued that the virus could have first
infected humans --- likely from a pangolin --- well before last fall, in
a form that did not cause sickness. It then evolved its pathogenic
features over time, while circulating. If that's the case, the question
``Who came first?'' may go without a conclusive answer for some time,
perhaps for good.

\hypertarget{how-long-youll-be-immune-after-infection}{%
\subsection{How long you'll be immune after
infection.}\label{how-long-youll-be-immune-after-infection}}

By Apoorva Mandavilli

Are people infected with the coronavirus protected from further
infection? And, if yes, for how long?

The answers to these questions have broad implications for reopening the
economy and allowing the public to live with less fear of infection in
the short term --- and for the effectiveness of vaccines in the long
term.

Scientists have made steady, if incremental, progress in getting to the
answers. When the body encounters any virus, it typically makes
antibodies, some of which are powerful enough to neutralize the pathogen
and prevent reinfection. It also produces large numbers of immune cells
that can kill the virus.

Most tests that look for antibodies to the coronavirus
\href{https://www.nytimes3xbfgragh.onion/2020/04/24/health/coronavirus-antibody-tests.html}{have
been flawed}. But at least one team with a reliable test reported that
most people, including those who were only mildly ill,
\href{https://www.nytimes3xbfgragh.onion/2020/05/07/health/coronavirus-antibody-prevalence.html}{make
powerful antibodies}. Data on immune cells has been slower to emerge,
but a few studies suggest a robust response from immune cells as well.

What remains unknown is how long this immunity will last. There have
been some reports of reinfection, but scientists have said that they are
a result either of faulty testing, or of viral remnants that circulate
long after the active infection has ended.

They are hopeful that based on other coronaviruses that cause the common
cold, SARS or MERS, immunity to the new coronavirus might last at least
a year, but it remains a mystery for now.

Advertisement

\protect\hyperlink{after-bottom}{Continue reading the main story}

\hypertarget{site-index}{%
\subsection{Site Index}\label{site-index}}

\hypertarget{site-information-navigation}{%
\subsection{Site Information
Navigation}\label{site-information-navigation}}

\begin{itemize}
\tightlist
\item
  \href{https://help.nytimes3xbfgragh.onion/hc/en-us/articles/115014792127-Copyright-notice}{©~2020~The
  New York Times Company}
\end{itemize}

\begin{itemize}
\tightlist
\item
  \href{https://www.nytco.com/}{NYTCo}
\item
  \href{https://help.nytimes3xbfgragh.onion/hc/en-us/articles/115015385887-Contact-Us}{Contact
  Us}
\item
  \href{https://www.nytco.com/careers/}{Work with us}
\item
  \href{https://nytmediakit.com/}{Advertise}
\item
  \href{http://www.tbrandstudio.com/}{T Brand Studio}
\item
  \href{https://www.nytimes3xbfgragh.onion/privacy/cookie-policy\#how-do-i-manage-trackers}{Your
  Ad Choices}
\item
  \href{https://www.nytimes3xbfgragh.onion/privacy}{Privacy}
\item
  \href{https://help.nytimes3xbfgragh.onion/hc/en-us/articles/115014893428-Terms-of-service}{Terms
  of Service}
\item
  \href{https://help.nytimes3xbfgragh.onion/hc/en-us/articles/115014893968-Terms-of-sale}{Terms
  of Sale}
\item
  \href{https://spiderbites.nytimes3xbfgragh.onion}{Site Map}
\item
  \href{https://help.nytimes3xbfgragh.onion/hc/en-us}{Help}
\item
  \href{https://www.nytimes3xbfgragh.onion/subscription?campaignId=37WXW}{Subscriptions}
\end{itemize}
