Sections

SEARCH

\protect\hyperlink{site-content}{Skip to
content}\protect\hyperlink{site-index}{Skip to site index}

\href{https://www.nytimes3xbfgragh.onion/section/health}{Health}

\href{https://myaccount.nytimes3xbfgragh.onion/auth/login?response_type=cookie\&client_id=vi}{}

\href{https://www.nytimes3xbfgragh.onion/section/todayspaper}{Today's
Paper}

\href{/section/health}{Health}\textbar{}Medical Workers Should Use
Respirator Masks, Not Surgical Masks

\url{https://nyti.ms/36R96ah}

\begin{itemize}
\item
\item
\item
\item
\item
\item
\end{itemize}

\hypertarget{the-coronavirus-outbreak}{%
\subsubsection{\texorpdfstring{\href{https://www.nytimes3xbfgragh.onion/news-event/coronavirus?name=styln-coronavirus-national\&region=TOP_BANNER\&variant=undefined\&block=storyline_menu_recirc\&action=click\&pgtype=Article\&impression_id=4aab1750-e3af-11ea-aa1e-5137522d6522}{The
Coronavirus
Outbreak}}{The Coronavirus Outbreak}}\label{the-coronavirus-outbreak}}

\begin{itemize}
\tightlist
\item
  live\href{https://www.nytimes3xbfgragh.onion/2020/08/21/world/covid-19-coronavirus.html?name=styln-coronavirus-national\&region=TOP_BANNER\&variant=undefined\&block=storyline_menu_recirc\&action=click\&pgtype=Article\&impression_id=4aab1751-e3af-11ea-aa1e-5137522d6522}{Latest
  Updates}
\item
  \href{https://www.nytimes3xbfgragh.onion/interactive/2020/us/coronavirus-us-cases.html?name=styln-coronavirus-national\&region=TOP_BANNER\&variant=undefined\&block=storyline_menu_recirc\&action=click\&pgtype=Article\&impression_id=4aab1752-e3af-11ea-aa1e-5137522d6522}{Maps
  and Cases}
\item
  \href{https://www.nytimes3xbfgragh.onion/interactive/2020/science/coronavirus-vaccine-tracker.html?name=styln-coronavirus-national\&region=TOP_BANNER\&variant=undefined\&block=storyline_menu_recirc\&action=click\&pgtype=Article\&impression_id=4aab1753-e3af-11ea-aa1e-5137522d6522}{Vaccine
  Tracker}
\item
  \href{https://www.nytimes3xbfgragh.onion/2020/08/19/us/colleges-closing-covid.html?name=styln-coronavirus-national\&region=TOP_BANNER\&variant=undefined\&block=storyline_menu_recirc\&action=click\&pgtype=Article\&impression_id=4aab1754-e3af-11ea-aa1e-5137522d6522}{Colleges
  Closing}
\item
  \href{https://www.nytimes3xbfgragh.onion/live/2020/08/21/business/stock-market-today-coronavirus?name=styln-coronavirus-national\&region=TOP_BANNER\&variant=undefined\&block=storyline_menu_recirc\&action=click\&pgtype=Article\&impression_id=4aab1755-e3af-11ea-aa1e-5137522d6522}{Economy}
\end{itemize}

Advertisement

\protect\hyperlink{after-top}{Continue reading the main story}

Supported by

\protect\hyperlink{after-sponsor}{Continue reading the main story}

\hypertarget{medical-workers-should-use-respirator-masks-not-surgical-masks}{%
\section{Medical Workers Should Use Respirator Masks, Not Surgical
Masks}\label{medical-workers-should-use-respirator-masks-not-surgical-masks}}

The surgical masks used in risky settings like hospitals offer much less
protection against the coronavirus, an analysis found.

\includegraphics{https://static01.graylady3jvrrxbe.onion/images/2020/06/01/science/01VIRUS-MASKS1/merlin_171764034_27de8610-56c3-4309-95cd-c6af76b18d51-articleLarge.jpg?quality=75\&auto=webp\&disable=upscale}

By
\href{https://www.nytimes3xbfgragh.onion/by/apoorva-mandavilli}{Apoorva
Mandavilli}

\begin{itemize}
\item
  Published June 1, 2020Updated June 26, 2020
\item
  \begin{itemize}
  \item
  \item
  \item
  \item
  \item
  \item
  \end{itemize}
\end{itemize}

A new analysis of 172 studies, funded by the World Health Organization,
confirms what scientists have said for months:
\href{https://www.nytimes3xbfgragh.onion/article/face-shield-mask-california-coronavirus.html}{N95}
and other respirator
\href{https://www.nytimes3xbfgragh.onion/article/face-shield-mask-california-coronavirus.html}{masks}
\href{https://doi.org/10.1016/S0140-6736(20)31142-9}{are far superior to
surgical or cloth masks} in protecting essential medical workers against
the coronavirus.

The results, published on Monday in The Lancet, make it clear that the
W.H.O. and the Centers for Disease Control and Prevention should
recommend that essential workers like nurses and emergency responders
wear
\href{https://www.nytimes3xbfgragh.onion/article/face-shield-mask-california-coronavirus.html}{N95
masks}, not just surgical masks, experts said.

``It's been disappointing that both the W.H.O. and the C.D.C. have
suggested that surgical
\href{https://www.nytimes3xbfgragh.onion/2020/06/02/health/coronavirus-face-masks-surveys.html}{masks}
are adequate, and they're clearly not,'' said David Michaels, a
professor at George Washington University who headed the Occupational
Safety and Health Administration under President Obama.

``Reliance on surgical masks has no doubt led to many workers being
infected,'' he said.

N95 masks offered 96 percent protection, the analysis found, while the
figure for surgical masks was 67 percent. The findings are particularly
important as the United States moves to reopen the economy, Dr. Michaels
said.

Workers in health care settings are not the only ones at high risk of
coronavirus infection: employees in
\href{https://www.nytimes3xbfgragh.onion/2020/05/25/business/coronavirus-meatpacking-plants-cases.html}{meatpacking
plants} and some farms are all also at high risk of coronavirus
infection and could benefit from N95 masks, he said.

The W.H.O. has not endorsed universal wearing of masks, although most of
its member governments adopted the practice months ago. The
organization's stance has frustrated many public health experts who see
masks as a simple, inexpensive and highly effective strategy to help
contain the pandemic, especially given that the virus can be transmitted
by people
\href{https://www.nytimes3xbfgragh.onion/2020/03/31/health/coronavirus-asymptomatic-transmission.html}{who
do not know they are ill}.

``When there is uncertainty and you don't know everything about a
disease, you have to be precautionary, which means you have to assume
the worst and provide the best for health care workers,'' said Dr. Raina
MacIntyre, an epidemiologist at the University of New South Wales in
Sydney, Australia, who wrote a commentary accompanying the paper.

\hypertarget{latest-updates-the-coronavirus-outbreak}{%
\section{\texorpdfstring{\href{https://www.nytimes3xbfgragh.onion/2020/08/21/world/covid-19-coronavirus.html?action=click\&pgtype=Article\&state=default\&region=MAIN_CONTENT_1\&context=storylines_live_updates}{Latest
Updates: The Coronavirus
Outbreak}}{Latest Updates: The Coronavirus Outbreak}}\label{latest-updates-the-coronavirus-outbreak}}

Updated 2020-08-21T13:00:19.184Z

\begin{itemize}
\tightlist
\item
  \href{https://www.nytimes3xbfgragh.onion/2020/08/21/world/covid-19-coronavirus.html?action=click\&pgtype=Article\&state=default\&region=MAIN_CONTENT_1\&context=storylines_live_updates\#link-6a60a19d}{`Be
  adults': Universities in the U.S. are warning students about
  gatherings as they return to campus.}
\item
  \href{https://www.nytimes3xbfgragh.onion/2020/08/21/world/covid-19-coronavirus.html?action=click\&pgtype=Article\&state=default\&region=MAIN_CONTENT_1\&context=storylines_live_updates\#link-324af071}{As
  he accepts the Democratic nomination, Biden knocks Trump's pandemic
  response.}
\item
  \href{https://www.nytimes3xbfgragh.onion/2020/08/21/world/covid-19-coronavirus.html?action=click\&pgtype=Article\&state=default\&region=MAIN_CONTENT_1\&context=storylines_live_updates\#link-191d44be}{South
  Korea threatens to detain people who obstruct virus-control efforts.}
\end{itemize}

\href{https://www.nytimes3xbfgragh.onion/2020/08/21/world/covid-19-coronavirus.html?action=click\&pgtype=Article\&state=default\&region=MAIN_CONTENT_1\&context=storylines_live_updates}{See
more updates}

More live coverage:
\href{https://www.nytimes3xbfgragh.onion/live/2020/08/21/business/stock-market-today-coronavirus?action=click\&pgtype=Article\&state=default\&region=MAIN_CONTENT_1\&context=storylines_live_updates}{Markets}

``This kind of denial, what purpose is it serving --- except to harm
health care workers?'' she said.

Previous studies have shown that the universal wearing of
\href{https://reliefweb.int/report/china/reduction-secondary-transmission-sars-cov-2-households-face-mask-use-disinfection-and}{masks
is effective}. If anything, the new study does so based on poorer
evidence, said Jeremy Howard, a distinguished research scientist at the
University of San Francisco, and a co-founder of the global \#Masks4All
movement.

``This is probably what the W.H.O. needs to change their guidance, even
though it shouldn't be,'' Mr. Howard said. ``Such is life. It will save
lives, so that's good.''

The new analysis also suggests that covering the eyes with
\href{https://www.nytimes3xbfgragh.onion/article/face-shield-mask-california-coronavirus.html}{face
shields}, goggles and glasses may provide additional safeguards for
health care workers and people in the community.

The report is among the first to lay out evidence specific to
coronaviruses, rather than extrapolating from data on other respiratory
viruses.

The coronavirus is thought to spread primarily through droplets expelled
when an infected person coughs, sneezes, sings, talks or even breathes.
Some infections may also be caused by people touching a contaminated
surface and then their mouth, nose or eyes.

Study after study has indicated that the virus can also spread via tiny
droplets less than 5 microns, called aerosols. (A micron equals one
millionth of a meter.)

But the W.H.O. has so far not acknowledged this risk and has not backed
the universal use of masks. ``There's this fear around saying airborne
--- but that's what it is,'' Dr. MacIntyre said.

A recent review in the journal Science
\href{https://science.sciencemag.org/content/early/2020/05/27/science.abc6197}{took
direct aim at the W.H.O.'s reluctance} on this issue, saying the
organization's recommendations for physical distancing and hand washing
were based on studies ``carried out in the 1930s.''

The C.D.C. did not recommend masks till April 3. Even now, its guidance
says surgical masks offer enough protection for health care workers
unless they are involved in
\href{https://www.cdc.gov/coronavirus/2019-ncov/hcp/guidance-risk-assesment-hcp.html}{procedures
that produce aerosols}.

The C.D.C. initially recommended N95 masks for all health care
personnel. Dr. MacIntyre said the agency later downgraded its
recommendations because of a shortage of N95 masks and personal
protective equipment.

``Guidelines should be based on evidence, not on supplies,'' she said.
``It's like telling an army, `Oh sorry, we've run out of guns, just take
these bows and arrows and face the enemy.'''

\href{https://www.nytimes3xbfgragh.onion/news-event/coronavirus?action=click\&pgtype=Article\&state=default\&region=MAIN_CONTENT_3\&context=storylines_faq}{}

\hypertarget{the-coronavirus-outbreak-}{%
\subsubsection{The Coronavirus Outbreak
›}\label{the-coronavirus-outbreak-}}

\hypertarget{frequently-asked-questions}{%
\paragraph{Frequently Asked
Questions}\label{frequently-asked-questions}}

Updated August 17, 2020

\begin{itemize}
\item ~
  \hypertarget{why-does-standing-six-feet-away-from-others-help}{%
  \paragraph{Why does standing six feet away from others
  help?}\label{why-does-standing-six-feet-away-from-others-help}}

  \begin{itemize}
  \tightlist
  \item
    The coronavirus spreads primarily through droplets from your mouth
    and nose, especially when you cough or sneeze. The C.D.C., one of
    the organizations using that measure,
    \href{https://www.nytimes3xbfgragh.onion/2020/04/14/health/coronavirus-six-feet.html?action=click\&pgtype=Article\&state=default\&region=MAIN_CONTENT_3\&context=storylines_faq}{bases
    its recommendation of six feet} on the idea that most large droplets
    that people expel when they cough or sneeze will fall to the ground
    within six feet. But six feet has never been a magic number that
    guarantees complete protection. Sneezes, for instance, can launch
    droplets a lot farther than six feet,
    \href{https://jamanetwork.com/journals/jama/fullarticle/2763852}{according
    to a recent study}. It's a rule of thumb: You should be safest
    standing six feet apart outside, especially when it's windy. But
    keep a mask on at all times, even when you think you're far enough
    apart.
  \end{itemize}
\item ~
  \hypertarget{i-have-antibodies-am-i-now-immune}{%
  \paragraph{I have antibodies. Am I now
  immune?}\label{i-have-antibodies-am-i-now-immune}}

  \begin{itemize}
  \tightlist
  \item
    As of right
    now,\href{https://www.nytimes3xbfgragh.onion/2020/07/22/health/covid-antibodies-herd-immunity.html?action=click\&pgtype=Article\&state=default\&region=MAIN_CONTENT_3\&context=storylines_faq}{that
    seems likely, for at least several months.} There have been
    frightening accounts of people suffering what seems to be a second
    bout of Covid-19. But experts say these patients may have a
    drawn-out course of infection, with the virus taking a slow toll
    weeks to months after initial exposure. People infected with the
    coronavirus typically
    \href{https://www.nature.com/articles/s41586-020-2456-9}{produce}
    immune molecules called antibodies, which are
    \href{https://www.nytimes3xbfgragh.onion/2020/05/07/health/coronavirus-antibody-prevalence.html?action=click\&pgtype=Article\&state=default\&region=MAIN_CONTENT_3\&context=storylines_faq}{protective
    proteins made in response to an
    infection}\href{https://www.nytimes3xbfgragh.onion/2020/05/07/health/coronavirus-antibody-prevalence.html?action=click\&pgtype=Article\&state=default\&region=MAIN_CONTENT_3\&context=storylines_faq}{.
    These antibodies may} last in the body
    \href{https://www.nature.com/articles/s41591-020-0965-6}{only two to
    three months}, which may seem worrisome, but that's perfectly normal
    after an acute infection subsides, said Dr. Michael Mina, an
    immunologist at Harvard University. It may be possible to get the
    coronavirus again, but it's highly unlikely that it would be
    possible in a short window of time from initial infection or make
    people sicker the second time.
  \end{itemize}
\item ~
  \hypertarget{im-a-small-business-owner-can-i-get-relief}{%
  \paragraph{I'm a small-business owner. Can I get
  relief?}\label{im-a-small-business-owner-can-i-get-relief}}

  \begin{itemize}
  \tightlist
  \item
    The
    \href{https://www.nytimes3xbfgragh.onion/article/small-business-loans-stimulus-grants-freelancers-coronavirus.html?action=click\&pgtype=Article\&state=default\&region=MAIN_CONTENT_3\&context=storylines_faq}{stimulus
    bills enacted in March} offer help for the millions of American
    small businesses. Those eligible for aid are businesses and
    nonprofit organizations with fewer than 500 workers, including sole
    proprietorships, independent contractors and freelancers. Some
    larger companies in some industries are also eligible. The help
    being offered, which is being managed by the Small Business
    Administration, includes the Paycheck Protection Program and the
    Economic Injury Disaster Loan program. But lots of folks have
    \href{https://www.nytimes3xbfgragh.onion/interactive/2020/05/07/business/small-business-loans-coronavirus.html?action=click\&pgtype=Article\&state=default\&region=MAIN_CONTENT_3\&context=storylines_faq}{not
    yet seen payouts.} Even those who have received help are confused:
    The rules are draconian, and some are stuck sitting on
    \href{https://www.nytimes3xbfgragh.onion/2020/05/02/business/economy/loans-coronavirus-small-business.html?action=click\&pgtype=Article\&state=default\&region=MAIN_CONTENT_3\&context=storylines_faq}{money
    they don't know how to use.} Many small-business owners are getting
    less than they expected or
    \href{https://www.nytimes3xbfgragh.onion/2020/06/10/business/Small-business-loans-ppp.html?action=click\&pgtype=Article\&state=default\&region=MAIN_CONTENT_3\&context=storylines_faq}{not
    hearing anything at all.}
  \end{itemize}
\item ~
  \hypertarget{what-are-my-rights-if-i-am-worried-about-going-back-to-work}{%
  \paragraph{What are my rights if I am worried about going back to
  work?}\label{what-are-my-rights-if-i-am-worried-about-going-back-to-work}}

  \begin{itemize}
  \tightlist
  \item
    Employers have to provide
    \href{https://www.osha.gov/SLTC/covid-19/standards.html}{a safe
    workplace} with policies that protect everyone equally.
    \href{https://www.nytimes3xbfgragh.onion/article/coronavirus-money-unemployment.html?action=click\&pgtype=Article\&state=default\&region=MAIN_CONTENT_3\&context=storylines_faq}{And
    if one of your co-workers tests positive for the coronavirus, the
    C.D.C.} has said that
    \href{https://www.cdc.gov/coronavirus/2019-ncov/community/guidance-business-response.html}{employers
    should tell their employees} -\/- without giving you the sick
    employee's name -\/- that they may have been exposed to the virus.
  \end{itemize}
\item ~
  \hypertarget{what-is-school-going-to-look-like-in-september}{%
  \paragraph{What is school going to look like in
  September?}\label{what-is-school-going-to-look-like-in-september}}

  \begin{itemize}
  \tightlist
  \item
    It is unlikely that many schools will return to a normal schedule
    this fall, requiring the grind of
    \href{https://www.nytimes3xbfgragh.onion/2020/06/05/us/coronavirus-education-lost-learning.html?action=click\&pgtype=Article\&state=default\&region=MAIN_CONTENT_3\&context=storylines_faq}{online
    learning},
    \href{https://www.nytimes3xbfgragh.onion/2020/05/29/us/coronavirus-child-care-centers.html?action=click\&pgtype=Article\&state=default\&region=MAIN_CONTENT_3\&context=storylines_faq}{makeshift
    child care} and
    \href{https://www.nytimes3xbfgragh.onion/2020/06/03/business/economy/coronavirus-working-women.html?action=click\&pgtype=Article\&state=default\&region=MAIN_CONTENT_3\&context=storylines_faq}{stunted
    workdays} to continue. California's two largest public school
    districts --- Los Angeles and San Diego --- said on July 13, that
    \href{https://www.nytimes3xbfgragh.onion/2020/07/13/us/lausd-san-diego-school-reopening.html?action=click\&pgtype=Article\&state=default\&region=MAIN_CONTENT_3\&context=storylines_faq}{instruction
    will be remote-only in the fall}, citing concerns that surging
    coronavirus infections in their areas pose too dire a risk for
    students and teachers. Together, the two districts enroll some
    825,000 students. They are the largest in the country so far to
    abandon plans for even a partial physical return to classrooms when
    they reopen in August. For other districts, the solution won't be an
    all-or-nothing approach.
    \href{https://bioethics.jhu.edu/research-and-outreach/projects/eschool-initiative/school-policy-tracker/}{Many
    systems}, including the nation's largest, New York City, are
    devising
    \href{https://www.nytimes3xbfgragh.onion/2020/06/26/us/coronavirus-schools-reopen-fall.html?action=click\&pgtype=Article\&state=default\&region=MAIN_CONTENT_3\&context=storylines_faq}{hybrid
    plans} that involve spending some days in classrooms and other days
    online. There's no national policy on this yet, so check with your
    municipal school system regularly to see what is happening in your
    community.
  \end{itemize}
\end{itemize}

She noted that N95 masks were inexpensive to produce and that other
countries, including Australia, had repurposed existing manufacturing
facilities to make the masks.

``We're not talking about making a space shuttle,'' she said. ``We're
talking about making a straightforward piece of equipment that's quite
cheap to manufacture.''

A C.D.C. spokesman was unable to address the Lancet study but said the
agency is constantly evaluating new science and ``adjusts its guidance
accordingly.''

No precaution is a guarantee against infection. But the new report
estimates how effective some of them seem to be.

Standing more than three feet away cuts the risk of transmission to 3
percent from 13 percent, the analysis found. Use of masks reduces the
odds of infection to 3 percent from 17 percent, and eye protection to 6
percent from 16 percent. The researchers emphasized that people should
also continue to wash their hands frequently.

The review also offers some practical information for disease models.
For example, it suggests that contact tracing for people who are
potentially exposed should include anyone who has been within six feet
of an infected person.

The analysis is based on observational studies conducted during the
coronavirus pandemic, as well as the SARS and MERS epidemics. It can
offer only moderate certainty --- although the study ``underplays the
results,'' Dr. Michaels said.

The evidence is strong enough to warrant changes to recommendations for
essential health care workers and to plan for mass production of N95
masks, Dr. Michaels said.

``The Trump administration needs to be less worried about producing
meat, and more about producing N95s and other P.P.E. needed to save the
lives of essential workers,'' he said.

Advertisement

\protect\hyperlink{after-bottom}{Continue reading the main story}

\hypertarget{site-index}{%
\subsection{Site Index}\label{site-index}}

\hypertarget{site-information-navigation}{%
\subsection{Site Information
Navigation}\label{site-information-navigation}}

\begin{itemize}
\tightlist
\item
  \href{https://help.nytimes3xbfgragh.onion/hc/en-us/articles/115014792127-Copyright-notice}{©~2020~The
  New York Times Company}
\end{itemize}

\begin{itemize}
\tightlist
\item
  \href{https://www.nytco.com/}{NYTCo}
\item
  \href{https://help.nytimes3xbfgragh.onion/hc/en-us/articles/115015385887-Contact-Us}{Contact
  Us}
\item
  \href{https://www.nytco.com/careers/}{Work with us}
\item
  \href{https://nytmediakit.com/}{Advertise}
\item
  \href{http://www.tbrandstudio.com/}{T Brand Studio}
\item
  \href{https://www.nytimes3xbfgragh.onion/privacy/cookie-policy\#how-do-i-manage-trackers}{Your
  Ad Choices}
\item
  \href{https://www.nytimes3xbfgragh.onion/privacy}{Privacy}
\item
  \href{https://help.nytimes3xbfgragh.onion/hc/en-us/articles/115014893428-Terms-of-service}{Terms
  of Service}
\item
  \href{https://help.nytimes3xbfgragh.onion/hc/en-us/articles/115014893968-Terms-of-sale}{Terms
  of Sale}
\item
  \href{https://spiderbites.nytimes3xbfgragh.onion}{Site Map}
\item
  \href{https://help.nytimes3xbfgragh.onion/hc/en-us}{Help}
\item
  \href{https://www.nytimes3xbfgragh.onion/subscription?campaignId=37WXW}{Subscriptions}
\end{itemize}
