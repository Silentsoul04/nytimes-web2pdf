Sections

SEARCH

\protect\hyperlink{site-content}{Skip to
content}\protect\hyperlink{site-index}{Skip to site index}

\href{https://www.nytimes3xbfgragh.onion/section/technology}{Technology}

\href{https://myaccount.nytimes3xbfgragh.onion/auth/login?response_type=cookie\&client_id=vi}{}

\href{https://www.nytimes3xbfgragh.onion/section/todayspaper}{Today's
Paper}

\href{/section/technology}{Technology}\textbar{}Zuckerberg Defends
Hands-Off Approach to Trump's Posts

\href{https://nyti.ms/300IMsV}{https://nyti.ms/300IMsV}

\begin{itemize}
\item
\item
\item
\item
\item
\item
\end{itemize}

\href{https://www.nytimes3xbfgragh.onion/news-event/george-floyd-protests-minneapolis-new-york-los-angeles?action=click\&pgtype=Article\&state=default\&region=TOP_BANNER\&context=storylines_menu}{Race
and America}

\begin{itemize}
\tightlist
\item
  \href{https://www.nytimes3xbfgragh.onion/2020/07/26/us/protests-portland-seattle-trump.html?action=click\&pgtype=Article\&state=default\&region=TOP_BANNER\&context=storylines_menu}{Protesters
  Return to Other Cities}
\item
  \href{https://www.nytimes3xbfgragh.onion/2020/07/24/us/portland-oregon-protests-white-race.html?action=click\&pgtype=Article\&state=default\&region=TOP_BANNER\&context=storylines_menu}{Portland
  at the Center}
\item
  \href{https://www.nytimes3xbfgragh.onion/2020/07/23/podcasts/the-daily/portland-protests.html?action=click\&pgtype=Article\&state=default\&region=TOP_BANNER\&context=storylines_menu}{Podcast:
  Showdown in Portland}
\item
  \href{https://www.nytimes3xbfgragh.onion/interactive/2020/07/16/us/black-lives-matter-protests-louisville-breonna-taylor.html?action=click\&pgtype=Article\&state=default\&region=TOP_BANNER\&context=storylines_menu}{45
  Days in Louisville}
\end{itemize}

Advertisement

\protect\hyperlink{after-top}{Continue reading the main story}

Supported by

\protect\hyperlink{after-sponsor}{Continue reading the main story}

\hypertarget{zuckerberg-defends-hands-off-approach-to-trumps-posts}{%
\section{Zuckerberg Defends Hands-Off Approach to Trump's
Posts}\label{zuckerberg-defends-hands-off-approach-to-trumps-posts}}

In a call with Facebook employees, who have protested the inaction on
Mr. Trump's messages, Mr. Zuckerberg said his decision was ``pretty
thorough.''

\includegraphics{https://static01.graylady3jvrrxbe.onion/images/2020/06/03/business/03facebook/03facebook-articleLarge.jpg?quality=75\&auto=webp\&disable=upscale}

\href{https://www.nytimes3xbfgragh.onion/by/mike-isaac}{\includegraphics{https://static01.graylady3jvrrxbe.onion/images/2018/02/16/multimedia/author-mike-isaac/author-mike-isaac-thumbLarge.jpg}}\href{https://www.nytimes3xbfgragh.onion/by/cecilia-kang}{\includegraphics{https://static01.graylady3jvrrxbe.onion/images/2019/01/29/multimedia/author-cecilia-kang/author-cecilia-kang-thumbLarge.png}}\href{https://www.nytimes3xbfgragh.onion/by/sheera-frenkel}{\includegraphics{https://static01.graylady3jvrrxbe.onion/images/2018/06/14/multimedia/author-sheera-frenkel/author-sheera-frenkel-thumbLarge.png}}

By \href{https://www.nytimes3xbfgragh.onion/by/mike-isaac}{Mike Isaac},
\href{https://www.nytimes3xbfgragh.onion/by/cecilia-kang}{Cecilia Kang}
and \href{https://www.nytimes3xbfgragh.onion/by/sheera-frenkel}{Sheera
Frenkel}

\begin{itemize}
\item
  Published June 2, 2020Updated June 3, 2020
\item
  \begin{itemize}
  \item
  \item
  \item
  \item
  \item
  \item
  \end{itemize}
\end{itemize}

SAN FRANCISCO --- Mark Zuckerberg, Facebook's chief executive, on
Tuesday stood firmly behind his decision not to do anything about
\href{https://www.nytimes3xbfgragh.onion/2020/06/02/us/politics/trump-bible-photo-op.html}{President
Trump's} inflammatory posts on the social network, saying that he had
made a ``tough decision'' but that it ``was pretty thorough.''

In a question-and-answer session with employees conducted over video
chat software, Mr. Zuckerberg sought to justify his position, which has
led to fierce internal dissent. The meeting, which had been scheduled
for Thursday, was moved up to Tuesday after hundreds of
\href{https://www.nytimes3xbfgragh.onion/2020/06/01/technology/facebook-employee-protest-trump.html}{employees
protested the inaction by staging a virtual ``walkout''} on Monday.

Facebook's principles and policies supporting free speech ``show that
the right action where we are right now is to leave this up,'' Mr.
Zuckerberg said on the call referring to Mr. Trump's posts. The audio of
the employee call was heard by The New York Times.

Mr. Zuckerberg said that though he knew many people would be upset with
Facebook, a policy review backed up his decision. He added that after he
made his determination, he received a phone call from President Trump on
Friday.

``I used that opportunity to make him know I felt this post was
inflammatory and harmful, and let him know where we stood on it,'' Mr.
Zuckerberg told Facebook employees. But though he voiced displeasure to
the president, he reiterated that Mr. Trump's message did not break the
social network's guidelines.

The Facebook chief held firm even as the pressure on him to rein in Mr.
Trump's messages intensified. Civil rights groups said late Monday after
meeting with Mr. Zuckerberg and Sheryl Sandberg, Facebook's chief
operating officer, that it was ``totally confounding'' that the company
was not taking a tougher stand on Mr. Trump's posts, which are often
aggressive and have heightened tensions over
\href{https://www.nytimes3xbfgragh.onion/news-event/george-floyd-protests-minneapolis-new-york-los-angeles?action=click\&pgtype=Article\&state=default\&module=styln-george-floyd\&variant=show\&region=TOP_BANNER\&context=storylines_menu}{protests
on police violence} in recent days.

Several Facebook employees have resigned over the lack of action, with
one publicly saying the company would end up ``on the wrong side of
history.'' And protesters showed up late Monday
to\href{https://padailypost.com/2020/06/02/nighttime-protest-blocked-near-police-station-demonstrators-also-go-to-zuckerbergs-house/}{Mr.
Zuckerberg's residential neighborhood} in Palo Alto, Calif., and also
headed toward the social network's headquarters in nearby Menlo Park.

The internal dissent began brewing last week after Facebook's rival,
Twitter, added labels to Mr.
\href{https://www.nytimes3xbfgragh.onion/2020/06/03/us/politics/trump-twitter-fact-check.html}{Trump's
tweets} that indicated
\href{https://www.nytimes3xbfgragh.onion/2020/05/29/technology/trump-twitter-minneapolis-george-floyd.html}{the
president was glorifying violence} and making inaccurate statements. The
same messages that Mr. Trump posted to Twitter also appeared on
Facebook. But unlike Twitter,
\href{https://www.nytimes3xbfgragh.onion/2020/05/29/technology/twitter-facebook-zuckerberg-trump.html}{Facebook
did not touch the president's posts}, including one in which Mr. Trump
said of the protests in Minneapolis: ``when the
\href{https://www.nytimes3xbfgragh.onion/2020/05/29/us/looting-starts-shooting-starts.html}{looting
starts, the shooting starts}.''

That decision led to internal criticism, with Facebook employees arguing
it was untenable to leave up Mr. Trump's messages that incited violence.
They said Mr. Zuckerberg was kowtowing to Republicans out of fear of
Facebook being regulated or broken up.

Mr. Zuckerberg and Ms. Sandberg have spent the past few days meeting
with employees, civil rights leaders and other angry parties to explain
the company's stance. Mr. Zuckerberg has said Facebook does not want to
be an ``arbiter of truth.'' He has also said that he stands for free
speech and that what world leaders post online is in the public interest
and newsworthy.

But in trying to placate everyone, Mr. Zuckerberg has failed to appease
almost anyone. Facebook employees have continued criticizing their
employer on Twitter, LinkedIn and on their personal Facebook pages. Some
circulated petitions calling for change. On Monday, hundreds of workers
participated in the virtual ``walkout'' by refusing to work and setting
their automated messages to one of protest.

Timothy Aveni, a Facebook software engineer who resigned after Mr.
Zuckerberg's decision to leave up Mr. Trump's posts, said on his
\href{https://www.facebookcorewwwi.onion/timothy.j.aveni/posts/3006224359465567}{Facebook
page} on Monday that the company wasn't enforcing its own rules to ban
speech that promotes violence.

``Facebook will keep moving the goalposts every time Trump escalates,
finding excuse after excuse not to act on increasingly dangerous
rhetoric,'' Mr. Aveni said.

\includegraphics{https://static01.graylady3jvrrxbe.onion/images/2020/06/02/technology/02facebook3/oakImage-1591120039942-articleLarge.jpg?quality=75\&auto=webp\&disable=upscale}

Politicians and civil rights organizations have also taken issue with
Mr. Zuckerberg's position.

On Monday evening, Vanita Gupta, who heads the Leadership Conference on
Civil and Human Rights, took part in a one-hour phone call with Mr.
Zuckerberg, Ms. Sandberg and other Facebook officials. Afterward, she
said Mr. Zuckerberg ``betrayed a lack of understanding'' and compared
Facebook's inaction on Mr. Trump's posts to its inaction in Myanmar and
the Philippines, where military and government leaders have
\href{https://www.nytimes3xbfgragh.onion/2018/10/15/technology/myanmar-facebook-genocide.html}{used
Facebook to spread disinformation and provoke violence}.

Later that evening, Ms. Sandberg posted on Facebook's internal message
board and described the conversation with civil rights leaders as a
``hard but meaningful'' one, according to a copy of the message viewed
by The Times.

Image

Vanita Gupta, who heads the Leadership Conference on Civil and Human
Rights, said Mr. Zuckerberg's inaction on President Trump's posts was
``totally confounding.''Credit...T.J. Kirkpatrick for The New York Times

On Tuesday in the virtual meeting with employees, Mr. Zuckerberg spent
30 minutes laying out what had happened with Mr. Trump's posts. He said
the president's looting-and-shooting message, which went up on Friday,
was immediately spotted by Facebook's policy team. Mr. Zuckerberg woke
up at 7:30 a.m. in Palo Alto that day to an email about the post. The
policy team called the White House, he said, telling officials there
that Mr. Trump's message was inflammatory.

Mr. Zuckerberg spent the rest of last Friday morning talking to policy
officials and other experts at Facebook. He ultimately decided Mr.
Trump's post had not broken Facebook's policies.

Mr. Zuckerberg said Mr. Trump's post relied on a call for ``state use of
force,'' which Facebook allows under its guidelines. He said that in the
future, the social network might reassess that policy, given the photos
and videos of excessive use of force by police that have spread across
social media in recent days.

After explaining his thought process, Mr. Zuckerberg took questions from
employees in the virtual meeting on Tuesday, according to a copy of the
call. One Facebook employee in New York expressed support for Mr.
Zuckerberg's position. But the vast majority of questions were pointed
and the call became increasingly contentious.

Mr. Zuckerberg was asked whether any black Facebook employees were
consulted in the decision-making process. He named one. A Facebook
employee in Austin, Texas, then said that he felt the company's
political speech policy wasn't working and needed to be changed.

One persistent feeling shared among Facebook's rank-and-file came out in
a direct moment between Mr. Zuckerberg and another employee during the
call.

``Why are the smartest people in the world focused on contorting and
twisting our policies to avoid antagonizing Trump?'' the employee asked.

In a statement, a Facebook spokeswoman said that ``open and honest
discussion has always been a part of Facebook's culture,'' and that Mr.
Zuckerberg was ``grateful'' for employees' feedback.

The call did little to soothe the feelings of employees. More than a
dozen current and former Facebook employees said the call only deepened
the frictions inside the company; some said that trying to persuade Mr.
Zuckerberg to change his mind was futile.

``It's crystal clear today that leadership refuses to stand with us,''
Brandon Dail, a Facebook engineer,
\href{https://twitter.com/aweary/status/1267885674805354498}{tweeted}
about the call.

Mike Isaac reported from San Francisco, Cecilia Kang from Washington and
Sheera Frenkel from Oakland, Calif.

Advertisement

\protect\hyperlink{after-bottom}{Continue reading the main story}

\hypertarget{site-index}{%
\subsection{Site Index}\label{site-index}}

\hypertarget{site-information-navigation}{%
\subsection{Site Information
Navigation}\label{site-information-navigation}}

\begin{itemize}
\tightlist
\item
  \href{https://help.nytimes3xbfgragh.onion/hc/en-us/articles/115014792127-Copyright-notice}{©~2020~The
  New York Times Company}
\end{itemize}

\begin{itemize}
\tightlist
\item
  \href{https://www.nytco.com/}{NYTCo}
\item
  \href{https://help.nytimes3xbfgragh.onion/hc/en-us/articles/115015385887-Contact-Us}{Contact
  Us}
\item
  \href{https://www.nytco.com/careers/}{Work with us}
\item
  \href{https://nytmediakit.com/}{Advertise}
\item
  \href{http://www.tbrandstudio.com/}{T Brand Studio}
\item
  \href{https://www.nytimes3xbfgragh.onion/privacy/cookie-policy\#how-do-i-manage-trackers}{Your
  Ad Choices}
\item
  \href{https://www.nytimes3xbfgragh.onion/privacy}{Privacy}
\item
  \href{https://help.nytimes3xbfgragh.onion/hc/en-us/articles/115014893428-Terms-of-service}{Terms
  of Service}
\item
  \href{https://help.nytimes3xbfgragh.onion/hc/en-us/articles/115014893968-Terms-of-sale}{Terms
  of Sale}
\item
  \href{https://spiderbites.nytimes3xbfgragh.onion}{Site Map}
\item
  \href{https://help.nytimes3xbfgragh.onion/hc/en-us}{Help}
\item
  \href{https://www.nytimes3xbfgragh.onion/subscription?campaignId=37WXW}{Subscriptions}
\end{itemize}
