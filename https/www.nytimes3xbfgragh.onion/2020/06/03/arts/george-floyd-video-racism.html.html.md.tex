Sections

SEARCH

\protect\hyperlink{site-content}{Skip to
content}\protect\hyperlink{site-index}{Skip to site index}

\href{https://www.nytimes3xbfgragh.onion/section/arts}{Arts}

\href{https://myaccount.nytimes3xbfgragh.onion/auth/login?response_type=cookie\&client_id=vi}{}

\href{https://www.nytimes3xbfgragh.onion/section/todayspaper}{Today's
Paper}

\href{/section/arts}{Arts}\textbar{}The Videos That Rocked America. The
Song That Knows Our Rage.

\url{https://nyti.ms/2MuoKPq}

\begin{itemize}
\item
\item
\item
\item
\item
\end{itemize}

\href{https://www.nytimes3xbfgragh.onion/news-event/george-floyd-protests-minneapolis-new-york-los-angeles?action=click\&pgtype=Article\&state=default\&region=TOP_BANNER\&context=storylines_menu}{Race
and America}

\begin{itemize}
\tightlist
\item
  \href{https://www.nytimes3xbfgragh.onion/2020/07/26/us/protests-portland-seattle-trump.html?action=click\&pgtype=Article\&state=default\&region=TOP_BANNER\&context=storylines_menu}{Protesters
  Return to Other Cities}
\item
  \href{https://www.nytimes3xbfgragh.onion/2020/07/24/us/portland-oregon-protests-white-race.html?action=click\&pgtype=Article\&state=default\&region=TOP_BANNER\&context=storylines_menu}{Portland
  at the Center}
\item
  \href{https://www.nytimes3xbfgragh.onion/2020/07/23/podcasts/the-daily/portland-protests.html?action=click\&pgtype=Article\&state=default\&region=TOP_BANNER\&context=storylines_menu}{Podcast:
  Showdown in Portland}
\item
  \href{https://www.nytimes3xbfgragh.onion/interactive/2020/07/16/us/black-lives-matter-protests-louisville-breonna-taylor.html?action=click\&pgtype=Article\&state=default\&region=TOP_BANNER\&context=storylines_menu}{45
  Days in Louisville}
\end{itemize}

Advertisement

\protect\hyperlink{after-top}{Continue reading the main story}

Supported by

\protect\hyperlink{after-sponsor}{Continue reading the main story}

Critic's Notebook

\hypertarget{the-videos-that-rocked-america-the-song-that-knows-our-rage}{%
\section{The Videos That Rocked America. The Song That Knows Our
Rage.}\label{the-videos-that-rocked-america-the-song-that-knows-our-rage}}

Awash in the ghastly video mosaic shot by black people's cameraphones, I
found myself doubled over the kitchen sink. Then a lyric gave me
strength.

\includegraphics{https://static01.graylady3jvrrxbe.onion/images/2020/06/04/arts/03unrest-morris1/merlin_173080962_76e5fcbc-802d-4956-aea5-80a085c640d3-articleLarge.jpg?quality=75\&auto=webp\&disable=upscale}

\href{https://www.nytimes3xbfgragh.onion/by/wesley-morris}{\includegraphics{https://static01.graylady3jvrrxbe.onion/images/2018/06/13/multimedia/author-wesley-morris/author-wesley-morris-thumbLarge.jpg}}

By \href{https://www.nytimes3xbfgragh.onion/by/wesley-morris}{Wesley
Morris}

\begin{itemize}
\item
  Published June 3, 2020Updated June 5, 2020
\item
  \begin{itemize}
  \item
  \item
  \item
  \item
  \item
  \end{itemize}
\end{itemize}

The most urgent filmmaking anybody's doing in this country right now is
by black people with camera phones. Their work comprises a ghastly
visual mosaic of mistreatment, at best, and whose victims are
international symbols of mourning:
\href{https://www.nytimes3xbfgragh.onion/2015/06/14/nyregion/eric-garner-police-chokehold-staten-island.html}{Eric
Garner},
\href{https://www.nytimes3xbfgragh.onion/2017/06/20/us/police-shooting-castile-trial-video.html}{Philando
Castile},
\href{https://www.nytimes3xbfgragh.onion/2019/05/07/us/sandra-bland-video-brian-encinia.html}{Sandra
Bland}. Art is not the intent. These videos are the stone truth. Quaking
proof of insult, seasick funerals. Livestreamed or uploaded, or
suppressed then suspiciously unearthed as found footage. Last week, the
archive grew by two, and now the nation's roiling.

First, a white dog walker called the police on
\href{https://www.nytimes3xbfgragh.onion/2020/05/27/nyregion/amy-cooper-christian-central-park-video.html}{Christian
Cooper}, a birder in Central Park. Her unrestrained dog disturbed this
man's peace. He asked that the dog be leashed (the park's rules, not
his) and its affronted owner told him that if he didn't stop recording
their interaction she'd tell the police that an ``African-American man
is threatening my life.'' He kept rolling --- actually, he kept
directing. ``Please call the cops,'' he calmly instructs her. ``Please
tell them whatever you like.'' And she does.

But it was how she told on him that you don't forget. Three times, she
informs the dispatcher that this man is her emergency. By the last
round, she's made herself hysterical. The person at the other end can
hear distress and can probably sense that the greater victim in this
exchange might be her mewling dog, choking because the grip on its
collar is so tight. When the call is over, Cooper thanks her --- for
leashing the dog, but for also endangering him, for living down to
herself, for quite a performance of umbrage. This woman has dialed 911,
but she's also got access to an ancient American network of
interpersonal fraud. She knows the advantage of her role. So, of course,
did he. That's why his camera's on. In case. The video would be the
counterfactual that might save his life.

That's not how it went for George Floyd, that same Monday, 1,200 miles
away. In this
\href{https://www.nytimes3xbfgragh.onion/2020/05/31/us/george-floyd-investigation.html}{video
footage}, captured, in part, by Darnella Frazier, the Minneapolis police
officers who bore down on his body appeared immune to the cameras
trained on them, immune to his gasping pleas for air and his mother.
There is a madness in how calm they appear as a grown man rasps and
begs. For stretches, one officer stands there like an inanimate object
that refuses to hear the bystanders beseeching on Floyd's behalf.
Another officer had rested his knee in the man's nape, comfortably. For
more than eight minutes --- eight of George Floyd's last --- it barely
moves, as though that's what God intended napes to be, a kneeler.

The Cooper and Floyd videos capture ancestral false alarms and
overreactions, centuries-old hatreds, miserable inequalities. (The dog
in one video fares better than the human being in the other.) These
videos are part of some newish optic tradition that dates back, at
least, to those abstract camcorder images of the L.A.P.D. going to town
on
\href{https://www.nytimes3xbfgragh.onion/1991/03/18/us/seven-minutes-los-angeles-special-report-videotaped-beating-officers-puts-full.html}{Rodney
King}, stories black people share to keep one another safe and warn
others; bystander evidence, filmed by all kinds of people, that has to
embarrass the wheels of justice into their slow grind. It's video that
is currently breaking open the United States once again.

\includegraphics{https://static01.graylady3jvrrxbe.onion/images/2020/06/03/arts/03unrest-morris/03unrest-morris-articleLarge.jpg?quality=75\&auto=webp\&disable=upscale}

This country manufactures only one product powerful enough to interrupt
the greatest health and economic crisis it's probably ever faced. We
make racism, the American virus and the underlying condition of black
woe. And the rage against it is strong enough to compel people to risk
catching one disease in order to combat the other --- in scores and
scores of American cities, in cities around the world. They're a tandem
now, the pandemic bold-underlining-italicizing what's endemic to us. The
underfunded hospitals, appalling factory conditions and unequal
education were readily evident last year, before Covid-19. Now, the
inadequacies and inequalities expedite death and compound estrangement.
The low-wage workers have been deemed essential yet remain paid
inessentially. The numbers of black, Latino and Indigenous people
infected, deceased and unemployed are out of whack with their share of
the population. And the president has yet to offer his condolences, in
earnest.

So maybe we were due for another round of unrest and conflagration.
Maybe, Black Lives Matter and Colin Kaepernick were simply ahead of
their time, even if the calls for respect that they marched and knelt
for remain absurdly longstanding. This explosion seems meant to occur in
the year in which we saw a video of a 25-year-old black man, a
\href{https://www.nytimes3xbfgragh.onion/article/ahmaud-arbery-shooting-georgia.html}{runner
named Ahmaud Arbery}, chased down by white men in Georgia and shot dead,
men who went on about their lives for months after.

It had to happen in a year in which police killed a sleeping black
woman,
\href{https://www.nytimes3xbfgragh.onion/article/breonna-taylor-police.html}{Breonna
Taylor}, wanted on suspicion of nothing. It had to be in the year of
lockdowns, masks, in-a-blink job loss and funerals no one could
physically attend. It had to be the year whose numbers refer to perfect
vision. People could, perhaps, see anew that, when it comes to certain
white people, what we call freedom is basically impunity. Freedom-plus.
Americans have watched that plus burn outside their homes.

Impunity permits politicians and TV hosts to lie about whatever and the
police to
\href{https://www.nytimes3xbfgragh.onion/2020/05/31/us/police-tactics-floyd-protests.html}{shoot
rubber bullets} at nonprotesters as if they were squirrels. Impunity is
what brings black men as different and differently eloquent as the
rapper-activist
\href{https://www.youtube.com/watch?v=sG0yrng0eY4\&feature=emb_logo}{Killer
Mike} and the Princeton professor
\href{https://www.youtube.com/watch?v=6ylb44LkG9A}{Dr. Eddie Glaude Jr.}
to anger and the verge of tears, in separate appearances before the
media on Friday. They both embodied a sentiment of the protests.
\emph{We've been trying to make this country great, but you won't let
us.}

Black Americans have come in peace, they've come armed. They've just
been trying to mind their business. Disappointment awaits, regardless.
Anytime the racial temperature goes up and hell pays a visit to earth,
the disappointment takes a holiday. And you fight. You fight because
you're tired. Yet you're tired because you've been fighting. For so
long. In waves, in loops, in vacuums, in vain.

Image

Patti LaBelle performing in Illinois in the 1980s.Credit...Paul
Natkin/Getty Images

I suppose this is all how I found myself doubled over the kitchen sink
on Sunday, bawling into a bowl of greens, a knife in one hand, the other
gathered into the loneliest fist that hand had ever made. I was doubled
over because Patti LaBelle had wrecked me.

Now, somehow, Patti's not for everybody. And I don't mean white people
(although I've heard the complaints). My mother was a black woman from
Philadelphia just like Patti and her feelings remained mixed. Not
something you need to know about my otherwise perfect mother; it's just
to say that Patti LaBelle is an unsettled matter. And her unorthodoxy
--- as a cookbook author, a vocalist and someone who believes that a
black woman's hair ought to be a wonder of the world --- makes her all
the more beloved to her partisans.

The last song on her 1985 album, ``Patti,'' is a live cover of a classic
written by \href{https://www.youtube.com/watch?v=PiOgnUe-T6A}{Kenny
Gamble and Leon Huff} and released in 1972: ``If You Don't Know Me by
Now.'' Harold Melvin \& the Blue Notes recorded the raw, plangent
definitive version; Simply Red the comparatively subdued hit 1989
incarnation. Both are excellent. It's a perfect song that
\href{https://www.youtube.com/watch?v=Tain2OF3Zyg}{LaBelle moves into
and refurnishes}. She flips and flexes every syllable. The word ``eye''
is elongated so that suddenly there's an ``o'' in there. Pure Patti. The
chorus --- ``if you don't know me by now, you will never, never, never
know me'' --- is left to the men doing her backing vocals, while LaBelle
spreads icing all over their cake.

This version is arranged as a march that keeps cresting: big drums,
wagging piano, bass that vamps. After four minutes, the band offers her
a clearing to do some trapeze work. And this is the moment --- in the
middle of a pandemic, with the country in some of its worst-ever shape,
with protesters on my street damning the police, with black America at
yet another wit's end --- that I heard a song I've listened to a hundred
times like I've never heard it before.

``I thought you knew me by now,'' she sings, ``but you don't.'' She's
off-book as they say, working on mood, instinct and fatigue. ``Heh,
heh,'' she says with a weary laugh before she starts talking to the
room.

``You break your back, you break your legs and you break your face,
trying to make these people know you in life. But somehow they just
don't wanna \emph{try} to,'' she says and goes on to wonder, ``Is it the
way I look?'' In her story, she's talking about a man and has a mirror
moment. ``Self,'' she asks, ``is it worth it?'' And, in her four-alarm
soprano, she lets out a naying, ``Uh-uh,'' only with more ``uhs'' than I
could count. ``I'm not going to try to prove myself no more,'' she
proclaims as her singers back her, firm yet softly, with that chorus
(``never, never, never''). For half of the remaining 90 seconds, she is
knocking everything off the emotional table she'd spent the previous
five minutes setting.

This is not a protest anthem. It's a lovers-at-a-crossroads jam. But
LaBelle is working this crowd. She's preaching about something that, at
my sink, at a crossroads, along with millions of other black people,
sounded like a much bigger love. Her exasperation felt transcendently
real and timelessly final. \emph{Enough.}

I heard a woman declaring her value. George Floyd was suspected of
having used a counterfeit bill at a corner store, which means his life
was worth less than money. I heard her thinking through an ultimatum now
being laid down in the streets of this country. You still think we're
monkeys, monsters, beasts, thugs, the living dead, minorities? If you
don't know that a black man, calling for his mother, his dead mother, is
so desperate for somebody to hear him that he's screaming for ghosts ---
or fears he's in the process of becoming one; if you don't know that we,
too, can run for leisure and sleep for rest; if you don't know that this
skin is neither your emergency nor an excuse to invent one, that the
emergency has tended to be you --- \emph{by now}? --- you will never,
never, never \ldots{}

Advertisement

\protect\hyperlink{after-bottom}{Continue reading the main story}

\hypertarget{site-index}{%
\subsection{Site Index}\label{site-index}}

\hypertarget{site-information-navigation}{%
\subsection{Site Information
Navigation}\label{site-information-navigation}}

\begin{itemize}
\tightlist
\item
  \href{https://help.nytimes3xbfgragh.onion/hc/en-us/articles/115014792127-Copyright-notice}{©~2020~The
  New York Times Company}
\end{itemize}

\begin{itemize}
\tightlist
\item
  \href{https://www.nytco.com/}{NYTCo}
\item
  \href{https://help.nytimes3xbfgragh.onion/hc/en-us/articles/115015385887-Contact-Us}{Contact
  Us}
\item
  \href{https://www.nytco.com/careers/}{Work with us}
\item
  \href{https://nytmediakit.com/}{Advertise}
\item
  \href{http://www.tbrandstudio.com/}{T Brand Studio}
\item
  \href{https://www.nytimes3xbfgragh.onion/privacy/cookie-policy\#how-do-i-manage-trackers}{Your
  Ad Choices}
\item
  \href{https://www.nytimes3xbfgragh.onion/privacy}{Privacy}
\item
  \href{https://help.nytimes3xbfgragh.onion/hc/en-us/articles/115014893428-Terms-of-service}{Terms
  of Service}
\item
  \href{https://help.nytimes3xbfgragh.onion/hc/en-us/articles/115014893968-Terms-of-sale}{Terms
  of Sale}
\item
  \href{https://spiderbites.nytimes3xbfgragh.onion}{Site Map}
\item
  \href{https://help.nytimes3xbfgragh.onion/hc/en-us}{Help}
\item
  \href{https://www.nytimes3xbfgragh.onion/subscription?campaignId=37WXW}{Subscriptions}
\end{itemize}
