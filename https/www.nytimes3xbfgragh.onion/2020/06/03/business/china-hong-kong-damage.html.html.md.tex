Sections

SEARCH

\protect\hyperlink{site-content}{Skip to
content}\protect\hyperlink{site-index}{Skip to site index}

\href{https://www.nytimes3xbfgragh.onion/section/business}{Business}

\href{https://myaccount.nytimes3xbfgragh.onion/auth/login?response_type=cookie\&client_id=vi}{}

\href{https://www.nytimes3xbfgragh.onion/section/todayspaper}{Today's
Paper}

\href{/section/business}{Business}\textbar{}Why China May Call the
World's Bluff on Hong Kong

\url{https://nyti.ms/2A06aMx}

\begin{itemize}
\item
\item
\item
\item
\item
\item
\end{itemize}

Advertisement

\protect\hyperlink{after-top}{Continue reading the main story}

Supported by

\protect\hyperlink{after-sponsor}{Continue reading the main story}

\hypertarget{why-china-may-call-the-worlds-bluff-on-hong-kong}{%
\section{Why China May Call the World's Bluff on Hong
Kong}\label{why-china-may-call-the-worlds-bluff-on-hong-kong}}

The U.S. looks weak. Business is falling in line. Protests have been
muted. For Beijing, the damage to the city and its own reputation from
seizing greater control may be worth it.

\includegraphics{https://static01.graylady3jvrrxbe.onion/images/2020/06/03/world/03hongkong-damages-1/merlin_172955151_6b092b78-7e3c-4d4c-bc2c-febab2e39129-articleLarge.jpg?quality=75\&auto=webp\&disable=upscale}

\href{https://www.nytimes3xbfgragh.onion/by/alexandra-stevenson}{\includegraphics{https://static01.graylady3jvrrxbe.onion/images/2018/02/20/multimedia/author-alexandra-stevenson/author-alexandra-stevenson-thumbLarge.jpg}}\href{https://www.nytimes3xbfgragh.onion/by/vivian-wang}{\includegraphics{https://static01.graylady3jvrrxbe.onion/images/2018/06/14/multimedia/author-vivian-wang/author-vivian-wang-thumbLarge-v2.png}}

By
\href{https://www.nytimes3xbfgragh.onion/by/alexandra-stevenson}{Alexandra
Stevenson} and
\href{https://www.nytimes3xbfgragh.onion/by/vivian-wang}{Vivian Wang}

\begin{itemize}
\item
  June 3, 2020
\item
  \begin{itemize}
  \item
  \item
  \item
  \item
  \item
  \item
  \end{itemize}
\end{itemize}

\href{https://cn.nytimes3xbfgragh.onion/china/20200604/china-hong-kong-damage/}{阅读简体中文版}\href{https://cn.nytimes3xbfgragh.onion/china/20200604/china-hong-kong-damage/zh-hant/}{閱讀繁體中文版}

HONG KONG --- China long depended on
\href{https://www.nytimes3xbfgragh.onion/2020/06/29/business/economy/us-halts-high-tech-exports-hong-kong.html}{Hong
Kong} to be everything it was not. The city's freewheeling capitalism
and personal freedoms, both absent from the mainland, made it one of the
world's premier financial hubs. Together, they flourished for decades.

Now China is doing what was once unthinkable: imposing its will on Hong
Kong in a way that could permanently damage the former British colony
economically and politically. In pushing for a new
\href{https://www.nytimes3xbfgragh.onion/2020/07/07/business/hong-kong-security-law-tech.html}{national
security law} that many fear will curtail the city's liberties, the
Chinese Communist Party is calculating that control and stability
outweigh the benefits the city has long provided.

Other countries are threatening to retaliate in ways that could leave
Hong Kong a shadow of its former self. The United States has
\href{https://www.nytimes3xbfgragh.onion/2020/05/29/us/politics/trump-hong-kong-china-WHO.html}{vowed
to end the special economic treatment} it has long granted the
territory. Britain has said it could open its doors to three million
Hong Kongers, laying the groundwork for a severe brain drain.

But Beijing sees its position as strong while the rest of the world is
divided and still recovering from the coronavirus pandemic. The United
States will hurt itself more by coming down hard against Hong Kong,
officials believe. Hong Kong's protest movement, at least for the
moment, seems demoralized.

And when it comes to the global economy, the Communist Party is wagering
that the world needs China, with or without Hong Kong. The response of
the business community has been muted so far. Even if it protested,
business has always come back to China, whether in the wake of the 1989
Tiananmen Square crackdown or the British handover of Hong Kong back to
China in 1997.

\includegraphics{https://static01.graylady3jvrrxbe.onion/images/2020/06/03/world/03hongkong-damages-2/merlin_172782960_abb00e92-df4b-433f-b70f-3024b2b7c353-articleLarge.jpg?quality=75\&auto=webp\&disable=upscale}

``There will be some unhappy people for some time,'' said John L.
Thornton, a former president of Goldman Sachs who has longstanding ties
with China's leadership. ``But the drum rolls, the dogs bark and the
caravan moves on. That's the political judgment. They have had a fair
amount of empirical evidence that the concerns will disappear.''

On Wednesday, HSBC said Peter Wong, its Asia-Pacific chief, had signed a
petition supporting the national security law.

Unquestionably, Hong Kong has declined in importance to China as the
mainland economy has surged. In 1997, Hong Kong's economic output was
nearly one-fifth the mainland's, making it a necessary growth engine for
Beijing. Deng Xiaoping, then China's top leader, had agreed to allow
Hong Kong to keep its business and personal freedoms for decades to
come, saying
\href{https://www.nytimes3xbfgragh.onion/1997/06/30/world/a-partly-alien-place-joins-china-today.html}{years
earlier} that ``there was no other possible solution.''

Today, Hong Kong's output is less than 3 percent of the mainland's.
While investors still prize Hong Kong's rule of law, low taxes and
transparent business environment, they have grown more accustomed to
doing business in mainland cities like Shanghai, where the stock market
is bigger than Hong Kong's by value.

Nevertheless, Washington believes Hong Kong is still too valuable for
China to jeopardize.

President Trump said last week that he would strip Hong Kong of the
special status granted to it by Washington. Depending on what he does,
it could subject Hong Kong to the same tariffs and trade restrictions
imposed on mainland China.

If the United States wants to raise the stakes sharply, it could harness
one of its major strengths: its vital role over the global financial
system.

Image

Protesters in Hong Kong's Causeway Bay area last week. Sporadic protests
against the law have been met with strong police responses.Credit...Lam
Yik Fei for The New York Times

China relies heavily on Hong Kong's unlimited access to U.S. dollars,
the world's de facto currency. China tightly limits the amount of its
currency that flows past its borders, making the Chinese renminbi less
useful in making global payments and loans, striking deals or
participating in international finance. About three-quarters of all
renminbi payments flow through Hong Kong, according to data from the
Society for Worldwide Interbank Financial Telecommunication, a network
that facilitates global financial transactions.

American retaliation may be enough to get many businesses to leave. In a
\href{https://amcham.org.hk/uploads/media/default/0001/08/e63cc524fe6b2d94c1b12b01c6249e144a394a9a.pdf}{survey}
released on Wednesday by the American Chamber of Commerce in Hong Kong,
more than a quarter of companies questioned said they were considering
moving elsewhere.

Individuals may leave, too. The British government, which says the
national security law violates the handover agreement, said it would
offer a path to citizenship to nearly three million Hong Kong residents
--- almost half the city's population --- if China proceeded.

``This would amount to one of the biggest changes in our visa system in
British history,'' Britain's prime minister, Boris Johnson,
\href{https://www.scmp.com/comment/opinion/article/3087252/hongkongers-fearing-their-way-life-britain-will-provide-alternative}{wrote
in an opinion piece} in the South China Morning Post, a Hong Kong
newspaper, on Wednesday. ``If it proves necessary, the British
government will take this step and take it willingly.''

Firms that help Hong Kong residents apply for British visas have seen a
surge in interest. One, British Connections, said 120 people had applied
for British travel documents between May 22 and May 31, up from 67 in
the same period last year.

Hong Kong residents have also
\href{https://www.nytimes3xbfgragh.onion/2020/01/03/business/hong-kong-protests-future.html}{explored
other options}, including Canada, Australia and Ireland. Those
departures could deprive the city of talent and embarrass Beijing to
boot, which is perhaps why China reacted furiously to Britain's
announcement.

``All Chinese compatriots residing in Hong Kong are Chinese nationals,''
Zhao Lijian, a spokesman for China's Foreign Ministry,
\href{https://www.fmprc.gov.cn/mfa_eng/xwfw_665399/s2510_665401/2511_665403/t1784166.shtml}{said
at a news conference}, adding that China could take unspecified
countermeasures.

Image

Workers at a booth trying to collect signatures in support of the
national security law.Credit...Lam Yik Fei for The New York Times

China's response has suggested that Beijing is willing to sacrifice Hong
Kong to get its way. Other Chinese cities, like Shanghai and Shenzhen,
have pledged to make investor-friendly legal and financial changes to
fight for Hong Kong's business. The resort island of Hainan has promised
to turn itself into a free-trade port like Hong Kong.

More broadly, China sees the risk as limited.

In the face of Mr. Trump's threat, for example, China is calculating
that he is bluffing. American business interests in Hong Kong are
extensive. If the White House takes the more drastic route of limiting
Hong Kong's access to U.S. dollars, Chinese banks have other ways to
maintain access to the global financial system, said Victor Shih, an
expert on the Chinese financial system at the University of California,
San Diego.

China also holds more than \$1 trillion in U.S. Treasury bills, which
accounts for more than 4 percent of America's total debt. While China
cannot quickly sell that debt without making major problems for itself,
such a move could cause disruptions globally.

Chinese officials also believe that Hong Kong's business elite,
historically a moderating force on Beijing, has been successfully
\href{https://www.nytimes3xbfgragh.onion/2020/05/31/business/hong-kong-china-business.html}{persuaded
or pressed} to go along. Many have extensive business holdings in the
mainland.

``We probably need not overinterpret it,'' Li Ka-shing, Hong Kong's
richest man, said of the law in a statement.

Some of Hong Kong's biggest investors contend that business will
continue as usual.

Weijian Shan, a major private equity investor in Hong Kong, recently
wrote a memoir detailing recollections of his difficult childhood under
the harsh policies of Mao Zedong. In a letter to his clients this week,
he expressed little concern about Beijing's new security law for Hong
Kong.

``There will not be any change in the rule of law, independent judicial
system or freedom of expression,'' he said.

China is also acting at a time of political strength. It has contained
the coronavirus within its borders, a feat few other countries have
managed. The moment may have emboldened China's top leader, Xi Jinping,
to take steps that his predecessors dared not.

Other rivals have been weakened. Mr. Trump is struggling to pass the
blame for American missteps in dealing with the outbreak and is
increasingly consumed with unrest at home.

Image

Chinese and Hong Kong flags fly side by side in the city's business
district.Credit...Lam Yik Fei for The New York Times

Other Western democracies, historically allies of the pro-democracy
movement in Hong Kong, are preoccupied with their own crises. The United
States, with its steady retreat from global leadership under Mr. Trump,
is in no position to rally them, say supporters of both the protesters
and Beijing.

```We expect foreign condemnation for everything we do' basically is
their attitude,'' said Andrew Nathan, a professor of political science
at Columbia University. ```You guys can bark all you want but you can't
bite, so what do we care?'''

Beijing's gamble has already yielded gains in one key arena: suppressing
the protests that inspired it to act in the first place. While some
protesters have vowed an even more determined fight against the new
security push, others acknowledged that the movement was
\href{https://t.co/hyDqjNF9Dl?amp=1}{fractured, tired and pessimistic}.

Peaceful mass protests have been barred by laws aimed at containing the
coronavirus. Those who join anyway are arrested en masse by an
increasingly aggressive police force. Many of the front-line protesters
who clashed, often violently, with the police have fled Hong Kong or
have been arrested.

A few activists have clung to hope that China still needs and wants the
world's approval.

``If the rest of the world doesn't trust China at all, they would have
to gang up against China. Is this a way forward for China and for Xi
Jinping?'' Martin Lee, a prominent veteran democracy supporter, said.
``We have to persuade them that it is ultimately and eminently in the
interest of China that they win the confidence of the rest of the
world.''

It is not clear that Beijing agrees. Mr. Lee, 81, who is sometimes
called the ``Father of Democracy'' in Hong Kong, was
\href{https://www.nytimes3xbfgragh.onion/2020/04/18/world/asia/hong-kong-arrests.html}{arrested}
in April for his participation in protests last year.

Image

The Kwai Tsing container terminal in Hong Kong, one of the city's
typically bustling port facilities.Credit...Lam Yik Fei for The New York
Times

Advertisement

\protect\hyperlink{after-bottom}{Continue reading the main story}

\hypertarget{site-index}{%
\subsection{Site Index}\label{site-index}}

\hypertarget{site-information-navigation}{%
\subsection{Site Information
Navigation}\label{site-information-navigation}}

\begin{itemize}
\tightlist
\item
  \href{https://help.nytimes3xbfgragh.onion/hc/en-us/articles/115014792127-Copyright-notice}{©~2020~The
  New York Times Company}
\end{itemize}

\begin{itemize}
\tightlist
\item
  \href{https://www.nytco.com/}{NYTCo}
\item
  \href{https://help.nytimes3xbfgragh.onion/hc/en-us/articles/115015385887-Contact-Us}{Contact
  Us}
\item
  \href{https://www.nytco.com/careers/}{Work with us}
\item
  \href{https://nytmediakit.com/}{Advertise}
\item
  \href{http://www.tbrandstudio.com/}{T Brand Studio}
\item
  \href{https://www.nytimes3xbfgragh.onion/privacy/cookie-policy\#how-do-i-manage-trackers}{Your
  Ad Choices}
\item
  \href{https://www.nytimes3xbfgragh.onion/privacy}{Privacy}
\item
  \href{https://help.nytimes3xbfgragh.onion/hc/en-us/articles/115014893428-Terms-of-service}{Terms
  of Service}
\item
  \href{https://help.nytimes3xbfgragh.onion/hc/en-us/articles/115014893968-Terms-of-sale}{Terms
  of Sale}
\item
  \href{https://spiderbites.nytimes3xbfgragh.onion}{Site Map}
\item
  \href{https://help.nytimes3xbfgragh.onion/hc/en-us}{Help}
\item
  \href{https://www.nytimes3xbfgragh.onion/subscription?campaignId=37WXW}{Subscriptions}
\end{itemize}
