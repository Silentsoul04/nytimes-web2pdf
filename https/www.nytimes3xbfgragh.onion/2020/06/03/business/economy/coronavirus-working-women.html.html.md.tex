Sections

SEARCH

\protect\hyperlink{site-content}{Skip to
content}\protect\hyperlink{site-index}{Skip to site index}

\href{https://www.nytimes3xbfgragh.onion/section/business/economy}{Economy}

\href{https://myaccount.nytimes3xbfgragh.onion/auth/login?response_type=cookie\&client_id=vi}{}

\href{https://www.nytimes3xbfgragh.onion/section/todayspaper}{Today's
Paper}

\href{/section/business/economy}{Economy}\textbar{}Pandemic Could Scar a
Generation of Working Mothers

\url{https://nyti.ms/2U7MHAr}

\begin{itemize}
\item
\item
\item
\item
\item
\item
\end{itemize}

\href{https://www.nytimes3xbfgragh.onion/spotlight/at-home?action=click\&pgtype=Article\&state=default\&region=TOP_BANNER\&context=at_home_menu}{At
Home}

\begin{itemize}
\tightlist
\item
  \href{https://www.nytimes3xbfgragh.onion/2020/07/28/books/time-for-a-literary-road-trip.html?action=click\&pgtype=Article\&state=default\&region=TOP_BANNER\&context=at_home_menu}{Take:
  A Literary Road Trip}
\item
  \href{https://www.nytimes3xbfgragh.onion/2020/07/29/magazine/bored-with-your-home-cooking-some-smoky-eggplant-will-fix-that.html?action=click\&pgtype=Article\&state=default\&region=TOP_BANNER\&context=at_home_menu}{Cook:
  Smoky Eggplant}
\item
  \href{https://www.nytimes3xbfgragh.onion/2020/07/27/travel/moose-michigan-isle-royale.html?action=click\&pgtype=Article\&state=default\&region=TOP_BANNER\&context=at_home_menu}{Look
  Out: For Moose}
\item
  \href{https://www.nytimes3xbfgragh.onion/interactive/2020/at-home/even-more-reporters-editors-diaries-lists-recommendations.html?action=click\&pgtype=Article\&state=default\&region=TOP_BANNER\&context=at_home_menu}{Explore:
  Reporters' Obsessions}
\end{itemize}

Advertisement

\protect\hyperlink{after-top}{Continue reading the main story}

Supported by

\protect\hyperlink{after-sponsor}{Continue reading the main story}

\hypertarget{pandemic-could-scar-a-generation-of-working-mothers}{%
\section{Pandemic Could Scar a Generation of Working
Mothers}\label{pandemic-could-scar-a-generation-of-working-mothers}}

Working from home has highlighted and compounded the heavier domestic
burden borne by women. Now office reopenings may force new career
sacrifices.

\includegraphics{https://static01.graylady3jvrrxbe.onion/images/2020/06/04/business/04virus-women1/merlin_173136084_ad030bf0-8574-474d-8ff4-ea058621d6e2-articleLarge.jpg?quality=75\&auto=webp\&disable=upscale}

\href{https://www.nytimes3xbfgragh.onion/by/patricia-cohen}{\includegraphics{https://static01.graylady3jvrrxbe.onion/images/2018/02/16/multimedia/author-patricia-cohen/author-patricia-cohen-thumbLarge.jpg}}\href{https://www.nytimes3xbfgragh.onion/by/tiffany-hsu}{\includegraphics{https://static01.graylady3jvrrxbe.onion/images/2018/12/06/multimedia/author-tiffany-hsu/author-tiffany-hsu-thumbLarge.png}}

By \href{https://www.nytimes3xbfgragh.onion/by/patricia-cohen}{Patricia
Cohen} and
\href{https://www.nytimes3xbfgragh.onion/by/tiffany-hsu}{Tiffany Hsu}

\begin{itemize}
\item
  Published June 3, 2020Updated June 30, 2020
\item
  \begin{itemize}
  \item
  \item
  \item
  \item
  \item
  \item
  \end{itemize}
\end{itemize}

Working during the pandemic has meant very different things for Virginia
Dressler and for her husband, Brandon.

As Mr. Dressler, a delivery driver, continued his routes near their home
in Newbury, Ohio, Ms. Dressler spent her days caring for their
3-year-old twins. Only after her husband came home at 6 p.m. could she
turn to her job as a digital projects librarian at Kent State
University, finishing her eight-hour shift from home about 2 a.m.

Later, Mr. Dressler was furloughed and took over some of the child-care
responsibilities. But now, with the economy reopening, the prospect of
being summoned back to campus fills Ms. Dressler with more anxiety: Day
care centers are just starting to reopen, with restrictions, so who will
take care of their children? ``All of these things are spinning around
in my head,'' she said. ``We're trying to come up with Plan A, Plan B
and Plan C.''

As the pandemic upends work and home life, women have carried an
outsized share of the burden, more likely to lose a job and more likely
to shoulder the load of closed schools and day care. For many working
mothers, the gradual reopening won't solve their problems, but compound
them --- forcing them out of the labor force or into part-time jobs
while increasing their responsibilities at home.

The impact could last a lifetime, reducing their earning potential and
work opportunities.

``We could have an entire generation of women who are hurt,'' Betsey
Stevenson, a professor of economics and public policy at the University
of Michigan, said of pregnant women and working mothers whose children
are too young to manage on their own. ``They may spend a significant
amount of time out of the work force, or their careers could just peter
out in terms of promotions.''

Women who drop out of the work force to take care of children often have
trouble getting back in, and the longer they stay out, the harder it is.

The economic crisis magnifies the downsides. Wage losses are much more
severe and enduring when they occur in recessions, and workers who lose
jobs now are likely to have less secure employment in the future.

``Even the limited gains made in the past decades are at risk of being
rolled back,''
\href{https://www.unfpa.org/sites/default/files/resource-pdf/COVID-19_A_Gender_Lens_Guidance_Note.pdf}{a
recent report from the United Nations} on the impact of the coronavirus
on women warned.

The setback comes at a striking moment. In February, right before the
outbreak began to spread in the United States, working women passed a
rare milestone --- making up more than half of the nation's civilian
nonfarm labor force. Still, they do a disproportionate share of the work
at home. Among married couples who work full time, women provide close
to 70 percent of child care during standard working hours, according to
\href{http://faculty.wcas.northwestern.edu/~mdo738/research/Alon_Doepke_Olmstead-Rumsey_Tertilt_COVID_2020.pdf}{recent
economic research}. That burden has been supersized as schools and other
activities shut down and help from cleaning services and babysitters has
been curtailed.

``This pandemic has exposed some weaknesses in American society that
were always there,'' said Ms. Stevenson, a former chief economist at the
U.S. Labor Department, ``and one of them is the incomplete transition of
women into truly equal roles in the labor market.''

Parents in the United States have nearly doubled the time they were
spending on education and household tasks before the coronavirus
outbreak, to 59 hours per week from 30, with mothers spending 15 hours
more on average than fathers, according to a
\href{https://www.bcg.com/publications/2020/helping-working-parents-ease-the-burden-of-covid-19.aspx}{report
from Boston Consulting Group}. Even before the pandemic, women with
children were more likely than men to be worried about their performance
reviews at work and their mental well-being and to be sleeping fewer
hours.

The inequities that existed before are now ``on steroids,'' said Claudia
Goldin, an economics professor at Harvard University. And since
workplaces tend to reward hours logged, she said, women are at a further
disadvantage. ``As work opens up, husbands have an edge,'' Ms. Goldin
said, and if the husband works more, the wife is going to have to work
less.

Ellen Kuwana, 51, was working 32 hours a week at her dream job, doing
scientific communications for biotech companies through a strategic
communications firm, as well as putting in up to 15 hours a week as a
freelance science editor.

The pandemic, though, meant her husband, a pediatric pulmonologist and
professor in Seattle, was working more than his usual 80-hour work
weeks. Her 17-year-old daughter had to take her Advanced Placement exams
and college tours online, and her 19-year-old daughter came home from
the University of California, Los Angeles. Ms. Kuwana has been buying
groceries for her parents, who have been in lockdown in their
independent living facility. She also began running a volunteer effort
that has delivered more than 12,000 meals to front-line workers.

In April, Ms. Kuwana quit her job, the best-paying work she's ever had.
She was spending more than eight hours a day hunched over her laptop at
her kitchen table for work, and then another six hours for the volunteer
effort, which she did not want to abandon. The effort aggravated the
tendinitis in her right elbow.

``It's a crazy time to quit a job, but it was a lot: the same workload,
but the work conditions had changed, the level of anxiety had changed
and so had the amount of distraction,'' she said. ``I had to get to the
point where I admitted to myself that I couldn't do it all.''

``But so much of my identity is tied up with my professional work that
it was hard for me to let that go,'' she added.

Family responsibilities as well as lower wages have always pushed women
in and out of the work force. Women often leave or lose jobs to care for
a sick child or aging relative. Meager wages make the work-home
trade-off harder to justify, even if the loss of a second paycheck may
lower a family's standard of living. In countries that offer more
comprehensive support for families --- like Germany, France, Canada and
Sweden --- a significantly larger proportion of women are in the labor
force.

And with day care centers and summer camps closed, and health concerns
lingering about grandparents and others who often make up the informal
network of backstop child care, some working women will have no choice
but to give up a job. Nor is it clear whether schools will open on a
regular routine rather than staggered or part-time schedules when the
fall term begins.

For single mothers, the pressure is intense.

Karin Ann Smith's paycheck barely covered her expenses when she was
working as a contractor for the U.S. Department of Education. She had
medical bills for her 13-year-old son, who has a condition that leaves
him constantly fatigued and pained, as well as student loans for her two
graduate degrees and \$1,650 a month in rent for an apartment in
Jupiter, Fla.

After Ms. Smith, 52, was laid off in mid-March, she was often so
overwhelmed that she hid in her bathroom with the shower running to
catch her breath. She did not receive unemployment insurance until two
months after applying, and then only after sending messages to every
state employment worker she could find on LinkedIn. Her landlord
threatened to evict her until she wangled rent assistance from the
county. Her \$500 in savings quickly evaporated, and she applied for
food stamps and sold some old toys on Facebook, even taking small
donations from sympathetic strangers on Twitter.

Ms. Smith does not expect to find another job before the fall --- long
after she exhausts her unemployment benefits. ``It's just too intense
--- I've thought about nothing else,'' she said. ``There's no help.
There's no break. When you're worried about keeping a roof over your
heads, when it's something that fundamental, you can't worry about
anything else, like whether your career is on track or your résumé is
good.''

Despite the miserable choices facing many working mothers, several
economists retain hopes that the increased pressure on families could
--- over the long term --- force structural and cultural changes that
could benefit women: a better child care system; more flexible work
arrangements; even a deeper appreciation of the sometimes overwhelming
demands of managing a household with children by partners stranded at
home for the first time.

``We find that men who can work from home do about 50 percent more child
care than men who cannot,'' said Matthias Doepke, an economist at
Northwestern University and a co-author of a recent study on the
disproportionately negative effect of the coronavirus outbreak on women.
``This may ultimately promote gender equality in the labor market.''

Companies like Salesforce, PepsiCo, Uber and Pinterest recently
\href{https://www.investinparents.com/}{signed a pledge} to offer more
flexibility and resources for working parents, and many businesses have
softened their stances on telecommuting. Staggered shifts and less
business travel are also likely to become more common.

``The effects of this shock'' --- both good and bad --- ``are likely to
outlast the actual epidemic,'' Mr. Doepke said.

In the near term, though, there is little relief in sight for working
mothers.

Mallory McMaster and her husband had intensely demanding jobs --- she
ran a communications firm in Cleveland, he worked for a start-up. Their
2-year-old son, Arlo, has been going to day care since he was 5 weeks
old.

But for the past two months, Ms. McMaster, 33, has worked from 3 to 8
a.m., then juggled her son and her job until noon, when her husband
takes over parenting. As her clients begin returning to their offices,
she is struggling to keep up.

``Everyone's scheduling all of these calls and meetings and planning
sessions because they want to hit the ground running,'' she said. ``This
would be a great time for businesses like mine to scale up, but I don't
have the time to find new clients, to update my website, because I don't
have child care. It's hindering me in a lot of ways that are going to
last much longer than the shutdown.''

Advertisement

\protect\hyperlink{after-bottom}{Continue reading the main story}

\hypertarget{site-index}{%
\subsection{Site Index}\label{site-index}}

\hypertarget{site-information-navigation}{%
\subsection{Site Information
Navigation}\label{site-information-navigation}}

\begin{itemize}
\tightlist
\item
  \href{https://help.nytimes3xbfgragh.onion/hc/en-us/articles/115014792127-Copyright-notice}{©~2020~The
  New York Times Company}
\end{itemize}

\begin{itemize}
\tightlist
\item
  \href{https://www.nytco.com/}{NYTCo}
\item
  \href{https://help.nytimes3xbfgragh.onion/hc/en-us/articles/115015385887-Contact-Us}{Contact
  Us}
\item
  \href{https://www.nytco.com/careers/}{Work with us}
\item
  \href{https://nytmediakit.com/}{Advertise}
\item
  \href{http://www.tbrandstudio.com/}{T Brand Studio}
\item
  \href{https://www.nytimes3xbfgragh.onion/privacy/cookie-policy\#how-do-i-manage-trackers}{Your
  Ad Choices}
\item
  \href{https://www.nytimes3xbfgragh.onion/privacy}{Privacy}
\item
  \href{https://help.nytimes3xbfgragh.onion/hc/en-us/articles/115014893428-Terms-of-service}{Terms
  of Service}
\item
  \href{https://help.nytimes3xbfgragh.onion/hc/en-us/articles/115014893968-Terms-of-sale}{Terms
  of Sale}
\item
  \href{https://spiderbites.nytimes3xbfgragh.onion}{Site Map}
\item
  \href{https://help.nytimes3xbfgragh.onion/hc/en-us}{Help}
\item
  \href{https://www.nytimes3xbfgragh.onion/subscription?campaignId=37WXW}{Subscriptions}
\end{itemize}
