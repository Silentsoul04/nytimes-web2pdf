Sections

SEARCH

\protect\hyperlink{site-content}{Skip to
content}\protect\hyperlink{site-index}{Skip to site index}

\href{https://www.nytimes3xbfgragh.onion/section/health}{Health}

\href{https://myaccount.nytimes3xbfgragh.onion/auth/login?response_type=cookie\&client_id=vi}{}

\href{https://www.nytimes3xbfgragh.onion/section/todayspaper}{Today's
Paper}

\href{/section/health}{Health}\textbar{}Contact Tracing With Your Phone:
It's Easier but There Are Tradeoffs

\href{https://nyti.ms/2zXSKkf}{https://nyti.ms/2zXSKkf}

\begin{itemize}
\item
\item
\item
\item
\item
\end{itemize}

\href{https://www.nytimes3xbfgragh.onion/news-event/coronavirus?action=click\&pgtype=Article\&state=default\&region=TOP_BANNER\&context=storylines_menu}{The
Coronavirus Outbreak}

\begin{itemize}
\tightlist
\item
  live\href{https://www.nytimes3xbfgragh.onion/2020/08/01/world/coronavirus-covid-19.html?action=click\&pgtype=Article\&state=default\&region=TOP_BANNER\&context=storylines_menu}{Latest
  Updates}
\item
  \href{https://www.nytimes3xbfgragh.onion/interactive/2020/us/coronavirus-us-cases.html?action=click\&pgtype=Article\&state=default\&region=TOP_BANNER\&context=storylines_menu}{Maps
  and Cases}
\item
  \href{https://www.nytimes3xbfgragh.onion/interactive/2020/science/coronavirus-vaccine-tracker.html?action=click\&pgtype=Article\&state=default\&region=TOP_BANNER\&context=storylines_menu}{Vaccine
  Tracker}
\item
  \href{https://www.nytimes3xbfgragh.onion/interactive/2020/07/29/us/schools-reopening-coronavirus.html?action=click\&pgtype=Article\&state=default\&region=TOP_BANNER\&context=storylines_menu}{What
  School May Look Like}
\item
  \href{https://www.nytimes3xbfgragh.onion/live/2020/07/31/business/stock-market-today-coronavirus?action=click\&pgtype=Article\&state=default\&region=TOP_BANNER\&context=storylines_menu}{Economy}
\end{itemize}

Advertisement

\protect\hyperlink{after-top}{Continue reading the main story}

Supported by

\protect\hyperlink{after-sponsor}{Continue reading the main story}

\hypertarget{contact-tracing-with-your-phone-its-easier-but-there-are-tradeoffs}{%
\section{Contact Tracing With Your Phone: It's Easier but There Are
Tradeoffs}\label{contact-tracing-with-your-phone-its-easier-but-there-are-tradeoffs}}

Lots of new app ideas are emerging to track Covid-19, but each has
issues around privacy, location accuracy and how much appeal it will
have to the public and to health officials.

\includegraphics{https://static01.graylady3jvrrxbe.onion/images/2020/05/28/science/00VIRUS-TECHTRACE1/merlin_171335304_2bf60532-cbd5-4f4c-a942-df354d88a6d0-articleLarge.jpg?quality=75\&auto=webp\&disable=upscale}

By \href{https://www.nytimes3xbfgragh.onion/by/matt-richtel}{Matt
Richtel}

\begin{itemize}
\item
  Published June 3, 2020Updated July 20, 2020
\item
  \begin{itemize}
  \item
  \item
  \item
  \item
  \item
  \end{itemize}
\end{itemize}

The handshake came first. Then the high-five, fist bump and more
recently, the elbow touch. Canadian researchers are now working on a new
greeting, the CanShake.

It is not a mere salutation. The CanShake --- which involves people
shaking their phones at each other upon meeting to transmit contact
information --- is one of many emerging concepts seeking to use
smartphones to do mass
\href{https://www.nytimes3xbfgragh.onion/2020/06/21/nyregion/nyc-contact-tracing.html}{contact
tracing} to track and contain the spread of Covid-19. All involve
harnessing common consumer technology to log people's location or
movements and match it against the location of people known to be sick.

There are dozens of versions, many already in practice around the world,
including in South Korea,
\href{https://www.reuters.com/article/us-health-coronavirus-singapore-technolo-idUSKBN2171ZQ}{Singapore},
China, Italy
and\href{https://www.nytimes3xbfgragh.onion/2020/03/16/world/middleeast/israel-coronavirus-cellphone-tracking.html}{Israel}.
But in the United States, privacy concerns and absence of national
policy have made the approach slower to catch on.

Efforts are piecemeal.
\href{https://www.nytimes3xbfgragh.onion/2020/07/20/technology/google-covid-tracker-app.html}{Google}
and Apple have a partnership underway to develop software for
smartphones that would enable them to continuously log information from
other devices. The MIT Media Lab has built contact tracing technology
too. Three states --- Alabama,
\href{https://www.nytimes3xbfgragh.onion/2020/04/29/business/coronavirus-cellphone-apps-contact-tracing.html}{North
Dakota and South Dakota} --- have said they have deployed or are
developing apps for tracking the virus.

The experimentation is happening as states, counties and cities are
working to train people for the traditional, more arduous approach to
contact tracing.

``There's an army of contact tracers being hired. Technology can make
this much more efficient,'' said Dr. Gunther Eysenbach, editor of the
Journal of Medical Internet Research, who is developing the CanShake.

George Rutherford, an epidemiologist at the University of California,
San Francisco who is leading training of 10,000 California contact
tracers, said digital ideas are bubbling up. ``We've gotten several
hundred people who want to show us their stuff,'' he said.

But he said, they rely on smartphones, and some lower-income people most
at risk from Covid-19 don't have them.

\includegraphics{https://static01.graylady3jvrrxbe.onion/images/2020/05/28/science/00VIRUS-TECHTRACE2/merlin_172750974_45d8fd94-c680-4fe8-b148-9de9cde149ff-articleLarge.jpg?quality=75\&auto=webp\&disable=upscale}

The traditional method of contact tracing is time consuming and
labor-intensive. It takes about 90 minutes for each case, Dr. Rutherford
said --- 60 minutes to interview the person who tests positive and 30
minutes to call or send texts to all the people the sick person
remembers being in contact with.

Whatever the technology, there are trade-offs among the major ways that
the information can be shared, stored and communicated: geolocation,
Bluetooth and QR codes.

\hypertarget{latest-updates-global-coronavirus-outbreak}{%
\section{\texorpdfstring{\href{https://www.nytimes3xbfgragh.onion/2020/08/01/world/coronavirus-covid-19.html?action=click\&pgtype=Article\&state=default\&region=MAIN_CONTENT_1\&context=storylines_live_updates}{Latest
Updates: Global Coronavirus
Outbreak}}{Latest Updates: Global Coronavirus Outbreak}}\label{latest-updates-global-coronavirus-outbreak}}

Updated 2020-08-02T07:42:09.613Z

\begin{itemize}
\tightlist
\item
  \href{https://www.nytimes3xbfgragh.onion/2020/08/01/world/coronavirus-covid-19.html?action=click\&pgtype=Article\&state=default\&region=MAIN_CONTENT_1\&context=storylines_live_updates\#link-34047410}{The
  U.S. reels as July cases more than double the total of any other
  month.}
\item
  \href{https://www.nytimes3xbfgragh.onion/2020/08/01/world/coronavirus-covid-19.html?action=click\&pgtype=Article\&state=default\&region=MAIN_CONTENT_1\&context=storylines_live_updates\#link-780ec966}{Top
  U.S. officials work to break an impasse over the federal jobless
  benefit.}
\item
  \href{https://www.nytimes3xbfgragh.onion/2020/08/01/world/coronavirus-covid-19.html?action=click\&pgtype=Article\&state=default\&region=MAIN_CONTENT_1\&context=storylines_live_updates\#link-2bc8948}{Its
  outbreak untamed, Melbourne goes into even greater lockdown.}
\end{itemize}

\href{https://www.nytimes3xbfgragh.onion/2020/08/01/world/coronavirus-covid-19.html?action=click\&pgtype=Article\&state=default\&region=MAIN_CONTENT_1\&context=storylines_live_updates}{See
more updates}

More live coverage:
\href{https://www.nytimes3xbfgragh.onion/live/2020/07/31/business/stock-market-today-coronavirus?action=click\&pgtype=Article\&state=default\&region=MAIN_CONTENT_1\&context=storylines_live_updates}{Markets}

\hypertarget{geolocation}{%
\subsection{Geolocation}\label{geolocation}}

This software typically runs in the background on phones to help with
location services like Google Maps. It can track people to within about
10 meters of their location, and be turned on and off voluntarily.

However, in other countries this technology has worked partly because it
has been used automatically, with governments taking the data without
asking permission.

After 3,000 people from the Diamond Princess cruise ship disembarked in
Taiwan in late January --- some of whom were later found to be infected
--- the Taiwanese government
\href{https://www.jmir.org/2020/5/e19540/}{tapped into geolocation data}
of individual cellphone users to look for contacts between its citizens
and the passengers.

The technology found 627,386 residents of Taiwan who had been in the
vicinity of the passengers, whose own location data was also taken using
other surveillance methods: the buses they took, the locations where
they used credit cards, security-camera footage and their phone data.

Image

German soldiers in Berlin in April tried out an app that promised to
show if someone nearby had tested positive for
coronavirus.Credit.../EPA, via Shutterstock

Those residents all received text messages and were offered tests if
they exhibited symptoms. Of 67 people tested, none were positive. Dr.
Eysenbach, who is an author of a paper on the test, said it was
effective but ``did not require informed consent'' and ``would in the
Western world be perceived as very privacy invasive.''

A report called ``\href{https://arxiv.org/pdf/2003.08567.pdf}{Apps Gone
Rogue,}'' published in April by the MIT Media Lab, found that many
international versions of contact-tracing technology ``expand mass
surveillance, limit individual freedoms and expose the most private
details about individuals.''

That said, use of geolocation software doesn't have to invade privacy,
partly because it can be turned off by a user who knows he or she might
be monitored. It also is possible to build applications that do not
allow movement history to be accessed by outside sources, said Ramesh
Raskar, an associate professor at the MIT Media Lab.

\hypertarget{bluetooth}{%
\subsection{Bluetooth}\label{bluetooth}}

Bluetooth, the technology that your phone uses to communicate with other
devices, can connect people to within one meter of one another and thus
is more precise than geolocation technology. But it potentially creates
privacy risk given that very precision.

The MIT Media lab has developed a contact-tracing concept that could use
Bluetooth or geolocation technology in ways its developers say would not
compromise individual liberties.

Safe Paths runs in the background of a person's phone --- with his or
her permission --- creating and storing a history of movements. If a
person tested positive, that individual's history would be downloaded to
a database. After that, other people who used the service could run
checks to see if their own movements had intersected with someone who
tested positive --- ``completely private,'' Mr. Raskar said, likening
the idea to someone checking for rain without having to reveal his or
her location.

The project is being developed with input from the Department of Health
and Human Services, Harvard University and the Mayo Clinic. Mr. Raskar
said several countries and 15 cities and states had expressed interest
to MIT in the technology, but declined to identify them.

Image

Singapore's contact-tracing app, TraceTogether, uses Bluetooth to
supplement the country's efforts to track and monitor coronavirus
cases.Credit...Catherine Lai/Agence France-Presse --- Getty Images

Apple and Google also use Bluetooth to let jurisdictions develop
contact-tracing apps.

The companies' technology offers privacy protections and is ``a
good-faith effort,'' said Gaurav Laroia, a lawyer for Free Press, a
nonprofit that is part of a consortium that includes the American Civil
Liberties Union. The larger issue, though, he said, is whether people
will choose to download these apps.

\href{https://www.nytimes3xbfgragh.onion/news-event/coronavirus?action=click\&pgtype=Article\&state=default\&region=MAIN_CONTENT_3\&context=storylines_faq}{}

\hypertarget{the-coronavirus-outbreak-}{%
\subsubsection{The Coronavirus Outbreak
›}\label{the-coronavirus-outbreak-}}

\hypertarget{frequently-asked-questions}{%
\paragraph{Frequently Asked
Questions}\label{frequently-asked-questions}}

Updated July 27, 2020

\begin{itemize}
\item ~
  \hypertarget{should-i-refinance-my-mortgage}{%
  \paragraph{Should I refinance my
  mortgage?}\label{should-i-refinance-my-mortgage}}

  \begin{itemize}
  \tightlist
  \item
    \href{https://www.nytimes3xbfgragh.onion/article/coronavirus-money-unemployment.html?action=click\&pgtype=Article\&state=default\&region=MAIN_CONTENT_3\&context=storylines_faq}{It
    could be a good idea,} because mortgage rates have
    \href{https://www.nytimes3xbfgragh.onion/2020/07/16/business/mortgage-rates-below-3-percent.html?action=click\&pgtype=Article\&state=default\&region=MAIN_CONTENT_3\&context=storylines_faq}{never
    been lower.} Refinancing requests have pushed mortgage applications
    to some of the highest levels since 2008, so be prepared to get in
    line. But defaults are also up, so if you're thinking about buying a
    home, be aware that some lenders have tightened their standards.
  \end{itemize}
\item ~
  \hypertarget{what-is-school-going-to-look-like-in-september}{%
  \paragraph{What is school going to look like in
  September?}\label{what-is-school-going-to-look-like-in-september}}

  \begin{itemize}
  \tightlist
  \item
    It is unlikely that many schools will return to a normal schedule
    this fall, requiring the grind of
    \href{https://www.nytimes3xbfgragh.onion/2020/06/05/us/coronavirus-education-lost-learning.html?action=click\&pgtype=Article\&state=default\&region=MAIN_CONTENT_3\&context=storylines_faq}{online
    learning},
    \href{https://www.nytimes3xbfgragh.onion/2020/05/29/us/coronavirus-child-care-centers.html?action=click\&pgtype=Article\&state=default\&region=MAIN_CONTENT_3\&context=storylines_faq}{makeshift
    child care} and
    \href{https://www.nytimes3xbfgragh.onion/2020/06/03/business/economy/coronavirus-working-women.html?action=click\&pgtype=Article\&state=default\&region=MAIN_CONTENT_3\&context=storylines_faq}{stunted
    workdays} to continue. California's two largest public school
    districts --- Los Angeles and San Diego --- said on July 13, that
    \href{https://www.nytimes3xbfgragh.onion/2020/07/13/us/lausd-san-diego-school-reopening.html?action=click\&pgtype=Article\&state=default\&region=MAIN_CONTENT_3\&context=storylines_faq}{instruction
    will be remote-only in the fall}, citing concerns that surging
    coronavirus infections in their areas pose too dire a risk for
    students and teachers. Together, the two districts enroll some
    825,000 students. They are the largest in the country so far to
    abandon plans for even a partial physical return to classrooms when
    they reopen in August. For other districts, the solution won't be an
    all-or-nothing approach.
    \href{https://bioethics.jhu.edu/research-and-outreach/projects/eschool-initiative/school-policy-tracker/}{Many
    systems}, including the nation's largest, New York City, are
    devising
    \href{https://www.nytimes3xbfgragh.onion/2020/06/26/us/coronavirus-schools-reopen-fall.html?action=click\&pgtype=Article\&state=default\&region=MAIN_CONTENT_3\&context=storylines_faq}{hybrid
    plans} that involve spending some days in classrooms and other days
    online. There's no national policy on this yet, so check with your
    municipal school system regularly to see what is happening in your
    community.
  \end{itemize}
\item ~
  \hypertarget{is-the-coronavirus-airborne}{%
  \paragraph{Is the coronavirus
  airborne?}\label{is-the-coronavirus-airborne}}

  \begin{itemize}
  \tightlist
  \item
    The coronavirus
    \href{https://www.nytimes3xbfgragh.onion/2020/07/04/health/239-experts-with-one-big-claim-the-coronavirus-is-airborne.html?action=click\&pgtype=Article\&state=default\&region=MAIN_CONTENT_3\&context=storylines_faq}{can
    stay aloft for hours in tiny droplets in stagnant air}, infecting
    people as they inhale, mounting scientific evidence suggests. This
    risk is highest in crowded indoor spaces with poor ventilation, and
    may help explain super-spreading events reported in meatpacking
    plants, churches and restaurants.
    \href{https://www.nytimes3xbfgragh.onion/2020/07/06/health/coronavirus-airborne-aerosols.html?action=click\&pgtype=Article\&state=default\&region=MAIN_CONTENT_3\&context=storylines_faq}{It's
    unclear how often the virus is spread} via these tiny droplets, or
    aerosols, compared with larger droplets that are expelled when a
    sick person coughs or sneezes, or transmitted through contact with
    contaminated surfaces, said Linsey Marr, an aerosol expert at
    Virginia Tech. Aerosols are released even when a person without
    symptoms exhales, talks or sings, according to Dr. Marr and more
    than 200 other experts, who
    \href{https://academic.oup.com/cid/article/doi/10.1093/cid/ciaa939/5867798}{have
    outlined the evidence in an open letter to the World Health
    Organization}.
  \end{itemize}
\item ~
  \hypertarget{what-are-the-symptoms-of-coronavirus}{%
  \paragraph{What are the symptoms of
  coronavirus?}\label{what-are-the-symptoms-of-coronavirus}}

  \begin{itemize}
  \tightlist
  \item
    Common symptoms
    \href{https://www.nytimes3xbfgragh.onion/article/symptoms-coronavirus.html?action=click\&pgtype=Article\&state=default\&region=MAIN_CONTENT_3\&context=storylines_faq}{include
    fever, a dry cough, fatigue and difficulty breathing or shortness of
    breath.} Some of these symptoms overlap with those of the flu,
    making detection difficult, but runny noses and stuffy sinuses are
    less common.
    \href{https://www.nytimes3xbfgragh.onion/2020/04/27/health/coronavirus-symptoms-cdc.html?action=click\&pgtype=Article\&state=default\&region=MAIN_CONTENT_3\&context=storylines_faq}{The
    C.D.C. has also} added chills, muscle pain, sore throat, headache
    and a new loss of the sense of taste or smell as symptoms to look
    out for. Most people fall ill five to seven days after exposure, but
    symptoms may appear in as few as two days or as many as 14 days.
  \end{itemize}
\item ~
  \hypertarget{does-asymptomatic-transmission-of-covid-19-happen}{%
  \paragraph{Does asymptomatic transmission of Covid-19
  happen?}\label{does-asymptomatic-transmission-of-covid-19-happen}}

  \begin{itemize}
  \tightlist
  \item
    So far, the evidence seems to show it does. A widely cited
    \href{https://www.nature.com/articles/s41591-020-0869-5}{paper}
    published in April suggests that people are most infectious about
    two days before the onset of coronavirus symptoms and estimated that
    44 percent of new infections were a result of transmission from
    people who were not yet showing symptoms. Recently, a top expert at
    the World Health Organization stated that transmission of the
    coronavirus by people who did not have symptoms was ``very rare,''
    \href{https://www.nytimes3xbfgragh.onion/2020/06/09/world/coronavirus-updates.html?action=click\&pgtype=Article\&state=default\&region=MAIN_CONTENT_3\&context=storylines_faq\#link-1f302e21}{but
    she later walked back that statement.}
  \end{itemize}
\end{itemize}

Bluetooth is also the technology behind the CanShake, an app in early
development. When two people were near each other, they would shake
their phones at each other to trigger a passing of their contact
information through a Bluetooth connection. The data would be logged in
each phone. Then, if either person got sick, the information could be
downloaded by the authorities, who would --- with the user's permission
--- warn those in the contact log.

``The idea is to replace the handshake with the CanShake. It alludes to
the idea that you `can shake' again --- not your hands but with your
phone,'' Mr. Eysenbac said.

\hypertarget{qr-codes}{%
\subsection{QR codes}\label{qr-codes}}

When coronavirus cases surged in South Korea this winter, hospitals
there asked people seeking tests or treatment to answer questions on
their phones before arriving, including whether they had a fever or
cough. After completing the responses, each person was sent a QR code to
their phone.

When the person arrived at the hospital, a scanner captured the code and
the individual's information and the person was directed to get a
coronavirus test or not.

Initially, this was seen as a way to process people without paperwork,
said Dr. Ki Mo-ran, a professor at the National Cancer Center Graduate
School of Cancer Science and Policy.

Image

QR technology in Mudanjiang, northeastern China, in
April.Credit...Agence France-Presse --- Getty Images

Now, the country is considering expanding the use of QR codes. In May,
Dr. Ki met with Prime Minister Chung Sye-kyun to recommend expansive use
of the technology for contact tracing. In an interview, Dr. Ki said she
described how it would scan visits by people to larger gatherings at
restaurants, churches and night clubs, for example.

The proposed expansion of this technology was prompted, she said, by an
outbreak that began in a nightclub. The government's policy at the time
was that visitors to such gatherings were required to sign in and leave
their contact information.

But she said that 30 percent of the visitors to the nightclub could not
be found because there was such a rush of people that not everyone gave
information or partial data that could not be traced.

Under the new rules, she said, ``people would generate a QR code, rather
than writing down'' their information. That code would be scanned when
they entered and the information ``would be connected to the
government,'' which, in the event of outbreak, could look for
intersections between the sick and those nearby.

The government is exploring this idea of a ``digital visitors list,''
for a six-month test at nightclubs, restaurants and bars. The government
would collect the data but would delete it after four weeks if it was
not needed to trace an outbreak.

The report from MIT Media Lab noted that one source of abuse from all
three technologies was that governments broadcast the location of people
who were infected. Singapore published maps designating whereabouts of
infected citizens while Korea sent text messages about their locations.
It didn't identify people by name, the report said, but it noted that
divulging locations was still ``making these places, and the businesses
occupying them, susceptible to boycott, harassment, and other punitive
measures.''

Dr. Ki acknowledged that privacy was a critical concern, but cautioned
that protecting public health may be worth trade-offs. ``Privacy is a
very important issue,'' she said, ``but nowadays even though we try to
protect personal privacy, it's very critical to save the community, so
we have to find the very appropriate balance.''

Advertisement

\protect\hyperlink{after-bottom}{Continue reading the main story}

\hypertarget{site-index}{%
\subsection{Site Index}\label{site-index}}

\hypertarget{site-information-navigation}{%
\subsection{Site Information
Navigation}\label{site-information-navigation}}

\begin{itemize}
\tightlist
\item
  \href{https://help.nytimes3xbfgragh.onion/hc/en-us/articles/115014792127-Copyright-notice}{©~2020~The
  New York Times Company}
\end{itemize}

\begin{itemize}
\tightlist
\item
  \href{https://www.nytco.com/}{NYTCo}
\item
  \href{https://help.nytimes3xbfgragh.onion/hc/en-us/articles/115015385887-Contact-Us}{Contact
  Us}
\item
  \href{https://www.nytco.com/careers/}{Work with us}
\item
  \href{https://nytmediakit.com/}{Advertise}
\item
  \href{http://www.tbrandstudio.com/}{T Brand Studio}
\item
  \href{https://www.nytimes3xbfgragh.onion/privacy/cookie-policy\#how-do-i-manage-trackers}{Your
  Ad Choices}
\item
  \href{https://www.nytimes3xbfgragh.onion/privacy}{Privacy}
\item
  \href{https://help.nytimes3xbfgragh.onion/hc/en-us/articles/115014893428-Terms-of-service}{Terms
  of Service}
\item
  \href{https://help.nytimes3xbfgragh.onion/hc/en-us/articles/115014893968-Terms-of-sale}{Terms
  of Sale}
\item
  \href{https://spiderbites.nytimes3xbfgragh.onion}{Site Map}
\item
  \href{https://help.nytimes3xbfgragh.onion/hc/en-us}{Help}
\item
  \href{https://www.nytimes3xbfgragh.onion/subscription?campaignId=37WXW}{Subscriptions}
\end{itemize}
