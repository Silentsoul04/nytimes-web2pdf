Sections

SEARCH

\protect\hyperlink{site-content}{Skip to
content}\protect\hyperlink{site-index}{Skip to site index}

\href{https://myaccount.nytimes3xbfgragh.onion/auth/login?response_type=cookie\&client_id=vi}{}

\href{https://www.nytimes3xbfgragh.onion/section/todayspaper}{Today's
Paper}

\href{/section/opinion}{Opinion}\textbar{}Cameras Won't Stop Police From
Killing

\url{https://nyti.ms/3gVfMZV}

\begin{itemize}
\item
\item
\item
\item
\item
\item
\end{itemize}

Advertisement

\protect\hyperlink{after-top}{Continue reading the main story}

\href{/section/opinion}{Opinion}

Supported by

\protect\hyperlink{after-sponsor}{Continue reading the main story}

\hypertarget{cameras-wont-stop-police-from-killing}{%
\section{Cameras Won't Stop Police From
Killing}\label{cameras-wont-stop-police-from-killing}}

Body cams have turned brutality into spectacle.

\href{https://www.nytimes3xbfgragh.onion/by/farhad-manjoo}{\includegraphics{https://static01.graylady3jvrrxbe.onion/images/2019/01/08/opinion/farhad-manjoo-opinion/farhad-manjoo-opinion-thumbLarge.png}}

By \href{https://www.nytimes3xbfgragh.onion/by/farhad-manjoo}{Farhad
Manjoo}

Opinion Columnist

\begin{itemize}
\item
  June 3, 2020
\item
  \begin{itemize}
  \item
  \item
  \item
  \item
  \item
  \item
  \end{itemize}
\end{itemize}

\includegraphics{https://static01.graylady3jvrrxbe.onion/images/2020/06/03/opinion/03manjoo1/merlin_173047794_244075a0-dee1-4968-9a2d-4c30b9ae861a-articleLarge.jpg?quality=75\&auto=webp\&disable=upscale}

A chilling word keeps coming to mind this week, like a scratched-up
record stuck on a lazy loop in my tweet-addled brain. \emph{Impunity}.

If you can bear it, watch one of the videos of
\href{https://www.nytimes3xbfgragh.onion/2020/06/04/us/politics/george-floyd-witness-maurice-lester-hall.html}{George
Floyd's} death last week at the hands of the Minneapolis Police
Department. Focus on the eyes of Derek Chauvin, the officer who has been
charged with murder and manslaughter for pressing his knee into Floyd's
neck for a torturous eight minutes and 46 seconds.

At several points, Chauvin makes smirking eye contact with the camera.
He even halfheartedly reaches for what looks like pepper spray when the
phone-wielding bystanders get a bit rowdy in their insistence that Floyd
is dying before their eyes. But the presence of the bystanders doesn't
stop him; it's almost as if Chauvin knows nothing can touch him.
\emph{Impunity} is the only word I can think of for it.

Keep a close eye, too, on Tou Thao, Chauvin's partner, who engages with
the crowd in the manner of a security guard at an amusement park. As
Chauvin pins Floyd down, Thao is almost polite in his colloquy with the
people recording the scene. It's as if he knows he's going to be all
over social media later, so he's going to play it cool.

I've watched the Floyd videos at least a dozen times, and every time,
it's Thao's composure that stiffens the hairs on the back of my neck.
Thao comes off as completely unashamed of the misconduct he
\href{https://www.nytimes3xbfgragh.onion/2020/06/04/us/politics/george-floyd-witness-maurice-lester-hall.html}{witnesses}
and, with his silence, encourages, in full public view.

Cameras were supposed to eliminate this sort of horror. Here, they
hardly make it better.

Ever since
\href{https://www.nytimes3xbfgragh.onion/1991/03/07/us/tape-of-beating-by-police-revives-charges-of-racism.html}{the
beating of Rodney King in Los Angeles in 1991}, America has been flooded
with videos --- captured by bystanders and by law enforcement officers
on dashboard and body cams --- that have highlighted the routine abuse
and killing of unarmed black people at the hands of the police.

As these cameras have become ubiquitous, we have gotten a better picture
of the scale of the horror. At times, as in
\href{https://www.theguardian.com/us-news/video/2014/dec/04/i-cant-breathe-eric-garner-chokehold-death-video}{the
death of Eric Garner on Staten Island in 2014}, bystanders have managed
to capture the precise moment at which police misconduct becomes fatal.

Yet in the Garner video, the police try to push the camera away. The
cops seemed at least embarrassed by it.

What's particularly nauseating about the Floyd videos is that the
officers know they're being watched, yet they are not deterred and don't
even seem bothered by the cameras. A similar shamelessness was on
display in the innumerable
\href{https://www.theverge.com/2020/5/31/21276044/police-violence-protest-george-floyd}{clips}
showing police officers brazenly assaulting protesters and journalists
during protests this weekend.

As I scrolled through
\href{https://twitter.com/fmanjoo/status/1267290935198486529?s=20}{endless
collections of these} online, I found it hard to escape the conclusion
that America's police forces are not just unfairly brutal --- they also
do not seem to care anymore about being caught on tape.

While videos have catalyzed protest movements like Black Lives Matter,
documenting police misconduct in America has had little effect in
reducing it.

Not long ago, many reformers saw video as a key way to improve policing.
In 2014, after the killing of Michael Brown by the police in Ferguson,
Mo., the Obama administration allocated funds to help police departments
purchase tens of thousands of body cameras. Even some civil liberties
groups
\href{https://www.aclu.org/other/police-body-mounted-cameras-right-policies-place-win-all?redirect=technology-and-liberty/police-body-mounted-cameras-right-policies-place-win-all\#control\%22\%3E\%20position\%20paper\%20\%3C/a\%3Esupporting\%20the\%20use\%20of\%20bodycams\%20by\%20police.}{endorsed
the idea}.

The theory was simple: If there were bad cops on the force, body cams
would root them out and make it easier to prosecute them.

But it hasn't worked out that way. One major study of body cameras in
American policing, which
\href{https://www.pnas.org/content/116/21/10329.short?rss=1}{followed
more than 2,000 officers in Washington, D.C.},
\href{https://www.nytimes3xbfgragh.onion/2017/10/20/upshot/a-big-test-of-police-body-cameras-defies-expectations.html}{found
that the cameras did little to alter police behavior}. Officers equipped
with cameras used force and faced complaints from civilians at rates
similar to those for officers who didn't have cameras.

What's more, in
\href{https://www.vox.com/policy-and-politics/2017/7/21/15983842/police-body-cameras-failures}{several
high-profile cases}, jurors were reluctant to convict, even with
eyewitness and body-cam videos capturing wrongdoing. In 2015, Michael
Slager, a police officer in North Charleston, S.C., who had been caught
on video shooting a black man named Walter Scott multiple times in the
back, was charged with murder. The trial ended in a hung jury when a
\href{https://www.nbcnews.com/news/crime-courts/jury-says-it-s-deadlocked-trial-officer-who-shot-walter-n691291}{lone
juror declared himself unable to convict}. (Slager later pleaded guilty
to the federal crime of violating Scott's civil rights and was sentenced
in 2017 to
\href{https://www.nytimes3xbfgragh.onion/2017/12/07/us/michael-slager-sentence-walter-scott.html}{20
years in prison}.)

One problem is that video is often open to interpretation --- where
critics of the police see clear brutality, jurors who are inclined to
give police officers the benefit of the doubt may
\href{https://www.cnn.com/2017/06/16/us/philando-castile-trial-verdict/index.html}{excuse}
as sins actions in the heat of the moment.

There are also a hodgepodge of policies governing body cameras.
\href{https://www.brennancenter.org/our-work/research-reports/police-body-worn-camera-policies}{Different
states have different rules} about when officers are supposed to turn
them on and
\href{https://www.abc10.com/article/news/local/release-of-police-body-cam-video-policies-varies-between-agencies/103-530783929}{who
gets access to the video when there are questions about officers'
conduct}. In some cases, officers equipped with body cameras have
conveniently
\href{https://www.vox.com/policy-and-politics/2017/7/17/15985442/minneapolis-police-shooting-justine-damond-video}{neglected
to turn them on}. On Monday, the mayor of Louisville, Ky.,
\href{https://www.cbsnews.com/news/steve-conrad-louisville-police-chief-fired-protest-shooting-death/}{fired
the city's police chief} after discovering that two officers involved in
the fatal shooting of a black business owner had not turned on their
body cameras.

What happens when, time and again, law enforcement officers are recorded
brutalizing citizens but left unpunished? I worry that police violence
will become even more normalized, turning into a crude spectacle that
loses even the ability to shock. How else to explain the orgy of
violence on display this weekend? A small selection:

Here are Atlanta police officers breaking into a car and attacking two
young people.

Here are members of the National Guard and the Minneapolis Police
Department firing paint rounds at people on a residential street.

Here is a New York Police Department vehicle ramming a crowd of
protesters in Brooklyn.

And here are two Australian journalists brutalized by the police in
Washington, D.C. --- one of many incidents in which members of the news
media appeared to have been deliberately
\href{https://www.nytimes3xbfgragh.onion/2020/06/01/business/media/reporters-protests-george-floyd.html}{targeted
by the police}.

``The whole world is watching'' is what American pundits might say to
China's leaders when they round up Uighurs to send to re-education
camps, or to Vladimir Putin of Russia when he
\href{https://www.nytimes3xbfgragh.onion/2019/12/24/world/europe/navalny-russia-putin.html}{banishes
dissidents} to an Arctic military base.

The phrase applies to our country, too. The whole world is watching and
has been for decades. Yet little changes, because merely watching is not
nearly enough.

\hypertarget{office-hours-with-farhad-manjoo}{%
\subsection{Office Hours With Farhad
Manjoo}\label{office-hours-with-farhad-manjoo}}

\emph{Farhad wants to}
\href{https://www.nytimes3xbfgragh.onion/2019/05/16/opinion/farhad-office-hours.html?module=inline}{\emph{chat
with readers on the phone}}\emph{. If you're interested in talking to a
New York Times columnist about anything that's on your mind, please fill
out this form. Farhad will select a few readers to call.}

\emph{The Times is committed to publishing}
\href{https://www.nytimes3xbfgragh.onion/2019/01/31/opinion/letters/letters-to-editor-new-york-times-women.html}{\emph{a
diversity of letters}} \emph{to the editor. We'd like to hear what you
think about this or any of our articles. Here are some}
\href{https://help.nytimes3xbfgragh.onion/hc/en-us/articles/115014925288-How-to-submit-a-letter-to-the-editor}{\emph{tips}}\emph{.
And here's our email:}
\href{mailto:letters@NYTimes.com}{\emph{letters@NYTimes.com}}\emph{.}

\emph{Follow The New York Times Opinion section on}
\href{https://www.facebookcorewwwi.onion/nytopinion}{\emph{Facebook}}\emph{,}
\href{http://twitter.com/NYTOpinion}{\emph{Twitter (@NYTopinion)}}
\emph{and}
\href{https://www.instagram.com/nytopinion/}{\emph{Instagram}}\emph{.}

Advertisement

\protect\hyperlink{after-bottom}{Continue reading the main story}

\hypertarget{site-index}{%
\subsection{Site Index}\label{site-index}}

\hypertarget{site-information-navigation}{%
\subsection{Site Information
Navigation}\label{site-information-navigation}}

\begin{itemize}
\tightlist
\item
  \href{https://help.nytimes3xbfgragh.onion/hc/en-us/articles/115014792127-Copyright-notice}{©~2020~The
  New York Times Company}
\end{itemize}

\begin{itemize}
\tightlist
\item
  \href{https://www.nytco.com/}{NYTCo}
\item
  \href{https://help.nytimes3xbfgragh.onion/hc/en-us/articles/115015385887-Contact-Us}{Contact
  Us}
\item
  \href{https://www.nytco.com/careers/}{Work with us}
\item
  \href{https://nytmediakit.com/}{Advertise}
\item
  \href{http://www.tbrandstudio.com/}{T Brand Studio}
\item
  \href{https://www.nytimes3xbfgragh.onion/privacy/cookie-policy\#how-do-i-manage-trackers}{Your
  Ad Choices}
\item
  \href{https://www.nytimes3xbfgragh.onion/privacy}{Privacy}
\item
  \href{https://help.nytimes3xbfgragh.onion/hc/en-us/articles/115014893428-Terms-of-service}{Terms
  of Service}
\item
  \href{https://help.nytimes3xbfgragh.onion/hc/en-us/articles/115014893968-Terms-of-sale}{Terms
  of Sale}
\item
  \href{https://spiderbites.nytimes3xbfgragh.onion}{Site Map}
\item
  \href{https://help.nytimes3xbfgragh.onion/hc/en-us}{Help}
\item
  \href{https://www.nytimes3xbfgragh.onion/subscription?campaignId=37WXW}{Subscriptions}
\end{itemize}
