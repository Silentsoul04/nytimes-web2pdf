Sections

SEARCH

\protect\hyperlink{site-content}{Skip to
content}\protect\hyperlink{site-index}{Skip to site index}

\href{https://www.nytimes3xbfgragh.onion/section/us}{U.S.}

\href{https://myaccount.nytimes3xbfgragh.onion/auth/login?response_type=cookie\&client_id=vi}{}

\href{https://www.nytimes3xbfgragh.onion/section/todayspaper}{Today's
Paper}

\href{/section/us}{U.S.}\textbar{}Officers Charged in George Floyd's
Death Not Likely to Present United Front

\url{https://nyti.ms/2Uad7By}

\begin{itemize}
\item
\item
\item
\item
\item
\end{itemize}

\href{https://www.nytimes3xbfgragh.onion/news-event/george-floyd-protests-minneapolis-new-york-los-angeles?action=click\&pgtype=Article\&state=default\&region=TOP_BANNER\&context=storylines_menu}{Race
and America}

\begin{itemize}
\tightlist
\item
  \href{https://www.nytimes3xbfgragh.onion/2020/07/26/us/protests-portland-seattle-trump.html?action=click\&pgtype=Article\&state=default\&region=TOP_BANNER\&context=storylines_menu}{Protesters
  Return to Other Cities}
\item
  \href{https://www.nytimes3xbfgragh.onion/2020/07/24/us/portland-oregon-protests-white-race.html?action=click\&pgtype=Article\&state=default\&region=TOP_BANNER\&context=storylines_menu}{Portland
  at the Center}
\item
  \href{https://www.nytimes3xbfgragh.onion/2020/07/23/podcasts/the-daily/portland-protests.html?action=click\&pgtype=Article\&state=default\&region=TOP_BANNER\&context=storylines_menu}{Podcast:
  Showdown in Portland}
\item
  \href{https://www.nytimes3xbfgragh.onion/interactive/2020/07/16/us/black-lives-matter-protests-louisville-breonna-taylor.html?action=click\&pgtype=Article\&state=default\&region=TOP_BANNER\&context=storylines_menu}{45
  Days in Louisville}
\end{itemize}

Advertisement

\protect\hyperlink{after-top}{Continue reading the main story}

Supported by

\protect\hyperlink{after-sponsor}{Continue reading the main story}

\hypertarget{officers-charged-in-george-floyds-death-not-likely-to-present-united-front}{%
\section{Officers Charged in George Floyd's Death Not Likely to Present
United
Front}\label{officers-charged-in-george-floyds-death-not-likely-to-present-united-front}}

Facing decades in prison and a bail of at least \$750,000, two former
Minneapolis officers blamed Derek Chauvin, and a third has cooperated
with investigators, their lawyers said.

\includegraphics{https://static01.graylady3jvrrxbe.onion/images/2020/06/04/us/04UNREST-SERVICERECORDS-2/merlin_173202597_0c7b1ce7-7c51-4b9d-9406-97193895a0cb-articleLarge.jpg?quality=75\&auto=webp\&disable=upscale}

By \href{https://www.nytimes3xbfgragh.onion/by/kim-barker}{Kim Barker},
\href{https://www.nytimes3xbfgragh.onion/by/john-eligon}{John Eligon},
\href{https://www.nytimes3xbfgragh.onion/by/richard-a-oppel-jr}{Richard
A. Oppel Jr.} and Matt Furber

\begin{itemize}
\item
  June 4, 2020
\item
  \begin{itemize}
  \item
  \item
  \item
  \item
  \item
  \end{itemize}
\end{itemize}

MINNEAPOLIS --- Two of the former police officers charged with aiding
and abetting in the killing of George Floyd turned on the senior officer
accused in the case, making for an extraordinary court appearance on
Thursday afternoon. A third officer was cooperating with the
authorities, a sign that the four fired officers would not be presenting
a united front.

Facing 40 years in prison and a bail of at least \$750,000, the former
officers Thomas Lane and
\href{https://www.nytimes3xbfgragh.onion/2020/06/27/us/minneapolis-police-officer-kueng.html}{J.
Alexander Kueng}, both rookies, blamed
\href{https://www.nytimes3xbfgragh.onion/2020/07/18/us/derek-chauvin-george-floyd.html}{Derek
Chauvin}, the senior officer at the scene and a training officer, their
lawyers said in court. The lawyer for Tou Thao, another former officer
charged in the case, said his client had cooperated with investigators
before they arrested Mr. Chauvin.

Mr. Chauvin, a white 19-year veteran, was captured on a graphic video on
May 25 kneeling for almost nine minutes on the neck of Mr. Floyd, who
was African-American, as the other three officers aided in the arrest.

Mr. Chauvin, 44, who did not appear in court on Thursday, faces the most
serious charges of the four men --- second-degree murder and
second-degree manslaughter.

In cases of excessive force, it is not common for officers to break
ranks, or cross what is often called the blue wall of silence. But
little about this case is typical: Mr. Floyd's death has unleashed a
movement, with demonstrations in more than 150 American cities against
police brutality and systemic racism.

The hearing --- which unfolded blocks from where Mr. Floyd was being
remembered in a packed, emotional memorial service --- was sparsely
attended because of threats from the coronavirus. Lawyers for the
defendants were flanked by National Guard soldiers and Hennepin County
sheriff's deputies as they entered the courthouse. Judge Paul R. Scoggin
and lawyers, when they addressed the court, were the only ones who did
not wear masks. All three defendants wore orange jumpsuits and light
blue surgical masks.

Earl Gray, the lawyer representing Mr. Lane, 37, told the court that Mr.
Chauvin was a training officer for new officers. He said that the day
Mr. Floyd died was Mr. Lane's fourth day on the force.

``They're required to call him `Sir,''' Mr. Gray told the court. ``He
has 20 years' experience. What is my client supposed to do but to follow
what the training officer said? Is that aiding and abetting a crime?''

Throughout the hearing, Mr. Lane kept looking over at seven people who
were there to support him.

Tom Plunkett, the lawyer representing Mr. Kueng, 26, said Mr. Chauvin
was his client's main training officer. Though police records show that
Mr. Kueng had become a police officer in December, he was only on his
third shift as a full-fledged officer when Mr. Floyd was killed, Mr.
Plunkett said.

The lawyer also argued that Mr. Kueng, who is African-American, and Mr.
Lane, who is white, had tried to stop Mr. Chauvin.

``At multiple times, Mr. Kueng and Mr. Lane directed their attention to
that 19-year veteran and said, `You shouldn't do this,''' Mr. Plunkett
said.

Eric Nelson, a lawyer for Mr. Chauvin, declined to comment on what the
lawyers for the other former officers said at Thursday's hearing.

Mr. Plunkett specializes in defending police officers accused of crimes.
But during the hearing he suggested that the Minneapolis Police
Department had lost its way.

``The events of this case are horrific. The tragedy that flows from
those events continues to grow,'' Mr. Plunkett said. ``I'm asking the
court to set bail on the individual, on the person, and not on an
institution that has lost its guidance.''

Robert Paule, the lawyer representing Mr. Thao, 34, said his client, who
is Hmong, had met with state agents and surrendered his service
revolver.

Still, Judge Scoggin denied all the lawyers' demands for reduced bail.
The
\href{https://www.nytimes3xbfgragh.onion/2020/06/29/us/derek-chauvin-court-hearing.html}{next
court appearance for three of the former officers} is scheduled for June
29. Mr. Chauvin is scheduled to be in court on Monday.

\includegraphics{https://static01.graylady3jvrrxbe.onion/images/2020/06/04/us/04UNREST-SERVICERECORDS-3/merlin_173206848_8dc6bff2-fb5e-4006-a564-b051062526ae-articleLarge.jpg?quality=75\&auto=webp\&disable=upscale}

The protests that have roiled the world began after a video released
online showed Mr. Chauvin holding down Mr. Floyd until he was
motionless. The police had responded to a 911 call that a counterfeit
\$20 bill was used to buy cigarettes at the Cup Foods corner store in
South Minneapolis.

When Mr. Lane and Mr. Kueng arrived, they ordered Mr. Floyd out of a
car, prosecutors say. Mr. Lane handcuffed him, and Mr. Floyd sat on the
ground and said, ``Thank you, man.'' He was calm, the statement of
probable cause said.

As they tried to walk Mr. Floyd to their squad car, he stiffened up and
fell to the ground. Mr. Floyd told them he was not resisting arrest but
was claustrophobic and did not want to get in the back seat of the car,
according to the arrest affidavit.

Soon, Mr. Chauvin showed up, with Mr. Thao at his side.

Mr. Chauvin quickly took charge. Cellphone video showed that Mr. Chauvin
placed his left knee on Mr. Floyd's neck. Mr. Lane held Mr. Floyd's
legs, and Mr. Kueng held his back. Mr. Thao stood between the officers
and onlookers, according to charging documents.

Calling for his mother and saying he was going to die, Mr. Floyd pleaded
for his life, repeating, ``I can't breathe.''

Mr. Lane called on Mr. Chauvin to roll Mr. Floyd over, and warned that
Mr. Floyd might be having a medical problem, Mr. Gray told the court.
Still, the lawyer said Mr. Chauvin refused to do anything.

The four officers were fired the next day. On Friday, Mr. Chauvin was
charged with third-degree murder and second-degree manslaughter. But on
Wednesday, as hundreds of protests continued across the nation, the
murder charge was upgraded and the three other former officers were also
charged.

According to personnel records released late Wednesday, the four
officers came to the force from different backgrounds, with some similar
stops along the way. Two worked at McDonald's and as security guards;
two took college courses on marginalized communities and stocked shelves
at stores; two earned the same degree at the same university.

Mr. Chauvin initially studied food preparation and once worked as a
security guard and a cook for McDonald's and a restaurant called
Tinucci's in the mid-1990s. He then took courses in law enforcement and
served in the U.S. Army as a military police officer in the late 1990s.

He became a part-time Minneapolis community service officer in 2001 and
eventually a full-time police officer. Mr. Chauvin appears to have been
reprimanded and possibly suspended after a woman complained in 2007 that
he needlessly removed her from her car. She said he searched her and put
her in the back of a squad car for driving 10 miles per hour over the
speed limit.

Mr. Chauvin was also the subject of
\href{https://www.nytimes3xbfgragh.onion/2020/05/30/us/derek-chauvin-george-floyd.html}{at
least 16 other misconduct complaints} over two decades.

Mr. Thao worked his entire life, starting at a McDonald's in the Twin
Cities suburb of Fridley when he was 14. After graduating from high
school, he became a stocker at a Cub Foods store in Crystal, Minn.

The following year, he started community college, taking classes like
Police and Community and Minority Groups, transcripts show. After
briefly working as a security guard, he dropped out of college about the
same time he was hired as a community service officer with the police,
in February 2008. He was laid off during budget cuts in December 2009,
three days after becoming a full-fledged officer, and rehired in 2012.

Over his career, Mr. Thao faced
\href{http://www.minneapolismn.gov/civilrights/policereview/cra_links-contacts}{at
least six complaints}. He and his former partner were also sued in 2017
by an African-American man who claimed they punched, kicked and kneed
him, leaving him with broken teeth and bruises.

Mr. Kueng graduated from high school in 2012. He then enrolled at Monroe
College, where he studied business administration before transferring to
a community college. He decided to major in criminal justice studies as
he worked stocking shelves at Target and then as a loss-prevention
detective at Macy's.

Eventually, Mr. Kueng transferred to the University of Minnesota, where
he majored in the sociology of law, criminology and deviance,
transcripts showed. He seemed to want to tackle the world: He studied
Russian and took classes like World Religions and World Politics and
Terrorist Networks. In spring 2018, he also enrolled in American Race
Relations.

After graduating in summer 2018, Mr. Kueng got his law enforcement
certificate. By this point, he was already working part-time as a
community service officer for Minneapolis police. He joined the police
academy, and became an officer in December. There are no records of any
complaints against him.

Mr. Lane came to policing later in life. He did not graduate from high
school,
\href{http://www.ci.minneapolis.mn.us/www/groups/public/@mpd/documents/webcontent/wcmsp-224779.pdf}{his
personnel file shows}, but he went on to get his G.E.D. He worked a
series of odd jobs from restaurant server to Home Depot sales associate
to nightclub bouncer. He eventually went to Century College; in 2016,
Mr. Lane graduated from the University of Minnesota with a bachelor's
degree in the sociology of law, criminology and deviance.

He also volunteered, helping Somali youth in the Cedar Riverside
community.

Mr. Lane started working in the criminal justice system as an assistant
probation officer for juveniles and as a juvenile correctional officer.
But in January 2019, he was accepted to the police academy, becoming an
officer in December, records show.

While in college, Mr. Lane worked security from 2015 to 2016 at the
Exchange, a nightclub in downtown Minneapolis. Ben Quam, who was the
general manager, said Mr. Lane was so calm, regardless of the situation,
that co-workers ``made fun of him for always being so boring and
even-keeled.''

The club usually had police officers stationed outside. Mr. Quam said he
believed that gave Mr. Lane a chance to speak to them, building his
interest in becoming an officer.

Mr. Quam said that he had not kept in touch with Mr. Lane, and that he
was still trying to process how the man he knew could be involved ``in
one of the most horrific things I've seen.''

``Everybody's just heartbroken that he could have been involved in
something so awful,'' Mr. Quam said.

Kim Barker, John Eligon and Matt Furber reported from Minneapolis and
Richard A. Oppel Jr. from New York. Tim Arango contributed reporting
from Los Angeles, and Dan Levin and Nicholas Bogel-Burroughs from New
York. Susan Beachy contributed research.

Advertisement

\protect\hyperlink{after-bottom}{Continue reading the main story}

\hypertarget{site-index}{%
\subsection{Site Index}\label{site-index}}

\hypertarget{site-information-navigation}{%
\subsection{Site Information
Navigation}\label{site-information-navigation}}

\begin{itemize}
\tightlist
\item
  \href{https://help.nytimes3xbfgragh.onion/hc/en-us/articles/115014792127-Copyright-notice}{©~2020~The
  New York Times Company}
\end{itemize}

\begin{itemize}
\tightlist
\item
  \href{https://www.nytco.com/}{NYTCo}
\item
  \href{https://help.nytimes3xbfgragh.onion/hc/en-us/articles/115015385887-Contact-Us}{Contact
  Us}
\item
  \href{https://www.nytco.com/careers/}{Work with us}
\item
  \href{https://nytmediakit.com/}{Advertise}
\item
  \href{http://www.tbrandstudio.com/}{T Brand Studio}
\item
  \href{https://www.nytimes3xbfgragh.onion/privacy/cookie-policy\#how-do-i-manage-trackers}{Your
  Ad Choices}
\item
  \href{https://www.nytimes3xbfgragh.onion/privacy}{Privacy}
\item
  \href{https://help.nytimes3xbfgragh.onion/hc/en-us/articles/115014893428-Terms-of-service}{Terms
  of Service}
\item
  \href{https://help.nytimes3xbfgragh.onion/hc/en-us/articles/115014893968-Terms-of-sale}{Terms
  of Sale}
\item
  \href{https://spiderbites.nytimes3xbfgragh.onion}{Site Map}
\item
  \href{https://help.nytimes3xbfgragh.onion/hc/en-us}{Help}
\item
  \href{https://www.nytimes3xbfgragh.onion/subscription?campaignId=37WXW}{Subscriptions}
\end{itemize}
