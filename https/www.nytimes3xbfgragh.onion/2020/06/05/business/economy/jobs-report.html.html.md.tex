Sections

SEARCH

\protect\hyperlink{site-content}{Skip to
content}\protect\hyperlink{site-index}{Skip to site index}

\href{https://www.nytimes3xbfgragh.onion/section/business/economy}{Economy}

\href{https://myaccount.nytimes3xbfgragh.onion/auth/login?response_type=cookie\&client_id=vi}{}

\href{https://www.nytimes3xbfgragh.onion/section/todayspaper}{Today's
Paper}

\href{/section/business/economy}{Economy}\textbar{}What to Make of the
Rebound in the U.S. Jobs Report

\url{https://nyti.ms/2AJPYPu}

\begin{itemize}
\item
\item
\item
\item
\item
\end{itemize}

\hypertarget{the-coronavirus-outbreak}{%
\subsubsection{\texorpdfstring{\href{https://www.nytimes3xbfgragh.onion/news-event/coronavirus?name=styln-coronavirus-markets\&region=TOP_BANNER\&variant=undefined\&block=storyline_menu_recirc\&action=click\&pgtype=Article\&impression_id=2b343a20-e385-11ea-b0af-051312a80ef8}{The
Coronavirus
Outbreak}}{The Coronavirus Outbreak}}\label{the-coronavirus-outbreak}}

\begin{itemize}
\tightlist
\item
  live\href{https://www.nytimes3xbfgragh.onion/2020/08/20/world/coronavirus-covid.html?name=styln-coronavirus-markets\&region=TOP_BANNER\&variant=undefined\&block=storyline_menu_recirc\&action=click\&pgtype=Article\&impression_id=2b343a21-e385-11ea-b0af-051312a80ef8}{Latest
  Updates}
\item
  \href{https://www.nytimes3xbfgragh.onion/interactive/2020/us/coronavirus-us-cases.html?name=styln-coronavirus-markets\&region=TOP_BANNER\&variant=undefined\&block=storyline_menu_recirc\&action=click\&pgtype=Article\&impression_id=2b346130-e385-11ea-b0af-051312a80ef8}{Maps
  and Cases}
\item
  \href{https://www.nytimes3xbfgragh.onion/interactive/2020/science/coronavirus-vaccine-tracker.html?name=styln-coronavirus-markets\&region=TOP_BANNER\&variant=undefined\&block=storyline_menu_recirc\&action=click\&pgtype=Article\&impression_id=2b346131-e385-11ea-b0af-051312a80ef8}{Vaccine
  Tracker}
\item
  \href{https://www.nytimes3xbfgragh.onion/2020/08/19/us/colleges-closing-covid.html?name=styln-coronavirus-markets\&region=TOP_BANNER\&variant=undefined\&block=storyline_menu_recirc\&action=click\&pgtype=Article\&impression_id=2b346132-e385-11ea-b0af-051312a80ef8}{Colleges
  Closing}
\item
  \href{https://www.nytimes3xbfgragh.onion/live/2020/08/20/business/stock-market-today-coronavirus?name=styln-coronavirus-markets\&region=TOP_BANNER\&variant=undefined\&block=storyline_menu_recirc\&action=click\&pgtype=Article\&impression_id=2b346133-e385-11ea-b0af-051312a80ef8}{Economy}
\end{itemize}

Advertisement

\protect\hyperlink{after-top}{Continue reading the main story}

Supported by

\protect\hyperlink{after-sponsor}{Continue reading the main story}

\hypertarget{what-to-make-of-the-rebound-in-the-us-jobs-report}{%
\section{What to Make of the Rebound in the U.S. Jobs
Report}\label{what-to-make-of-the-rebound-in-the-us-jobs-report}}

Many of the gains came in restaurant work as 2.5 million jobs were added
in May and the unemployment rate fell to 13.3 percent.

\hypertarget{cumulative-monthly-change-in-jobs-since-september-2010}{%
\subsubsection{Cumulative monthly change in jobs since September
2010}\label{cumulative-monthly-change-in-jobs-since-september-2010}}

\hypertarget{job-losses-in-march-and-april-nearly-wiped-out-the-previous-113-months-of-job-gains-but-may-showed-a-partial-comeback}{%
\paragraph{Job losses in March and April nearly wiped out the previous
113 months of job gains, but May showed a partial
comeback.}\label{job-losses-in-march-and-april-nearly-wiped-out-the-previous-113-months-of-job-gains-but-may-showed-a-partial-comeback}}

--1.4

million

in March

+20 million

+22,106,000 jobs

from Oct. 2010 through Feb. 2020

+15

--20.7

million

in April

+10

+5

+2.5

million

in May

0

`12

`14

`16

`18

`20

130 million total jobs

in September 2010

--1,373,000

in March

+20 million

+22,106,000 jobs

from Oct. 2010 through Feb. 2020

+15

--20,687,000

in April

+10

+5

130 million total jobs in Sept. 2010

+2,509,000

in May

0

`12

`14

`15

`17

`19

`11

`13

`16

`18

`20

By Ella Koeze·Source: Bureau of Labor Statistics

\href{https://www.nytimes3xbfgragh.onion/by/ben-casselman}{\includegraphics{https://static01.graylady3jvrrxbe.onion/images/2018/11/09/multimedia/author-ben-casselman/author-ben-casselman-thumbLarge.png}}

By \href{https://www.nytimes3xbfgragh.onion/by/ben-casselman}{Ben
Casselman}

\begin{itemize}
\item
  June 5, 2020
\item
  \begin{itemize}
  \item
  \item
  \item
  \item
  \item
  \end{itemize}
\end{itemize}

The job market halted its pandemic-induced collapse in May as employers
brought back millions of workers and the unemployment rate unexpectedly
declined.

Tens of millions are still out of work, and the unemployment rate, which
fell to 13.3 percent from 14.7 percent in April, remains worse than in
any previous postwar recession. The rate would have been higher had it
not been for data-collection issues.

Nonetheless, after weeks of data depicting enormous economic
destruction, Friday's report from the Labor Department offered a glimmer
of hope. Employers added 2.5 million jobs in May, defying economists'
expectations of further losses and holding the prospect that the rebound
from the economic crisis could be faster than forecast.

Job growth was concentrated in industries hit hardest early in the
crisis, like leisure, hospitality and retail work. But manufacturing,
health care and professional services added jobs as well, possibly
signaling that the damage did not spread as deeply into the economy as
many feared.

Major stock indexes surged on the news, and President Trump hailed the
report in remarks outside the White House, saying the rebound ``leads us
onto a long period of growth.''

``We will go back to having the greatest economy anywhere in the world,
nothing close, and I think we're going to have a very good upcoming few
months,'' Mr. Trump said.

All the same, economists warn that it will take far longer for the
economy to climb out of the hole than it did to fall into it.

And even as the economy shows signs of revival, the United States is
confirming more than 20,000 new coronavirus cases a day, with counts
rising in particular in the South and the West.

While employers recalled temporarily laid-off or furloughed workers in
May, the number of permanent job losses rose, a sign that some
businesses didn't survive the shutdown, or expect demand to stay
depressed as the economy reopens. Others are bringing back workers at
reduced hours: The number of people working part time because they
couldn't find full-time work barely budged. And millions more people
have been laid off in the weeks since the data released Friday was
collected in mid-May.

``The surprise to me in this report is that the recovery was earlier
than we expected, but the next question is whether it will be faster
than we expected,'' said Daniel Zhao, senior economist at the career
site Glassdoor. The increase in permanent job losses, he said, ``is a
concerning sign for the length of the recovery because every layoff that
turns permanent makes a full recovery harder.''

\hypertarget{latest-updates-the-coronavirus-outbreak-and-the-economy}{%
\section{\texorpdfstring{\href{https://www.nytimes3xbfgragh.onion/live/2020/08/20/business/stock-market-today-coronavirus?action=click\&pgtype=Article\&state=default\&region=MAIN_CONTENT_1\&context=storylines_live_updates}{Latest
Updates: The Coronavirus Outbreak and the
Economy}}{Latest Updates: The Coronavirus Outbreak and the Economy}}\label{latest-updates-the-coronavirus-outbreak-and-the-economy}}

\href{https://www.nytimes3xbfgragh.onion/live/2020/08/20/business/stock-market-today-coronavirus?action=click\&pgtype=Article\&state=default\&region=MAIN_CONTENT_1\&context=storylines_live_updates\#the-producer-of-unhinged-makes-a-big-bet-on-audiences-returning-to-theaters}{10h
ago}

\href{https://www.nytimes3xbfgragh.onion/live/2020/08/20/business/stock-market-today-coronavirus?action=click\&pgtype=Article\&state=default\&region=MAIN_CONTENT_1\&context=storylines_live_updates\#the-producer-of-unhinged-makes-a-big-bet-on-audiences-returning-to-theaters}{The
producer of `Unhinged' makes a big bet on audiences returning to
theaters.}

\href{https://www.nytimes3xbfgragh.onion/live/2020/08/20/business/stock-market-today-coronavirus?action=click\&pgtype=Article\&state=default\&region=MAIN_CONTENT_1\&context=storylines_live_updates\#american-airlines-to-stop-flights-to-15-cities-after-government-aid-ends}{18h
ago}

\href{https://www.nytimes3xbfgragh.onion/live/2020/08/20/business/stock-market-today-coronavirus?action=click\&pgtype=Article\&state=default\&region=MAIN_CONTENT_1\&context=storylines_live_updates\#american-airlines-to-stop-flights-to-15-cities-after-government-aid-ends}{American
Airlines to stop flights to 15 cities after government aid ends.}

\href{https://www.nytimes3xbfgragh.onion/live/2020/08/20/business/stock-market-today-coronavirus?action=click\&pgtype=Article\&state=default\&region=MAIN_CONTENT_1\&context=storylines_live_updates\#without-school-plays-and-assemblies-a-technicians-livelihood-withers}{19h
ago}

\href{https://www.nytimes3xbfgragh.onion/live/2020/08/20/business/stock-market-today-coronavirus?action=click\&pgtype=Article\&state=default\&region=MAIN_CONTENT_1\&context=storylines_live_updates\#without-school-plays-and-assemblies-a-technicians-livelihood-withers}{Without
school plays and assemblies, a technician's livelihood withers.}

\href{https://www.nytimes3xbfgragh.onion/live/2020/08/20/business/stock-market-today-coronavirus?action=click\&pgtype=Article\&state=default\&region=MAIN_CONTENT_1\&context=storylines_live_updates}{See
more updates}

More live coverage:
\href{https://www.nytimes3xbfgragh.onion/2020/08/20/world/coronavirus-covid.html?action=click\&pgtype=Article\&state=default\&region=MAIN_CONTENT_1\&context=storylines_live_updates}{Global}

Economists said the gains in May indicated that Congress and the Federal
Reserve had at least partly succeeded in limiting permanent economic
damage by providing trillions of dollars in assistance to households and
businesses.

But that aid is now in jeopardy, and economists warned that there was no
guarantee the job market would continue to improve without it.

``The economy is still being very much buffered by stimulus,'' said
Michelle Meyer, head of U.S. economics at Bank of America. ``When that
starts to wane, we will learn a lot more about the underlying health of
the recovery.''

The Labor Department cautioned that data-collection issues troubling the
agency throughout the crisis continued last month. Some temporarily
jobless workers were characterized as employed in May; had they been
counted correctly, the department said, the unemployment rate would have
topped 16 percent.

By Ella Koeze·Employment rates are seasonally adjusted. \textbar{}
Source: Bureau of Labor Statistics

The official unemployment rate also leaves out many people who are not
actively looking for jobs or are working reduced hours. A broader
measure of unemployment and underemployment, which includes those
groups, was 21.2 percent in May, down from April but higher than any
other month on record.

For some demographic groups, the crisis is deeper. Early job losses were
concentrated among low-wage workers in service jobs, many of them black
and Latino. The unemployment rate for black workers rose slightly, to
16.8 percent, although mostly because more people were looking for work.
The jobless rate for Latinos fell but was still the highest of any
racial or ethnic group, at 17.6 percent.

Even if they didn't anticipate the May bounce, many economists had
expected that unemployment would begin to ease as states reopened and
businesses called employees back to work.

More than half of the month's job gains --- 1.4 million --- were in
restaurants and bars, many of which received assistance under the
federal Paycheck Protection Program. Friday's report suggests that the
program, along with other elements of the government's response, helped
offset at least some of the economic damage caused by the shutdown,
which should allow for a faster recovery.

\emph{{[}How do you feel about going back to work?}
\href{https://www.nytimes3xbfgragh.onion/2020/05/15/us/as-cities-reopen-how-do-you-feel-about-going-out.html}{\emph{Share
your story}}\emph{.{]}}

``If it is the case that more people were able to hold on to their jobs
and remain attached to the labor force, then the recovery will be more
efficient,'' Ms. Meyer said. ``People will have more income to spend,
they will have more consistency as far as their employment, and
confidence will be higher.''

At Beef `O' Brady's, a Florida-based chain of more than 150 sports bars,
business was down 62 percent in April, when its dining rooms were closed
nationwide and its only business came from takeout. But only a handful
of the chain's restaurants have closed permanently, in part because
nearly all of its franchisees received Paycheck Protection Program
loans.

``The damage would have been much greater without P.P.P., I can tell you
that,'' said Chris Elliott, the chief executive.

Now business has begun to pick up as states gradually allow restaurants
to reopen. In the last week of May, sales were down about 15 percent,
Mr. Elliott said, and customers appear eager to eat out again.

The longer-run outlook is uncertain. If business stays at its current
level, many franchisees will struggle to eke out a profit, he said, and
many locations are losing money. That won't be sustainable.

``There are going to be franchise owners that if they can't reach 15
percent, or it doesn't improve incrementally over time, they're going to
get fatigued, and I think some of them are at risk of just throwing in
the towel,'' Mr. Elliott said.

At the same time, employment in nearly every sector remains far below
where it was before the crisis. Many economists expect an initial
resurgence in at least some kinds of business. But it isn't clear how
strong that upturn will be, or what will come after.

``It's the jump and then the crawl, and the question is how high is the
jump and then how long does the crawl take,'' said Nick Bunker, who
leads North American economic research at the Indeed Hiring Lab.

When Mike Lowe flew to Florida in early March to visit his mother, he
had a successful freelance business doing web and graphic design, and a
part-time gig in dog day care. A week later, he arrived home in
Portland, Ore., to a text message from the dog business telling him not
to go into work. He was let go entirely within days, even as his
freelance clients began calling to cancel orders.

Two and a half months later, Oregon has begun to reopen, but Mr. Lowe,
51, is treading water. One freelance project looks likely to resume
soon, but another client, a local bar, told him this week that it would
shut down permanently --- its business relied on live music, which seems
unlikely to come back anytime soon. The owner of the dog day care says
she hopes to bring him back at reduced hours but isn't sure when
business will rebound sufficiently to make that possible.

``I'd say I'm just in wait-and-see mode at the moment,'' he said.

\hypertarget{percent-change-in-jobs-from-previous-month-by-industry}{%
\subsubsection{Percent change in jobs from previous month, by
industry}\label{percent-change-in-jobs-from-previous-month-by-industry}}

\hypertarget{the-leisure-and-hospitality-industry-which-includes-restaurants-had-a-severe-loss-in-april-but-bounced-back-the-most-in-may}{%
\paragraph{The leisure and hospitality industry, which includes
restaurants, had a severe loss in April but bounced back the most in
May.}\label{the-leisure-and-hospitality-industry-which-includes-restaurants-had-a-severe-loss-in-april-but-bounced-back-the-most-in-may}}

By Ella Koeze·Source: Bureau of Labor Statistics

Like many laid-off workers, Mr. Lowe is able to get by largely because
of the \$600 a week in extra unemployment pay that Congress approved as
part of its emergency funding bill in March. But that benefit is set to
run out at the end of July, and it is far from clear that Congress will
extend it. Economists warn that pulling away support too early could
stall the recovery.

``Right now, the government is plugging a good deal of that hole for
households, but how long will that last, we don't know,'' said Ellen
Zentner, chief U.S. economist for Morgan Stanley. She noted that the
unemployment rate will almost certainly remain elevated in August,
adding, ``It's a very tough time to pull support away from households
when the unemployment rate is still that high.''

Even now, as some businesses start to bring back workers, layoffs are
continuing.
\href{https://www.nytimes3xbfgragh.onion/2020/06/04/business/economy/coronavirus-unemployment-claims.html}{Nearly
two million workers} filed first-time claims for state unemployment
benefits last week, more than double the worst week of any previous
recession. State and local governments cut nearly half a million jobs in
May, and millions more such layoffs are likely in coming months in
response to plunging tax revenues.

In Jackson, Mich., a small city about 70 miles west of Detroit, the
school board voted last week to cut more than 40 positions in response
to a multimillion-dollar budget shortfall. Jeff Beal, the district's
superintendent, said he worried about the impact the cuts would have on
education and on the local economy. But he said the district had little
choice.

Among the cuts: the assistant superintendent for human resources, which
means Mr. Beal will have to inform laid-off workers himself.

``Now that that position has been eliminated, that responsibility falls
to me,'' he said. ``I'm going to have to make a lot of very personal,
very painful phone calls this week.''

Katie Rogers contributed reporting.

Advertisement

\protect\hyperlink{after-bottom}{Continue reading the main story}

\hypertarget{site-index}{%
\subsection{Site Index}\label{site-index}}

\hypertarget{site-information-navigation}{%
\subsection{Site Information
Navigation}\label{site-information-navigation}}

\begin{itemize}
\tightlist
\item
  \href{https://help.nytimes3xbfgragh.onion/hc/en-us/articles/115014792127-Copyright-notice}{©~2020~The
  New York Times Company}
\end{itemize}

\begin{itemize}
\tightlist
\item
  \href{https://www.nytco.com/}{NYTCo}
\item
  \href{https://help.nytimes3xbfgragh.onion/hc/en-us/articles/115015385887-Contact-Us}{Contact
  Us}
\item
  \href{https://www.nytco.com/careers/}{Work with us}
\item
  \href{https://nytmediakit.com/}{Advertise}
\item
  \href{http://www.tbrandstudio.com/}{T Brand Studio}
\item
  \href{https://www.nytimes3xbfgragh.onion/privacy/cookie-policy\#how-do-i-manage-trackers}{Your
  Ad Choices}
\item
  \href{https://www.nytimes3xbfgragh.onion/privacy}{Privacy}
\item
  \href{https://help.nytimes3xbfgragh.onion/hc/en-us/articles/115014893428-Terms-of-service}{Terms
  of Service}
\item
  \href{https://help.nytimes3xbfgragh.onion/hc/en-us/articles/115014893968-Terms-of-sale}{Terms
  of Sale}
\item
  \href{https://spiderbites.nytimes3xbfgragh.onion}{Site Map}
\item
  \href{https://help.nytimes3xbfgragh.onion/hc/en-us}{Help}
\item
  \href{https://www.nytimes3xbfgragh.onion/subscription?campaignId=37WXW}{Subscriptions}
\end{itemize}
