Sections

SEARCH

\protect\hyperlink{site-content}{Skip to
content}\protect\hyperlink{site-index}{Skip to site index}

\href{https://myaccount.nytimes3xbfgragh.onion/auth/login?response_type=cookie\&client_id=vi}{}

\href{https://www.nytimes3xbfgragh.onion/section/todayspaper}{Today's
Paper}

\href{/section/opinion}{Opinion}\textbar{}`Get Your Knee Off Our Necks'

\url{https://nyti.ms/3h0uj6w}

\begin{itemize}
\item
\item
\item
\item
\item
\item
\end{itemize}

Advertisement

\protect\hyperlink{after-top}{Continue reading the main story}

\href{/section/opinion}{Opinion}

Supported by

\protect\hyperlink{after-sponsor}{Continue reading the main story}

\hypertarget{get-your-knee-off-our-necks}{%
\section{`Get Your Knee Off Our
Necks'}\label{get-your-knee-off-our-necks}}

All that is needed for rebellion against relentless oppression is a
spark.

\href{https://www.nytimes3xbfgragh.onion/by/roger-cohen}{\includegraphics{https://static01.graylady3jvrrxbe.onion/images/2014/11/01/opinion/cohen-circular/cohen-circular-thumbLarge-v6.png}}

By \href{https://www.nytimes3xbfgragh.onion/by/roger-cohen}{Roger Cohen}

Opinion Columnist

\begin{itemize}
\item
  June 5, 2020
\item
  \begin{itemize}
  \item
  \item
  \item
  \item
  \item
  \item
  \end{itemize}
\end{itemize}

\includegraphics{https://static01.graylady3jvrrxbe.onion/images/2020/06/05/opinion/05cohen1/merlin_173208594_c111b73f-f8e7-4643-85ae-183b13eb98de-articleLarge.jpg?quality=75\&auto=webp\&disable=upscale}

\href{https://cn.nytimes3xbfgragh.onion/opinion/20200610/george-floyd-protests/}{阅读简体中文版}\href{https://cn.nytimes3xbfgragh.onion/opinion/20200610/george-floyd-protests/zh-hant/}{閱讀繁體中文版}

\emph{Get your knee off our necks.}

Marcus Delespinasse, weary-eyed, stands on Broadway in the late
afternoon. ``The culture of America,'' he tells me after I approached
him on the street, ``is that it's OK to treat blacks this way. That cop
knew George Floyd would not make it. He still kept his knee there.''

Yes, Derek Chauvin, who has been charged with second-degree murder, kept
his knee on Floyd's neck
\href{https://www.nytimes3xbfgragh.onion/2020/05/31/us/george-floyd-investigation.html}{for
almost nine minutes}. A powerful white man asphyxiating a powerless
black man, a scene with a long American history, part of the nation's
iconography. Chauvin was a \emph{training officer} for the other cops at
the scene. His blithe expression said, ``Watch me kill.''

``Get your knee off our necks,'' is the Rev. Al Sharpton's phrase for
the uprising of 2020. The ``knee'' has been there for a while. It was in
the Constitution's three-fifths clause that set the census value of a
slave at 60 percent of a free human being. The ``knee'' is slavery and
Jim Crow and lynching and segregation in schools and transportation and
neighborhoods and on and on and on through all the inflections of
systemic state oppression of African-Americans that allowed Chauvin to
believe he had the right as a white man to do what he did.

``Being black in America should not be a death sentence,'' Jacob Frey,
the mayor of Minneapolis, the
\href{https://www.nytimes3xbfgragh.onion/2020/06/01/us/minneapolis-racism-minnesota.html}{liberal
city where Floyd was killed}, said. He had to say that more than a
half-century after the civil rights movement. Think about it.

Get your knee off our necks.

Aged 52 and unemployed, Delespinasse is black. I feel despair as I write
that sentence. So-and-so is white. So-and-so is black. All those
parentheses running through copy, the refrain of failure. To explain
what exactly? America's societal fracture; America's original sin;
America's shame that life, liberty and the pursuit of happiness have
never been equally afforded its citizens. I might have written: Aged 52
and unemployed, Delespinasse is a human being. The likes of Chauvin
still cannot see that.

``You look at that video and think that could be me, or my cousin, or my
uncle,'' Delespinasse tells me. ``Police have impunity. No wonder young
people are enraged. That cop with his knee resting there sums up the
savageness of white apathy.''

Delespinasse looks out with those weary eyes on a ghostly New York.
First the hum-and-honking of the city gave way to pandemic-induced
silence interspersed with ambulance sirens. Now, after the looting, the
sound of New York is the screeching of electric saws cutting plywood to
board up broken windows and the rumbling bursts of electric screwdrivers
fixing the panels in place. This is the audio of a great city's
disaster. This is the audio of a virus that sharpened the inequities of
American dysfunction.

Get your knee off our necks.

There is no right to pillage and burn in the United States. But human
beings will react to entrenched state violence, in extreme cases a
license to kill, which is what black Americans have confronted for
centuries. All that is needed for rebellion against relentless
oppression is a spark.
\href{https://www.poetryfoundation.org/poems/46548/harlem}{What happens
to a dream deferred,} asked Langston Hughes? ``\emph{Does it explode?''}

\emph{The savageness of white apathy}: a striking phrase, and sometimes
it is worse than apathy. Consider Amy Cooper, that highly educated white
woman caught on video in Central Park. She found herself saying she
would tell the police there is ``an African-American man threatening my
life.'' Because a black man, Christian Cooper (no relation), an avid
birder, had properly asked her to leash her dog. It's important to call
such racist aggression by its name.

Those impulses are what President Trump, a racist who
\href{https://www.washingtonpost.com/news/the-fix/wp/2017/06/16/theyre-rapists-presidents-trump-campaign-launch-speech-two-years-later-annotated/}{launched
his successful campaign} in 2015 by calling Mexicans entering the
country ``rapists,'' plays on. Violence and division are his elements.
He has no other. Hence his recent threat to deploy the military to quash
``domestic terror,'' his repeated talk of ``domination,'' his
encouragement to violence couched in endless references to Second
Amendment rights, and his tweeting support for Senator Tom Cotton, a
prominent Republican, who
\href{https://twitter.com/TomCottonAR/status/1267459561675468800}{called
in a tweet} for the deployment of ``10th Mountain, 82nd Airborne, 1st
Cav, 3rd Infantry --- whatever it takes to restore order.''

Whatever it takes to do what? To stop the lawbreakers and looters, Trump
and Cotton would say with breathtaking disingenuousness. The military is
not needed for that.

No, the point would be this: to assert with a great show of force, after
the slow-motion murder of George Floyd by a white police officer, that
the oppressive system that produced this act is not about to change and
armed white male power in America is inviolable. That is Trump's
fundamental credo. His Bible-brandishing, American Gothic portrait this
week outside St. John's Episcopal Church in Washington is one of the
most disturbing portraits of psychopathic self-importance seen since
1933.

Get your knee off our necks --- and American democracy.

Trump was widely dismissed in 2015. He was dismissed in 2016, for that
matter, until he won. A fringe loony, he would burn out. Turned out tens
of millions of Americans thought like him.

Cotton followed up on his tweet with his now infamous
\href{https://www.nytimes3xbfgragh.onion/2020/06/03/opinion/tom-cotton-protests-military.html}{send-in-the-troops
Op-Ed} in The New York Times. The piece was wrong, repugnant, mistimed
and flawed. It was also extremely relevant and very dangerous to ignore.
I prefer to read it and vote with rage than experience again, in
November, the consequences of complacent liberal ignorance.

\emph{The Times is committed to publishing}
\href{https://www.nytimes3xbfgragh.onion/2019/01/31/opinion/letters/letters-to-editor-new-york-times-women.html}{\emph{a
diversity of letters}} \emph{to the editor. We'd like to hear what you
think about this or any of our articles. Here are some}
\href{https://help.nytimes3xbfgragh.onion/hc/en-us/articles/115014925288-How-to-submit-a-letter-to-the-editor}{\emph{tips}}\emph{.
And here's our email:}
\href{mailto:letters@NYTimes.com}{\emph{letters@NYTimes.com}}\emph{.}

\emph{Follow The New York Times Opinion section on}
\href{https://www.facebookcorewwwi.onion/nytopinion}{\emph{Facebook}}\emph{,}
\href{http://twitter.com/NYTOpinion}{\emph{Twitter (@NYTopinion)}}
\emph{and}
\href{https://www.instagram.com/nytopinion/}{\emph{Instagram}}\emph{.}

Advertisement

\protect\hyperlink{after-bottom}{Continue reading the main story}

\hypertarget{site-index}{%
\subsection{Site Index}\label{site-index}}

\hypertarget{site-information-navigation}{%
\subsection{Site Information
Navigation}\label{site-information-navigation}}

\begin{itemize}
\tightlist
\item
  \href{https://help.nytimes3xbfgragh.onion/hc/en-us/articles/115014792127-Copyright-notice}{©~2020~The
  New York Times Company}
\end{itemize}

\begin{itemize}
\tightlist
\item
  \href{https://www.nytco.com/}{NYTCo}
\item
  \href{https://help.nytimes3xbfgragh.onion/hc/en-us/articles/115015385887-Contact-Us}{Contact
  Us}
\item
  \href{https://www.nytco.com/careers/}{Work with us}
\item
  \href{https://nytmediakit.com/}{Advertise}
\item
  \href{http://www.tbrandstudio.com/}{T Brand Studio}
\item
  \href{https://www.nytimes3xbfgragh.onion/privacy/cookie-policy\#how-do-i-manage-trackers}{Your
  Ad Choices}
\item
  \href{https://www.nytimes3xbfgragh.onion/privacy}{Privacy}
\item
  \href{https://help.nytimes3xbfgragh.onion/hc/en-us/articles/115014893428-Terms-of-service}{Terms
  of Service}
\item
  \href{https://help.nytimes3xbfgragh.onion/hc/en-us/articles/115014893968-Terms-of-sale}{Terms
  of Sale}
\item
  \href{https://spiderbites.nytimes3xbfgragh.onion}{Site Map}
\item
  \href{https://help.nytimes3xbfgragh.onion/hc/en-us}{Help}
\item
  \href{https://www.nytimes3xbfgragh.onion/subscription?campaignId=37WXW}{Subscriptions}
\end{itemize}
