Sections

SEARCH

\protect\hyperlink{site-content}{Skip to
content}\protect\hyperlink{site-index}{Skip to site index}

\href{https://www.nytimes3xbfgragh.onion/section/us}{U.S.}

\href{https://myaccount.nytimes3xbfgragh.onion/auth/login?response_type=cookie\&client_id=vi}{}

\href{https://www.nytimes3xbfgragh.onion/section/todayspaper}{Today's
Paper}

\href{/section/us}{U.S.}\textbar{}Research Shows Students Falling Months
Behind During Virus Disruptions

\url{https://nyti.ms/3cHuCQp}

\begin{itemize}
\item
\item
\item
\item
\item
\end{itemize}

\href{https://www.nytimes3xbfgragh.onion/news-event/coronavirus?action=click\&pgtype=Article\&state=default\&region=TOP_BANNER\&context=storylines_menu}{The
Coronavirus Outbreak}

\begin{itemize}
\tightlist
\item
  live\href{https://www.nytimes3xbfgragh.onion/2020/08/04/world/coronavirus-cases.html?action=click\&pgtype=Article\&state=default\&region=TOP_BANNER\&context=storylines_menu}{Latest
  Updates}
\item
  \href{https://www.nytimes3xbfgragh.onion/interactive/2020/us/coronavirus-us-cases.html?action=click\&pgtype=Article\&state=default\&region=TOP_BANNER\&context=storylines_menu}{Maps
  and Cases}
\item
  \href{https://www.nytimes3xbfgragh.onion/interactive/2020/science/coronavirus-vaccine-tracker.html?action=click\&pgtype=Article\&state=default\&region=TOP_BANNER\&context=storylines_menu}{Vaccine
  Tracker}
\item
  \href{https://www.nytimes3xbfgragh.onion/2020/08/02/us/covid-college-reopening.html?action=click\&pgtype=Article\&state=default\&region=TOP_BANNER\&context=storylines_menu}{College
  Reopening}
\item
  \href{https://www.nytimes3xbfgragh.onion/live/2020/08/04/business/stock-market-today-coronavirus?action=click\&pgtype=Article\&state=default\&region=TOP_BANNER\&context=storylines_menu}{Economy}
\end{itemize}

Advertisement

\protect\hyperlink{after-top}{Continue reading the main story}

Supported by

\protect\hyperlink{after-sponsor}{Continue reading the main story}

\hypertarget{research-shows-students-falling-months-behind-during-virus-disruptions}{%
\section{Research Shows Students Falling Months Behind During Virus
Disruptions}\label{research-shows-students-falling-months-behind-during-virus-disruptions}}

The abrupt switch to remote learning wiped out academic gains for many
students in America, and widened racial and economic gaps. Catching up
in the fall won't be easy.

\includegraphics{https://static01.graylady3jvrrxbe.onion/images/2020/06/03/us/00VIRUS-LOSTLEARNING-gandy/merlin_173084370_74fdc4d9-cd76-49a9-875d-0a4e239cd615-articleLarge.jpg?quality=75\&auto=webp\&disable=upscale}

\href{https://www.nytimes3xbfgragh.onion/by/dana-goldstein}{\includegraphics{https://static01.graylady3jvrrxbe.onion/images/2018/06/12/multimedia/author-dana-goldstein/author-dana-goldstein-thumbLarge.png}}

By \href{https://www.nytimes3xbfgragh.onion/by/dana-goldstein}{Dana
Goldstein}

\begin{itemize}
\item
  Published June 5, 2020Updated June 10, 2020
\item
  \begin{itemize}
  \item
  \item
  \item
  \item
  \item
  \end{itemize}
\end{itemize}

While a nation of burned-out,
\href{https://www.nytimes3xbfgragh.onion/2020/04/06/us/coronavirus-schools-attendance-absent.html}{involuntary
home schoolers} slogs to the finish line of a disrupted academic year, a
picture is emerging of the extent of the learning loss among children in
America, and the size of the gaps schools will be asked to fill when
they reopen.

It is not pretty.

New research suggests that by September, most students will have fallen
behind where they would have been if they had stayed in classrooms, with
some losing the equivalent of a full school year's worth of academic
gains. Racial and socioeconomic achievement gaps will most likely widen
\href{https://www.nytimes3xbfgragh.onion/2020/05/09/us/coronavirus-public-private-school.html}{because
of disparities} in access to computers, home internet connections and
direct instruction from teachers.

And the crisis is far from over. The harm to students could grow if
schools continue to teach fully or partly online in the fall, or if they
reopen with significant budget cuts because of the economic downturn.
High school dropout rates could increase, researchers say, while younger
children could miss out on foundational concepts in phonics and
fractions that prepare them for a lifetime of learning and working.

In South Los Angeles, Danielle Gandy has spent countless difficult hours
guiding her energetic 6-year-old, Cadynce, through online meetings and
assignments provided by her charter school. Still, Ms. Gandy is under no
illusion that Cadynce has completed the normal kindergarten curriculum,
and is especially concerned about her progress in math.

``Looking at the work the teacher has done, I applaud her,'' Ms. Gandy
said, ``but it's maybe a fraction of what they would be learning if they
were in an actual school setting. If they are transitioning into first
grade, will there be time to catch up and get them up to par?''

Teachers across the country share such worries. In Aurora, Colo.,
outside Denver, Clint Silva, a seventh-grade social studies teacher, was
planning to spend the spring working with his students on research
skills. For one remote assignment, he asked them to create a primary
source about
\href{https://www.nytimes3xbfgragh.onion/news-event/coronavirus}{the
pandemic} that future historians could consult.

\hypertarget{latest-updates-global-coronavirus-outbreak}{%
\section{\texorpdfstring{\href{https://www.nytimes3xbfgragh.onion/2020/08/04/world/coronavirus-cases.html?action=click\&pgtype=Article\&state=default\&region=MAIN_CONTENT_1\&context=storylines_live_updates}{Latest
Updates: Global Coronavirus
Outbreak}}{Latest Updates: Global Coronavirus Outbreak}}\label{latest-updates-global-coronavirus-outbreak}}

Updated 2020-08-04T21:18:49.389Z

\begin{itemize}
\tightlist
\item
  \href{https://www.nytimes3xbfgragh.onion/2020/08/04/world/coronavirus-cases.html?action=click\&pgtype=Article\&state=default\&region=MAIN_CONTENT_1\&context=storylines_live_updates\#link-1228a480}{Novavax
  sees encouraging results from two studies of its experimental
  vaccine.}
\item
  \href{https://www.nytimes3xbfgragh.onion/2020/08/04/world/coronavirus-cases.html?action=click\&pgtype=Article\&state=default\&region=MAIN_CONTENT_1\&context=storylines_live_updates\#link-4825b93}{Public
  and private schools in Maryland and elsewhere are divided over
  in-person instruction.}
\item
  \href{https://www.nytimes3xbfgragh.onion/2020/08/04/world/coronavirus-cases.html?action=click\&pgtype=Article\&state=default\&region=MAIN_CONTENT_1\&context=storylines_live_updates\#link-50f7386d}{The
  United Nations calls on policymakers to `plan thoroughly for school
  reopenings.'}
\end{itemize}

\href{https://www.nytimes3xbfgragh.onion/2020/08/04/world/coronavirus-cases.html?action=click\&pgtype=Article\&state=default\&region=MAIN_CONTENT_1\&context=storylines_live_updates}{See
more updates}

More live coverage:
\href{https://www.nytimes3xbfgragh.onion/live/2020/08/04/business/stock-market-today-coronavirus?action=click\&pgtype=Article\&state=default\&region=MAIN_CONTENT_1\&context=storylines_live_updates}{Markets}

But a majority of his students have not consistently engaged with remote
assignments. They are
\href{https://www.nytimes3xbfgragh.onion/2020/04/30/us/coronavirus-high-school-grades.html}{not
receiving traditional grades}, and some have parents who are working
outside the home or who are not tech-savvy, and are unable to assist
with online schooling.

``We know this isn't a good way to teach,'' Mr. Silva said. ``We want to
hold kids accountable. We want to see their progress, be in the
classroom with them and see them struggle and overcome that. Instead, we
are logging in for an hour a day, and kids are turning their cameras off
and staying quiet and not talking to us.''

Research can now estimate the size of the learning loss students have
experienced under such conditions. Because regular standardized testing
has been suspended, some of the research uses past disruptions to
learning --- such as natural disasters or even summer break --- to
project the potential impact of the current crisis. Other studies look
at schools that used online learning software before the coronavirus
shutdown, and check to see how students performed using the same
programs from home.

The average student could begin the next school year having lost as much
as a third of the expected progress from the previous year in reading
and half of the expected progress in math, according to a
\href{https://www.edworkingpapers.com/ai20-226}{working paper} from
NWEA, a nonprofit organization, and scholars at Brown University and the
University of Virginia.

A separate analysis of 800,000 students from
\href{https://tracktherecovery.org/}{researchers at Brown and Harvard}
looked at how Zearn, an online math program, was used both before and
after schools closed in March. It found that through late April, student
progress in math decreased by about half in classrooms located in
low-income ZIP codes, by a third in classrooms in middle-income ZIP
codes and not at all in classrooms in high-income ZIP codes.

When all of the impacts are taken into account, the average student
could fall seven months behind academically, while black and Hispanic
students could experience even greater learning losses, equivalent to 10
months for black children and nine months for Latinos, according to
\href{https://www.mckinsey.com/industries/public-sector/our-insights/covid-19-and-student-learning-in-the-united-states-the-hurt-could-last-a-lifetime}{an
analysis from McKinsey \& Company}, the consulting group.

There are several reasons low-income, black and Hispanic students appear
to be suffering the most through the crisis. The Center on Reinventing
Public Education, a think tank, will release an analysis next week of
the pandemic learning policies of 477 school districts. It found that
only a fifth have required live teaching over video, and that wealthy
school districts were twice as likely to provide such teaching as
low-income districts.

Rural students have been especially cut off from their teachers. Only 27
percent of their districts required any instruction while schools were
closed, according to the center.

While almost every school has provided assignments for students to
complete independently, that does not necessarily mean that teachers
conducted remote lessons. Schools with many poor students sometimes
chose to relax instructional expectations on teachers because they knew
families did not have reliable access to home computers or internet
connections able to stream video.

The disparities in educational progress do not appear to be caused by
any lack of effort on the part of families. The poorest parents spent
about the same amount of time during school closures assisting their
children with learning --- 13 hours per week --- as those making over
\$200,000 per year, according to a May
\href{https://www.census.gov/data/tables/2020/demo/hhp2.html}{Census
Bureau survey} of households with children.

\href{https://www.nytimes3xbfgragh.onion/news-event/coronavirus?action=click\&pgtype=Article\&state=default\&region=MAIN_CONTENT_3\&context=storylines_faq}{}

\hypertarget{the-coronavirus-outbreak-}{%
\subsubsection{The Coronavirus Outbreak
›}\label{the-coronavirus-outbreak-}}

\hypertarget{frequently-asked-questions}{%
\paragraph{Frequently Asked
Questions}\label{frequently-asked-questions}}

Updated August 4, 2020

\begin{itemize}
\item ~
  \hypertarget{i-have-antibodies-am-i-now-immune}{%
  \paragraph{I have antibodies. Am I now
  immune?}\label{i-have-antibodies-am-i-now-immune}}

  \begin{itemize}
  \tightlist
  \item
    As of right
    now,\href{https://www.nytimes3xbfgragh.onion/2020/07/22/health/covid-antibodies-herd-immunity.html?action=click\&pgtype=Article\&state=default\&region=MAIN_CONTENT_3\&context=storylines_faq}{that
    seems likely, for at least several months.} There have been
    frightening accounts of people suffering what seems to be a second
    bout of Covid-19. But experts say these patients may have a
    drawn-out course of infection, with the virus taking a slow toll
    weeks to months after initial exposure. People infected with the
    coronavirus typically
    \href{https://www.nature.com/articles/s41586-020-2456-9}{produce}
    immune molecules called antibodies, which are
    \href{https://www.nytimes3xbfgragh.onion/2020/05/07/health/coronavirus-antibody-prevalence.html?action=click\&pgtype=Article\&state=default\&region=MAIN_CONTENT_3\&context=storylines_faq}{protective
    proteins made in response to an
    infection}\href{https://www.nytimes3xbfgragh.onion/2020/05/07/health/coronavirus-antibody-prevalence.html?action=click\&pgtype=Article\&state=default\&region=MAIN_CONTENT_3\&context=storylines_faq}{.
    These antibodies may} last in the body
    \href{https://www.nature.com/articles/s41591-020-0965-6}{only two to
    three months}, which may seem worrisome, but that's perfectly normal
    after an acute infection subsides, said Dr. Michael Mina, an
    immunologist at Harvard University. It may be possible to get the
    coronavirus again, but it's highly unlikely that it would be
    possible in a short window of time from initial infection or make
    people sicker the second time.
  \end{itemize}
\item ~
  \hypertarget{im-a-small-business-owner-can-i-get-relief}{%
  \paragraph{I'm a small-business owner. Can I get
  relief?}\label{im-a-small-business-owner-can-i-get-relief}}

  \begin{itemize}
  \tightlist
  \item
    The
    \href{https://www.nytimes3xbfgragh.onion/article/small-business-loans-stimulus-grants-freelancers-coronavirus.html?action=click\&pgtype=Article\&state=default\&region=MAIN_CONTENT_3\&context=storylines_faq}{stimulus
    bills enacted in March} offer help for the millions of American
    small businesses. Those eligible for aid are businesses and
    nonprofit organizations with fewer than 500 workers, including sole
    proprietorships, independent contractors and freelancers. Some
    larger companies in some industries are also eligible. The help
    being offered, which is being managed by the Small Business
    Administration, includes the Paycheck Protection Program and the
    Economic Injury Disaster Loan program. But lots of folks have
    \href{https://www.nytimes3xbfgragh.onion/interactive/2020/05/07/business/small-business-loans-coronavirus.html?action=click\&pgtype=Article\&state=default\&region=MAIN_CONTENT_3\&context=storylines_faq}{not
    yet seen payouts.} Even those who have received help are confused:
    The rules are draconian, and some are stuck sitting on
    \href{https://www.nytimes3xbfgragh.onion/2020/05/02/business/economy/loans-coronavirus-small-business.html?action=click\&pgtype=Article\&state=default\&region=MAIN_CONTENT_3\&context=storylines_faq}{money
    they don't know how to use.} Many small-business owners are getting
    less than they expected or
    \href{https://www.nytimes3xbfgragh.onion/2020/06/10/business/Small-business-loans-ppp.html?action=click\&pgtype=Article\&state=default\&region=MAIN_CONTENT_3\&context=storylines_faq}{not
    hearing anything at all.}
  \end{itemize}
\item ~
  \hypertarget{what-are-my-rights-if-i-am-worried-about-going-back-to-work}{%
  \paragraph{What are my rights if I am worried about going back to
  work?}\label{what-are-my-rights-if-i-am-worried-about-going-back-to-work}}

  \begin{itemize}
  \tightlist
  \item
    Employers have to provide
    \href{https://www.osha.gov/SLTC/covid-19/standards.html}{a safe
    workplace} with policies that protect everyone equally.
    \href{https://www.nytimes3xbfgragh.onion/article/coronavirus-money-unemployment.html?action=click\&pgtype=Article\&state=default\&region=MAIN_CONTENT_3\&context=storylines_faq}{And
    if one of your co-workers tests positive for the coronavirus, the
    C.D.C.} has said that
    \href{https://www.cdc.gov/coronavirus/2019-ncov/community/guidance-business-response.html}{employers
    should tell their employees} -\/- without giving you the sick
    employee's name -\/- that they may have been exposed to the virus.
  \end{itemize}
\item ~
  \hypertarget{should-i-refinance-my-mortgage}{%
  \paragraph{Should I refinance my
  mortgage?}\label{should-i-refinance-my-mortgage}}

  \begin{itemize}
  \tightlist
  \item
    \href{https://www.nytimes3xbfgragh.onion/article/coronavirus-money-unemployment.html?action=click\&pgtype=Article\&state=default\&region=MAIN_CONTENT_3\&context=storylines_faq}{It
    could be a good idea,} because mortgage rates have
    \href{https://www.nytimes3xbfgragh.onion/2020/07/16/business/mortgage-rates-below-3-percent.html?action=click\&pgtype=Article\&state=default\&region=MAIN_CONTENT_3\&context=storylines_faq}{never
    been lower.} Refinancing requests have pushed mortgage applications
    to some of the highest levels since 2008, so be prepared to get in
    line. But defaults are also up, so if you're thinking about buying a
    home, be aware that some lenders have tightened their standards.
  \end{itemize}
\item ~
  \hypertarget{what-is-school-going-to-look-like-in-september}{%
  \paragraph{What is school going to look like in
  September?}\label{what-is-school-going-to-look-like-in-september}}

  \begin{itemize}
  \tightlist
  \item
    It is unlikely that many schools will return to a normal schedule
    this fall, requiring the grind of
    \href{https://www.nytimes3xbfgragh.onion/2020/06/05/us/coronavirus-education-lost-learning.html?action=click\&pgtype=Article\&state=default\&region=MAIN_CONTENT_3\&context=storylines_faq}{online
    learning},
    \href{https://www.nytimes3xbfgragh.onion/2020/05/29/us/coronavirus-child-care-centers.html?action=click\&pgtype=Article\&state=default\&region=MAIN_CONTENT_3\&context=storylines_faq}{makeshift
    child care} and
    \href{https://www.nytimes3xbfgragh.onion/2020/06/03/business/economy/coronavirus-working-women.html?action=click\&pgtype=Article\&state=default\&region=MAIN_CONTENT_3\&context=storylines_faq}{stunted
    workdays} to continue. California's two largest public school
    districts --- Los Angeles and San Diego --- said on July 13, that
    \href{https://www.nytimes3xbfgragh.onion/2020/07/13/us/lausd-san-diego-school-reopening.html?action=click\&pgtype=Article\&state=default\&region=MAIN_CONTENT_3\&context=storylines_faq}{instruction
    will be remote-only in the fall}, citing concerns that surging
    coronavirus infections in their areas pose too dire a risk for
    students and teachers. Together, the two districts enroll some
    825,000 students. They are the largest in the country so far to
    abandon plans for even a partial physical return to classrooms when
    they reopen in August. For other districts, the solution won't be an
    all-or-nothing approach.
    \href{https://bioethics.jhu.edu/research-and-outreach/projects/eschool-initiative/school-policy-tracker/}{Many
    systems}, including the nation's largest, New York City, are
    devising
    \href{https://www.nytimes3xbfgragh.onion/2020/06/26/us/coronavirus-schools-reopen-fall.html?action=click\&pgtype=Article\&state=default\&region=MAIN_CONTENT_3\&context=storylines_faq}{hybrid
    plans} that involve spending some days in classrooms and other days
    online. There's no national policy on this yet, so check with your
    municipal school system regularly to see what is happening in your
    community.
  \end{itemize}
\end{itemize}

Administrators and teachers know they will need to catch students up in
the fall, perhaps through reviewing skills and content that would have
normally been covered this school year. But they face major hurdles and
competing priorities. Preparing school buildings to meet new state and
federal health guidelines --- including smaller class sizes, temperature
checks and increased access to sinks, soap, personal protective
equipment and disinfectants --- requires money and careful planning.

It is just as important to improve the quality of remote learning, given
the likelihood that schools in many parts of the country will face
\href{https://www.nytimes3xbfgragh.onion/2020/04/28/us/coronavirus-schools-reopen.html}{continued
intermittent closures} to contain the virus, and that some parents will
simply choose not to send their children to classrooms before a vaccine
is available.

Students are also expected to need a greatly increased level of social
and emotional support from counselors and therapists, in part because of
the impact of spending months in social isolation, often while families
experienced job loss, economic hardship and health distress.

All of this will need to happen as schools face significant budget cuts
that will not be offset by the federal infusion that has been promised
so far, according to Marguerite Roza, a school finance expert at
Georgetown University.

Schools could freeze hiring, especially for support roles like reading
specialists and counselors, and might cancel programs like pre-K and
after-school enrichment, she said.

For protesters
\href{https://www.nytimes3xbfgragh.onion/2020/06/03/us/live-george-floyd-protests-today.html}{flooding
the nation's streets} in response to the death of George Floyd, a black
man killed while Minneapolis police officers arrested him, the idea that
school budgets could face greater cuts than
\href{https://www.inquirer.com/news/philadelphia/philadelphia-police-funding-increase-george-floyd-protests-20200602.html}{police}
\href{https://gothamist.com/news/nypd-defends-its-massive-budget-social-services-and-youth-programs-are-cut}{budgets}
as cities deal with the economic impact of the pandemic has emerged as a
major concern, and yet more evidence of racial inequality.

Already, New York City, the nation's largest school district,
\href{https://www.amny.com/education-2/carranza-warns-of-massive-nyc-public-school-cuts-in-face-of-horrific-budget/}{has
said} it would slow down the expansion of its universal pre-K program to
3-year-olds. California's urban schools
\href{https://www.documentcloud.org/documents/6894024-2020-21StateBud-Urban-Let051820.html}{have
warned} that budget cuts proposed by Gov. Gavin Newsom could make it
impossible for them to reopen safely while simultaneously helping
students catch up academically.

In Broward County, Fla., north of Miami, Iman Cassells Alleyne, an
elementary school special education teacher, spent much of the spring
semester filming herself giving remote lessons on multiplication and
phonics, even as she home-schooled her own three sons. She wanted to
provide one-on-one tutoring and reached out to students numerous times,
but many were not able to regularly get online for remote learning
because of issues at home.

Her students have learning disabilities and behavioral disorders that
make school challenging under normal circumstances. Now, she is
concerned they will fall even further behind.

``If we continue doing things the way we do them,'' she said, ``we won't
be able to fill those gaps.''

Advertisement

\protect\hyperlink{after-bottom}{Continue reading the main story}

\hypertarget{site-index}{%
\subsection{Site Index}\label{site-index}}

\hypertarget{site-information-navigation}{%
\subsection{Site Information
Navigation}\label{site-information-navigation}}

\begin{itemize}
\tightlist
\item
  \href{https://help.nytimes3xbfgragh.onion/hc/en-us/articles/115014792127-Copyright-notice}{©~2020~The
  New York Times Company}
\end{itemize}

\begin{itemize}
\tightlist
\item
  \href{https://www.nytco.com/}{NYTCo}
\item
  \href{https://help.nytimes3xbfgragh.onion/hc/en-us/articles/115015385887-Contact-Us}{Contact
  Us}
\item
  \href{https://www.nytco.com/careers/}{Work with us}
\item
  \href{https://nytmediakit.com/}{Advertise}
\item
  \href{http://www.tbrandstudio.com/}{T Brand Studio}
\item
  \href{https://www.nytimes3xbfgragh.onion/privacy/cookie-policy\#how-do-i-manage-trackers}{Your
  Ad Choices}
\item
  \href{https://www.nytimes3xbfgragh.onion/privacy}{Privacy}
\item
  \href{https://help.nytimes3xbfgragh.onion/hc/en-us/articles/115014893428-Terms-of-service}{Terms
  of Service}
\item
  \href{https://help.nytimes3xbfgragh.onion/hc/en-us/articles/115014893968-Terms-of-sale}{Terms
  of Sale}
\item
  \href{https://spiderbites.nytimes3xbfgragh.onion}{Site Map}
\item
  \href{https://help.nytimes3xbfgragh.onion/hc/en-us}{Help}
\item
  \href{https://www.nytimes3xbfgragh.onion/subscription?campaignId=37WXW}{Subscriptions}
\end{itemize}
