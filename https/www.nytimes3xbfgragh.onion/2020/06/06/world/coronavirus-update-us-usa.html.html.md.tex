Sections

SEARCH

\protect\hyperlink{site-content}{Skip to
content}\protect\hyperlink{site-index}{Skip to site index}

\href{https://www.nytimes3xbfgragh.onion/section/world}{World}

\href{https://myaccount.nytimes3xbfgragh.onion/auth/login?response_type=cookie\&client_id=vi}{}

\href{https://www.nytimes3xbfgragh.onion/section/todayspaper}{Today's
Paper}

\href{/section/world}{World}\textbar{}China Defends Its Coronavirus
Response

\url{https://nyti.ms/2AO5g5L}

\begin{itemize}
\item
\item
\item
\item
\item
\end{itemize}

\hypertarget{the-coronavirus-outbreak}{%
\subsubsection{\texorpdfstring{\href{https://www.nytimes3xbfgragh.onion/news-event/coronavirus?name=styln-coronavirus-national\&region=TOP_BANNER\&variant=undefined\&block=storyline_menu_recirc\&action=click\&pgtype=Article\&impression_id=a18de280-e384-11ea-a237-f96f036501c6}{The
Coronavirus
Outbreak}}{The Coronavirus Outbreak}}\label{the-coronavirus-outbreak}}

\begin{itemize}
\tightlist
\item
  live\href{https://www.nytimes3xbfgragh.onion/2020/08/20/world/coronavirus-covid.html?name=styln-coronavirus-national\&region=TOP_BANNER\&variant=undefined\&block=storyline_menu_recirc\&action=click\&pgtype=Article\&impression_id=a18e0990-e384-11ea-a237-f96f036501c6}{Latest
  Updates}
\item
  \href{https://www.nytimes3xbfgragh.onion/interactive/2020/us/coronavirus-us-cases.html?name=styln-coronavirus-national\&region=TOP_BANNER\&variant=undefined\&block=storyline_menu_recirc\&action=click\&pgtype=Article\&impression_id=a18e0991-e384-11ea-a237-f96f036501c6}{Maps
  and Cases}
\item
  \href{https://www.nytimes3xbfgragh.onion/interactive/2020/science/coronavirus-vaccine-tracker.html?name=styln-coronavirus-national\&region=TOP_BANNER\&variant=undefined\&block=storyline_menu_recirc\&action=click\&pgtype=Article\&impression_id=a18e0992-e384-11ea-a237-f96f036501c6}{Vaccine
  Tracker}
\item
  \href{https://www.nytimes3xbfgragh.onion/2020/08/19/us/colleges-closing-covid.html?name=styln-coronavirus-national\&region=TOP_BANNER\&variant=undefined\&block=storyline_menu_recirc\&action=click\&pgtype=Article\&impression_id=a18e0993-e384-11ea-a237-f96f036501c6}{Colleges
  Closing}
\item
  \href{https://www.nytimes3xbfgragh.onion/live/2020/08/20/business/stock-market-today-coronavirus?name=styln-coronavirus-national\&region=TOP_BANNER\&variant=undefined\&block=storyline_menu_recirc\&action=click\&pgtype=Article\&impression_id=a18e0994-e384-11ea-a237-f96f036501c6}{Economy}
\end{itemize}

Advertisement

\protect\hyperlink{after-top}{Continue reading the main story}

Supported by

\protect\hyperlink{after-sponsor}{Continue reading the main story}

\hypertarget{china-defends-its-coronavirus-response}{%
\section{China Defends Its Coronavirus
Response}\label{china-defends-its-coronavirus-response}}

Government officials denied they had suppressed information about the
outbreak and said China had set a strong example for how to combat it.

\begin{itemize}
\item
  Published June 6, 2020Updated June 10, 2020
\item
  \begin{itemize}
  \item
  \item
  \item
  \item
  \item
  \end{itemize}
\end{itemize}

\hypertarget{heres-what-you-need-to-know}{%
\subsubsection{Here's what you need to
know:}\label{heres-what-you-need-to-know}}

\begin{itemize}
\tightlist
\item
  \protect\hyperlink{link-285448fa}{China denies that it concealed the
  early spread of the virus.}
\item
  \protect\hyperlink{link-4015bf10}{U.S. states face different
  challenges in controlling the spread of the virus.}
\item
  \protect\hyperlink{link-30049c69}{Japan's embrace of face masks may be
  the secret to its virus-fighting success.}
\item
  \protect\hyperlink{link-8fd8820}{In Britain, Prince William volunteers
  at a crisis help line.}
\item
  \protect\hyperlink{link-3cfab740}{Brazil has removed coronavirus data
  from a Health Ministry website.}
\item
  \protect\hyperlink{link-435acae9}{Huge crowds turn out around the
  world to protest racism, and experts fear a coronavirus spike.}
\item
  \protect\hyperlink{link-117b4aa5}{Some parts of the United States face
  financial ruin but few infections.}
\end{itemize}

\includegraphics{https://static01.graylady3jvrrxbe.onion/images/2020/07/06/world/06virus-briefing-china-wuhan/merlin_168749316_eb1959bd-4325-4694-945a-782535fac37d-articleLarge.jpg?quality=75\&auto=webp\&disable=upscale}

\hypertarget{china-denies-that-it-concealed-the-early-spread-of-the-virus}{%
\subsection{China denies that it concealed the early spread of the
virus.}\label{china-denies-that-it-concealed-the-early-spread-of-the-virus}}

The Chinese government on Sunday strongly defended its handling of the
coronavirus, pushing back at criticism that officials had suppressed
early reports of the outbreak and contending instead that China had set
a strong example for how to combat it.

A top Chinese official said at a news conference in Beijing that the
government and state news media had provided early, timely and extensive
information since the first cases appeared in Hubei Province late last
year. In an apparent reference to the Trump administration's
\href{https://www.nytimes3xbfgragh.onion/2020/05/29/health/virus-who.html}{numerous
assertions} that China is to blame for the subsequent pandemic, he
complained bitterly about what he described as foreign lies and
slanders.

``Those are completely unwarranted and unreasonable,'' said the
official, Xu Lin, who oversees the State Council Information Office. On
Sunday the agency published a detailed
\href{http://english.scio.gov.cn/whitepapers/2020-06/07/content_76135269.htm}{report}
on China's epidemic response.

Ma Xiaowei, the minister in charge of the National Health Commission,
also said that China had ``not delayed in any way'' the release of
information about the disease.

The report is an attempt by the Chinese government to provide a
comprehensive narrative for the epidemic, ignoring any early missteps.
It describes local and provincial officials as having faithfully
implemented the instructions of the central government, without
mentioning that some of these officials were hurriedly replaced after
initial problems.

Critics have pointed out that while Chinese scientists moved quickly to
identify the new virus and share their findings internationally,
political leaders were slower to act, ordering police investigations of
doctors who
\href{https://www.nytimes3xbfgragh.onion/2020/02/01/world/asia/china-coronavirus.html}{tried
to sound the alarm} in late December.

As the U.S. and other countries struggle to bring their outbreaks under
control, China has largely returned to normal life, with its last
remaining high-risk area, a district in the northeastern city of Jilin,
lowering its epidemic response level on Sunday. The government reported
six new cases across the country on Sunday, including five that
originated abroad and one that was transmitted locally in the southern
island province of Hainan. Since the outbreak began, the Chinese
mainland has recorded more than 89,000 cases and more than 4,600 deaths.

\hypertarget{us-states-face-different-challenges-in-controlling-the-spread-of-the-virus}{%
\subsection{U.S. states face different challenges in controlling the
spread of the
virus.}\label{us-states-face-different-challenges-in-controlling-the-spread-of-the-virus}}

Image

A group at the Mystic Aquarium in Mystic, Conn., in May. As states move
ahead with reopening plans, they may face different futures based on
factors like how stressed their health care systems have
been.Credit...Christopher Capozziello for The New York Times

With states beginning to allow varying degrees of economic reopening,
large protests against police brutality being held in dozens of cities
and warmer weather inviting people outside, forecasters tracking the
Covid-19 pandemic in the United States are approaching a difficult
juncture.

While the portrait of the country over all has improved significantly in
recent weeks, epidemiologists have cautioned that different states are
likely to experience very different challenges now in measuring and
controlling the virus's spread.

According to
\href{https://www.nytimes3xbfgragh.onion/interactive/2020/us/coronavirus-us-cases.html}{data}
compiled by The New York Times, more than a third of states are still
seeing new infections increasing. But as many of them move ahead with
reopening plans, their outcomes may depend on factors like how stressed
their health care systems have been and how far they are along the
curve.

In some relatively large states such as North Carolina and Arizona,
increased testing suggests that infections are still climbing quickly
and may spike further as more people venture out.

In another group are states that have achieved modest declines in new
cases, but where the sheer number of people already infected remains the
main source of concern. Even as states such as Maryland or Connecticut
have seen small declines in new infections, both still have alarmingly
high counts per capita, which have taxed health care systems for weeks.

The fear for states in the second category is that with scores of people
already infected, recent declines could be quickly erased through
increased social contact in the months ahead, threatening health care
systems anew.

\href{https://www.nytimes3xbfgragh.onion/interactive/2020/world/coronavirus-maps.html}{}

\includegraphics{https://static01.graylady3jvrrxbe.onion/images/2020/03/03/world/coronavirus-map-promo/coronavirus-map-promo-articleLarge-v723.png}

\hypertarget{coronavirus-map-tracking-the-global-outbreak}{%
\subsection{Coronavirus Map: Tracking the Global
Outbreak}\label{coronavirus-map-tracking-the-global-outbreak}}

The virus has infected more than 22,659,500 people and has been detected
in nearly every country.

\hypertarget{japans-embrace-of-face-masks-may-be-the-secret-to-its-virus-fighting-success}{%
\subsection{Japan's embrace of face masks may be the secret to its
virus-fighting
success.}\label{japans-embrace-of-face-masks-may-be-the-secret-to-its-virus-fighting-success}}

Image

Shoppers in Tokyo last month. Face masks were a common sight in the city
long before the coronavirus arrived.Credit...Noriko Hayashi for The New
York Times

When the coronavirus arrived in Japan, people did what they normally do:
They put on masks.

Face coverings are nothing new there. During flu and hay fever seasons,
trains are crowded with commuters half-hidden behind white surgical
masks. Employees with colds, worried about the stigma of missing work,
throw one on and soldier into the office.

\href{https://www.nytimes3xbfgragh.onion/interactive/2020/world/coronavirus-maps.html\#countries}{Japan}
has reported more than 17,000 infections and just over 900 deaths, while
the United States, with a population roughly two and a half times as
large, has
\href{https://www.nytimes3xbfgragh.onion/interactive/2020/us/coronavirus-us-cases.html}{topped
1.9 million cases and is approaching 110,000 deaths}.

``Japan, I think a lot of people agree, kind of did everything wrong,
with poor social distancing, karaoke bars still open and public transit
packed near the zone where the worst outbreaks were happening,'' Jeremy
Howard, a researcher at the University of San Francisco who has studied
\href{https://www.preprints.org/manuscript/202004.0203/v2}{the use of
masks}, said of the country's early response. ``But the one thing that
Japan did right was masks.''

During the pandemic, scientists have found a correlation between high
levels of mask-wearing --- whether as a matter of culture or policy ---
and success in containing the virus.

``I think there is definitely evidence coming out of Covid that Japan,
as well as other countries which practice mask-wearing, tend to do much
better in flattening the curve,'' said
\href{https://medicine.yale.edu/profile/akiko_iwasaki/}{Akiko Iwasaki},
a professor of immunobiology at Yale.

\hypertarget{in-britain-prince-william-volunteers-at-a-crisis-help-line}{%
\subsection{In Britain, Prince William volunteers at a crisis help
line.}\label{in-britain-prince-william-volunteers-at-a-crisis-help-line}}

Prince William counts as one among thousands of British volunteers
assisting on a crisis help line during the coronavirus lockdown,
Kensington Palace announced in a message marking the end of Volunteers'
Week.

\hypertarget{latest-updates-the-coronavirus-outbreak}{%
\section{\texorpdfstring{\href{https://www.nytimes3xbfgragh.onion/2020/08/20/world/coronavirus-covid.html?action=click\&pgtype=Article\&state=default\&region=MAIN_CONTENT_1\&context=storylines_live_updates}{Latest
Updates: The Coronavirus
Outbreak}}{Latest Updates: The Coronavirus Outbreak}}\label{latest-updates-the-coronavirus-outbreak}}

Updated 2020-08-21T07:46:15.883Z

\begin{itemize}
\tightlist
\item
  \href{https://www.nytimes3xbfgragh.onion/2020/08/20/world/coronavirus-covid.html?action=click\&pgtype=Article\&state=default\&region=MAIN_CONTENT_1\&context=storylines_live_updates\#link-68774d88}{Shutdowns,
  warnings and scoldings follow alarming incidents on college campuses.}
\item
  \href{https://www.nytimes3xbfgragh.onion/2020/08/20/world/coronavirus-covid.html?action=click\&pgtype=Article\&state=default\&region=MAIN_CONTENT_1\&context=storylines_live_updates\#link-26b58724}{Biden
  knocks Trump's pandemic response, and outlines a national strategy.}
\item
  \href{https://www.nytimes3xbfgragh.onion/2020/08/20/world/coronavirus-covid.html?action=click\&pgtype=Article\&state=default\&region=MAIN_CONTENT_1\&context=storylines_live_updates\#link-4e542da3}{U.S.
  health agencies announce moves to confront the flu season and
  plummeting child vaccination rates.}
\end{itemize}

\href{https://www.nytimes3xbfgragh.onion/2020/08/20/world/coronavirus-covid.html?action=click\&pgtype=Article\&state=default\&region=MAIN_CONTENT_1\&context=storylines_live_updates}{See
more updates}

More live coverage:
\href{https://www.nytimes3xbfgragh.onion/live/2020/08/20/business/stock-market-today-coronavirus?action=click\&pgtype=Article\&state=default\&region=MAIN_CONTENT_1\&context=storylines_live_updates}{Markets}

``I'm going to share a little secret with you guys, but I'm actually on
the platform volunteering,'' Prince William, the Duke of Cambridge, said
during a video call in which he and his wife, Catherine, the Duchess of
Cambridge, thanked volunteers for their work.

In a statement,
\href{https://www.royal.uk/duke-and-duchess-cambridge-thank-volunteers}{Kensington
Palace said} Prince William had been volunteering for Shout85258, the
country's first 24/7 crisis text line, which the couple launched in 2019
with Prince Harry and Meghan, the Duke and Duchess of Sussex.

``The Duke is one of more than 2000 Crisis Volunteers who are trained to
support anyone, anytime, whatever their crisis may be,'' the palace
said. ``Last month, the Duke and Duchess marked the service's first
anniversary by speaking to five Shout volunteers via video call.''

More than 300,000 text conversations have taken place between volunteers
and people needing mental health support,
\href{https://apnews.com/526a713bd3d291c9023ed2f61cd64e41}{The
Associated Press reported}. More than half of the people texting are
under 25.

\href{https://www.youtube.com/watch?v=tGdL9zfdE1Q}{In the full
eight-minute video that the palace shared on social media}, Catherine
noted that although the coronavirus pandemic had been ``such pressure
for everybody,'' communities pulled together and people stepped up to
volunteer.

There have been at least 284,800 confirmed cases of the coronavirus in
Britain, according to the British
\href{https://coronavirus.data.gov.uk/}{public health authorities}. At
least 40,000 people have died from the virus.

\hypertarget{brazil-has-removed-coronavirus-data-from-a-health-ministry-website}{%
\subsection{Brazil has removed coronavirus data from a Health Ministry
website.}\label{brazil-has-removed-coronavirus-data-from-a-health-ministry-website}}

Image

Grave diggers burying a coronavirus victim at the Vila Formosa cemetery
in São Paulo in May.Credit...Victor Moriyama for The New York Times

\href{https://www.nytimes3xbfgragh.onion/2020/06/10/world/americas/bolsonaro-coup-coronavirus-brazil.html}{Brazil's}
government on Friday removed comprehensive numbers on
\href{https://www.nytimes3xbfgragh.onion/2020/06/10/world/americas/bolsonaro-coup-coronavirus-brazil.html}{coronavirus
cases} and deaths from the Health Ministry's website, claiming without
offering evidence that state officials had been reporting inflated
figures to secure more federal funding.

Carlos Wizard, a businessman recently appointed by President Jair
Bolsonaro to a top job in the ministry, told the newspaper O Globo on
Friday that the government suspects state officials have been including
deaths from other causes in the coronavirus tallies they report to the
federal government.

``Local officials, driven purely by a desire to get more funding for
their cities, labeled everyone as Covid,'' Wizard said. ``We're
reviewing those deaths.''

The accusation outraged public health experts. Several noted that Brazil
has a sophisticated health surveillance system and that there is a broad
consensus among epidemiologists that a lack of testing worldwide has
resulted in a gross undercount of deaths from Covid-19, the disease
caused by the coronavirus. And an
\href{https://www.nytimes3xbfgragh.onion/interactive/2020/04/21/world/coronavirus-missing-deaths.html}{analysis
by The New York Times} found that coronavirus deaths in five Brazilian
cities appeared to be vastly underreported.

Mr. Bolsonaro has come under withering criticism at home and abroad for
his cavalier handling of the pandemic. He has sabotaged quarantine
guidelines issued at the state level, calling them ruinous for economic
growth. On Friday,
\href{https://www.haaretz.com/world-news/americas/bolsonaro-threatens-who-exit-as-coronavirus-kills-a-brazilian-per-minute-1.8900578}{he
threatened to pull Brazil out of the World Health Organization}, which
has urged countries with increasing outbreaks to adopt social distancing
guidelines.

As of Saturday, Brazil had more than 669,000 confirmed cases, second
only to the United States, and more than 35,000 deaths. In recent days,
Brazil has led the world in the number of new deaths reported each day.

The National Council of Health Secretaries, which represents municipal
health officials, called Mr. Wizard's accusation outrageous.

``This authoritarian, insensitive, inhumane and unethical attempt to
erase people who have died from Covid-19 will fail,'' the council said.
``We are not mercenaries of death.''

As the country's caseload exploded in recent weeks, Mr. Bolsonaro fired
his health minister and replaced him with a doctor who lasted less than
a month on the job.

Since mid-May, the health ministry has been led by an active duty
general with no medical experience, and military officers have stepped
into several top jobs as career health officials resigned.

Global Roundup

\hypertarget{huge-crowds-turn-out-around-the-world-to-protest-racism-and-experts-fear-a-coronavirus-spike}{%
\subsection{Huge crowds turn out around the world to protest racism, and
experts fear a coronavirus
spike.}\label{huge-crowds-turn-out-around-the-world-to-protest-racism-and-experts-fear-a-coronavirus-spike}}

Image

Protesters at a Black Lives Matter rally in Brisbane, Australia, on
Saturday.Credit...Glenn Hunt/EPA, via Shutterstock

From London to Sydney, crowds of people around the world defied public
health warnings and turned out in solidarity with U.S.
\href{https://www.nytimes3xbfgragh.onion/2020/06/06/us/george-floyd-protests.html}{protesters
calling for justice} in the death of an African-American man, George
Floyd, killed in policy custody in Minneapolis. Health experts have
warned that the
\href{https://www.nytimes3xbfgragh.onion/2020/06/06/us/george-floyd-protests.html?action=click\&module=Top\%20Stories\&pgtype=Homepage\#link-2301563b}{demonstrations
could accelerate the spread of the coronavirus}, especially among people
not wearing masks.

In Australia, huge crowds turned out in Sydney, Melbourne and many other
communities in support of the Black Lives Matter movement calling for an
end to systemic racism and Aboriginal deaths in police custody.

The health minister in Britain urged residents not to gather for
demonstrations in London, Manchester and Birmingham. But large crowds
appeared --- despite the cold weather, the rain and warnings by the
police that mass gatherings would violate the rule that only six people
from different households could gather outside during the pandemic.

Police in
\href{https://www.psni.police.uk/news/Latest-News/060620-planned-protests-in-belfast-and-derrylondonderry-have-ended/}{Northern
Ireland announced on Saturday} that organizers of ``Black Lives Matter''
protests in Belfast and Derry will be reported to the Public Prosecution
Service for breaking coronavirus restrictions.

``We estimate there were less than 500 people in attendance at each
event and a significant number of Community Resolution Notices (CRNS)
and fines were issued,'' said Assistant Chief Constable Alan Todd, who
added that ``Health Protection Regulations are in place to protect us
all during this pandemic and it is everyone's responsibility to adhere
to them to protect our society.''

In Paris,
\href{https://twitter.com/CVersailles/status/1269154953844703232}{on a
day the Palace of Versailles reopened} to visitors and tourists, the
authorities barred people from gathering in front of the United States
Embassy, but thousands protested there anyway in the late afternoon, as
well as near the Eiffel Tower, echoing a protest earlier this week that
drew nearly 20,000 people in memory of Adama Traoré, a Frenchman who
died in police custody in 2016. On Friday,
\href{https://twitter.com/prefpolice/status/1269002972807278595}{police
officially banned protests on the Champ-de-Mars for June 6}, citing
coronavirus concerns.

And in the German cities of Berlin and Cologne, thousands responded to
social media calls to take to the streets to honor Mr. Floyd. The
protests came after a week of demonstrations in cities like Hamburg and
Frankfurt.

Fury against racism and police brutality has also brought crowds into
the streets of Belgium, Canada, Sweden and Zimbabwe. In other parts of
the world:

\begin{itemize}
\tightlist
\item
  \textbf{Art Basel, the centerpiece of the European art market
  calendar,}
  \textbf{\href{https://www.nytimes3xbfgragh.onion/2020/06/06/arts/design/art-basel-cancels-fair-virus.html}{is
  canceled}.} The 50th anniversary edition of the event in Basel,
  Switzerland, was to feature more than 250 international galleries and
  had already been postponed.
\end{itemize}

\begin{itemize}
\item
  \textbf{Saudi Arabia reimposed a curfew} in the Red Sea city of Jeddah
  from 3 p.m. to 6 a.m. for two weeks starting on Saturday, halted
  prayers in the city's mosques and suspended work in offices because of
  a rise in the spread of the coronavirus, the state news agency SPA
  reported.
\item
  \textbf{Russia on Saturday reported 8,855 new cases} of the
  coronavirus, pushing the total number of infections to 458,689, and
  197 deaths in the past 24 hours. The nationwide death toll has reached
  5,725.
\end{itemize}

Economic roundup

\hypertarget{some-parts-of-the-united-states-face-financial-ruin-but-few-infections}{%
\subsection{Some parts of the United States face financial ruin but few
infections.}\label{some-parts-of-the-united-states-face-financial-ruin-but-few-infections}}

Image

The first night the restaurant Bin707 was opened in Grand Junction,
Colo., on Monday.Credit...Benjamin Rasmussen for The New York Times

In Corpus Christi, the oil and gas and vacation town on the southeastern
coast of Texas, it can be tough to find people who have experienced the
coronavirus's devastation, or even know someone who has. But people hit
with job losses or business closures? They are everywhere.

Theresa Thompson has been furloughed from her position as a catering and
events manager at a Holiday Inn. Richard Lomax has seen sales fall by
more than 90 percent at the two restaurants his family owns. Brett
Oetting, chief executive of the tourism office, has been working with
countless businesses struggling to navigate the economic collapse.

None of them knows anyone local who has been sickened by the virus.

In corners of the United States facing financial ruin, but where the
coronavirus hasn't arrived in full,
\href{https://www.nytimes3xbfgragh.onion/interactive/2020/06/06/business/economy/high-unemployment-few-coronavirus-cases.html}{a
New York Times analysis} of economic and infection data helps explain
why some see reopening as long overdue. The sharp disconnect between
extreme economic pain and limited health impact presents local officials
and businesses with difficult choices, even after
\href{https://www.nytimes3xbfgragh.onion/2020/06/05/business/economy/jobs-report.html}{Friday's
encouraging jobs report} suggested more of the country was returning to
work.

``In the first two weeks when they said this was coming, I was like,
`Let's all stay in, hunker down, and if we all do this, that can help
while we figure out what is going on,''' said Stephanie Anderson, a real
estate agent in Satellite Beach, Fla.

But since ``places here aren't producing mass death,'' she said, ``don't
tell me I can't open my business in a responsible manner.''

Some business owners and workers in these communities have embraced
reopening because of their firsthand experiences. Many are angry or
confused. Others plead for caution. But most agree the virus has not
posed the local public health threat that so many were expecting ---
even while acknowledging that
\href{https://www.nytimes3xbfgragh.onion/2020/05/20/us/coronavirus-reopening-50-states.html}{things
could get worse} and the numbers would most likely already be higher
with more testing.

Here are some other recent developments on the economic impact of the
pandemic:

\begin{itemize}
\item
  \textbf{African-Americans and Latinos}
  \textbf{\href{https://www.nytimes3xbfgragh.onion/2020/06/06/business/economy/jobs-report-minorities.html}{are
  especially vulnerable}} to job losses in the pandemic and at a
  disadvantage in getting government support.
\item
  Economists warn that encouraging job numbers may mask \textbf{a huge,
  slower-moving slump in demand}, and
  \href{https://www.nytimes3xbfgragh.onion/2020/06/06/upshot/coronavirus-economic-crisis.html}{imperil
  stimulus measures required to combat it}.
\end{itemize}

\hypertarget{in-new-york-a-big-sigh-of-relief-as-the-daily-death-toll-falls-to-35}{%
\subsection{In New York, a `big sigh of relief' as the daily death toll
falls to
35.}\label{in-new-york-a-big-sigh-of-relief-as-the-daily-death-toll-falls-to-35}}

Image

A sign hanging in Brooklyn. The city will start to reopen Monday,
allowing retail stores to offer curbside or in-store
pickup.Credit...Hilary Swift for The New York Times

The weekend ahead of New York City's start of gradual reopening, Gov.
Andrew M. Cuomo reported 35 new coronavirus deaths statewide, a drop of
seven from the day before and the lowest daily total in the last two
months.

``This is really, really good news compared to where we were,'' Mr.
Cuomo said Saturday during his daily briefing in Albany. ``This is a big
sigh of relief.''

Under Phase 1 of reopening, set to begin Monday, retail stores will be
allowed to open for curbside or in-store pickup, and nonessential
construction and manufacturing can resume, returning as many as 400,000
people to the work force.

``You want to talk about a turnaround --- this one, my friends, is going
to go in the history books,'' Mr. Cuomo said. ``There is no state in the
United States that has gone from where we were to where we are.''

Mr. Cuomo also announced he was expanding the occupancy guidelines for
houses of worship, which could now admit up to 25 percent of the
building's occupancy. It is unclear if the measure applies statewide or
only in locations that have reached Phase 2. All regions of the state
except New York City \href{https://forward.ny.gov/}{are in the first or
second phase of reopening}.

Across the Hudson River, Gov. Philip D. Murphy of New Jersey announced
60 new virus-related deaths Saturday via
\href{https://twitter.com/GovMurphy/status/1269316516962545666?s=20}{social
media}, bringing the state's toll to 12,106. The figure was a drop from
the 79 new deaths reported the previous day. He also reported 606 new
confirmed positive cases, totaling 163,893 cases in the state.

While New York City's shutdown has successfully flattened the number of
infections,
\href{https://www.nytimes3xbfgragh.onion/2020/06/06/business/economy/coronavirus-closings-strategy.html}{a
study has found} that the economic cost could have been reduced by a
third or more by strategically choosing neighborhoods to close,
calibrating the risk of infection for local residents and workers with
the impact on local jobs.

\hypertarget{the-economy-wont-be-easily-healed-denying-that-reality-wont-help-fix-it}{%
\subsection{The economy won't be easily healed. Denying that reality
won't help fix
it.}\label{the-economy-wont-be-easily-healed-denying-that-reality-wont-help-fix-it}}

Image

Credit...Adam Maida

For the first time in three months there is a scent of economic optimism
in the air. Employers added millions of jobs to their payrolls in May,
and the jobless rate fell, a big surprise to forecasters who expected
further losses. Businesses are reopening, and the rate of coronavirus
deaths has edged down. The Trump administration has begun pointing to
what are likely to be impressive growth numbers as the economy starts to
pull out of its deep hole.

All of that is good news. But there are clear signs that the collapse of
economic activity has set in motion problems that will play out over
many months, or maybe many years. If not contained, they could cause
human misery on a mass scale and create lasting scars for families.

The fabric of the economy has been ripped, with damage done to millions
of interconnections --- between workers and employers, companies and
their suppliers, borrowers and lenders. Both the historical evidence
from severe economic crises and the data available today point to
enormous delayed effects.

While the government can't wave a wand and bring back industries that
are semi-permanently shuttered, it can act --- and has acted --- to try
to keep demand for goods and services at pre-crisis levels. That, in
turn, can smooth the path for other sectors to grow so that there is not
a prolonged depression of jobs, income and investment, with a resulting
reduction in the economy's long-term potential.

\hypertarget{study-finds-some-rays-of-optimism-for-women-considering-pregnancy}{%
\subsection{Study finds some rays of optimism for women considering
pregnancy.}\label{study-finds-some-rays-of-optimism-for-women-considering-pregnancy}}

Image

Ebru Celik holds her newborn baby boy after recovering from Covid-19 in
Instanbul, Turkey, in May.Credit...Chris Mcgrath/Getty Images

How the coronavirus might affect pregnant women and newborns has been a
major concern since the outbreaks began. A
\href{https://slack-redir.net/link?url=https\%3A\%2F\%2Fjamanetwork.com\%2Fjournals\%2Fjama\%2Ffullarticle\%2F2767060\%3FguestAccessKey\%3Dcdf30829-77bc-4a44-88ec-5eaee89b19ec\%26utm_source\%3DFor_The_Media\%26utm_medium\%3Dreferral\%26utm_campaign\%3Dftm_links\%26utm_content\%3Dtfl\%26utm_term\%3D060520}{new
report in the medical journal JAMA} has both reassuring and worrisome
findings, with caveats that there is limited data and still much
unknown.

So far, compared to the general population, pregnant women do not seem
to have an increased risk of severe illness if they contract the virus,
the report said. Of 147 pregnant women with Covid-19 in China, 8 percent
had severe disease and 1 percent had critical illness --- rates that
were actually lower than those in the rest of the population, where 14
percent had severe disease and 6 percent were critically ill. In New
York City, a report on 43 pregnant women with Covid-19 found that their
rates of severe disease were similar to those in other adults.

\href{https://www.nytimes3xbfgragh.onion/news-event/coronavirus?action=click\&pgtype=Article\&state=default\&region=MAIN_CONTENT_3\&context=storylines_faq}{}

\hypertarget{the-coronavirus-outbreak-}{%
\subsubsection{The Coronavirus Outbreak
›}\label{the-coronavirus-outbreak-}}

\hypertarget{frequently-asked-questions}{%
\paragraph{Frequently Asked
Questions}\label{frequently-asked-questions}}

Updated August 17, 2020

\begin{itemize}
\item ~
  \hypertarget{why-does-standing-six-feet-away-from-others-help}{%
  \paragraph{Why does standing six feet away from others
  help?}\label{why-does-standing-six-feet-away-from-others-help}}

  \begin{itemize}
  \tightlist
  \item
    The coronavirus spreads primarily through droplets from your mouth
    and nose, especially when you cough or sneeze. The C.D.C., one of
    the organizations using that measure,
    \href{https://www.nytimes3xbfgragh.onion/2020/04/14/health/coronavirus-six-feet.html?action=click\&pgtype=Article\&state=default\&region=MAIN_CONTENT_3\&context=storylines_faq}{bases
    its recommendation of six feet} on the idea that most large droplets
    that people expel when they cough or sneeze will fall to the ground
    within six feet. But six feet has never been a magic number that
    guarantees complete protection. Sneezes, for instance, can launch
    droplets a lot farther than six feet,
    \href{https://jamanetwork.com/journals/jama/fullarticle/2763852}{according
    to a recent study}. It's a rule of thumb: You should be safest
    standing six feet apart outside, especially when it's windy. But
    keep a mask on at all times, even when you think you're far enough
    apart.
  \end{itemize}
\item ~
  \hypertarget{i-have-antibodies-am-i-now-immune}{%
  \paragraph{I have antibodies. Am I now
  immune?}\label{i-have-antibodies-am-i-now-immune}}

  \begin{itemize}
  \tightlist
  \item
    As of right
    now,\href{https://www.nytimes3xbfgragh.onion/2020/07/22/health/covid-antibodies-herd-immunity.html?action=click\&pgtype=Article\&state=default\&region=MAIN_CONTENT_3\&context=storylines_faq}{that
    seems likely, for at least several months.} There have been
    frightening accounts of people suffering what seems to be a second
    bout of Covid-19. But experts say these patients may have a
    drawn-out course of infection, with the virus taking a slow toll
    weeks to months after initial exposure. People infected with the
    coronavirus typically
    \href{https://www.nature.com/articles/s41586-020-2456-9}{produce}
    immune molecules called antibodies, which are
    \href{https://www.nytimes3xbfgragh.onion/2020/05/07/health/coronavirus-antibody-prevalence.html?action=click\&pgtype=Article\&state=default\&region=MAIN_CONTENT_3\&context=storylines_faq}{protective
    proteins made in response to an
    infection}\href{https://www.nytimes3xbfgragh.onion/2020/05/07/health/coronavirus-antibody-prevalence.html?action=click\&pgtype=Article\&state=default\&region=MAIN_CONTENT_3\&context=storylines_faq}{.
    These antibodies may} last in the body
    \href{https://www.nature.com/articles/s41591-020-0965-6}{only two to
    three months}, which may seem worrisome, but that's perfectly normal
    after an acute infection subsides, said Dr. Michael Mina, an
    immunologist at Harvard University. It may be possible to get the
    coronavirus again, but it's highly unlikely that it would be
    possible in a short window of time from initial infection or make
    people sicker the second time.
  \end{itemize}
\item ~
  \hypertarget{im-a-small-business-owner-can-i-get-relief}{%
  \paragraph{I'm a small-business owner. Can I get
  relief?}\label{im-a-small-business-owner-can-i-get-relief}}

  \begin{itemize}
  \tightlist
  \item
    The
    \href{https://www.nytimes3xbfgragh.onion/article/small-business-loans-stimulus-grants-freelancers-coronavirus.html?action=click\&pgtype=Article\&state=default\&region=MAIN_CONTENT_3\&context=storylines_faq}{stimulus
    bills enacted in March} offer help for the millions of American
    small businesses. Those eligible for aid are businesses and
    nonprofit organizations with fewer than 500 workers, including sole
    proprietorships, independent contractors and freelancers. Some
    larger companies in some industries are also eligible. The help
    being offered, which is being managed by the Small Business
    Administration, includes the Paycheck Protection Program and the
    Economic Injury Disaster Loan program. But lots of folks have
    \href{https://www.nytimes3xbfgragh.onion/interactive/2020/05/07/business/small-business-loans-coronavirus.html?action=click\&pgtype=Article\&state=default\&region=MAIN_CONTENT_3\&context=storylines_faq}{not
    yet seen payouts.} Even those who have received help are confused:
    The rules are draconian, and some are stuck sitting on
    \href{https://www.nytimes3xbfgragh.onion/2020/05/02/business/economy/loans-coronavirus-small-business.html?action=click\&pgtype=Article\&state=default\&region=MAIN_CONTENT_3\&context=storylines_faq}{money
    they don't know how to use.} Many small-business owners are getting
    less than they expected or
    \href{https://www.nytimes3xbfgragh.onion/2020/06/10/business/Small-business-loans-ppp.html?action=click\&pgtype=Article\&state=default\&region=MAIN_CONTENT_3\&context=storylines_faq}{not
    hearing anything at all.}
  \end{itemize}
\item ~
  \hypertarget{what-are-my-rights-if-i-am-worried-about-going-back-to-work}{%
  \paragraph{What are my rights if I am worried about going back to
  work?}\label{what-are-my-rights-if-i-am-worried-about-going-back-to-work}}

  \begin{itemize}
  \tightlist
  \item
    Employers have to provide
    \href{https://www.osha.gov/SLTC/covid-19/standards.html}{a safe
    workplace} with policies that protect everyone equally.
    \href{https://www.nytimes3xbfgragh.onion/article/coronavirus-money-unemployment.html?action=click\&pgtype=Article\&state=default\&region=MAIN_CONTENT_3\&context=storylines_faq}{And
    if one of your co-workers tests positive for the coronavirus, the
    C.D.C.} has said that
    \href{https://www.cdc.gov/coronavirus/2019-ncov/community/guidance-business-response.html}{employers
    should tell their employees} -\/- without giving you the sick
    employee's name -\/- that they may have been exposed to the virus.
  \end{itemize}
\item ~
  \hypertarget{what-is-school-going-to-look-like-in-september}{%
  \paragraph{What is school going to look like in
  September?}\label{what-is-school-going-to-look-like-in-september}}

  \begin{itemize}
  \tightlist
  \item
    It is unlikely that many schools will return to a normal schedule
    this fall, requiring the grind of
    \href{https://www.nytimes3xbfgragh.onion/2020/06/05/us/coronavirus-education-lost-learning.html?action=click\&pgtype=Article\&state=default\&region=MAIN_CONTENT_3\&context=storylines_faq}{online
    learning},
    \href{https://www.nytimes3xbfgragh.onion/2020/05/29/us/coronavirus-child-care-centers.html?action=click\&pgtype=Article\&state=default\&region=MAIN_CONTENT_3\&context=storylines_faq}{makeshift
    child care} and
    \href{https://www.nytimes3xbfgragh.onion/2020/06/03/business/economy/coronavirus-working-women.html?action=click\&pgtype=Article\&state=default\&region=MAIN_CONTENT_3\&context=storylines_faq}{stunted
    workdays} to continue. California's two largest public school
    districts --- Los Angeles and San Diego --- said on July 13, that
    \href{https://www.nytimes3xbfgragh.onion/2020/07/13/us/lausd-san-diego-school-reopening.html?action=click\&pgtype=Article\&state=default\&region=MAIN_CONTENT_3\&context=storylines_faq}{instruction
    will be remote-only in the fall}, citing concerns that surging
    coronavirus infections in their areas pose too dire a risk for
    students and teachers. Together, the two districts enroll some
    825,000 students. They are the largest in the country so far to
    abandon plans for even a partial physical return to classrooms when
    they reopen in August. For other districts, the solution won't be an
    all-or-nothing approach.
    \href{https://bioethics.jhu.edu/research-and-outreach/projects/eschool-initiative/school-policy-tracker/}{Many
    systems}, including the nation's largest, New York City, are
    devising
    \href{https://www.nytimes3xbfgragh.onion/2020/06/26/us/coronavirus-schools-reopen-fall.html?action=click\&pgtype=Article\&state=default\&region=MAIN_CONTENT_3\&context=storylines_faq}{hybrid
    plans} that involve spending some days in classrooms and other days
    online. There's no national policy on this yet, so check with your
    municipal school system regularly to see what is happening in your
    community.
  \end{itemize}
\end{itemize}

But whether the infection can cause birth defects, miscarriage,
premature birth or stillbirth is not yet known. Newborns have become
infected, but it's not clear whether they contracted the virus before,
during or after birth, or if breastfeeding can transmit the virus.

Even so, the report says that for women who are wondering whether this
is a safe time to conceive, ``based on limited data, there does not seem
to be a compelling reason to recommend delaying pregnancy.''

\hypertarget{europe-has-lost-some-of-the-last-witnesses-to-its-grim-history}{%
\subsection{Europe has lost some of the last witnesses to its grim
history.}\label{europe-has-lost-some-of-the-last-witnesses-to-its-grim-history}}

Image

On Italy's Liberation Day in April, residents of Milan placed flowers in
memory of the partisans who died during World War II. The ranks of those
who remember the war have been diminished by the
coronavirus.Credit...Alessandro Grassani for The New York Times

For years, Gildo Negri visited schools to share his stories about
blowing up bridges and cutting electrical wires to sabotage Nazis and
fascists during World War II. In January, the 89-year-old made another
visit, leaving his nursing home outside Milan to help students plant
trees in honor of Italians deported to concentration camps.

But at the end of February, as Europe's first outbreak of the
coronavirus spread through Mr. Negri's nursing home, it fatally infected
him, too.

The virus, which is so lethal to the old, has hastened the departure of
these last witnesses and forced the cancellation of commemorations. It
has also created an opportunity for rising political forces who seek to
recast the history of the last century in order to play a greater role
in remaking the present one.

Throughout Europe, radical right-wing parties with histories of
Holocaust denial, Mussolini infatuation and fascist motifs have gained
traction in recent years.

\hypertarget{more-americans-use-home-aides-than-nursing-homes-but-the-sector-has-been-forgotten-in-the-pandemic}{%
\subsection{More Americans use home aides than nursing homes, but the
sector has been `forgotten' in the
pandemic.}\label{more-americans-use-home-aides-than-nursing-homes-but-the-sector-has-been-forgotten-in-the-pandemic}}

Image

Jennifer Washington, a home health care aide who juggles multiple
clients in Oakland, Calif.Credit...Jim Wilson/The New York Times

Much of the attention to the toll Covid-19 has taken on older adults has
rightly focused on long-term care facilities. Their residents and
employees account for almost 40 percent of the nation's deaths,
according to an updated
\href{https://www.nytimes3xbfgragh.onion/interactive/2020/05/09/us/coronavirus-cases-nursing-homes-us.html}{New
York Times analysis}.

But far more Americans --- nearly six million,
\href{https://www.ncbi.nlm.nih.gov/pmc/articles/PMC5052697/}{by one
estimate} --- rely on paid home care than live-in nursing homes and
assisted living combined. And both workers and clients have cause for
worry.

Even more than nursing home employees, home care workers are poorly paid
hourly workers and often lack health insurance; half rely on some form
of public assistance. Not only do many home care workers serve several
clients each week, but to piece together a living they may
simultaneously work for several agencies or for nursing homes, or hold
outside jobs.

Those conditions increase infection risks, and not only for their frail
older clients. Almost a third of home care workers, a heavily female
work force, are themselves over 55, and most are black or Hispanic,
groups that have proved particularly vulnerable to Covid-19.

Personal protective equipment, or P.P.E., has proved hard to acquire,
however. With hospitals and nursing homes scrambling for supplies,
``this was the forgotten sector,'' said Dr. Nathan Stall, a geriatrician
at the University of Toronto.

``Home care workers are probably unknowingly involved in the
transmission of Covid-19, especially when they're not equipped with
sufficient P.P.E.,'' he added.

\hypertarget{how-to-stay-safe-as-life-restarts}{%
\subsection{How to stay safe as life
restarts.}\label{how-to-stay-safe-as-life-restarts}}

Image

A line practicing social distancing at a Whole Foods in Detroit in
May.Credit...Brittany Greeson for The New York Times

When the country was under lockdown, at least the rules were mostly
clear. Essential workers ventured out; everyone else sheltered in.

Now states are lifting restrictions, but detailed guidance about
navigating the minutiae of everyday life is still hard to come by ---
and anyway, there's never going to be a ready solution to every
problematic circumstance you may encounter.

As you tiptoe toward normalization --- whatever that is, given these
times --- try to follow three precautions: avoid contact, confinement
and crowds. And make realistic choices.

\hypertarget{contact}{%
\subsubsection{Contact}\label{contact}}

You need to continue with social distancing precautions. That means
wearing masks, washing hands well and often, and keeping a six-foot
distance from one another. No hugs, no handshakes.

Any 15-minute face-to-face conversation between people who are within
six feet of one another constitutes close contact, said Dr. Muge Cevik,
an expert on infectious diseases and virology at University of Saint
Andrews School of Medicine in Scotland.

\hypertarget{confinement}{%
\subsubsection{Confinement}\label{confinement}}

Indoor activities in confined enclosed spaces, even large ones, are more
conducive to spreading the virus than events held outside, especially if
the air inside the building is being recirculated or the windows don't
open.

\hypertarget{crowds}{%
\subsubsection{Crowds}\label{crowds}}

Large groups are risky, even outdoors. They mean more people, more
contacts --- and more potential sources of infection.

\hypertarget{choices}{%
\subsubsection{Choices}\label{choices}}

People at
\href{https://www.cdc.gov/coronavirus/2019-ncov/need-extra-precautions/people-at-higher-risk.html}{high
risk for developing severe disease if they become infected with the
coronavirus} --- including those 65 and over, residents of nursing homes
and long-term care facilities, people with compromised immune systems,
chronic lung or kidney disease, heart conditions or severe obesity ---
will want to take the greatest of precautions.

But young healthy adults and children should also consider the
protection of people around them, including family members, colleagues
or friends who are vulnerable, said Dr. Barbara Taylor, an infectious
disease specialist at The University of Texas Health Science Center at
San Antonio.

\hypertarget{deforestation-in-the-amazon-is-soaring-as-the-pandemic-cripples-enforcement}{%
\subsection{Deforestation in the Amazon is soaring as the pandemic
cripples
enforcement.}\label{deforestation-in-the-amazon-is-soaring-as-the-pandemic-cripples-enforcement}}

Image

A fire burning in the Amazon rainforest near the city of Porto Velho in
Brazil last year.Credit...Victor Moriyama for The New York Times

President Jair Bolsonaro of Brazil has enabled increased razing of the
Amazon rainforest. The coronavirus has accelerated that destruction.

Illegal loggers, miners and land grabbers have cleared vast areas of the
Amazon with impunity in recent months as law enforcement efforts were
hobbled by the pandemic.

The fallout from the pandemic has exacerbated the
\href{https://www.nytimes3xbfgragh.onion/2020/04/19/world/americas/bolsonaro-brazil-amazon-indigenous.html}{ecological
degradation set in motion by government policies under Mr. Bolsonaro},
who favors expanding commercial development in the Amazon and views
environmental regulations as a hindrance to economic growth. But some
career civil servants are still working to enforce environmental
protections.

An estimated 464 square miles of Amazon tree cover was slashed from
January to April, a 55 percent increase from the same period last year
and an area roughly 20 times the size of Manhattan, according to
Brazil's National Institute for Space Research, a government agency that
tracks deforestation with satellite images.

Already last year, deforestation in the Amazon had reached levels not
seen since 2008.

At the same time, the coronavirus
has\href{https://covid.saude.gov.br/}{killed more than 34,000 peopl}e in
Brazil, which now has the highest daily number of deaths in the world.

\hypertarget{new-york-could-have-taken-a-more-targeted-approach-to-the-shutdown-researchers-say}{%
\subsection{New York could have taken a more targeted approach to the
shutdown, researchers
say.}\label{new-york-could-have-taken-a-more-targeted-approach-to-the-shutdown-researchers-say}}

Image

A view of a closed business in Brooklyn in April.Credit...Demetrius
Freeman for The New York Times

As Covid-19 cases took off in New York in March, Gov. Andrew M. Cuomo
imposed a lockdown of nonessential businesses to slow the spread of the
coronavirus, calling it
``\href{https://www.nytimes3xbfgragh.onion/2020/03/20/us/ny-ca-stay-home-order.html}{the
most drastic action we can take}.''

Now researchers say more targeted approaches --- in New York and
elsewhere --- might have protected public health with less economic
pain.

Businesses in New York City, where an initial phase of reopening is set
to begin Monday, have been mostly shut down for 11 weeks. But a study
has found that the economic cost could have been reduced by a third or
more by strategically choosing neighborhoods to close, calibrating the
risk of infection for local residents and workers with the impact on
local jobs.

\hypertarget{coronavirus-helped-britains-homeless-get-off-the-street-but-maybe-not-for-long}{%
\subsection{Coronavirus helped Britain's homeless get off the street.
But maybe not for
long.}\label{coronavirus-helped-britains-homeless-get-off-the-street-but-maybe-not-for-long}}

Image

Homeless people and others in need lining up for free meals in Trafalgar
Square in London in April.

As part of Britain's effort to contain the spread of the virus, the
government required local councils in England and Wales to provide
emergency accommodation in budget hotels to every homeless person living
on the streets.

Since the beginning of the coronavirus lockdown, more than 90 percent of
people sleeping on the street have been offered a place to stay,
\href{https://www.gov.uk/government/news/6-000-new-supported-homes-as-part-of-landmark-commitment-to-end-rough-sleeping}{according
to government statistics.}

Homeless charities say the initial success of the program has proved
what they have long maintained: that an injection of funding and support
from the government can quickly and effectively bring people off the
streets.

``It was an amazing effort, and it shows what you can do when you have
the political will and a willingness to spend the money,'' said Dominic
Williamson, the executive director of strategy and policy for the
British homeless charity
\href{https://www.mungos.org/homelessness/how-to-help-during-coronavirus/?gclid=EAIaIQobChMIxp2_7pvj6QIVgbTtCh2X-wwDEAAYASAAEgIER_D_BwE}{St.
Mungo's}.

Reporting was contributed by Keith Bradsher, Aimee Ortiz, Neil Irwin,
Andrea Salcedo, Zach Montague, Michael H. Keller, Steve Eder, Karl
Russell, Denise Grady, Ernesto Londoño, Letícia Casado, Jason Horowitz,
Damien Cave, Livia Albeck-Ripka, Iliana Magra, Ceylan Yeginsu, Elian
Peltier, Yonette Joseph, Roni Rabin, Eduardo Porter, **** Patricia
Cohen, Ernesto Londoño, Manuela Andreoni, Leticia Casado, Ben Casselman
and Paula Span.

Advertisement

\protect\hyperlink{after-bottom}{Continue reading the main story}

\hypertarget{site-index}{%
\subsection{Site Index}\label{site-index}}

\hypertarget{site-information-navigation}{%
\subsection{Site Information
Navigation}\label{site-information-navigation}}

\begin{itemize}
\tightlist
\item
  \href{https://help.nytimes3xbfgragh.onion/hc/en-us/articles/115014792127-Copyright-notice}{©~2020~The
  New York Times Company}
\end{itemize}

\begin{itemize}
\tightlist
\item
  \href{https://www.nytco.com/}{NYTCo}
\item
  \href{https://help.nytimes3xbfgragh.onion/hc/en-us/articles/115015385887-Contact-Us}{Contact
  Us}
\item
  \href{https://www.nytco.com/careers/}{Work with us}
\item
  \href{https://nytmediakit.com/}{Advertise}
\item
  \href{http://www.tbrandstudio.com/}{T Brand Studio}
\item
  \href{https://www.nytimes3xbfgragh.onion/privacy/cookie-policy\#how-do-i-manage-trackers}{Your
  Ad Choices}
\item
  \href{https://www.nytimes3xbfgragh.onion/privacy}{Privacy}
\item
  \href{https://help.nytimes3xbfgragh.onion/hc/en-us/articles/115014893428-Terms-of-service}{Terms
  of Service}
\item
  \href{https://help.nytimes3xbfgragh.onion/hc/en-us/articles/115014893968-Terms-of-sale}{Terms
  of Sale}
\item
  \href{https://spiderbites.nytimes3xbfgragh.onion}{Site Map}
\item
  \href{https://help.nytimes3xbfgragh.onion/hc/en-us}{Help}
\item
  \href{https://www.nytimes3xbfgragh.onion/subscription?campaignId=37WXW}{Subscriptions}
\end{itemize}
