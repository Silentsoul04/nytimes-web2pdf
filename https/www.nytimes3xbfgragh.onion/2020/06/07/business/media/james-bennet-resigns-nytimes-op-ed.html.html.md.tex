Sections

SEARCH

\protect\hyperlink{site-content}{Skip to
content}\protect\hyperlink{site-index}{Skip to site index}

\href{https://www.nytimes3xbfgragh.onion/section/business/media}{Media}

\href{https://myaccount.nytimes3xbfgragh.onion/auth/login?response_type=cookie\&client_id=vi}{}

\href{https://www.nytimes3xbfgragh.onion/section/todayspaper}{Today's
Paper}

\href{/section/business/media}{Media}\textbar{}James Bennet Resigns as
New York Times Opinion Editor

\url{https://nyti.ms/3h3F6N9}

\begin{itemize}
\item
\item
\item
\item
\item
\item
\end{itemize}

\href{https://www.nytimes3xbfgragh.onion/news-event/george-floyd-protests-minneapolis-new-york-los-angeles?action=click\&pgtype=Article\&state=default\&region=TOP_BANNER\&context=storylines_menu}{Race
and America}

\begin{itemize}
\tightlist
\item
  \href{https://www.nytimes3xbfgragh.onion/2020/07/26/us/protests-portland-seattle-trump.html?action=click\&pgtype=Article\&state=default\&region=TOP_BANNER\&context=storylines_menu}{Protesters
  Return to Other Cities}
\item
  \href{https://www.nytimes3xbfgragh.onion/2020/07/24/us/portland-oregon-protests-white-race.html?action=click\&pgtype=Article\&state=default\&region=TOP_BANNER\&context=storylines_menu}{Portland
  at the Center}
\item
  \href{https://www.nytimes3xbfgragh.onion/2020/07/23/podcasts/the-daily/portland-protests.html?action=click\&pgtype=Article\&state=default\&region=TOP_BANNER\&context=storylines_menu}{Podcast:
  Showdown in Portland}
\item
  \href{https://www.nytimes3xbfgragh.onion/interactive/2020/07/16/us/black-lives-matter-protests-louisville-breonna-taylor.html?action=click\&pgtype=Article\&state=default\&region=TOP_BANNER\&context=storylines_menu}{45
  Days in Louisville}
\end{itemize}

Advertisement

\protect\hyperlink{after-top}{Continue reading the main story}

Supported by

\protect\hyperlink{after-sponsor}{Continue reading the main story}

\hypertarget{james-bennet-resigns-as-new-york-times-opinion-editor}{%
\section{James Bennet Resigns as New York Times Opinion
Editor}\label{james-bennet-resigns-as-new-york-times-opinion-editor}}

A. G. Sulzberger noted ``a significant breakdown in our editing
processes'' before the publication of an Op-Ed by a United States
senator calling for a military response to civic unrest.

\includegraphics{https://static01.graylady3jvrrxbe.onion/images/2020/06/04/business/04UNREST-MEDIA-BENNET/04UNREST-MEDIA-BENNET-articleLarge-v2.jpg?quality=75\&auto=webp\&disable=upscale}

\href{https://www.nytimes3xbfgragh.onion/by/marc-tracy}{\includegraphics{https://static01.graylady3jvrrxbe.onion/images/2018/02/20/multimedia/author-marc-tracy/author-marc-tracy-thumbLarge.jpg}}

By \href{https://www.nytimes3xbfgragh.onion/by/marc-tracy}{Marc Tracy}

\begin{itemize}
\item
  June 7, 2020
\item
  \begin{itemize}
  \item
  \item
  \item
  \item
  \item
  \item
  \end{itemize}
\end{itemize}

James Bennet resigned on Sunday from his job as the editorial page
editor of The New York Times, days after the newspaper's opinion
section, which he oversaw, published a much-criticized Op-Ed by a United
States senator calling for a military response to civic unrest in
American cities.

``Last week we saw a significant breakdown in our editing processes, not
the first we've experienced in recent years,'' said A. G. Sulzberger,
the publisher, in a note to the staff on Sunday announcing Mr. Bennet's
departure.

In a brief interview, Mr. Sulzberger added: ``Both of us concluded that
James would not be able to lead the team through the next leg of change
that is required.''

At an all-staff virtual meeting on Friday, Mr. Bennet, 54, apologized
for the Op-Ed, saying that it should not have been published and that it
had not been edited carefully enough. An editors' note posted late
Friday noted factual inaccuracies and a ``needlessly harsh'' tone. ``The
essay fell short of our standards and should not have been published,''
the note said.

The
\href{https://www.nytimes3xbfgragh.onion/2020/06/03/business/tom-cotton-op-ed.html}{Op-Ed},
by Senator Tom Cotton, Republican of Arkansas, had ``Send In the
Troops'' as its headline. ``One thing above all else will restore order
to our streets: an overwhelming show of force to disperse, detain and
ultimately deter lawbreakers,'' he wrote. The piece, published on
Wednesday, drew anger from readers and Times journalists. Mr. Bennet
declined to comment.

Mr. Bennet's swift fall from one of the most powerful positions in
American journalism comes as
\href{https://www.nytimes3xbfgragh.onion/2020/06/07/us/protests-today-george-floyd-video.html}{hundreds
of thousands of people} have marched in recent weeks in protest of
racism in law enforcement and society. The protests were set in motion
when George Floyd, a black man in Minneapolis, died last month after
being handcuffed and pinned to the ground by a white police officer's
knee.

The foment
\href{https://www.nytimes3xbfgragh.onion/2020/06/07/business/media/inside-the-revolts-erupting-in-americas-big-newsrooms.html}{has
reached other newsrooms}. On Saturday night, Stan Wischnowski
\href{https://www.nytimes3xbfgragh.onion/2020/06/06/business/media/editor-philadephia-inquirer-resigns.html}{resigned}
as top editor of The Philadelphia Inquirer days after an article in the
newspaper about the effects of protests on the urban landscape carried
the headline ``Buildings Matter, Too.'' The headline prompted an apology
published in The Inquirer, a heated staff meeting and a ``sickout'' by
dozens of journalists at the paper.

Mr. Bennet's tenure as editorial page editor, which started in 2016, was
marked by several missteps. Last spring, The Times
\href{https://www.nytimes3xbfgragh.onion/2019/04/28/business/ny-times-anti-semitic-cartoon.html}{apologized}
for an anti-Semitic cartoon that appeared in the Opinion pages of its
international edition.

Last August, a federal appellate court
\href{https://www.nytimes3xbfgragh.onion/2019/08/06/business/sarah-palin-lawsuit-new-york-times.html}{found}
that Sarah Palin, the former vice-presidential candidate, could proceed
with a defamation lawsuit against The Times over an editorial edited by
Mr. Bennet that inaccurately linked her statements to the 2011 shooting
of a congresswoman.

During Mr. Bennet's first year on the job, two Times national security
reporters
\href{https://www.buzzfeednews.com/article/stevenperlberg/russia-critic-sparks-feud-at-the-new-york-times}{publicly
objected} to an Op-Ed by the journalist Louise Mensch, who cited her own
reporting on United States law enforcement's purported monitoring of the
Trump presidential campaign. Times reporters who had covered the same
story, along with reporters at other outlets, were skeptical of her
claim.

Mr. Bennet worked and held key jobs in the Times newsroom from 1991
until 2006, when he left the newspaper to become the editor of The
Atlantic. Since his return, he has widely been considered a
\href{https://www.nytimes3xbfgragh.onion/2020/06/07/business/media/james-bennet-resigns-nytimes-op-ed.html}{possible
successor} to Dean Baquet, who has been in charge of the newsroom for
six years.

In his four years as editorial page editor, Mr. Bennet sought to expand
Opinion's range, making it more responsive to breaking news and better
positioned to cover the tech industry. While he hired several
progressive columnists and contributors, he also added conservative
voices to the traditionally liberal department.

He reduced the number of unsigned editorials and encouraged editorial
board members to write more signed opinion pieces; one editorial board
member, Brent Staples, won the Pulitzer Prize for editorial writing last
year for a series of opinion columns on race in America.

Under Mr. Bennet, the opinion section also published investigative
journalism, developed newsletters and a podcast. It also
\href{https://www.nytimes3xbfgragh.onion/2018/09/05/business/media/new-york-times-trump-anonymous.html}{published}
a much-discussed Op-Ed by an anonymous Trump administration official who
described a ``quiet resistance'' within the federal government.

The most prominent conservative columnist hired by Mr. Bennet, Bret
Stephens, angered many readers with his
\href{https://www.nytimes3xbfgragh.onion/2017/04/28/opinion/climate-of-complete-certainty.html}{inaugural
Times column}, in which he chastised the ``moral superiority'' of those
who look down on climate-change skeptics. Late last year, Mr. Stephens
published another column, headlined
\href{https://www.nytimes3xbfgragh.onion/2019/12/27/opinion/jewish-culture-genius-iq.html}{``The
Secrets of Jewish Genius,''} that led to widespread criticism. After a
review, the editors appended a note to the column and re-edited it to
remove a reference to a study cited in the original version after it was
revealed that one of the study's authors had promoted racist views.

Mr. Bennet is the brother of Michael Bennet, a U.S. senator from
Colorado, and he recused himself from presidential campaign coverage
during his brother's unsuccessful run for this year's Democratic
nomination.

Katie Kingsbury, a deputy editorial page editor, will be the acting
editorial page editor through the November election, Mr. Sulzberger said
in his memo to the staff. Jim Dao, the deputy editorial page editor who
oversees Op-Eds, is stepping down from his position, which was on the
Times masthead, and taking a new job in the newsroom. Mr. Baquet, the
executive editor, said Sunday that he and Mr. Dao had just started
discussing possible jobs for Mr. Dao. Mr. Dao did not reply to a request
for comment.

Ms. Kingsbury, 41, was hired in 2017. Previously she was on The Boston
Globe's editorial board, where she won a Pulitzer for editorial writing
and edited another Pulitzer-winning series.

In a note to the Opinion staff Sunday, Ms. Kingsbury, who declined to
comment for this article, said that until a more ``technical solution''
is in place, anyone who sees ``any piece of Opinion journalism ---
including headlines or social posts or photos or you name it --- that
gives you the slightest pause, please call or text me immediately.''

Senator Cotton's Op-Ed prompted criticism on social media from many
Times employees from different departments, an online protest that was
led by African-American staff members. Much of the dissent included
tweets that said the Op-Ed ``puts Black @NYTimes staff in danger.''
Times employees objected despite a company policy instructing them not
to post partisan comments on social media or take sides on issues in
public forums.

In addition, more than 800 staff members had signed a letter by Thursday
evening protesting the Op-Ed's publication. The letter, addressed to
high-ranking editors in the opinion and news divisions, as well as New
York Times Company executives, argued that Mr. Cotton's essay contained
misinformation, such as his depiction of the role of ``antifa'' in the
protests.

Mr. Sulzberger said at the Friday town hall meeting and in his note on
Sunday that a rethinking of Opinion was necessary for an era in which
readers are likely to come upon Op-Eds in social media posts, divorced
from their print context next to the editorial page.

Advertisement

\protect\hyperlink{after-bottom}{Continue reading the main story}

\hypertarget{site-index}{%
\subsection{Site Index}\label{site-index}}

\hypertarget{site-information-navigation}{%
\subsection{Site Information
Navigation}\label{site-information-navigation}}

\begin{itemize}
\tightlist
\item
  \href{https://help.nytimes3xbfgragh.onion/hc/en-us/articles/115014792127-Copyright-notice}{©~2020~The
  New York Times Company}
\end{itemize}

\begin{itemize}
\tightlist
\item
  \href{https://www.nytco.com/}{NYTCo}
\item
  \href{https://help.nytimes3xbfgragh.onion/hc/en-us/articles/115015385887-Contact-Us}{Contact
  Us}
\item
  \href{https://www.nytco.com/careers/}{Work with us}
\item
  \href{https://nytmediakit.com/}{Advertise}
\item
  \href{http://www.tbrandstudio.com/}{T Brand Studio}
\item
  \href{https://www.nytimes3xbfgragh.onion/privacy/cookie-policy\#how-do-i-manage-trackers}{Your
  Ad Choices}
\item
  \href{https://www.nytimes3xbfgragh.onion/privacy}{Privacy}
\item
  \href{https://help.nytimes3xbfgragh.onion/hc/en-us/articles/115014893428-Terms-of-service}{Terms
  of Service}
\item
  \href{https://help.nytimes3xbfgragh.onion/hc/en-us/articles/115014893968-Terms-of-sale}{Terms
  of Sale}
\item
  \href{https://spiderbites.nytimes3xbfgragh.onion}{Site Map}
\item
  \href{https://help.nytimes3xbfgragh.onion/hc/en-us}{Help}
\item
  \href{https://www.nytimes3xbfgragh.onion/subscription?campaignId=37WXW}{Subscriptions}
\end{itemize}
