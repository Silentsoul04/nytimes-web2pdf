Sections

SEARCH

\protect\hyperlink{site-content}{Skip to
content}\protect\hyperlink{site-index}{Skip to site index}

\href{https://myaccount.nytimes3xbfgragh.onion/auth/login?response_type=cookie\&client_id=vi}{}

\href{https://www.nytimes3xbfgragh.onion/section/todayspaper}{Today's
Paper}

Soothing Incense, Recommended by T Editors

\url{https://nyti.ms/2UqYlGK}

\begin{itemize}
\item
\item
\item
\item
\item
\end{itemize}

Advertisement

\protect\hyperlink{after-top}{Continue reading the main story}

Supported by

\protect\hyperlink{after-sponsor}{Continue reading the main story}

\hypertarget{soothing-incense-recommended-by-t-editors}{%
\section{Soothing Incense, Recommended by T
Editors}\label{soothing-incense-recommended-by-t-editors}}

Scented smoke for the good of your space and state of mind.

\includegraphics{https://static01.graylady3jvrrxbe.onion/images/2020/06/09/t-magazine/09tmag-incense-slide-9DWG/09tmag-incense-slide-9DWG-articleLarge.jpg?quality=75\&auto=webp\&disable=upscale}

June 9, 2020

\begin{itemize}
\item
\item
\item
\item
\item
\end{itemize}

From around the third century B.C. to the second century A.D., incense
was sufficiently sought after to warrant its own trade route, the
Incense Route, on which frankincense and myrrh were ferried from
Ethiopia, Somalia and southern Arabia and through the Negev Desert to
various Mediterranean and Asian destinations, and has by now spent
millenniums as a fixture of myriad countries and cultures. In ancient
Egypt, it was used to ward off evil forces. Buddhist monks, too, relied
(and still rely) on incense to purify their meditation spaces, while in
Japan, the Koh-Do (incense ceremony) has been popular since courtiers
and samurai took it up in the 16th century. Of course, incense also just
smells nice, its scent hanging in the room even after the physical
material has burned to a tidy pile of ash. But protection and
cleanliness feel like particularly appealing aims right now, as does the
idea of creating a ritual, and thus imposing a modicum of regularity and
intention amid so much uncertainty.

\hypertarget{nag-champa-by-satya-sai-baba}{%
\subsubsection{\texorpdfstring{\textbf{\href{https://www.iherb.com/pr/Sai-Baba-Satya-Nag-Champa-Agarbatti-Incense-Sticks-100-g/37611?gclid=EAIaIQobChMIpfLruoXG6QIVTLLICh2AEwCkEAQYAiABEgIdYvD_BwE\&gclsrc=aw.ds}{Nag
Champa by Satya Sai
Baba}}}{Nag Champa by Satya Sai Baba}}\label{nag-champa-by-satya-sai-baba}}

In college, I worked at a feminist bookseller called Sisterhood
Bookstore in Los Angeles. It was opened in 1972 by two former
sisters-in-law, both of whom divorced the brothers. As one can imagine,
the store was small, being that it was filled only with books for and
about women, so when I worked there, I worked alone. But during my
training, I remember Julie, the manager, showing me how to open. After
unlocking the door, she instructed, I would walk into the bathroom and
fire up a stick of
\href{https://www.iherb.com/pr/Sai-Baba-Satya-Nag-Champa-Agarbatti-Incense-Sticks-100-g/37611?gclid=EAIaIQobChMIpfLruoXG6QIVTLLICh2AEwCkEAQYAiABEgIdYvD_BwE\&gclsrc=aw.ds}{Nag
Champa incense} (\$10.85). I'd watch the smoke waft out of the restroom
and into the front of the store. The first hint of sandalwood was
followed by the sweetness of plumeria, and it took a while for the full
scent to fill the space. Incense smells best after it's finished
burning, when it's a lingering note that has permeated your clothing,
your skin, your hair, and you can't quite recall why everything smells
the same. But for me, during those solo eight-hour shifts, it was also a
marker of time. Once I realized the smoke had stopped, I'd light up
another one. --- MINJU PAK

\hypertarget{ground-by-bodha}{%
\subsubsection{\texorpdfstring{\textbf{\href{https://bodha.com/shop/ground-ritual-incense}{Ground
by Bodha}}}{Ground by Bodha}}\label{ground-by-bodha}}

Incense stimulates my senses in a way that makes me feel more grounded
and alert, so it's fitting that I favor
\href{https://bodha.com/shop/ground-ritual-incense}{Bodha's Ground
variety} (\$32). I discovered it one day in the dead of winter in a
small West Village boutique. As soon as I passed through the door, my
nose was struck with the warming scents of hinoki, cedar wood and
frankincense, and I knew I needed the smell in my home. Another thing I
like about these sticks, which are made in Japan with some of the best
organic woods and essential oils, is that the burn is smokeless. As a
result, I feel good about what I'm inhaling and consider them a sort of
antidote to the pollution that comes with city living and inevitably
seeps in from outdoors. --- KRISTINA SAMULEWSKI

\hypertarget{sandalwood-by-hayashi-ryushodo-and-commune}{%
\subsubsection{\texorpdfstring{\textbf{\href{https://www.communedesign.com/shop/objects/commune-incense-sticks-4405788377146/}{Sandalwood
by Hayashi Ryushodo and
Commune}}}{Sandalwood by Hayashi Ryushodo and Commune}}\label{sandalwood-by-hayashi-ryushodo-and-commune}}

I've collected artisanal incense during travels to many of its
modern-day capitals --- from Big Sur to Baja, Mexico, to Tokyo --- but
\href{https://www.communedesign.com/shop/objects/commune-incense-sticks-4405788377146/}{my
current favorite} comes from the Los Angeles-based design firm Commune.
Working with the incense master Keijirou Hayashi, who runs Hayashi
Ryushodo, his family's 186-year-old shop in Kyoto, they packaged more
than 100 hand-rolled incense sticks into a vibey, minimalist white box
(\$32) that's taken pride of place on one of my bookshelves. I love the
sandalwood scent, which is musty and a bit mustardy; it's one of the
purest expressions of the aromatic wood that I've smelled in my lifetime
--- and the sticks are a pleasure to burn in late spring, when they make
my entire small apartment smell like an ancestral forest after it's
rained. --- KURT SOLLER

Image

Côte d'Azur by Oribe.Credit...Courtesy of Oribe

Image

Ground by Bodha.Credit...Courtesy of Bodha

\hypertarget{yakushima-by-astier-de-villatte}{%
\subsubsection{\texorpdfstring{\textbf{\href{https://www.johnderian.com/products/yakushima-incense}{Yakushima
by Astier de
Villatte}}}{Yakushima by Astier de Villatte}}\label{yakushima-by-astier-de-villatte}}

Even before
\href{https://www.nytimes3xbfgragh.onion/2018/05/17/t-magazine/yakushima-japan-hayao-miyazaki-princess-mononoke.html}{I
visited} the Japanese island of Yakushima for myself, I had already seen
it, thanks to Hayao Miyazaki's ``Princess Mononoke'' (1997), whose
enchanted forest is based on the old-growth cedar forest that dominates
the isle. I'd also smelled it --- an ancient, green, haunted scent, of
millennia-old cedar trees and moss --- thanks to
\href{https://www.johnderian.com/products/yakushima-incense}{this
incense} (\$50), which I burn year-round. It makes winters darker and
cozier, and in the summer, it keeps bugs away. I also like to give it to
people who're nice enough to let me stay at their country houses; you
can just put a few sticks in a jar or glass and take it outside to burn
the night away. --- HANYA YANAGIHARA

\hypertarget{cuxf4te-dazur-by-oribe}{%
\subsubsection{\texorpdfstring{\textbf{\href{https://www.oribe.com/oribestorefront/oribe/en/Shop-All/Body-\%26-Fragrance/C\%C3\%B4te-d\%27Azur-Incense/p/400623}{Côte
d'Azur by Oribe}}}{Côte d'Azur by Oribe}}\label{cuxf4te-dazur-by-oribe}}

Long ago, my good friend worked at a store called Sunshine Daydream.
Ever since, the smell of patchouli has reminded me of chatting with her
over the case of handblown glass pipes and then, whenever her boss
appeared, pretending to shop for a tie-dye tapestry. Fond as those
memories are, my preference is for less earthy scents, and I recently
discovered that, two years ago, the hair-care brand Oribe, along with
the help of artisans on the Japanese island of Awaji, reconceived its
signature Côte d'Azur scent
\href{https://www.oribe.com/oribestorefront/oribe/en/Shop-All/Body-\%26-Fragrance/C\%C3\%B4te-d\%27Azur-Incense/p/400623}{in
incense form} (\$65). The blend balances floral notes including
Calabrian bergamot, tuberose and butterfly jasmine with those of
sandalwood and amber, and its smooth scent feels especially perfect for
this summer, when I probably won't be going to a salon, let alone the
French Riviera. --- KATE GUADAGNINO

\hypertarget{frankincense-resin-from-mountain-rose-herbs}{%
\subsubsection{\texorpdfstring{\textbf{\href{https://www.mountainroseherbs.com/products/frankincense-resin/profile}{Frankincense
Resin from Mountain Rose
Herbs}}}{Frankincense Resin from Mountain Rose Herbs}}\label{frankincense-resin-from-mountain-rose-herbs}}

A couple of years ago, I found myself struggling with daily anxiety that
would pop up right as I got home from work, giving me way too many
sleepless nights. I'm a huge supporter of seeking professional help when
it comes to mental health, but I was also curious about CBD and other
natural alternatives as a starting point. After trying a few products
that didn't prove all that effective, I happened to read **** an article
about archaeology and stumbled across a paragraph on
\href{https://www.mountainroseherbs.com/products/frankincense-resin/profile}{how
frankincense} (from \$9.25) has been used for centuries to combat stress
and depression. Archaeology nerd that I am, I had to try it, and found
that not only did it settle my mental state, but that I also loved the
aroma, which is how I imagine Arrakis, the spice-rich planet of Frank
Herbert's 1965 science fiction novel ``Dune,'' might smell. I have since
learned, however, that we're losing
\href{https://www.nytimes3xbfgragh.onion/2019/07/05/science/frankincense-trees-collapse.html}{Boswellia
sacra trees}, from which frankincense is harvested, at an alarming rate,
and am now on the hunt for an ecologically sustainable source or
alternative. --- CARTER LOVE

Image

Frankincense tears.Credit...Green Cottage Creek

Image

Yakushima by Astier de Villatte.Credit...Courtesy of Astier de Villatte

\hypertarget{oulan-bator-by-astier-de-villatte}{%
\subsubsection{\texorpdfstring{\textbf{\href{https://www.johnderian.com/products/oulan-bator-incense}{Oulan
Bator by Astier de
Villatte}}}{Oulan Bator by Astier de Villatte}}\label{oulan-bator-by-astier-de-villatte}}

I am not a frequent incense burner, but I have never been able to help
myself with Astier de Villatte, in part because their earthenware
incense holders are so beautiful. I prefer to burn
\href{https://www.johnderian.com/products/oulan-bator-incense}{Oulan
Bator} (\$50) --- which is evocative of leather and ambergris --- but
I've also been intrigued by another one of their scents, Atelier de
Balthus, which nods to the 20th-century Polish-French artist and
promises, with the smell of turpentine mixed with smoke, honey, tobacco
and cedar wood, to transport you to a beautiful studio with big windows
that overlook the meadows. --- THESSALY LA FORCE

\hypertarget{sandalwood-from-the-good-liver}{%
\subsubsection{\texorpdfstring{\textbf{\href{https://good-liver.com/products/incense-sandalwood}{Sandalwood
from the Good
Liver}}}{Sandalwood from the Good Liver}}\label{sandalwood-from-the-good-liver}}

I don't quite remember how I first came across
\href{https://good-liver.com/products/incense-sandalwood}{this incense}
(\$16), but for the past two years or so, I've regularly bought boxes of
it online from the Los Angeles general store the Good Liver. It's a
traditional Japanese sandalwood incense that smells something like a
forest floor and is made with natural ingredients in the Shiga
Prefecture. According to the store's website, it is burned ``at well
regarded temples in Shiga and Kyoto,'' and something about the fact that
it has a sacred usage and is seemingly only available in the States at
this one shop makes me feel like a little princeling whenever I order it
(even though it's relatively inexpensive). But I intersperse burning it
with phases of using Muji's Japanese Scents incense, which costs even
less and has a slightly lighter, more powdery scent --- plus, the sticks
are just a couple of inches in length, so they never burn long enough to
give you a headache. --- ALICE NEWELL-HANSON

\hypertarget{12am-by-cinnamon-projects}{%
\subsubsection{\texorpdfstring{\textbf{\href{https://www.cinnamonprojects.com/product/incense-solo-12am/}{12AM
by Cinnamon
Projects}}}{12AM by Cinnamon Projects}}\label{12am-by-cinnamon-projects}}

Because I gravitate toward crisp fragrances, I've generally chosen
candles or essential oils for scenting a space --- incense always felt
like something better suited to a head shop than my living room.
However,
\href{https://www.cinnamonprojects.com/product/incense-solo-12am/}{this
warm, woody blend} (\$30) from the New York creative agency Cinnamon
Projects has completely changed my perspective. Made according to
traditional methods in Japan with notes of amber and clove, its scent is
cozy but, thanks to additional notes of lavender and oakmoss, not
cloying. Plus, the brand's handcrafted bronze and semiprecious stone
burners make using it feel ceremonial in a way that lighting a candle
simply does not. --- CAITIE KELLY

Advertisement

\protect\hyperlink{after-bottom}{Continue reading the main story}

\hypertarget{site-index}{%
\subsection{Site Index}\label{site-index}}

\hypertarget{site-information-navigation}{%
\subsection{Site Information
Navigation}\label{site-information-navigation}}

\begin{itemize}
\tightlist
\item
  \href{https://help.nytimes3xbfgragh.onion/hc/en-us/articles/115014792127-Copyright-notice}{©~2020~The
  New York Times Company}
\end{itemize}

\begin{itemize}
\tightlist
\item
  \href{https://www.nytco.com/}{NYTCo}
\item
  \href{https://help.nytimes3xbfgragh.onion/hc/en-us/articles/115015385887-Contact-Us}{Contact
  Us}
\item
  \href{https://www.nytco.com/careers/}{Work with us}
\item
  \href{https://nytmediakit.com/}{Advertise}
\item
  \href{http://www.tbrandstudio.com/}{T Brand Studio}
\item
  \href{https://www.nytimes3xbfgragh.onion/privacy/cookie-policy\#how-do-i-manage-trackers}{Your
  Ad Choices}
\item
  \href{https://www.nytimes3xbfgragh.onion/privacy}{Privacy}
\item
  \href{https://help.nytimes3xbfgragh.onion/hc/en-us/articles/115014893428-Terms-of-service}{Terms
  of Service}
\item
  \href{https://help.nytimes3xbfgragh.onion/hc/en-us/articles/115014893968-Terms-of-sale}{Terms
  of Sale}
\item
  \href{https://spiderbites.nytimes3xbfgragh.onion}{Site Map}
\item
  \href{https://help.nytimes3xbfgragh.onion/hc/en-us}{Help}
\item
  \href{https://www.nytimes3xbfgragh.onion/subscription?campaignId=37WXW}{Subscriptions}
\end{itemize}
