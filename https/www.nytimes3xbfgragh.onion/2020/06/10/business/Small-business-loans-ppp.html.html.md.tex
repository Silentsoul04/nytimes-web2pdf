Sections

SEARCH

\protect\hyperlink{site-content}{Skip to
content}\protect\hyperlink{site-index}{Skip to site index}

\href{https://www.nytimes3xbfgragh.onion/section/business}{Business}

\href{https://myaccount.nytimes3xbfgragh.onion/auth/login?response_type=cookie\&client_id=vi}{}

\href{https://www.nytimes3xbfgragh.onion/section/todayspaper}{Today's
Paper}

\href{/section/business}{Business}\textbar{}\$130 Billion in
Small-Business Aid Still Hasn't Been Used

\url{https://nyti.ms/2YnYs78}

\begin{itemize}
\item
\item
\item
\item
\item
\end{itemize}

\href{https://www.nytimes3xbfgragh.onion/news-event/coronavirus?action=click\&pgtype=Article\&state=default\&region=TOP_BANNER\&context=storylines_menu}{The
Coronavirus Outbreak}

\begin{itemize}
\tightlist
\item
  live\href{https://www.nytimes3xbfgragh.onion/2020/08/04/world/coronavirus-cases.html?action=click\&pgtype=Article\&state=default\&region=TOP_BANNER\&context=storylines_menu}{Latest
  Updates}
\item
  \href{https://www.nytimes3xbfgragh.onion/interactive/2020/us/coronavirus-us-cases.html?action=click\&pgtype=Article\&state=default\&region=TOP_BANNER\&context=storylines_menu}{Maps
  and Cases}
\item
  \href{https://www.nytimes3xbfgragh.onion/interactive/2020/science/coronavirus-vaccine-tracker.html?action=click\&pgtype=Article\&state=default\&region=TOP_BANNER\&context=storylines_menu}{Vaccine
  Tracker}
\item
  \href{https://www.nytimes3xbfgragh.onion/2020/08/02/us/covid-college-reopening.html?action=click\&pgtype=Article\&state=default\&region=TOP_BANNER\&context=storylines_menu}{College
  Reopening}
\item
  \href{https://www.nytimes3xbfgragh.onion/live/2020/08/04/business/stock-market-today-coronavirus?action=click\&pgtype=Article\&state=default\&region=TOP_BANNER\&context=storylines_menu}{Economy}
\end{itemize}

Advertisement

\protect\hyperlink{after-top}{Continue reading the main story}

Supported by

\protect\hyperlink{after-sponsor}{Continue reading the main story}

\hypertarget{130-billion-in-small-business-aid-still-hasnt-been-used}{%
\section{\$130 Billion in Small-Business Aid Still Hasn't Been
Used}\label{130-billion-in-small-business-aid-still-hasnt-been-used}}

Owners are becoming reluctant to borrow from the federal Paycheck
Protection Program. Some are even returning money.

\includegraphics{https://static01.graylady3jvrrxbe.onion/images/2020/06/08/business/08virus-sbareturns3/merlin_173210475_ea8e1475-742b-4751-80cc-9d92f998846b-articleLarge.jpg?quality=75\&auto=webp\&disable=upscale}

\href{https://www.nytimes3xbfgragh.onion/by/stacy-cowley}{\includegraphics{https://static01.graylady3jvrrxbe.onion/images/2018/10/03/multimedia/author-stacy-cowley/author-stacy-cowley-thumbLarge.png}}

By \href{https://www.nytimes3xbfgragh.onion/by/stacy-cowley}{Stacy
Cowley}

\begin{itemize}
\item
  Published June 10, 2020Updated July 6, 2020
\item
  \begin{itemize}
  \item
  \item
  \item
  \item
  \item
  \end{itemize}
\end{itemize}

In April, when the federal government offered \$349 billion in loans to
small businesses reeling from government shutdown orders in the
pandemic,
\href{https://www.nytimes3xbfgragh.onion/2020/04/16/business/coronavirus-sba-loans-out-of-money.html}{the
funding ran out in just 13 days}, prompting Congress to swiftly approve
a second round of \$310 billion.

Small businesses have since grown more wary of taking the money.

As of Tuesday, more than \$130 billion was left in the fund, known as
the
\href{https://www.nytimes3xbfgragh.onion/2020/07/06/us/ppp-small-business-loans.html}{Paycheck
Protection Program}. Even more striking was the fact that on many days
last month, more money was being returned than borrowed, according to
data from the Small Business Administration, which is overseeing the
program --- highlighting its messy execution and confusing rules that
deterred some small businesses from using the money.

Thousands of companies that got loans have sent the money back,
according to lenders. For some owners, the program's terms were too
restrictive; for others, the criteria for loan forgiveness was too
murky. Some public companies that received these loans
\href{https://www.nytimes3xbfgragh.onion/2020/05/18/business/ppp-deadline-return-funds-coronavirus.html}{returned
them after a public outcry}, and in the initial rush, some borrowers
\href{https://www.reuters.com/article/us-health-coronavirus-usa-ppp-exclusive/exclusive-us-small-business-program-handed-out-virus-aid-to-many-borrowers-twice-idUSKBN2391S9}{accidentally
got duplicate loans} that they, too, returned.

A total of around \$12 billion was returned, Treasury Secretary Steven
Mnuchin said at a Senate hearing on Wednesday. The amount of loans
outstanding under the program dropped to \$510.2 billion at the end of
May,
\href{https://content.sba.gov/sites/default/files/2020-05/PPP_Report_Net_200518_0.pdf}{from
\$513.3 billion} in the middle of the month, according to data from the
Small Business Administration.

By Tuesday, the amount of approved loans had inched back up to \$511.4
billion --- indicating that changes Congress made to the program last
week to make it less restrictive could be pushing more money out the
door.

But obstacles remain. The program's chaotic execution has ``chilled the
willingness of many small businesses to even apply for loans during the
second round of
\href{https://www.nytimes3xbfgragh.onion/2020/06/30/business/paycheck-protection-program-coronavirus.html}{P.P.P.}
funding, and has caused many businesses to return disbursed loans out of
fear of doing something wrong,'' Tony Wilkinson, the chief executive of
the National Association of Government Guaranteed Lenders, a trade
group, said last week at a hearing of the Pandemic Response
Accountability Committee, an oversight group.

The turn of events is notable for a signature program of Congress's
\$2.2 trillion coronavirus relief package, which only a couple of months
ago was caught in
\href{https://www.nytimes3xbfgragh.onion/2020/04/16/business/coronavirus-sba-loans-out-of-money.html}{an
intense borrowing frenzy} by desperate business owners. After all, small
businesses are still in distress. Even as states begin to reopen,
millions of stores around the country remain shuttered and
\href{https://www.nytimes3xbfgragh.onion/article/small-business-bankruptcy-coronavirus.html}{could
go out of business}.

\hypertarget{latest-updates-economy}{%
\section{\texorpdfstring{\href{https://www.nytimes3xbfgragh.onion/live/2020/08/04/business/stock-market-today-coronavirus?action=click\&pgtype=Article\&state=default\&region=MAIN_CONTENT_1\&context=storylines_live_updates}{Latest
Updates:
Economy}}{Latest Updates: Economy}}\label{latest-updates-economy}}

\href{https://www.nytimes3xbfgragh.onion/live/2020/08/04/business/stock-market-today-coronavirus?action=click\&pgtype=Article\&state=default\&region=MAIN_CONTENT_1\&context=storylines_live_updates\#fox-corporations-plunging-profit-is-cushioned-by-fox-news}{19m
ago}

\href{https://www.nytimes3xbfgragh.onion/live/2020/08/04/business/stock-market-today-coronavirus?action=click\&pgtype=Article\&state=default\&region=MAIN_CONTENT_1\&context=storylines_live_updates\#fox-corporations-plunging-profit-is-cushioned-by-fox-news}{Fox
Corporation's plunging profit is cushioned by Fox News.}

\href{https://www.nytimes3xbfgragh.onion/live/2020/08/04/business/stock-market-today-coronavirus?action=click\&pgtype=Article\&state=default\&region=MAIN_CONTENT_1\&context=storylines_live_updates\#trading-in-kodak-shares-comes-under-scrutiny}{43m
ago}

\href{https://www.nytimes3xbfgragh.onion/live/2020/08/04/business/stock-market-today-coronavirus?action=click\&pgtype=Article\&state=default\&region=MAIN_CONTENT_1\&context=storylines_live_updates\#trading-in-kodak-shares-comes-under-scrutiny}{Trading
in Kodak shares comes under scrutiny.}

\href{https://www.nytimes3xbfgragh.onion/live/2020/08/04/business/stock-market-today-coronavirus?action=click\&pgtype=Article\&state=default\&region=MAIN_CONTENT_1\&context=storylines_live_updates\#disney-lost-4-7-billion-last-quarter-but-its-newest-business-was-a-big-hit}{2h
ago}

\href{https://www.nytimes3xbfgragh.onion/live/2020/08/04/business/stock-market-today-coronavirus?action=click\&pgtype=Article\&state=default\&region=MAIN_CONTENT_1\&context=storylines_live_updates\#disney-lost-4-7-billion-last-quarter-but-its-newest-business-was-a-big-hit}{Disney
lost \$4.7 billion last quarter, but its newest business was a big hit.}

\href{https://www.nytimes3xbfgragh.onion/live/2020/08/04/business/stock-market-today-coronavirus?action=click\&pgtype=Article\&state=default\&region=MAIN_CONTENT_1\&context=storylines_live_updates}{See
more updates}

More live coverage:
\href{https://www.nytimes3xbfgragh.onion/2020/08/04/world/coronavirus-cases.html?action=click\&pgtype=Article\&state=default\&region=MAIN_CONTENT_1\&context=storylines_live_updates}{Global}

On Wednesday last week, Congress moved to
\href{https://www.nytimes3xbfgragh.onion/2020/06/03/business/stock-market-today-coronavirus.html\#link-7a1554a7}{loosen
the program's rules and give businesses more flexibility} in spending
their aid, and President Trump signed the bill on Friday. The change was
widely praised by small-businesses advocacy groups and will help many
borrowers.

The amended rules could help the remaining \$130 billion move faster.
``My expectation is that we will definitely see businesses that were on
the sidelines now take it," Mr. Mnuchin said.

But having the terms of their loans revised on the fly yet again ---
which has happened repeatedly since the program began in April --- is a
nightmare for borrowers as they struggle to salvage their companies.

``The boundaries move like a video game,'' said Caren Griffin, the owner
of \href{https://spauniversaire.com/}{Spa Universaire}, a hotel spa in
downtown Denver. Her business has been shut down since mid-March and
won't reopen until July, at the earliest.

Ms. Griffin was on the verge of returning her \$66,272 loan because she
didn't think she would be able to use it within eight weeks, as the
program originally specified. The new changes
\href{https://www.jdsupra.com/legalnews/congress-adds-more-flexibility-to-71347/}{give
her 24 weeks}, allowing her more time to decide whether to use the money
when --- and if --- she reopens.

She isn't sure she will. Refitting her spa to comply with new safety
guidelines will be expensive, and no one knows when customers will be
willing to get high-touch services like massages and facials. Ms.
Griffin, who is 63, wonders if she would be better off closing the
business and retiring.

``I'm running through a dozen different scenarios for what our cash-flow
structure might look like if we reopened with changes in our hours and
services,'' she said. ``We won't go back to normal. That's clear.''

For many small businesses that depend on foot traffic, like restaurants
and nail salons, even the more relaxed relief terms might not be enough.

``I cried the day I sent it back,'' said Shelly Ross, the owner of
\href{https://talesofthekitty.com/}{Tales of the Kitty}, a cat-sitting
service in San Francisco, who recently gave up on using her \$75,000
loan and returned the money. ``I thought this would save my business,
but I was worried about being financially ruined if it wasn't forgiven,
and no one could give me any real answers about that.''

Ms. Ross started Tales of the Kitty in 2003 and expanded it into a
thriving venture with 14 employees and a packed schedule of 10,000
client visits a year. In March, her sales plummeted because of the
pandemic, forcing her to lay people off.

To have their loans fully forgiven, companies must keep the same number
of employees on the payroll as before the pandemic, at the same wages.
That's a hard bar to clear for business owners whose ventures remain
shut. Ms. Ross expects her sales to stay slow at least through
Thanksgiving.

The paycheck program was largely meant to keep workers off unemployment
rolls by funneling money to their employers, and it made progress toward
that goal. Job numbers released on Friday
\href{https://www.nytimes3xbfgragh.onion/2020/06/05/business/economy/jobs-report.html}{showed
gains in industries} that received small-business loans and suggested
that the program had helped offset at least some of the economic damage
caused by the shutdown.

More than 4.5 million companies, from solo gig workers and small
restaurateurs to retailers and professional services firms with fewer
than 500 employees, got help. For most,
\href{https://www.nytimes3xbfgragh.onion/2020/05/31/business/virus-ppp-loan-small-businesses.html}{it
was a lifeline}.

But many lenders are ready to be done with the program. Even though the
government will keep processing loans until June 30, Wells Fargo stopped
taking new applications last month. So did many community banks,
including OceanFirst, a regional bank in Toms River, N.J., that has lent
\$500 million to 3,000 businesses.

Requests had slowed to a trickle, said Christopher Maher, the bank's
chief executive. He cut off new lending because he wanted to pivot his
employees toward the next major program challenge: loan forgiveness.

``It's going to be much more difficult to work through the repayment
calculations and documentation than it was to make these loans in the
first place,'' Mr. Maher said.

The paperwork for proving that a loan complied with the terms is
extensive and complicated; the Small Business Administration's
\href{https://www.sba.gov/sites/default/files/2020-05/3245-0407\%20SBA\%20Form\%203508\%20PPP\%20Forgiveness\%20Application.pdf}{11-page
application} for loan forgiveness is much more intricate than the loan
application itself. Any portion that is not forgiven becomes a debt that
must be repaid within five years. (The initial term was two years.)

The lack of clarity around loan forgiveness cemented Ms. Ross's decision
to return her loan. She considered simply paying her workers to stay
home for eight weeks, which the program allows, but she worried about
having to lay them off again when the money ran out.

Then, the week after she got her loan, the Small Business Administration
released its forgiveness application. Ms. Ross tried to run the
calculations for her business, but her staff is mostly part-time
employees with variable hours. She consulted her accountant, her
bookkeeper, a lawyer and her lender to figure out how much of her loan
would qualify for elimination.

None could give her definitive answers. Scared she would be stuck with a
big debt, Ms. Ross sent the money back.

Many trade groups for lenders and small businesses, such as the Consumer
Bankers Association and the Small Business Majority, have asked for
blanket forgiveness for loans of less than \$150,000. So far, the Small
Business Administration and the Treasury Department --- which is calling
the shots on most of the paycheck program's terms --- have not indicated
any willingness to grant that.

Adam Markowitz, an accountant in Florida who is working with dozens of
clients who took paycheck loans, said he was unable to figure out
whether his own loan, for \$34,500, would be fully forgiven. He's
waiting for further guidance on the many unresolved technical questions.

``The only consistent thing about this program is that it's been a mess
at every stage,'' he said.

Advertisement

\protect\hyperlink{after-bottom}{Continue reading the main story}

\hypertarget{site-index}{%
\subsection{Site Index}\label{site-index}}

\hypertarget{site-information-navigation}{%
\subsection{Site Information
Navigation}\label{site-information-navigation}}

\begin{itemize}
\tightlist
\item
  \href{https://help.nytimes3xbfgragh.onion/hc/en-us/articles/115014792127-Copyright-notice}{©~2020~The
  New York Times Company}
\end{itemize}

\begin{itemize}
\tightlist
\item
  \href{https://www.nytco.com/}{NYTCo}
\item
  \href{https://help.nytimes3xbfgragh.onion/hc/en-us/articles/115015385887-Contact-Us}{Contact
  Us}
\item
  \href{https://www.nytco.com/careers/}{Work with us}
\item
  \href{https://nytmediakit.com/}{Advertise}
\item
  \href{http://www.tbrandstudio.com/}{T Brand Studio}
\item
  \href{https://www.nytimes3xbfgragh.onion/privacy/cookie-policy\#how-do-i-manage-trackers}{Your
  Ad Choices}
\item
  \href{https://www.nytimes3xbfgragh.onion/privacy}{Privacy}
\item
  \href{https://help.nytimes3xbfgragh.onion/hc/en-us/articles/115014893428-Terms-of-service}{Terms
  of Service}
\item
  \href{https://help.nytimes3xbfgragh.onion/hc/en-us/articles/115014893968-Terms-of-sale}{Terms
  of Sale}
\item
  \href{https://spiderbites.nytimes3xbfgragh.onion}{Site Map}
\item
  \href{https://help.nytimes3xbfgragh.onion/hc/en-us}{Help}
\item
  \href{https://www.nytimes3xbfgragh.onion/subscription?campaignId=37WXW}{Subscriptions}
\end{itemize}
