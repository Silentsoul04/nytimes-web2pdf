Sections

SEARCH

\protect\hyperlink{site-content}{Skip to
content}\protect\hyperlink{site-index}{Skip to site index}

\href{https://myaccount.nytimes3xbfgragh.onion/auth/login?response_type=cookie\&client_id=vi}{}

\href{https://www.nytimes3xbfgragh.onion/section/todayspaper}{Today's
Paper}

\href{/section/opinion}{Opinion}\textbar{}{[}Black Lives Matter Is
Winning

\begin{itemize}
\item
\item
\item
\item
\item
\item
\end{itemize}

Advertisement

\protect\hyperlink{after-top}{Continue reading the main story}

\href{/section/opinion}{Opinion}

Supported by

\protect\hyperlink{after-sponsor}{Continue reading the main story}

\hypertarget{black-lives-matter-is-winning}{%
\section{{[}Black Lives Matter Is
Winning}\label{black-lives-matter-is-winning}}

Activists set out to show that police brutality was pervasive. The
police have now made that clear.

\href{https://www.nytimes3xbfgragh.onion/by/farhad-manjoo}{\includegraphics{https://static01.graylady3jvrrxbe.onion/images/2019/01/08/opinion/farhad-manjoo-opinion/farhad-manjoo-opinion-thumbLarge.png}}

By \href{https://www.nytimes3xbfgragh.onion/by/farhad-manjoo}{Farhad
Manjoo}

Opinion Columnist

\begin{itemize}
\item
  June 10, 2020
\item
  \begin{itemize}
  \item
  \item
  \item
  \item
  \item
  \item
  \end{itemize}
\end{itemize}

\includegraphics{https://static01.graylady3jvrrxbe.onion/images/2020/06/10/opinion/10manjoo1/merlin_173249757_84ae29e9-1d4e-4476-a22d-37dad769fffe-articleLarge.jpg?quality=75\&auto=webp\&disable=upscale}

\hypertarget{listen-to-this-op-ed}{%
\subsubsection{Listen to This Op-Ed}\label{listen-to-this-op-ed}}

Audio Recording by Audm

\emph{To hear more audio stories from publishers like The New York
Times, download}
\href{https://www.audm.com/?utm_source=nytopinion\&utm_medium=embed\&utm_campaign=black_lives_winning}{\emph{Audm
for iPhone or Android}}\emph{.}

It's wondrous, isn't it, how the people just keep coming out? Day after
day, night after night, in dozens of cities, braving a deadly virus and
brutal retaliation, they continue to pack the streets in uncountable
numbers, demanding equality and justice --- and, finally, prompting what
feels like real change.

How did this happen? How did
\href{https://www.nytimes3xbfgragh.onion/interactive/2020/06/10/upshot/black-lives-matter-attitudes.html}{Black
Lives Matter}, a hashtag-powered movement that has been building for
years, bring America to what looks like a turning point?

I have a theory: The protests exploded in scale and intensity because
the police seemed to go out of their way to illustrate exactly the
arguments that Black Lives Matter has been raising online since 2013.

For the last two weeks, the police reaction to the movement has been so
unhinged, and so well documented, that it couldn't help but feed support
for the protests. American public opinion may have tipped in favor of
Black Lives Matter for good.

By ``the police,'' I mean not just state and municipal police across the
country, but also the federal officers from various agencies that
cracked down on protesters in front of the White House, as well as their
supporters and political patrons, from police chiefs to mayors to the
attorney general and the president himself.

Black Lives Matter \href{https://blacklivesmatter.com/about/}{aims} to
highlight the depth of brutality, injustice and unaccountability that
American society, especially law enforcement, harbors toward black
people. Many protesters set out to call attention to the unchecked power
of the police, their military weaponry and their capricious use of it.
They wanted to show that the problem of policing in America is more than
that of individual bad officers; the problem is a culture that protects
wrongdoers, tolerates
\href{https://apnews.com/a172fb01bdb74b4159b39da390d9e79e}{mendacity},
rewards blind loyalty and is fiercely resistant to change. More deeply,
it is a law enforcement culture that does not regard black lives as
worthy of protection.

And what did the cops do? They responded with a display of organized,
unchecked power --- on camera, in a way that many Americans might never
be able to unsee.

To understand why this moment may prompt structural change, it is worth
putting the latest protests into a larger context. To me, the past two
weeks have felt like an echo of that heady moment late in 2017, after
The
\href{https://www.nytimes3xbfgragh.onion/2017/10/05/us/harvey-weinstein-harassment-allegations.html}{New
York Times} and The
\href{https://www.newyorker.com/news/news-desk/from-aggressive-overtures-to-sexual-assault-harvey-weinsteins-accusers-tell-their-stories}{New
Yorker} exposed Harvey Weinstein's history of sexual assault. At the
time, \#MeToo, as an online rallying cry against sexual abuse and
harassment, was
\href{https://www.ajc.com/news/world/who-tarana-burke-meet-the-woman-who-started-the-too-movement-decade-ago/i8NEiuFHKaIvBh9ucukidK/}{more
than a decade old}. The Weinstein story didn't create that movement,
just as the
\href{https://www.nytimes3xbfgragh.onion/video/us/100000007159353/george-floyd-arrest-death-video.html}{videos
of George Floyd's death} at the hands of the Minneapolis police didn't
create Black Lives Matter.

\includegraphics{https://static01.graylady3jvrrxbe.onion/images/2020/06/10/opinion/10manjoo2/merlin_173180409_1a8c2fbf-d894-44e5-b25b-73eac9621144-articleLarge.jpg?quality=75\&auto=webp\&disable=upscale}

Instead, the Weinstein news broke the dam. Since then, \#MeToo activism
\href{https://www.nbcnews.com/storyline/sexual-misconduct/weinstein-here-s-growing-list-men-accused-sexual-misconduct-n816546}{has
gone on to upend} society in a way that felt revolutionary.

It feels like the dam is breaking again.

The movement behind Black Lives Matter has taken to the streets before
--- but nothing on this scale, with this intensity. And not with these
results. The National Football League was once a powerful and bitter
rival;
\href{https://sports.yahoo.com/the-nf-ls-black-lives-matter-statement-was-years-in-the-making-013616877.html}{now
it has}embraced the movement, though it still has not apologized to or
signed
\href{https://www.nytimes3xbfgragh.onion/2020/06/05/sports/football/george-floyd-kaepernick-kneeling-nfl-protests.html}{Colin
Kaepernick}, the player who first knelt in protest against police
brutality.

Politicians at every level
\href{https://www.theatlantic.com/politics/archive/2020/06/mitt-romney-black-lives-matter/612808/}{are
professing newfound support}, and, right before our eyes, the
\href{https://www.nytimes3xbfgragh.onion/2019/02/26/us/politics/overton-window-democrats.html}{Overton
window} of acceptable public discourse about police reform has shifted
to include terms like ``demilitarize,'' ``defund'' and ``abolish.''

It's not clear how far the politics will go, but the shifts so far are
significant. ``Never before in the history of modern polling has the
country expressed such widespread agreement on racism's pervasiveness in
policing, and in society at large,''
The\href{https://www.nytimes3xbfgragh.onion/2020/06/05/us/politics/polling-george-floyd-protests-racism.html}{Times
reported last week}.

More important, we are no longer just talking about imposing new limits
on how the police can operate. We're finally asking more substantive
political questions: What roles should be reserved for the police in our
cities, and what roles would better be served by hiring more teachers,
social workers or mental health experts?

In Los Angeles, where leaders on the left and the right have long
showered resources on the police,
\href{https://www.latimes.com/california/story/2020-06-05/eric-garcetti-lapd-budget-cuts-10000-officers-protests}{the
mayor has now proposed} spending \$250 million more on social services
and \$150 million less on policing. Last week, New York's mayor, Bill de
Blasio, resisted cutting the \$6 billion police budget; on Sunday, he
\href{https://www.nytimes3xbfgragh.onion/2020/06/07/nyregion/deblasio-nypd-funding.html}{promised
future cuts}. And in Minneapolis, a veto-proof majority of City Council
members pledged to dismantle the city's police department.

The proximate cause of the latest protests was the horror of George
Floyd's death. But we've seen videos of cops killing black men before
and they have rarely led to criminal prosecution, let alone broad
societal upheaval.

What's happening now is about more than that video. Just as, after the
Weinstein story broke, when women came forward with stories too numerous
to ignore or dismiss, what we've seen in the last two weeks are episodes
of excessive force too blatant and numerous to conclude that the problem
is one of a few isolated cases.

The evidence of police brutality has become too widespread even for
elected officials to ignore. They can no longer easily coddle police
unions in exchange for political support; now ignoring police misconduct
will become a political liability, and perhaps something will change.

Alex Vitale, a sociologist and the author of ``The End of Policing,''
which argues for a wholesale dismantling of American policing, told me
that he has high hopes for structural change because organizers had laid
the groundwork for it. ``My reason for optimism is that before
Minneapolis happened, there were already dozens of campaigns to divert
police funding,'' he said. ``So that's why that demand emerged so
quickly --- people were already doing that work.''

Vitale also suggested that the movement can take hold permanently, that
what's happening now has cracked ``the `ideological armor''' of policing
in America.

I think he's right.

Image

Police officers~waiting for members of the December 12th Movement and
other protesters in Brooklyn on June 1.Credit...Anthony Geathers for The
New York Times

\hypertarget{office-hours-with-farhad-manjoo}{%
\subsection{Office Hours With Farhad
Manjoo}\label{office-hours-with-farhad-manjoo}}

\emph{Farhad wants to}
\href{https://www.nytimes3xbfgragh.onion/2019/05/16/opinion/farhad-office-hours.html?module=inline}{\emph{chat
with readers on the phone}}\emph{. If you're interested in talking to a
New York Times columnist about anything that's on your mind, please fill
out this form. Farhad will select a few readers to call.}

\emph{The Times is committed to publishing}
\href{https://www.nytimes3xbfgragh.onion/2019/01/31/opinion/letters/letters-to-editor-new-york-times-women.html}{\emph{a
diversity of letters}} \emph{to the editor. We'd like to hear what you
think about this or any of our articles. Here are some}
\href{https://help.nytimes3xbfgragh.onion/hc/en-us/articles/115014925288-How-to-submit-a-letter-to-the-editor}{\emph{tips}}\emph{.
And here's our email:}
\href{mailto:letters@NYTimes.com}{\emph{letters@NYTimes.com}}\emph{.}

\emph{Follow The New York Times Opinion section on}
\href{https://www.facebookcorewwwi.onion/nytopinion}{\emph{Facebook}}\emph{,}
\href{http://twitter.com/NYTOpinion}{\emph{Twitter (@NYTopinion)}}
\emph{and}
\href{https://www.instagram.com/nytopinion/}{\emph{Instagram}}\emph{.}

Advertisement

\protect\hyperlink{after-bottom}{Continue reading the main story}

\hypertarget{site-index}{%
\subsection{Site Index}\label{site-index}}

\hypertarget{site-information-navigation}{%
\subsection{Site Information
Navigation}\label{site-information-navigation}}

\begin{itemize}
\tightlist
\item
  \href{https://help.nytimes3xbfgragh.onion/hc/en-us/articles/115014792127-Copyright-notice}{©~2020~The
  New York Times Company}
\end{itemize}

\begin{itemize}
\tightlist
\item
  \href{https://www.nytco.com/}{NYTCo}
\item
  \href{https://help.nytimes3xbfgragh.onion/hc/en-us/articles/115015385887-Contact-Us}{Contact
  Us}
\item
  \href{https://www.nytco.com/careers/}{Work with us}
\item
  \href{https://nytmediakit.com/}{Advertise}
\item
  \href{http://www.tbrandstudio.com/}{T Brand Studio}
\item
  \href{https://www.nytimes3xbfgragh.onion/privacy/cookie-policy\#how-do-i-manage-trackers}{Your
  Ad Choices}
\item
  \href{https://www.nytimes3xbfgragh.onion/privacy}{Privacy}
\item
  \href{https://help.nytimes3xbfgragh.onion/hc/en-us/articles/115014893428-Terms-of-service}{Terms
  of Service}
\item
  \href{https://help.nytimes3xbfgragh.onion/hc/en-us/articles/115014893968-Terms-of-sale}{Terms
  of Sale}
\item
  \href{https://spiderbites.nytimes3xbfgragh.onion}{Site Map}
\item
  \href{https://help.nytimes3xbfgragh.onion/hc/en-us}{Help}
\item
  \href{https://www.nytimes3xbfgragh.onion/subscription?campaignId=37WXW}{Subscriptions}
\end{itemize}
