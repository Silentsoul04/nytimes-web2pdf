Sections

SEARCH

\protect\hyperlink{site-content}{Skip to
content}\protect\hyperlink{site-index}{Skip to site index}

\href{/section/business}{Business}\textbar{}One Bar. Twelve Weeks.
Seventeen Lives in Lockdown.

\url{https://nyti.ms/3hhFUy3}

\begin{itemize}
\item
\item
\item
\item
\item
\item
\end{itemize}

\href{https://www.nytimes3xbfgragh.onion/news-event/coronavirus?action=click\&pgtype=Article\&state=default\&region=TOP_BANNER\&context=storylines_menu}{The
Coronavirus Outbreak}

\begin{itemize}
\tightlist
\item
  live\href{https://www.nytimes3xbfgragh.onion/2020/08/03/world/coronavirus-covid-19.html?action=click\&pgtype=Article\&state=default\&region=TOP_BANNER\&context=storylines_menu}{Latest
  Updates}
\item
  \href{https://www.nytimes3xbfgragh.onion/interactive/2020/us/coronavirus-us-cases.html?action=click\&pgtype=Article\&state=default\&region=TOP_BANNER\&context=storylines_menu}{Maps
  and Cases}
\item
  \href{https://www.nytimes3xbfgragh.onion/interactive/2020/science/coronavirus-vaccine-tracker.html?action=click\&pgtype=Article\&state=default\&region=TOP_BANNER\&context=storylines_menu}{Vaccine
  Tracker}
\item
  \href{https://www.nytimes3xbfgragh.onion/2020/08/02/us/covid-college-reopening.html?action=click\&pgtype=Article\&state=default\&region=TOP_BANNER\&context=storylines_menu}{College
  Reopening}
\item
  \href{https://www.nytimes3xbfgragh.onion/live/2020/08/03/business/stock-market-today-coronavirus?action=click\&pgtype=Article\&state=default\&region=TOP_BANNER\&context=storylines_menu}{Economy}
\end{itemize}

\includegraphics{https://static01.graylady3jvrrxbe.onion/images/2020/06/14/business/00virus-hatch/merlin_170816781_2dbce79d-2427-4b1b-9094-8e8441b850db-articleLarge.jpg?quality=75\&auto=webp\&disable=upscale}

The Great Read

\hypertarget{one-bar-twelve-weeks-seventeen-lives-in-lockdown}{%
\section{One Bar. Twelve Weeks. Seventeen Lives in
Lockdown.}\label{one-bar-twelve-weeks-seventeen-lives-in-lockdown}}

Following an Oakland tavern and its staff --- from the tattooed
bartender to the undocumented cleaner --- as they weather the economic
and emotional fallout of the coronavirus.

Louwenda Kachingwe, the owner of the Hatch, is known as Pancho. ``The
logical thing would just be to shut it down,'' he said at one point.
``But in this case, it's about the people.''Credit...Jim Wilson/The New
York Times

Supported by

\protect\hyperlink{after-sponsor}{Continue reading the main story}

\href{https://www.nytimes3xbfgragh.onion/by/jack-nicas}{\includegraphics{https://static01.graylady3jvrrxbe.onion/images/2018/11/26/multimedia/author-jack-nicas/author-jack-nicas-thumbLarge.png}}

By \href{https://www.nytimes3xbfgragh.onion/by/jack-nicas}{Jack Nicas}

\begin{itemize}
\item
  June 11, 2020
\item
  \begin{itemize}
  \item
  \item
  \item
  \item
  \item
  \item
  \end{itemize}
\end{itemize}

OAKLAND, Calif. --- Cold beer flowed, soul music played and regulars
lined the redwood bar to order tequila shots and tater tots. No one wore
masks, many hugged, and the staff passed a joint out front.

A scene from a bygone era, it was March 16, the final night before the
nation's first effective coronavirus lockdown. Hours earlier, officials
across the San Francisco Bay Area had ordered most businesses to close
at midnight for at least three weeks. In response, the staff of the
Hatch, a cozy locals' bar in downtown Oakland, opened its doors to bring
people together for one last night of drinks --- and pay.

Robin Easterbrook, the Hatch's tattooed manager, served beers and
whiskey to a municipal-bonds trader and a fried-chicken cook, two stools
apart.

``We're six years running, so hopefully something like this doesn't wipe
us out,'' she said from behind the bar, her dyed orange-and-pink hair
peeking out from a baseball cap. ``It's frustrating, because I don't
have all the answers to give to our team, other than my word that we're
going to do our best to make sure that you get taken care of.''

Behind a curtain, Santos, a 56-year-old Guatemalan immigrant, pressed
burgers to the grill. He and his six children in the Bay Area had all
received word that day that they no longer had jobs. He planned to
return to the three-bedroom house on the outskirts of Oakland that he
shared with 11 family members and stay put. ``I want to respect the
law,'' he said in Spanish. ``But my worry is my rent, food.'' The
Hatch's other cook, Leonardo Garcia, fried fresh tortillas into chips
and packed meat into the freezer for the bar's sudden hibernation.

\includegraphics{https://static01.graylady3jvrrxbe.onion/images/2020/06/14/business/00virus-hatch2/00virus-hatch2-articleLarge.jpg?quality=75\&auto=webp\&disable=upscale}

Shelter-in-place orders in the United States and beyond have forced
millions of businesses to close, some for good. Amid that loss, there
are countless stories of the places that people loved and that made
their communities special --- like my local watering hole, the Hatch.

It's a laid-back melting pot of a bar, where the art is abundant and the
cans of Tecate are \$3. The night before the lockdown, I persuaded the
staff to share their finances and lives over the next three months.

On that Monday night, Kenny Bloom, a bartender, was fairly zen. ``Am I
worried? Of course I'm worried. Is it the end of the world, though?
No,'' he said. ``If society crashes, whatever. We'll rebuild it.''
Antoine Towers, the bouncer, was more fatalistic. He forecast riots.
``Three weeks of not making any money?'' he said, standing outside the
bar. ``People are going to do what they have to do.''

Not all of the bar's 17 employees were there for the impromptu farewell
party. Abel Oleson, another bartender, had served drinks that afternoon
but was now quarantined inside his home, worried about his asthma and
nearly empty bank account. Maria, an undocumented immigrant who cleaned
the Hatch each morning, was already in bed, set to awake around dawn for
a final scrubbing of the floors.

Eventually, the owner entered. Wearing a black beanie over his
dreadlocks, Louwenda Kachingwe, known to everyone as Pancho, looked
relaxed as he greeted customers and employees with a broad smile. Then
he started to do some math. The rent was \$6,200, and two days earlier,
he had agreed to take on the \$2,600 lease next door for a long-planned
sister bar.

``I know,'' he said. ``It's a problem.'' He burst out in nervous
laughter.

``Are we laughing, or are we crying?'' Ms. Easterbrook asked.

``That's the crying laugh,'' he replied.

On a scale of one to 10, Mr. Kachingwe said, his anxiety about the bar's
future was a nine. ``Three weeks is doable. But what I've been told is,
it's probably realistically eight weeks,'' he said. ``If it's eight
weeks, then I'm just like: How does anyone survive that?''

\hypertarget{closed-mouths-dont-get-fed}{%
\subsection{`Closed mouths don't get
fed'}\label{closed-mouths-dont-get-fed}}

Image

Robin Easterbrook, center, and Mr. Kachingwe, right, board up the bar on
March 17.Credit...Jim Wilson/The New York Times

The Hatch used to be a Hawaiian barbecue joint with orange carpets. ``It
was nasty,'' Mr. Kachingwe said. He leased it anyway, tore out the
carpets and built wooden tables and benches with some friends.

The Hatch opened in November 2014, a side-street hideaway just off
Oakland's main drag. It offered a single India pale ale ** and a
bacon-wrapped hot dog. (Both are still on the menu.) Customers drank out
of red Solo cups and sat on chairs borrowed from a restaurant down the
street. Mr. Kachingwe slept on a couch upstairs.

Mr. Kachingwe, 40, was used to risks. His childhood took him from
Chegutu, a village in Zimbabwe without electricity, to Coralville, Iowa,
where his stepfather had gotten a job as a professor when he was 10.
Fifteen years later, he moved to San Francisco and ended up managing a
popular Senegalese spot known for its dance floor. (The owner promoted
him because he didn't drink.) Eventually he thought owning his own place
would give him more time to play guitar and write screenplays (wrong),
so he looked to Oakland.

``The first goal was completely selfish,'' he said. ``And then
immediately, you realize it has nothing to do with you.''

The Hatch grew into a community hub*.* The narrow two-story bar,
jacketed in reclaimed fence boards and local art, became a hangout for
the city's artists, musicians and writers, as well as waiters,
bartenders and baristas. Upstairs were free comedy and rap shows, and
Mr. Kachingwe hung a bedsheet to project obscure movies. The success
helped drive sales from about \$250,000 in its first year to more than
\$700,000 in 2019.

\hypertarget{latest-updates-economy}{%
\section{\texorpdfstring{\href{https://www.nytimes3xbfgragh.onion/live/2020/08/03/business/stock-market-today-coronavirus?action=click\&pgtype=Article\&state=default\&region=MAIN_CONTENT_1\&context=storylines_live_updates}{Latest
Updates:
Economy}}{Latest Updates: Economy}}\label{latest-updates-economy}}

\href{https://www.nytimes3xbfgragh.onion/live/2020/08/03/business/stock-market-today-coronavirus?action=click\&pgtype=Article\&state=default\&region=MAIN_CONTENT_1\&context=storylines_live_updates\#the-chicago-fed-president-says-its-up-to-congress-to-save-the-economy}{11h
ago}

\href{https://www.nytimes3xbfgragh.onion/live/2020/08/03/business/stock-market-today-coronavirus?action=click\&pgtype=Article\&state=default\&region=MAIN_CONTENT_1\&context=storylines_live_updates\#the-chicago-fed-president-says-its-up-to-congress-to-save-the-economy}{The
Chicago Fed president says it's up to Congress to save the economy.}

\href{https://www.nytimes3xbfgragh.onion/live/2020/08/03/business/stock-market-today-coronavirus?action=click\&pgtype=Article\&state=default\&region=MAIN_CONTENT_1\&context=storylines_live_updates\#faa-says-boeing-has-effectively-mitigated-defects-in-the-737-max}{11h
ago}

\href{https://www.nytimes3xbfgragh.onion/live/2020/08/03/business/stock-market-today-coronavirus?action=click\&pgtype=Article\&state=default\&region=MAIN_CONTENT_1\&context=storylines_live_updates\#faa-says-boeing-has-effectively-mitigated-defects-in-the-737-max}{F.A.A.
says Boeing has `effectively mitigated' defects in the 737 Max.}

\href{https://www.nytimes3xbfgragh.onion/live/2020/08/03/business/stock-market-today-coronavirus?action=click\&pgtype=Article\&state=default\&region=MAIN_CONTENT_1\&context=storylines_live_updates\#small-businesses-got-emergency-loans-but-not-what-they-expected}{14h
ago}

\href{https://www.nytimes3xbfgragh.onion/live/2020/08/03/business/stock-market-today-coronavirus?action=click\&pgtype=Article\&state=default\&region=MAIN_CONTENT_1\&context=storylines_live_updates\#small-businesses-got-emergency-loans-but-not-what-they-expected}{Small
businesses got emergency loans, but not what they expected.}

\href{https://www.nytimes3xbfgragh.onion/live/2020/08/03/business/stock-market-today-coronavirus?action=click\&pgtype=Article\&state=default\&region=MAIN_CONTENT_1\&context=storylines_live_updates}{See
more updates}

More live coverage:
\href{https://www.nytimes3xbfgragh.onion/2020/08/03/world/coronavirus-covid-19.html?action=click\&pgtype=Article\&state=default\&region=MAIN_CONTENT_1\&context=storylines_live_updates}{Global}

Now ---~on March 17, a Tuesday --- Mr. Kachingwe and Ms. Easterbrook
were packing up the booze and boarding up the windows. On top of the
\$8,800 in rent, an advertising contract with Yelp was \$1,000 a month.
The point-of-sale system was \$284. Cable and internet, \$180. The alarm
system, \$165. Ms. Easterbrook poured a beer and then remembered that
the six tapped kegs would also soon go flat --- another \$1,200. ``We're
about to find out what we can and can't pay for,'' Mr. Kachingwe said.

Within days, he had a creative plan for survival: Use the Hatch's tiny
kitchen to cater meals for a government operations hub a few miles away.
He had already met with city officials.

``We have a saying that closed mouths don't get fed,'' he said. ``My
main goal is just trying to figure out how we can possibly, one,
survive, and two, get people anything in their pockets.''

A few days later, Mr. Kachingwe said the catering idea was looking
unlikely. But he had a new plan: The Hatch would become a takeout joint.
``We're going to see what it looks like,'' Mr. Kachingwe said. ``Because
I have no clue.''

\hypertarget{bien-gracias-a-dios}{%
\subsection{`Bien, gracias a Dios'}\label{bien-gracias-a-dios}}

Image

Santos lives with 11 family members. He and his six children in the Bay
Area all received word they lost their jobs on the same day.Credit...Jim
Wilson/The New York Times

Santos is the sort of man who blesses you when he meets you. ``Que Dios
te bendiga,'' he says. When you ask him how he is, he often replies,
``Bien, gracias a Dios.'' And when you part ways, he offers another
blessing for the road.

By early April, weeks into joblessness, he was still optimistic. He had
begun taking morning walks in the hills and leading nightly family
prayer circles. ``Almost every day, we are praying that God takes
control,'' said Santos, who asked to be identified only by his first
name because of his family's immigration status.

God had carried him through difficult times before, he said. His first
five years in the United States were spent working 70 hours a week in a
buffet for less than \$5 an hour and a bed in a house crammed with
colleagues. After authorities raided the place for human trafficking, he
lost his job and his home. Eventually he found dishwashing work and
welcomed four of his children to the United States from Guatemala.

In 2016, he began working the grill at the Hatch and settled in there,
making roughly \$2,500 a month. He sent a slice of his pay to his wife
and remaining two children in Guatemala, whom he hasn't seen in 12
years. Last year, he won legal residence. Life had been improving.

Now he wasn't sure if he could pay rent. When I visited him one day in
early April at the three-bedroom house beneath elevated train tracks
that he shared with four children and seven grandchildren, he swung open
the iron security gate, wiped his bleary eyes and smiled to reveal his
missing teeth.

Standing in the driveway, amid scattered children's toys and bags of
empty cans collected from the Hatch, he said the family had just barely
made their \$2,860 rent for March, plus about \$500 more for utilities.
He and his children had applied for jobs at a recycling plant and a
tortilla-chip factory, but were turned away. ``We are OK right now,
assuming this ends in a week or two,'' he said. ``But if this goes on
longer, then it really worries me.''

I asked him if he had any questions. He hesitated. If it wasn't too much
trouble, he said before blessing me, did I know of any place that was
hiring?

\hypertarget{right-now-our-worry-is-financial}{%
\subsection{`Right now our worry is
financial'}\label{right-now-our-worry-is-financial}}

Image

Maria on her porch. She lost her job as the Hatch's cleaner just as her
back pain got worse.Credit...Jim Wilson/The New York Times

A few miles away, Maria, the Hatch's cleaner, was having a more
difficult start. Her nagging back pain had sharpened, and at times, she
couldn't walk. She started paying for \$100 chiropractor sessions, a
service not covered by her health insurance, cutting sharply into her
slim savings. (We agreed to publish only her first name because of her
immigration status.)

Like Santos, Maria and her husband have scraped by since they arrived in
the Bay Area from Coeneo, a small town in Mexico, in 1998. Her husband
labored in the smoky kitchens of cramped East Bay restaurants, working
under fake \$20 documents, while she collected bottles and cans and
raised their children.

Since 2016, Maria, 55, has taken the 7 a.m. bus to the Hatch to clean,
earning sometimes \$400 a week, or about \$20,000 a year. (Her husband
sometimes helped her, but was otherwise unemployed.) The family could
hardly afford life before the virus. Their share of the monthly rent
costs \$1,000; Maria's stepdaughter, who works at a Toyota dealership,
has helped keep them afloat.

``Right now our worry is financial. To pay the rent, to pay the bills,
to buy food,'' Maria said in Spanish. ``Because the diseases themselves,
thank God, haven't touched us yet.''

Image

Abel Oleson, a bartender at the Hatch, was worried about his asthma and
empty bank account as the lockdown began.Credit...Jim Wilson/The New
York Times

Mr. Oleson, the bartender, was also struggling. After a grocery run and
a \$270 late phone bill, he said he had just \$20 to his name and plenty
more in debt. His girlfriend had left her job at a marijuana dispensary,
in part to protect Mr. Oleson, who worried his asthma put him at
particular risk of the virus. ``I'm completely broke,'' he said, adding
an expletive. ``This couldn't have come at a worse time.''

Mr. Oleson, 34 and from Portland, Ore., is the kind of hipster who plays
1970s soul vinyls during his DJ sets but wears 1980s metal shirts around
town. When we met, he had a bushy beard beneath his mask and lots of
tattoos he couldn't explain. ``No story,'' he said, when I asked him
about a goat head spitting fire on his triceps.

He took home roughly \$2,000 a month at the Hatch and spent about half
on rent and bills. When the lockdown began, he immediately filed for
unemployment and food stamps. Mr. Oleson said he also felt lucky to have
another safety net: his sister. She could lend him more money if needed;
she was already doing his laundry and had sewn him a mask. ``By May,
hopefully it'll swing back,'' he said, ``and I'll only be a couple
thousand in debt.''

\hypertarget{the-logical-thing-would-just-be-to-shut-it-down}{%
\subsection{`The logical thing would just be to shut it
down'}\label{the-logical-thing-would-just-be-to-shut-it-down}}

Image

Starting an experiment: Takeout.Credit...Jim Wilson/The New York Times

Takeout, it turns out, is not so easy. To start, the Hatch didn't have a
phone, so Mr. Kachingwe posted his cell on the website. He and Ms.
Easterbrook had to scour stores to find takeout containers, beer
growlers and Mason jars for to-go cocktails. Their point-of-sale system
couldn't handle internet orders, so they contracted with a new company,
which required building an online menu. One weekday in late March, I
walked into the bar to find Ms. Easterbrook hunched over her laptop,
painstakingly adding custom sauce options for all the varieties of
chicken wings.

Yelp, the online-review site, locked them out because of an overdue
bill, so they couldn't alert prospective customers. The refrigerator
broke. And finally, there was what to do about
\href{https://www.nytimes3xbfgragh.onion/2020/06/09/technology/delivery-apps-restaurants-fees-virus.html}{the
delivery apps}.

UberEats, Grubhub, DoorDash, Postmates and Caviar have long been
operating what many restaurant owners describe as parasitic businesses.
According to the restaurant industry, they intercept customers and then
take roughly 30 percent of the sale for facilitating the online order
and delivery. The apps says they provide restaurants with customers they
wouldn't otherwise have.

Mr. Kachingwe considered the apps' 30 percent cut and decided to go it
alone, calling it ``a moral stand.'' He bet that the Hatch could
generate enough business with Instagram posts, fliers in nearby condo
buildings and the loyalty of its regulars. He would run the deliveries
himself.

On March 28, two weeks after the lockdown began, the Hatch opened again
for business. Ms. Easterbrook manned the empty bar to make Mason-jar
margaritas and pack orders while Santos and Mr. Garcia split days on the
grill.

In the first week, the Hatch received nine orders for \$369. I was one
of them.

I asked Mr. Kachingwe how he felt about his decision. ``Honestly, pretty
good,'' he said. He knew business would be slow at first. ``Next week
should be better. Right?''

Image

The Hatch tried to avoid working with delivery apps by promoting its
takeout service on Instagram. It didn't work. (And no, that's not a real
mustache.)

Five days went by, and I watched the Hatch's Instagram account post
increasingly desperate pitches. There were mentions of free delivery,
Taco Tuesdays,
\href{https://www.nytimes3xbfgragh.onion/2020/04/02/arts/television/tiger-king-style.html}{Tiger
King} tie-ins and eventually just a shot of Ms. Easterbrook waiting,
bored, by the phone.

I called Mr. Kachingwe back. He was signing up with Grubhub. Ms.
Easterbrook had canvassed other restaurants and convinced him. ``She's
like, `Everybody's in the same boat. They hate it. They don't want to do
it. But what is the choice?''' Mr. Kachingwe recounted. He asked me to
hold; he was getting a drive-through coronavirus test. (Negative.)

Grubhub quickly caused sales to climb, but it wasn't enough to cover
overhead and payroll. The Hatch brought in about \$3,250 in April and
\$1,500 in May. Before the virus, the bar averaged roughly \$70,000 a
month. By May, Mr. Kachingwe had burned through roughly \$20,000 in
emergency funds and another \$20,000 of his personal money. ``The
logical thing would just be to shut it down,'' he said. ``But in this
case, it's about the people.''

\hypertarget{i-couldnt-stand-the-pain-anymore}{%
\subsection{`I couldn't stand the pain
anymore'}\label{i-couldnt-stand-the-pain-anymore}}

When I met Maria at her Oakland home on May 1, her husband pushed her to
the door in a chair. She could no longer afford the chiropractor, and
now she couldn't walk. When she rose for a moment, her face twisted. A
week earlier, she had gone to the emergency room. ``I couldn't stand the
pain anymore,'' she said in Spanish. ``It grabs me and, for a few
seconds, won't even let me move.'' The doctor gave her pain pills, and
ordered more tests. While she talked, her 18-year-old daughter sat at
her feet, looking at the floor.

Weeks earlier, on the phone, Maria had broken into tears telling me
that, to pay rent, she spent the \$800 she had been saving for her
daughter's high-school graduation gift. After that, she stopped
answering my calls and texts for a while. Eventually she got back in
touch and explained her silence: She had stopped paying the phone bill.

On that call, I asked her if she had enough to eat. Her voice choked.
``I'm sorry,'' she said. ``It's very, very difficult.''

The situation was also getting dire at Santos's home. He had returned to
the Hatch, but was only notching 15 hours a week at most. The roughly
\$225 each week barely helped the family make April rent, but now May
was uncertain.

His children kept applying for jobs but hadn't gotten callbacks. He
stopped paying his car insurance. They cut back on meat and began going
to a nearby food bank. Their meals were simple, like eggs and beans.
Someone from his grandchildren's school called and wondered why the
children weren't attending virtual classes. That answer was easy, Santos
said. The family doesn't have a computer.

They continued to pray each night. ``Humanity has reached an extreme
that has offended God. So we ask for mercy,'' he recited to me.

Image

Santos prepares onion rings. The takeout business brought him back to
work, but only for 15 hours a week. He was still unsure he could make
rent.Credit...Jim Wilson/The New York Times

A month earlier, the federal government had approved \$2.2 trillion in
stimulus funds, but Santos and Maria were not caught by the safety net.
Santos and his children were ineligible for unemployment and stimulus
payments because of their immigration and tax statuses. Maria, as an
undocumented immigrant, also did not qualify.

Maria lamented that she paid taxes like legal residents but had few of
their rights. President Trump ``describes us as thieves but that isn't
the case,'' she said. ``We are working people. We are people who want to
provide for our family.''

California officials were offering \$500 payments for undocumented
immigrants, but to Maria, they would simply complicate her effort to
become a citizen. ``It's not worth the trouble,'' she said. ``Tell Trump
if you see him that we're starving, and he is at his heart's content at
home.''

Mr. Oleson, on the other hand, had landed softly. On April 7, three
weeks after the lockdown began, he received his first unemployment
check, at \$421 a week. ``That'll at least help me pay my rent,'' he
said. A week later, the federal stimulus funds arrived; he received the
\$1,200 lump sum, and his unemployment checks started including an extra
\$600 a week. ``I'm ecstatic,'' he said. ``It's nice to see that shit
actually works.''

He was now making more than \$1,000 a week, roughly double his pay at
the Hatch. He planned to replace his broken laptop and pay off debts,
though he admitted he was eyeing a new TV, too. ``I want to be
relatively cautious and not just like go nuts right now,'' he said,
``which I'm very tempted to do.''

\hypertarget{the-clock-is-ticking}{%
\subsection{`The clock is ticking'}\label{the-clock-is-ticking}}

Image

Ms. Easterbrook staging food for a photo shoot for the Hatch's new
website.Credit...Jim Wilson/The New York Times

More than a month after the Hatch first closed its door, Mr. Kachingwe's
catering idea had fallen apart, the takeout business was bleeding money,
and local officials were hinting that the bar wouldn't be welcoming back
customers anytime soon.

New rules for bars were expected to require six feet between most
customers. At the Hatch, there is not much more than six feet from the
bar to the back wall. ``With that, we can't operate,'' Mr. Kachingwe
said. ``That's not even realistic.''

But Mr. Kachingwe was focusing his attention on a possible lifeline: The
stimulus package. Since the federal government announced \$349 billion
in forgivable loans for small businesses, Mr. Kachingwe had been
researching how to apply. ``We've got to find a way to make this work
because people are depending on us, not just for financial stability,
but emotional health,'' he said. He knew his employees' rent and food
bills were piling up. ``The clock is ticking,'' he said.

He applied with JPMorgan Chase as soon as he could, starting a
maddening, weeklong back-and-forth that resulted in his first
application not going through and his second getting rejected because it
was too late. Mr. Kachingwe was never able to get a person on the phone,
while the banks'
\href{https://www.nytimes3xbfgragh.onion/2020/04/22/business/sba-loans-ppp-coronavirus.html}{richest
clients received white-glove service}.

He rushed to apply with another lender, this time a start-up called
Lendio. ``If we got it, then we would be OK,'' he said. ``But without
it, then it just falls on me to finance everything.''

Then he saw the headlines: The federal money had run out. Later that
day, his typically bright outlook had darkened. ``I'm doing terrible,''
he said. ``It's such a mess.''

Several hours later, he sent me a text showing an email from Lendio. The
email began by noting that the federal funds had been exhausted. Then it
went on. ``Good news! Your application has been accepted,'' it said.
``This means funds have been reserved for you.'' He sent me crying
emojis.

Image

``If we got it, then we would be OK,'' Mr. Kachingwe said of stimulus
funds. ``But without it, then it just falls on me to finance
everything.''Credit...Jim Wilson/The New York Times

On May 15, four weeks later, I peered through the Hatch's window as I
waited for Mr. Kachingwe. The bar that two months earlier had buzzed
with people was now cluttered with power cables, kitchen gear and other
oddities, like a disco ball, a motorcycle fender and a broken fence
board on which someone had scrawled, ``Your girl smells like cat food.''

Mr. Kachingwe double parked his Prius (his other ride is a motorcycle)
and let me inside, his dreadlocks held back by a gray ninja headband.
``We got the money,'' he said as he sat down on a stool, ``and it's a
nightmare.''

To make the \$72,500 loan forgivable, he had to spend it all in eight
weeks, including three-quarters of it on payroll. Yet he had few workers
to pay. Some employees, like Mr. Oleson, were loath to lose their
unemployment checks. And Mr. Kachingwe wasn't sure if he could pay
himself. ``I can't get anyone to take this money,'' he said.

That wasn't entirely true; some of his workers would have been happy to
take a fat paycheck. In reality, Mr. Kachingwe was worried about handing
out the government's money and then being on the hook. If he didn't
spend the funds precisely to the government's specification --- and have
the receipts to prove it --- he would have a new pile of debt on top of
his failing business.

``It's one thing to spend my personal money, but it's another to spend
\$73,000 of the government's money and have to pay it back,'' he said.
``This can ruin people.''

\hypertarget{we-can-say-we-did-everything-we-could}{%
\subsection{`We can say we did everything we
could'}\label{we-can-say-we-did-everything-we-could}}

When I called Maria after Memorial Day, her husband answered. ``She will
call you back,'' he told me in Spanish. ``We're at the hospital.'' After
she returned home a week later, she shared the bad news: Doctors had
found cancer in her hip. She was getting more tests and was unsure how
much treatment her insurance would cover. She had decided to take one of
the \$500 checks for undocumented California residents but, after dozens
of calls, she couldn't get through on the government hotline. ``It's
always busy,'' she said.

Santos had become difficult to reach, and when I did, he was stressed.
His doctor was urging him to undergo surgery on a painful hernia in his
abdomen, and his family didn't pay the rent due on June 1. ``I can't say
what's going to happen,'' he said.

Even Mr. Oleson, relatively flush, was struggling. He has long suffered
from depression, and the money hadn't rid him of anxiety. ``It's a very
easy time to get depressed,'' he said. Then, over Memorial Day weekend,
he learned that his mother had colon cancer. He wanted to fly to New
Mexico to see her, but was nervous about the virus. ``Life still goes on
in the pandemic,'' he said. ``I guess no other traumas stop.'' (Days
later, I spotted him hoisting a sign at a downtown protest.)

Finally, Mr. Kachingwe told me that someone had broken into the Hatch
and stolen \$19,000 worth of stuff, including laptops, liquor and a
bike. ``It happens,'' he said. But at least the Hatch was spared any
damage from looters during the first nights of demonstrations against
racism and police brutality. Mr. Kachingwe had put up a sign identifying
it as a black-owned business. The Money Mart across the street was
destroyed. (On Sunday, Mr. Kachingwe and Ms. Easterbrook helped organize
neighbors
\href{https://www.youtube.com/watch?v=Os8jt3AFWz4\&feature=youtu.be}{to
paint the words ``Black Lives Matter'' on the pavement} outside the bar
and down three city blocks.)

Image

Painting ``Black Lives Matter'' on the street outside the
Hatch.Credit...Jim Wilson/The New York Times

Mr. Kachingwe had a final plan to save the business. The planned sister
bar next door would become a flower shop that also sold beers to go.
(Ms. Easterbrook is a trained florist.) As for the Hatch,
\href{https://covid19.ca.gov/pdf/guidance-restaurants-bars.pdf}{new
rules} effectively blocked it from opening to customers this summer, so
Mr. Kachingwe built a takeout window into the kitchen, found an
ice-cream supplier and bought four slushie machines. Then he secured a
verbal commitment from the city to close the street to most vehicles,
allowing him to move tables outside for customers. The plan seemed
firmer than his previous ideas.

``At the very least, we can say that we did everything that we could,''
said Mr. Kachingwe, who decided last week that he would use the federal
loan.

``At some point in time, the Hatch would have had to close,'' he added.
``The fact that people had a chance to relax here and escape from
whatever problems they were having --- to me, that's a victory.''

Reporting was contributed by Kirla Oyola-Seal and Sonia Duarte Nicas.

Advertisement

\protect\hyperlink{after-bottom}{Continue reading the main story}

\hypertarget{site-index}{%
\subsection{Site Index}\label{site-index}}

\hypertarget{site-information-navigation}{%
\subsection{Site Information
Navigation}\label{site-information-navigation}}

\begin{itemize}
\tightlist
\item
  \href{https://help.nytimes3xbfgragh.onion/hc/en-us/articles/115014792127-Copyright-notice}{©~2020~The
  New York Times Company}
\end{itemize}

\begin{itemize}
\tightlist
\item
  \href{https://www.nytco.com/}{NYTCo}
\item
  \href{https://help.nytimes3xbfgragh.onion/hc/en-us/articles/115015385887-Contact-Us}{Contact
  Us}
\item
  \href{https://www.nytco.com/careers/}{Work with us}
\item
  \href{https://nytmediakit.com/}{Advertise}
\item
  \href{http://www.tbrandstudio.com/}{T Brand Studio}
\item
  \href{https://www.nytimes3xbfgragh.onion/privacy/cookie-policy\#how-do-i-manage-trackers}{Your
  Ad Choices}
\item
  \href{https://www.nytimes3xbfgragh.onion/privacy}{Privacy}
\item
  \href{https://help.nytimes3xbfgragh.onion/hc/en-us/articles/115014893428-Terms-of-service}{Terms
  of Service}
\item
  \href{https://help.nytimes3xbfgragh.onion/hc/en-us/articles/115014893968-Terms-of-sale}{Terms
  of Sale}
\item
  \href{https://spiderbites.nytimes3xbfgragh.onion}{Site Map}
\item
  \href{https://help.nytimes3xbfgragh.onion/hc/en-us}{Help}
\item
  \href{https://www.nytimes3xbfgragh.onion/subscription?campaignId=37WXW}{Subscriptions}
\end{itemize}
