Sections

SEARCH

\protect\hyperlink{site-content}{Skip to
content}\protect\hyperlink{site-index}{Skip to site index}

\href{https://www.nytimes3xbfgragh.onion/podcasts/the-daily}{The Daily}

\href{https://myaccount.nytimes3xbfgragh.onion/auth/login?response_type=cookie\&client_id=vi}{}

\href{https://www.nytimes3xbfgragh.onion/section/todayspaper}{Today's
Paper}

\href{/podcasts/the-daily}{The Daily}\textbar{}Georgia's Election
Meltdown

\begin{itemize}
\item
\item
\item
\item
\item
\item
\end{itemize}

\begin{itemize}
\item
  \href{https://www.nytimes3xbfgragh.onion/2020/08/04/us/elections/primary-election-michigan-arizona-kansas.html?action=click\&pgtype=Article\&state=default\&region=TOP_BANNER\&context=storylines_menu}{Election
  Updates}
\item
  \href{https://www.nytimes3xbfgragh.onion/article/biden-vice-president-2020.html?action=click\&pgtype=Article\&state=default\&region=TOP_BANNER\&context=storylines_menu}{Biden's
  V.P. Search}
\item
  \href{https://www.nytimes3xbfgragh.onion/interactive/2020/07/24/us/politics/trump-biden-campaign-donors.html?action=click\&pgtype=Article\&state=default\&region=TOP_BANNER\&context=storylines_menu}{Map
  of Donations}
\item
  \href{https://www.nytimes3xbfgragh.onion/interactive/2020/us/elections/delegate-count-primary-results.html?action=click\&pgtype=Article\&state=default\&region=TOP_BANNER\&context=storylines_menu}{Delegate
  Count}
\item
  \href{https://www.nytimes3xbfgragh.onion/interactive/2019/us/politics/2020-presidential-candidates.html?action=click\&pgtype=Article\&state=default\&region=TOP_BANNER\&context=storylines_menu}{The
  Candidates}
\item
  \href{https://www.nytimes3xbfgragh.onion/newsletters/politics?action=click\&pgtype=Article\&state=default\&region=TOP_BANNER\&context=storylines_menu}{Politics
  Newsletter}
\end{itemize}

Advertisement

\protect\hyperlink{after-top}{Continue reading the main story}

transcript

Back to The Daily

bars

0:00/26:27

-26:27

transcript

\hypertarget{georgias-election-meltdown}{%
\subsection{Georgia's Election
Meltdown}\label{georgias-election-meltdown}}

\hypertarget{hosted-by-michael-barbaro-and-caitlin-dickerson-produced-by-eric-krupke-alexandra-leigh-young-robert-jimison-and-stella-tan-with-help-from-sydney-harper-and-edited-by-mj-davis-lin}{%
\subsubsection{Hosted by Michael Barbaro and Caitlin Dickerson; produced
by Eric Krupke, Alexandra Leigh Young, Robert Jimison and Stella Tan;
with help from Sydney Harper; and edited by M.J. Davis
Lin}\label{hosted-by-michael-barbaro-and-caitlin-dickerson-produced-by-eric-krupke-alexandra-leigh-young-robert-jimison-and-stella-tan-with-help-from-sydney-harper-and-edited-by-mj-davis-lin}}

\hypertarget{why-the-states-troubled-primary-elections-this-week-may-be-a-preview-of-graver-battles-coming-in-the-general-election}{%
\paragraph{Why the state's troubled primary elections this week may be a
preview of graver battles coming in the general
election.}\label{why-the-states-troubled-primary-elections-this-week-may-be-a-preview-of-graver-battles-coming-in-the-general-election}}

Thursday, June 11th, 2020

\begin{itemize}
\item
  caitlin dickerson\\
  From The New York Times, I'm Caitlin Dickerson. This is ``The Daily.''
\item
  {[}music{]}\\
  Today: A full-scale meltdown of new voting systems in Georgia is
  alarming Democratic leaders ahead of the state's general election in
  November. My colleague, Astead Herndon, on why voting access in
  Georgia has become a national issue for the party. It's Thursday, June
  11.

  OK, Astead, so tell me what happened on Tuesday in Georgia.
\item
  astead herndon\\
  Tuesday was Georgia's primary elections, where they were slated to
  send Senate candidates and House candidates ahead to November's
  general election, but ---
\item
  archived recording\\
  After twice being delayed due to the coronavirus pandemic, finally
  primary election day, and some of the polls, simply did not go as
  planned.
\end{itemize}

astead herndon

What we saw on Tuesday did not look much like an election at all.

\begin{itemize}
\tightlist
\item
  archived recording\\
  Our newsroom is flooded with emails, calls, tweets, texts of voters
  reporting issues and irregularities at precincts across the metro.
\end{itemize}

astead herndon

At the beginning of the day, polling sites were not opening on time, and
then it became very clear that they weren't adequately staffed.

\begin{itemize}
\tightlist
\item
  archived recording\\
  Health concerns kept many longtime poll workers from showing up today,
  leaving inexperienced volunteers to run new voting machines for the
  first time.
\end{itemize}

astead herndon

Also, there were problems with the machines that were at the polling
sites.

\begin{itemize}
\item
  archived recording 1\\
  Poll workers said they had difficulties turning on the voter check-in
  computers, and encoding voter access cards, and installing touch
  screens.
\item
  archived recording 2\\
  They had printer problems, missing some electrical plugs, as well.
\end{itemize}

astead herndon

There are also fewer polling places to begin with because of the
coronavirus pandemic, so the virus has added more emphasis on mail-in
ballots and absentee ballots, many of which some Georgia residents said
they did not receive in the mail. And this created massive lines ---

\begin{itemize}
\tightlist
\item
  archived recording\\
  We saw repeated over and over people standing, sitting, waiting for
  the opportunity to have their say in our state's political future.
\end{itemize}

astead herndon

--- causing people to wait more than four or five hours in some cases.

\begin{itemize}
\item
  archived recording 1\\
  53 years I've been voting, and never have seen a line like this in 53
  years.
\item
  archived recording 2\\
  This is wrong. This is America. This is a crisis in our world to make
  us not exercise our right to vote.
\end{itemize}

astead herndon

It also caused some people to turn away, just throwing up their hands
and saying, you know, they can't spend a whole day waiting for a line
that they don't know is going to move.

\begin{itemize}
\tightlist
\item
  archived recording\\
  The system is a joke, and we're not laughing.
\end{itemize}

caitlin dickerson

So why was this happening, Astead? What's the reason for all this chaos?

astead herndon

Tuesday was a confluence of local and state problems. And what you hear
from the counties that were particularly affected was that certainly,
their machines and their processes did not work, and they take some
blame for that. But what Democrats say is a larger problem is a state
and Republican administration system that runs the elections process
that is not interested in helping these counties succeed.

caitlin dickerson

What I hear you saying is that what happened on Tuesday was not simply a
fluke.

astead herndon

Right. The roots of Georgia's fights over ballot access and voting
rights start way before Tuesday.

\begin{itemize}
\tightlist
\item
  archived recording\\
  David, what has been the reaction there in Washington to the Supreme
  Court effectively hobbling the Voting Rights Act?
\end{itemize}

astead herndon

In 2013, the Supreme Court opened the door for states to have more
autonomy in changing their voting procedures without input from the
federal government.

\begin{itemize}
\tightlist
\item
  archived recording (david leonhardt)\\
  You see Democrats very upset about this rule, And you see Republicans
  who have come out so far praising it, saying the Voting Rights Act has
  done its work. It may not be needed anymore.
\end{itemize}

astead herndon

And that allowed states like Georgia, states that had historically been
closely watched in the South, to really overhaul their ballot process.
This has included closing polling locations across the state that have
predominantly been in Democratic and African-American communities. And
also, they passed in 2017 what's called the Exact Match Law, which means
when someone registers for the ballot, if there is any difference
between that registration and the identification the state has on file
--- whether that is a misplaced letter or an incorrect hyphen --- it
allows the state to throw out that ballot registration. That has led to
thousands of people being purged from Georgia's voting rolls. And both
of these things, closing the polling locations and the Exact Match Law,
have disproportionately impacted minority communities, and black
communities especially. I remember in 2018 ---

\begin{itemize}
\tightlist
\item
  archived recording\\
  In Georgia, a record-breaking two million early votes were cast, and
  all eyes are focused on the state's race for governor.
\end{itemize}

astead herndon

--- being in Georgia for the closely-watched governor's race between
Stacey Abrams and Brian Kemp.

\begin{itemize}
\tightlist
\item
  archived recording\\
  A poll released today shows that Georgia's secretary of state and
  Republican candidate Brian Kemp leads the Democratic candidate, Stacey
  Abrams, by just one point.
\end{itemize}

astead herndon

And you would be at people's homes, and you would watch them look up
whether their voting registration was still on file. And many would be
shocked to find out that they had been purged even without their
knowledge. And this came in the middle of a governor's race that was
just as much about voting rights as it was about Democrat versus
Republican.

\begin{itemize}
\tightlist
\item
  archived recording (stacey abrams)\\
  I'm Stacey Abrams, and I'm running for governor, because where you
  come from shouldn't determine how far you can go.
\end{itemize}

astead herndon

Stacey Abrams, who had previously been the House Minority Leader in
Georgia, had built a career off of registering new voters, bringing
people --- new people into the process, and kind of a vision of a blue
Georgia on the backs of a multiracial coalition that had yet to be
achieved.

\begin{itemize}
\item
  archived recording (stacey abrams)\\
  The blue wave is African-American. {[}CHEERING{]}

  It's white. It's Latino. It's Asian Pacific Islander.
\item
  archived recording (crowd)\\
  Yes!
\item
  archived recording (stacey abrams)\\
  It is made up of those who've been told that they are not worthy of
  being here.
\item
  archived recording (crowd)\\
  Yes!
\item
  archived recording (stacey abrams)\\
  It is comprised of those who are documented and undocumented.
\item
  archived recording (crowd)\\
  Yes!
\end{itemize}

astead herndon

And she was facing the secretary of state, Brian Kemp ---

\begin{itemize}
\tightlist
\item
  archived recording (brian kemp)\\
  Well, thankfully, the truth here is very simple. Georgians should
  simply watch what she says. You'll know that she's talking about this
  election, and talking about illegals voting for her in this election.
  They filed a lawsuit.
\end{itemize}

astead herndon

--- who had refused to recuse himself from overseeing the state's
election, even as he ran. And this race was wrapped up in accusations of
voter suppression.

\begin{itemize}
\tightlist
\item
  archived recording (stacey abrams)\\
  My worry is that he's using his position as secretary of state to tilt
  the playing field in his direction.
\end{itemize}

astead herndon

And from Republicans about voter fraud.

\begin{itemize}
\tightlist
\item
  archived recording (brian kemp)\\
  I think hardworking Georgians should decide who their governor is, not
  people here illegally like my opponent wants.
\end{itemize}

astead herndon

There was a real sense that whoever won this would be determining the
direction, and most importantly, would be the referee for the state's
elections going forward.

And on election day ---

\begin{itemize}
\item
  archived recording 1\\
  Good morning. There is no lull in this line, and you can see people
  lined up here.
\item
  archived recording 2\\
  The worst of the issues was in Fulton County. At the Pittman Park
  location, only three voting machines were sent, but eight were
  supposed to be there.
\item
  archived recording 3\\
  I live in East Point, and I updated my address at least two times
  before election day. And on Tuesday when I went to my polling place,
  they denied me a ballot.
\end{itemize}

astead herndon

Stacey Abrams lost by a little less than 55,000 votes. And when she
lost, accused Republicans of voter suppression tactics that changed the
outcome of the race.

\begin{itemize}
\tightlist
\item
  archived recording (stacey abrams)\\
  Democracy only works when we work for it, and apparently today, when
  we stand in line for hours to meet it at the ballot box, that's when
  democracy works.
\end{itemize}

astead herndon

For a while, Abrams wouldn't concede to Kemp.

\begin{itemize}
\tightlist
\item
  archived recording (stacey abrams)\\
  Friends, friends, we are still on the verge of history, and the best
  is yet to come. {[}CHEERING{]}
\end{itemize}

caitlin dickerson

Astead, how do Republicans respond to these allegations from Democrats
that the prior election was unfair?

astead herndon

On the defensive side, Republicans say that there is not evidence that
they are proactively trying to suppress votes. They flip the blame,
saying that it is local Democratic officials in these areas who have not
lived up to their task in administrating clean elections. They also say
that they are focused on things like voter fraud, which we should note
does not have real evidence. And they justify things like exact match as
a tool to combat this voter fraud. But they've also done offensive
moves. The state purchased new voting machines after criticism that the
previous ones were not safe, and a court ordered to do so, and those
were used for the first time in Tuesday's election.

caitlin dickerson

And based on what you saw Tuesday, those measures to address problems in
the electoral system, they don't seem to have worked.

astead herndon

Certainly, those measures do not meet the scope of the crisis. So for
whatever new voting machines, or for whatever back and forth this
happening between county and state officials, what is clear is that
voting in Georgia does not go the way voting should be. But for
Republicans, they'll say that the vast majority of Georgia's county's,
150 out of 159, had fine days on Tuesday. But it's important to
recognize that those nine counties that had the biggest issues on
Tuesday, they're not only Democratic areas, but those are the counties
that have the largest minority populations in the state.

caitlin dickerson

So from everything you've said, it sounds like Democrats would see the
problems with Tuesday's election as being just a continuation of voting
issues that have plagued these same communities in the past, and that
they feel Republicans have either ignored or even made worse.

astead herndon

That's what Democrats will tell you. I remember running into the state
Democratic chair when she was trying to vote. It took her five hours on
what was her 10-year wedding anniversary. And she was talking about how
familiar it felt, and encouraging people in the line to hold that
feeling with them as they look towards the general election. But when
you look at the reaction across the country and how much interest there
was in what was happening in Georgia, I think a part of that is because
the national Democratic and Republican parties realize just how
important this state is --- not just for November, but what could be a
preview of how Southern politics is changing in the future.

{[}music{]}

caitlin dickerson

We'll be right back.

Astead, what do you mean? Why is Georgia such a key state for the
Democratic party?

astead herndon

For decades now, Democrats have been virtually shut out of the South. It
has been almost impossible for the party to find consistent success in
getting a candidate elected to statewide office, whether that's a
governor or the Senate, and in presidential elections. And what
Democrats have been trying to do over the past decade is create a
grassroots momentum that can change the way that they operate in the
South. And Georgia has been the focal point of that.

caitlin dickerson

Astead, help me understand the Democrats' strategy in Georgia.

astead herndon

It basically breaks down to three areas.

The first is just the changing demographics of the state. New
industries, particularly movie and film, have caused an influx of a new
Southerner, as some folks called it, who is living in places like
Atlanta and the metro areas, that has made the South their home in the
way that has given Democrats a new type of voter to target. Another key
point of the strategy is in registering Georgians who may not have
participated in previous elections. So that includes predominately young
people and people of color, and going to those communities that have
kind of felt distant from the political process and bringing them along
and involved. The third piece, which has been accelerated in the last
three years, has been trying to persuade a white, often college-educated
voter, who probably had voted Republican before, that Democrats are now
more acceptable party. And this is something that Democrats have said
Donald Trump is their best recruiter for. That there's a type of upscale
Southerner who doesn't like the incivility that they feel coming from
the White House, and is just not as much of a hardened Republican as
maybe some others. This is where Democrats, combining all of those
three, think they can make big inroads.

caitlin dickerson

So it sounds like Democrats see Georgia as ripe for flipping from red to
blue because of these shifts you're talking about.

astead herndon

Yes. They see it as their most likely opportunity to deliver a blue
state in the South for Joe Biden in November, and in the U.S. Senate.
But they also see it as a gateway to a playbook that other Southern
states can replicate. The thought process is, if Georgia can put it
together after years and years of coming close, that allows places like
South Carolina, places like Texas, to have a real roadmap on how
Democrats can make inroads. What they're missing is a victory to prove
to other states and to prove to the Democratic party that the South is
worth investing in.

caitlin dickerson

And how likely is it that this victory you're describing is actually
going to happen?

astead herndon

While it's certainly a possibility, you have to note that Georgia has
been kind of fool's gold for Democrats for some years now, which makes
the kind of conundrum for what the national party and Joe Biden's
campaign should do this year. Should they invest in Georgia, which is
the only state in the country that has both its Senate seats up in
November? Or, do they spend that money, that time, that investment in
states that they know are more likely to be the tipping point for the
electoral college? It's kind of a choice between playing it safe or
putting all their chips on the table.

caitlin dickerson

So in light of what they saw on Tuesday, which of these two strategies
do you think the Democratic leadership is leaning toward right now?

astead herndon

In the short term, what Biden chooses to prioritize for the November
election, we don't really know. But one of the best ways that the
campaign can signal its intentions is through the vice presidential
selection. If Joe Biden was to select someone who represents the kind of
new Southern democrat --- someone like Stacey Abrams or Keisha Lance
Bottoms, or even Val Demings, the representative in Florida --- that
could signal that the campaign is trying to unlock this type of new
Democratic future in the region that we've talked about. And I don't
think that you can separate race from this question also. The South and
Southern Democrats are overwhelmingly black, and those are the same
people that helped revive Joe Biden's campaign after he was struggling
in Iowa, New Hampshire and Nevada. To me, an important question as we
look towards November is, will Joe Biden try to reward those communities
with an increased focus on them as he moves towards the general
election, or is the primary over and this is all about just the ways
that the campaign believes it needs to beat Donald Trump?

caitlin dickerson

So we've been talking about how important Georgia is to the Democratic
party in 2020, but I can imagine that for that same reason, Georgia is
equally as important to Republicans. So what are they doing to hold on
to the state?

astead herndon

I think like Democrats, Georgia Republicans have short-term and
long-term considerations. In the short term, they just think the state
remains kind of structurally red. But in the long term, Republicans will
concede that the demographics of the state are not moving in their
direction. And what they need to do to stop this kind of rising tide is
to appeal to kind of new communities there. And there's kind of a pitch
that, we should tell them that the reason you're leaving California, or
New York, or other places is because those states have high taxes and
Georgia's business friendly. The ``why'' liberals have wanted to come
here is because of the kind of conservative values, and that's what we
should try to hold on to. The problem is, when the President has so
defined the parties by kind of social and cultural concerns, can the
state Republican make a pitch to an immigrant community, a black
professional, around Republicanism with that not being tied in to what
Trump has made the focus of the party?

caitlin dickerson

You're talking about this cultural clash going on in the country, and
that's very top of mind for a lot of Americans right now, obviously. So
can you put this election we're talking about into the context of this
broader cultural moment that we are all living right now?

astead herndon

Mm-hmm. For both Democrats and Republicans, I think that this moment,
this re-emergence of race and racial justice as the country's top even
electoral or voting concern, plays into the strategies that we have laid
out. For the Republican side, when we talk about the way that state
Republicans and the President have tried to appeal to voters, you've
seen Republicans in the last week or so try to make ``defund the
police'' a scare tactic to bring back that suburban voter. You've seen
them try to focus on the more destructive or looting aspects of the
protests to discredit the movement as a whole. But frankly, public
opinion shows that there has been widespread agreement around police
brutality as a growing issue, and I think that's important to note about
what candidates for both sides are saying right now in Georgia. Doug
Collins, the representative on the Republican side who is running for
Senate, he was the member who wrote and helped pass the First Step Act,
the criminal justice reform that President Trump signed into law. And
this is a deeply conservative representative who has made that criminal
justice pitch a part of his appeal, even in minority communities. And on
the Democratic side, the Senate candidates are running very explicit
campaigns about race and criminal justice, and about inequalities that
were kind of unfathomable in the South years ago. They say that the
times are changing, that you don't have to be cagey or calibrate to the
ideological middle on things like race. That white Democrats are willing
and open to talking about things in explicit terms, and they think that
that can be a winning strategy.

caitlin dickerson

Astead, you've been describing how important Georgia could be in the
2020 elections. So what does what happened on Tuesday night tell us
about what we might expect?

astead herndon

I think Tuesday is a signal for both the country and the parties of
things that we might have to expect come November. For one, if elected
officials do not proactively prepare for an election that could be
upended by virus concerns, we might have lines like we saw on Tuesday.
If they're not prepared to count as thousands and thousands of absentee
and mail-in ballots in ways that are unprecedented in presidential
history, we might not get results from key states on election night. And
even more so, about the type of messages that politicians are giving to
the public right now, if people don't feel as if going to the ballot box
and voting is a process that is equitable and fair to them, it is going
to be harder for particularly Democratic politicians to tell their base,
this is where you should put your energy. This is how you make change.
What we saw on Tuesday was not an encouraging scene.

caitlin dickerson

Thank you so much, Astead.

astead herndon

Thank you, Caitlin.

caitlin dickerson

We'll be right back.

Here's what else you need to know today.

\begin{itemize}
\tightlist
\item
  archived recording (philonise floyd)\\
  The man who took his life, who suffocated him for 8 minutes and 46
  seconds, he still called him ``sir'' as he begged for his life. I
  can't tell you the kind of pain you feel when you watch something like
  that.
\end{itemize}

caitlin dickerson

On Wednesday, George Floyd's brother, Philonise Floyd, testified before
Congress.

\begin{itemize}
\tightlist
\item
  archived recording (philonise floyd)\\
  George wasn't hurting anyone that day. He didn't deserve to die over
  \$20. I'm asking you, is that what a black man is worth, \$20? This is
  2020. Enough is enough.
\end{itemize}

caitlin dickerson

Speaking to the House Judiciary Committee, Floyd called on lawmakers to
pass reforms that would address police brutality and racial
discrimination.

\begin{itemize}
\tightlist
\item
  archived recording (philonise floyd)\\
  If his death ends up changing the world for the better --- and I think
  it will --- then he died as he lived. It is on you to make sure his
  death is not in vain.
\end{itemize}

caitlin dickerson

House Democrats are expected to pass a reform bill this month that would
make it easier to track, prosecute and punish police misconduct. But
Senate Republicans have announced plans to draft their own reform bill.

That's it for ``The Daily.'' I'm Caitlin Dickerson. See you tomorrow.

\href{https://www.nytimes3xbfgragh.onion/column/the-daily}{\includegraphics{https://static01.graylady3jvrrxbe.onion/images/2017/01/29/podcasts/the-daily-album-art/the-daily-album-art-square320-v4.png}The
Daily}Subscribe:

\begin{itemize}
\tightlist
\item
  \href{https://itunes.apple.com/us/podcast/id1200361736}{Apple
  Podcasts}
\item
  \href{https://www.google.com/podcasts?feed=aHR0cHM6Ly9yc3MuYXJ0MTkuY29tL3RoZS1kYWlseQ\%3D\%3D}{Google
  Podcasts}
\end{itemize}

\hypertarget{georgias-election-meltdown-1}{%
\section{Georgia's Election
Meltdown}\label{georgias-election-meltdown-1}}

\hypertarget{why-the-states-troubled-primary-elections-this-week-may-be-a-preview-of-graver-battles-coming-in-the-general-election-1}{%
\subsection{Why the state's troubled primary elections this week may be
a preview of graver battles coming in the general
election.}\label{why-the-states-troubled-primary-elections-this-week-may-be-a-preview-of-graver-battles-coming-in-the-general-election-1}}

Hosted by Michael Barbaro and Caitlin Dickerson; produced by Eric
Krupke, Alexandra Leigh Young, Robert Jimison and Stella Tan; with help
from Sydney Harper; and edited by M.J. Davis Lin

Transcript

transcript

Back to The Daily

bars

0:00/26:27

-0:00

transcript

\hypertarget{georgias-election-meltdown-2}{%
\subsection{Georgia's Election
Meltdown}\label{georgias-election-meltdown-2}}

\hypertarget{hosted-by-michael-barbaro-and-caitlin-dickerson-produced-by-eric-krupke-alexandra-leigh-young-robert-jimison-and-stella-tan-with-help-from-sydney-harper-and-edited-by-mj-davis-lin-1}{%
\subsubsection{Hosted by Michael Barbaro and Caitlin Dickerson; produced
by Eric Krupke, Alexandra Leigh Young, Robert Jimison and Stella Tan;
with help from Sydney Harper; and edited by M.J. Davis
Lin}\label{hosted-by-michael-barbaro-and-caitlin-dickerson-produced-by-eric-krupke-alexandra-leigh-young-robert-jimison-and-stella-tan-with-help-from-sydney-harper-and-edited-by-mj-davis-lin-1}}

\hypertarget{why-the-states-troubled-primary-elections-this-week-may-be-a-preview-of-graver-battles-coming-in-the-general-election-2}{%
\paragraph{Why the state's troubled primary elections this week may be a
preview of graver battles coming in the general
election.}\label{why-the-states-troubled-primary-elections-this-week-may-be-a-preview-of-graver-battles-coming-in-the-general-election-2}}

Thursday, June 11th, 2020

\begin{itemize}
\item
  caitlin dickerson\\
  From The New York Times, I'm Caitlin Dickerson. This is ``The Daily.''
\item
  {[}music{]}\\
  Today: A full-scale meltdown of new voting systems in Georgia is
  alarming Democratic leaders ahead of the state's general election in
  November. My colleague, Astead Herndon, on why voting access in
  Georgia has become a national issue for the party. It's Thursday, June
  11.

  OK, Astead, so tell me what happened on Tuesday in Georgia.
\item
  astead herndon\\
  Tuesday was Georgia's primary elections, where they were slated to
  send Senate candidates and House candidates ahead to November's
  general election, but ---
\item
  archived recording\\
  After twice being delayed due to the coronavirus pandemic, finally
  primary election day, and some of the polls, simply did not go as
  planned.
\end{itemize}

astead herndon

What we saw on Tuesday did not look much like an election at all.

\begin{itemize}
\tightlist
\item
  archived recording\\
  Our newsroom is flooded with emails, calls, tweets, texts of voters
  reporting issues and irregularities at precincts across the metro.
\end{itemize}

astead herndon

At the beginning of the day, polling sites were not opening on time, and
then it became very clear that they weren't adequately staffed.

\begin{itemize}
\tightlist
\item
  archived recording\\
  Health concerns kept many longtime poll workers from showing up today,
  leaving inexperienced volunteers to run new voting machines for the
  first time.
\end{itemize}

astead herndon

Also, there were problems with the machines that were at the polling
sites.

\begin{itemize}
\item
  archived recording 1\\
  Poll workers said they had difficulties turning on the voter check-in
  computers, and encoding voter access cards, and installing touch
  screens.
\item
  archived recording 2\\
  They had printer problems, missing some electrical plugs, as well.
\end{itemize}

astead herndon

There are also fewer polling places to begin with because of the
coronavirus pandemic, so the virus has added more emphasis on mail-in
ballots and absentee ballots, many of which some Georgia residents said
they did not receive in the mail. And this created massive lines ---

\begin{itemize}
\tightlist
\item
  archived recording\\
  We saw repeated over and over people standing, sitting, waiting for
  the opportunity to have their say in our state's political future.
\end{itemize}

astead herndon

--- causing people to wait more than four or five hours in some cases.

\begin{itemize}
\item
  archived recording 1\\
  53 years I've been voting, and never have seen a line like this in 53
  years.
\item
  archived recording 2\\
  This is wrong. This is America. This is a crisis in our world to make
  us not exercise our right to vote.
\end{itemize}

astead herndon

It also caused some people to turn away, just throwing up their hands
and saying, you know, they can't spend a whole day waiting for a line
that they don't know is going to move.

\begin{itemize}
\tightlist
\item
  archived recording\\
  The system is a joke, and we're not laughing.
\end{itemize}

caitlin dickerson

So why was this happening, Astead? What's the reason for all this chaos?

astead herndon

Tuesday was a confluence of local and state problems. And what you hear
from the counties that were particularly affected was that certainly,
their machines and their processes did not work, and they take some
blame for that. But what Democrats say is a larger problem is a state
and Republican administration system that runs the elections process
that is not interested in helping these counties succeed.

caitlin dickerson

What I hear you saying is that what happened on Tuesday was not simply a
fluke.

astead herndon

Right. The roots of Georgia's fights over ballot access and voting
rights start way before Tuesday.

\begin{itemize}
\tightlist
\item
  archived recording\\
  David, what has been the reaction there in Washington to the Supreme
  Court effectively hobbling the Voting Rights Act?
\end{itemize}

astead herndon

In 2013, the Supreme Court opened the door for states to have more
autonomy in changing their voting procedures without input from the
federal government.

\begin{itemize}
\tightlist
\item
  archived recording (david leonhardt)\\
  You see Democrats very upset about this rule, And you see Republicans
  who have come out so far praising it, saying the Voting Rights Act has
  done its work. It may not be needed anymore.
\end{itemize}

astead herndon

And that allowed states like Georgia, states that had historically been
closely watched in the South, to really overhaul their ballot process.
This has included closing polling locations across the state that have
predominantly been in Democratic and African-American communities. And
also, they passed in 2017 what's called the Exact Match Law, which means
when someone registers for the ballot, if there is any difference
between that registration and the identification the state has on file
--- whether that is a misplaced letter or an incorrect hyphen --- it
allows the state to throw out that ballot registration. That has led to
thousands of people being purged from Georgia's voting rolls. And both
of these things, closing the polling locations and the Exact Match Law,
have disproportionately impacted minority communities, and black
communities especially. I remember in 2018 ---

\begin{itemize}
\tightlist
\item
  archived recording\\
  In Georgia, a record-breaking two million early votes were cast, and
  all eyes are focused on the state's race for governor.
\end{itemize}

astead herndon

--- being in Georgia for the closely-watched governor's race between
Stacey Abrams and Brian Kemp.

\begin{itemize}
\tightlist
\item
  archived recording\\
  A poll released today shows that Georgia's secretary of state and
  Republican candidate Brian Kemp leads the Democratic candidate, Stacey
  Abrams, by just one point.
\end{itemize}

astead herndon

And you would be at people's homes, and you would watch them look up
whether their voting registration was still on file. And many would be
shocked to find out that they had been purged even without their
knowledge. And this came in the middle of a governor's race that was
just as much about voting rights as it was about Democrat versus
Republican.

\begin{itemize}
\tightlist
\item
  archived recording (stacey abrams)\\
  I'm Stacey Abrams, and I'm running for governor, because where you
  come from shouldn't determine how far you can go.
\end{itemize}

astead herndon

Stacey Abrams, who had previously been the House Minority Leader in
Georgia, had built a career off of registering new voters, bringing
people --- new people into the process, and kind of a vision of a blue
Georgia on the backs of a multiracial coalition that had yet to be
achieved.

\begin{itemize}
\item
  archived recording (stacey abrams)\\
  The blue wave is African-American. {[}CHEERING{]}

  It's white. It's Latino. It's Asian Pacific Islander.
\item
  archived recording (crowd)\\
  Yes!
\item
  archived recording (stacey abrams)\\
  It is made up of those who've been told that they are not worthy of
  being here.
\item
  archived recording (crowd)\\
  Yes!
\item
  archived recording (stacey abrams)\\
  It is comprised of those who are documented and undocumented.
\item
  archived recording (crowd)\\
  Yes!
\end{itemize}

astead herndon

And she was facing the secretary of state, Brian Kemp ---

\begin{itemize}
\tightlist
\item
  archived recording (brian kemp)\\
  Well, thankfully, the truth here is very simple. Georgians should
  simply watch what she says. You'll know that she's talking about this
  election, and talking about illegals voting for her in this election.
  They filed a lawsuit.
\end{itemize}

astead herndon

--- who had refused to recuse himself from overseeing the state's
election, even as he ran. And this race was wrapped up in accusations of
voter suppression.

\begin{itemize}
\tightlist
\item
  archived recording (stacey abrams)\\
  My worry is that he's using his position as secretary of state to tilt
  the playing field in his direction.
\end{itemize}

astead herndon

And from Republicans about voter fraud.

\begin{itemize}
\tightlist
\item
  archived recording (brian kemp)\\
  I think hardworking Georgians should decide who their governor is, not
  people here illegally like my opponent wants.
\end{itemize}

astead herndon

There was a real sense that whoever won this would be determining the
direction, and most importantly, would be the referee for the state's
elections going forward.

And on election day ---

\begin{itemize}
\item
  archived recording 1\\
  Good morning. There is no lull in this line, and you can see people
  lined up here.
\item
  archived recording 2\\
  The worst of the issues was in Fulton County. At the Pittman Park
  location, only three voting machines were sent, but eight were
  supposed to be there.
\item
  archived recording 3\\
  I live in East Point, and I updated my address at least two times
  before election day. And on Tuesday when I went to my polling place,
  they denied me a ballot.
\end{itemize}

astead herndon

Stacey Abrams lost by a little less than 55,000 votes. And when she
lost, accused Republicans of voter suppression tactics that changed the
outcome of the race.

\begin{itemize}
\tightlist
\item
  archived recording (stacey abrams)\\
  Democracy only works when we work for it, and apparently today, when
  we stand in line for hours to meet it at the ballot box, that's when
  democracy works.
\end{itemize}

astead herndon

For a while, Abrams wouldn't concede to Kemp.

\begin{itemize}
\tightlist
\item
  archived recording (stacey abrams)\\
  Friends, friends, we are still on the verge of history, and the best
  is yet to come. {[}CHEERING{]}
\end{itemize}

caitlin dickerson

Astead, how do Republicans respond to these allegations from Democrats
that the prior election was unfair?

astead herndon

On the defensive side, Republicans say that there is not evidence that
they are proactively trying to suppress votes. They flip the blame,
saying that it is local Democratic officials in these areas who have not
lived up to their task in administrating clean elections. They also say
that they are focused on things like voter fraud, which we should note
does not have real evidence. And they justify things like exact match as
a tool to combat this voter fraud. But they've also done offensive
moves. The state purchased new voting machines after criticism that the
previous ones were not safe, and a court ordered to do so, and those
were used for the first time in Tuesday's election.

caitlin dickerson

And based on what you saw Tuesday, those measures to address problems in
the electoral system, they don't seem to have worked.

astead herndon

Certainly, those measures do not meet the scope of the crisis. So for
whatever new voting machines, or for whatever back and forth this
happening between county and state officials, what is clear is that
voting in Georgia does not go the way voting should be. But for
Republicans, they'll say that the vast majority of Georgia's county's,
150 out of 159, had fine days on Tuesday. But it's important to
recognize that those nine counties that had the biggest issues on
Tuesday, they're not only Democratic areas, but those are the counties
that have the largest minority populations in the state.

caitlin dickerson

So from everything you've said, it sounds like Democrats would see the
problems with Tuesday's election as being just a continuation of voting
issues that have plagued these same communities in the past, and that
they feel Republicans have either ignored or even made worse.

astead herndon

That's what Democrats will tell you. I remember running into the state
Democratic chair when she was trying to vote. It took her five hours on
what was her 10-year wedding anniversary. And she was talking about how
familiar it felt, and encouraging people in the line to hold that
feeling with them as they look towards the general election. But when
you look at the reaction across the country and how much interest there
was in what was happening in Georgia, I think a part of that is because
the national Democratic and Republican parties realize just how
important this state is --- not just for November, but what could be a
preview of how Southern politics is changing in the future.

{[}music{]}

caitlin dickerson

We'll be right back.

Astead, what do you mean? Why is Georgia such a key state for the
Democratic party?

astead herndon

For decades now, Democrats have been virtually shut out of the South. It
has been almost impossible for the party to find consistent success in
getting a candidate elected to statewide office, whether that's a
governor or the Senate, and in presidential elections. And what
Democrats have been trying to do over the past decade is create a
grassroots momentum that can change the way that they operate in the
South. And Georgia has been the focal point of that.

caitlin dickerson

Astead, help me understand the Democrats' strategy in Georgia.

astead herndon

It basically breaks down to three areas.

The first is just the changing demographics of the state. New
industries, particularly movie and film, have caused an influx of a new
Southerner, as some folks called it, who is living in places like
Atlanta and the metro areas, that has made the South their home in the
way that has given Democrats a new type of voter to target. Another key
point of the strategy is in registering Georgians who may not have
participated in previous elections. So that includes predominately young
people and people of color, and going to those communities that have
kind of felt distant from the political process and bringing them along
and involved. The third piece, which has been accelerated in the last
three years, has been trying to persuade a white, often college-educated
voter, who probably had voted Republican before, that Democrats are now
more acceptable party. And this is something that Democrats have said
Donald Trump is their best recruiter for. That there's a type of upscale
Southerner who doesn't like the incivility that they feel coming from
the White House, and is just not as much of a hardened Republican as
maybe some others. This is where Democrats, combining all of those
three, think they can make big inroads.

caitlin dickerson

So it sounds like Democrats see Georgia as ripe for flipping from red to
blue because of these shifts you're talking about.

astead herndon

Yes. They see it as their most likely opportunity to deliver a blue
state in the South for Joe Biden in November, and in the U.S. Senate.
But they also see it as a gateway to a playbook that other Southern
states can replicate. The thought process is, if Georgia can put it
together after years and years of coming close, that allows places like
South Carolina, places like Texas, to have a real roadmap on how
Democrats can make inroads. What they're missing is a victory to prove
to other states and to prove to the Democratic party that the South is
worth investing in.

caitlin dickerson

And how likely is it that this victory you're describing is actually
going to happen?

astead herndon

While it's certainly a possibility, you have to note that Georgia has
been kind of fool's gold for Democrats for some years now, which makes
the kind of conundrum for what the national party and Joe Biden's
campaign should do this year. Should they invest in Georgia, which is
the only state in the country that has both its Senate seats up in
November? Or, do they spend that money, that time, that investment in
states that they know are more likely to be the tipping point for the
electoral college? It's kind of a choice between playing it safe or
putting all their chips on the table.

caitlin dickerson

So in light of what they saw on Tuesday, which of these two strategies
do you think the Democratic leadership is leaning toward right now?

astead herndon

In the short term, what Biden chooses to prioritize for the November
election, we don't really know. But one of the best ways that the
campaign can signal its intentions is through the vice presidential
selection. If Joe Biden was to select someone who represents the kind of
new Southern democrat --- someone like Stacey Abrams or Keisha Lance
Bottoms, or even Val Demings, the representative in Florida --- that
could signal that the campaign is trying to unlock this type of new
Democratic future in the region that we've talked about. And I don't
think that you can separate race from this question also. The South and
Southern Democrats are overwhelmingly black, and those are the same
people that helped revive Joe Biden's campaign after he was struggling
in Iowa, New Hampshire and Nevada. To me, an important question as we
look towards November is, will Joe Biden try to reward those communities
with an increased focus on them as he moves towards the general
election, or is the primary over and this is all about just the ways
that the campaign believes it needs to beat Donald Trump?

caitlin dickerson

So we've been talking about how important Georgia is to the Democratic
party in 2020, but I can imagine that for that same reason, Georgia is
equally as important to Republicans. So what are they doing to hold on
to the state?

astead herndon

I think like Democrats, Georgia Republicans have short-term and
long-term considerations. In the short term, they just think the state
remains kind of structurally red. But in the long term, Republicans will
concede that the demographics of the state are not moving in their
direction. And what they need to do to stop this kind of rising tide is
to appeal to kind of new communities there. And there's kind of a pitch
that, we should tell them that the reason you're leaving California, or
New York, or other places is because those states have high taxes and
Georgia's business friendly. The ``why'' liberals have wanted to come
here is because of the kind of conservative values, and that's what we
should try to hold on to. The problem is, when the President has so
defined the parties by kind of social and cultural concerns, can the
state Republican make a pitch to an immigrant community, a black
professional, around Republicanism with that not being tied in to what
Trump has made the focus of the party?

caitlin dickerson

You're talking about this cultural clash going on in the country, and
that's very top of mind for a lot of Americans right now, obviously. So
can you put this election we're talking about into the context of this
broader cultural moment that we are all living right now?

astead herndon

Mm-hmm. For both Democrats and Republicans, I think that this moment,
this re-emergence of race and racial justice as the country's top even
electoral or voting concern, plays into the strategies that we have laid
out. For the Republican side, when we talk about the way that state
Republicans and the President have tried to appeal to voters, you've
seen Republicans in the last week or so try to make ``defund the
police'' a scare tactic to bring back that suburban voter. You've seen
them try to focus on the more destructive or looting aspects of the
protests to discredit the movement as a whole. But frankly, public
opinion shows that there has been widespread agreement around police
brutality as a growing issue, and I think that's important to note about
what candidates for both sides are saying right now in Georgia. Doug
Collins, the representative on the Republican side who is running for
Senate, he was the member who wrote and helped pass the First Step Act,
the criminal justice reform that President Trump signed into law. And
this is a deeply conservative representative who has made that criminal
justice pitch a part of his appeal, even in minority communities. And on
the Democratic side, the Senate candidates are running very explicit
campaigns about race and criminal justice, and about inequalities that
were kind of unfathomable in the South years ago. They say that the
times are changing, that you don't have to be cagey or calibrate to the
ideological middle on things like race. That white Democrats are willing
and open to talking about things in explicit terms, and they think that
that can be a winning strategy.

caitlin dickerson

Astead, you've been describing how important Georgia could be in the
2020 elections. So what does what happened on Tuesday night tell us
about what we might expect?

astead herndon

I think Tuesday is a signal for both the country and the parties of
things that we might have to expect come November. For one, if elected
officials do not proactively prepare for an election that could be
upended by virus concerns, we might have lines like we saw on Tuesday.
If they're not prepared to count as thousands and thousands of absentee
and mail-in ballots in ways that are unprecedented in presidential
history, we might not get results from key states on election night. And
even more so, about the type of messages that politicians are giving to
the public right now, if people don't feel as if going to the ballot box
and voting is a process that is equitable and fair to them, it is going
to be harder for particularly Democratic politicians to tell their base,
this is where you should put your energy. This is how you make change.
What we saw on Tuesday was not an encouraging scene.

caitlin dickerson

Thank you so much, Astead.

astead herndon

Thank you, Caitlin.

caitlin dickerson

We'll be right back.

Here's what else you need to know today.

\begin{itemize}
\tightlist
\item
  archived recording (philonise floyd)\\
  The man who took his life, who suffocated him for 8 minutes and 46
  seconds, he still called him ``sir'' as he begged for his life. I
  can't tell you the kind of pain you feel when you watch something like
  that.
\end{itemize}

caitlin dickerson

On Wednesday, George Floyd's brother, Philonise Floyd, testified before
Congress.

\begin{itemize}
\tightlist
\item
  archived recording (philonise floyd)\\
  George wasn't hurting anyone that day. He didn't deserve to die over
  \$20. I'm asking you, is that what a black man is worth, \$20? This is
  2020. Enough is enough.
\end{itemize}

caitlin dickerson

Speaking to the House Judiciary Committee, Floyd called on lawmakers to
pass reforms that would address police brutality and racial
discrimination.

\begin{itemize}
\tightlist
\item
  archived recording (philonise floyd)\\
  If his death ends up changing the world for the better --- and I think
  it will --- then he died as he lived. It is on you to make sure his
  death is not in vain.
\end{itemize}

caitlin dickerson

House Democrats are expected to pass a reform bill this month that would
make it easier to track, prosecute and punish police misconduct. But
Senate Republicans have announced plans to draft their own reform bill.

That's it for ``The Daily.'' I'm Caitlin Dickerson. See you tomorrow.

Previous

More episodes ofThe Daily

\href{https://www.nytimes3xbfgragh.onion/2020/08/04/podcasts/the-daily/mail-in-voting-president-trump.html?action=click\&module=audio-series-bar\&region=header\&pgtype=Article}{\includegraphics{https://static01.graylady3jvrrxbe.onion/images/2020/07/30/us/politics/04daily/30trump-election1-thumbLarge.jpg}}

August 4, 2020Is the U.S. Ready to Vote by Mail?

\href{https://www.nytimes3xbfgragh.onion/2020/08/03/podcasts/the-daily/algorithmic-justice-racism.html?action=click\&module=audio-series-bar\&region=header\&pgtype=Article}{\includegraphics{https://static01.graylady3jvrrxbe.onion/images/2020/06/24/business/03daily/24michigan-arrest1-thumbLarge.jpg}}

August 3, 2020~~•~ 28:13Wrongfully Accused by an Algorithm

\href{https://www.nytimes3xbfgragh.onion/2020/08/02/podcasts/the-daily/on-female-rage.html?action=click\&module=audio-series-bar\&region=header\&pgtype=Article}{\includegraphics{https://static01.graylady3jvrrxbe.onion/images/2018/01/21/magazine/21mag-femaleanger1-copy/21mag-femaleanger1-thumbLarge.jpg}}

August 2, 2020The Sunday Read: `On Female Rage'

\href{https://www.nytimes3xbfgragh.onion/2020/07/31/podcasts/the-daily/vanessa-guillen-military-metoo.html?action=click\&module=audio-series-bar\&region=header\&pgtype=Article}{\includegraphics{https://static01.graylady3jvrrxbe.onion/images/2020/07/12/us/politics/31daily/00dc-army-metoo-thumbLarge.jpg}}

July 31, 2020A \#MeToo Moment in the Military

\href{https://www.nytimes3xbfgragh.onion/2020/07/30/podcasts/the-daily/congress-facebook-amazon-google-apple.html?action=click\&module=audio-series-bar\&region=header\&pgtype=Article}{\includegraphics{https://static01.graylady3jvrrxbe.onion/images/2020/07/30/reader-center/30daily/merlin_175077825_5ebc931b-baa1-489a-960c-34e4d845e997-thumbLarge.jpg}}

July 30, 2020~~•~ 35:19The Big Tech Hearing

\href{https://www.nytimes3xbfgragh.onion/2020/07/29/podcasts/the-daily/china-trump-foreign-policy.html?action=click\&module=audio-series-bar\&region=header\&pgtype=Article}{\includegraphics{https://static01.graylady3jvrrxbe.onion/images/2020/07/26/world/29daily/00china-us-clash1-thumbLarge.jpg}}

July 29, 2020~~•~ 28:40Confronting China

\href{https://www.nytimes3xbfgragh.onion/2020/07/28/podcasts/the-daily/unemployment-benefits-coronavirus.html?action=click\&module=audio-series-bar\&region=header\&pgtype=Article}{\includegraphics{https://static01.graylady3jvrrxbe.onion/images/2020/07/23/business/28daily/23virus-uiexplain1-thumbLarge.jpg}}

July 28, 2020~~•~ 26:13Why \$600 Checks Are Tearing Republicans Apart

\href{https://www.nytimes3xbfgragh.onion/2020/07/27/podcasts/the-daily/new-york-hospitals-covid.html?action=click\&module=audio-series-bar\&region=header\&pgtype=Article}{\includegraphics{https://static01.graylady3jvrrxbe.onion/images/2020/07/27/world/27daily-hospitals/27daily-hospitals-thumbLarge.jpg}}

July 27, 2020~~•~ 33:28The Mistakes New York Made

\href{https://www.nytimes3xbfgragh.onion/2020/07/26/podcasts/the-daily/the-accusation-the-sunday-read.html?action=click\&module=audio-series-bar\&region=header\&pgtype=Article}{\includegraphics{https://static01.graylady3jvrrxbe.onion/images/2020/03/22/magazine/26audm-2/22mag-titleix-thumbLarge.jpg}}

July 26, 2020The Sunday Read: `The Accusation'

\href{https://www.nytimes3xbfgragh.onion/2020/07/24/podcasts/the-daily/mlb-baseball-season-coronavirus.html?action=click\&module=audio-series-bar\&region=header\&pgtype=Article}{\includegraphics{https://static01.graylady3jvrrxbe.onion/images/2020/07/22/sports/24daily/22mlb-previewlede1-thumbLarge.jpg}}

July 24, 2020~~•~ 45:34The Battle for a Baseball Season

\href{https://www.nytimes3xbfgragh.onion/2020/07/23/podcasts/the-daily/portland-protests.html?action=click\&module=audio-series-bar\&region=header\&pgtype=Article}{\includegraphics{https://static01.graylady3jvrrxbe.onion/images/2020/07/22/us/23daily-image/22portland-tactics02-thumbLarge.jpg}}

July 23, 2020~~•~ 30:04The Showdown in Portland

\href{https://www.nytimes3xbfgragh.onion/2020/07/22/podcasts/the-daily/school-reopenings-coronavirus.html?action=click\&module=audio-series-bar\&region=header\&pgtype=Article}{\includegraphics{https://static01.graylady3jvrrxbe.onion/images/2020/07/12/science/22daily/00virus-schools-reopen01-thumbLarge.jpg}}

July 22, 2020~~•~ 27:24The Science of School Reopenings

\href{https://www.nytimes3xbfgragh.onion/column/the-daily}{See All
Episodes ofThe Daily}

Next

June 11, 2020

\begin{itemize}
\item
\item
\item
\item
\item
\item
\end{itemize}

\emph{\textbf{Listen and subscribe to our podcast from your mobile
device:}}\\
\textbf{\href{https://itunes.apple.com/us/podcast/the-daily/id1200361736?mt=2}{\emph{Via
Apple Podcasts}}} \emph{\textbf{\textbar{}}}
\textbf{\href{https://open.spotify.com/show/3IM0lmZxpFAY7CwMuv9H4g?si=SfuMSC55R1qprFsRZU3_zw}{\emph{Via
Spotify}}} \emph{\textbf{\textbar{}}}
\textbf{\href{http://www.stitcher.com/podcast/the-new-york-times/the-daily-10}{\emph{Via
Stitcher}}}

A full-scale meltdown of new voting systems in Georgia is alarming
Democratic leaders --- and revealing a new national playing field ---
ahead of the general election in November. Today, we explore why voting
access in Georgia has become a national issue for the party.

\textbf{On today's episode:}

\begin{itemize}
\tightlist
\item
  \href{https://www.nytimes3xbfgragh.onion/by/astead-w-herndon}{Astead
  W. Herndon}, who covers national politics for The New York Times.
\end{itemize}

\includegraphics{https://static01.graylady3jvrrxbe.onion/images/2020/06/09/us/politics/09voting-georgia2/09voting-georgia2-videoSixteenByNine3000-v2.jpg}

\textbf{Background reading:}

\begin{itemize}
\item
  Long lines and malfunctioning voting machines marred Georgia primary
  elections, renewing attention on voting rights there, and raising
  questions about how to ensure
  \href{https://www.nytimes3xbfgragh.onion/2020/06/09/us/politics/atlanta-voting-georgia-primary.html}{access
  to voting in the general election}.
\item
  With both Senate seats in play and President Trump up for re-election
  in November, Georgia Democrats are telling anyone who will listen:
  This time
  \href{https://www.nytimes3xbfgragh.onion/2020/06/09/us/politics/georgia-primary-election-senate-race-jon-ossoff.html}{will
  be different}.
\end{itemize}

\emph{Tune in, and tell us what you think. Email us at}
\href{mailto:thedaily@NYTimes.com}{\emph{thedaily@NYTimes.com}}\emph{.
Follow Michael Barbaro on Twitter:}
\href{https://twitter.com/mikiebarb}{\emph{@mikiebarb}}\emph{. And if
you're interested in advertising with ``The Daily,'' write to us at}
\href{mailto:thedaily-ads@NYTimes.com}{\emph{thedaily-ads@NYTimes.com}}\emph{.}

Astead W. Herndon contributed reporting.

``The Daily'' is made by Theo Balcomb, Andy Mills, Lisa Tobin, Rachel
Quester, Lynsea Garrison, Annie Brown, Clare Toeniskoetter, Paige
Cowett, Michael Simon Johnson, Brad Fisher, Larissa Anderson, Wendy
Dorr, Chris Wood, Jessica Cheung, Stella Tan, Alexandra Leigh Young,
Jonathan Wolfe, Lisa Chow, Eric Krupke, Marc Georges, Luke Vander Ploeg,
Adizah Eghan, Kelly Prime, Julia Longoria, Sindhu Gnanasambandan, M.J.
Davis Lin, Austin Mitchell, Sayre Quevedo, Neena Pathak, Dan Powell,
Dave Shaw, Sydney Harper, Daniel Guillemette, Hans Buetow, Robert
Jimison, Mike Benoist, Bianca Giaever and Asthaa Chaturvedi. Our theme
music is by Jim Brunberg and Ben Landsverk of Wonderly. Special thanks
to Sam Dolnick, Mikayla Bouchard, Lauren Jackson, Julia Simon, Mahima
Chablani and Nora Keller.

\hypertarget{our-2020-election-guide}{%
\section{Our 2020 Election Guide}\label{our-2020-election-guide}}

Updated Aug. 4, 2020

\begin{itemize}
\item
  \begin{center}\rule{0.5\linewidth}{\linethickness}\end{center}

  \hypertarget{the-latest}{%
  \subsection{The Latest}\label{the-latest}}

  \begin{itemize}
  \tightlist
  \item
    Five states are holding primary elections Tuesday, with voters in
    Arizona, Kansas, Michigan, Missouri and Washington State choosing
    nominees for Congress and local offices.
    \href{https://www.nytimes3xbfgragh.onion/2020/08/04/us/elections/primary-election-michigan-arizona-kansas.html?action=click\&pgtype=Article\&state=default\&region=BELOW_MAIN_CONTENT\&context=storylines_guide}{Follow
    live election updates here.}
  \end{itemize}
\item
  \begin{center}\rule{0.5\linewidth}{\linethickness}\end{center}

  \hypertarget{bidens-vp-search}{%
  \subsection{Biden's V.P. Search}\label{bidens-vp-search}}

  \begin{itemize}
  \tightlist
  \item
    \href{https://www.nytimes3xbfgragh.onion/article/biden-vice-president-2020.html?action=click\&pgtype=Article\&state=default\&region=BELOW_MAIN_CONTENT\&context=storylines_guide}{Here
    are 13 women} who have been under consideration to be Joe Biden's
    running mate, and why each might be chosen --- and might not be.
  \end{itemize}
\item
  \begin{center}\rule{0.5\linewidth}{\linethickness}\end{center}

  \hypertarget{keep-up-with-our-coverage}{%
  \subsection{Keep Up With Our
  Coverage}\label{keep-up-with-our-coverage}}

  \begin{itemize}
  \tightlist
  \item
    Get an
    \href{https://www.nytimes3xbfgragh.onion/newsletters/politics?action=click\&pgtype=Article\&state=default\&region=BELOW_MAIN_CONTENT\&context=storylines_guide}{email}
    recapping the day's news
  \end{itemize}

  \begin{itemize}
  \tightlist
  \item
    Download our mobile app on
    \href{https://apps.apple.com/us/app/nytimes/id284862083?ls=1\&mat_click_id=5c79ae7455014fd1bd66b5610c05b8f2-20191112-16948\&referrer=mat_click_id\%3D5c79ae7455014fd1bd66b5610c05b8f2-20191112-16948\%26link_click_id\%3D722930677036718082}{iOS}
    and
    \href{http://a.localytics.com/android?id=com.nytimes.android\&referrer=utm_source\%3Dother_nyt_mobile_web\%26utm_medium\%3DWeb\%2520page\%26utm_term\%3DGeneral\%2520Mobile\%2520Page\%26utm_campaign\%3DNYT\%2520Mobile\%2520General\%2520Page}{Android}
    and turn on Breaking News and Politics alerts
  \end{itemize}
\end{itemize}

Advertisement

\protect\hyperlink{after-bottom}{Continue reading the main story}

\hypertarget{site-index}{%
\subsection{Site Index}\label{site-index}}

\hypertarget{site-information-navigation}{%
\subsection{Site Information
Navigation}\label{site-information-navigation}}

\begin{itemize}
\tightlist
\item
  \href{https://help.nytimes3xbfgragh.onion/hc/en-us/articles/115014792127-Copyright-notice}{©~2020~The
  New York Times Company}
\end{itemize}

\begin{itemize}
\tightlist
\item
  \href{https://www.nytco.com/}{NYTCo}
\item
  \href{https://help.nytimes3xbfgragh.onion/hc/en-us/articles/115015385887-Contact-Us}{Contact
  Us}
\item
  \href{https://www.nytco.com/careers/}{Work with us}
\item
  \href{https://nytmediakit.com/}{Advertise}
\item
  \href{http://www.tbrandstudio.com/}{T Brand Studio}
\item
  \href{https://www.nytimes3xbfgragh.onion/privacy/cookie-policy\#how-do-i-manage-trackers}{Your
  Ad Choices}
\item
  \href{https://www.nytimes3xbfgragh.onion/privacy}{Privacy}
\item
  \href{https://help.nytimes3xbfgragh.onion/hc/en-us/articles/115014893428-Terms-of-service}{Terms
  of Service}
\item
  \href{https://help.nytimes3xbfgragh.onion/hc/en-us/articles/115014893968-Terms-of-sale}{Terms
  of Sale}
\item
  \href{https://spiderbites.nytimes3xbfgragh.onion}{Site Map}
\item
  \href{https://help.nytimes3xbfgragh.onion/hc/en-us}{Help}
\item
  \href{https://www.nytimes3xbfgragh.onion/subscription?campaignId=37WXW}{Subscriptions}
\end{itemize}
