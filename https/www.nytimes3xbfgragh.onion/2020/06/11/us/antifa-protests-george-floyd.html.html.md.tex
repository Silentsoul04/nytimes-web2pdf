Sections

SEARCH

\protect\hyperlink{site-content}{Skip to
content}\protect\hyperlink{site-index}{Skip to site index}

\href{https://www.nytimes3xbfgragh.onion/section/us}{U.S.}

\href{https://myaccount.nytimes3xbfgragh.onion/auth/login?response_type=cookie\&client_id=vi}{}

\href{https://www.nytimes3xbfgragh.onion/section/todayspaper}{Today's
Paper}

\href{/section/us}{U.S.}\textbar{}Federal Arrests Show No Sign That
Antifa Plotted Protests

\url{https://nyti.ms/37pmWkK}

\begin{itemize}
\item
\item
\item
\item
\item
\item
\end{itemize}

\href{https://www.nytimes3xbfgragh.onion/news-event/george-floyd-protests-minneapolis-new-york-los-angeles?action=click\&pgtype=Article\&state=default\&region=TOP_BANNER\&context=storylines_menu}{Race
and America}

\begin{itemize}
\tightlist
\item
  \href{https://www.nytimes3xbfgragh.onion/2020/07/26/us/protests-portland-seattle-trump.html?action=click\&pgtype=Article\&state=default\&region=TOP_BANNER\&context=storylines_menu}{Protesters
  Return to Other Cities}
\item
  \href{https://www.nytimes3xbfgragh.onion/2020/07/24/us/portland-oregon-protests-white-race.html?action=click\&pgtype=Article\&state=default\&region=TOP_BANNER\&context=storylines_menu}{Portland
  at the Center}
\item
  \href{https://www.nytimes3xbfgragh.onion/2020/07/23/podcasts/the-daily/portland-protests.html?action=click\&pgtype=Article\&state=default\&region=TOP_BANNER\&context=storylines_menu}{Podcast:
  Showdown in Portland}
\item
  \href{https://www.nytimes3xbfgragh.onion/interactive/2020/07/16/us/black-lives-matter-protests-louisville-breonna-taylor.html?action=click\&pgtype=Article\&state=default\&region=TOP_BANNER\&context=storylines_menu}{45
  Days in Louisville}
\end{itemize}

Advertisement

\protect\hyperlink{after-top}{Continue reading the main story}

Supported by

\protect\hyperlink{after-sponsor}{Continue reading the main story}

\hypertarget{federal-arrests-show-no-sign-that-antifa-plotted-protests}{%
\section{Federal Arrests Show No Sign That Antifa Plotted
Protests}\label{federal-arrests-show-no-sign-that-antifa-plotted-protests}}

Despite claims by President Trump and Attorney General William P. Barr,
there is scant evidence that loosely organized anti-fascists are a
significant player in protests.

\includegraphics{https://static01.graylady3jvrrxbe.onion/images/2020/06/10/us/10UNREST-ANTIFA-la/merlin_173182728_cd1a5374-60f8-40cc-80bb-19c16e8224a7-articleLarge.jpg?quality=75\&auto=webp\&disable=upscale}

By \href{https://www.nytimes3xbfgragh.onion/by/neil-macfarquhar}{Neil
MacFarquhar},
\href{https://www.nytimes3xbfgragh.onion/by/alan-feuer}{Alan Feuer} and
\href{https://www.nytimes3xbfgragh.onion/by/adam-goldman}{Adam Goldman}

\begin{itemize}
\item
  Published June 11, 2020Updated June 12, 2020
\item
  \begin{itemize}
  \item
  \item
  \item
  \item
  \item
  \item
  \end{itemize}
\end{itemize}

Inciting a riot. Hurling a Molotov cocktail. Plotting to sow
destruction. Those are some of the most serious charges brought by
federal prosecutors against demonstrators at
\href{https://www.nytimes3xbfgragh.onion/news-event/george-floyd-protests-minneapolis-new-york-los-angeles}{protests
across the country} in recent weeks.

But despite cries from President Trump and others in his administration,
none of those charged with serious federal crimes amid the unrest have
been linked so far to the loose collective of
\href{https://www.nytimes3xbfgragh.onion/article/what-antifa-trump.html}{anti-fascist
activists known as antifa}.

A review of the arrests of dozens of people on federal charges reveals
no known effort by antifa to perpetrate a coordinated campaign of
violence. Some criminal complaints described vague, anti-government
political leanings among suspects, but a majority of the violent acts
that have taken place at protests have been attributed by federal
prosecutors to individuals with no affiliation to any particular group.

Even so, Attorney General William P. Barr has blamed antifa for
orchestrating the mass protests, which broke out in cities and towns
across the country after the death in police custody of
\href{https://www.nytimes3xbfgragh.onion/article/george-floyd-who-is.html}{George
Floyd}. ``There is clearly some high degree of organization involved at
some of these events and coordinated tactics that we are seeing,'' Mr.
Barr said. ``Some of it relates to antifa, some of it relates to groups
that act very much like antifa.''

Mr. Trump has sought to expand and exploit accusations against what he
has called the involvement of ``radical leftists'' in the protests. At
one point the president said that antifa would be declared a ``terrorist
organization,'' although it is not a single organization nor does any
\href{https://www.nytimes3xbfgragh.onion/2020/02/25/us/domestic-terrorism-laws.html}{American
law allow} using that designation against a domestic group. On Tuesday,
the president suggested on Twitter, without providing any evidence, that
a 75-year-old Buffalo protester hospitalized after being knocked down by
the police could be
\href{https://www.nytimes3xbfgragh.onion/2020/06/09/nyregion/who-is-martin-gugino-buffalo-police.html}{``an
ANTIFA provocateur.''}

Mr. Trump and other Republicans have also sought to raise campaign funds
off the unsubstantiated ****** accusations. ``Stand with President Trump
against antifa!'' read a banner advertisement on Mr. Trump's re-election
campaign website this week.

Marjorie Greene, a congressional candidate in Georgia, produced a
campaign ad showing her armed with an AR-15-style rifle and threatening
antifa activists. ``You won't burn our churches, loot our businesses or
destroy our homes,'' she said.

Asked why the myriad criminal complaints do not single out antifa, Mr.
Barr said on Fox News this week that preliminary charges did not require
linking suspects to a particular group, adding that there was ``a
witches' brew of extremist groups that are trying to exploit this
situation on all sides.''

F.B.I. agents and federal prosecutors have pursued charges aggressively
against rioters, looters and others accused of wreaking havoc during the
demonstrations. Law enforcement officials have relied on a variety of
federal statutes to make arrests, including conspiracy to commit arson,
starting a riot, civil disorder and possession of a Molotov cocktail.

The most serious case that has emerged in federal court involved three
men in Nevada linked to a loose, national network of far-right
extremists advocating the overthrow of the U.S. government. They were
arrested on May 30 on charges of trying to foment violence during Black
Lives Matter protests.

Given the sheer volume of thousands of arrests nationwide in recent
weeks, officials cautioned that many investigations remain in the early
stages with investigators still trying to determine affiliations. In
addition, state and local court documents are far harder to search
comprehensively.

However, interviews with several major police departments and a review
of hundreds of newspaper articles about arrests around the country
revealed no evidence of an organized political effort behind the looting
and other violence.

``We saw no organized effort of antifa here in Los Angeles,'' said Josh
Rubenstein, the spokesman for the Los Angeles Police Department.

Asked in an interview about the involvement of antifa or other extremist
groups in Minneapolis, Medaria Arradondo, the chief of police, said,
``As I sit here today, I have not received any sort of official
information identifying any of the groups.''

In one example where antifa is mentioned, the police in Austin, Texas,
said members of the Red Guards, a Maoist organization, were involved in
organizing the looting of a Target store. The Red Guards have been
associated with past antifa protests in Austin, but local activists said
they were largely estranged from the group.

While anarchists and anti-fascists openly acknowledged being part of the
immense crowds, they call the scale, intensity and durability of the
protests far beyond anything they might dream of organizing. Some
tactics used at the protests, like the wearing of all black and the
shattering of store windows, are reminiscent of those used by anarchist
groups, say those who study such movements.

In Portland, Ore., those affiliated with Rose City Antifa said they had
supported the continuing protests. But the city's antifa actions have
long involved a wide range of people, some who dress in black apparel
and face coverings and others who show up in everyday clothing to
denounce far-right extremists and police militarization. There have also
been various far-left activities in Seattle, including people who have
spray-painted anarchist symbols on public property.

Antifa has roots in the Occupy Wall Street protests of a decade ago and
the demonstrations against the World Trade Organization in the 1990s.
During Mr. Trump's inauguration, antifa activists marched in Washington
vandalizing businesses and at one point setting fire to a limousine.

Over the next several months, its followers disrupted events hosted by
right-wing speakers like Ann Coulter and Milo Yiannopoulos. When the far
right fought back, organizing its own public protests, anti-fascist
activists met them on the streets in what often turned into violent
confrontations, culminating in
\href{https://www.nytimes3xbfgragh.onion/2017/08/12/us/charlottesville-protest-white-nationalist.html}{the
bloody rally in Charlottesville, Va.}, in 2017.

Anarchists and others accuse officials of trying to assign blame to
extremists rather than accept the idea that millions of Americans from a
variety of political backgrounds have been on the streets demanding
change. Numerous experts also called the participation of extremist
organizations overstated.

``A significant number of people in positions of authority are pushing a
false narrative about antifa being behind a lot of this activity,'' said
J.M. Berger, the author of the book ``Extremism'' and an authority on
militant movements. ``These are just unbelievably large protests at a
time of great turmoil in this country, and there is surprisingly little
violence given the size of this movement.''

Christopher Wray, the F.B.I. director,
\href{https://reason.com/2019/07/24/ted-cruz-wants-antifa-investigated-by-the-fbi/}{told
the Senate Judiciary Committee} in July that the agency ``considers
antifa more of an ideology than an organization.''

In Las Vegas, the complaint filed in U.S. District Court said the three
suspects called themselves members of the ``boogaloo,'' which is
described as a far-right movement ``to signify a coming civil war and/or
fall of civilization.''

At a protest, the three strapped on bulletproof vests, grabbed their
rifles and waded into the crowd, hoping to provoke clashes between
protesters and the police, according to court papers. One taunted police
officers, yelling in their faces, while a second chided protesters
``that peaceful protests don't accomplish anything and they needed to be
violent,'' the complaint said.

When that failed, they plotted to blow up an electric substation along
the route of the demonstration in the hope that would prompt more
violence between the police and protesters, according to the complaint.
They were arrested after preparing Molotov cocktails from gasoline and
lemonade bottles before a march.

Robert M. Draskovich Jr., a lawyer for one of the accused, Stephen T.
Parshall, 35, said his client denied all the charges.

Individuals associated with the
\href{https://www.nytimes3xbfgragh.onion/2020/05/03/us/coronavirus-extremists.html}{boogaloo
movement} have been out in force at numerous demonstrations in the past
few years, clad in their distinctive combat dress and armed with rifles.
They often claim that they appear armed in public to underscore their
commitment to Second Amendment rights, or to protect local businesses.

But online, boogaloo discussion groups overflow with racist statements
and threats to exploit any unrest to incite a race war that will bring
about a new government system.

In Denver, the police seized a small arsenal including three assault
rifles, numerous magazines, several bulletproof vests and other military
paraphernalia from the car trunk of a self-professed ``boogaloo''
adherent headed toward a protest, a man who had previously livestreamed
his own support for armed confrontations with the police.

After a demonstration in Athens, Ga., on May 31 ended with the National
Guard being called in and tear gas being fired to clear protesters away
from the gates of the University of Georgia, Chief Cleveland L. Spruill
wrote a lengthy memo spelling out his concerns around extremist
involvement in the protests.

Given the volatile mix of protesters, including armed men, he said, he
feared a repeat of Charlottesville. Some participants called such fears
overblown given the overall peaceful tenor of the protest.

\includegraphics{https://static01.graylady3jvrrxbe.onion/images/2020/06/10/us/10UNREST-ANTIFA-bk/merlin_173303703_8d942968-55ee-4382-bd97-e069c714b12e-articleLarge.jpg?quality=75\&auto=webp\&disable=upscale}

In New York, the police briefed reporters on May 31, claiming that
radical anarchists from outside the state had plotted ahead of protests
by setting up encrypted communications systems, arranging for street
medics and collecting bail funds.

Within five days, however, Dermot F. Shea, the city's police
commissioner, acknowledged that most of the hundreds of people arrested
at the
\href{https://www.nytimes3xbfgragh.onion/2020/06/11/nyregion/nyc-george-floyd-protests.html}{protests
in New York} were actually New Yorkers who took advantage of the chaos
to commit crimes and were not motivated by political ideology. John
Miller, the police official who had briefed reporters, told CNN that
most looting in New York had been committed by ``regular criminal
groups.''

In Austin, court documents said several members of the Red Guards
participated in burglarizing a Target store, including a woman who
streamed the event on Facebook Live, encouraging people to come ``even
if you do not want to loot,'' one affidavit said.

Although the court documents identified the Red Guards as part of the
city's anti-fascist umbrella organization, several Austin activists
described the group as either defunct or estranged because of its
penchant for troubling acts such as laying a dead cat on the doorstep of
a business involved in a gentrification dispute.

Kit O'Connell, a longtime radical leftist activist and community
organizer in Austin, said that shortly after Mr. Trump's election, the
group took part in anti-fascist protests in the city against a local
white supremacist group and scuffled separately with Act for America, an
anti-Muslim organization.

``They've been an influence at the protests but they're not in charge
--- no one's really in charge,'' Mr. O'Connell said.

Carl Guthrie, a lawyer for Samuel Miller, one of those charged with
burglary, denied that his client had any connection to the Red Guards.
He called such accusations ``a transparent, incendiary attempt to
distract from the problems plaguing our society --- systemic racism and
state-sponsored murder.''

Experts on extremism said the few suspects arrested with overt political
goals fall under the broad category of ``accelerationists,'' groups that
hope to exploit any public unrest to further their own anti-government
goals.

Tim Arango, Mike Baker and John Eligon contributed reporting.

Advertisement

\protect\hyperlink{after-bottom}{Continue reading the main story}

\hypertarget{site-index}{%
\subsection{Site Index}\label{site-index}}

\hypertarget{site-information-navigation}{%
\subsection{Site Information
Navigation}\label{site-information-navigation}}

\begin{itemize}
\tightlist
\item
  \href{https://help.nytimes3xbfgragh.onion/hc/en-us/articles/115014792127-Copyright-notice}{©~2020~The
  New York Times Company}
\end{itemize}

\begin{itemize}
\tightlist
\item
  \href{https://www.nytco.com/}{NYTCo}
\item
  \href{https://help.nytimes3xbfgragh.onion/hc/en-us/articles/115015385887-Contact-Us}{Contact
  Us}
\item
  \href{https://www.nytco.com/careers/}{Work with us}
\item
  \href{https://nytmediakit.com/}{Advertise}
\item
  \href{http://www.tbrandstudio.com/}{T Brand Studio}
\item
  \href{https://www.nytimes3xbfgragh.onion/privacy/cookie-policy\#how-do-i-manage-trackers}{Your
  Ad Choices}
\item
  \href{https://www.nytimes3xbfgragh.onion/privacy}{Privacy}
\item
  \href{https://help.nytimes3xbfgragh.onion/hc/en-us/articles/115014893428-Terms-of-service}{Terms
  of Service}
\item
  \href{https://help.nytimes3xbfgragh.onion/hc/en-us/articles/115014893968-Terms-of-sale}{Terms
  of Sale}
\item
  \href{https://spiderbites.nytimes3xbfgragh.onion}{Site Map}
\item
  \href{https://help.nytimes3xbfgragh.onion/hc/en-us}{Help}
\item
  \href{https://www.nytimes3xbfgragh.onion/subscription?campaignId=37WXW}{Subscriptions}
\end{itemize}
