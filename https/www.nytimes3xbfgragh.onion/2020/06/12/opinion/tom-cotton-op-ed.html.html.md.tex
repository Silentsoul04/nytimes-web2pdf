Sections

SEARCH

\protect\hyperlink{site-content}{Skip to
content}\protect\hyperlink{site-index}{Skip to site index}

\href{https://myaccount.nytimes3xbfgragh.onion/auth/login?response_type=cookie\&client_id=vi}{}

\href{https://www.nytimes3xbfgragh.onion/section/todayspaper}{Today's
Paper}

\href{/section/opinion}{Opinion}\textbar{}Bret Stephens: What The Times
Got Wrong

\url{https://nyti.ms/37m4GZv}

\begin{itemize}
\item
\item
\item
\item
\item
\item
\end{itemize}

Advertisement

\protect\hyperlink{after-top}{Continue reading the main story}

\href{/section/opinion}{Opinion}

Supported by

\protect\hyperlink{after-sponsor}{Continue reading the main story}

\hypertarget{bret-stephens-what-the-times-got-wrong}{%
\section{Bret Stephens: What The Times Got
Wrong}\label{bret-stephens-what-the-times-got-wrong}}

Tom Cotton speaks for a large part of this country. Will we not listen?

\href{https://www.nytimes3xbfgragh.onion/by/bret-stephens}{\includegraphics{https://static01.graylady3jvrrxbe.onion/images/2017/08/27/insider/bretstephens/bretstephens-thumbLarge-v6.png}}

By \href{https://www.nytimes3xbfgragh.onion/by/bret-stephens}{Bret
Stephens}

Opinion Columnist

\begin{itemize}
\item
  June 12, 2020
\item
  \begin{itemize}
  \item
  \item
  \item
  \item
  \item
  \item
  \end{itemize}
\end{itemize}

\includegraphics{https://static01.graylady3jvrrxbe.onion/images/2020/06/12/opinion/12stephens1/merlin_168181410_2494ca03-e379-41bd-9903-35dd212d5105-articleLarge.jpg?quality=75\&auto=webp\&disable=upscale}

\emph{Acting editorial page editor Kathleen Kingsbury}
\href{https://www.nytimes3xbfgragh.onion/2020/06/12/opinion/tom-cotton-new-york-times.html}{\emph{wrote
about the decision}} \emph{to publish our writers' responses to the Tom
Cotton Op-Ed in Friday's edition of our Opinion Today newsletter.}

\begin{center}\rule{0.5\linewidth}{\linethickness}\end{center}

Last week's decision by this newspaper to disavow an Op-Ed by Senator
Tom Cotton is a gift to the enemies of a free press --- free in the
sense of one that doesn't quiver and cave in the face of an outrage mob.
It is a violation of the principles that are supposed to sustain the
profession, particularly our obligation to give readers a picture of the
world as it really is.

And, as the paper dismisses distinguished journalists along with
controversial opinions, it's an invitation to intellectual cowardice.

\href{https://www.nytimes3xbfgragh.onion/2020/06/03/opinion/tom-cotton-protests-military.html}{Start
with the Op-Ed itself}, in which Senator Cotton, an Arkansas Republican,
called on the federal government to deploy active-duty troops to
American cities in the wake of looting and rioting that accompanied
overwhelmingly peaceful protests.

I don't agree with Cotton's view. I know of nobody at The Times who
agrees with it. The Wall Street Journal's editorial page
\href{https://www.wsj.com/articles/dont-call-in-the-troops-11591054305}{doesn't
agree with it}. Ditto for much of the mainstream media, at least its
more liberal precincts.

Then again, isn't this the biggest problem these outlets have faced in
recent years --- being of a single mind on subjects that sharply divide
the nation? Isn't that how we got into trouble in 2016, with our
rock-solid belief that Donald Trump couldn't possibly win?

In the week of the
\href{https://www.nytimes3xbfgragh.onion/2020/06/03/opinion/tom-cotton-protests-military.html}{Op-Ed}'s
publication, an
\href{https://abcnews.go.com/Politics/52-americans-support-deploying-military-control-violent-protests/story?id=71097167}{ABC
News/Ipsos poll} found that 52 percent of Americans favored deploying
troops to help quell violent unrest in American cities. That's not a
political fringe unworthy of consideration. And Tom Cotton isn't some
nobody you'll never hear from again. He has the pulse of his party, the
ear of the president and an eye on higher office. Readers deserve an
unvarnished look at who this man is and what he stands for.

Many critics of the piece's publication think otherwise. The paper's
editors' note said the senator's
\href{https://www.nytimes3xbfgragh.onion/2020/06/03/opinion/tom-cotton-protests-military.html}{Op-Ed}
didn't meet The Times's editorial standards. To which one might ask:
Would the paper have stood by the article if Cotton had made a
\emph{better} case for sending in troops, with stronger legal arguments
and a nicer tone? Or were the piece's supposed flaws a pretext for
achieving the politically desired result by a paper that lost its nerve
in the face of a staff revolt?

A second criticism is that the paper could have examined Cotton's views
without giving him an unmediated platform; that his proposal should have
been evaluated by the news department, not published uncritically in the
Opinion pages; and that his arguments went beyond the moral pale.

But the value of Cotton's
\href{https://www.nytimes3xbfgragh.onion/2020/06/03/opinion/tom-cotton-protests-military.html}{Op-Ed}
doesn't lie in its goodness or rightness. It lies in the fact that
Cotton is a leading spokesman for a major current of public opinion. To
suggest our readers should not have the chance to examine his opinions
for themselves is to patronize them. To say they should look up his
opinions elsewhere --- say, his Twitter feed --- is to betray our
responsibility as a newspaper of record. And to claim that his argument
is too repugnant for publication is to write off half of America --- a
remarkable about-face for a paper that, after 2016, fretted that it was
\href{https://www.nytimes3xbfgragh.onion/2020/01/31/podcasts/the-daily/2020-election.html}{out
of touch with the country we live in}.

The most serious criticism is that publication of the piece puts black
lives at risk, including members of the Times staff.

That's a vital consideration, especially now, and one about which no
responsible publisher can be indifferent. No one can look away from the
deaths of black Americans at the hands of the police, and the overall
rise in reported hate crimes in recent years.

But as important as it is to try to \emph{keep} people safe against
genuine threats, it is not the duty of the paper to make people
\emph{feel} safe by refusing to publish a dismaying
\href{https://www.nytimes3xbfgragh.onion/2020/06/03/opinion/tom-cotton-protests-military.html}{Op-Ed}.
Even if one concedes that Cotton's call to send in the troops poses
potential risks, it poses those risks whether his call appears in these
pages or not. To know Cotton's views is, if nothing else, to be better
armed against them.

The same goes for any other type of knowledge, however unpleasant:
Having more of it is always a source of strength --- a belief that lies
at the core of our profession.

Or, I should say, used to. There is a spirit of ferocious intellectual
intolerance sweeping the country and much of the journalistic
establishment with it. Contrary opinions aren't just wrong but unworthy
of discussion. The range of political views deemed morally unfit for
publication seems to grow ever wider. Arthur Miller once said a good
newspaper is ``a nation talking to itself.'' What kind of paper will The
Times be if half the nation doesn't get to be even an occasional part of
that conversation?

All this is a tragedy. We have an obligation as journalists to be
rigorous in fact and argument. We also have an obligation to keep
undeniably hateful ideas, like Holocaust denial or racism, out of the
editorial pages. But serious journalism, complete with a vigorous
exchange of ideas, cannot survive in an atmosphere in which modest
intellectual risk-taking or minor offenses against new ideological
orthodoxies risk professional ruin.

It's also an irony. Who, after all, has gained the most from the turmoil
at The Times? That would be Tom Cotton, who first got the benefit of a
public furor that helped make his piece the most read
\href{https://www.nytimes3xbfgragh.onion/2020/06/03/opinion/tom-cotton-protests-military.html}{Op-Ed}
in The Times last week --- and then got to pose as a tribune of free
speech against the censorious leftists and stampeded editors at the
``Fake News.''

If that's a victory for Cotton's ideological opponents, I wonder what
defeat looks like.

\emph{The Times is committed to publishing}
\href{https://www.nytimes3xbfgragh.onion/2019/01/31/opinion/letters/letters-to-editor-new-york-times-women.html}{\emph{a
diversity of letters}} \emph{to the editor. We'd like to hear what you
think about this or any of our articles. Here are some}
\href{https://help.nytimes3xbfgragh.onion/hc/en-us/articles/115014925288-How-to-submit-a-letter-to-the-editor}{\emph{tips}}\emph{.
And here's our email:}
\href{mailto:letters@NYTimes.com}{\emph{letters@NYTimes.com}}\emph{.}

\emph{Follow The New York Times Opinion section on}
\href{https://www.facebookcorewwwi.onion/nytopinion}{\emph{Facebook}}\emph{,}
\href{http://twitter.com/NYTOpinion}{\emph{Twitter (@NYTopinion)}}
\emph{and}
\href{https://www.instagram.com/nytopinion/}{\emph{Instagram}}\emph{.}

Advertisement

\protect\hyperlink{after-bottom}{Continue reading the main story}

\hypertarget{site-index}{%
\subsection{Site Index}\label{site-index}}

\hypertarget{site-information-navigation}{%
\subsection{Site Information
Navigation}\label{site-information-navigation}}

\begin{itemize}
\tightlist
\item
  \href{https://help.nytimes3xbfgragh.onion/hc/en-us/articles/115014792127-Copyright-notice}{©~2020~The
  New York Times Company}
\end{itemize}

\begin{itemize}
\tightlist
\item
  \href{https://www.nytco.com/}{NYTCo}
\item
  \href{https://help.nytimes3xbfgragh.onion/hc/en-us/articles/115015385887-Contact-Us}{Contact
  Us}
\item
  \href{https://www.nytco.com/careers/}{Work with us}
\item
  \href{https://nytmediakit.com/}{Advertise}
\item
  \href{http://www.tbrandstudio.com/}{T Brand Studio}
\item
  \href{https://www.nytimes3xbfgragh.onion/privacy/cookie-policy\#how-do-i-manage-trackers}{Your
  Ad Choices}
\item
  \href{https://www.nytimes3xbfgragh.onion/privacy}{Privacy}
\item
  \href{https://help.nytimes3xbfgragh.onion/hc/en-us/articles/115014893428-Terms-of-service}{Terms
  of Service}
\item
  \href{https://help.nytimes3xbfgragh.onion/hc/en-us/articles/115014893968-Terms-of-sale}{Terms
  of Sale}
\item
  \href{https://spiderbites.nytimes3xbfgragh.onion}{Site Map}
\item
  \href{https://help.nytimes3xbfgragh.onion/hc/en-us}{Help}
\item
  \href{https://www.nytimes3xbfgragh.onion/subscription?campaignId=37WXW}{Subscriptions}
\end{itemize}
