Sections

SEARCH

\protect\hyperlink{site-content}{Skip to
content}\protect\hyperlink{site-index}{Skip to site index}

\href{https://www.nytimes3xbfgragh.onion/section/opinion/sunday}{Sunday
Review}

\href{https://myaccount.nytimes3xbfgragh.onion/auth/login?response_type=cookie\&client_id=vi}{}

\href{https://www.nytimes3xbfgragh.onion/section/todayspaper}{Today's
Paper}

\href{/section/opinion/sunday}{Sunday Review}\textbar{}What the Pandemic
Reveals About the Male Ego

\url{https://nyti.ms/2AyQgJh}

\begin{itemize}
\item
\item
\item
\item
\item
\item
\end{itemize}

Advertisement

\protect\hyperlink{after-top}{Continue reading the main story}

\href{/section/opinion}{Opinion}

Supported by

\protect\hyperlink{after-sponsor}{Continue reading the main story}

\hypertarget{what-the-pandemic-reveals-about-the-male-ego}{%
\section{What the Pandemic Reveals About the Male
Ego}\label{what-the-pandemic-reveals-about-the-male-ego}}

Why are the rates of coronavirus deaths far lower in many female-led
countries?

\href{https://www.nytimes3xbfgragh.onion/column/nicholas-kristof}{\includegraphics{https://static01.graylady3jvrrxbe.onion/images/2018/04/03/opinion/nicholas-kristof/nicholas-kristof-thumbLarge-v2.png}}

By
\href{https://www.nytimes3xbfgragh.onion/column/nicholas-kristof}{Nicholas
Kristof}

Opinion Columnist

\begin{itemize}
\item
  June 13, 2020
\item
  \begin{itemize}
  \item
  \item
  \item
  \item
  \item
  \item
  \end{itemize}
\end{itemize}

\includegraphics{https://static01.graylady3jvrrxbe.onion/images/2020/06/15/opinion/sunday/15Kristof2/merlin_171421221_ec3ba2ca-4f93-47c8-9727-f5ad9ab7f258-articleLarge.jpg?quality=75\&auto=webp\&disable=upscale}

Are female leaders better at fighting a pandemic?

I compiled
\href{https://ourworldindata.org/grapher/total-covid-deaths-per-million}{death
rates} from the coronavirus for 21 countries around the world, 13 led by
men and eight by women. The male-led countries suffered an average of
214 coronavirus-related deaths per million inhabitants. Those led by
women lost only one-fifth as many, 36 per million.

If the United States had the coronavirus death rate of the average
female-led country, 102,000 American lives would have been saved out of
the
\href{https://www.nytimes3xbfgragh.onion/interactive/2020/us/coronavirus-us-cases.html}{114,000
lost}.

``Countries led by women do seem to be particularly successful in
fighting the coronavirus,'' noted Anne W. Rimoin, an epidemiologist at
U.C.L.A. ``New Zealand, Denmark, Finland, Germany, Iceland, Norway have
done so well perhaps due to the leadership and management styles
attributed to their female leaders.''

Let's start by acknowledging that there have been plenty of wretched
female leaders over the years. Indeed, according to research I once did
for
\href{https://www.penguinrandomhouse.com/books/95840/half-the-sky-by-nicholas-d-kristof-and-sheryl-wudunn/}{a
book}, female leaders around the world haven't been clearly better than
male counterparts even at improving girls' education or reducing
maternal mortality.

There has been solid research that it makes a difference to have more
women on boards and in grass-roots positions, but evidence that they
make better presidents or prime ministers has been lacking ---
\href{https://www.nytimes3xbfgragh.onion/2020/05/15/world/coronavirus-women-leaders.html}{until
Covid-19 came along}.

It's not that the leaders who best managed the virus were all women. But
those who bungled the response were \emph{all} men, and mostly a
particular type: authoritarian, vainglorious and blustering. Think of
Boris Johnson in Britain, Jair Bolsonaro in Brazil, Ayatollah Ali
Khamenei in Iran and Donald Trump in the United States.

Virtually every country that has experienced coronavirus mortality at a
rate of more than 150 per million inhabitants is male-led.

``I don't think it's a coincidence that some of the best-run places have
been run by women: New Zealand, Germany, Taiwan,'' mused Susan Rice, who
was national security adviser under President Barack Obama. ``And where
we've seen things go most badly wrong --- the U.S., Brazil, Russia, the
U.K. --- it's a lot of male ego and bluster.''

I think the divergence has a great deal to do with that ego and bluster.

``We often joke that men drivers never ask for directions,'' observed
Dr. Ezekiel Emanuel of the University of Pennsylvania. ``I actually
think there's something to that also in terms of women's leadership, in
terms of recognizing expertise and asking experts for advice, and men
sort of barreling ahead like they got it.''

He has a point. Those leaders who handled the virus best were those who
humbly consulted public health experts and acted quickly, and many were
women; in contrast, male authoritarians who botched the response were
suspicious of experts and too full of themselves.

``I really get it,'' Trump
\href{https://www.whitehouse.gov/briefings-statements/remarks-president-trump-tour-centers-disease-control-prevention-atlanta-ga/}{said}
when he visited the Centers for Disease Control and Prevention in March.
Surrounded by medical experts, he added, ``Maybe I have natural
ability,'' and he wondered aloud if he should have become a scientist.

(Given that Trump
\href{https://www.nytimes3xbfgragh.onion/2020/03/15/opinion/trump-coronavirus.html}{said}
in January that Covid-19 was ``totally under control,'' he has his
answer. And peer review might not have been kind to his ideas about
\href{https://www.nytimes3xbfgragh.onion/2020/04/24/us/politics/trump-inject-disinfectant-bleach-coronavirus.html}{bleach}.)

Image

Jacinda Ardern, prime minister of New Zealand.Credit...Pool photo by
Mark Mitchell

Image

Angela Merkel, Germany's chancellor.Credit...Pool photo by Andreas Gora

While women have generally outshone men as international leaders, that
does not seem true within the United States. Some female governors have
done better, others worse, so there isn't an obvious gender gap at home.

It's also possible that this isn't about female leaders but about the
kind of country that chooses a woman to lead it.

Companies with more female executives on average
\href{https://www.nytimes3xbfgragh.onion/2013/10/24/opinion/kristof-twitter-women-power.html}{perform
better} than those with fewer women, but analysts think that the reason
isn't just the brilliance of women leaders. Rather, companies that are
culturally open to having senior women are also more willing to embrace
other innovations, and it may be this innovative spirit that leads to
higher profitability. Likewise, countries willing to elect female prime
ministers may be those more inclined to listen to epidemiologists.

Yet I think that there's also a difference in the leadership itself.

``Women lead often in a very different style from men,'' said Margot
Wallstrom, a former Swedish foreign minister, citing examples from
Norway, Germany and New Zealand of women with low-key, inclusive and
evidence-based leadership.

Wallstrom also noted that public health is a traditional ``home turf''
concern for many women leaders. Grant Miller, an expert in health
economics at Stanford University,
\href{https://fsi.stanford.edu/news/when-it-comes-health-men-have-long-benefited-women-power}{found}
that as states, one by one, granted the vote to women in the late 19th
and early 20th centuries, those states then also invested more in
sanitation and public health --- saving some 20,000 children's lives a
year. Boys were thus huge beneficiaries of women's suffrage.

One trap for female politicians is that brashness can be effective for
male candidates, but
\href{https://www.nytimes3xbfgragh.onion/2008/02/10/opinion/10kristof.html}{researchers
find} that male and female voters alike are turned off by women who seem
self-promotional. That forces women in politics to master the art of
communicating effectively in a low-key way --- just what's needed in a
pandemic.

``Perhaps the skills that have led them to reach the top,'' said Rimoin,
the U.C.L.A. epidemiologist, ``are the same skills that are currently
needed to bring a country together.''

\emph{The Times is committed to publishing}
\href{https://www.nytimes3xbfgragh.onion/2019/01/31/opinion/letters/letters-to-editor-new-york-times-women.html}{\emph{a
diversity of letters}} \emph{to the editor. We'd like to hear what you
think about this or any of our articles. Here are some}
\href{https://help.nytimes3xbfgragh.onion/hc/en-us/articles/115014925288-How-to-submit-a-letter-to-the-editor}{\emph{tips}}\emph{.
And here's our email:}
\href{mailto:letters@NYTimes.com}{\emph{letters@NYTimes.com}}\emph{.}

Advertisement

\protect\hyperlink{after-bottom}{Continue reading the main story}

\hypertarget{site-index}{%
\subsection{Site Index}\label{site-index}}

\hypertarget{site-information-navigation}{%
\subsection{Site Information
Navigation}\label{site-information-navigation}}

\begin{itemize}
\tightlist
\item
  \href{https://help.nytimes3xbfgragh.onion/hc/en-us/articles/115014792127-Copyright-notice}{©~2020~The
  New York Times Company}
\end{itemize}

\begin{itemize}
\tightlist
\item
  \href{https://www.nytco.com/}{NYTCo}
\item
  \href{https://help.nytimes3xbfgragh.onion/hc/en-us/articles/115015385887-Contact-Us}{Contact
  Us}
\item
  \href{https://www.nytco.com/careers/}{Work with us}
\item
  \href{https://nytmediakit.com/}{Advertise}
\item
  \href{http://www.tbrandstudio.com/}{T Brand Studio}
\item
  \href{https://www.nytimes3xbfgragh.onion/privacy/cookie-policy\#how-do-i-manage-trackers}{Your
  Ad Choices}
\item
  \href{https://www.nytimes3xbfgragh.onion/privacy}{Privacy}
\item
  \href{https://help.nytimes3xbfgragh.onion/hc/en-us/articles/115014893428-Terms-of-service}{Terms
  of Service}
\item
  \href{https://help.nytimes3xbfgragh.onion/hc/en-us/articles/115014893968-Terms-of-sale}{Terms
  of Sale}
\item
  \href{https://spiderbites.nytimes3xbfgragh.onion}{Site Map}
\item
  \href{https://help.nytimes3xbfgragh.onion/hc/en-us}{Help}
\item
  \href{https://www.nytimes3xbfgragh.onion/subscription?campaignId=37WXW}{Subscriptions}
\end{itemize}
