Sections

SEARCH

\protect\hyperlink{site-content}{Skip to
content}\protect\hyperlink{site-index}{Skip to site index}

\href{https://www.nytimes3xbfgragh.onion/section/nyregion}{New York}

\href{https://myaccount.nytimes3xbfgragh.onion/auth/login?response_type=cookie\&client_id=vi}{}

\href{https://www.nytimes3xbfgragh.onion/section/todayspaper}{Today's
Paper}

\href{/section/nyregion}{New York}\textbar{}How 2 Lives Collided in
Central Park, Rattling the Nation

\href{https://nyti.ms/3hpStHH}{https://nyti.ms/3hpStHH}

\begin{itemize}
\item
\item
\item
\item
\item
\item
\end{itemize}

\href{https://www.nytimes3xbfgragh.onion/news-event/george-floyd-protests-minneapolis-new-york-los-angeles?action=click\&pgtype=Article\&state=default\&region=TOP_BANNER\&context=storylines_menu}{Race
and America}

\begin{itemize}
\tightlist
\item
  \href{https://www.nytimes3xbfgragh.onion/2020/07/26/us/protests-portland-seattle-trump.html?action=click\&pgtype=Article\&state=default\&region=TOP_BANNER\&context=storylines_menu}{Protesters
  Return to Other Cities}
\item
  \href{https://www.nytimes3xbfgragh.onion/2020/07/24/us/portland-oregon-protests-white-race.html?action=click\&pgtype=Article\&state=default\&region=TOP_BANNER\&context=storylines_menu}{Portland
  at the Center}
\item
  \href{https://www.nytimes3xbfgragh.onion/2020/07/23/podcasts/the-daily/portland-protests.html?action=click\&pgtype=Article\&state=default\&region=TOP_BANNER\&context=storylines_menu}{Podcast:
  Showdown in Portland}
\item
  \href{https://www.nytimes3xbfgragh.onion/interactive/2020/07/16/us/black-lives-matter-protests-louisville-breonna-taylor.html?action=click\&pgtype=Article\&state=default\&region=TOP_BANNER\&context=storylines_menu}{45
  Days in Louisville}
\end{itemize}

Advertisement

\protect\hyperlink{after-top}{Continue reading the main story}

Supported by

\protect\hyperlink{after-sponsor}{Continue reading the main story}

\hypertarget{how-2-lives-collided-in-central-park-rattling-the-nation}{%
\section{How 2 Lives Collided in Central Park, Rattling the
Nation}\label{how-2-lives-collided-in-central-park-rattling-the-nation}}

The inside story of the Black birder and the white woman who called the
police on him. Their encounter stirred wrenching conversations about
racism and white privilege.

\includegraphics{https://static01.graylady3jvrrxbe.onion/images/2020/06/16/nyregion/00CENTRALPARK-TWOLIVES-top/00CENTRALPARK-TWOLIVES-top-articleLarge.jpg?quality=75\&auto=webp\&disable=upscale}

\href{https://www.nytimes3xbfgragh.onion/by/sarah-maslin-nir}{\includegraphics{https://static01.graylady3jvrrxbe.onion/images/2018/06/13/multimedia/author-sarah-maslin-nir/author-sarah-maslin-nir-thumbLarge.jpg}}

By \href{https://www.nytimes3xbfgragh.onion/by/sarah-maslin-nir}{Sarah
Maslin Nir}

\begin{itemize}
\item
  Published June 14, 2020Updated July 7, 2020
\item
  \begin{itemize}
  \item
  \item
  \item
  \item
  \item
  \item
  \end{itemize}
\end{itemize}

\href{https://www.nytimes3xbfgragh.onion/es/2020/06/15/espanol/central-park-racismo-amy-cooper.html}{Leer
en
español}\href{https://cn.nytimes3xbfgragh.onion/usa/20200617/central-park-amy-cooper-christian-racism/}{阅读简体中文版}\href{https://cn.nytimes3xbfgragh.onion/usa/20200617/central-park-amy-cooper-christian-racism/zh-h}{閱讀繁體中文版}

Christian Cooper began his Memorial Day like most of his May mornings,
searching for Blackburnian warblers, scarlet tanagers and other
songbirds that wing their way into Central Park.

In his Lower East Side apartment, Mr. Cooper, 57, slung on his prize
possession, his Swarovski binoculars --- a pricey 50th birthday present
from his late father. Leaving his boyfriend asleep in bed, he biked
three miles away, to the semi-wild section of the park, the Ramble.

Around the same time,
\href{https://www.nytimes3xbfgragh.onion/2020/07/06/nyregion/amy-cooper-false-report-charge.html}{Amy
Cooper}, 40, who is not related to Christian Cooper, left her apartment
on the Upper West Side at the edge of the Hudson River. She was with her
dog, Henry, a blond cocker spaniel she had rescued and whose romps
around the city she chronicled on a dedicated Instagram account.

It was in the Ramble that the two Coopers' lives collided, an encounter
that was brief but would reverberate in New York City and beyond,
stirring anguished conversations about racism and hypocrisy in one of
the nation's most progressive cities.

Only a few hours later, George Floyd would be killed in Minneapolis when
a police officer pinned Mr. Floyd's neck under his knee. The two
Memorial Day incidents captured on video two facets of entrenched racism
Black people experience: one the horrors of police brutality, the other
the routine humiliations and threats in daily life.

Just before 8 a.m., Mr. Cooper was startled from his quiet birding by
Ms. Cooper, who was loudly calling after her dog, he said. He asked her
to leash Henry, as the park rules required. She refused.

They exchanged words, and as he recorded on his phone, she threatened to
report that ``an African-American man is threatening my life,'' a false
accusation. Then as Mr. Cooper continued to film, she called 911.

The video clip shows that before and during the 911 call,
\href{https://www.nytimes3xbfgragh.onion/2020/05/29/nyregion/Amy-Cooper-Central-Park-racism.html}{she
referred to Mr. Cooper as ``African-American,'' three times}. Mr.
Cooper's sister later posted the clip to Twitter, where it has been
\href{https://twitter.com/melodyMcooper/status/1264965252866641920}{viewed
more than 40 million times}.

Their lives have gone in drastically different directions since then.
\href{https://www.nytimes3xbfgragh.onion/2020/05/26/nyregion/amy-cooper-dog-central-park.html}{Amy
Cooper was fired from her high-level finance job}, temporarily
surrendered her dog and has been vilified as the embodiment of racism
and white privilege. Christian Cooper has appeared on ``The View'' and
has become such a celebrated figure that a congressional candidate in
the Bronx publicized
\href{https://www.nydailynews.com/news/politics/ny-chris-cooper-ritchie-torres-endorsement-20200607-3uualrgirrbsfpuvgxhue2pu4u-story.html}{Mr.
Cooper's endorsement}.

His experience has also been highlighted by prominent Black politicians,
from former President Barack Obama to the city's public advocate,
Jumaane Williams, during the protests over Mr. Floyd's death.

Mr. Cooper said the encounter touched a nerve and evoked a long history
of racism. ``It's not about her,'' he said in an interview.

``What she did was tap into a deep vein of racial bias,'' Mr. Cooper
added. ``And it is that deep vein of racial bias that keeps cropping up
that led to much more serious events and much more serious repercussions
than my little dust-up with Amy Cooper --- the murder of George Floyd,
the murder of
\href{https://www.nytimes3xbfgragh.onion/article/ahmaud-arbery-shooting-georgia.html}{Ahmaud
Arbery}\emph{,} ****** and before that
\href{https://www.nytimes3xbfgragh.onion/topic/person/amadou-diallo}{Amadou
Diallo} and
\href{https://www.nytimes3xbfgragh.onion/2000/03/17/nyregion/undercover-police-in-manhattan-kill-an-unarmed-man-in-a-scuffle.html}{Patrick
Dorismond} and
\href{https://www.nytimes3xbfgragh.onion/2015/06/14/nyregion/eric-garner-police-chokehold-staten-island.html}{Eric
Garner} and
\href{https://www.nytimes3xbfgragh.onion/2015/12/30/magazine/tamir-rice-and-the-color-of-fear.html}{Tamir
Rice}.''

Before that day, Mr. Cooper and Ms. Cooper were both successful
professionals with prestigious degrees and a love of animals, which drew
them to that haven in the city, Central Park. But a deeper look at their
lives shows that their encounter was to some extent a telling reflection
of their personalities.

Mr. Cooper warmly embraces serious nerdiness, memorizing bird song and
learning bits of the Klingon language from Star Trek. But he also has an
activist's bent, bristling at society's injustices.

He once set up his own nonprofit group to help elect Democrats, and he
used his love of comic books to break barriers by creating one of the
first gay Star Trek characters.

Among Central Park's birders, he is considered to be a mentor --- even
to those who disapprove of his preferred tactic to protect the birds'
sanctuary: He deploys treats to tempt unleashed dogs so that their
owners tether them. (During the Central Park encounter, he pulled out
one such treat for Ms. Cooper's dog.)

Ms. Cooper, an immigrant from Canada, can be sensitive and caring,
according to her friends, but also seems to have a more contentious
side. Neighbors said she had a tendency to get into personal disputes.

Her personal life once spilled into court. A few years ago, according to
a lawsuit she filed, she had become involved with a married man and had
lent him \$65,000. When he did not leave his wife for her, she filed the
suit in Manhattan to get back the money, before settling.

\includegraphics{https://static01.graylady3jvrrxbe.onion/images/2020/05/26/nyregion/26Central-Park1/26Central-Park1-videoSixteenByNineJumbo1600.jpg}

Though Ms. Cooper issued an apology to Mr. Cooper after their encounter,
she has not since spoken publicly. The authorities are reviewing whether
she can be charged with filing a false police report.

Ms. Cooper did not respond to multiple requests for comment.

``I reacted emotionally and made false assumptions about his intentions
when, in fact, I was the one who was acting inappropriately by not
having my dog on a leash,''
\href{https://www.prnewswire.com/news-releases/statement-from-amy-cooper-on-central-park-incident-301065492.html}{she
wrote in her apology}.

``I am well aware of the pain that misassumptions and insensitive
statements about race cause,'' she said. ``I would never have imagined
that I would be involved in the type of incident that occurred.''

Since the video of their encounter went viral,
\href{https://www.nytimes3xbfgragh.onion/2020/05/27/nyregion/amy-cooper-christian-central-park-video.html}{Mr.
Cooper has expressed some ambivalence} about what has happened to Ms.
Cooper's life.

``I'm not excusing the racism, but I don't know if her life needed to be
torn apart,'' he said a day after the video went viral.

``There are certain dark societal impulses that she, as a white woman,
facing in a conflict with a Black man, that she thought she could
marshal to her advantage,'' Mr. Cooper said. ``She went there.''

\hypertarget{the-dog-walker}{%
\subsection{The dog walker}\label{the-dog-walker}}

\includegraphics{https://static01.graylady3jvrrxbe.onion/images/2020/06/10/nyregion/00CENTRALPARK-TWOLIVES-1/merlin_173341023_6aa88abb-d085-4b0a-8407-1f0b2f24528e-articleLarge.jpg?quality=75\&auto=webp\&disable=upscale}

Ms. Cooper's building on the Upper West Side was once known as Trump
Place, but the name was removed in a
\href{https://www.nytimes3xbfgragh.onion/2016/10/25/nyregion/trump-place-buildings-name-change.html}{symbolic
action against the president} by liberal residents.

Around the building, Ms. Cooper was known for her attachment to her
cocker spaniel, Henry. She was described as a constant presence on
morning walks and at doggy birthday parties.

``From what I saw, she was very devoted to her animals,'' said Maria
Meade, 60, who lives in a nearby building. ``The only thing I'll tell
you is she never spoke directly to a person. She always spoke through
her dog, and in a baby voice. It was really bizarre.''

It is not possible to determine to what extent recollections of Ms.
Cooper's behavior are now shaded by news of her encounter in Central
Park. Still, some residents said they held her at arm's length because
of what they described as her combative behavior with other dog walkers
and the building staff.

Another neighbor, Marisol De Leon, 40, said Ms. Cooper frequently walked
Henry unleashed, and became irate when told not to. ``There was a sense
of entitlement,'' Ms. De Leon said.

Alison Faircloth, 37, a neighbor and dog owner, recalled that last
winter, she came upon Ms. Cooper on the verge of tears outside the
building's lobby. A doorman had cursed at her for no reason, Ms. Cooper
told her. Ms. Cooper vowed to get the doorman fired, Ms. Faircloth said.

But when Ms. Faircloth asked the doorman what had happened, he told her
that Ms. Cooper had complained about a broken elevator, then cursed at
him after she barged into a security booth and had to be removed by a
guard.

``There's always a narrative from her about someone who has done her
wrong,'' Ms. Faircloth said.

The building's management declined to comment.

Before arriving in New York Ms. Cooper lived in Ontario, where she
attended the University of Waterloo. She obtained a master's degree at
the University of Chicago Booth School of Business, according to her
résumé.

She has worked at Lehman Brothers, Citigroup and A.I.G., according to
her résumé. She spent the past five years at Franklin Templeton, rising
to become a vice president of insurance portfolio management, making
investments for insurance companies.

It was on that corporate ladder that she met Martin Priest, a married
colleague at Lehman Brothers, where, her résumé said, she worked from
2005 to 2008.

In a lawsuit filed in 2015, when she was no longer dating Mr. Priest,
Ms. Cooper sought repayment of \$65,000. She said she had given him the
money to help speed his divorce and pay another woman he was involved
with to abort her pregnancy, according to court records.

In the lawsuit, Ms. Cooper said Mr. Priest preyed on her emotions to get
the money, promising it would help them to be together.

Instead, she said she discovered that his wife, Tianna, whom he was
divorcing, was pregnant --- and Mr. Priest was planning to marry a third
woman, who was also pregnant, the lawsuit said.

``She was naïve, devastated, heartbroken,'' said a person involved in
the case, who spoke on condition of anonymity because the settlement is
private.

In an interview, Mr. Priest denied that he'd had a romantic relationship
with Ms. Cooper, though he admitted to borrowing the money. He called
her a ``stalker'' who fictionalized their relationship, then erupted
when it did not go her way.

In an unusual twist, since the lawsuit, Ms. Cooper has developed a close
friendship with Tianna Priest, who is now divorced from Martin Priest,
after Ms. Cooper exposed his infidelity to her. Ms. Cooper and Ms.
Priest now spend holidays together.

Ms. Priest declined to comment on the Central Park encounter, but
praised Ms. Cooper's professionalism.

``Work, work, work, work, work --- she's a workaholic,'' Ms. Priest
said. ``She loves numbers, so she gets it and she's good at it.''

To Ms. Priest's family, Ms. Cooper is a hero, who saved Ms. Priest from
a toxic marriage, said Tom Selby, Ms. Priest's father. He blamed his
former son-in-law: ``Amy is just another one of his victims,'' Mr. Selby
said.

A day after the video went viral, internet commenters noted that the
Instagram account dedicated to Henry documented injuries that the dog
had suffered. That evening, under pressure, Ms. Cooper returned the dog
to Abandoned Angels Cocker Spaniel Rescue.

On June 3, the organization said it had given Henry back to Ms. Cooper
at her request after its veterinarian found that the dog was in good
health.

\hypertarget{the-bird-watcher}{%
\subsection{The bird-watcher}\label{the-bird-watcher}}

Image

Credit...Brittainy Newman/The New York Times

On a family road trip when he was 11 years old, Christian Cooper was
given a copy of ``The Birds of North America'' to keep him entertained.
By the end of the excursion in a Volkswagen bus with his sister, Melody,
and their parents, two schoolteachers from Long Island, he had memorized
the entire text, he said, and was identifying the birds that flew by.

He was equally enthralled by comic books, which he parlayed into a
career after he graduated from Harvard with a degree in political
science.

```The X-Men' was a perfect parable for the gay experience,'' he
\href{https://www.wired.com/1998/12/can-miss-thang-save-earth/?fbclid=IwAR3iEjyOmx-OGot3VBzWzcUVRvey1E7nKano-bHvLn4wi_QHX5zfWK28B4E}{told
Wired Magazine} in an interview in 1998. ``The X-Men looked like
everyone else, but they learned a deep secret in adolescence that made
them different.''

In the late 1980s, Mr. Cooper served on the board of directors of GLAAD,
formerly the Gay and Lesbian Alliance Against Defamation, and set up his
own political action committee to support Democrats for the New York
Senate, according to a biography on
``\href{http://gayusatv.org/chris-cooper/?fbclid=IwAR3vZj4cMWhBjhY8aYthS2C1yapBYQiazkpQrwUs7R0E-iqyK8GWrM18E3w}{Gay
USA},'' a televised news show about gay issues, which Mr. Cooper
occasionally hosts.

In 1998 he launched ``Queer Nation,'' a pioneering gay web comic that
envisioned L.G.B.T.Q. superheroes fighting the scourge of a right-wing
world order. It was partly inspired by his parents, he told Wired, who
were active in the civil rights movement.

Mr. Cooper is now a senior editor at Health Science Communications, a
public relations agency for the health care industry. But his résumé
does not diminish the universality to his experience as a Black man,
some have pointed out.

``I have no doubt that if the police had showed up in the Ramble, a
wooded area of the park where Chris had gone bird watching, my brother's
Ivy League degree and impressive résumé would not have protected him,''
his sister,
\href{https://www.nytimes3xbfgragh.onion/2020/05/31/opinion/chris-cooper-central-park.html?smtyp=cur\&smid=tw-nytimes}{Melody,
wrote in an Op-Ed} in The New York Times ** on May 31.

\href{https://mariejavins.blogspot.com/2020/05/she-went-there.html}{In a
series of posts}, Marie Javins, a former colleague, tried to make sense
of what happened.

``If you'd asked me `What do you think Christian would be famous for,'
I'd have guessed for something he'd written, a science-fiction book or
maybe the Star Trek comic he used to write where he introduced the first
gay character in the history of Star Trek,'' she wrote.

She said she never would have expected it would be because a white woman
``used the term `African-American man' as a weapon.''

\hypertarget{the-ramble}{%
\subsection{The Ramble}\label{the-ramble}}

Image

Credit...Brittainy Newman/The New York Times

Above all, the constants in Mr. Cooper's life have been the thrushes,
sparrows and swallows of Central Park.

Just beside the 79th Street Transverse, the semi-wild part of the park
called the Ramble is a haven this time of year for migrating birds.

There, special rules to protect them --- including that dogs be leashed
at all times --- often render it a microcosm of the city's tensions:
between nature and urban life; between solitude and socializing.

Mr. Cooper is a well-known presence there, a mentor to neophyte birders
who carries gravitas as a member of the board of the New York City
Audubon Society.

``He has a method of dealing with dogs. He'll say, `Can you please leash
your dog?' and if they refuse he starts giving the dogs treats,'' said
Zach McDargh, 29, a research scientist. ``Dog owners hate that.''

At about 8 a.m. on Memorial Day, as Henry bounded through the Ramble,
and his owner refused to leash him as was required and as she was asked,
Mr. Cooper fished in his pockets for those treats.

``Look, if you're going to do what you want, I'm going to do what I
want, but you're not going to like it,'' he recalled saying,
\href{https://m.facebookcorewwwi.onion/story.php?story_fbid=10158742137255229\&id=671885228}{in
his Facebook post about the incident}, before he took out his phone to
film the scofflaw behavior.

The video recorded Ms. Cooper as she lunged at him, then threatened to
call 911 and claimed that he was threatening her life.

Officers responded to a report of an assault that never happened. The
police later characterized it as a ``verbal dispute.''

``I was conscious of the fact I was now a target of the cops, and by
target, I don't mean that they are going to necessarily kill me,'' Mr.
Cooper said later. ``That's never a comfortable feeling when you're
Black and under suspicion.''

That morning, aware that the police would most likely be arriving
shortly, Mr. Cooper recalled his next steps clearly.

He picked back up his Swarovski binoculars that hung around his neck and
continued to look for splashes of feathers atop the London plane trees.

``I was adamant about that,'' Mr. Cooper said. ``I birded my way out as
I normally do.''

Nate Schweber and Alex Traub contributed reporting. Kitty Bennett
provided research.

Advertisement

\protect\hyperlink{after-bottom}{Continue reading the main story}

\hypertarget{site-index}{%
\subsection{Site Index}\label{site-index}}

\hypertarget{site-information-navigation}{%
\subsection{Site Information
Navigation}\label{site-information-navigation}}

\begin{itemize}
\tightlist
\item
  \href{https://help.nytimes3xbfgragh.onion/hc/en-us/articles/115014792127-Copyright-notice}{©~2020~The
  New York Times Company}
\end{itemize}

\begin{itemize}
\tightlist
\item
  \href{https://www.nytco.com/}{NYTCo}
\item
  \href{https://help.nytimes3xbfgragh.onion/hc/en-us/articles/115015385887-Contact-Us}{Contact
  Us}
\item
  \href{https://www.nytco.com/careers/}{Work with us}
\item
  \href{https://nytmediakit.com/}{Advertise}
\item
  \href{http://www.tbrandstudio.com/}{T Brand Studio}
\item
  \href{https://www.nytimes3xbfgragh.onion/privacy/cookie-policy\#how-do-i-manage-trackers}{Your
  Ad Choices}
\item
  \href{https://www.nytimes3xbfgragh.onion/privacy}{Privacy}
\item
  \href{https://help.nytimes3xbfgragh.onion/hc/en-us/articles/115014893428-Terms-of-service}{Terms
  of Service}
\item
  \href{https://help.nytimes3xbfgragh.onion/hc/en-us/articles/115014893968-Terms-of-sale}{Terms
  of Sale}
\item
  \href{https://spiderbites.nytimes3xbfgragh.onion}{Site Map}
\item
  \href{https://help.nytimes3xbfgragh.onion/hc/en-us}{Help}
\item
  \href{https://www.nytimes3xbfgragh.onion/subscription?campaignId=37WXW}{Subscriptions}
\end{itemize}
