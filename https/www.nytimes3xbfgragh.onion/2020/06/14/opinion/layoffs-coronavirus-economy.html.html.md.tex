Sections

SEARCH

\protect\hyperlink{site-content}{Skip to
content}\protect\hyperlink{site-index}{Skip to site index}

\href{https://myaccount.nytimes3xbfgragh.onion/auth/login?response_type=cookie\&client_id=vi}{}

\href{https://www.nytimes3xbfgragh.onion/section/todayspaper}{Today's
Paper}

\href{/section/opinion}{Opinion}\textbar{}More People Will Be Fired in
the Pandemic. Let's Talk About It.

\url{https://nyti.ms/2ADgbj3}

\begin{itemize}
\item
\item
\item
\item
\item
\item
\end{itemize}

Advertisement

\protect\hyperlink{after-top}{Continue reading the main story}

\href{/section/opinion}{Opinion}

Supported by

\protect\hyperlink{after-sponsor}{Continue reading the main story}

\hypertarget{more-people-will-be-fired-in-the-pandemic-lets-talk-about-it}{%
\section{More People Will Be Fired in the Pandemic. Let's Talk About
It.}\label{more-people-will-be-fired-in-the-pandemic-lets-talk-about-it}}

For starters, don't do it on Zoom. And consider not doing it at all.

\href{https://www.nytimes3xbfgragh.onion/by/jennifer-senior}{\includegraphics{https://static01.graylady3jvrrxbe.onion/images/2018/10/26/opinion/jennifer-senior/jennifer-senior-thumbLarge.png}}

By \href{https://www.nytimes3xbfgragh.onion/by/jennifer-senior}{Jennifer
Senior}

Opinion columnist

\begin{itemize}
\item
  June 14, 2020
\item
  \begin{itemize}
  \item
  \item
  \item
  \item
  \item
  \item
  \end{itemize}
\end{itemize}

\includegraphics{https://static01.graylady3jvrrxbe.onion/images/2020/06/15/opinion/15Senior/merlin_168730191_6d2050f8-1226-4039-a1ef-0fabf814cdf7-articleLarge.jpg?quality=75\&auto=webp\&disable=upscale}

\hypertarget{listen-to-this-op-ed}{%
\subsubsection{Listen to This Op-Ed}\label{listen-to-this-op-ed}}

Audio Recording by Audm

\emph{To hear more audio stories from publishers like The New York
Times,
download}\href{https://www.audm.com/?utm_source=nytmag\&utm_medium=embed\&utm_campaign=left_behind_draper}{**}\href{https://www.audm.com/?utm_source=nytopinion\&utm_medium=embed\&utm_campaign=more_fired_pandemic}{\emph{Audm
for iPhone or Android}}\emph{.}

Years ago, I stumbled across some startling research by economists in
England and Australia: It takes
\href{https://books.google.com/books?id=E1Ic-dZtrBIC\&pg=PT101\&lpg=PT101\&dq=andrew+oswald+unemployment+death+of+a+spouse\&source=bl\&ots=gfUL7Y4GvI\&sig=ACfU3U2KXUi5OPsW78e0CwyFsEdFNB6aZg\&hl=en\&sa=X\&ved=2ahUKEwie2qna_P7pAhWuTTABHfoIBmwQ6AEwBnoECAgQAQ\#v=onepage\&q=andrew\%20oswald\%20unemployment\%20death\%20of\%20a\%20spouse\&f=false}{longer
to adapt} to the pain of unemployment than to losing a loved one.

The notion completely violated my intuition, at first. Then I considered
how very personal getting fired is --- it's often taken as a referendum
on your character, your competence --- and what kind of crisis of
meaning it can create (What am I here for?), and how thoroughly
depleting debt and chronic economic insecurity can be. It really wasn't
so strange at all.

It's time to talk about layoffs. They wound people not just
economically, but emotionally and spiritually, and it looks like we're
due for another round.
\href{https://www.bloomberg.com/graphics/2020-how-many-more-jobs-will-be-lost/}{Bloomberg
Economics} predicted as much in early June, and last week, we began to
see it, with entities as diverse as
\href{https://www.reuters.com/article/us-bp-jobs-exclusive/exclusive-bp-to-cut-15-of-workforce-sources-idUSKBN23F1D9}{BP},
the
\href{https://www.denverpost.com/2020/06/09/university-denver-layoffs-salary-cuts-covid/}{University
of Denver} and the
\href{https://week.com/2020/06/09/peoria-city-council-makes-first-round-of-job-cuts/}{city
of Peoria} shedding employees.

Thursday, the Labor Department reported that more than
\href{https://www.nytimes3xbfgragh.onion/2020/06/11/business/economy/unemployment-claims-coronavirus.html?campaign_id=9\&emc=edit_nn_20200612\&instance_id=19339\&nl=the-morning\&regi_id=78243797\&segment_id=30751\&te=1\&user_id=271fccc23f80ce773b97febcc97e1904}{1.5
million} Americans had filed new state unemployment claims. A grim
\href{https://www.nytimes3xbfgragh.onion/2020/06/12/business/economy/federal-reserve-economy-coronavirus.html}{Friday
report} from the Federal Reserve to Congress noted, ``The path ahead is
extraordinarily uncertain.''

One of the many unwelcome lessons we've already learned during this
recession is just how terrible companies are at firing people. Last
month, WW International, the company formerly known as Weight Watchers,
laid off employees in an
\href{https://www.nytimes3xbfgragh.onion/2020/05/22/business/weight-watchers-firings-zoom.html}{arpeggio
of simultaneous Zoom calls} lasting just three minutes each.

\includegraphics{https://static01.graylady3jvrrxbe.onion/images/2020/06/22/autossell/22Senior-twitter-thumb/22Senior-twitter-thumb-videoSixteenByNineJumbo1600.jpg}

A few weeks before that, Bird, the Santa Monica-based scooter rental
company, lured 406 employees to a mysterious Zoom ``webinar'' only to
have them stare at a slide that read ``Covid-19'' while
\href{https://dot.la/bird-layoffs-meeting-story-2645612465.html}{a
disembodied female voice} told them their services were no longer
required. ``It felt like a Black Mirror episode,''
\href{https://dot.la/bird-layoffs-meeting-story-2645612465.html}{said a
former employee}. (If only. At least there'd have been the possibility
of a
\href{https://www.nytimes3xbfgragh.onion/2014/12/22/arts/television/jon-hamm-stars-in-black-mirror-christmas-special.html}{star
turn by Jon Hamm}.)

Letting people go is not a natural instinct, even if you're a sadist.
Our own president, who achieved household fame as the guy who fired
people on TV,
\href{https://www.washingtonpost.com/outlook/2018/09/25/president-trump-is-afraid-fire-people/}{is
a boneless chicken about it in real life}, deputizing the unpleasant
task to underlings or doing the deed by letter, even by
\href{https://foreignpolicy.com/2019/09/10/bolton-trump-fires-national-security-advisor-via-twitter/}{tweet}.

But there is a right way and a wrong way to pink-slip someone, and every
boss in America should be trained in the art of having this difficult
conversation. As Joel Brockner, a professor of organizational behavior
at Columbia Business School, explained to me, layoffs take a less
extreme toll if there's ``procedural fairness'' associated with them ---
if employees are given reasonable advance notice, for instance, and the
resources to ask follow-up questions and find new job leads. Artless
layoffs, he added, don't just devastate those who've been let go, but
the survivors left behind.

``It can be devastating to productivity, devastating to morale,'' he
told me. The more fair layoffs are, he's found, the happier and more
committed the remaining workers are six months later.

And this, perhaps, is where the logic of these pandemic layoffs must
itself be called into question. That's what I discovered after speaking
to Wayne Cascio, a management professor at the University of Colorado,
Denver. He and two colleagues recently completed a study of every
publicly traded company on the New York Stock Exchange from 1980 to
2016. The companies that delayed layoffs as long as they could ---
whether by cutting salaries, furloughing employees, or even running in
the red --- saw higher stock returns, two years later, than comparable
companies that fired people from the start.

Businesses currently mulling layoffs should remember this.

Sometimes layoffs can't be helped, obviously: A restaurant closes; its
staff must go. But if a business or institution endures, there's a whole
body of literature suggesting that layoffs don't ultimately help the
bottom line once the economy heats back up. Experienced and dedicated
humans are hard to replace.

Recently, I called James Guthrie, a co-author of one of the most
accessible and often-cited papers that argued as much. (It's called
``\href{https://econpapers.repec.org/article/inmororsc/v_3a19_3ay_3a2008_3ai_3a1_3ap_3a108-123.htm}{Dumb
and Dumber}.'') He's an associate dean at the school of business at the
University of Kansas --- which, like many universities in the United
States, is struggling to stay afloat. Yet I discovered it wasn't
economic efficiencies that interested him most at this moment. It was
fairness. Now, he told me, is the time for every organization to express
its values.

``If we at the university had to resort to layoffs, we'd be laying off
some of the most vulnerable staff --- who happen to be the lowest-paid
employees: the custodians, the maintenance crew, the receptionists,'' he
told me. He finds the thought quite troubling. He's started arguing for
pay cuts and furloughs of the better-heeled faculty and administrators
instead. ``It's both more effective,'' he said, ``and more just.''

A number of cautionary tales are going to emerge during this annus
horribilis. But if we want to survive this recession with our dignity
and our sanity intact, it's clear we should keep two things in mind: How
people are laid off matters. And layoffs should be a last resort.
They're often the lazy way out.

\emph{The Times is committed to publishing}
\href{https://www.nytimes3xbfgragh.onion/2019/01/31/opinion/letters/letters-to-editor-new-york-times-women.html}{\emph{a
diversity of letters}} \emph{to the editor. We'd like to hear what you
think about this or any of our articles. Here are some}
\href{https://help.nytimes3xbfgragh.onion/hc/en-us/articles/115014925288-How-to-submit-a-letter-to-the-editor}{\emph{tips}}\emph{.
And here's our email:}
\href{mailto:letters@NYTimes.com}{\emph{letters@NYTimes.com}}\emph{.}

\emph{Follow The New York Times Opinion section on}
\href{https://www.facebookcorewwwi.onion/nytopinion}{\emph{Facebook}}\emph{,}
\href{http://twitter.com/NYTOpinion}{\emph{Twitter (@NYTopinion)}}
\emph{and}
\href{https://www.instagram.com/nytopinion/}{\emph{Instagram}}\emph{.}

Advertisement

\protect\hyperlink{after-bottom}{Continue reading the main story}

\hypertarget{site-index}{%
\subsection{Site Index}\label{site-index}}

\hypertarget{site-information-navigation}{%
\subsection{Site Information
Navigation}\label{site-information-navigation}}

\begin{itemize}
\tightlist
\item
  \href{https://help.nytimes3xbfgragh.onion/hc/en-us/articles/115014792127-Copyright-notice}{©~2020~The
  New York Times Company}
\end{itemize}

\begin{itemize}
\tightlist
\item
  \href{https://www.nytco.com/}{NYTCo}
\item
  \href{https://help.nytimes3xbfgragh.onion/hc/en-us/articles/115015385887-Contact-Us}{Contact
  Us}
\item
  \href{https://www.nytco.com/careers/}{Work with us}
\item
  \href{https://nytmediakit.com/}{Advertise}
\item
  \href{http://www.tbrandstudio.com/}{T Brand Studio}
\item
  \href{https://www.nytimes3xbfgragh.onion/privacy/cookie-policy\#how-do-i-manage-trackers}{Your
  Ad Choices}
\item
  \href{https://www.nytimes3xbfgragh.onion/privacy}{Privacy}
\item
  \href{https://help.nytimes3xbfgragh.onion/hc/en-us/articles/115014893428-Terms-of-service}{Terms
  of Service}
\item
  \href{https://help.nytimes3xbfgragh.onion/hc/en-us/articles/115014893968-Terms-of-sale}{Terms
  of Sale}
\item
  \href{https://spiderbites.nytimes3xbfgragh.onion}{Site Map}
\item
  \href{https://help.nytimes3xbfgragh.onion/hc/en-us}{Help}
\item
  \href{https://www.nytimes3xbfgragh.onion/subscription?campaignId=37WXW}{Subscriptions}
\end{itemize}
