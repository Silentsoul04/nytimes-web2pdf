Sections

SEARCH

\protect\hyperlink{site-content}{Skip to
content}\protect\hyperlink{site-index}{Skip to site index}

\href{https://www.nytimes3xbfgragh.onion/section/world/australia}{Australia}

\href{https://myaccount.nytimes3xbfgragh.onion/auth/login?response_type=cookie\&client_id=vi}{}

\href{https://www.nytimes3xbfgragh.onion/section/todayspaper}{Today's
Paper}

\href{/section/world/australia}{Australia}\textbar{}A Stimulus Backlash
Delivers a Global Warning: Value Female Workers

\url{https://nyti.ms/37uZxOw}

\begin{itemize}
\item
\item
\item
\item
\item
\end{itemize}

\href{https://www.nytimes3xbfgragh.onion/news-event/coronavirus?action=click\&pgtype=Article\&state=default\&region=TOP_BANNER\&context=storylines_menu}{The
Coronavirus Outbreak}

\begin{itemize}
\tightlist
\item
  live\href{https://www.nytimes3xbfgragh.onion/2020/08/04/world/coronavirus-cases.html?action=click\&pgtype=Article\&state=default\&region=TOP_BANNER\&context=storylines_menu}{Latest
  Updates}
\item
  \href{https://www.nytimes3xbfgragh.onion/interactive/2020/us/coronavirus-us-cases.html?action=click\&pgtype=Article\&state=default\&region=TOP_BANNER\&context=storylines_menu}{Maps
  and Cases}
\item
  \href{https://www.nytimes3xbfgragh.onion/interactive/2020/science/coronavirus-vaccine-tracker.html?action=click\&pgtype=Article\&state=default\&region=TOP_BANNER\&context=storylines_menu}{Vaccine
  Tracker}
\item
  \href{https://www.nytimes3xbfgragh.onion/2020/08/02/us/covid-college-reopening.html?action=click\&pgtype=Article\&state=default\&region=TOP_BANNER\&context=storylines_menu}{College
  Reopening}
\item
  \href{https://www.nytimes3xbfgragh.onion/live/2020/08/04/business/stock-market-today-coronavirus?action=click\&pgtype=Article\&state=default\&region=TOP_BANNER\&context=storylines_menu}{Economy}
\end{itemize}

Advertisement

\protect\hyperlink{after-top}{Continue reading the main story}

Supported by

\protect\hyperlink{after-sponsor}{Continue reading the main story}

\hypertarget{a-stimulus-backlash-delivers-a-global-warning-value-female-workers}{%
\section{A Stimulus Backlash Delivers a Global Warning: Value Female
Workers}\label{a-stimulus-backlash-delivers-a-global-warning-value-female-workers}}

Australia is pouring millions into the male-dominated construction
industry while ending free child care. Critics say that is illogical and
reflects sexist biases.

\includegraphics{https://static01.graylady3jvrrxbe.onion/images/2020/06/13/world/13oz-childcare01/merlin_173465433_6020cddd-c14e-497c-a7c4-c41c9d9017d1-articleLarge.jpg?quality=75\&auto=webp\&disable=upscale}

\href{https://www.nytimes3xbfgragh.onion/by/damien-cave}{\includegraphics{https://static01.graylady3jvrrxbe.onion/images/2018/10/08/multimedia/author-damien-cave/author-damien-cave-thumbLarge.png}}

By \href{https://www.nytimes3xbfgragh.onion/by/damien-cave}{Damien Cave}

\begin{itemize}
\item
  June 14, 2020
\item
  \begin{itemize}
  \item
  \item
  \item
  \item
  \item
  \end{itemize}
\end{itemize}

SYDNEY, Australia --- Shelley Duggan became an essential worker when
Australia declared that her job at a suburban child care center must be
preserved so that doctors and nurses would have a place to send their
children during the coronavirus outbreak.

Australia even made child care free, subsidizing wages and leading Ms.
Duggan to believe that the country was finally learning to respect her
work in a multibillion-dollar industry that is overwhelmingly female.

Now, though, that faith is shattered. With the country striding back
toward normalcy, the first industry that the government is cutting from
the subsidy program is Ms. Duggan's. And just as free child care is
ending, extra stimulus will be pumped into the construction industry ---
a contrast that many say reflects old sexist biases.

``They've thrown us under the bus,'' said Ms. Duggan, 41, a mother of
three with degrees in education and psychology. ``I've been working
through the whole thing while trying to home-school my children as well.
It just goes to show that they are not valuing what we do.''

As countries
\href{https://www.nytimes3xbfgragh.onion/2020/06/09/world/coronavirus-reopenings.html}{relax
coronavirus lockdowns} and redouble efforts to bring their economies
back to life, Australia's throwback approach to stimulus spending has
prompted a furious outcry. The message for nations a step behind
Australia on the path to reopening comes from many sectors of society:
This is not
\href{https://rooseveltinstitute.org/wpa-flawed-model-women-inspiration-progress/}{the
1930s} or 1950s; economic priorities must match the times and account
for women's essential roles and sacrifices.

``One thing the Covid crisis has shown us is how important women's work
really is,'' said Rae Cooper, a professor of gender, work and employment
relations at the University of Sydney Business School. ``They've kept us
alive and kept our society running. Policy needs to catch up with what
women are doing.''

\includegraphics{https://static01.graylady3jvrrxbe.onion/images/2020/06/13/world/13oz-childcare02/13oz-childcare02-articleLarge.jpg?quality=75\&auto=webp\&disable=upscale}

The pandemic has placed a disproportionate burden on women. They have
put themselves on the contagion's front lines, in health care, in caring
for older adults and in education, outnumbering men at almost every
turn. Their
\href{https://www.nytimes3xbfgragh.onion/2020/05/20/us/women-economy-jobs-coronavirus-gender.html}{unpaid
work} managing families has expanded. And studies show that they are
\href{https://nwlc-ciw49tixgw5lbab.stackpathdns.com/wp-content/uploads/2020/05/Jobs-Day-April-Factsheet.pdf}{suffering
higher unemployment} because they are overrepresented in retail,
restaurants and other service industries crippled by lockdowns.

The trend lines are global. The coronavirus has weakened the already
precarious position of women in the economies
\href{https://www.nytimes3xbfgragh.onion/2020/06/09/world/asia/india-coronavirus-women-economy.html}{of
countries such as India} and
\href{https://www.smh.com.au/business/markets/shock-absorbers-japanese-women-bear-brunt-of-recession-as-pandemic-unravels-abe-s-womenomics-20200612-p551wt.html}{Japan},
and threatens to reverse their economic gains in many places, including
the
\href{https://www.nytimes3xbfgragh.onion/2020/06/03/business/economy/coronavirus-working-women.html}{United
States} and Europe.

But Australia is an especially revealing example because its early
approach to the pandemic set up what could have been a transformational
moment.

\hypertarget{latest-updates-global-coronavirus-outbreak}{%
\section{\texorpdfstring{\href{https://www.nytimes3xbfgragh.onion/2020/08/04/world/coronavirus-cases.html?action=click\&pgtype=Article\&state=default\&region=MAIN_CONTENT_1\&context=storylines_live_updates}{Latest
Updates: Global Coronavirus
Outbreak}}{Latest Updates: Global Coronavirus Outbreak}}\label{latest-updates-global-coronavirus-outbreak}}

Updated 2020-08-04T22:07:26.192Z

\begin{itemize}
\tightlist
\item
  \href{https://www.nytimes3xbfgragh.onion/2020/08/04/world/coronavirus-cases.html?action=click\&pgtype=Article\&state=default\&region=MAIN_CONTENT_1\&context=storylines_live_updates\#link-2daa96b5}{As
  talks drag on, McConnell signals openness to jobless aid extension
  that Republicans have opposed.}
\item
  \href{https://www.nytimes3xbfgragh.onion/2020/08/04/world/coronavirus-cases.html?action=click\&pgtype=Article\&state=default\&region=MAIN_CONTENT_1\&context=storylines_live_updates\#link-1228a480}{Novavax
  sees encouraging results from two studies of its experimental
  vaccine.}
\item
  \href{https://www.nytimes3xbfgragh.onion/2020/08/04/world/coronavirus-cases.html?action=click\&pgtype=Article\&state=default\&region=MAIN_CONTENT_1\&context=storylines_live_updates\#link-4825b93}{Public
  and private schools in Maryland and elsewhere are divided over
  in-person instruction.}
\end{itemize}

\href{https://www.nytimes3xbfgragh.onion/2020/08/04/world/coronavirus-cases.html?action=click\&pgtype=Article\&state=default\&region=MAIN_CONTENT_1\&context=storylines_live_updates}{See
more updates}

More live coverage:
\href{https://www.nytimes3xbfgragh.onion/live/2020/08/04/business/stock-market-today-coronavirus?action=click\&pgtype=Article\&state=default\&region=MAIN_CONTENT_1\&context=storylines_live_updates}{Markets}

Most Australians had to blink twice when their conservative prime
minister, Scott Morrison, moved to protect the early childhood education
sector in April. The cost of care for children under 5 had been
\href{https://www.abs.gov.au/ausstats/abs@.nsf/cat/4402.0}{surging for
years} without much interest from Parliament, and Mr. Morrison has often
been criticized for his macho ``rugby bloke'' manner. He was scorned
last year after saying that women should rise at work only
\href{https://www.sbs.com.au/news/scott-morrison-says-the-rise-of-women-should-not-come-at-the-expense-of-men}{if
their gains didn't come at the expense of men}.

And yet, as he turned to scientists for the country's public health
response to the virus, Mr. Morrison --- the father of two school-age
daughters --- yielded to crisis logic for an industry in which 91
percent of the workers are women. As parents pulled their children out
of child care centers, the government agreed to cover half of the fees,
included child care in its national
\href{https://www.ato.gov.au/general/jobkeeper-payment/}{wage subsidy
program} and declared care would be free for all.

The decision meant that many doctors, nurses and other essential workers
could put in extra hours. Working parents nationwide, including Fernanda
Fain-Binda, 37, a freelance writer in Melbourne, let out sighs of
relief.

``When the child care fees became free, it was this incredible weight
off our mind,'' said Ms. Fain-Binda, who has a 5-year-old daughter and a
2-year-old son. ``The moment the lockdown started, we actually increased
my son's days because we knew we needed that.''

Image

Two-year-old Ray Selby-Smith and his mother Fernanda Fain-Binda at a
playground near their home in Melbourne. ``When the child care fees
became free, it was this incredible weight off my mind,'' said Ms.
Fain-Binda.Credit...Christina Simons for The New York Times

The government did not exactly make it easy for the industry. In most
cases, federal assistance failed to match pre-pandemic revenue. Many
workers had their hours cut. Ms. Duggan, who works in a suburb of
Melbourne, said she took a pay cut even with the government support,
leaving her with a wary appreciation for Mr. Morrison's assistance.

``It was a backhanded compliment,'' she said. ```We really need you, and
the economy needs you, but we don't want to pay you, because we don't
think you're that important.'''

Eventually, the soft whisper of disrespect started to sound like a
shout. One after another, federal and state officials crowed about plans
to prioritize infrastructure projects and bolster manufacturing and
construction --- industries that are 70 percent to 95 percent male.

On June 4, Mr. Morrison announced a plan to support builders with 688
million Australian dollars, about \$475 million, in grants for the
construction or renovation of homes.

Four days later, his government announced the end of free child care and
subsidies for early-childhood educators, despite promises that the
program would last until September.

Asked how she felt about the pairing, Dana Lightbody, 42, a single
mother of 3-year-old twins in Sydney, offered a common response: ``Aw
mate, it's anger, seething anger.''

\href{https://www.nytimes3xbfgragh.onion/news-event/coronavirus?action=click\&pgtype=Article\&state=default\&region=MAIN_CONTENT_3\&context=storylines_faq}{}

\hypertarget{the-coronavirus-outbreak-}{%
\subsubsection{The Coronavirus Outbreak
›}\label{the-coronavirus-outbreak-}}

\hypertarget{frequently-asked-questions}{%
\paragraph{Frequently Asked
Questions}\label{frequently-asked-questions}}

Updated August 4, 2020

\begin{itemize}
\item ~
  \hypertarget{i-have-antibodies-am-i-now-immune}{%
  \paragraph{I have antibodies. Am I now
  immune?}\label{i-have-antibodies-am-i-now-immune}}

  \begin{itemize}
  \tightlist
  \item
    As of right
    now,\href{https://www.nytimes3xbfgragh.onion/2020/07/22/health/covid-antibodies-herd-immunity.html?action=click\&pgtype=Article\&state=default\&region=MAIN_CONTENT_3\&context=storylines_faq}{that
    seems likely, for at least several months.} There have been
    frightening accounts of people suffering what seems to be a second
    bout of Covid-19. But experts say these patients may have a
    drawn-out course of infection, with the virus taking a slow toll
    weeks to months after initial exposure. People infected with the
    coronavirus typically
    \href{https://www.nature.com/articles/s41586-020-2456-9}{produce}
    immune molecules called antibodies, which are
    \href{https://www.nytimes3xbfgragh.onion/2020/05/07/health/coronavirus-antibody-prevalence.html?action=click\&pgtype=Article\&state=default\&region=MAIN_CONTENT_3\&context=storylines_faq}{protective
    proteins made in response to an
    infection}\href{https://www.nytimes3xbfgragh.onion/2020/05/07/health/coronavirus-antibody-prevalence.html?action=click\&pgtype=Article\&state=default\&region=MAIN_CONTENT_3\&context=storylines_faq}{.
    These antibodies may} last in the body
    \href{https://www.nature.com/articles/s41591-020-0965-6}{only two to
    three months}, which may seem worrisome, but that's perfectly normal
    after an acute infection subsides, said Dr. Michael Mina, an
    immunologist at Harvard University. It may be possible to get the
    coronavirus again, but it's highly unlikely that it would be
    possible in a short window of time from initial infection or make
    people sicker the second time.
  \end{itemize}
\item ~
  \hypertarget{im-a-small-business-owner-can-i-get-relief}{%
  \paragraph{I'm a small-business owner. Can I get
  relief?}\label{im-a-small-business-owner-can-i-get-relief}}

  \begin{itemize}
  \tightlist
  \item
    The
    \href{https://www.nytimes3xbfgragh.onion/article/small-business-loans-stimulus-grants-freelancers-coronavirus.html?action=click\&pgtype=Article\&state=default\&region=MAIN_CONTENT_3\&context=storylines_faq}{stimulus
    bills enacted in March} offer help for the millions of American
    small businesses. Those eligible for aid are businesses and
    nonprofit organizations with fewer than 500 workers, including sole
    proprietorships, independent contractors and freelancers. Some
    larger companies in some industries are also eligible. The help
    being offered, which is being managed by the Small Business
    Administration, includes the Paycheck Protection Program and the
    Economic Injury Disaster Loan program. But lots of folks have
    \href{https://www.nytimes3xbfgragh.onion/interactive/2020/05/07/business/small-business-loans-coronavirus.html?action=click\&pgtype=Article\&state=default\&region=MAIN_CONTENT_3\&context=storylines_faq}{not
    yet seen payouts.} Even those who have received help are confused:
    The rules are draconian, and some are stuck sitting on
    \href{https://www.nytimes3xbfgragh.onion/2020/05/02/business/economy/loans-coronavirus-small-business.html?action=click\&pgtype=Article\&state=default\&region=MAIN_CONTENT_3\&context=storylines_faq}{money
    they don't know how to use.} Many small-business owners are getting
    less than they expected or
    \href{https://www.nytimes3xbfgragh.onion/2020/06/10/business/Small-business-loans-ppp.html?action=click\&pgtype=Article\&state=default\&region=MAIN_CONTENT_3\&context=storylines_faq}{not
    hearing anything at all.}
  \end{itemize}
\item ~
  \hypertarget{what-are-my-rights-if-i-am-worried-about-going-back-to-work}{%
  \paragraph{What are my rights if I am worried about going back to
  work?}\label{what-are-my-rights-if-i-am-worried-about-going-back-to-work}}

  \begin{itemize}
  \tightlist
  \item
    Employers have to provide
    \href{https://www.osha.gov/SLTC/covid-19/standards.html}{a safe
    workplace} with policies that protect everyone equally.
    \href{https://www.nytimes3xbfgragh.onion/article/coronavirus-money-unemployment.html?action=click\&pgtype=Article\&state=default\&region=MAIN_CONTENT_3\&context=storylines_faq}{And
    if one of your co-workers tests positive for the coronavirus, the
    C.D.C.} has said that
    \href{https://www.cdc.gov/coronavirus/2019-ncov/community/guidance-business-response.html}{employers
    should tell their employees} -\/- without giving you the sick
    employee's name -\/- that they may have been exposed to the virus.
  \end{itemize}
\item ~
  \hypertarget{should-i-refinance-my-mortgage}{%
  \paragraph{Should I refinance my
  mortgage?}\label{should-i-refinance-my-mortgage}}

  \begin{itemize}
  \tightlist
  \item
    \href{https://www.nytimes3xbfgragh.onion/article/coronavirus-money-unemployment.html?action=click\&pgtype=Article\&state=default\&region=MAIN_CONTENT_3\&context=storylines_faq}{It
    could be a good idea,} because mortgage rates have
    \href{https://www.nytimes3xbfgragh.onion/2020/07/16/business/mortgage-rates-below-3-percent.html?action=click\&pgtype=Article\&state=default\&region=MAIN_CONTENT_3\&context=storylines_faq}{never
    been lower.} Refinancing requests have pushed mortgage applications
    to some of the highest levels since 2008, so be prepared to get in
    line. But defaults are also up, so if you're thinking about buying a
    home, be aware that some lenders have tightened their standards.
  \end{itemize}
\item ~
  \hypertarget{what-is-school-going-to-look-like-in-september}{%
  \paragraph{What is school going to look like in
  September?}\label{what-is-school-going-to-look-like-in-september}}

  \begin{itemize}
  \tightlist
  \item
    It is unlikely that many schools will return to a normal schedule
    this fall, requiring the grind of
    \href{https://www.nytimes3xbfgragh.onion/2020/06/05/us/coronavirus-education-lost-learning.html?action=click\&pgtype=Article\&state=default\&region=MAIN_CONTENT_3\&context=storylines_faq}{online
    learning},
    \href{https://www.nytimes3xbfgragh.onion/2020/05/29/us/coronavirus-child-care-centers.html?action=click\&pgtype=Article\&state=default\&region=MAIN_CONTENT_3\&context=storylines_faq}{makeshift
    child care} and
    \href{https://www.nytimes3xbfgragh.onion/2020/06/03/business/economy/coronavirus-working-women.html?action=click\&pgtype=Article\&state=default\&region=MAIN_CONTENT_3\&context=storylines_faq}{stunted
    workdays} to continue. California's two largest public school
    districts --- Los Angeles and San Diego --- said on July 13, that
    \href{https://www.nytimes3xbfgragh.onion/2020/07/13/us/lausd-san-diego-school-reopening.html?action=click\&pgtype=Article\&state=default\&region=MAIN_CONTENT_3\&context=storylines_faq}{instruction
    will be remote-only in the fall}, citing concerns that surging
    coronavirus infections in their areas pose too dire a risk for
    students and teachers. Together, the two districts enroll some
    825,000 students. They are the largest in the country so far to
    abandon plans for even a partial physical return to classrooms when
    they reopen in August. For other districts, the solution won't be an
    all-or-nothing approach.
    \href{https://bioethics.jhu.edu/research-and-outreach/projects/eschool-initiative/school-policy-tracker/}{Many
    systems}, including the nation's largest, New York City, are
    devising
    \href{https://www.nytimes3xbfgragh.onion/2020/06/26/us/coronavirus-schools-reopen-fall.html?action=click\&pgtype=Article\&state=default\&region=MAIN_CONTENT_3\&context=storylines_faq}{hybrid
    plans} that involve spending some days in classrooms and other days
    online. There's no national policy on this yet, so check with your
    municipal school system regularly to see what is happening in your
    community.
  \end{itemize}
\end{itemize}

The conference and training company she co-owns has lost most of its
business. In the past few weeks, she said she had sold a garage's worth
of old baby accessories to make ends meet. In July, she will again
confront child care charges of more than \$200 a day for her toddlers.

Image

Dana Lightbody with her children in their living room. In July, Ms.
Lightbody will be confronting child care charges of more than \$200 a
day for her toddlers.Credit...Anna Maria Antoinette D'Addario for The
New York Times

Professor Cooper, the gender and employment specialist, said that many
Australians are outraged by the government's approach. ``It's absolutely
valuing men's work over women's work,'' she said. ``It is so obvious and
so clear to the public that it's really quite shocking.''

Mr. Morrison, for his part, seemed surprised by the indignation. ``When
we put the new arrangements for child care in place during the worst
parts of this crisis, we were very clear that it would not be a
permanent arrangement,'' he said in Parliament on Wednesday. He noted
that the usual subsidies, on a sliding scale for income, would soon
return.

But critics say that the prime minister is missing the point. The
Australia Institute, an independent think tank,
\href{https://www.tai.org.au/content/new-analysis-more-women-unemployed-yet-stimulus-favours-male-dominated-industries}{recently
found} that stimulus spending on construction would create fewer jobs
for both men and women than spending a similar amount on health,
education and tourism or entertainment, which are expected to be future
growth areas for economies worldwide.

The focus on hard-hat workers also ignores the fact that 55 percent of
the newly unemployed in Australia are women --- and that in
\href{https://www.abc.net.au/news/2019-07-30/working-mothers-more-stressed-than-fathers-report-finds/11365632}{survey
after survey}, women report that high-quality, affordable child care is
one of the most important factors shaping their careers.

For years, Australians have
\href{https://www.smh.com.au/education/looking-to-swedish-model-of-childcare-and-education-20150518-gh48hj.html}{envied
Scandinavian countries of similar wealth} that offer free universal care
from
\href{https://www.weforum.org/agenda/2019/03/nordic-nations-best-places-for-parents-children/\#:~:text=Finland\%20provides\%20free\%20universal\%20daycare,formal\%20education\%20at\%20age\%20seven.\&text=In\%20Sweden\%2C\%20parents\%20are\%20entitled,at\%2080\%25\%20of\%20their\%20salary}{as
early as eight months of age} --- and for a brief moment, they had it.

``We saw for a couple of months that it's possible to think more broadly
than we have in the past and possibly change the gender dynamics,''
Professor Cooper said. ``And now it's gone. It's a real slap in the face
to be removing stimulus in female-dominated areas at precisely the
moment when women are dealing with an additional burden.''

Ms. Duggan said that even now, after Australia has mostly stamped out
the coronavirus, the public health crisis and economic struggles for
women have lingered. At her child care center, roughly a quarter of the
work force was stuck at home this week waiting for results from Covid-19
tests for themselves or for their children, she noted. Many have already
used their sick leave to get through April and May.

``We're told we're essential because people need to go to work and keep
the economy running,'' she said. ``But once the economy starts running
again, then we go back to being unimportant.''

Advertisement

\protect\hyperlink{after-bottom}{Continue reading the main story}

\hypertarget{site-index}{%
\subsection{Site Index}\label{site-index}}

\hypertarget{site-information-navigation}{%
\subsection{Site Information
Navigation}\label{site-information-navigation}}

\begin{itemize}
\tightlist
\item
  \href{https://help.nytimes3xbfgragh.onion/hc/en-us/articles/115014792127-Copyright-notice}{©~2020~The
  New York Times Company}
\end{itemize}

\begin{itemize}
\tightlist
\item
  \href{https://www.nytco.com/}{NYTCo}
\item
  \href{https://help.nytimes3xbfgragh.onion/hc/en-us/articles/115015385887-Contact-Us}{Contact
  Us}
\item
  \href{https://www.nytco.com/careers/}{Work with us}
\item
  \href{https://nytmediakit.com/}{Advertise}
\item
  \href{http://www.tbrandstudio.com/}{T Brand Studio}
\item
  \href{https://www.nytimes3xbfgragh.onion/privacy/cookie-policy\#how-do-i-manage-trackers}{Your
  Ad Choices}
\item
  \href{https://www.nytimes3xbfgragh.onion/privacy}{Privacy}
\item
  \href{https://help.nytimes3xbfgragh.onion/hc/en-us/articles/115014893428-Terms-of-service}{Terms
  of Service}
\item
  \href{https://help.nytimes3xbfgragh.onion/hc/en-us/articles/115014893968-Terms-of-sale}{Terms
  of Sale}
\item
  \href{https://spiderbites.nytimes3xbfgragh.onion}{Site Map}
\item
  \href{https://help.nytimes3xbfgragh.onion/hc/en-us}{Help}
\item
  \href{https://www.nytimes3xbfgragh.onion/subscription?campaignId=37WXW}{Subscriptions}
\end{itemize}
