Sections

SEARCH

\protect\hyperlink{site-content}{Skip to
content}\protect\hyperlink{site-index}{Skip to site index}

\href{https://myaccount.nytimes3xbfgragh.onion/auth/login?response_type=cookie\&client_id=vi}{}

\href{https://www.nytimes3xbfgragh.onion/section/todayspaper}{Today's
Paper}

\href{/section/opinion}{Opinion}\textbar{}The Court-Martial of Donald J.
Trump

\url{https://nyti.ms/2B46slP}

\begin{itemize}
\item
\item
\item
\item
\item
\item
\end{itemize}

Advertisement

\protect\hyperlink{after-top}{Continue reading the main story}

\href{/section/opinion}{Opinion}

Supported by

\protect\hyperlink{after-sponsor}{Continue reading the main story}

\hypertarget{the-court-martial-of-donald-j-trump}{%
\section{The Court-Martial of Donald J.
Trump}\label{the-court-martial-of-donald-j-trump}}

The intensifying rift between the military and the president demands
attention.

\href{https://www.nytimes3xbfgragh.onion/by/frank-bruni}{\includegraphics{https://static01.graylady3jvrrxbe.onion/images/2018/04/03/opinion/frank-bruni/frank-bruni-thumbLarge.png}}

By \href{https://www.nytimes3xbfgragh.onion/by/frank-bruni}{Frank Bruni}

Opinion Columnist

\begin{itemize}
\item
  June 15, 2020
\item
  \begin{itemize}
  \item
  \item
  \item
  \item
  \item
  \item
  \end{itemize}
\end{itemize}

\includegraphics{https://static01.graylady3jvrrxbe.onion/images/2020/06/15/opinion/15Bruni/merlin_173507868_b9ad2a5c-2264-4b5f-8a41-593475393afc-articleLarge.jpg?quality=75\&auto=webp\&disable=upscale}

The first time I saw President Trump referred to as
``\href{https://www.cnn.com/2018/02/06/politics/tammy-duckworth-cadet-bone-spurs/index.html}{Cadet
Bone Spurs}'' I laughed, the second time I smiled and the third time I
cringed. It's an apt slur, but it lumps him together with all the other
politicians whose military huzzahs contradict their personal histories
and whose insult to our men and women in uniform can be reduced to
dodging the draft.

Trump's twisted and utterly transactional relationship with America's
armed forces is a bigger insult than that. For all his lip service to
military service, his actions reveal a crude take on those who perform
it.

And they have led now to a remarkable and remarkably public reappraisal
--- even repudiation --- of him by people in the armed services, their
leaders and veterans.

Some are finally coming around to a cleareyed view of a corrupt
president. Others are venting a distaste for Trump that they'd
previously downplayed or kept to themselves.

Even the most dutiful soldier has a breaking point, and even a culture
of deference finds
\href{https://time.com/5849362/military-leaders-denounce-trump/}{its
moment of defiance}.

Late last week Army Gen. Mark Milley, the chairman of the Joint Chiefs
of Staff, issued an extraordinary apology for his participation in that
awful presidential photo op made possible by the use of tear gas against
peaceful protesters. But as Helene Cooper
\href{https://www.nytimes3xbfgragh.onion/2020/06/11/us/politics/trump-milley-military-protests-lafayette-square.html?action=click\&module=RelatedLinks\&pgtype=Article}{noted}
in a story in The Times, that's just one example of an intensifying
friction between the president and military leaders. Many of them don't
share his opposition to renaming bases that honor Confederate officers
and disagreed with his push to have armed forces quell demonstrations.

``Trump's Actions Rattle the Military World'' was the headline on
\href{https://www.nytimes3xbfgragh.onion/2020/06/12/us/politics/trump-polls-military-approval.html?smid=tw-share}{a
separate story} in The Times by Jennifer Steinhauer. Her conversations
with members of the military, their families and veterans made clear
that they might not back Trump to the extent that they did in 2016.

Then there are the generals and admirals, silent by custom but
\href{https://www.cnn.com/2020/06/05/politics/military-leaders-trump-floyd-protests/index.html}{silent
no more}. What we've seen and heard from them over the past two weeks is
unprecedented in my adult lifetime, a jolting departure from their norm
of mutely supporting a sitting president, no matter their differences
with him.

Trump has been
\href{https://www.theatlantic.com/politics/archive/2020/06/james-mattis-denounces-trump-protests-militarization/612640/}{denounced
by Marine Corps Gen. James Mattis} and reprimanded by Marine Corps Gen.
John Kelly, both of whom held top jobs in his administration. ``I think
we need to look harder at who we elect,''
\href{https://www.youtube.com/watch?v=_zAu3cZdQVQ}{Kelly said} in an
interview for the online platform SALT Talks.

Trump has been upbraided by Navy Adm. Mike Mullen and Air Force Gen.
Richard Myers, each of whom served as the chairman of the Joint Chiefs
of Staff under President George W. Bush. ``I'm glad I don't have to
advise this president,''
\href{https://www.cnn.com/2020/06/04/politics/trump-protests-former-joint-chiefs-chairman-cnntv/index.html}{Myers
said} in a CNN interview.

Together these admonishments amount to a metaphoric court-martial of the
commander in chief.

Trump campaigned by arguing that presidents before him had abused the
military by deploying troops to places --- Iraq, Afghanistan --- where
the justification was suspect, the mission ill-fated and the end point
invisible. He promised to avoid such costly entanglements while
nonetheless spending more on military equipment. He wanted lots of it
and he wanted it to gleam, just like his casinos. He told members of the
armed forces that they'd never known a friend like him in the White
House.

But what a nasty tongue and temper this friend has. At the start of his
candidacy, he grossly mocked John McCain, who had been tortured for
years in North Vietnam, by saying that he preferred war heroes who
didn't get captured.

He praised generals, sure, but only to assert his superiority to them.
``I know more about ISIS than the generals do --- believe me,'' he said
at one point in his candidacy, a cockamamie coda to his earlier boast
that ``there's nobody bigger or better at the military than I am.''

He
\href{https://www.npr.org/2017/10/23/559558075/trump-call-controversy-renews-spotlight-on-gold-star-families}{attacked
Gold Star families}, rage-tweeting against the father of Army Capt.
Humayun Khan, who was killed in Iraq, and the widow of Army Sgt. La
David Johnson, who was killed in Niger. They had dared to criticize him,
and he put his vanity over their grief.

There's no reverence in Trump, only convenience and expedience. Nearly
two years and a hell of a lot of golf passed between his inauguration
and the first time he could rouse himself
\href{https://www.nytimes3xbfgragh.onion/2018/12/26/us/politics/trump-iraq-troops-visit.html}{to
visit troops} in a foreign combat zone.

During an earlier trip abroad in late 2018, he
\href{https://www.politico.eu/article/donald-trump-rain-world-war-1-castigated-for-missing-event-paris-armistice/}{abruptly
canceled} his participation in an event at an American cemetery and
World War I memorial in France when rain meant that he'd have to drive
instead of taking a quicker helicopter flight.

The following year, again in France, he used the commemoration of the
75th anniversary of the D-Day invasion in
\href{https://www.nytimes3xbfgragh.onion/2019/06/06/world/europe/trump-france-dday.html}{Normandy
as the backdrop} for an interview in which he called Robert Mueller ``a
fool'' and Nancy Pelosi ``a disaster.'' Everything to Trump is show
business, and the graves of warriors are fitting props for a tirade on
Fox News.

Trump approaches and appraises the military as he does all else:
\emph{What's in it for me?} He needlessly sent troops to our southern
border in the fall of 2018 because it advanced a narrative, which he
contrived to help the Republican Party in the midterms, that America was
being threatened by an invasion of migrants.

``My military,''
\href{https://www.businessinsider.com/trump-my-generals-my-military-2017-10}{he
has said}, and it's no slip of the tongue. He sees the military as a
vessel for his own glorification, to which end he openly yearned for a
military parade in Washington, a titanic tribute to all the metal and
munitions under \emph{his} control.

``My generals,'' he has also said, referring to the bevy of them ---
Mattis, his first defense secretary; Kelly, his second chief of staff;
Army Lt. Gen. Michael Flynn, his first national security adviser; Army
Lt. Gen. H.R. McMaster, his second --- with which he stocked his
administration. The phrase is the giveaway that, to him, they were
trophies, their stars and medals merely ornaments on his ego.

And they were meant to defer, lest there be doubt of his own dominance.
So he lied not only about firing Mattis, who in fact resigned, but also
about having given him the nickname ``Mad Dog.''

For Trump the military is a commercial enterprise and commodity. His
complaints about N.A.T.O. boil down to balance sheets, focusing on the
\emph{financial} disparity between the United States's military
contribution and other countries', as if our servicemen and servicewomen
are service providers.

They rightly see themselves as more than that --- and as more than the
brutes of Trump's childish imagination. Last November he
\href{https://www.nytimes3xbfgragh.onion/2019/11/15/us/trump-pardons.html}{cleared
three members} of the armed forces who had been accused or convicted of
war crimes. He did so \emph{against} the wishes of Defense Secretary
Mark Esper and Army Secretary Ryan McCarthy, who worried about the
military code of justice being undermined.

Code? Justice? Trump thinks and speaks in the language of wins, losses,
brawn and bloodshed. He only pantomimes principle. His supposed
reluctance to send troops into foreign lands gave way over recent weeks
to his readiness to have them occupy our own land and engage in combat
with their fellow Americans.

That he didn't expect them to push back proves how little he understands
them and how far short he sells them. They bring more than muscle to
what they do. They bring heart, soul and intellect. Which is more than
can be said for their commander in chief.

\emph{I invite you to sign up for my free}
\href{https://www.nytimes3xbfgragh.onion/newsletters/frank-bruni}{\emph{weekly
email newsletter}}\emph{. You can follow me on Twitter
(}\href{https://twitter.com/FrankBruni}{\emph{@FrankBruni}}\emph{).}

\emph{Listen to}
\href{https://www.nytimes3xbfgragh.onion/column/the-argument}{\emph{``The
Argument'' podcast}} \emph{every Thursday morning, with Ross Douthat,
Michelle Goldberg and me.}

Advertisement

\protect\hyperlink{after-bottom}{Continue reading the main story}

\hypertarget{site-index}{%
\subsection{Site Index}\label{site-index}}

\hypertarget{site-information-navigation}{%
\subsection{Site Information
Navigation}\label{site-information-navigation}}

\begin{itemize}
\tightlist
\item
  \href{https://help.nytimes3xbfgragh.onion/hc/en-us/articles/115014792127-Copyright-notice}{©~2020~The
  New York Times Company}
\end{itemize}

\begin{itemize}
\tightlist
\item
  \href{https://www.nytco.com/}{NYTCo}
\item
  \href{https://help.nytimes3xbfgragh.onion/hc/en-us/articles/115015385887-Contact-Us}{Contact
  Us}
\item
  \href{https://www.nytco.com/careers/}{Work with us}
\item
  \href{https://nytmediakit.com/}{Advertise}
\item
  \href{http://www.tbrandstudio.com/}{T Brand Studio}
\item
  \href{https://www.nytimes3xbfgragh.onion/privacy/cookie-policy\#how-do-i-manage-trackers}{Your
  Ad Choices}
\item
  \href{https://www.nytimes3xbfgragh.onion/privacy}{Privacy}
\item
  \href{https://help.nytimes3xbfgragh.onion/hc/en-us/articles/115014893428-Terms-of-service}{Terms
  of Service}
\item
  \href{https://help.nytimes3xbfgragh.onion/hc/en-us/articles/115014893968-Terms-of-sale}{Terms
  of Sale}
\item
  \href{https://spiderbites.nytimes3xbfgragh.onion}{Site Map}
\item
  \href{https://help.nytimes3xbfgragh.onion/hc/en-us}{Help}
\item
  \href{https://www.nytimes3xbfgragh.onion/subscription?campaignId=37WXW}{Subscriptions}
\end{itemize}
