Sections

SEARCH

\protect\hyperlink{site-content}{Skip to
content}\protect\hyperlink{site-index}{Skip to site index}

\href{https://www.nytimes3xbfgragh.onion/section/us}{U.S.}

\href{https://myaccount.nytimes3xbfgragh.onion/auth/login?response_type=cookie\&client_id=vi}{}

\href{https://www.nytimes3xbfgragh.onion/section/todayspaper}{Today's
Paper}

\href{/section/us}{U.S.}\textbar{}Civil Rights Law Protects Gay and
Transgender Workers, Supreme Court Rules

\url{https://nyti.ms/2B7rZtS}

\begin{itemize}
\item
\item
\item
\item
\item
\item
\end{itemize}

Advertisement

\protect\hyperlink{after-top}{Continue reading the main story}

Supported by

\protect\hyperlink{after-sponsor}{Continue reading the main story}

\hypertarget{civil-rights-law-protects-gay-and-transgender-workers-supreme-court-rules}{%
\section{Civil Rights Law Protects Gay and Transgender Workers, Supreme
Court
Rules}\label{civil-rights-law-protects-gay-and-transgender-workers-supreme-court-rules}}

The court said the language of the Civil Rights Act of 1964, which
prohibits sex discrimination, applies to discrimination based on sexual
orientation and gender identity.

\includegraphics{https://static01.graylady3jvrrxbe.onion/images/2020/06/15/us/politics/15dc-scotus-lgbt/merlin_173534901_83f3865e-6ba4-4f1b-8d6c-a7f54db8cce8-articleLarge.jpg?quality=75\&auto=webp\&disable=upscale}

\href{https://www.nytimes3xbfgragh.onion/by/adam-liptak}{\includegraphics{https://static01.graylady3jvrrxbe.onion/images/2018/07/13/multimedia/author-adam-liptak/author-adam-liptak-thumbLarge-v3.png}}

By \href{https://www.nytimes3xbfgragh.onion/by/adam-liptak}{Adam Liptak}

\begin{itemize}
\item
  Published June 15, 2020Updated June 16, 2020
\item
  \begin{itemize}
  \item
  \item
  \item
  \item
  \item
  \item
  \end{itemize}
\end{itemize}

WASHINGTON --- The Supreme Court
\href{https://www.supremecourt.gov/opinions/19pdf/17-1618_hfci.pdf}{ruled
on Monday} that a landmark civil rights law protects
\href{https://www.nytimes3xbfgragh.onion/2020/06/16/podcasts/the-daily/supreme-court-lgbtq.html}{gay
and transgender workers} from workplace discrimination, handing the
movement for L.G.B.T. equality a long-sought and unexpected victory.

``An employer who fires an individual merely for being gay or
transgender defies the law,'' Justice Neil M. Gorsuch wrote for the
majority in the 6-to-3 ruling.

That opinion and two dissents, spanning 168 pages, touched on a host of
flash points in the culture wars involving the L.G.B.T. community ---
bathrooms, locker rooms, sports, pronouns and religious objections to
same-sex marriage. The decision, the first major case on transgender
rights, came amid widespread demonstrations, some protesting violence
aimed at transgender people of color.

Until Monday's decision, it was legal in more than half of the states to
fire workers for being gay, bisexual or transgender. The vastly
consequential decision thus extended workplace protections to millions
of people across the nation, continuing a series of Supreme Court
\href{https://www.nytimes3xbfgragh.onion/2015/06/27/us/supreme-court-same-sex-marriage.html}{victories
for gay rights} even after President Trump transformed the court with
his two appointments.

The decision achieved a decades-long goal of gay rights proponents, one
they had initially considered much easier to achieve than a
constitutional right to same-sex marriage. But even as the Supreme Court
established that right in 2015, workplace discrimination remained lawful
in most of the country. An employee who married a same-sex partner in
the morning could be fired that afternoon for being gay.

Monday's lopsided ruling, coming from a fundamentally conservative
court, was a surprise. Justice Gorsuch, who was Mr. Trump's first
appointment to the court, was joined by Chief Justice John G. Roberts
Jr. and Justices Ruth Bader Ginsburg, Stephen G. Breyer, Sonia Sotomayor
and Elena Kagan.

Supporters of L.G.B.T. rights were elated by the ruling, which they said
was long overdue.

``This is a simple and profound victory for L.G.B.T. civil rights,''
said Suzanne B. Goldberg, a law professor at Columbia. ``Many of us
feared that the court was poised to gut sex discrimination protections
and allow employers to discriminate based on sexual orientation and
gender identity, yet it declined the federal government's invitation to
take that damaging path.''

In remarks to reporters, Mr. Trump said he accepted the ruling. ``I've
read the decision,'' he said, ``and some people were surprised, but
they've ruled and we live with their decision.'' He added that it was a
``very powerful decision, actually.''

The Trump administration had urged the court to rule against gay and
transgender workers, and it has
\href{https://www.nytimes3xbfgragh.onion/2019/01/22/us/politics/transgender-ban-military-supreme-court.html}{barred
most transgender people} from serving in the military. The Department of
Health and Human Services
\href{https://www.nytimes3xbfgragh.onion/2020/06/12/us/politics/trump-transgender-rights.html}{issued
a regulation} on Friday that undid protections for transgender patients
against discrimination by doctors, hospitals and health insurance
companies.

Those actions involved different laws from the one at issue on Monday,
and the Supreme Court has allowed the military ban to go into effect
while lawsuits challenging it proceed. Still, the court's ruling
suggested that a new era in transgender rights has arrived.

The decision, covering two sets of cases, was the court's first on
lesbian, gay, bisexual and transgender rights since the
\href{https://www.nytimes3xbfgragh.onion/2018/06/27/us/politics/anthony-kennedy-retire-supreme-court.html?module=inline}{retirement
in 2018 of Justice Anthony M. Kennedy}, who wrote the majority opinions
in all four of the court's major gay rights decisions. Proponents of
those rights had worried that his departure would halt the progress of
the movement toward equality.

\includegraphics{https://static01.graylady3jvrrxbe.onion/images/2017/01/29/podcasts/the-daily-album-art/the-daily-album-art-articleInline-v2.jpg?quality=75\&auto=webp\&disable=upscale}

\hypertarget{listen-to-the-daily-a-landmark-supreme-court-ruling}{%
\subsubsection{Listen to `The Daily': A Landmark Supreme Court
Ruling}\label{listen-to-the-daily-a-landmark-supreme-court-ruling}}

A surprise majority of judges ruled that the Civil Rights Act protects
gay and transgender people from workplace discrimination.

transcript

Back to The Daily

bars

0:00/23:39

-23:39

transcript

\hypertarget{listen-to-the-daily-a-landmark-supreme-court-ruling-1}{%
\subsection{Listen to `The Daily': A Landmark Supreme Court
Ruling}\label{listen-to-the-daily-a-landmark-supreme-court-ruling-1}}

\hypertarget{hosted-by-michael-barbaro-produced-by-annie-brown-luke-vander-ploeg-asthaa-chaturvedi-and-sydney-harper-and-edited-by-mj-davis-lin-and-lisa-chow}{%
\subsubsection{Hosted by Michael Barbaro, produced by Annie Brown, Luke
Vander Ploeg, Asthaa Chaturvedi and Sydney Harper, and edited by M.J.
Davis Lin and Lisa
Chow}\label{hosted-by-michael-barbaro-produced-by-annie-brown-luke-vander-ploeg-asthaa-chaturvedi-and-sydney-harper-and-edited-by-mj-davis-lin-and-lisa-chow}}

\hypertarget{a-surprise-majority-of-judges-ruled-that-the-civil-rights-act-protects-gay-and-transgender-people-from-workplace-discrimination}{%
\paragraph{A surprise majority of judges ruled that the Civil Rights Act
protects gay and transgender people from workplace
discrimination.}\label{a-surprise-majority-of-judges-ruled-that-the-civil-rights-act-protects-gay-and-transgender-people-from-workplace-discrimination}}

\begin{itemize}
\item
  aimee stephens\\
  My name is Aimee Stephens. I'm 58 years old, and I live in Redford,
  Mich.
\item
  michael barbaro\\
  Aimee, I wonder if you could read from the letter that you handed your
  boss.
\item
  aimee stephens\\
  Sure.

  ``Dear friends and co-workers. I have known many of you for some time
  now, and I count you all as my friends. What I must tell you is very
  difficult for me and is taking all the courage I can muster. I am
  writing this both to inform you of a significant change in my life and
  to ask for your patience, understanding, and support, which I would
  treasure greatly. I have a gender identity disorder that I have
  struggled with my entire life. I have felt imprisoned in a body that
  does not match my mind, and this has caused me great despair and
  loneliness. With the support of my loving wife, I have decided to
  become the person that my mind already is. At the end of my vacation,
  on Aug. 26, 2013, I will return to work as my true self, Aimee
  Australia Stephens, in appropriate business attire. I realize that
  some of you may have trouble understanding this. In truth, I have had
  to live with it every day of my life, and even I do not fully
  understand it myself. It is my wish that I can continue my work at
  R.G. \& G.R. Harris Funeral Homes doing what I have always done, which
  is my best.''
\item
  {[}music{]}
\item
  aimee stephens\\
  I gave it to the boss. And then two weeks later, he came back with his
  own letter, which was my letter of dismissal. Basically, his letter to
  me was that, your services are no longer needed. This is what we're
  offering. You have 21 days to make up your mind. But if you accept
  severance package, you will have to agree to keep your mouth shut and
  not ever talk about this to anyone. And I didn't think I could live
  with that the rest of my life. At that point, I knew I had to do
  something. After all, this was not only happening to me, but to
  thousands of others. And the only thing I knew to do was basically to
  take it to court. That's what I did.
\item
  {[}music{]}
\item
  {[}crowd chanting{]}
\item
  archived recording\\
  We're coming on the air because of a major civil rights decision out
  of the United States Supreme Court.
\end{itemize}

michael barbaro

From The New York Times, I'm Michael Barbaro. This is ``The Daily.''

\begin{itemize}
\tightlist
\item
  archived recording\\
  The decision now is clear from the Supreme Court ---
\end{itemize}

michael barbaro

Today ---

\begin{itemize}
\item
  archived recording\\
  --- they have issued a ruling that now bans discrimination by
  employers against transgender individuals and gay individuals.
\item
  crowd chanting\\
  Trans lives matter! Trans lives matter! Trans lives matter!
\end{itemize}

michael barbaro

Adam Liptak on the surprise majority that decided the case.

It's Tuesday, June 16.

Adam, tell us about this ruling on Monday.

adam liptak

The Supreme Court issued a huge ruling, a very consequential ruling. It
said that all across the nation, it's no longer permissible for
employers to fire people simply for being gay or transgender. Now, you
might think that's already the state of the world. But in 27 states,
there's no federal protection for gay and transgender workers. Gay
people have a constitutional right to get married. They have since 2015.
They can get married on a Monday morning, and when their employer found
out about it Monday afternoon, they could be fired without consequence,
simply for being gay. Until Monday's ruling from the Supreme Court.

michael barbaro

So in the national debate over the rights for gay and transgender
people, this was a kind of untouched area --- employment.

adam liptak

Right, it's sort of surprising. I mean, most people I think, think it's
unlawful to discriminate against them for being gay or transgender. But
until this Supreme Court ruling on Monday, people were without
protection in about half the nation.

michael barbaro

And Adam, remind us of the specific cases that are involved in this
ruling. I know we've talked about them in the past on the show.

adam liptak

Yeah, so there are actually three separate cases, two of them involving
gay men, one involving a transgender woman. The cases involving gay men
were a government worker in Georgia and a skydiving instructor, both of
whom alleged in their lawsuits that they'd been fired for being gay. And
the third was a transgender woman named Aimee Stephens, who, Michael,
your listeners may remember, because ---

michael barbaro

Right.

adam liptak

--- you had a conversation with her and she described how, when she
announced she was going to assume the gender identity that she believed
was hers, the reaction of the funeral home for whom she worked was to
fire her.

michael barbaro

Right, this letter that she had spent years composing in her head and on
paper that told her friends and her colleagues and her boss who she was,
was actually what ended up getting her fired.

adam liptak

That's right.

michael barbaro

And, Adam, what was the central legal question posed in these three
different cases?

adam liptak

The question in the case is whether Title VII of the Civil Rights Act of
1964 --- a landmark piece of civil rights legislation which prohibits
discrimination based on race, religion, ethnicity, and sex --- applies
to sexual orientation and gender identity discrimination. And drilling
down just a little bit more, the key question is whether the phrase
``discrimination because of sex'' covers sexual orientation and gender
identity.

michael barbaro

Right, and as I recall from talking about these oral arguments with you
many months ago, the case very much rested, not just on what the entire
court thought of the phrase ``because of sex'' and what it meant, but
specifically what the conservative justices on the court, who are now in
the majority --- what they thought that that phrase meant.

adam liptak

That's right. I mean, you have a court where the four more liberal votes
are pretty much locked in. You know what they're going to do, and they
have to pick up a conservative vote. And the question with this court
was, the conservative vote they would ordinarily be sure of picking up
was that of Justice Anthony Kennedy, who wrote all four of the major gay
rights decisions before this one. But he retired in 2018, replaced by
Justice Brett Kavanaugh. And so the court has a five justice
conservative majority. And for the liberals to win, they'd have to pick
off at least one of them.

michael barbaro

And it sounds like they did that.

adam liptak

Oh, yeah, in fact they got two, Justice Gorsuch, President Trump's first
appointee, and they also picked up Chief Justice John G. Roberts. So you
got a 6-3 decision in the end. And, Michael, just to put that in
context, this is a very conservative court. This is a court that gay
rights advocates were terrified of. So to get a 6-3 victory from this
court on a consequential, stunning, vastly important decision is really
something.

michael barbaro

So let's talk about these two conservative justices who sided with the
liberal justices in this case.

adam liptak

Well, the key justice is Justice Gorsuch. He writes the majority
opinion. He's the only one whose language we have in front of us and
whose reasoning we know for certain. And he says it is impossible to
discriminate against a person for being homosexual or transgender
without discriminating against that individual based on sex. Consider,
for example, an employer with two employees, both of whom are attracted
to men. The two individuals are, to the employer's mind, materially
identical in all respects, except that one is a man and the other a
woman. If the employer fires the male employee for no reason other than
the fact he is attracted to men, the employer discriminates against him
for traits or actions it tolerates in his female colleague.

michael barbaro

So Gorsuch is arguing, you can't divorce discrimination based on sexual
identity, sexual orientation, from gender, and perhaps cultural
expectations of gender.

adam liptak

That's right. He says, listen, it may not be the only factor, but it's a
factor and that's good enough for this law.

michael barbaro

Right, and therefore, by his logic, that word, that phrase, ``because of
sex'' in 1964, clearly applies to gay and transgender people today.

adam liptak

Exactly right.

michael barbaro

And so what explanation do the other conservative justices who did not
join Gorsuch in the majority give for breaking with him, and with
Roberts, if Gorsuch and Roberts found a pretty conservative
justification for extending these rights to gay and transgender people?

adam liptak

So there are two separate dissents, but the theme that runs through both
of them, Michael, is that it's just not a natural way to read this set
of words. That in 1964, nobody thought that they were prohibiting
discrimination against gay people and transgender people. And Justice
Kavanaugh in his dissent says, even today, when you ask people what
``because of sex'' means to them, they will not typically say oh, that
means because of sexual orientation, because of gender identity. Then
Alito and Thomas, but not Kavanaugh, go on to talk about what they view
as the very pernicious consequences of the majority decision, which they
say will have an impact on restrooms and locker rooms and college sports
and maybe professional sports and religious employers and freedom of
speech. Justice Alito even says it might prohibit people from using
anything other than the preferred pronouns of the people to whom they
talk.

michael barbaro

Right, but of course in the end, those three justices were outvoted. And
two of their conservative allies went in the other direction. And I have
to say, and I don't know if this is the case for you, it seems
surprising that the majority opinion in this landmark gay and
transgender rights case was written by a conservative member of the
court.

adam liptak

Oh, yeah. No, I think it's a big surprise to me, big surprise to gay
rights advocates, big surprise to the L.G.B.T. community. I will say
this, Michael, that if you'd asked me in September what the outcome of
this case was going to be, I would have said it's classic 5-4,
conservatives against liberals.

michael barbaro

You said that on our show. You said this was ---

adam liptak

Right, but once it was argued, once we saw Gorsuch struggling with this
textual question at the argument in September, I started to think that
there was a live possibility it would be 5-4, the liberals plus Gorsuch.
And I guess the chief justice came along for the ride.

michael barbaro

But that makes me wonder, does a ruling like this, and the composition
of the majority, does that make you conclude that ultimately, we don't
really know this relatively new Supreme Court, this conservative
majority court, as well as perhaps, everyone thinks that they do?

adam liptak

That's a really important point. People on the left are very unhappy
that President Trump got to appoint two people to the Supreme Court. But
those two people don't vote together all that often. Overall, their
voting will be conservative. But they're individual people with
individual jurisprudential commitments, and they will from time to time
surprise you as Justice Gorsuch surprised us in this one.

michael barbaro

We'll be right back.

Adam, how is this ruling being received by those who are champions of
religious liberty? Because I have to imagine that they are not looking
favorably on a ruling that says every employer, including employers run
by people whose religion says that being homosexual is wrong, would
welcome this ruling.

adam liptak

Sure, they're nervous about this ruling. And in dissent, Justice Alito
said they're right to be nervous, that this ruling can make religious
people and religious employers on the hook for employment discrimination
if they just follow the dictates of their faith. Justice Gorsuch, who is
ordinarily very sympathetic to those kinds of claims of religion said,
listen, that's not this case. We'll deal with that case down the line.
But he did say, let me tell you, first of all, Title VII itself has an
exemption for religious employers. There are other laws and
constitutional provisions which can protect religious people and
religious employers. So his basic answer is, we'll get back to you on
that.

michael barbaro

Hm, so this ruling may leave open the possibility that an employer could
bring future cases that could make it all the way up to the Supreme
Court challenging this ruling on the grounds of religious liberty,
saying this ruling infringes on my right to practice my religion the way
I see fit.

adam liptak

Yeah, clashes between religious employers and their employees are
commonplace, and we're waiting, even in this term, for a decision on
whether employment discrimination laws apply to Catholic school
teachers. So that clash is something that's very much on the front
burner at the court, but we don't have an answer yet.

michael barbaro

Adam, you've been covering the Supreme Court for The Times for more than
a decade. And you have watched this debate over L.G.B.T.Q. rights play
out before the justices on many occasions. Where does this decision
stack up in that history of the decisions that they have made?

adam liptak

Well, for gay rights, it easily ranks with the top three. It ranks with
the decisions in which the court struck down a Texas law making gay sex
a crime. It stands with the decision establishing a constitutional right
to same sex marriage. And now for gay people, we have this enormously
consequential decision protecting them from employment discrimination.
And let's not forget, for transgender people, we have the first major
transgender rights case from the Supreme Court ever.

michael barbaro

So by definition, this is a historic case when it comes to rights.

adam liptak

Some historic cases are symbolic only. This historic case will have a
real-world impact for lots and lots of people.

michael barbaro

It's interesting that this decision comes three and a half years into an
administration, the Trump administration, that has repeatedly taken
actions to restrict the rights of transgender Americans, in particular.
You know, banning them from serving in the military, telling the
military to stop paying for gender confirmation surgery. And just about
a week ago, narrowing the definition of sex discrimination in the
Affordable Care Act to omit protections for transgender people. So how
does Monday's decision affect those? Because after all, the United
States government is a major employer, right?

adam liptak

Yeah, so as an employer, it's subject to Title VII like other kinds of
employers. When it's talking about health care, when it's talking about
the military, those are different statutes, and whether it has the power
or not to disadvantage transgender people is an open question. This
decision of course, gives you some sense that challenges to Trump
administration actions would meet with positive reception at the court.
But they're different statutes in different settings, and the president
gets a lot of deference when it's the military who's involved. We do
have a quick sense that President Trump is prepared to accept the
Supreme Court's decision on Monday. He was asked about it at a press
availability and he said, they ruled, and we live with their decision, a
very powerful decision.

michael barbaro

Speaking of that kind of atmospheric change, I have noticed that in the
protests that we've been seeing all over the United States for the past
few weeks, that in addition to protesting against racism, demonstrators
have taken up the issue of trans rights and calling for the protection
of black trans people, for example.

adam liptak

Yeah, so society is moving very fast on these issues. The protests
reflect that. One thing that struck me that also reflects it is that
more than 200 major corporations filed briefs in these cases saying,
please subject us to these laws, please make it possible to sue us.
Because the commitment among very large parts of society to equality for
black people, gay people, trans people is moving quite quickly in the
direction of equality. And the Supreme Court, which is seldom very far
out of step with the American public, as reflected in this decision,
seems to agree.

michael barbaro

Adam, what has been the response from the plaintiffs in this case? I
remember speaking with Aimee Stephens after the oral arguments, and she
had some real doubts about whether the court was going to ultimately
rule in her favor.

adam liptak

Yeah, that's right, Michael. There are three plaintiffs in the three
cases, two of them have sadly died, including Aimee Stephens, who died
just a few weeks ago. So she didn't see the result of her lawsuit. But
she did sketch out some thoughts in anticipation that the court might
rule in her favor. So she wrote these words. ``Firing me because I'm
transgender was discrimination, plain and simple. And I am glad the
court recognized that what happened to me is wrong and illegal. I am
thankful that the court said my transgender siblings and I have a place
in our laws. It made me feel safer and more included in society.''

michael barbaro

Thank you, Adam.

adam liptak

Thank you, Michael.

{[}music{]}

aimee stephens

They asked me a question. And that question was, are you willing to see
this through to the end?

And I told them then that I was raised on a farm, that I was used to
hard work, and that I didn't give up so easily. They've had people, I
guess, in the past, who started this process, and it can get to you to
the point that you just want it to be over. And you say well, I'm done.
I'm not going any further. Or perhaps they try to settle out of court.

I had in my mind what I needed to do, and it wasn't to really settle out
of court. It wasn't to just give up and walk away. And that yes, I would
see this to the end.

michael barbaro

We'll be right back.

Here's what else you need to know today.

\begin{itemize}
\tightlist
\item
  archived recording (dermot shea)\\
  Good afternoon, everyone. It's been a tough few weeks for the
  N.Y.P.D., for the city, really, for the whole country.
\end{itemize}

michael barbaro

In a major reform by the nation's largest police force, New York City is
disbanding its anti-crime unit, a team of 600 officers who patrol the
city in plain clothes that has been involved in some of the city's most
notorious police shootings.

\begin{itemize}
\tightlist
\item
  archived recording (dermot shea)\\
  Make no mistake, this is a seismic shift in the culture of how the
  N.Y.P.D. polices this great city. It will be felt immediately in the
  communities that we protect.
\end{itemize}

michael barbaro

The decision makes the N.Y.P.D. one of the first police departments in
the country to begin defunding and dismantling its operations in the
wake of nationwide protests.

\begin{itemize}
\tightlist
\item
  archived recording (dermot shea)\\
  We can do it better. We can do it smarter. And we will.
\end{itemize}

michael barbaro

And on Monday, the Food and Drug Administration reversed course and
revoked its emergency authorization of two malaria drugs,
hydroxychloroquine and chloroquine, as treatments for Covid-19. In
March, the F.D.A. allowed the drugs to be used by hospitals treating
patients with the coronavirus. But studies since then have shown that
the drugs are unlikely to be effective, despite claims by President
Trump, who has repeatedly promoted both of them, and who said he had
taken one of them himself.

That's it for ``The Daily.'' I'm Michael Barbaro, see you tomorrow.

\href{https://www.nytimes3xbfgragh.onion/interactive/2020/06/15/us/supreme-court-major-cases-2020.html}{}

\includegraphics{https://static01.graylady3jvrrxbe.onion/images/2020/06/18/us/-promo-1592492336549/-promo-1592492336549-articleLarge.jpg}

\hypertarget{the-supreme-court-aligned-with-public-opinion-in-most-major-cases-this-term}{%
\subsection{The Supreme Court Aligned With Public Opinion in Most Major
Cases This
Term}\label{the-supreme-court-aligned-with-public-opinion-in-most-major-cases-this-term}}

The justices confronted an unusually potent mix of political and social
issues in the middle of both a presidential election year and a public
health crisis.

The Supreme Court is generally not very far out of step with popular
opinion, and large majorities of Americans oppose employment
discrimination based on sexual orientation, and substantial ones oppose
it when based on gender identity. More than 200 major corporations
\href{https://www.supremecourt.gov/DocketPDF/17/17-1618/106953/20190703115551379_2019.07.03\%20-\%20Amicus\%20Brief\%20of\%20206\%20Businesses\%20in\%20Support\%20of\%20Employees.pdf}{filed
a brief} supporting the gay and transgender employees in the cases
before the court.

The decision was both symbolic and consequential, and it followed in the
tradition of landmark rulings on discrimination. Unlike Brown v. Board
of Education, the 1954 decision that said racially segregated public
schools violated the Constitution; Loving v. Virginia, the 1967 decision
that struck down bans on interracial marriage; and Obergefell v. Hodges,
the 2015 decision that struck down state bans on same-sex marriage, the
new decision did not involve constitutional rights.

Instead, the question for the justices was the meaning of a statute,
Title VII of the Civil Rights Act of 1964, which bars employment
discrimination based on race, religion, national origin and sex. They
had to decide whether that last prohibition --- discrimination ``because
of sex'' --- applies to many millions of gay and transgender workers.

Justice Gorsuch wrote that it did.

``An employer who fires an individual for being homosexual or
transgender fires that person for traits or actions it would not have
questioned in members of a different sex,'' he wrote.

``It is impossible,'' Justice Gorsuch wrote, ``to discriminate against a
person for being homosexual or transgender without discriminating
against that individual based on sex.''

The decision will allow people who say they were discriminated against
in the workplace based on their sexual orientation or gender identity to
file lawsuits, just as people claiming race and sex discrimination may.
The plaintiffs will have to offer evidence, of course, and employers may
respond that they had reasons unrelated to discrimination for their
decisions.

Justice Samuel A. Alito Jr., in a dissent joined by Justice Clarence
Thomas, wrote that the majority had abandoned its judicial role.

``There is only one word for what the court has done today:
legislation,'' Justice Alito wrote. ``The document that the court
releases is in the form of a judicial opinion interpreting a statute,
but that is deceptive.''

``A more brazen abuse of our authority to interpret statutes is hard to
recall,'' he wrote. ``The court tries to convince readers that it is
merely enforcing the terms of the statute, but that is preposterous.''

The common understanding of sex discrimination in 1964, Justice Alito
wrote, was bias against women or men and did not encompass
discrimination based on sexual orientation and gender identity. If
Congress wanted to protect gay and transgender workers, he wrote, it
could pass a new law.

``Discrimination `because of sex' was not understood as having anything
to do with discrimination because of sexual orientation or transgender
status'' in 1964, he wrote. ``Any such notion would have clashed in
spectacular fashion with the societal norms of the day.''

Justice Alito added that the majority's decision would have pernicious
consequences.

He said the majority left open, for instance, questions about access to
restrooms and locker rooms. ``For women who have been victimized by
sexual assault or abuse,'' he wrote, ``the experience of seeing an
unclothed person with the anatomy of a male in a confined and sensitive
location such as a bathroom or locker room can cause serious
psychological harm.''

Nor did the majority address, he said, how its ruling would affect
sports, college housing, religious employers, health care or free
speech.

``After today's decision,'' Justice Alito wrote, ``plaintiffs may claim
that the failure to use their preferred pronoun violates one of the
federal laws prohibiting sex discrimination.''

``Although the court does not want to think about the consequences of
its decision, we will not be able to avoid those issues for long,'' he
wrote. ``The entire federal judiciary will be mired for years in
disputes about the reach of the court's reasoning.''

Justice Gorsuch responded that the court's ruling was narrow. ``We do
not purport to address bathrooms, locker rooms or anything else of the
kind,'' he wrote. ``Whether other policies and practices might or might
not qualify as unlawful discrimination or find justifications under
other provisions of Title VII are questions for future cases, not
these.''

He added that Title VII itself included protections for religious
employers and that a separate federal law and the First Amendment also
allow religious groups latitude in their employment decisions.

Justice Brett M. Kavanaugh, Mr. Trump's other appointment to the court,
issued a separate dissent making a point about statutory interpretation.
``Courts must follow ordinary meaning, not literal meaning,'' he wrote,
adding that the ordinary meaning of ``because of sex'' does not cover
discrimination based on sexual orientation or gender identity.

``Seneca Falls was not Stonewall,'' he wrote. ``The women's rights
movement was not (and is not) the gay rights movement, although many
people obviously support or participate in both. So to think that sexual
orientation discrimination is just a form of sex discrimination is not
just a mistake of language and psychology, but also a mistake of history
and sociology.''

The court considered two sets of cases.
\href{https://www.supremecourt.gov/oral_arguments/argument_transcripts/2019/17-1618_7k47.pdf}{The
first} concerned a pair of lawsuits from gay men who said they were
fired because of their sexual orientation:
\href{http://www.scotusblog.com/case-files/cases/bostock-v-clayton-county-georgia/}{Bostock
v. Clayton County, Ga.}, No. 17-1618, and
\href{http://www.scotusblog.com/case-files/cases/altitude-express-inc-v-zarda/}{Altitude
Express Inc. v. Zarda}, No. 17-1623.

The first case was filed by Gerald Bostock, who was fired from a
government program that helped neglected and abused children in Clayton
County, Ga., just south of Atlanta, after he joined a gay softball
league.

The second was brought by a skydiving instructor, Donald Zarda, who also
said he was fired because he was gay. His dismissal followed a complaint
from a female customer who had expressed concerns about being strapped
to Mr. Zarda during a tandem dive. Mr. Zarda, hoping to reassure the
customer, told her that he was ``100 percent gay.''

The case on gender identity,
\href{http://www.scotusblog.com/case-files/cases/r-g-g-r-harris-funeral-homes-inc-v-equal-opportunity-employment-commission/}{R.G.
\& G.R. Harris Funeral Homes Inc. v. Equal Employment Opportunity
Commission}, No. 18-107, was brought by a transgender woman,
\href{https://www.nytimes3xbfgragh.onion/2020/05/12/us/aimee-stephens-supreme-court-dead.html}{Aimee
Stephens}, who was fired from a Michigan funeral home after she
announced in 2013 that she was a transgender woman and would start
working in women's clothing.

Mr. Zarda died in an accident in 2014, and
\href{https://www.nytimes3xbfgragh.onion/2020/05/12/us/aimee-stephens-supreme-court-dead.html}{Ms.
Stephens died} on May 12. Their estates continued to pursue their cases
after their deaths.

Critics sometimes say that the Congress does not hide elephants in mouse
holes, Justice Gorsuch wrote on Monday, meaning that lawmakers do not
take enormous steps with vague terms or in asides.

``We can't deny that today's holding --- that employers are prohibited
from firing employees on the basis of homosexuality or transgender
status --- is an elephant,'' he wrote. ``But where's the mouse hole?
Title VII's prohibition of sex discrimination in employment is a major
piece of federal civil rights legislation. It is written in starkly
broad terms. It has repeatedly produced unexpected applications, at
least in the view of those on the receiving end of them.''

``This elephant,'' he wrote, ``has never hidden in a mouse hole; it has
been standing before us all along.''

Advertisement

\protect\hyperlink{after-bottom}{Continue reading the main story}

\hypertarget{site-index}{%
\subsection{Site Index}\label{site-index}}

\hypertarget{site-information-navigation}{%
\subsection{Site Information
Navigation}\label{site-information-navigation}}

\begin{itemize}
\tightlist
\item
  \href{https://help.nytimes3xbfgragh.onion/hc/en-us/articles/115014792127-Copyright-notice}{©~2020~The
  New York Times Company}
\end{itemize}

\begin{itemize}
\tightlist
\item
  \href{https://www.nytco.com/}{NYTCo}
\item
  \href{https://help.nytimes3xbfgragh.onion/hc/en-us/articles/115015385887-Contact-Us}{Contact
  Us}
\item
  \href{https://www.nytco.com/careers/}{Work with us}
\item
  \href{https://nytmediakit.com/}{Advertise}
\item
  \href{http://www.tbrandstudio.com/}{T Brand Studio}
\item
  \href{https://www.nytimes3xbfgragh.onion/privacy/cookie-policy\#how-do-i-manage-trackers}{Your
  Ad Choices}
\item
  \href{https://www.nytimes3xbfgragh.onion/privacy}{Privacy}
\item
  \href{https://help.nytimes3xbfgragh.onion/hc/en-us/articles/115014893428-Terms-of-service}{Terms
  of Service}
\item
  \href{https://help.nytimes3xbfgragh.onion/hc/en-us/articles/115014893968-Terms-of-sale}{Terms
  of Sale}
\item
  \href{https://spiderbites.nytimes3xbfgragh.onion}{Site Map}
\item
  \href{https://help.nytimes3xbfgragh.onion/hc/en-us}{Help}
\item
  \href{https://www.nytimes3xbfgragh.onion/subscription?campaignId=37WXW}{Subscriptions}
\end{itemize}
