Sections

SEARCH

\protect\hyperlink{site-content}{Skip to
content}\protect\hyperlink{site-index}{Skip to site index}

\href{https://myaccount.nytimes3xbfgragh.onion/auth/login?response_type=cookie\&client_id=vi}{}

\href{https://www.nytimes3xbfgragh.onion/section/todayspaper}{Today's
Paper}

\href{/section/opinion}{Opinion}\textbar{}White America Wants Me to
Conform. I Won't Do It.

\begin{itemize}
\item
\item
\item
\item
\item
\item
\end{itemize}

Advertisement

\protect\hyperlink{after-top}{Continue reading the main story}

\href{/section/opinion}{Opinion}

Supported by

\protect\hyperlink{after-sponsor}{Continue reading the main story}

THE STONE

\hypertarget{white-america-wants-me-to-conform-i-wont-do-it}{%
\section{White America Wants Me to Conform. I Won't Do
It.}\label{white-america-wants-me-to-conform-i-wont-do-it}}

Even at elite universities, I was exposed to the disease that has
endangered black lives for so long.

By Chris Lebron

Mr. Lebron is an associate professor of philosophy at Johns Hopkins and
the author of ``The Making of Black Lives Matter: A Brief History of an
Idea.''

\begin{itemize}
\item
  June 16, 2020
\item
  \begin{itemize}
  \item
  \item
  \item
  \item
  \item
  \item
  \end{itemize}
\end{itemize}

\includegraphics{https://static01.graylady3jvrrxbe.onion/images/2020/06/16/opinion/16stone-lebron3/merlin_170065476_8c2008f2-363c-4aa2-bc69-394ff748cd43-articleLarge.jpg?quality=75\&auto=webp\&disable=upscale}

It could have been the day I died.

It was a fall morning in 2000 and I was on the New York City subway,
heading from the Lower East Side to my not very glamorous
stock-brokering job in the Financial District. It was casual Friday and
I took that theme to heart, wearing sweats and a leather jacket. In my
car was a large white youth with a backpack who was making it his
business to pace the car banging on the doors and windows. The conductor
had apparently called ahead for the police, and a group of officers met
the train and approached my car from the platform when we pulled in. I
just wanted to get to work, so I stood at the door ready to leave this
scene, but the police swarmed my position. Being a good citizen, I went
to move to another door so that they could apprehend the culprit.

Instead, I was met by an officer yelling with the kind of authority that
surely got him an A at the academy, ``Don't you move!'' They also
trained his comrades such that each and every officer put his hand on
his gun, ready to riddle my body with justice. The now infamous Amy
Cooper has given white women a bad name recently, but that day it was a
white woman's panicked plea that saved me: ``No! No! Over there!'' she
yelled, pointing to our rambunctious fellow rider. I went to work, alive
but also a little dead inside.

As the nation is being rocked by black, brown and white rage over police
killings of George Floyd, Breonna Taylor and so many other black
Americans, I think of that day frequently. The rage is flaring now
because it has become entirely too normal for a white man with power to
casually, so very casually, apply enough pressure to a vulnerable black
neck and cause the soul to vacate in an untimely fashion. America's
propensity to dispense with black life is a sickness, a pathology that
authorizes public murder for the sake of white supremacy.

Three major medical associations recently declared
\href{https://www.cnn.com/2020/06/01/health/racism-public-health-issue-police-brutality-wellness-bn/index.html}{racism
and police brutality public health crises}. But I had long ago begun
thinking of racism as a kind of social disease. I even gave it a name
--- Racial Diminishment Syndrome. This disease, like the coronavirus, is
hard to detect, highly contagious and often deadly. Many of the infected
exhibit no symptoms, but may be ``spreaders.'' When R.D.S. is active in
public spaces (almost always), social distancing will decrease the
likelihood of extreme illness or untimely death.

Consider recent cases like Ahmaud Arbery and Atatiana Jefferson: Cause
of death --- R.D.S. by way of jogging near white people and standing in
a window where the police could see you, respectively. George Floyd's
alleged offense was passing a fake \$20 bill at a convenience store.
Corporate barons rob the American people daily to the tune of millions,
but it was Floyd who got a knee to the neck.

However, R.D.S. need not resolve morbidly. For instance, I am still
alive. The syndrome more commonly results in discomfort, inconvenience
and the sort of pains that eventually go away but the memories of which
do not. Here we are talking about being pulled over for driving while
black; a hotel patron assuming I am staff while walking the hall to my
own room; professional colleagues failing to consider my point of view;
and on and on. Social distancing can help prevent this kind of exposure,
but it goes only so far.

In 2007, my wife and I moved to Charlottesville, Va. Before arriving I
had been heartened by its electoral map --- bright blue surrounded by
socially menacing red. Once there, I soon learned that a blue town is in
some ways worse than a red one because everyone is possessed of the
conviction of their own racial virtues, and they're almost all very
wrong. My first three years in Charlottesville were spent coldly coming
to terms with its radical segregation and the absence of a black middle
class. I observed as the police harassed homeless black men on the
beloved Downtown Mall while the white frat boys got to shamelessly
litter the streets surrounding the University of Virginia with beer
kegs. Dionysus surely considered these misfits his chosen ones.

By 2010, nine years after the day I could have died, I was hardly
leaving the house. When I did venture out, I kept to myself, avoided
small talk, went straight home after doing what I needed to do, grateful
when I finally made it back to the safe comfort of my own home. Nothing
in particular was happening in the world other than America just being
America.

With middle age looming on the horizon, my tolerance for being a social
other and possibly in danger just by walking out my front door was
atrophying. The equation was becoming clearer in my mind: Me + white
spaces = precarity. At the University of Virginia, where I was an
assistant professor, I received lessons from senior colleagues who had
the power to make or break my career on the need for humility in work I
sought to publish. Then there was the time that a colleague, upon
learning my wife and I had accepted positions at Yale, saw fit to walk
into my office and quip, ``If I were angry at you, I would tell you to
go {[}expletive{]} \ldots{} \emph{but I'm not!}'' He \emph{was} angry,
and he \emph{did} effectively just tell me that. Social distance was
needed; this man was a vector of R.D.S.

When the Black Lives Matter movement took hold in the wake of Trayvon
Martin's murder in 2012, it was almost as if an incantation had been
whispered into the ether, because for the next five years America turned
into what looked like a sizzle reel for a black snuff film, as images of
shot black body after choked black body after broken black body after
dragged on the sidewalk black body after violently removed from the
public pool black body made their way to our computers and phones. But
this was just the most grotesque presentation of R.D.S. My own
experiences on the ground were more mundane, but terrifying in ways one
can't quite put into words.

The northwestern edge of Yale's campus is rimmed with expensive shops.
The highlight of these is an Apple store. One especially sunny and
optimistic-feeling day, as I was walking back to my office from grabbing
lunch I witnessed a scene that triggered my subway memory. About 10
police officers and six vehicles, some of them vans big enough for
several suspects, had converged on the body of a lone weeping young
black male, about 20 years old by my guess. The police had him sitting
in full display on the curb instead of in a car or wagon, thus a large
white audience of Yale students were learning just how dangerous the New
Haven natives were.

As I passed, I heard this young man sob: ``What you expect me to do? I'm
tired, I'm tired!'' Maybe his onlookers were confused about his fatigue
but I wasn't. He was tired of a mega-rich institution that thrived
despite the black poverty that circled the institution like a Trumpian
wall. He was tired of things like Yale building two new residential
colleges at a cost of hundreds of millions of dollars, yet to look at
the construction crews at the building site right next to one of the
blackest areas of New Haven, you'd swear someone had said, ``Hire anyone
except those black people over there.''

I didn't care about whatever property he allegedly lifted from the Apple
store because I know what had been lifted from him and others on the
social edges --- a sense of being full and fully respected members of
the richest nation on earth. As I walked by this young man I could only
wish he had kept his social distance. R.D.S. will get you.

By the time I reached my present job at Johns Hopkins University, I had
essentially given up. When the small number of my black colleagues
decided to challenge the university's
\href{https://publicsafetyinitiatives.jhu.edu/assets/uploads/sites/8/2019/01/proposal_summary.pdf}{wish
to establish an armed police force on campus}, one likely to be staffed
by former officers from the Baltimore Police Department
---\href{https://mappingpoliceviolence.org/compare-police-departments}{one
of the deadliest in the nation}--- I never bothered to join them.
Valiant as their attempt was, I know this: When fearful whites and
co-opted blacks decide the scariest people on earth are poor blacks,
absolutely nothing can stop them from putting the police between them
and the black folks they help to keep scary.

The resolution went forward despite opposition and passed, but last week
the administration decided to delay the arrival of the armed force by
two years. In the face of entire cities defunding or disbanding the
police, this can't help but strike me as a hedge for a return of the
status quo, rotten as it is. If these black people won't stay in their
designated spaces, the police will help remind them. It will be a great
surprise if I am not driven to my keyboard within the next few years
writing about our campus's very own George Floyd moment. In the
meantime, I keep my distance --- I don't want to be a candidate for such
a moment.

It is not only instances that can result in physical harm I avoid. I
almost never attend casual faculty functions. I don't go out for drinks.
I don't entertain for dinner parties and I don't seek to ingratiate
myself into the lives of my white colleagues. I have a great deal of
respect for the many white academics I have worked with. But some of
them remain vectors of R.D.S. nonetheless. I know so much about many of
these people because I know what it is white America needs me to be for
it to allow me inside. What they need is a version of myself that
acquiesces and conforms, that is never displeased or contrary --- or
angry.

I won't do it. I'll social distance. It's already hard enough to breathe
in America. Every day you feel like you're living with a knee on your
neck. It's a sickness. And I am not immune.

Chris Lebron is an associate professor of philosophy at Johns Hopkins
and the author of ``The Making of Black Lives Matter: A Brief History of
an Idea.''

\begin{center}\rule{0.5\linewidth}{\linethickness}\end{center}

\emph{\textbf{Now in print:}}
\emph{``}\href{http://bitly.com/1MW2kN3}{\emph{Modern Ethics in 77
Arguments}}\emph{'' and ``}\href{http://bitly.com/1MW2kN3}{\emph{The
Stone Reader: Modern Philosophy in 133 Arguments}}\emph{,'' with essays
from the series, edited by Peter Catapano and Simon Critchley, published
by Liveright Books.}

\emph{The Times is committed to publishing}
\href{https://www.nytimes3xbfgragh.onion/2019/01/31/opinion/letters/letters-to-editor-new-york-times-women.html}{\emph{a
diversity of letters}} \emph{to the editor. We'd like to hear what you
think about this or any of our articles. Here are some}
\href{https://help.nytimes3xbfgragh.onion/hc/en-us/articles/115014925288-How-to-submit-a-letter-to-the-editor}{\emph{tips}}\emph{.
And here's our email:}
\href{mailto:letters@NYTimes.com}{\emph{letters@NYTimes.com}}\emph{.}

\emph{Follow The New York Times Opinion section on}
\href{https://www.facebookcorewwwi.onion/nytopinion}{\emph{Facebook}}\emph{,}
\href{http://twitter.com/NYTOpinion}{\emph{Twitter (@NYTopinion)}}
\emph{and}
\href{https://www.instagram.com/nytopinion/}{\emph{Instagram}}\emph{.}

Advertisement

\protect\hyperlink{after-bottom}{Continue reading the main story}

\hypertarget{site-index}{%
\subsection{Site Index}\label{site-index}}

\hypertarget{site-information-navigation}{%
\subsection{Site Information
Navigation}\label{site-information-navigation}}

\begin{itemize}
\tightlist
\item
  \href{https://help.nytimes3xbfgragh.onion/hc/en-us/articles/115014792127-Copyright-notice}{©~2020~The
  New York Times Company}
\end{itemize}

\begin{itemize}
\tightlist
\item
  \href{https://www.nytco.com/}{NYTCo}
\item
  \href{https://help.nytimes3xbfgragh.onion/hc/en-us/articles/115015385887-Contact-Us}{Contact
  Us}
\item
  \href{https://www.nytco.com/careers/}{Work with us}
\item
  \href{https://nytmediakit.com/}{Advertise}
\item
  \href{http://www.tbrandstudio.com/}{T Brand Studio}
\item
  \href{https://www.nytimes3xbfgragh.onion/privacy/cookie-policy\#how-do-i-manage-trackers}{Your
  Ad Choices}
\item
  \href{https://www.nytimes3xbfgragh.onion/privacy}{Privacy}
\item
  \href{https://help.nytimes3xbfgragh.onion/hc/en-us/articles/115014893428-Terms-of-service}{Terms
  of Service}
\item
  \href{https://help.nytimes3xbfgragh.onion/hc/en-us/articles/115014893968-Terms-of-sale}{Terms
  of Sale}
\item
  \href{https://spiderbites.nytimes3xbfgragh.onion}{Site Map}
\item
  \href{https://help.nytimes3xbfgragh.onion/hc/en-us}{Help}
\item
  \href{https://www.nytimes3xbfgragh.onion/subscription?campaignId=37WXW}{Subscriptions}
\end{itemize}
