Sections

SEARCH

\protect\hyperlink{site-content}{Skip to
content}\protect\hyperlink{site-index}{Skip to site index}

\href{https://www.nytimes3xbfgragh.onion/section/world/asia}{Asia
Pacific}

\href{https://myaccount.nytimes3xbfgragh.onion/auth/login?response_type=cookie\&client_id=vi}{}

\href{https://www.nytimes3xbfgragh.onion/section/todayspaper}{Today's
Paper}

\href{/section/world/asia}{Asia Pacific}\textbar{}Worst Clash in Decades
on Disputed India-China Border Kills 20 Indian Troops

\url{https://nyti.ms/30KpacT}

\begin{itemize}
\item
\item
\item
\item
\item
\item
\end{itemize}

Advertisement

\protect\hyperlink{after-top}{Continue reading the main story}

Supported by

\protect\hyperlink{after-sponsor}{Continue reading the main story}

\hypertarget{worst-clash-in-decades-on-disputed-india-china-border-kills-20-indian-troops}{%
\section{Worst Clash in Decades on Disputed India-China Border Kills 20
Indian
Troops}\label{worst-clash-in-decades-on-disputed-india-china-border-kills-20-indian-troops}}

Under nationalist leaders, the nuclear-armed rivals have been
increasingly assertive toward neighbors, risking open conflict. They
have contested their Himalayan border for generations.

\includegraphics{https://static01.graylady3jvrrxbe.onion/images/2020/06/16/world/16india-china2/merlin_173582595_29c6957d-37f0-4340-90f5-20e2fa36dc03-articleLarge.jpg?quality=75\&auto=webp\&disable=upscale}

\href{https://www.nytimes3xbfgragh.onion/by/jeffrey-gettleman}{\includegraphics{https://static01.graylady3jvrrxbe.onion/images/2018/10/10/multimedia/author-jeffrey-gettleman/author-jeffrey-gettleman-thumbLarge.png}}\href{https://www.nytimes3xbfgragh.onion/by/hari-kumar}{\includegraphics{https://static01.graylady3jvrrxbe.onion/images/2019/12/13/reader-center/author-hari-kumar/author-hari-kumar-thumbLarge.png}}\href{https://www.nytimes3xbfgragh.onion/by/sameer-yasir}{\includegraphics{https://static01.graylady3jvrrxbe.onion/images/2019/11/22/reader-center/author-sameer-yasir/author-sameer-yasir-thumbLarge.png}}

By
\href{https://www.nytimes3xbfgragh.onion/by/jeffrey-gettleman}{Jeffrey
Gettleman}, \href{https://www.nytimes3xbfgragh.onion/by/hari-kumar}{Hari
Kumar} and
\href{https://www.nytimes3xbfgragh.onion/by/sameer-yasir}{Sameer Yasir}

\begin{itemize}
\item
  Published June 16, 2020Updated June 29, 2020
\item
  \begin{itemize}
  \item
  \item
  \item
  \item
  \item
  \item
  \end{itemize}
\end{itemize}

NEW DELHI --- The worst
\href{https://www.nytimes3xbfgragh.onion/2020/06/17/world/asia/india-china-border-clashes.html}{border
clash between India and China} in more than 40 years left 20 Indian
soldiers dead and dozens believed captured, Indian officials said on
Tuesday, raising tensions between nuclear-armed rivals who have
increasingly been flexing their diplomatic and military muscle.

For the past several weeks, after
\href{https://www.nytimes3xbfgragh.onion/2020/05/30/world/asia/india-china-border.html}{a
series of brawls} along their disputed border, China and India have been
building up their forces in the remote Galwan Valley, high up in the
Himalayas.

As they dug into opposing positions, adding tinder to a long-smoldering
conflict, China took an especially muscular posture, sending in
artillery, armored personnel carriers, dump trucks and excavators. On
Monday night, a huge fight broke out between Chinese and Indian troops
in roughly the same barren area where these two nations, the world's
most populous, had fought a war in 1962.

Military and political analysts say the two countries do not want a
further escalation --- particularly India, where military forces are
nowhere near as powerful as China's --- but they may struggle to find a
way out of the conflict that does not hint at backing down.

Both countries and their nationalist leaders, President Xi Jinping of
China and Prime Minister
\href{https://www.nytimes3xbfgragh.onion/2020/06/17/world/asia/india-china-border-clashes.html}{Narendra
Modi} of India, have taken increasingly assertive postures that pose
real risks of the conflict spinning out of control.

``Neither PM Modi or President Xi want a war, but neither can relinquish
their territorial claims either,'' said
\href{https://carnegieendowment.org/experts/198}{Ashley J. Tellis,}a
senior fellow at the Carnegie Endowment for International Peace in
Washington.

\includegraphics{https://static01.graylady3jvrrxbe.onion/images/2020/06/16/world/16india-china/16india-china-articleLarge.jpg?quality=75\&auto=webp\&disable=upscale}

What's happening along the Himalayan border is an unusual kind of
warfare. As in the brawls last month, Chinese and Indian soldiers fought
fiercely without firing a shot --- at least that's what officials on
both sides contend. They say the soldiers followed their de facto border
code not to use firearms and went at each other with fists, rocks and
wooden clubs, some possibly studded with nails or wrapped in barbed
wire.

At first, India's military said only three Indian troops had been killed
in the clash, where the Ladakh region of India abuts Aksai Chin, an area
controlled by China but claimed by both countries. But late Tuesday
night, a military spokesman said that 17 other Indian soldiers had
succumbed to injuries sustained in the clash, bringing the total dead to
20.

An Indian commander said dozens of soldiers were missing, apparently
captured by the Chinese. Indian television channels reported that
several Chinese soldiers had been killed, as well, citing high-level
Indian government sources. Chinese officials did not comment on that.

It's not clear what India can do now. Mr. Modi and his Hindu nationalist
party have pursued a forceful foreign policy that emphasizes India's
growing role in the world and last year, after a devastating suicide
attack that India blamed on a Pakistani terror group, Mr. Modi ordered
airstrikes on
\href{https://www.nytimes3xbfgragh.onion/2020/06/29/world/asia/pakistan-stock-exchange-shooting.html}{Pakistan},
bringing the two countries to the brink of war.

But India is in no shape to risk a war against China --- especially now,
as it slips deeper into the economic and health crisis caused by the
coronavirus, which has cost the country more than 100 million jobs.

``Whatever India might want to do it's not in a position to do,'' said
Bharat Karnad, a professor of security studies at the Center for Policy
Research at New Delhi.

``The Modi government is in a difficult position,'' he said. ``This is
bound to escalate.''

And, he added, ``we are not prepared for this kind of escalation.''

Mr. Xi has been doubling
\href{https://www.nytimes3xbfgragh.onion/2020/05/24/world/asia/china-hong-kong-taiwan.html}{down
on China's territorial claims across Asia}, backing up arguments with
the threat of force or sometimes even the use of force. In recent weeks,
the Chinese have tightened their grip on the semiautonomous region of
Hong Kong; menaced Taiwan; and sunk a Vietnamese fishing boat in the
South China Sea.

Image

China's leader, Xi Jinping, and Prime Minister Narendra Modi of India in
Mamallapuram, India, last year.Credit...Associated Press

The upshot, scholars say, is a dangerous break from the past. China and
India, with their growing ambitions and growing militaries are
increasingly bumping up against each other along their 2,100-mile
border.

``Over the past several decades there's been incessant confrontation
between China and India, but proudly there have been no shots fired or
deaths,'' said Long Xingchun, a professor at China West Normal
University in southwest China who studies relations with India.

But now, he added, neither side is as willing to compromise, raising the
risks of more fighting, even if the countries don't actually want to go
to war.

``There was no reason for this to happen,'' he said. ``Unless it was a
military standoff that got out of control.''

The State Department in Washington said the United States was ``closely
monitoring'' the dispute. ``Both India and China have expressed a desire
to de-escalate, and we support a peaceful resolution of the current
situation,'' the department said in a statement.

It's difficult to know the exact trigger or severity of the fighting.
\href{https://www.indiatoday.in/india/story/india-china-border-faceoff-exclusive-maps-pangong-tso-ladakh-galwan-1688727-2020-06-13}{The
Galwan Valley} is a rocky, icy, inhospitable slice of the Himalayas,
off-limits to all but a few lonely herdsmen and Indian and Chinese
troops. Each side was quick to blame the other for violence along the
Line of Actual Control, the boundary that emerged from the 1962 war.

``The Indian military broke their promises and once again crossed the
Line of Actual Control to engage in illegal activities,'' Col. Zhang
Shuili, a Chinese military spokesman, said
\href{http://m.news.cctv.com/2020/06/16/ARTI5CoAVcHZBNXfje73CE1y200616.shtml}{in
a statement}. ``They deliberately launched a provocative assault,
leading to an intense physical clash that caused death and injury.''

Line of Actual

Control

CHINA

PAKISTAN

Diamer Bhasha dam

CHINA

NEPAL

GILGIT-

BALTISTAN

Controlled by Pakistan

Area of detail

Disputed borders

or cease-fire lines

CHINA

INDIA

Galwan Valley

Controlled by India

Bay of

Bengal

Pangong Tso

Arabian

Sea

JAMMU

AND KASHMIR

LADAKH

CHINA

CHINA

PAKISTAN

Lipulekh

Pass

INDIA

Naku La

NEPAL

BHUTAN

100 miles

BANGLADESH

Line of Actual

Control

Diamer Bhasha dam

CHINA

CHINA

PAKISTAN

GILGIT-

BALTISTAN

Controlled by Pakistan

NEPAL

Disputed borders

or cease-fire lines

CHINA

Area of detail

Galwan Valley

Controlled by India

INDIA

Pangong Tso

JAMMU

AND KASHMIR

LADAKH

CHINA

Bay of

Bengal

CHINA

Arabian

Sea

PAKISTAN

Lipulekh

Pass

INDIA

Naku La

NEPAL

BHUTAN

100 miles

BANGLADESH

Line of Actual

Control

Diamer

Bhasha dam

CHINA

CHINA

GILGIT-

BALTISTAN

Controlled by Pakistan

PAKISTAN

Disputed borders

or cease-fire lines

NEPAL

CHINA

Area of detail

Galwan Valley

Controlled by India

INDIA

LADAKH

JAMMU

AND KASHMIR

Pangong Tso

Bay of

Bengal

CHINA

CHINA

Arabian

Sea

PAKISTAN

Lipulekh

Pass

INDIA

Naku La

NEPAL

100 miles

Line of Actual

Control

CHINA

CHINA

GILGIT-

BALTISTAN

Controlled by Pakistan

1

Disputed borders

or cease-fire lines

Area of detail

CHINA

2

INDIA

Controlled by India

LADAKH

3

JAMMU

AND KASHMIR

Bay of

Bengal

Arabian

Sea

CHINA

CHINA

PAKISTAN

4

INDIA

5

NEPAL

100 miles

1

2

3

4

5

Diamer Bhasha dam

Galwan Valley

Pangong Tso

Lipulekh Pass

Naku La

CHINA

Disputed borders

or cease-fire lines

1

Ctrl. by Pakistan

CHINA

2

3

Ctrl. by India

CHINA

CHINA

PAKISTAN

4

INDIA

NEPAL

5

100 miles

1

2

3

4

5

Diamer Bhasha dam

Galwan Valley

Pangong Tso

Lipulekh Pass

Naku La

CHINA

PAKISTAN

NEPAL

Area of detail

INDIA

Bay of

Bengal

Arabian

Sea

Source: Satellite image via Microsoft Corporation Earthstar Geographics

By Jugal K. Patel

An Indian military official said the clash started during a meeting
attended by hundreds of soldiers on both sides who had come together
along the border to discuss efforts to de-escalate tensions. For the
past week, Indian military officials had been reassuring the Indian
public that the border was calming down and that they were having
productive talks with the Chinese, through diplomatic and military
channels.

But, according to the Indians, Chinese officers insulted them at the
meeting on Monday night, which triggered a fight between soldiers that
quickly spiraled into a major melee.

``A violent face-off happened as a result of an attempt by the Chinese
side to unilaterally change the status quo,'' said a statement from
Anurag Srivastava, an Indian government spokesman. ``Both sides suffered
casualties that could have been avoided had the agreement at the higher
level been scrupulously followed by the Chinese side.''

China and India said Tuesday they were committed to resolving their
differences through dialogue, but again, it wasn't clear if the military
commanders along the border were standing down or gearing up.

Also, if brawls in May were any indication, the casualties from Monday
may end up higher than reported. The clashes in May, which erupted at
several border points, were first reported as minor. Only weeks later
was it revealed that Chinese and Indian soldiers had been briefly
captured and some beaten so badly they required airlifting to hospitals
hundreds of miles away.

Both countries run patrols along the disputed border and the soldiers
are under strict orders not to shoot but that doesn't stop them from
throwing rocks. Or fighting with crude weapons.

``This was an incident waiting to happen,'' said H.S. Panag, a retired
Indian general, of the latest fighting.

The surge in violence is a product of the protracted dispute between
India and China over the precise location of their jagged Himalayan
border, which cuts through a desolate landscape home to few people or
resources that would be easy to extract. Both sides maintain
high-elevation military installations facing each other, and armed
skirmishes continued through the late 1960s and mid-70s.

Image

Crowds lining the streets to watch as Indian troops drive through the
Ladakh region during border clashes between India and China in
1962.Credit...Radloff/Hulton Archive, via Getty Images

The spark for the recent tensions seemed to have been a road to a remote
air force base that the Indian Army is building through the Galwan
Valley. Military analysts say that the road is fully within Indian
territory but that the Chinese are determined to frustrate India's
efforts to upgrade its military positions.

And the wider backdrop is that India and China have been competing for
influence on many fronts across South Asia.

Several countries, such as Nepal and Sri Lanka, that were once reliable
Indian allies have recently tilted toward China, wooed by Chinese
investment. And Pakistan, India's archenemy, is now fully aligned with
China, working hand in hand with the Chinese military.

But that doesn't necessarily mean that the China-India relationship will
worsen. Mr. Modi and Mr. Xi have held
\href{https://www.nytimes3xbfgragh.onion/2019/10/11/world/asia/narendra-modi-xi-jinping-india-china.html}{several
friendly summit meetings}.

And last year China helped Mr. Modi during his re-election campaign by
agreeing to a United Nations resolution that designated Masood Azhar, a
Pakistani militant, as a global terrorist. China had refused to do this
for a long time but after determined lobbying by India and others,
\href{https://www.lowyinstitute.org/the-interpreter/masood-azhar-global-terrorist-and-implications-pakistan}{China
relented}, handing Mr. Modi a diplomatic victory at a crucial time.

But then India did something that aggravated China. A few months after
Mr. Modi won a landslide election, his home minister, Amit Shah,
\href{https://economictimes.indiatimes.com/news/politics-and-nation/pok-aksai-chin-part-of-jammu-and-kashmir-will-sacrifice-our-lives-for-it-amit-shah-in-lok-sabha/videoshow/70550153.cms?from=mdr}{vowed
to take back Aksai Chin}. During a speech in Parliament about the
disputed region of Kashmir, Mr. Shah said that Aksai Chin and all of
Kashmir belonged to India and India would ``sacrifice life for this.''

Analysts say the Aksai Chin issue, along with India's warming relations
with the United States, have become irritants to China, which may now be
using its military to harass India as payback.

Amber Wang contributed research from Beijing.

Advertisement

\protect\hyperlink{after-bottom}{Continue reading the main story}

\hypertarget{site-index}{%
\subsection{Site Index}\label{site-index}}

\hypertarget{site-information-navigation}{%
\subsection{Site Information
Navigation}\label{site-information-navigation}}

\begin{itemize}
\tightlist
\item
  \href{https://help.nytimes3xbfgragh.onion/hc/en-us/articles/115014792127-Copyright-notice}{©~2020~The
  New York Times Company}
\end{itemize}

\begin{itemize}
\tightlist
\item
  \href{https://www.nytco.com/}{NYTCo}
\item
  \href{https://help.nytimes3xbfgragh.onion/hc/en-us/articles/115015385887-Contact-Us}{Contact
  Us}
\item
  \href{https://www.nytco.com/careers/}{Work with us}
\item
  \href{https://nytmediakit.com/}{Advertise}
\item
  \href{http://www.tbrandstudio.com/}{T Brand Studio}
\item
  \href{https://www.nytimes3xbfgragh.onion/privacy/cookie-policy\#how-do-i-manage-trackers}{Your
  Ad Choices}
\item
  \href{https://www.nytimes3xbfgragh.onion/privacy}{Privacy}
\item
  \href{https://help.nytimes3xbfgragh.onion/hc/en-us/articles/115014893428-Terms-of-service}{Terms
  of Service}
\item
  \href{https://help.nytimes3xbfgragh.onion/hc/en-us/articles/115014893968-Terms-of-sale}{Terms
  of Sale}
\item
  \href{https://spiderbites.nytimes3xbfgragh.onion}{Site Map}
\item
  \href{https://help.nytimes3xbfgragh.onion/hc/en-us}{Help}
\item
  \href{https://www.nytimes3xbfgragh.onion/subscription?campaignId=37WXW}{Subscriptions}
\end{itemize}
