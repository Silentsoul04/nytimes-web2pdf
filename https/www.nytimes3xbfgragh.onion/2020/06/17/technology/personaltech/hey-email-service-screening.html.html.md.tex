Sections

SEARCH

\protect\hyperlink{site-content}{Skip to
content}\protect\hyperlink{site-index}{Skip to site index}

\href{https://www.nytimes3xbfgragh.onion/section/technology/personaltech}{Personal
Tech}

\href{https://myaccount.nytimes3xbfgragh.onion/auth/login?response_type=cookie\&client_id=vi}{}

\href{https://www.nytimes3xbfgragh.onion/section/todayspaper}{Today's
Paper}

\href{/section/technology/personaltech}{Personal Tech}\textbar{}Tackling
the Unsolvable Problem: The Bottomless Email Inbox

\url{https://nyti.ms/37NXHsB}

\begin{itemize}
\item
\item
\item
\item
\item
\item
\end{itemize}

Advertisement

\protect\hyperlink{after-top}{Continue reading the main story}

Supported by

\protect\hyperlink{after-sponsor}{Continue reading the main story}

tech fix

\hypertarget{tackling-the-unsolvable-problem-the-bottomless-email-inbox}{%
\section{Tackling the Unsolvable Problem: The Bottomless Email
Inbox}\label{tackling-the-unsolvable-problem-the-bottomless-email-inbox}}

For \$99 a year, Hey wants to help us restore some control. The new
service has a way to go --- and so does email, come to think of it.

\includegraphics{https://static01.graylady3jvrrxbe.onion/images/2020/06/17/business/17Techfix-illo/17Techfix-illo-articleLarge.gif?quality=75\&auto=webp\&disable=upscale}

\href{https://www.nytimes3xbfgragh.onion/by/brian-x-chen}{\includegraphics{https://static01.graylady3jvrrxbe.onion/images/2018/02/16/multimedia/author-brian-x-chen/author-brian-x-chen-thumbLarge.jpg}}

By \href{https://www.nytimes3xbfgragh.onion/by/brian-x-chen}{Brian X.
Chen}

\begin{itemize}
\item
  June 17, 2020
\item
  \begin{itemize}
  \item
  \item
  \item
  \item
  \item
  \item
  \end{itemize}
\end{itemize}

Over the decade-plus that I have been writing about consumer technology,
the one subject I have avoided tackling is the
\href{https://bits.blogs.nytimes3xbfgragh.onion/2010/08/23/reinventing-e-mail-one-message-at-a-time/}{misery
of email}.

That's because email, which has been around for as long as we can
remember using the internet, is a source of pain with no treatment plan.
It's out of control: Anyone, from exes to marketers, can message us.
It's annoying: Once we buy something online and share our email address,
the business bombards us with useless messages.
\href{https://www.nytimes3xbfgragh.onion/2019/02/15/opinion/sunday/email-etiquette.html}{It
also stokes rude behavior}: Who hasn't ignored the avalanche of emails
that arrive?

So when I heard that the makers of
\href{https://basecamp.com/}{Basecamp}, a popular online collaboration
tool, were trying to reinvent email with a new app and service, I had to
see what they came up with. This week, Basecamp unveiled Hey, a
\$99-a-year service that offers a cleaner interface for navigating
email.

Its prime selling point is a screening tool that we can use to decide
who emails us, which theoretically helps us regain control of our
inboxes. That isn't too different from the ability to block senders in
services like Gmail, but Hey has people screen them by default.

``The reason people hate email is because they don't control it
anymore,'' said Jason Fried, chief executive of Basecamp. ``By flipping
this around and giving you control, it's actually a radical change.''

Email today is dominated by the tech behemoths: Google, Apple, Yahoo and
Microsoft. Several smaller companies have emerged recently, claiming to
offer superior products. Superhuman, an invitation-only \$30-a-month
service that rolled out last year, promised the
``\href{https://www.nytimes3xbfgragh.onion/2019/06/27/technology/superhuman-email.html}{fastest
email experience ever made}.'' It was immediately scrutinized for
privacy violations.

Paid email services are usually designed for businesses. Hey is one of
the rare ones that are consumer focused, so I felt it was worth trying.

But after about a week of testing Hey, I'm sad to report that I didn't
feel I had regained control of my inbox. I suspect most of us will
continue to feel that free services like Gmail are good enough --- and
when something is free and good enough, it's tough to beat. Hey has
taken a thoughtful first step, but it will have to do more to persuade
people to pay \$99 a year.

What's more, I walked away convinced that email as a whole is so broken
that many of us have taken most of our conversations elsewhere. More on
this later.

\hypertarget{heres-what-hey-does}{%
\subsection{Here's what Hey does.}\label{heres-what-hey-does}}

\includegraphics{https://static01.graylady3jvrrxbe.onion/images/2020/06/17/business/17techfix_hey/17techfix_hey-articleLarge.png?quality=75\&auto=webp\&disable=upscale}

Similar to Gmail in its early days, Hey is an invitation-only service.
To sign up for an account, you send an email to
\href{mailto:iwant@hey.com}{\nolinkurl{iwant@hey.com}} and tell the
company how you feel about email. Then you receive a sign-up code.

You can get to your Hey inbox through a web browser or apps made for
Apple, Android, Windows and Linux devices. (Apple users may run into
issues downloading the app: Basecamp said on Tuesday that
\href{https://twitter.com/dhh/status/1272968382329942017}{Apple had
rejected a new version of Hey} from its App Store because of issues
related to its policy for charging for subscriptions.)

Hey's star feature is the screening tool. When you first get an email
from someone, a message at the top of the screen invites you to screen
the sender. Then you are taken to the Screener menu, which shows a list
of any first-time senders and gives you the option to click Yes or No to
receiving emails from that address.

Hey also includes other benefits:

\begin{itemize}
\item
  \textbf{Anti-tracking technology.}
  \href{https://hey.com/spy-trackers/}{Email trackers} come in many
  forms, including a single invisible pixel or special web fonts, and
  marketers frequently use them to detect when someone opens a message
  and even where that person is when the email is opened. Hey
  automatically detects emails containing trackers and alerts you when
  they have been blocked. That's a step ahead of free services like
  Gmail, which offers minimal protections against tracking.
\item
  \textbf{A place for receipts.} When you get a receipt from a business
  or a trip itinerary, you can click the Move button to send it to a
  Paper Trail, which is essentially a folder for important documents.
\item
  \textbf{Other email management tools.} When you receive an important
  email that you don't want to forget about, you can pin it so that a
  preview of the message remains at the bottom of the app screen. You
  can also flag emails you want to reply to later by tapping the Later
  button.
\end{itemize}

\hypertarget{test-test}{%
\subsection{Test, test.}\label{test-test}}

To test Hey, I set up two of my Gmail accounts to automatically forward
all messages to my @hey.com email address so I could check whether the
screening tools might help my inbox feel less overwhelming.

Image

The screening tool was helpful in some cases, but tedious or impractical
in others.Credit...Basecamp

I soon found flaws. In some instances, the screening tool was helpful: I
filtered out obnoxious emails from political campaigns, Priceline and
Enterprise Rent-A-Car, simply by clicking No to those senders.

But when it came to businesses I wanted to hear from, screening became a
chore. For example, I get too many marketing emails from my home
insurance company, but I want to hear from it about receipts or policy
changes. So do I filter it out? Sometimes, businesses send important
messages and marketing emails from different email addresses, but not
always, and filtering out only the spammy addresses became tedious.

In other cases, screening became entirely impractical. Last week, I used
Yelp to contact some electronics repair shops for quotes on fixing my
smartphone. Every reply came from a newly generated email address from
Yelp. So to read responses from one repair shop, I had to use the
screener to approve messages from more than 10 email addresses. This
issue stems from how Yelp designed its reply system, but it illustrated
that Hey's screener doesn't always work well.

Mr. Fried said that this was just the beginning for Hey, and that the
screener would improve over time. ``It's not perfect,'' he said, adding
it was still ``a lot better than the alternative --- which is zero
control over who can email you.''

Hey's anti-tracking technology also felt incomplete. The service
primarily blocks tracking pixels and special fonts, which, when loaded,
ping external servers to inform a third party when you have opened an
email.

But web trackers also live elsewhere in emails. When you click on a
hyperlinked word in a sentence or on a photo for a Uniqlo sweatshirt,
that may also alert a third party that you opened the email and
interacted with its contents.
(\href{https://www.nytimes3xbfgragh.onion/2015/11/19/technology/personaltech/foiling-electronic-snoops-in-email.html}{In
the past, email tracker blockers I tested} stripped out hyperlinks
containing trackers.)

Mr. Fried said people generally understood that email links went to
websites that tracked them. I respectfully disagree. My concern is that
when Hey informs people that trackers have been blocked, they will get a
false sense of security.

I thoroughly enjoyed some of Hey's other features. I loved the pin tool
to keep important emails at the forefront of the app while doing work.
The Paper Trail was also a nice feature for keeping receipts tidy.

But I still wouldn't pay for Hey because of some of the flaws I
experienced.

\hypertarget{will-email-ever-be-fun-again}{%
\subsection{Will email ever be fun
again?}\label{will-email-ever-be-fun-again}}

Testing Hey made me reminisce about a time when email brought joy. In
the days of AOL in the 1990s, we relied on email to send notes to
friends and family.

When Gmail emerged in the mid-2000s, Google offered a free, searchable
inbox with more storage, eliminating the need to delete emails. But that
widely used service hasn't changed much since its inception.

In other words, email became boring.

Plenty of companies have attempted a more delightful email experience.
In 2013,
\href{https://www.theverge.com/2013/3/15/4110532/dropbox-reportedly-paid-around-100-million-for-mailbox}{Dropbox
acquired Mailbox}, an app that helped users declutter their inboxes, for
\$100 million. Dropbox killed the app in 2015, after concluding it could
not
``\href{https://web.archive.org/web/20151229002407/https://www.mailboxapp.com/}{fundamentally
fix email}.''

After testing Hey, I looked closely at my devices and noticed a trend.
The overwhelming majority of my digital conversations with family,
friends and co-workers happen on messaging apps like iMessage, Google
Hangouts and Slack. My email accounts have turned into a passive channel
for receiving receipts and newsletters.

This may be the case for many people. People from 16 to 44 years old
spend more time in apps like Facebook Messenger, WhatsApp and Twitter
than they use Gmail, and people older than that spend more time using
Gmail than messaging apps, said Amir Ghodrati, director of market
insights at App Annie, a research firm.

So where does this leave us?

Email may no longer be fun, because many of us have moved on. But it may
also never be gone, because it's a universal communication platform that
lets anyone talk to anyone, which makes it both horrible and great at
the same time. Perhaps that is just a reality we will have to accept.

In the meantime, my iPhone mailbox says I have about 118,000 unread
email messages. I'll get right on to ignoring them.

Advertisement

\protect\hyperlink{after-bottom}{Continue reading the main story}

\hypertarget{site-index}{%
\subsection{Site Index}\label{site-index}}

\hypertarget{site-information-navigation}{%
\subsection{Site Information
Navigation}\label{site-information-navigation}}

\begin{itemize}
\tightlist
\item
  \href{https://help.nytimes3xbfgragh.onion/hc/en-us/articles/115014792127-Copyright-notice}{©~2020~The
  New York Times Company}
\end{itemize}

\begin{itemize}
\tightlist
\item
  \href{https://www.nytco.com/}{NYTCo}
\item
  \href{https://help.nytimes3xbfgragh.onion/hc/en-us/articles/115015385887-Contact-Us}{Contact
  Us}
\item
  \href{https://www.nytco.com/careers/}{Work with us}
\item
  \href{https://nytmediakit.com/}{Advertise}
\item
  \href{http://www.tbrandstudio.com/}{T Brand Studio}
\item
  \href{https://www.nytimes3xbfgragh.onion/privacy/cookie-policy\#how-do-i-manage-trackers}{Your
  Ad Choices}
\item
  \href{https://www.nytimes3xbfgragh.onion/privacy}{Privacy}
\item
  \href{https://help.nytimes3xbfgragh.onion/hc/en-us/articles/115014893428-Terms-of-service}{Terms
  of Service}
\item
  \href{https://help.nytimes3xbfgragh.onion/hc/en-us/articles/115014893968-Terms-of-sale}{Terms
  of Sale}
\item
  \href{https://spiderbites.nytimes3xbfgragh.onion}{Site Map}
\item
  \href{https://help.nytimes3xbfgragh.onion/hc/en-us}{Help}
\item
  \href{https://www.nytimes3xbfgragh.onion/subscription?campaignId=37WXW}{Subscriptions}
\end{itemize}
