Sections

SEARCH

\protect\hyperlink{site-content}{Skip to
content}\protect\hyperlink{site-index}{Skip to site index}

\href{https://www.nytimes3xbfgragh.onion/section/books/review}{Book
Review}

\href{https://myaccount.nytimes3xbfgragh.onion/auth/login?response_type=cookie\&client_id=vi}{}

\href{https://www.nytimes3xbfgragh.onion/section/todayspaper}{Today's
Paper}

\href{/section/books/review}{Book Review}\textbar{}`Odd, Unpopular and
Reticent': The Books That Sing to Wayne Koestenbaum

\url{https://nyti.ms/2BgSCfZ}

\begin{itemize}
\item
\item
\item
\item
\item
\end{itemize}

Advertisement

\protect\hyperlink{after-top}{Continue reading the main story}

Supported by

\protect\hyperlink{after-sponsor}{Continue reading the main story}

\href{/column/by-the-book}{By the Book}

\hypertarget{odd-unpopular-and-reticent-the-books-that-sing-to-wayne-koestenbaum}{%
\section{`Odd, Unpopular and Reticent': The Books That Sing to Wayne
Koestenbaum}\label{odd-unpopular-and-reticent-the-books-that-sing-to-wayne-koestenbaum}}

\includegraphics{https://static01.graylady3jvrrxbe.onion/images/2020/06/21/books/review/21ByTheBook/21ByTheBook-articleLarge.jpg?quality=75\&auto=webp\&disable=upscale}

June 18, 2020

\begin{itemize}
\item
\item
\item
\item
\item
\end{itemize}

\emph{``When I was younger, I craved novels,'' says the poet and
cultural critic, whose latest book, ``Figure It Out,'' is a collection
of essays. ``Now I read more poetry and nonfiction.''}

\textbf{What books are on your nightstand?}

A formidable pile. Alejandra Pizarnik, ``Diana's Tree'' (translated by
Yvette Siegert). Jakov Lind, ``Ergo'' (translated by Ralph Manheim).
Mónica de la Torre, ``Repetition Nineteen.'' Hannah Arendt, ``The
Origins of Totalitarianism.'' Oyinkan Braithwaite, ``My Sister, the
Serial Killer.'' George Oppen, ``Selected Prose, Daybooks, and Papers''
(edited by Stephen Cope). Joon Oluchi Lee, ``Neotenica.'' Hervé Guibert,
``Written in Invisible Ink: Selected Stories'' (edited and translated by
Jeffrey Zuckerman). Lara Mimosa Montes, ``Thresholes.'' Abigail Child,
``This Is Called Moving: A Critical Poetics of Film.'' Renee Gladman,
``Calamities.''

\textbf{What's the last great book you read?}

Magda Szabo, ``Katalin Street,'' translated by Len Rix. A dead girl
comes back to semi-existence and hovers near the homes of neighbors who
never treated her nicely enough while she was alive. Earlier this year,
to school myself on the history of innovative celluloid magic, I finally
read P. Adams Sitney's ``Visionary Film.''

\textbf{Are there any classic novels that you only recently read for the
first time?}

Does the ``Aeneid'' count as a novel? I read the Robert Fagles
translation. Dido, the ``tragic queen of Carthage,'' had long been part
of my bloodstream, but I needed to confront her sorrow up close, in the
verse that is her homeland.

\textbf{Can a great book be badly written? What other criteria can
overcome bad prose?}

Some world-altering books arrive wrapped in dense, difficult prose,
requiring prolonged mastication. I've never been able to make it through
Hegel's ``Phenomenology of Spirit.'' I'm being judged, not Hegel.

\textbf{Describe your ideal reading experience.}

While temping as a Kelly Girl in 1982, I gulped down Charles Dickens's
``Dombey and Son.'' I was working for a week as a receptionist, but no
customers or clients showed up. Sometimes the phone would ring. But
mostly I was on my own with Dickens.

\textbf{What's your favorite book no one else has heard of?}

Stephen Rodefer's ``Four Lectures'' (published in 1982 by the Figures, a
magnificent press run by Geoffrey Young) changed the tempo of my
poetry-writing life. I was inspired by Rodefer's non sequiturs, by his
distractibility, by his diction's obstreperousness --- road blocks that
made me giddy.

\textbf{Which writers --- novelists, playwrights, critics, journalists,
poets --- working today do you admire most?}

Impossible question to answer succinctly. I depend --- for nourishment,
ideas, ecstasy --- on hundreds. Among my favorite living poets is
Friederike Mayröcker. Close to my heart, always, as role model, is Joyce
Carol Oates. Jamaica Kincaid, Hilton Als, Joan Didion:
style-triumvirate. Slide into my DMs if you want more recommendations.

\textbf{What book, if any, most contributed to your artistic
development?}

``The Collected Poems of Frank O'Hara.'' He taught me how to be quick.

\textbf{Do you count any books as guilty pleasures?}

I love reading movie-star biographies, a genre I take seriously.
Patricia Bosworth's biography of Montgomery Clift is a classic: terse
yet juicy.

\textbf{Has a book ever brought you closer to another person, or come
between you?}

I received, when I was 20, a slim edition of Wordsworth's selected
poems, as a gift from a young man I loved. I didn't dare to hope that he
loved me back. But his inscription gave me a foretaste.

\textbf{What's the most interesting thing you learned from a book
recently?}

From Jackie Wang's ``Carceral Capitalism,'' I learned about the sinister
existence of a predictive policing software called PredPol. Don't Google
it, or ads for PredPol will start showing up on your Twitter feed.

\textbf{Which subjects do you wish more authors would write about?}

Their inner filth. But, please, in clean sentences. I can't say it's
radical to wish for more writers to describe \emph{accurately} their
sexual fantasies; but my appetite for carnal truth is unsated.

\textbf{Have you ever changed your opinion of a book based on
information about the author, or anything else?}

I should have changed my mind about Ezra Pound's poems, but I didn't. I
can't erase, from my ear, his elegiac line ``Quick eyes gone under
earth's lid.''

\textbf{How do you organize your books?}

Prose and poetry dwell in separate areas, alphabetized by author.
Certain writers defect from category. Boris Pasternak's prose memoir
``Safe Conduct'' lives next to his book-length poem ``My Sister ---
Life'' in the poetry section. Books in French have their own cul-de-sac.
I keep together the books of certain publishers: all of my Green Integer
books, for example, are on one shelf. On another set of shelves --- the
\emph{salon des refusés} --- are the books that depress me too much, or
are too miscellaneous and dreary, to be included in the regular area.
But I can't discard these wastrels.

\textbf{What book might people be surprised to find on your shelves?}

Thich Nhat Hanh, ``Happiness.'' Are you shocked? I was going to mention
``For Every Young Heart: Connie Francis Talks to Teenagers,'' but that
didn't seem surprising enough.

\textbf{What's the best book you've ever received as a gift?}

My parents gave me a Gallimard ``Collection Folio'' paperback of André
Gide's ``L'immoraliste'' for Christmas when I was a teen. I didn't read
it until much later. Some messages take time to detonate.

\textbf{How have your reading tastes changed over time?}

When I was younger, I craved novels: ``Jane Eyre'' and ``Sons and
Lovers'' schooled me in ambivalence, flame, escape. Now, I read more
poetry and nonfiction. I seek out books that are odd, unpopular and
reticent. But a fluent novel --- especially a monologue, in the maniacal
mode of Thomas Bernhard --- can still sock it to me.

\textbf{You're organizing a literary dinner party. Which three writers,
dead or alive, do you invite?}

Jean Rhys, Emily Dickinson, Max Jacob.

\textbf{Disappointing, overrated, just not good. What book did you feel
as if you were supposed to like, and didn't?}

Edmund Spenser's ``The Faerie Queene'' is entrancing, but I was
disappointed to discover that it was not about fairies and queens.

\textbf{What books are you embarrassed not to have read yet?}

Thomas Mann, ``The Magic Mountain*.''* Thomas Mann, ``Buddenbrooks*.''*
Thomas Mann, ``Doctor Faustus*.''* But I will make amends --- as soon as
Susan Bernofsky publishes her new translation of ``The Magic Mountain.''

\textbf{What do you plan to read next?}

Whim will determine what's next. Many offerings entice. Cathy Park Hong,
``Minor Feelings: An Asian American Reckoning.'' Douglas A. Martin,\\
``Wolf*.''* Asiya Wadud, ``Syncope*.''* I might start with Steve
Benson's ``Blue Book''; I'm late to its idiosyncratic party. I open it
and discover these lines: ``he's up on the wall nude his arms / wrapped
around his head talking like / or about David Hockney David Cassidy.''
How could I resist?

Advertisement

\protect\hyperlink{after-bottom}{Continue reading the main story}

\hypertarget{site-index}{%
\subsection{Site Index}\label{site-index}}

\hypertarget{site-information-navigation}{%
\subsection{Site Information
Navigation}\label{site-information-navigation}}

\begin{itemize}
\tightlist
\item
  \href{https://help.nytimes3xbfgragh.onion/hc/en-us/articles/115014792127-Copyright-notice}{©~2020~The
  New York Times Company}
\end{itemize}

\begin{itemize}
\tightlist
\item
  \href{https://www.nytco.com/}{NYTCo}
\item
  \href{https://help.nytimes3xbfgragh.onion/hc/en-us/articles/115015385887-Contact-Us}{Contact
  Us}
\item
  \href{https://www.nytco.com/careers/}{Work with us}
\item
  \href{https://nytmediakit.com/}{Advertise}
\item
  \href{http://www.tbrandstudio.com/}{T Brand Studio}
\item
  \href{https://www.nytimes3xbfgragh.onion/privacy/cookie-policy\#how-do-i-manage-trackers}{Your
  Ad Choices}
\item
  \href{https://www.nytimes3xbfgragh.onion/privacy}{Privacy}
\item
  \href{https://help.nytimes3xbfgragh.onion/hc/en-us/articles/115014893428-Terms-of-service}{Terms
  of Service}
\item
  \href{https://help.nytimes3xbfgragh.onion/hc/en-us/articles/115014893968-Terms-of-sale}{Terms
  of Sale}
\item
  \href{https://spiderbites.nytimes3xbfgragh.onion}{Site Map}
\item
  \href{https://help.nytimes3xbfgragh.onion/hc/en-us}{Help}
\item
  \href{https://www.nytimes3xbfgragh.onion/subscription?campaignId=37WXW}{Subscriptions}
\end{itemize}
