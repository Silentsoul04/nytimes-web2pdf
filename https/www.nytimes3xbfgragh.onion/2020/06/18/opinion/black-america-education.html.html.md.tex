Sections

SEARCH

\protect\hyperlink{site-content}{Skip to
content}\protect\hyperlink{site-index}{Skip to site index}

\href{https://myaccount.nytimes3xbfgragh.onion/auth/login?response_type=cookie\&client_id=vi}{}

\href{https://www.nytimes3xbfgragh.onion/section/todayspaper}{Today's
Paper}

\href{/section/opinion}{Opinion}\textbar{}How Moderates Failed Black
America

\url{https://nyti.ms/37HAIPz}

\begin{itemize}
\item
\item
\item
\item
\item
\end{itemize}

Advertisement

\protect\hyperlink{after-top}{Continue reading the main story}

\href{/section/opinion}{Opinion}

Supported by

\protect\hyperlink{after-sponsor}{Continue reading the main story}

\hypertarget{how-moderates-failed-black-america}{%
\section{How Moderates Failed Black
America}\label{how-moderates-failed-black-america}}

Better education is not leading to equality.

\href{https://www.nytimes3xbfgragh.onion/by/david-brooks}{\includegraphics{https://static01.graylady3jvrrxbe.onion/images/2018/04/03/opinion/david-brooks/david-brooks-thumbLarge-v2.png}}

By \href{https://www.nytimes3xbfgragh.onion/by/david-brooks}{David
Brooks}

Opinion Columnist

\begin{itemize}
\item
  June 18, 2020
\item
  \begin{itemize}
  \item
  \item
  \item
  \item
  \item
  \end{itemize}
\end{itemize}

\includegraphics{https://static01.graylady3jvrrxbe.onion/images/2020/06/18/opinion/18brooks1/merlin_134050107_214b4e96-1ac8-4f63-92ba-00822874959a-articleLarge.jpg?quality=75\&auto=webp\&disable=upscale}

We Americans believe in education. We tend to assume that if you help a
young person get a good education and the right skills, then she'll be
able to make her way in American society. Opportunity will be bountiful.
Social harmony will reign.

This formula has not worked for many African-Americans.

Over generations, great gains have been made in improving black
students' education. In 1968, just 54 percent of young black adults had
a high school diploma. Today, 92 percent do. In 1968, about 9 percent of
young African-American adults had completed college. Today, roughly 23
percent have.

And yet these gains have not led to the kind of progress that those of
us who preach the gospel of the American dream would have predicted and
that all young people are entitled to.

The median income for a white head of household with a college degree is
\$106,600. The median income for a comparable black college graduate is
\href{https://www.pewsocialtrends.org/2016/06/27/1-demographic-trends-and-economic-well-being/}{only
\$82,300}. As my colleagues on
\href{https://www.nytimes3xbfgragh.onion/2017/09/20/opinion/college-racial-income-gap.html}{the
editorial page noted} in 2017, black college graduates earned about 21
percent less per hour than white college graduates. Over all, black
families
\href{https://www.nytimes3xbfgragh.onion/interactive/2017/09/18/upshot/black-white-wealth-gap-perceptions.html}{earn
\$57.30 for every \$100} white families earn. These pay gaps have been
widening since 1979, not shrinking.

If the income gaps are bad, the wealth gaps are horrendous. In 2020, the
wealth gap between white and black families is as wide as it was in
1968. Ray Boshara of the St. Louis Fed
\href{https://www.washingtonpost.com/business/2020/06/04/economic-divide-black-households/\%20https:/www.washingtonpost.com/opinions/black-college-graduates-are-losing-wealth-heres-what-can-help/2017/04/12/cd83ba64-1ba4-11e7-9887-1a5314b56a08_story.html?arc404=true}{pointed
out} that between 1992 and 2013, college-educated whites saw the value
of their assets soar by 86 percent, while their black counterparts saw
theirs fall by 55 percent.

College-educated blacks tend to have higher student loan burdens. When a
young white adult wants to buy a home, his parents will often help him
out. When a young black man is at that stage, he's more likely to be
sending money \emph{to} his parents to help \emph{them} out.

It turns out that increasing educational opportunities does not by
itself reduce income disparities. Nor does it reduce social disparities.
Minority students who graduated college were supposed to enter a less
racist America. They have not. Seventeen percent of college-educated
blacks
\href{https://www.pewresearch.org/fact-tank/2019/05/02/for-black-americans-experiences-of-racial-discrimination-vary-by-education-level-gender/}{say
they face discrimination ``regularly,''} compared with 9 percent of high
school-educated blacks. Half of all black Americans with at least some
college said they've feared for their personal safety because of their
race, compared with roughly a third of those with less education.

Integrated colleges try to create welcoming homes for minority students,
and they try to prepare them for adulthood. It's not entirely working.
Fifty-five percent of the black graduates of historically black colleges
and universities
\href{https://www.insidehighered.com/news/2015/10/28/survey-finds-big-differences-between-black-hbcu-graduates-those-who-attended-other\#:~:text=Black\%20graduates\%20of\%20historically\%20black,ongoing\%20Gallup\%2DPurdue\%20University\%20study.}{strongly
agree that their school prepared them well} for life. Only 29 percent of
black graduates of non-H.B.C.U. schools feel that way. About half of
black H.B.C.U. graduates say their college was ``the perfect school''
for them. Only 34 percent of black non-H.B.C.U. alumni say the same.

The gospel of the American dream teaches that as people make it in
America they will feel more accepted by America, more at home in
America. This is not happening for many African-Americans.

Perhaps you saw Chris Lebron's
\href{https://www.nytimes3xbfgragh.onion/2020/06/16/opinion/black-academia-racism.html}{Times
Op-Ed this week}. Lebron has taught at the University of Virginia and
Yale and now teaches philosophy at Johns Hopkins. A great American
success story.

But Lebron writes about how unsafe he feels. ``I almost never attend
casual faculty functions. I don't go out for drinks. I don't entertain
for dinner parties and I don't seek to ingratiate myself into the lives
of my white colleagues. \ldots{} It's already hard enough to breath in
America. Every day you feel like you're living with a knee on your
neck.''

This profound alienation is not unusual, especially within the upper
echelons of American society. Read Critical Race Theory, which started
at elite law schools. \href{https://ta-nehisicoates.com/books/}{Read
Ta-Nehisi Coates}, the most celebrated American writer right now.
\href{https://www.ibramxkendi.com/how-to-be-an-antiracist-1}{Read Ibram
X. Kendi}, one of the best-selling authors right now, who has lost faith
in the way schooling is done. He calls Barack Obama's Race to the Top
program a racist extension of previous efforts that discriminated
against black students.

Or \href{https://wwnorton.com/books/9781631496141}{read
``Afropessimism,''} by Frank B. Wilderson III, of UC Irvine, who argues
that anti-black violence is intrinsic to world history: ``Blacks are not
going to be genocided like Native Americans. We \emph{are} being
genocided, but genocided \emph{and} regenerated because the
\emph{spectacle} of Black death is essential to the mental health of the
world.''

Many conservatives and moderates say these ideas come from campus
culture. People read Foucault and develop an alienated view of the
world. The blunt facts, however, suggest that, overstated or not, these
writers are responding to something real in the world, something real in
the world both of the less educated \emph{and} of the highly educated.
People are responding to the failure of the mainstream, moderate,
progressive formula for how to create a more equal pluralist America.

I'm a moderate guy, but the evidence doesn't support moderation when it
comes to racial equity. Better schooling is essential to creating a fair
and equal America. But it is not nearly enough.

\emph{The Times is committed to publishing}
\href{https://www.nytimes3xbfgragh.onion/2019/01/31/opinion/letters/letters-to-editor-new-york-times-women.html}{\emph{a
diversity of letters}} \emph{to the editor. We'd like to hear what you
think about this or any of our articles. Here are some}
\href{https://help.nytimes3xbfgragh.onion/hc/en-us/articles/115014925288-How-to-submit-a-letter-to-the-editor}{\emph{tips}}\emph{.
And here's our email:}
\href{mailto:letters@NYTimes.com}{\emph{letters@NYTimes.com}}\emph{.}

\emph{Follow The New York Times Opinion section on}
\href{https://www.facebookcorewwwi.onion/nytopinion}{\emph{Facebook}}\emph{,}
\href{http://twitter.com/NYTOpinion}{\emph{Twitter (@NYTopinion)}}
\emph{and}
\href{https://www.instagram.com/nytopinion/}{\emph{Instagram}}\emph{.}

Advertisement

\protect\hyperlink{after-bottom}{Continue reading the main story}

\hypertarget{site-index}{%
\subsection{Site Index}\label{site-index}}

\hypertarget{site-information-navigation}{%
\subsection{Site Information
Navigation}\label{site-information-navigation}}

\begin{itemize}
\tightlist
\item
  \href{https://help.nytimes3xbfgragh.onion/hc/en-us/articles/115014792127-Copyright-notice}{©~2020~The
  New York Times Company}
\end{itemize}

\begin{itemize}
\tightlist
\item
  \href{https://www.nytco.com/}{NYTCo}
\item
  \href{https://help.nytimes3xbfgragh.onion/hc/en-us/articles/115015385887-Contact-Us}{Contact
  Us}
\item
  \href{https://www.nytco.com/careers/}{Work with us}
\item
  \href{https://nytmediakit.com/}{Advertise}
\item
  \href{http://www.tbrandstudio.com/}{T Brand Studio}
\item
  \href{https://www.nytimes3xbfgragh.onion/privacy/cookie-policy\#how-do-i-manage-trackers}{Your
  Ad Choices}
\item
  \href{https://www.nytimes3xbfgragh.onion/privacy}{Privacy}
\item
  \href{https://help.nytimes3xbfgragh.onion/hc/en-us/articles/115014893428-Terms-of-service}{Terms
  of Service}
\item
  \href{https://help.nytimes3xbfgragh.onion/hc/en-us/articles/115014893968-Terms-of-sale}{Terms
  of Sale}
\item
  \href{https://spiderbites.nytimes3xbfgragh.onion}{Site Map}
\item
  \href{https://help.nytimes3xbfgragh.onion/hc/en-us}{Help}
\item
  \href{https://www.nytimes3xbfgragh.onion/subscription?campaignId=37WXW}{Subscriptions}
\end{itemize}
