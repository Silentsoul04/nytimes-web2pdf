Sections

SEARCH

\protect\hyperlink{site-content}{Skip to
content}\protect\hyperlink{site-index}{Skip to site index}

\href{https://www.nytimes3xbfgragh.onion/section/travel}{Travel}

\href{https://myaccount.nytimes3xbfgragh.onion/auth/login?response_type=cookie\&client_id=vi}{}

\href{https://www.nytimes3xbfgragh.onion/section/todayspaper}{Today's
Paper}

\href{/section/travel}{Travel}\textbar{}Help! It's Been Months. I'm
Still in a Travel Mess.

\url{https://nyti.ms/2YIWQVM}

\begin{itemize}
\item
\item
\item
\item
\item
\end{itemize}

\hypertarget{the-coronavirus-outbreak}{%
\subsubsection{\texorpdfstring{\href{https://www.nytimes3xbfgragh.onion/news-event/coronavirus?name=styln-coronavirus-national\&region=TOP_BANNER\&variant=undefined\&block=storyline_menu_recirc\&action=click\&pgtype=Article\&impression_id=d127f790-e3b3-11ea-a68a-479821f8f864}{The
Coronavirus
Outbreak}}{The Coronavirus Outbreak}}\label{the-coronavirus-outbreak}}

\begin{itemize}
\tightlist
\item
  live\href{https://www.nytimes3xbfgragh.onion/2020/08/21/world/covid-19-coronavirus.html?name=styln-coronavirus-national\&region=TOP_BANNER\&variant=undefined\&block=storyline_menu_recirc\&action=click\&pgtype=Article\&impression_id=d1281ea0-e3b3-11ea-a68a-479821f8f864}{Latest
  Updates}
\item
  \href{https://www.nytimes3xbfgragh.onion/interactive/2020/us/coronavirus-us-cases.html?name=styln-coronavirus-national\&region=TOP_BANNER\&variant=undefined\&block=storyline_menu_recirc\&action=click\&pgtype=Article\&impression_id=d1281ea1-e3b3-11ea-a68a-479821f8f864}{Maps
  and Cases}
\item
  \href{https://www.nytimes3xbfgragh.onion/interactive/2020/science/coronavirus-vaccine-tracker.html?name=styln-coronavirus-national\&region=TOP_BANNER\&variant=undefined\&block=storyline_menu_recirc\&action=click\&pgtype=Article\&impression_id=d12845b0-e3b3-11ea-a68a-479821f8f864}{Vaccine
  Tracker}
\item
  \href{https://www.nytimes3xbfgragh.onion/2020/08/19/us/colleges-closing-covid.html?name=styln-coronavirus-national\&region=TOP_BANNER\&variant=undefined\&block=storyline_menu_recirc\&action=click\&pgtype=Article\&impression_id=d12845b1-e3b3-11ea-a68a-479821f8f864}{Colleges
  Closing}
\item
  \href{https://www.nytimes3xbfgragh.onion/live/2020/08/21/business/stock-market-today-coronavirus?name=styln-coronavirus-national\&region=TOP_BANNER\&variant=undefined\&block=storyline_menu_recirc\&action=click\&pgtype=Article\&impression_id=d12845b2-e3b3-11ea-a68a-479821f8f864}{Economy}
\end{itemize}

Advertisement

\protect\hyperlink{after-top}{Continue reading the main story}

Supported by

\protect\hyperlink{after-sponsor}{Continue reading the main story}

Tripped Up

\hypertarget{help-its-been-months-im-still-in-a-travel-mess}{%
\section{Help! It's Been Months. I'm Still in a Travel
Mess.}\label{help-its-been-months-im-still-in-a-travel-mess}}

Unclear policies, confusing customer-service protocols and
not-yet-fulfilled refunds continue to be an issue. Our columnist sees
what she can do.

\includegraphics{https://static01.graylady3jvrrxbe.onion/images/2020/06/18/travel/18tripped-up-rev/18tripped-up-rev-articleLarge.jpg?quality=75\&auto=webp\&disable=upscale}

By Sarah Firshein

\begin{itemize}
\item
  June 18, 2020
\item
  \begin{itemize}
  \item
  \item
  \item
  \item
  \item
  \end{itemize}
\end{itemize}

I've received thousands of emails from Times readers since the
coronavirus put the brakes on travel in mid-March. And if there's one
thing that's obvious, it's that the aftershocks of canceled trips
persist. Unclear policies, confusing customer-service protocols and
not-yet-fulfilled refunds continue to be an issue --- months later, and
even as the world reopens.

This week, I dipped into my email inbox, grab-bag style, for three
questions to tackle in brief. Have your own? Ask me at
\href{mailto:travel@NYTimes.com}{\nolinkurl{travel@NYTimes.com}}.

\hypertarget{dear-tripped-up}{%
\subsubsection{\texorpdfstring{\textbf{Dear Tripped
Up,}}{Dear Tripped Up,}}\label{dear-tripped-up}}

In early February, I used Orbitz to book a round-trip Qantas Airways
flight from San Francisco to Melbourne, scheduled to depart in April.
The flight was canceled. Who should one lobby for refunds in this case:
the online travel agency or the airline? Meredith

\hypertarget{hi-meredith}{%
\subsubsection{Hi Meredith,}\label{hi-meredith}}

I've gotten a lot of questions about this issue. Many readers report
being herded like cattle from an online travel agency to an airline, and
back --- then around again.

Always start with the company that sold you the ticket. O.T.A.s like
Orbitz can change most reservations (save for certain types on some
low-cost carriers), so you're well within reason to let the customer
service representative do your bidding with the airline.

\hypertarget{latest-updates-the-coronavirus-outbreak}{%
\section{\texorpdfstring{\href{https://www.nytimes3xbfgragh.onion/2020/08/21/world/covid-19-coronavirus.html?action=click\&pgtype=Article\&state=default\&region=MAIN_CONTENT_1\&context=storylines_live_updates}{Latest
Updates: The Coronavirus
Outbreak}}{Latest Updates: The Coronavirus Outbreak}}\label{latest-updates-the-coronavirus-outbreak}}

Updated 2020-08-21T13:33:48.544Z

\begin{itemize}
\tightlist
\item
  \href{https://www.nytimes3xbfgragh.onion/2020/08/21/world/covid-19-coronavirus.html?action=click\&pgtype=Article\&state=default\&region=MAIN_CONTENT_1\&context=storylines_live_updates\#link-6a60a19d}{`Be
  adults': Universities in the U.S. are warning students about
  gatherings as they return to campus.}
\item
  \href{https://www.nytimes3xbfgragh.onion/2020/08/21/world/covid-19-coronavirus.html?action=click\&pgtype=Article\&state=default\&region=MAIN_CONTENT_1\&context=storylines_live_updates\#link-324af071}{As
  he accepts the Democratic nomination, Biden knocks Trump's pandemic
  response.}
\item
  \href{https://www.nytimes3xbfgragh.onion/2020/08/21/world/covid-19-coronavirus.html?action=click\&pgtype=Article\&state=default\&region=MAIN_CONTENT_1\&context=storylines_live_updates\#link-191d44be}{South
  Korea threatens to detain people who obstruct virus-control efforts.}
\end{itemize}

\href{https://www.nytimes3xbfgragh.onion/2020/08/21/world/covid-19-coronavirus.html?action=click\&pgtype=Article\&state=default\&region=MAIN_CONTENT_1\&context=storylines_live_updates}{See
more updates}

More live coverage:
\href{https://www.nytimes3xbfgragh.onion/live/2020/08/21/business/stock-market-today-coronavirus?action=click\&pgtype=Article\&state=default\&region=MAIN_CONTENT_1\&context=storylines_live_updates}{Markets}

I've
\href{https://www.nytimes3xbfgragh.onion/2020/05/01/travel/trip-refund-airlines.html}{said
it before} and I'll say it again: When a flight to, from or within the
United States is canceled because of the pandemic, you're entitled to a
cash refund. And if you're not getting that as an initial option from
sites like Orbitz, it's because they can only process refunds after
they've been granted that right by the end provider; here, the airline.
That's why
\href{https://www.nytimes3xbfgragh.onion/2020/04/03/travel/coronavirus-refund-travel-ota.html?smtyp=cur\&smid=tw-nytimestravelhttps://www.nytimes3xbfgragh.onion/2020/04/03/travel/coronavirus-refund-travel-ota.html}{getting
refunds from}O.T.A.s can be complicated, especially during this global
crisis. These agencies are squished between an unprecedented number of
consumer requests and airlines that stall or stonewall --- or perhaps
never pick up the phone. If you're hitting roadblocks like these, ask
that your case be escalated (Orbitz and other O.T.A.s have a hierarchy
of urgency). Don't be shy about following up.

That persistence seemed to work for you, even without my help: By the
time I emailed Orbitz, a spokeswoman confirmed that the refund was
already in process. Yes, it's now June, and although call volumes in
March and April have leveled off for most agencies by now, the current
environment is still far from normal --- patience is key.

\hypertarget{dear-tripped-up-1}{%
\subsubsection{\texorpdfstring{\textbf{Dear Tripped
Up,}}{Dear Tripped Up,}}\label{dear-tripped-up-1}}

I have a flight to France coming up next month. Delta rebooked me on a
different flight on my original departure date, which seemed fine --- I
understand flexibility is key nowadays. But when I hopped online to
double-check my options, I didn't see the new flight on Delta's
schedule. In fact, it didn't appear until a few days later. What's the
story? Sabine

\hypertarget{hi-sabine}{%
\subsubsection{Hi Sabine,}\label{hi-sabine}}

Although airline technology has improved in recent years, the pandemic
has placed extraordinary stress on the system.

Even as the world begins to open up, flight schedules will be slimmer
than normal this summer. Delta, like all airlines, has been compressing
its schedule into fewer flights, so it's possible that the flight you
were rebooked on was kept ``off-sale'' to give already-booked passengers
dibs before being made available to new passengers. It's also possible
that the new flight was full for a period of time, during which time it
wasn't appearing for sale online.

There are complexities to publishing and adjusting airline schedules,
for sure, but there is one upside to a system that's so in flux: mistake
fares, or bargain-basement price glitches, that eagle-eyed consumers can
score before they're corrected. There are several websites that comb for
these deals, including \href{https://www.secretflying.com/}{Secret
Flying,} \href{https://scottscheapflights.com/}{Scott's Cheap Flights}
and \href{https://www.airfarewatchdog.com/}{Airfarewatchdog.}

\hypertarget{dear-tripped-up-2}{%
\subsubsection{\texorpdfstring{\textbf{Dear Tripped
Up,}}{Dear Tripped Up,}}\label{dear-tripped-up-2}}

I paid a \$125 ``miles redeposit fee'' when I canceled my United
Airlines reward flight a couple of months ago. How can I get that money
back? Jen

\hypertarget{hi-jen}{%
\subsubsection{Hi Jen,}\label{hi-jen}}

Save for the famously flexible Southwest, most airlines have long
imposed some sort of fee --- usually between \$75 and \$150 --- when
passengers change or cancel tickets booked with miles. These so-called
``restocking fees'' tend to lessen or disappear with elite status.

\href{https://www.nytimes3xbfgragh.onion/news-event/coronavirus?action=click\&pgtype=Article\&state=default\&region=MAIN_CONTENT_3\&context=storylines_faq}{}

\hypertarget{the-coronavirus-outbreak-}{%
\subsubsection{The Coronavirus Outbreak
›}\label{the-coronavirus-outbreak-}}

\hypertarget{frequently-asked-questions}{%
\paragraph{Frequently Asked
Questions}\label{frequently-asked-questions}}

Updated August 17, 2020

\begin{itemize}
\item ~
  \hypertarget{why-does-standing-six-feet-away-from-others-help}{%
  \paragraph{Why does standing six feet away from others
  help?}\label{why-does-standing-six-feet-away-from-others-help}}

  \begin{itemize}
  \tightlist
  \item
    The coronavirus spreads primarily through droplets from your mouth
    and nose, especially when you cough or sneeze. The C.D.C., one of
    the organizations using that measure,
    \href{https://www.nytimes3xbfgragh.onion/2020/04/14/health/coronavirus-six-feet.html?action=click\&pgtype=Article\&state=default\&region=MAIN_CONTENT_3\&context=storylines_faq}{bases
    its recommendation of six feet} on the idea that most large droplets
    that people expel when they cough or sneeze will fall to the ground
    within six feet. But six feet has never been a magic number that
    guarantees complete protection. Sneezes, for instance, can launch
    droplets a lot farther than six feet,
    \href{https://jamanetwork.com/journals/jama/fullarticle/2763852}{according
    to a recent study}. It's a rule of thumb: You should be safest
    standing six feet apart outside, especially when it's windy. But
    keep a mask on at all times, even when you think you're far enough
    apart.
  \end{itemize}
\item ~
  \hypertarget{i-have-antibodies-am-i-now-immune}{%
  \paragraph{I have antibodies. Am I now
  immune?}\label{i-have-antibodies-am-i-now-immune}}

  \begin{itemize}
  \tightlist
  \item
    As of right
    now,\href{https://www.nytimes3xbfgragh.onion/2020/07/22/health/covid-antibodies-herd-immunity.html?action=click\&pgtype=Article\&state=default\&region=MAIN_CONTENT_3\&context=storylines_faq}{that
    seems likely, for at least several months.} There have been
    frightening accounts of people suffering what seems to be a second
    bout of Covid-19. But experts say these patients may have a
    drawn-out course of infection, with the virus taking a slow toll
    weeks to months after initial exposure. People infected with the
    coronavirus typically
    \href{https://www.nature.com/articles/s41586-020-2456-9}{produce}
    immune molecules called antibodies, which are
    \href{https://www.nytimes3xbfgragh.onion/2020/05/07/health/coronavirus-antibody-prevalence.html?action=click\&pgtype=Article\&state=default\&region=MAIN_CONTENT_3\&context=storylines_faq}{protective
    proteins made in response to an
    infection}\href{https://www.nytimes3xbfgragh.onion/2020/05/07/health/coronavirus-antibody-prevalence.html?action=click\&pgtype=Article\&state=default\&region=MAIN_CONTENT_3\&context=storylines_faq}{.
    These antibodies may} last in the body
    \href{https://www.nature.com/articles/s41591-020-0965-6}{only two to
    three months}, which may seem worrisome, but that's perfectly normal
    after an acute infection subsides, said Dr. Michael Mina, an
    immunologist at Harvard University. It may be possible to get the
    coronavirus again, but it's highly unlikely that it would be
    possible in a short window of time from initial infection or make
    people sicker the second time.
  \end{itemize}
\item ~
  \hypertarget{im-a-small-business-owner-can-i-get-relief}{%
  \paragraph{I'm a small-business owner. Can I get
  relief?}\label{im-a-small-business-owner-can-i-get-relief}}

  \begin{itemize}
  \tightlist
  \item
    The
    \href{https://www.nytimes3xbfgragh.onion/article/small-business-loans-stimulus-grants-freelancers-coronavirus.html?action=click\&pgtype=Article\&state=default\&region=MAIN_CONTENT_3\&context=storylines_faq}{stimulus
    bills enacted in March} offer help for the millions of American
    small businesses. Those eligible for aid are businesses and
    nonprofit organizations with fewer than 500 workers, including sole
    proprietorships, independent contractors and freelancers. Some
    larger companies in some industries are also eligible. The help
    being offered, which is being managed by the Small Business
    Administration, includes the Paycheck Protection Program and the
    Economic Injury Disaster Loan program. But lots of folks have
    \href{https://www.nytimes3xbfgragh.onion/interactive/2020/05/07/business/small-business-loans-coronavirus.html?action=click\&pgtype=Article\&state=default\&region=MAIN_CONTENT_3\&context=storylines_faq}{not
    yet seen payouts.} Even those who have received help are confused:
    The rules are draconian, and some are stuck sitting on
    \href{https://www.nytimes3xbfgragh.onion/2020/05/02/business/economy/loans-coronavirus-small-business.html?action=click\&pgtype=Article\&state=default\&region=MAIN_CONTENT_3\&context=storylines_faq}{money
    they don't know how to use.} Many small-business owners are getting
    less than they expected or
    \href{https://www.nytimes3xbfgragh.onion/2020/06/10/business/Small-business-loans-ppp.html?action=click\&pgtype=Article\&state=default\&region=MAIN_CONTENT_3\&context=storylines_faq}{not
    hearing anything at all.}
  \end{itemize}
\item ~
  \hypertarget{what-are-my-rights-if-i-am-worried-about-going-back-to-work}{%
  \paragraph{What are my rights if I am worried about going back to
  work?}\label{what-are-my-rights-if-i-am-worried-about-going-back-to-work}}

  \begin{itemize}
  \tightlist
  \item
    Employers have to provide
    \href{https://www.osha.gov/SLTC/covid-19/standards.html}{a safe
    workplace} with policies that protect everyone equally.
    \href{https://www.nytimes3xbfgragh.onion/article/coronavirus-money-unemployment.html?action=click\&pgtype=Article\&state=default\&region=MAIN_CONTENT_3\&context=storylines_faq}{And
    if one of your co-workers tests positive for the coronavirus, the
    C.D.C.} has said that
    \href{https://www.cdc.gov/coronavirus/2019-ncov/community/guidance-business-response.html}{employers
    should tell their employees} -\/- without giving you the sick
    employee's name -\/- that they may have been exposed to the virus.
  \end{itemize}
\item ~
  \hypertarget{what-is-school-going-to-look-like-in-september}{%
  \paragraph{What is school going to look like in
  September?}\label{what-is-school-going-to-look-like-in-september}}

  \begin{itemize}
  \tightlist
  \item
    It is unlikely that many schools will return to a normal schedule
    this fall, requiring the grind of
    \href{https://www.nytimes3xbfgragh.onion/2020/06/05/us/coronavirus-education-lost-learning.html?action=click\&pgtype=Article\&state=default\&region=MAIN_CONTENT_3\&context=storylines_faq}{online
    learning},
    \href{https://www.nytimes3xbfgragh.onion/2020/05/29/us/coronavirus-child-care-centers.html?action=click\&pgtype=Article\&state=default\&region=MAIN_CONTENT_3\&context=storylines_faq}{makeshift
    child care} and
    \href{https://www.nytimes3xbfgragh.onion/2020/06/03/business/economy/coronavirus-working-women.html?action=click\&pgtype=Article\&state=default\&region=MAIN_CONTENT_3\&context=storylines_faq}{stunted
    workdays} to continue. California's two largest public school
    districts --- Los Angeles and San Diego --- said on July 13, that
    \href{https://www.nytimes3xbfgragh.onion/2020/07/13/us/lausd-san-diego-school-reopening.html?action=click\&pgtype=Article\&state=default\&region=MAIN_CONTENT_3\&context=storylines_faq}{instruction
    will be remote-only in the fall}, citing concerns that surging
    coronavirus infections in their areas pose too dire a risk for
    students and teachers. Together, the two districts enroll some
    825,000 students. They are the largest in the country so far to
    abandon plans for even a partial physical return to classrooms when
    they reopen in August. For other districts, the solution won't be an
    all-or-nothing approach.
    \href{https://bioethics.jhu.edu/research-and-outreach/projects/eschool-initiative/school-policy-tracker/}{Many
    systems}, including the nation's largest, New York City, are
    devising
    \href{https://www.nytimes3xbfgragh.onion/2020/06/26/us/coronavirus-schools-reopen-fall.html?action=click\&pgtype=Article\&state=default\&region=MAIN_CONTENT_3\&context=storylines_faq}{hybrid
    plans} that involve spending some days in classrooms and other days
    online. There's no national policy on this yet, so check with your
    municipal school system regularly to see what is happening in your
    community.
  \end{itemize}
\end{itemize}

Although commonplace in normal times, these fees felt particularly
out-of-step at the start of the pandemic. Facing increased pressure in
March, many carriers waived them; United took a bit longer than some
others to reverse its policy but eventually fell in line. On June 1, the
airline announced that it will continue to waive redeposit fees on all
award tickets with 2020 departure dates.

Refunds and waivers
\href{https://www.nytimes3xbfgragh.onion/2020/04/11/travel/coronavirus-travel-trip-refunds.html}{have
been a moving target} since the pandemic started, and if you got
ensnared in an unfavorable policy before it changed for the better, you
have plenty of company. If you're committed to getting that restocking
fee reversed, your best bet is to call the airline --- a United
spokeswoman wouldn't commit to this as an across-the-board option, but
she said the company will address the issue on a case-by-case basis.

\begin{center}\rule{0.5\linewidth}{\linethickness}\end{center}

\href{https://twitter.com/sfirshein?lang=en}{Sarah Firshein} is a
Brooklyn-based writer. If you need advice about a best-laid travel plan
that went awry, \textbf{\href{mailto:travel@NYTimes.com}{send an email
to travel@NYTimes.com}.}

\begin{center}\rule{0.5\linewidth}{\linethickness}\end{center}

\textbf{LET'S GO, SAFELY.} \emph{Discover more Travel coverage by
following us on}
\href{https://twitter.com/nytimestravel}{\emph{Twitter}} \emph{and}
\href{https://www.facebookcorewwwi.onion/nytimestravel/}{\emph{Facebook}}\emph{.
And}
\href{https://www.nytimes3xbfgragh.onion/newsletters/traveldispatch?action=click\&module=inline\&pgtype=Article}{\emph{sign
up for our}} **
\href{https://www.nytimes3xbfgragh.onion/newsletters/traveldispatch}{\emph{Travel
Dispatch newsletter}}\emph{: Each week you'll receive tips on traveling
smarter, stories on hot destinations and access to photos from all over
the world.}

Advertisement

\protect\hyperlink{after-bottom}{Continue reading the main story}

\hypertarget{site-index}{%
\subsection{Site Index}\label{site-index}}

\hypertarget{site-information-navigation}{%
\subsection{Site Information
Navigation}\label{site-information-navigation}}

\begin{itemize}
\tightlist
\item
  \href{https://help.nytimes3xbfgragh.onion/hc/en-us/articles/115014792127-Copyright-notice}{©~2020~The
  New York Times Company}
\end{itemize}

\begin{itemize}
\tightlist
\item
  \href{https://www.nytco.com/}{NYTCo}
\item
  \href{https://help.nytimes3xbfgragh.onion/hc/en-us/articles/115015385887-Contact-Us}{Contact
  Us}
\item
  \href{https://www.nytco.com/careers/}{Work with us}
\item
  \href{https://nytmediakit.com/}{Advertise}
\item
  \href{http://www.tbrandstudio.com/}{T Brand Studio}
\item
  \href{https://www.nytimes3xbfgragh.onion/privacy/cookie-policy\#how-do-i-manage-trackers}{Your
  Ad Choices}
\item
  \href{https://www.nytimes3xbfgragh.onion/privacy}{Privacy}
\item
  \href{https://help.nytimes3xbfgragh.onion/hc/en-us/articles/115014893428-Terms-of-service}{Terms
  of Service}
\item
  \href{https://help.nytimes3xbfgragh.onion/hc/en-us/articles/115014893968-Terms-of-sale}{Terms
  of Sale}
\item
  \href{https://spiderbites.nytimes3xbfgragh.onion}{Site Map}
\item
  \href{https://help.nytimes3xbfgragh.onion/hc/en-us}{Help}
\item
  \href{https://www.nytimes3xbfgragh.onion/subscription?campaignId=37WXW}{Subscriptions}
\end{itemize}
