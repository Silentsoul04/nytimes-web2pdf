Sections

SEARCH

\protect\hyperlink{site-content}{Skip to
content}\protect\hyperlink{site-index}{Skip to site index}

\href{https://www.nytimes3xbfgragh.onion/section/health}{Health}

\href{https://myaccount.nytimes3xbfgragh.onion/auth/login?response_type=cookie\&client_id=vi}{}

\href{https://www.nytimes3xbfgragh.onion/section/todayspaper}{Today's
Paper}

\href{/section/health}{Health}\textbar{}Older Adults May Be Left Out of
Some Covid-19 Trials

\url{https://nyti.ms/2YKtt5u}

\begin{itemize}
\item
\item
\item
\item
\item
\item
\end{itemize}

\href{https://www.nytimes3xbfgragh.onion/news-event/coronavirus?action=click\&pgtype=Article\&state=default\&region=TOP_BANNER\&context=storylines_menu}{The
Coronavirus Outbreak}

\begin{itemize}
\tightlist
\item
  live\href{https://www.nytimes3xbfgragh.onion/2020/08/01/world/coronavirus-covid-19.html?action=click\&pgtype=Article\&state=default\&region=TOP_BANNER\&context=storylines_menu}{Latest
  Updates}
\item
  \href{https://www.nytimes3xbfgragh.onion/interactive/2020/us/coronavirus-us-cases.html?action=click\&pgtype=Article\&state=default\&region=TOP_BANNER\&context=storylines_menu}{Maps
  and Cases}
\item
  \href{https://www.nytimes3xbfgragh.onion/interactive/2020/science/coronavirus-vaccine-tracker.html?action=click\&pgtype=Article\&state=default\&region=TOP_BANNER\&context=storylines_menu}{Vaccine
  Tracker}
\item
  \href{https://www.nytimes3xbfgragh.onion/interactive/2020/07/29/us/schools-reopening-coronavirus.html?action=click\&pgtype=Article\&state=default\&region=TOP_BANNER\&context=storylines_menu}{What
  School May Look Like}
\item
  \href{https://www.nytimes3xbfgragh.onion/live/2020/07/31/business/stock-market-today-coronavirus?action=click\&pgtype=Article\&state=default\&region=TOP_BANNER\&context=storylines_menu}{Economy}
\end{itemize}

Advertisement

\protect\hyperlink{after-top}{Continue reading the main story}

Supported by

\protect\hyperlink{after-sponsor}{Continue reading the main story}

The New Old Age

\hypertarget{older-adults-may-be-left-out-of-some-covid-19-trials}{%
\section{Older Adults May Be Left Out of Some Covid-19
Trials}\label{older-adults-may-be-left-out-of-some-covid-19-trials}}

The coronavirus disproportionately affects seniors, but they may not be
included in important clinical trials for vaccines and treatments.

\includegraphics{https://static01.graylady3jvrrxbe.onion/images/2020/06/23/science/23SCI-NOACVTRIALS/23SCI-NOACVTRIALS-articleLarge.jpg?quality=75\&auto=webp\&disable=upscale}

By \href{https://www.nytimes3xbfgragh.onion/by/paula-span}{Paula Span}

\begin{itemize}
\item
  June 19, 2020
\item
  \begin{itemize}
  \item
  \item
  \item
  \item
  \item
  \item
  \end{itemize}
\end{itemize}

Picture the day --- in six months, a year and a half or in 2023 --- when
university researchers or a pharmaceutical company announces a
breakthrough against the virus that causes Covid-19.

Maybe it's a successful vaccine or an effective treatment, a discovery
that brings hope and relief --- especially to the older adults most
vulnerable to the disease, eager to return to their pre-pandemic lives.

Imagine, however, that the researchers neglected to enroll many people
in their 70s or 80s in the clinical trials that established the
effectiveness of this treatment. As a result, it's unclear how much the
treatment will benefit older people, what risks it might pose, or if the
dose that works for younger people is the best one for an older
population.

Could that really happen? Certainly there is
\href{https://onlinelibrary.wiley.com/doi/abs/10.1111/j.1532-5415.2010.03032.x}{a
long history} of older people being excluded from clinical trials, even
when the diseases in question disproportionately affected this group.
They have been underrepresented, for instance, in studies of
\href{https://onlinelibrary.wiley.com/doi/10.1111/jgs.14833}{heart
disease},
\href{https://ascopubs.org/doi/abs/10.1200/JCO.2003.08.010}{cancer},
\href{https://onlinelibrary.wiley.com/doi/abs/10.1111/jgs.12215}{diabetes}
and
\href{https://onlinelibrary.wiley.com/doi/abs/10.1111/jgs.14840}{osteoporosis}.

``Ideally, the patients enrolled in a randomized clinical trial reflect
the demographics of the disease,'' said Dr. Mark Sloan, a hematologist
leading a Covid-19 drug study at Boston Medical Center, in an email.
``Unfortunately, this is seldom the case.''

Now, Dr. Sharon K. Inouye, a geriatrician at Harvard Medical School and
Hebrew SeniorLife, is sounding an alarm. She points out that in the race
to find drugs and vaccines to fight the pandemic --- in which
\href{https://www.cdc.gov/nchs/nvss/vsrr/covid_weekly/index.htm\#AgeAndSex}{80
percent of American} deaths have occurred in people over age 65 --- a
substantial proportion of studies may be excluding older subjects,
purposely or inadvertently.

``A year from now, when these trials are published, I don't want to see
that there's no one in them over 75,'' she said. ``If they create a drug
that works really well in healthy 50- and 60-year-olds, they've missed
the boat.''

She and her team have reviewed 241 interventional Covid-19 studies
undertaken in the United States and listed on clinicaltrials.gov, a site
maintained by a division of the National Institutes of Health.

They found that 37 of these trials --- testing drugs, vaccines and
devices --- set specific age limits and would not enroll participants
older than 75, 80 or 85. A few even excluded those over 65.

Another group of 27 trials set no maximum age but used study designs
that could nevertheless disqualify many older adults.

\hypertarget{latest-updates-global-coronavirus-outbreak}{%
\section{\texorpdfstring{\href{https://www.nytimes3xbfgragh.onion/2020/08/01/world/coronavirus-covid-19.html?action=click\&pgtype=Article\&state=default\&region=MAIN_CONTENT_1\&context=storylines_live_updates}{Latest
Updates: Global Coronavirus
Outbreak}}{Latest Updates: Global Coronavirus Outbreak}}\label{latest-updates-global-coronavirus-outbreak}}

Updated 2020-08-01T19:54:00.494Z

\begin{itemize}
\tightlist
\item
  \href{https://www.nytimes3xbfgragh.onion/2020/08/01/world/coronavirus-covid-19.html?action=click\&pgtype=Article\&state=default\&region=MAIN_CONTENT_1\&context=storylines_live_updates\#link-3ac56579}{Top
  officials work to break impasse over jobless benefit.}
\item
  \href{https://www.nytimes3xbfgragh.onion/2020/08/01/world/coronavirus-covid-19.html?action=click\&pgtype=Article\&state=default\&region=MAIN_CONTENT_1\&context=storylines_live_updates\#link-8796723}{The
  virus picks up dangerous speed in the Midwest, and in areas that had
  seen success.}
\item
  \href{https://www.nytimes3xbfgragh.onion/2020/08/01/world/coronavirus-covid-19.html?action=click\&pgtype=Article\&state=default\&region=MAIN_CONTENT_1\&context=storylines_live_updates\#link-25930521}{Thousands
  in Berlin protest Germany's coronavirus measures.}
\end{itemize}

\href{https://www.nytimes3xbfgragh.onion/2020/08/01/world/coronavirus-covid-19.html?action=click\&pgtype=Article\&state=default\&region=MAIN_CONTENT_1\&context=storylines_live_updates}{See
more updates}

More live coverage:
\href{https://www.nytimes3xbfgragh.onion/live/2020/07/31/business/stock-market-today-coronavirus?action=click\&pgtype=Article\&state=default\&region=MAIN_CONTENT_1\&context=storylines_live_updates}{Markets}

Some excluded people with illnesses common among the older population,
like hypertension or diabetes, even if participants controlled the
disease through medication. ``Surrogates for age exclusion,'' Dr. Inouye
said.

In other cases, broad exclusion criteria gave the investigator
inordinate discretion. Dr. Inouye pointed to one trial barring subjects
with ``any physical examination findings, and/or history of any illness,
concomitant medications or recent live vaccines that, in the opinion of
the study investigator, might confound the results of the study or pose
an additional risk to the participant.''

That allows investigators to exclude people for reasons ranging from
legitimate safety issues to the researchers' own convenience. One
scenario that worried Dr. Inouye: ``If you have an older adult who
appears hard of hearing, and it takes twice as long to explain the study
and obtain consent, the investigator just won't do it.''

Other studies, relying on digital technology, might rule out people who
lack smartphones or the ability to use email.

Overall, when Dr. Inouye compiled preliminary results, which have not
yet been published, she found that about one-quarter of interventional
trials in the United States could exclude or underrepresent older
adults.

``To have them be this gravely impacted and not include them is
immoral,'' said Dr. Louise Aronson, the author of the best-selling book
``Elderhood'' and a geriatrician at the University of California, San
Francisco. ``It seems crazy.''

In response to similar concerns, the N.I.H. began last year to require
the studies it funds to include
``\href{https://grants.nih.gov/grants/guide/notice-files/NOT-OD-18-116.html}{individuals
across the life span,}'' unless investigators provide an ``acceptable
justification'' for exclusion. If they cannot, N.I.H. won't award the
grants.

``It's a pretty visible guidepost, reflecting national standards,'' Dr.
Aronson said. But most clinical trials are privately funded and need not
follow N.I.H. policy.

Although Dr. Inouye systematically reviewed only American trials, her
team also noticed age limits in Covid-19 studies in France, Italy,
China, Colombia and other countries, some set as low as age 60.

Whether such trials will actually exclude many older subjects probably
will not be clear for some time. Several investigators, asked about the
number of enrolled participants over age 80, said they were only
starting recruitment or could not yet provide data.

Some investigators might make accommodations. A study at the University
of California, San Francisco, for example, examining the use of the
antibiotic azithromycin, drew Dr. Inouye's attention because
participants had to submit questionnaires online. But Catherine
Oldenburg, a co-principal investigator, explained in an email that
researchers would also allow participants to complete interviews by
phone or have a proxy submit the online forms.

\href{https://www.nytimes3xbfgragh.onion/news-event/coronavirus?action=click\&pgtype=Article\&state=default\&region=MAIN_CONTENT_3\&context=storylines_faq}{}

\hypertarget{the-coronavirus-outbreak-}{%
\subsubsection{The Coronavirus Outbreak
›}\label{the-coronavirus-outbreak-}}

\hypertarget{frequently-asked-questions}{%
\paragraph{Frequently Asked
Questions}\label{frequently-asked-questions}}

Updated July 27, 2020

\begin{itemize}
\item ~
  \hypertarget{should-i-refinance-my-mortgage}{%
  \paragraph{Should I refinance my
  mortgage?}\label{should-i-refinance-my-mortgage}}

  \begin{itemize}
  \tightlist
  \item
    \href{https://www.nytimes3xbfgragh.onion/article/coronavirus-money-unemployment.html?action=click\&pgtype=Article\&state=default\&region=MAIN_CONTENT_3\&context=storylines_faq}{It
    could be a good idea,} because mortgage rates have
    \href{https://www.nytimes3xbfgragh.onion/2020/07/16/business/mortgage-rates-below-3-percent.html?action=click\&pgtype=Article\&state=default\&region=MAIN_CONTENT_3\&context=storylines_faq}{never
    been lower.} Refinancing requests have pushed mortgage applications
    to some of the highest levels since 2008, so be prepared to get in
    line. But defaults are also up, so if you're thinking about buying a
    home, be aware that some lenders have tightened their standards.
  \end{itemize}
\item ~
  \hypertarget{what-is-school-going-to-look-like-in-september}{%
  \paragraph{What is school going to look like in
  September?}\label{what-is-school-going-to-look-like-in-september}}

  \begin{itemize}
  \tightlist
  \item
    It is unlikely that many schools will return to a normal schedule
    this fall, requiring the grind of
    \href{https://www.nytimes3xbfgragh.onion/2020/06/05/us/coronavirus-education-lost-learning.html?action=click\&pgtype=Article\&state=default\&region=MAIN_CONTENT_3\&context=storylines_faq}{online
    learning},
    \href{https://www.nytimes3xbfgragh.onion/2020/05/29/us/coronavirus-child-care-centers.html?action=click\&pgtype=Article\&state=default\&region=MAIN_CONTENT_3\&context=storylines_faq}{makeshift
    child care} and
    \href{https://www.nytimes3xbfgragh.onion/2020/06/03/business/economy/coronavirus-working-women.html?action=click\&pgtype=Article\&state=default\&region=MAIN_CONTENT_3\&context=storylines_faq}{stunted
    workdays} to continue. California's two largest public school
    districts --- Los Angeles and San Diego --- said on July 13, that
    \href{https://www.nytimes3xbfgragh.onion/2020/07/13/us/lausd-san-diego-school-reopening.html?action=click\&pgtype=Article\&state=default\&region=MAIN_CONTENT_3\&context=storylines_faq}{instruction
    will be remote-only in the fall}, citing concerns that surging
    coronavirus infections in their areas pose too dire a risk for
    students and teachers. Together, the two districts enroll some
    825,000 students. They are the largest in the country so far to
    abandon plans for even a partial physical return to classrooms when
    they reopen in August. For other districts, the solution won't be an
    all-or-nothing approach.
    \href{https://bioethics.jhu.edu/research-and-outreach/projects/eschool-initiative/school-policy-tracker/}{Many
    systems}, including the nation's largest, New York City, are
    devising
    \href{https://www.nytimes3xbfgragh.onion/2020/06/26/us/coronavirus-schools-reopen-fall.html?action=click\&pgtype=Article\&state=default\&region=MAIN_CONTENT_3\&context=storylines_faq}{hybrid
    plans} that involve spending some days in classrooms and other days
    online. There's no national policy on this yet, so check with your
    municipal school system regularly to see what is happening in your
    community.
  \end{itemize}
\item ~
  \hypertarget{is-the-coronavirus-airborne}{%
  \paragraph{Is the coronavirus
  airborne?}\label{is-the-coronavirus-airborne}}

  \begin{itemize}
  \tightlist
  \item
    The coronavirus
    \href{https://www.nytimes3xbfgragh.onion/2020/07/04/health/239-experts-with-one-big-claim-the-coronavirus-is-airborne.html?action=click\&pgtype=Article\&state=default\&region=MAIN_CONTENT_3\&context=storylines_faq}{can
    stay aloft for hours in tiny droplets in stagnant air}, infecting
    people as they inhale, mounting scientific evidence suggests. This
    risk is highest in crowded indoor spaces with poor ventilation, and
    may help explain super-spreading events reported in meatpacking
    plants, churches and restaurants.
    \href{https://www.nytimes3xbfgragh.onion/2020/07/06/health/coronavirus-airborne-aerosols.html?action=click\&pgtype=Article\&state=default\&region=MAIN_CONTENT_3\&context=storylines_faq}{It's
    unclear how often the virus is spread} via these tiny droplets, or
    aerosols, compared with larger droplets that are expelled when a
    sick person coughs or sneezes, or transmitted through contact with
    contaminated surfaces, said Linsey Marr, an aerosol expert at
    Virginia Tech. Aerosols are released even when a person without
    symptoms exhales, talks or sings, according to Dr. Marr and more
    than 200 other experts, who
    \href{https://academic.oup.com/cid/article/doi/10.1093/cid/ciaa939/5867798}{have
    outlined the evidence in an open letter to the World Health
    Organization}.
  \end{itemize}
\item ~
  \hypertarget{what-are-the-symptoms-of-coronavirus}{%
  \paragraph{What are the symptoms of
  coronavirus?}\label{what-are-the-symptoms-of-coronavirus}}

  \begin{itemize}
  \tightlist
  \item
    Common symptoms
    \href{https://www.nytimes3xbfgragh.onion/article/symptoms-coronavirus.html?action=click\&pgtype=Article\&state=default\&region=MAIN_CONTENT_3\&context=storylines_faq}{include
    fever, a dry cough, fatigue and difficulty breathing or shortness of
    breath.} Some of these symptoms overlap with those of the flu,
    making detection difficult, but runny noses and stuffy sinuses are
    less common.
    \href{https://www.nytimes3xbfgragh.onion/2020/04/27/health/coronavirus-symptoms-cdc.html?action=click\&pgtype=Article\&state=default\&region=MAIN_CONTENT_3\&context=storylines_faq}{The
    C.D.C. has also} added chills, muscle pain, sore throat, headache
    and a new loss of the sense of taste or smell as symptoms to look
    out for. Most people fall ill five to seven days after exposure, but
    symptoms may appear in as few as two days or as many as 14 days.
  \end{itemize}
\item ~
  \hypertarget{does-asymptomatic-transmission-of-covid-19-happen}{%
  \paragraph{Does asymptomatic transmission of Covid-19
  happen?}\label{does-asymptomatic-transmission-of-covid-19-happen}}

  \begin{itemize}
  \tightlist
  \item
    So far, the evidence seems to show it does. A widely cited
    \href{https://www.nature.com/articles/s41591-020-0869-5}{paper}
    published in April suggests that people are most infectious about
    two days before the onset of coronavirus symptoms and estimated that
    44 percent of new infections were a result of transmission from
    people who were not yet showing symptoms. Recently, a top expert at
    the World Health Organization stated that transmission of the
    coronavirus by people who did not have symptoms was ``very rare,''
    \href{https://www.nytimes3xbfgragh.onion/2020/06/09/world/coronavirus-updates.html?action=click\&pgtype=Article\&state=default\&region=MAIN_CONTENT_3\&context=storylines_faq\#link-1f302e21}{but
    she later walked back that statement.}
  \end{itemize}
\end{itemize}

Still, the rush to learn more about the deadly coronavirus could
exacerbate the problem, because including older people can make research
slower, more complicated and more expensive.

In clinical trials, ``you want to control as many factors as possible,''
Dr. Aronson said. Most older adults have other illnesses and take
multiple medications, so-called confounding variables that make it
difficult to distinguish the effects of the drug or vaccine being
studied.

Older people also suffer more side effects. ``Nearly all drugs are less
toxic when given to younger, healthier people,'' Dr. Sloan said in an
email. Focusing on them produces fewer adverse effects that must be
reported, ``and thereby improves chances for F.D.A. approval.''

Physical disabilities, which make it harder for seniors to reach study
sites, or hearing and vision impairments requiring large-print forms or
audio amplification, can further decrease participation. Investigators
may need to take the extra step of obtaining family consent if a patient
is incapacitated.

``The same things that put you at higher risk for Covid make it harder
to study you,'' Dr. Aronson said.

But that shouldn't prevent researchers from incorporating the population
most at risk from the coronavirus. ``If that's the real-world scenario,
you've got to deal with it,'' said Susan Peschin, the president of the
nonprofit Alliance for Aging Research.

The Alliance wants the F.D.A., which is now preparing guidance for
including
\href{https://www.fda.gov/regulatory-information/search-fda-guidance-documents/inclusion-older-adults-cancer-clinical-trials}{older
adults in cancer research}, to issue guidelines covering all medical
products for conditions that primarily impact the elderly.

The F.D.A. has issued this kind of guidance as far back as 1993, and it
released additional guidelines in 2012 and again
\href{https://www.fda.gov/regulatory-information/search-fda-guidance-documents/enhancing-diversity-clinical-trial-populations-eligibility-criteria-enrollment-practices-and-trial}{last
year}, said Dr. Harpreet Singh, a division director at the agency's
Oncology Center of Excellence --- although such guidance is not legally
binding or enforceable.

Dr. Singh, a geriatrician and oncologist, acknowledged in an interview
that the agency could do more to promote the inclusion of seniors. And
when there is reason to suspect that trials are overlooking them, ``it
should be called out and examined.''

That's what Dr. Inouye has undertaken, although she worries that she may
be too late. Covid-19 studies already underway can still revamp their
criteria to include older participants --- there are reliable strategies
to achieve that --- but investigators are working at an accelerated
pace, under great pressure to produce results.

For researchers, the professional and financial stakes are high. But
they are even higher for older Americans.

``These are the people getting very ill from Covid and dying from it,''
Dr. Inouye said. ``If people are genuinely committed to inclusion, let's
do what we need to do.''

Advertisement

\protect\hyperlink{after-bottom}{Continue reading the main story}

\hypertarget{site-index}{%
\subsection{Site Index}\label{site-index}}

\hypertarget{site-information-navigation}{%
\subsection{Site Information
Navigation}\label{site-information-navigation}}

\begin{itemize}
\tightlist
\item
  \href{https://help.nytimes3xbfgragh.onion/hc/en-us/articles/115014792127-Copyright-notice}{©~2020~The
  New York Times Company}
\end{itemize}

\begin{itemize}
\tightlist
\item
  \href{https://www.nytco.com/}{NYTCo}
\item
  \href{https://help.nytimes3xbfgragh.onion/hc/en-us/articles/115015385887-Contact-Us}{Contact
  Us}
\item
  \href{https://www.nytco.com/careers/}{Work with us}
\item
  \href{https://nytmediakit.com/}{Advertise}
\item
  \href{http://www.tbrandstudio.com/}{T Brand Studio}
\item
  \href{https://www.nytimes3xbfgragh.onion/privacy/cookie-policy\#how-do-i-manage-trackers}{Your
  Ad Choices}
\item
  \href{https://www.nytimes3xbfgragh.onion/privacy}{Privacy}
\item
  \href{https://help.nytimes3xbfgragh.onion/hc/en-us/articles/115014893428-Terms-of-service}{Terms
  of Service}
\item
  \href{https://help.nytimes3xbfgragh.onion/hc/en-us/articles/115014893968-Terms-of-sale}{Terms
  of Sale}
\item
  \href{https://spiderbites.nytimes3xbfgragh.onion}{Site Map}
\item
  \href{https://help.nytimes3xbfgragh.onion/hc/en-us}{Help}
\item
  \href{https://www.nytimes3xbfgragh.onion/subscription?campaignId=37WXW}{Subscriptions}
\end{itemize}
