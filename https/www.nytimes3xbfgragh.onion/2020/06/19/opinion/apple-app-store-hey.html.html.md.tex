Sections

SEARCH

\protect\hyperlink{site-content}{Skip to
content}\protect\hyperlink{site-index}{Skip to site index}

\href{https://myaccount.nytimes3xbfgragh.onion/auth/login?response_type=cookie\&client_id=vi}{}

\href{https://www.nytimes3xbfgragh.onion/section/todayspaper}{Today's
Paper}

\href{/section/opinion}{Opinion}\textbar{}Is It Finally Hammer Time for
Apple and Its App Store?

\url{https://nyti.ms/3ebfyvL}

\begin{itemize}
\item
\item
\item
\item
\item
\end{itemize}

Advertisement

\protect\hyperlink{after-top}{Continue reading the main story}

\href{/section/opinion}{Opinion}

Supported by

\protect\hyperlink{after-sponsor}{Continue reading the main story}

\hypertarget{is-it-finally-hammer-time-for-apple-and-its-app-store}{%
\section{Is It Finally Hammer Time for Apple and Its App
Store?}\label{is-it-finally-hammer-time-for-apple-and-its-app-store}}

The outcome of the company's latest scuffle could have a lasting impact
on the power dynamic between Big Tech and the rest of us.

\includegraphics{https://static01.graylady3jvrrxbe.onion/images/2018/08/02/opinion/02swisher/02swisher-thumbLarge.png}

By Kara Swisher

Ms. Swisher covers technology and is a contributing opinion writer.

\begin{itemize}
\item
  June 19, 2020
\item
  \begin{itemize}
  \item
  \item
  \item
  \item
  \item
  \end{itemize}
\end{itemize}

\includegraphics{https://static01.graylady3jvrrxbe.onion/images/2020/06/22/opinion/22Swisher/19Swisher-articleLarge.jpg?quality=75\&auto=webp\&disable=upscale}

The tech faceplant of the week goes to Mark Zuckerberg, the Facebook
chief executive, who recently released a video in which he and his wife
said they were ``deeply shaken and disgusted by President Trump's
divisive and incendiary rhetoric at a time when our nation so
desperately needs unity.'' It was a case study in how \emph{not} to read
the room, especially when you are in charge of a huge service that has
been one of the main digital tools that has helped the president divide
and set fire to the body politic.

Mr. Zuckerberg is an easy go-to when it comes to pointing out the
hypocrisy of the powerful playacting as the aggrieved. It's irksome, and
it's also easy to mock, but in truth most of the battles between the
digital Davids and Goliaths are much more complex.

Consider, for example, the fight this week between Apple and a small
tech company --- a quarrel that you might want to pay attention to since
the outcome could have a lasting impact on the power dynamic between Big
Tech and the rest of us.

The kerfuffle between the trillion-dollar tech giant and the founders of
Basecamp, a nifty project management tool, centers on an innovative new
email service that Basecamp created called Hey.com. As part of the
\href{https://www.nytimes3xbfgragh.onion/2020/06/17/technology/personaltech/hey-email-service-screening.html}{Hey.com}
rollout, the company submitted the mobile app to Apple's App Store. A
spot there is critical for its success.

And how --- given that access to the mobile universe is controlled by
just two companies: Apple and Google. As one person intimately familiar
with the mobile ecosystem noted to me, Apple and Google are the ``two
tollbooths'' for us all.

Tollbooth is just the right metaphor. While you can use various services
like Hey.com on the web through browsers, when it comes to mobile, app
developers are subject to whatever guidelines Apple and Google impose
and the fees they charge.

Apple has asserted its curatorial might most strongly, by far, often in
the interests of taming the sprawling and enormous app deluge. Their
oversight includes efforts to protect privacy and eliminate dangerous
developers who attempt to foist spam and malware on consumers. Mistakes
slip through, but Apple runs a tidy ship.

Yet Apple has also changed rules in ways that many developers find
capricious and unfair and, more to the point, scary. While complaints
have been raised for a long time about what Ben Thompson of Stratechery
calls Apple's ``rent-seeking'' practices, many developers do not want to
speak out for fear of falling afoul of Apple and, worse, getting banned
from its store.

But not Basecamp's iconoclastic and outspoken founders, Jason Fried and
David Heinemeier Hansson, who took to Twitter and other media to
complain loudly after the Hey.com app had been accepted by Apple and
then flagged for being in violation of its rules last week. In practice,
that means Hey.com cannot make crucial bug updates. The company was
awaiting an appeal decision by Apple that came down Thursday and that
demands that Basecamp make changes in order to keep its Hey.com app on
the platform.

Apple executives had told me and others earlier this week that it erred
in its original approval of Hey.com, since the email service, which
costs \$99 per year, did not include the Apple-friendly in-app payment
system as required by its rules.

The payment system, the only one available to developers when using
Apple's App Store, means Apple gets a huge cut when a customer buys
digital goods like an app subscription (30 percent for the first year
and 15 percent thereafter). Apple does not get a fee when customers sign
up through the app's company website, and it's worth noting that 84
percent of apps are free, and developers pay nothing to Apple.

There is no doubt that Apple's payment system is convenient for some
developers. And Apple correctly touts that it has created a lucrative
business platform for many companies (\$519 billion in overall revenue
in 2019). But because of Apple's cut, some companies try to minimize
in-app transactions, either by avoiding offering in-app subscriptions at
all (Netflix) or charging more for that particular sign-up (Spotify).
Hey.com did not include Apple's payment tech in its app.

Apple has made exceptions for some services that fall into what it calls
a ``reader'' category and has given other services a pass for a variety
of reasons; it has even struck individual deals to bypass grabbing a
cut. Still, Apple has decided thus far that Hey.com does not merit
special treatment, even though there are also some subscription email
apps that don't offer in-app purchase technology and are allowed to
operate on Apple's platform (for now).

Apple's approach can be confusing --- and definitely irritating. It
hasn't sat well with Basecamp, as evidenced in a tweetstorm by Mr.
Heinemeier Hansson. While he has a reputation for speaking out, this
time Mr. Heinemeier Hansson has ended up in one of the more epic App
Store controversies (best boiled down to his accusation that Apple was
acting like ``gangsters''). He likened Apple's actions to smashing
windows and burning down stores.

``They count on developers to stay quiet,'' he said in an agitated
interview this week, referring to Apple executives. ``We thought we were
fighting an email market dominated by Google, but these were the real
heavies.''

Apple obviously does not agree with this assessment.

``These rules have been around the App Store since the day we started,''
said Phil Schiller, who is in charge of the App Store and noted that the
company reviews over 100,000 apps every week. ``There is no perfection
here. We will admit it when we make a mistake and acknowledge that we
also need to learn and grow,'' he said. ``But this was rejected for a
good reason.''

This dispute comes just as the European Union said this week that it had
\href{https://www.nytimes3xbfgragh.onion/2020/06/16/business/apple-app-store-european-union-antitrust.html}{opened
an investigation} into whether Apple unfairly shakes down developers
vying to distribute their wares on the App Store.

The longtime Silicon Valley thorn
\href{https://www.nytimes3xbfgragh.onion/2019/11/19/technology/tech-regulator-europe.html}{Margrethe
Vestager}, the European Union's antitrust head, said the company had now
``obtained a `gatekeeper' role,'' and, because of that, ``we need to
ensure that Apple's rules do not distort competition in markets where
Apple is competing with other app developers.''

She was referring to problems first raised by Spotify, the popular music
service, which is now in direct competition with Apple Music. Spotify's
complaints of Apple's anticompetitive behavior, filed more than a year
ago, led to the European Union move. Spotify's top lawyer, Horacio
Gutierrez, said in an interview with me that the situation with Hey.com
is strikingly familiar.

``I have to say a lot of Hey's experience is reminiscent of what we have
been living,'' he said. ``The reality is Apple continues to move the
goal posts and change the rules to its advantage and the detriment of
developers.''

At a press event last week about the European Union investigation, Mr.
Gutierrez used a sports metaphor, too, saying that ``Apple acts as
stadium owner, referee and player and tilts the playing field to favor
its own services.'' He added that ``their selective and capricious
enforcement is designed to put companies like ours at an untenable
competitive disadvantage.''

``We don't want special treatment or a free ride,'' Mr. Gutierrez said.
``We want fair treatment and the opportunity to compete without
artificial obstacles put in our way,'' a point also made by the Basecamp
team.

Spotify has thus pushed for a choice of payment systems within the App
Store rather than having to use Apple's proprietary system, along with
more ability to communicate with its own users.

Unhappy developers are not new for Apple or to any tech company. But
Apple is not just facing regulatory issues in Europe.

The Justice Department has an open and active investigation specifically
looking into Apple monopolization and self-preferencing in the App
Store, said sources. The agency has already revealed that it is
officially looking at Google over a range of practices around market
power. And Congress is also prepping a big antitrust hearing soon, which
Apple and other tech giants will likely attend.

Mr. Heinemeier Hansson testified before Congress last year, excoriating
Apple, as well as Facebook and Google, before the House antitrust
subcommittee. He ended his jeremiad with a plea, ``Help us, Congress.
You're our only hope.''

He sounded more hopeless this week, calling dealing with Apple's app
approval process a ``wheel of misfortune.'' And, referring to the now
defunct browser that ran right into another tech giant's cross hairs,
Mr. Heinemeier Hansson sent a text to me later: ``The wheels of
legislative justice turn slow. It didn't help Netscape any that
Microsoft got some penalties years after its air supply had been cut
off.''

It goes without saying that this is not how Apple sees itself, having
spent years brandishing its image as the breaker and not maker of chains
--- see the famous
\href{https://www.youtube.com/watch?v=VtvjbmoDx-I}{1984 Apple Macintosh
commercial}.

The question is whether anyone throwing a hammer at power these days can
throw it hard enough to make a difference.

\emph{The Times is committed to publishing}
\href{https://www.nytimes3xbfgragh.onion/2019/01/31/opinion/letters/letters-to-editor-new-york-times-women.html}{\emph{a
diversity of letters}} \emph{to the editor. We'd like to hear what you
think about this or any of our articles. Here are some}
\href{https://help.nytimes3xbfgragh.onion/hc/en-us/articles/115014925288-How-to-submit-a-letter-to-the-editor}{\emph{tips}}\emph{.
And here's our email:}
\href{mailto:letters@NYTimes.com}{\emph{letters@NYTimes.com}}\emph{.}

\emph{Follow The New York Times Opinion section on}
\href{https://www.facebookcorewwwi.onion/nytopinion}{\emph{Facebook}}\emph{,}
\href{http://twitter.com/NYTOpinion}{\emph{Twitter (@NYTopinion)}}
\emph{and}
\href{https://www.instagram.com/nytopinion/}{\emph{Instagram}}\emph{,
and sign up for the}
\href{http://www.nytimes3xbfgragh.onion/newsletters/opiniontoday/}{\emph{Opinion
Today newsletter}}\emph{.}

Advertisement

\protect\hyperlink{after-bottom}{Continue reading the main story}

\hypertarget{site-index}{%
\subsection{Site Index}\label{site-index}}

\hypertarget{site-information-navigation}{%
\subsection{Site Information
Navigation}\label{site-information-navigation}}

\begin{itemize}
\tightlist
\item
  \href{https://help.nytimes3xbfgragh.onion/hc/en-us/articles/115014792127-Copyright-notice}{©~2020~The
  New York Times Company}
\end{itemize}

\begin{itemize}
\tightlist
\item
  \href{https://www.nytco.com/}{NYTCo}
\item
  \href{https://help.nytimes3xbfgragh.onion/hc/en-us/articles/115015385887-Contact-Us}{Contact
  Us}
\item
  \href{https://www.nytco.com/careers/}{Work with us}
\item
  \href{https://nytmediakit.com/}{Advertise}
\item
  \href{http://www.tbrandstudio.com/}{T Brand Studio}
\item
  \href{https://www.nytimes3xbfgragh.onion/privacy/cookie-policy\#how-do-i-manage-trackers}{Your
  Ad Choices}
\item
  \href{https://www.nytimes3xbfgragh.onion/privacy}{Privacy}
\item
  \href{https://help.nytimes3xbfgragh.onion/hc/en-us/articles/115014893428-Terms-of-service}{Terms
  of Service}
\item
  \href{https://help.nytimes3xbfgragh.onion/hc/en-us/articles/115014893968-Terms-of-sale}{Terms
  of Sale}
\item
  \href{https://spiderbites.nytimes3xbfgragh.onion}{Site Map}
\item
  \href{https://help.nytimes3xbfgragh.onion/hc/en-us}{Help}
\item
  \href{https://www.nytimes3xbfgragh.onion/subscription?campaignId=37WXW}{Subscriptions}
\end{itemize}
