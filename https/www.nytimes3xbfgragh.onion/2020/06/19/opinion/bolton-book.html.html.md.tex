Sections

SEARCH

\protect\hyperlink{site-content}{Skip to
content}\protect\hyperlink{site-index}{Skip to site index}

\href{https://myaccount.nytimes3xbfgragh.onion/auth/login?response_type=cookie\&client_id=vi}{}

\href{https://www.nytimes3xbfgragh.onion/section/todayspaper}{Today's
Paper}

\href{/section/opinion}{Opinion}\textbar{}The Doom Where It Happened

\url{https://nyti.ms/37HM5qU}

\begin{itemize}
\item
\item
\item
\item
\item
\item
\end{itemize}

Advertisement

\protect\hyperlink{after-top}{Continue reading the main story}

\href{/section/opinion}{Opinion}

Supported by

\protect\hyperlink{after-sponsor}{Continue reading the main story}

\hypertarget{the-doom-where-it-happened}{%
\section{The Doom Where It Happened}\label{the-doom-where-it-happened}}

Was John Bolton gullible or cynical? Yes.

\href{https://www.nytimes3xbfgragh.onion/by/bret-stephens}{\includegraphics{https://static01.graylady3jvrrxbe.onion/images/2017/08/27/insider/bretstephens/bretstephens-thumbLarge-v6.png}}

By \href{https://www.nytimes3xbfgragh.onion/by/bret-stephens}{Bret
Stephens}

Opinion Columnist

\begin{itemize}
\item
  June 19, 2020
\item
  \begin{itemize}
  \item
  \item
  \item
  \item
  \item
  \item
  \end{itemize}
\end{itemize}

\includegraphics{https://static01.graylady3jvrrxbe.onion/images/2020/06/19/opinion/19stephens1/merlin_173649384_ba53fc0d-fee6-462b-96bd-68ce83db8af6-articleLarge.jpg?quality=75\&auto=webp\&disable=upscale}

Karl Popper once said, ``A theory that explains everything, explains
nothing.'' John Bolton's forthcoming memoir of his 17-month stint as
Donald Trump's national security adviser, ``The Room Where It
Happened,'' offers a corollary: A book that tells all, yet somehow
manages to tell nothing.

Bolton writes that Trump's Ukraine quid pro quo was ``bad policy,
questionable legally, and unacceptable as presidential behavior.'' We
knew that. He writes that Vladimir Putin ``had to be laughing
uproariously at what he had gotten away with in Helsinki.'' Knew that.
He writes that, for Trump, ``obstruction of justice {[}is{]} a way of
life.'' Knew that, too.

Bolton writes that Trump tried to bend the criminal justice system to do
favors for China's Xi Jinping and Turkey's Recep Tayyip Erdogan. Not
surprised. He writes that Mike Pompeo, obsequious with Trump in public,
privately thinks his boss is ``so full of shit.'' Not surprised. He
writes that, in the midst of a trade negotiation, Trump pleaded with Xi
to help ``ensure he'd win'' the 2020 election. Not surprised. He writes
that Trump thought that China's construction of concentration camps for
ethnic Uighurs was ``exactly the right thing to do.'' Appalled --- but
not surprised.

Bolton writes that he is ``hard-pressed to identify any significant
Trump decision during my White House tenure that wasn't driven by
re-election calculations.'' Yep. He writes that Trump's ``thinking was
like an archipelago of dots \ldots{} leaving the rest of us to discern
--- or create --- policy.'' Sounds right. He writes that Trump seemed
not to know that the U.K. was a nuclear power and wondered whether
Finland was part of Russia.

Wait, is Trump some sort of ignoramus?

And on it goes: One knockout revelation after another, ultimately
revealing very little. If by now you haven't concluded that Donald Trump
is ``erratic,'' ``irrational,'' ``foolish,'' and ``stunningly
uninformed'' --- among the epithets Bolton applies to the president ---
this book isn't likely to convince you.

But the larger question looming over Bolton's book isn't about its
subject. It's about its author --- and everyone else who joined the
administration without illusions, participated in it without defiance,
and exited it without shame. How do people like Bolton, Reince Priebus,
Gary Cohn, Rex Tillerson or (when he eventually departs) Pompeo justify
their witting, willing service to this witless, wicked president?

In
\href{https://www.theatlantic.com/magazine/archive/2020/07/trumps-collaborators/612250/}{a
superb essay in The Atlantic}, Anne Applebaum draws on the inspiration
of Czeslaw Milosz's ``The Captive Mind'' to address the question. There
is the relief, and pleasure, of political conformity. There is the
allure of power, or proximity to it. There is a profit motive. There is
a kind of savior complex, in which officials like the
\href{https://www.nytimes3xbfgragh.onion/2018/09/05/opinion/trump-white-house-anonymous-resistance.html}{``Anonymous''
New York Times Op-Ed} writer from 2018 claim to form part of a secret
resistance within the upper reaches of government.

All true. And all eminently applicable to Bolton. But there's an
additional factor at work, described by Hannah Arendt in ``The Origins
of Totalitarianism.'' How do demagogues get away with their nonstop
lying --- even with those who at some level understand they are being
lied to? Arendt observed ``a curiously varying mixture of gullibility
and cynicism with which each member, depending upon his rank and
standing in the movement, is expected to react to the changing lying
statements of the leaders and the central unchanging ideological fiction
of the movement.''

A similar mix seems to explain Bolton's behavior.

It took cynicism to work for a president whose character he disdained
and whose worldview he opposed. It took gullibility to think he could
blunt or influence either. It took cynicism to observe the president
commit multiple potentially impeachable offenses and then sit out
impeachment on the pathetic excuse that Democrats were going about it
the wrong way and that his testimony would have made no meaningful
difference. It took gullibility to assume his book would have any effect
on Trump's re-election prospects now. It took cynicism to reap profits
thanks to a president he betrayed and a nation he let down. It took
gullibility to imagine he'd be applauded as a courageous truth-teller
when his motives are so nakedly vindictive and mercenary.

Above all, it took astonishing foolishness for Bolton to imagine that
his book would advance the thing he claims to care about most --- a
hawkish vision of U.S. foreign policy. That vision will now be forever
tarred by its association with him, a man considered a lunatic by most
liberals and a Judas by many conservatives.

I write all this as someone who shares many of Bolton's hawkish
foreign-policy views. I'm also someone who urged Bolton, while he was
still in office,
\href{https://www.google.com/amp/s/www.nytimes3xbfgragh.onion/2018/07/19/opinion/mike-pompeo-john-bolton-resign.amp.html}{to
resign on principle}. It's a shame he didn't do so while he still had a
chance to preserve his honor, but it isn't a surprise. Only the truly
gullible can act totally cynically and imagine they can escape history's
damning verdict.

\emph{The Times is committed to publishing}
\href{https://www.nytimes3xbfgragh.onion/2019/01/31/opinion/letters/letters-to-editor-new-york-times-women.html}{\emph{a
diversity of letters}} \emph{to the editor. We'd like to hear what you
think about this or any of our articles. Here are some}
\href{https://help.nytimes3xbfgragh.onion/hc/en-us/articles/115014925288-How-to-submit-a-letter-to-the-editor}{\emph{tips}}\emph{.
And here's our email:}
\href{mailto:letters@NYTimes.com}{\emph{letters@NYTimes.com}}\emph{.}

\emph{Follow The New York Times Opinion section on}
\href{https://www.facebookcorewwwi.onion/nytopinion}{\emph{Facebook}}\emph{,}
\href{http://twitter.com/NYTOpinion}{\emph{Twitter (@NYTopinion)}}
\emph{and}
\href{https://www.instagram.com/nytopinion/}{\emph{Instagram}}\emph{.}

Advertisement

\protect\hyperlink{after-bottom}{Continue reading the main story}

\hypertarget{site-index}{%
\subsection{Site Index}\label{site-index}}

\hypertarget{site-information-navigation}{%
\subsection{Site Information
Navigation}\label{site-information-navigation}}

\begin{itemize}
\tightlist
\item
  \href{https://help.nytimes3xbfgragh.onion/hc/en-us/articles/115014792127-Copyright-notice}{©~2020~The
  New York Times Company}
\end{itemize}

\begin{itemize}
\tightlist
\item
  \href{https://www.nytco.com/}{NYTCo}
\item
  \href{https://help.nytimes3xbfgragh.onion/hc/en-us/articles/115015385887-Contact-Us}{Contact
  Us}
\item
  \href{https://www.nytco.com/careers/}{Work with us}
\item
  \href{https://nytmediakit.com/}{Advertise}
\item
  \href{http://www.tbrandstudio.com/}{T Brand Studio}
\item
  \href{https://www.nytimes3xbfgragh.onion/privacy/cookie-policy\#how-do-i-manage-trackers}{Your
  Ad Choices}
\item
  \href{https://www.nytimes3xbfgragh.onion/privacy}{Privacy}
\item
  \href{https://help.nytimes3xbfgragh.onion/hc/en-us/articles/115014893428-Terms-of-service}{Terms
  of Service}
\item
  \href{https://help.nytimes3xbfgragh.onion/hc/en-us/articles/115014893968-Terms-of-sale}{Terms
  of Sale}
\item
  \href{https://spiderbites.nytimes3xbfgragh.onion}{Site Map}
\item
  \href{https://help.nytimes3xbfgragh.onion/hc/en-us}{Help}
\item
  \href{https://www.nytimes3xbfgragh.onion/subscription?campaignId=37WXW}{Subscriptions}
\end{itemize}
