Sections

SEARCH

\protect\hyperlink{site-content}{Skip to
content}\protect\hyperlink{site-index}{Skip to site index}

\href{https://www.nytimes3xbfgragh.onion/section/technology}{Technology}

\href{https://myaccount.nytimes3xbfgragh.onion/auth/login?response_type=cookie\&client_id=vi}{}

\href{https://www.nytimes3xbfgragh.onion/section/todayspaper}{Today's
Paper}

\href{/section/technology}{Technology}\textbar{}Two Gamers Played `The
Last of Us Part II.' They Were Blown Away.

\url{https://nyti.ms/2UYMZdq}

\begin{itemize}
\item
\item
\item
\item
\item
\end{itemize}

Advertisement

\protect\hyperlink{after-top}{Continue reading the main story}

Supported by

\protect\hyperlink{after-sponsor}{Continue reading the main story}

\hypertarget{two-gamers-played-the-last-of-us-part-ii-they-were-blown-away}{%
\section{Two Gamers Played `The Last of Us Part II.' They Were Blown
Away.}\label{two-gamers-played-the-last-of-us-part-ii-they-were-blown-away}}

When the original video game was transcendental, can a sequel top it?
Two Times reporters debate the answer.

\includegraphics{https://static01.graylady3jvrrxbe.onion/images/2020/06/21/business/19lastofus-conversation-01/19lastofus-conversation-01-articleLarge-v2.jpg?quality=75\&auto=webp\&disable=upscale}

\href{https://www.nytimes3xbfgragh.onion/by/mike-isaac}{\includegraphics{https://static01.graylady3jvrrxbe.onion/images/2018/02/16/multimedia/author-mike-isaac/author-mike-isaac-thumbLarge.jpg}}\href{https://www.nytimes3xbfgragh.onion/by/conor-dougherty}{\includegraphics{https://static01.graylady3jvrrxbe.onion/images/2018/07/27/multimedia/author-conor-dougherty/author-conor-dougherty-thumbLarge.png}}

By \href{https://www.nytimes3xbfgragh.onion/by/mike-isaac}{Mike Isaac}
and \href{https://www.nytimes3xbfgragh.onion/by/conor-dougherty}{Conor
Dougherty}

\begin{itemize}
\item
  June 19, 2020
\item
  \begin{itemize}
  \item
  \item
  \item
  \item
  \item
  \end{itemize}
\end{itemize}

Hardcore gamers and post-apocalyptica enthusiasts have waited years for
a sequel to
\href{https://www.nytimes3xbfgragh.onion/2013/06/14/arts/video-games/in-the-video-game-the-last-of-us-survival-favors-the-man.html}{The
Last of Us}, an immersive video game set in a dark, dismal future after
a fungus ravaged Earth and zombies have become the ruling class. The
game unspooled through the lens of two unlikely companions: a hardened,
Texan smuggler named Joel and a 14-year-old, streetwise girl named
Ellie.

\href{https://www.nytimes3xbfgragh.onion/2020/06/19/business/last-of-us-2-review.html}{The
Last of Us Part II} hits stores on Friday, when
\href{https://www.nytimes3xbfgragh.onion/2020/04/21/technology/personaltech/coronavirus-video-game-production.html}{the
studio Naughty Dog} releases what is a winding, punishing follow-up that
explores even more complex ideas and a richer character universe.

Mike Isaac and Conor Dougherty, two New York Times reporters, spent
weeks playing the game in the run up to release. They discussed their
first impressions of the game.

\textbf{Mike:} Let me start with something simple. I'm a huge fan of the
first game, which I thought was perfect as a stand-alone piece. The
story was compelling, and the depth of the characters and their evolving
relationships really felt about as powerful to me as any other part of
the game. I enjoyed the narrative far beyond the shoot em' up zombie
killing.

So I was a bit worried when I saw the sequel was coming, and that the
game studio might be messing up a good thing. But after what felt like
dozens of emotional, gut-wrenching hours playing Part II, I'm glad they
followed up.

Character relationships have become richer, the game's universe has
expanded, and more complicated questions around morality, tribalism and
vengeance are being asked of the characters --- and of ourselves.

I left the game wiped, emotionally. But I do feel like it enriched my
idea of what video games should be. I read your review and really dug
it. But would you consider this game ``fun?''

\includegraphics{https://static01.graylady3jvrrxbe.onion/images/2020/06/21/business/19lastofus-conversation-03/19lastofus-conversation-03-articleLarge.jpg?quality=75\&auto=webp\&disable=upscale}

\textbf{Conor:} If you find this fun in the traditional sense, you're
twisted. But sad movies aren't fun either, and I found the game
interesting and worthwhile in that it told an effective story and
expanded my sense of what big-budget games can do.

\textbf{Mike:} Yeah, that's fair. I won't go too heavily into spoilers,
but I appreciated some of the risks the studio decided to take here
right from the get-go.

The way they are willing to kill off characters made me feel a sense of,
I don't know, unease? Like the world they inhabit is actually dangerous,
and that no one is really safe. Kind of like what an actual zombie
apocalypse might look like.

I know you loved the first one. What struck you as the most novel thing
about the sequel?

\textbf{Conor:} How emotional it made me. At the end I was scared to
finish because I was so worried what Ellie might do, and that's a
feeling I've never had in a game. It left me sad and exhausted, as you
say, but you have to marvel at the storytelling.

\textbf{Mike:} Agreed. It was a feeling few games really give me, which
is why it felt like more than a game? I made the allusion to an
interactive film, given the numerous cut-scenes and story development
cinematics.

One thing that really stuck with me was the level of detail and care
they put into hammering home the points they were trying to make. In
most games, for instance, murder costs nothing, and is often largely the
point of playing. Take
\href{https://www.nytimes3xbfgragh.onion/2014/11/11/arts/video-games/call-of-duty-advanced-warfare-stars-kevin-spacey.html}{Call
of Duty} or some other shooter game where your goal is to just ratchet
up the highest body count.

I'm not saying killing in this game isn't part of it. But each death
takes an emotional toll on the character and, I would argue, on the
player. They do this thing where when you kill a random enemy, one of
their allies might cry out ``Steve!'' or ``Jill!'' or whatever. It's
subtle, but gives an entire back story to someone that in other games
might just be any old NPC (non-player character). And all it took was a
name.

\textbf{Conor:} That dog, man. That dog.

\textbf{Mike:} Ugh. That tore me \emph{up.} And you know how I am with
dogs.

\textbf{Conor:} That's the riskiest thing this game does --- it makes
you feel gross playing it.

In the first game you kill things in defense of a child, which made me
feel OK about all the gore. With this new one --- again, no spoilers ---
it's less about working toward a satisfying ending than it is seeing how
someone else reacts to an event whose outcome you already know.

The surprises are surprises of character and perspective, and I thought
that was deep. Did you find yourself changing your feelings toward the
Abby character, who you're supposed to hate at the beginning and feel
more mixed about at the end?

\textbf{Mike:} Yeah, totally. I was impressed with the way they were
able to flip my feelings on their head, from the starting point of the
story to the ending.

My one big criticism of the game is its length; at points it felt
interminable, and I almost found myself wanting the game to end
precisely because it was a long slog of brutal violence and gore.

Image

The game has a rich character universe.Credit...Brian Finke for The New
York Times

But the flip side of that is what they were able to accomplish with
character development. We had already spent an entire game with the
Ellie character in Part 1. I think, to some degree, we needed just as
much time with Abby to really connect with her. And I think we did.

\textbf{Conor:} And that's what makes these games so amazing: You spend
so much time with these characters and unlike movies get to inhabit
them, so the story unfolds in a way that feels really authentic,
assuming you can handle the zombie killing. (BTW: We haven't even talked
about zombies.) Every medium has its pluses and minuses but the thing
about games is they are immersive, and this one really got into me.

\textbf{Mike:} Indeed. Well, here's to hoping that the next big-budget,
highly anticipated game release can make us equally sad in the future.

Thanks for the chat, my fellow gamer. Now go get to replaying TLOU2
again! You know you want to.

\textbf{Conor:} I'll never forget it, but I'm not sure I can do it
again.

Advertisement

\protect\hyperlink{after-bottom}{Continue reading the main story}

\hypertarget{site-index}{%
\subsection{Site Index}\label{site-index}}

\hypertarget{site-information-navigation}{%
\subsection{Site Information
Navigation}\label{site-information-navigation}}

\begin{itemize}
\tightlist
\item
  \href{https://help.nytimes3xbfgragh.onion/hc/en-us/articles/115014792127-Copyright-notice}{©~2020~The
  New York Times Company}
\end{itemize}

\begin{itemize}
\tightlist
\item
  \href{https://www.nytco.com/}{NYTCo}
\item
  \href{https://help.nytimes3xbfgragh.onion/hc/en-us/articles/115015385887-Contact-Us}{Contact
  Us}
\item
  \href{https://www.nytco.com/careers/}{Work with us}
\item
  \href{https://nytmediakit.com/}{Advertise}
\item
  \href{http://www.tbrandstudio.com/}{T Brand Studio}
\item
  \href{https://www.nytimes3xbfgragh.onion/privacy/cookie-policy\#how-do-i-manage-trackers}{Your
  Ad Choices}
\item
  \href{https://www.nytimes3xbfgragh.onion/privacy}{Privacy}
\item
  \href{https://help.nytimes3xbfgragh.onion/hc/en-us/articles/115014893428-Terms-of-service}{Terms
  of Service}
\item
  \href{https://help.nytimes3xbfgragh.onion/hc/en-us/articles/115014893968-Terms-of-sale}{Terms
  of Sale}
\item
  \href{https://spiderbites.nytimes3xbfgragh.onion}{Site Map}
\item
  \href{https://help.nytimes3xbfgragh.onion/hc/en-us}{Help}
\item
  \href{https://www.nytimes3xbfgragh.onion/subscription?campaignId=37WXW}{Subscriptions}
\end{itemize}
