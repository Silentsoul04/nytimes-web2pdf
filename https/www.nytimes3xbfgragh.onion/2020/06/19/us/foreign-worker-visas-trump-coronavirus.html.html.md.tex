Sections

SEARCH

\protect\hyperlink{site-content}{Skip to
content}\protect\hyperlink{site-index}{Skip to site index}

\href{https://www.nytimes3xbfgragh.onion/section/us}{U.S.}

\href{https://myaccount.nytimes3xbfgragh.onion/auth/login?response_type=cookie\&client_id=vi}{}

\href{https://www.nytimes3xbfgragh.onion/section/todayspaper}{Today's
Paper}

\href{/section/us}{U.S.}\textbar{}Businesses Brace for Possible Limits
on Foreign Worker Visas

\url{https://nyti.ms/30YJkQK}

\begin{itemize}
\item
\item
\item
\item
\item
\end{itemize}

Advertisement

\protect\hyperlink{after-top}{Continue reading the main story}

Supported by

\protect\hyperlink{after-sponsor}{Continue reading the main story}

\hypertarget{businesses-brace-for-possible-limits-on-foreign-worker-visas}{%
\section{Businesses Brace for Possible Limits on Foreign Worker
Visas}\label{businesses-brace-for-possible-limits-on-foreign-worker-visas}}

Citing the economic slump, the president could act this week to limit
H-1B, L-1 and other visas as well as a program allowing foreign students
to work in the United States after they graduate.

\includegraphics{https://static01.graylady3jvrrxbe.onion/images/2020/06/19/us/19VIRUS-VISAS/merlin_173668047_40b0825f-c8ca-436e-a3ed-1c94a2eec1f7-articleLarge.jpg?quality=75\&auto=webp\&disable=upscale}

\href{https://www.nytimes3xbfgragh.onion/by/miriam-jordan/}{\includegraphics{https://static01.graylady3jvrrxbe.onion/images/2018/02/16/multimedia/author-miriam-jordan/author-miriam-jordan-thumbLarge-v2.png}}

By \href{https://www.nytimes3xbfgragh.onion/by/miriam-jordan/}{Miriam
Jordan}

\begin{itemize}
\item
  Published June 19, 2020Updated June 22, 2020
\item
  \begin{itemize}
  \item
  \item
  \item
  \item
  \item
  \end{itemize}
\end{itemize}

Maya Nasr was admitted to the Massachusetts Institute of Technology at
16 to study aerospace engineering. Currently a doctoral student there,
she has been working on NASA's next Mars rover, slated for launch this
summer.

``By the time I finish my Ph.D., I will have spent 10 years in the U.S.
researching what I am passionate about --- getting people to Mars and
human space exploration,'' said Ms. Nasr, 23, who is Lebanese. ``I would
really like to stay here and work in this field.''

But recently she has been wracked with worry that the economic downturn
that has left millions of Americans unemployed could threaten the visa
program that would allow her to work as a foreigner in the United States
once she graduates. ``If I had to, I would consider Canada, the U.K. or
Europe, but the U.S. is the place,'' she said.

President Trump is expected to issue an executive order early this week
to temporarily suspend various work visas that businesses rely on to
hire foreigners, and also lay the groundwork for regulatory changes that
would limit employment opportunities for foreign graduates of U.S.
universities like Ms. Nasr.

The details and scope of the plan remain unclear, and it is still a work
in progress. But the coming order has elicited an extraordinary response
from a diverse coalition that includes universities and sectors spanning
manufacturing, technology and consulting, which have been inundating the
White House with letters and phone calls.

``What is scaring so many different groups here is that the order is
going to impact everyone,'' said Shev Dalal-Dheini, director of
government relations at the American Immigration Lawyers Association.

Mr. Trump on April 22 signed an executive order that suspended for 60
days the issuance of green cards for applicants outside the country,
describing it as protection for unemployed Americans amid the
coronavirus pandemic. But he stopped short of suspending visas and
programs that allow U.S. employers to hire foreign workers.

Additional limitations on foreign skilled and seasonal workers are now
in the works, said several people who have been communicating with the
administration in an attempt to narrow the scope of the proposed new
limitations. ``What is being proposed is significant; we just don't know
how far it will go,'' Ms. Dalal-Dheini said.

A key target is expected to be the H-1B visa, often issued to computer
programmers and other skilled workers who critics say often displace
Americans from such jobs. The order is also expected to temporarily halt
other visas, including L-1s, for executives transferred within
companies; H-2Bs, for seasonal workers who often work in landscaping and
hospitality; and J-1s, issued to au pairs, students on work-study summer
programs and others.

The administration is also likely to freeze or downsize a popular
program, Optional Practical Training, that enables graduates of U.S.
universities like Ms. Nasr to work in the country for up to three years
if they are in science, technology or mathematics fields. Once in the
program, the graduates are often sponsored for H-1Bs by their employers,
who eventually may help them get green cards.

In the 2019 fiscal year, almost 139,000 new H-1B petitions were
approved; 77,000 L-1 visas were issued, as well as 66,000 H-2Bs, visas
for unskilled workers that Mr. Trump has regularly used to staff his
resorts. There were about 200,000 new J-1 workers hired in 2018,
according to the latest data available. About 225,000 graduates of U.S.
universities were authorized to remain in the country to work.

The visa suspensions would likely extend into the next fiscal year,
starting Oct. 1, according to people familiar with the planning.

The proposed limits on foreign worker visas, if adopted, would be the
latest
\href{https://www.nytimes3xbfgragh.onion/2020/06/12/us/politics/coronavirus-trump-immigration-policies.html?searchResultPosition=2}{restriction
on immigration} imposed by the Trump administration since
\href{https://www.nytimes3xbfgragh.onion/2020/05/03/us/coronavirus-immigration-stephen-miller-public-health.html}{the
onset of the coronavirus pandemic}.

Immigration opponents argue that recipients of ``nonimmigrant'' visas
compete with Americans for jobs, and that the present unemployment level
justifies a clampdown.

``The president should insist that certain employers who have found it
to be cheaper and more convenient to hire visa workers instead cast down
their bucket here first, and get used to hiring U.S. workers again,''
said Jessica Vaughan, policy director at the Center for Immigration
Studies, which lobbies to curb immigration. She said she was regularly
consulted by White House aides on the matter.

The planned new restrictions could hamper the ability of companies to
relocate personnel, affecting both American multinationals that transfer
employees from abroad to the United States and foreign firms that send
employees to do stints in U.S. cities.

The National Association of Manufacturers, which represents 14,000 small
and large companies, sent a letter to the White House on June 1 warning
that the planned immigration restrictions would stymie recovery efforts.

``We urge you to avoid immigration actions, either temporary or long
term, that would cause uncertainty and impose great costs on our nation
at this critical time,'' said the letter, signed by the association's
president, Jay Timmons.

Two groups affiliated with the industrialist Charles Koch last week sent
a letter to Larry Kudlow, director of the National Economic Council, and
Jared Kushner, senior adviser to Mr. Trump, urging the administration to
``refrain from imposing additional barriers'' on visas.

The letter from Americans for Prosperity and the Libre Initiative said
research had shown that immigrants were key to an economic rebound.

Thomas J. Donohue, chief executive of the U.S. Chamber of Commerce, also
sent a letter to Mr. Trump last week calling on him to reconsider a visa
ban.

Several Republican senators have voiced their opposition to elements of
the proposed order. But influential members of Mr. Trump's entourage
believe that a pause in visas, coupled with additional regulations,
would induce employers to hire Americans and pay them higher wages.

There could be risks in seriously reducing the number of foreign
workers. Many businesses have learned during the pandemic that employees
can work remotely from almost anywhere. If immigration is further
restricted, companies might feel compelled to move jobs abroad,
defenders of the visa programs caution, a trend that could undermine the
U.S. economy in the long term.

``I have spoken to the administration and they are open-minded about
this, but I emphasized that if we go too far on restricting immigration,
from C.E.O.s I talked to, there is a risk companies will move jobs
outside the country,'' said Rich Lesser, chief executive of the Boston
Consulting Group, an American global management consultancy.

The additional regulatory changes now under review would affect many
other people already in the United States.

Among the changes under consideration are eliminating the H-4 visa,
which allows spouses of H-1B visa holders to work; ending the Optional
Practical Training program, or shortening the time foreign graduates can
work in the United States; and tightening H-1B requirements along with a
\$20,000 fee per applicant sponsored for the visa.

While some Republican lawmakers have called for rescinding the O.P.T.
program, 21 Republican members of Congress this month called for it to
remain intact, saying in a letter to the Trump administration that it
was a magnet for foreign students --- a key source of revenue for many
universities, whose numbers have been falling.

The Association of American Universities issued an action alert to
presidents and chancellors, urging them to engage the administration on
the executive order.

``The message you are sending to international students here, and those
considering coming here, is that we don't value them,'' said Benjamin
Lane, vice chair of the external affairs board at M.I.T.'s graduate
student council. International students represent 45 percent of M.I.T.
graduate students.

Fan Chen, a data scientist from China, was hired by a U.S. multinational
in Washington after earning a master's degree at the University of
Michigan.

For three consecutive years, the company, which initially had hired her
under the O.P.T. program, sponsored her for an H-1B visa. However, she
was not selected in the lottery that U.S. Citizenship and Immigration
Services holds when the number of applications exceeds the allotted
annual number of visas.

To keep employing Ms. Fan, the company transferred her last year to its
branch in the United Kingdom, where she has continued to work with her
U.S.-based team.

Ms. Chen and her manager were ecstatic when she was chosen for a visa in
this year's round. Then news began to circulate about the impending
executive order, and it is no longer certain that she will be able to
return to the United States in the fall.

The company's lawyer and human resources personnel have been trying to
reassure her. ``They are trying to make it sound like this order isn't
likely to happen,'' Ms. Chen said. ``But they are also saying we might
have to think of Plan B.''

Advertisement

\protect\hyperlink{after-bottom}{Continue reading the main story}

\hypertarget{site-index}{%
\subsection{Site Index}\label{site-index}}

\hypertarget{site-information-navigation}{%
\subsection{Site Information
Navigation}\label{site-information-navigation}}

\begin{itemize}
\tightlist
\item
  \href{https://help.nytimes3xbfgragh.onion/hc/en-us/articles/115014792127-Copyright-notice}{©~2020~The
  New York Times Company}
\end{itemize}

\begin{itemize}
\tightlist
\item
  \href{https://www.nytco.com/}{NYTCo}
\item
  \href{https://help.nytimes3xbfgragh.onion/hc/en-us/articles/115015385887-Contact-Us}{Contact
  Us}
\item
  \href{https://www.nytco.com/careers/}{Work with us}
\item
  \href{https://nytmediakit.com/}{Advertise}
\item
  \href{http://www.tbrandstudio.com/}{T Brand Studio}
\item
  \href{https://www.nytimes3xbfgragh.onion/privacy/cookie-policy\#how-do-i-manage-trackers}{Your
  Ad Choices}
\item
  \href{https://www.nytimes3xbfgragh.onion/privacy}{Privacy}
\item
  \href{https://help.nytimes3xbfgragh.onion/hc/en-us/articles/115014893428-Terms-of-service}{Terms
  of Service}
\item
  \href{https://help.nytimes3xbfgragh.onion/hc/en-us/articles/115014893968-Terms-of-sale}{Terms
  of Sale}
\item
  \href{https://spiderbites.nytimes3xbfgragh.onion}{Site Map}
\item
  \href{https://help.nytimes3xbfgragh.onion/hc/en-us}{Help}
\item
  \href{https://www.nytimes3xbfgragh.onion/subscription?campaignId=37WXW}{Subscriptions}
\end{itemize}
