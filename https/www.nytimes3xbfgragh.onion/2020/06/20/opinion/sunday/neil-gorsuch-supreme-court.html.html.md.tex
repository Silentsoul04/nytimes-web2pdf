Sections

SEARCH

\protect\hyperlink{site-content}{Skip to
content}\protect\hyperlink{site-index}{Skip to site index}

\href{https://www.nytimes3xbfgragh.onion/section/opinion/sunday}{Sunday
Review}

\href{https://myaccount.nytimes3xbfgragh.onion/auth/login?response_type=cookie\&client_id=vi}{}

\href{https://www.nytimes3xbfgragh.onion/section/todayspaper}{Today's
Paper}

\href{/section/opinion/sunday}{Sunday Review}\textbar{}The Tempting of
Neil Gorsuch

\url{https://nyti.ms/2UYTU6k}

\begin{itemize}
\item
\item
\item
\item
\item
\end{itemize}

Advertisement

\protect\hyperlink{after-top}{Continue reading the main story}

\href{/section/opinion}{Opinion}

Supported by

\protect\hyperlink{after-sponsor}{Continue reading the main story}

\hypertarget{the-tempting-of-neil-gorsuch}{%
\section{The Tempting of Neil
Gorsuch}\label{the-tempting-of-neil-gorsuch}}

Another conservative justice's arc bends toward juristocracy.

\href{https://www.nytimes3xbfgragh.onion/by/ross-douthat}{\includegraphics{https://static01.graylady3jvrrxbe.onion/images/2018/04/03/opinion/ross-douthat/ross-douthat-thumbLarge.png}}

By \href{https://www.nytimes3xbfgragh.onion/by/ross-douthat}{Ross
Douthat}

Opinion Columnist

\begin{itemize}
\item
  June 20, 2020
\item
  \begin{itemize}
  \item
  \item
  \item
  \item
  \item
  \end{itemize}
\end{itemize}

\includegraphics{https://static01.graylady3jvrrxbe.onion/images/2020/06/21/opinion/sunday/21Douthat/21Douthat-articleLarge.jpg?quality=75\&auto=webp\&disable=upscale}

It might surprise contemporary Americans that for most of our history,
what we call ``culture war'' debates --- arguments about rights, social
justice, the moral organization of society --- were often settled
through democratic deliberation, rather than the kind of ruling the
Supreme Court just delivered on gay and transgender civil rights.
Congress debated and passed laws. State legislatures did the same.
Constitutional amendments were proposed, passed, ratified --- and when
necessary, repealed.

This was true even when the debates in question led to the Civil War. In
1864, while Grant and Sherman prepared their offensives, Abraham Lincoln
didn't demand that the Supreme Court declare slavery unconstitutional.
Instead he pushed the Senate to amend the Constitution to abolish it.

Subsequent battles over Catholicism and public education, women's
suffrage and temperance all had similar legislative goals. The long
struggle for civil rights was aided by Brown v. Board of Education and
Loving v. Virginia, but the crucial action was in Congress, where the
major civil rights laws ultimately passed. The following decade,
feminists naturally sought their own constitutional amendment, the
E.R.A., and its defeat was seen as a milestone in conservatism's rise.

All of those battles belong to a lost world. Today constitutional
amendments have become unimaginable, Congress barely legislates, and the
Supreme Court manages our social and cultural debates. Our affirmative
action system was designed by Lewis Powell and amended by Sandra Day
O'Connor. The boundaries of voting rights and free expression are
policed by John Roberts. Our abortion laws reflect the preferences of
Anthony Kennedy. And now anti-discrimination law and religious liberty
protections will reflect what Neil Gorsuch, author of the new decision,
thinks is right and good.

Occasionally, a conservative ruling or Republican appointment threatens
to inspire a left-wing revolt against the juristocracy. But the courts
have not yet claimed as much power over economic policy as over social
policy, and the willingness of Republican appointees to swim leftward on
social issues has reassured liberals that judicial power is just a
natural extension of meritocracy.

This means it's been left to religious conservatives --- the losers in
many of the court's culture-war decisions, going back to the school
prayer rulings after World War II --- to make the consistent case
against
\href{https://www.firstthings.com/article/1996/11/the-end-of-democracy-the-judicial-usurpation-of-politics}{the
judicial usurpation of politics}.

In making that case conservatives have championed constrained schools of
legal interpretation, originalism and textualism, against a
values-driven jurisprudence. A ``living constitutionalism'' naturally
usurps democratic powers, the argument goes, in a way that a
jurisprudence bound to textual language or original intent does not.

There was power and plausibility in this view, especially as embodied in
the brilliance of the late Antonin Scalia. But it always reflected a
slightly naïve view of how power works and grows.

For one thing, the law's ambiguities provide ample space for even a mind
that imagines itself constrained --- even Scalia's mind, in some cases
--- to argue its way into ruling on behalf of its ideological
objectives. Meanwhile politics abhors a power vacuum, and our
juristocracy has claimed new powers in part because Congress doesn't
want them, a tendency that originalism is powerless to change.

And the public seems to have accepted this abdication. ``The main
question in American social life,'' the blogger Tanner Greer recently
observed, ``is not `how do we make that happen?' but `how do we get
management to take our side?' '' The Supreme Court, clothed in
meritocratic authority, seems more like \emph{management} than Congress.

All of these tendencies converged in Gorsuch's decision. The goal of his
ruling, civil rights protections for gay and transgender Americans, is
widely shared; the problem is that Congress has no desire to negotiate
over the uncertain implications --- for religious liberty, single-sex
institutions, transgender athletes, and more. So Gorsuch (with Roberts's
support) took the burden on himself, discovering the desired protections
in the text of the 1964 Civil Rights Act (an act of sophistry, not
interpretation)
\href{https://www.nationalreview.com/2020/06/what-are-the-adverse-consequences-of-gorsuchs-bostock-ruling/}{and
then suggesting} that all the uncertainties would be worked out in
future cases --- in other words, by Neil Gorsuch, arbiter of sexual and
religious liberties alike.

That a textualist philosophy and a Federalist Society pedigree didn't
restrain him from this self-aggrandizement suggests the conservative
legal movement needs either
\href{https://www.firstthings.com/web-exclusives/2020/06/a-morally-empty-jurisprudence}{a
new theory of its purpose}, a new personnel strategy, or both.

But outside the right, the welcome afforded Gorsuch's ruling --- which
reached the popular outcome, and relieved our legislators of a
responsibility they didn't want --- is a telling indication of how our
system is understood to work. We may officially have three branches of
government, but Americans seem to accept that it's more like 2.25: A
presidency that acts unilaterally whenever possible, a high court that
checks the White House and settles culture wars, and a Congress that
occasionally bestirs itself to pass a budget.

What sort of Republic this is, and whether we will keep it, is for a
higher court than Neil Gorsuch's to decide.

\emph{The Times is committed to publishing}
\href{https://www.nytimes3xbfgragh.onion/2019/01/31/opinion/letters/letters-to-editor-new-york-times-women.html}{\emph{a
diversity of letters}} \emph{to the editor. We'd like to hear what you
think about this or any of our articles. Here are some}
\href{https://help.nytimes3xbfgragh.onion/hc/en-us/articles/115014925288-How-to-submit-a-letter-to-the-editor}{\emph{tips}}\emph{.
And here's our email:}
\href{mailto:letters@NYTimes.com}{\emph{letters@NYTimes.com}}\emph{.}

\emph{Follow The New York Times Opinion section on}
\href{https://www.facebookcorewwwi.onion/nytopinion}{\emph{Facebook}}\emph{,}
\href{http://twitter.com/NYTOpinion}{\emph{Twitter (@NYTOpinion)}}
\emph{and}
\href{https://www.instagram.com/nytopinion/}{\emph{Instagram}}\emph{,
join the Facebook political discussion group,}
\href{https://www.facebookcorewwwi.onion/groups/votingwhilefemale/}{\emph{Voting
While Female}}\emph{.}

Advertisement

\protect\hyperlink{after-bottom}{Continue reading the main story}

\hypertarget{site-index}{%
\subsection{Site Index}\label{site-index}}

\hypertarget{site-information-navigation}{%
\subsection{Site Information
Navigation}\label{site-information-navigation}}

\begin{itemize}
\tightlist
\item
  \href{https://help.nytimes3xbfgragh.onion/hc/en-us/articles/115014792127-Copyright-notice}{©~2020~The
  New York Times Company}
\end{itemize}

\begin{itemize}
\tightlist
\item
  \href{https://www.nytco.com/}{NYTCo}
\item
  \href{https://help.nytimes3xbfgragh.onion/hc/en-us/articles/115015385887-Contact-Us}{Contact
  Us}
\item
  \href{https://www.nytco.com/careers/}{Work with us}
\item
  \href{https://nytmediakit.com/}{Advertise}
\item
  \href{http://www.tbrandstudio.com/}{T Brand Studio}
\item
  \href{https://www.nytimes3xbfgragh.onion/privacy/cookie-policy\#how-do-i-manage-trackers}{Your
  Ad Choices}
\item
  \href{https://www.nytimes3xbfgragh.onion/privacy}{Privacy}
\item
  \href{https://help.nytimes3xbfgragh.onion/hc/en-us/articles/115014893428-Terms-of-service}{Terms
  of Service}
\item
  \href{https://help.nytimes3xbfgragh.onion/hc/en-us/articles/115014893968-Terms-of-sale}{Terms
  of Sale}
\item
  \href{https://spiderbites.nytimes3xbfgragh.onion}{Site Map}
\item
  \href{https://help.nytimes3xbfgragh.onion/hc/en-us}{Help}
\item
  \href{https://www.nytimes3xbfgragh.onion/subscription?campaignId=37WXW}{Subscriptions}
\end{itemize}
