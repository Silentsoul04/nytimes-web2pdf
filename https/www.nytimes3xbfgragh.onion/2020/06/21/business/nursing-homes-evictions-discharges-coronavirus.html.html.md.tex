Sections

SEARCH

\protect\hyperlink{site-content}{Skip to
content}\protect\hyperlink{site-index}{Skip to site index}

\href{https://www.nytimes3xbfgragh.onion/section/business}{Business}

\href{https://myaccount.nytimes3xbfgragh.onion/auth/login?response_type=cookie\&client_id=vi}{}

\href{https://www.nytimes3xbfgragh.onion/section/todayspaper}{Today's
Paper}

\href{/section/business}{Business}\textbar{}`They Just Dumped Him Like
Trash': Nursing Homes Evict Vulnerable Residents

\url{https://nyti.ms/37ME7g9}

\begin{itemize}
\item
\item
\item
\item
\item
\end{itemize}

\href{https://www.nytimes3xbfgragh.onion/news-event/coronavirus?action=click\&pgtype=Article\&state=default\&region=TOP_BANNER\&context=storylines_menu}{The
Coronavirus Outbreak}

\begin{itemize}
\tightlist
\item
  live\href{https://www.nytimes3xbfgragh.onion/2020/08/03/world/coronavirus-covid-19.html?action=click\&pgtype=Article\&state=default\&region=TOP_BANNER\&context=storylines_menu}{Latest
  Updates}
\item
  \href{https://www.nytimes3xbfgragh.onion/interactive/2020/us/coronavirus-us-cases.html?action=click\&pgtype=Article\&state=default\&region=TOP_BANNER\&context=storylines_menu}{Maps
  and Cases}
\item
  \href{https://www.nytimes3xbfgragh.onion/interactive/2020/science/coronavirus-vaccine-tracker.html?action=click\&pgtype=Article\&state=default\&region=TOP_BANNER\&context=storylines_menu}{Vaccine
  Tracker}
\item
  \href{https://www.nytimes3xbfgragh.onion/2020/08/02/us/covid-college-reopening.html?action=click\&pgtype=Article\&state=default\&region=TOP_BANNER\&context=storylines_menu}{College
  Reopening}
\item
  \href{https://www.nytimes3xbfgragh.onion/live/2020/08/03/business/stock-market-today-coronavirus?action=click\&pgtype=Article\&state=default\&region=TOP_BANNER\&context=storylines_menu}{Economy}
\end{itemize}

Advertisement

\protect\hyperlink{after-top}{Continue reading the main story}

Supported by

\protect\hyperlink{after-sponsor}{Continue reading the main story}

\hypertarget{they-just-dumped-him-like-trash-nursing-homes-evict-vulnerable-residents}{%
\section{`They Just Dumped Him Like Trash': Nursing Homes Evict
Vulnerable
Residents}\label{they-just-dumped-him-like-trash-nursing-homes-evict-vulnerable-residents}}

Nursing homes across the country are kicking out old and disabled
residents and sending them to homeless shelters and rundown motels.

\includegraphics{https://static01.graylady3jvrrxbe.onion/images/2020/06/18/business/00Virus-Evictions-lakeview-01/merlin_173678574_d42ebf58-15fe-4d54-ba06-d55a9b73da76-articleLarge.jpg?quality=75\&auto=webp\&disable=upscale}

By
\href{https://www.nytimes3xbfgragh.onion/by/jessica-silver-greenberg}{Jessica
Silver-Greenberg} and
\href{https://www.nytimes3xbfgragh.onion/by/amy-julia-harris}{Amy Julia
Harris}

\begin{itemize}
\item
  Published June 21, 2020Updated July 23, 2020
\item
  \begin{itemize}
  \item
  \item
  \item
  \item
  \item
  \end{itemize}
\end{itemize}

On a chilly afternoon in April, Los Angeles police found an old,
disoriented man crumpled on a Koreatown sidewalk.

Several days earlier, RC Kendrick, an 88-year-old with dementia, was
living at Lakeview Terrace, a
\href{https://www.nytimes3xbfgragh.onion/2020/07/23/nyregion/nursing-homes-deaths-cuomo.html}{nursing
home} with a history of regulatory problems. His family had placed him
there to make sure he got round-the-clock care after his condition
deteriorated and he began disappearing for days at a time.

But on April 6, the nursing home deposited Mr. Kendrick at an
unregulated boardinghouse --- without bothering to inform his family.
Less than 24 hours later, Mr. Kendrick was wandering the city alone.

According to three Lakeview employees, Mr. Kendrick's ouster came as the
nursing home was telling staff members to try to clear out
less-profitable residents to make room for a new class of customers who
would generate more revenue: patients with Covid-19.

Image

A missing person report for RC Kendrick in January, shortly before his
family sent him to Lakeview Terrace.Credit...Los Angeles County
Sheriff's Department

More than any other institution in America, nursing homes have come to
symbolize the deadly destruction of the coronavirus crisis. More than
51,000 residents and employees of nursing homes and long-term care
facilities have died, representing more than 40 percent of the total
death toll in the United States.

But even as they have been ravaged, nursing homes have also been
enlisted in the response to the outbreak. They are taking on
coronavirus-stricken patients to ease the burden on overwhelmed
hospitals --- and, at times, to bolster their bottom lines.

A Lakeview official said the company's evictions were appropriate and
weren't an attempt to free space for Covid-19 patients. But similar
scenes are playing out at nursing homes nationwide. They are kicking out
old and disabled residents --- among the people most susceptible to the
coronavirus --- and shunting them into homeless shelters, rundown motels
and other unsafe facilities, according to 22 watchdogs in 16 states, as
well as dozens of elder-care lawyers, social workers and former nursing
home executives.

Many of the evictions, known as involuntary discharges, appear to
violate federal rules that require nursing homes to place residents in
safe locations and to provide them with at least 30 days' notice before
forcing them to leave.

While the popular conception of nursing homes is of places where elderly
people live, much of their business is caring for patients of all ages
and income levels who are recovering from surgery or acute illnesses
like strokes. Medicare often pays for short-term rehabilitation stints;
Medicaid covers longer-term stays for poor people.

Nursing homes have long had a financial incentive to evict Medicaid
patients in favor of those who pay through private insurance or
Medicare, which reimburses nursing homes at a much higher rate than
Medicaid. More than 10,000 residents and their families complained to
watchdogs about being discharged in 2018, the most recent year for which
data are available.

The pandemic has intensified the situation.

With nursing homes not allowing visitors, there is less outside scrutiny
of their practices. Fifteen state-funded ombudsmen said in interviews
that some homes appear to be taking advantage of that void to evict
vulnerable residents.

\includegraphics{https://static01.graylady3jvrrxbe.onion/images/2020/06/18/business/00virus-evictions-lakeview-02/merlin_173678664_2a591080-2b55-498f-ad45-3c70a8ac2266-articleLarge.jpg?quality=75\&auto=webp\&disable=upscale}

Many nursing homes are struggling in part because one of their most
profitable businesses --- post-surgery rehab --- has withered as states
restricted hospitals from performing nonessential services.

\hypertarget{latest-updates-economy}{%
\section{\texorpdfstring{\href{https://www.nytimes3xbfgragh.onion/live/2020/08/03/business/stock-market-today-coronavirus?action=click\&pgtype=Article\&state=default\&region=MAIN_CONTENT_1\&context=storylines_live_updates}{Latest
Updates:
Economy}}{Latest Updates: Economy}}\label{latest-updates-economy}}

\href{https://www.nytimes3xbfgragh.onion/live/2020/08/03/business/stock-market-today-coronavirus?action=click\&pgtype=Article\&state=default\&region=MAIN_CONTENT_1\&context=storylines_live_updates\#the-chicago-fed-president-says-its-up-to-congress-to-save-the-economy}{11h
ago}

\href{https://www.nytimes3xbfgragh.onion/live/2020/08/03/business/stock-market-today-coronavirus?action=click\&pgtype=Article\&state=default\&region=MAIN_CONTENT_1\&context=storylines_live_updates\#the-chicago-fed-president-says-its-up-to-congress-to-save-the-economy}{The
Chicago Fed president says it's up to Congress to save the economy.}

\href{https://www.nytimes3xbfgragh.onion/live/2020/08/03/business/stock-market-today-coronavirus?action=click\&pgtype=Article\&state=default\&region=MAIN_CONTENT_1\&context=storylines_live_updates\#faa-says-boeing-has-effectively-mitigated-defects-in-the-737-max}{11h
ago}

\href{https://www.nytimes3xbfgragh.onion/live/2020/08/03/business/stock-market-today-coronavirus?action=click\&pgtype=Article\&state=default\&region=MAIN_CONTENT_1\&context=storylines_live_updates\#faa-says-boeing-has-effectively-mitigated-defects-in-the-737-max}{F.A.A.
says Boeing has `effectively mitigated' defects in the 737 Max.}

\href{https://www.nytimes3xbfgragh.onion/live/2020/08/03/business/stock-market-today-coronavirus?action=click\&pgtype=Article\&state=default\&region=MAIN_CONTENT_1\&context=storylines_live_updates\#small-businesses-got-emergency-loans-but-not-what-they-expected}{14h
ago}

\href{https://www.nytimes3xbfgragh.onion/live/2020/08/03/business/stock-market-today-coronavirus?action=click\&pgtype=Article\&state=default\&region=MAIN_CONTENT_1\&context=storylines_live_updates\#small-businesses-got-emergency-loans-but-not-what-they-expected}{Small
businesses got emergency loans, but not what they expected.}

\href{https://www.nytimes3xbfgragh.onion/live/2020/08/03/business/stock-market-today-coronavirus?action=click\&pgtype=Article\&state=default\&region=MAIN_CONTENT_1\&context=storylines_live_updates}{See
more updates}

More live coverage:
\href{https://www.nytimes3xbfgragh.onion/2020/08/03/world/coronavirus-covid-19.html?action=click\&pgtype=Article\&state=default\&region=MAIN_CONTENT_1\&context=storylines_live_updates}{Global}

Treating Covid-19 patients quickly became a popular way to fill that
financial void.

Last fall, the Centers for Medicare and Medicaid changed the formula for
reimbursing nursing homes, making it more profitable to take in sicker
patients for a short period of time. Covid-19 patients can bring in at
least \$600 more a day in Medicare dollars than people with relatively
mild health issues, according to nursing home executives and state
officials.

``They could be big money for nursing homes,'' said David Grabowski, a
professor of health care policy at Harvard Medical School.

It is not always about the money. Several states, including New York,
New Jersey and California, urged nursing homes to accept Covid-19
patients to help relieve pressure on hospitals. Some nursing home
employees worried that would endanger their vulnerable residents.

There is no national data on the number of nursing home residents who
have been moved into homeless shelters, motels and other facilities. The
New York Times contacted more than 80 state-funded nursing-home
ombudsmen in 46 states for a tally of involuntary discharges during the
pandemic at facilities they monitor. Twenty six ombudsmen, from 18
states, provided figures to The Times: a total of more than 6,400
discharges, many to homeless shelters.

``We're dealing with unsafe discharges, whether it be to a homeless
shelter or to unlicensed facilities, on a daily basis, and Covid-19 has
made this all more urgent,'' said Molly Davies, the Los Angeles
ombudsman, whose office works with residents at about 400 nursing homes.

In Connecticut, a nursing-home resident was told he had less than a week
to pack his things and move to a homeless shelter, according to the
resident's lawyer. In Philadelphia, a nursing home planned to discharge
a resident with schizophrenia to one of the city's office of homeless
services sites, but it was closed during the pandemic. A lawyer said she
intervened to stop the eviction on the grounds that it was unsafe.

Image

The Silvercrest Center is one of many nursing homes that have tried to
evict residents during the pandemic.~Credit...Hiroko Masuike/The New
York Times

In New York City, the epicenter of the pandemic, nursing homes tried to
discharge at least 27 residents to homeless shelters from February
through May, according to data from the New York City Department of
Homeless Services. Ombudsmen and city officials blocked many of the
discharges, which they said were medically unsafe.

But those figures are most likely a dramatic undercount. ``What we're
seeing is just the tip of the iceberg,'' said Susan Dooha, executive
director of Center for Independence of the Disabled, a nonprofit group
that is the home of the Long Term Care Ombudsman Program in New York
City.

Traditionally, ombudsmen would regularly go to nursing homes. In March,
though, ombudsmen --- and residents' families --- were required to stop
visiting. Evictions followed.

``It felt opportunistic, where some homes were basically seizing the
moment when everyone is looking the other way to move people out,'' said
Laurie Facciarossa Brewer, a long-term care ombudsman in New Jersey.

Nursing homes are allowed to evict residents if they aren't able to pay
for their care, are endangering others in the facility or have
sufficiently recovered. Under federal law, before discharging patients
against their will, nursing homes are required to give formal notice to
the resident and to the ombudsman's office. They must also find a safe
alternative location for the resident to go, whether that is an assisted
living facility, an apartment or, in rare circumstances, a homeless
shelter.

But some homes have figured out a workaround: They pressure residents to
leave. Many residents assume they have no choice, and the nursing homes
often do not report them to ombudsmen.

That is what David Mellor said happened to him. Mr. Mellor, 54, was
recovering from spinal surgery that left him numb from the neck down at
a nursing home in Fremont, Calif. In April, Mr. Mellor said, the staff
at the Windsor Park Care Center, an 85-bed facility, told him that he
had to go to a hotel to clear the way for coronavirus patients. Mr.
Mellor, who had been trying to arrange long-term housing, felt he had no
choice and agreed to leave.

``I saw what was going on,'' Mr. Mellor said. ``They were forcing people
out.'' At the Radisson Hotel in Oakland, which was being used to house
the homeless, Mr. Mellor said there was no one to help him learn to walk
again or to assist him with the medications he takes to control his
blood sugar and pain.

A spokesman for the Windsor Park Care Center declined to comment. It is
part of a chain owned by Lee Samson, a major
\href{https://timesofsandiego.com/politics/2019/04/05/trump-raises-4-million-for-re-election-campaign-at-los-angeles-dinners/}{fund-raiser}
for President Trump. ``Whatever my political affiliation, Windsor's
commitment to protecting its residents will never be compromised,'' Mr.
Samson said.

Nursing home evictions can be disruptive and dangerous during normal
times --- and even more so during a pandemic that preys on the elderly
and those with underlying medical conditions.

In March, seven groups that represent nursing home residents
\href{https://www.yumpu.com/en/document/read/63189113/moratorium-on-nursing-home-discharge-letter}{wrote
to New York's health department}, urging it to stop nursing homes from
evicting residents because they are ``particularly vulnerable to the
Covid-19 virus.'' Such discharges, especially to homeless shelters, they
wrote, ``pose particular public health risks, due to the close living
quarters in shelters.'' The letter also warned that sending patients
from nursing homes --- hotbeds of the coronavirus --- into the community
could hasten the spread of the disease.

Advocates for nursing home residents have also urged California's health
department to halt evictions.

While at least four states have restricted nursing homes from evicting
patients during the pandemic, New York and California have not. Some
companies appear to be taking advantage.

In California, Rockport Healthcare Services, which manages the state's
largest chain of for-profit nursing homes, has repeatedly been cited by
state regulators for illegal evictions.

On March 31, with Covid-19 cases soaring, a Rockport executive wrote in
an email to colleagues that they should begin ``discharge planning
immediately,'' noting that any discharges should be done safely.

Dr. Michael Wasserman, who was the chief executive of Rockport until
2018, said that was code to kick out the least-lucrative residents.
``You are looking to replace the poorest, least profitable patients with
the highest paying ones,'' said Dr. Wasserman, who resigned after
clashing with the chain's owner.

This spring, Los Angeles County designated three of Rockport's nursing
homes as preferred destinations for Covid-19 patients. Since then, one
of them has tried unsuccessfully to evict at least two residents against
their will, according to a lawyer who was contacted by the residents'
families.

David Silver, the chief executive of Rockport, said the company was
trying to be a good partner to the state by making room for an expected
surge of Covid-19 patients. ``This has absolutely nothing to do with
money,'' he said. He declined to comment on individual residents, citing
confidentiality.

In New York City, the Silvercrest Center for Nursing and Rehabilitation
in Queens tried to evict more than 20 residents at one point in March,
according to residents and elder care lawyers. Employees at Silvercrest
--- including the director of social services --- told residents or
family members that the discharges were necessary to free beds for
Covid-19 patients.

Image

Abraham Hightower in the window of his room at the Silvercrest Center.
He said the home has tried to evict him three times.Credit...Hiroko
Masuike/The New York Times

Abraham Hightower, a 57-year-old man on Medicaid who suffers from kidney
problems and high blood pressure, arrived at Silvercrest in January.
Since then, the home has tried to evict him three times.

In February, Silvercrest tried to send him to a Best Western hotel that
New York City uses as a homeless shelter, according to Mr. Hightower and
his lawyer. He appealed and an administrative judge determined that such
a facility was not appropriate given his health needs.

Mr. Hightower said he was told by Silvercrest employees that they were
evicting residents to make way for Covid-19 patients. In March, he
received another discharge notice, this time sending him to a homeless
shelter in Manhattan, according to records reviewed by The Times. When
Mr. Hightower appealed, Silvercrest backed down.

This month, Silvercrest issued the third eviction notice. Mr.
Hightower's appeal is pending.

``They just want to get rid of me,'' he said.

Michael Tretola, the president of Silvercrest, declined to comment on
Mr. Hightower's case or to say how many residents have been evicted.
``The health and safety of every patient under our care is always our
first concern,'' he said.

Lakeview Terrace in Los Angeles, which evicted the 88-year-old Mr.
Kendrick, has a history of illegally ousting residents. In February
2019, the Los Angeles city attorney, Mike Feuer, reached a \$600,000
settlement with the nursing home to resolve accusations that it had
illegally evicted mentally ill and homeless residents. As part of that
settlement, in which Lakeview denied wrongdoing, prosecutors appointed
someone to monitor the facility. As the coronavirus intensified in
March, the monitor had to stop visiting.

Around this time, said three Lakeview employees, who weren't authorized
to speak publicly, their superiors began encouraging them to find ways
to discharge residents to make room for coronavirus patients.

On April 6, the staff moved Mr. Kendrick to an unlicensed boardinghouse
in Van Nuys, Calif., about 20 miles away.

The next day, the police called Mr. Kendrick's nephew, Darryl Kennedy.
They had found his uncle, who had wandered away from the boardinghouse,
Mr. Kennedy said.

``They just dumped him like trash,'' Mr. Kennedy said.

David Weaver, the administrator of Lakeview Terrace, wouldn't say why
Mr. Kendrick was evicted, citing confidentiality, but he said all of the
nursing home's discharges were ``clinically appropriate.''

Mr. Weaver said that while Lakeview --- which has space for 99 patients
--- has discharged or transferred 16 residents since March, it had not
done so to make room for coronavirus patients and in fact had not
knowingly admitted any.

After the police found Mr. Kendrick, Mr. Kennedy agreed to let his uncle
stay with him, even though he could not provide the level of supervision
that Mr. Kendrick would have received at Lakeview.

About a month later, Mr. Kennedy woke up at 3 a.m. to find Mr. Kendrick
standing over him with a steak knife. His uncle stabbed him in the back
and the head. Mr. Kennedy called the police. He needed 30 stitches.

Mr. Kendrick turned 89 on May 6. He spent his birthday at the Los
Angeles County jail, about four miles from Lakeview Terrace.

Advertisement

\protect\hyperlink{after-bottom}{Continue reading the main story}

\hypertarget{site-index}{%
\subsection{Site Index}\label{site-index}}

\hypertarget{site-information-navigation}{%
\subsection{Site Information
Navigation}\label{site-information-navigation}}

\begin{itemize}
\tightlist
\item
  \href{https://help.nytimes3xbfgragh.onion/hc/en-us/articles/115014792127-Copyright-notice}{©~2020~The
  New York Times Company}
\end{itemize}

\begin{itemize}
\tightlist
\item
  \href{https://www.nytco.com/}{NYTCo}
\item
  \href{https://help.nytimes3xbfgragh.onion/hc/en-us/articles/115015385887-Contact-Us}{Contact
  Us}
\item
  \href{https://www.nytco.com/careers/}{Work with us}
\item
  \href{https://nytmediakit.com/}{Advertise}
\item
  \href{http://www.tbrandstudio.com/}{T Brand Studio}
\item
  \href{https://www.nytimes3xbfgragh.onion/privacy/cookie-policy\#how-do-i-manage-trackers}{Your
  Ad Choices}
\item
  \href{https://www.nytimes3xbfgragh.onion/privacy}{Privacy}
\item
  \href{https://help.nytimes3xbfgragh.onion/hc/en-us/articles/115014893428-Terms-of-service}{Terms
  of Service}
\item
  \href{https://help.nytimes3xbfgragh.onion/hc/en-us/articles/115014893968-Terms-of-sale}{Terms
  of Sale}
\item
  \href{https://spiderbites.nytimes3xbfgragh.onion}{Site Map}
\item
  \href{https://help.nytimes3xbfgragh.onion/hc/en-us}{Help}
\item
  \href{https://www.nytimes3xbfgragh.onion/subscription?campaignId=37WXW}{Subscriptions}
\end{itemize}
