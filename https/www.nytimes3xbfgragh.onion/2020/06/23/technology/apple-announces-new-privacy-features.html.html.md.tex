Sections

SEARCH

\protect\hyperlink{site-content}{Skip to
content}\protect\hyperlink{site-index}{Skip to site index}

\href{https://www.nytimes3xbfgragh.onion/section/technology}{Technology}

\href{https://myaccount.nytimes3xbfgragh.onion/auth/login?response_type=cookie\&client_id=vi}{}

\href{https://www.nytimes3xbfgragh.onion/section/todayspaper}{Today's
Paper}

\href{/section/technology}{Technology}\textbar{}Apple Announces New
Privacy Features

\href{https://nyti.ms/2CrFjKa}{https://nyti.ms/2CrFjKa}

\begin{itemize}
\item
\item
\item
\item
\item
\end{itemize}

Advertisement

\protect\hyperlink{after-top}{Continue reading the main story}

Supported by

\protect\hyperlink{after-sponsor}{Continue reading the main story}

on tech

\hypertarget{apple-announces-new-privacy-features}{%
\section{Apple Announces New Privacy
Features}\label{apple-announces-new-privacy-features}}

Apple says it will make it tougher for apps to track you. It will also
help you wash your hands.

\includegraphics{https://static01.graylady3jvrrxbe.onion/images/2020/06/23/business/23ontech-videostill/23ontech-videostill-mediumSquareAt3X.png}

\href{https://www.nytimes3xbfgragh.onion/by/brian-x-chen}{\includegraphics{https://static01.graylady3jvrrxbe.onion/images/2018/02/16/multimedia/author-brian-x-chen/author-brian-x-chen-thumbLarge.jpg}}

By \href{https://www.nytimes3xbfgragh.onion/by/brian-x-chen}{Brian X.
Chen}

\begin{itemize}
\item
  June 23, 2020
\item
  \begin{itemize}
  \item
  \item
  \item
  \item
  \item
  \end{itemize}
\end{itemize}

\emph{This article is part of the On Tech newsletter. You can}
\href{https://www.nytimes3xbfgragh.onion/newsletters/signup/OT}{\emph{sign
up here}} \emph{to receive it weekdays.}

Greetings, friends! I'm your personal tech columnist. I'll be taking the
wheel for today's intro.

This week is a big one for Apple, which is hosting its
\href{https://www.nytimes3xbfgragh.onion/2020/06/22/technology/apple-macs-intel-chips.html}{annual
conference for the software developers} who make apps for Apple phones,
tablets, watches and computers. The company started the weeklong event
with a video presentation streamed on Monday, outlining its new
features.

The presentation was chock-full of jargon and deeply technical stuff
that only engineers would understand. But here's the news that you may
care about:

\begin{itemize}
\item
  \textbf{Apple is making it tougher for apps to track you.} Unbeknown
  to many of us, thousands of apps that we lovingly use on our
  smartphones have
  \href{https://www.nytimes3xbfgragh.onion/2019/07/03/technology/personaltech/fingerprinting-track-devices-what-to-do.html}{invisible
  trackers running in the background}. The trackers may be collecting
  and sharing our personal information, like our location, email address
  and phone number, with businesses and other entities for the purpose
  of serving targeted ads. You can't opt in or opt out of app tracking.

  That may soon change. Apple said that beginning this fall with its
  next mobile operating system, iOS 14, it will require so-called
  third-party apps to ask for your permission to track you.

  Apple will also give iPhone and iPad users more control over how their
  location is shared. Instead of sharing your precise location with an
  app, you will be able to share your approximate location, giving a
  developer a rough idea of where you are. That could be helpful if you
  are using a news app, for example, and you want to see articles about
  your hometown but don't want to share precisely where you live.

  In the past, Apple and Google have required apps to ask for permission
  to access sensors, such as your camera and microphone. These new
  protections expand on Apple's efforts to give users greater
  transparency and control over the data collected about us. (Your move,
  Google.)
\item
  \textbf{The Mac is about to make a big shift}. Have you ever noticed
  how sluggish Macs feel compared to Apple's mobile devices? Macs use
  Intel processors, but mobile processors have outpaced Intel chips in
  terms of speed and power efficiency. Even the cheapest new Apple
  phone, the
  \href{https://www.nytimes3xbfgragh.onion/2020/05/06/technology/personaltech/apple-iphone-se-review.html}{\$399
  iPhone SE}, by some measures outperforms the most powerful Mac
  laptops, which cost more than \$2,500.

  That's why it's a big deal that this week Apple announced the
  \href{https://www.nytimes3xbfgragh.onion/2020/06/19/technology/apple-intel-breakup.html}{beginning
  of the Mac's transition to Apple-made silicon}, which will be based on
  the same chip architecture powering iPhones and iPads. If all goes
  well, we can expect Macs with snappier performance and much longer
  battery life, and they should also be able to run iPhone and iPad
  apps.

  The transition to Apple chips is expected to take two years. If you
  buy a Mac in 2022, it should have the horsepower of an iPad, but work
  with a mouse and keyboard.
\item
  \textbf{The Apple Watch is trying to be more helpful during the
  pandemic.} Apple said the next version of the Apple Watch operating
  system, WatchOS 7, would take advantage of the watch's motion sensors
  to detect when you are washing your hands --- and start a 20-second
  timer to ensure you scrub thoroughly. The watch will also use its
  sensors to track sleep patterns. These are relatively minor new
  features, but we could all probably use better sleep and hygiene these
  days.
\end{itemize}

\begin{center}\rule{0.5\linewidth}{\linethickness}\end{center}

\hypertarget{seeking-therapy-through-screens}{%
\subsection{Seeking therapy through
screens}\label{seeking-therapy-through-screens}}

\emph{Thanks to Brian for tag-teaming with me. This is Shira Ovide for
the rest of today's dispatch.}

The last few months have been A LOT. As the stresses on our bodies,
finances, families and minds have piled up, it's gotten more complicated
to seek help as physicians and mental health specialists paused seeing
patients in person because of coronavirus fears.

But a silver lining, said Dr. Joy Harden Bradford, a psychologist and
founder of the mental health resource
\href{https://therapyforblackgirls.com/about/}{Therapy for Black Girls},
is that moving therapy sessions online has been more rewarding for some
people --- although that hasn't been true for everyone.

For some, taking face-to-face interaction out of therapy makes difficult
conversations easier. ``Adding a screen is just enough of a barrier
where it maybe feels safer to share something,'' Dr. Bradford said.

Dr. Bradford had advice for people seeking therapy right now through
online appointments. During virtual sessions, try to find a private,
personal space --- even if that means taking a video call alone in your
car, while on a walk or in the bathroom with the door closed.

She also advised people to give themselves a time to transition after
sessions end, rather than jumping right back into family or work
obligations.

Dr. Bradford also said people shouldn't be afraid to find a new
therapist if their current one isn't a good match, for example a
practitioner who isn't responsive to the added stresses that some black
people are feeling.

For people worried about costs, Dr. Bradford said some individual
therapists and apps such as Talkspace offer free appointments, and a
number of health insurers have been waiving co-payments for people to
speak to therapists virtually during the coronavirus.

And nonprofit groups including the Loveland Foundation, Open Path
Psychotherapy Collective and the Boris Lawrence Henson Foundation offer
financial assistance or lower-cost options for people who are seeking
therapy.

\begin{center}\rule{0.5\linewidth}{\linethickness}\end{center}

\hypertarget{before-we-go-}{%
\subsection{Before we go \ldots{}}\label{before-we-go-}}

\begin{itemize}
\item
  \textbf{Never waste a crisis, I guess?} Companies are repurposing
  their technology products as
  \href{https://www.nytimes3xbfgragh.onion/2020/06/22/business/virus-office-workplace-return.html}{untested
  and sometimes comical-sounding coronavirus-fighting systems}. My
  colleagues
  \href{https://www.nytimes3xbfgragh.onion/by/natasha-singer}{Natasha
  Singer} and
  \href{https://www.nytimes3xbfgragh.onion/by/julie-creswell}{Julie
  Creswell} write about software for tracking product inventory that has
  been remade into an employee virus-tracking system. Income
  verification software has been pitched for workers to report their
  health status to employers.
\item
  \textbf{I want to know everything about K-pop fans:} The **** New York
  Times pop music writer
  \href{https://www.nytimes3xbfgragh.onion/by/joe-coscarelli}{Joe
  Coscarelli}
  \href{https://www.nytimes3xbfgragh.onion/2020/06/22/arts/music/k-pop-fans-trump-politics.html}{explains
  the internet might of Korean pop music fans}, who often use their
  online savvy to support their favorite boy bands and now are
  \href{https://www.nytimes3xbfgragh.onion/2020/06/21/style/tiktok-trump-rally-tulsa.html}{taking
  credit for helping inflate attendance expectations} for President
  Trump's Tulsa rally.

  Also, read the Times Opinion writer
  \href{https://www.nytimes3xbfgragh.onion/by/charlie-warzel}{Charlie
  Warzel} fretting about
  \href{https://www.nytimes3xbfgragh.onion/2020/06/22/opinion/trump-protest-gen-z.html}{oversimplifying
  young people's motivations} in the online information wars.
\item
  \textbf{Your regular reminder to be cautious of what you see and share
  online:} A Times team found 41 U.S. cities and towns where false
  rumors spread about anti-fascist activists coming to cause mayhem.
  \href{https://www.nytimes3xbfgragh.onion/by/davey-alba}{Davey Alba}
  and Ben Decker traced the origins of how friends, neighbors and
  trusted businesses in four towns
  \href{https://www.nytimes3xbfgragh.onion/2020/06/22/technology/antifa-local-disinformation.html}{sometimes
  unwittingly spread bogus fears} and forced law enforcement to respond.
\end{itemize}

\hypertarget{hugs-to-this}{%
\subsubsection{Hugs to this}\label{hugs-to-this}}

\href{https://twitter.com/wildbirdfund/status/1272639253898571776}{Rescued
baby swans}! Look how fuzzy they are! (Thanks to my colleague Dodai
Stewart for sharing this one.)

\begin{center}\rule{0.5\linewidth}{\linethickness}\end{center}

\emph{We want to hear from you. Tell us what you think of this
newsletter and what else you'd like us to explore. You can reach us at}
\href{mailto:ontech@NYTimes.com?subject=On\%20Tech\%20Feedback}{\emph{ontech@NYTimes.com.}}
**

\emph{Get this newsletter in your inbox every
weekday;}\href{https://www.nytimes3xbfgragh.onion/newsletters/signup/OT}{\emph{please
sign up here}}\emph{.}

Advertisement

\protect\hyperlink{after-bottom}{Continue reading the main story}

\hypertarget{site-index}{%
\subsection{Site Index}\label{site-index}}

\hypertarget{site-information-navigation}{%
\subsection{Site Information
Navigation}\label{site-information-navigation}}

\begin{itemize}
\tightlist
\item
  \href{https://help.nytimes3xbfgragh.onion/hc/en-us/articles/115014792127-Copyright-notice}{©~2020~The
  New York Times Company}
\end{itemize}

\begin{itemize}
\tightlist
\item
  \href{https://www.nytco.com/}{NYTCo}
\item
  \href{https://help.nytimes3xbfgragh.onion/hc/en-us/articles/115015385887-Contact-Us}{Contact
  Us}
\item
  \href{https://www.nytco.com/careers/}{Work with us}
\item
  \href{https://nytmediakit.com/}{Advertise}
\item
  \href{http://www.tbrandstudio.com/}{T Brand Studio}
\item
  \href{https://www.nytimes3xbfgragh.onion/privacy/cookie-policy\#how-do-i-manage-trackers}{Your
  Ad Choices}
\item
  \href{https://www.nytimes3xbfgragh.onion/privacy}{Privacy}
\item
  \href{https://help.nytimes3xbfgragh.onion/hc/en-us/articles/115014893428-Terms-of-service}{Terms
  of Service}
\item
  \href{https://help.nytimes3xbfgragh.onion/hc/en-us/articles/115014893968-Terms-of-sale}{Terms
  of Sale}
\item
  \href{https://spiderbites.nytimes3xbfgragh.onion}{Site Map}
\item
  \href{https://help.nytimes3xbfgragh.onion/hc/en-us}{Help}
\item
  \href{https://www.nytimes3xbfgragh.onion/subscription?campaignId=37WXW}{Subscriptions}
\end{itemize}
