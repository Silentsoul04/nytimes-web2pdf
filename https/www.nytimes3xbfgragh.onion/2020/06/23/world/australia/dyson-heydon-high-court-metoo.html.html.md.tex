Sections

SEARCH

\protect\hyperlink{site-content}{Skip to
content}\protect\hyperlink{site-index}{Skip to site index}

\href{https://www.nytimes3xbfgragh.onion/section/world/australia}{Australia}

\href{https://myaccount.nytimes3xbfgragh.onion/auth/login?response_type=cookie\&client_id=vi}{}

\href{https://www.nytimes3xbfgragh.onion/section/todayspaper}{Today's
Paper}

\href{/section/world/australia}{Australia}\textbar{}A Sexual Harasser
Spent Years on Australia's Top Court, an Inquiry Finds

\url{https://nyti.ms/3erXqOf}

\begin{itemize}
\item
\item
\item
\item
\item
\end{itemize}

Advertisement

\protect\hyperlink{after-top}{Continue reading the main story}

Supported by

\protect\hyperlink{after-sponsor}{Continue reading the main story}

\hypertarget{a-sexual-harasser-spent-years-on-australias-top-court-an-inquiry-finds}{%
\section{A Sexual Harasser Spent Years on Australia's Top Court, an
Inquiry
Finds}\label{a-sexual-harasser-spent-years-on-australias-top-court-an-inquiry-finds}}

Dyson Heydon, a judge on the country's highest court for a decade,
harassed at least six women, a court inquiry found. He has denied the
accusations.

\includegraphics{https://static01.graylady3jvrrxbe.onion/images/2020/06/23/world/23oz-judge-1/merlin_173824860_3c83e2e9-f671-4edd-ab9a-ccd92396856e-articleLarge.jpg?quality=75\&auto=webp\&disable=upscale}

\href{https://www.nytimes3xbfgragh.onion/by/damien-cave}{\includegraphics{https://static01.graylady3jvrrxbe.onion/images/2018/10/08/multimedia/author-damien-cave/author-damien-cave-thumbLarge.png}}\href{https://www.nytimes3xbfgragh.onion/by/isabella-kwai}{\includegraphics{https://static01.graylady3jvrrxbe.onion/images/2019/09/17/reader-center/author-isabella-kwai/author-isabella-kwai-thumbLarge.png}}

By \href{https://www.nytimes3xbfgragh.onion/by/damien-cave}{Damien Cave}
and \href{https://www.nytimes3xbfgragh.onion/by/isabella-kwai}{Isabella
Kwai}

\begin{itemize}
\item
  June 23, 2020
\item
  \begin{itemize}
  \item
  \item
  \item
  \item
  \item
  \end{itemize}
\end{itemize}

SYDNEY, Australia --- For years there were private whispers and warnings
from young women about a pattern of abuse by one of Australia's most
powerful men --- a judge on the country's highest court.

Now, an internal court inquiry has found that the judge, Dyson Heydon,
had indeed sexually harassed at least six young women on his staff,
reigniting \#MeToo outrage in a
\href{https://www.nytimes3xbfgragh.onion/2019/06/05/world/australia/journalist-raids.html}{secretive},
\href{https://www.nytimes3xbfgragh.onion/2019/02/26/world/australia/australia-sex-abuse-secrecy.html}{proudly
masculine country} where many women are still calling for a fuller
reckoning.

The findings led Chief Justice Susan Kiefel to issue a statement Monday
confirming the investigation's conclusions. ``We are ashamed that this
could have happened at the High Court of Australia,'' she said. ``We
have made a sincere apology to the six women whose complaints were borne
out.''

The inquiry has cast a harsh spotlight on the relationship between
senior judges and often-younger associates, who are top-level law
graduates chosen for their promise and placed under intense pressure in
situations where power flows one way. The investigation also laid bare
what many Australian lawyers have described as a male-dominated and
insular culture where harassment is rife and speaking up costly.

Though nonstatutory and lacking the ability to mete out punishment for
the now-retired Mr. Heydon, the internal findings have already prompted
promises of reform. Chief Justice Kiefel said the investigation, led by
Vivienne Thom, a former inspector-general of intelligence and security,
included a series of recommendations that would be adopted by the High
Court. But many of them have more to do with training associates than
powerful justices.

Through his lawyers, Mr. Heydon, 77, who retired in 2013, has denied
``any allegation of predatory behavior or breaches of the law,''
according to a statement. Though he apologized if ``any conduct of his
has caused offense,'' the statement added that the offense had been
``inadvertent and unintended.''

The statement also noted that the inquiry had not been conducted by a
lawyer, judge or tribunal member and had not allowed Mr. Heydon's
representatives to cross-examine those who made the complaints.

His response has created an extraordinary standoff between the chief
justice and Mr. Heydon, a former colleague and conservative appointee
who served as a justice on the High Court for a decade and who had long
been respected as one of the country's foremost legal authorities. The
inquiry started in March 2019 after two of the judge's former associates
notified Chief Justice Kiefel that they had been harassed.

But the pattern of alleged abuse stretched back years. Many of the
details
were\href{https://www.smh.com.au/national/high-court-inquiry-finds-former-justice-dyson-heydon-sexually-harassed-associates-20200622-p5550w.html}{first
revealed} on Monday in an article by The Sydney Morning Herald, which
started reporting on Mr. Heydon three years ago, when two former
associates shared their stories with investigative reporters.

One former legal assistant, Chelsea Tabart, said that in 2012, on her
first day working for Mr. Heydon, he suggested they stop for a drink
after an office dinner. He took her to a private room at a prestigious
club, she said, and put his hand on her thigh. She was 22 at the time.

Another former associate, Rachael Patterson Collins, said that in 2005,
when she was 26, Mr. Heydon made repeated advances toward her --- on one
occasion asking if he could kiss her.

When she later asked him why, as a married man, he was behaving this
way, she said he replied, ``Because you're beautiful.''

The Herald's reporters went on to tally numerous accusations of
predatory behavior by Mr. Heydon from several other women, including
claims of stalking and forcible touching.

One judge said he once slid his hand between her thighs at a
professional law dinner.

In another case, Noor Blumer, a Canberra lawyer and former president of
the ACT Law Society who was seated next to Mr. Heydon at a University of
Canberra event in April 2013, said he groped her under the table before
later trying to kiss her without her consent in an empty room. According
to the University of Canberra, Mr. Heydon was asked to leave the event
after a student also complained about ``inappropriate behavior'' that
same evening.

``I was upset and disgusted by it all,'' Ms. Blumer told
\href{https://www.smh.com.au/politics/federal/the-judge-s-hands-became-very-busy-under-the-table-lawyer-says-heydon-groped-her-20200622-p554zg.html}{The
Sydney Morning Herald}. Ms. Blumer declined to comment further when
reached by phone.

\includegraphics{https://static01.graylady3jvrrxbe.onion/images/2020/06/23/world/23oz-judge-2/merlin_100063450_ba483eb5-695d-4fae-879a-a8e5882e9879-articleLarge.jpg?quality=75\&auto=webp\&disable=upscale}

Prime Minister Scott Morrison said on Tuesday that the accusations
against Mr. Heydon were ``very disturbing and very concerning.'' Some
lawmakers have called for the former justice to be stripped of awards
such as Companion of the Order of Australia, awarded to Australians of
merit. But Mr. Morrison said it was too soon to make such a decision,
adding that there should be ``proper processes'' to determine whether
such honors should be revoked.

The evidence so far suggests that Mr. Heydon was known to be a threat to
women for much of his career, including his years serving on the High
Court from 2003 to 2013, but he was never held accountable.

Josh Bornstein, a lawyer for three of Mr. Heydon's former associates,
\href{https://www.abc.net.au/radio/melbourne/programs/mornings/dyson-heydon-shattered-associates-career-aspirations/12383230}{told
reporters with the Australian Broadcasting Corporation} that
``hundreds'' in the industry had known about Mr. Heydon's behavior. He
said that the justice's actions had ``annihilated'' the women's career
aspirations. All three clients have abandoned their legal ambitions
because of their experience with Mr. Heydon, he said.

The judge had stood at the peak of a deeply hierarchical profession, Mr.
Bornstein said, and for his accusers, harassment arrived with their
first job at the bottom of the industry. ``That shattered their passion
and their faith in the law,'' he said.

Now, he added, his clients are seeking claims for compensation for the
harm they suffered as a result of the harassment.

Many Australian women --- lawyers and others --- said on Tuesday that
they hoped the inquiry and its fallout would help other women come
forward, but they also worried that the fall of one powerful figure
would not be enough to create the deeper changes that they say are
needed.

``Based on what's been reported, this was an open secret, and that
speaks to a fundamentally toxic culture,'' said Nina Funnell, a longtime
advocate for sexual assault survivors in Australia. ``Cultural change
takes a huge amount of time and labor, and while cases like this
absolutely refocus people and highlight the need for that labor, the
headlines themselves actually don't produce the cultural change.''

The problems within Australia's legal profession have been well known
for years. In a 2013 survey by the Law Council of Australia, one in six
female respondents indicated they had experienced inappropriate behavior
in the workplace.

``Since then, there has been more research saying it's even more than
that,'' said Pauline Wright, the president of the Law Council. ``It's
actually closer to one in two.''

Hopefully, Ms. Wright said, the High Court inquiry would restore some
credibility to the profession. ``What people need to understand, and
what they can take away from this, is that when they complain about
something in the legal profession, they will be treated with respect,''
Ms. Wright said. ``They will be taken seriously.''

Still,
\href{https://www.nytimes3xbfgragh.onion/2019/05/01/business/australia-sexual-harassment-nondisclosure-agreement.html}{many
obstacles remain}.

Australia's
\href{https://www.nytimes3xbfgragh.onion/2019/06/05/world/australia/journalist-raids.html}{defamation
laws heavily favor plaintiffs} and have kept many accusations of
harassment hidden in reporters' files for fear of a costly lawsuit.
Complaints to state bar associations have frequently gone nowhere.

Spurred by the investigation's recommendations, the court has agreed to
clarify that confidentiality agreements for legal assistants relate only
to the work of the court and make clear that their duties do not extend
to an obligation to attend social functions.

``I'd be cautious celebrating too soon because we haven't changed the
structural barriers --- such as changing community attitudes, which are
still steeped in victim blaming --- or having reporting mechanisms which
are well functioning and swift and we haven't done anything yet about
defamation,'' Ms. Funnell said.

``It's almost the exception that proves the rule,'' she added. ``It's
one old goat.''

Advertisement

\protect\hyperlink{after-bottom}{Continue reading the main story}

\hypertarget{site-index}{%
\subsection{Site Index}\label{site-index}}

\hypertarget{site-information-navigation}{%
\subsection{Site Information
Navigation}\label{site-information-navigation}}

\begin{itemize}
\tightlist
\item
  \href{https://help.nytimes3xbfgragh.onion/hc/en-us/articles/115014792127-Copyright-notice}{©~2020~The
  New York Times Company}
\end{itemize}

\begin{itemize}
\tightlist
\item
  \href{https://www.nytco.com/}{NYTCo}
\item
  \href{https://help.nytimes3xbfgragh.onion/hc/en-us/articles/115015385887-Contact-Us}{Contact
  Us}
\item
  \href{https://www.nytco.com/careers/}{Work with us}
\item
  \href{https://nytmediakit.com/}{Advertise}
\item
  \href{http://www.tbrandstudio.com/}{T Brand Studio}
\item
  \href{https://www.nytimes3xbfgragh.onion/privacy/cookie-policy\#how-do-i-manage-trackers}{Your
  Ad Choices}
\item
  \href{https://www.nytimes3xbfgragh.onion/privacy}{Privacy}
\item
  \href{https://help.nytimes3xbfgragh.onion/hc/en-us/articles/115014893428-Terms-of-service}{Terms
  of Service}
\item
  \href{https://help.nytimes3xbfgragh.onion/hc/en-us/articles/115014893968-Terms-of-sale}{Terms
  of Sale}
\item
  \href{https://spiderbites.nytimes3xbfgragh.onion}{Site Map}
\item
  \href{https://help.nytimes3xbfgragh.onion/hc/en-us}{Help}
\item
  \href{https://www.nytimes3xbfgragh.onion/subscription?campaignId=37WXW}{Subscriptions}
\end{itemize}
