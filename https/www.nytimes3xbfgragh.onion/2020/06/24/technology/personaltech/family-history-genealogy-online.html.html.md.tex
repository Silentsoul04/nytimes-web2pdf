Sections

SEARCH

\protect\hyperlink{site-content}{Skip to
content}\protect\hyperlink{site-index}{Skip to site index}

\href{https://www.nytimes3xbfgragh.onion/section/technology/personaltech}{Personal
Tech}

\href{https://myaccount.nytimes3xbfgragh.onion/auth/login?response_type=cookie\&client_id=vi}{}

\href{https://www.nytimes3xbfgragh.onion/section/todayspaper}{Today's
Paper}

\href{/section/technology/personaltech}{Personal Tech}\textbar{}How to
Dig Up Family History Online

\href{https://nyti.ms/3fUvR0p}{https://nyti.ms/3fUvR0p}

\begin{itemize}
\item
\item
\item
\item
\item
\end{itemize}

\href{https://www.nytimes3xbfgragh.onion/spotlight/at-home?action=click\&pgtype=Article\&state=default\&region=TOP_BANNER\&context=at_home_menu}{At
Home}

\begin{itemize}
\tightlist
\item
  \href{https://www.nytimes3xbfgragh.onion/2020/07/28/books/time-for-a-literary-road-trip.html?action=click\&pgtype=Article\&state=default\&region=TOP_BANNER\&context=at_home_menu}{Take:
  A Literary Road Trip}
\item
  \href{https://www.nytimes3xbfgragh.onion/2020/07/29/magazine/bored-with-your-home-cooking-some-smoky-eggplant-will-fix-that.html?action=click\&pgtype=Article\&state=default\&region=TOP_BANNER\&context=at_home_menu}{Cook:
  Smoky Eggplant}
\item
  \href{https://www.nytimes3xbfgragh.onion/2020/07/27/travel/moose-michigan-isle-royale.html?action=click\&pgtype=Article\&state=default\&region=TOP_BANNER\&context=at_home_menu}{Look
  Out: For Moose}
\item
  \href{https://www.nytimes3xbfgragh.onion/interactive/2020/at-home/even-more-reporters-editors-diaries-lists-recommendations.html?action=click\&pgtype=Article\&state=default\&region=TOP_BANNER\&context=at_home_menu}{Explore:
  Reporters' Obsessions}
\end{itemize}

Advertisement

\protect\hyperlink{after-top}{Continue reading the main story}

Supported by

\protect\hyperlink{after-sponsor}{Continue reading the main story}

Tech Tip

\hypertarget{how-to-dig-up-family-history-online}{%
\section{How to Dig Up Family History
Online}\label{how-to-dig-up-family-history-online}}

Digitized newspaper archives and hyperlocal historical sources can help
you understand how your ancestors lived.

\includegraphics{https://static01.graylady3jvrrxbe.onion/images/2020/06/24/technology/personaltech/24TECHTIP_TOP/24TECHTIP_TOP-articleLarge.jpg?quality=75\&auto=webp\&disable=upscale}

\href{https://www.nytimes3xbfgragh.onion/by/j-d-biersdorfer}{\includegraphics{https://static01.graylady3jvrrxbe.onion/images/2018/06/14/multimedia/author-j-d-biersdorfer/author-j-d-biersdorfer-thumbLarge.png}}

By \href{https://www.nytimes3xbfgragh.onion/by/j-d-biersdorfer}{J. D.
Biersdorfer}

\begin{itemize}
\item
  June 24, 2020
\item
  \begin{itemize}
  \item
  \item
  \item
  \item
  \item
  \end{itemize}
\end{itemize}

Long before the internet made it easy to share the nuances of daily
life,
\href{https://en.wikipedia.org/wiki/Wikipedia:List_of_online_newspaper_archives\#United_States}{local
newspapers} and other regional publications reported the business,
society and civic news of the people in the community. For budding
genealogists, finding an ancestor in an old microfilmed newspaper and
reading contemporaneous accounts of her turn in the school play or his
all-city bowling championship provide a glimpse into the past that's
more textured than a chart of names and dates.

Taking a more narrative approach to the family story can be a
time-consuming detective project with no guaranteed results. But once
you have a name and know when and where the person lived, you can start
your quest to find out \emph{how} they lived. Here's how to get started.

\hypertarget{digging-up-your-roots}{%
\subsection{Digging Up Your Roots}\label{digging-up-your-roots}}

If you're just beginning to climb your family tree and need names on the
branches, a subscription service like
\href{https://www.ancestry.com/}{Ancestry} or
\href{https://www.myheritage.com}{MyHeritage} can be an easy place to
start gathering information. In addition to billions of digitized
records (like census data, draft rolls and religious registries), these
services include tutorials, articles, message boards and other tools to
help learn you learn how to find your people.

Image

Large subscription genealogy services like Ancestry host billions of
historical records from the National Archives and elsewhere that you can
search.Credit...The New York Times

When you get some names pinned to your tree, you may also start to
receive hints of possible undiscovered relatives from the site's
algorithms or the service's other members to help you along. If you're
not sure you want to commit to a regular subscription fee, look for a
free trial period.

\hypertarget{finding-alternative-resources}{%
\subsection{Finding Alternative
Resources}\label{finding-alternative-resources}}

Sleuthing on a budget? Visit the
\href{https://www.archives.gov}{National Archives} site and its
\href{https://www.archives.gov/research/genealogy}{``Resources for
Genealogists''} page for links to information on finding
\href{https://www.archives.gov/research/land/land-records}{land
records},
\href{https://www.archives.gov/research/immigration}{immigration} and
\href{https://www.archives.gov/research/immigration/naturalization}{naturalization}
documents,
\href{https://www.archives.gov/research/genealogy/census}{census} data,
\href{https://www.archives.gov/veterans}{military-service papers}, and
more. While not all government records may be free or digitized, the
National Archives hosts a
\href{https://www.archives.gov/research/alic/reference/ethnic-heritage.html}{page
of links} from other genealogy sites where you can look for information.

\includegraphics{https://static01.graylady3jvrrxbe.onion/images/2020/06/24/technology/personaltech/24TECHTIP_02/24TECHTIP_02-articleLarge.jpg?quality=75\&auto=webp\&disable=upscale}

Some ancestors are harder to trace than others. For families severed by
slavery or overlooked by government, the site has an
\href{https://www.archives.gov/research/genealogy/ethnic-heritage}{Ethnic
Heritage} section with tips for finding
\href{https://www.archives.gov/research/african-americans\#toc-online-resources}{African-American}
ancestors, as well as for those searching for
\href{https://www.archives.gov/research/chinese-americans}{Chinese},
\href{https://www.archives.gov/research/hispanic}{Hispanic/Latino},
\href{https://www.archives.gov/research/japanese-americans}{Japanese} or
\href{https://www.archives.gov/research/native-americans}{Native
American} forebears.

\href{https://www.familysearch.org/en/}{FamilySearch}, run by the Church
of Jesus Christ of Latter-day Saints, requires only a free account to
search its billions of historical records.
\href{https://www.geni.com}{Geni.com} (owned by MyHeritage) has free
basic family-tree building services and a large social community that
encourages members to work together. Immigration museums may also have
free online databases, like the
\href{https://www.libertyellisfoundation.org/passenger}{Statue of
Liberty-Ellis Island Foundation Passenger Search}.

\hypertarget{diving-into-the-archives}{%
\subsection{Diving Into the Archives}\label{diving-into-the-archives}}

Once you have pinned your ancestors to specific places and years, look
for local media from that time. Business dealings, town government
activity, social gatherings and obituaries were often reported in 19th-
and 20th-century papers. But be warned: In addition to sometimes florid
writing, articles from certain eras and areas can be rife with the
unchecked misogyny, racism and xenophobia of the day.

The \href{http://www.loc.gov/rr/news/oltitles.html}{Newspaper Archives,
Indexes and Morgues} section of the \href{https://www.loc.gov}{Library
of Congress} site has links to many digitized publications, including
\href{https://chroniclingamerica.loc.gov/newspapers/?state=\&ethnicity=African+American\&language=}{African-American},
\href{https://www.wdl.org/en/search/?collection=cherokee-phoenix-and-indians-advocate-newspaper\&grouping=15576}{Cherokee}
and
\href{http://contentdm-landing.library.arizona.edu/contentdm/mmap/}{Mexican-American}
newspapers.

\href{https://www.theancestorhunt.com}{The Ancestor Hunt} genealogy site
has a section devoted to
\href{https://www.theancestorhunt.com/newspaper-research-links.html}{finding
historical newspapers online}, and the
\href{https://elephind.com}{Elephind} site lets you search a growing
collection of digitized international newspapers. Some archives are
free, some charge to view the microfilmed images, and search
capabilities vary.

Image

The search function of Newspapers.com can help you find your ancestors
in the archive and on the digitized copy of printed page.Credit...The
New York Times

\href{https://www.newspapers.com}{Newspapers.com} is an archive with
more than 17,000 digitized publications dating from the 1700s. After the
free trial, subscriptions start at about \$8 a month, but you can
\href{https://www.newspapers.com/basics/\#h-search}{search},
\href{https://www.newspapers.com/basics/\#h-clips}{clip},
\href{https://www.newspapers.com/basics/\#h-print}{save and print} the
articles you find.

\hypertarget{finding-further-reading}{%
\subsection{Finding Further Reading}\label{finding-further-reading}}

Libraries and
\href{https://en.wikipedia.org/wiki/List_of_historical_societies\#County_and_local_societies}{historical/genealogical
societies} may also have books and periodicals that recorded the
development of the area and the people who lived there, although you may
have to visit in person to look at the original material if it has not
been scanned. (Some libraries also offer free access to the commercial
genealogy services.)

Image

If your family has been tied to an area for generations, you may be able
to find your forebears in a local history book that has been digitized
by Google or an independent archivist.Credit...The New York Times

As settlements grew, local historians often wrote books that chronicled
that development and its founding families. Many of these volumes are
now digitized in the public domain; search
\href{https://books.google.com}{Google Books} or the
\href{https://archive.org}{Internet Archive} for the town or county in
question.

Your relatives may also appear in the
\href{https://www.familysearch.org/wiki/en/United_States_Vital_Records}{vital
records bureaus} of the states where they lived. The
\href{https://www.rootsweb.com}{RootsWeb} site offers tips on searching
in its
\href{https://wiki.rootsweb.com/wiki/index.php/Red_Book:_American_State,_County,_and_Town_Sources}{Red
Book} collection of American state, county and town resources.

Image

Even if you can't visit the cemetery, a search on the Find a Grave site
might lead you right to a relative's plot.Credit...The New York Times

And finally, if burial was the family tradition, try the
\href{https://www.findagrave.com}{Find a Grave} site, a searchable
database of cemeteries; like Newspapers.com, it's owned by Ancestry. The
site is still growing and often includes published obituaries and photos
of grave sites so you can remotely visit and see where your ancestors
ultimately landed.

Advertisement

\protect\hyperlink{after-bottom}{Continue reading the main story}

\hypertarget{site-index}{%
\subsection{Site Index}\label{site-index}}

\hypertarget{site-information-navigation}{%
\subsection{Site Information
Navigation}\label{site-information-navigation}}

\begin{itemize}
\tightlist
\item
  \href{https://help.nytimes3xbfgragh.onion/hc/en-us/articles/115014792127-Copyright-notice}{©~2020~The
  New York Times Company}
\end{itemize}

\begin{itemize}
\tightlist
\item
  \href{https://www.nytco.com/}{NYTCo}
\item
  \href{https://help.nytimes3xbfgragh.onion/hc/en-us/articles/115015385887-Contact-Us}{Contact
  Us}
\item
  \href{https://www.nytco.com/careers/}{Work with us}
\item
  \href{https://nytmediakit.com/}{Advertise}
\item
  \href{http://www.tbrandstudio.com/}{T Brand Studio}
\item
  \href{https://www.nytimes3xbfgragh.onion/privacy/cookie-policy\#how-do-i-manage-trackers}{Your
  Ad Choices}
\item
  \href{https://www.nytimes3xbfgragh.onion/privacy}{Privacy}
\item
  \href{https://help.nytimes3xbfgragh.onion/hc/en-us/articles/115014893428-Terms-of-service}{Terms
  of Service}
\item
  \href{https://help.nytimes3xbfgragh.onion/hc/en-us/articles/115014893968-Terms-of-sale}{Terms
  of Sale}
\item
  \href{https://spiderbites.nytimes3xbfgragh.onion}{Site Map}
\item
  \href{https://help.nytimes3xbfgragh.onion/hc/en-us}{Help}
\item
  \href{https://www.nytimes3xbfgragh.onion/subscription?campaignId=37WXW}{Subscriptions}
\end{itemize}
