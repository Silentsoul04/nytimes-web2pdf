Sections

SEARCH

\protect\hyperlink{site-content}{Skip to
content}\protect\hyperlink{site-index}{Skip to site index}

\href{https://www.nytimes3xbfgragh.onion/section/politics}{Politics}

\href{https://myaccount.nytimes3xbfgragh.onion/auth/login?response_type=cookie\&client_id=vi}{}

\href{https://www.nytimes3xbfgragh.onion/section/todayspaper}{Today's
Paper}

\href{/section/politics}{Politics}\textbar{}Biden Takes Dominant Lead as
Voters Reject Trump on Virus and Race

\url{https://nyti.ms/2YticaX}

\begin{itemize}
\item
\item
\item
\item
\item
\item
\end{itemize}

\begin{itemize}
\item
  \href{https://www.nytimes3xbfgragh.onion/2020/08/04/us/elections/primary-election-michigan-arizona-kansas.html?action=click\&pgtype=Article\&state=default\&region=TOP_BANNER\&context=storylines_menu}{Election
  Updates}
\item
  \href{https://www.nytimes3xbfgragh.onion/article/biden-vice-president-2020.html?action=click\&pgtype=Article\&state=default\&region=TOP_BANNER\&context=storylines_menu}{Biden's
  V.P. Search}
\item
  \href{https://www.nytimes3xbfgragh.onion/interactive/2020/07/24/us/politics/trump-biden-campaign-donors.html?action=click\&pgtype=Article\&state=default\&region=TOP_BANNER\&context=storylines_menu}{Map
  of Donations}
\item
  \href{https://www.nytimes3xbfgragh.onion/interactive/2020/us/elections/delegate-count-primary-results.html?action=click\&pgtype=Article\&state=default\&region=TOP_BANNER\&context=storylines_menu}{Delegate
  Count}
\item
  \href{https://www.nytimes3xbfgragh.onion/interactive/2019/us/politics/2020-presidential-candidates.html?action=click\&pgtype=Article\&state=default\&region=TOP_BANNER\&context=storylines_menu}{The
  Candidates}
\item
  \href{https://www.nytimes3xbfgragh.onion/newsletters/politics?action=click\&pgtype=Article\&state=default\&region=TOP_BANNER\&context=storylines_menu}{Politics
  Newsletter}
\end{itemize}

Advertisement

\protect\hyperlink{after-top}{Continue reading the main story}

Supported by

\protect\hyperlink{after-sponsor}{Continue reading the main story}

\hypertarget{biden-takes-dominant-lead-as-voters-reject-trump-on-virus-and-race}{%
\section{Biden Takes Dominant Lead as Voters Reject Trump on Virus and
Race}\label{biden-takes-dominant-lead-as-voters-reject-trump-on-virus-and-race}}

A New York Times/Siena College poll finds that Joseph R. Biden Jr. is
ahead of the president by 14 points, leading among women and nonwhite
voters and cutting into his support with white voters.

\href{https://www.nytimes3xbfgragh.onion/by/alexander-burns}{\includegraphics{https://static01.graylady3jvrrxbe.onion/images/2018/09/25/multimedia/author-alexander-burns/author-alexander-burns-thumbLarge-v2.png}}\href{https://www.nytimes3xbfgragh.onion/by/jonathan-martin}{\includegraphics{https://static01.graylady3jvrrxbe.onion/images/2018/11/06/multimedia/author-jonathan-martin/author-jonathan-martin-thumbLarge.png}}\href{https://www.nytimes3xbfgragh.onion/by/matt-stevens}{\includegraphics{https://static01.graylady3jvrrxbe.onion/images/2019/04/03/multimedia/author-matt-stevens/author-matt-stevens-thumbLarge.png}}

By
\href{https://www.nytimes3xbfgragh.onion/by/alexander-burns}{Alexander
Burns},
\href{https://www.nytimes3xbfgragh.onion/by/jonathan-martin}{Jonathan
Martin} and
\href{https://www.nytimes3xbfgragh.onion/by/matt-stevens}{Matt Stevens}

\begin{itemize}
\item
  Published June 24, 2020Updated July 31, 2020
\item
  \begin{itemize}
  \item
  \item
  \item
  \item
  \item
  \item
  \end{itemize}
\end{itemize}

NYT Upshot/Siena College poll

of registered voters

50\%

Biden

36\%

Trump

14\%

Other

NYT Upshot/

Siena College poll

of registered voters

50\%

Biden

36\%

Trump

14\%

Other

``Other'' includes those who would vote for another candidate, would not
vote or did not know.·Based on a New York Times/Siena College poll of
1,337 registered voters from June 17 to June 22.

Joseph R.
\href{https://www.nytimes3xbfgragh.onion/2020/07/17/us/trump-biden-2020-election.html}{Biden}
Jr. has taken a commanding lead over President
\href{https://www.nytimes3xbfgragh.onion/2020/07/17/us/trump-biden-2020-election.html}{Trump}
in the 2020 race, building a wide advantage among women and nonwhite
voters and making deep inroads with some traditionally
Republican-leaning groups that have shifted away from Mr. Trump
following his ineffective
\href{https://www.nytimes3xbfgragh.onion/2020/04/11/us/politics/coronavirus-trump-response.html}{response
to the coronavirus pandemic}, according to a new national
\href{https://www.nytimes3xbfgragh.onion/2020/07/20/upshot/biden-trump-poll.html}{poll}
of registered voters by The New York Times and Siena College.

Mr.
\href{https://www.nytimes3xbfgragh.onion/2020/07/03/upshot/joe-biden-voters-coronavirus.html}{Biden}
is currently ahead of Mr. Trump by 14 percentage points, garnering 50
percent of the vote compared with 36 percent for Mr. Trump. That is
among the most dismal showings of Mr. Trump's presidency, and a sign
that he is the clear underdog right now in his fight for a second term.

Mr. Trump has been an unpopular president for virtually his entire time
in office. He has made few efforts since his election in 2016 to broaden
his support beyond the right-wing base that vaulted him into office with
only
\href{https://www.nytimes3xbfgragh.onion/elections/2016/results/president}{46
percent} of the popular vote and a modest victory in the Electoral
College.

But among a striking cross-section of voters, the distaste for Mr. Trump
has deepened as his administration failed to stop a deadly disease that
crippled the economy and then as he responded to a wave of
racial-justice protests with angry bluster and militaristic threats. The
dominant picture that emerges from the poll is of a country ready to
reject a president whom a strong majority of voters regard as failing
the greatest tests confronting his administration.

Mr. Biden leads Mr. Trump by enormous margins with black and Hispanic
voters, and women and young people appear on track to choose Mr. Biden
by an even wider margin than they favored Hillary Clinton over Mr. Trump
in 2016. But the former vice president has also drawn even with Mr.
Trump among male voters, whites and people in middle age and older ---
groups that have typically been the backbones of Republican electoral
success, including Mr. Trump's in 2016.

\hypertarget{if-the-2020-presidential-election-were-held-today-whom-would-you-vote-for}{%
\subsubsection{If the 2020 presidential election were held today, whom
would you vote
for?}\label{if-the-2020-presidential-election-were-held-today-whom-would-you-vote-for}}

Trump ahead

Biden ahead

All reg. voters

+14 pct. pts.

(n=1,337)

Female

+22

GENDER

(725)

Male

+3

(612)

18 to 34

+34

AGE

(245)

35 to 49

+23

(287)

50 to 64

+1

(356)

65 and older

+2

(400)

Black

+74

RACE AND

EDUCATION

(150)

Hispanic

+39

(145)

+28

White, college

(436)

+1

White

(870)

+19

White, no coll.

(427)

Democrat

+85

PARTY

IDENTIFICATION

(466)

Independent

+21

(401)

Republican

+85

(341)

+83

Very liberal

IDEOLOGY

(177)

Somewhat liberal

+69

(260)

Moderate

+33

(332)

Somewhat conservative

+32

(281)

Very conservative

+73

(195)

All reg. voters

Biden +14

(n=1,337)

GENDER

Female

Biden +22

(725)

Male

Biden +3

(612)

AGE

18 to 34

Biden +34

(245)

35 to 49

Biden +23

(287)

50 to 64

Trump +1

(356)

65 and older

Biden +2

(400)

RACE AND EDUCATION

Black

Biden +74

(150)

Hispanic

Biden +39

(145)

White, college

Biden +28

(436)

White

Trump +1

(870)

White, no coll.

Trump +19

(427)

PARTY IDENTIFICATION

Democrat

Biden +85

(466)

Independent

Biden +21

(401)

Republican

Trump +85

(341)

IDEOLOGY

Very liberal

Biden +83

(177)

Somewhat liberal

Biden +69

(260)

Moderate

Biden +33

(332)

Somewhat conservative

Trump +32

(281)

Very conservative

Trump +73

(195)

Sample sizes may not add to the total because some demographic
characteristics of respondents are unknown.·Based on a New York
Times/Siena College poll of 1,337 registered voters from June 17 to June
22.

Arlene Myles, 75, of Denver, said she had been a Republican for nearly
six decades before switching her registration to independent earlier
this year during Mr. Trump's impeachment trial. Ms. Myles said that when
Mr. Trump was first elected, she had resolved to ``give him a chance,''
but had since concluded that he and his party were irredeemable.

``I was one of those people who stuck by Nixon until he was waving
goodbye,'' Ms. Myles said. ``I thought I was a good Republican and
thought they had my values, but they have gone down the tubes these last
few years.''

Ms. Myles said she planned to vote for Mr. Biden, expressing only one
misgiving: ``I wish he was younger,'' she said.

Most stark may be Mr. Biden's towering advantage among white women with
college degrees, who support him over Mr. Trump by 39 percentage points.
In 2016, exit polls found that group preferred Mrs. Clinton to Mr. Trump
by just 7 percentage points. The poll also found that Mr. Biden has
narrowed Mr. Trump's advantage with less-educated white voters.

\hypertarget{latest-updates-2020-election}{%
\section{\texorpdfstring{\href{https://www.nytimes3xbfgragh.onion/2020/08/04/us/elections/primary-election-michigan-arizona-kansas.html?action=click\&pgtype=Article\&state=default\&region=MAIN_CONTENT_1\&context=storylines_live_updates}{Latest
Updates: 2020
Election}}{Latest Updates: 2020 Election}}\label{latest-updates-2020-election}}

Updated 2020-08-04T19:43:49.779Z

\begin{itemize}
\tightlist
\item
  \href{https://www.nytimes3xbfgragh.onion/2020/08/04/us/elections/primary-election-michigan-arizona-kansas.html?action=click\&pgtype=Article\&state=default\&region=MAIN_CONTENT_1\&context=storylines_live_updates\#link-3924dd44}{Two
  G.O.P. Senate primaries offer --- what else? --- a test of loyalty to
  Trump.}
\item
  \href{https://www.nytimes3xbfgragh.onion/2020/08/04/us/elections/primary-election-michigan-arizona-kansas.html?action=click\&pgtype=Article\&state=default\&region=MAIN_CONTENT_1\&context=storylines_live_updates\#link-32b39e33}{President
  Trump is suddenly a big supporter of mail-in voting --- in Florida.}
\item
  \href{https://www.nytimes3xbfgragh.onion/2020/08/04/us/elections/primary-election-michigan-arizona-kansas.html?action=click\&pgtype=Article\&state=default\&region=MAIN_CONTENT_1\&context=storylines_live_updates\#link-6d019753}{Election
  experts warn Congress about widespread disenfranchisement of voters of
  color in November.}
\end{itemize}

\href{https://www.nytimes3xbfgragh.onion/2020/08/04/us/elections/primary-election-michigan-arizona-kansas.html?action=click\&pgtype=Article\&state=default\&region=MAIN_CONTENT_1\&context=storylines_live_updates}{See
more updates}

The exodus of white voters from the G.O.P. has been especially
pronounced among younger voters, an ominous trend for a party that was
already heavily reliant on older Americans.

Fifty-two percent of whites under 45 said they supported Mr. Biden while
only 30 percent said they supported Mr. Trump. And their opposition is
intense: More than twice as many younger whites viewed the president
very unfavorably than very favorably.

Tom Diamond, 31, a Republican in Fort Worth, Texas, said he planned to
vote for Mr. Trump but would do so with real misgivings. He called the
president a ``poor leader'' who had mishandled the pandemic and said Mr.
Biden seemed ``like a guy you can trust.'' But Mr. Trump held views
closer to his own on the economy, health care and abortion.

``Part of you just feels icky voting for him,'' Mr. Diamond said. ``But
definitely from a policy perspective, that's where my vote's going to
go.''

Some unease toward Mr. Trump stems from voters' racial attitudes.
According to the poll, white voters under 45 are overwhelmingly
supportive of the Black Lives Matter movement, while older whites are
more tepid in their views toward racial justice activism. And nearly 70
percent of whites under 45 said they believed
\href{https://www.nytimes3xbfgragh.onion/2020/05/31/us/george-floyd-investigation.html}{the
killing of George Floyd} was part of a broader pattern of excessive
police violence toward African-Americans rather than an isolated
incident.

What's striking, though, is that even among white seniors, one of Mr.
Trump's strongest constituencies, he has damaged himself with his
conduct. About two-fifths of whites over 65 said they disapproved of Mr.
Trump's handling of both the coronavirus and race relations.

Mr. Trump retains a few points of strength in the poll that could offer
him a way to regain a footing in the race, and the feeble condition of
his candidacy right now may well represent his low point in a campaign
with four and a half months still to go. In 2016, Mr. Trump often
trailed Mrs. Clinton in national polls by slimmer margins, and
ultimately overcame her lead in the popular vote with razor-thin
victories in key swing states.

His
\href{https://www.nytimes3xbfgragh.onion/2020/06/28/us/politics/trump-approval-older-voters-coronavirus.html}{approval
rating} is still narrowly positive on the issue of the economy, with 50
percent of voters giving him favorable marks compared with 45 percent
saying the opposite. Should the fall campaign become a referendum on
which candidate is better equipped to restore prosperity after the
pandemic has subsided, that could give Mr. Trump a new opening to press
his case.

The president is also still ahead of Mr. Biden among white voters
without college degrees, who hold disproportionate influence in
presidential elections because of how central the Midwest is to
capturing 270 electoral votes.

Yet if Mr. Trump still has a significant measure of credibility with
voters on the economy, he lacks any apparent political strength on the
most urgent issues of the moment: the pandemic and the national
reckoning on policing and race.

Nearly three-fifths of voters disapprove of Mr. Trump's handling of the
coronavirus pandemic, including majorities of white voters and men.
Self-described moderate voters disapproved of Mr. Trump on the
coronavirus by a margin of more than two to one.

Most of the country is also rejecting Mr. Trump's call to reopen the
economy as quickly as possible, even at the cost of exposing people to
greater health risks. By a 21-point margin, voters said the federal
government should prioritize containing the coronavirus, even if it
hurts the economy, a view that aligns them with Mr. Biden.

Just a third of voters said the government should focus on restarting
the economy even if that entails greater public-health risks.

That debate could become the central focus of the campaign in the coming
weeks, as coronavirus outbreaks grow rapidly in a number of
Republican-led states that have resisted the strict lockdown measures
imposed in the spring by Democratic states like New York and California.

The public also does not share Mr. Trump's resistance to mask wearing.
The president has declined to don a mask in nearly all public
appearances, even as top health officials in his administration have
urged Americans to do so as a precaution against spreading the
coronavirus. In the poll, 54 percent of people said they always wear a
mask when they expect to be in proximity to other people, while another
22 percent said they usually wear a mask.

Just 22 percent said they rarely or never wear a mask.

Mr. Trump's job approval on race relations was just as dismal. Sixty-one
percent of voters said they disapproved of Mr. Trump's handling of race,
versus 33 percent who said they approved. By a similar margin, voters
said they disapproved of his response to the protests after the death of
Mr. Floyd.

Mr. Trump has sought several times in the last month to use
demonstrations against the police as a political wedge issue, forcing
Democrats to align themselves squarely either with law-enforcement
agencies or with the most strident anti-police demonstrators.

The poll suggested most voters were rejecting that binary choice, as
well as Mr. Trump's harsh characterization of protesters: Large
majorities said they had a positive overall assessment of both the Black
Lives Matter movement and the police.

\hypertarget{more-voters-feel-strongly-about-mr-trump-than-they-do-about-mr-biden}{%
\subsubsection{More voters feel strongly about Mr. Trump than they do
about Mr.
Biden}\label{more-voters-feel-strongly-about-mr-trump-than-they-do-about-mr-biden}}

Voter impressions of ...

Trump

Biden

Very

favorable

Very

unfavorable

Very

favorable

Very

unfavorable

27\%

50\%

26\%

27\%

ALL REG. VOTERS

(n=1,337)

NONWHITE

Age 18 to 29

11\%

68\%

21\%

15\%

(88)

Age 30 to 44

15\%

61\%

29\%

14\%

(100)

Age 45 to 64

22\%

62\%

45\%

19\%

(150)

Age 65 and older

13\%

70\%

60\%

16\%

(85)

WHITE

Age 18 to 29

23\%

46\%

4\%

28\%

(89)

Age 30 to 44

17\%

58\%

20\%

22\%

(156)

Age 45 to 64

36\%

39\%

22\%

35\%

(305)

Age 65 and older

42\%

38\%

27\%

38\%

(320)

Voter

impressions of:

Trump

Very

favorable

Very

unfavorable

n=

27\%

50\%

ALL REG. VOTERS

(1,337)

NON-WHITE

Age 18 to 29

11\%

68\%

(88)

Age 30 to 44

15\%

61\%

(100)

Age 45 to 64

22\%

62\%

(150)

Age 65 and older

13\%

70\%

(85)

WHITE

Age 18 to 29

23\%

46\%

(89)

Age 30 to 44

17\%

58\%

(156)

Age 45 to 64

36\%

39\%

(305)

Age 65 and older

42\%

38\%

(320)

Voter

impressions of:

Biden

Very

favorable

Very

unfavorable

n=

26\%

27\%

ALL REG. VOTERS

(1,337)

NONWHITE

21\%

15\%

Age 18 to 29

(88)

29\%

14\%

Age 30 to 44

(100)

45\%

19\%

Age 45 to 64

(150)

60\%

16\%

Age 65 and older

(85)

WHITE

4\%

28\%

Age 18 to 29

(89)

20\%

22\%

Age 30 to 44

(156)

22\%

35\%

Age 45 to 64

(305)

27\%

38\%

Age 65 and older

(320)

Sample sizes may not add to the total because some demographic
characteristics of respondents are unknown.·Based on a New York
Times/Siena College poll of 1,337 registered voters from June 17 to June
22.

The picture of Mr. Biden that emerges from the poll is one of a broadly
acceptable candidate who inspires relatively few strong feelings in
either direction. He is seen favorably by about half of voters and
unfavorably by 42 percent. Only a quarter said they saw him very
favorably, equaling the share that sees him in very negative terms.

Mr. Trump, by contrast, is seen very favorably by 27 percent of voters
and very unfavorably by 50 percent.

Harry Hoyt, 72, of York County in Southern Maine, said he has sometimes
voted for Republican presidential candidates in the past and cast a
grudging vote for Mrs. Clinton in 2016. He felt better this time about
his plan to vote for Mr. Biden.

``Biden would be a better candidate than Trump, simply because he's a
nice person,'' Mr. Hoyt said. ``One of the most important things to me
is the character of the man in charge of our country.''

Significantly, one group that saw Mr. Biden as far more than just
acceptable was black voters. Fifty-six percent of black respondents in
the poll said they saw Mr. Biden very favorably, a far more enthusiastic
judgment than from any other constituency.

The limited passion for Mr. Biden among other Democratic constituencies
does not appear to be affecting his position against Mr. Trump. Though
only 13 percent of people under 30 said they had a very favorable
opinion of the former vice president, that group is backing Mr. Biden
over Mr. Trump by 34 percentage points.

Nicholas Angelos, a 20-year-old voter in Bloomington, Ind., who said he
supported Senator Bernie Sanders in the Democratic primaries, said he
would vote for Mr. Biden as the ``lesser of two evils.'' He said he
believed the former vice president would ``try his best,'' in contrast
to Mr. Trump, whom he described as ``an autocrat'' and ``anti-science.''

``We all have to compromise,'' said Mr. Angelos, who described himself
as very liberal. He added of Mr. Biden, ``I don't think he's anything
special.''

For the moment, voters also appear unpersuaded by one of the primary
attack lines Mr. Trump and his party have used against Mr. Biden: the
claim that, at age 77, he is simply too old for the presidency. Mr.
Trump, 74, has mocked Mr. Biden's mental acuity frequently over the last
few months and his campaign has run television advertisements that cast
Mr. Biden as absent-minded and inarticulate.

But whatever reservations voters may have about Mr. Biden's age, three
in five said they disagreed with the claim that he was too old to be an
effective president. The percentage of voters who agreed, 36 percent,
exactly matched Mr. Trump's existing support in the presidential race.

Lindsay Clark, 37, who lives in the suburbs of Salt Lake City, was among
the voters who said she would probably vote for Mr. Trump because she
was unsure Mr. Biden was ``physically and mentally up to the task'' of
being president. But Ms. Clark expressed little admiration for Mr.
Trump, whom she called unpresidential.

Ms. Clark, who voted for a third-party candidate in 2016, said she was
hard-pressed to name something she really liked about Mr. Trump,
eventually settling on the idea that he expressed himself bluntly.

``I was just trying to think if I could think of something off the top
of my head that I was like, `Yes, I loved when you did that!''' she said
of Mr. Trump. ``And I kind of just can't.''

\includegraphics{https://static01.graylady3jvrrxbe.onion/images/2017/01/29/podcasts/the-daily-album-art/the-daily-album-art-articleInline-v2.jpg?quality=75\&auto=webp\&disable=upscale}

\hypertarget{listen-to-the-daily-the-voters-trump-is-losing}{%
\subsubsection{Listen to `The Daily': The Voters Trump Is
Losing}\label{listen-to-the-daily-the-voters-trump-is-losing}}

In six states crucial to the president's election hopes, our polling
found a potentially decisive shift in opinion.

transcript

Back to The Daily

bars

0:00/25:37

-25:37

transcript

\hypertarget{listen-to-the-daily-the-voters-trump-is-losing-1}{%
\subsection{Listen to `The Daily': The Voters Trump Is
Losing}\label{listen-to-the-daily-the-voters-trump-is-losing-1}}

\hypertarget{hosted-by-michael-barbaro-produced-by-alexandra-leigh-young-sydney-harper-and-eric-krupke-with-help-from-robert-jimison-and-edited-by-mj-davis-lin-and-lisa-chow}{%
\subsubsection{Hosted by Michael Barbaro; produced by Alexandra Leigh
Young, Sydney Harper and Eric Krupke; with help from Robert Jimison; and
edited by M.J. Davis Lin and Lisa
Chow}\label{hosted-by-michael-barbaro-produced-by-alexandra-leigh-young-sydney-harper-and-eric-krupke-with-help-from-robert-jimison-and-edited-by-mj-davis-lin-and-lisa-chow}}

\hypertarget{in-six-states-crucial-to-the-presidents-election-hopes-our-polling-found-a-potentially-decisive-shift-in-opinion}{%
\paragraph{In six states crucial to the president's election hopes, our
polling found a potentially decisive shift in
opinion.}\label{in-six-states-crucial-to-the-presidents-election-hopes-our-polling-found-a-potentially-decisive-shift-in-opinion}}

\begin{itemize}
\item
  michael barbaro\\
  From The New York Times, I'm Michael Barbaro. This is ``The Daily.''

  Today: This fall's presidential race is likely to be decided by a
  handful of battleground states won by President Trump in 2016. Nate
  Cohn on what a major new poll from The New York Times found about how
  voters in those states view the president and his Democratic rival.

  It's Thursday, June 25.

  Nate, the last time we talked about polling in the presidential race,
  it was still the Democratic primary. It was back in November. And the
  polling that you all did showed that no matter which Democrat Donald
  Trump would face, he was doing OK.
\item
  nate cohn\\
  Yeah. The polls showed that Donald Trump was pretty competitive. He
  was in a close race against all of his major Democratic rivals. He
  wasn't always ahead. He was losing to Joe Biden, for instance. But he
  was in a pretty good and highly competitive spot, even at a moment
  where he was facing imminent prospect of impeachment in the House of
  Representatives. And so after all of that, for him still to be so
  close, it sure seemed like he was in a pretty decent position, all
  considered.
\item
  michael barbaro\\
  So what was your thinking going into this next big poll that you all
  just finished?
\item
  nate cohn\\
  Well, I did not think that Donald Trump would be doing quite as well
  as he had done in October. And there have been a lot of polls over the
  last couple of months indicating that Joe Biden has had a gradually
  building lead, both nationwide and in the battleground states. But
  given that we had such good results for the president in the past, I
  thought there was a pretty distinct possibility that we would show a
  race that was, you know, even if not extremely close, still
  competitive.
\item
  michael barbaro\\
  And what happened once this polling got underway?
\item
  nate cohn\\
  Well, we get the results back every morning from the last night of
  interviews. And from the start, it was pretty clear that this was a
  very different set of polling data.

  The first morning's numbers were really bad for Trump. And you know,
  it's something you try not to pay any attention to. It's just one day
  of interviews, and the numbers change a lot. But you know, the next
  morning, and the next morning after that, it was bad again. And in the
  end, we polled for 14 days. And for all 14, the numbers were bad for
  the president. And they never got better.
\item
  michael barbaro\\
  And just how bad?
\item
  nate cohn\\
  Really bad. We did seven different polls. We have a national survey
  that showed Joe Biden leading by 14 points, 50 to 36 percent.
\item
  michael barbaro\\
  Wow.
\item
  nate cohn\\
  We have six battleground state polls. These are the six states that
  Donald Trump won in 2016, but that were most closely fought last time
  --- Pennsylvania, Michigan, Wisconsin, Florida, Arizona and North
  Carolina. And here again, Joe Biden had a considerable lead. He was up
  9 percentage points across those six states, including a lead of at
  least six points in every one of them.
\item
  michael barbaro\\
  Wow. So that is legitimately, seriously bad polling data for the
  incumbent President, Donald Trump?
\item
  nate cohn\\
  Yeah. I mean, if you look back historically, it is really hard to find
  an example of an incumbent president sitting in such a bad position
  heading into re-election. It's worse than Jimmy Carter in 1980 at this
  stage, or even in the final polls, for instance.
\item
  michael barbaro\\
  Huh.
\item
  nate cohn\\
  There's just not much for the president to hang his hat on here.
\item
  michael barbaro\\
  So let's talk about what the battleground state polls show,
  specifically about why voters seem to be souring on President Trump.
  What's the story that the data from these battleground state polls
  tell you about why that is?
\item
  nate cohn\\
  The polls tell a really simple story. They say that voters across the
  battleground states have concluded that the president has failed to
  meet the most important crises of our political moment. They think
  he's failed on the coronavirus. They think he's failed on race
  relations. And they think he's failed in the protests, and so on. And
  as a result, there has been a rebellion among white voters in the
  battleground states, the very voters that four years ago were
  responsible for the president's persistent strength in these states.
\item
  michael barbaro\\
  So before we get to the revolt of these white voters, which seems very
  important, let's talk about these issues that voters do not think the
  president has properly managed and maybe go through a few of them, one
  by one.
\item
  nate cohn\\
  I think it's worth just breaking them down into two groups. One is the
  coronavirus. Voters disapprove of the way he's handled it by a wide
  margin. 56 percent of voters in the battleground states say they
  disapprove, includes a significant number of people who voted for him
  in 2016. And it's not just that they take issue with the president's
  effectiveness. There's an underlying disagreement between the
  president and the electorate about priorities.
\item
  michael barbaro\\
  What do you mean?
\item
  nate cohn\\
  So we asked voters whether they thought the federal government's
  priority should be to limit the spread of the coronavirus, even if it
  hurts the economy. Or if they think the federal government's priority
  should be to restart the economy, even if it increases the risk to
  public health. And voters in the battleground states said, by a
  20-plus point margin, that they thought the priority ought to be to
  limit the spread of the coronavirus. And incredibly, that even
  includes the people who have lost their jobs as a result of the
  coronavirus over the last few months.
\item
  michael barbaro\\
  I want to pause on this. Because this seems really fascinating. By 20
  percent, voters favor conquering the virus over reopening the American
  economy. Because that is the complete opposite of the message from the
  president.
\item
  nate cohn\\
  It's the complete opposite. And it really goes against so much of what
  we usually assume about American politics, right? It's the economy,
  stupid. But here we have a rare issue where voters very explicitly are
  prioritizing something else. I can't think of another comparison for
  it except war.
\item
  michael barbaro\\
  Hm.
\item
  nate cohn\\
  You know? In World War II, it's not like if you had asked voters, what
  should be the priority, you know, helping the economy or beating the
  Japanese, that they would be like, oh, it's the economy, stupid.
\item
  michael barbaro\\
  Right. You're saying with lives on the line, voters are telling us in
  this poll that the economy takes something of a backseat.
\item
  nate cohn\\
  That's right.
\item
  michael barbaro\\
  What seems interesting is that the president is banking on an economic
  recovery, or goodwill over his management of the economy prior to the
  pandemic, to win re-election. But it sounds like from what you're
  saying, that wouldn't necessarily give him a boost, because voters are
  not prioritizing the economy the way they usually do.
\item
  nate cohn\\
  Yeah. I mean, he's right to think that's his advantage. I mean,
  astonishingly, given the overall economic numbers, voters in the
  battleground states say they approve of the president's handling of
  the economy, by a 15-point margin. That's a ---
\item
  michael barbaro\\
  Wow.
\item
  nate cohn\\
  --- what, 30 points different than his overall approval rating? And
  the reason for that disconnect is just fundamental and simple. The
  coronavirus is more important. And they may appreciate what he's done
  on the economy. But in this case, emphasizing the economy isn't what
  they're looking for.
\item
  michael barbaro\\
  So what about the second big issue that you said voters disapprove of
  the president's management of, which is race and the protests over
  race and policing?
\item
  nate cohn\\
  Right. Just a whole spectrum of different issues relating to race and
  criminal justice and the protests, the president's ratings are even
  worse on those issues than they are on the coronavirus. And here
  again, the president has this fundamental disconnect with the
  electorate, where his priorities aren't the same as theirs. I mean, we
  asked whether they would rather have a candidate who says that we need
  to be tough on protests that go too far, or whether they would rather
  have a candidate who says we need to focus on the cause of protests,
  even when they go too far. And voters said, by a 40-point margin, that
  they would rather have the candidate who focuses on the cause of the
  protests, even when they're going too far. And it's also including a
  significant number of people who backed the president.
\item
  michael barbaro\\
  So the gulf between how the president is talking about these protests
  and how voters across the battleground states are thinking about these
  protests is enormous.
\item
  nate cohn\\
  Yeah. And I wouldn't have guessed that, personally.
\item
  michael barbaro\\
  So Nate, how does this disapproval of how the president's handling
  these major crises, how does that explain this concept you touched on
  earlier of white voters revolting against the President?
\item
  nate cohn\\
  Well, a pretty significant number of white voters in the battleground
  states do not side with the president, either his handling of the
  coronavirus or on these racial issues. And these are now the issues
  that are most important in the minds of voters in the battleground
  states. And so while maybe in 2016, a lot of these voters were focused
  on who would do the best on trade or immigration, now what's on their
  mind is who is the best job handling protests and coronavirus. They're
  not so sure they want to vote for the President anymore. In our poll,
  nearly 15 percent of the people who say they voted for President Trump
  in 2016 aren't willing to say they support him against Joe Biden. And
  7 percent of those voters say there's almost no chance they'll vote
  for him again.
\item
  michael barbaro\\
  So 7 percent of his base plans on flipping, and 15 percent is very
  open to flipping, because of the way he has handled all of these
  complex questions over the past few months or year?
\item
  nate cohn\\
  Yeah. And I should note they may not necessarily flip. They could vote
  third party. They could stay home. They have other options here. But a
  meaningful number of them at the moment say they would back Joe Biden
  for president.
\item
  michael barbaro\\
  Wow.
\item
  nate cohn\\
  So at the moment, the president's coalition, this core base of support
  among white voters that got him to victory in 2016, just is not there
  for him anymore. It has suffered serious defections, mainly at its
  periphery, you know, not the people wearing MAGA hats. But the sort of
  people who voted for Obama in 2012, backed Donald Trump in 2016
  because they liked what he had to say about illegal immigration and
  trade, are not putting up with him right now.
\item
  michael barbaro\\
  And who is this voter, this white voter who is now turning on the
  president four years later?
\item
  nate cohn\\
  It's a pretty broad group of white voters. It includes
  college-educated white voters. It includes white voters without a
  college degree. It includes young voters. Young white voters now back
  Joe Biden by 20 points.
\item
  michael barbaro\\
  Wow.
\item
  nate cohn\\
  It includes older white voters. At the moment, the president is losing
  among seniors, who were the bedrock of his support in 2016.
\item
  michael barbaro\\
  Wow. You know, Nate, I'm curious if there's a prototypical swing state
  region that you think of as being illustrative of everything that
  you're describing here, this alienation, this disapproval of the way
  the president is handling everything since we last polled these
  battleground states.
\item
  nate cohn\\
  Well, I could indulge you on that question. I could tell you about how
  in the Green Bay region of Wisconsin or in Northeast Pennsylvania or
  something that Joe Biden is now ahead, and Donald Trump won there big
  in 2016. But you know, frankly, it's true everywhere. There are no
  exceptions here. There's no place where the president's holding up
  fairly well and I can be like, that's the place where it's
  particularly bad. There's no region that epitomizes this more than any
  other. I could choose anywhere and tell you that the president appears
  to be losing ground among white voters. It's true in the battleground
  states. It's true nationally. It's true everywhere.
\item
  michael barbaro\\
  We'll be right back.

  OK. Nate, let's talk about what these polls, both the national poll,
  but especially the six battleground state polls, have told us about
  Joe Biden and why he seems to be leading so handsomely over Donald
  Trump.
\item
  nate cohn\\
  You know, my honest answer is that I think our answers on Joe Biden
  are kind of boring.
\item
  michael barbaro\\
  Boring how?
\item
  nate cohn\\
  You know, the voters, they like Joe Biden. He's got a 50 percent
  favorability rating. It's fine. They don't seem to have a very strong
  opinion of him either way. They do think he would do a better job than
  Trump on almost every issue, except the economy and China. But there's
  not a groundswelling of support for him. And there's not particularly
  deep opposition to him either. Instead, 55 percent of voters say there
  is at least some chance they would vote for him. So that to me
  indicates that a pretty broad swath of the electorate is at least
  considering the guy at this stage.
\item
  michael barbaro\\
  Hm. And what you're describing is an alternative, rather than a
  charismatic figure that voters are looking to with great ardor.
\item
  nate cohn\\
  I think that's right. A lot of the divides that you might think about
  in American politics today, like between young and old, and so on.
  They don't really even exist on Joe Biden. Everyone just kind of has a
  modestly favorable view of him. And there are many circumstances in
  which maybe that's not the exact candidate you'd want to run for
  president. But it may be exactly the right candidate at a time when a
  clear majority of the electorate has resolved that it does not want to
  reelect the current president.
\item
  michael barbaro\\
  Nate, you're painting a scenario in which Joe Biden seems to be
  thriving as a reflection of President Trump's weakness. And that's a
  familiar concept in presidential campaigns. But I went through the
  battleground state poll that you sent me about 24 hours ago. And I was
  really struck by the number of issues on which respondents to the poll
  said they thought that Joe Biden would do a better job than President
  Trump. It was immigration. It was protests. It was the pandemic. And
  so how do we know whether that is a reflection of President Trump, or
  a true sense that people believe Joe Biden is inherently the best
  person to do those things?
\item
  nate cohn\\
  You know, it's a great question. Unfortunately, the way that we've
  asked this question to voters, I don't think we can disentangle
  whether they think Joe Biden would be good versus believing that
  Donald Trump is bad. We're just asking voters whether they think Joe
  Biden would do a better job than Donald Trump. And so I don't think we
  can pull out the varying effects of Joe Biden and Donald Trump on
  that. I do think, though, that one thing that stands out to me is how
  much the results of those questions matches up with Trump's approval
  rating on those issues, which at least to me implies that it's more
  about the president than it is about Joe Biden.
\item
  michael barbaro\\
  Hm. In other words, disapproval of the president is kind of mirrored
  in the data of approval or the belief that Joe Biden would do
  something better?
\item
  nate cohn\\
  That's right.
\item
  michael barbaro\\
  This data clearly suggests there's not a tremendous amount of passion
  around Joe Biden. And yet, he's developed this very significant lead
  over President Trump. And so I wonder whether that means that the
  electability case for Joe Biden, which was debated endlessly
  throughout the primary, has kind of proven to be exactly what Joe
  Biden and the people around him said it would be now that there's a
  two person race, Biden versus Trump.
\item
  nate cohn\\
  Well, I don't think we can run the counterfactual and see whether
  Elizabeth Warren would be leading today if she had been the Democratic
  nominee or Bernie Sanders. What I think we can say is that the case
  for Joe Biden's electability is playing out here. The case for Joe
  Biden's electability was always that a sufficient number of voters do
  not want to re-elect the president, so choose someone who maximizes
  the appeal of the Democratic candidate with the broadest number of
  voters. I don't know whether that means that a different candidate
  would be doing better or worse than Joe Biden. But what we do know is
  that Joe Biden is up nine points across battleground states that voted
  for Trump last time. He's up 14 points nationwide. And he would have a
  distinct chance of winning by the widest margin of any candidate in my
  lifetime.
\item
  michael barbaro\\
  Well, Nate, that leads me to a very important and delicate question
  that involves you and me. Because four years ago, we were having a
  conversation about polling. And Donald Trump was down in those polls.
  I was the host of a different show called ``The Run Up.'' You were the
  first guest on the first episode. And the title of that episode was,
  ``Could Hillary Clinton Win in a Landslide?'' So I think you know
  where this question is now headed.
\item
  nate cohn\\
  How could I not?
\item
  michael barbaro\\
  So what is different about this lead, if anything, that makes you and
  should make us trust it?
\item
  nate cohn\\
  Well, I think we should all approach this with a lot of humility. I
  mean, polling is tough the day before the election. It's really tough
  five months before the election. The national environment has changed
  a lot in the last five months. It can change a lot in the next five
  months. That said, I think that this lead is different from Hillary
  Clinton's lead. It doesn't mean that Donald Trump can't win. But it is
  different. It's different in two ways. One, it's a wider lead. Joe
  Biden is up by more. If the polls were just as wrong as they were in
  2016, and the election were held tomorrow, Joe Biden would still win.
  I mean, the polls could be more wrong. There's no law of polling that
  says that 2016 is the worst case scenario or something. But this is a
  bigger advantage than Hillary Clinton had down the final stretch. It's
  a more persistent advantage. And it's wider than even her peaks. The
  second difference is a methodological one, which is that we are much
  more focused on the battleground states this time than we were in
  2016. In 2016, we had really good national polling. And we did not
  have very much good polling in the battleground states. We have
  resolved to flip that around, to try and deal with the things that
  went wrong four years ago. None of that ensures that our results will
  be perfect if the election were held tomorrow. But it is a reason why
  it is less likely that these polls would be fundamentally wrong in the
  same way that so many state polls were fundamentally wrong four years
  ago.
\item
  michael barbaro\\
  A key caveat, however, seems to be that the lead that Joe Biden has
  developed over these past few months has occurred when he has been
  largely invisible. Right? I mean, this has been a very unusual
  campaign in the sense that the President is as visible as he's ever
  been, every day --- at the White House, on Twitter and now campaign
  rallies. Whereas Joe Biden has been largely quarantined and kind of
  offstage. Inevitably, that will change. And I wonder how much that
  could begin to alter some of the dynamics of these polls.
\item
  nate cohn\\
  I think that it absolutely could begin to help narrow the race. And
  this was the pattern in 2016, by the way. There were these moments of
  the race when the talk was always about Donald Trump. And he would get
  in these fights. The news would be all about how Donald Trump said
  this ridiculous stuff. All these Republicans would be criticizing him.
  And then Hillary Clinton would have this lead. And then two weeks
  later, something would happen, when it was her emails, or the
  ``deplorables'' line, or random health scares, and so on. And Hillary
  Clinton's lead would fall right back down to a more competitive race.
  And I think it is certainly possible that if voters focus as much on
  Joe Biden at any point in this election cycle as they focused on
  Hillary Clinton, that would probably lead to a tighter race than the
  one we have now.
\item
  {[}music{]}
\item
  michael barbaro\\
  Perhaps put more simply, what you're saying is that campaigning in
  absentia, essentially, is working for Joe Biden. And it's working
  pretty well.
\item
  nate cohn\\
  Absolutely. I mean, the fundamentals of the race right now are that
  Donald Trump is really unpopular. Donald Trump is the defining feature
  of the race. That adds up to a big win for Joe Biden right now. If
  that formulation changes, and Joe Biden is just as important in the
  minds of voters as Donald Trump, maybe Joe Biden's lead would shrink a
  lot. Maybe it would shrink a little. I don't know. But what I can say
  is that this particular dynamic is working out really well for Joe
  Biden.
\item
  michael barbaro\\
  Nate, thank you very much. And I want to warn you that we are going to
  be doing this a lot for the next five months, but not too much.
\item
  nate cohn\\
  Thank you for having me. And I am looking forward to it.
\item
  michael barbaro\\
  We'll be right back.

  Here's what else you need to know today.
\item
  archived recording (joyette holmes)\\
  Good afternoon. Today the Sun County grand jury did return an
  indictment against Travis McMichael, Greg McMichael and William
  Bryant.
\end{itemize}

michael barbaro

On Wednesday, four months after a 25-year-old black man, Ahmaud Arbery,
was chased down and killed while jogging in South Georgia, three white
men were indicted on charges ranging from malice murder to false
imprisonment.

\begin{itemize}
\tightlist
\item
  archived recording (joyette holmes)\\
  This is another positive step, another great step for finding justice
  for Ahmaud, for finding justice for this family and the community
  beyond.
\end{itemize}

michael barbaro

Local police were slow to arrest and charge the three men, prompting the
first in a series of public protests in the past few months over the
extrajudicial killing of black Americans. And the U.S. recorded nearly
37,000 new coronavirus infections on Wednesday, the highest single-day
total since the pandemic began. With infections rising so quickly, the
City of Houston said it was running out of intensive care beds. The
governor of Texas urged residents to stay inside. And Washington and
North Carolina said they would require that masks be worn in public.

\begin{itemize}
\tightlist
\item
  archived recording (ned lamont)\\
  We're going to have a quarantine on visitors from those states that
  have a positivity rate north of 10 percent over a seven day moving
  average. So it could change a little bit over time. Hopefully, those
  states that are so grave --- Florida, Texas, Arizona --- will come
  back with a lower infection rate.
\end{itemize}

michael barbaro

In the Northeast, where infection rates have declined, the governors of
three neighboring states --- New York, New Jersey and Connecticut ---
said they would require a two-week quarantine for travelers coming from
states where cases are surging.

\begin{itemize}
\tightlist
\item
  archived recording (ned lamont)\\
  What that means in terms of quarantine will be enforced differently in
  different states. What we want to do here in the state of Connecticut
  is one, if you come up from those states and you haven't tested, and
  you haven't had a negative test, you're coming up here, you've got to
  quarantine for 14 days.
\end{itemize}

michael barbaro

That's it for ``The Daily.'' I'm Michael Barbaro. See you tomorrow.

\hypertarget{new-york-timessiena-college-polls}{%
\subsubsection{New York Times/Siena College
polls}\label{new-york-timessiena-college-polls}}

We asked thousands of voters across the country about President Trump,
Joseph R. Biden Jr., the coronavirus pandemic, Black Lives Matter and
more.

\begin{itemize}
\item
  \href{https://www.nytimes3xbfgragh.onion/2020/06/24/us/politics/trump-biden-poll-nyt-upshot-siena-college.html}{\includegraphics{https://static01.graylady3jvrrxbe.onion/newsgraphics/2020/06/23/2020-pollingribbon/3133357315cec7c0cac77287fdc6f8ec7774b74f/thumbs/allstates.jpg}}

  \href{https://www.nytimes3xbfgragh.onion/2020/06/24/us/politics/trump-biden-poll-nyt-upshot-siena-college.html}{}

  \hypertarget{biden-takes-dominant-lead-as-voters-reject-trump-on-virus-and-race-1}{%
  \section{Biden Takes Dominant Lead as Voters Reject Trump on Virus and
  Race}\label{biden-takes-dominant-lead-as-voters-reject-trump-on-virus-and-race-1}}

  June 24, 2020
\item
  \href{https://www.nytimes3xbfgragh.onion/2020/06/25/upshot/poll-2020-biden-battlegrounds.html}{\includegraphics{https://static01.graylady3jvrrxbe.onion/newsgraphics/2020/06/23/2020-pollingribbon/3133357315cec7c0cac77287fdc6f8ec7774b74f/thumbs/battlestates.jpg}}

  \href{https://www.nytimes3xbfgragh.onion/2020/06/25/upshot/poll-2020-biden-battlegrounds.html}{}

  \hypertarget{showing-strength-with-white-voters-biden-builds-lead-in-battleground-states}{%
  \section{Showing Strength With White Voters, Biden Builds Lead in
  Battleground
  States}\label{showing-strength-with-white-voters-biden-builds-lead-in-battleground-states}}

  June 25, 2020
\item
  \href{https://www.nytimes3xbfgragh.onion/2020/06/25/us/politics/trump-senate-republicans-poll.html}{\includegraphics{https://static01.graylady3jvrrxbe.onion/newsgraphics/2020/06/23/2020-pollingribbon/3133357315cec7c0cac77287fdc6f8ec7774b74f/thumbs/altraces.jpg}}

  \href{https://www.nytimes3xbfgragh.onion/2020/06/25/us/politics/trump-senate-republicans-poll.html}{}

  \hypertarget{trumps-sagging-popularity-drags-down-republican-senate-candidates}{%
  \section{Trump's Sagging Popularity Drags Down Republican Senate
  Candidates}\label{trumps-sagging-popularity-drags-down-republican-senate-candidates}}

  June 25, 2020
\item
  \href{https://www.nytimes3xbfgragh.onion/2020/06/26/us/politics/biden-vice-president-voters.html}{\includegraphics{https://static01.graylady3jvrrxbe.onion/newsgraphics/2020/06/23/2020-pollingribbon/3133357315cec7c0cac77287fdc6f8ec7774b74f/thumbs/vp.png}}

  \href{https://www.nytimes3xbfgragh.onion/2020/06/26/us/politics/biden-vice-president-voters.html}{}

  \hypertarget{biden-is-getting-a-lot-of-advice-on-his-vp-heres-what-voters-think}{%
  \section{Biden Is Getting a Lot of Advice on His V.P. Here's What
  Voters
  Think.}\label{biden-is-getting-a-lot-of-advice-on-his-vp-heres-what-voters-think}}

  June 26, 2020
\end{itemize}

Here are
\href{https://int.graylady3jvrrxbe.onion/data/documenttools/crosstabs0624release/18307fed6cb2dc5a/full.pdf}{more
results} and
\href{https://www.nytimes3xbfgragh.onion/2020/06/23/upshot/poll-2020-election-method.html}{our
explanation} of the
\href{https://int.graylady3jvrrxbe.onion/data/documenttools/nyt-siena-poll-methodology-june-2020/f6f533b4d07f4cbe/full.pdf}{methodology}
for the poll.

\hypertarget{our-2020-election-guide}{%
\section{Our 2020 Election Guide}\label{our-2020-election-guide}}

Updated Aug. 4, 2020

\begin{itemize}
\item
  \begin{center}\rule{0.5\linewidth}{\linethickness}\end{center}

  \hypertarget{the-latest}{%
  \subsection{The Latest}\label{the-latest}}

  \begin{itemize}
  \tightlist
  \item
    Five states are holding primary elections Tuesday, with voters in
    Arizona, Kansas, Michigan, Missouri and Washington State choosing
    nominees for Congress and local offices.
    \href{https://www.nytimes3xbfgragh.onion/2020/08/04/us/elections/primary-election-michigan-arizona-kansas.html?action=click\&pgtype=Article\&state=default\&region=BELOW_MAIN_CONTENT\&context=storylines_guide}{Follow
    live election updates here.}
  \end{itemize}
\item
  \begin{center}\rule{0.5\linewidth}{\linethickness}\end{center}

  \hypertarget{bidens-vp-search}{%
  \subsection{Biden's V.P. Search}\label{bidens-vp-search}}

  \begin{itemize}
  \tightlist
  \item
    \href{https://www.nytimes3xbfgragh.onion/article/biden-vice-president-2020.html?action=click\&pgtype=Article\&state=default\&region=BELOW_MAIN_CONTENT\&context=storylines_guide}{Here
    are 13 women} who have been under consideration to be Joe Biden's
    running mate, and why each might be chosen --- and might not be.
  \end{itemize}
\item
  \begin{center}\rule{0.5\linewidth}{\linethickness}\end{center}

  \hypertarget{keep-up-with-our-coverage}{%
  \subsection{Keep Up With Our
  Coverage}\label{keep-up-with-our-coverage}}

  \begin{itemize}
  \tightlist
  \item
    Get an
    \href{https://www.nytimes3xbfgragh.onion/newsletters/politics?action=click\&pgtype=Article\&state=default\&region=BELOW_MAIN_CONTENT\&context=storylines_guide}{email}
    recapping the day's news
  \end{itemize}

  \begin{itemize}
  \tightlist
  \item
    Download our mobile app on
    \href{https://apps.apple.com/us/app/nytimes/id284862083?ls=1\&mat_click_id=5c79ae7455014fd1bd66b5610c05b8f2-20191112-16948\&referrer=mat_click_id\%3D5c79ae7455014fd1bd66b5610c05b8f2-20191112-16948\%26link_click_id\%3D722930677036718082}{iOS}
    and
    \href{http://a.localytics.com/android?id=com.nytimes.android\&referrer=utm_source\%3Dother_nyt_mobile_web\%26utm_medium\%3DWeb\%2520page\%26utm_term\%3DGeneral\%2520Mobile\%2520Page\%26utm_campaign\%3DNYT\%2520Mobile\%2520General\%2520Page}{Android}
    and turn on Breaking News and Politics alerts
  \end{itemize}
\end{itemize}

Advertisement

\protect\hyperlink{after-bottom}{Continue reading the main story}

\hypertarget{site-index}{%
\subsection{Site Index}\label{site-index}}

\hypertarget{site-information-navigation}{%
\subsection{Site Information
Navigation}\label{site-information-navigation}}

\begin{itemize}
\tightlist
\item
  \href{https://help.nytimes3xbfgragh.onion/hc/en-us/articles/115014792127-Copyright-notice}{©~2020~The
  New York Times Company}
\end{itemize}

\begin{itemize}
\tightlist
\item
  \href{https://www.nytco.com/}{NYTCo}
\item
  \href{https://help.nytimes3xbfgragh.onion/hc/en-us/articles/115015385887-Contact-Us}{Contact
  Us}
\item
  \href{https://www.nytco.com/careers/}{Work with us}
\item
  \href{https://nytmediakit.com/}{Advertise}
\item
  \href{http://www.tbrandstudio.com/}{T Brand Studio}
\item
  \href{https://www.nytimes3xbfgragh.onion/privacy/cookie-policy\#how-do-i-manage-trackers}{Your
  Ad Choices}
\item
  \href{https://www.nytimes3xbfgragh.onion/privacy}{Privacy}
\item
  \href{https://help.nytimes3xbfgragh.onion/hc/en-us/articles/115014893428-Terms-of-service}{Terms
  of Service}
\item
  \href{https://help.nytimes3xbfgragh.onion/hc/en-us/articles/115014893968-Terms-of-sale}{Terms
  of Sale}
\item
  \href{https://spiderbites.nytimes3xbfgragh.onion}{Site Map}
\item
  \href{https://help.nytimes3xbfgragh.onion/hc/en-us}{Help}
\item
  \href{https://www.nytimes3xbfgragh.onion/subscription?campaignId=37WXW}{Subscriptions}
\end{itemize}
