Sections

SEARCH

\protect\hyperlink{site-content}{Skip to
content}\protect\hyperlink{site-index}{Skip to site index}

\href{https://www.nytimes3xbfgragh.onion/section/arts/music}{Music}

\href{https://myaccount.nytimes3xbfgragh.onion/auth/login?response_type=cookie\&client_id=vi}{}

\href{https://www.nytimes3xbfgragh.onion/section/todayspaper}{Today's
Paper}

\href{/section/arts/music}{Music}\textbar{}The Dixie Chicks Change Their
Name, Dropping the `Dixie'

\href{https://nyti.ms/37YYJ4S}{https://nyti.ms/37YYJ4S}

\begin{itemize}
\item
\item
\item
\item
\item
\end{itemize}

\href{https://www.nytimes3xbfgragh.onion/news-event/george-floyd-protests-minneapolis-new-york-los-angeles?action=click\&pgtype=Article\&state=default\&region=TOP_BANNER\&context=storylines_menu}{Race
and America}

\begin{itemize}
\tightlist
\item
  \href{https://www.nytimes3xbfgragh.onion/2020/07/26/us/protests-portland-seattle-trump.html?action=click\&pgtype=Article\&state=default\&region=TOP_BANNER\&context=storylines_menu}{Protesters
  Return to Other Cities}
\item
  \href{https://www.nytimes3xbfgragh.onion/2020/07/24/us/portland-oregon-protests-white-race.html?action=click\&pgtype=Article\&state=default\&region=TOP_BANNER\&context=storylines_menu}{Portland
  at the Center}
\item
  \href{https://www.nytimes3xbfgragh.onion/2020/07/23/podcasts/the-daily/portland-protests.html?action=click\&pgtype=Article\&state=default\&region=TOP_BANNER\&context=storylines_menu}{Podcast:
  Showdown in Portland}
\item
  \href{https://www.nytimes3xbfgragh.onion/interactive/2020/07/16/us/black-lives-matter-protests-louisville-breonna-taylor.html?action=click\&pgtype=Article\&state=default\&region=TOP_BANNER\&context=storylines_menu}{45
  Days in Louisville}
\end{itemize}

Advertisement

\protect\hyperlink{after-top}{Continue reading the main story}

Supported by

\protect\hyperlink{after-sponsor}{Continue reading the main story}

\hypertarget{the-dixie-chicks-change-their-name-dropping-the-dixie}{%
\section{The Dixie Chicks Change Their Name, Dropping the
`Dixie'}\label{the-dixie-chicks-change-their-name-dropping-the-dixie}}

The platinum-selling country trio will be known as the Chicks, the
latest example of sweeping cultural changes brought on by nationwide
protests spotlighting racial inequality.

\includegraphics{https://static01.graylady3jvrrxbe.onion/images/2020/06/27/arts/25chicks/25chicks-articleLarge.jpg?quality=75\&auto=webp\&disable=upscale}

\href{https://www.nytimes3xbfgragh.onion/by/ben-sisario}{\includegraphics{https://static01.graylady3jvrrxbe.onion/images/2018/02/20/multimedia/author-ben-sisario/author-ben-sisario-thumbLarge.jpg}}

By \href{https://www.nytimes3xbfgragh.onion/by/ben-sisario}{Ben Sisario}

\begin{itemize}
\item
  Published June 25, 2020Updated July 8, 2020
\item
  \begin{itemize}
  \item
  \item
  \item
  \item
  \item
  \end{itemize}
\end{itemize}

The
\href{https://www.nytimes3xbfgragh.onion/2020/07/08/arts/music/dixie-chicks-gaslighter.html}{Dixie
Chicks} are now the Chicks.

The platinum-selling country trio, which in 2003 became pariahs in
Nashville for criticizing President George W. Bush on the eve of the
American-led invasion of Iraq, has changed its name, apparently in tacit
acknowledgment of criticism over its use of the word ``Dixie,'' a
nostalgic nickname for the Civil War-era South.

The group made the change stealthily on Thursday,
\href{https://youtu.be/xwBjF_VVFvE}{releasing a new video as the Chicks}
and \href{https://twitter.com/thechicks}{adjusting its social media
presence}. Representatives for the band confirmed the new name.

But the three women of the group ---~Natalie Maines, Emily Strayer and
Martie Maguire, who have been among the most outspoken figures in the
conservative world of country music ---~made little immediate comment.

In a brief statement on its new website, the band states simply: ``We
want to meet this moment.'' The new video, ``March March,'' features
images of current and historical protests --- for women's rights, gay
rights, environmental causes and Black Lives Matter.

The group's decision comes as nationwide protests over police violence
against black people have put a renewed spotlight on racial inequalities
of all kinds --- from corporate brands with
\href{https://www.nytimes3xbfgragh.onion/2020/06/17/business/aunt-jemima-racial-stereotype.html}{problematic
logos} to media organizations with a lack of diversity in their top
ranks.

For the Dixie Chicks, the pressure had come over its use of the word
Dixie, with commentary in the news media pushing the group to change its
name just as the country debates issues like removing
\href{https://www.nytimes3xbfgragh.onion/2020/06/03/us/confederate-statues-george-floyd.html}{Confederate
monuments}.

The name change comes ahead of the release of the group's first album in
14 years, ``Gaslighter,'' due out on July 17.

It is perhaps the highest-profile example of a musical act rechristening
itself over questions of historical and social resonance. This month,
the country group Lady Antebellum --- which has won five Grammy Awards
--- announced it would become
\href{https://www.nytimes3xbfgragh.onion/2020/06/11/arts/music/lady-antebellum-name-change-lady-a.html}{Lady
A}, saying, ``Our hearts have been stirred with conviction, our eyes
opened wide to the injustices, inequality and biases black women and men
have always faced and continue to face everyday.''

But the Dixie Chicks occupy an even greater level of fame. Once a
darling of country radio, the group has crossed over to become a banner
mainstream act and
\href{https://ew.com/article/2013/08/09/dixie-chicks-nude-ew-cover-court-yard-hounds/}{magazine
cover subject}, even
\href{https://www.allure.com/story/dixie-chicks-april-2020-cover-interview}{today}.
The Dixie Chicks have sold at least 33 million albums in the United
States, according to the Recording Industry Association of America, and
won 13 Grammys, including album of the year for its 2006 release
``Taking the Long Way.''

That album was the Dixie Chicks' defiant response to its abrupt ejection
from the Nashville establishment, after Maines, the group's lead singer,
\href{https://www.nytimes3xbfgragh.onion/2006/05/21/arts/music/21pare.html}{told
a London audience} in March 2003: ``We do not want this war, this
violence, and we're ashamed that the president of the United States is
from Texas.''

The fallout from that statement was immediate and fierce, with fans
\href{https://www.youtube.com/watch?v=nQ8WzYc32JQ}{protesting the group}
on tour and country radio stations dropping once-ubiquitous hits like
``Wide Open Spaces'' and ``Without You'' from playlists. Ever since,
such brutal rejection as a consequence of political speech has been so
feared that it has become a verb ---
\href{https://www.billboard.com/articles/columns/country/7685038/country-music-politics}{``Dixie
Chicked.}'' Taylor Swift's recent documentary, ``Miss Americana,''
showed that the fear loomed over even her. (Swift also had the Dixie
Chicks as guests on her latest album, ``Lover.'')

Quick rebrandings can be complicated. Soon after Lady Antebellum
announced its new name, it emerged that at least one other act --- a
\href{https://www.rollingstone.com/pro/news/lady-antebellum-lady-a-country-blues-1013919/}{black
blues singer from Seattle} --- had already been releasing music under
the name Lady A for years, leading to
\href{https://www.newsday.com/entertainment/music/lady-antebellum-lady-a-1.45795425}{awkward
settlement talks}.

The Dixie Chicks faced a similar situation, with a
\href{https://www.noted.co.nz/culture/culture-music/the-chicks-the-girls-next-door-who-became-kiwi-music-royalty}{1960s
pop group from New Zealand}, but apparently resolved it in advance.

``A sincere and heartfelt thank you goes out to `The Chicks' of NZ for
their gracious gesture in allowing us to share their name,'' Maines,
Strayer and Maguire said in a statement circulated by a spokeswoman.
``We are honored to co-exist together in the world with these
exceptionally talented sisters. Chicks Rock!''

Exactly how the Chicks will handle the sudden change was unclear. Its
social media accounts were quickly swapped on Thursday morning, but some
merchandise, like advance vinyl copies of ``Gaslighter,'' has already
been put on sale
\href{https://www.target.com/p/dixie-chicks-gaslighter-target-exclusive-vinyl/-/A-79694783}{under
the old name}.

The Dixie Chicks, founded as a bluegrass group in Dallas in 1989, took
its name in reference to
``\href{https://www.youtube.com/watch?v=yaHEfJApEVM}{Dixie Chicken},'' a
1973 album by the country-rock group Little Feat. The lyrics to that
album's title track ---~``If you'll be my Dixie chicken/I'll be your
Tennessee lamb'' ---~contain the kind of casual references to ``Dixie''
that have turned up repeatedly in country songs, with little mainstream
controversy. But as criticism of a romanticized slavery-era South has
grown, they have drawn new scrutiny.

The Dixie Chicks became major country hitmakers in the years after
Maines joined in 1995, and were hailed by critics for blending sharp
bluegrass skills with pop sensibilities. Maines has also become known
for being outspoken on progressive causes like the release of the
\href{https://www.nytimes3xbfgragh.onion/2011/08/20/us/20arkansas.html}{West
Memphis Three}, a group of teenage boys who were convicted of a murder
in 1994 but later released after widespread doubts about their case.
Maines's \href{https://twitter.com/1NatalieMaines}{Twitter account}
currently states simply: ``Black Lives Matter.''

But as protests over police brutality and the killing of George Floyd
continued, and Americans re-examined institutions and brands with ties
to racist stereotypes, the Dixie Chicks were quickly targeted with
demands to make a change. In a recent
\href{https://variety.com/2020/music/opinion/dixie-chicks-name-rethink-lady-antebellum-1234636972/}{opinion
article} in Variety, the entertainment trade publication, the journalist
Jeremy Helligar said that the term Dixie ``conjures a time and a place
of bondage.''

Advertisement

\protect\hyperlink{after-bottom}{Continue reading the main story}

\hypertarget{site-index}{%
\subsection{Site Index}\label{site-index}}

\hypertarget{site-information-navigation}{%
\subsection{Site Information
Navigation}\label{site-information-navigation}}

\begin{itemize}
\tightlist
\item
  \href{https://help.nytimes3xbfgragh.onion/hc/en-us/articles/115014792127-Copyright-notice}{©~2020~The
  New York Times Company}
\end{itemize}

\begin{itemize}
\tightlist
\item
  \href{https://www.nytco.com/}{NYTCo}
\item
  \href{https://help.nytimes3xbfgragh.onion/hc/en-us/articles/115015385887-Contact-Us}{Contact
  Us}
\item
  \href{https://www.nytco.com/careers/}{Work with us}
\item
  \href{https://nytmediakit.com/}{Advertise}
\item
  \href{http://www.tbrandstudio.com/}{T Brand Studio}
\item
  \href{https://www.nytimes3xbfgragh.onion/privacy/cookie-policy\#how-do-i-manage-trackers}{Your
  Ad Choices}
\item
  \href{https://www.nytimes3xbfgragh.onion/privacy}{Privacy}
\item
  \href{https://help.nytimes3xbfgragh.onion/hc/en-us/articles/115014893428-Terms-of-service}{Terms
  of Service}
\item
  \href{https://help.nytimes3xbfgragh.onion/hc/en-us/articles/115014893968-Terms-of-sale}{Terms
  of Sale}
\item
  \href{https://spiderbites.nytimes3xbfgragh.onion}{Site Map}
\item
  \href{https://help.nytimes3xbfgragh.onion/hc/en-us}{Help}
\item
  \href{https://www.nytimes3xbfgragh.onion/subscription?campaignId=37WXW}{Subscriptions}
\end{itemize}
