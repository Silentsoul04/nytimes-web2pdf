Sections

SEARCH

\protect\hyperlink{site-content}{Skip to
content}\protect\hyperlink{site-index}{Skip to site index}

\href{https://www.nytimes3xbfgragh.onion/section/arts}{Arts}

\href{https://myaccount.nytimes3xbfgragh.onion/auth/login?response_type=cookie\&client_id=vi}{}

\href{https://www.nytimes3xbfgragh.onion/section/todayspaper}{Today's
Paper}

\href{/section/arts}{Arts}\textbar{}Five Nonbinary Comics on This
Moment: `I'm Not Some New Buzzword'

\url{https://nyti.ms/31lcFVx}

\begin{itemize}
\item
\item
\item
\item
\item
\end{itemize}

Advertisement

\protect\hyperlink{after-top}{Continue reading the main story}

Supported by

\protect\hyperlink{after-sponsor}{Continue reading the main story}

\hypertarget{five-nonbinary-comics-on-this-moment-im-not-some-new-buzzword}{%
\section{Five Nonbinary Comics on This Moment: `I'm Not Some New
Buzzword'}\label{five-nonbinary-comics-on-this-moment-im-not-some-new-buzzword}}

These comedians talk about their experiences in the industry, and
wanting audiences to look beyond their gender identity.

\includegraphics{https://static01.graylady3jvrrxbe.onion/images/2020/06/25/arts/25nonbinary-comics-promo/2-threeByTwoMediumAt2X.jpg}

\href{https://www.nytimes3xbfgragh.onion/by/shane-oneill}{\includegraphics{https://static01.graylady3jvrrxbe.onion/images/2018/12/10/multimedia/author-shane-oneill/author-shane-oneill-thumbLarge.png}}

By \href{https://www.nytimes3xbfgragh.onion/by/shane-oneill}{Shane
O'Neill}

\begin{itemize}
\item
  June 25, 2020
\item
  \begin{itemize}
  \item
  \item
  \item
  \item
  \item
  \end{itemize}
\end{itemize}

The stand-up comedian Jes Tom has a go-to pronoun joke: ``I like when
people call me `they.' It makes me feel less lonely.''

Tom is part of a small but growing group of comedians that doesn't
exclusively identify with the gender labels of ``male'' and ``female,''
and before the pandemic were regularly performing stand-up sets around
New York City.

Over several months, we spoke with five of these comedians about the
joys and frustrations of explaining their gender onstage and the
entertainment industry's newfound interest in nonbinary performers. They
also talked about the challenges of working in a field embroiled in a
continuing discussion about \href{https://nyti.ms/31tBFrA}{what is} and
\href{https://nyti.ms/2V4Xy0p}{isn't} off limits in comedy.

All this coincides with a greater awareness of
\href{https://www.nytimes3xbfgragh.onion/2019/06/04/magazine/gender-nonbinary.html}{gender
diversity in the culture} at large. In 2019,
\href{https://www.nytimes3xbfgragh.onion/2019/12/10/us/merriam-webster-they-word-year.html}{Merriam-Webster's
word of the year} was the singular pronoun ``they,'' and both the singer
Sam Smith and Jonathan Van Ness of ``Queer Eye'' came out as nonbinary.
The Supreme Court recently ruled that the Civil Rights Act of 1964
\href{https://www.nytimes3xbfgragh.onion/2020/06/15/us/gay-transgender-workers-supreme-court.html}{protects
gay and transgender workers} from workplace discrimination. And on the
state and federal level there are proposals that would allow a third
gender option on passports and
\href{https://www.nytimes3xbfgragh.onion/2019/05/29/us/nonbinary-drivers-licenses.html}{drivers'
licenses}.

Comedy clubs remain closed, but these comedians were largely in good
spirits when we reconnected with them recently to find out how they were
managing as the city slowly emerges from its shutdown.

\hypertarget{the-embodiment-of-nonbinary}{%
\subsection{The Embodiment of
Nonbinary}\label{the-embodiment-of-nonbinary}}

\includegraphics{https://static01.graylady3jvrrxbe.onion/images/2020/06/26/arts/25nonbinary-comics-jes2/merlin_166117701_ff76823f-a4cf-46b7-a5cb-abd8a5a597a7-articleLarge.jpg?quality=75\&auto=webp\&disable=upscale}

\href{https://www.jestomdotcom.com/}{Jes Tom}, 29, has a swagger that is
at once preening, arch, acidic and world-weary. When I said I found them
surprisingly menacing onstage, they clapped in delight and responded,
``That is the nicest thing anyone's ever said to me.''

An Asian-American comedian who also hosts a weekly cooking show on
Instagram Live called
``\href{https://www.instagram.com/p/CBjfafOjRAR/}{Iron Jes},'' Tom is
aware of Hollywood's growing appetite for people who look like them.
``For better or worse, I think that the embodiment of nonbinary that I
am is really hot in the mainstream world right now. By which I mean
specifically a thin, of-color but still a relatively light-skinned
person, East Asian but still perfect English-speaking, assigned female
at birth but still kind of masculine leaning.'' Indeed, Tom fits this
prototype perfectly.

Image

Jes Tom performing a set at Union Hall last winter.Credit...Michael
George for The New York Times

Tom acknowledged that their identity has helped land jobs, including
commercial work that has helped sustain them during the pandemic. ``As a
\href{https://www.youtube.com/watch?v=htU38RhRmto}{nonbinary trans queer
Asian-American stand-up comedian}, I probably get a lot more paying gigs
than white straight {[}male{]} comedians do,'' they said. ``Which is not
to say I have it any easier systemically.

``This is like the marginalized artist double-edged sword, right?'' Tom
added. ``On the one hand, I look forward to the day when my whole career
is not about this identity part of myself. On the other hand, I feel
very grateful that this marginalized identity part of myself gets me
working, gets me attention.''

Being able to work in such a niche is a far cry from when they started
doing comedy in 2013. Back then they were usually the only gender
nonconforming person in the room. ``There was, as far as I know, no
queer comedy open mic,'' Tom recalled. ``By which I mean I Googled it
and nothing came up.''

\hypertarget{free-speech-is-a-two-way-street}{%
\subsection{`Free Speech Is a Two-Way
Street'}\label{free-speech-is-a-two-way-street}}

Image

James Tison onstage at Club Cumming, which hosts Tison's ``Snowflake
Mic'' night. The monthly open mic, on hiatus during the coronavirus
shutdown, is meant to be a safe space for L.G.B.T.Q.
performers.Credit...Michael George for The New York Times

``If you don't know, the open mic scene in New York can be shockingly
transphobic, misogynistic, racist,''
\href{https://www.jamestison.com/}{James Tison} chirped from the stage
at Club Cumming last fall.

It was the third installment of the monthly showcase ``The Snowflake
Mic,'' and Tison was explaining the ethos of the open mic night.

``You're welcome to say whatever you want. But free speech is
fortunately a two-way street,'' Tison cautioned the participating
comics. ``So if you come onto this stage and you rattle off some hack
premise about how trans women aren't women or if you think you found
some clever new way to say the `N-word' into a microphone and you're a
white person, this audience is allowed to boo you, they're allowed to
come up to you after and say `Hey, I didn't like that,' and you have to
take it.''

Image

Credit...Michael George for The New York Times

As a stand-up, Tison, 33, has performed at open mics in New York City
for years and regularly encountered heckling and harassment both onstage
and off. In response, they curated a list of
``\href{https://www.jamestison.com/faq}{nontoxic open mics}'' on their
website and created ``The Snowflake Mic,'' reclaiming the word that is
used to describe an overly sensitive person.

Not everyone appreciated the joke. A few comedians took to Twitter
extolling the virtue of tough rooms and hostile crowds as important
preparation for a nasty and difficult industry.

But Tison bristled at the idea that a comedy night by and for L.G.B.T.Q.
comedians represents a radical departure from comedy at large or a
softening of jokes. ``I don't think anybody has neutral material,'' they
said. ``It's a made-up concept. There's just a lot of straight men in
the business and we call that neutral.''

``The Snowflake Mic'' has been on hold since March, but Tison plans to
restart it as soon as possible. For Tison, hosting the mic is about
labor as much as it is about comedy.

Image

Credit...Michael George for The New York Times

``Open mics are the entry point for the entire field,'' Tison wrote in
an email. ``They're the only way to actually get better at stand-up, but
they're also where you build professional relationships with like-minded
folks that lead to future gigs. We're talking about job opportunities in
a job market.''

Of course, that job market and the open mics that feed it are largely
gone right now. Over the past few months, Tison has spent the time they
would normally have spent onstage working on a podcast for the
\href{https://shop.naral.org/products/book}{abortion rights group
NARAL}, writing a pilot, and uploading videos to
\href{https://www.tiktok.com/@james.tison?lang=en}{TikTok}, which they
describe as ``methadone'' to the heroin of live comedy performance.

They've also turned their attention and anxieties to their own home. ``I
built a desk,'' Tison told me. ``I repainted my bathroom twice.''

\hypertarget{ripping-apart-the-reality-were-in}{%
\subsection{Ripping Apart `the Reality We're
In'}\label{ripping-apart-the-reality-were-in}}

Image

``We created an alternative scene in an industry that didn't have space
for us. So now we're doing that but in actual life,'' Lorelei Ramirez
said about their work and activism during the current political
climate.Credit...Michael George for The New York Times

For anyone concerned that nonbinary comedians represent politically
correct or sanitized comedy, Lorelei Ramirez, 30, might ease their
worries. At their \href{https://www.moma.org/calendar/events/5931}{MoMA
PS1} show last fall, Ramirez told a joke about getting caught
masturbating in a public library and mused about how a choir of
pedophiles might sound.

Ramirez's PS1 show also included a musical performance using their voice
and a looping pedal, a live projection of a digital drawing, and a
circus-themed art installation, complete with a stilt walker, popcorn,
and framed portraits of clowns painted by Ramirez. ``Nonbinary'' applies
not only to Ramirez's gender, but to their entire approach to
performance and art.

While they do tell jokes about their gender identity (``I'm nonbinary,
so I'm not like other girls. Because I'm a person.''), Ramirez often
veers into absurdist body horror monologues delivered in voices that can
test the line between **** cutesy and creepy.

Image

Lorelei Ramirez onstage at C'mon Everybody in Brooklyn.Credit...Michael
George for The New York Times

``I like to jokingly rip apart the reality that we're in,'' Ramirez
explained over Zoom. ``I just like to have people imagine the craziest
thing that they can.'' Which, in the case of Lorelei's comedy can
include a long non sequitur about being followed home and killed by a
one-foot-tall imp with a giant head.

Ramirez's performances during the lockdown have mostly been limited to a
weekly live drawing show called
``\href{https://www.twitch.tv/lramirez8095}{Art Is Easy},'' on Twitch.
They've also been working with friends to provide food and wellness
services to community organizers and activists.

Though not performing regularly, Ramirez said they're still using their
``comedy brain and production brain'' amid the current political
climate. ``We created an alternative scene in an industry that didn't
have space for us,'' Ramirez said. ``So now we're doing that but in
actual life, not just this small scene. And it still applies.''

\hypertarget{a-duty-to-free-other-people}{%
\subsection{A Duty to `Free Other
People'}\label{a-duty-to-free-other-people}}

Image

``All language is wrong,'' Peter Smith said, referring to the concept of
nonbinary gender identity. ``To pick an identity still has a binary
nature to it because there's still a decision that needs to be
made.''Credit...Michael George for The New York Times

In an elegiac solo performance at Ars Nova, Peter Smith, 29, portrayed
Princess Diana in a work that featured original songs and monologues
scattered among a virtuosic lip sync of Diana's 1995 BBC interview with
the journalist Martin Bashir. It was tragic and disorienting, and
occasionally hilarious.

\href{http://ptrsmth.com/}{Smith}'s performance in ``Diana'' had more in
common with esoteric theatrical performers like
\href{http://dickiebeau.com/}{Dickie Beau} or
\href{https://www.lypsinka.com/}{Lypsinka} than your average stand-up
comedian, but Smith still sees value in being an openly trans comedian
playing for mainstream comedy audiences.

``If you are free from something, it's your duty to free other people,''
Smith told me. ``Just seeing someone exist and have fun is liberating.''

Image

Peter Smith hosting ``The Bongo Hour,'' a flashy variety show, at the
Slipper Room in December.Credit...Michael George for The New York Times

Smith is philosophical about the concept of nonbinary gender identity.
``All language is wrong,'' Smith said. ``To pick an identity still has a
binary nature to it because there's still a decision that needs to be
made.''

Smith has frequently performed at comedy venues like Caroline's on
Broadway, but their career has been nothing if not eclectic. They've
played the titular role in a production of Leonard Bernstein's ``Peter
Pan,'' worked as a wardrobe production assistant for independent films,
and assisted the painter George Condo.

They also hosted two variety shows in New York City with their frequent
collaborator, Sandy Honig: the flashy high-gloss burlesque show ``The
Bongo Hour'' and the self-consciously casual weekly comedy show ``Pig,''
which ended its run in 2019.

Image

Credit...Michael George for The New York Times

In 2020, Smith starred in the new musical ``XY'' at the Village Theater
in Seattle, appeared alongside Honig in Adult Swim's ``Three Busy
Debras,'' and delivered a powerhouse rendition of the Beach Boys' ``God
Only Knows'' in an episode of Hulu's ``Shrill.''

Since March, Smith has eschewed traditional comedy in favor of
songwriting and community organizing, sometimes collaborating with
Lorelei Ramirez. Does Smith miss live performance? Of course.

``But my desire to get back on the stage does not come close to my
desire for getting people together and activating them,'' they said.
``Everything that's happening now is very live.''

\hypertarget{i-always-wanted-to-do-it-all}{%
\subsection{`I Always Wanted to Do It
All'}\label{i-always-wanted-to-do-it-all}}

Image

``I feel like the body that I've been given is like a weapon,'' Spike
Einbinder said. ``It's something that I use as a tool because I don't
want it to be used against me.''Credit...Michael George for The New York
Times

A summer camp talent show was Spike Einbinder's first brush with
performing comedy. They pantomimed picking tomatoes off a hamburger
while Bonnie Tyler's ``Total Eclipse of the Heart'' played on a tape
recorder. ``I was never like, `I'm doing just one thing,''' Einbinder
said. ``I always wanted to do it all.''

Einbinder, who insisted in our conversations that their age was 5,412,
fuses elements of theater, drag and performance art in their comedy
sets. They have appeared onstage as demons, golems, or in the case of
their alter-ego, \href{https://youtu.be/pxg8rkB5qMM}{Candy Dish}, a
green-skinned swamp creature who cut her teeth as a comedian working
``the Bog Belt.''

During their more traditional stand-up sets, Einbinder sometimes uses
their trans identity to toy with their audience. ``I'll say `Who here
thinks I'm a girl?''' Einbinder said. ``And usually people are too
afraid to answer that.''

Image

Spike Einbinder, whose sets fuse elements of theater, drag and
performance art, at Otto's Shrunken Head in the East
Village.Credit...Michael George for The New York Times

``I feel like the body that I've been given is like a weapon,''
Einbinder told me on the phone. ``It's something that I use as a tool
because I don't want it to be used against me.''

Einbinder's eclectic performances made them a favorite at alternative
and queer comedy nights, and they've also appeared on HBO's
``\href{https://www.hbo.com/high-maintenance/season-3/5-payday}{High
Maintenance}'' and ``\href{https://www.hbo.com/los-espookys}{Los
Espookys},'' which was written by their best friend, Julio Torres.

``I would like to think that the reason that I get roles and the reason
that I get booked is because I'm a singular, unique person who is funny
and not because I'm some new buzzword,'' they said. ``I've always been
this way.''

Advertisement

\protect\hyperlink{after-bottom}{Continue reading the main story}

\hypertarget{site-index}{%
\subsection{Site Index}\label{site-index}}

\hypertarget{site-information-navigation}{%
\subsection{Site Information
Navigation}\label{site-information-navigation}}

\begin{itemize}
\tightlist
\item
  \href{https://help.nytimes3xbfgragh.onion/hc/en-us/articles/115014792127-Copyright-notice}{©~2020~The
  New York Times Company}
\end{itemize}

\begin{itemize}
\tightlist
\item
  \href{https://www.nytco.com/}{NYTCo}
\item
  \href{https://help.nytimes3xbfgragh.onion/hc/en-us/articles/115015385887-Contact-Us}{Contact
  Us}
\item
  \href{https://www.nytco.com/careers/}{Work with us}
\item
  \href{https://nytmediakit.com/}{Advertise}
\item
  \href{http://www.tbrandstudio.com/}{T Brand Studio}
\item
  \href{https://www.nytimes3xbfgragh.onion/privacy/cookie-policy\#how-do-i-manage-trackers}{Your
  Ad Choices}
\item
  \href{https://www.nytimes3xbfgragh.onion/privacy}{Privacy}
\item
  \href{https://help.nytimes3xbfgragh.onion/hc/en-us/articles/115014893428-Terms-of-service}{Terms
  of Service}
\item
  \href{https://help.nytimes3xbfgragh.onion/hc/en-us/articles/115014893968-Terms-of-sale}{Terms
  of Sale}
\item
  \href{https://spiderbites.nytimes3xbfgragh.onion}{Site Map}
\item
  \href{https://help.nytimes3xbfgragh.onion/hc/en-us}{Help}
\item
  \href{https://www.nytimes3xbfgragh.onion/subscription?campaignId=37WXW}{Subscriptions}
\end{itemize}
