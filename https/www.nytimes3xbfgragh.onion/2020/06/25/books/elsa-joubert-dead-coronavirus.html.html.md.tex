Sections

SEARCH

\protect\hyperlink{site-content}{Skip to
content}\protect\hyperlink{site-index}{Skip to site index}

\href{https://www.nytimes3xbfgragh.onion/section/books}{Books}

\href{https://myaccount.nytimes3xbfgragh.onion/auth/login?response_type=cookie\&client_id=vi}{}

\href{https://www.nytimes3xbfgragh.onion/section/todayspaper}{Today's
Paper}

\href{/section/books}{Books}\textbar{}Elsa Joubert, 97, Dies; Afrikaans
Writer Explored Black Reality

\url{https://nyti.ms/31iSRCe}

\begin{itemize}
\item
\item
\item
\item
\item
\end{itemize}

\href{https://www.nytimes3xbfgragh.onion/news-event/coronavirus?action=click\&pgtype=Article\&state=default\&region=TOP_BANNER\&context=storylines_menu}{The
Coronavirus Outbreak}

\begin{itemize}
\tightlist
\item
  live\href{https://www.nytimes3xbfgragh.onion/2020/08/04/world/coronavirus-covid-19.html?action=click\&pgtype=Article\&state=default\&region=TOP_BANNER\&context=storylines_menu}{Latest
  Updates}
\item
  \href{https://www.nytimes3xbfgragh.onion/interactive/2020/us/coronavirus-us-cases.html?action=click\&pgtype=Article\&state=default\&region=TOP_BANNER\&context=storylines_menu}{Maps
  and Cases}
\item
  \href{https://www.nytimes3xbfgragh.onion/interactive/2020/science/coronavirus-vaccine-tracker.html?action=click\&pgtype=Article\&state=default\&region=TOP_BANNER\&context=storylines_menu}{Vaccine
  Tracker}
\item
  \href{https://www.nytimes3xbfgragh.onion/2020/08/02/us/covid-college-reopening.html?action=click\&pgtype=Article\&state=default\&region=TOP_BANNER\&context=storylines_menu}{College
  Reopening}
\item
  \href{https://www.nytimes3xbfgragh.onion/live/2020/08/03/business/stock-market-today-coronavirus?action=click\&pgtype=Article\&state=default\&region=TOP_BANNER\&context=storylines_menu}{Economy}
\end{itemize}

Advertisement

\protect\hyperlink{after-top}{Continue reading the main story}

Supported by

\protect\hyperlink{after-sponsor}{Continue reading the main story}

Those We've Lost

\hypertarget{elsa-joubert-97-dies-afrikaans-writer-explored-black-reality}{%
\section{Elsa Joubert, 97, Dies; Afrikaans Writer Explored Black
Reality}\label{elsa-joubert-97-dies-afrikaans-writer-explored-black-reality}}

Her novel ``The Long Journey of Poppie Nongena'' was compared to Alan
Paton's ``Cry, the Beloved Country'' in arousing white opinion against
apartheid. She had Covid-19.

\includegraphics{https://static01.graylady3jvrrxbe.onion/images/2020/06/26/obituaries/24Joubert-print/24Joubert-articleLarge.jpg?quality=75\&auto=webp\&disable=upscale}

By \href{https://www.nytimes3xbfgragh.onion/by/alan-cowell}{Alan Cowell}

\begin{itemize}
\item
  Published June 25, 2020Updated June 26, 2020
\item
  \begin{itemize}
  \item
  \item
  \item
  \item
  \item
  \end{itemize}
\end{itemize}

\emph{This obituary is part of a series about people who have died in
the coronavirus pandemic. Read about others}
\href{https://www.nytimes3xbfgragh.onion/interactive/2020/obituaries/people-died-coronavirus-obituaries.html}{\emph{here}}\emph{.}

Elsa Joubert, one of South Africa's best-known writers in the Afrikaans
language, whose apartheid-era novel ``The Long Journey of Poppie
Nongena'' opened the eyes of many white South Africans to the harsh
treatment that the black majority had been enduring largely out of their
sight, died on June 14 in Cape Town. She was 97.

She had received a diagnosis of Covid-19, her son, Nico Steytler, told
South African news media.

Ms. Joubert belonged to a group of dissident writers in Afrikaans --- a
language derived from the 17th-century Dutch spoken by South Arica's
first white settlers --- who called themselves
\href{https://www.britannica.com/topic/Sestigers}{``Die Sestigers''}
(the Sixtyers, or writers of the 1960s).

Her work ranged from novels to autobiography to travelogues, but among
her books it was ``Poppie Nongena'' that struck the most resounding
chord in South Africa. First published in 1978 in Afrikaans as ``Die
Swerfjare van Poppie Nongena,'' the novel tells of a black woman's
struggle to keep her family together in the face of oppressive apartheid
laws intended to control the lives of the black majority from cradle to
grave.

As the writer and fellow Die Sestiger
\href{https://www.britannica.com/biography/Andre-Philippus-Brink\#ref113215}{André
Brink} put it in an essay, quoted in her obituary in
\href{https://johannesburgreviewofbooks.com/2020/06/15/elsa-joubert-1922-2020-rip/}{The
Johannesburg Review of Books}, the novel, based on the life of an actual
South African woman, ``caused a furor in Afrikaner circles.''

He added, ``It would not be an overstatement to say that, in this
fictionalized biography, Elsa Joubert has done for Afrikaners what
\href{https://www.nytimes3xbfgragh.onion/1988/04/13/obituaries/alan-paton-author-who-fought-against-apartheid-is-dead-at-85.html}{Paton's}
`Cry, the Beloved Country' did for white readers'' 30 years earlier in
arousing world opinion against apartheid.

``Poppie Nongena'' was translated into 13 languages and won a host of
South African literary awards.

Image

Ms. Joubert's novel~``The Long Journey of Poppie Nongena'' was
translated from Afrikaans into 13 languages. It won a host of South
African literary awards.

Elsabé Antoinette Murray Joubert was born on Oct. 19, 1922, in the Cape
settlement of Paarl, which was closely associated with the Afrikaners'
campaign for official recognition of their language. She was educated in
a racially segregated system that pre-dated apartheid.

Before becoming an author, Ms. Joubert was a teacher in the remote
Eastern Cape town of Cradock, which would later be a crucible of black
resistance. She married Klaas Steytler, a writer, in 1950 and had three
children with him, Elsabé, Henriette and Nico. Mr. Steytler died in
1998. (Complete information on her survivors was not immediately
available.)

In preparing to write ``Poppie Nongena'' Ms. Joubert had long
conversations with the woman on whom she based the title character. Ms.
Joubert said that only the woman's name in the book, Poppie Nongena, was
an invention.

Ms. Joubert trod a fine line as a white woman seeking to articulate the
plight of a black protagonist at a time when many white South Africans
displayed scant curiosity about the lives of black people, who most
often occupied the most menial of positions.

The gulf between the Sestigers and many other Afrikaners produced what
Mr. Brink,
\href{https://www.nytimes3xbfgragh.onion/2015/02/08/books/andre-brink-south-african-literary-figure-who-ran-afoul-of-censors-dies-at-79.html}{who
died in 2015}, called a ``cultural schizophrenia.'' In their early work,
he said, ``they could not reconcile their cosmopolitan outlook with the
laager mentality of Afrikanerdom,'' referring to a circle-the-wagons
defensiveness.

They ``finally resolved the conflicts within themselves by `coming home'
to Africa in the fullest sense of the word,'' he added, coming to see
their identity as part of a common African heritage.

``Poppie Nongena'' appears on a
\href{https://www.ascleiden.nl/content/webdossiers/africas-100-best-books-20th-century}{list
of the best 100 African books} of the 20th century, as compiled in 2002
by the African Studies Center at Leiden University in the Netherlands.
It inspired a play, adapted by Ms. Joubert and Sandra Kotze (it had
\href{https://www.nytimes3xbfgragh.onion/1982/03/28/theater/theater-the-long-journey-of-poppie-nongena.html}{its
New York premiere} Off Broadway in 1982), and
\href{https://www.imdb.com/title/tt9248912/?ref_=ttfc_fc_tt}{a South
African movie in 2019}. Ms. Joubert was awarded high honors by the
post-apartheid government in the early 1990s.

In 1995 she published what some reviewers took as a counter-story, ``Die
Reise van Isobelle'' (translated into English in 2002 as ``The Long
Journey of Isobelle''), which explored the blinkered lives of women in
an Afrikaner family over a century.

Ms. Joubert's literary career spanned decades. Her brief debut novel,
``Ons Wag Op Die Kaptein,'' appeared in 1963 and was published in
English in 1982 as ``To Die at Sunset.'' She published a final volume of
autobiography, ``Cul-de-sac,'' in English in 2019. The memoir, in which
she contemplates the vagaries and indignities of aging, was published in
Afrikaans as ``Spertyd,'' or ``Deadline,'' in 2017.

J. M. Coetzee, the South African Nobel laureate in literature, said of
``Cul-de-sac,'' ``Seldom have the humiliations of old age been so
nakedly laid open.''

In her last months, when the coronavirus pandemic forced Ms. Joubert to
live under lockdown in a care home in Cape Town, her writing took
briefer, more urgent form. In an
\href{https://www.news24.com/arts/literature/iconic-sa-author-elsa-joubert-97-dies-of-covid-19-20200614}{open
letter} in May, she appealed plaintively and passionately for a
relaxation of the quarantine rules that prevented care home residents
from seeing close relatives.

``We are in the last months and weeks of our lives,'' she wrote, ``and
we who live in homes or institutions, however wonderful, are totally cut
off from our family members.''

``I'm suffering. Telephone calls, videos, Skype and much more help, but
it's not enough,'' she wrote. ``It's not the same.''

\href{https://www.nytimes3xbfgragh.onion/interactive/2020/obituaries/people-died-coronavirus-obituaries.html?action=click\&pgtype=Article\&state=default\&region=BELOW_MAIN_CONTENT\&context=covid_obits_promo}{}

\hypertarget{those-weve-lost}{%
\section{Those We've Lost}\label{those-weve-lost}}

The coronavirus pandemic has taken an incalculable death toll. This
series is designed to put names and faces to the numbers.

Read more

\includegraphics{https://static01.graylady3jvrrxbe.onion/images/2020/07/30/obituaries/30Pedro/30Pedro-square640.jpg}

\hypertarget{bernaldina-josuxe9-pedro}{%
\section{Bernaldina José Pedro}\label{bernaldina-josuxe9-pedro}}

d. Boa Vista, Brazil

Leader among the Indigenous Macuxi

\includegraphics{https://static01.graylady3jvrrxbe.onion/images/2020/07/31/obituaries/31Swing/merlin_175167783_8913bc90-0d64-43f3-a655-1bb1bf1601c9-square640.jpg}

\hypertarget{john-eric-swing}{%
\section{John Eric Swing}\label{john-eric-swing}}

d. Fountain Valley, Calif.

Champion of Filipino-Americans

\includegraphics{https://static01.graylady3jvrrxbe.onion/images/2020/07/27/obituaries/27Victor/merlin_175001436_38b11f8e-227a-4e2c-9821-7618af9b2524-square640.jpg}

\hypertarget{victor-victor}{%
\section{Victor Victor}\label{victor-victor}}

d. Santo Domingo, Dominican Republic

Beloved musician of the Dominican Republic

\includegraphics{https://static01.graylady3jvrrxbe.onion/images/2020/07/31/obituaries/31Negron/merlin_175160169_516322ae-fd23-4969-b6b2-193ced371105-square640.jpg}

\hypertarget{dr-eddie-negruxf3n}{%
\section{Dr. Eddie Negrón}\label{dr-eddie-negruxf3n}}

d. Fort Walton Beach, Fla.

Internist on Florida's Emerald Coast

\includegraphics{https://static01.graylady3jvrrxbe.onion/images/2020/07/30/obituaries/30Dobson/merlin_175115928_f6b9271c-8f05-4fe1-a38a-5ca4a58f8935-square640.jpg}

\hypertarget{dobby-dobson}{%
\section{Dobby Dobson}\label{dobby-dobson}}

d. Coral Springs, Fla.

Jamaican singer and songwriter

\includegraphics{https://static01.graylady3jvrrxbe.onion/images/2020/08/01/obituaries/28Gonzalez/merlin_175002771_beb57888-3951-409a-ae13-03a94b2e962e-square640.jpg}

\hypertarget{waldemar-gonzalez}{%
\section{Waldemar Gonzalez}\label{waldemar-gonzalez}}

d. White Plains, N.Y.

Teacher and social worker

Advertisement

\protect\hyperlink{after-bottom}{Continue reading the main story}

\hypertarget{site-index}{%
\subsection{Site Index}\label{site-index}}

\hypertarget{site-information-navigation}{%
\subsection{Site Information
Navigation}\label{site-information-navigation}}

\begin{itemize}
\tightlist
\item
  \href{https://help.nytimes3xbfgragh.onion/hc/en-us/articles/115014792127-Copyright-notice}{©~2020~The
  New York Times Company}
\end{itemize}

\begin{itemize}
\tightlist
\item
  \href{https://www.nytco.com/}{NYTCo}
\item
  \href{https://help.nytimes3xbfgragh.onion/hc/en-us/articles/115015385887-Contact-Us}{Contact
  Us}
\item
  \href{https://www.nytco.com/careers/}{Work with us}
\item
  \href{https://nytmediakit.com/}{Advertise}
\item
  \href{http://www.tbrandstudio.com/}{T Brand Studio}
\item
  \href{https://www.nytimes3xbfgragh.onion/privacy/cookie-policy\#how-do-i-manage-trackers}{Your
  Ad Choices}
\item
  \href{https://www.nytimes3xbfgragh.onion/privacy}{Privacy}
\item
  \href{https://help.nytimes3xbfgragh.onion/hc/en-us/articles/115014893428-Terms-of-service}{Terms
  of Service}
\item
  \href{https://help.nytimes3xbfgragh.onion/hc/en-us/articles/115014893968-Terms-of-sale}{Terms
  of Sale}
\item
  \href{https://spiderbites.nytimes3xbfgragh.onion}{Site Map}
\item
  \href{https://help.nytimes3xbfgragh.onion/hc/en-us}{Help}
\item
  \href{https://www.nytimes3xbfgragh.onion/subscription?campaignId=37WXW}{Subscriptions}
\end{itemize}
