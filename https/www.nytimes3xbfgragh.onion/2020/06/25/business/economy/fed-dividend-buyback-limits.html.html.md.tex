Sections

SEARCH

\protect\hyperlink{site-content}{Skip to
content}\protect\hyperlink{site-index}{Skip to site index}

\href{https://www.nytimes3xbfgragh.onion/section/business/economy}{Economy}

\href{https://myaccount.nytimes3xbfgragh.onion/auth/login?response_type=cookie\&client_id=vi}{}

\href{https://www.nytimes3xbfgragh.onion/section/todayspaper}{Today's
Paper}

\href{/section/business/economy}{Economy}\textbar{}Fed Limits Bank
Payouts and Suspends Share Buybacks as Pandemic Grinds On

\url{https://nyti.ms/2YyZmPr}

\begin{itemize}
\item
\item
\item
\item
\item
\end{itemize}

\hypertarget{the-coronavirus-outbreak}{%
\subsubsection{\texorpdfstring{\href{https://www.nytimes3xbfgragh.onion/news-event/coronavirus?name=styln-coronavirus-markets\&region=TOP_BANNER\&variant=undefined\&block=storyline_menu_recirc\&action=click\&pgtype=Article\&impression_id=fd8f52a0-e387-11ea-bb43-632ae03b21d4}{The
Coronavirus
Outbreak}}{The Coronavirus Outbreak}}\label{the-coronavirus-outbreak}}

\begin{itemize}
\tightlist
\item
  live\href{https://www.nytimes3xbfgragh.onion/2020/08/20/world/coronavirus-covid.html?name=styln-coronavirus-markets\&region=TOP_BANNER\&variant=undefined\&block=storyline_menu_recirc\&action=click\&pgtype=Article\&impression_id=fd8f52a1-e387-11ea-bb43-632ae03b21d4}{Latest
  Updates}
\item
  \href{https://www.nytimes3xbfgragh.onion/interactive/2020/us/coronavirus-us-cases.html?name=styln-coronavirus-markets\&region=TOP_BANNER\&variant=undefined\&block=storyline_menu_recirc\&action=click\&pgtype=Article\&impression_id=fd960960-e387-11ea-bb43-632ae03b21d4}{Maps
  and Cases}
\item
  \href{https://www.nytimes3xbfgragh.onion/interactive/2020/science/coronavirus-vaccine-tracker.html?name=styln-coronavirus-markets\&region=TOP_BANNER\&variant=undefined\&block=storyline_menu_recirc\&action=click\&pgtype=Article\&impression_id=fd960961-e387-11ea-bb43-632ae03b21d4}{Vaccine
  Tracker}
\item
  \href{https://www.nytimes3xbfgragh.onion/2020/08/19/us/colleges-closing-covid.html?name=styln-coronavirus-markets\&region=TOP_BANNER\&variant=undefined\&block=storyline_menu_recirc\&action=click\&pgtype=Article\&impression_id=fd960962-e387-11ea-bb43-632ae03b21d4}{Colleges
  Closing}
\item
  \href{https://www.nytimes3xbfgragh.onion/live/2020/08/20/business/stock-market-today-coronavirus?name=styln-coronavirus-markets\&region=TOP_BANNER\&variant=undefined\&block=storyline_menu_recirc\&action=click\&pgtype=Article\&impression_id=fd960963-e387-11ea-bb43-632ae03b21d4}{Economy}
\end{itemize}

Advertisement

\protect\hyperlink{after-top}{Continue reading the main story}

Supported by

\protect\hyperlink{after-sponsor}{Continue reading the main story}

\hypertarget{fed-limits-bank-payouts-and-suspends-share-buybacks-as-pandemic-grinds-on}{%
\section{Fed Limits Bank Payouts and Suspends Share Buybacks as Pandemic
Grinds
On}\label{fed-limits-bank-payouts-and-suspends-share-buybacks-as-pandemic-grinds-on}}

About a quarter of banks would near minimum capital levels in a double
dip recession, the Fed said, as it announced its stress-test results.

\includegraphics{https://static01.graylady3jvrrxbe.onion/images/2020/06/25/business/25DC-StressTest-01/merlin_172627287_74c022dd-a667-45c1-8cd8-b5cbf22a0c51-articleLarge.jpg?quality=75\&auto=webp\&disable=upscale}

By \href{https://www.nytimes3xbfgragh.onion/by/jeanna-smialek}{Jeanna
Smialek}, \href{https://www.nytimes3xbfgragh.onion/by/peter-eavis}{Peter
Eavis} and \href{https://www.nytimes3xbfgragh.onion/by/kate-kelly}{Kate
Kelly}

\begin{itemize}
\item
  June 25, 2020
\item
  \begin{itemize}
  \item
  \item
  \item
  \item
  \item
  \end{itemize}
\end{itemize}

WASHINGTON --- The Federal Reserve on Thursday temporarily restricted
shareholder payouts by the nation's biggest banks, barring them from
buying back their own stocks or increasing dividend payments in the
third quarter as regulators try to ensure banks remain strong enough to
keep lending through the pandemic-induced downturn.

The decision to limit payouts is an admission by the Fed that large
financial institutions, while far better off than they were in the
financial crisis, remain vulnerable to an economic downturn unlike any
other in modern history. With virus cases across the United States still
surging and business activity subdued, it remains unclear when and how
robustly the economy will recover.

Some of the Fed's own loss projections for banks, in fact, suggest that
the eventual hit to loans in a bad scenario could be far worse than in
the aftermath of 2008.

Still, the Fed stopped short of barring banks from paying dividends next
quarter, as some lawmakers and former regulators have urged --- a
decision that drew public criticism from one of the Fed's current
governors, who said not taking stronger measures could ``impair the
recovery.''

The Fed, which devised its primary stress test scenarios before the
virus tore through the economy, will require the 34 biggest banks to
resubmit and update their capital plans later this year, something it
has usually required only for banks that failed to pass. Those plans
detail how the banks intend to proceed with share buybacks and dividend
increases in light of the pandemic, and the Fed said that resubmitting
them ``will help firms reassess their capital needs.''

It will also allow the Fed to reserve the right to run additional
analyses, and potentially restrict payouts further, down the road.

``Today's actions by the board to preserve the high levels of capital in
the U.S. banking system are an acknowledgment of both the strength of
our largest banks as well as the high degree of uncertainty we face,''
Randal K. Quarles, the Fed's vice chairman of supervision, said in a
statement.

\hypertarget{latest-updates-the-coronavirus-outbreak-and-the-economy}{%
\section{\texorpdfstring{\href{https://www.nytimes3xbfgragh.onion/live/2020/08/20/business/stock-market-today-coronavirus?action=click\&pgtype=Article\&state=default\&region=MAIN_CONTENT_1\&context=storylines_live_updates}{Latest
Updates: The Coronavirus Outbreak and the
Economy}}{Latest Updates: The Coronavirus Outbreak and the Economy}}\label{latest-updates-the-coronavirus-outbreak-and-the-economy}}

\href{https://www.nytimes3xbfgragh.onion/live/2020/08/20/business/stock-market-today-coronavirus?action=click\&pgtype=Article\&state=default\&region=MAIN_CONTENT_1\&context=storylines_live_updates\#the-producer-of-unhinged-makes-a-big-bet-on-audiences-returning-to-theaters}{10h
ago}

\href{https://www.nytimes3xbfgragh.onion/live/2020/08/20/business/stock-market-today-coronavirus?action=click\&pgtype=Article\&state=default\&region=MAIN_CONTENT_1\&context=storylines_live_updates\#the-producer-of-unhinged-makes-a-big-bet-on-audiences-returning-to-theaters}{The
producer of `Unhinged' makes a big bet on audiences returning to
theaters.}

\href{https://www.nytimes3xbfgragh.onion/live/2020/08/20/business/stock-market-today-coronavirus?action=click\&pgtype=Article\&state=default\&region=MAIN_CONTENT_1\&context=storylines_live_updates\#american-airlines-to-stop-flights-to-15-cities-after-government-aid-ends}{19h
ago}

\href{https://www.nytimes3xbfgragh.onion/live/2020/08/20/business/stock-market-today-coronavirus?action=click\&pgtype=Article\&state=default\&region=MAIN_CONTENT_1\&context=storylines_live_updates\#american-airlines-to-stop-flights-to-15-cities-after-government-aid-ends}{American
Airlines to stop flights to 15 cities after government aid ends.}

\href{https://www.nytimes3xbfgragh.onion/live/2020/08/20/business/stock-market-today-coronavirus?action=click\&pgtype=Article\&state=default\&region=MAIN_CONTENT_1\&context=storylines_live_updates\#without-school-plays-and-assemblies-a-technicians-livelihood-withers}{19h
ago}

\href{https://www.nytimes3xbfgragh.onion/live/2020/08/20/business/stock-market-today-coronavirus?action=click\&pgtype=Article\&state=default\&region=MAIN_CONTENT_1\&context=storylines_live_updates\#without-school-plays-and-assemblies-a-technicians-livelihood-withers}{Without
school plays and assemblies, a technician's livelihood withers.}

\href{https://www.nytimes3xbfgragh.onion/live/2020/08/20/business/stock-market-today-coronavirus?action=click\&pgtype=Article\&state=default\&region=MAIN_CONTENT_1\&context=storylines_live_updates}{See
more updates}

More live coverage:
\href{https://www.nytimes3xbfgragh.onion/2020/08/20/world/coronavirus-covid.html?action=click\&pgtype=Article\&state=default\&region=MAIN_CONTENT_1\&context=storylines_live_updates}{Global}

The central bank's annual stress tests assess how the banks would fare
under dire scenarios that include high unemployment and severe market
turbulence. While those tests are meant to be hypothetical, this year's
scenarios were set before the pandemic, and some of the economic
projections now look benign compared to reality. To compensate for that,
the Fed ran an additional analysis to gauge how the banks would perform
under coronavirus recessions of varying severity.

The hypothetical scenarios included a sharp bounce-back, an extended
``U''-shaped downturn, and a double-dip ``W'' recession.

In aggregate, under those severe analyses, loan losses for the 34 banks
ranged from \$560 billion to \$700 billion, and overall capital ratios
declined from 12 percent in the fourth quarter of 2019 to between 9.5
percent and 7.7 percent.

The Fed did not release results for individual banks, as it does with
the annual stress tests. But it did show that about a quarter of banks
would nearly breach minimum capital ratios in a double-dip recession
scenario, based on
\href{https://www.nytimes3xbfgragh.onion/2020/06/25/business/economy/fed-limits-bank-payouts-and-suspends-share-buybacks-as-pandemic-grinds-on.html}{the
report}.

Given those results, the central bank will cap dividends to the amount
paid in the second quarter, with an additional limitation based on
recent earnings. While the eight largest banks had voluntarily suspended
share buybacks through the second quarter, the Fed's move will broaden
and extend that limitation.

The Financial Services Forum, which represents the chief executives of
the biggest U.S. banks, issued a muted statement saying that its members
``appreciate the Federal Reserve's work to promote financial stability
during such extraordinary economic uncertainty and understand its
decision regarding capital returns through the third quarter.''

Others felt that the Fed could have gone further to shore up the
financial system. Officials could have placed formal restrictions on
shareholder payouts earlier in the coronavirus crisis, and the decision
to do so now is a sign that regulators believe the financial system
could face threats if the downturn drags on. But the fact that the Fed's
demands are not stricter could limit the amount of buffer that banks
have on hand to absorb losses and make loans to households and companies
should borrowers struggle to repay debts over the coming months.

``A lot of this seems to be about preserving options,'' said Daniel
Tarullo, a former Fed governor and the original architect of much of the
stress-testing regime who is now at Harvard. ``That's inconsistent with
the idea of acting early in response to a major shock.''

Lael Brainard, a Fed governor who was nominated and confirmed during the
Obama administration, objected to the fact that banks are still allowed
to pay out dividends in any fashion.

``The payouts will amount to a depletion of loss-absorbing capital,''
she wrote in a statement. ``This is inconsistent with the purpose of the
stress tests, which is to be forward-looking by preserving resilience,
not backward-looking by authorizing payouts based on net income from
past quarters that had already been paid out.''

Banks have been pushing the Fed to allow them to continue paying
dividends, worried that restricting the regular payouts will hit their
stock prices. But watchdog groups have been critical of the Fed's
leniency, pointing out that in the 2008 financial crisis, officials
allowed money to walk out the door by failing to curb payouts, worsening
the financial situation for struggling banks that ultimately failed.

For the largest banks, buybacks make up a bigger share of overall
capital distributions while dividends are a smaller chunk. Of the \$143
billion that the six biggest banks spent on capital distributions last
year, \$107 billion went to buybacks and \$36 billion to dividends.

Even without across-the-board dividend restrictions, the performance on
the normal stress tests could hamper some banks' ability to continue
payouts.

JPMorgan Chase, Citigroup, Bank of America and Wells Fargo, the four
largest banks in the United States, all came through the stress tests
with sufficient capital, according to a New York Times analysis of the
Fed's results. But capital at the fifth largest, Goldman Sachs, fell
slightly below the required level, according to the analysis. The result
could complicate any plans the Wall Street firm had for paying out
capital to its shareholders if it doesn't rise to the required amount by
late this year as part of a new regulatory framework.

The Fed's stress tests were introduced after the 2008 financial crisis
as a way of making sure regulators had an up-to-date grasp of the risks
in the banking system --- something they lacked before the housing
market crash. The exams focus on how much capital a bank would have left
after the different stress scenarios.

Capital is money banks don't have to pay back to creditors and
depositors. The more capital they have, the more losses they can
theoretically absorb.

Investors appear to have doubts about how the banks will perform in the
recession.

Bank stocks are down by nearly a third this year, compared with a 5
percent decline for the S\&P 500 stock index. Unlike in 2008, the banks
aren't the epicenter of the crisis and have actually experienced large
inflows of money rather than outflows. Lending to corporations soared in
the first quarter as companies drew on their credit lines and trading
revenue surged at banks with large Wall Street operations. But their
first-quarter earnings took a hit as they added billions of dollars to
their loan loss reserves in anticipation of widespread defaults.

The Fed's sensitivity analysis on Thursday suggested that banks could
see a 10.3 percent four-quarter loan loss rate if the economy
experiences a protracted downturn, and a 9.9 percent rate in the case of
a double-dip recession. Those figures exceed the rates experienced by
banks in the global financial crisis, when a smaller number of financial
institutions experienced loss rates of 6.8 percent.

Back then, an early version of the stress test estimated that loss rates
could total 9.1 percent over nine quarters.

That said, the sensitivity analyses were not full stress tests and did
have some limitations. For instance, they did not incorporate the
effects of government supports like the Paycheck Protection Program,
which has funneled money to small businesses to keep them paying bills
during widespread shutdowns.

Advertisement

\protect\hyperlink{after-bottom}{Continue reading the main story}

\hypertarget{site-index}{%
\subsection{Site Index}\label{site-index}}

\hypertarget{site-information-navigation}{%
\subsection{Site Information
Navigation}\label{site-information-navigation}}

\begin{itemize}
\tightlist
\item
  \href{https://help.nytimes3xbfgragh.onion/hc/en-us/articles/115014792127-Copyright-notice}{©~2020~The
  New York Times Company}
\end{itemize}

\begin{itemize}
\tightlist
\item
  \href{https://www.nytco.com/}{NYTCo}
\item
  \href{https://help.nytimes3xbfgragh.onion/hc/en-us/articles/115015385887-Contact-Us}{Contact
  Us}
\item
  \href{https://www.nytco.com/careers/}{Work with us}
\item
  \href{https://nytmediakit.com/}{Advertise}
\item
  \href{http://www.tbrandstudio.com/}{T Brand Studio}
\item
  \href{https://www.nytimes3xbfgragh.onion/privacy/cookie-policy\#how-do-i-manage-trackers}{Your
  Ad Choices}
\item
  \href{https://www.nytimes3xbfgragh.onion/privacy}{Privacy}
\item
  \href{https://help.nytimes3xbfgragh.onion/hc/en-us/articles/115014893428-Terms-of-service}{Terms
  of Service}
\item
  \href{https://help.nytimes3xbfgragh.onion/hc/en-us/articles/115014893968-Terms-of-sale}{Terms
  of Sale}
\item
  \href{https://spiderbites.nytimes3xbfgragh.onion}{Site Map}
\item
  \href{https://help.nytimes3xbfgragh.onion/hc/en-us}{Help}
\item
  \href{https://www.nytimes3xbfgragh.onion/subscription?campaignId=37WXW}{Subscriptions}
\end{itemize}
