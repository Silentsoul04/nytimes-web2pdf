Sections

SEARCH

\protect\hyperlink{site-content}{Skip to
content}\protect\hyperlink{site-index}{Skip to site index}

\href{https://myaccount.nytimes3xbfgragh.onion/auth/login?response_type=cookie\&client_id=vi}{}

\href{https://www.nytimes3xbfgragh.onion/section/todayspaper}{Today's
Paper}

\href{/section/opinion}{Opinion}\textbar{}America Is Facing 5 Epic
Crises All at Once

\url{https://nyti.ms/2Zet7UQ}

\begin{itemize}
\item
\item
\item
\item
\item
\item
\end{itemize}

Advertisement

\protect\hyperlink{after-top}{Continue reading the main story}

\href{/section/opinion}{Opinion}

Supported by

\protect\hyperlink{after-sponsor}{Continue reading the main story}

\hypertarget{america-is-facing-5-epic-crises-all-at-once}{%
\section{America Is Facing 5 Epic Crises All at
Once}\label{america-is-facing-5-epic-crises-all-at-once}}

This is not the time to obsess about symbolism.

\href{https://www.nytimes3xbfgragh.onion/by/david-brooks}{\includegraphics{https://static01.graylady3jvrrxbe.onion/images/2018/04/03/opinion/david-brooks/david-brooks-thumbLarge-v2.png}}

By \href{https://www.nytimes3xbfgragh.onion/by/david-brooks}{David
Brooks}

Opinion Columnist

\begin{itemize}
\item
  June 25, 2020
\item
  \begin{itemize}
  \item
  \item
  \item
  \item
  \item
  \item
  \end{itemize}
\end{itemize}

\includegraphics{https://static01.graylady3jvrrxbe.onion/images/2020/06/27/opinion/25brooksWeb/merlin_173727378_812b7d4e-3b86-4952-8daf-dc0aa7cd78e9-articleLarge.jpg?quality=75\&auto=webp\&disable=upscale}

There are five gigantic changes happening in America right now. The
first is that we are losing the fight against Covid-19. Our behavior
doesn't have anything to do with the reality around us. We just got
tired so we're giving up.

Second, all Americans, but especially white Americans, are undergoing a
rapid education on the burdens African-Americans carry every day. This
education is continuing, but already public opinion is shifting with
astonishing speed.

Third, we're in the middle of a political realignment. The American
public is vehemently rejecting Donald Trump's Republican Party. The most
telling sign is that the party has even given up on itself, a
personality cult whose cult leader is over.

Fourth, a quasi-religion is seeking control of America's cultural
institutions. The acolytes of this quasi-religion, Social Justice, hew
to a simplifying ideology: History is essentially a power struggle
between groups, some of which are oppressors and others of which are
oppressed. Viewpoints are not explorations of truth; they are weapons
that dominant groups use to maintain their place in the power structure.
Words can thus be a form of violence that has to be regulated.

Fifth, we could be on the verge of a prolonged economic depression.
State and household budgets are in meltdown, some businesses are failing
and many others are on the brink, the continuing health emergency will
mean economic activity cannot fully resume.

These five changes, each reflecting a huge crisis and hitting all at
once, have created a moral, spiritual and emotional disaster. Americans
are now
\href{https://www.norc.org/Research/Projects/Pages/covid-response-tracking-study.aspx}{less
happy} than at any time since they started measuring happiness nearly 50
years ago. Americans now express
\href{https://news.gallup.com/poll/312644/national-pride-falls-record-low.aspx}{less
pride} in their nation than at any time since Gallup started measuring
it 20 years ago.

Americans look around the world and see that other nations are beating
Covid-19 and we are failing. Americans look around and see
state-sponsored violence --- rhetorical and actual --- inflicted on
their fellow citizens. America doesn't seem very exceptional.

In times like this, you've got to have a theory of change.

The loudest theory of change is coming from the Social Justice movement.
This movement emerged from elite universities, and its basic premise is
that if you can change the cultural structures you can change society.

Members of this movement pay intense attention to cultural symbols ---
to language, statues, the names of buildings. They pay enormous
attention to repeating certain slogans, such as ``defund the police,''
which may or may not have anything to do with policy, and to lifting up
symbolic gestures, like kneeling before a football game. It's a very apt
method for change in an age of social media because it's very
performative.

The Social Justice activists focus on the cultural levers of power.
Their most talked about action is canceling people. Some person, usually
mildly progressive, will say something politically ``problematic'' and
his or her job will be
\href{https://nymag.com/intelligencer/2020/06/case-for-liberalism-tom-cotton-new-york-times-james-bennet.html}{terminated}.
In this way new boundaries are established for what has to be said and
what cannot be said.

The Social Justice activists sometimes claim that if you don't like
their tactics then you are not fighting for racial equity or economic
justice or whatever. But those movements all existed long before Social
Justice affixed itself to them and tried to change their methods.

The core problem is that the Social Justice theory of change doesn't
produce much actual change. Corporations are happy to adopt some woke
symbols and hold a few consciousness-raising seminars and go on their
merry way. Worse, this method has no theory of politics.

How exactly is all this cultural agitation going to lead to legislation
that will decrease income disparities, create better housing policies or
tackle the big challenges that I listed above? That part is never
spelled out. In fact, the Sturm und Drang makes political work harder.
You can't purify your way to a governing majority.

The Social Justice methodology is ultimately not a solution to our
problem, it's a symptom of our problem. Over the last half century,
we've turned politics from a practical way to solve common problems into
a cultural arena to display resentments. Donald Trump is the ultimate
performer in this paralyzed arena.

If you think the interplay of these five gigantic changes is going to
fit into some neat ideological narrative, you're probably wrong. If you
think we can deal with a racial disparity, reform militaristic police
departments and address an existential health crisis and a prolonged
economic depression by taking the culture war up another notch, I think
you're mistaken.

Dealing with these problems is going to take government. It's going to
take actual lawmaking, actual budgeting, complex compromises --- all the
boring, dogged work of government that is more C-SPAN than Instagram.

I know a lot of people aren't excited about him, but I thank God that
Joe Biden is going to be nominated by the Democratic Party. He came to
public life when it wasn't about performing your zeal, it was about
crafting coalitions and legislating. He exudes a spirit that is about
empathy and friendship not animosity and canceling. The pragmatic spirit
of the New Deal is a more apt guide for the years ahead than the spirit
of critical theory symbology.

\emph{The Times is committed to publishing}
\href{https://www.nytimes3xbfgragh.onion/2019/01/31/opinion/letters/letters-to-editor-new-york-times-women.html}{\emph{a
diversity of letters}} \emph{to the editor. We'd like to hear what you
think about this or any of our articles. Here are some}
\href{https://help.nytimes3xbfgragh.onion/hc/en-us/articles/115014925288-How-to-submit-a-letter-to-the-editor}{\emph{tips}}\emph{.
And here's our email:}
\href{mailto:letters@NYTimes.com}{\emph{letters@NYTimes.com}}\emph{.}

\emph{Follow The New York Times Opinion section on}
\href{https://www.facebookcorewwwi.onion/nytopinion}{\emph{Facebook}}\emph{,}
\href{http://twitter.com/NYTOpinion}{\emph{Twitter (@NYTopinion)}}
\emph{and}
\href{https://www.instagram.com/nytopinion/}{\emph{Instagram}}\emph{.}

Advertisement

\protect\hyperlink{after-bottom}{Continue reading the main story}

\hypertarget{site-index}{%
\subsection{Site Index}\label{site-index}}

\hypertarget{site-information-navigation}{%
\subsection{Site Information
Navigation}\label{site-information-navigation}}

\begin{itemize}
\tightlist
\item
  \href{https://help.nytimes3xbfgragh.onion/hc/en-us/articles/115014792127-Copyright-notice}{©~2020~The
  New York Times Company}
\end{itemize}

\begin{itemize}
\tightlist
\item
  \href{https://www.nytco.com/}{NYTCo}
\item
  \href{https://help.nytimes3xbfgragh.onion/hc/en-us/articles/115015385887-Contact-Us}{Contact
  Us}
\item
  \href{https://www.nytco.com/careers/}{Work with us}
\item
  \href{https://nytmediakit.com/}{Advertise}
\item
  \href{http://www.tbrandstudio.com/}{T Brand Studio}
\item
  \href{https://www.nytimes3xbfgragh.onion/privacy/cookie-policy\#how-do-i-manage-trackers}{Your
  Ad Choices}
\item
  \href{https://www.nytimes3xbfgragh.onion/privacy}{Privacy}
\item
  \href{https://help.nytimes3xbfgragh.onion/hc/en-us/articles/115014893428-Terms-of-service}{Terms
  of Service}
\item
  \href{https://help.nytimes3xbfgragh.onion/hc/en-us/articles/115014893968-Terms-of-sale}{Terms
  of Sale}
\item
  \href{https://spiderbites.nytimes3xbfgragh.onion}{Site Map}
\item
  \href{https://help.nytimes3xbfgragh.onion/hc/en-us}{Help}
\item
  \href{https://www.nytimes3xbfgragh.onion/subscription?campaignId=37WXW}{Subscriptions}
\end{itemize}
