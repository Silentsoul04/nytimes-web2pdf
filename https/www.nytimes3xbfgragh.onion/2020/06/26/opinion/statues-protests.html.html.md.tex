Sections

SEARCH

\protect\hyperlink{site-content}{Skip to
content}\protect\hyperlink{site-index}{Skip to site index}

\href{https://myaccount.nytimes3xbfgragh.onion/auth/login?response_type=cookie\&client_id=vi}{}

\href{https://www.nytimes3xbfgragh.onion/section/todayspaper}{Today's
Paper}

\href{/section/opinion}{Opinion}\textbar{}After the Statues Fall

\url{https://nyti.ms/3i2iTjf}

\begin{itemize}
\item
\item
\item
\item
\item
\item
\end{itemize}

Advertisement

\protect\hyperlink{after-top}{Continue reading the main story}

\href{/section/opinion}{Opinion}

Supported by

\protect\hyperlink{after-sponsor}{Continue reading the main story}

\hypertarget{after-the-statues-fall}{%
\section{After the Statues Fall}\label{after-the-statues-fall}}

Some deserve to be toppled. But monuments to those who sought to make
the union more perfect should stand.

\href{https://www.nytimes3xbfgragh.onion/by/bret-stephens}{\includegraphics{https://static01.graylady3jvrrxbe.onion/images/2017/08/27/insider/bretstephens/bretstephens-thumbLarge-v6.png}}

By \href{https://www.nytimes3xbfgragh.onion/by/bret-stephens}{Bret
Stephens}

Opinion Columnist

\begin{itemize}
\item
  June 26, 2020
\item
  \begin{itemize}
  \item
  \item
  \item
  \item
  \item
  \item
  \end{itemize}
\end{itemize}

\includegraphics{https://static01.graylady3jvrrxbe.onion/images/2020/06/26/opinion/26stephensWeb/26stephensWeb-articleLarge.jpg?quality=75\&auto=webp\&disable=upscale}

Regarding statues, monuments, and other public tributes to those once
deemed great --- which to do away with and which to keep --- four
familiar words can guide our choices: a more perfect union.

Did Jefferson Davis or Robert E. Lee fight for a more perfect union? No.
They fought for disunion. Outside of museums, grave sites, or private
collections, there should be no statues of either man, or of their
senior confederates.

Likewise, John C. Calhoun believed in slavery as a positive good and
nullification as a state's right. He utterly fails the more perfect
union test, which is why Yale was right when in 2017 it rechristened the
residential college previously named for him.

Likewise, Forts Bragg, Hood, Benning and seven other military
installations named for Confederate generals should be renamed. The
Constitution is specific in defining treason narrowly as ``levying war''
against the United States. It is dangerous for the government to name
buildings or facilities for those who betrayed it --- and incredible
that the fact escaped wide notice until now.

Likewise, we should never honor public figures who, by the standards of
their own time as well as ours, abused a public trust. Case in point:
The more we learn about J. Edgar Hoover, the more outrageous it is that
the F.B.I. building in Washington is
\href{https://www.wsj.com/articles/SB112182505647390371}{still named for
him}.

These are the easy cases. Equally easy are the opposite cases.

Hans Christian Heg, an ardent abolitionist whose statue in Madison,
Wis., was pulled down this week, fell at the Battle of Chickamauga
trying to make a more perfect union. Robert Gould Shaw, who commanded
one of the first black Union regiments and whose monument in Boston was
defaced last month, was killed at Fort Wagner trying to make a more
perfect union. Ulysses Grant, who did more than any other general to
defeat the Confederacy and more than any other president to defeat the
Klan, and whose statue in San Francisco was pulled down last week,
devoted his life to trying to make a more perfect union.

What about, say, Andrew Jackson, Teddy Roosevelt, Thomas Jefferson or
George Washington?

The central case against Washington and Jefferson is that they were
slaveholders, albeit ones who knew slavery was wrong. If their fault lay
in being creatures of their time, their greatness was in their ability
to look past it. An unbroken moral thread connects the Declaration of
Independence to the Gettysburg Address to Martin Luther King's ``I Have
A Dream'' speech. An unbroken political thread connects the first
president to the 16th to the 44th. It's impossible to imagine any union,
much less the possibility of a more perfect one, without them.

Jackson and Roosevelt? Both were avowed racists, and the Indian Removal
Act of 1830, enacted during Jackson's presidency, stands with the
Chinese Exclusion Act as one of the most shameful pieces of American
legislation.

But there's historical irony in the fact that some of today's
progressives are eager to bring down statues to the two most progressive
presidents of their times. Roosevelt busted trusts, championed
conservation, and caused a scandal by
\href{https://www.npr.org/2012/05/14/152684575/teddy-roosevelts-shocking-dinner-with-washington}{inviting
Booker T. Washington} to dine with his family in the White House. Were
those not acts in the service of a more perfect union?

As for Jackson, his egalitarianism, distrust of big money, and battle
with Calhoun over nullification make him much more the political
progenitor of Bernie Sanders than of Donald Trump. If it's OK to knock
Old Hickory off his pedestal now, is any reformist leader of the more
recent past --- F.D.R., for instance, or even Barack Obama --- safe from
the furies of the future? It's hard to build progressive politics on a
continually undermined foundation.

That isn't to say that every statue is worth preserving. New York's
Museum of Natural History just decided
\href{https://www.nytimes3xbfgragh.onion/2020/06/21/arts/design/roosevelt-statue-to-be-removed-from-museum-of-natural-history.html}{to
bring down the equestrian bronze} of the 26th president, not so much on
his account as because of the placement on his flanks of a Native
American and an African figure. Fine. But since the museum is largely
dedicated to Roosevelt's legacy as a statesman, scholar, and naturalist,
isn't the right way to do it to replace it with another T.R. statue ---
this time as a man in the arena rather than as a figure in the saddle?

Such a statue might be a useful reminder that the men and women who most
deserve to be shaped in metal or carved in stone weren't made from them.
And that acknowledging the fallibility of our national heroes and the
limitations of their time needn't make them less heroic and may often
make them more. And that there's a vast difference between thinking
critically about the past, for the sake of learning from it, and
behaving destructively toward the past, with the aim of erasing it.

A great debate about who should remain on which pedestals can be a
healthy one. The right's idea that we must preserve the worst figures to
protect the best is idiotic. The left's idea that we should bring down
the best because we know who they were at their worst is no less so. An
intelligent society should be able to make intelligent distinctions,
starting with the one between those who made our union more perfect and
those who made it less.

\emph{The Times is committed to publishing}
\href{https://www.nytimes3xbfgragh.onion/2019/01/31/opinion/letters/letters-to-editor-new-york-times-women.html}{\emph{a
diversity of letters}} \emph{to the editor. We'd like to hear what you
think about this or any of our articles. Here are some}
\href{https://help.nytimes3xbfgragh.onion/hc/en-us/articles/115014925288-How-to-submit-a-letter-to-the-editor}{\emph{tips}}\emph{.
And here's our email:}
\href{mailto:letters@NYTimes.com}{\emph{letters@NYTimes.com}}\emph{.}

\emph{Follow The New York Times Opinion section on}
\href{https://www.facebookcorewwwi.onion/nytopinion}{\emph{Facebook}}\emph{,}
\href{http://twitter.com/NYTOpinion}{\emph{Twitter (@NYTopinion)}}
\emph{and}
\href{https://www.instagram.com/nytopinion/}{\emph{Instagram}}\emph{.}

Advertisement

\protect\hyperlink{after-bottom}{Continue reading the main story}

\hypertarget{site-index}{%
\subsection{Site Index}\label{site-index}}

\hypertarget{site-information-navigation}{%
\subsection{Site Information
Navigation}\label{site-information-navigation}}

\begin{itemize}
\tightlist
\item
  \href{https://help.nytimes3xbfgragh.onion/hc/en-us/articles/115014792127-Copyright-notice}{©~2020~The
  New York Times Company}
\end{itemize}

\begin{itemize}
\tightlist
\item
  \href{https://www.nytco.com/}{NYTCo}
\item
  \href{https://help.nytimes3xbfgragh.onion/hc/en-us/articles/115015385887-Contact-Us}{Contact
  Us}
\item
  \href{https://www.nytco.com/careers/}{Work with us}
\item
  \href{https://nytmediakit.com/}{Advertise}
\item
  \href{http://www.tbrandstudio.com/}{T Brand Studio}
\item
  \href{https://www.nytimes3xbfgragh.onion/privacy/cookie-policy\#how-do-i-manage-trackers}{Your
  Ad Choices}
\item
  \href{https://www.nytimes3xbfgragh.onion/privacy}{Privacy}
\item
  \href{https://help.nytimes3xbfgragh.onion/hc/en-us/articles/115014893428-Terms-of-service}{Terms
  of Service}
\item
  \href{https://help.nytimes3xbfgragh.onion/hc/en-us/articles/115014893968-Terms-of-sale}{Terms
  of Sale}
\item
  \href{https://spiderbites.nytimes3xbfgragh.onion}{Site Map}
\item
  \href{https://help.nytimes3xbfgragh.onion/hc/en-us}{Help}
\item
  \href{https://www.nytimes3xbfgragh.onion/subscription?campaignId=37WXW}{Subscriptions}
\end{itemize}
