Sections

SEARCH

\protect\hyperlink{site-content}{Skip to
content}\protect\hyperlink{site-index}{Skip to site index}

\href{https://www.nytimes3xbfgragh.onion/section/world/australia}{Australia}

\href{https://myaccount.nytimes3xbfgragh.onion/auth/login?response_type=cookie\&client_id=vi}{}

\href{https://www.nytimes3xbfgragh.onion/section/todayspaper}{Today's
Paper}

\href{/section/world/australia}{Australia}\textbar{}Australian
Politician's Home Raided in Chinese Influence Inquiry

\url{https://nyti.ms/384V1GM}

\begin{itemize}
\item
\item
\item
\item
\item
\end{itemize}

Advertisement

\protect\hyperlink{after-top}{Continue reading the main story}

Supported by

\protect\hyperlink{after-sponsor}{Continue reading the main story}

\hypertarget{australian-politicians-home-raided-in-chinese-influence-inquiry}{%
\section{Australian Politician's Home Raided in Chinese Influence
Inquiry}\label{australian-politicians-home-raided-in-chinese-influence-inquiry}}

The case is the first high-profile criminal investigation of Chinese
influence peddling to be made public since Australia passed foreign
interference laws two years ago.

\includegraphics{https://static01.graylady3jvrrxbe.onion/images/2020/06/26/world/26oz-raid-1/merlin_173940033_e30a9733-4219-45ca-a8f9-a2e733e3cda3-articleLarge.jpg?quality=75\&auto=webp\&disable=upscale}

\href{https://www.nytimes3xbfgragh.onion/by/damien-cave}{\includegraphics{https://static01.graylady3jvrrxbe.onion/images/2018/10/08/multimedia/author-damien-cave/author-damien-cave-thumbLarge.png}}

By \href{https://www.nytimes3xbfgragh.onion/by/damien-cave}{Damien Cave}

\begin{itemize}
\item
  June 26, 2020
\item
  \begin{itemize}
  \item
  \item
  \item
  \item
  \item
  \end{itemize}
\end{itemize}

SYDNEY, Australia --- The Australian authorities raided the home and
office of a state lawmaker on Friday as part of a sweeping investigation
into allegations of a Chinese government plot to manipulate the
country's politics and policy.

Shaoquett Moselmane, a Labor politician from a Sydney suburb, recently
\href{https://www.smh.com.au/politics/nsw/unswerving-leadership-nsw-labor-mp-praises-china-s-coronavirus-response-20200330-p54fbg.html}{praised}
China's top leader, Xi Jinping, for his response to the coronavirus
pandemic, a message sharply at odds with the Australian government's
subsequent call for a global inquiry into the outbreak's origins.

Australia's national security agency confirmed that a search warrant had
been executed as part of an ongoing investigation, and Labor Party
leaders quickly announced that Mr. Moselmane's membership would be
suspended. The authorities have not made public any evidence of illegal
activities by Mr. Moselmane, who could not be reached for comment.

The case is the first high-profile criminal investigation of Chinese
influence peddling to be made public since Australia passed a suite of
foreign interference and espionage laws two years ago. The measures were
aimed directly at Beijing's attempt to shape the country's politics
through donations, promises and pressure on politicians at every level
of government.

The inquiry promises to further inflame tensions with Beijing, which
have been accelerating since Australia began pushing in April for a
fact-finding mission into the pandemic. China has bristled at criticism
of its handling of the virus since it first emerged in Wuhan.

Image

Mr. Moselmane has strongly praised China's response to the coronavirus.

Officials in Beijing have retaliated against Australia by slashing
agricultural imports and leveling threats of additional economic harm.
They have warned Chinese tourists to avoid Australia because of what
they described as racist mistreatment, and have issued similar guidance
to students not to study in the country. That is a potentially a huge
blow to Australia's universities, which rely on international students
for billions of dollars in revenue.

And last week, Prime Minister Scott Morrison announced that a
``sophisticated state-based cyberactor'' --- widely believed to be China
--- was
\href{https://www.abc.net.au/news/2020-06-19/cyber-attacks-likely-huawei-5g-ban-payback-from-china-spy-agency/12374374}{attacking
a wide range of Australian government entities}.

Peter Jennings, the executive director of the Australian Strategic
Policy Institute, a nonpartisan think tank, called the inquiry involving
Mr. Moselmane a test case for prosecutors and the country's more
aggressive effort to stand up to China after years of accommodation and
unfettered trade.

``It's just another signpost on the journey,'' Mr. Jennings said. ``What
we've seen over a few years now is a deterioration in the relationship
with China, which has largely come with the Chinese government pushing
the limits of what is acceptable behavior on a range of fronts and
Australian authorities pushing back, which China doesn't like.''

The investigation tied to the raids on Friday, which were
\href{https://www.smh.com.au/politics/nsw/nsw-labor-mp-shaoquett-moselmane-to-be-suspended-after-raids-probing-china-links-20200626-p556gd.html}{first
reported} by The Sydney Morning Herald, most likely began before the
pandemic. One former official said the case had been building for
months. Mr. Jennings, who worked for more than a decade in senior
defense and intelligence roles, said it was probably one of many
investigations.

Under the mandate of the new laws, which broadened the definition of
espionage, he said the authorities would be exploring whether covert
activities or agents directed by Beijing had sought to influence
Australian politicians or their employees, and if they did, whether
those involved were aware of the Chinese government's efforts.

``From the moment the new legislation appeared, I anticipated that our
intelligence agencies would be looking for a case to test the laws out
on,'' Mr. Jennings said. He added: ``This is possibly one of the most
egregious ones, but there is a system at play that affects more than one
person. There is a network of connections that now I think will be drawn
out.''

The Chinese government has long treated Australia as a petri dish for
experiments in influence, and
\href{https://www.nytimes3xbfgragh.onion/2019/05/20/world/australia/australia-china.html}{its
playbook} is relatively well known.

Chinese consulates and other agencies tend to work closely with civic
organizations in Australia that are tied to the
\href{http://www.zytzb.gov.cn/}{United Front Work Department}, the
party's arm for dealing with overseas Chinese. Leaders of these
organizations frequently mix cultural events with politics and donations
to political parties and candidates considered friendly to Beijing, or
at least open to influence. Chinese-language media outlets directly or
indirectly controlled by Beijing promote those they favor, and condemn
those who resist.

The influence efforts led to the downfall of a promising Labor
politician in 2017. The lawmaker, Sam Dastyari, was an up-and-comer
known for his fund-raising when he
\href{https://www.nytimes3xbfgragh.onion/2017/12/12/world/australia/sam-dastyari-resigns-china.html}{resigned}
amid accusations that he had pushed China's foreign policy interests
after taking money from Chinese-born political donors.

\includegraphics{https://static01.graylady3jvrrxbe.onion/images/2020/06/26/world/26oz-raid-3/merlin_130695969_4c14162a-b630-4f67-82a4-fceeb158d711-articleLarge.jpg?quality=75\&auto=webp\&disable=upscale}

Mr. Moselmane's known ties to the Chinese government appear to be
related more to personnel and travel. He has taken several privately
funded trips to China over the past decade, with records showing costs
covered by Chinese government officials or agencies.

He also came under scrutiny for hiring a part-time staff member, John
Zhang, who has been linked by Chinese websites to a propaganda training
course run by the Overseas Chinese Affairs Office, which is believed to
be part of the United Front.

Mr. Moselmane has been especially bold in his pro-China pronouncements.
Earlier this year, he praised Mr. Xi's ``unswerving leadership'' in
handling the coronavirus crisis. At a 2018 event in the New South Wales
Parliament, he declared: ``The only way for China to reach its potential
is for China to force a change to the rules and create a new world
order.''

Some Australian officials, including George Brandis, Australia's top
diplomat in London who drafted the foreign interference laws when he was
attorney general, have argued that the country's new measures put it at
the forefront of the global effort to rein in China's attempted
violations of sovereignty and democratic principles.

But regardless of the current case's outcome, some international
analysts are less certain that any legal consequences will alter the
calculations or actions of a rising superpower determined to bend the
world to its will.

``If the case is prosecuted and leads to a verdict against the Chinese
United Front organization or officials, it will certainly be
embarrassing for Beijing,'' said Bonnie Glaser, director of the China
Power Project at the Center for Strategic and International Studies.

``I doubt that it would result in a radical change in such efforts,
however,'' she added. ``They might alter their tactics, but the UF
operations are essential for promoting Chinese interests and won't be
easily abandoned.''

Advertisement

\protect\hyperlink{after-bottom}{Continue reading the main story}

\hypertarget{site-index}{%
\subsection{Site Index}\label{site-index}}

\hypertarget{site-information-navigation}{%
\subsection{Site Information
Navigation}\label{site-information-navigation}}

\begin{itemize}
\tightlist
\item
  \href{https://help.nytimes3xbfgragh.onion/hc/en-us/articles/115014792127-Copyright-notice}{©~2020~The
  New York Times Company}
\end{itemize}

\begin{itemize}
\tightlist
\item
  \href{https://www.nytco.com/}{NYTCo}
\item
  \href{https://help.nytimes3xbfgragh.onion/hc/en-us/articles/115015385887-Contact-Us}{Contact
  Us}
\item
  \href{https://www.nytco.com/careers/}{Work with us}
\item
  \href{https://nytmediakit.com/}{Advertise}
\item
  \href{http://www.tbrandstudio.com/}{T Brand Studio}
\item
  \href{https://www.nytimes3xbfgragh.onion/privacy/cookie-policy\#how-do-i-manage-trackers}{Your
  Ad Choices}
\item
  \href{https://www.nytimes3xbfgragh.onion/privacy}{Privacy}
\item
  \href{https://help.nytimes3xbfgragh.onion/hc/en-us/articles/115014893428-Terms-of-service}{Terms
  of Service}
\item
  \href{https://help.nytimes3xbfgragh.onion/hc/en-us/articles/115014893968-Terms-of-sale}{Terms
  of Sale}
\item
  \href{https://spiderbites.nytimes3xbfgragh.onion}{Site Map}
\item
  \href{https://help.nytimes3xbfgragh.onion/hc/en-us}{Help}
\item
  \href{https://www.nytimes3xbfgragh.onion/subscription?campaignId=37WXW}{Subscriptions}
\end{itemize}
