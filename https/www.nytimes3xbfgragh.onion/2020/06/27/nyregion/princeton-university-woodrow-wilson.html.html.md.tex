Sections

SEARCH

\protect\hyperlink{site-content}{Skip to
content}\protect\hyperlink{site-index}{Skip to site index}

\href{https://www.nytimes3xbfgragh.onion/section/nyregion}{New York}

\href{https://myaccount.nytimes3xbfgragh.onion/auth/login?response_type=cookie\&client_id=vi}{}

\href{https://www.nytimes3xbfgragh.onion/section/todayspaper}{Today's
Paper}

\href{/section/nyregion}{New York}\textbar{}Princeton Will Remove
Woodrow Wilson's Name From School

\url{https://nyti.ms/3ganazn}

\begin{itemize}
\item
\item
\item
\item
\item
\end{itemize}

Advertisement

\protect\hyperlink{after-top}{Continue reading the main story}

Supported by

\protect\hyperlink{after-sponsor}{Continue reading the main story}

\hypertarget{princeton-will-remove-woodrow-wilsons-name-from-school}{%
\section{Princeton Will Remove Woodrow Wilson's Name From
School}\label{princeton-will-remove-woodrow-wilsons-name-from-school}}

University trustees concluded that Wilson's ``racist thinking and
policies make him an inappropriate namesake for a school or college,''
Princeton's president said on Saturday.

\includegraphics{https://static01.graylady3jvrrxbe.onion/images/2020/06/28/multimedia/27xp-unrest-princeton1-print/merlin_136092333_6c6d600f-51df-43d0-9a7c-6ade3082ed7d-articleLarge.jpg?quality=75\&auto=webp\&disable=upscale}

By Bryan Pietsch

\begin{itemize}
\item
  June 27, 2020
\item
  \begin{itemize}
  \item
  \item
  \item
  \item
  \item
  \end{itemize}
\end{itemize}

Princeton University will remove Woodrow Wilson's name from its public
policy school and one of its residential colleges, the university's
president said on Saturday --- a move that comes four years after it
decided to keep the name over the objections of student protests.

The university's board of trustees found that Wilson's ``racist thinking
and policies make him an inappropriate namesake for a school or college
whose scholars, students and alumni must stand firmly against racism in
all its forms,'' Princeton's president, Christopher L. Eisgruber, said
in a
\href{https://www.princeton.edu/news/2020/06/27/president-eisgrubers-message-community-removal-woodrow-wilson-name-public-policy}{statement}.

``Wilson's racism was significant and consequential even by the
standards of his own time,'' Mr. Eisgruber said. Wilson was the
university's president from 1902 to 1910 before becoming the U.S.
president in 1913.

Wilson had overseen the resegregation of federal government offices,
including the Treasury Department. In a meeting in the Oval Office with
the civil rights leader Monroe Trotter, Wilson said, ``Segregation is
not humiliating, but a benefit, and ought to be so regarded by you
gentlemen,'' \href{http://historymatters.gmu.edu/d/5719/}{according to a
transcript} of the meeting.

Monmouth University in New Jersey said last week that
\href{https://www.nytimes3xbfgragh.onion/2020/06/21/nyregion/monmouth-university-woodrow-wilson.html}{it
would remove} Wilson's name from its marquee building after
administrators, professors and students said that the former president
held abhorrent views on race and reinstituted segregation in the federal
work force.

The decision contrasted with a vote by Princeton trustees in 2016 to
\href{https://www.nytimes3xbfgragh.onion/2016/04/05/nyregion/princeton-board-votes-to-keep-woodrow-wilsons-name-on-campus-buildings.html}{keep
Wilson's name} on campus buildings and programs, despite student
protests that led to a review of his legacy there.

The university's trustees said in a
\href{https://www.princeton.edu/news/2020/06/27/board-trustees-decision-removing-woodrow-wilsons-name-public-policy-school-and}{statement}
that it had questioned whether it was appropriate to name a school for
``a racist who segregated the nation's Civil Service after it had been
integrated for decades.''

``The question has been made more urgent by the recent killings of
Breonna Taylor, Ahmaud Arbery, George Floyd and Rayshard Brooks, which
have served as tragic reminders of the ongoing need for all of us to
stand against racism and for equality and justice,'' the statement
continued.

\includegraphics{https://static01.graylady3jvrrxbe.onion/images/2020/06/28/multimedia/27xp-unrest-princeton2-print/merlin_101852425_948b2623-8c10-469b-84e9-95cb38615c24-articleLarge.jpg?quality=75\&auto=webp\&disable=upscale}

Students in the Woodrow Wilson School of Public and International
Affairs sent
\href{https://drive.google.com/file/d/1lpxEuLY_5rV4VTwC1Ly8sa2r77XuMVEE/view}{a
letter} dated June 22 to the university's administration asking for the
school's name to be changed, among a list of other demands.

Renaming the school is ``the most basic step the university could have
taken,'' said Ally McGowen, a rising senior at the public policy school,
who is black. The students behind the letter said the university had not
consulted with them before its announcement on Saturday.

``This is more than a name,'' Ms. McGowen said. The students had asked
that the university underwrite research into reparations and that the
public policy school's faculty and curriculum be diversified. The
students noted their demands are ``nothing new,'' having been raised in
2015 by students in the Black Justice League at the university.

The discomfort over the school's name was widespread, said Ananya
Agustin Malhotra, a 2020 graduate of the policy school.

Princeton had already planned to retire the former president's name from
Wilson College, a residential community of about 1,000 students that
includes dormitories, dining facilities and extracurricular programming.
But rather than asking students ``to identify with the name of a racist
president for the next two years,'' Princeton will ``accelerate'' the
retiring of the name, Mr. Eisgruber said.

The community will be renamed First College, acknowledging its history
as one of the first residential colleges at Princeton, AnneMarie
Luijendijk, head of the college and a professor of religion, said in
\href{https://wilsoncollege.princeton.edu/news/we-are-now-first-college}{a
note} to students.

Professor Luijendijk said she ``always found it hard'' to ask students
to ``cheer for Wilson'' during intramural sporting events. The renaming
is an important step, she said, adding that students often struggled
with the ``constant reminder'' of Wilson's actions.

Julia Chaffers, a black Princeton student who resided in the college,
which is not affiliated with the Pennsylvania liberal arts college of
the same name, wrote in a 2018
\href{https://www.dailyprincetonian.com/article/2018/12/what-it-means-to-live-in-wilson}{opinion
piece} for the university's newspaper that upon entering Princeton, she
``felt a mix of excitement for the coming adventures but also a
discomfort with the name I would now be adopting as my home.''

``To name a residential college after Wilson, while ignoring the fact
that he did not believe white and black people belonged on equal terms
in the same spaces, is ridiculous,'' Ms. Chaffers wrote.

Residential colleges at Princeton are ``really central to your identity
on campus,'' especially as a freshman, Ms. Chaffers, who is a rising
junior, said in an interview on Saturday.

Introductions at the university would often begin by asking about the
college someone was in, and identifying Wilson College could be ``really
challenging for black students,'' she said.

Ms. Chaffers said the renaming is a ``really important turning point''
in recognizing other changes that need to be addressed at Princeton.
``But it's not the end of the road,'' she added.

Advertisement

\protect\hyperlink{after-bottom}{Continue reading the main story}

\hypertarget{site-index}{%
\subsection{Site Index}\label{site-index}}

\hypertarget{site-information-navigation}{%
\subsection{Site Information
Navigation}\label{site-information-navigation}}

\begin{itemize}
\tightlist
\item
  \href{https://help.nytimes3xbfgragh.onion/hc/en-us/articles/115014792127-Copyright-notice}{©~2020~The
  New York Times Company}
\end{itemize}

\begin{itemize}
\tightlist
\item
  \href{https://www.nytco.com/}{NYTCo}
\item
  \href{https://help.nytimes3xbfgragh.onion/hc/en-us/articles/115015385887-Contact-Us}{Contact
  Us}
\item
  \href{https://www.nytco.com/careers/}{Work with us}
\item
  \href{https://nytmediakit.com/}{Advertise}
\item
  \href{http://www.tbrandstudio.com/}{T Brand Studio}
\item
  \href{https://www.nytimes3xbfgragh.onion/privacy/cookie-policy\#how-do-i-manage-trackers}{Your
  Ad Choices}
\item
  \href{https://www.nytimes3xbfgragh.onion/privacy}{Privacy}
\item
  \href{https://help.nytimes3xbfgragh.onion/hc/en-us/articles/115014893428-Terms-of-service}{Terms
  of Service}
\item
  \href{https://help.nytimes3xbfgragh.onion/hc/en-us/articles/115014893968-Terms-of-sale}{Terms
  of Sale}
\item
  \href{https://spiderbites.nytimes3xbfgragh.onion}{Site Map}
\item
  \href{https://help.nytimes3xbfgragh.onion/hc/en-us}{Help}
\item
  \href{https://www.nytimes3xbfgragh.onion/subscription?campaignId=37WXW}{Subscriptions}
\end{itemize}
