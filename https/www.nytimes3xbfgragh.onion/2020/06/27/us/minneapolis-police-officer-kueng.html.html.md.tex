Sections

SEARCH

\protect\hyperlink{site-content}{Skip to
content}\protect\hyperlink{site-index}{Skip to site index}

\href{https://www.nytimes3xbfgragh.onion/section/us}{U.S.}

\href{https://myaccount.nytimes3xbfgragh.onion/auth/login?response_type=cookie\&client_id=vi}{}

\href{https://www.nytimes3xbfgragh.onion/section/todayspaper}{Today's
Paper}

\href{/section/us}{U.S.}\textbar{}The Black Officer Who Detained George
Floyd Had Pledged to Fix the Police

\url{https://nyti.ms/383XPEa}

\begin{itemize}
\item
\item
\item
\item
\item
\item
\end{itemize}

\href{https://www.nytimes3xbfgragh.onion/news-event/george-floyd-protests-minneapolis-new-york-los-angeles?action=click\&pgtype=Article\&state=default\&region=TOP_BANNER\&context=storylines_menu}{Race
and America}

\begin{itemize}
\tightlist
\item
  \href{https://www.nytimes3xbfgragh.onion/2020/07/26/us/protests-portland-seattle-trump.html?action=click\&pgtype=Article\&state=default\&region=TOP_BANNER\&context=storylines_menu}{Protesters
  Return to Other Cities}
\item
  \href{https://www.nytimes3xbfgragh.onion/2020/07/24/us/portland-oregon-protests-white-race.html?action=click\&pgtype=Article\&state=default\&region=TOP_BANNER\&context=storylines_menu}{Portland
  at the Center}
\item
  \href{https://www.nytimes3xbfgragh.onion/2020/07/23/podcasts/the-daily/portland-protests.html?action=click\&pgtype=Article\&state=default\&region=TOP_BANNER\&context=storylines_menu}{Podcast:
  Showdown in Portland}
\item
  \href{https://www.nytimes3xbfgragh.onion/interactive/2020/07/16/us/black-lives-matter-protests-louisville-breonna-taylor.html?action=click\&pgtype=Article\&state=default\&region=TOP_BANNER\&context=storylines_menu}{45
  Days in Louisville}
\end{itemize}

Advertisement

\protect\hyperlink{after-top}{Continue reading the main story}

Supported by

\protect\hyperlink{after-sponsor}{Continue reading the main story}

\hypertarget{the-black-officer-who-detained-george-floyd-had-pledged-to-fix-the-police}{%
\section{The Black Officer Who Detained George Floyd Had Pledged to Fix
the
Police}\label{the-black-officer-who-detained-george-floyd-had-pledged-to-fix-the-police}}

Alex Kueng is one of four former officers accused of crimes in the
killing of Mr. Floyd, which happened on his third shift. His decision to
join the force had frayed friendships.

\includegraphics{https://static01.graylady3jvrrxbe.onion/images/2020/06/28/us/28unrest-kueng-p1/merlin_173379090_b4abd1d3-c5f5-4f72-9a99-d2cb3d676c46-articleLarge.jpg?quality=75\&auto=webp\&disable=upscale}

\href{https://www.nytimes3xbfgragh.onion/by/kim-barker}{\includegraphics{https://static01.graylady3jvrrxbe.onion/images/2018/06/13/multimedia/author-kim-barker/author-kim-barker-thumbLarge.jpg}}

By \href{https://www.nytimes3xbfgragh.onion/by/kim-barker}{Kim Barker}

\begin{itemize}
\item
  June 27, 2020
\item
  \begin{itemize}
  \item
  \item
  \item
  \item
  \item
  \item
  \end{itemize}
\end{itemize}

MINNEAPOLIS --- There were two black men at the scene of the police
killing in Minneapolis last month that roiled the nation. One, George
Floyd, was sprawled on the asphalt, with a white officer's knee on his
neck. The other black man, Alex Kueng, was a rookie police officer who
held his back as Mr. Floyd struggled to breathe.

Mr. Floyd, whose name has been painted on murals and scrawled on protest
signs, has been laid to rest. Mr. Kueng, who faces charges of aiding and
abetting in Mr. Floyd's death, is out on bail,
\href{https://twitter.com/worldstar/status/1274746480486764544?s=21}{hounded
at the supermarket by strangers} and denounced by some family members.

Long before Mr. Kueng was arrested, he had wrestled with the issue of
police abuse of black people, joining the force in part to help protect
people close to him from police aggression. He argued that diversity
could force change in a Police Department long accused of racism.

He had seen one sibling arrested and treated poorly, in his view, by
sheriff's deputies. He had found himself defending his decision to join
the police force, saying he thought it was the best way to fix a broken
system. He had clashed with friends over whether public demonstrations
could actually make things better.

``He said, `Don't you think that that needs to be done from the
inside?''' his mother, Joni Kueng, recalled him saying after he watched
protesters block a highway years ago. ``That's part of the reason why he
wanted to become a police officer --- and a black police officer on top
of it --- is to bridge that gap in the community, change the narrative
between the officers and the black community.''

As hundreds of thousands of people demonstrated against the police after
Mr. Floyd's killing on May 25, Mr. Kueng became part of a national
debate over police violence toward black people, a symbol of the very
sort of policing he had long said he wanted to stop.

Derek Chauvin, the officer who placed his knee on Mr. Floyd's neck for
more than eight minutes, has been most widely associated with the case.
He faces charges of second-degree murder and second-degree manslaughter;
Mr. Kueng and two other former officers were charged with aiding and
abetting the killing. At 26, Mr. Kueng was the youngest and least
experienced officer at the scene, on only his third shift as a full
officer.

The arrest of Mr. Kueng, whose mother is white and whose father was from
Nigeria, has brought anguish to his friends and family. ``It's a gut
punch,'' Ms. Kueng said. ``Here you are, you've raised this child, you
know who he is inside and out. We're such a racially diverse family. To
be wrapped up in a racially motivated incident like this is just
unfathomable.''

Two of Mr. Kueng's siblings, Taylor and Radiance, both of whom are
African-American, called for the arrests of all four officers, including
their brother. They joined protests in Minneapolis.

In a
\href{https://www.facebookcorewwwi.onion/watch/live/?v=2793585684211644\&ref=watch_permalink}{Facebook
Live video}, Taylor Kueng, 21, appeared with the head of the local
N.A.A.C.P. to speak of the injustice that befell Mr. Floyd,
acknowledging being related to Mr. Kueng but never mentioning his name.

Mr. Kueng's sister Radiance posted a video of Mr. Floyd's final minutes
on Facebook. ``Just broke my heart,'' she wrote. In an interview, she
said that as a black man, her brother should have intervened. She said
she planned to change her last name in part because she did not want to
be associated with her brother's actions.

``I don't care if it was his third day at work or not,'' she said. ``He
knows right from wrong.''

\includegraphics{https://static01.graylady3jvrrxbe.onion/images/2020/06/28/us/28unrest-kueng-print2/merlin_173174286_592c560d-5fc7-4890-9fd1-25bbd2e333e0-articleLarge.jpg?quality=75\&auto=webp\&disable=upscale}

\hypertarget{a-full-house}{%
\subsection{A Full House}\label{a-full-house}}

Through his life, Mr. Kueng straddled two worlds, black and white.

Mr. Kueng, whose full name is J. Alexander Kueng (pronounced ``king''),
was raised by his mother, whom he lived with until last year. His father
was absent.

As a child, Mr. Kueng sometimes asked for siblings. Joni Kueng, who
lived in the Shingle Creek neighborhood in north Minneapolis, signed up
with an African-American adoption agency.

When Alex was 5, Ms. Kueng brought home a baby boy who had been
abandoned at a hospital. Alex soon asked for a sister; Radiance arrived
when he was 11. Taylor and a younger brother came in 2009, when Alex was
about 16.

Radiance Kueng, 21, said their adoptive mother did not talk about race.
``Race was not really a topic in our household, unfortunately,'' she
said. ``For her adopting as many black kids as she did --- I didn't get
that conversation from her. I feel like that should have been a
conversation that was had.''

Growing up, Mr. Kueng and his family made repeated trips to Haiti,
helping at an orphanage. Mr. Kueng and his siblings took a break from
school to volunteer there after the earthquake in 2010.

Joni Kueng, 56, likes to say that the Kuengs are a family of doers, not
talkers.

``I had to stay out of the race conversations because I was the minority
in the household,'' Ms. Kueng said in her first interview since her
son's arrest. She said that race was not an issue with her, but that she
was conflicted. ``It didn't really matter, but it does matter to them
because they are African-American. And so they had to be able to have an
outlet to tell their stories and their experience as well, especially
having a white mom.''

Ms. Kueng taught math at the schools her children went to, where the
student body was often mostly Hmong, African-American and Latino.
Classmates described Alex Kueng as friends with everyone, a master of
juggling a soccer ball and a defender against bullies. Photos portray
him with a sly smile.

Darrow Jones said he first met Mr. Kueng on the playground when he was
6. Mr. Jones was trying to finish his multiplication homework. Mr. Kueng
helped Mr. Jones and then invited him into a game of tag.

When Mr. Jones's mother died in 2008, Ms. Kueng took him in for as long
as a month at a time.

By high school, Mr. Kueng had found soccer, and soon that was all he
wanted to do. He became captain of the soccer team; he wanted to turn
pro. The quote next to his senior yearbook picture proclaimed, ``We
ignore failures and strive for success.''

Mr. Kueng went to Monroe College in New Rochelle, N.Y., to play soccer
and study business. But after surgery on both knees, soccer proved
impossible. Mr. Kueng quit. Back in Minneapolis, he enrolled in
technical college and supported himself catching shoplifters at Macy's.

About that time, he started talking about joining the police, Ms. Kueng
recalled. She said she was nervous, for his safety and also because of
the troubled relationship between the Minneapolis police and residents.

Given his background, Mr. Kueng thought he had the ability to bridge the
gap between white and black worlds, Mr. Jones said. He often did not see
the same level of racism that friends felt. Mr. Jones, who is black,
recalled a road trip a few years ago to Utah with Mr. Kueng, a white
friend and Mr. Kueng's girlfriend, who is Hmong. Mr. Jones said he had
to explain to Mr. Kueng why people were staring at the group.

``Once we got to Utah, we walked into a store, and literally everybody's
eyes were on us,'' recalled Mr. Jones, whose skin is darker than Mr.
Kueng's. ``I said, `Alex, that's because you're walking in here with a
black person. The reason they're staring at us is because you're here
with me.'''

By February 2019, Mr. Kueng had made up his mind: He signed up as a
police cadet.

Only a few months later, his sibling Taylor, a longtime supporter of
Black Lives Matter who had volunteered as a counselor at a black
heritage camp and as a mentor to at-risk black youths, had a
confrontation with law enforcement.

Taylor Kueng and a friend saw local sheriff's deputies questioning two
men in a downtown Minneapolis shopping district about drinking in
public. They intervened. Taylor Kueng used a cellphone
\href{https://kstp.com/news/naacp-hennepin-county-sheriff-investigate-arrest-caught-on-video-tape/5390358/}{to
record video of the deputies} putting the friend, in a striped summer
dress, on the ground. ``You're hurting me!'' the friend shouted.

As the confrontation continued, a deputy turned to Taylor Kueng and
said, ``Put your hands behind your back.'' ``For what?'' Taylor Kueng
asked several times. ``Because,'' said the deputy, threatening to use
his Taser.

Taylor Kueng called home. Mr. Kueng and their mother rushed to get bail
and then to the jail. ``Don't worry, I got you,'' Mr. Kueng told his
sibling, hugging Taylor, their mother recalled.

Mr. Kueng reminded his sibling that those were sheriff's deputies, not
the city force he was joining, and criticized their behavior, his mother
recalled.

After Taylor Kueng's video went public, the city dropped the misdemeanor
charges of disorderly conduct and obstructing the legal process. The
sheriff's office announced an official review of the arrests, which
resulted in no discipline.

Image

Ms. Kueng said she was nervous when her son wanted to become a police
officer because of concern for his safety and the troubled relationship
between the Minneapolis police and residents.Credit...Victor J. Blue for
The New York Times

\hypertarget{diverging-paths}{%
\subsection{Diverging Paths}\label{diverging-paths}}

Mr. Kueng's choice to become a police officer caused a rift in his
friendship with Mr. Jones.

``It was very clear where we stood on that,'' said Mr. Jones, a Black
Lives Matter supporter who protested on the streets after the deaths of
\href{https://www.nytimes3xbfgragh.onion/2015/11/20/us/minneapolis-protest-police-jamar-clark.html}{Jamar
Clark} and
\href{https://www.nytimes3xbfgragh.onion/2016/07/13/us/philando-castile-minnesota-police-shooting.html}{Philando
Castile} at the hands of Minneapolis-area police. ``Our fundamental
disagreement around law enforcement is not that I believe cops are bad
people. I just believe that the system needs to be completely wiped out
and replaced. It's the difference between reform and rebuilding.''

After Mr. Kueng became a cadet, Mr. Jones went from seeing Mr. Kueng
twice a month to maybe three times a year. He said he did not even tell
Mr. Kueng when the police pursued him for nothing and then let him go.

In December, Mr. Kueng graduated from the police academy. For most of
his field training, Mr. Chauvin, with 19 years on the job, was his
training officer.

At one point, Mr. Kueng, upset, called his mother. He said he had done
something during training that bothered a supervising officer, who
reamed him out. Ms. Kueng did not know if that supervisor was Mr.
Chauvin.

Mr. Chauvin also extended Mr. Kueng's training period. He felt Mr. Kueng
was meeting too often with a fellow police trainee, Thomas Lane, when
responding to calls, rather than handling the calls on his own, Ms.
Kueng said.

But on May 22, Mr. Kueng officially became one of about 80 black
officers on a police force of almost 900. In recent years, the
department, not as racially diverse as the city's population, has
\href{https://www.npr.org/2014/09/03/345308385/minneapolis-pd-makes-an-effort-to-hire-more-minority-officers}{tried
to increase the number of officers of color}, with limited success.

That evening, other officers held a small party at the Third Precinct
station to celebrate Mr. Kueng's promotion. The next evening, he worked
his first full shift as an officer, inside the station. On that Sunday,
he worked the 4 p.m. to 2 a.m. patrol shift, his first on the streets.

On May 25, Mr. Kueng's third day on the job, Mr. Kueng and Mr. Lane, now
partnered up despite both being freshly minted rookies, were the first
officers to answer a call of a counterfeit \$20 bill being passed at a
corner store. They found Mr. Floyd in a car outside.

After they failed to get Mr. Floyd into the back of a squad car, Mr.
Chauvin and Tou Thao, another officer, showed up.

As Mr. Chauvin jammed his knee into the back of Mr. Floyd's neck, Mr.
Kueng held down Mr. Floyd's back,
\href{https://www.ag.state.mn.us/Office/Communications/2020/docs/Complaint_kueng.pdf}{according
to a probable cause statement} filed by prosecutors.

Mr. Chauvin kept his knee there as Mr. Floyd repeated ``I can't
breathe'' and ``mama'' and ``please.'' Through the passing minutes, Mr.
Kueng did nothing to intervene, prosecutors say. After Mr. Floyd stopped
moving, Mr. Kueng checked Mr. Floyd's pulse. ``I couldn't find one,''
Mr. Kueng told the other officers.

Critics of the police said the fact that none of the junior officers
stopped Mr. Chauvin showed that the system itself needed to be
overhauled.

``How do you as an individual think that you're going to be able to
change that system, especially when you're going in at a low level?''
said Michelle Gross, president of Communities United Against Police
Brutality in Minneapolis. ``You're not going to feel OK to say, `Stop,
senior officer.' The culture is such, that that kind of intervening
would be greatly discouraged.''

All four officers have been fired. All four face 40 years in prison. Mr.
Kueng, who was released on bail on June 19, declined through his lawyer
to be interviewed. He is set to appear in court on Monday.

A day after Mr. Floyd's death, Mr. Jones learned that Mr. Kueng was one
of the officers who had been present. Around midnight, Mr. Jones called
Mr. Kueng. They talked for 40 minutes --- about what, Mr. Jones would
not say --- and they cried.

``I'm feeling a lot of sadness and a lot of disappointment,'' Mr. Jones
said. ``A lot of us believe he should have stepped in and should have
done something.''

He added: ``It's really hard. Because I do have those feelings and I
won't say I don't. But though I feel sad about what's occurred, he still
has my unwavering support. Because we grew up together, and I love
him.''

Mr. Jones said he had gone to the protests but could not bring himself
to join in.

Eric Killelea contributed reporting. Susan C. Beachy contributed
research.

Advertisement

\protect\hyperlink{after-bottom}{Continue reading the main story}

\hypertarget{site-index}{%
\subsection{Site Index}\label{site-index}}

\hypertarget{site-information-navigation}{%
\subsection{Site Information
Navigation}\label{site-information-navigation}}

\begin{itemize}
\tightlist
\item
  \href{https://help.nytimes3xbfgragh.onion/hc/en-us/articles/115014792127-Copyright-notice}{©~2020~The
  New York Times Company}
\end{itemize}

\begin{itemize}
\tightlist
\item
  \href{https://www.nytco.com/}{NYTCo}
\item
  \href{https://help.nytimes3xbfgragh.onion/hc/en-us/articles/115015385887-Contact-Us}{Contact
  Us}
\item
  \href{https://www.nytco.com/careers/}{Work with us}
\item
  \href{https://nytmediakit.com/}{Advertise}
\item
  \href{http://www.tbrandstudio.com/}{T Brand Studio}
\item
  \href{https://www.nytimes3xbfgragh.onion/privacy/cookie-policy\#how-do-i-manage-trackers}{Your
  Ad Choices}
\item
  \href{https://www.nytimes3xbfgragh.onion/privacy}{Privacy}
\item
  \href{https://help.nytimes3xbfgragh.onion/hc/en-us/articles/115014893428-Terms-of-service}{Terms
  of Service}
\item
  \href{https://help.nytimes3xbfgragh.onion/hc/en-us/articles/115014893968-Terms-of-sale}{Terms
  of Sale}
\item
  \href{https://spiderbites.nytimes3xbfgragh.onion}{Site Map}
\item
  \href{https://help.nytimes3xbfgragh.onion/hc/en-us}{Help}
\item
  \href{https://www.nytimes3xbfgragh.onion/subscription?campaignId=37WXW}{Subscriptions}
\end{itemize}
