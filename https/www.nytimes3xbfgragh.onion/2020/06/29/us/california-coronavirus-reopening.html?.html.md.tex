Sections

SEARCH

\protect\hyperlink{site-content}{Skip to
content}\protect\hyperlink{site-index}{Skip to site index}

\href{https://www.nytimes3xbfgragh.onion/section/us}{U.S.}

\href{https://myaccount.nytimes3xbfgragh.onion/auth/login?response_type=cookie\&client_id=vi}{}

\href{https://www.nytimes3xbfgragh.onion/section/todayspaper}{Today's
Paper}

\href{/section/us}{U.S.}\textbar{}`Our Luck May Have Run Out':
California's Case Count Explodes

\url{https://nyti.ms/31xhdrZ}

\begin{itemize}
\item
\item
\item
\item
\item
\item
\end{itemize}

\href{https://www.nytimes3xbfgragh.onion/news-event/coronavirus?action=click\&pgtype=Article\&state=default\&region=TOP_BANNER\&context=storylines_menu}{The
Coronavirus Outbreak}

\begin{itemize}
\tightlist
\item
  live\href{https://www.nytimes3xbfgragh.onion/2020/08/04/world/coronavirus-cases.html?action=click\&pgtype=Article\&state=default\&region=TOP_BANNER\&context=storylines_menu}{Latest
  Updates}
\item
  \href{https://www.nytimes3xbfgragh.onion/interactive/2020/us/coronavirus-us-cases.html?action=click\&pgtype=Article\&state=default\&region=TOP_BANNER\&context=storylines_menu}{Maps
  and Cases}
\item
  \href{https://www.nytimes3xbfgragh.onion/interactive/2020/science/coronavirus-vaccine-tracker.html?action=click\&pgtype=Article\&state=default\&region=TOP_BANNER\&context=storylines_menu}{Vaccine
  Tracker}
\item
  \href{https://www.nytimes3xbfgragh.onion/2020/08/02/us/covid-college-reopening.html?action=click\&pgtype=Article\&state=default\&region=TOP_BANNER\&context=storylines_menu}{College
  Reopening}
\item
  \href{https://www.nytimes3xbfgragh.onion/live/2020/08/04/business/stock-market-today-coronavirus?action=click\&pgtype=Article\&state=default\&region=TOP_BANNER\&context=storylines_menu}{Economy}
\end{itemize}

Advertisement

\protect\hyperlink{after-top}{Continue reading the main story}

Supported by

\protect\hyperlink{after-sponsor}{Continue reading the main story}

\hypertarget{our-luck-may-have-run-out-californias-case-count-explodes}{%
\section{`Our Luck May Have Run Out': California's Case Count
Explodes}\label{our-luck-may-have-run-out-californias-case-count-explodes}}

Los Angeles County, which has been averaging more than 2,000 new cases
each day, surpassed 100,000 total cases on Monday.

\includegraphics{https://static01.graylady3jvrrxbe.onion/images/2020/06/29/us/29VIRUS-CALIFORNIA-la-testing/merlin_174045042_6f256f6d-a8c5-4085-a813-18101c388129-articleLarge.jpg?quality=75\&auto=webp\&disable=upscale}

\href{https://www.nytimes3xbfgragh.onion/by/shawn-hubler}{\includegraphics{https://static01.graylady3jvrrxbe.onion/images/2020/06/05/reader-center/author-shawn-hubler/author-shawn-hubler-thumbLarge.png}}\href{https://www.nytimes3xbfgragh.onion/by/thomas-fuller}{\includegraphics{https://static01.graylady3jvrrxbe.onion/images/2018/06/12/multimedia/author-thomas-fuller/author-thomas-fuller-thumbLarge.png}}

By \href{https://www.nytimes3xbfgragh.onion/by/shawn-hubler}{Shawn
Hubler} and
\href{https://www.nytimes3xbfgragh.onion/by/thomas-fuller}{Thomas
Fuller}

\begin{itemize}
\item
  Published June 29, 2020Updated July 23, 2020
\item
  \begin{itemize}
  \item
  \item
  \item
  \item
  \item
  \item
  \end{itemize}
\end{itemize}

\emph{{[}Sign up to}
\href{https://www.nytimes3xbfgragh.onion/newsletters/california-today}{\emph{receive
California Today}}\emph{, our newsletter from the Golden State, every
weekday.{]}}

SACRAMENTO --- Only a few weeks ago, thousands of Southern
\href{https://www.nytimes3xbfgragh.onion/2020/07/16/us/california-coronavirus-cases.html}{Californians}
were flocking to beaches, Disneyland was announcing it would soon reopen
and Whoopi Goldberg was lauding Gov. Gavin Newsom on ``The View'' for
the state's progress in combating the
\href{https://www.nytimes3xbfgragh.onion/2020/07/16/us/california-coronavirus-cases.html}{coronavirus}.
The worst, many in
\href{https://www.nytimes3xbfgragh.onion/2020/07/14/us/california-counties-reopening.html}{California}
thought, was behind them.

In fact, an alarming surge in cases up and down the state was only just
beginning.

Over the past week
\href{https://www.nytimes3xbfgragh.onion/2020/07/23/us/california-covid-19-cases.html}{California's
case count} has exploded, surpassing 200,000 known infections, and
forcing Mr. Newsom to roll back the state's
\href{https://www.nytimes3xbfgragh.onion/2020/06/30/us/coronavirus-california-cases-rising.html}{reopening}
in some counties. On Monday, he said the number of people hospitalized
in
\href{https://www.nytimes3xbfgragh.onion/2020/06/30/us/coronavirus-california-cases-rising.html}{California}
had risen 43 percent over the past two weeks.

Los Angeles County, which has been averaging more than 2,000 new cases
each day, surpassed 100,000 total cases on Monday, with the virus
actively infecting one in every 140 people, according to local health
officials. More than 2,800 cases were announced in the county on Monday,
the most of any day during the pandemic.

More than 7,000 new cases were announced across California on Monday,
its highest single-day total of the pandemic.

On Sunday, Mr. Newsom shut down the bars in a half-dozen counties,
including Los Angeles County and in the Central Valley, and recommended
that another eight counties voluntarily close their nightspots and
gathering places. On Friday, Imperial County, along the Mexican border,
was
\href{https://www.nytimes3xbfgragh.onion/2020/06/26/us/coronavirus-florida-texas-bars-closing.html}{told
to return to a stay-at-home order}. And Disneyland has since rescinded
its decision to open its gates.

California was the first state to shut down and one of the most
aggressive in fighting the virus. But the state that was so proactive in
combating the spread of the coronavirus is now being forced to ask
itself what went wrong.

``To some extent I think our luck may have run out,'' said Dr. Bob
Wachter, a professor and chair of the department of medicine at the
University of California, San Francisco. ``This is faster and worse than
I expected. You have to have a ton of respect for this thing. It is
nasty and it just lurks and waits to stomp on you if you let your guard
down for a second.''

On Monday, the governors of New York and New Jersey said they were
reconsidering plans to allow indoor dining in the coming days because
they were so alarmed by
\href{https://www.nytimes3xbfgragh.onion/2020/06/14/us/coronavirus-united-states.html}{the
rise in coronavirus cases in the South and the West}.

The head start that California appeared to enjoy --- the companies that
allowed employees to work from home as early as February, the governor
who warned residents in daily briefings to stay home and appeared to be
listened to --- was not protective enough in the long run.

\hypertarget{latest-updates-global-coronavirus-outbreak}{%
\section{\texorpdfstring{\href{https://www.nytimes3xbfgragh.onion/2020/08/04/world/coronavirus-cases.html?action=click\&pgtype=Article\&state=default\&region=MAIN_CONTENT_1\&context=storylines_live_updates}{Latest
Updates: Global Coronavirus
Outbreak}}{Latest Updates: Global Coronavirus Outbreak}}\label{latest-updates-global-coronavirus-outbreak}}

Updated 2020-08-04T23:01:34.126Z

\begin{itemize}
\tightlist
\item
  \href{https://www.nytimes3xbfgragh.onion/2020/08/04/world/coronavirus-cases.html?action=click\&pgtype=Article\&state=default\&region=MAIN_CONTENT_1\&context=storylines_live_updates\#link-2daa96b5}{As
  talks drag on, McConnell signals openness to jobless aid extension
  that Republicans have opposed.}
\item
  \href{https://www.nytimes3xbfgragh.onion/2020/08/04/world/coronavirus-cases.html?action=click\&pgtype=Article\&state=default\&region=MAIN_CONTENT_1\&context=storylines_live_updates\#link-1228a480}{Novavax
  sees encouraging results from two studies of its experimental
  vaccine.}
\item
  \href{https://www.nytimes3xbfgragh.onion/2020/08/04/world/coronavirus-cases.html?action=click\&pgtype=Article\&state=default\&region=MAIN_CONTENT_1\&context=storylines_live_updates\#link-4825b93}{Public
  and private schools in Maryland and elsewhere are divided over
  in-person instruction.}
\end{itemize}

\href{https://www.nytimes3xbfgragh.onion/2020/08/04/world/coronavirus-cases.html?action=click\&pgtype=Article\&state=default\&region=MAIN_CONTENT_1\&context=storylines_live_updates}{See
more updates}

More live coverage:
\href{https://www.nytimes3xbfgragh.onion/live/2020/08/04/business/stock-market-today-coronavirus?action=click\&pgtype=Article\&state=default\&region=MAIN_CONTENT_1\&context=storylines_live_updates}{Markets}

Younger people appear to account for the large surge in new cases, as
they have in many other states. Latinos, who make up a large swath of
the state's essential work force, have also recently seen
\href{https://www.nytimes3xbfgragh.onion/2020/06/26/us/corona-virus-latinos.html}{consistently
high case counts}.

And just as in Texas and Florida, the state's reopening appears to have
triggered a large resurgence. Pressured in part by businesses, church
groups and conservatives, Mr. Newsom ceded control of much of the timing
of reopening to local officials who were eager to regain a sense of
normalcy and stem economic losses. The result was a decentralized,
haphazard process that sowed confusion and gave residents a false sense
that they were in the clear.

Unlike people in the Northeast, many Californians did not have a sense
of urgency or immediacy toward the virus because infection rates had
been so low for months. There were no overflowing morgues or ambulance
sirens at all hours.

In a state with 40 million people, outbreaks have been heterogeneous:
San Quentin State Prison on the San Francisco Bay, food-processing
plants in the Central Valley, nursing homes, dense urban neighborhoods
filled with essential workers and family gatherings in remote rural
communities have all accounted for clusters of cases.

But health experts and state officials say the ultimate reasons for the
surge lie in the millions of individual decisions made across the vast
state.

Mayor Eric Garcetti of Los Angeles blamed ``irrational exuberance.''

``A lot of people didn't stick with the plan,'' the mayor said in an
interview on Friday. ``The idea was, we would do a move, wait three
weeks, check the impact, take the next move.''

Instead, Mr. Garcetti said, the reopening ``was like a tidal wave ---
one move led to the next, led to the next, led to the next. And then we
had the protest on top of that, and other things. And we have yet to be
able to identify where spread is happening and what we can do to crank
it down.''

\includegraphics{https://static01.graylady3jvrrxbe.onion/images/2020/06/29/us/29VIRUS-CALIFORNIA-losfeliz/merlin_174028461_3df19ece-f207-4261-85c3-18bcc34e0094-articleLarge.jpg?quality=75\&auto=webp\&disable=upscale}

State Senator Richard Pan, a Sacramento physician who led the state's
push to tighten immunization requirements, said that the state might
have flattened its curve at first, but that it never bent it down toward
zero.

``How this disease spreads is all about the margins,'' Dr. Pan said.
``All it takes is, like, 5 percent more people doing more high-risk
behavior to change its direction.''

Dr. Pan blames partisanship and misinformation spread by President
Trump. On the weekend after June 20, when bars reopened in Los Angeles
County, an estimated 500,000 people visited nightspots. Additionally,
half of the restaurants visited by county inspectors are not complying
with new public health rules, according to health officials.

``I'm frustrated because it's not that we don't know what to do,'' Dr.
Pan said. ``We know what to do. We're just not doing it.''

California was one of the earliest states to get hit by the virus. Soon
after it first appeared on American shores, Silicon Valley, south of San
Francisco, was considered a major hot spot, seeded by travelers arriving
from China.

Counties across the Bay Area banded together to announce
\href{https://www.nytimes3xbfgragh.onion/2020/03/16/us/california-covid-19.html}{the
nation's first stay-at-home order} on March 16, followed by a statewide
order three days later. Cellphone data showed that Californians moved
around less than people in other states, and infection rates stabilized
at a plateau well below the levels experts had projected, making
military field hospitals and sports arenas and auditoriums --- all
mobilized in case of a shortage of beds --- unnecessary.

By May, with low case counts remaining steady, Mr. Newsom was coming
under increasing pressure to reopen. Harmeet K. Dhillon, a civil rights
attorney and member of the Republican National Committee, filed more
than a dozen lawsuits related to the reopening. If residents could
congregate at Costco, they should be able to go to church, she argued.
Elon Musk, the head of Tesla, railed that his Bay Area car factory was
forced to shut and
\href{https://www.nytimes3xbfgragh.onion/2020/05/09/business/coronavirus-elon-musk-tesla-california.html}{threatened
to move the company's headquarters} out of California.

\href{https://www.nytimes3xbfgragh.onion/news-event/coronavirus?action=click\&pgtype=Article\&state=default\&region=MAIN_CONTENT_3\&context=storylines_faq}{}

\hypertarget{the-coronavirus-outbreak-}{%
\subsubsection{The Coronavirus Outbreak
›}\label{the-coronavirus-outbreak-}}

\hypertarget{frequently-asked-questions}{%
\paragraph{Frequently Asked
Questions}\label{frequently-asked-questions}}

Updated August 4, 2020

\begin{itemize}
\item ~
  \hypertarget{i-have-antibodies-am-i-now-immune}{%
  \paragraph{I have antibodies. Am I now
  immune?}\label{i-have-antibodies-am-i-now-immune}}

  \begin{itemize}
  \tightlist
  \item
    As of right
    now,\href{https://www.nytimes3xbfgragh.onion/2020/07/22/health/covid-antibodies-herd-immunity.html?action=click\&pgtype=Article\&state=default\&region=MAIN_CONTENT_3\&context=storylines_faq}{that
    seems likely, for at least several months.} There have been
    frightening accounts of people suffering what seems to be a second
    bout of Covid-19. But experts say these patients may have a
    drawn-out course of infection, with the virus taking a slow toll
    weeks to months after initial exposure. People infected with the
    coronavirus typically
    \href{https://www.nature.com/articles/s41586-020-2456-9}{produce}
    immune molecules called antibodies, which are
    \href{https://www.nytimes3xbfgragh.onion/2020/05/07/health/coronavirus-antibody-prevalence.html?action=click\&pgtype=Article\&state=default\&region=MAIN_CONTENT_3\&context=storylines_faq}{protective
    proteins made in response to an
    infection}\href{https://www.nytimes3xbfgragh.onion/2020/05/07/health/coronavirus-antibody-prevalence.html?action=click\&pgtype=Article\&state=default\&region=MAIN_CONTENT_3\&context=storylines_faq}{.
    These antibodies may} last in the body
    \href{https://www.nature.com/articles/s41591-020-0965-6}{only two to
    three months}, which may seem worrisome, but that's perfectly normal
    after an acute infection subsides, said Dr. Michael Mina, an
    immunologist at Harvard University. It may be possible to get the
    coronavirus again, but it's highly unlikely that it would be
    possible in a short window of time from initial infection or make
    people sicker the second time.
  \end{itemize}
\item ~
  \hypertarget{im-a-small-business-owner-can-i-get-relief}{%
  \paragraph{I'm a small-business owner. Can I get
  relief?}\label{im-a-small-business-owner-can-i-get-relief}}

  \begin{itemize}
  \tightlist
  \item
    The
    \href{https://www.nytimes3xbfgragh.onion/article/small-business-loans-stimulus-grants-freelancers-coronavirus.html?action=click\&pgtype=Article\&state=default\&region=MAIN_CONTENT_3\&context=storylines_faq}{stimulus
    bills enacted in March} offer help for the millions of American
    small businesses. Those eligible for aid are businesses and
    nonprofit organizations with fewer than 500 workers, including sole
    proprietorships, independent contractors and freelancers. Some
    larger companies in some industries are also eligible. The help
    being offered, which is being managed by the Small Business
    Administration, includes the Paycheck Protection Program and the
    Economic Injury Disaster Loan program. But lots of folks have
    \href{https://www.nytimes3xbfgragh.onion/interactive/2020/05/07/business/small-business-loans-coronavirus.html?action=click\&pgtype=Article\&state=default\&region=MAIN_CONTENT_3\&context=storylines_faq}{not
    yet seen payouts.} Even those who have received help are confused:
    The rules are draconian, and some are stuck sitting on
    \href{https://www.nytimes3xbfgragh.onion/2020/05/02/business/economy/loans-coronavirus-small-business.html?action=click\&pgtype=Article\&state=default\&region=MAIN_CONTENT_3\&context=storylines_faq}{money
    they don't know how to use.} Many small-business owners are getting
    less than they expected or
    \href{https://www.nytimes3xbfgragh.onion/2020/06/10/business/Small-business-loans-ppp.html?action=click\&pgtype=Article\&state=default\&region=MAIN_CONTENT_3\&context=storylines_faq}{not
    hearing anything at all.}
  \end{itemize}
\item ~
  \hypertarget{what-are-my-rights-if-i-am-worried-about-going-back-to-work}{%
  \paragraph{What are my rights if I am worried about going back to
  work?}\label{what-are-my-rights-if-i-am-worried-about-going-back-to-work}}

  \begin{itemize}
  \tightlist
  \item
    Employers have to provide
    \href{https://www.osha.gov/SLTC/covid-19/standards.html}{a safe
    workplace} with policies that protect everyone equally.
    \href{https://www.nytimes3xbfgragh.onion/article/coronavirus-money-unemployment.html?action=click\&pgtype=Article\&state=default\&region=MAIN_CONTENT_3\&context=storylines_faq}{And
    if one of your co-workers tests positive for the coronavirus, the
    C.D.C.} has said that
    \href{https://www.cdc.gov/coronavirus/2019-ncov/community/guidance-business-response.html}{employers
    should tell their employees} -\/- without giving you the sick
    employee's name -\/- that they may have been exposed to the virus.
  \end{itemize}
\item ~
  \hypertarget{should-i-refinance-my-mortgage}{%
  \paragraph{Should I refinance my
  mortgage?}\label{should-i-refinance-my-mortgage}}

  \begin{itemize}
  \tightlist
  \item
    \href{https://www.nytimes3xbfgragh.onion/article/coronavirus-money-unemployment.html?action=click\&pgtype=Article\&state=default\&region=MAIN_CONTENT_3\&context=storylines_faq}{It
    could be a good idea,} because mortgage rates have
    \href{https://www.nytimes3xbfgragh.onion/2020/07/16/business/mortgage-rates-below-3-percent.html?action=click\&pgtype=Article\&state=default\&region=MAIN_CONTENT_3\&context=storylines_faq}{never
    been lower.} Refinancing requests have pushed mortgage applications
    to some of the highest levels since 2008, so be prepared to get in
    line. But defaults are also up, so if you're thinking about buying a
    home, be aware that some lenders have tightened their standards.
  \end{itemize}
\item ~
  \hypertarget{what-is-school-going-to-look-like-in-september}{%
  \paragraph{What is school going to look like in
  September?}\label{what-is-school-going-to-look-like-in-september}}

  \begin{itemize}
  \tightlist
  \item
    It is unlikely that many schools will return to a normal schedule
    this fall, requiring the grind of
    \href{https://www.nytimes3xbfgragh.onion/2020/06/05/us/coronavirus-education-lost-learning.html?action=click\&pgtype=Article\&state=default\&region=MAIN_CONTENT_3\&context=storylines_faq}{online
    learning},
    \href{https://www.nytimes3xbfgragh.onion/2020/05/29/us/coronavirus-child-care-centers.html?action=click\&pgtype=Article\&state=default\&region=MAIN_CONTENT_3\&context=storylines_faq}{makeshift
    child care} and
    \href{https://www.nytimes3xbfgragh.onion/2020/06/03/business/economy/coronavirus-working-women.html?action=click\&pgtype=Article\&state=default\&region=MAIN_CONTENT_3\&context=storylines_faq}{stunted
    workdays} to continue. California's two largest public school
    districts --- Los Angeles and San Diego --- said on July 13, that
    \href{https://www.nytimes3xbfgragh.onion/2020/07/13/us/lausd-san-diego-school-reopening.html?action=click\&pgtype=Article\&state=default\&region=MAIN_CONTENT_3\&context=storylines_faq}{instruction
    will be remote-only in the fall}, citing concerns that surging
    coronavirus infections in their areas pose too dire a risk for
    students and teachers. Together, the two districts enroll some
    825,000 students. They are the largest in the country so far to
    abandon plans for even a partial physical return to classrooms when
    they reopen in August. For other districts, the solution won't be an
    all-or-nothing approach.
    \href{https://bioethics.jhu.edu/research-and-outreach/projects/eschool-initiative/school-policy-tracker/}{Many
    systems}, including the nation's largest, New York City, are
    devising
    \href{https://www.nytimes3xbfgragh.onion/2020/06/26/us/coronavirus-schools-reopen-fall.html?action=click\&pgtype=Article\&state=default\&region=MAIN_CONTENT_3\&context=storylines_faq}{hybrid
    plans} that involve spending some days in classrooms and other days
    online. There's no national policy on this yet, so check with your
    municipal school system regularly to see what is happening in your
    community.
  \end{itemize}
\end{itemize}

Mr. Newsom localized the reopening process, allowing counties to move at
different speeds, repeatedly declaring that ``localism is
determinative,'' and vowing to collaborate with county governments, not
issue orders. Church congregations were allowed to meet with
restrictions.

Advocates for reopening like Ms. Dhillon felt vindicated.

``I feel that our lawsuits were responsible for large sectors of
California's economy opening up much sooner than the governor originally
intended,'' she said, adding that she fielded countless calls from
business owners. ``People are absolutely devastated.''

But Mr. Newsom was also criticized by those who worried the state was
reopening too quickly.

Dr. Sara Cody, the chief health officer of Santa Clara County and the
architect of the Bay Area's stay-at-home orders, said the system was
bewildering to residents who cross county lines regularly.

``For the public it's incredibly confusing,'' Dr. Cody said in an
interview. ``What's the message? How can it be that something is OK here
and in the adjoining county it's not?''

The measures have become even more disparate in the past few weeks. In
Napa and Sonoma Counties, wine tastings and restaurant meals are
permitted both indoors and outside. In San Francisco, restaurant dining
is only available outdoors. Mayor London Breed of San Francisco
announced last week that the city would postpone the reopening,
scheduled for Monday, of hair and nail salons, massage shops, museums,
tattoo parlors and outdoor bars.

Diana Dooley, a former state secretary of Health and Human Services who
dealt with the Ebola and Zika pandemics during her tenure, said she had
watched ``with great empathy'' as the crisis gripped California.
Initially, she said, ``it looked like the Bay Area was driving the
decisions.'' And as the virus spread, Californians were generally
compliant.

``But after several months, the impatient people have made top-down
orders very hard to enforce,'' she said.

The result, she said, is a sense in some parts of the state of ``a kind
of liberty gone rampant.''

``People want to go to bars, they want to go to picnics,'' she said.
``These protest rallies have heightened that sense of `I want to be in a
crowd.' We're coming up on the Fourth of July and people want to be in
connection with each other.''

``You can only lead if people follow,'' Ms. Dooley said. ``Newsom has
done a pretty good job of creating awareness, and people in California
are more inclined to believe this is serious. But what they can do about
it, we're still getting to.''

David Townsend, a veteran Democratic political consultant in the state,
said California's size and political complexity pose a considerable
challenge. Although the Legislature is overwhelmingly Democratic, more
than a fifth of the electorate is Republican.

``You have the Inland Empire doing one thing, Los Angeles doing another,
Orange County --- it's pretty hard to corral everybody in California and
get them to do the right thing. It's just so big.''

Mr. Townsend said the pressure would be tough for any governor.

``How do you put the genie back into the bottle?'' Mr. Townsend added.
``I'm not sure there's much more he can do.''

Shawn Hubler reported from Sacramento and Thomas Fuller from Moraga,
Calif. Mitch Smith contributed reporting from Chicago.

Advertisement

\protect\hyperlink{after-bottom}{Continue reading the main story}

\hypertarget{site-index}{%
\subsection{Site Index}\label{site-index}}

\hypertarget{site-information-navigation}{%
\subsection{Site Information
Navigation}\label{site-information-navigation}}

\begin{itemize}
\tightlist
\item
  \href{https://help.nytimes3xbfgragh.onion/hc/en-us/articles/115014792127-Copyright-notice}{©~2020~The
  New York Times Company}
\end{itemize}

\begin{itemize}
\tightlist
\item
  \href{https://www.nytco.com/}{NYTCo}
\item
  \href{https://help.nytimes3xbfgragh.onion/hc/en-us/articles/115015385887-Contact-Us}{Contact
  Us}
\item
  \href{https://www.nytco.com/careers/}{Work with us}
\item
  \href{https://nytmediakit.com/}{Advertise}
\item
  \href{http://www.tbrandstudio.com/}{T Brand Studio}
\item
  \href{https://www.nytimes3xbfgragh.onion/privacy/cookie-policy\#how-do-i-manage-trackers}{Your
  Ad Choices}
\item
  \href{https://www.nytimes3xbfgragh.onion/privacy}{Privacy}
\item
  \href{https://help.nytimes3xbfgragh.onion/hc/en-us/articles/115014893428-Terms-of-service}{Terms
  of Service}
\item
  \href{https://help.nytimes3xbfgragh.onion/hc/en-us/articles/115014893968-Terms-of-sale}{Terms
  of Sale}
\item
  \href{https://spiderbites.nytimes3xbfgragh.onion}{Site Map}
\item
  \href{https://help.nytimes3xbfgragh.onion/hc/en-us}{Help}
\item
  \href{https://www.nytimes3xbfgragh.onion/subscription?campaignId=37WXW}{Subscriptions}
\end{itemize}
