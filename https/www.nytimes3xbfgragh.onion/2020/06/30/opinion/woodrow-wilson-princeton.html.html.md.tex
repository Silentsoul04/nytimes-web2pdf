Sections

SEARCH

\protect\hyperlink{site-content}{Skip to
content}\protect\hyperlink{site-index}{Skip to site index}

\href{https://myaccount.nytimes3xbfgragh.onion/auth/login?response_type=cookie\&client_id=vi}{}

\href{https://www.nytimes3xbfgragh.onion/section/todayspaper}{Today's
Paper}

\href{/section/opinion}{Opinion}\textbar{}The Ghost of Woodrow Wilson

\url{https://nyti.ms/2BQoCrK}

\begin{itemize}
\item
\item
\item
\item
\item
\item
\end{itemize}

Advertisement

\protect\hyperlink{after-top}{Continue reading the main story}

\href{/section/opinion}{Opinion}

Supported by

\protect\hyperlink{after-sponsor}{Continue reading the main story}

\hypertarget{the-ghost-of-woodrow-wilson}{%
\section{The Ghost of Woodrow
Wilson}\label{the-ghost-of-woodrow-wilson}}

The debts our institutions owe to the problematic past can't just be
canceled.

\href{https://www.nytimes3xbfgragh.onion/by/ross-douthat}{\includegraphics{https://static01.graylady3jvrrxbe.onion/images/2018/04/03/opinion/ross-douthat/ross-douthat-thumbLarge.png}}

By \href{https://www.nytimes3xbfgragh.onion/by/ross-douthat}{Ross
Douthat}

Opinion Columnist

\begin{itemize}
\item
  June 30, 2020
\item
  \begin{itemize}
  \item
  \item
  \item
  \item
  \item
  \item
  \end{itemize}
\end{itemize}

\includegraphics{https://static01.graylady3jvrrxbe.onion/images/2020/06/30/opinion/30douthat1/merlin_101852905_2d80c0b6-6e03-4382-9a5d-64dfd5ce6fbc-articleLarge.jpg?quality=75\&auto=webp\&disable=upscale}

When it comes to hating Woodrow Wilson, I was an early adopter. Raised
with the bland liberal history that hailed the 28th president as a
visionary for championing the League of Nations, I picked up in college
what was then a contrarian, mostly right-wing perspective --- that many
of Wilson's legacies were disastrous, including an imperial
understanding of the presidency that's deformed our constitutional
structure ever since, the messianic style in American foreign policy
that gave us Vietnam and Iraq, and a solidification of Jim Crow under a
scientific-racist guise.

Now his racism has finally prompted Princeton University, which once had
Wilson as its president, to remove his name from its prominent school of
public and international affairs. This move was made under pressure from
left-wing activists, but it also answered conservatives who had invoked
Wilson's name to suggest that progressive racists might be unjustly
spared from cancellation.

For this Wilson-despiser, his fall was a clarifying moment. I expected
to be at least a little pleased and justified when the name was gone.
Instead, the decision just seemed fundamentally dishonest, a case study
in what goes wrong when iconoclasm moves beyond Confederates to
encompass the wider American inheritance.

Our civil religion, back when it had more true believers, sometimes
\href{https://en.wikipedia.org/wiki/The_Apotheosis_of_Washington}{treated
departed presidents like saints}. But our monuments and honorifics exist
primarily to honor deeds, not to issue canonizations --- to express
gratitude for some specific act, to acknowledge some specific debt, to
trace a line back to some worthwhile inheritance.

Thus when you enter their Washington, D.C., memorials, you'll see Thomas
Jefferson honored as the man who expressed the founding's highest ideals
and Abraham Lincoln as the president who made good on their promise.
That the first was a hypocrite slave owner and the second a pragmatist
who had to be pushed into liberating the slaves is certainly relevant to
our assessment of their characters. But they remain the author of the
Declaration of Independence and the savior of the union, and you can't
embrace either legacy, the union or ``we hold these truths \ldots{}''
without acknowledging that these gifts came down through them.

To repudiate an honor or dismantle a memorial, then, makes moral sense
only if you intend to repudiate the specific deeds that it memorializes.
In the case of Confederate monuments, that's exactly what we should want
to do. Their objective purpose was to valorize a cause that we are
grateful met defeat, there is no debt we owe J.E.B. Stuart or Nathan
Bedford Forrest that needs to be remembered, and if they are put away we
will become more morally consistent, not less, in how we think about
that chapter in our past.

But just as Jefferson's memorial wasn't built to celebrate his
slaveholding, the Woodrow Wilson School of Public and International
Affairs wasn't named for Wilson to honor him for being a segregationist.
It was named for him because he helped create precisely the institutions
that the school exists to staff --- our domestic administrative state
and our global foreign policy apparatus --- and because he was the
presidential progenitor of the idealistic, interventionist worldview
that has animated that foreign policy community ever since.

Which means, in turn, that the school will remain \emph{his} school,
whatever name gets slapped upon it, so long as it pursues the projects
of enlightened progressive administration and global superpowerdom.
Obviously there are people, right and left, who would prefer that one or
both of those projects be abandoned. But they aren't likely to be
running the renamed school. Instead, it will continue to be run by
21st-century Wilsonians --- who will now act as if their worldview
sprang from nowhere, that its progenitor did not exist, effectively
repudiating their benefactor while accepting his inheritance.

Or consider a different example, one raised by puckish conservatives in
the last few weeks: The case of Yale University, named for a
17th-century merchant, official and dealer in slaves named Elihu Yale.
What is honored and memorialized in the school's name (and this is true
of many schools) is exactly one deed from Yale's often wicked and
dishonest life --- the donation of his money to the young college. The
name ``Yale'' doesn't honor old Elihu's slaving; it simply pays the
school's debt to him, acknowledging that Yale owes part of its very
existence to a rich man's desire to see ill-gotten money put to better
use.

Now some might suggest that Yale's existence is \emph{not} in fact a
good thing, and that honoring the man whose money helped establish it is
therefore a mistake. But if Yale is bad in this profound sense, then
renaming the school won't magically make it good; it will remain the
same bad place, continue taking money from today's Elihu Yales (how much
money touched by slave labor in China fills Yale's coffers even now?),
and all it will have done is added self-righteous amnesia and historical
ingratitude to its list of sins.

Or consider a case with wider application --- the monuments to
Christopher Columbus, like the one removed from a small park in my
hometown, New Haven, Conn., just last week. These statues acknowledge
the general debt that the New World's colonists, settlers and immigrants
owe to the man who connected Europe and the Americas, along with (in
most cases) the specific desire of Italian-American immigrants to
acknowledge and lay claim to an Italian explorer. And just as Yale's
debt to Elihu exists so long as anyone believes that Yale is good and
worth preserving, the American debt to Columbus's audacity exists so
long as we are grateful to have had ancestors who crossed the seas to
settle here --- notwithstanding his cruelty in governing Hispaniola or
any other crime.

Again, as in the previous examples, you can believe that gratitude of
any sort is the wrong emotion to feel for 1492; you can believe that the
settlement of the Americas was a purely wicked project whose fruits
should be redistributed and whose legacy abjured. This belief is
consistent with taking down the statues of Columbus; indeed it's
consistent with smashing them.

But unless the endgame of New Haven's removal of Columbus is the
expropriation of white property (Yale's property, I suppose, especially)
and its redistribution to the Pequots and Mohegans, then a consistent
rejection of Columbus's legacy isn't what my city is embracing. Instead,
it's just doing the same thing as Princeton: keeping the inheritance,
but repudiating the benefactor. Keeping the gains, but making a big show
of pronouncing them ill gotten.

If this dance eventually falters, and the true radicals take over, maybe
I will regret being too critical of its hypocrisies. (The Committee for
the De-settling of the Americas can wave this column in my face when
they come to expropriate my house.) But that possibility is one reason
not to accentuate historical ingratitude so glaringly, lest the people
who really pine for some genuine Year Zero take you up on the implied
offer.

Meanwhile, for now the ingratitude is being presented as a clear moral
advance, and it is not. To enjoy an inheritance that comes from flawed
men by pretending that it comes from nowhere, through nobody, is a
betrayal of memory, not its rectification --- an act of
self-righteousness that may not bring the revolution, but does make our
ruling class that much less fit to rule.

\emph{The Times is committed to publishing}
\href{https://www.nytimes3xbfgragh.onion/2019/01/31/opinion/letters/letters-to-editor-new-york-times-women.html}{\emph{a
diversity of letters}} \emph{to the editor. We'd like to hear what you
think about this or any of our articles. Here are some}
\href{https://help.nytimes3xbfgragh.onion/hc/en-us/articles/115014925288-How-to-submit-a-letter-to-the-editor}{\emph{tips}}\emph{.
And here's our email:}
\href{mailto:letters@NYTimes.com}{\emph{letters@NYTimes.com}}\emph{.}

\emph{Follow The New York Times Opinion section on}
\href{https://www.facebookcorewwwi.onion/nytopinion}{\emph{Facebook}}\emph{,}
\href{http://twitter.com/NYTOpinion}{\emph{Twitter (@NYTOpinion)}}
\emph{and}
\href{https://www.instagram.com/nytopinion/}{\emph{Instagram}}\emph{,
join the Facebook political discussion group,}
\href{https://www.facebookcorewwwi.onion/groups/votingwhilefemale/}{\emph{Voting
While Female}}\emph{.}

Advertisement

\protect\hyperlink{after-bottom}{Continue reading the main story}

\hypertarget{site-index}{%
\subsection{Site Index}\label{site-index}}

\hypertarget{site-information-navigation}{%
\subsection{Site Information
Navigation}\label{site-information-navigation}}

\begin{itemize}
\tightlist
\item
  \href{https://help.nytimes3xbfgragh.onion/hc/en-us/articles/115014792127-Copyright-notice}{©~2020~The
  New York Times Company}
\end{itemize}

\begin{itemize}
\tightlist
\item
  \href{https://www.nytco.com/}{NYTCo}
\item
  \href{https://help.nytimes3xbfgragh.onion/hc/en-us/articles/115015385887-Contact-Us}{Contact
  Us}
\item
  \href{https://www.nytco.com/careers/}{Work with us}
\item
  \href{https://nytmediakit.com/}{Advertise}
\item
  \href{http://www.tbrandstudio.com/}{T Brand Studio}
\item
  \href{https://www.nytimes3xbfgragh.onion/privacy/cookie-policy\#how-do-i-manage-trackers}{Your
  Ad Choices}
\item
  \href{https://www.nytimes3xbfgragh.onion/privacy}{Privacy}
\item
  \href{https://help.nytimes3xbfgragh.onion/hc/en-us/articles/115014893428-Terms-of-service}{Terms
  of Service}
\item
  \href{https://help.nytimes3xbfgragh.onion/hc/en-us/articles/115014893968-Terms-of-sale}{Terms
  of Sale}
\item
  \href{https://spiderbites.nytimes3xbfgragh.onion}{Site Map}
\item
  \href{https://help.nytimes3xbfgragh.onion/hc/en-us}{Help}
\item
  \href{https://www.nytimes3xbfgragh.onion/subscription?campaignId=37WXW}{Subscriptions}
\end{itemize}
