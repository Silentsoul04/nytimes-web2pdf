Sections

SEARCH

\protect\hyperlink{site-content}{Skip to
content}\protect\hyperlink{site-index}{Skip to site index}

\href{https://www.nytimes3xbfgragh.onion/section/us}{U.S.}

\href{https://myaccount.nytimes3xbfgragh.onion/auth/login?response_type=cookie\&client_id=vi}{}

\href{https://www.nytimes3xbfgragh.onion/section/todayspaper}{Today's
Paper}

\href{/section/us}{U.S.}\textbar{}Why a Pediatric Group Is Pushing to
Reopen Schools This Fall

\url{https://nyti.ms/38egCwR}

\begin{itemize}
\item
\item
\item
\item
\item
\item
\end{itemize}

\href{https://www.nytimes3xbfgragh.onion/news-event/coronavirus?action=click\&pgtype=Article\&state=default\&region=TOP_BANNER\&context=storylines_menu}{The
Coronavirus Outbreak}

\begin{itemize}
\tightlist
\item
  live\href{https://www.nytimes3xbfgragh.onion/2020/08/04/world/coronavirus-cases.html?action=click\&pgtype=Article\&state=default\&region=TOP_BANNER\&context=storylines_menu}{Latest
  Updates}
\item
  \href{https://www.nytimes3xbfgragh.onion/interactive/2020/us/coronavirus-us-cases.html?action=click\&pgtype=Article\&state=default\&region=TOP_BANNER\&context=storylines_menu}{Maps
  and Cases}
\item
  \href{https://www.nytimes3xbfgragh.onion/interactive/2020/science/coronavirus-vaccine-tracker.html?action=click\&pgtype=Article\&state=default\&region=TOP_BANNER\&context=storylines_menu}{Vaccine
  Tracker}
\item
  \href{https://www.nytimes3xbfgragh.onion/2020/08/02/us/covid-college-reopening.html?action=click\&pgtype=Article\&state=default\&region=TOP_BANNER\&context=storylines_menu}{College
  Reopening}
\item
  \href{https://www.nytimes3xbfgragh.onion/live/2020/08/04/business/stock-market-today-coronavirus?action=click\&pgtype=Article\&state=default\&region=TOP_BANNER\&context=storylines_menu}{Economy}
\end{itemize}

Advertisement

\protect\hyperlink{after-top}{Continue reading the main story}

Supported by

\protect\hyperlink{after-sponsor}{Continue reading the main story}

\hypertarget{why-a-pediatric-group-is-pushing-to-reopen-schools-this-fall}{%
\section{Why a Pediatric Group Is Pushing to Reopen Schools This
Fall}\label{why-a-pediatric-group-is-pushing-to-reopen-schools-this-fall}}

Guidelines from the American Academy of Pediatrics encourage ``having
students physically present in school.'' Dr. Sean O'Leary, an author of
that advice, explains why.

\includegraphics{https://static01.graylady3jvrrxbe.onion/images/2020/06/30/us/30virus-pediatrics-01/30virus-pediatrics-01-articleLarge-v2.jpg?quality=75\&auto=webp\&disable=upscale}

\href{https://www.nytimes3xbfgragh.onion/by/dana-goldstein}{\includegraphics{https://static01.graylady3jvrrxbe.onion/images/2018/06/12/multimedia/author-dana-goldstein/author-dana-goldstein-thumbLarge.png}}

By \href{https://www.nytimes3xbfgragh.onion/by/dana-goldstein}{Dana
Goldstein}

\begin{itemize}
\item
  June 30, 2020
\item
  \begin{itemize}
  \item
  \item
  \item
  \item
  \item
  \item
  \end{itemize}
\end{itemize}

The American Academy of Pediatrics has a reputation as conservative and
cautious, which is what you would expect from an organization devoted to
protecting children's health. But this week, the academy
\href{https://twitter.com/DanaGoldstein/status/1277566240526123008}{made
a splash} with advice about reopening schools that appears to be
somewhat at odds with what administrators are hearing from some federal
and state health officials.

The Centers for Disease Control and Prevention, for example, have
advised that remote learning is the safest option. But the academy's
guidelines strongly recommend that students be ``physically present in
school'' as much as possible, and emphasize that there are major health,
social and educational risks to keeping children at home.

\href{https://www.cudoctors.com/Find_A_Doctor/Profile/8599}{Dr. Sean
O'Leary}, a pediatrics infectious disease specialist at the University
of Colorado Anschutz Medical Campus, helped write
\href{https://services.aap.org/en/pages/2019-novel-coronavirus-covid-19-infections/clinical-guidance/covid-19-planning-considerations-return-to-in-person-education-in-schools/}{the
academy's guidelines}. He is a father of two children, 12 and 16, and a
survivor of
\href{https://www.nytimes3xbfgragh.onion/news-event/coronavirus}{Covid-19}
who is still experiencing some symptoms after he and his wife contracted
the coronavirus in March.

``I absolutely take this seriously,'' Dr. O'Leary said. ``I'm still
sick.'' But he explained why the academy was emphasizing the need to get
students back in classrooms.

\emph{This interview has been condensed and edited for clarity.}

\textbf{The academy guidelines place a big emphasis on the importance of
physical school over remote learning. Can you summarize why?}

As pediatricians, many of us have recognized already the impact that
having schools closed even for a couple months had on children. At the
same time, a lot of us are parents. We experienced our own kids doing
online learning. There really wasn't a lot of learning happening. Now
we're seeing
\href{https://www.nytimes3xbfgragh.onion/2020/06/05/us/coronavirus-education-lost-learning.html}{studies
documenting this}. Kids being home led to increases in behavioral health
problems. There were reports of increased rates of abuse.

Of course, the reason they were at home was to help control the
pandemic. But we know a lot more now than we did then, when schools
first closed. We're still learning more every day.

\hypertarget{latest-updates-global-coronavirus-outbreak}{%
\section{\texorpdfstring{\href{https://www.nytimes3xbfgragh.onion/2020/08/04/world/coronavirus-cases.html?action=click\&pgtype=Article\&state=default\&region=MAIN_CONTENT_1\&context=storylines_live_updates}{Latest
Updates: Global Coronavirus
Outbreak}}{Latest Updates: Global Coronavirus Outbreak}}\label{latest-updates-global-coronavirus-outbreak}}

Updated 2020-08-04T21:18:49.389Z

\begin{itemize}
\tightlist
\item
  \href{https://www.nytimes3xbfgragh.onion/2020/08/04/world/coronavirus-cases.html?action=click\&pgtype=Article\&state=default\&region=MAIN_CONTENT_1\&context=storylines_live_updates\#link-1228a480}{Novavax
  sees encouraging results from two studies of its experimental
  vaccine.}
\item
  \href{https://www.nytimes3xbfgragh.onion/2020/08/04/world/coronavirus-cases.html?action=click\&pgtype=Article\&state=default\&region=MAIN_CONTENT_1\&context=storylines_live_updates\#link-4825b93}{Public
  and private schools in Maryland and elsewhere are divided over
  in-person instruction.}
\item
  \href{https://www.nytimes3xbfgragh.onion/2020/08/04/world/coronavirus-cases.html?action=click\&pgtype=Article\&state=default\&region=MAIN_CONTENT_1\&context=storylines_live_updates\#link-50f7386d}{The
  United Nations calls on policymakers to `plan thoroughly for school
  reopenings.'}
\end{itemize}

\href{https://www.nytimes3xbfgragh.onion/2020/08/04/world/coronavirus-cases.html?action=click\&pgtype=Article\&state=default\&region=MAIN_CONTENT_1\&context=storylines_live_updates}{See
more updates}

More live coverage:
\href{https://www.nytimes3xbfgragh.onion/live/2020/08/04/business/stock-market-today-coronavirus?action=click\&pgtype=Article\&state=default\&region=MAIN_CONTENT_1\&context=storylines_live_updates}{Markets}

This virus is different from most of the respiratory viruses we deal
with every year. School-age kids clearly play a role in driving
influenza rates within communities. That doesn't seem to be the case
with Covid-19. And it seems like in countries where they have reopened
schools, it plays a much smaller role in driving spread of disease than
we would expect.

\textbf{Back in March, there was this idea of children as silent
superspreaders who put older adults and other vulnerable people at
immense risk. Has that picture changed?}

What we have seen so far
\href{https://academic.oup.com/cid/article/doi/10.1093/cid/ciaa794/5862649}{in}
\href{https://www.thelancet.com/journals/lanchi/article/PIIS2352-4642(20)30095-X/fulltext\#seccestitle20}{the}
\href{https://adc.bmj.com/content/105/7/618}{literature} --- and
anecdotally, as well --- is that kids really do seem to be both
\href{https://pubmed.ncbi.nlm.nih.gov/32546824/}{less likely to catch
the infection} and
\href{https://pediatrics.aappublications.org/content/early/2020/05/22/peds.2020-004879}{less
likely to spread the infection}. It seems to be even more true for
younger kids, under 10 or under 12. And older kids seem to play less of
a role than adults.

Here in Colorado, I've been following our state health department
website very closely. They update data every day and include the
outbreaks in the state they are investigating. As you can imagine, there
are lots and lots in long-term care facilities and skilled nursing
homes, some in restaurants and grocery stores. There have been a
\href{https://covid19.colorado.gov/data/outbreak-data}{total of four} in
child care centers, and we do have a lot of child care centers open. In
almost every one of those cases, transmission was between two adults.
The kids in the centers are not spreading Covid-19. I'm hearing the same
thing from other states, as well.

\includegraphics{https://static01.graylady3jvrrxbe.onion/images/2020/06/30/us/30virus-pediatrics-02/merlin_170643639_7250b6c2-579b-466b-a145-c8c0629b3633-articleLarge.jpg?quality=75\&auto=webp\&disable=upscale}

\textbf{The academy's guidelines talk about balancing the need for
physical distance with children's educational and developmental needs,
such as the need for hands-on play. They suggest that if older students
are masked, three feet of distance between desks might be sufficient,
compared to the six feet recommended by the C.D.C. Why is your advice
different?}

I don't know that we're different.
\href{https://www.cdc.gov/coronavirus/2019-ncov/community/schools-childcare/schools.html}{The
C.D.C. said} six feet if ``feasible.'' The point we are trying to make
is, that's really not feasible. When you consider the overall health of
children and really the community at large, adhering to a six-foot rule,
which would mean having a lot of kids at home, may not be in the best
interest of overall health. Something has to give.

From our perspective as pediatricians, the downsides of having kids at
home versus in school are outweighed by the small incremental gain you
would get from having kids six feet apart as opposed to five, four or
three. When you add into that other mitigation measures like mask
wearing, particularly for older kids, and frequent hand washing, you can
bring the risk down.

I do think it's a balance. I'm not going to come out here and say on
June 30 that everything is going to be perfect in the coming school
year. There will be cases of Covid-19 in schools even where they make
their best efforts. But we have to balance that with the overall health
of children.

\textbf{As I talk to school administrators, most are planning
temperature checks. The academy guidelines warn this could be
impractical and take away instructional time. Can you say more about why
you're skeptical that this is the right strategy?}

Do the harms outweigh the benefits? In this case, if it means students
are congregating, it could increase the risk of spread. And we don't
have great evidence that temperature screening is helpful. That's for a
couple reasons. One, a lot of kids who have Covid-19, perhaps the
majority, never get a fever. To use fever as a screen and assume that's
going to be good enough? You will miss a kid. And many fevers are not
going to be Covid-19. Kids should not go to a school with a fever,
period.

\href{https://www.nytimes3xbfgragh.onion/news-event/coronavirus?action=click\&pgtype=Article\&state=default\&region=MAIN_CONTENT_3\&context=storylines_faq}{}

\hypertarget{the-coronavirus-outbreak-}{%
\subsubsection{The Coronavirus Outbreak
›}\label{the-coronavirus-outbreak-}}

\hypertarget{frequently-asked-questions}{%
\paragraph{Frequently Asked
Questions}\label{frequently-asked-questions}}

Updated August 4, 2020

\begin{itemize}
\item ~
  \hypertarget{i-have-antibodies-am-i-now-immune}{%
  \paragraph{I have antibodies. Am I now
  immune?}\label{i-have-antibodies-am-i-now-immune}}

  \begin{itemize}
  \tightlist
  \item
    As of right
    now,\href{https://www.nytimes3xbfgragh.onion/2020/07/22/health/covid-antibodies-herd-immunity.html?action=click\&pgtype=Article\&state=default\&region=MAIN_CONTENT_3\&context=storylines_faq}{that
    seems likely, for at least several months.} There have been
    frightening accounts of people suffering what seems to be a second
    bout of Covid-19. But experts say these patients may have a
    drawn-out course of infection, with the virus taking a slow toll
    weeks to months after initial exposure. People infected with the
    coronavirus typically
    \href{https://www.nature.com/articles/s41586-020-2456-9}{produce}
    immune molecules called antibodies, which are
    \href{https://www.nytimes3xbfgragh.onion/2020/05/07/health/coronavirus-antibody-prevalence.html?action=click\&pgtype=Article\&state=default\&region=MAIN_CONTENT_3\&context=storylines_faq}{protective
    proteins made in response to an
    infection}\href{https://www.nytimes3xbfgragh.onion/2020/05/07/health/coronavirus-antibody-prevalence.html?action=click\&pgtype=Article\&state=default\&region=MAIN_CONTENT_3\&context=storylines_faq}{.
    These antibodies may} last in the body
    \href{https://www.nature.com/articles/s41591-020-0965-6}{only two to
    three months}, which may seem worrisome, but that's perfectly normal
    after an acute infection subsides, said Dr. Michael Mina, an
    immunologist at Harvard University. It may be possible to get the
    coronavirus again, but it's highly unlikely that it would be
    possible in a short window of time from initial infection or make
    people sicker the second time.
  \end{itemize}
\item ~
  \hypertarget{im-a-small-business-owner-can-i-get-relief}{%
  \paragraph{I'm a small-business owner. Can I get
  relief?}\label{im-a-small-business-owner-can-i-get-relief}}

  \begin{itemize}
  \tightlist
  \item
    The
    \href{https://www.nytimes3xbfgragh.onion/article/small-business-loans-stimulus-grants-freelancers-coronavirus.html?action=click\&pgtype=Article\&state=default\&region=MAIN_CONTENT_3\&context=storylines_faq}{stimulus
    bills enacted in March} offer help for the millions of American
    small businesses. Those eligible for aid are businesses and
    nonprofit organizations with fewer than 500 workers, including sole
    proprietorships, independent contractors and freelancers. Some
    larger companies in some industries are also eligible. The help
    being offered, which is being managed by the Small Business
    Administration, includes the Paycheck Protection Program and the
    Economic Injury Disaster Loan program. But lots of folks have
    \href{https://www.nytimes3xbfgragh.onion/interactive/2020/05/07/business/small-business-loans-coronavirus.html?action=click\&pgtype=Article\&state=default\&region=MAIN_CONTENT_3\&context=storylines_faq}{not
    yet seen payouts.} Even those who have received help are confused:
    The rules are draconian, and some are stuck sitting on
    \href{https://www.nytimes3xbfgragh.onion/2020/05/02/business/economy/loans-coronavirus-small-business.html?action=click\&pgtype=Article\&state=default\&region=MAIN_CONTENT_3\&context=storylines_faq}{money
    they don't know how to use.} Many small-business owners are getting
    less than they expected or
    \href{https://www.nytimes3xbfgragh.onion/2020/06/10/business/Small-business-loans-ppp.html?action=click\&pgtype=Article\&state=default\&region=MAIN_CONTENT_3\&context=storylines_faq}{not
    hearing anything at all.}
  \end{itemize}
\item ~
  \hypertarget{what-are-my-rights-if-i-am-worried-about-going-back-to-work}{%
  \paragraph{What are my rights if I am worried about going back to
  work?}\label{what-are-my-rights-if-i-am-worried-about-going-back-to-work}}

  \begin{itemize}
  \tightlist
  \item
    Employers have to provide
    \href{https://www.osha.gov/SLTC/covid-19/standards.html}{a safe
    workplace} with policies that protect everyone equally.
    \href{https://www.nytimes3xbfgragh.onion/article/coronavirus-money-unemployment.html?action=click\&pgtype=Article\&state=default\&region=MAIN_CONTENT_3\&context=storylines_faq}{And
    if one of your co-workers tests positive for the coronavirus, the
    C.D.C.} has said that
    \href{https://www.cdc.gov/coronavirus/2019-ncov/community/guidance-business-response.html}{employers
    should tell their employees} -\/- without giving you the sick
    employee's name -\/- that they may have been exposed to the virus.
  \end{itemize}
\item ~
  \hypertarget{should-i-refinance-my-mortgage}{%
  \paragraph{Should I refinance my
  mortgage?}\label{should-i-refinance-my-mortgage}}

  \begin{itemize}
  \tightlist
  \item
    \href{https://www.nytimes3xbfgragh.onion/article/coronavirus-money-unemployment.html?action=click\&pgtype=Article\&state=default\&region=MAIN_CONTENT_3\&context=storylines_faq}{It
    could be a good idea,} because mortgage rates have
    \href{https://www.nytimes3xbfgragh.onion/2020/07/16/business/mortgage-rates-below-3-percent.html?action=click\&pgtype=Article\&state=default\&region=MAIN_CONTENT_3\&context=storylines_faq}{never
    been lower.} Refinancing requests have pushed mortgage applications
    to some of the highest levels since 2008, so be prepared to get in
    line. But defaults are also up, so if you're thinking about buying a
    home, be aware that some lenders have tightened their standards.
  \end{itemize}
\item ~
  \hypertarget{what-is-school-going-to-look-like-in-september}{%
  \paragraph{What is school going to look like in
  September?}\label{what-is-school-going-to-look-like-in-september}}

  \begin{itemize}
  \tightlist
  \item
    It is unlikely that many schools will return to a normal schedule
    this fall, requiring the grind of
    \href{https://www.nytimes3xbfgragh.onion/2020/06/05/us/coronavirus-education-lost-learning.html?action=click\&pgtype=Article\&state=default\&region=MAIN_CONTENT_3\&context=storylines_faq}{online
    learning},
    \href{https://www.nytimes3xbfgragh.onion/2020/05/29/us/coronavirus-child-care-centers.html?action=click\&pgtype=Article\&state=default\&region=MAIN_CONTENT_3\&context=storylines_faq}{makeshift
    child care} and
    \href{https://www.nytimes3xbfgragh.onion/2020/06/03/business/economy/coronavirus-working-women.html?action=click\&pgtype=Article\&state=default\&region=MAIN_CONTENT_3\&context=storylines_faq}{stunted
    workdays} to continue. California's two largest public school
    districts --- Los Angeles and San Diego --- said on July 13, that
    \href{https://www.nytimes3xbfgragh.onion/2020/07/13/us/lausd-san-diego-school-reopening.html?action=click\&pgtype=Article\&state=default\&region=MAIN_CONTENT_3\&context=storylines_faq}{instruction
    will be remote-only in the fall}, citing concerns that surging
    coronavirus infections in their areas pose too dire a risk for
    students and teachers. Together, the two districts enroll some
    825,000 students. They are the largest in the country so far to
    abandon plans for even a partial physical return to classrooms when
    they reopen in August. For other districts, the solution won't be an
    all-or-nothing approach.
    \href{https://bioethics.jhu.edu/research-and-outreach/projects/eschool-initiative/school-policy-tracker/}{Many
    systems}, including the nation's largest, New York City, are
    devising
    \href{https://www.nytimes3xbfgragh.onion/2020/06/26/us/coronavirus-schools-reopen-fall.html?action=click\&pgtype=Article\&state=default\&region=MAIN_CONTENT_3\&context=storylines_faq}{hybrid
    plans} that involve spending some days in classrooms and other days
    online. There's no national policy on this yet, so check with your
    municipal school system regularly to see what is happening in your
    community.
  \end{itemize}
\end{itemize}

\textbf{As you were preparing these guidelines, did the emergence of the
potentially deadly}
\textbf{\href{https://www.nytimes3xbfgragh.onion/2020/06/29/well/family/caring-for-children-with-multisystem-inflammatory-syndrome.html}{pediatric
inflammatory syndrome}} \textbf{linked to Covid-19 sway you at all?}

We talked about it. It is by any measure a rare condition. And so it's
something we have to pay attention to and figure out what causes it and
the best treatment. We should also point out that even those kids who
have gotten very sick, the vast majority of them have recovered and done
well.

\textbf{The guidelines emphasize that teachers and school staff members
should stay physically distanced to the greatest extent possible and
conduct meetings remotely. But I am hearing from a lot of teachers. Many
are, frankly, scared to go back to school before a vaccine is available.
A few have told me that they feel that their health is treated as
expendable. What's your message to them?}

We're pediatricians. We're not educators. We don't want to tread in
space where we don't belong. But what I would say is it depends on the
level of risk for the individual person. Every district I have talked to
here in Colorado? They are making major considerations for their
teachers, trying to figure out how to keep them safe.

There are a couple of things we know now that we didn't know when we
closed schools down in March. One is that masks really do
\href{https://www.thelancet.com/journals/lancet/article/PIIS0140-6736(20)31142-9/fulltext}{seem
to work}. They are very effective. Two, physical distancing works as
well. If they are taking as many precautions as they can, I think the
risk is pretty low.

Some of these are very personal decisions. But schools can do a lot of
things to really make the environment as safe as possible.

\textbf{What do you hope is next in terms of local schools making
specific plans to reopen?}

How this gets rolled out in August or September when schools reopen is
really dependent on what is going on at that time with the virus. If
you're in a state that is doing well with very few cases, all of the
measures in the guidance are far less important. But if you're in a
place where the virus is raging, all of those things become much more
important.

As a country, we have to get on the same page. It's a tragedy to me that
the virus has become a political issue. It's costing tens of thousands,
potentially hundreds of thousands, of lives.

Reopening schools is so important for the kids, but really for the
entire community. So much of our world relies on kids being in school
and parents being able to work. Trying to work from home with the kids
home is disproportionately impacting women. So it goes beyond just the
health of the child, which is, of course, very important. As a country,
we should be doing everything we can right now, for lots of reasons, to
make sure we can safely reopen schools in the fall.

Advertisement

\protect\hyperlink{after-bottom}{Continue reading the main story}

\hypertarget{site-index}{%
\subsection{Site Index}\label{site-index}}

\hypertarget{site-information-navigation}{%
\subsection{Site Information
Navigation}\label{site-information-navigation}}

\begin{itemize}
\tightlist
\item
  \href{https://help.nytimes3xbfgragh.onion/hc/en-us/articles/115014792127-Copyright-notice}{©~2020~The
  New York Times Company}
\end{itemize}

\begin{itemize}
\tightlist
\item
  \href{https://www.nytco.com/}{NYTCo}
\item
  \href{https://help.nytimes3xbfgragh.onion/hc/en-us/articles/115015385887-Contact-Us}{Contact
  Us}
\item
  \href{https://www.nytco.com/careers/}{Work with us}
\item
  \href{https://nytmediakit.com/}{Advertise}
\item
  \href{http://www.tbrandstudio.com/}{T Brand Studio}
\item
  \href{https://www.nytimes3xbfgragh.onion/privacy/cookie-policy\#how-do-i-manage-trackers}{Your
  Ad Choices}
\item
  \href{https://www.nytimes3xbfgragh.onion/privacy}{Privacy}
\item
  \href{https://help.nytimes3xbfgragh.onion/hc/en-us/articles/115014893428-Terms-of-service}{Terms
  of Service}
\item
  \href{https://help.nytimes3xbfgragh.onion/hc/en-us/articles/115014893968-Terms-of-sale}{Terms
  of Sale}
\item
  \href{https://spiderbites.nytimes3xbfgragh.onion}{Site Map}
\item
  \href{https://help.nytimes3xbfgragh.onion/hc/en-us}{Help}
\item
  \href{https://www.nytimes3xbfgragh.onion/subscription?campaignId=37WXW}{Subscriptions}
\end{itemize}
