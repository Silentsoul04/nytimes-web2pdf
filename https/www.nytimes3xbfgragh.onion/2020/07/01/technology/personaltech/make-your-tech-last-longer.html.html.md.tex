Sections

SEARCH

\protect\hyperlink{site-content}{Skip to
content}\protect\hyperlink{site-index}{Skip to site index}

\href{https://www.nytimes3xbfgragh.onion/section/technology/personaltech}{Personal
Tech}

\href{https://myaccount.nytimes3xbfgragh.onion/auth/login?response_type=cookie\&client_id=vi}{}

\href{https://www.nytimes3xbfgragh.onion/section/todayspaper}{Today's
Paper}

\href{/section/technology/personaltech}{Personal Tech}\textbar{}How to
Make Your Tech Last Longer

\url{https://nyti.ms/2ZoQpaR}

\begin{itemize}
\item
\item
\item
\item
\item
\end{itemize}

Advertisement

\protect\hyperlink{after-top}{Continue reading the main story}

Supported by

\protect\hyperlink{after-sponsor}{Continue reading the main story}

Tech Fix

\hypertarget{how-to-make-your-tech-last-longer}{%
\section{How to Make Your Tech Last
Longer}\label{how-to-make-your-tech-last-longer}}

In a pandemic-induced recession, it's more important than ever to take
care of our smartphones and other gadgets.

\includegraphics{https://static01.graylady3jvrrxbe.onion/images/2020/07/02/business/02Techfix-illo/01Techfix-illo-articleLarge.gif?quality=75\&auto=webp\&disable=upscale}

\href{https://www.nytimes3xbfgragh.onion/by/brian-x-chen}{\includegraphics{https://static01.graylady3jvrrxbe.onion/images/2018/02/16/multimedia/author-brian-x-chen/author-brian-x-chen-thumbLarge.jpg}}

By \href{https://www.nytimes3xbfgragh.onion/by/brian-x-chen}{Brian X.
Chen}

\begin{itemize}
\item
  July 1, 2020
\item
  \begin{itemize}
  \item
  \item
  \item
  \item
  \item
  \end{itemize}
\end{itemize}

With so many people
\href{https://www.nytimes3xbfgragh.onion/2020/06/25/business/economy/coronavirus-unemployment-claims.html}{becoming
unemployed} in the pandemic-induced recession, we have no choice but to
handle our technology differently. Put another way: We need to make our
tech last longer.

We generally do a poor job of this. As soon as a device like a
smartphone starts to feel slow or its
\href{https://www.nytimes3xbfgragh.onion/2018/01/31/technology/personaltech/iphone-battery-slowing-down.html}{battery
deteriorates}, we conclude that it's time to buy a new one --- so we
upgrade.

``People just like to spend, spend, spend --- they say, I don't want to
fix it, I'll just get a new phone,'' said Shakeel Taiyab, who runs a
business repairing phones and computers in South San Francisco. ``It's
\$200 to fix your phone, now you want to spend another \$1,500? People
don't do the math.''

The tech companies have generally helped nurture this behavior, of
course. Many phone makers gave us incentives to buy new devices
regularly, for example, while offering scarce education on steps to help
our tech endure.

But with so many people now dealing with shrinking funds, making our
tech last longer makes common sense and is not so hard. Basic
maintenance includes replacing batteries, cleaning out dust and purging
unnecessary files that bog down our devices.

If we put a small amount of time into caring for our gadgets, they can
last indefinitely. We'd also be doing the world a favor. By elongating
the life of our gadgets, we put more use into the energy, materials and
human labor invested in creating the product.

We don't have to do all this on our own, either. There are people out
there willing to help. With
\href{https://www.nytimes3xbfgragh.onion/aponline/2020/06/19/business/bc-us-apple-store-closures.html}{retail
stores for Apple} and
\href{https://www.nytimes3xbfgragh.onion/aponline/2020/06/26/business/ap-microsoft-store-closures.html}{Microsoft
shutting down} in the pandemic and unable to service our gadgets, hiring
a local fixer is now as good an idea as ever. It just takes some
homework.

So here are some of the most effective steps you can take to squeeze as
much life as possible out of your phones, tablets and computers without
breaking the bank.

\hypertarget{check-your-battery}{%
\subsection{Check Your Battery}\label{check-your-battery}}

Start thinking about your device batteries as if they were car tires and
replace accordingly. Because batteries can be charged only a finite
number of times before they deteriorate, they will be one of the first
things to go on mobile devices and laptops.

For different types of gadgets, your battery mileage may vary.
Generally, a smartphone battery will last about two years, and ones for
a laptop or smartphone will last three or four years before needing to
be replaced, said Kyle Wiens, chief executive of iFixit, a site that
offers instructions on do-it-yourself gadget repairs.

So how do you assess whether a battery needs replacing? There are
various software tools you can use to check its health:

\begin{itemize}
\item
  Apple users with iPhones and iPads can open the Settings, then tap
  Battery and select Battery Health.
\item
  Android devices have third-party apps
  like\href{https://nl.nytimes3xbfgragh.onion/f/a/CnFx0v5P5lLYaOrA5NfJ6A~~/AAAAAQA~/RgRf-f_rP0T3aHR0cHM6Ly9wbGF5Lmdvb2dsZS5jb20vc3RvcmUvYXBwcy9kZXRhaWxzP2lkPWNvbS5kaWdpYml0ZXMuYWNjdWJhdHRlcnkmaGw9ZW5fVVMmdGU9MSZubD1wZXJzb25hbC10ZWNoJmVtYz1lZGl0X2N0XzIwMjAwMTA5P2NhbXBhaWduX2lkPTM4Jmluc3RhbmNlX2lkPTE1MDY0JnNlZ21lbnRfaWQ9MjAxODUmdXNlcl9pZD1mODgyNmUzMGY2Mzg1OThjMzAwZWIwODczMTQ5ZDI5NiZyZWdpX2lkPTY1MjU0MTc0ZWRpdF9jdF8yMDIwMDEwOVcDbnl0QgoAKut6F16kiLoqUhZicmlhbi5jaGVuQG55dGltZXMuY29tWAQAAAAA}{AccuBattery}
  that can do a reading on your battery's health.
\item
  Mac users can click on the Apple icon, then About This Mac and then
  System Report. Then click on Power to see a reading on battery health.
\item
  Windows users can download the app
  \href{https://www.nirsoft.net/utils/battery_information_view.html}{BatteryInfoView}
  to measure battery health.
\end{itemize}

In general, pay attention to a battery's remaining capacity. The lower
the capacity, the more short-lived your device gets. If your capacity is
less than 60 percent, you should think about replacing the battery.

\hypertarget{do-a-deep-clean}{%
\subsection{Do a Deep Clean}\label{do-a-deep-clean}}

Gadgets need regular cleaning. Dirt and debris clogging up our equipment
can contribute to overheating, which shortens the life of our
electronics.

So Mr. Wiens recommended this regular cleaning routine:

\begin{itemize}
\item
  For mobile devices, look inside the ports. Dust and food crumbs easily
  get lodged inside the charging ports, which causes the phone to charge
  more slowly. Shine a flashlight into the port for a close look. Use
  canned aerosol or a sewing needle to remove any debris.
\item
  For computers, blow out the fans once a year. Opening up the case of a
  laptop or desktop machine reveals its fans. The fans accumulate lots
  of dust and hair over time, and the dirtier the fans are, the hotter
  your device gets. Take a small vacuum cleaner or a can of aerosol to
  get rid of the gunk.
\end{itemize}

\hypertarget{declutter-your-data}{%
\subsection{Declutter Your Data}\label{declutter-your-data}}

The more device storage you use up, the slower a gadget gets. So set a
calendar reminder to do a data purge at least once a year.

On iPhones, Apple offers the
tool\href{https://support.apple.com/en-us/HT201656}{iPhone Storage},
which shows a list of apps that take up the most data and when they were
last used; on Android devices, Google offers a similar tool called
\href{https://play.google.com/store/apps/details?id=com.google.android.apps.nbu.files\&hl=en_US}{Files}.
Use these tools to delete any apps that you haven't touched in more than
six months.

To do a quick purge with computers, open a folder and sort the files by
when they were last opened. From there, you can immediately eliminate
files and apps you have not opened in years.

If purging files is too cumbersome, there are shortcuts. Some Android
phones have a slot for inserting a memory card, where you can load games
and videos. Similarly, with computers you can plug in an external hard
drive and store large files there. That will free up storage on the
device so that the operating system runs faster, said Vincent Lai,
director of the Fixers Collective, a social club in New York that
repairs aging devices to extend their lives.

\hypertarget{protect-your-gear}{%
\subsection{Protect Your Gear}\label{protect-your-gear}}

Most smartphone owners already know to wear
\href{https://www.npd.com/wps/portal/npd/us/news/press-releases/one-in-four-smartphone-owners-dont-use-a-case-to-protect-their-phone-according-to-npd/}{cases
to protect their devices from drops}. That's wise --- a good case
protects your phone from scratches and absorbs impact in the corners,
edges and the back of your device. Carrying a phone without a case is
similar to driving a car without bumpers.

Especially if you're accident prone, a screen protector is another
safeguard. Small scratches on a screen can weaken the glass, increasing
the likelihood for it to shatter the next time it's dropped. Wirecutter,
a New York Times site that tests products,
\href{https://www.nytimes3xbfgragh.onion/wirecutter/reviews/best-iphone-x-screen-protectors/}{recommends
protectors} from Maxboost and TechMatte, which cost \$8 to \$14.

\hypertarget{find-a-fixer}{%
\subsection{Find a Fixer}\label{find-a-fixer}}

If any of the above intimidates you, there are plenty of professionals
who can help.

A few weeks ago, I opened my iPhone camera app and noticed something
odd: The image was shaky even though my hand was still. I concluded
something had gone wrong with the hardware.

So I went on Yelp and searched ``iPhone repair.'' The site loaded a tool
for getting quotes from multiple fixers in my area. About a dozen repair
shops all responded within a day, and I was surprised by how widely
their responses varied. Some quoted \$80 to \$100 and said the repair
would take one or two hours. Others said the job could take up to two
days and cost \$140 to \$180.

To cross-reference, I checked iFixit's guide on replacing an iPhone
camera, which said the repair should take about an hour and the part
could cost \$120.

The response from
\href{https://www.yelp.com/biz/shakeel-the-iphone-repair-guy-south-san-francisco-2}{Shakeel
the iPhone Repair Guy}, Mr. Taiyab's shop, was the most terse: ``I know
exactly what's wrong with your camera. Please bring it by --- I can fix
your phone within five minutes.'' His quote was \$80.

His repair shop had more than 1,000 Yelp reviews with an average of five
stars. I scanned them and saw no warning signs of fakes. Above all, I
liked the confidence in his response.

I drove to his shop. There, Mr. Taiyab quickly opened up my phone with a
screwdriver and said a moving part inside the camera, which controls the
optical zoom, had broken --- possibly from a drop --- so the camera
module had to be replaced. True to his word, he replaced it in a few
minutes and charged \$80. My photos now look as clear as they did when
the phone was brand-new.

The lessons? When collecting quotes, pay attention to the price and
customer reviews, and do a web search on the repair to get a realistic
idea of the true cost and difficulty of the task. (Just because a repair
shop says it will take longer doesn't mean it will do a better job.)

The most important lesson: When you find a good gadget repair shop,
reward it with your loyalty. I took my wife's seven-year-old MacBook Pro
to Mr. Taiyab's this week to replace the battery. It took him 15 minutes
and cost \$140 --- time and money well spent for a computer that should
last us many years to come.

Advertisement

\protect\hyperlink{after-bottom}{Continue reading the main story}

\hypertarget{site-index}{%
\subsection{Site Index}\label{site-index}}

\hypertarget{site-information-navigation}{%
\subsection{Site Information
Navigation}\label{site-information-navigation}}

\begin{itemize}
\tightlist
\item
  \href{https://help.nytimes3xbfgragh.onion/hc/en-us/articles/115014792127-Copyright-notice}{©~2020~The
  New York Times Company}
\end{itemize}

\begin{itemize}
\tightlist
\item
  \href{https://www.nytco.com/}{NYTCo}
\item
  \href{https://help.nytimes3xbfgragh.onion/hc/en-us/articles/115015385887-Contact-Us}{Contact
  Us}
\item
  \href{https://www.nytco.com/careers/}{Work with us}
\item
  \href{https://nytmediakit.com/}{Advertise}
\item
  \href{http://www.tbrandstudio.com/}{T Brand Studio}
\item
  \href{https://www.nytimes3xbfgragh.onion/privacy/cookie-policy\#how-do-i-manage-trackers}{Your
  Ad Choices}
\item
  \href{https://www.nytimes3xbfgragh.onion/privacy}{Privacy}
\item
  \href{https://help.nytimes3xbfgragh.onion/hc/en-us/articles/115014893428-Terms-of-service}{Terms
  of Service}
\item
  \href{https://help.nytimes3xbfgragh.onion/hc/en-us/articles/115014893968-Terms-of-sale}{Terms
  of Sale}
\item
  \href{https://spiderbites.nytimes3xbfgragh.onion}{Site Map}
\item
  \href{https://help.nytimes3xbfgragh.onion/hc/en-us}{Help}
\item
  \href{https://www.nytimes3xbfgragh.onion/subscription?campaignId=37WXW}{Subscriptions}
\end{itemize}
