\href{/section/business}{Business}\textbar{}In the Covid-19 Economy, You
Can Have a Kid or a Job. You Can't Have Both.

\url{https://nyti.ms/3ggPKir}

\begin{itemize}
\item
\item
\item
\item
\item
\item
\end{itemize}

\href{https://www.nytimes3xbfgragh.onion/spotlight/at-home?action=click\&pgtype=Article\&state=default\&region=TOP_BANNER\&context=at_home_menu}{At
Home}

\begin{itemize}
\tightlist
\item
  \href{https://www.nytimes3xbfgragh.onion/2020/07/28/books/time-for-a-literary-road-trip.html?action=click\&pgtype=Article\&state=default\&region=TOP_BANNER\&context=at_home_menu}{Take:
  A Literary Road Trip}
\item
  \href{https://www.nytimes3xbfgragh.onion/2020/07/29/magazine/bored-with-your-home-cooking-some-smoky-eggplant-will-fix-that.html?action=click\&pgtype=Article\&state=default\&region=TOP_BANNER\&context=at_home_menu}{Cook:
  Smoky Eggplant}
\item
  \href{https://www.nytimes3xbfgragh.onion/2020/07/27/travel/moose-michigan-isle-royale.html?action=click\&pgtype=Article\&state=default\&region=TOP_BANNER\&context=at_home_menu}{Look
  Out: For Moose}
\item
  \href{https://www.nytimes3xbfgragh.onion/interactive/2020/at-home/even-more-reporters-editors-diaries-lists-recommendations.html?action=click\&pgtype=Article\&state=default\&region=TOP_BANNER\&context=at_home_menu}{Explore:
  Reporters' Obsessions}
\end{itemize}

\includegraphics{https://static01.graylady3jvrrxbe.onion/images/2020/07/05/business/02Schoolparent-illo/02Schoolparent-illo-articleLarge.jpg?quality=75\&auto=webp\&disable=upscale}

Sections

\protect\hyperlink{site-content}{Skip to
content}\protect\hyperlink{site-index}{Skip to site index}

\hypertarget{in-the-covid-19-economy-you-can-have-a-kid-or-a-job-you-cant-have-both}{%
\section{In the Covid-19 Economy, You Can Have a Kid or a Job. You Can't
Have
Both.}\label{in-the-covid-19-economy-you-can-have-a-kid-or-a-job-you-cant-have-both}}

Our struggle is not an emotional concern. We are not burned out. We are
being crushed by an economy that has bafflingly declared working parents
inessential.

Credit...Taylor Callery

Supported by

\protect\hyperlink{after-sponsor}{Continue reading the main story}

By Deb Perelman

\begin{itemize}
\item
  Published July 2, 2020Updated July 8, 2020
\item
  \begin{itemize}
  \item
  \item
  \item
  \item
  \item
  \item
  \end{itemize}
\end{itemize}

\href{https://www.nytimes3xbfgragh.onion/es/2020/07/09/espanol/estilos-de-vida/trabajo-hijos-coronavirus.html}{Leer
en español}

\hypertarget{listen-to-this-article}{%
\subsubsection{Listen to This Article}\label{listen-to-this-article}}

Audio Recording by Audm

\emph{To hear more audio stories from publishers like The New York
Times,
download}\href{https://www.audm.com/?utm_source=nytmag\&utm_medium=embed\&utm_campaign=left_behind_draper}{**}\href{https://www.audm.com/?utm_source=nyt\&utm_medium=embed\&utm_campaign=covid_kid_job}{\emph{Audm
for iPhone or Android}}\emph{.}

Last week, I received an email from my children's principal, sharing
some of the first details about plans to reopen New York City schools
this fall. The message explained that the city's Department of
Education, following federal guidelines, will require each student to
have 65 square feet of classroom space. Not everyone will be allowed in
the building at once. The upshot is that my children will be able to
\href{https://www.nytimes3xbfgragh.onion/2020/06/26/us/coronavirus-schools-reopen-fall.html}{physically
attend school} one out of every three weeks.

At the same time, many adults --- at least the lucky ones that have held
onto their jobs --- are supposed to be back at work as the economy
reopens. What is confusing to me is that these two plans are moving
forward apace without any consideration of the
\href{https://www.nytimes3xbfgragh.onion/2020/07/08/us/drisana-rios-lawsuit-hub-international.html}{working
parents} who will be ground up in the gears when they collide.

Let me say the quiet part loud: In the Covid-19 economy, you're allowed
only a kid \emph{or} a job.

Why isn't anyone talking about this? Why are we not hearing a primal
scream so deafening that no plodding policy can be implemented without
addressing the people buried by it? Why am I, a food blogger best known
for such hits as the All-Butter Really Flaky Pie Dough and The `I Want
Chocolate Cake' Cake, sounding the alarm on this? I think it's because
when
\href{https://www.nytimes3xbfgragh.onion/2020/04/27/upshot/coronavirus-exposes-workplace-truths.html?action=click\&module=RelatedLinks\&pgtype=Article}{you're
home schooling all day}, and not performing the work you were hired to
do until the wee hours of the morning, and do it on repeat for 106 days
(not that anyone is counting), you might be a bit too fried to funnel
your rage effectively.

For months, I've been muttering about this --- in group texts, in secret
Facebook groups for moms, in masked encounters when I bump into a parent
friend on the street. We all ask one another why we aren't making more
noise. The consensus is that everyone agrees this is a catastrophe, but
we are
\href{https://www.nytimes3xbfgragh.onion/2020/04/27/nyregion/coronavirus-homeschooling-parents.html?action=click\&module=RelatedLinks\&pgtype=Article}{too
bone-tired to raise our voices} above a groan, let alone scream through
a megaphone. Every single person
\href{https://www.nytimes3xbfgragh.onion/2020/06/29/opinion/coronavirus-school-reopening.html}{confesses
burnout, despair,} feeling like they are losing their minds, knowing in
their guts that this is untenable.

It should be obvious, but a nonnegotiable precondition of ``getting back
to normal'' is that families need a normal to return to as well. But as
soon as you express this, the conversation quickly gets clouded with
tangential and irrelevant arguments that would get you kicked off any
school debate team.

``But we don't even know if it's
\href{https://www.nytimes3xbfgragh.onion/2020/06/12/upshot/epidemiologists-decisions-children-school-coronavirus.html}{safe
to send kids back to school},'' is absolutely correct, but it's not the
central issue here. The sadder flip side --- the friend who told me that
if their school reopens, her children are going back whether it's safe
or not because she cannot afford to not work --- edges closer.

``Why do you want teachers to get sick?'' isn't my agenda either, but
it's hard to imagine that a system in which each child will spend two
weeks out of every three being handed off among various caretakers only
to reconvene in a classroom, infinitely increasing the number of
potential virus-carrying interactions, protects a teacher more than a
consistent pod of students week in and out with minimized external
interactions.

``You shouldn't have had kids if you can't take care of them,'' is
comically troll-like, but has come up so often, one might wonder if
you're supposed to educate your children at night. Or perhaps you should
have been paying for some all-age day care backup that sat empty while
kids were at school in case the school you were paying taxes to keep
open and that requires, by law, that your child attend abruptly closed
for the year.

``Why aren't you enjoying the extra quality time with your kid?'' lays
bare what is really simmering below the surface --- a retrograde view
that maybe one parent
(\href{https://www.nytimes3xbfgragh.onion/2020/06/03/business/economy/coronavirus-working-women.html}{they
mean the mom}) shouldn't be working, that doing so is bad for children,
that it's selfish to pursue financial gains (or solvency, as working
parents will tell you). It is a sentiment so deeply woven into our
cultural psyche that making the reasonable suggestion that one shouldn't
have to abandon a career or livelihood if offices reopen before schools,
day cares and camps do is viewed as a chance to redeliberate this.

It is not, and you're off the debate team, too.

I've heard from parents who have the luck of a grandparent who can swoop
in, or the deep pockets for a full-time nanny or a private tutor for
their child when schools are closed. That all sounds enviable, but it
would be absurd to let policy be guided by people with cushioning. If
you have the privilege to opt out of the work force and wish to, enjoy
it. But don't wield it as a stick to poke others with because far more
people are being forced to ``opt out'' this year and will never
professionally or financially recover.

I resent articles that view the struggle of working parents this year as
an emotional concern. We are not burned out because life is hard this
year. We are burned out because we are being rolled over by the wheels
of an economy that has bafflingly declared working parents inessential.

\includegraphics{https://static01.graylady3jvrrxbe.onion/images/2020/07/02/business/02SchoolParent-02/merlin_171532341_6bf69907-be52-4e33-902b-e0104088b403-articleLarge.jpg?quality=75\&auto=webp\&disable=upscale}

\hypertarget{part-time-teachers-full-time-parents}{%
\subsection{Part-time teachers, full-time
parents}\label{part-time-teachers-full-time-parents}}

For context, let me tell you how the last few months have been for my
family. The first few weeks of
\href{https://www.nytimes3xbfgragh.onion/2020/05/01/nyregion/coronavirus-new-york-update.html}{school
and business closures} were jaw-clenchingly stressful. I am
self-employed and worked full-time from home already, so that part
required no transition. But I needed to use this flexibility to ensure
that my husband, who would normally have been at his office, didn't miss
a meeting, call or email, while I managed the
\href{https://www.nytimes3xbfgragh.onion/2020/06/13/health/school-learning-online-education.html}{remote-learning
curriculums} of our two children, one in pre-K, one in fifth grade. I
compensated by working until about 2 a.m. each night.

Three weeks later, our marital work-balance stress evaporated as my
husband was put on furlough. He took over home schooling and basically
everything else as I became the sole breadwinner, trying to work as hard
as I could, at every hour. Last week, he was fully laid off.

Despite our own financial strain, we've continued to pay the nanny who
used to help shuttle the kids around while we worked, even though she
hasn't worked for us since March. Even if we asked for her help in home
schooling our children this fall, who would do so for her school-age
children? When will my husband be able to look for work? How can he go
back to work if there's no one to watch the kids?

And I speak from a position of significant privilege. We were, until
recently, a two-income family with savings, paying for more than the
minimum of child care hours that we needed each day just to cover
what-ifs, living in one of the most expensive cities on earth. We have
laptops, tablets, Wi-Fi, and didn't think twice before panic-ordering
pencils, paper, markers and anything else we thought might help our
children.

But my family, as a social and economic unit, cannot operate forever in
the framework authorities envision for the fall. There are so many ways
that the situation we've been thrust into, in which businesses are
planning to reopen without any conversation about the repercussions on
families with school-age children, is even more untenable for others.

Under the best of circumstances, the
\href{https://www.nytimes3xbfgragh.onion/2020/06/05/us/coronavirus-education-lost-learning.html}{impact
on children will still be significant}. Students will lose most of a
year of learning as parents --- their new untrained teachers --- cannot
supervise in any meaningful way while Zooming into the office. At best,
the kids will be crabby and stir-crazy as they don't get enough physical
activity because they're now tethered to their parents' work spaces all
day, running around the living room in lieu of fresh air. Without social
interactions with other children, they constantly seek parental
attention in bad ways, further straining the mood at home. And these are
ideal scenarios.

But what about kids who cannot learn remotely? What about kids who need
services that are tied to schools? Or those who are at higher risk for
complications if they get the virus and might not be able to go back
even one week out of the three?

When learning plans for children with special needs could not be
followed appropriately this year, academic gains for many students were
quickly wiped out. Remote learning has already widened racial and
socioeconomic achievement gaps because of disparities in access to
technology tutors. As parents are crushed by the Covid economy, so are
the children who need the most support. It's no wonder the
\href{https://www.nytimes3xbfgragh.onion/2020/06/30/us/coronavirus-schools-reopening-guidelines-aap.html}{American
Academy of Pediatrics released a statement}this weekend urging that
students be physically present in school as much as possible this fall.

The long-term losses for professional adults will be incalculable, too,
and will disproportionately affect mothers. Working mothers all over the
country feel that they're being pushed out of the labor force or into
part-time jobs as their responsibilities at home have increased tenfold.

Even those who found a short-term solution because they had the luxury
to hit the pause button on their projects and careers this spring to
manage the effects of the pandemic --- predicated on the assumption that
the fall would bring a return to school and child care --- may now have
no choice but to leave the work force. A friend just applied for a job
and tells me she cannot even imagine how she would be able to take it if
her children aren't truly back in school. There's an idea that people
can walk away from careers and just pick them up where they left off,
even though we know that women who
\href{https://www.nytimes3xbfgragh.onion/2014/09/07/upshot/a-child-helps-your-career-if-youre-a-man.html}{drop
out of the work force} to take care of children often have trouble
getting back in.

And lest you think it's everyone vs. teachers, I cannot imagine a group
this situation is less fair to. Teachers are supposed to teach in the
classroom full-time but simultaneously manage remote learning? Even in
non-pandemic times, teachers would tell you that they already work
unpaid overtime on nights and weekends, just planning and grading.
Where, exactly, will the extra hours come from? For teachers with their
own school-age children, the situation isn't just untenable, it's
impossible.

\hypertarget{the-wealthy-win-again}{%
\subsection{The wealthy win. Again.}\label{the-wealthy-win-again}}

Without a doubt, reopening schools is a colossal undertaking. There are
no easy solutions to finding enough space for students to socially
distance, ensuring teachers and staff are protected, adding more sinks
and cleaning staff, and implementing widespread temperature checks,
testing and contact tracing.

But after nearly four months since the lockdowns began --- four months
of working all hours, at remarkable stress levels, while our children
have gone without play dates and playgrounds and all of the other
stimuli that help them thrive --- most parents have been shocked to find
that state governments don't have any creative or even plausible
solutions.

For parents who cannot simply sort it out, our national response feels
more like a dystopian novel where only the wealthy get to limit their
exposure and survive the pandemic unscathed. Allowing workplaces to
reopen while schools, camps and day cares remain closed tells a
generation of working parents that it's fine if they lose their jobs,
insurance and livelihoods in the process. It's outrageous, and I fear if
we don't make the loudest amount of noise possible over this, we will be
erased from the economy.

Deb Perelman is a New York writer and the creator of the food blog
\href{http://smittenkitchen.com/}{smittenkitchen.com}.

Advertisement

\protect\hyperlink{after-bottom}{Continue reading the main story}

\hypertarget{site-index}{%
\subsection{Site Index}\label{site-index}}

\hypertarget{site-information-navigation}{%
\subsection{Site Information
Navigation}\label{site-information-navigation}}

\begin{itemize}
\tightlist
\item
  \href{https://help.nytimes3xbfgragh.onion/hc/en-us/articles/115014792127-Copyright-notice}{©~2020~The
  New York Times Company}
\end{itemize}

\begin{itemize}
\tightlist
\item
  \href{https://www.nytco.com/}{NYTCo}
\item
  \href{https://help.nytimes3xbfgragh.onion/hc/en-us/articles/115015385887-Contact-Us}{Contact
  Us}
\item
  \href{https://www.nytco.com/careers/}{Work with us}
\item
  \href{https://nytmediakit.com/}{Advertise}
\item
  \href{http://www.tbrandstudio.com/}{T Brand Studio}
\item
  \href{https://www.nytimes3xbfgragh.onion/privacy/cookie-policy\#how-do-i-manage-trackers}{Your
  Ad Choices}
\item
  \href{https://www.nytimes3xbfgragh.onion/privacy}{Privacy}
\item
  \href{https://help.nytimes3xbfgragh.onion/hc/en-us/articles/115014893428-Terms-of-service}{Terms
  of Service}
\item
  \href{https://help.nytimes3xbfgragh.onion/hc/en-us/articles/115014893968-Terms-of-sale}{Terms
  of Sale}
\item
  \href{https://spiderbites.nytimes3xbfgragh.onion}{Site Map}
\item
  \href{https://help.nytimes3xbfgragh.onion/hc/en-us}{Help}
\item
  \href{https://www.nytimes3xbfgragh.onion/subscription?campaignId=37WXW}{Subscriptions}
\end{itemize}
