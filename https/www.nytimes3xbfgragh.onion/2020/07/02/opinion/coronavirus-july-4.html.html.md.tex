Sections

SEARCH

\protect\hyperlink{site-content}{Skip to
content}\protect\hyperlink{site-index}{Skip to site index}

\href{https://myaccount.nytimes3xbfgragh.onion/auth/login?response_type=cookie\&client_id=vi}{}

\href{https://www.nytimes3xbfgragh.onion/section/todayspaper}{Today's
Paper}

\href{/section/opinion}{Opinion}\textbar{}The National Humiliation We
Need

\url{https://nyti.ms/38kCOp5}

\begin{itemize}
\item
\item
\item
\item
\item
\end{itemize}

Advertisement

\protect\hyperlink{after-top}{Continue reading the main story}

\href{/section/opinion}{Opinion}

Supported by

\protect\hyperlink{after-sponsor}{Continue reading the main story}

\hypertarget{the-national-humiliation-we-need}{%
\section{The National Humiliation We
Need}\label{the-national-humiliation-we-need}}

July 4 and America's crisis of the spirit.

\href{https://www.nytimes3xbfgragh.onion/by/david-brooks}{\includegraphics{https://static01.graylady3jvrrxbe.onion/images/2018/04/03/opinion/david-brooks/david-brooks-thumbLarge-v2.png}}

By \href{https://www.nytimes3xbfgragh.onion/by/david-brooks}{David
Brooks}

Opinion Columnist

\begin{itemize}
\item
  July 2, 2020
\item
  \begin{itemize}
  \item
  \item
  \item
  \item
  \item
  \end{itemize}
\end{itemize}

\includegraphics{https://static01.graylady3jvrrxbe.onion/images/2020/07/02/opinion/02brooksWeb/02brooksWeb-articleLarge.jpg?quality=75\&auto=webp\&disable=upscale}

We Americans enter the July 4 weekend of 2020 humiliated as almost never
before. We had one collective project this year and that was to crush
Covid-19, and we failed.

On Wednesday, we had about 50,000 new positive tests, a record. Other
nations are beating the disease while our infection lines shoot upward
as sharply as they did in March.

This failure will lead to other failures. A third of Americans show
signs of clinical anxiety or depression,
\href{https://www.washingtonpost.com/health/2020/05/26/americans-with-depression-anxiety-pandemic/?arc404=true}{according
to} the Census Bureau. Suspected drug overdose deaths
\href{https://www.washingtonpost.com/health/2020/07/01/coronavirus-drug-overdose/}{surged}
by 42 percent in May. Small businesses, colleges and community hubs will
close.

At least Americans are not in denial about the nation's turmoil of the
last three months. According to a
\href{https://www.people-press.org/2020/06/30/publics-mood-turns-grim-trump-trails-biden-on-most-personal-traits-major-issues/}{Pew
survey}, 71 percent of Americans are angry about the state of the
country right now and 66 percent are fearful. Only 17 percent are proud.

Americans are reacting in two positive ways. We're seeing incredible
shifts in attitudes toward race. Roughly 60 percent of Americans now
believe that African-Americans face a great deal or a lot of
discrimination. People have been waiting for a white backlash since the
riots, or since the statues started toppling. There isn't much if any
evidence of a backlash. There's evidence of a fore-lash.

Second, Americans have decided to get rid of Donald Trump. His
mishandling of Covid-19 hurt him among seniors. His racist catcalls in a
time of racial reckoning have damaged him among all groups.

I'll be delighted when Trump goes, but it's worth pointing out that it
wasn't only because of Donald Trump that Americans never really locked
down, and then started moving around again in late April.

It wasn't Trump who went out to bars in Tempe, Austin and Los Angeles in
June. It wasn't Trump who put on hospital gowns and told the American
people you could suspend the lockdown if your cause was just. Once you
told people they could suspend the lockdown for one thing, they were
going to suspend it for others.

Our fixation on the awfulness of Donald Trump has distracted us from the
larger problems and rendered us strangely passive in the face of them.
Sure, this was a Republican failure, but it was also a collective
failure, and it follows a few decades of collective failures.

On the day Trump leaves office, we'll still have a younger generation
with worse life prospects than their parents had faced. We'll still have
a cultural elite that knows little about people in red America and daily
sends the message that they are illegitimate. We'll still have yawning
inequalities, residential segregation, crumbling social capital, a
crisis in family formation.

Trump's rise in 2016 was a symptom of all these crises, long before he
had a chance to become an additional cause of them.

What's the core problem? Damon Linker
\href{https://theweek.com/articles/922812/coronavirus-revealing-shattered-country}{is
on to a piece of it}: ``It amounts to a refusal on the part of lots of
Americans to think in terms of the social whole --- of what's best for
the community, of the common or public good. Each of us thinks we know
what's best for ourselves.''

I'd add that this individualism, atomism and selfishness is downstream
from a deeper crisis of legitimacy. In 1970, in a moment like our own,
Irving Kristol
\href{https://www.nationalaffairs.com/public_interest/detail/when-virtue-loses-all-her-loveliness-some-reflections-on-capitalism-and-the-free-society}{wrote},
``In the same way as men cannot for long tolerate a sense of spiritual
meaninglessness in their individual lives, so they cannot for long
accept a society in which power, privilege, and property are not
distributed according to some morally meaningful criteria.''

A lot of people look around at the conditions of this country --- how
Black Americans are treated, how communities are collapsing, how
Washington doesn't work --- and none of it makes sense. None of it
inspires faith, confidence. In none of it do they feel a part.

If you don't breathe the spirit of the nation, if you don't have a
fierce sense of belonging to each other, you're not going to sacrifice
for the common good. We're confronted with a succession of wicked
problems and it turns out we're not even capable of putting on a
friggin' mask.

In the days leading up to this July 4 weekend, I've been thinking about
a scene in ``Good Will Hunting.'' We've seen Will perform all these
mathematical feats and flights of verbal brilliance, but the Robin
Williams character sits him down on a park bench and confronts him with
a rot at the core of his character. ``I look at you; I don't see an
intelligent, confident man; I see a cocky, scared'' kid.

The last three years have been like that Robin Williams speech for a
whole nation --- an intervention, a truth-telling. I had hopes that the
crisis would bring us together, but it's made everything harder and
worse. And now I worry less about populism or radical wokeness than
about a pervasive loss of national faith.

What's lurking, I hope, somewhere deep down inside is our shared
ferocious love for our common country and a vision for the role America
could play as the great pluralist beacon of the 21st century.

July 4 would be a good day to find that faith.

\emph{The Times is committed to publishing}
\href{https://www.nytimes3xbfgragh.onion/2019/01/31/opinion/letters/letters-to-editor-new-york-times-women.html}{\emph{a
diversity of letters}} \emph{to the editor. We'd like to hear what you
think about this or any of our articles. Here are some}
\href{https://help.nytimes3xbfgragh.onion/hc/en-us/articles/115014925288-How-to-submit-a-letter-to-the-editor}{\emph{tips}}\emph{.
And here's our email:}
\href{mailto:letters@NYTimes.com}{\emph{letters@NYTimes.com}}\emph{.}

\emph{Follow The New York Times Opinion section on}
\href{https://www.facebookcorewwwi.onion/nytopinion}{\emph{Facebook}}\emph{,}
\href{http://twitter.com/NYTOpinion}{\emph{Twitter (@NYTopinion)}}
\emph{and}
\href{https://www.instagram.com/nytopinion/}{\emph{Instagram}}\emph{.}

Advertisement

\protect\hyperlink{after-bottom}{Continue reading the main story}

\hypertarget{site-index}{%
\subsection{Site Index}\label{site-index}}

\hypertarget{site-information-navigation}{%
\subsection{Site Information
Navigation}\label{site-information-navigation}}

\begin{itemize}
\tightlist
\item
  \href{https://help.nytimes3xbfgragh.onion/hc/en-us/articles/115014792127-Copyright-notice}{©~2020~The
  New York Times Company}
\end{itemize}

\begin{itemize}
\tightlist
\item
  \href{https://www.nytco.com/}{NYTCo}
\item
  \href{https://help.nytimes3xbfgragh.onion/hc/en-us/articles/115015385887-Contact-Us}{Contact
  Us}
\item
  \href{https://www.nytco.com/careers/}{Work with us}
\item
  \href{https://nytmediakit.com/}{Advertise}
\item
  \href{http://www.tbrandstudio.com/}{T Brand Studio}
\item
  \href{https://www.nytimes3xbfgragh.onion/privacy/cookie-policy\#how-do-i-manage-trackers}{Your
  Ad Choices}
\item
  \href{https://www.nytimes3xbfgragh.onion/privacy}{Privacy}
\item
  \href{https://help.nytimes3xbfgragh.onion/hc/en-us/articles/115014893428-Terms-of-service}{Terms
  of Service}
\item
  \href{https://help.nytimes3xbfgragh.onion/hc/en-us/articles/115014893968-Terms-of-sale}{Terms
  of Sale}
\item
  \href{https://spiderbites.nytimes3xbfgragh.onion}{Site Map}
\item
  \href{https://help.nytimes3xbfgragh.onion/hc/en-us}{Help}
\item
  \href{https://www.nytimes3xbfgragh.onion/subscription?campaignId=37WXW}{Subscriptions}
\end{itemize}
