Sections

SEARCH

\protect\hyperlink{site-content}{Skip to
content}\protect\hyperlink{site-index}{Skip to site index}

\href{https://myaccount.nytimes3xbfgragh.onion/auth/login?response_type=cookie\&client_id=vi}{}

\href{https://www.nytimes3xbfgragh.onion/section/todayspaper}{Today's
Paper}

\href{/section/opinion}{Opinion}\textbar{}Trump's Virus Is Spreading,
and His Economy Is Stalling

\url{https://nyti.ms/31EpHxJ}

\begin{itemize}
\item
\item
\item
\item
\item
\item
\end{itemize}

Advertisement

\protect\hyperlink{after-top}{Continue reading the main story}

\href{/section/opinion}{Opinion}

Supported by

\protect\hyperlink{after-sponsor}{Continue reading the main story}

\hypertarget{trumps-virus-is-spreading-and-his-economy-is-stalling}{%
\section{Trump's Virus Is Spreading, and His Economy Is
Stalling}\label{trumps-virus-is-spreading-and-his-economy-is-stalling}}

The president leads a transition to sickness.

\href{https://www.nytimes3xbfgragh.onion/by/paul-krugman}{\includegraphics{https://static01.graylady3jvrrxbe.onion/images/2018/04/02/opinion/paul-krugman/paul-krugman-thumbLarge.png}}

By \href{https://www.nytimes3xbfgragh.onion/by/paul-krugman}{Paul
Krugman}

Opinion Columnist

\begin{itemize}
\item
  July 2, 2020
\item
  \begin{itemize}
  \item
  \item
  \item
  \item
  \item
  \item
  \end{itemize}
\end{itemize}

\includegraphics{https://static01.graylady3jvrrxbe.onion/images/2020/07/03/opinion/02krugman1/merlin_174147207_762954af-fe13-4195-8a20-30185ae658ca-articleLarge.jpg?quality=75\&auto=webp\&disable=upscale}

Just over two weeks ago The Wall Street Journal published an
\href{https://www.wsj.com/articles/there-isnt-a-coronavirus-second-wave-11592327890}{opinion
piece} by Vice President Mike Pence titled ``There Isn't a Coronavirus
`Second Wave.''' The article was supposed to reassure the nation.

What it provided, instead, was a clear illustration of the delusions and
magical thinking that have marked every step of the Trump
administration's response to Covid-19, producing an epic policy
disaster.

Put it this way: By now, according to Trump officials and sycophants, we
were supposed to be seeing a fading pandemic and a roaring recovery.
Instead, we have a fading recovery and a roaring pandemic.

About the pandemic: The Pence article cheerily declared that ``cases
have stabilized,'' with the daily average number of new cases only
20,000. Even that figure, as it happens, was five times the number in
the
\href{https://ourworldindata.org/coronavirus-data-explorer?zoomToSelection=true\&casesMetric=true\&dailyFreq=true\&smoothing=7\&country=USA~EuropeanUnion\&pickerMetric=location\&pickerSort=asc}{European
Union}, which has a third more people than America does. Since then,
however, new cases have soared, hitting
\href{https://covidtracking.com/data}{more than 50,000} by
\href{https://www.nytimes3xbfgragh.onion/interactive/2020/us/coronavirus-us-cases.html}{some
counts} on Wednesday.

Indeed, at this point Arizona, with seven million people, is reporting
around as many new cases each day as the whole E.U., with 446 million
people.

Some Trump supporters are still trying to dismiss the upswing in cases
as an illusion created by more testing. But it isn't. Cases have grown
\href{https://covidtracking.com/data}{far more than testing} has.
Hospitalizations have shot up in
\href{https://covidtracking.com/data/state/arizona}{Arizona} and
\href{https://covidtracking.com/data/state/texas}{Texas}, which are at
the leading edge of the new surge; in both states, hospitals are in
\href{https://abcnews.go.com/US/leaving-hospital-tears-arizona-doctors-worry-icus-fill/story?id=71519097}{crisis}
\href{https://www.propublica.org/article/internal-messages-reveal-crisis-at-houston-hospitals-as-coronavirus-cases-surge}{mode}.
(Florida, which is probably in the same situation, hasn't been releasing
hospitalization data.)

The one piece of slightly good news is that deaths from the coronavirus
are still falling, in part because the new wave of infections is hitting
people younger than the first wave did, in part perhaps because doctors
have gotten better at treating the disease. But Covid-19 can be
debilitating and cause
\href{https://www.nytimes3xbfgragh.onion/2020/07/01/health/coronavirus-recovery-survivors.html?action=click\&module=Top\%20Stories\&pgtype=Homepage}{long-term
damage}, even when it doesn't kill.

Also, deaths are a lagging indicator. In Arizona, where the jump in
cases began about two weeks before the rest of the Sunbelt,
\href{https://twitter.com/COVID19Tracking/status/1278479854715809797}{deaths
are rising}.

The thing is, Covid-19's resurgence was utterly predictable --- and
predicted. When Donald Trump declared that we would
``\href{https://www.prnewsonline.com/transition-to-greatness-web-domain-trump}{transition
to greatness}'' --- which is to say, rush to reopen the economy despite
a still-rampant pandemic --- epidemiologists warned that this could set
off a new wave of infections. They were right.

And economists \href{https://t.co/4LxtfRQW1T?amp=1}{warned} that while
relaxing social distancing would lead to a brief period of job growth,
these gains would be short-lived, that premature reopening would be
self-defeating even in economic terms. They were also right.

Don't be fooled by the big jobs number in Thursday's employment report
--- a number that still left us down
\href{https://fred.stlouisfed.org/series/PAYEMS}{almost 15 million jobs}
from February. The report was a snapshot of the economy during the
``\href{https://t.co/PwzxFkZJGI?amp=1}{reference period},'' basically
the second week of June. So it's telling us what was happening
\emph{before} the Covid-19 surge became apparent.

We don't have official data for what has happened since then, but a
variety of real-time indicators
\href{https://www.nytimes3xbfgragh.onion/2020/07/01/business/economic-recovery-virus-surge.html}{suggest}
that the recovery has stalled or even gone backward. Indeed, things
started falling apart even before states began reversing some of their
previous moves to reopen. \href{https://www.nber.org/papers/w27432}{Fear
of infection} will do that: Many people will avoid going out whatever
their governors may say.

As a result, unemployment, still in double digits, probably won't get
much better for a long time.

Now, there isn't a one-to-one correspondence between jobs and the spread
of the pandemic. If we had all worn masks and avoided stupid policies
like reopening bars and resuming large indoor gatherings, we probably
could have had \href{https://www.nber.org/papers/w27244}{substantial job
gains} without surging infections. But we didn't, largely because Trump
and Republican governors refused to take sensible actions (and in many
cases prevented mayors and other local officials from acting sensibly on
their own).

Nor can we simply hit the reset button. Activities we could have safely
resumed two months ago, when infection rates were low, aren't safe to
continue given today's much higher Covid-19 prevalence. That is, we're
in worse shape, even economically, than we would have been if Trump and
his allies had taken the pandemic seriously early on.

The really frightening aspect about where we are is that Trump and his
people don't seem to have learned anything from their coronavirus
debacle. On Wednesday --- Wednesday! --- Trump insisted, as he has at
every stage of the pandemic, that the coronavirus will
``\href{https://www.cnn.com/2020/07/01/politics/donald-trump-masks-coronavirus/index.html}{sort
of just disappear}.''

And the Trumpists are crowing about the June employment number, with no
apparent awareness that it's out of date and the situation has probably
worsened in recent weeks.

The sad, even terrifying thing is that Trumpian delusions of success
will impose a heavy price on the rest of us.

Right now we should be going all-out to bring the Covid-19 surge under
control and making sure that Americans keep getting the economic aid
they need. In reality, neither of those things is likely to happen.
Infections and hospitalizations will soar further, and millions of
Americans will lose crucial economic lifelines in a few weeks.

The next four months are going to be very, very ugly.

\emph{The Times is committed to publishing}
\href{https://www.nytimes3xbfgragh.onion/2019/01/31/opinion/letters/letters-to-editor-new-york-times-women.html}{\emph{a
diversity of letters}} \emph{to the editor. We'd like to hear what you
think about this or any of our articles. Here are some}
\href{https://help.nytimes3xbfgragh.onion/hc/en-us/articles/115014925288-How-to-submit-a-letter-to-the-editor}{\emph{tips}}\emph{.
And here's our email:}
\href{mailto:letters@NYTimes.com}{\emph{letters@NYTimes.com}}\emph{.}

\emph{Follow The New York Times Opinion section on}
\href{https://www.facebookcorewwwi.onion/nytopinion}{\emph{Facebook}}\emph{,}
\href{http://twitter.com/NYTOpinion}{\emph{Twitter (@NYTopinion)}}
\emph{and}
\href{https://www.instagram.com/nytopinion/}{\emph{Instagram}}\emph{.}

Advertisement

\protect\hyperlink{after-bottom}{Continue reading the main story}

\hypertarget{site-index}{%
\subsection{Site Index}\label{site-index}}

\hypertarget{site-information-navigation}{%
\subsection{Site Information
Navigation}\label{site-information-navigation}}

\begin{itemize}
\tightlist
\item
  \href{https://help.nytimes3xbfgragh.onion/hc/en-us/articles/115014792127-Copyright-notice}{©~2020~The
  New York Times Company}
\end{itemize}

\begin{itemize}
\tightlist
\item
  \href{https://www.nytco.com/}{NYTCo}
\item
  \href{https://help.nytimes3xbfgragh.onion/hc/en-us/articles/115015385887-Contact-Us}{Contact
  Us}
\item
  \href{https://www.nytco.com/careers/}{Work with us}
\item
  \href{https://nytmediakit.com/}{Advertise}
\item
  \href{http://www.tbrandstudio.com/}{T Brand Studio}
\item
  \href{https://www.nytimes3xbfgragh.onion/privacy/cookie-policy\#how-do-i-manage-trackers}{Your
  Ad Choices}
\item
  \href{https://www.nytimes3xbfgragh.onion/privacy}{Privacy}
\item
  \href{https://help.nytimes3xbfgragh.onion/hc/en-us/articles/115014893428-Terms-of-service}{Terms
  of Service}
\item
  \href{https://help.nytimes3xbfgragh.onion/hc/en-us/articles/115014893968-Terms-of-sale}{Terms
  of Sale}
\item
  \href{https://spiderbites.nytimes3xbfgragh.onion}{Site Map}
\item
  \href{https://help.nytimes3xbfgragh.onion/hc/en-us}{Help}
\item
  \href{https://www.nytimes3xbfgragh.onion/subscription?campaignId=37WXW}{Subscriptions}
\end{itemize}
