Sections

SEARCH

\protect\hyperlink{site-content}{Skip to
content}\protect\hyperlink{site-index}{Skip to site index}

\href{https://www.nytimes3xbfgragh.onion/section/travel}{Travel}

\href{https://myaccount.nytimes3xbfgragh.onion/auth/login?response_type=cookie\&client_id=vi}{}

\href{https://www.nytimes3xbfgragh.onion/section/todayspaper}{Today's
Paper}

\href{/section/travel}{Travel}\textbar{}Venice Tourism May Never Be the
Same. It Could Be Better.

\url{https://nyti.ms/2VEYSW9}

\begin{itemize}
\item
\item
\item
\item
\item
\item
\end{itemize}

\hypertarget{the-coronavirus-outbreak}{%
\subsubsection{\texorpdfstring{\href{https://www.nytimes3xbfgragh.onion/news-event/coronavirus?name=styln-coronavirus-national\&region=TOP_BANNER\&variant=undefined\&block=storyline_menu_recirc\&action=click\&pgtype=Article\&impression_id=63cab970-e3b4-11ea-81b0-ed0ce085a510}{The
Coronavirus
Outbreak}}{The Coronavirus Outbreak}}\label{the-coronavirus-outbreak}}

\begin{itemize}
\tightlist
\item
  live\href{https://www.nytimes3xbfgragh.onion/2020/08/21/world/covid-19-coronavirus.html?name=styln-coronavirus-national\&region=TOP_BANNER\&variant=undefined\&block=storyline_menu_recirc\&action=click\&pgtype=Article\&impression_id=63cae080-e3b4-11ea-81b0-ed0ce085a510}{Latest
  Updates}
\item
  \href{https://www.nytimes3xbfgragh.onion/interactive/2020/us/coronavirus-us-cases.html?name=styln-coronavirus-national\&region=TOP_BANNER\&variant=undefined\&block=storyline_menu_recirc\&action=click\&pgtype=Article\&impression_id=63cae081-e3b4-11ea-81b0-ed0ce085a510}{Maps
  and Cases}
\item
  \href{https://www.nytimes3xbfgragh.onion/interactive/2020/science/coronavirus-vaccine-tracker.html?name=styln-coronavirus-national\&region=TOP_BANNER\&variant=undefined\&block=storyline_menu_recirc\&action=click\&pgtype=Article\&impression_id=63cae082-e3b4-11ea-81b0-ed0ce085a510}{Vaccine
  Tracker}
\item
  \href{https://www.nytimes3xbfgragh.onion/2020/08/19/us/colleges-closing-covid.html?name=styln-coronavirus-national\&region=TOP_BANNER\&variant=undefined\&block=storyline_menu_recirc\&action=click\&pgtype=Article\&impression_id=63cae083-e3b4-11ea-81b0-ed0ce085a510}{Colleges
  Closing}
\item
  \href{https://www.nytimes3xbfgragh.onion/live/2020/08/21/business/stock-market-today-coronavirus?name=styln-coronavirus-national\&region=TOP_BANNER\&variant=undefined\&block=storyline_menu_recirc\&action=click\&pgtype=Article\&impression_id=63cae084-e3b4-11ea-81b0-ed0ce085a510}{Economy}
\end{itemize}

Advertisement

\protect\hyperlink{after-top}{Continue reading the main story}

Supported by

\protect\hyperlink{after-sponsor}{Continue reading the main story}

\hypertarget{venice-tourism-may-never-be-the-same-it-could-be-better}{%
\section{Venice Tourism May Never Be the Same. It Could Be
Better.}\label{venice-tourism-may-never-be-the-same-it-could-be-better}}

The pandemic crushed the tourism industry in Venice and other
overtouristed cities. But many see this as an opportunity to rethink a
``tourism monoculture.''

\includegraphics{https://static01.graylady3jvrrxbe.onion/images/2020/07/02/travel/02travel-venice3/merlin_170044923_b665dd9c-cc2a-4327-992a-a7a8010dd41d-articleLarge.jpg?quality=75\&auto=webp\&disable=upscale}

By Anna Momigliano

\begin{itemize}
\item
  July 2, 2020
\item
  \begin{itemize}
  \item
  \item
  \item
  \item
  \item
  \item
  \end{itemize}
\end{itemize}

Long before Venice became the destination of choice for millions of
international holidaymakers, locals had a tradition of flânerie, an
aimless stroll through the city's ** calli, or walkways. They would bump
into acquaintances for a chat and the occasional drink, an ombra de vin,
a ``shadow of wine,'' as it's called in the lagoon.

That tradition has been picked up again. The pandemic crushed the
tourism industry, curtailing the hordes of annual visitors that made
flânerie a near impossibility, and now many residents --- particularly
those furloughed or laid off --- have more time and space to enjoy the
city's slow pace and faded beauty. But money is tight, for that sip of
wine and everything else. Local taverns have begun accepting promises of
future payments from regulars.

``People are like, I'll pay you in September, when hopefully tourists
will be back,'' said Matteo Secchi, an unemployed hotel concierge. ``If
we don't help each other, who will?''

Mr. Secchi, a native Venetian, started working in tourism when he was
still in high school, 30 years ago. ``My first job was to escort
tourists from hotels to Murano's glass shops,'' he said. ``Since I can
remember, tourism has been our only economy, we thought it was a
bottomless well, like oil for the Saudis.''

Venice certainly wasn't alone. The economies of other European cities
--- Barcelona, Prague and others --- grew to rely heavily on tourism,
leaving them now particularly exposed to the side effects of the
Covid-19 pandemic.

But there's a new feeling many residents and local travel operators
share: The crisis creates an opportunity to make future travel to and in
their cities and regions more sustainable. This crossroads is sparking
conversations on how to make tourism less taxing and more beneficial on
urban infrastructure and for its local inhabitants.

In Venice, residents and local leaders hope their city can develop an
economy that doesn't revolve entirely around tourism, one that would
draw international investors, expand the footprint of the city's two
universities and turn its empty buildings into environmental research
facilities.

Yes, the pandemic has shuttered Venice's lodging industry, said Claudio
Scarpa, the president of Associazione Veneziana Albergatori, a body
representing 430 hotels in Venice, but ``it is also a precious occasion
to rethink tourism.''

``This is the time to reclaim this city,'' he said, ``Or in a couple of
years we'll get back to complaining about overtourism.''

Other Venetians echoed that sentiment.

``We have to act now, before mass tourism will be back at full capacity,
because we won't get a second chance,'' said Paolo Costa, a former mayor
of Venice and an economics professor who also served as the dean of Ca'
Foscari University of Venice.

\includegraphics{https://static01.graylady3jvrrxbe.onion/images/2020/07/02/travel/02travel-venice4/02travel-venice4-articleLarge.jpg?quality=75\&auto=webp\&disable=upscale}

\hypertarget{an-attraction-for-centuries}{%
\subsection{An attraction for
centuries}\label{an-attraction-for-centuries}}

The uniqueness of this Italian city has made it a worldwide attraction
for centuries. And, tellingly, Venice's rise as a travel destination
coincided with its decline as an economic powerhouse, said Ezio Micelli,
an expert of urban transformation at Iuav University of Venice.

As a city-state, Venice thrived as a commercial and financial hub for
much of the Middle Ages. Its location midway between Constantinople and
Western Europe made it an ideal junction for the trade of spices, silk
and salt. ``It was the capital of capitalism,'' Mr. Micelli said.

But as the center of trade moved from the Mediterranean to the Atlantic,
Venice lost centrality and by the end of the 18th century, when it fell
under foreign rule, its decline was unstoppable. It was then that
wealthy Europeans started visiting Italy's art-rich cities, including
Venice, in a tradition known as ``the Grand Tour.'' Lord Byron and
Stendhal were among the city's earliest holidaymakers. By the 19th
century, Venice's Lido became the place of pilgrimage for Europe's
well-off bourgeoise (think of Thomas Mann's ``Death in Venice'').

But by the late 20th century, Venice became what economists describe as
a ``tourism monoculture,'' borrowing the term from the risky
agricultural practice of growing a single crop.

Image

Crowds of tourists swarm Riva degli Schiavoni, by St. Mark's
Square.Credit...Susan Wright for The New York Times

\hypertarget{too-many-of-them}{%
\subsection{`Too many of them'}\label{too-many-of-them}}

Before Covid-19, hotels in and around Venice annually hosted 10.2
million mostly international guests, according
to\href{https://www.istat.it/it/files/2019/11/Movimento-turistico-in-Italia-2018.pdf}{Italy's
bureau of statistics}. But this figure --- an estimate at best --- does
not account for day-trippers, who disembark from cruise ships, the train
station and bus tours. One estimate puts the actual number of tourists
around 20 million annually --- largely concentrated in an area of two
square miles and 50,000 residents. They contributed 3 billion euros, or
about \$3.3 billion, a year.

``Tourists grew gradually, year by year, and before we realized it,
there were too many of them, just like a boiling frog,'' Mr. Micelli
said.

\hypertarget{latest-updates-the-coronavirus-outbreak}{%
\section{\texorpdfstring{\href{https://www.nytimes3xbfgragh.onion/2020/08/21/world/covid-19-coronavirus.html?action=click\&pgtype=Article\&state=default\&region=MAIN_CONTENT_1\&context=storylines_live_updates}{Latest
Updates: The Coronavirus
Outbreak}}{Latest Updates: The Coronavirus Outbreak}}\label{latest-updates-the-coronavirus-outbreak}}

Updated 2020-08-21T13:40:30.090Z

\begin{itemize}
\tightlist
\item
  \href{https://www.nytimes3xbfgragh.onion/2020/08/21/world/covid-19-coronavirus.html?action=click\&pgtype=Article\&state=default\&region=MAIN_CONTENT_1\&context=storylines_live_updates\#link-6a60a19d}{`Be
  adults': Universities in the U.S. are warning students about
  gatherings as they return to campus.}
\item
  \href{https://www.nytimes3xbfgragh.onion/2020/08/21/world/covid-19-coronavirus.html?action=click\&pgtype=Article\&state=default\&region=MAIN_CONTENT_1\&context=storylines_live_updates\#link-324af071}{As
  he accepts the Democratic nomination, Biden knocks Trump's pandemic
  response.}
\item
  \href{https://www.nytimes3xbfgragh.onion/2020/08/21/world/covid-19-coronavirus.html?action=click\&pgtype=Article\&state=default\&region=MAIN_CONTENT_1\&context=storylines_live_updates\#link-191d44be}{South
  Korea threatens to detain people who obstruct virus-control efforts.}
\end{itemize}

\href{https://www.nytimes3xbfgragh.onion/2020/08/21/world/covid-19-coronavirus.html?action=click\&pgtype=Article\&state=default\&region=MAIN_CONTENT_1\&context=storylines_live_updates}{See
more updates}

More live coverage:
\href{https://www.nytimes3xbfgragh.onion/live/2020/08/21/business/stock-market-today-coronavirus?action=click\&pgtype=Article\&state=default\&region=MAIN_CONTENT_1\&context=storylines_live_updates}{Markets}

The mass tourism of recent decades was a result of globalization,
home-sharing platforms, cheap airfares and emerging economies. Ryanair,
easyJet and other low-cost carriers began flying into the Marco Polo
airport,
\href{https://www.port.venice.it/files/page/portofvenice4-2019_0.pdf}{cruise
ships alone} brought in 1.6 million visitors each year, and the growing
strength of the Chinese and other Asian economies allowed new tourists
to join the crowds of Europeans and North Americans.

Especially in the high season between May and October, and during
Carnival in February, Venice was impossibly crowded --- particularly in
its narrow calli, some just two meters, or six-and-a-half feet, wide.

When Dr. Micelli, the urban studies professor, would visit a brother who
lives on one of the city's most touristy streets, he sometimes could not
get out of the door.

``It's like a flood, literally. So I just have to wait,'' Mr. Micelli
said. Occasionally the local police would declare some calli one-way.
``I guess Venice is the only place in the world where you need one-way
pedestrian streets.''

Cristina Giussani, a bookshop owner, often walked home with heavy
groceries because the vaporetto, the water buses that serve as public
transportation, would be swarmed with hordes of tourists. She considers
the famous Rialto Bridge off-limits between 10 a.m. and 5 p.m.,
``because it's impossible to cross it without throwing elbows.''

Tourism changed the soul of the lagoon. Grocery stores turned into
souvenir shops, and rising housing costs and an increasing lack of
services pushed residents out. With more than 8,000 apartments listed on
Airbnb, Venice has Italy's highest Airbnb-to-population ratio.

The city's historical center, consisting of two islands, had at its peak
in the 1950s,
\href{https://nuovavenezia.gelocal.it/venezia/cronaca/2015/02/23/news/da-175-000-a-56-000-abitanti-cosi-si-svuota-venezia-1.10921951}{175,000
residents}.

In 2009, the population fell below 60,000, the conventional threshold to
be considered a city in Italy. A
\href{https://www.nytimes3xbfgragh.onion/2009/11/15/world/europe/15venice.html}{mock
funeral} was organized, with a coffin wrapped in the city's
1,500-year-old flag.

Today, the center of the city has some 50,000 residents.

``Being a resident in Venice feels like being part of the resistance,''
Ms. Giussani said.

\hypertarget{i-knew-it-had-to-be-tourism}{%
\subsection{`I knew it had to be
tourism'}\label{i-knew-it-had-to-be-tourism}}

Approximately 25,000 Venetians are now directly employed in tourism. And
even if the figure includes those who commute into the historical center
from the city's other areas, many other Venetians in the city center
rely on the industry indirectly.

``If you sell groceries, if you are a lawyer or an accountant, your main
clients are fellow Venetians who make money either directly from tourism
or from other Venetians who make money from tourism,'' said Stefano
Croce, who heads the local association of tour guides.

It wasn't a planned choice, as much as the result of a vicious cycle.
The more touristy Venice became, the more residents were pushed out; the
fewer the residents, the more those who remained struggled to find
employment outside of tourism, thus reinforcing the pattern.

Before he became a guide five years ago, Mr. Croce commuted to Padua and
worked in architecture. ``When I decided I wanted to work in my own
city, I knew it had to be tourism,'' he said. His son, a neuroscientist,
moved to Scotland.

Many Venetians found the situation unsustainable, but, until recently,
few did anything to change it. ``As long as mass tourism was there,
there were ideas, but they never gained traction because the status quo
was so convenient,'' said Mr. Costa, the former mayor.

``The same people who complain that overtourism is making their lives
impossible are renting their apartments to tourists on Airbnb,'' said
Guido Moltedo, editor of the Venice-based magazine
\href{https://ytali.com/}{Ytali.}

``It's a complicated place.''

Mr. Secchi, the hotel concierge, is also an activist fighting for the
lagoon's residents. Fifteen years ago, he founded the grassroots
organization ``\href{http://www.venessia.com/}{Venessia}'' (Venice, in
Venetian dialect), which keeps track of the declining local population.

But while his organization lobbies officials to create subsidized
housing to locals, and ``put some limits to the renting of the
apartments to tourists,'' Mr. Secchi also lists three rooms of his
apartment on Airbnb. ``I have to, if I want to pay my own rent.''

Mr. Secchi sees no contradiction in his livelihood and his passion for
Venice.

``Tourism is a great resource, but residents shouldn't be treated as
second class,'' he said.

Image

Before the pandemic shut down international travel, large cruise ships
were a daily sight from many calli of Venice.Credit...Miguel
Medina/Agence France-Presse --- Getty Images

\hypertarget{day-trippers-and-cruise-ships}{%
\subsection{Day trippers and cruise
ships}\label{day-trippers-and-cruise-ships}}

``The longer a visitor stays, the smaller his impact on the territory,''
said Magda Antonioli Corigliano, a tourism industry scholar at Milan's
Bocconi University. Day trippers tend to have a particularly harmful
impact, she argues, because they are on a constant move, and always
crowding the same spots around St. Mark's and the Rialto.

``If you have only one day, you want to see as much as you can, so you
run here and there, take a lot of vaporettos*,*'' Ms. Antonioli
Corigliano said. Overnight visitors can enjoy the lagoon at a slower
pace, and venture beyond its most obvious spots, contributing less to
pedestrian traffic jams.

Then there are the cruise ships, docking at the Marittima port and
navigating through the Giudecca Canal and St. Mark's basin. Though
responsible for a fraction of day trippers, they unload a significant
amount at a time, as well as causing a significant impact on the city's
environment because of the amount of fuel used.

``A cruise is a very energy-intensive way you can take a holiday,'' said
Jane Da Mosto, a scientist who heads the environmentalist group
\href{https://weareherevenice.org/}{We Are Here Venice}, which opposes
the presence of cruise ships.

Cruise ships bring money, but not all goes to Venice's historical
center.

A 2013 study by Ca' Foscari University estimated the overall business
brought to the city from the cruise industry to be around 290 million
euros annually. The study considered direct and indirect business with
the government as well as privately owned companies, and included fuel,
food supplies, laundry services and money spent by cruise day-trippers
in the city (as little as 19 euros or around \$21 per capita, if they
didn't spend the night).

The small amount of taxes paid to Venice's Port Authority were included
in that figure: Last year the authority, run by the central government's
transportation ministry, received 5.6 million euros from cruise vessels,
a spokesman said. This money goes to the running the authority itself,
and includes maintenance of the city's canals.

In 2012, the central government approved a law banning cruise ships from
the St. Mark's basin and the Giudecca Canal, to lessen overcrowding in
those areas, but it has yet to be enforced. And even if it were, Ms. Da
Mosto said, it will do little to contain the damage.

Even if cruise ships were to dock in Marghera, the nearby port on the
mainland, Ms. Da Mosto said that the vessels would cause the same
environmental impact. The only difference is that they would do it a few
miles away.

\hypertarget{the-covid-crisis}{%
\subsection{The Covid crisis}\label{the-covid-crisis}}

Six months ago, Venice's overtourism came to a sudden halt.

The number of tourists in the city plummeted first in November, when a
series of unusually high tides spurred cancellations. Tourism almost
disappeared beginning in late February, when the Covid-19 pandemic
prompted authorities to cancel the Carnival and, soon after, declare a
nationwide lockdown.

\href{https://www.nytimes3xbfgragh.onion/news-event/coronavirus?action=click\&pgtype=Article\&state=default\&region=MAIN_CONTENT_3\&context=storylines_faq}{}

\hypertarget{the-coronavirus-outbreak-}{%
\subsubsection{The Coronavirus Outbreak
›}\label{the-coronavirus-outbreak-}}

\hypertarget{frequently-asked-questions}{%
\paragraph{Frequently Asked
Questions}\label{frequently-asked-questions}}

Updated August 17, 2020

\begin{itemize}
\item ~
  \hypertarget{why-does-standing-six-feet-away-from-others-help}{%
  \paragraph{Why does standing six feet away from others
  help?}\label{why-does-standing-six-feet-away-from-others-help}}

  \begin{itemize}
  \tightlist
  \item
    The coronavirus spreads primarily through droplets from your mouth
    and nose, especially when you cough or sneeze. The C.D.C., one of
    the organizations using that measure,
    \href{https://www.nytimes3xbfgragh.onion/2020/04/14/health/coronavirus-six-feet.html?action=click\&pgtype=Article\&state=default\&region=MAIN_CONTENT_3\&context=storylines_faq}{bases
    its recommendation of six feet} on the idea that most large droplets
    that people expel when they cough or sneeze will fall to the ground
    within six feet. But six feet has never been a magic number that
    guarantees complete protection. Sneezes, for instance, can launch
    droplets a lot farther than six feet,
    \href{https://jamanetwork.com/journals/jama/fullarticle/2763852}{according
    to a recent study}. It's a rule of thumb: You should be safest
    standing six feet apart outside, especially when it's windy. But
    keep a mask on at all times, even when you think you're far enough
    apart.
  \end{itemize}
\item ~
  \hypertarget{i-have-antibodies-am-i-now-immune}{%
  \paragraph{I have antibodies. Am I now
  immune?}\label{i-have-antibodies-am-i-now-immune}}

  \begin{itemize}
  \tightlist
  \item
    As of right
    now,\href{https://www.nytimes3xbfgragh.onion/2020/07/22/health/covid-antibodies-herd-immunity.html?action=click\&pgtype=Article\&state=default\&region=MAIN_CONTENT_3\&context=storylines_faq}{that
    seems likely, for at least several months.} There have been
    frightening accounts of people suffering what seems to be a second
    bout of Covid-19. But experts say these patients may have a
    drawn-out course of infection, with the virus taking a slow toll
    weeks to months after initial exposure. People infected with the
    coronavirus typically
    \href{https://www.nature.com/articles/s41586-020-2456-9}{produce}
    immune molecules called antibodies, which are
    \href{https://www.nytimes3xbfgragh.onion/2020/05/07/health/coronavirus-antibody-prevalence.html?action=click\&pgtype=Article\&state=default\&region=MAIN_CONTENT_3\&context=storylines_faq}{protective
    proteins made in response to an
    infection}\href{https://www.nytimes3xbfgragh.onion/2020/05/07/health/coronavirus-antibody-prevalence.html?action=click\&pgtype=Article\&state=default\&region=MAIN_CONTENT_3\&context=storylines_faq}{.
    These antibodies may} last in the body
    \href{https://www.nature.com/articles/s41591-020-0965-6}{only two to
    three months}, which may seem worrisome, but that's perfectly normal
    after an acute infection subsides, said Dr. Michael Mina, an
    immunologist at Harvard University. It may be possible to get the
    coronavirus again, but it's highly unlikely that it would be
    possible in a short window of time from initial infection or make
    people sicker the second time.
  \end{itemize}
\item ~
  \hypertarget{im-a-small-business-owner-can-i-get-relief}{%
  \paragraph{I'm a small-business owner. Can I get
  relief?}\label{im-a-small-business-owner-can-i-get-relief}}

  \begin{itemize}
  \tightlist
  \item
    The
    \href{https://www.nytimes3xbfgragh.onion/article/small-business-loans-stimulus-grants-freelancers-coronavirus.html?action=click\&pgtype=Article\&state=default\&region=MAIN_CONTENT_3\&context=storylines_faq}{stimulus
    bills enacted in March} offer help for the millions of American
    small businesses. Those eligible for aid are businesses and
    nonprofit organizations with fewer than 500 workers, including sole
    proprietorships, independent contractors and freelancers. Some
    larger companies in some industries are also eligible. The help
    being offered, which is being managed by the Small Business
    Administration, includes the Paycheck Protection Program and the
    Economic Injury Disaster Loan program. But lots of folks have
    \href{https://www.nytimes3xbfgragh.onion/interactive/2020/05/07/business/small-business-loans-coronavirus.html?action=click\&pgtype=Article\&state=default\&region=MAIN_CONTENT_3\&context=storylines_faq}{not
    yet seen payouts.} Even those who have received help are confused:
    The rules are draconian, and some are stuck sitting on
    \href{https://www.nytimes3xbfgragh.onion/2020/05/02/business/economy/loans-coronavirus-small-business.html?action=click\&pgtype=Article\&state=default\&region=MAIN_CONTENT_3\&context=storylines_faq}{money
    they don't know how to use.} Many small-business owners are getting
    less than they expected or
    \href{https://www.nytimes3xbfgragh.onion/2020/06/10/business/Small-business-loans-ppp.html?action=click\&pgtype=Article\&state=default\&region=MAIN_CONTENT_3\&context=storylines_faq}{not
    hearing anything at all.}
  \end{itemize}
\item ~
  \hypertarget{what-are-my-rights-if-i-am-worried-about-going-back-to-work}{%
  \paragraph{What are my rights if I am worried about going back to
  work?}\label{what-are-my-rights-if-i-am-worried-about-going-back-to-work}}

  \begin{itemize}
  \tightlist
  \item
    Employers have to provide
    \href{https://www.osha.gov/SLTC/covid-19/standards.html}{a safe
    workplace} with policies that protect everyone equally.
    \href{https://www.nytimes3xbfgragh.onion/article/coronavirus-money-unemployment.html?action=click\&pgtype=Article\&state=default\&region=MAIN_CONTENT_3\&context=storylines_faq}{And
    if one of your co-workers tests positive for the coronavirus, the
    C.D.C.} has said that
    \href{https://www.cdc.gov/coronavirus/2019-ncov/community/guidance-business-response.html}{employers
    should tell their employees} -\/- without giving you the sick
    employee's name -\/- that they may have been exposed to the virus.
  \end{itemize}
\item ~
  \hypertarget{what-is-school-going-to-look-like-in-september}{%
  \paragraph{What is school going to look like in
  September?}\label{what-is-school-going-to-look-like-in-september}}

  \begin{itemize}
  \tightlist
  \item
    It is unlikely that many schools will return to a normal schedule
    this fall, requiring the grind of
    \href{https://www.nytimes3xbfgragh.onion/2020/06/05/us/coronavirus-education-lost-learning.html?action=click\&pgtype=Article\&state=default\&region=MAIN_CONTENT_3\&context=storylines_faq}{online
    learning},
    \href{https://www.nytimes3xbfgragh.onion/2020/05/29/us/coronavirus-child-care-centers.html?action=click\&pgtype=Article\&state=default\&region=MAIN_CONTENT_3\&context=storylines_faq}{makeshift
    child care} and
    \href{https://www.nytimes3xbfgragh.onion/2020/06/03/business/economy/coronavirus-working-women.html?action=click\&pgtype=Article\&state=default\&region=MAIN_CONTENT_3\&context=storylines_faq}{stunted
    workdays} to continue. California's two largest public school
    districts --- Los Angeles and San Diego --- said on July 13, that
    \href{https://www.nytimes3xbfgragh.onion/2020/07/13/us/lausd-san-diego-school-reopening.html?action=click\&pgtype=Article\&state=default\&region=MAIN_CONTENT_3\&context=storylines_faq}{instruction
    will be remote-only in the fall}, citing concerns that surging
    coronavirus infections in their areas pose too dire a risk for
    students and teachers. Together, the two districts enroll some
    825,000 students. They are the largest in the country so far to
    abandon plans for even a partial physical return to classrooms when
    they reopen in August. For other districts, the solution won't be an
    all-or-nothing approach.
    \href{https://bioethics.jhu.edu/research-and-outreach/projects/eschool-initiative/school-policy-tracker/}{Many
    systems}, including the nation's largest, New York City, are
    devising
    \href{https://www.nytimes3xbfgragh.onion/2020/06/26/us/coronavirus-schools-reopen-fall.html?action=click\&pgtype=Article\&state=default\&region=MAIN_CONTENT_3\&context=storylines_faq}{hybrid
    plans} that involve spending some days in classrooms and other days
    online. There's no national policy on this yet, so check with your
    municipal school system regularly to see what is happening in your
    community.
  \end{itemize}
\end{itemize}

Mr. Scarpa, the president of the hotel body, said that the sudden drop
in tourism could cost the city more than 1 billion euros in lost
revenues. About 10,000 Venetians have been furloughed in the hotel
industry alone, Mr. Scarpa said. The recovery, he added, will to be
slow, as hotels expect only one-third of the usual number of visitors
for the high season this year.

Italy's central government has vowed to help the tourism industry by
providing aid packages and tax breaks for struggling hotels and
restaurants, but other sectors have also been hit hard.

Tour guides are one such group. They are often self-employed and thus
not eligible for long-term unemployment benefits; short-term subsidies
for the self-employed, issued by the central government, ended when the
lockdown was lifted but before international travelers were allowed
back. In June, tour guides held protests in several Italian cities,
including Venice.

``There are a lot of grievances in the profession,'' said Mr. Croce, the
tour guide. He pointed out that most guides work with international
tourists. ``When the lockdown was lifted, restaurants and cafes could go
back to business, but we couldn't. It's not fair that we are getting the
same treatment.''

Since Italy lifted its restriction on movement in early June, the lagoon
has seen few visitors, the vast majority of them day-trippers from the
surrounding Veneto region.

Image

Venice in early June, just before the restrictive measures halting
travel were lifted.~Credit...Alessandro Grassani for The New York Times

\hypertarget{geography-and-oversight}{%
\subsection{Geography and oversight}\label{geography-and-oversight}}

Venetians have long grappled with the same questions: How can we make
tourism more sustainable? And how can we stop relying solely on it?

Why little to no significant change has been implemented so far begins
with geography and government oversight.

Today's Venice is more than its medieval origins in the lagoon. From an
administrative point of view, it is a large city of more 250,000
inhabitants, consisting of neighborhoods on the mainland as well as
several islands in the lagoon.

But historical Venice, which is what people mean when they use the word
colloquially, is two islands. One large, fish-shaped island cut in half
by the Grand Canal --- technically, the ``island of Venice,'' but often
just called ``the fish''--- and a smaller island, the Giudecca.
Overtourism is largely concentrated within two of the larger island's
six neighborhoods.

``Venice is two cities, there's the land, with their problems, and
there's the lagoon, with our problems,'' said Mr. Moltedo, the editor.
He noted that Venice's past and present administrations are a reflection
of the mainland population, which is larger and not as affected by
overtourism.

Ms. Giussani, the bookstore owner, also noted that groups that have long
opposed overtourism were disorganized, and rarely coordinated their
approach. But she argued that now people seemed more open to ``create a
network.''

These groups are currently pressuring the City Council, which governs
tourism decisions, together with the regional government, to limit
access to the historical center with a system of quotas and bookings
(residents and visitors with hotel reservations would be excluded).
Mayor Luigi Brugnaro wrote in an email that his administration is
working on the booking system ``as a short-term goal.''

The government, he added, hopes ``to regulate the tourists flows so that
they can be compatible with the daily lives of the residents.''

In the meantime, the hotel industry plans to promote Venice as a
Christmas destination for wealthy international holidaymakers, creating
special cultural packages in partnership with museums, said Mr. Scarpa,
the official at the local hotel group.

\hypertarget{the-role-of-universities}{%
\subsection{The role of universities}\label{the-role-of-universities}}

But most of all, Venice's two universities are actively working on
revitalizing the city's population.

``People tend to think that everyone in Venice is either a tourist or a
resident, but in the middle there's another group, temporary residents,
who are part of the social fabric and breathe new life into it,'' said
Michele Bugliesi, the dean of Ca' Foscari, Venice's largest university.

The school, he said, is already a pull factor for temporary residents
--- ``It's remarkable how easy we get visiting professors,'' said Mr.
Bugliesi --- but later this year it plans to open a business incubator,
with the goal of attracting forward-thinking entrepreneurs.

In late 2018, partnering with the \href{https://www.iit.it/}{Italian
Institute of Technology}, Ca' Foscari launched a center for the
application of \href{https://ccht.iit.it/}{technologies to the
preservation of cultural heritage}, which is now expanding. In 2018, the
university also founded, in partnership with Italy's National Research
Council, a \href{https://www.cmcc.it/it/cmcccafoscari}{program on
climate change}. It is expected to expand; beginning next semester, it
will offer a new English-language degree in environmental humanities,
one that is targeted to international students.

Iuav, a small public-arts college, is converting empty
bed-and-breakfasts into dorms for its 4,000 students, most of whom were
commuters. Mr. Brugnaro, the current mayor, wrote that he is also
planning some incentives to bring in new residents.

Taken alone, these three projects aren't enough to repopulate Venice.
But Mr. Bugliesi thinks they have the potential to create ``a critical
mass that would set off a chain reaction.''

Dreams of attracting multinational corporations, prestigious
institutions and digital nomads, transforming Venice into something of a
blend of Brussels and Berlin, have been discussed for years, and are a
recurrent theme when one discusses the future of the city with educated
Venetians.

``Arts foundations and research institutes from all over the world
should have an interest to open a chapter here, but we have to offer
them incentives,'' said Camilla Seibezzi, an art curator.

Also frequently mentioned is that the city's symbiotic relationship with
the sea makes the place ideal for any private or public institution
interested in climate change.

And locals love to argue that the city's stunning beauty and its unique
car-free lifestyle makes Venice an ideal place of residence for creative
people and digital nomads.

``I really don't understand why more people don't move here, when one
can simply work from remote and enjoy all this beauty and silence,''
said Mr. Moltedo, the editor, who moved from Rome seven years ago.

And, for the first time, Venice may have the space to dedicate to new
projects.

``Very soon, Venice will end up with lots of empty buildings, because
some hotels will have to close. Now it's the time to think about what to
do with them,'' Mr. Costa said.

``Before the pandemic, every project, every idea had to carve out space
from overtourism. But now, there's a whole world out there.''

\emph{\textbf{Follow New York Times Travel}}
\emph{on}\href{https://www.instagram.com/nytimestravel/}{\emph{Instagram}}\emph{,}\href{https://twitter.com/nytimestravel}{\emph{Twitter}}
\emph{and}\href{https://www.facebookcorewwwi.onion/nytimestravel/}{\emph{Facebook}}\emph{.
And}\href{https://www.nytimes3xbfgragh.onion/newsletters/traveldispatch}{\emph{sign
up for our weekly Travel Dispatch newsletter}} \emph{to receive expert
tips on traveling smarter and inspiration for your next vacation.}

Advertisement

\protect\hyperlink{after-bottom}{Continue reading the main story}

\hypertarget{site-index}{%
\subsection{Site Index}\label{site-index}}

\hypertarget{site-information-navigation}{%
\subsection{Site Information
Navigation}\label{site-information-navigation}}

\begin{itemize}
\tightlist
\item
  \href{https://help.nytimes3xbfgragh.onion/hc/en-us/articles/115014792127-Copyright-notice}{©~2020~The
  New York Times Company}
\end{itemize}

\begin{itemize}
\tightlist
\item
  \href{https://www.nytco.com/}{NYTCo}
\item
  \href{https://help.nytimes3xbfgragh.onion/hc/en-us/articles/115015385887-Contact-Us}{Contact
  Us}
\item
  \href{https://www.nytco.com/careers/}{Work with us}
\item
  \href{https://nytmediakit.com/}{Advertise}
\item
  \href{http://www.tbrandstudio.com/}{T Brand Studio}
\item
  \href{https://www.nytimes3xbfgragh.onion/privacy/cookie-policy\#how-do-i-manage-trackers}{Your
  Ad Choices}
\item
  \href{https://www.nytimes3xbfgragh.onion/privacy}{Privacy}
\item
  \href{https://help.nytimes3xbfgragh.onion/hc/en-us/articles/115014893428-Terms-of-service}{Terms
  of Service}
\item
  \href{https://help.nytimes3xbfgragh.onion/hc/en-us/articles/115014893968-Terms-of-sale}{Terms
  of Sale}
\item
  \href{https://spiderbites.nytimes3xbfgragh.onion}{Site Map}
\item
  \href{https://help.nytimes3xbfgragh.onion/hc/en-us}{Help}
\item
  \href{https://www.nytimes3xbfgragh.onion/subscription?campaignId=37WXW}{Subscriptions}
\end{itemize}
