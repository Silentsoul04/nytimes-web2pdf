\href{/section/opinion}{Opinion}\textbar{}Reading Orwell for the Fourth
of July

\url{https://nyti.ms/38oWb0d}

\begin{itemize}
\item
\item
\item
\item
\item
\item
\end{itemize}

\includegraphics{https://static01.graylady3jvrrxbe.onion/images/2020/07/06/opinion/03stephensNewe/03stephensNewe-articleLarge.jpg?quality=75\&auto=webp\&disable=upscale}

Sections

\protect\hyperlink{site-content}{Skip to
content}\protect\hyperlink{site-index}{Skip to site index}

\href{/section/opinion}{Opinion}

\hypertarget{reading-orwell-for-the-fourth-of-july}{%
\section{Reading Orwell for the Fourth of
July}\label{reading-orwell-for-the-fourth-of-july}}

As we celebrate freedom, speaking freely is in danger.

George Orwell, date unknown.Credit...Associated Press

Supported by

\protect\hyperlink{after-sponsor}{Continue reading the main story}

\href{https://www.nytimes3xbfgragh.onion/by/bret-stephens}{\includegraphics{https://static01.graylady3jvrrxbe.onion/images/2017/08/27/insider/bretstephens/bretstephens-thumbLarge-v6.png}}

By \href{https://www.nytimes3xbfgragh.onion/by/bret-stephens}{Bret
Stephens}

Opinion Columnist

\begin{itemize}
\item
  July 3, 2020
\item
  \begin{itemize}
  \item
  \item
  \item
  \item
  \item
  \item
  \end{itemize}
\end{itemize}

\href{https://cn.nytimes3xbfgragh.onion/opinion/20200706/orwell-fourth-of-july/}{阅读简体中文版}\href{https://cn.nytimes3xbfgragh.onion/opinion/20200706/orwell-fourth-of-july/zh-hant/}{閱讀繁體中文版}

This Fourth of July, it's worth taking stock of the state of freedom ---
and of our attitudes toward it --- at home and around the world.

In Russia, Vladimir Putin
\href{https://www.nytimes3xbfgragh.onion/2020/07/01/world/europe/putin-referendum-vote-russia.html?action=click\&module=Top\%20Stories\&pgtype=Homepage}{just
won a ``plebiscite''} ratifying his right to stay in power until the
year 2036. In Hong Kong,
\href{https://www.nytimes3xbfgragh.onion/2020/07/01/world/asia/hong-kong-security-law-china.html}{a
new security law came into effect} effectively putting an end to the
right of peaceful protest. In Poland, a
\href{https://foreignpolicy.com/2020/05/04/poland-is-showing-the-world-how-not-to-run-a-pandemic-election/}{runoff
election will decide} if the country continues its slide toward
illiberalism.

In the United States, these stories barely make a dent on public
consciousness. Conservatives and liberals alike have ceased to care very
much about the denial of freedom to others.

We also have our own problems with freedom.

For once, the main problem isn't Donald Trump. The president may be an
instinctual fascist, a wannabe autocrat. But, after nearly four years in
power, he's been unmasked as an incompetent one.

Trump may have privately praised Xi Jinping for building concentration
camps for Uighurs. Congress still passed legislation to impose sanctions
on China for them. He may want to bring Russia back to the G7. The other
six won't let him. He may have sought to abolish DACA for the Dreamers.
John Roberts decided otherwise. He may call the press an ``enemy of the
American people.'' That enemy still operates without restraint when it
comes to slamming him.

To adapt the Lloyd Bentsen line, Donald John Trump, you're no Recep
Tayyip Erdogan.

The more serious problem today comes from the left: from liberal elites
who, when tested, lack the courage of their liberal convictions; from
so-called progressives whose core convictions were never liberal to
begin with; from administrative types at nonprofits and corporations
who, with only vague convictions of their own, don't want to be on the
wrong side of a P.R. headache.

This has been the great cultural story of the last few years. It is
\href{https://www.nytimes3xbfgragh.onion/2018/09/04/opinion/bannon-new-yorker-festival-remnick.html}{typified
by incidents} such as The New Yorker's David Remnick thinking it would
be a good idea to interview Steve Bannon for the magazine's annual
festival --- until a Twitter mob and some members of his own staff
decided otherwise. Or by The Washington Post
\href{https://www.washingtonpost.com/local/social-issues/blackface-incident-at-washington-post-cartoonists-2018-halloween-party-resurfaces-amid-protests/2020/06/17/66f09bde-af2e-11ea-856d-5054296735e5_story.html}{devoting
3,000 words} to destroying the life of a private person of no particular
note because in 2018 she wore blackface, with ironic intent, at a
Halloween party. Or by big corporations
\href{https://www.nytimes3xbfgragh.onion/2020/06/26/business/media/Facebook-advertising-boycott.html}{pulling
ads from Facebook} while demanding the company do more to censor forms
of speech they deem impermissible.

These stories matter because an idea is at risk. That's the idea that
people who cannot speak freely will not be able to think clearly, and
that no society can long flourish when contrarians are treated as
heretics.

That idea, old as Socrates, formerly had powerful institutional
defenders, especially in the form of universities, news media, book
publishers, free-speech groups and major philanthropies.

But those defenders are, on account of one excuse or another,
capitulating to people who claim free speech for themselves (but not for
others), who believe all the old patriarchal hierarchies must go (so
that new ``intersectional'' hierarchies may arise), who are in a
perpetual fervor to rewrite the past (all the better to control the
future), and who demand cringing public apologies from those who have
sinned against an ever-more radical ideological standard (while those
apologies won't save them from being fired).

As in so much else, George Orwell was here before us. In connection to
the recent vandalism of monuments and destruction of statues, a line
from ``1984'' has been making the rounds --- ``every book has been
rewritten, every picture has been repainted, every statue and street and
building has been renamed, every date has been altered.'' But the Orwell
essay to which I keep returning is a little jewel from 1946,
``\href{https://www.orwellfoundation.com/the-orwell-foundation/orwell/essays-and-other-works/the-prevention-of-literature/}{The
Prevention of Literature}.''

Orwell's concern then was not just with Russian totalitarianism, but
with the arguments used by much of the Western intelligentsia to justify
repression.

``What is sinister,'' he wrote, ``is that the conscious enemies of
liberty are those to whom liberty ought to mean most.'' He was
particularly calling out Western scientists who admired the Soviet Union
for its technical prowess and were utterly indifferent to Stalin's
persecution of writers and artists. ``They do not see that any attack on
intellectual liberty, and on the concept of objective truth, threatens
in the long run every department of thought.''

\emph{Every department of thought.} Right now, all the Twitter furors,
the angry rows over publication decisions, the canceled speeches and
books, the semantic battles about which words take an uppercase and
which don't, may seem remote to those who care about more tangible
issues: depression, disease, police abuse, urban decline. Yet the issue
that counts the most is whether the institutions that are supposed to
champion liberal ideals will muster the moral confidence to survive. On
this July 4, it's very much in doubt.

\emph{The Times is committed to publishing}
\href{https://www.nytimes3xbfgragh.onion/2019/01/31/opinion/letters/letters-to-editor-new-york-times-women.html}{\emph{a
diversity of letters}} \emph{to the editor. We'd like to hear what you
think about this or any of our articles. Here are some}
\href{https://help.nytimes3xbfgragh.onion/hc/en-us/articles/115014925288-How-to-submit-a-letter-to-the-editor}{\emph{tips}}\emph{.
And here's our email:}
\href{mailto:letters@NYTimes.com}{\emph{letters@NYTimes.com}}\emph{.}

\emph{Follow The New York Times Opinion section on}
\href{https://www.facebookcorewwwi.onion/nytopinion}{\emph{Facebook}}\emph{,}
\href{http://twitter.com/NYTOpinion}{\emph{Twitter (@NYTopinion)}}
\emph{and}
\href{https://www.instagram.com/nytopinion/}{\emph{Instagram}}\emph{.}

Advertisement

\protect\hyperlink{after-bottom}{Continue reading the main story}

\hypertarget{site-index}{%
\subsection{Site Index}\label{site-index}}

\hypertarget{site-information-navigation}{%
\subsection{Site Information
Navigation}\label{site-information-navigation}}

\begin{itemize}
\tightlist
\item
  \href{https://help.nytimes3xbfgragh.onion/hc/en-us/articles/115014792127-Copyright-notice}{©~2020~The
  New York Times Company}
\end{itemize}

\begin{itemize}
\tightlist
\item
  \href{https://www.nytco.com/}{NYTCo}
\item
  \href{https://help.nytimes3xbfgragh.onion/hc/en-us/articles/115015385887-Contact-Us}{Contact
  Us}
\item
  \href{https://www.nytco.com/careers/}{Work with us}
\item
  \href{https://nytmediakit.com/}{Advertise}
\item
  \href{http://www.tbrandstudio.com/}{T Brand Studio}
\item
  \href{https://www.nytimes3xbfgragh.onion/privacy/cookie-policy\#how-do-i-manage-trackers}{Your
  Ad Choices}
\item
  \href{https://www.nytimes3xbfgragh.onion/privacy}{Privacy}
\item
  \href{https://help.nytimes3xbfgragh.onion/hc/en-us/articles/115014893428-Terms-of-service}{Terms
  of Service}
\item
  \href{https://help.nytimes3xbfgragh.onion/hc/en-us/articles/115014893968-Terms-of-sale}{Terms
  of Sale}
\item
  \href{https://spiderbites.nytimes3xbfgragh.onion}{Site Map}
\item
  \href{https://help.nytimes3xbfgragh.onion/hc/en-us}{Help}
\item
  \href{https://www.nytimes3xbfgragh.onion/subscription?campaignId=37WXW}{Subscriptions}
\end{itemize}
