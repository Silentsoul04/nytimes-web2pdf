Sections

SEARCH

\protect\hyperlink{site-content}{Skip to
content}\protect\hyperlink{site-index}{Skip to site index}

\href{https://www.nytimes3xbfgragh.onion/section/health}{Health}

\href{https://myaccount.nytimes3xbfgragh.onion/auth/login?response_type=cookie\&client_id=vi}{}

\href{https://www.nytimes3xbfgragh.onion/section/todayspaper}{Today's
Paper}

\href{/section/health}{Health}\textbar{}The Pandemic's Big Mystery: How
Deadly Is the Coronavirus?

\url{https://nyti.ms/2ZBdjfc}

\begin{itemize}
\item
\item
\item
\item
\item
\end{itemize}

\href{https://www.nytimes3xbfgragh.onion/news-event/coronavirus?action=click\&pgtype=Article\&state=default\&region=TOP_BANNER\&context=storylines_menu}{The
Coronavirus Outbreak}

\begin{itemize}
\tightlist
\item
  live\href{https://www.nytimes3xbfgragh.onion/2020/08/04/world/coronavirus-cases.html?action=click\&pgtype=Article\&state=default\&region=TOP_BANNER\&context=storylines_menu}{Latest
  Updates}
\item
  \href{https://www.nytimes3xbfgragh.onion/interactive/2020/us/coronavirus-us-cases.html?action=click\&pgtype=Article\&state=default\&region=TOP_BANNER\&context=storylines_menu}{Maps
  and Cases}
\item
  \href{https://www.nytimes3xbfgragh.onion/interactive/2020/science/coronavirus-vaccine-tracker.html?action=click\&pgtype=Article\&state=default\&region=TOP_BANNER\&context=storylines_menu}{Vaccine
  Tracker}
\item
  \href{https://www.nytimes3xbfgragh.onion/2020/08/02/us/covid-college-reopening.html?action=click\&pgtype=Article\&state=default\&region=TOP_BANNER\&context=storylines_menu}{College
  Reopening}
\item
  \href{https://www.nytimes3xbfgragh.onion/live/2020/08/04/business/stock-market-today-coronavirus?action=click\&pgtype=Article\&state=default\&region=TOP_BANNER\&context=storylines_menu}{Economy}
\end{itemize}

Advertisement

\protect\hyperlink{after-top}{Continue reading the main story}

Supported by

\protect\hyperlink{after-sponsor}{Continue reading the main story}

\hypertarget{the-pandemics-big-mystery-how-deadly-is-the-coronavirus}{%
\section{The Pandemic's Big Mystery: How Deadly Is the
Coronavirus?}\label{the-pandemics-big-mystery-how-deadly-is-the-coronavirus}}

Even with more than 500,000 dead worldwide, scientists are struggling to
learn how often the virus kills. Here's why.

\includegraphics{https://static01.graylady3jvrrxbe.onion/images/2020/07/04/science/04virus-fatality/merlin_173454330_b2784934-b893-42fa-9f57-c4b66f881947-articleLarge.jpg?quality=75\&auto=webp\&disable=upscale}

\href{https://www.nytimes3xbfgragh.onion/by/donald-g-mcneil-jr}{\includegraphics{https://static01.graylady3jvrrxbe.onion/images/2018/06/13/multimedia/author-donald-g-mcneil-jr/author-donald-g-mcneil-jr-thumbLarge-v4.png}}

By
\href{https://www.nytimes3xbfgragh.onion/by/donald-g-mcneil-jr}{Donald
G. McNeil Jr.}

\begin{itemize}
\item
  Published July 4, 2020Updated July 22, 2020
\item
  \begin{itemize}
  \item
  \item
  \item
  \item
  \item
  \end{itemize}
\end{itemize}

\href{https://www.nytimes3xbfgragh.onion/es/2020/07/06/espanol/tasa-mortalidad-fatalidad-coronavirus.html}{Leer
en español}

More than six months into the pandemic, the
\href{https://www.nytimes3xbfgragh.onion/2020/07/22/us/florida-mother-2-children-covid-19.html}{coronavirus}
has
\href{https://www.nytimes3xbfgragh.onion/2020/07/15/health/coronavirus-schools-reopening.html}{infected}
more than 11 million people worldwide,
\href{https://www.nytimes3xbfgragh.onion/2020/07/13/us/arizona-coronavirus-mark-anthony-urquiza.html}{killing}
more than 525,000. But despite the increasing toll, scientists still do
not have a definitive answer to one of the most fundamental questions
about the virus: How deadly is it?

A firm estimate could help governments predict how many
\href{https://www.nytimes3xbfgragh.onion/2020/07/22/us/florida-mother-2-children-covid-19.html}{deaths}
would ensue if the virus spread out of control. The figure, usually
called the infection fatality rate, could tell health officials what to
expect as the pandemic spreads to densely populated nations like Brazil,
Nigeria and India.

In even poorer countries, where lethal threats like measles and malaria
are constant and where hard budget choices are routine, the number could
help officials decide whether to spend more on oxygen concentrators or
ventilators, or on measles shots and mosquito nets.

The question became even more complex last month, when the Centers for
Disease Control and Prevention released data suggesting that for every
documented infection in the United States,
\href{https://www.cdc.gov/coronavirus/2019-ncov/cases-updates/commercial-lab-surveys.html}{there
were 10 other cases on average that had gone unrecorded}, probably
because they were very mild or asymptomatic.

If there are many more asymptomatic infections than once thought, then
the virus may be less deadly than it has appeared. But even that
calculation is a difficult one.

On Thursday, after the World Health Organization held a two-day online
meeting of 1,300 scientists from around the world, the agency's chief
scientist, Dr. Soumya Swaminathan, said the consensus for now was that
the I.F.R. is about 0.6 percent --- which means that the risk of death
is less than 1 percent.

Although she did not note this, 0.6 percent of the world's population is
47 million people, and 0.6 percent of the American population is 2
million people. The virus remains a major threat.

At present, countries have very different case fatality rates, or
C.F.R.'s, which measure deaths among patients known to have had
Covid-19. In most cases, that number is highest in countries that have
had the virus the longest.

According to
\href{https://www.nytimes3xbfgragh.onion/interactive/2020/world/coronavirus-maps.html}{data
gathered by The New York Times}, China had reported 90,294 cases as of
Friday and 4,634 deaths, which is a C.F.R. of 5 percent. The United
States was very close to that mark. It has had 2,811,447 cases and
129,403 deaths, about 4.6 percent.

Those percentages are far higher rates than the
\href{https://wwwnc.cdc.gov/eid/article/12/1/05-0979_article}{2.5
percent death rate often ascribed to the 1918 flu pandemic}. Still, it
is difficult to measure fatality rates during pandemics, especially at
the beginning.

\hypertarget{latest-updates-global-coronavirus-outbreak}{%
\section{\texorpdfstring{\href{https://www.nytimes3xbfgragh.onion/2020/08/04/world/coronavirus-cases.html?action=click\&pgtype=Article\&state=default\&region=MAIN_CONTENT_1\&context=storylines_live_updates}{Latest
Updates: Global Coronavirus
Outbreak}}{Latest Updates: Global Coronavirus Outbreak}}\label{latest-updates-global-coronavirus-outbreak}}

Updated 2020-08-04T21:57:55.984Z

\begin{itemize}
\tightlist
\item
  \href{https://www.nytimes3xbfgragh.onion/2020/08/04/world/coronavirus-cases.html?action=click\&pgtype=Article\&state=default\&region=MAIN_CONTENT_1\&context=storylines_live_updates\#link-2daa96b5}{As
  talks drag on, McConnell signals openness to jobless aid extension
  that Republicans have opposed.}
\item
  \href{https://www.nytimes3xbfgragh.onion/2020/08/04/world/coronavirus-cases.html?action=click\&pgtype=Article\&state=default\&region=MAIN_CONTENT_1\&context=storylines_live_updates\#link-1228a480}{Novavax
  sees encouraging results from two studies of its experimental
  vaccine.}
\item
  \href{https://www.nytimes3xbfgragh.onion/2020/08/04/world/coronavirus-cases.html?action=click\&pgtype=Article\&state=default\&region=MAIN_CONTENT_1\&context=storylines_live_updates\#link-4825b93}{Public
  and private schools in Maryland and elsewhere are divided over
  in-person instruction.}
\end{itemize}

\href{https://www.nytimes3xbfgragh.onion/2020/08/04/world/coronavirus-cases.html?action=click\&pgtype=Article\&state=default\&region=MAIN_CONTENT_1\&context=storylines_live_updates}{See
more updates}

More live coverage:
\href{https://www.nytimes3xbfgragh.onion/live/2020/08/04/business/stock-market-today-coronavirus?action=click\&pgtype=Article\&state=default\&region=MAIN_CONTENT_1\&context=storylines_live_updates}{Markets}

In the chaos that ensues when a new virus hits a city hard, thousands of
people may die and be buried without ever being tested, and certainly
without them all being autopsied.

It is never entirely clear how many died of the virus and how many died
of heart attacks, strokes or other ills. That has happened in both New
York City and in Wuhan, China, where the outbreak began.

Normally, once the chaos has subsided, more testing is done and more
mild cases are found --- and because the denominator of the fraction
rises, fatality rates fall. But the results are not always consistent or
predictable.

Ten sizable countries, most of them in Western Europe, have tested
bigger percentages of their populations than has the United States,
according to
\href{https://www.worldometers.info/coronavirus/}{Worldometer}, which
gathers statistics. They are Iceland, Denmark, Spain, Portugal, Belgium,
Ireland, Italy, Britain, Israel and New Zealand.

But their case fatality rates
\href{https://fivethirtyeight.com/features/coronavirus-case-counts-are-meaningless/?campaign_id=9\&emc=edit_nn_20200605\&instance_id=19111\&nl=the-morning\&regi_id=3423762\&segment_id=30155\&te=1\&user_id=2f7d5ee538b6384a048aa9442c9b48c2}{vary
wildly}: Iceland's is less than 1 percent, New Zealand's and Israel's
are below 2 percent. Belgium, by comparison, is at 16 percent, and Italy
and Britain at 14 percent.

Both figures --- the infection fatality rate and the case fatality rate
--- can differ quite a bit by country.

So far, in most countries, about 20 percent of all confirmed Covid-19
patients become ill enough to need supplemental oxygen or even more
advanced hospital care, said Dr. Janet Diaz, head of clinical care for
the W.H.O.'s emergencies program.

Whether those patients survive depends on a host of factors, including
age, underlying illnesses and the level of medical care available.

Death rates are expected to be lower in countries with younger
populations and less obesity, which are often the poorest countries.
Conversely, the figures should be higher in countries that lack oxygen
tanks, ventilators and dialysis machines, and where many people live far
from hospitals. Those are also often the poorest countries.

\includegraphics{https://static01.graylady3jvrrxbe.onion/images/2020/07/04/science/04virus-fatality2/merlin_174192792_d189f060-40b4-48dd-9e3a-c4e166f91e08-articleLarge.jpg?quality=75\&auto=webp\&disable=upscale}

The W.H.O. and various charities are
\href{https://www.nytimes3xbfgragh.onion/2020/06/23/health/coronavirus-oxygen-africa.html}{scrambling
to purchase oxygen equipment} for poor and middle-income nations in
which the coronavirus is spreading.

And now, new factors are being introduced into the equation. For
example, new evidence that people with Type A blood
\href{https://www.nytimes3xbfgragh.onion/2020/06/03/health/coronavirus-blood-type-genetics.html}{are
more likely to fall deathly ill} could change risk calculations. Type A
blood
\href{https://en.wikipedia.org/wiki/Blood_type_distribution_by_country}{is
relatively rare} in West Africa and South Asia, and very rare among the
Indigenous peoples of South America.

Before this week's meeting, the W.H.O. had no official I.F.R. estimate,
Oliver Morgan, the agency's director of health emergency information and
risk assessment, said in an interview in early June.

Instead, it had relied on a mix of data sent in by member countries and
by academic groups, and on a meta-analysis done in May by scientists at
the University of Wollongong and James Cook University in Australia.

Those researchers looked at 267 studies in more than a dozen countries,
and then chose the 25 they considered the most accurate, weighting them
for accuracy and averaged the data. They concluded that the global
I.F.R. was 0.64 percent.

\href{https://www.nytimes3xbfgragh.onion/news-event/coronavirus?action=click\&pgtype=Article\&state=default\&region=MAIN_CONTENT_3\&context=storylines_faq}{}

\hypertarget{the-coronavirus-outbreak-}{%
\subsubsection{The Coronavirus Outbreak
›}\label{the-coronavirus-outbreak-}}

\hypertarget{frequently-asked-questions}{%
\paragraph{Frequently Asked
Questions}\label{frequently-asked-questions}}

Updated August 4, 2020

\begin{itemize}
\item ~
  \hypertarget{i-have-antibodies-am-i-now-immune}{%
  \paragraph{I have antibodies. Am I now
  immune?}\label{i-have-antibodies-am-i-now-immune}}

  \begin{itemize}
  \tightlist
  \item
    As of right
    now,\href{https://www.nytimes3xbfgragh.onion/2020/07/22/health/covid-antibodies-herd-immunity.html?action=click\&pgtype=Article\&state=default\&region=MAIN_CONTENT_3\&context=storylines_faq}{that
    seems likely, for at least several months.} There have been
    frightening accounts of people suffering what seems to be a second
    bout of Covid-19. But experts say these patients may have a
    drawn-out course of infection, with the virus taking a slow toll
    weeks to months after initial exposure. People infected with the
    coronavirus typically
    \href{https://www.nature.com/articles/s41586-020-2456-9}{produce}
    immune molecules called antibodies, which are
    \href{https://www.nytimes3xbfgragh.onion/2020/05/07/health/coronavirus-antibody-prevalence.html?action=click\&pgtype=Article\&state=default\&region=MAIN_CONTENT_3\&context=storylines_faq}{protective
    proteins made in response to an
    infection}\href{https://www.nytimes3xbfgragh.onion/2020/05/07/health/coronavirus-antibody-prevalence.html?action=click\&pgtype=Article\&state=default\&region=MAIN_CONTENT_3\&context=storylines_faq}{.
    These antibodies may} last in the body
    \href{https://www.nature.com/articles/s41591-020-0965-6}{only two to
    three months}, which may seem worrisome, but that's perfectly normal
    after an acute infection subsides, said Dr. Michael Mina, an
    immunologist at Harvard University. It may be possible to get the
    coronavirus again, but it's highly unlikely that it would be
    possible in a short window of time from initial infection or make
    people sicker the second time.
  \end{itemize}
\item ~
  \hypertarget{im-a-small-business-owner-can-i-get-relief}{%
  \paragraph{I'm a small-business owner. Can I get
  relief?}\label{im-a-small-business-owner-can-i-get-relief}}

  \begin{itemize}
  \tightlist
  \item
    The
    \href{https://www.nytimes3xbfgragh.onion/article/small-business-loans-stimulus-grants-freelancers-coronavirus.html?action=click\&pgtype=Article\&state=default\&region=MAIN_CONTENT_3\&context=storylines_faq}{stimulus
    bills enacted in March} offer help for the millions of American
    small businesses. Those eligible for aid are businesses and
    nonprofit organizations with fewer than 500 workers, including sole
    proprietorships, independent contractors and freelancers. Some
    larger companies in some industries are also eligible. The help
    being offered, which is being managed by the Small Business
    Administration, includes the Paycheck Protection Program and the
    Economic Injury Disaster Loan program. But lots of folks have
    \href{https://www.nytimes3xbfgragh.onion/interactive/2020/05/07/business/small-business-loans-coronavirus.html?action=click\&pgtype=Article\&state=default\&region=MAIN_CONTENT_3\&context=storylines_faq}{not
    yet seen payouts.} Even those who have received help are confused:
    The rules are draconian, and some are stuck sitting on
    \href{https://www.nytimes3xbfgragh.onion/2020/05/02/business/economy/loans-coronavirus-small-business.html?action=click\&pgtype=Article\&state=default\&region=MAIN_CONTENT_3\&context=storylines_faq}{money
    they don't know how to use.} Many small-business owners are getting
    less than they expected or
    \href{https://www.nytimes3xbfgragh.onion/2020/06/10/business/Small-business-loans-ppp.html?action=click\&pgtype=Article\&state=default\&region=MAIN_CONTENT_3\&context=storylines_faq}{not
    hearing anything at all.}
  \end{itemize}
\item ~
  \hypertarget{what-are-my-rights-if-i-am-worried-about-going-back-to-work}{%
  \paragraph{What are my rights if I am worried about going back to
  work?}\label{what-are-my-rights-if-i-am-worried-about-going-back-to-work}}

  \begin{itemize}
  \tightlist
  \item
    Employers have to provide
    \href{https://www.osha.gov/SLTC/covid-19/standards.html}{a safe
    workplace} with policies that protect everyone equally.
    \href{https://www.nytimes3xbfgragh.onion/article/coronavirus-money-unemployment.html?action=click\&pgtype=Article\&state=default\&region=MAIN_CONTENT_3\&context=storylines_faq}{And
    if one of your co-workers tests positive for the coronavirus, the
    C.D.C.} has said that
    \href{https://www.cdc.gov/coronavirus/2019-ncov/community/guidance-business-response.html}{employers
    should tell their employees} -\/- without giving you the sick
    employee's name -\/- that they may have been exposed to the virus.
  \end{itemize}
\item ~
  \hypertarget{should-i-refinance-my-mortgage}{%
  \paragraph{Should I refinance my
  mortgage?}\label{should-i-refinance-my-mortgage}}

  \begin{itemize}
  \tightlist
  \item
    \href{https://www.nytimes3xbfgragh.onion/article/coronavirus-money-unemployment.html?action=click\&pgtype=Article\&state=default\&region=MAIN_CONTENT_3\&context=storylines_faq}{It
    could be a good idea,} because mortgage rates have
    \href{https://www.nytimes3xbfgragh.onion/2020/07/16/business/mortgage-rates-below-3-percent.html?action=click\&pgtype=Article\&state=default\&region=MAIN_CONTENT_3\&context=storylines_faq}{never
    been lower.} Refinancing requests have pushed mortgage applications
    to some of the highest levels since 2008, so be prepared to get in
    line. But defaults are also up, so if you're thinking about buying a
    home, be aware that some lenders have tightened their standards.
  \end{itemize}
\item ~
  \hypertarget{what-is-school-going-to-look-like-in-september}{%
  \paragraph{What is school going to look like in
  September?}\label{what-is-school-going-to-look-like-in-september}}

  \begin{itemize}
  \tightlist
  \item
    It is unlikely that many schools will return to a normal schedule
    this fall, requiring the grind of
    \href{https://www.nytimes3xbfgragh.onion/2020/06/05/us/coronavirus-education-lost-learning.html?action=click\&pgtype=Article\&state=default\&region=MAIN_CONTENT_3\&context=storylines_faq}{online
    learning},
    \href{https://www.nytimes3xbfgragh.onion/2020/05/29/us/coronavirus-child-care-centers.html?action=click\&pgtype=Article\&state=default\&region=MAIN_CONTENT_3\&context=storylines_faq}{makeshift
    child care} and
    \href{https://www.nytimes3xbfgragh.onion/2020/06/03/business/economy/coronavirus-working-women.html?action=click\&pgtype=Article\&state=default\&region=MAIN_CONTENT_3\&context=storylines_faq}{stunted
    workdays} to continue. California's two largest public school
    districts --- Los Angeles and San Diego --- said on July 13, that
    \href{https://www.nytimes3xbfgragh.onion/2020/07/13/us/lausd-san-diego-school-reopening.html?action=click\&pgtype=Article\&state=default\&region=MAIN_CONTENT_3\&context=storylines_faq}{instruction
    will be remote-only in the fall}, citing concerns that surging
    coronavirus infections in their areas pose too dire a risk for
    students and teachers. Together, the two districts enroll some
    825,000 students. They are the largest in the country so far to
    abandon plans for even a partial physical return to classrooms when
    they reopen in August. For other districts, the solution won't be an
    all-or-nothing approach.
    \href{https://bioethics.jhu.edu/research-and-outreach/projects/eschool-initiative/school-policy-tracker/}{Many
    systems}, including the nation's largest, New York City, are
    devising
    \href{https://www.nytimes3xbfgragh.onion/2020/06/26/us/coronavirus-schools-reopen-fall.html?action=click\&pgtype=Article\&state=default\&region=MAIN_CONTENT_3\&context=storylines_faq}{hybrid
    plans} that involve spending some days in classrooms and other days
    online. There's no national policy on this yet, so check with your
    municipal school system regularly to see what is happening in your
    community.
  \end{itemize}
\end{itemize}

The C.D.C. relies on a ``symptomatic case fatality ratio'' that ``is not
necessarily equivalent to the number of reported deaths per reported
cases.'' The best estimate for the United States is 0.4 percent,
according to
\href{https://www.cdc.gov/coronavirus/2019-ncov/hcp/planning-scenarios.html}{a
set of planning scenarios} released in late May.

The agency did not respond to requests to explain how it arrived at that
figure, or why it was so much lower than the W.H.O.'s estimate. By
comparison, 0.4 percent of the United States population is 1.3 million
people.

The 25 studies that the Australian researchers considered the most
accurate relied on very different methodologies. One report, for
example, was based on diagnostic PCR tests of all passengers and crew
aboard the Diamond Princess, the cruise ship that docked in Japan after
it was overcome by the coronavirus. Another study drew data from an
antibody survey of 38,000 Spaniards, while another included only 1,104
Swedes.

The current W.H.O. estimate is based on later, larger studies of how
many people have antibodies in their blood; future studies may further
refine the figure, Dr. Swaminathan said.

But there is ``a lot of uncertainty'' about how many silent and untested
carriers there are, Dr. Morgan of the W.H.O. said.

To arrive at the C.D.C.'s new estimate, researchers tested samples from
11,933 people for antibodies to the coronavirus in six regions in the
United States. New York City reported 53,803 cases by April 1, but the
actual number of infections was 12 times higher --- nearly 642,000, the
agency estimated.

New York City's prevalence of 7 percent in the C.D.C. study was well
below the 21 percent estimated in
\href{https://www.sciencedirect.com/science/article/pii/S1047279720302015}{a
state survey} in April. But that number was based on people recruited at
supermarkets, and so the results may have been biased toward people out
shopping during a pandemic --- often the young, who have been less
affected.

The global fatality rates could still change. With one or two
exceptions, like
\href{https://www.nytimes3xbfgragh.onion/2020/05/18/world/middleeast/iran-coronavirus-surge.html}{Iran}
and
\href{https://www.nytimes3xbfgragh.onion/2020/04/23/world/americas/ecuador-deaths-coronavirus.html}{Ecuador},
the pandemic first struck wealthier countries in Asia, Western Europe
and North America where advanced medical care was available.

Now it is spreading widely in India, Brazil, Mexico, Nigeria and other
countries where millions are crowded into slums, lockdowns have been
relatively brief and hospitals have few resources.

But the death rates may also shift in wealthier northern countries as
winter approaches. Most of the spread of the virus in Europe and North
America has taken place during mild or warm weather in the spring and
summer.

Many experts fear that infections and deaths will shoot up in the fall
as colder weather forces people indoors, where they are more likely to
infect one another. Discipline about wearing masks and avoiding
breathing on one another will be even more important then.

In each of the eight influenza pandemics to hit the United States since
1763, a relatively mild first wave --- no matter what time of year it
arrived --- was followed by a larger, much more lethal wave a few months
later, noted Michael T. Osterholm, director of the Center for Infectious
Disease Research and Policy at the University of Minnesota.

More than a third of all the people killed by the Spanish flu, which
lasted from March 1918 to late 1920, died in the short stretch between
September and December 1918 --- about six months after a first,
relatively mild version of what may have been the same virus broke out
in western Kansas.

``We will go much higher in the next 12 to 18 months,'' Dr. Osterholm
said. Because this is a coronavirus, not influenza, it may not follow
the same pattern, but it is ``a much more efficient transmitter than
influenza.''

Advertisement

\protect\hyperlink{after-bottom}{Continue reading the main story}

\hypertarget{site-index}{%
\subsection{Site Index}\label{site-index}}

\hypertarget{site-information-navigation}{%
\subsection{Site Information
Navigation}\label{site-information-navigation}}

\begin{itemize}
\tightlist
\item
  \href{https://help.nytimes3xbfgragh.onion/hc/en-us/articles/115014792127-Copyright-notice}{©~2020~The
  New York Times Company}
\end{itemize}

\begin{itemize}
\tightlist
\item
  \href{https://www.nytco.com/}{NYTCo}
\item
  \href{https://help.nytimes3xbfgragh.onion/hc/en-us/articles/115015385887-Contact-Us}{Contact
  Us}
\item
  \href{https://www.nytco.com/careers/}{Work with us}
\item
  \href{https://nytmediakit.com/}{Advertise}
\item
  \href{http://www.tbrandstudio.com/}{T Brand Studio}
\item
  \href{https://www.nytimes3xbfgragh.onion/privacy/cookie-policy\#how-do-i-manage-trackers}{Your
  Ad Choices}
\item
  \href{https://www.nytimes3xbfgragh.onion/privacy}{Privacy}
\item
  \href{https://help.nytimes3xbfgragh.onion/hc/en-us/articles/115014893428-Terms-of-service}{Terms
  of Service}
\item
  \href{https://help.nytimes3xbfgragh.onion/hc/en-us/articles/115014893968-Terms-of-sale}{Terms
  of Sale}
\item
  \href{https://spiderbites.nytimes3xbfgragh.onion}{Site Map}
\item
  \href{https://help.nytimes3xbfgragh.onion/hc/en-us}{Help}
\item
  \href{https://www.nytimes3xbfgragh.onion/subscription?campaignId=37WXW}{Subscriptions}
\end{itemize}
