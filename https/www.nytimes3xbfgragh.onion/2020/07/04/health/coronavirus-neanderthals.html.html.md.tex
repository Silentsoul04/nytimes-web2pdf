Sections

SEARCH

\protect\hyperlink{site-content}{Skip to
content}\protect\hyperlink{site-index}{Skip to site index}

\href{https://www.nytimes3xbfgragh.onion/section/health}{Health}

\href{https://myaccount.nytimes3xbfgragh.onion/auth/login?response_type=cookie\&client_id=vi}{}

\href{https://www.nytimes3xbfgragh.onion/section/todayspaper}{Today's
Paper}

\href{/section/health}{Health}\textbar{}DNA Inherited From Neanderthals
May Increase Risk of Covid-19

\url{https://nyti.ms/2YX7Ekv}

\begin{itemize}
\item
\item
\item
\item
\item
\item
\end{itemize}

\hypertarget{the-coronavirus-outbreak}{%
\subsubsection{\texorpdfstring{\href{https://www.nytimes3xbfgragh.onion/news-event/coronavirus?name=styln-coronavirus-national\&region=TOP_BANNER\&block=storyline_menu_recirc\&action=click\&pgtype=Article\&impression_id=4e7d9530-efba-11ea-bf6c-d9d97a131239\&variant=undefined}{The
Coronavirus
Outbreak}}{The Coronavirus Outbreak}}\label{the-coronavirus-outbreak}}

\begin{itemize}
\tightlist
\item
  live\href{https://www.nytimes3xbfgragh.onion/2020/09/05/world/coronavirus-covid.html?name=styln-coronavirus-national\&region=TOP_BANNER\&block=storyline_menu_recirc\&action=click\&pgtype=Article\&impression_id=4e7d9531-efba-11ea-bf6c-d9d97a131239\&variant=undefined}{Latest
  Updates}
\item
  \href{https://www.nytimes3xbfgragh.onion/interactive/2020/us/coronavirus-us-cases.html?name=styln-coronavirus-national\&region=TOP_BANNER\&block=storyline_menu_recirc\&action=click\&pgtype=Article\&impression_id=4e7d9532-efba-11ea-bf6c-d9d97a131239\&variant=undefined}{Maps
  and Cases}
\item
  \href{https://www.nytimes3xbfgragh.onion/interactive/2020/science/coronavirus-vaccine-tracker.html?name=styln-coronavirus-national\&region=TOP_BANNER\&block=storyline_menu_recirc\&action=click\&pgtype=Article\&impression_id=4e7d9533-efba-11ea-bf6c-d9d97a131239\&variant=undefined}{Vaccine
  Tracker}
\item
  \href{https://www.nytimes3xbfgragh.onion/2020/09/02/your-money/eviction-moratorium-covid.html?name=styln-coronavirus-national\&region=TOP_BANNER\&block=storyline_menu_recirc\&action=click\&pgtype=Article\&impression_id=4e7dbc40-efba-11ea-bf6c-d9d97a131239\&variant=undefined}{Eviction
  Moratorium}
\item
  \href{https://www.nytimes3xbfgragh.onion/interactive/2020/09/02/magazine/food-insecurity-hunger-us.html?name=styln-coronavirus-national\&region=TOP_BANNER\&block=storyline_menu_recirc\&action=click\&pgtype=Article\&impression_id=4e7dbc41-efba-11ea-bf6c-d9d97a131239\&variant=undefined}{American
  Hunger}
\end{itemize}

Advertisement

\protect\hyperlink{after-top}{Continue reading the main story}

Supported by

\protect\hyperlink{after-sponsor}{Continue reading the main story}

\hypertarget{dna-inherited-from-neanderthals-may-increase-risk-of-covid-19}{%
\section{DNA Inherited From Neanderthals May Increase Risk of
Covid-19}\label{dna-inherited-from-neanderthals-may-increase-risk-of-covid-19}}

The stretch of six genes seems to increase the risk of severe illness
from the coronavirus.

\includegraphics{https://static01.graylady3jvrrxbe.onion/images/2020/07/04/multimedia/04virus-neanderthals-pix/04virus-neanderthals-pix-articleLarge.jpg?quality=75\&auto=webp\&disable=upscale}

\href{https://www.nytimes3xbfgragh.onion/by/carl-zimmer}{\includegraphics{https://static01.graylady3jvrrxbe.onion/images/2018/06/12/multimedia/author-carl-zimmer/author-carl-zimmer-thumbLarge.png}}

By \href{https://www.nytimes3xbfgragh.onion/by/carl-zimmer}{Carl Zimmer}

\begin{itemize}
\item
  Published July 4, 2020Updated July 8, 2020
\item
  \begin{itemize}
  \item
  \item
  \item
  \item
  \item
  \item
  \end{itemize}
\end{itemize}

A stretch of DNA
\href{https://www.nytimes3xbfgragh.onion/2020/06/03/health/coronavirus-blood-type-genetics.html}{linked
to Covid-19} was passed down from Neanderthals 60,000 years ago,
according to a new study.

Scientists don't yet know why this particular segment increases the
\href{https://www.nytimes3xbfgragh.onion/2020/07/08/health/coronavirus-risk-factors.html}{risk}
of severe illness from the coronavirus. But the
\href{https://www.biorxiv.org/content/10.1101/2020.07.03.186296v1.full.pdf+html}{new
findings}, which were posted online on Friday and have not yet been
published in a scientific journal, show how some clues to modern health
stem from ancient history.

``This interbreeding effect that happened 60,000 years ago is still
having an impact today,'' said Joshua Akey, a geneticist at Princeton
University who was not involved in the new study.

This piece of the genome, which spans six genes on Chromosome 3, has had
a puzzling journey through human history, the study found. The variant
is now common in Bangladesh, where 63 percent of people carry at least
one copy. Across all of South Asia, almost one-third of people have
inherited the segment.

Elsewhere, however, the segment is far less common. Only 8 percent of
Europeans carry it, and just 4 percent have it in East Asia. It is
almost completely absent in Africa.

It's not clear what evolutionary pattern produced this distribution over
the past 60,000 years. ``That's the \$10,000 question,'' said Hugo
Zeberg, a geneticist at the Karolinska Institute in Sweden who was one
of the authors of the new study.

One possibility is that the Neanderthal version is harmful and has been
getting rarer over all. It's also possible that the segment improved
people's health in South Asia, perhaps providing a strong immune
response to viruses in the region.

\hypertarget{latest-updates-the-coronavirus-outbreak}{%
\section{\texorpdfstring{\href{https://www.nytimes3xbfgragh.onion/2020/09/04/world/covid-19-coronavirus.html?action=click\&pgtype=Article\&state=default\&region=MAIN_CONTENT_1\&context=storylines_live_updates}{Latest
Updates: The Coronavirus
Outbreak}}{Latest Updates: The Coronavirus Outbreak}}\label{latest-updates-the-coronavirus-outbreak}}

Updated 2020-09-05T12:05:40.998Z

\begin{itemize}
\tightlist
\item
  \href{https://www.nytimes3xbfgragh.onion/2020/09/04/world/covid-19-coronavirus.html?action=click\&pgtype=Article\&state=default\&region=MAIN_CONTENT_1\&context=storylines_live_updates\#link-1654f6ad}{Research
  connects vaping to a higher chance of catching the virus --- and
  suffering its worst effects.}
\item
  \href{https://www.nytimes3xbfgragh.onion/2020/09/04/world/covid-19-coronavirus.html?action=click\&pgtype=Article\&state=default\&region=MAIN_CONTENT_1\&context=storylines_live_updates\#link-52e4198a}{Another
  college football game won't be played as planned.}
\item
  \href{https://www.nytimes3xbfgragh.onion/2020/09/04/world/covid-19-coronavirus.html?action=click\&pgtype=Article\&state=default\&region=MAIN_CONTENT_1\&context=storylines_live_updates\#link-181cef0}{Pharmaceutical
  companies plan a joint pledge on safety standards as they move
  vaccines to the marketplace.}
\end{itemize}

\href{https://www.nytimes3xbfgragh.onion/2020/09/04/world/covid-19-coronavirus.html?action=click\&pgtype=Article\&state=default\&region=MAIN_CONTENT_1\&context=storylines_live_updates}{See
more updates}

More live coverage:
\href{https://www.nytimes3xbfgragh.onion/live/2020/09/04/business/stock-market-today-coronavirus?action=click\&pgtype=Article\&state=default\&region=MAIN_CONTENT_1\&context=storylines_live_updates}{Markets}

``One should stress that at this point this is pure speculation,'' said
Dr. Zeberg's co-author, Svante Paabo, the director of the Max Planck
Institute for Evolutionary Anthropology in Leipzig, Germany.

Researchers are only beginning to understand why Covid-19 is more
dangerous for some people than others. Older people are more likely to
become severely ill than younger ones. Men are at more risk than women.

Social inequality matters, too. In the United States, Black people are
\href{https://www.cdc.gov/coronavirus/2019-ncov/need-extra-precautions/racial-ethnic-minorities.html}{far
more likely} than white people to become severely ill from the
coronavirus, for example, most likely due in part to the country's
history of
\href{https://www.nytimes3xbfgragh.onion/2020/04/29/magazine/racial-disparities-covid-19.html}{systemic
racism}. It has left Black people with a high rate of chronic diseases
such as diabetes, as well as living conditions and jobs that may
increase exposure to the virus.

Genes play a role as well. Last month, researchers compared people in
Italy and Spain who became very sick with
\href{https://www.nytimes3xbfgragh.onion/2020/07/08/health/coronavirus-risk-factors.html}{Covid-19}
to those who had only mild infections. They found two places in the
genome
\href{https://www.nytimes3xbfgragh.onion/2020/06/03/health/coronavirus-blood-type-genetics.html}{associated
with a greater risk}. One is on Chromosome 9 and includes ABO, a gene
that determines blood type. The other is the Neanderthal segment on
Chromosome 3.

But these genetic findings are being rapidly updated as more people
infected with the coronavirus are studied. Just last week, an
international group of scientists called the
\href{https://www.covid19hg.org/}{Covid-19 Host Genetics Initiative}
released a new set of data downplaying the risk of blood type. ``The
jury is still out on ABO,'' said Mark Daly, a geneticist at Harvard
Medical School who is a member of the initiative.

The new data showed an even stronger link between the disease and the
Chromosome 3 segment. People who carry two copies of the variant are
three times more likely to suffer from severe illness than people who do
not.

After the new batch of data came out on Monday, Dr. Zeberg decided to
find out if the Chromosome 3 segment was passed down from Neanderthals.

About 60,000 years ago, some ancestors of modern humans
\href{https://www.nytimes3xbfgragh.onion/2020/01/31/science/neanderthal-dna-africa.html}{expanded
out of Africa} and swept across Europe, Asia and Australia. These people
encountered Neanderthals and interbred. Once Neanderthal DNA entered our
gene pool, it spread down through the generations, long after
Neanderthals became extinct.

Most Neanderthal genes turned out to be harmful to modern humans. They
may have been a burden on people's health or made it harder to have
children. As a result, Neanderthal genes became rarer, and many
disappeared from our gene pool.

But some genes appear to have provided an evolutionary edge and have
become quite common. In May, Dr. Zeberg, Dr. Paabo and Dr. Janet Kelso,
also of the Max Planck Institute, discovered that one-third of European
women have a
\href{https://academic.oup.com/mbe/article/doi/10.1093/molbev/msaa119/5841671}{Neanderthal
hormone receptor}. It is associated with increased fertility and fewer
miscarriages.

\href{https://www.nytimes3xbfgragh.onion/news-event/coronavirus?action=click\&pgtype=Article\&state=default\&region=MAIN_CONTENT_3\&context=storylines_faq}{}

\hypertarget{the-coronavirus-outbreak-}{%
\subsubsection{The Coronavirus Outbreak
›}\label{the-coronavirus-outbreak-}}

\hypertarget{frequently-asked-questions}{%
\paragraph{Frequently Asked
Questions}\label{frequently-asked-questions}}

Updated September 4, 2020

\begin{itemize}
\item ~
  \hypertarget{what-are-the-symptoms-of-coronavirus}{%
  \paragraph{What are the symptoms of
  coronavirus?}\label{what-are-the-symptoms-of-coronavirus}}

  \begin{itemize}
  \tightlist
  \item
    In the beginning, the coronavirus
    \href{https://www.nytimes3xbfgragh.onion/article/coronavirus-facts-history.html?action=click\&pgtype=Article\&state=default\&region=MAIN_CONTENT_3\&context=storylines_faq\#link-6817bab5}{seemed
    like it was primarily a respiratory illness}~--- many patients had
    fever and chills, were weak and tired, and coughed a lot, though
    some people don't show many symptoms at all. Those who seemed
    sickest had pneumonia or acute respiratory distress syndrome and
    received supplemental oxygen. By now, doctors have identified many
    more symptoms and syndromes. In April,
    \href{https://www.nytimes3xbfgragh.onion/2020/04/27/health/coronavirus-symptoms-cdc.html?action=click\&pgtype=Article\&state=default\&region=MAIN_CONTENT_3\&context=storylines_faq}{the
    C.D.C. added to the list of early signs}~sore throat, fever, chills
    and muscle aches. Gastrointestinal upset, such as diarrhea and
    nausea, has also been observed. Another telltale sign of infection
    may be a sudden, profound diminution of one's
    \href{https://www.nytimes3xbfgragh.onion/2020/03/22/health/coronavirus-symptoms-smell-taste.html?action=click\&pgtype=Article\&state=default\&region=MAIN_CONTENT_3\&context=storylines_faq}{sense
    of smell and taste.}~Teenagers and young adults in some cases have
    developed painful red and purple lesions on their fingers and toes
    --- nicknamed ``Covid toe'' --- but few other serious symptoms.
  \end{itemize}
\item ~
  \hypertarget{why-is-it-safer-to-spend-time-together-outside}{%
  \paragraph{Why is it safer to spend time together
  outside?}\label{why-is-it-safer-to-spend-time-together-outside}}

  \begin{itemize}
  \tightlist
  \item
    \href{https://www.nytimes3xbfgragh.onion/2020/05/15/us/coronavirus-what-to-do-outside.html?action=click\&pgtype=Article\&state=default\&region=MAIN_CONTENT_3\&context=storylines_faq}{Outdoor
    gatherings}~lower risk because wind disperses viral droplets, and
    sunlight can kill some of the virus. Open spaces prevent the virus
    from building up in concentrated amounts and being inhaled, which
    can happen when infected people exhale in a confined space for long
    stretches of time, said Dr. Julian W. Tang, a virologist at the
    University of Leicester.
  \end{itemize}
\item ~
  \hypertarget{why-does-standing-six-feet-away-from-others-help}{%
  \paragraph{Why does standing six feet away from others
  help?}\label{why-does-standing-six-feet-away-from-others-help}}

  \begin{itemize}
  \tightlist
  \item
    The coronavirus spreads primarily through droplets from your mouth
    and nose, especially when you cough or sneeze. The C.D.C., one of
    the organizations using that measure,
    \href{https://www.nytimes3xbfgragh.onion/2020/04/14/health/coronavirus-six-feet.html?action=click\&pgtype=Article\&state=default\&region=MAIN_CONTENT_3\&context=storylines_faq}{bases
    its recommendation of six feet}~on the idea that most large droplets
    that people expel when they cough or sneeze will fall to the ground
    within six feet. But six feet has never been a magic number that
    guarantees complete protection. Sneezes, for instance, can launch
    droplets a lot farther than six feet,
    \href{https://jamanetwork.com/journals/jama/fullarticle/2763852}{according
    to a recent study}. It's a rule of thumb: You should be safest
    standing six feet apart outside, especially when it's windy. But
    keep a mask on at all times, even when you think you're far enough
    apart.
  \end{itemize}
\item ~
  \hypertarget{i-have-antibodies-am-i-now-immune}{%
  \paragraph{I have antibodies. Am I now
  immune?}\label{i-have-antibodies-am-i-now-immune}}

  \begin{itemize}
  \tightlist
  \item
    As of right
    now,\href{https://www.nytimes3xbfgragh.onion/2020/07/22/health/covid-antibodies-herd-immunity.html?action=click\&pgtype=Article\&state=default\&region=MAIN_CONTENT_3\&context=storylines_faq}{~that
    seems likely, for at least several months.}~There have been
    frightening accounts of people suffering what seems to be a second
    bout of Covid-19. But experts say these patients may have a
    drawn-out course of infection, with the virus taking a slow toll
    weeks to months after initial exposure.~People infected with the
    coronavirus typically
    \href{https://www.nature.com/articles/s41586-020-2456-9}{produce}~immune
    molecules called antibodies, which are
    \href{https://www.nytimes3xbfgragh.onion/2020/05/07/health/coronavirus-antibody-prevalence.html?action=click\&pgtype=Article\&state=default\&region=MAIN_CONTENT_3\&context=storylines_faq}{protective
    proteins made in response to an
    infection}\href{https://www.nytimes3xbfgragh.onion/2020/05/07/health/coronavirus-antibody-prevalence.html?action=click\&pgtype=Article\&state=default\&region=MAIN_CONTENT_3\&context=storylines_faq}{.
    These antibodies may}~last in the body
    \href{https://www.nature.com/articles/s41591-020-0965-6}{only two to
    three months}, which may seem worrisome, but that's~perfectly normal
    after an acute infection subsides, said Dr. Michael Mina, an
    immunologist at Harvard University. It may be possible to get the
    coronavirus again, but it's highly unlikely that it would be
    possible in a short window of time from initial infection or make
    people sicker the second time.
  \end{itemize}
\item ~
  \hypertarget{what-are-my-rights-if-i-am-worried-about-going-back-to-work}{%
  \paragraph{What are my rights if I am worried about going back to
  work?}\label{what-are-my-rights-if-i-am-worried-about-going-back-to-work}}

  \begin{itemize}
  \tightlist
  \item
    Employers have to provide
    \href{https://www.osha.gov/SLTC/covid-19/standards.html}{a safe
    workplace}~with policies that protect everyone equally.
    \href{https://www.nytimes3xbfgragh.onion/article/coronavirus-money-unemployment.html?action=click\&pgtype=Article\&state=default\&region=MAIN_CONTENT_3\&context=storylines_faq}{And
    if one of your co-workers tests positive for the coronavirus, the
    C.D.C.}~has said that
    \href{https://www.cdc.gov/coronavirus/2019-ncov/community/guidance-business-response.html}{employers
    should tell their employees}~-\/- without giving you the sick
    employee's name -\/- that they may have been exposed to the virus.
  \end{itemize}
\end{itemize}

Dr. Zeberg knew that other Neanderthal genes that are common today even
\href{https://www.nytimes3xbfgragh.onion/2018/10/04/science/neanderthal-genes-viruses.html}{help
us fight viruses}. When modern humans expanded into Asia and Europe,
they may have encountered new viruses against which Neanderthals had
already evolved defenses. We have held onto those genes ever since.

Dr. Zeberg looked at Chromosome 3 in an online database of Neanderthal
genomes. He found that the version that raises people's risk of severe
Covid-19 is the same version found in a Neanderthal who lived in Croatia
50,000 years ago. ``I texted Svante immediately,'' Dr. Zeberg said in an
interview, referring to Dr. Paabo.

Dr. Paabo was on vacation in a cottage in the remote Swedish
countryside. Dr. Zeberg showed up the next day, and they worked day and
night until they posted the study online on Friday.

``It's the most crazy vacation I've ever had in this cottage,'' Dr.
Paabo said.

Tony Capra, a geneticist at Vanderbilt University who was not involved
in the study, thought it was plausible that the Neanderthal chunk of DNA
originally provided a benefit --- perhaps even against other viruses.
``But that was 40,000 years ago, and here we are now,'' he said.

It's possible that an immune response that worked against ancient
viruses has ended up overreacting against the new coronavirus. People
who develop severe cases of Covid-19 typically do so because their
immune systems launch uncontrolled attacks that end up scarring their
lungs and causing inflammation.

Dr. Paabo said the DNA segment may account in part for why people of
Bangladeshi descent are
\href{https://www.theguardian.com/world/2020/jun/19/south-asians-in-uk-most-likely-to-die-of-covid-19-study-finds}{dying
at a high rate} of Covid-19 in the United Kingdom.

It's an open question whether this Neanderthal segment continues to keep
a strong link to Covid-19 as Dr. Zeberg and other researchers study more
patients. And it may take discoveries of the segment in ancient fossils
of modern humans to understand why it became so common in some places
but not others.

But Dr. Zeberg said that the 60,000-year journey of this chunk of DNA in
our species might help explain why it's so dangerous today.

``Its evolutionary history may give us some clues,'' Dr. Zeberg said.

\textbf{\emph{{[}}\href{http://on.fb.me/1paTQ1h}{\emph{Like the Science
Times page on Facebook.}}} ****** \emph{\textbar{} Sign up for the}
\textbf{\href{http://nyti.ms/1MbHaRU}{\emph{Science Times
newsletter.}}\emph{{]}}}

Advertisement

\protect\hyperlink{after-bottom}{Continue reading the main story}

\hypertarget{site-index}{%
\subsection{Site Index}\label{site-index}}

\hypertarget{site-information-navigation}{%
\subsection{Site Information
Navigation}\label{site-information-navigation}}

\begin{itemize}
\tightlist
\item
  \href{https://help.nytimes3xbfgragh.onion/hc/en-us/articles/115014792127-Copyright-notice}{©~2020~The
  New York Times Company}
\end{itemize}

\begin{itemize}
\tightlist
\item
  \href{https://www.nytco.com/}{NYTCo}
\item
  \href{https://help.nytimes3xbfgragh.onion/hc/en-us/articles/115015385887-Contact-Us}{Contact
  Us}
\item
  \href{https://www.nytco.com/careers/}{Work with us}
\item
  \href{https://nytmediakit.com/}{Advertise}
\item
  \href{http://www.tbrandstudio.com/}{T Brand Studio}
\item
  \href{https://www.nytimes3xbfgragh.onion/privacy/cookie-policy\#how-do-i-manage-trackers}{Your
  Ad Choices}
\item
  \href{https://www.nytimes3xbfgragh.onion/privacy}{Privacy}
\item
  \href{https://help.nytimes3xbfgragh.onion/hc/en-us/articles/115014893428-Terms-of-service}{Terms
  of Service}
\item
  \href{https://help.nytimes3xbfgragh.onion/hc/en-us/articles/115014893968-Terms-of-sale}{Terms
  of Sale}
\item
  \href{https://spiderbites.nytimes3xbfgragh.onion}{Site Map}
\item
  \href{https://help.nytimes3xbfgragh.onion/hc/en-us}{Help}
\item
  \href{https://www.nytimes3xbfgragh.onion/subscription?campaignId=37WXW}{Subscriptions}
\end{itemize}
