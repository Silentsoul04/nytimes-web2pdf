Sections

SEARCH

\protect\hyperlink{site-content}{Skip to
content}\protect\hyperlink{site-index}{Skip to site index}

\href{https://www.nytimes3xbfgragh.onion/section/business}{Business}

\href{https://myaccount.nytimes3xbfgragh.onion/auth/login?response_type=cookie\&client_id=vi}{}

\href{https://www.nytimes3xbfgragh.onion/section/todayspaper}{Today's
Paper}

\href{/section/business}{Business}\textbar{}Atlantic Coast Pipeline
Canceled as Delays and Costs Mount

\url{https://nyti.ms/31LHYch}

\begin{itemize}
\item
\item
\item
\item
\item
\end{itemize}

Advertisement

\protect\hyperlink{after-top}{Continue reading the main story}

Supported by

\protect\hyperlink{after-sponsor}{Continue reading the main story}

\hypertarget{atlantic-coast-pipeline-canceled-as-delays-and-costs-mount}{%
\section{Atlantic Coast Pipeline Canceled as Delays and Costs
Mount}\label{atlantic-coast-pipeline-canceled-as-delays-and-costs-mount}}

The natural gas project would have crossed the Appalachian Trail.
Dominion Energy, one of the pipeline's two partners, also announced the
sale of its gas transmission and storage assets.

\includegraphics{https://static01.graylady3jvrrxbe.onion/images/2020/07/05/us/politics/05pipeline/merlin_169451838_1a2a42b5-6080-4465-bd95-e0b19da351cd-articleLarge.jpg?quality=75\&auto=webp\&disable=upscale}

\href{https://www.nytimes3xbfgragh.onion/by/ivan-penn}{\includegraphics{https://static01.graylady3jvrrxbe.onion/images/2018/06/12/multimedia/author-ivan-penn/author-ivan-penn-thumbLarge.png}}

By \href{https://www.nytimes3xbfgragh.onion/by/ivan-penn}{Ivan Penn}

\begin{itemize}
\item
  July 5, 2020
\item
  \begin{itemize}
  \item
  \item
  \item
  \item
  \item
  \end{itemize}
\end{itemize}

Two of the nation's largest utility companies announced on Sunday that
they had
\href{https://www.prnewswire.com/news-releases/dominion-energy-and-duke-energy-cancel-the-atlantic-coast-pipeline-301088177.html}{canceled
the Atlantic Coast Pipeline}, which would have carried natural gas
across the Appalachian Trail, as delays and rising costs threatened the
viability of the project.

Duke Energy and Dominion Energy said that lawsuits, mainly from
environmentalists aimed at blocking the project, had increased costs to
as much as \$8 billion from about \$4.5 billion to \$5 billion when it
was first announced in 2014. The utilities said they had begun
developing the project ``in response to a lack of energy supply and
delivery diversification for millions of families, businesses, schools
and national defense installations across North Carolina and Virginia.''

The two energy companies won a victory just last month in the
\href{https://www.nytimes3xbfgragh.onion/2020/02/24/us/supreme-court-appalachian-trail.html}{Supreme
Court over a permit} from the U.S. Forest Service, but said that
``recent developments have created an unacceptable layer of uncertainty
and anticipated delays'' for the pipeline. They cited the potential for
further legal challenges.

Dominion also said on Sunday that it was
\href{https://www.prnewswire.com/news-releases/dominion-energy-agrees-to-sell-gas-transmission-storage-assets-to-berkshire-hathaway-energy-strategic-repositioning-toward-pure-play-state-regulated-sustainability-focused-utility-operations-301088176.html}{selling
all of its gas transmission and storage assets} to an affiliate of
Warren Buffett's Berkshire Hathaway in a deal valued at \$9.7 billion.

Environmental groups have long criticized Dominion and Duke for their
continued development of fossil fuel projects. The two companies have
argued that they have increasingly added renewable energy sources to
produce electricity that include wind, solar and hydro power, but they
also contend that they need natural gas for the times when those clean
energy resources are not available.

``For almost six years we have worked diligently and invested billions
of dollars to complete the project and deliver the much-needed
infrastructure to our customers and communities,'' executives for
Dominion and Duke said in a prepared statement. ``This announcement
reflects the increasing legal uncertainty that overhangs large-scale
energy and industrial infrastructure development in the United States.''

Gillian Giannetti, a lawyer with the Sustainable FERC Project at the
Natural Resources Defense Council, quickly issued a statement in support
of the utilities' move. ``The costly and unneeded Atlantic Coast
Pipeline would have threatened waterways and communities across its
600-mile path,'' she said. ``As they abandon this dirty pipe dream,
Dominion and Duke should now pivot to investing more in energy
efficiency, wind and solar --- that's how to provide jobs and a better
future for all.''

Dominion's agreement with Berkshire Hathaway Energy includes the sale of
more than 7,700 miles of natural gas storage and transmission pipelines
and about 900 billion cubic feet of gas storage. As part of the deal,
Berkshire Hathaway will assume about \$5.7 billion in debt and make a
\$4 billion cash payment to Dominion at the closing.

Dominion, which is based in Richmond, Va., provides electricity and
natural gas to more than seven million customers across 20 states.
Berkshire Hathaway Energy's operations include a portfolio of more than
\$100 billion in assets and provide service to 12 million electric and
natural gas users.

``We are very proud to be adding such a great portfolio of natural gas
assets to our already strong energy business,'' Mr. Buffett said in a
statement.

Advertisement

\protect\hyperlink{after-bottom}{Continue reading the main story}

\hypertarget{site-index}{%
\subsection{Site Index}\label{site-index}}

\hypertarget{site-information-navigation}{%
\subsection{Site Information
Navigation}\label{site-information-navigation}}

\begin{itemize}
\tightlist
\item
  \href{https://help.nytimes3xbfgragh.onion/hc/en-us/articles/115014792127-Copyright-notice}{©~2020~The
  New York Times Company}
\end{itemize}

\begin{itemize}
\tightlist
\item
  \href{https://www.nytco.com/}{NYTCo}
\item
  \href{https://help.nytimes3xbfgragh.onion/hc/en-us/articles/115015385887-Contact-Us}{Contact
  Us}
\item
  \href{https://www.nytco.com/careers/}{Work with us}
\item
  \href{https://nytmediakit.com/}{Advertise}
\item
  \href{http://www.tbrandstudio.com/}{T Brand Studio}
\item
  \href{https://www.nytimes3xbfgragh.onion/privacy/cookie-policy\#how-do-i-manage-trackers}{Your
  Ad Choices}
\item
  \href{https://www.nytimes3xbfgragh.onion/privacy}{Privacy}
\item
  \href{https://help.nytimes3xbfgragh.onion/hc/en-us/articles/115014893428-Terms-of-service}{Terms
  of Service}
\item
  \href{https://help.nytimes3xbfgragh.onion/hc/en-us/articles/115014893968-Terms-of-sale}{Terms
  of Sale}
\item
  \href{https://spiderbites.nytimes3xbfgragh.onion}{Site Map}
\item
  \href{https://help.nytimes3xbfgragh.onion/hc/en-us}{Help}
\item
  \href{https://www.nytimes3xbfgragh.onion/subscription?campaignId=37WXW}{Subscriptions}
\end{itemize}
