Sections

SEARCH

\protect\hyperlink{site-content}{Skip to
content}\protect\hyperlink{site-index}{Skip to site index}

\href{https://www.nytimes3xbfgragh.onion/section/nyregion}{New York}

\href{https://myaccount.nytimes3xbfgragh.onion/auth/login?response_type=cookie\&client_id=vi}{}

\href{https://www.nytimes3xbfgragh.onion/section/todayspaper}{Today's
Paper}

\href{/section/nyregion}{New York}\textbar{}How You Get Your Berries:
Migrant Workers Who Fear Virus, but Toil On

\url{https://nyti.ms/3itW2gt}

\begin{itemize}
\item
\item
\item
\item
\item
\end{itemize}

\href{https://www.nytimes3xbfgragh.onion/news-event/coronavirus?action=click\&pgtype=Article\&state=default\&region=TOP_BANNER\&context=storylines_menu}{The
Coronavirus Outbreak}

\begin{itemize}
\tightlist
\item
  live\href{https://www.nytimes3xbfgragh.onion/2020/08/01/world/coronavirus-covid-19.html?action=click\&pgtype=Article\&state=default\&region=TOP_BANNER\&context=storylines_menu}{Latest
  Updates}
\item
  \href{https://www.nytimes3xbfgragh.onion/interactive/2020/us/coronavirus-us-cases.html?action=click\&pgtype=Article\&state=default\&region=TOP_BANNER\&context=storylines_menu}{Maps
  and Cases}
\item
  \href{https://www.nytimes3xbfgragh.onion/interactive/2020/science/coronavirus-vaccine-tracker.html?action=click\&pgtype=Article\&state=default\&region=TOP_BANNER\&context=storylines_menu}{Vaccine
  Tracker}
\item
  \href{https://www.nytimes3xbfgragh.onion/interactive/2020/07/29/us/schools-reopening-coronavirus.html?action=click\&pgtype=Article\&state=default\&region=TOP_BANNER\&context=storylines_menu}{What
  School May Look Like}
\item
  \href{https://www.nytimes3xbfgragh.onion/live/2020/07/31/business/stock-market-today-coronavirus?action=click\&pgtype=Article\&state=default\&region=TOP_BANNER\&context=storylines_menu}{Economy}
\end{itemize}

Advertisement

\protect\hyperlink{after-top}{Continue reading the main story}

Supported by

\protect\hyperlink{after-sponsor}{Continue reading the main story}

\hypertarget{how-you-get-your-berries-migrant-workers-who-fear-virus-but-toil-on}{%
\section{How You Get Your Berries: Migrant Workers Who Fear Virus, but
Toil
On}\label{how-you-get-your-berries-migrant-workers-who-fear-virus-but-toil-on}}

Many laborers in New Jersey follow the ripening of crops up the East
Coast. Each influx of new workers brings the risk of a fresh outbreak.

\includegraphics{https://static01.graylady3jvrrxbe.onion/images/2020/07/05/nyregion/00njvirus-migrant/00njvirus-migrant-articleLarge.jpg?quality=75\&auto=webp\&disable=upscale}

By \href{https://www.nytimes3xbfgragh.onion/by/tracey-tully}{Tracey
Tully}

\begin{itemize}
\item
  July 5, 2020
\item
  \begin{itemize}
  \item
  \item
  \item
  \item
  \item
  \end{itemize}
\end{itemize}

HAMMONTON, N.J. --- Workers at the largest blueberry farm in the
Northeast move through the fields in small groups, fingers dancing with
the speed of musicians as they pick bushes heavy with fruit.

The more they gather, the more they are paid during a season that lasts
only about seven weeks.

Barring rain, they work seven days a week; there is no time for illness.

But everywhere there are reminders of the coronavirus and its power to
sweep quickly through tightly packed farm camps.

It is the reason laborers who live and toil close together wear
bandannas across their faces in the hot sun and work separated by
plexiglass in the fruit packaging facility.

It is what had them standing in line on a steamy morning, weeks before
picking started, to be tested for the virus at the large farm in
southern New Jersey, Atlantic Blueberry Company in Hammonton.

``It feels a little uncomfortable,'' said Angel Rodriguez, who works in
the farm's packaging facility. ``You don't know if somebody is
contagious.''

Mr. Rodriguez, 34, left Puerto Rico in March to begin working his way up
the East Coast, stopping for two months in Florida before arriving in
late May in Atlantic County, the hub of New Jersey's
\href{https://www.nj.gov/agriculture/news/press/2019/approved/press190701.html\#:~:text=Blueberries\%20were\%20the\%20No.,on\%209\%2C000\%20acres\%20last\%20year.}{thriving
blueberry industry}.

He is one of an estimated
\href{https://www.ncbi.nlm.nih.gov/pmc/articles/PMC4040280/\#R10}{22,000}
seasonal workers who tend and harvest crops in New Jersey, nicknamed the
\href{https://www.state.nj.us/nj/about/facts/nickname/\#:~:text=Abraham\%20Browning\%20of\%20Camden\%20is,the\%20nickname\%20the\%20Garden\%20State.\&text=Browning\%20said\%20that\%20our\%20Garden,The\%20name\%20stuck\%20ever\%20since.}{Garden
State} for its robust agriculture industry.

Like Mr. Rodriguez, many laborers follow the ripening crops up the
Eastern Seaboard, starting in
\href{https://www.nytimes3xbfgragh.onion/2020/06/25/us/coronavirus-florida-texas-california-arizona.html}{Florida},
where
\href{https://www.nytimes3xbfgragh.onion/2020/06/18/us/florida-coronavirus-immokalee-farmworkers.html}{migrant
living quarters have been ravaged} by the virus, and working their way
north to Maine.

Making life even more perilous this year, they have been deemed
\href{https://www.cisa.gov/sites/default/files/publications/CISA-Guidance-on-Essential-Critical-Infrastructure-Workers-1-20-508c.pdf}{essential
workers} --- exempt from stay-at-home orders and a
\href{https://www.nytimes3xbfgragh.onion/2020/06/24/nyregion/ny-coronavirus-states-quarantine.html?referringSource=articleShare}{14-day
quarantine rule} in New Jersey for people coming from states where the
virus is spreading quickly. With each influx of new workers comes the
risk of a fresh outbreak.

\hypertarget{latest-updates-global-coronavirus-outbreak}{%
\section{\texorpdfstring{\href{https://www.nytimes3xbfgragh.onion/2020/08/01/world/coronavirus-covid-19.html?action=click\&pgtype=Article\&state=default\&region=MAIN_CONTENT_1\&context=storylines_live_updates}{Latest
Updates: Global Coronavirus
Outbreak}}{Latest Updates: Global Coronavirus Outbreak}}\label{latest-updates-global-coronavirus-outbreak}}

Updated 2020-08-01T19:54:00.494Z

\begin{itemize}
\tightlist
\item
  \href{https://www.nytimes3xbfgragh.onion/2020/08/01/world/coronavirus-covid-19.html?action=click\&pgtype=Article\&state=default\&region=MAIN_CONTENT_1\&context=storylines_live_updates\#link-3ac56579}{Top
  officials work to break impasse over jobless benefit.}
\item
  \href{https://www.nytimes3xbfgragh.onion/2020/08/01/world/coronavirus-covid-19.html?action=click\&pgtype=Article\&state=default\&region=MAIN_CONTENT_1\&context=storylines_live_updates\#link-8796723}{The
  virus picks up dangerous speed in the Midwest, and in areas that had
  seen success.}
\item
  \href{https://www.nytimes3xbfgragh.onion/2020/08/01/world/coronavirus-covid-19.html?action=click\&pgtype=Article\&state=default\&region=MAIN_CONTENT_1\&context=storylines_live_updates\#link-25930521}{Thousands
  in Berlin protest Germany's coronavirus measures.}
\end{itemize}

\href{https://www.nytimes3xbfgragh.onion/2020/08/01/world/coronavirus-covid-19.html?action=click\&pgtype=Article\&state=default\&region=MAIN_CONTENT_1\&context=storylines_live_updates}{See
more updates}

More live coverage:
\href{https://www.nytimes3xbfgragh.onion/live/2020/07/31/business/stock-market-today-coronavirus?action=click\&pgtype=Article\&state=default\&region=MAIN_CONTENT_1\&context=storylines_live_updates}{Markets}

In New Jersey, 3,900 farmworkers had been tested as of Thursday and 193
were positive for the virus, according to the state's Department of
Health. Of these, 14 migrant workers who had nowhere to remain isolated
were placed in quarantine at a state-run field hospital at the Atlantic
City Convention Center.

\includegraphics{https://static01.graylady3jvrrxbe.onion/images/2020/07/05/nyregion/00njvirus-migrant-02/00njvirus-migrant-02-articleLarge.jpg?quality=75\&auto=webp\&disable=upscale}

``It's a little dangerous,'' said Felix Nieves, 56, who works as a
supervisor at Atlantic Blueberry. The 1,300-acre farm is considered the
biggest blueberry producer in the Northeast.

``But farming never stops. The fruit will not wait for this to pass.''

The first round of testing at
\href{https://www.atlanticblueberry.com/history}{Atlantic Blueberry} was
done early in the season, before most workers had arrived. Three of the
first 56 people tested were positive for the virus.

The health risks posed by the virus have made testing a priority at the
sprawling farm, according to an owner, Paul Galletta.

``As often as they can come, we will test,'' Mr. Galletta said of the
health workers who wore white jumpsuits, masks, face shields and gloves
as they gathered nasal swabs. They have returned three times.

A sick work force during a short growing season could be financially
catastrophic.

``This crop comes in, virus or no virus,'' said Denny Doyle, president
of the New Jersey Blueberry Industry Advisory Council.

Atlantic Blueberry purchased 3,000 bandannas and gave each worker two
--- one to wear, one to wash --- and hung fire-retardant cloth between
beds in the dormitories where hundreds of laborers live during the
season. Mr. Doyle said the farm also purchased several additional buses
to create extra space on the shuttles that run to and from the fields.

Agriculture is New Jersey's third-largest industry. The state is among
the nation's
\href{https://www.nj.gov/agriculture/pdf/2018Annual\%20ReportFINAL.pdf}{top
producers} of
\href{https://www.nj.gov/agriculture/news/press/2019/approved/press190701.html\#:~:text=Blueberries\%20were\%20the\%20No.,on\%209\%2C000\%20acres\%20last\%20year.}{blueberries},
cranberries, peaches and eggplant.

In May, state health officials arranged for four federally qualified
health centers to begin testing and
\href{https://nj.gov/health/cd/documents/topics/NCOV/COVID_MigrantFarmWorkerGuidance_5.20.2020.pdf}{issued
safety guidelines} that offered a range of ambitious --- some say
impractical --- suggestions for farm owners. Farmers were told to avoid
bunked beds, require masks and create separate housing for anyone who
tested positive for the virus, among other recommendations.

There are no penalties for noncompliance.

New Jersey's 5 percent rate of infection among farmworkers may actually
be higher. Day laborers who do not live on the farms are unlikely to be
among those tested by the health centers. Workers who are tested in
private medical practices are not included in the tally.

The testing program is also voluntary, and 57 farms have barred medical
teams from doing on-site testing, according to Dr. Lori Talbot, who
treats migrant farm workers and viewed the list of noncompliant farms
that was sent to the state's health and labor departments.

Image

Agriculture is New Jersey's third-largest industry.~``Farming never
stops,'' said Felix Nieves, a supervisor at Atlantic Blueberry Company.
``The fruit will not wait for this to pass.''Credit...Chang W. Lee/The
New York Times

Dr. Talbot, who runs a clinic in
\href{https://www.nytimes3xbfgragh.onion/2019/10/01/nyregion/missing-child-nj-dulce-alavez.html}{Bridgeton,
N.J.}, said 18 percent of the 200 farmworkers she tested in May were
positive for the coronavirus; many were asymptomatic, but two patients
died of Covid-19.

``This is just a whole new level of pain for farmworkers,'' Dr. Talbot
said. ``They're arriving now, and they're coming from places with high
rates of infection.''

The state's health commissioner, Judith M. Persichilli, has cited the
prevalence of cases among farmworkers as one of the possible reasons the
positivity rate in South Jersey is now higher than it is in other parts
of the state.

Linda Flake, the chief executive of Southern Jersey Family Medical
Center, one of the four health centers coordinating testing, said the
perception that workers might carry the virus breeds a fear that, in
ways, is worse than the risk of the disease itself.

``Fingers are being pointed at the farmworkers,'' she said. ``I'm more
concerned about them being stigmatized.''

In May at a large agricultural greenhouse in Oneida, N.Y., Green Empire
Farms, one in four workers contracted the virus, according to a
spokeswoman for Madison County, Samantha Field. Community backlash
followed, playing out on social media and in panicked telephone calls.

\href{https://www.nytimes3xbfgragh.onion/news-event/coronavirus?action=click\&pgtype=Article\&state=default\&region=MAIN_CONTENT_3\&context=storylines_faq}{}

\hypertarget{the-coronavirus-outbreak-}{%
\subsubsection{The Coronavirus Outbreak
›}\label{the-coronavirus-outbreak-}}

\hypertarget{frequently-asked-questions}{%
\paragraph{Frequently Asked
Questions}\label{frequently-asked-questions}}

Updated July 27, 2020

\begin{itemize}
\item ~
  \hypertarget{should-i-refinance-my-mortgage}{%
  \paragraph{Should I refinance my
  mortgage?}\label{should-i-refinance-my-mortgage}}

  \begin{itemize}
  \tightlist
  \item
    \href{https://www.nytimes3xbfgragh.onion/article/coronavirus-money-unemployment.html?action=click\&pgtype=Article\&state=default\&region=MAIN_CONTENT_3\&context=storylines_faq}{It
    could be a good idea,} because mortgage rates have
    \href{https://www.nytimes3xbfgragh.onion/2020/07/16/business/mortgage-rates-below-3-percent.html?action=click\&pgtype=Article\&state=default\&region=MAIN_CONTENT_3\&context=storylines_faq}{never
    been lower.} Refinancing requests have pushed mortgage applications
    to some of the highest levels since 2008, so be prepared to get in
    line. But defaults are also up, so if you're thinking about buying a
    home, be aware that some lenders have tightened their standards.
  \end{itemize}
\item ~
  \hypertarget{what-is-school-going-to-look-like-in-september}{%
  \paragraph{What is school going to look like in
  September?}\label{what-is-school-going-to-look-like-in-september}}

  \begin{itemize}
  \tightlist
  \item
    It is unlikely that many schools will return to a normal schedule
    this fall, requiring the grind of
    \href{https://www.nytimes3xbfgragh.onion/2020/06/05/us/coronavirus-education-lost-learning.html?action=click\&pgtype=Article\&state=default\&region=MAIN_CONTENT_3\&context=storylines_faq}{online
    learning},
    \href{https://www.nytimes3xbfgragh.onion/2020/05/29/us/coronavirus-child-care-centers.html?action=click\&pgtype=Article\&state=default\&region=MAIN_CONTENT_3\&context=storylines_faq}{makeshift
    child care} and
    \href{https://www.nytimes3xbfgragh.onion/2020/06/03/business/economy/coronavirus-working-women.html?action=click\&pgtype=Article\&state=default\&region=MAIN_CONTENT_3\&context=storylines_faq}{stunted
    workdays} to continue. California's two largest public school
    districts --- Los Angeles and San Diego --- said on July 13, that
    \href{https://www.nytimes3xbfgragh.onion/2020/07/13/us/lausd-san-diego-school-reopening.html?action=click\&pgtype=Article\&state=default\&region=MAIN_CONTENT_3\&context=storylines_faq}{instruction
    will be remote-only in the fall}, citing concerns that surging
    coronavirus infections in their areas pose too dire a risk for
    students and teachers. Together, the two districts enroll some
    825,000 students. They are the largest in the country so far to
    abandon plans for even a partial physical return to classrooms when
    they reopen in August. For other districts, the solution won't be an
    all-or-nothing approach.
    \href{https://bioethics.jhu.edu/research-and-outreach/projects/eschool-initiative/school-policy-tracker/}{Many
    systems}, including the nation's largest, New York City, are
    devising
    \href{https://www.nytimes3xbfgragh.onion/2020/06/26/us/coronavirus-schools-reopen-fall.html?action=click\&pgtype=Article\&state=default\&region=MAIN_CONTENT_3\&context=storylines_faq}{hybrid
    plans} that involve spending some days in classrooms and other days
    online. There's no national policy on this yet, so check with your
    municipal school system regularly to see what is happening in your
    community.
  \end{itemize}
\item ~
  \hypertarget{is-the-coronavirus-airborne}{%
  \paragraph{Is the coronavirus
  airborne?}\label{is-the-coronavirus-airborne}}

  \begin{itemize}
  \tightlist
  \item
    The coronavirus
    \href{https://www.nytimes3xbfgragh.onion/2020/07/04/health/239-experts-with-one-big-claim-the-coronavirus-is-airborne.html?action=click\&pgtype=Article\&state=default\&region=MAIN_CONTENT_3\&context=storylines_faq}{can
    stay aloft for hours in tiny droplets in stagnant air}, infecting
    people as they inhale, mounting scientific evidence suggests. This
    risk is highest in crowded indoor spaces with poor ventilation, and
    may help explain super-spreading events reported in meatpacking
    plants, churches and restaurants.
    \href{https://www.nytimes3xbfgragh.onion/2020/07/06/health/coronavirus-airborne-aerosols.html?action=click\&pgtype=Article\&state=default\&region=MAIN_CONTENT_3\&context=storylines_faq}{It's
    unclear how often the virus is spread} via these tiny droplets, or
    aerosols, compared with larger droplets that are expelled when a
    sick person coughs or sneezes, or transmitted through contact with
    contaminated surfaces, said Linsey Marr, an aerosol expert at
    Virginia Tech. Aerosols are released even when a person without
    symptoms exhales, talks or sings, according to Dr. Marr and more
    than 200 other experts, who
    \href{https://academic.oup.com/cid/article/doi/10.1093/cid/ciaa939/5867798}{have
    outlined the evidence in an open letter to the World Health
    Organization}.
  \end{itemize}
\item ~
  \hypertarget{what-are-the-symptoms-of-coronavirus}{%
  \paragraph{What are the symptoms of
  coronavirus?}\label{what-are-the-symptoms-of-coronavirus}}

  \begin{itemize}
  \tightlist
  \item
    Common symptoms
    \href{https://www.nytimes3xbfgragh.onion/article/symptoms-coronavirus.html?action=click\&pgtype=Article\&state=default\&region=MAIN_CONTENT_3\&context=storylines_faq}{include
    fever, a dry cough, fatigue and difficulty breathing or shortness of
    breath.} Some of these symptoms overlap with those of the flu,
    making detection difficult, but runny noses and stuffy sinuses are
    less common.
    \href{https://www.nytimes3xbfgragh.onion/2020/04/27/health/coronavirus-symptoms-cdc.html?action=click\&pgtype=Article\&state=default\&region=MAIN_CONTENT_3\&context=storylines_faq}{The
    C.D.C. has also} added chills, muscle pain, sore throat, headache
    and a new loss of the sense of taste or smell as symptoms to look
    out for. Most people fall ill five to seven days after exposure, but
    symptoms may appear in as few as two days or as many as 14 days.
  \end{itemize}
\item ~
  \hypertarget{does-asymptomatic-transmission-of-covid-19-happen}{%
  \paragraph{Does asymptomatic transmission of Covid-19
  happen?}\label{does-asymptomatic-transmission-of-covid-19-happen}}

  \begin{itemize}
  \tightlist
  \item
    So far, the evidence seems to show it does. A widely cited
    \href{https://www.nature.com/articles/s41591-020-0869-5}{paper}
    published in April suggests that people are most infectious about
    two days before the onset of coronavirus symptoms and estimated that
    44 percent of new infections were a result of transmission from
    people who were not yet showing symptoms. Recently, a top expert at
    the World Health Organization stated that transmission of the
    coronavirus by people who did not have symptoms was ``very rare,''
    \href{https://www.nytimes3xbfgragh.onion/2020/06/09/world/coronavirus-updates.html?action=click\&pgtype=Article\&state=default\&region=MAIN_CONTENT_3\&context=storylines_faq\#link-1f302e21}{but
    she later walked back that statement.}
  \end{itemize}
\end{itemize}

``There was a lot of community outrage,'' Ms. Field said. ``A lot of
people were blaming them.''

Yet the risk of spread is most pronounced within the cramped camps
themselves. Of 100 laborers tested at a
\href{https://www.nytimes3xbfgragh.onion/2020/06/18/us/florida-coronavirus-immokalee-farmworkers.html}{watermelon
farm in Florida}, 90 were found to have the virus, according to
Florida's governor.

In New Jersey, at Cassaday Farms in Gloucester County, 70 of the 90
workers contracted the virus, according to the owner, George Cassaday.

Mr. Cassaday asked Southern Jersey Family Medical Center to conduct
testing after an older worker became ill and was hospitalized for about
a week. None of the other workers showed severe symptoms, said Mr.
Cassaday, who also contracted the virus; he was tested after he could no
longer smell his favorite flowers, hyacinths.

Most of his employees travel each spring from Mexico on
\href{https://www.uscis.gov/working-united-states/temporary-workers/h-2a-temporary-agricultural-workers}{H-2A}
worker visas, and stay for the harvest of early and late-season crops,
including broccoli, corn, strawberries and squash.

He says his business is as dependent on their health as he is on their
trust.

``I eat with the men. I visit them in Mexico,'' Mr. Cassaday said.
``We're one big family.''

Image

Atlantic Blueberry Company's packaging facility was adding plexiglass
dividers to separate its work stations.Credit...Chang W. Lee/The New
York Times

At least half the nation's farmworkers are believed to be undocumented,
according to Bruce Goldstein, president of Farmworker Justice, a
national advocacy organization focused on labor standards and
occupational safety.

``What we're hearing from all over is that people are too fearful of
being fired or deported to ask for improvement of the health and safety
practices,'' Mr. Goldstein said.

Migrant farmworkers are not included in the categories of foreign
\href{https://www.nytimes3xbfgragh.onion/2020/06/22/us/politics/trump-h1b-work-visas.html}{workers
barred in June by President Trump} from entering the country. But
finding enough people to work the fields has been a problem on farms in
the United States long before the coronavirus.

There has been a fivefold increase in the number of H-2A visas requested
and approved since 2005, climbing to 258,000 last year --- ``one of the
clearest indicators of the scarcity of farm labor,'' according to a
\href{https://www.ers.usda.gov/topics/farm-economy/farm-labor/\#size}{report}
from the United States Department of Agriculture.

On New Jersey's blueberry farms, a labor shortage over the last several
years has led to an expanded use of machinery to harvest the fruit,
which can be damaged in the process and then must be sold frozen, not
fresh.

\href{https://www.cdc.gov/coronavirus/2019-ncov/community/guidance-agricultural-workers.html}{Guidance
issued} by the United States for the safety of the nation's estimated
2.4 million farmworkers is not mandatory. A New Jersey state senator, M.
Teresa Ruiz, has introduced a bill to make the state's recommendations
binding.

``Each state is on their own,'' said Amy Liebman, director of
environmental and occupational health at the Migrant Clinicians Network,
a nonprofit for health care organizations. ``In some cases, each farm is
on their own.''

Sara A. Quandt, a professor and medical anthropologist who teaches
epidemiology and prevention at Wake Forest School of Medicine, has begun
a survey of farmworkers' understanding of the virus and social
distancing.

She said she was frustrated by ``victim blaming'' and suggestions that
the infection rate in migrant communities is somehow linked to poor
hygiene.

``There is inherent racism going on,'' Professor Quandt said, ``that
perhaps their lives are not worth quite as much and perhaps it's their
own darn fault.''

Advertisement

\protect\hyperlink{after-bottom}{Continue reading the main story}

\hypertarget{site-index}{%
\subsection{Site Index}\label{site-index}}

\hypertarget{site-information-navigation}{%
\subsection{Site Information
Navigation}\label{site-information-navigation}}

\begin{itemize}
\tightlist
\item
  \href{https://help.nytimes3xbfgragh.onion/hc/en-us/articles/115014792127-Copyright-notice}{©~2020~The
  New York Times Company}
\end{itemize}

\begin{itemize}
\tightlist
\item
  \href{https://www.nytco.com/}{NYTCo}
\item
  \href{https://help.nytimes3xbfgragh.onion/hc/en-us/articles/115015385887-Contact-Us}{Contact
  Us}
\item
  \href{https://www.nytco.com/careers/}{Work with us}
\item
  \href{https://nytmediakit.com/}{Advertise}
\item
  \href{http://www.tbrandstudio.com/}{T Brand Studio}
\item
  \href{https://www.nytimes3xbfgragh.onion/privacy/cookie-policy\#how-do-i-manage-trackers}{Your
  Ad Choices}
\item
  \href{https://www.nytimes3xbfgragh.onion/privacy}{Privacy}
\item
  \href{https://help.nytimes3xbfgragh.onion/hc/en-us/articles/115014893428-Terms-of-service}{Terms
  of Service}
\item
  \href{https://help.nytimes3xbfgragh.onion/hc/en-us/articles/115014893968-Terms-of-sale}{Terms
  of Sale}
\item
  \href{https://spiderbites.nytimes3xbfgragh.onion}{Site Map}
\item
  \href{https://help.nytimes3xbfgragh.onion/hc/en-us}{Help}
\item
  \href{https://www.nytimes3xbfgragh.onion/subscription?campaignId=37WXW}{Subscriptions}
\end{itemize}
