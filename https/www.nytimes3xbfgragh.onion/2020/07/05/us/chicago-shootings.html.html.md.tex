Sections

SEARCH

\protect\hyperlink{site-content}{Skip to
content}\protect\hyperlink{site-index}{Skip to site index}

\href{https://www.nytimes3xbfgragh.onion/section/us}{U.S.}

\href{https://myaccount.nytimes3xbfgragh.onion/auth/login?response_type=cookie\&client_id=vi}{}

\href{https://www.nytimes3xbfgragh.onion/section/todayspaper}{Today's
Paper}

\href{/section/us}{U.S.}\textbar{}Chicago Gun Violence Spikes and
Increasingly Finds the Youngest Victims

\url{https://nyti.ms/3gyoVqd}

\begin{itemize}
\item
\item
\item
\item
\item
\item
\end{itemize}

\href{https://www.nytimes3xbfgragh.onion/news-event/george-floyd-protests-minneapolis-new-york-los-angeles?action=click\&pgtype=Article\&state=default\&region=TOP_BANNER\&context=storylines_menu}{Race
and America}

\begin{itemize}
\tightlist
\item
  \href{https://www.nytimes3xbfgragh.onion/2020/07/26/us/protests-portland-seattle-trump.html?action=click\&pgtype=Article\&state=default\&region=TOP_BANNER\&context=storylines_menu}{Protesters
  Return to Other Cities}
\item
  \href{https://www.nytimes3xbfgragh.onion/2020/07/24/us/portland-oregon-protests-white-race.html?action=click\&pgtype=Article\&state=default\&region=TOP_BANNER\&context=storylines_menu}{Portland
  at the Center}
\item
  \href{https://www.nytimes3xbfgragh.onion/2020/07/23/podcasts/the-daily/portland-protests.html?action=click\&pgtype=Article\&state=default\&region=TOP_BANNER\&context=storylines_menu}{Podcast:
  Showdown in Portland}
\item
  \href{https://www.nytimes3xbfgragh.onion/interactive/2020/07/16/us/black-lives-matter-protests-louisville-breonna-taylor.html?action=click\&pgtype=Article\&state=default\&region=TOP_BANNER\&context=storylines_menu}{45
  Days in Louisville}
\end{itemize}

Advertisement

\protect\hyperlink{after-top}{Continue reading the main story}

Supported by

\protect\hyperlink{after-sponsor}{Continue reading the main story}

\hypertarget{chicago-gun-violence-spikes-and-increasingly-finds-the-youngest-victims}{%
\section{Chicago Gun Violence Spikes and Increasingly Finds the Youngest
Victims}\label{chicago-gun-violence-spikes-and-increasingly-finds-the-youngest-victims}}

Nine children under the age of 18 have been shot dead in Chicago since
June 20.

\includegraphics{https://static01.graylady3jvrrxbe.onion/images/2020/07/03/us/00chicago-shootings01/merlin_174135867_468e3362-3d7c-45fa-af17-674de7887fe9-articleLarge.jpg?quality=75\&auto=webp\&disable=upscale}

By \href{https://www.nytimes3xbfgragh.onion/by/neil-macfarquhar}{Neil
MacFarquhar} and Robert Chiarito

\begin{itemize}
\item
  Published July 5, 2020Updated July 22, 2020
\item
  \begin{itemize}
  \item
  \item
  \item
  \item
  \item
  \item
  \end{itemize}
\end{itemize}

As Yasmin Miller drove home from a laundromat in Chicago's Englewood
neighborhood last weekend, a gunman in another car peppered her red
Hyundai sedan with bullets, grazing her head and striking her son,
Sincere Gaston, in the chest. Sincere died in his car seat. He was 20
months old.

On June 20, a man fired gunshots through the back of a dark blue S.U.V.,
wounding the 27-year-old man driving and hitting his stepson, Mekhi
James, in the back, killing him. Mekhi was 3.

Two girls, both aged 3, were hospitalized with gunshot wounds in
separate incidents in recent days --- one after her mother thought she
heard fireworks and turned around to see her daughter collapsed on the
ground.

These were just the toddlers.

In all, nine children under 18 have been killed since June 20 as Chicago
reels from another wave of gun violence. The last two were killed on
Saturday evening. A 14-year-old boy was shot to death on Chicago's South
Side. A 7-year-old girl was struck in the forehead by a bullet when
three gunmen opened fire on a July 4 street party on the city's West
Side, the police said.

``The Windy City is becoming the Bloody City,'' said the Rev. Michael L.
Pfleger of Saint Sabina Church, calling it the worst period in the 45
years he has worked on social issues. ``I have never seen the despair,
hopelessness and anger all mixed together at the level it is right
now.''

The violence comes amid a wrenching debate nationwide about policing in
the wake of the death of George Floyd in Minneapolis at the hands of the
police. Those who defend the police say that the violence shows they
need more support, not less, and that it is people living in high-crime
areas who most need effective policing. Critics say the violence shows
how the police are failing the public, how deeply residents distrust
officers and the need for reforms and the transfer of funds to address
underlying problems, including unemployment, mental illness and drug
use.

At least 336 people have been murdered in Chicago through July 2,
according to the Chicago Police Department; because murders typically
increase in the summer, the city is on track to match the 778 deaths in
2016, its deadliest year since the mid-1990s. (New York City, with
almost three times the population, had 176 murders as of June 28.)

Chicago had 658 murders in 2017, 567 in 2018 and 492 in 2019, according
to Chicago police records.

Ahead of the July 4 weekend, Mayor Lori Lightfoot made an appeal to
young men, who she said were responsible for the bulk of the shootings.
``Think about the number of children that have been killed just in the
last two weeks,'' she said at a news conference. ``Families that will
not recover from this hardship. Mothers' hearts that are broken,
fathers' hearts that are destroyed, grandparents who are living in
mourning.''

Chicago is not alone. Before the coronavirus hit, homicides were
escalating nationwide in early 2020, and although the lockdown brought a
pause, they began rising again as the stay-at-home measures were lifted.
A
\href{https://craftmediabucket.s3.amazonaws.com/uploads/COVID-19-Homicide_061520_Final.pdf}{national
study} showed that homicide rates fell in 39 of 64 major cities during
April and began creeping up in May.

The pandemic has added significant stress on the communities that
already suffer the most violence. Impoverished neighborhoods like
Englewood also have some of the highest rates of Covid-19 infections and
deaths. Over all, there have been 53,375 known coronavirus cases in
Chicago and at least 2,631 deaths, according to statistics from the
state.

Unemployment in some of the most affected areas rose to 35 percent from
28 percent during the pandemic, Father Pfleger said. ``That is the
tragedy,'' he said. ``The bad situation in this city got even worse with
the pandemic. It exposed the reality that Black and brown communities
are disproportionately affected.''

``Because this is not one crisis, this is two crises operating at the
same time, this could in fact be worse than what we saw in 2016,'' said
Thomas Abt, a senior fellow at the Council on Criminal Justice and one
author of the nationwide homicide study by Arnold Ventures, a
philanthropy focused on criminal justice.

Distrust of the police is also a contributing factor as many residents
of the hardest-hit neighborhoods feel reluctant to call on law
enforcement, perhaps even more so since the death of Mr. Floyd and the
nationwide protests against police brutality that followed it.

Image

Balloons were released during the memorial for Sincere Gaston in Chicago
last week.Credit...Taylor Glascock for The New York Times

Image

Thomas Gaston, Sincere's father, hugged Father Michael Pfleger before
the vigil last week.Credit...Taylor Glascock for The New York Times

\includegraphics{https://static01.graylady3jvrrxbe.onion/images/2020/07/03/us/00chicago-shootings04/00chicago-shootings04-articleLarge.jpg?quality=75\&auto=webp\&disable=upscale}

People who have lost trust in the police are more prone to settle scores
on their own, experts said. ``The lack of trust, the lack of confidence
in police and the lack of willingness to use police, I think is going to
have a broader effect,'' Mr. Abt said.

The police too are feeling the strain as they try to confront both the
violence in the city and the pandemic. ``All of the people and
organizations that we usually depend on to respond to homicide and
violent crime are overburdened right now,'' Mr. Abt said.

Chicago's new police superintendent, David O. Brown, who took the job in
April, had vowed to keep murders this year below 300. That benchmark has
already fallen.

Mr. Brown called the open-air drug markets on street corners ``the
precursors'' to much of the violence, with the drug sellers employing
teenagers with no criminal history so they will be released if caught.

Asked about how they are addressing the gun violence, he said that the
police are confiscating guns --- 4,629 so far this year, over 10,000
last year. He repeatedly appealed to the public for help, saying that
residents knew something about the perpetrators in most cases.

A low rate in solving murders --- it hovers around 20 percent --- and
the lack of protection for witnesses both play into the continued high
murder rate, criminologists said. Murderers do not expect to get caught
and witnesses feel intimidated, they said.

The Chicago Police Department let its community policing program wither
about two decades ago, said Wesley G. Skogan, of the Institute for
Policy Research at Northwestern University. Now, young police officers
canvassing unfamiliar blocks have found that residents do not open their
doors out of fear of being seen talking to a police officer, he said.

Thomas Ahern, a Police Department spokesman, disputed the notion that
community policing was being neglected. He cited Operation Clean, which
works to spruce up neighborhoods including fixing streetlights,
repairing damaged buildings and removing graffiti.

Many residents think that is not enough, however. The city needs to do
more to protect witnesses, said the Rev. Ira Acree of Greater St. John
Bible Church. ``People want to tell, but they are afraid,'' Mr. Acree
told a community meeting that he organized to discuss the shootings,
adding that people approach him repeatedly about doing the right thing.
They tell him, he said, ``I want to go to heaven, but I do not want to
go this week.''

He called the death of children ``heartbreaking'' for the community.
``There was a time even in the gangs, there was some code of ethics, you
would not bother the kids or the old ladies,'' he said. ``They were off
limits.''The debate over rising violence is also tangled in both local
and national politics.

President Trump weighed in on the killings in late June, sending a
letter addressed to Gov. J.B. Pritzker and Mayor Lightfoot, saying that
the federal government could help revitalize distressed neighborhoods
but ``you must establish law and order.'' The mayor accused the
president of trying to play politics rather than to help.

Kimberly M. Foxx, the prosecutor for Cook County, has been a strong
advocate for reducing the prison population through measures like
release without bail, erasing marijuana convictions and not prosecuting
low-level crimes like shoplifting.

The police union, also at odds with Ms. Lightfoot over her criticism of
some of their actions during the recent unrest, opposes the bail
policies.

Image

The scene where Mekhi James, 3, was killed in a shooting in Chicago in
June.~Credit...John J. Kim/Chicago Tribune, via Associated Press

Chicago's toll has mounted steadily since Memorial Day weekend --- when
85 people were shot and 24 killed --- which usually ushers in summer
violence. During a 24-hour period the next weekend, 18 people were
murdered, the worst day in decades.

Some experts attribute the high numbers of children being killed to
collateral damage from gunmen leaving their fingers on the triggers of
automatic weapons that they have never been trained to shoot.

For example, Amaria Jones, 13, was showing her mother a dance step when
a bullet tore through a window and a television set before striking the
girl in the neck, killing her. The gunman had opened fire from more than
a block away, the police said.

At a memorial for Sincere Gaston, a giant poster bearing the words
``Enough Is Enough'' showed the bright-eyed toddler grasping a
green-topped milk bottle.

His parents, Thomas Gaston, 27, and Ms. Miller, complained that the
police treated them like suspects, even though Mr. Gaston has
participated in an anti-gang program. He was the intended target of the
shooting that killed his son, the police said.

Ms. Miller said that detectives initially prevented her from seeing her
son, demanding that she first divulge information about who might have
carried out the killing. ``Have some compassion for us, it hurts,'' she
said.

John Catanzara, the head of the police union, defended the decision,
saying that investigators needed to collect as many details as possible
while events were still fresh.

On the hot, humid day the memorial was held, about 100 people gathered
under a white tent erected in an empty lot, releasing a flurry of red
and blue balloons in Sincere's honor. ``He lit up the room. Everybody
loved him,'' his mother said. ``I can do nothing without that little
boy. I feel lifeless, I am lifeless.''

Mitch Smith contributed reporting.

Advertisement

\protect\hyperlink{after-bottom}{Continue reading the main story}

\hypertarget{site-index}{%
\subsection{Site Index}\label{site-index}}

\hypertarget{site-information-navigation}{%
\subsection{Site Information
Navigation}\label{site-information-navigation}}

\begin{itemize}
\tightlist
\item
  \href{https://help.nytimes3xbfgragh.onion/hc/en-us/articles/115014792127-Copyright-notice}{©~2020~The
  New York Times Company}
\end{itemize}

\begin{itemize}
\tightlist
\item
  \href{https://www.nytco.com/}{NYTCo}
\item
  \href{https://help.nytimes3xbfgragh.onion/hc/en-us/articles/115015385887-Contact-Us}{Contact
  Us}
\item
  \href{https://www.nytco.com/careers/}{Work with us}
\item
  \href{https://nytmediakit.com/}{Advertise}
\item
  \href{http://www.tbrandstudio.com/}{T Brand Studio}
\item
  \href{https://www.nytimes3xbfgragh.onion/privacy/cookie-policy\#how-do-i-manage-trackers}{Your
  Ad Choices}
\item
  \href{https://www.nytimes3xbfgragh.onion/privacy}{Privacy}
\item
  \href{https://help.nytimes3xbfgragh.onion/hc/en-us/articles/115014893428-Terms-of-service}{Terms
  of Service}
\item
  \href{https://help.nytimes3xbfgragh.onion/hc/en-us/articles/115014893968-Terms-of-sale}{Terms
  of Sale}
\item
  \href{https://spiderbites.nytimes3xbfgragh.onion}{Site Map}
\item
  \href{https://help.nytimes3xbfgragh.onion/hc/en-us}{Help}
\item
  \href{https://www.nytimes3xbfgragh.onion/subscription?campaignId=37WXW}{Subscriptions}
\end{itemize}
