Sections

SEARCH

\protect\hyperlink{site-content}{Skip to
content}\protect\hyperlink{site-index}{Skip to site index}

\href{https://www.nytimes3xbfgragh.onion/section/world/asia}{Asia
Pacific}

\href{https://myaccount.nytimes3xbfgragh.onion/auth/login?response_type=cookie\&client_id=vi}{}

\href{https://www.nytimes3xbfgragh.onion/section/todayspaper}{Today's
Paper}

\href{/section/world/asia}{Asia Pacific}\textbar{}Hong Kong, Changed
Overnight, Navigates Its New Reality

\url{https://nyti.ms/3gryBCH}

\begin{itemize}
\item
\item
\item
\item
\item
\end{itemize}

Advertisement

\protect\hyperlink{after-top}{Continue reading the main story}

Supported by

\protect\hyperlink{after-sponsor}{Continue reading the main story}

\hypertarget{hong-kong-changed-overnight-navigates-its-new-reality}{%
\section{Hong Kong, Changed Overnight, Navigates Its New
Reality}\label{hong-kong-changed-overnight-navigates-its-new-reality}}

In a city where China has made some ideas suddenly dangerous, people are
trying to figure out where the boundaries lie, and how their lives have
changed.

\includegraphics{https://static01.graylady3jvrrxbe.onion/images/2020/07/04/world/00hongkong1/merlin_174186198_fff2e9bb-8398-4bd7-97b3-1056a2af8a3e-articleLarge.jpg?quality=75\&auto=webp\&disable=upscale}

By \href{https://www.nytimes3xbfgragh.onion/by/vivian-wang}{Vivian
Wang}, Elaine Yu and
\href{https://www.nytimes3xbfgragh.onion/by/tiffany-may}{Tiffany May}

\begin{itemize}
\item
  Published July 5, 2020Updated July 13, 2020
\item
  \begin{itemize}
  \item
  \item
  \item
  \item
  \item
  \end{itemize}
\end{itemize}

\href{https://cn.nytimes3xbfgragh.onion/china/20200706/hong-kong-security-law/}{阅读简体中文版}\href{https://cn.nytimes3xbfgragh.onion/china/20200706/hong-kong-security-law/zh-hant/}{閱讀繁體中文版}

HONG KONG --- A barge draped with enormous red banners celebrating
\href{https://www.nytimes3xbfgragh.onion/2020/07/13/world/asia/hong-kong-elections-security.html}{China's
new security law} was sailing across
\href{https://www.nytimes3xbfgragh.onion/2020/07/13/podcasts/the-daily/hong-kong.html}{Hong
Kong's famed Victoria Harbor} only hours after the legislation passed.
The police now hoist a purple sign warning protesters that
\href{https://www.nytimes3xbfgragh.onion/2020/07/01/world/asia/hong-kong-purple-flag.html}{their
chants could be criminal}. Along major roads throughout the city,
neon-colored flags hailing a new era of stability and prosperity stand
erect as soldiers.

In recent days, as China took a victory lap over the law it imposed on
the city Tuesday, the defiant masses who once filled
\href{https://www.nytimes3xbfgragh.onion/2020/07/13/world/asia/hong-kong-elections-security.html}{Hong
Kong's} streets in protest have largely gone quiet. Sticky notes that
had plastered the walls of pro-democracy businesses vanished, taken down
by owners suddenly fearful of the words scribbled on them. Parents
whispered about whether to stop their children from singing a popular
protest song, while activists devised coded ways to express
now-dangerous ideas.

Seemingly overnight, Hong Kong was visibly and viscerally different, its
more than seven million people left to navigate what the law would mean
to their lives. The territory's distinct culture of political activism
and free speech, at times brazenly directed at China's ruling Communist
Party, appeared to be in peril.

For some who had been alarmed by the ferocity of
\href{https://www.nytimes3xbfgragh.onion/news-event/hong-kong-protests}{last
year's unrest}, which at times transformed
\href{https://www.nytimes3xbfgragh.onion/2019/08/04/world/asia/hong-kong-protests.html}{shopping
districts},
\href{https://www.nytimes3xbfgragh.onion/2019/08/24/world/asia/hong-kong-protests.html}{neighborhoods}
and
\href{https://www.nytimes3xbfgragh.onion/2019/11/13/world/asia/hong-kong-protests-students.html}{university
campuses} into
\href{https://www.nytimes3xbfgragh.onion/interactive/2019/11/18/world/asia/hong-kong-protest-universities.html}{smoke-filled
battlefields}, the law brought relief and optimism. For others, who had
hoped the desperate protest campaign would help secure long-cherished
freedoms, it signaled a new era of fear and uncertainty.

``This is home,'' said Ming Tse, sitting in the cafe he manages, which
once loudly supported the protesters. ``But I don't think this place
loves us anymore.''

For months, Mr. Tse's love for his home was advertised at his shop in
the working-class neighborhood of North Point. The oat milk carton at
the cash register sat behind postcards of protest art. A poster
condemned the police shootings of two student demonstrators. Even after
opponents of the movement threatened to vandalize the shop last fall,
the decorations stayed.

\includegraphics{https://static01.graylady3jvrrxbe.onion/images/2020/07/04/world/00hongkong2/merlin_174186150_5363c3bb-4ae6-48e8-b077-4e9d229b9788-articleLarge.jpg?quality=75\&auto=webp\&disable=upscale}

\includegraphics{https://static01.graylady3jvrrxbe.onion/images/2017/01/29/podcasts/the-daily-album-art/the-daily-album-art-articleInline-v2.jpg?quality=75\&auto=webp\&disable=upscale}

\hypertarget{listen-to-the-daily-a-turning-point-for-hong-kong}{%
\subsubsection{Listen to `The Daily': A Turning Point For Hong
Kong}\label{listen-to-the-daily-a-turning-point-for-hong-kong}}

How new legislation has chilled Hong Kong's pro-democracy movement ---
and potentially altered the city forever.

transcript

Back to The Daily

bars

0:00/27:34

-27:34

transcript

\hypertarget{listen-to-the-daily-a-turning-point-for-hong-kong-1}{%
\subsection{Listen to `The Daily': A Turning Point For Hong
Kong}\label{listen-to-the-daily-a-turning-point-for-hong-kong-1}}

\hypertarget{hosted-by-michael-barbaro-produced-by-jessica-cheung-with-help-from-eric-krupke-and-clare-toeniskoetter-and-edited-by-mj-davis-lin-and-lisa-tobin}{%
\subsubsection{Hosted by Michael Barbaro; produced by Jessica Cheung;
with help from Eric Krupke and Clare Toeniskoetter; and edited by M.J.
Davis Lin and Lisa
Tobin}\label{hosted-by-michael-barbaro-produced-by-jessica-cheung-with-help-from-eric-krupke-and-clare-toeniskoetter-and-edited-by-mj-davis-lin-and-lisa-tobin}}

\hypertarget{how-new-legislation-has-chilled-hong-kongs-pro-democracy-movement--and-potentially-altered-the-city-forever}{%
\paragraph{How new legislation has chilled Hong Kong's pro-democracy
movement --- and potentially altered the city
forever.}\label{how-new-legislation-has-chilled-hong-kongs-pro-democracy-movement--and-potentially-altered-the-city-forever}}

\begin{itemize}
\item
  michael barbaro\\
  From The New York Times, I'm Michael Barbaro. This is ``The Daily.''

  Today: A new security law is bringing China's harsh approach to free
  speech to Hong Kong in an attempt to stamp out protests there. My
  colleague, Austin Ramzy, on the fallout.

  It's Monday, July 13.

  Austin, when did you first start to hear about this new security law
  in Hong Kong?
\item
  austin ramzy\\
  I first heard about it in mid-May.

  I was going to work one morning. I was outside the subway station in
  the neighborhood where I live on Hong Kong Island called Wan Chai.
  It's a really crowded part of town. And as I was passing into the
  station, I noticed a petition booth. And they were collecting
  signatures for something called Article 23, which is a security law
  that the Hong Kong government had tried and failed to pass 17 years
  ago.
\item
  michael barbaro\\
  And when you say security law, what do you mean?
\item
  austin ramzy\\
  It was a law that outlawed things like secession and subversion. But
  people worried that it was so broad that it could outlaw all sorts of
  behavior in Hong Kong. And so people took to the streets. And there
  was a mass protest on July 1, 2003. And shortly after that, the
  government backed down and has never picked up the law since.
\item
  michael barbaro\\
  So when you see this idea reemerge of a security law on your way into
  the subway, what are you thinking?
\item
  austin ramzy\\
  It was a sense of disbelief, really. I mean, it's something that's
  sort of part of the political environment that sort of comes up from
  time to time. But everyone sort of says no, it's not possible, that
  there's no way the government can push this through. And certainly not
  now, after a year of the most intense protests Hong Kong has ever
  seen. Things were dying down a little bit with the coronavirus and
  police sort of being more aggressive. It seemed like the government
  sort of had the protesters on the back foot. And so the idea that they
  would do something that would encourage people to come out again in
  mass numbers, it just seemed inconceivable to me.
\item
  michael barbaro\\
  Mhm. OK, so what ends up happening next?
\item
  austin ramzy\\
  So a few weeks later ---
\item
  archived recording\\
  (IN MANDARIN) Now, let's move onto item number 3.
\end{itemize}

austin ramzy

China's Congress meets in Beijing.

\begin{itemize}
\tightlist
\item
  archived recording\\
  {[}SPEAKING MANDARIN{]}
\end{itemize}

austin ramzy

A friend and I were discussing going out for drinks on the night that
the Congress starts. And my friend, who's a former China correspondent,
said, are you sure you won't be busy on the start of the National
People's Congress? And I said, oh, of course. There's nothing that's
going to affect Hong Kong.

\begin{itemize}
\tightlist
\item
  archived recording\\
  (IN MANDARIN) Please vote.
\end{itemize}

austin ramzy

And then it emerged that the National People's Congress planned to pass
a security law for Hong Kong.

\begin{itemize}
\tightlist
\item
  archived recording\\
  (IN MANDARIN) Approved by 2878 votes to 1. 6 abstentions. The law is
  passed. China!
\end{itemize}

austin ramzy

It's a shock to everyone in the city. It really emerges out of nowhere.

michael barbaro

So a security law, it sounds like very much like what you saw a petition
for in your neighborhood a few weeks before.

austin ramzy

Exactly, very much like the security law that the Hong Kong government
had tried to pass for 17 years and never succeeded.

michael barbaro

Austin, how exactly can China do that, pass a law that regulates Hong
Kong, that from what you're saying Hong Kong does not want for itself
--- it's rejected it. Because my sense is that Hong Kong has its own
legislature and retains a fair bit of independence from China.

austin ramzy

That's right. So Hong Kong is a former British colony, returned to
Chinese control in 1997. And under what's called ``one country, two
systems,'' it is supposed to operate with a high degree of autonomy for
50 years after that. And so by China's legislature passing a law like
this in 2020, it's acting way ahead of schedule and doing something that
it's not really supposed to do until 2047.

michael barbaro

So this is not how this is supposed to work, China crafting laws that
directly govern Hong Kong's freedoms?

austin ramzy

No, that's not how it's supposed to work.

michael barbaro

And Austin, what does this law actually do? What does it actually say?

austin ramzy

Well, nobody knows exactly, because at that point, the law has not been
written. It's a brief outline, sort of authorizing the Chinese
government to write this law. And so it targets secession, subversion,
terrorist activities and collusion with foreign powers. But we're all
left to wonder what exactly those words mean, what exactly the Chinese
government has in store for Hong Kong.

michael barbaro

Mm-hmm. And I'm curious, what is the government and the leadership in
Hong Kong saying after Beijing says it's going to be writing this law?
Are they filling in the gaps here? Are they just as confused as you are?

austin ramzy

Well, the first thing they say is that this is very welcome. This is
exactly what Hong Kong needs. This will help with all the problems we've
been facing, the unrest and the violence on the streets. But when
pressed on what exactly the law will say, they have to acknowledge that
they don't know.

\begin{itemize}
\tightlist
\item
  archived recording (carrie lam)\\
  {[}SPEAKING CANTONESE{]}
\end{itemize}

austin ramzy

So at one point, Carrie Lam, the chief executive of Hong Kong, the top
official here, has asked about this.

\begin{itemize}
\item
  archived recording\\
  Why should people take your comments and your minister's comments
  about the national security law seriously if you guys have not read
  details in the clauses?
\item
  archived recording (carrie lam)\\
  Thank you for those questions. Well first, yes, you are right. We have
  not seen the complete details of the proposed legislation. We were
  commenting on what we have seen. OK? There was just published by the
  Xinhua News Agency, which contains actually quite a lot of details.
\end{itemize}

austin ramzy

So even Carrie Lam, someone who is seen as a very pro-Beijing figure,
she is sort of caught like a deer in the headlights and trying to defend
the law, at the same time acknowledging that she doesn't know exactly
what it is.

\begin{itemize}
\tightlist
\item
  archived recording (carrie lam)\\
  So based on that, I have given you my comments and my understanding.
\end{itemize}

michael barbaro

So I have to imagine that the citizens of Hong Kong are deeply skeptical
and worried about this law and not following their leaders' advice to
blindly support it.

austin ramzy

That's right. People are very worried about the law. At the same time,
that lack of clarity paralyzes people, because this law is something
that targets dissent. And people grow very worried that anything they do
or say could be made a crime under this law. And so you begin to see
people deleting social media accounts and becoming very worried about
things they might say that could later come back to haunt them under
this new law.

michael barbaro

That makes me wonder if China understood that the ambiguity of this all
--- authorizing a law but not necessarily explaining what would be in
it, using words like sedition and terrorism but not defining them --- if
that was deliberate?

austin ramzy

Yes, definitely. In fact, there were some officials who make this clear,
that this ambiguity is by design. And it's basically meant to intimidate
people.

\begin{itemize}
\tightlist
\item
  archived recording\\
  Friends from the press, good morning.
\end{itemize}

austin ramzy

And one of them, a man named Zhang Xioaming gives a press conference on
the law. And he gives a stern warning that it's not something you want
to mess with.

\begin{itemize}
\tightlist
\item
  archived recording\\
  If those in Hong Kong who defy and challenge the authority of the
  central government and undermine stability in Hong Kong are allowed to
  have their way, Hong Kong would be the loser. There is no doubt to
  that.
\end{itemize}

austin ramzy

And then he said something really striking to me.

\begin{itemize}
\tightlist
\item
  archived recording\\
  This law will be the Sword of Damocles, hanging over a tiny group of
  criminals who endanger national security, who interfere in Hong Kong
  affairs.
\end{itemize}

austin ramzy

He says that the law is like the Sword of Damocles hanging over Hong
Kong.

michael barbaro

And remind me of that parable.

austin ramzy

The Sword of Damocles is a mythological tale about a man who wants to be
king. And the king agrees to allow him to sit on the throne. But as part
of the agreement, there's a sword that's hanging above him from a single
horse's hair. And the man is so frightened that the sword will fall on
him that he begs to no longer be on the throne.

michael barbaro

Right, because at any moment, that horse hair can break and the sword
will fall. Which is no way to be king.

austin ramzy

Right, the idea is that it's such a miserable experience that no one
would want to tap into them.

michael barbaro

Mm-hmm. So as this parable applies to Hong Kong, if I have this correct,
Damocles is the people of Hong Kong. The king is China. And the sword is
this ambiguous security law that makes everything feel precarious and
has everyone living in a certain amount of fear.

austin ramzy

That's right. There's this sense of fear that in this place that's known
for its free speech, suddenly anything you say could potentially be used
against you. And you could potentially end up in prison. And that is
what's hanging over Hong Kong so precariously.

michael barbaro

We'll be right back.

So Austin, when does it finally become clear precisely what this
national security law actually says and does?

austin ramzy

Well, the language of the national security law only becomes clear when
it's released. Nobody knows exactly when it's going to come out. People
are looking at the government website, waiting, waiting, checking. And
then finally at about 11:00 p.m. the night of June 30, it lands.

michael barbaro

And when it lands --- and I assume you're one of those people checking
the website over and over again --- what does it actually say? What's
the wording?

austin ramzy

Well, it's quite long. It's 66 articles and six sections. And as it
starts out, it describes the law as something that's meant to uphold one
country, two systems. But as I read through it, it becomes clear that
the law undermines that.

Because in these 66 articles, it begins to define the words that have
been hanging over Hong Kong for weeks --- secession, subversion,
terrorism, collusion with foreign powers. All things that are being
criminalized under this new law. So let me just read one of these
definitions --- subversion. Subversion, it says, is, quote, ``A person
who undermines the basic system of the People's Republic of China,
overthrowing the central power of the People's Republic of China or the
body of power in Hong Kong.'' It goes on to define subversion as, quote,
``Seriously interfering in, disrupting or undermining the performance of
duties and functions in accordance with the law by the body of power of
the People's Republic of China or the Hong Kong Special Administrative
Region.''

michael barbaro

Those words --- interfering, disrupting, undermining --- that feels
pretty broad. It feels like a lot of activities could be construed as
falling onto those words.

austin ramzy

That's right. And many of the activities that took place --- the
protests over the past year --- could fall under those words.

michael barbaro

And so once this law goes into effect, and scary words like
``subversion'' are now being defined --- and defined in ways that, if I
lived in Hong Kong, would make me think that any challenge to authority
might suddenly be illegal --- what ends up happening on the streets?

austin ramzy

So the law goes into effect at 11:00 p.m. on June 30. And then an hour
later, it's July 1. And that's astonishing timing, because July 1 is a
national holiday. It's a very important day in Hong Kong. It's the day
that Hong Kong was handed back from Britain to China. It's also become a
day of protest. In fact, the original protest that led to the first
security law being blocked happened on July 1, 2003. And every year
since then, there have been protests on this day.

michael barbaro

Huh. So this law that could make protesting far more dangerous has been
released the night before this day of protest.

austin ramzy

That's right.

michael barbaro

And is that a coincidence?

austin ramzy

No, I don't think it's a coincidence at all. By dropping this at the
eleventh hour, it's intended to create uncertainty. People are waking up
on July 1, deciding whether they want to go protest. They know this law
is in effect. And it creates a great sense of risk for people going out
on the street.

michael barbaro

So on the morning of July 1, what do you do and what do you see?

austin ramzy

So I got up early. I went out to catch a cab by the waterfront. And I
noticed this barge with these huge characters --- 20 feet high, yellow
on red --- that said, celebrate the national security law. And this
barge was towed through Victoria Harbour in the middle of Hong Kong. And
you could see it from either side, these huge characters welcoming the
new security law. And this is something that is a style that is very
much reminiscent of Chinese propaganda. Now it's in the middle of Hong
Kong.

And then I arrived at this neighborhood, Causeway Bay, where every year
the protest march on July 1 usually begins. And right away, I could see
there was a large crowd.

It was sort of hard to tell who was who, because it's a shopping
neighborhood. It's a holiday. There's lots of people out. But it becomes
quickly clear that there are protesters out everywhere on the streets.

\begin{itemize}
\tightlist
\item
  archived recording (protestors)\\
  {[}NON-ENGLISH SPEECH{]}
\end{itemize}

austin ramzy

I can see police who are trying to block off parts of the streets in
Causeway Bay.

\begin{itemize}
\tightlist
\item
  archived recording\\
  {[}CHATTER AND SIRENS{]}
\end{itemize}

austin ramzy

You can hear people shouting at police.

\begin{itemize}
\tightlist
\item
  archived recording (protestors)\\
  {[}NON-ENGLISH SPEECH{]}
\end{itemize}

austin ramzy

And you can see police officers raising a purple banner that tells
people that they are in violation of the new security law.

\begin{itemize}
\tightlist
\item
  archived recording\\
  {[}NON-ENGLISH SPEECH{]}
\end{itemize}

michael barbaro

What is the purple banner saying?

austin ramzy

It's quite lengthy. And it's like a fine print banner. It says, ``This
is a police warning. You are displaying flags or banners, or chanting
slogans, or conducting yourself with an intent, such as secession or
subversion, which may constitute offenses under the Hong Kong national
security law.'' It's basically showing that the police will now be out
policing words, policing language.

michael barbaro

So what happens next?

austin ramzy

Police begin closing in on these protesters.

They pepper spray some people. And they begin making the first arrests
under this new law.

They announce that the first arrest has been made under the national
security law. They say that they stopped and searched a man, and in his
bag they found a Hong Kong independence flag. And he hadn't even
displayed the flag. He had just had it in his bag. And police found it
when they searched him.

michael barbaro

That very much does feel like the thought police, because a flag
unfurled is a thought unexpressed. And it's almost as if this law now
says that even thinking something, not even saying it, can be a
violation of the law.

austin ramzy

That's right. And in Hong Kong, a city that had some of the most robust
protections for free speech in Asia, to suddenly have this happen ---
people are being arrested for words in their possession --- it's a real
shock to people here.

michael barbaro

So how does July 1 come to an end?

austin ramzy

So protests continue throughout the afternoon. By the end of the day,
the police have arrested 370 people, including 10 under the new national
security laws. One of them is a 15-year-old girl, who like that man had
a Hong Kong independence flag that she was waving.

michael barbaro

So basically they arrest a child.

austin ramzy

A child.

michael barbaro

Does it feel to you that this law is accomplishing what it very much
seems its goal is, which is to repress free speech and keep dissent
against China contained in Hong Kong?

austin ramzy

Yes. You can already see the effects of this law. Throughout the city,
there are a number of restaurants and cafes that support the protest
movement that are normally covered in signs and banners and stuff. And
many of them have taken those down. And the walls are now bare. Sources
have contacted me and asked that I delete a messaging history with them.
And now sometimes, when you ask people for comments or interview, people
who would normally be very willing to talk say not right now, that
they're worried about the law and they just can't talk.

michael barbaro

So people actually worry that their communications with you may violate
these laws?

austin ramzy

That's right. That's right. They're worried either that --- not just
that they might say something on the record that could get them in
trouble, but that some sort of private message that we shared some point
in the past could incriminate them. And so they're trying to delete all
of that.

michael barbaro

I'm reminded of what you said at the beginning of our conversation, that
the people of Hong Kong felt that they had time, that they had decades
until this kind of a change was going to come from mainland China. And
so I wonder if it's starting to feel like people are being robbed of a
big chunk of time to be what they have been, and what they think they're
supposed to be allowed to be in Hong Kong.

austin ramzy

Yes. I think in Hong Kong there's always this sense that what makes the
city unique is always under threat, and has always been gradually
disappearing, but that it would have been a slow process. And year by
year, Hong Kong would lose a little bit to the point where by 2047, you
couldn't tell the difference from the rest of the country. But now it
feels like that process has accelerated wildly. And so in a matter of a
couple weeks, we have lost decades.

michael barbaro

Which I guess leads to the inevitable question: Do people feel like it's
time to leave Hong Kong if they don't have decades left?

austin ramzy

Yes, many people have thought about it, discuss it. I hear it from
everyone I know. But, of course, not everyone can just pick up and
leave. And people love this city. People don't want to try to recreate
their life here in Vancouver or New York or London. They want Hong Kong
to be the place that they love. But for many of them, it feels like that
that's rapidly disappearing.

michael barbaro

Austin, thank you very much. And stay safe there.

austin ramzy

Thank you, Michael.

michael barbaro

We'll be right back.

Here's what else you need to know today. On Sunday, Florida reported
more than 15,000 new cases of the coronavirus, marking the highest
single-day total in any state since the start of the pandemic. The surge
was driven by record infection rates in and around Florida's biggest
cities, including Miami, Fort Lauderdale, Orlando and Pensacola. In
Miami-Dade County, at least six hospitals have reached full capacity.
And over the weekend congressional Democrats, including Representative
Adam Schiff on ABC, reacted with fury to President Trump's decision to
commute the prison sentence of his friend and former adviser Roger
Stone.

\begin{itemize}
\item
  archived recording\\
  Stone was convicted in part for false statements he made to your
  committee. What's your response to the president?
\item
  archived recording (adam schiff)\\
  I think anyone who cares about the rule of law in this country is
  nauseated by the fact that the president has commuted the sentence of
  someone who willfully lied to Congress, covered up for the president,
  intimidated witnesses, obstructed the investigation.
\end{itemize}

michael barbaro

Stone, who was scheduled to report to prison within days, had been
convicted of obstructing a congressional investigation into Trump's 2016
campaign and possible ties to Russia.

\begin{itemize}
\tightlist
\item
  archived recording (adam schiff)\\
  He lied to cover up and protect the president. And the president,
  through this commutation, is basically saying, if you lie for me, if
  you cover up for me, if you have my back, then I will make sure that
  you get a get-out-of-jail-free card.
\end{itemize}

michael barbaro

Two Republican senators, Mitt Romney and Pat Toomey, joined Democrats in
condemning the commutation, with Romney calling it an act of, quote,
``unprecedented historic corruption.''

That's it for ``The Daily.'' For the rest of the week we're revisiting
people we met in the early weeks of the pandemic and hearing what's
happened to them since our original episodes first ran. I'm Michael
Barbaro. See you tomorrow.

But on Thursday, Mr. Tse, 34, took everything down.
\href{https://hk.appledaily.com/local/20200702/7WZXFIHG7JRSASWKGI4TUIUMFQ/}{News
reports} said police officers had interrogated owners of restaurants
with similar protest paraphernalia. The security law criminalizes
``subversion'' of the government, a crime that the police say
encompasses speech such as political slogans.

All that remained was a small plastic dinosaur on the counter, wearing a
yellow hard hat. That inexpensive yet tough headgear, worn by protesters
who fought with the police, had become a symbol of their scrappy
fortitude.

``I don't know if they are so sensitive,'' Mr. Tse said. ``It's just a
helmet on a dinosaur.''

He paused, then reconsidered: ``Actually, everything is sensitive.''

That the
\href{https://www.nytimes3xbfgragh.onion/2020/07/01/world/asia/hong-kong-security-law-china.html}{lines
of criminality had been redrawn} became clear on Friday, when the
authorities charged a 24-year-old man with terrorism and inciting
separatism --- the first person to be indicted under the new law. With a
``Liberate Hong Kong'' flag mounted on the back of his motorcycle, the
man careened into a group of police officers on Wednesday, the
anniversary of Hong Kong's return to China from British rule.

Most years, that holiday draws large pro-democracy rallies. But this
time, they were banned. Protests were scattered, and the police swept in
and arrested hundreds. Ten people, including a 15-year-old girl, were
accused of ``inciting subversion,'' a vaguely defined crime under the
new law; some had merely waved flags, bearing slogans that had never
been explicitly outlawed.

A few dozen relatives and social workers waited on Thursday outside a
police station in North Point where more than 100 of those arrested were
being held. Such vigils had become a rite for protesters' loved ones.

Image

Police officers detaining a protester in the Causeway Bay district on
Wednesday. Hundreds were arrested that day, the anniversary of Hong
Kong's return to China from British rule.Credit...Lam Yik Fei for The
New York Times

But this one felt more perilous, with crimes under the security law
punishable by life imprisonment in the most serious cases. A Chinese
official said Wednesday that the law was meant to hang over would-be
troublemakers like the sword of Damocles.

The police collected DNA samples and searched the homes of the 10 people
arrested on suspicion of inciting subversion --- measures that seemed
excessive when applied to people accused only of possessing pamphlets,
said Janet Pang, a lawyer who is helping some of them.

``You're supposed to only use power that is necessary, and that's how
the law should be,'' she said.

Shortly after noon on Thursday, a pro-democracy activist, Tam Tak-chi,
emerged from the station, where he had spent the night after being
detained. Mr. Tam met a young man inside who said he had been arrested
after the police found a banner in his bag reading ``Hong Kong
Independence, the Only Way Out.'' The man wept on his shoulder, Mr. Tam
said.

The Hong Kong government has insisted that free speech is not under
threat. But on Saturday, the city's public library system said that
books by some prominent activists had been removed from circulation
while officials reviewed whether they violated the new law.

The censorship has crept even into private homes.

In June of last year, Katie Lam took her two young sons to a large
rally. Her older son wore a cap that read ``Hong Konger'' and raised a
handmade sign saying, ``Don't shoot us.''

Now Ms. Lam, a data analyst, is anxious about what her sons say at home.
One of them is having a birthday party in two weeks, and Ms. Lam
wondered if she should hide a print displayed on the piano that reads
``Liberate Hong Kong, Revolution of Our Times,'' a slogan that the
government says could be considered subversive.

The boys loved singing ``Glory to Hong Kong,'' the unofficial anthem of
the protest movement. She worries that the neighbors will hear it.

``Even though we all knew it would happen one day,'' she said of China's
intervention, ``it's still painful.''

Image

Supporters of a detained protester held blank signs outside a courthouse
on Friday.Credit...Lam Yik Fei for The New York Times

But in some corners of the city, China's move has been welcome.

The successive blows of the unrest, followed by the coronavirus
pandemic, emptied malls and grounded flights, eviscerating
\href{https://www.nytimes3xbfgragh.onion/2019/11/14/business/hong-kong-protests-recession.html}{Hong
Kong's economy}. The security law,
\href{https://www.reuters.com/article/us-hongkong-protests-poll-exclusive/exclusive-support-dips-for-hong-kong-democracy-protests-as-national-security-law-looms-poll-idUSKBN23W3KO}{however
unpopular}, seemed poised to end the
\href{https://www.nytimes3xbfgragh.onion/2019/10/17/world/asia/hong-kong-protests.html}{monthslong
impasse} over the protests.

It was Hong Kong's prosperity and worldliness that drew Harry He, 33, to
the city from mainland China 10 years ago. He earned master's degrees in
finance and engineering and fell in love with his new home: its
efficient public transportation, its high food-safety standards. He got
married, found work as an insurance agent, bought a home, had a
daughter.

Last year shattered that serenity. Once, while he was eating at a
restaurant with friends, masked protesters smashed a nearby sushi
restaurant owned by a company seen as pro-Beijing, he said. His mainland
clients began avoiding Hong Kong.

Mr. He said he had supported the protesters at first. But he soon grew
convinced that the authorities needed to restore stability, and that the
security law would do so.

``I just don't want to see violence again,'' he said in an interview in
his office tower in Tsim Sha Tsui, a luxury shopping district that was
\href{https://www.scmp.com/news/hong-kong/politics/article/3038385/busy-hong-kong-tourist-district-dealing-destructive}{battered
by street fighting}. ``I just want Hong Kong to be as developed and
prosperous as before.''

Still, even some who embraced stability wondered about its price.

Just as core to Hong Kong's identity as its freewheeling capitalism has
been its proud, even gleeful, outspokenness. Street booths often lined
the city's busiest shopping districts, blasting dueling political
messages. Tiny bookstores crammed into overpriced commercial spaces
hawked volumes that were banned in the mainland.

Image

Demonstrators supporting the security law on Tuesday, hours before it
went into effect. Some carried the Chinese flag.Credit...Lam Yik Fei for
The New York Times

Xu Zhe, a 22-year-old recent college graduate, said the law was needed
to address the ``terrorism'' committed by some protesters. He had been
horrified by a clash in November, when some demonstrators poured
gasoline on a man who had scolded them, then
\href{https://www.nytimes3xbfgragh.onion/2019/11/11/world/hong-kong-protests.html}{set
him ablaze}.

But Mr. Xu also worried that the law could be used to clamp down on
dissent, including speech. Mr. Xu, who grew up on the mainland before
attending university in Hong Kong, had never had a chance to protest at
home. Last year, he attended his first demonstration, a small gathering
against violence.

If Hong Kongers lost the right to protest, he said, ``I would feel
deeply, deeply regretful.''

Few people in the city know the price of protest better than Rowena He,
a historian at the Chinese University of Hong Kong. For more than two
decades, Professor He has studied the 1989 Tiananmen Square crackdown,
when Chinese troops gunned down protesters in Beijing.

Her office is an informal museum of the massacre, with a miniature
replica on her bookshelf of the ``Goddess of Democracy'' statue that the
Tiananmen protesters erected shortly before the killings.

On Wednesday, the day after the security law was enacted, one of
Professor He's students decided to walk around Hong Kong, documenting a
city on the cusp of change. He sent her a photo of a row of Chinese
flags, flapping in the wind. On a sidewalk railing nearby, a banner
supporting a pastor imprisoned on the mainland had been ripped in half.

``You are a real historian,'' Professor He responded.

Image

A changed city's skyline on Friday.Credit...Lam Yik Fei for The New York
Times

Even as old markers of resistance have come down, subtler ones have
surfaced. Some protesters have turned to puns and created new meaning
from well-worn phrases, a tactic long adopted by mainland internet users
to skirt government censorship.

On Wednesday, in one of the city's commercial hubs, someone had
\href{https://twitter.com/USPhongkong/status/1278281464140820485}{spray-painted}
``Arise, ye who refuse to be slaves'' --- the opening line of China's
national anthem.

And one shop, in place of protest slogans,
\href{https://twitter.com/krislc/status/1278729190628458498/photo/1}{hung
up} nearly two dozen posters of propaganda from Mao-era China, including
one that proclaimed: ``Revolution is not a crime, rebellion is
reasonable.''

Bella Huang contributed research.

Advertisement

\protect\hyperlink{after-bottom}{Continue reading the main story}

\hypertarget{site-index}{%
\subsection{Site Index}\label{site-index}}

\hypertarget{site-information-navigation}{%
\subsection{Site Information
Navigation}\label{site-information-navigation}}

\begin{itemize}
\tightlist
\item
  \href{https://help.nytimes3xbfgragh.onion/hc/en-us/articles/115014792127-Copyright-notice}{©~2020~The
  New York Times Company}
\end{itemize}

\begin{itemize}
\tightlist
\item
  \href{https://www.nytco.com/}{NYTCo}
\item
  \href{https://help.nytimes3xbfgragh.onion/hc/en-us/articles/115015385887-Contact-Us}{Contact
  Us}
\item
  \href{https://www.nytco.com/careers/}{Work with us}
\item
  \href{https://nytmediakit.com/}{Advertise}
\item
  \href{http://www.tbrandstudio.com/}{T Brand Studio}
\item
  \href{https://www.nytimes3xbfgragh.onion/privacy/cookie-policy\#how-do-i-manage-trackers}{Your
  Ad Choices}
\item
  \href{https://www.nytimes3xbfgragh.onion/privacy}{Privacy}
\item
  \href{https://help.nytimes3xbfgragh.onion/hc/en-us/articles/115014893428-Terms-of-service}{Terms
  of Service}
\item
  \href{https://help.nytimes3xbfgragh.onion/hc/en-us/articles/115014893968-Terms-of-sale}{Terms
  of Sale}
\item
  \href{https://spiderbites.nytimes3xbfgragh.onion}{Site Map}
\item
  \href{https://help.nytimes3xbfgragh.onion/hc/en-us}{Help}
\item
  \href{https://www.nytimes3xbfgragh.onion/subscription?campaignId=37WXW}{Subscriptions}
\end{itemize}
