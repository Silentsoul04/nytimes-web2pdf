Sections

SEARCH

\protect\hyperlink{site-content}{Skip to
content}\protect\hyperlink{site-index}{Skip to site index}

\href{https://myaccount.nytimes3xbfgragh.onion/auth/login?response_type=cookie\&client_id=vi}{}

\href{https://www.nytimes3xbfgragh.onion/section/todayspaper}{Today's
Paper}

\href{/section/opinion}{Opinion}\textbar{}How a Covid-19 Vaccine Could
Cost Americans Dearly

\url{https://nyti.ms/3gB9uO5}

\begin{itemize}
\item
\item
\item
\item
\item
\item
\end{itemize}

Advertisement

\protect\hyperlink{after-top}{Continue reading the main story}

\href{/section/opinion}{Opinion}

Supported by

\protect\hyperlink{after-sponsor}{Continue reading the main story}

\hypertarget{how-a-covid-19-vaccine-could-cost-americans-dearly}{%
\section{How a Covid-19 Vaccine Could Cost Americans
Dearly}\label{how-a-covid-19-vaccine-could-cost-americans-dearly}}

The United States is the only developed nation unable to balance cost,
efficacy and social good in setting prices.

\href{https://www.nytimes3xbfgragh.onion/by/elisabeth-rosenthal}{\includegraphics{https://static01.graylady3jvrrxbe.onion/images/2019/09/25/opinion/elisabeth-rosenthal/elisabeth-rosenthal-thumbLarge.png}}

By
\href{https://www.nytimes3xbfgragh.onion/by/elisabeth-rosenthal}{Elisabeth
Rosenthal}

Dr. Rosenthal is a contributing opinion writer.

\begin{itemize}
\item
  July 6, 2020
\item
  \begin{itemize}
  \item
  \item
  \item
  \item
  \item
  \item
  \end{itemize}
\end{itemize}

\includegraphics{https://static01.graylady3jvrrxbe.onion/images/2020/07/02/opinion/00rosenthalWeb/merlin_173432817_71b1d0dc-2676-4dc8-9cf6-43fc61ba0c20-articleLarge.jpg?quality=75\&auto=webp\&disable=upscale}

Yes, of course, Americans' health is priceless, and reining in a deadly
virus that has trashed the economy would be invaluable.

But a
\href{https://www.nytimes3xbfgragh.onion/2020/07/09/health/regeneron-monoclonal-antibodies.html}{Covid-19
vaccine} will have an actual price tag. And given the prevailing
business-centric model of American drug pricing, it could well be budget
breaking, perhaps making it unavailable to many.

The last vaccine to quell a global viral scourge was the polio
inoculation, which ended outbreaks that
\href{https://www.cdc.gov/polio/what-is-polio/polio-us.html}{killed}
thousands and paralyzed tens of thousands each year in the United
States. The March of Dimes Foundation covered the drug cost for a free
national vaccination program.

It came in the mid-1950s, before health insurance for outpatient care
was common, before new drugs were protected by multiple patents, before
medical research was regarded as a way to become rich. It was not
patented because it was not considered patentable under the standards at
the time.

Now we are looking for viral deliverance when drug development is one of
the world's most lucrative businesses, ownership of drug patents is
disputed in endless court battles, and monopoly power often lets
manufacturers set any price, no matter how extraordinary. A new cancer
treatment can cost a half-million dollars and old staples like insulin
have risen manifold in price to thousands of dollars annually.

And the American government has no effective way to fight back.

\href{https://www.aarp.org/health/medicare-qa-tool/does-medicare-cover-shingles-shot/}{Recent
vaccines} targeting more limited populations, such as a meningitis B
vaccine for college students and the shingles vaccine for older adults,
have a retail cost of \$300 to \$400 for a full course.

If a Covid-19 vaccine yields a price of, say, \$500 a course,
vaccinating the entire population would bring a company over \$150
billion, almost all of it profit.

Kevin Schulman, a physician-economist at the Stanford Graduate School of
Business, called that amount ``staggering.'' But Katherine Baicker, dean
of the University of Chicago Harris School of Public Policy, said that
from society's perspective ``\$150 billion might not be an unreasonable
sum'' to pay to tame an epidemic that has left millions unemployed and
cost the economy trillions.

Every other developed country has evolved schemes to set or negotiate
prices, while balancing cost, efficacy and social good. The United
States instead has let business calculations drive drug price tags,
forcing us to accept and absorb ever higher costs. That feels
particularly galling for treatments and vaccines against Covid-19, whose
development and production is being subsidized and incentivized with
billions in federal investment.

When AZT, the first effective drug for combating the virus that causes
AIDS, was introduced in 1992, it was priced at
\href{https://www.nytimes3xbfgragh.onion/1987/03/21/us/us-approves-drug-to-prolong-lives-of-aids-patients.html}{up
to \$10,000 a year} or about \$800 a month. It was the most expensive
prescription drug in history, at that time. The price was widely
denounced as
``\href{https://www.nytimes3xbfgragh.onion/1989/08/28/opinion/azt-s-inhuman-cost.html\#:~:text=At\%20\%248\%2C000\%20a\%20year\%20for,program\%20that\%20ends\%20in\%20September.}{inhuman}.''
Today that price gets you some drugs for toenail fungus.

Investors already smell big money for a Covid-19 vaccine.

The market cap of Moderna, a small Boston area company that has
partnered with the National Institutes of Health in the vaccine race,
has \href{https://ycharts.com/companies/MRNA/market_cap}{tripled} since
Feb. 20, to \$23 billion from \$7 billion, turning its chief executive
into an overnight billionaire. While Moderna's vaccine is regarded as a
strong contender, the company has
\href{https://www.wsj.com/articles/inside-moderna-the-covid-vaccine-front-runner-with-no-track-record-and-an-unsparing-ceo-11593615205}{never
brought a successful drug to market}.

Manufacturers have traditionally claimed that only the lure of windfall
profits would encourage them to take the necessary risks, since drug
development is expensive and there's no way of knowing whether they're
putting their money on a horse that will finish first, or scratch.

More recently they have justified high prices by comparing them to the
costs they would prevent. Expensive hepatitis C drugs, they say, avoid
the need for a \$1 million liver transplant. No matter that the
comparison being made is to the highly inflated costs of treating
disease in American hospitals.

Such logic would be disastrous if it were applied to a successful Covid
vaccine. Covid-19 has shut down countless businesses, creating
record-high unemployment. And the medical consequences of severe
Covid-19 mean weeks of highly expensive intensive care.

``Maybe the economic value of the Covid vaccine is a trillion and even
if the expense to the company was a billion. That's 1,000 times return
on investment,'' said Dr. Schulman. ``No economic theory would support
that.''

In 2015, the Senate Finance Committee came up with a simpler explanation
for high drug prices. After reviewing 20,000 pages of company documents,
it found that Gilead had, what the committee's ranking Democratic member
Ron Wyden of Oregon
\href{https://www.finance.senate.gov/ranking-members-news/wyden-grassley-sovaldi-investigation-finds-revenue-driven-pricing-strategy-behind-84-000-hepatitis-drug}{called},
``a calculated scheme for pricing and marketing its hepatitis C drug
based on one primary goal, maximizing revenue.''

In setting prices, drug makers rarely acknowledge the considerable
federal funding and research that has helped develop their products;
they have not offered taxpayer-investors financial payback.

The Biomedical Advanced Research and Development Authority, a federal
agency known as Barda, is
\href{https://www.europeanpharmaceuticalreview.com/news/117327/barda-to-give-moderna-up-to-483-million-for-covid-19-vaccine-development/}{giving}
Moderna up to \$483 million for late-stage development of its vaccine.

The basic science that has allowed the small company to move so rapidly
was
\href{https://www.patientsforaffordabledrugs.org/2020/05/10/covid-blog-moderna/}{developed}
with a huge prior infusion of federal money to develop a treatment for
diseases like Zika.

Francis Collins, the head of the National Institutes of Health, has said
the government has some intellectual property rights. Moderna seems to
\href{https://www.axios.com/moderna-nih-coronavirus-vaccine-ownership-agreements-22051c42-2dee-4b19-938d-099afd71f6a0.html}{dispute
that view}, saying that it is ``not aware of any I.P. that would prevent
us from commercializing'' a Covid-19 vaccine.

Likewise, AstraZeneca, another top competitor, has received a Barda
\href{https://www.hhs.gov/about/news/2020/05/21/trump-administration-accelerates-astrazeneca-covid-19-vaccine-to-be-available-beginning-in-october.html}{promise}
of up to \$1.2 billion for commercializing a product derived from
research at the University of Oxford.

There is no simple, direct mechanism for regulators or legislators to
control pricing. Our laws, in fact, favor business: Medicare is not
allowed to engage in price negotiations for medicines covered by its
part D drug plan. The Food and Drug Administration, which will have to
approve the manufacturer's vaccine for use as `safe and effective,' is
not allowed to consider proposed cost. The panels that recommend
approval of new drugs generally have no idea how they will be priced.

``The idea that we would allow ourselves to be held hostage in an
emergency is mind boggling,'' said David Mitchell, head of
\href{https://www.patientsforaffordabledrugsnow.org/press-releases-and-news-content/2020/6/22/patients-for-affordable-drugs-now-endorses-bipartisan-legislation-to-ensure-reasonable-prices-for-taxpayer-funded-covid-19-drugs}{Patients
for Affordable Drugs}, an advocacy group.

That's why a bipartisan coalition in the House recently
\href{https://www.healthleadersmedia.com/welcome-ad?toURL=/finance/bipartisan-bills-aim-prevent-price-gouging-taxpayer-funded-covid-19-drugs}{proposed}
two new bills to prevent ``price gouging'' for ``taxpayer funded
Covid-19 drugs'' to ensure affordable pricing.

The exact mechanisms for enacting the provisions therein --- such as
requiring manufacturers to reveal their actual development costs ---
remain unclear. The industry has previously protected development data
as a trade secret. The bills would also require ``reasonable pricing
clauses'' be included in agreements between drug companies and agencies
funding their work. They propose waiving exclusive licenses for Covid-19
drugs, allowing competitors to sell the same products so long as they
paid the patent holder royalties.

Other countries, such as Britain, take a more head on approach: a
national body does a cost-benefit analysis regarding the price at which
a new drug is worth being made available to its citizens. Health
authorities then use that information to negotiate with a drug maker on
price and to develop a national reimbursement scheme.

We could too, but would need to consider mechanisms outside of our
current box --- at least for this national emergency.

The federal government could, for example, invoke a never-before-used
power called
``\href{https://www.nytimes3xbfgragh.onion/2019/07/06/opinion/drug-pricing-trump.html}{march-in
rights,}'' through which it can override a patent holder's rights if it
doesn't make its medicines ``available to the public on reasonable
terms.'' (Unfortunately, in already-signed
\href{https://www.axios.com/federal-government-barda-contracts-moderna-regeneron-aaf9fde2-2ee1-46fb-8465-0d573e6af1ed.html}{agreements
with Barda}, some drug makers have explicitly watered down or eliminated
that proviso.)

We could, alternatively, allow Medicare to negotiate drug prices --- a
proposal that has been raised by politicians and beaten back by industry
again and again. We would then need to restrict markup for a Covid-19
vaccine for the private market. Otherwise we'd get the kinds of
\href{https://www.nytimes3xbfgragh.onion/2020/06/16/upshot/coronavirus-test-cost-varies-widely.html}{results}
emerging from the Covid testing industry, where Medicare pays \$100 for
the test but some labs charge insurers over \$2,000.

There is already reason to worry that our deliverance from coronavirus
will cost us plenty. Barda paid AztraZeneca up to \$1.2 billion toward
development, production and delivery of its candidate vaccine, in order
to secure 300 million doses in October. Britain
\href{https://www.genengnews.com/news/astrazeneca-wins-1-2b-from-barda-to-develop-manufacture-covid-19-vac}{paid}
the equivalent of \$80 million to secure 100 million doses in September
--- one-fifth of what the United States government agreed to pay per
dose.

Ms. Baicker, the public policy school dean, thinks public scrutiny will
prevent outrageous pricing. The industry has made various pledges,
trying to balance corporate citizenry, while making eager investors
happy:
\href{https://www.astrazeneca.com/media-centre/articles/2020/astrazeneca-takes-next-steps-towards-broad-and-equitable-access-to-oxford-universitys-potential-covid-19-vaccine.html}{Astra
Zeneca has promised} one billion doses for low- and middle-income
countries. Johnson and Johnson says it would make the Covid-19 vaccine
available on a ``not for profit basis'' at \$10 for ``emergency pandemic
use.''

We've heard such offers before. Pharmaceutical companies routinely
provide coupons to cover patient co-payments for expensive drugs, so
that we don't squawk when they charge our insurance company tens of
thousands for the medicine, driving up premiums year after year. A
naloxone injector to reverse heroin overdoses is given free to some
clinics, but priced at thousands for the rest.

And it won't feel like a bargain if we get free or cheap vaccines during
a pandemic, but pay dearly for annual Covid-19 shots thereafter.

Drug companies deserve a reasonable profit for taking on this urgent
task of creating a Covid-19 vaccine. But we deserve a return, too.

So before these invaluable vaccines hit the market, we should talk about
an actual price. Otherwise we will be stuck paying dearly for shots that
the rest of the world will get for much less.

\emph{The Times is committed to publishing}
\href{https://www.nytimes3xbfgragh.onion/2019/01/31/opinion/letters/letters-to-editor-new-york-times-women.html}{\emph{a
diversity of letters}} \emph{to the editor. We'd like to hear what you
think about this or any of our articles. Here are some}
\href{https://help.nytimes3xbfgragh.onion/hc/en-us/articles/115014925288-How-to-submit-a-letter-to-the-editor}{\emph{tips}}\emph{.
And here's our email:}
\href{mailto:letters@NYTimes.com}{\emph{letters@NYTimes.com}}\emph{.}

\emph{Follow The New York Times Opinion section on}
\href{https://www.facebookcorewwwi.onion/nytopinion}{\emph{Facebook}}\emph{,}
\href{http://twitter.com/NYTOpinion}{\emph{Twitter (@NYTopinion)}}
\emph{and}
\href{https://www.instagram.com/nytopinion/}{\emph{Instagram}}\emph{.}

Advertisement

\protect\hyperlink{after-bottom}{Continue reading the main story}

\hypertarget{site-index}{%
\subsection{Site Index}\label{site-index}}

\hypertarget{site-information-navigation}{%
\subsection{Site Information
Navigation}\label{site-information-navigation}}

\begin{itemize}
\tightlist
\item
  \href{https://help.nytimes3xbfgragh.onion/hc/en-us/articles/115014792127-Copyright-notice}{©~2020~The
  New York Times Company}
\end{itemize}

\begin{itemize}
\tightlist
\item
  \href{https://www.nytco.com/}{NYTCo}
\item
  \href{https://help.nytimes3xbfgragh.onion/hc/en-us/articles/115015385887-Contact-Us}{Contact
  Us}
\item
  \href{https://www.nytco.com/careers/}{Work with us}
\item
  \href{https://nytmediakit.com/}{Advertise}
\item
  \href{http://www.tbrandstudio.com/}{T Brand Studio}
\item
  \href{https://www.nytimes3xbfgragh.onion/privacy/cookie-policy\#how-do-i-manage-trackers}{Your
  Ad Choices}
\item
  \href{https://www.nytimes3xbfgragh.onion/privacy}{Privacy}
\item
  \href{https://help.nytimes3xbfgragh.onion/hc/en-us/articles/115014893428-Terms-of-service}{Terms
  of Service}
\item
  \href{https://help.nytimes3xbfgragh.onion/hc/en-us/articles/115014893968-Terms-of-sale}{Terms
  of Sale}
\item
  \href{https://spiderbites.nytimes3xbfgragh.onion}{Site Map}
\item
  \href{https://help.nytimes3xbfgragh.onion/hc/en-us}{Help}
\item
  \href{https://www.nytimes3xbfgragh.onion/subscription?campaignId=37WXW}{Subscriptions}
\end{itemize}
