Sections

SEARCH

\protect\hyperlink{site-content}{Skip to
content}\protect\hyperlink{site-index}{Skip to site index}

\href{https://myaccount.nytimes3xbfgragh.onion/auth/login?response_type=cookie\&client_id=vi}{}

\href{https://www.nytimes3xbfgragh.onion/section/todayspaper}{Today's
Paper}

\href{/section/opinion}{Opinion}\textbar{}How America Lost the War on
Covid-19

\href{https://nyti.ms/3gyqJzi}{https://nyti.ms/3gyqJzi}

\begin{itemize}
\item
\item
\item
\item
\item
\item
\end{itemize}

Advertisement

\protect\hyperlink{after-top}{Continue reading the main story}

\href{/section/opinion}{Opinion}

Supported by

\protect\hyperlink{after-sponsor}{Continue reading the main story}

\hypertarget{how-america-lost-the-war-on-covid-19}{%
\section{How America Lost the War on
Covid-19}\label{how-america-lost-the-war-on-covid-19}}

It wasn't because of our culture, it was because of our leadership.

\href{https://www.nytimes3xbfgragh.onion/by/paul-krugman}{\includegraphics{https://static01.graylady3jvrrxbe.onion/images/2018/04/02/opinion/paul-krugman/paul-krugman-thumbLarge.png}}

By \href{https://www.nytimes3xbfgragh.onion/by/paul-krugman}{Paul
Krugman}

Opinion Columnist

\begin{itemize}
\item
  July 6, 2020
\item
  \begin{itemize}
  \item
  \item
  \item
  \item
  \item
  \item
  \end{itemize}
\end{itemize}

\includegraphics{https://static01.graylady3jvrrxbe.onion/images/2020/07/06/opinion/06krugman1/merlin_174243468_6644214d-6a6a-4308-a373-8548e3f6fddf-articleLarge.jpg?quality=75\&auto=webp\&disable=upscale}

When did America start losing its war against the coronavirus? How did
we find ourselves international pariahs, not even allowed to travel to
Europe?

I'd suggest that the turning point was way back on April 17, the day
that Donald Trump tweeted ``LIBERATE MINNESOTA,'' followed by
``\href{https://www.detroitnews.com/story/news/politics/2020/04/17/trump-tweets-liberate-michigan-other-states-democratic-governors/5152037002/}{LIBERATE
MICHIGAN}'' and ``LIBERATE VIRGINIA.'' In so doing, he effectively
declared White House support for protesters demanding an end to the
lockdowns governors had instituted to bring Covid-19 under control.

As it happens, the Democratic governors Trump was targeting in those
tweets stood firm. But Republican governors in Arizona, Florida, Texas
and elsewhere soon lifted stay-at-home orders and ended many
restrictions on business operations. They also, following Trump's lead,
refused to require that people wear masks, and Texas and Arizona denied
local governments the right to impose such requirements. They waved away
warnings from health experts that premature and careless reopening could
lead to a new wave of infections.

And the virus came.

The initial outbreak of Covid-19, centered on New York, should have
taught us to be wary. Rising rates of infection can seem like a minor
concern at first, especially if you don't have adequate testing, until
they explode with terrifying speed.

But neither Republican politicians nor the Trump administration was
willing to heed that lesson. By the second week of June new Covid-19
cases were surging in Arizona and clearly on the rise in Texas. Yet the
governors of
\href{https://www.texastribune.org/2020/06/12/texas-coronavirus-reopening-cases/}{both}
\href{https://www.bizjournals.com/phoenix/news/2020/06/11/arizona-keep-reopening-despite-covid-19-surge.html}{states}
dismissed calls for a pause in reopening, insisting that things were
under control.

And on June 16, of course, The Wall Street Journal published an
\href{https://www.wsj.com/articles/there-isnt-a-coronavirus-second-wave-11592327890}{opinion
article} by Vice President Mike Pence declaring that there wasn't and
wouldn't be a coronavirus second wave. Given the Trump administration's
track record, this virtually guaranteed that the wave was about to hit.
And so it was.

Over the past three weeks things have quickly gotten very grim.
Hospitals in Arizona and Texas are in crisis. And, yes, it was premature
reopening that did it, both directly and by sending a signal to
individuals that the risk was past.

But why did America bungle Covid-19 so badly?

There has been a fair bit of commentary to the effect that our failed
pandemic response was deeply rooted in American culture. We are, the
argument goes, too libertarian, too distrustful of government, too
unwilling to accept even slight inconveniences to protect others.

And there's surely something to this. I don't think any other advanced
country (but are we still an advanced country?) has a comparable number
of people who respond with rage when asked to wear a mask in a
supermarket. There definitely isn't any other advanced country where
demonstrators against public health measures would wave guns around and
invade state capitols. And the Republican Party is more or less unique
among major Western political parties in its hostility to science in
general.

But what strikes me, when looking at America's extraordinary pandemic
failure, is how top-down it all was.

Those anti-lockdown demonstrations weren't spontaneous, grass-roots
affairs. Many were organized and coordinated by
\href{https://www.nytimes3xbfgragh.onion/2020/04/21/us/politics/coronavirus-protests-trump.html}{conservative
political activists}, some with close ties to the Trump campaign, and
financed in part by
\href{https://www.washingtonpost.com/politics/inside-the-conservative-networks-backing-anti-quarantine-protests/2020/04/22/da75c81e-83fe-11ea-a3eb-e9fc93160703_story.html}{right-wing
billionaires}.

And the rush to reopen in Sunbelt states was less a response to popular
demand than a case of Republican governors following Trump's lead.

The main driving force behind reopening, as far as I can tell, was the
administration's desire to have big job gains leading into November, so
that it could do what it knew how to do --- boast about economic
success. Actually dealing with the pandemic just wasn't Trump's kind of
thing.

In that case, however, why has Trump refused to wear a face mask or
encourage others to do so? After all, wider use of masks would be one
way to limit infections without shutting down the economy.

Well, Trump's vanity --- his belief that wearing a mask would make him
look silly, or mess up his makeup, or something --- has surely played a
role. But it's also true that masks remind people that we haven't
controlled the coronavirus --- and Trump wants people to forget that
awkward fact.

The irony is that Trump's willingness to trade deaths for jobs and
political gain has backfired.

Reopening did lead to large job increases in May and June, as around a
\href{https://fred.stlouisfed.org/series/PAYEMS}{third} of the workers
laid off as a result of the pandemic were rehired. But Trump's
\href{https://projects.fivethirtyeight.com/trump-approval-ratings/?ex_cid=rrpromo}{job
approval} and
\href{https://projects.economist.com/us-2020-forecast/president}{electoral
prospects} just kept sliding.

And even in purely economic terms the rush to reopen is probably
failing. The last official employment number was a snapshot from the
second week of June; a variety of
\href{https://twitter.com/SaraEisen/status/1279897244494450694}{short}-\href{https://twitter.com/paulkrugman/status/1279885912059641857}{term}
\href{https://www.cnbc.com/2020/06/26/the-employment-recovery-may-be-slowing-as-virus-cases-climb.html}{indicators}
suggest that growth slowed or even went into reverse soon afterward,
especially in states where Covid-19 cases are spiking.

In any case, the point is that America's defeat at the hands of the
coronavirus didn't happen because victory was impossible. Nor was it
because we as a nation were incapable of responding. No, we lost because
Trump and those around him decided that it was in their political
interests to let the virus run wild.

\emph{The Times is committed to publishing}
\href{https://www.nytimes3xbfgragh.onion/2019/01/31/opinion/letters/letters-to-editor-new-york-times-women.html}{\emph{a
diversity of letters}} \emph{to the editor. We'd like to hear what you
think about this or any of our articles. Here are some}
\href{https://help.nytimes3xbfgragh.onion/hc/en-us/articles/115014925288-How-to-submit-a-letter-to-the-editor}{\emph{tips}}\emph{.
And here's our email:}
\href{mailto:letters@NYTimes.com}{\emph{letters@NYTimes.com}}\emph{.}

\emph{Follow The New York Times Opinion section on}
\href{https://www.facebookcorewwwi.onion/nytopinion}{\emph{Facebook}}\emph{,}
\href{http://twitter.com/NYTOpinion}{\emph{Twitter (@NYTopinion)}}
\emph{and}
\href{https://www.instagram.com/nytopinion/}{\emph{Instagram}}\emph{.}

Advertisement

\protect\hyperlink{after-bottom}{Continue reading the main story}

\hypertarget{site-index}{%
\subsection{Site Index}\label{site-index}}

\hypertarget{site-information-navigation}{%
\subsection{Site Information
Navigation}\label{site-information-navigation}}

\begin{itemize}
\tightlist
\item
  \href{https://help.nytimes3xbfgragh.onion/hc/en-us/articles/115014792127-Copyright-notice}{©~2020~The
  New York Times Company}
\end{itemize}

\begin{itemize}
\tightlist
\item
  \href{https://www.nytco.com/}{NYTCo}
\item
  \href{https://help.nytimes3xbfgragh.onion/hc/en-us/articles/115015385887-Contact-Us}{Contact
  Us}
\item
  \href{https://www.nytco.com/careers/}{Work with us}
\item
  \href{https://nytmediakit.com/}{Advertise}
\item
  \href{http://www.tbrandstudio.com/}{T Brand Studio}
\item
  \href{https://www.nytimes3xbfgragh.onion/privacy/cookie-policy\#how-do-i-manage-trackers}{Your
  Ad Choices}
\item
  \href{https://www.nytimes3xbfgragh.onion/privacy}{Privacy}
\item
  \href{https://help.nytimes3xbfgragh.onion/hc/en-us/articles/115014893428-Terms-of-service}{Terms
  of Service}
\item
  \href{https://help.nytimes3xbfgragh.onion/hc/en-us/articles/115014893968-Terms-of-sale}{Terms
  of Sale}
\item
  \href{https://spiderbites.nytimes3xbfgragh.onion}{Site Map}
\item
  \href{https://help.nytimes3xbfgragh.onion/hc/en-us}{Help}
\item
  \href{https://www.nytimes3xbfgragh.onion/subscription?campaignId=37WXW}{Subscriptions}
\end{itemize}
