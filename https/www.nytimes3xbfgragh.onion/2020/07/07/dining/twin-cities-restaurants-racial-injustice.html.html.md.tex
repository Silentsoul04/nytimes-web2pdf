Sections

SEARCH

\protect\hyperlink{site-content}{Skip to
content}\protect\hyperlink{site-index}{Skip to site index}

\href{https://www.nytimes3xbfgragh.onion/section/food}{Food}

\href{https://myaccount.nytimes3xbfgragh.onion/auth/login?response_type=cookie\&client_id=vi}{}

\href{https://www.nytimes3xbfgragh.onion/section/todayspaper}{Today's
Paper}

\href{/section/food}{Food}\textbar{}In the Twin Cities' Robust Dining
Scene, Few Black-Owned Businesses

\url{https://nyti.ms/31ReXvM}

\begin{itemize}
\item
\item
\item
\item
\item
\item
\end{itemize}

\href{https://www.nytimes3xbfgragh.onion/news-event/george-floyd-protests-minneapolis-new-york-los-angeles?action=click\&pgtype=Article\&state=default\&region=TOP_BANNER\&context=storylines_menu}{Race
and America}

\begin{itemize}
\tightlist
\item
  \href{https://www.nytimes3xbfgragh.onion/2020/07/26/us/protests-portland-seattle-trump.html?action=click\&pgtype=Article\&state=default\&region=TOP_BANNER\&context=storylines_menu}{Protesters
  Return to Other Cities}
\item
  \href{https://www.nytimes3xbfgragh.onion/2020/07/24/us/portland-oregon-protests-white-race.html?action=click\&pgtype=Article\&state=default\&region=TOP_BANNER\&context=storylines_menu}{Portland
  at the Center}
\item
  \href{https://www.nytimes3xbfgragh.onion/2020/07/23/podcasts/the-daily/portland-protests.html?action=click\&pgtype=Article\&state=default\&region=TOP_BANNER\&context=storylines_menu}{Podcast:
  Showdown in Portland}
\item
  \href{https://www.nytimes3xbfgragh.onion/interactive/2020/07/16/us/black-lives-matter-protests-louisville-breonna-taylor.html?action=click\&pgtype=Article\&state=default\&region=TOP_BANNER\&context=storylines_menu}{45
  Days in Louisville}
\end{itemize}

Advertisement

\protect\hyperlink{after-top}{Continue reading the main story}

Supported by

\protect\hyperlink{after-sponsor}{Continue reading the main story}

\hypertarget{in-the-twin-cities-robust-dining-scene-few-black-owned-businesses}{%
\section{In the Twin Cities' Robust Dining Scene, Few Black-Owned
Businesses}\label{in-the-twin-cities-robust-dining-scene-few-black-owned-businesses}}

In the wake of George Floyd's killing, chefs and restaurateurs are
looking for ways to fix the racial imbalance.

\includegraphics{https://static01.graylady3jvrrxbe.onion/images/2020/07/08/dining/07minn-restaurants1/merlin_174286434_49081dbf-2c44-4881-9676-cc71790c55c0-articleLarge.jpg?quality=75\&auto=webp\&disable=upscale}

By Mecca Bos

\begin{itemize}
\item
  Published July 7, 2020Updated July 21, 2020
\item
  \begin{itemize}
  \item
  \item
  \item
  \item
  \item
  \item
  \end{itemize}
\end{itemize}

MINNEAPOLIS --- Before the coronavirus decimated the restaurant
industry, Minneapolis and St. Paul were being hailed around the country
for their up-and-coming culinary scene.

\href{https://www.nytimes3xbfgragh.onion/2019/09/10/dining/ann-kim-chef-minneapolis.html}{Ann
Kim}, the owner of three acclaimed restaurants including the
Korean-influenced pizzeria Young Joni, won a James Beard award last year
for Best Chef Midwest. Lauded chefs like Tim McKee, Alex Roberts, Doug
Flicker and Gavin Kaysen have kept the region on the lips and must-visit
lists of gastro-tourists. Minneapolis has large Southeast Asian and East
African neighborhoods with mom-and-pop restaurants that rival any of
their kind in the nation.

But Minnesota has a marked scarcity of food businesses owned or operated
by African-Americans. Of the more than half-million businesses that the
2012 census reported in the state, about 20,000 were Black-owned.

There are so few African-American food and beverage establishments in
the Twin Cities that they are nearly impossible to find if you don't
know where to look --- even on the Northside of Minneapolis, where more
than half the residents are Black (in a city where less than 20 percent
of the population is African-American).

Since the killing of George Floyd and amid continued protests against
racial injustice, some restaurant owners, chefs and others have begun
talking about ways to change that imbalance.

``I am a Black restaurateur and I am changing it by existing,'' said
Jared Brewington, 41, who has opened two restaurants. One of them, Funky
Grits, at East 38th Street and Chicago Avenue, was directly across the
street from the site of Mr. Floyd's fatal encounter with the police,
before closing last October.

The other, \href{https://www.thightimesbirdhouse.com/}{Thigh Times
Birdhouse}, in the city's gentrified North Loop neighborhood, closed
during the pandemic. Mr. Brewington hopes to reopen in another location,
and he has shifted his focus to plans for a new bar and restaurant,
Cologne House Cafe, in rural Cologne, Minn., about 36 miles from
Minneapolis., where he lives with his wife, Jenn, and their young
daughter.

He and his business partner, Ben Brickweg, are working on a
crowdsourcing platform to help L.G.B.T. people and people of color find
capital to start businesses. He hopes it will inspire white people to
invest in those projects. ``Call it a guilt bucket, I don't care,'' Mr.
Brewington said.

Dawn Drouillard, 49, and Eden Fitzgerald, 46, are the owners of
\href{http://www.fabulouscatering.com/}{Fabulous Catering}, known for
their work for high-end clients and galas. The pandemic has so stymied
their business that they plan to shut down.

But rather than quickly sell their building to the ``white development
people'' who they say are already showing interest, Ms. Drouillard said
they hope to find a way for people of color to acquire it.

``There are so many glaring inequities in our city,'' said Ms.
Drouillard, who is partly of Anishinaabe heritage, and a Grand Portage
tribal descendant. ``We have a small, little piece of the pie to give.
The only asset we have is our building.''

\includegraphics{https://static01.graylady3jvrrxbe.onion/images/2020/07/08/dining/07minn-restaurants3/merlin_174286407_e5180d0f-aa91-4639-a136-bf0dd32c926d-articleLarge.jpg?quality=75\&auto=webp\&disable=upscale}

To that end, the partners have turned their facilities --- just over a
mile from the Third Precinct police station that went up in flames in
the days after Mr. Floyd's death --- over to a food-relief operation
called F12 People's Kitchen, in which volunteers serve about 1,200 meals
a week to homeless people living in a neighborhood encampment. The group
is trying to secure funding --- through grass-roots efforts on social
media, as well as via
\href{https://www.google.com/search?q=nexus+community+partners\&oq=Nexus+Community+Partners\&aqs=chrome.0.0j46j0l2.343j0j9\&sourceid=chrome\&ie=UTF-8}{Nexus
Community Partners}, an organization that helps community wealth
building --- and hopes ultimately to buy the Fabulous Catering building.

The collective envisions running it as a community kitchen that small
food businesses can use, and where community members can simply get a
hot meal if they need one. Gina Peña, 26, who identifies as ``a young,
queer, Black person who loves to cook,'' is currently working as the
group's lead cook.

Eddie Wu, 41, who with his wife, Eve Wu, 40, owns the celebrated
breakfast-and-lunch diner \href{http://www.cookstp.com/}{Cook St. Paul},
said they are talking about not reopening whenever the governor gives
the green light, and instead turning the diner over to people of color
to run.

``Actions speak louder than words,'' said Mr. Wu, a white man who took
his Korean wife's surname. ``I can use what privilege I have. Enough
with white people.''

Image

Eve and Eddie Wu are planning to offer six-month residencies to chefs of
color at their restaurant, Cook St. Paul.Credit...Jenn Ackerman for The
New York Times

Mr. Wu has never been content to quietly run a restaurant. When Philando
Castile was
\href{https://www.nytimes3xbfgragh.onion/2017/06/21/us/video-police-shooting-philando-castile-trial.html}{fatally
shot} by a police officer in St. Paul in 2016, Mr. Wu placed signs
reading ``Black Lives Matter'' and ``Philando Castile Matters'' on his
marquee, drawing a backlash from some white customers.

Before Mr. Wu bought the diner in 2013, it housed Serlin's Cafe, a
67-year old, white-owned restaurant slinging pancakes and hash browns.
Under Mr. Wu's ownership, the diner changed much the same way as its
East Side neighborhood, which has attracted a large Southeast Asian
population, serving dishes like bibimbap alongside the basted eggs.

Mr. Wu also offered the space to rising chefs and restaurateurs for
pop-ups before pop-ups became commonplace in Minnesota, charging them
little more than the night's operating expenses. Now, the Wus want to
extend that model, offering chefs of color, especially Black chefs,
residencies of six months or longer, and perhaps living quarters
upstairs.

Mr. Wu said he wants to see exceptional chefs of color get some of the
recognition that now goes to ``the boring white dudes that we've got a
never-ending supply of here in Minnesota --- which includes me.''

Allena Dancer, a 44-year-old Black woman who ran two one-night pop-ups
at Cook St. Paul, is in talks with the Wus about taking on the first
residency this summer. She said her project, called the Peach Eatery,
would serve ``soul comfort food,'' like her peach cobbler and her
husband's Memphis-inspired smoked meat.

``For me, this means getting my name out there without worrying about
all the overhead,'' Ms. Dancer said. ``People know Cook. This will give
me a name to show people that there is good soul food, and it's
consistent, and it's here to stay. My daughter is 3, and she already
wants a kitchen. I want to leave a legacy.''

\emph{Follow} \href{https://twitter.com/nytfood}{\emph{NYT Food on
Twitter}} \emph{and}
\href{https://www.instagram.com/nytcooking/}{\emph{NYT Cooking on
Instagram}}\emph{,}
\href{https://www.facebookcorewwwi.onion/nytcooking/}{\emph{Facebook}}\emph{,}
\href{https://www.youtube.com/nytcooking}{\emph{YouTube}} \emph{and}
\href{https://www.pinterest.com/nytcooking/}{\emph{Pinterest}}\emph{.}
\href{https://www.nytimes3xbfgragh.onion/newsletters/cooking}{\emph{Get
regular updates from NYT Cooking, with recipe suggestions, cooking tips
and shopping advice}}\emph{.}

Advertisement

\protect\hyperlink{after-bottom}{Continue reading the main story}

\hypertarget{site-index}{%
\subsection{Site Index}\label{site-index}}

\hypertarget{site-information-navigation}{%
\subsection{Site Information
Navigation}\label{site-information-navigation}}

\begin{itemize}
\tightlist
\item
  \href{https://help.nytimes3xbfgragh.onion/hc/en-us/articles/115014792127-Copyright-notice}{©~2020~The
  New York Times Company}
\end{itemize}

\begin{itemize}
\tightlist
\item
  \href{https://www.nytco.com/}{NYTCo}
\item
  \href{https://help.nytimes3xbfgragh.onion/hc/en-us/articles/115015385887-Contact-Us}{Contact
  Us}
\item
  \href{https://www.nytco.com/careers/}{Work with us}
\item
  \href{https://nytmediakit.com/}{Advertise}
\item
  \href{http://www.tbrandstudio.com/}{T Brand Studio}
\item
  \href{https://www.nytimes3xbfgragh.onion/privacy/cookie-policy\#how-do-i-manage-trackers}{Your
  Ad Choices}
\item
  \href{https://www.nytimes3xbfgragh.onion/privacy}{Privacy}
\item
  \href{https://help.nytimes3xbfgragh.onion/hc/en-us/articles/115014893428-Terms-of-service}{Terms
  of Service}
\item
  \href{https://help.nytimes3xbfgragh.onion/hc/en-us/articles/115014893968-Terms-of-sale}{Terms
  of Sale}
\item
  \href{https://spiderbites.nytimes3xbfgragh.onion}{Site Map}
\item
  \href{https://help.nytimes3xbfgragh.onion/hc/en-us}{Help}
\item
  \href{https://www.nytimes3xbfgragh.onion/subscription?campaignId=37WXW}{Subscriptions}
\end{itemize}
