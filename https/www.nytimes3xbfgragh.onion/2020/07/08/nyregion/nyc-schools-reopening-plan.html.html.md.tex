Sections

SEARCH

\protect\hyperlink{site-content}{Skip to
content}\protect\hyperlink{site-index}{Skip to site index}

\href{https://www.nytimes3xbfgragh.onion/section/nyregion}{New York}

\href{https://myaccount.nytimes3xbfgragh.onion/auth/login?response_type=cookie\&client_id=vi}{}

\href{https://www.nytimes3xbfgragh.onion/section/todayspaper}{Today's
Paper}

\href{/section/nyregion}{New York}\textbar{}N.Y.C. Schools, Nation's
Largest District, Will Not Fully Reopen in Fall

\url{https://nyti.ms/3fnOU3z}

\begin{itemize}
\item
\item
\item
\item
\item
\item
\end{itemize}

\href{https://www.nytimes3xbfgragh.onion/news-event/coronavirus?action=click\&pgtype=Article\&state=default\&region=TOP_BANNER\&context=storylines_menu}{The
Coronavirus Outbreak}

\begin{itemize}
\tightlist
\item
  live\href{https://www.nytimes3xbfgragh.onion/2020/08/04/world/coronavirus-covid-19.html?action=click\&pgtype=Article\&state=default\&region=TOP_BANNER\&context=storylines_menu}{Latest
  Updates}
\item
  \href{https://www.nytimes3xbfgragh.onion/interactive/2020/us/coronavirus-us-cases.html?action=click\&pgtype=Article\&state=default\&region=TOP_BANNER\&context=storylines_menu}{Maps
  and Cases}
\item
  \href{https://www.nytimes3xbfgragh.onion/interactive/2020/science/coronavirus-vaccine-tracker.html?action=click\&pgtype=Article\&state=default\&region=TOP_BANNER\&context=storylines_menu}{Vaccine
  Tracker}
\item
  \href{https://www.nytimes3xbfgragh.onion/2020/08/02/us/covid-college-reopening.html?action=click\&pgtype=Article\&state=default\&region=TOP_BANNER\&context=storylines_menu}{College
  Reopening}
\item
  \href{https://www.nytimes3xbfgragh.onion/live/2020/08/03/business/stock-market-today-coronavirus?action=click\&pgtype=Article\&state=default\&region=TOP_BANNER\&context=storylines_menu}{Economy}
\end{itemize}

Advertisement

\protect\hyperlink{after-top}{Continue reading the main story}

Supported by

\protect\hyperlink{after-sponsor}{Continue reading the main story}

\hypertarget{nyc-schools-nations-largest-district-will-not-fully-reopen-in-fall}{%
\section{N.Y.C. Schools, Nation's Largest District, Will Not Fully
Reopen in
Fall}\label{nyc-schools-nations-largest-district-will-not-fully-reopen-in-fall}}

Classroom attendance in September will be limited to only one to three
days a week in an effort to continue to curb the outbreak, the mayor
said.

\includegraphics{https://static01.graylady3jvrrxbe.onion/images/2020/07/08/nyregion/08NYVIRUS-SCHOOLS-REOPENING1/08NYVIRUS-SCHOOLS-REOPENING1-videoSixteenByNine3000.jpg}

\href{https://www.nytimes3xbfgragh.onion/by/eliza-shapiro}{\includegraphics{https://static01.graylady3jvrrxbe.onion/images/2018/12/28/multimedia/author-eliza-shapiro/author-eliza-shapiro-thumbLarge.png}}

By \href{https://www.nytimes3xbfgragh.onion/by/eliza-shapiro}{Eliza
Shapiro}

\begin{itemize}
\item
  Published July 8, 2020Updated July 17, 2020
\item
  \begin{itemize}
  \item
  \item
  \item
  \item
  \item
  \item
  \end{itemize}
\end{itemize}

About four months after 1.1 million New York City children
\href{https://www.nytimes3xbfgragh.onion/2020/03/15/nyregion/nyc-schools-closed.html}{were
forced into online learning}, Mayor Bill de Blasio announced on
Wednesday that public schools would still not fully
\href{https://www.nytimes3xbfgragh.onion/article/new-york-phase-reopening.html}{reopen}
in September, saying that classroom attendance would instead be limited
to only one to three days a week in an effort to continue to curb the
coronavirus outbreak.

The mayor's release of his plan for the system, by far the nation's
largest,
\href{https://www.nytimes3xbfgragh.onion/2020/07/06/nyregion/nyc-school-reopening-plan.html}{capped
weeks of intense debate among elected officials, educators and public
health experts} over how to bring children back safely to 1,800 public
schools.

The decision to opt for only a
\href{https://www.nytimes3xbfgragh.onion/2020/07/17/nyregion/phase-4-nyc-reopen.html}{partial
reopening}, which is most likely the only way to accommodate students in
school buildings while maintaining social distancing, may hinder
hundreds of thousands of parents from returning to their pre-pandemic
work lives, undermining the recovery of the
\href{https://www.nytimes3xbfgragh.onion/2020/07/07/nyregion/nyc-unemployment.html}{sputtering
local economy}.

Still, the staggered schedules in
\href{https://www.nytimes3xbfgragh.onion/article/new-york-phase-reopening.html}{New
York City} schools for September reflect a
\href{https://www.nytimes3xbfgragh.onion/2020/06/26/us/coronavirus-schools-reopen-fall.html}{growing
trend among school systems, universities and colleges around the
country}, which are all trying to find ways of balancing the urgent need
to bring students back to classrooms and campuses while also reducing
density to prevent the spread of the virus.

``Everyone is looking to the public school system to indicate the bigger
direction of New York City,'' Mr. de Blasio said Wednesday.

Under the mayor's plan, there will probably be no more than a dozen
people in a classroom at a time, including teachers and aides, a stark
change from typical class size in New York City schools, which can hover
around 30 children.

Educators widely consider online learning to be a poor substitute for
the classroom, especially for younger children and
\href{https://www.nytimes3xbfgragh.onion/2020/04/16/nyregion/special-education-coronavirus-nyc.html}{those
with special needs}. The shift has also created enormous challenges for
parents who have struggled helping their children learn even as they
have had to maintain jobs from home or, if they are essential workers,
had to scramble for child care.

Still, like New York City's,
\href{https://www.nytimes3xbfgragh.onion/2020/06/26/us/coronavirus-schools-reopen-fall.html}{many
school districts around the country are planning on not reopening
fully}, and instead will use a mix of in-person and remote learning
indefinitely.

President Trump threatened on Wednesday
\href{https://www.nytimes3xbfgragh.onion/2020/07/08/us/politics/trump-schools-reopening.html}{to
cut off federal funding to school districts that do not reopen in person
this fall}. On Tuesday he said that the social, psychological and
educational costs of keeping children at home would be worse than the
virus itself. Education policy is largely controlled by state and local
officials, so Mr. Trump does not have authority over whether systems
reopen.

The details of
\href{https://www.nytimes3xbfgragh.onion/2020/07/16/upshot/coronavirus-school-reopening-private-public-gap.html}{reopening}
will vary widely between districts depending on the virus' spread, which
is why a return to school may look very different in New York, where
transmission is currently low, than in Phoenix, where cases are
increasing.

\hypertarget{latest-updates-global-coronavirus-outbreak}{%
\section{\texorpdfstring{\href{https://www.nytimes3xbfgragh.onion/2020/08/04/world/coronavirus-covid-19.html?action=click\&pgtype=Article\&state=default\&region=MAIN_CONTENT_1\&context=storylines_live_updates}{Latest
Updates: Global Coronavirus
Outbreak}}{Latest Updates: Global Coronavirus Outbreak}}\label{latest-updates-global-coronavirus-outbreak}}

Updated 2020-08-04T10:03:05.885Z

\begin{itemize}
\tightlist
\item
  \href{https://www.nytimes3xbfgragh.onion/2020/08/04/world/coronavirus-covid-19.html?action=click\&pgtype=Article\&state=default\&region=MAIN_CONTENT_1\&context=storylines_live_updates\#link-6b644638}{`Long
  days, long nights': Washington prepares for a prolonged fight over
  virus relief.}
\item
  \href{https://www.nytimes3xbfgragh.onion/2020/08/04/world/coronavirus-covid-19.html?action=click\&pgtype=Article\&state=default\&region=MAIN_CONTENT_1\&context=storylines_live_updates\#link-7af9fca0}{Israel's
  rocky reopening of its schools may be a lesson for the U.S.}
\item
  \href{https://www.nytimes3xbfgragh.onion/2020/08/04/world/coronavirus-covid-19.html?action=click\&pgtype=Article\&state=default\&region=MAIN_CONTENT_1\&context=storylines_live_updates\#link-33bf9168}{Hurricane
  Isaias arrives in North Carolina as officials along the East Coast
  scramble.}
\end{itemize}

\href{https://www.nytimes3xbfgragh.onion/2020/08/04/world/coronavirus-covid-19.html?action=click\&pgtype=Article\&state=default\&region=MAIN_CONTENT_1\&context=storylines_live_updates}{See
more updates}

More live coverage:
\href{https://www.nytimes3xbfgragh.onion/live/2020/08/03/business/stock-market-today-coronavirus?action=click\&pgtype=Article\&state=default\&region=MAIN_CONTENT_1\&context=storylines_live_updates}{Markets}

In New York,
\href{https://www.nytimes3xbfgragh.onion/2020/07/14/arts/design/cuomo-covid-poster-new-york.html}{Gov.
Andrew M. Cuomo} has authority over when schools across the state,
including in New York City, can reopen. Mr. Cuomo and Mr. de Blasio have
\href{https://www.nytimes3xbfgragh.onion/2020/04/12/nyregion/schools-cuomo-de-blasio-nyc-coronavirus.html}{long
feuded over schools}, and Mr. Cuomo could still halt the mayor's
timeline for
\href{https://www.nytimes3xbfgragh.onion/2020/07/16/upshot/coronavirus-school-reopening-private-public-gap.html}{reopening}.

The governor did not contradict Mr. de Blasio on Wednesday. Instead, he
reiterated that he has the ultimate decision about whether to reopen
schools at all, and that his office will make those decisions in the
first week of August. The governor said he expected some parents to
decide whether to send their children back to school in person the night
before reopening.

Under Mr. de Blasio's plan, school principals will spend July
determining which of three staggered schedule options to adopt. That
decision will depend on how many students and staff can fit into school
buildings while social distancing, and on how many families want their
children to return to school in the first place.

School leaders will let parents know in August which days children can
report to school, and which days they will learn remotely.

The city's models are based on current federal guidelines that recommend
six feet of distance between students.

After Mr. Trump said the Centers for Disease Control and Prevention's
school reopening guidelines were too onerous, Vice President Mike Pence
said Wednesday
\href{https://www.nytimes3xbfgragh.onion/2020/07/08/world/coronavirus-updates.html?action=click\&module=Top\%20Stories\&pgtype=Homepage}{that
the C.D.C. would issue updated guidelines on schools next week}. If the
guidance is more relaxed, it could mean that New York and other
districts across the country may be able to accommodate more children in
person come fall --- if districts believe the new recommendations are
safe.

Officials in Massachusetts, for example,
\href{https://www.npr.org/2020/07/02/886845449/massachusetts-education-commissioner-on-states-plan-to-reopen-schools-in-the-fal}{have
already said schools could reopen there with at least three feet of
distance} between children.

Schools that can accommodate at least half of the student body with
distancing guidelines will be able to educate children in person two or
three days every week, while schools that can fit only about a third of
students will have children attend one or two days a week.

A
\href{https://www.nytimes3xbfgragh.onion/2020/04/16/nyregion/special-education-coronavirus-nyc.html}{relatively
small number of students with special needs} in schools that already had
very small class sizes could return for alternating full weeks or even
full-time.

Though school is scheduled to begin in early September, the options
announced Wednesday could still change significantly.

After New York become the national epicenter of the pandemic this
spring, the city flattened the curve significantly, but a significant
drop --- or spike --- could alter how schools reopen.

City Hall does not yet know precisely how many parents are planning to
keep their children home from school but will begin formally asking
families next week. If the number of students who opt for full-time
remote learning turns out to be much higher or lower than anticipated,
the models could change again. Like many urban school districts, New
York has moved away from neighborhood high schools to schools that admit
students from all over the city --- many of whom have long trips on
public transportation.

Reopening public schools, even in a limited capacity, is the biggest and
most complex obstacle on New York's long path to a full reopening.

The vast majority of the city's public school students are low-income,
and many of their parents and caretakers are essential workers who had
little choice but to report to work even at the height of the pandemic.

A Department of Education survey of about 400,000 parents found that
about 75 percent of families are tentatively willing to send their
children back to school.

Perhaps the biggest unanswered question of the reopening effort is how
working families will find child care for the days when their children
cannot be physically present in school. Mr. de Blasio said the city
would look to maximize classroom space wherever it can, but
acknowledged, ``this is something we're going to be building as we go
along.''

\href{https://www.nytimes3xbfgragh.onion/2020/07/07/nyregion/nyc-unemployment.html}{New
York is experiencing its worst financial disaster since the 1970s}, and
getting as many people back to work as possible is an urgent priority
for Mr. de Blasio's administration.

\href{https://www.nytimes3xbfgragh.onion/news-event/coronavirus?action=click\&pgtype=Article\&state=default\&region=MAIN_CONTENT_3\&context=storylines_faq}{}

\hypertarget{the-coronavirus-outbreak-}{%
\subsubsection{The Coronavirus Outbreak
›}\label{the-coronavirus-outbreak-}}

\hypertarget{frequently-asked-questions}{%
\paragraph{Frequently Asked
Questions}\label{frequently-asked-questions}}

Updated August 3, 2020

\begin{itemize}
\item ~
  \hypertarget{im-a-small-business-owner-can-i-get-relief}{%
  \paragraph{I'm a small-business owner. Can I get
  relief?}\label{im-a-small-business-owner-can-i-get-relief}}

  \begin{itemize}
  \tightlist
  \item
    The
    \href{https://www.nytimes3xbfgragh.onion/article/small-business-loans-stimulus-grants-freelancers-coronavirus.html?action=click\&pgtype=Article\&state=default\&region=MAIN_CONTENT_3\&context=storylines_faq}{stimulus
    bills enacted in March} offer help for the millions of American
    small businesses. Those eligible for aid are businesses and
    nonprofit organizations with fewer than 500 workers, including sole
    proprietorships, independent contractors and freelancers. Some
    larger companies in some industries are also eligible. The help
    being offered, which is being managed by the Small Business
    Administration, includes the Paycheck Protection Program and the
    Economic Injury Disaster Loan program. But lots of folks have
    \href{https://www.nytimes3xbfgragh.onion/interactive/2020/05/07/business/small-business-loans-coronavirus.html?action=click\&pgtype=Article\&state=default\&region=MAIN_CONTENT_3\&context=storylines_faq}{not
    yet seen payouts.} Even those who have received help are confused:
    The rules are draconian, and some are stuck sitting on
    \href{https://www.nytimes3xbfgragh.onion/2020/05/02/business/economy/loans-coronavirus-small-business.html?action=click\&pgtype=Article\&state=default\&region=MAIN_CONTENT_3\&context=storylines_faq}{money
    they don't know how to use.} Many small-business owners are getting
    less than they expected or
    \href{https://www.nytimes3xbfgragh.onion/2020/06/10/business/Small-business-loans-ppp.html?action=click\&pgtype=Article\&state=default\&region=MAIN_CONTENT_3\&context=storylines_faq}{not
    hearing anything at all.}
  \end{itemize}
\item ~
  \hypertarget{what-are-my-rights-if-i-am-worried-about-going-back-to-work}{%
  \paragraph{What are my rights if I am worried about going back to
  work?}\label{what-are-my-rights-if-i-am-worried-about-going-back-to-work}}

  \begin{itemize}
  \tightlist
  \item
    Employers have to provide
    \href{https://www.osha.gov/SLTC/covid-19/standards.html}{a safe
    workplace} with policies that protect everyone equally.
    \href{https://www.nytimes3xbfgragh.onion/article/coronavirus-money-unemployment.html?action=click\&pgtype=Article\&state=default\&region=MAIN_CONTENT_3\&context=storylines_faq}{And
    if one of your co-workers tests positive for the coronavirus, the
    C.D.C.} has said that
    \href{https://www.cdc.gov/coronavirus/2019-ncov/community/guidance-business-response.html}{employers
    should tell their employees} -\/- without giving you the sick
    employee's name -\/- that they may have been exposed to the virus.
  \end{itemize}
\item ~
  \hypertarget{should-i-refinance-my-mortgage}{%
  \paragraph{Should I refinance my
  mortgage?}\label{should-i-refinance-my-mortgage}}

  \begin{itemize}
  \tightlist
  \item
    \href{https://www.nytimes3xbfgragh.onion/article/coronavirus-money-unemployment.html?action=click\&pgtype=Article\&state=default\&region=MAIN_CONTENT_3\&context=storylines_faq}{It
    could be a good idea,} because mortgage rates have
    \href{https://www.nytimes3xbfgragh.onion/2020/07/16/business/mortgage-rates-below-3-percent.html?action=click\&pgtype=Article\&state=default\&region=MAIN_CONTENT_3\&context=storylines_faq}{never
    been lower.} Refinancing requests have pushed mortgage applications
    to some of the highest levels since 2008, so be prepared to get in
    line. But defaults are also up, so if you're thinking about buying a
    home, be aware that some lenders have tightened their standards.
  \end{itemize}
\item ~
  \hypertarget{what-is-school-going-to-look-like-in-september}{%
  \paragraph{What is school going to look like in
  September?}\label{what-is-school-going-to-look-like-in-september}}

  \begin{itemize}
  \tightlist
  \item
    It is unlikely that many schools will return to a normal schedule
    this fall, requiring the grind of
    \href{https://www.nytimes3xbfgragh.onion/2020/06/05/us/coronavirus-education-lost-learning.html?action=click\&pgtype=Article\&state=default\&region=MAIN_CONTENT_3\&context=storylines_faq}{online
    learning},
    \href{https://www.nytimes3xbfgragh.onion/2020/05/29/us/coronavirus-child-care-centers.html?action=click\&pgtype=Article\&state=default\&region=MAIN_CONTENT_3\&context=storylines_faq}{makeshift
    child care} and
    \href{https://www.nytimes3xbfgragh.onion/2020/06/03/business/economy/coronavirus-working-women.html?action=click\&pgtype=Article\&state=default\&region=MAIN_CONTENT_3\&context=storylines_faq}{stunted
    workdays} to continue. California's two largest public school
    districts --- Los Angeles and San Diego --- said on July 13, that
    \href{https://www.nytimes3xbfgragh.onion/2020/07/13/us/lausd-san-diego-school-reopening.html?action=click\&pgtype=Article\&state=default\&region=MAIN_CONTENT_3\&context=storylines_faq}{instruction
    will be remote-only in the fall}, citing concerns that surging
    coronavirus infections in their areas pose too dire a risk for
    students and teachers. Together, the two districts enroll some
    825,000 students. They are the largest in the country so far to
    abandon plans for even a partial physical return to classrooms when
    they reopen in August. For other districts, the solution won't be an
    all-or-nothing approach.
    \href{https://bioethics.jhu.edu/research-and-outreach/projects/eschool-initiative/school-policy-tracker/}{Many
    systems}, including the nation's largest, New York City, are
    devising
    \href{https://www.nytimes3xbfgragh.onion/2020/06/26/us/coronavirus-schools-reopen-fall.html?action=click\&pgtype=Article\&state=default\&region=MAIN_CONTENT_3\&context=storylines_faq}{hybrid
    plans} that involve spending some days in classrooms and other days
    online. There's no national policy on this yet, so check with your
    municipal school system regularly to see what is happening in your
    community.
  \end{itemize}
\item ~
  \hypertarget{is-the-coronavirus-airborne}{%
  \paragraph{Is the coronavirus
  airborne?}\label{is-the-coronavirus-airborne}}

  \begin{itemize}
  \tightlist
  \item
    The coronavirus
    \href{https://www.nytimes3xbfgragh.onion/2020/07/04/health/239-experts-with-one-big-claim-the-coronavirus-is-airborne.html?action=click\&pgtype=Article\&state=default\&region=MAIN_CONTENT_3\&context=storylines_faq}{can
    stay aloft for hours in tiny droplets in stagnant air}, infecting
    people as they inhale, mounting scientific evidence suggests. This
    risk is highest in crowded indoor spaces with poor ventilation, and
    may help explain super-spreading events reported in meatpacking
    plants, churches and restaurants.
    \href{https://www.nytimes3xbfgragh.onion/2020/07/06/health/coronavirus-airborne-aerosols.html?action=click\&pgtype=Article\&state=default\&region=MAIN_CONTENT_3\&context=storylines_faq}{It's
    unclear how often the virus is spread} via these tiny droplets, or
    aerosols, compared with larger droplets that are expelled when a
    sick person coughs or sneezes, or transmitted through contact with
    contaminated surfaces, said Linsey Marr, an aerosol expert at
    Virginia Tech. Aerosols are released even when a person without
    symptoms exhales, talks or sings, according to Dr. Marr and more
    than 200 other experts, who
    \href{https://academic.oup.com/cid/article/doi/10.1093/cid/ciaa939/5867798}{have
    outlined the evidence in an open letter to the World Health
    Organization}.
  \end{itemize}
\end{itemize}

Finding ways to plug the enormous gaps in child care is sure to be a
citywide effort that does not rely solely on the Department of
Education, since school buildings will be fully occupied by September.

The city will have to find other public and private space to accommodate
thousands of children a day. The city Board of Health recently
\href{https://ny.chalkbeat.org/2020/7/7/21316621/nyc-preschools-can-reopen}{authorized
the reopening of child care centers}, but those centers have only a tiny
fraction of the capacity the city will need come September.

Anand Raghunath, a parent of two children who attend school in East
Harlem, spent Wednesday morning frantically plotting out child care
options for the fall.

After he heard the mayor's announcement, Mr. Raghunath said he turned to
his wife, who works in a hospital, and asked, ``What are we going to
do?''

He said the proposal, with students attending school physically for a
range of one to three days a week, does not allow his family to do much
specific planning. For now, Mr. Raghunath's strategy is to have his
mother-in-law fly in from California and watch the children while he and
his wife work.

``We're all at a standstill here,'' he said.

Restarting school even a few days a week presents myriad logistical
hurdles.

Many of the city's school buildings are over a century old, with poor
air ventilation and cramped classrooms and hallways. Drastic budget cuts
have left many schools with less money to hire teachers and staff ---
all while the city estimates that about one in five current teachers
will receive medical exemptions to work remotely.

Schools Chancellor Richard A. Carranza said Wednesday that the city
would try to bring as many Department of Education employees with
teaching certificates into classrooms as possible.

Though union leaders have raised alarms about whether schools will have
enough personal protective gear and nurses to safely reopen, the city
has said it will deep clean schools each night and have sanitizer and
disinfectant in all classrooms and common spaces. Some teachers have
said they did not have enough resources to keep schools clean when the
virus was spreading in March.

``We can make up learning for students,'' Mr. Carranza said. ``We cannot
bring a student back who is infected and passes away.'' All teachers and
students will be expected to wear masks throughout the school day come
fall.

Mr. de Blasio on Wednesday laid out three schedules for principals to
consider.

The most generous model would apply to schools that can accommodate at
least half of their students while adhering to social distancing. In
that scenario, two cohorts of students would cycle in and out on
alternating days that remain consistent throughout the semester.

For example, one group might attend school on Tuesdays and Thursdays,
with the second group in classes on Wednesdays and Fridays, while
Mondays alternate weekly between the two groups.

More crowded schools will have three groups of students who attend
school just once or twice a week. Those students will all have five days
of in person instruction every three weeks, but some middle and high
school students might not be in school on any consistent day week to
week.

Mr. Cuomo will have some oversight regarding how New York's hundreds of
private, parochial and charter schools plan to reopen this fall. There
are many charter schools that share buildings with district schools and
will likely have to follow the same basic staggering plans, though
charters in private spaces could have more flexibility.

Private schools that already had smaller class sizes and spacious
buildings may be able to accommodate more children in person than most
public schools.

Advertisement

\protect\hyperlink{after-bottom}{Continue reading the main story}

\hypertarget{site-index}{%
\subsection{Site Index}\label{site-index}}

\hypertarget{site-information-navigation}{%
\subsection{Site Information
Navigation}\label{site-information-navigation}}

\begin{itemize}
\tightlist
\item
  \href{https://help.nytimes3xbfgragh.onion/hc/en-us/articles/115014792127-Copyright-notice}{©~2020~The
  New York Times Company}
\end{itemize}

\begin{itemize}
\tightlist
\item
  \href{https://www.nytco.com/}{NYTCo}
\item
  \href{https://help.nytimes3xbfgragh.onion/hc/en-us/articles/115015385887-Contact-Us}{Contact
  Us}
\item
  \href{https://www.nytco.com/careers/}{Work with us}
\item
  \href{https://nytmediakit.com/}{Advertise}
\item
  \href{http://www.tbrandstudio.com/}{T Brand Studio}
\item
  \href{https://www.nytimes3xbfgragh.onion/privacy/cookie-policy\#how-do-i-manage-trackers}{Your
  Ad Choices}
\item
  \href{https://www.nytimes3xbfgragh.onion/privacy}{Privacy}
\item
  \href{https://help.nytimes3xbfgragh.onion/hc/en-us/articles/115014893428-Terms-of-service}{Terms
  of Service}
\item
  \href{https://help.nytimes3xbfgragh.onion/hc/en-us/articles/115014893968-Terms-of-sale}{Terms
  of Sale}
\item
  \href{https://spiderbites.nytimes3xbfgragh.onion}{Site Map}
\item
  \href{https://help.nytimes3xbfgragh.onion/hc/en-us}{Help}
\item
  \href{https://www.nytimes3xbfgragh.onion/subscription?campaignId=37WXW}{Subscriptions}
\end{itemize}
