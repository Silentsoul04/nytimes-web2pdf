Sections

SEARCH

\protect\hyperlink{site-content}{Skip to
content}\protect\hyperlink{site-index}{Skip to site index}

\href{https://www.nytimes3xbfgragh.onion/section/technology/personaltech}{Personal
Tech}

\href{https://myaccount.nytimes3xbfgragh.onion/auth/login?response_type=cookie\&client_id=vi}{}

\href{https://www.nytimes3xbfgragh.onion/section/todayspaper}{Today's
Paper}

\href{/section/technology/personaltech}{Personal Tech}\textbar{}How to
Buy Tech That Lasts and Lasts

\url{https://nyti.ms/3fdIhR5}

\begin{itemize}
\item
\item
\item
\item
\item
\item
\end{itemize}

\href{https://www.nytimes3xbfgragh.onion/spotlight/at-home?action=click\&pgtype=Article\&state=default\&region=TOP_BANNER\&context=at_home_menu}{At
Home}

\begin{itemize}
\tightlist
\item
  \href{https://www.nytimes3xbfgragh.onion/2020/07/28/books/time-for-a-literary-road-trip.html?action=click\&pgtype=Article\&state=default\&region=TOP_BANNER\&context=at_home_menu}{Take:
  A Literary Road Trip}
\item
  \href{https://www.nytimes3xbfgragh.onion/2020/07/29/magazine/bored-with-your-home-cooking-some-smoky-eggplant-will-fix-that.html?action=click\&pgtype=Article\&state=default\&region=TOP_BANNER\&context=at_home_menu}{Cook:
  Smoky Eggplant}
\item
  \href{https://www.nytimes3xbfgragh.onion/2020/07/27/travel/moose-michigan-isle-royale.html?action=click\&pgtype=Article\&state=default\&region=TOP_BANNER\&context=at_home_menu}{Look
  Out: For Moose}
\item
  \href{https://www.nytimes3xbfgragh.onion/interactive/2020/at-home/even-more-reporters-editors-diaries-lists-recommendations.html?action=click\&pgtype=Article\&state=default\&region=TOP_BANNER\&context=at_home_menu}{Explore:
  Reporters' Obsessions}
\end{itemize}

Advertisement

\protect\hyperlink{after-top}{Continue reading the main story}

Supported by

\protect\hyperlink{after-sponsor}{Continue reading the main story}

tech fix

\hypertarget{how-to-buy-tech-that-lasts-and-lasts}{%
\section{How to Buy Tech That Lasts and
Lasts}\label{how-to-buy-tech-that-lasts-and-lasts}}

All of our tech products will one day become obsolete, but here are some
strategies to buying gadgets that you can enjoy for many years.

\includegraphics{https://static01.graylady3jvrrxbe.onion/images/2020/07/09/business/09Techfix-illo/07Techfix-illo-articleLarge.gif?quality=75\&auto=webp\&disable=upscale}

\href{https://www.nytimes3xbfgragh.onion/by/brian-x-chen}{\includegraphics{https://static01.graylady3jvrrxbe.onion/images/2018/02/16/multimedia/author-brian-x-chen/author-brian-x-chen-thumbLarge.jpg}}

By \href{https://www.nytimes3xbfgragh.onion/by/brian-x-chen}{Brian X.
Chen}

\begin{itemize}
\item
  July 8, 2020
\item
  \begin{itemize}
  \item
  \item
  \item
  \item
  \item
  \item
  \end{itemize}
\end{itemize}

When we buy a gadget these days, we rarely assume that it will endure.

We expect to play a video game console only for as long as companies
make games for it. We expect to use a smartphone or a laptop for just as
long as the battery has juice or until it can no longer run important
software.

At some point, we feel that we must upgrade. We must have the latest and
greatest camera. We must have apps that run faster. We must have
brighter screens.

Here's the thing: This is all the doing of marketing professionals,
seared into our subconscious. The reality is that consumer electronics,
such as your phone, computer or tablet, can last for many years. It just
takes some research to obtain tech that will endure. This exercise will
be increasingly important in a pandemic-induced recession, which has
forced many of us to tighten our spending.

``It's a matter of buying what you need, not what the company is telling
you that you need,'' said Carole Mars, the director of technical
development and innovation at the
\href{https://www.sustainabilityconsortium.org/}{Sustainability
Consortium}, which studies the sustainability of consumer goods.

Strategically choosing tech with a longer shelf life is not intuitive.
It involves assessing how easy or not it is to repair a particular
product and determining when it makes sense to invest more money. Here
are some questions to consider for the long run.

\hypertarget{is-the-tech-easy-to-repair}{%
\subsection{Is the tech easy to
repair?}\label{is-the-tech-easy-to-repair}}

The next time you shop for an electronic product, try this exercise:
Before you buy it, find out whether you or a professional can easily fix
it. If so, then go for it. If it's too difficult, make it a hard pass.

Vincent Lai, who works for the
\href{http://www.nytimes3xbfgragh.onion/2009/10/08/garden/08seen.html?module=inline}{Fixers'
Collective}, a social club in New York that repairs aging devices,
offered several approaches to assessing whether a gadget can be
straightforwardly fixed:

\begin{itemize}
\item
  \textbf{Consult iFixit, a website that offers instructions on gadget
  repairs.} For some products, the site tears apart gadgets and does an
  analysis on its ease of repair. Apple's iPhone SE, for example, has a
  \href{https://www.ifixit.com/Teardown/iPhone+SE+2020+Teardown/133066}{repairability
  score of 6 out of 10} (10 being the easiest to repair), so it could be
  a device worth considering for the long haul.
\item
  \textbf{Check if local technicians can service the device.} Plenty of
  technicians have the parts and ability to service popular phones like
  iPhones and Samsung Galaxy devices. But if you want to buy a handset
  from a less popular brand, like OnePlus or Motorola, it's worth
  calling around first to find out if anyone can fix it in case
  something goes wrong.
\item
  \textbf{Find out whether there's a community of enthusiasts.}
  Sometimes there are no local fixers who can help with a product, but
  there may be enthusiasts who write their own guides that you can
  follow. While you probably can't find someone to repair a Philips
  Sonicare ****** electric toothbrush that is out of warranty, there are
  \href{https://www.ifixit.com/Guide/How+to+fix+a+Philips+Sonicare+DiamondClean+HX9340+when+it+is+getting+less+powerful+or+has+a+weak+brush./76240}{instructions
  on how to service it on iFixit.}
\end{itemize}

\hypertarget{is-the-battery-replaceable}{%
\subsection{Is the battery
replaceable?}\label{is-the-battery-replaceable}}

One of the clearest indicators of a product's durability is whether the
batteries are replaceable. Gadgets that work without wires are powered
by a lithium-ion battery, which can be charged only a finite number of
times before it deteriorates.

Fortunately, most phones and laptops have batteries that can be replaced
by professionals. But more compact products have components that are
glued together and tightly sealed up, making their batteries impossible
to replace. Wireless earphones like
\href{https://www.nytimes3xbfgragh.onion/2019/04/03/technology/personaltech/apple-airpods-review.html}{Apple's
AirPods} and
\href{https://www.bose.com/en_us/products/headphones/over_ear_headphones/quietcomfort-35-wireless-ii.html\#v=qc35_ii_black}{Bose's
QuietComfort 35} are examples of popular products with irreplaceable
batteries. Once the batteries die, you have to buy a brand-new pair.

So if you're buying anything with a battery --- including digital
picture frames, wireless security cameras and Bluetooth speakers --- do
a web search to see if the battery can be replaced. If not, consider it
disposable.

\hypertarget{is-the-product-reliable}{%
\subsection{Is the product reliable?}\label{is-the-product-reliable}}

Like household appliances, tech products have failure rates --- the
ratio of working to defective units. These rates can give you a sense of
a brand's reliability.

Consumer Reports, well known for publishing reliability ratings for
household appliances, compiles similar reliability data for smartphones,
laptops, tablets, TVs and printers by surveying subscribers who own the
products.

People tend to have more problems with products that have moving parts,
like printers with ink cartridges, than with electronics like TVs or
tablets, said Jerry Beilinson, a technology editor at Consumer Reports.
Brother printers fared well in the publication's surveys. For phones,
Apple and Samsung had strong reliability ratings.

Mr. Lai of the Fixers' Collective recommends a grass-roots approach to
assessing reliability. He reads web forums like Reddit to see what
people are saying about a product. If a large number of customers report
problems with the device, he said, he steers clear.

\hypertarget{should-i-spend-more}{%
\subsection{Should I spend more?}\label{should-i-spend-more}}

Another rule of thumb to consider is investing more in a product to make
it last. That doesn't mean you have to buy the most expensive phone or
computer on the market. But it does mean investing in configurations
that will make you happier in the long run, said Nick Guy, a senior
staff writer for
\href{https://www.nytimes3xbfgragh.onion/wirecutter/}{Wirecutter}, a New
York Times publication that tests products.

Let's use an iPad as an example. If you wanted
\href{https://www.nytimes3xbfgragh.onion/2020/04/21/technology/personaltech/apple-ipad-gadget-pandemic.html}{an
iPad}, you could pay \$329 for the base model with 32 gigabytes of
storage. But it's probably a better idea to spend \$429 on the model
with 128 gigabytes of storage --- that's quadruple the capacity, which
you can use to hold apps, games, photos and videos for years to come.

In tech parlance, this strategy is known as ``futureproofing.''

If you're turned off by the idea of spending more, there's a way around
that. You can look to buy the same higher-priced product
\href{https://www.nytimes3xbfgragh.onion/2016/04/28/technology/personaltech/taking-the-stigma-out-of-buying-usedelectronics.html}{refurbished}
--- meaning it was returned by a customer and restored to its former
glory --- for a significant discount, Dr. Mars said.

\hypertarget{is-the-software-easy-to-update}{%
\subsection{Is the software easy to
update?}\label{is-the-software-easy-to-update}}

Because many modern gadgets, like smartphones and tablets, mostly lack
moving parts, their software plays a strong role in determining their
longevity. After a company stops providing software updates to a device,
you can expect to run into problems, such as your favorite apps ceasing
to work properly.

This is where an iPhone has an edge over an Android. Each year, when
Apple releases a new operating system for the iPhone, it generally works
on phones as far back as five years ago. (Apple's
\href{https://www.apple.com/ios/ios-14-preview/}{iOS 14}, due for
release this fall, will support the iPhone 6S from 2015.) That means
when you buy an iPhone, it will probably get new features and stability
improvements for at least five years.

Android users will have a tougher time. Typically, manufacturers provide
software updates to Android devices for two or three years.

To get around that, Android users might turn to the grass-roots
community. For some Android phones, Mr. Lai said, there are enthusiasts
who offer so-called ROMs, or custom-made operating systems, which can be
installed to keep the software up to date. Check the website
\href{https://www.xda-developers.com/the-most-popular-custom-roms-on-xda/}{XDA
Developers} to see whether tinkerers are building custom software for
the Android phone you intend to buy.

\hypertarget{does-it-solve-a-problem}{%
\subsection{Does it solve a problem?}\label{does-it-solve-a-problem}}

Many so-called smart home gadgets --- ordinary appliances with wireless
sensors and an internet connection --- offer interesting benefits, like
a refrigerator with a camera that sends an alert to our phone when the
milk is running low.

Just keep in mind that smart home products can create more problems than
they solve. A trash can that automatically opens its lid when you wave
your hand over it may feel magical, but it relies on batteries and
moving parts that eventually wear out.

``If it moves, if it flashes, if it can connect to the internet and
tattle on you, it's an electronic,'' Dr. Mars said, ``and you're
inheriting all the issues that come with an electronic.''

It all comes back to buying what you truly need. Sometimes a
\href{https://www.nytimes3xbfgragh.onion/2018/02/21/technology/personaltech/smart-things-dumb-stuff.html}{``dumb''
product} will do just fine.

Advertisement

\protect\hyperlink{after-bottom}{Continue reading the main story}

\hypertarget{site-index}{%
\subsection{Site Index}\label{site-index}}

\hypertarget{site-information-navigation}{%
\subsection{Site Information
Navigation}\label{site-information-navigation}}

\begin{itemize}
\tightlist
\item
  \href{https://help.nytimes3xbfgragh.onion/hc/en-us/articles/115014792127-Copyright-notice}{©~2020~The
  New York Times Company}
\end{itemize}

\begin{itemize}
\tightlist
\item
  \href{https://www.nytco.com/}{NYTCo}
\item
  \href{https://help.nytimes3xbfgragh.onion/hc/en-us/articles/115015385887-Contact-Us}{Contact
  Us}
\item
  \href{https://www.nytco.com/careers/}{Work with us}
\item
  \href{https://nytmediakit.com/}{Advertise}
\item
  \href{http://www.tbrandstudio.com/}{T Brand Studio}
\item
  \href{https://www.nytimes3xbfgragh.onion/privacy/cookie-policy\#how-do-i-manage-trackers}{Your
  Ad Choices}
\item
  \href{https://www.nytimes3xbfgragh.onion/privacy}{Privacy}
\item
  \href{https://help.nytimes3xbfgragh.onion/hc/en-us/articles/115014893428-Terms-of-service}{Terms
  of Service}
\item
  \href{https://help.nytimes3xbfgragh.onion/hc/en-us/articles/115014893968-Terms-of-sale}{Terms
  of Sale}
\item
  \href{https://spiderbites.nytimes3xbfgragh.onion}{Site Map}
\item
  \href{https://help.nytimes3xbfgragh.onion/hc/en-us}{Help}
\item
  \href{https://www.nytimes3xbfgragh.onion/subscription?campaignId=37WXW}{Subscriptions}
\end{itemize}
