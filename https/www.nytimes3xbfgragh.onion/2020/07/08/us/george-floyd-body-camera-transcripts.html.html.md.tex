Sections

SEARCH

\protect\hyperlink{site-content}{Skip to
content}\protect\hyperlink{site-index}{Skip to site index}

\href{https://www.nytimes3xbfgragh.onion/section/us}{U.S.}

\href{https://myaccount.nytimes3xbfgragh.onion/auth/login?response_type=cookie\&client_id=vi}{}

\href{https://www.nytimes3xbfgragh.onion/section/todayspaper}{Today's
Paper}

\href{/section/us}{U.S.}\textbar{}New Transcripts Detail Last Moments
for George Floyd

\url{https://nyti.ms/2O53mkD}

\begin{itemize}
\item
\item
\item
\item
\item
\item
\end{itemize}

\href{https://www.nytimes3xbfgragh.onion/news-event/george-floyd-protests-minneapolis-new-york-los-angeles?action=click\&pgtype=Article\&state=default\&region=TOP_BANNER\&context=storylines_menu}{Race
and America}

\begin{itemize}
\tightlist
\item
  \href{https://www.nytimes3xbfgragh.onion/2020/07/26/us/protests-portland-seattle-trump.html?action=click\&pgtype=Article\&state=default\&region=TOP_BANNER\&context=storylines_menu}{Protesters
  Return to Other Cities}
\item
  \href{https://www.nytimes3xbfgragh.onion/2020/07/24/us/portland-oregon-protests-white-race.html?action=click\&pgtype=Article\&state=default\&region=TOP_BANNER\&context=storylines_menu}{Portland
  at the Center}
\item
  \href{https://www.nytimes3xbfgragh.onion/2020/07/23/podcasts/the-daily/portland-protests.html?action=click\&pgtype=Article\&state=default\&region=TOP_BANNER\&context=storylines_menu}{Podcast:
  Showdown in Portland}
\item
  \href{https://www.nytimes3xbfgragh.onion/interactive/2020/07/16/us/black-lives-matter-protests-louisville-breonna-taylor.html?action=click\&pgtype=Article\&state=default\&region=TOP_BANNER\&context=storylines_menu}{45
  Days in Louisville}
\end{itemize}

Advertisement

\protect\hyperlink{after-top}{Continue reading the main story}

Supported by

\protect\hyperlink{after-sponsor}{Continue reading the main story}

\hypertarget{new-transcripts-detail-last-moments-for-george-floyd}{%
\section{New Transcripts Detail Last Moments for George
Floyd}\label{new-transcripts-detail-last-moments-for-george-floyd}}

``They'll kill me. They'll kill me,'' Mr. Floyd pleaded, according to a
body camera transcript in court filings by a former officer who wants
the charges against him dismissed.

\includegraphics{https://static01.graylady3jvrrxbe.onion/images/2020/07/08/us/08minneapolis/08minneapolis-articleLarge.jpg?quality=75\&auto=webp\&disable=upscale}

By
\href{https://www.nytimes3xbfgragh.onion/by/richard-a-oppel-jr}{Richard
A. Oppel Jr.} and
\href{https://www.nytimes3xbfgragh.onion/by/kim-barker}{Kim Barker}

\begin{itemize}
\item
  Published July 8, 2020Updated July 29, 2020
\item
  \begin{itemize}
  \item
  \item
  \item
  \item
  \item
  \item
  \end{itemize}
\end{itemize}

George Floyd's dying moments have played on an endless loop, horrifying
the world and prompting a spasm of street protests, but newly released
evidence reveals an even more desperate scene than previously known in
the moments before an officer pressed his knee into Mr. Floyd's neck.

Mr. Floyd uttered ``I can't breathe'' not a handful of times, as
previous videotapes showed, but more than 20 times in all. He cried out
not just for his dead mother but for his children too. Before his final
breaths, Mr. Floyd gasped: ``They'll kill me. They'll kill me.''

As Mr. Floyd shouted for his life, an officer yelled back at him to
``stop talking, stop yelling, it takes a heck of a lot of oxygen to
talk.''

The
\href{https://int.graylady3jvrrxbe.onion/data/documenthelper/7070-exhibit-final07072020/4b81216735f2203a08cb/optimized/full.pdf\#page=1}{chilling
transcripts} of
\href{https://www.nytimes3xbfgragh.onion/2020/07/29/us/george-floyd-memorial.html}{Minneapolis}
police body camera footage, made public on Wednesday, were filed in
state court as part of an effort by one of the officers on the scene,
Thomas Lane, 37, to have charges that he aided and abetted Mr. Floyd's
murder thrown out by a judge.

Mr. Floyd, 46, died after another officer,
\href{https://www.nytimes3xbfgragh.onion/2020/07/18/us/derek-chauvin-george-floyd.html}{Derek
Chauvin}, 44, pressed his knee down onto Mr. Floyd's neck for more than
eight minutes until he was no longer moving.

\href{https://www.nytimes3xbfgragh.onion/interactive/2020/06/28/us/i-cant-breathe-police-arrest.html}{}

\includegraphics{https://static01.graylady3jvrrxbe.onion/images/2020/06/28/us/icantbreathe-promo/icantbreathe-promo-articleLarge.jpg}

\hypertarget{three-words-70-cases-the-tragic-history-of-i-cant-breathe}{%
\subsection{Three Words. 70 Cases. The Tragic History of `I Can't
Breathe.'}\label{three-words-70-cases-the-tragic-history-of-i-cant-breathe}}

The deaths of Eric Garner in New York and George Floyd in Minnesota
created national outrage over the use of deadly police restraints. There
were many others you didn't hear about.

Mr. Chauvin, who was on the force for 19 years, faces second-degree
murder and second-degree manslaughter charges in Mr. Floyd's death and
up to 40 years in prison if he is convicted. Mr. Lane and J. Alexander
Kueng, 26, who were both rookie officers, and Tou Thao, 34, also face 40
years in prison if convicted on charges of aiding and abetting Mr.
Floyd's murder. All four officers were fired.

Even before he was on the ground, Mr. Floyd said he was in physical
distress, telling officers who were trying to get him into a squad car
that he was claustrophobic and could not breathe.

At one point, according to one transcript, he said: ``Momma, I love you.
Tell my kids I love them. I'm dead.''

At another point, Mr. Chauvin asked if Mr. Floyd was high on something;
Mr. Lane said he assumed so, and Mr. Kueng said they had found a pipe on
him. One
\href{https://www.nytimes3xbfgragh.onion/article/george-floyd-autopsy-michael-baden.html}{autopsy
report} found traces of illegal drugs in Mr. Floyd's body.

``Relax,'' Mr. Thao told Mr. Floyd.

``I can't breathe,'' Mr. Floyd said.

``You're fine,'' Mr. Kueng replied. ``You're talking fine.''

``Deep breath,'' Mr. Lane added.

\hypertarget{transcripts-of-george-floyds-last-minutes}{%
\subsection{Transcripts of George Floyd's Last
Minutes}\label{transcripts-of-george-floyds-last-minutes}}

Newly released transcripts offer one of the most thorough and dramatic
accounts yet of the moments before Mr. Floyd's death.

\includegraphics{https://int.graylady3jvrrxbe.onion/data/documenthelper/7070-exhibit-final07072020/4b81216735f2203a08cb/optimized/thumbnail.png}

The new court filings include 82 pages of body camera transcripts as
well as the 60-page transcript of Mr. Lane's interview with
investigators from Minnesota's Bureau of Criminal Apprehension.

In that interview, when he was asked whether he felt at the time that
Mr. Floyd was having a medical emergency, Mr. Lane replied, ``Yeah, I
felt maybe that something was going on.''

At the end of the interview, though, Mr. Lane's lawyer, Earl Gray,
objected when an investigator asked Mr. Lane whether he felt that either
he or Mr. Chauvin had contributed to Mr. Floyd's death.

``You're not going to answer that,'' Mr. Gray said. Mr. Lane did not
answer the question.

Much of what had been
\href{https://www.nytimes3xbfgragh.onion/2020/05/31/us/george-floyd-investigation.html}{known
about Mr. Floyd's final moments} had come from
\href{https://www.nytimes3xbfgragh.onion/2020/05/26/us/george-floyd-minneapolis-police.html}{bystander
video}, surveillance footage and probable cause statements released by
prosecutors when they filed charges against the officers. But the body
camera transcripts, and Mr. Lane's interview with investigators, provide
more details about Mr. Floyd's exchange with officers, and how
vociferously and persistently he had pleaded with them that he was
having a medical emergency.

The filings include what Mr. Gray described as pictures from inside the
car Mr. Floyd was sitting in when Mr. Lane first approached him.
Officers had been called after a nearby store employee reported that Mr.
Floyd had passed a counterfeit \$20 bill. The pictures show two crumpled
\$20 bills that Mr. Gray said were counterfeit and that he said were
found lodged between the center console and the passenger's seat.

The filings also indicate that an ambulance, called early in the
encounter, did not respond right away and initially went to the wrong
spot.

According to the transcripts, Mr. Lane called for an ambulance after Mr.
Floyd's mouth started bleeding. Mr. Lane told investigators it was
likely when Mr. Floyd banged his face on the glass inside of the squad
car.

Mr. Lane then upgraded that ambulance request, from a less-serious
``Code 2'' to a more serious ``Code 3,'' after Mr. Floyd had repeatedly
said he could not breathe and the officers discussed whether he could be
high on drugs.

\includegraphics{https://static01.graylady3jvrrxbe.onion/images/2020/05/27/autossell/flyod-site-1-white-box/flyod-site-1-white-box-videoSixteenByNineJumbo1600.jpg}

The transcripts zero in on the most critical moments of Mr. Floyd's
restraint by officers.

After Mr. Floyd says that the officers were going to kill him, Mr.
Chauvin said, according to one of the transcripts, ``Then stop talking,
stop yelling, it takes a heck of a lot of oxygen to talk.''

While Mr. Floyd was being restrained on the ground, on his stomach, with
Mr. Chauvin's knee pressed onto his neck, Mr. Lane asked whether Mr.
Floyd should be turned onto his side.

Mr. Chauvin said, ``No, he's staying put where we got him.''

Mr. Lane then said he was worried Mr. Floyd might be having a medical
emergency.

``Well that's why we got the ambulance coming,'' Mr. Chauvin responded,
according to one of the transcripts.

``OK, I suppose,'' Mr. Lane replied, adding soon after, ``I think he's
passing out.''

At that moment, a bystander shouted: ``He's not even breathing right
now, bro, you think that's cool? You think that's cool, right?'' Other
onlookers repeatedly asked if Mr. Floyd had a pulse.

``You got one?'' Mr. Lane asked. ``I can't find one,'' Mr. Kueng said.
``Huh?'' Mr. Chauvin replied. Mr. Kueng tried again, and again said he
could not find a pulse.

More than two minutes then went by, according to timestamps on the
transcript of Mr. Kueng's body camera footage. Still, Mr. Chauvin kept
his knee on Mr. Floyd's neck, videos show.

After the ambulance arrived, Mr. Lane rode with Mr. Floyd to a hospital
alongside ambulance workers and performed chest compressions on him. One
worker told Mr. Lane the ambulance waited to respond because it was
called in as a ``code 2 mouth injury.''

''And then as we're sitting here, I'm like, `Now it says Code 3, I just
don't understand,''' the worker said, explaining what had happened.
``And then we figured out where it was so, and then one of your officers
was like, `Hey, hey ding-dongs, you're at the wrong spot.'''

The filings were the latest effort by Mr. Lane, who held Mr. Floyd's
legs while he was on the ground, to argue that he does not bear the
responsibility for Mr. Floyd's death that prosecutors say he does.

At Mr. Lane's first court appearance a month ago, his lawyer, Mr. Gray,
sought to emphasize Mr. Chauvin's status as a senior officer who helped
train rookies, and that the fateful encounter with Mr. Floyd had
occurred on Mr. Lane's fourth day on the force.

``They're required to call him `Sir,''' Mr. Gray said in court about Mr.
Chauvin, who served as a field training officer, or F.T.O. ``He has 20
years' experience. What is my client supposed to do but to follow what
the training officer said? Is that aiding and abetting a crime?''

In the court papers filed this week where he asks the judge to dismiss
the charges against Mr. Lane, Mr. Gray argued that Mr. Lane, as a new
officer, was taking his cues from Mr. Chauvin. He also stated that Mr.
Lane believed that Mr. Floyd was on drugs ``based on his behavior.''

After Mr. Chauvin refused to turn Mr. Floyd onto his side, Mr. Gray
wrote in his filings, ``Lane listened to F.T.O. Chauvin and thought it
made sense because there are times when a person who is OD'ing or passed
out one minute but then comes back really aggressive.''

John Eligon and Matt Furber contributed reporting.

Advertisement

\protect\hyperlink{after-bottom}{Continue reading the main story}

\hypertarget{site-index}{%
\subsection{Site Index}\label{site-index}}

\hypertarget{site-information-navigation}{%
\subsection{Site Information
Navigation}\label{site-information-navigation}}

\begin{itemize}
\tightlist
\item
  \href{https://help.nytimes3xbfgragh.onion/hc/en-us/articles/115014792127-Copyright-notice}{©~2020~The
  New York Times Company}
\end{itemize}

\begin{itemize}
\tightlist
\item
  \href{https://www.nytco.com/}{NYTCo}
\item
  \href{https://help.nytimes3xbfgragh.onion/hc/en-us/articles/115015385887-Contact-Us}{Contact
  Us}
\item
  \href{https://www.nytco.com/careers/}{Work with us}
\item
  \href{https://nytmediakit.com/}{Advertise}
\item
  \href{http://www.tbrandstudio.com/}{T Brand Studio}
\item
  \href{https://www.nytimes3xbfgragh.onion/privacy/cookie-policy\#how-do-i-manage-trackers}{Your
  Ad Choices}
\item
  \href{https://www.nytimes3xbfgragh.onion/privacy}{Privacy}
\item
  \href{https://help.nytimes3xbfgragh.onion/hc/en-us/articles/115014893428-Terms-of-service}{Terms
  of Service}
\item
  \href{https://help.nytimes3xbfgragh.onion/hc/en-us/articles/115014893968-Terms-of-sale}{Terms
  of Sale}
\item
  \href{https://spiderbites.nytimes3xbfgragh.onion}{Site Map}
\item
  \href{https://help.nytimes3xbfgragh.onion/hc/en-us}{Help}
\item
  \href{https://www.nytimes3xbfgragh.onion/subscription?campaignId=37WXW}{Subscriptions}
\end{itemize}
