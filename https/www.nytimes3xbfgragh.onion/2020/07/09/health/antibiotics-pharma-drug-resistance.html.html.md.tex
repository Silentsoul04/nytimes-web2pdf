Sections

SEARCH

\protect\hyperlink{site-content}{Skip to
content}\protect\hyperlink{site-index}{Skip to site index}

\href{https://www.nytimes3xbfgragh.onion/section/health}{Health}

\href{https://myaccount.nytimes3xbfgragh.onion/auth/login?response_type=cookie\&client_id=vi}{}

\href{https://www.nytimes3xbfgragh.onion/section/todayspaper}{Today's
Paper}

\href{/section/health}{Health}\textbar{}Drug Giants Create Fund to
Bolster Struggling Antibiotic Start-Ups

\url{https://nyti.ms/2ZTm8kv}

\begin{itemize}
\item
\item
\item
\item
\item
\item
\end{itemize}

Advertisement

\protect\hyperlink{after-top}{Continue reading the main story}

Supported by

\protect\hyperlink{after-sponsor}{Continue reading the main story}

\hypertarget{drug-giants-create-fund-to-bolster-struggling-antibiotic-start-ups}{%
\section{Drug Giants Create Fund to Bolster Struggling Antibiotic
Start-Ups}\label{drug-giants-create-fund-to-bolster-struggling-antibiotic-start-ups}}

New medicines are desperately needed to treat a growing number of
drug-resistant infections, but many companies developing the drugs are
short on cash and investments.

\includegraphics{https://static01.graylady3jvrrxbe.onion/images/2020/07/10/science/09antibiotics-print/merlin_152334294_d66d1aea-43f3-476d-a43e-d1cbe05dcb21-articleLarge.jpg?quality=75\&auto=webp\&disable=upscale}

By \href{https://www.nytimes3xbfgragh.onion/by/andrew-jacobs}{Andrew
Jacobs}

\begin{itemize}
\item
  July 9, 2020
\item
  \begin{itemize}
  \item
  \item
  \item
  \item
  \item
  \item
  \end{itemize}
\end{itemize}

Twenty of the world's largest pharmaceutical companies on Thursday
announced the creation of a \$1 billion fund to buoy financially
strapped biotech start-ups that are developing new antibiotics to treat
the
\href{https://www.nytimes3xbfgragh.onion/2019/04/06/health/drug-resistant-candida-auris.html}{mounting
number} of
\href{https://www.nytimes3xbfgragh.onion/2020/01/17/health/antibiotics-resistance-new-drugs.html}{drug-resistant
infections} responsible for hundreds of thousands of deaths each year.

The fund, created in partnership with the World Health Organization and
financed by drug behemoths that include Roche, Merck, and Johnson \&
Johnson, will offer a short-term but desperately needed lifeline for
some of the three dozen small antibiotic companies, many of them based
in the United States, that have been struggling to draw investment amid
a collapsing antibiotics industry.

Over the past year, three
\href{https://www.nytimes3xbfgragh.onion/2019/12/25/health/antibiotics-new-resistance.html}{American
antibiotic start-ups} with promising drugs have gone bankrupt, and many
of the remaining companies are quickly running out of cash.

The new \href{https://amractionfund.com/}{AMR Action Fund} will make
investments in roughly two dozen companies that have already identified
a promising drug with the goal of bringing two to four novel antibiotics
to the market within a decade, according to the International Federation
of Pharmaceutical Manufacturers and Associations, an industry trade
group that is administering the fund.

Recipients will be chosen by an advisory panel made up of drug company
executives, scientists and other experts in the field. The companies
will also provide free expertise to biotech companies with promising
drugs as they navigate the clinical and regulatory hurdles needed to
bring an antimicrobial compound from laboratory to market.

``Antibiotics are the mortar that holds the entire health care system
together,'' said David A. Ricks, the chief executive of Eli Lilly, who
helped spearhead the effort. ``We make drugs for diabetes, cancer and
immunological conditions, but you couldn't treat any of them without
effective antibiotics.''

In an interview, Mr. Ricks said he was well aware of the irony that Eli
Lilly and many of the other companies contributing to the fund were once
the giants of antibiotic development but have long since abandoned the
field because of their inability to earn money on the drugs. ``We know
firsthand how broken the system is,'' he said.

\href{https://www.nytimes3xbfgragh.onion/2019/04/06/health/drug-resistant-candida-auris.html}{The
crisis} stems from the peculiar economics and biochemical quirks of
drugs that kill bacteria and fungi. The more often antimicrobial drugs
are used, the more likely they are to lose their efficacy as pathogens
survive and mutate. Efforts to promote antibiotic stewardship mean that
new drugs are used as a last resort, limiting the ability of companies
to earn back the billions of dollars it can take to create a new
product.

``It's been a really tough time for companies doing antibiotic discovery
despite the tremendous unmet need,'' said Zachary Zimmerman, the chief
executive of \href{https://forgetherapeutics.com/}{Forge Therapeutics},
a San Diego company that has several new drugs in the pipeline. He said
the fund would provide critical help for companies that have already
spent millions identifying an innovative compound but lack the money to
carry out the costly clinical trials needed to gain regulatory approval.
``A fund like this can really help us get through that valley of
death,'' Mr. Zimmerman said.

The collapse of the antibiotic market has dramatically reduced the
number of promising drugs. Between 1980 and 2009, the Food and Drug
Administration approved 61 new antibiotics for systemic use; over the
past decade that number has shrunk to 15, and a third of the companies
behind those medicines have since gone belly up. Those backing the fund
acknowledge that the effort is largely a stopgap measure. Industry
executives and public health experts say that fixing the broken
marketplace for antibiotics would require sweeping government
intervention to create financial incentives for drug companies,
including policy changes that would increase reimbursements for
lifesaving drugs kept under lock and key and used only when existing
therapies fail. Legislation that would address the problem has not
gained traction in recent years.

Drug-resistant infections kill 700,000 people a year across the globe,
\href{https://www.nytimes3xbfgragh.onion/2019/04/29/health/un-drug-resistance-antibiotics.html}{according
to the United Nations}, which has warned that the death toll could rise
to 10 million by 2050 without concerted action.

Dr. Peter Beyer, a senior adviser at the W.H.O. who led the effort to
create the new fund, said the threat of antimicrobial resistance rivaled
that of the coronavirus pandemic, but it was a slow-rolling crisis that
could feel abstract to political leaders focused on the next election
cycle.

``Hopefully this fund can bridge the gap until politicians realize the
urgency of antimicrobial resistance,'' he said.

Everly Macario, a public health expert at the University of Chicago
Medicine who focuses on antimicrobial resistance, understands how
abstract the threat can feel. In 2004, her 18-month- old son, Simon,
died from a drug-resistant staph infection within 24 hours of arriving
at a hospital emergency room with breathing difficulties.

``People think drug-resistant infections are something that affects
other people,'' she said. ``But one day, all of us, both young and old,
will need an antibiotic. A world in which antibiotics no longer work is
something that should terrify everyone.''

\textbf{\emph{{[}}\href{http://on.fb.me/1paTQ1h}{\emph{Like the Science
Times page on Facebook.}}} ****** \emph{\textbar{} Sign up for the}
\textbf{\href{http://nyti.ms/1MbHaRU}{\emph{Science Times
newsletter.}}\emph{{]}}}

Advertisement

\protect\hyperlink{after-bottom}{Continue reading the main story}

\hypertarget{site-index}{%
\subsection{Site Index}\label{site-index}}

\hypertarget{site-information-navigation}{%
\subsection{Site Information
Navigation}\label{site-information-navigation}}

\begin{itemize}
\tightlist
\item
  \href{https://help.nytimes3xbfgragh.onion/hc/en-us/articles/115014792127-Copyright-notice}{©~2020~The
  New York Times Company}
\end{itemize}

\begin{itemize}
\tightlist
\item
  \href{https://www.nytco.com/}{NYTCo}
\item
  \href{https://help.nytimes3xbfgragh.onion/hc/en-us/articles/115015385887-Contact-Us}{Contact
  Us}
\item
  \href{https://www.nytco.com/careers/}{Work with us}
\item
  \href{https://nytmediakit.com/}{Advertise}
\item
  \href{http://www.tbrandstudio.com/}{T Brand Studio}
\item
  \href{https://www.nytimes3xbfgragh.onion/privacy/cookie-policy\#how-do-i-manage-trackers}{Your
  Ad Choices}
\item
  \href{https://www.nytimes3xbfgragh.onion/privacy}{Privacy}
\item
  \href{https://help.nytimes3xbfgragh.onion/hc/en-us/articles/115014893428-Terms-of-service}{Terms
  of Service}
\item
  \href{https://help.nytimes3xbfgragh.onion/hc/en-us/articles/115014893968-Terms-of-sale}{Terms
  of Sale}
\item
  \href{https://spiderbites.nytimes3xbfgragh.onion}{Site Map}
\item
  \href{https://help.nytimes3xbfgragh.onion/hc/en-us}{Help}
\item
  \href{https://www.nytimes3xbfgragh.onion/subscription?campaignId=37WXW}{Subscriptions}
\end{itemize}
