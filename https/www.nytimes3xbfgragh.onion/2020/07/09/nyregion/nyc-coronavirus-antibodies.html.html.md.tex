Sections

SEARCH

\protect\hyperlink{site-content}{Skip to
content}\protect\hyperlink{site-index}{Skip to site index}

\href{https://www.nytimes3xbfgragh.onion/section/nyregion}{New York}

\href{https://myaccount.nytimes3xbfgragh.onion/auth/login?response_type=cookie\&client_id=vi}{}

\href{https://www.nytimes3xbfgragh.onion/section/todayspaper}{Today's
Paper}

\href{/section/nyregion}{New York}\textbar{}68\% Have Antibodies in This
Clinic. Can a Neighborhood Beat a Next Wave?

\begin{itemize}
\item
\item
\item
\item
\item
\item
\end{itemize}

\hypertarget{the-coronavirus-outbreak}{%
\subsubsection{\texorpdfstring{\href{https://www.nytimes3xbfgragh.onion/news-event/coronavirus?name=styln-coronavirus-national\&region=TOP_BANNER\&variant=undefined\&block=storyline_menu_recirc\&action=click\&pgtype=Article\&impression_id=3583c990-e38b-11ea-97c1-25b780906619}{The
Coronavirus
Outbreak}}{The Coronavirus Outbreak}}\label{the-coronavirus-outbreak}}

\begin{itemize}
\tightlist
\item
  live\href{https://www.nytimes3xbfgragh.onion/2020/08/20/world/coronavirus-covid.html?name=styln-coronavirus-national\&region=TOP_BANNER\&variant=undefined\&block=storyline_menu_recirc\&action=click\&pgtype=Article\&impression_id=3583c991-e38b-11ea-97c1-25b780906619}{Latest
  Updates}
\item
  \href{https://www.nytimes3xbfgragh.onion/interactive/2020/us/coronavirus-us-cases.html?name=styln-coronavirus-national\&region=TOP_BANNER\&variant=undefined\&block=storyline_menu_recirc\&action=click\&pgtype=Article\&impression_id=3583c992-e38b-11ea-97c1-25b780906619}{Maps
  and Cases}
\item
  \href{https://www.nytimes3xbfgragh.onion/interactive/2020/science/coronavirus-vaccine-tracker.html?name=styln-coronavirus-national\&region=TOP_BANNER\&variant=undefined\&block=storyline_menu_recirc\&action=click\&pgtype=Article\&impression_id=3583c993-e38b-11ea-97c1-25b780906619}{Vaccine
  Tracker}
\item
  \href{https://www.nytimes3xbfgragh.onion/2020/08/19/us/colleges-closing-covid.html?name=styln-coronavirus-national\&region=TOP_BANNER\&variant=undefined\&block=storyline_menu_recirc\&action=click\&pgtype=Article\&impression_id=3583c994-e38b-11ea-97c1-25b780906619}{Colleges
  Closing}
\item
  \href{https://www.nytimes3xbfgragh.onion/live/2020/08/20/business/stock-market-today-coronavirus?name=styln-coronavirus-national\&region=TOP_BANNER\&variant=undefined\&block=storyline_menu_recirc\&action=click\&pgtype=Article\&impression_id=3583c995-e38b-11ea-97c1-25b780906619}{Economy}
\end{itemize}

Advertisement

\protect\hyperlink{after-top}{Continue reading the main story}

Supported by

\protect\hyperlink{after-sponsor}{Continue reading the main story}

\hypertarget{68-have-antibodies-in-this-clinic-can-a-neighborhood-beat-a-next-wave}{%
\section{68\% Have Antibodies in This Clinic. Can a Neighborhood Beat a
Next
Wave?}\label{68-have-antibodies-in-this-clinic-can-a-neighborhood-beat-a-next-wave}}

Data from those tested at a storefront medical office in Queens is
leading to a deeper understanding of the outbreak's scope in New York.

\includegraphics{https://static01.graylady3jvrrxbe.onion/images/2020/07/06/nyregion/00nyvirus-antibody1/merlin_171942420_5217c6f3-d568-4aaa-bad1-664185bfadd5-articleLarge.jpg?quality=75\&auto=webp\&disable=upscale}

\href{https://www.nytimes3xbfgragh.onion/by/joseph-goldstein}{\includegraphics{https://static01.graylady3jvrrxbe.onion/images/2018/07/16/multimedia/author-joseph-goldstein/author-joseph-goldstein-thumbLarge.png}}

By \href{https://www.nytimes3xbfgragh.onion/by/joseph-goldstein}{Joseph
Goldstein}

\begin{itemize}
\item
  Published July 9, 2020Updated July 10, 2020
\item
  \begin{itemize}
  \item
  \item
  \item
  \item
  \item
  \item
  \end{itemize}
\end{itemize}

At a clinic in Corona, a working-class neighborhood in Queens, more than
68 percent of people tested positive for
\href{https://www.nytimes3xbfgragh.onion/2020/08/20/nyregion/nyc-coronavirus-antibody-testing.html}{antibodies
to the new coronavirus}. At another clinic in Jackson Heights, Queens,
that number was 56 percent. But at a clinic in Cobble Hill, a mostly
white and wealthy neighborhood in Brooklyn, only 13 percent of people
tested positive for antibodies.

As it has swept through
\href{https://www.nytimes3xbfgragh.onion/2020/07/23/nyregion/coronavirus-testing-nyc.html}{New
York}, the
\href{https://www.nytimes3xbfgragh.onion/2020/07/23/nyregion/coronavirus-testing-nyc.html}{coronavirus}
has exposed stark inequalities in nearly every aspect of city life, from
who has been most affected to
\href{https://www.nytimes3xbfgragh.onion/2020/07/01/nyregion/Coronavirus-hospitals.html}{how
the health care system cared} for those patients. Many lower-income
neighborhoods, where Black and Latino residents make up a large part of
the population, were hard hit, while many wealthy neighborhoods suffered
much less.

But now, as the city braces for a possible second wave of the virus,
some of those vulnerabilities may flip, with the affluent neighborhoods
becoming most at risk of a surge. According to antibody test results
from CityMD that were shared with The New York Times, some neighborhoods
were so exposed to the virus during the peak of the epidemic in March
and April that they might have some protection during a second wave.

``Some communities might have herd immunity,'' said Dr. Daniel Frogel, a
senior vice president for operations at CityMD, which plays a key role
in the city's testing program.

The CityMD statistics --- which Dr. Frogel provided during an interview
and which reflect tests done between late April and late June --- appear
to present the starkest picture yet of how infection rates have diverged
across neighborhoods in the city.

As of June 26, CityMD had administered about 314,000
\href{https://www.nytimes3xbfgragh.onion/2020/08/19/nyregion/new-york-city-antibody-test.html}{antibody
tests} in New York City. Citywide, 26 percent of the tests came back
positive.

But Dr. Frogel said the testing results in Jackson Heights and Corona
seemed to ``jump off the map.''

While stopping short of predicting that those neighborhoods would be
protected against a major new outbreak of the virus --- a
\href{https://www.nytimes3xbfgragh.onion/interactive/2020/05/28/upshot/coronavirus-herd-immunity.html}{phenomenon
known as herd immunity} --- several epidemiologists said that the
different levels of antibody prevalence across the city are likely to
play a role in what happens next, assuming that antibodies do in fact
offer significant protection against future infection.

\hypertarget{latest-updates-the-coronavirus-outbreak}{%
\section{\texorpdfstring{\href{https://www.nytimes3xbfgragh.onion/2020/08/20/world/coronavirus-covid.html?action=click\&pgtype=Article\&state=default\&region=MAIN_CONTENT_1\&context=storylines_live_updates}{Latest
Updates: The Coronavirus
Outbreak}}{Latest Updates: The Coronavirus Outbreak}}\label{latest-updates-the-coronavirus-outbreak}}

Updated 2020-08-21T07:46:15.883Z

\begin{itemize}
\tightlist
\item
  \href{https://www.nytimes3xbfgragh.onion/2020/08/20/world/coronavirus-covid.html?action=click\&pgtype=Article\&state=default\&region=MAIN_CONTENT_1\&context=storylines_live_updates\#link-68774d88}{Shutdowns,
  warnings and scoldings follow alarming incidents on college campuses.}
\item
  \href{https://www.nytimes3xbfgragh.onion/2020/08/20/world/coronavirus-covid.html?action=click\&pgtype=Article\&state=default\&region=MAIN_CONTENT_1\&context=storylines_live_updates\#link-26b58724}{Biden
  knocks Trump's pandemic response, and outlines a national strategy.}
\item
  \href{https://www.nytimes3xbfgragh.onion/2020/08/20/world/coronavirus-covid.html?action=click\&pgtype=Article\&state=default\&region=MAIN_CONTENT_1\&context=storylines_live_updates\#link-4e542da3}{U.S.
  health agencies announce moves to confront the flu season and
  plummeting child vaccination rates.}
\end{itemize}

\href{https://www.nytimes3xbfgragh.onion/2020/08/20/world/coronavirus-covid.html?action=click\&pgtype=Article\&state=default\&region=MAIN_CONTENT_1\&context=storylines_live_updates}{See
more updates}

More live coverage:
\href{https://www.nytimes3xbfgragh.onion/live/2020/08/20/business/stock-market-today-coronavirus?action=click\&pgtype=Article\&state=default\&region=MAIN_CONTENT_1\&context=storylines_live_updates}{Markets}

``In the future, the infection rate should really be lower in minority
communities,'' said Kitaw Demissie, an epidemiologist and the dean of
the School of Public Health at SUNY Downstate Medical Center in
Brooklyn.

Dr. Ted Long, the executive director of the city's contact-tracing
program, said that while much remained unknown about the strength and
duration of the protection that antibodies offer, he was hopeful that
hard-hit communities like Corona would have some degree of protection
because of their high rate of positive tests. ``We hope that that will
confer greater herd immunity,'' he said.

Neighborhoods that had relatively low infection rates --- and where few
residents have antibodies --- are especially vulnerable going forward.
There could be some degree of ``catch up'' among neighborhoods, said
Prof. Denis Nash, an epidemiology professor at the CUNY School of Public
Health.

But he added that even if infection rate were to climb in wealthier
neighborhoods, ``there are advantages to being in the neighborhoods that
are hit later.'' For one, doctors have become somewhat more adept at
treating severe cases.

\includegraphics{https://static01.graylady3jvrrxbe.onion/images/2020/07/06/nyregion/00nyvirus-antibody3/merlin_172855548_19fd9af8-0a17-4cd9-a582-7b86c39c3067-articleLarge.jpg?quality=75\&auto=webp\&disable=upscale}

Some epidemiologists and virologists cautioned that not enough data
exists to conclude that any areas have herd immunity. For starters, the
fact that 68.4 percent of tests taken at an urgent care center in Corona
came back positive does not mean that 68.4 percent of residents had been
infected.

``For sure, the persons who are seeking antibody testing probably have a
higher likelihood of being positive than the general population,'' said
Professor Nash. ``If you went out in Corona and tested a representative
sample, it wouldn't be 68 percent.''

So far, the federal government has released relatively little data from
antibody testing --- making the CityMD data all the more striking. The
Centers for Disease Control and Prevention, for instance,
\href{https://www.cdc.gov/coronavirus/2019-ncov/cases-updates/commercial-lab-surveys.html}{has
published} limited data that suggested that 6.93 percent of residents in
New York City and part of Long Island had antibodies. But that survey
was based on samples collected mainly in March, before many infected New
Yorkers might have developed antibodies.

New York State conducted a more comprehensive survey on antibody rates,
which involved testing some 28,419 people across the state. That survey
suggested that roughly 21.6 percent of New York City residents had
antibodies. But it also revealed a much higher rate in some
neighborhoods. While the state has released
\href{https://www.medrxiv.org/content/10.1101/2020.05.25.20113050v1.full.pdf}{little
data from Queens}, its numbers showed that in Flatbush, Brooklyn, for
example, about 45 percent of those tested had antibodies.

The CityMD data provides similar conclusions. At a location in Bushwick,
a Brooklyn neighborhood which has a large Hispanic population and where
the median household income is below the citywide average, some 35
percent of antibody tests were positive, according to Dr. Frogel.

Image

More than 56 percent of patients at one clinic in Jackson Heights,~
Queens, tested positive for coronavirus antibodies.~Credit...Juan
Arredondo for The New York Times

Dr. Frogel said that across the Bronx, which has had the city's
\href{https://www1.nyc.gov/site/doh/covid/covid-19-data-deaths.page}{highest
death rate} from Covid-19, about 37 percent of antibody tests were
turning up positive.

\href{https://www.nytimes3xbfgragh.onion/news-event/coronavirus?action=click\&pgtype=Article\&state=default\&region=MAIN_CONTENT_3\&context=storylines_faq}{}

\hypertarget{the-coronavirus-outbreak-}{%
\subsubsection{The Coronavirus Outbreak
›}\label{the-coronavirus-outbreak-}}

\hypertarget{frequently-asked-questions}{%
\paragraph{Frequently Asked
Questions}\label{frequently-asked-questions}}

Updated August 17, 2020

\begin{itemize}
\item ~
  \hypertarget{why-does-standing-six-feet-away-from-others-help}{%
  \paragraph{Why does standing six feet away from others
  help?}\label{why-does-standing-six-feet-away-from-others-help}}

  \begin{itemize}
  \tightlist
  \item
    The coronavirus spreads primarily through droplets from your mouth
    and nose, especially when you cough or sneeze. The C.D.C., one of
    the organizations using that measure,
    \href{https://www.nytimes3xbfgragh.onion/2020/04/14/health/coronavirus-six-feet.html?action=click\&pgtype=Article\&state=default\&region=MAIN_CONTENT_3\&context=storylines_faq}{bases
    its recommendation of six feet} on the idea that most large droplets
    that people expel when they cough or sneeze will fall to the ground
    within six feet. But six feet has never been a magic number that
    guarantees complete protection. Sneezes, for instance, can launch
    droplets a lot farther than six feet,
    \href{https://jamanetwork.com/journals/jama/fullarticle/2763852}{according
    to a recent study}. It's a rule of thumb: You should be safest
    standing six feet apart outside, especially when it's windy. But
    keep a mask on at all times, even when you think you're far enough
    apart.
  \end{itemize}
\item ~
  \hypertarget{i-have-antibodies-am-i-now-immune}{%
  \paragraph{I have antibodies. Am I now
  immune?}\label{i-have-antibodies-am-i-now-immune}}

  \begin{itemize}
  \tightlist
  \item
    As of right
    now,\href{https://www.nytimes3xbfgragh.onion/2020/07/22/health/covid-antibodies-herd-immunity.html?action=click\&pgtype=Article\&state=default\&region=MAIN_CONTENT_3\&context=storylines_faq}{that
    seems likely, for at least several months.} There have been
    frightening accounts of people suffering what seems to be a second
    bout of Covid-19. But experts say these patients may have a
    drawn-out course of infection, with the virus taking a slow toll
    weeks to months after initial exposure. People infected with the
    coronavirus typically
    \href{https://www.nature.com/articles/s41586-020-2456-9}{produce}
    immune molecules called antibodies, which are
    \href{https://www.nytimes3xbfgragh.onion/2020/05/07/health/coronavirus-antibody-prevalence.html?action=click\&pgtype=Article\&state=default\&region=MAIN_CONTENT_3\&context=storylines_faq}{protective
    proteins made in response to an
    infection}\href{https://www.nytimes3xbfgragh.onion/2020/05/07/health/coronavirus-antibody-prevalence.html?action=click\&pgtype=Article\&state=default\&region=MAIN_CONTENT_3\&context=storylines_faq}{.
    These antibodies may} last in the body
    \href{https://www.nature.com/articles/s41591-020-0965-6}{only two to
    three months}, which may seem worrisome, but that's perfectly normal
    after an acute infection subsides, said Dr. Michael Mina, an
    immunologist at Harvard University. It may be possible to get the
    coronavirus again, but it's highly unlikely that it would be
    possible in a short window of time from initial infection or make
    people sicker the second time.
  \end{itemize}
\item ~
  \hypertarget{im-a-small-business-owner-can-i-get-relief}{%
  \paragraph{I'm a small-business owner. Can I get
  relief?}\label{im-a-small-business-owner-can-i-get-relief}}

  \begin{itemize}
  \tightlist
  \item
    The
    \href{https://www.nytimes3xbfgragh.onion/article/small-business-loans-stimulus-grants-freelancers-coronavirus.html?action=click\&pgtype=Article\&state=default\&region=MAIN_CONTENT_3\&context=storylines_faq}{stimulus
    bills enacted in March} offer help for the millions of American
    small businesses. Those eligible for aid are businesses and
    nonprofit organizations with fewer than 500 workers, including sole
    proprietorships, independent contractors and freelancers. Some
    larger companies in some industries are also eligible. The help
    being offered, which is being managed by the Small Business
    Administration, includes the Paycheck Protection Program and the
    Economic Injury Disaster Loan program. But lots of folks have
    \href{https://www.nytimes3xbfgragh.onion/interactive/2020/05/07/business/small-business-loans-coronavirus.html?action=click\&pgtype=Article\&state=default\&region=MAIN_CONTENT_3\&context=storylines_faq}{not
    yet seen payouts.} Even those who have received help are confused:
    The rules are draconian, and some are stuck sitting on
    \href{https://www.nytimes3xbfgragh.onion/2020/05/02/business/economy/loans-coronavirus-small-business.html?action=click\&pgtype=Article\&state=default\&region=MAIN_CONTENT_3\&context=storylines_faq}{money
    they don't know how to use.} Many small-business owners are getting
    less than they expected or
    \href{https://www.nytimes3xbfgragh.onion/2020/06/10/business/Small-business-loans-ppp.html?action=click\&pgtype=Article\&state=default\&region=MAIN_CONTENT_3\&context=storylines_faq}{not
    hearing anything at all.}
  \end{itemize}
\item ~
  \hypertarget{what-are-my-rights-if-i-am-worried-about-going-back-to-work}{%
  \paragraph{What are my rights if I am worried about going back to
  work?}\label{what-are-my-rights-if-i-am-worried-about-going-back-to-work}}

  \begin{itemize}
  \tightlist
  \item
    Employers have to provide
    \href{https://www.osha.gov/SLTC/covid-19/standards.html}{a safe
    workplace} with policies that protect everyone equally.
    \href{https://www.nytimes3xbfgragh.onion/article/coronavirus-money-unemployment.html?action=click\&pgtype=Article\&state=default\&region=MAIN_CONTENT_3\&context=storylines_faq}{And
    if one of your co-workers tests positive for the coronavirus, the
    C.D.C.} has said that
    \href{https://www.cdc.gov/coronavirus/2019-ncov/community/guidance-business-response.html}{employers
    should tell their employees} -\/- without giving you the sick
    employee's name -\/- that they may have been exposed to the virus.
  \end{itemize}
\item ~
  \hypertarget{what-is-school-going-to-look-like-in-september}{%
  \paragraph{What is school going to look like in
  September?}\label{what-is-school-going-to-look-like-in-september}}

  \begin{itemize}
  \tightlist
  \item
    It is unlikely that many schools will return to a normal schedule
    this fall, requiring the grind of
    \href{https://www.nytimes3xbfgragh.onion/2020/06/05/us/coronavirus-education-lost-learning.html?action=click\&pgtype=Article\&state=default\&region=MAIN_CONTENT_3\&context=storylines_faq}{online
    learning},
    \href{https://www.nytimes3xbfgragh.onion/2020/05/29/us/coronavirus-child-care-centers.html?action=click\&pgtype=Article\&state=default\&region=MAIN_CONTENT_3\&context=storylines_faq}{makeshift
    child care} and
    \href{https://www.nytimes3xbfgragh.onion/2020/06/03/business/economy/coronavirus-working-women.html?action=click\&pgtype=Article\&state=default\&region=MAIN_CONTENT_3\&context=storylines_faq}{stunted
    workdays} to continue. California's two largest public school
    districts --- Los Angeles and San Diego --- said on July 13, that
    \href{https://www.nytimes3xbfgragh.onion/2020/07/13/us/lausd-san-diego-school-reopening.html?action=click\&pgtype=Article\&state=default\&region=MAIN_CONTENT_3\&context=storylines_faq}{instruction
    will be remote-only in the fall}, citing concerns that surging
    coronavirus infections in their areas pose too dire a risk for
    students and teachers. Together, the two districts enroll some
    825,000 students. They are the largest in the country so far to
    abandon plans for even a partial physical return to classrooms when
    they reopen in August. For other districts, the solution won't be an
    all-or-nothing approach.
    \href{https://bioethics.jhu.edu/research-and-outreach/projects/eschool-initiative/school-policy-tracker/}{Many
    systems}, including the nation's largest, New York City, are
    devising
    \href{https://www.nytimes3xbfgragh.onion/2020/06/26/us/coronavirus-schools-reopen-fall.html?action=click\&pgtype=Article\&state=default\&region=MAIN_CONTENT_3\&context=storylines_faq}{hybrid
    plans} that involve spending some days in classrooms and other days
    online. There's no national policy on this yet, so check with your
    municipal school system regularly to see what is happening in your
    community.
  \end{itemize}
\end{itemize}

The CityMD in Corona, on Junction Boulevard, serves a predominantly
Hispanic neighborhood whose residents include many construction workers
and restaurant employees. Many had to work throughout the pandemic,
raising their risk of infection.

Angela Rasmussen, a virologist at Columbia University, called the high
positive rate in Corona ``a stunning finding.'' Epidemiologists said the
rate showed the limits of New York's strategy in curtailing the virus:
While public health measures may have slowed the spread in some
neighborhoods, they did far less for others.

There are reasons parts of Queens
\href{https://www.nytimes3xbfgragh.onion/2020/04/09/nyregion/coronavirus-queens-corona-jackson-heights-elmhurst.html}{were
hit so hard}. Homes in Elmhurst and parts of Corona are
\href{https://www.nytimes3xbfgragh.onion/interactive/2019/10/23/nyregion/basements-queens-immigrants.html}{especially
crowded} --- the
\href{https://www.nytimes3xbfgragh.onion/2016/03/01/nyregion/overcrowding-worsens-in-new-york-as-working-families-double-up.html}{highest
rate of household crowding} in the city, according to census bureau data
from 2014. Given that
\href{https://www.who.int/docs/default-source/coronaviruse/who-china-joint-mission-on-covid-19-final-report.pdf}{transmission
among family members} is a leading driver of the disease's spread, it is
unsurprising that
\href{https://www.wsj.com/articles/covid-19-households-spread-coronavirus-families-navajo-california-second-wave-11591553896}{crowded
households} have been associated
\href{https://www.cuimc.columbia.edu/news/crowded-homes-poor-neighborhoods-linked-covid-19}{with
higher risk of infection}.

For residents of Corona, the main sources of employment are jobs in
hospitality, including restaurants, as well as construction and
manufacturing,
\href{https://www.cccnewyork.org/wp-content/uploads/2019/12/FINAL-CCC-Elmhurst-Corona_SinglePages_1216.pdf}{according
to a 2019} report by the Citizens' Committee for Children of New York.
Many construction workers and restaurant employees showed up to work
throughout the pandemic, elevating their risk of infection.

``Our plan did not really accommodate essential workers as it did people
privileged enough --- for lack of a better word --- to socially distance
themselves,'' Professor Nash said. He said that one lesson of the past
few months was that the city needed to better protect essential workers
--- everyone from grocery store employees to pharmacy cashiers --- and
make sure they had sufficient protective equipment.

Epidemiologists have estimated that at least 60 percent of a population
--- and perhaps as much as 80 percent --- would need immunity
\href{https://www.nytimes3xbfgragh.onion/interactive/2020/05/28/upshot/coronavirus-herd-immunity.html}{before
``herd immunity'' is reached}, and the virus can no longer spread widely
in that community.

But scientists say it would be a mistake to base public health decisions
off antibody rates across a population.

``Just looking at seroprevalence alone can't really be used to make
actionable public health decisions,'' Dr. Rasmussen, the virologist at
Columbia, said.

One reason is that the accuracy of the antibody tests is not fully
known, nor is the extent of immunity conferred by antibodies or how long
that immunity lasts. Dr. Rasmussen noted that the ``magical number of 60
percent for herd immunity'' assumes that everyone infected has complete
protection from a second infection. ``But what about people with partial
protection?'' she asked. ``They may not get sick, but they can get
infected and pass it along.''

``It is premature to discuss herd immunity, since we are still learning
what the presence of Covid-19 antibodies means to an individual and
whether, or for how long, that conveys immunity; and we don't know how
the level of immunity in a single community translates into herd
immunity,'' said Jonah Bruno, a spokesman for the state Department of
Health.

He said he was unsurprised by the high rate in Corona, and senior
officials with the city's contact-tracing program and public hospital
system agree. ``We know this area was disproportionately affected,''
said Dr. Andrew Wallach, a senior official in the city's public hospital
system, ``so this just confirms what we've seen clinically.''

Advertisement

\protect\hyperlink{after-bottom}{Continue reading the main story}

\hypertarget{site-index}{%
\subsection{Site Index}\label{site-index}}

\hypertarget{site-information-navigation}{%
\subsection{Site Information
Navigation}\label{site-information-navigation}}

\begin{itemize}
\tightlist
\item
  \href{https://help.nytimes3xbfgragh.onion/hc/en-us/articles/115014792127-Copyright-notice}{©~2020~The
  New York Times Company}
\end{itemize}

\begin{itemize}
\tightlist
\item
  \href{https://www.nytco.com/}{NYTCo}
\item
  \href{https://help.nytimes3xbfgragh.onion/hc/en-us/articles/115015385887-Contact-Us}{Contact
  Us}
\item
  \href{https://www.nytco.com/careers/}{Work with us}
\item
  \href{https://nytmediakit.com/}{Advertise}
\item
  \href{http://www.tbrandstudio.com/}{T Brand Studio}
\item
  \href{https://www.nytimes3xbfgragh.onion/privacy/cookie-policy\#how-do-i-manage-trackers}{Your
  Ad Choices}
\item
  \href{https://www.nytimes3xbfgragh.onion/privacy}{Privacy}
\item
  \href{https://help.nytimes3xbfgragh.onion/hc/en-us/articles/115014893428-Terms-of-service}{Terms
  of Service}
\item
  \href{https://help.nytimes3xbfgragh.onion/hc/en-us/articles/115014893968-Terms-of-sale}{Terms
  of Sale}
\item
  \href{https://spiderbites.nytimes3xbfgragh.onion}{Site Map}
\item
  \href{https://help.nytimes3xbfgragh.onion/hc/en-us}{Help}
\item
  \href{https://www.nytimes3xbfgragh.onion/subscription?campaignId=37WXW}{Subscriptions}
\end{itemize}
