Sections

SEARCH

\protect\hyperlink{site-content}{Skip to
content}\protect\hyperlink{site-index}{Skip to site index}

\href{/section/opinion/sunday}{Sunday Review}\textbar{}I've Seen a
Future Without Cars, and It's Amazing

\url{https://nyti.ms/2DBIUGp}

\begin{itemize}
\item
\item
\item
\item
\item
\item
\end{itemize}

\hypertarget{the-space-cars-take}{%
\subsection{The space cars take}\label{the-space-cars-take}}

The island of Manhattan is about 23 square miles.

Much of it is covered in \textbf{roadway, street parking and parking
garages}.

If you added up all the space Manhattan devotes to cars, you'd have an
area nearly four times as large as Central Park.

What if we gave that space back to New Yorkers?

Source: Practice for Architecture and Urbanism

Supported by

\protect\hyperlink{after-sponsor}{Continue reading the main story}

\hypertarget{ive-seen-a-future-without-cars-and-its-amazing}{%
\section{I've Seen a Future Without Cars, and It's
Amazing}\label{ive-seen-a-future-without-cars-and-its-amazing}}

Why do American cities waste so much space on cars?

\href{https://www.nytimes3xbfgragh.onion/by/farhad-manjoo}{\includegraphics{https://static01.graylady3jvrrxbe.onion/images/2019/01/08/opinion/farhad-manjoo-opinion/farhad-manjoo-opinion-thumbLarge.png}}

By \href{https://www.nytimes3xbfgragh.onion/by/farhad-manjoo}{Farhad
Manjoo}

Opinion Columnist

\begin{itemize}
\item
  July 9, 2020
\item
  \begin{itemize}
  \item
  \item
  \item
  \item
  \item
  \item
  \end{itemize}
\end{itemize}

As coronavirus lockdowns crept across the globe this winter and spring,
an unusual sound fell over the world's metropolises: the hush of streets
that were suddenly, blessedly free of cars. City dwellers reported
hearing\href{https://www.nytimes3xbfgragh.onion/interactive/2020/05/22/upshot/coronavirus-quiet-city-noise.html}{bird
song, wind and the rustling of leaves}. (Along with, in New York City,
the intermittent screams of sirens.)

You could smell the absence of cars, too. From
\href{https://www.nydailynews.com/coronavirus/ny-corionavirus-environmental-benefit-unlikely-to-last-20200406-vx5v3dn6evhbhdjdmarcyihleu-story.html}{New
York} to
\href{https://www.lamag.com/citythinkblog/air-quality-covid/}{Los
Angeles} to New Delhi, air pollution plummeted, and the soupy,
exhaust-choked haze over the world's dirtiest cities
\href{https://www.theguardian.com/environment/2020/apr/11/positively-alpine-disbelief-air-pollution-falls-lockdown-coronavirus}{lifted
to reveal brilliant blue skies}.

\includegraphics{https://static01.graylady3jvrrxbe.onion/images/2020/07/10/autossell/17Manjoo_Twitter_thumb/17Manjoo_Twitter_thumb-videoSixteenByNineJumbo1600.jpg}

Cars took a break from killing people, too.
\href{https://www.nytimes3xbfgragh.onion/2020/03/10/nyregion/nyc-deaths-pedestrian-cycling.html}{About
10 pedestrians} die on New York City's streets in an ordinary month.
Under lockdown, the city went a record
\href{https://www.cnn.com/2020/05/13/us/new-york-city-pedestrian-deaths/index.html}{two
months without a single pedestrian fatality}. In California, vehicle
collisions plummeted 50 percent, reducing accidents resulting in
injuries or death by
\href{https://roadecology.ucdavis.edu/files/content/projects/COVID_CHIPs_Impacts_updated_415.pdf}{about
6,000 per month}.

As the roads became freer of cars, they grew full of possibility.
\href{https://www.ktvu.com/news/rollerskaters-rejoice-throwback-activity-has-revival-during-pandemic}{Rollerblading}
and
\href{https://shop-eat-surf.com/2020/03/skateboard-sales-jump-during-crisis/}{skateboarding}
have come back into fashion. Sales of
\href{https://www.nytimes3xbfgragh.onion/2020/05/18/nyregion/bike-shortage-coronavirus.html}{bicycles}
and
\href{https://electrek.co/2020/05/01/electric-bike-sales-skyrocket-during-lockdown/}{electric
bikes} have skyrocketed.

But there is a catch: Cities are beginning to cautiously open back up
again, and people are wondering how they're going to get in to work.
Many are worried about the spread of the virus on public transit. Are
cars our only option? How will we find space for all of them?

In much of Manhattan, the average speed of traffic before the pandemic
had fallen to 7 miles per hour. In Midtown, it was less than 5 m.p.h.
That's only slightly faster than walking and slower than riding a bike.
Will traffic soon be worse than ever?

Not if we choose another path.

Rather than stumble back into car dependency, cities can begin to undo
\href{https://www.washingtonpost.com/news/in-theory/wp/2016/02/29/the-car-century-was-a-mistake-its-time-to-move-on/}{their
worst mistake}:
\href{https://www.newyorker.com/magazine/2019/07/29/was-the-automotive-era-a-terrible-mistake}{giving
up} so much of their land to the automobile.

The pandemic should not stop us. There is
\href{https://www.theatlantic.com/ideas/archive/2020/06/fear-transit-bad-cities/612979/}{little
evidence} that public transit is responsible for the spread of the
coronavirus in New York or elsewhere; some cities with heavily used
transit systems, including Hong Kong, have been able to avoid terrible
tolls from the virus.

If riders wear face masks --- and if there are enough subway cars,
buses, bike lanes and pedestrian paths for people to avoid intense
overcrowding --- transit might be no less safe than cars, in terms of
the risk of the spread of disease.
\href{https://mobilitylab.org/2016/09/08/transit-10-times-safer-driving-makes-communities-safer-says-new-apta-report/}{In
all other measures of safety,} transit is far safer than cars.

What's that you say? There \emph{aren't} enough buses in your city to
avoid overcrowding, and they're too slow, anyway? Pedestrian space is
\emph{already} hard to find? Well, right. That's car dependency. And
it's exactly why cities need to plan for a future of fewer cars, a
future in which owning an automobile, even an electric one, is neither
the only way nor the best way to get around town.

A few weeks ago, I began talking to Vishaan Chakrabarti, a former New
York City urban-planning official and the founder of Practice for
Architecture and Urbanism, a Manhattan-based architecture firm. Like
\href{https://www.nytimes3xbfgragh.onion/2020/06/20/opinion/pandemic-automobile-cities.html?action=click\&module=Opinion\&pgtype=Homepage}{other
urbanists}, Chakrabarti believes that the pandemic has created an
opportunity for New York and other cities to reduce their reliance on
cars.

Manhattan, already one of the most car-free places in the country, is
the best place to start. Chakrabarti's firm, known as PAU, had been
working on an intricate proposal to show what it might look and feel
like to live in a city liberated from cars, to show how much better life
in New York might be with one simple change: Most cars would be banished
from Manhattan.

PAU's proposal would not ban all motor vehicles, just privately owned
cars. There would still be delivery trucks, paratransit, emergency
vehicles, and taxicabs and rideshare cars, if you needed them.

But private cars account for so many of Manhattan's vehicles that
banning them would instantly improve life for just about everyone who
lives and works in New York.

\hypertarget{delancey-street}{%
\subsection{Delancey Street}\label{delancey-street}}

In parts of downtown, pedestrians have to cross wide roads designed to
carry traffic from the Williamsburg and Manhattan Bridges.

In a car-free world, the city could expand sidewalks to give those
pedestrians more space.

Two-way bike lanes could replace car lanes in both directions. A
concrete barrier would protect bikers.

Dedicated bus lanes, free of car traffic, would efficiently shuttle
people in and out of Manhattan and relieve congestion on the subway
system.

Source: Practice for Architecture and Urbanism

You already know what's terrible about cars: They're dirty. They're
dangerous. They're expensive to buy and maintain, and environmentally
hazardous to produce and operate. Automobiles kill around 90,000
Americans every year
---~\href{https://ohsonline.com/articles/2019/02/18/nsc-motor-vehicle-deaths.aspx\#:~:text=National\%20Safety\%20Council.-,NSC\%20estimates\%20that\%20in\%202018\%2C\%2040\%2C000\%20people\%20died\%20in\%20car,percent\%20decrease\%20from\%202017\%20figures.}{about
40,000}in car accidents, and an estimated 50,000 more from long-term
exposure to
\href{https://www.sciencedirect.com/science/article/abs/pii/S1352231013004548}{air
pollution emitted by cars}.

But Chakrabarti is among a group of urbanists who've been calling
attention to a less-discussed problem with cars. Automobiles are not
just dangerous and bad for the environment; they are also profoundly
wasteful of the land around us, taking up way too much physical space to
transport too few people. It's geometry.

In most American cities, wherever you look, you will see
\href{https://oldurbanist.blogspot.com/2011/12/we-are-25-looking-at-street-area.html}{a
landscape constructed primarily for the movement and storage of
automobiles}, not for the enjoyment of people: endless wide boulevards
and freeways for cars to move swiftly; each road lined with parking
spaces for cars at rest; retail establishments ringed with spots for
cars; houses built around garages for cars; and a gas station, for cars
to feed, on every other corner.

In the most car-dependent cities, the amount of space devoted to
automobiles reaches truly ridiculous levels. In Los Angeles, for
instance, land for parking
\href{https://la.curbed.com/2018/11/30/18119646/los-angeles-parking-lots-total-size-development}{exceeds
the entire land area of Manhattan}, enough space to house almost a
million more people at Los Angeles's prevailing density.

This isn't a big deal in the parts of America where space is seemingly
endless. But in the most populated cities, physical space is just about
the most precious resource there is. The land value of Manhattan alone
is
\href{https://www.bloomberg.com/news/articles/2018-04-24/manhattan-s-land-value-is-an-incredible-1-74-trillion}{estimated
to top \$1.7 trillion}. Why are we giving so much of it to cars?

Without cars, Manhattan's streets could give priority to more equitable
and accessible ways of getting around, including an extensive system of
bike ``superhighways'' and
\href{https://www.itdp.org/library/standards-and-guides/the-bus-rapid-transit-standard/what-is-brt/}{bus
rapid transit} --- a bus system with dedicated lanes in the roadway,
creating a service that approaches the capacity, speed and efficiency of
the subway, at a fraction of the cost.

Eliminating most cars in Manhattan would also significantly clean up the
air for the entire region. It would free up space for new housing and
create hundreds of acres of new parks and pedestrian promenades,
improving the fundamental health, beauty and livability of America's
largest metropolis.

There have been
\href{https://www.dissentmagazine.org/article/banning-cars-from-manhattan}{several}
\href{https://ny.curbed.com/2019/10/3/20896837/nyc-traffic-history-pedestrian-cars}{proposals}
to ban cars in Manhattan, and the city has been working on a system to
impose a toll on cars south of 60th Street. (This congestion-pricing
project was scheduled to start early next year, but it has been
\href{https://www.nydailynews.com/coronavirus/ny-coronavirus-congestion-pricing-on-hold-20200422-idyclofw4nbnnbc3ftbhjpjfpm-story.html}{delayed
by the pandemic}.)

What distinguishes PAU's proposal is its visual appeal. Chakrabarti says
his firm aimed to show, at a street level, how much better life without
cars might be for most New Yorkers. ``This is an amazing way to live,''
he said.

\hypertarget{residential-streets}{%
\subsection{Residential streets}\label{residential-streets}}

Parking spots and piles of trash dominate much of the space on a typical
residential street in Manhattan.

Eliminating parking would create space for large trash receptacles and
more bike lanes. Additional crosswalks would make it easier for people
to safely cross the street.

Source: Practice for Architecture and Urbanism

Any proposal to ban cars had better look amazing, because in America,
the automobile has never been just a way of getting from A to B. More
than a century of car ads and a good deal of hagiographic cultural
propaganda has done a job on a lot of us. For many Americans, cars are
not just a consumer product but a rite of passage, a symbol of national
pride, and an expression of liberty nearly as fundamental as anything
promised in the Bill of Rights.

I know, because I, too, have long loved cars. I love them viscerally,
the way a dog loves a bone, or an Instagrammer loves a sunset, and I am
as surprised as anyone to be calling for their eradication from cities.

As a teenager growing up in Southern California, America's center of car
culture, I spent endless hours lusting after the vehicles in car
magazines; these days my appetites are whetted digitally, with
\href{https://www.youtube.com/channel/UCsqjHFMB_JYTaEnf_vmTNqg}{ridiculously
detailed car-review videos} on YouTube. My current ride is a car that
only European automobile nerds would appreciate: an apple-red Volkswagen
Golf R, a ``hot hatch'' that does 0 to 60
\href{https://youtu.be/meSVfbFEwH8}{in under five environmentally
disastrous seconds}, which I bought only because driving it very fast
touched me in unmentionable places.

Yet when I got my speedy ride, I quickly realized it was kind of
pointless, because most of the time there's too much traffic where I
live to go any faster than a golf cart. This is the drab reality of
driving you'll never see in car ads ---~a daily, rage-inducing grind of
traffic, parking and shelling out to fill up; an option that many people
choose not for any love affair with cars, but often because driving is
the least-inconvenient way of getting around where they live and work.

I was receptive to Chakrabarti's proposal because in the last few years,
I've grown increasingly disillusioned about America's tolerance for the
public health and environmental damage caused by cars, not to mention
the frustrations of commuting by car. And I'm losing hope that the car
industry will be able to fix the damage anytime soon.

I've spent much of the last decade watching Silicon Valley take on that
industry, and I once had great expectations that techies would soon make
cars substantially cleaner, safer, more efficient, more convenient and
cheaper to operate.

But many of their innovations are turning into a bust --- or, at the
very least, are not making enough of a difference. Uber and Lyft once
promised to reduce traffic through car-pooling. In fact,
\href{https://www.theverge.com/2019/8/6/20756945/uber-lyft-tnc-vmt-traffic-congestion-study-fehr-peers}{ride-hailing
services have greatly worsened traffic} in many big cities.

Tesla turned the electric car into a mainstream object of lust ---~but
most of the rest of the auto industry is struggling
\href{https://www.marketwatch.com/story/teslas-competitors-find-that-going-electric-has-its-own-set-of-problems-2019-11-04}{to
get consumers to switch over} from gas, so it could take
\href{http://energyfuse.org/americas-aging-vehicles-delay-rate-fleet-turnover/}{15
years or more} to electrify America's entire fleet. The largest
automakers still make
\href{https://www.nytimes3xbfgragh.onion/2019/09/12/business/suv-sedan-detroit-fight.html}{most
of their profits} from
\href{https://www.curbed.com/2018/9/27/17909270/pedestrian-deaths-suv-car-design}{dangerous},
gas-guzzling S.U.V.s that will be on the roads for
\href{http://energyfuse.org/americas-aging-vehicles-delay-rate-fleet-turnover/}{years
to come}, and automakers continue to
\href{https://www.latimes.com/business/story/2019-10-31/epa-trump-emissions-lawsuit}{mount
aggressive legal and lobbying campaigns} against mileage standards.

Electric cars are no environmental panacea ---~they are more efficient
than gas-powered cars, but they still consume
\href{https://www.latimes.com/business/la-fi-electric-car-cobalt-battery-20180222-story.html}{a
lot of resources to produce}, and if they result in people driving more,
\href{https://www.bloomberg.com/news/articles/2019-09-23/electric-vehicles-alone-won-t-stop-climate-change}{they
may not greatly reduce overall emissions}.

Then there's the accident-free, self-driving car ---~the auto industry's
holy grail. Don't hold your breath: The dream is
\href{https://www.investors.com/news/self-driving-cars-hit-delays-driverless-cars-timeline/}{proving
to be far trickier} than many carmakers imagined, and
\href{https://www.nytimes3xbfgragh.onion/interactive/2019/10/04/opinion/self-driving-cars-safety.html}{cars
will remain reliably deadly} for years to come.

When he wanted to underscore the unexpected nature of invention, Steve
Jobs was fond of using a version of a line widely attributed to Henry
Ford: ``If I'd asked customers what they wanted, they would have told
me, `A faster horse!''' Silicon Valley's collective quest for a better
car has begun to look similarly narrow: What if Ubers and Teslas are
just faster horses --- and what if the real way to revolutionize
transportation is to think beyond the car entirely?

A more straightforward campaign against the automobile has been winning
results around the world. This is a movement by urban planners,
community groups and far-thinking elected officials to reduce the amount
of land cars occupy.

The effort has resulted in the wresting of major tracts of land away
from cars in some of the world's largest cities. Late in Michael
Bloomberg's tenure as mayor, \href{http://www.jsadikkhan.com/}{Janette
Sadik-Khan}, the transportation commissioner, pedestrianized large
sections of New York City,
\href{https://ny.curbed.com/2017/4/19/15358234/times-square-snohetta-before-after-photos}{including
Times Square}, and created
\href{https://www.bicycling.com/news/a20006501/this-woman-built-400-miles-of-bike-lanes-in-new-york-city/}{hundreds
of miles of new bike lanes}. Last year, the city banned cars from part
of 14th Street in Manhattan, resulting in
\href{https://www.ny1.com/nyc/all-boroughs/news/2019/10/18/14th-street-busway-is-making-trips-faster--encouraging-more-trips}{faster
crosstown bus service}.

\href{https://www.sfchronicle.com/bayarea/article/Car-free-Market-What-happens-to-the-side-14999923.php}{Market
Street in San Francisco has been turned into a car-free promenade}. And
in Paris, Mayor Anne Hidalgo has made taking away land from cars the
centerpiece of her politics, and it's working. Traffic in Paris has
fallen by
\href{https://www.bloomberg.com/news/articles/2018-01-19/how-paris-shifted-away-from-the-car}{40
percent in the last decade}; last month,
\href{https://www.newstatesman.com/world/europe/2020/06/how-anne-hidalgos-anti-car-policies-won-her-re-election-paris}{Hidalgo
handily won re-election}.

\hypertarget{manhattan-reimagined}{%
\subsection{Manhattan reimagined}\label{manhattan-reimagined}}

\hypertarget{manhattan-reimagined-1}{%
\subsubsection{Manhattan reimagined}\label{manhattan-reimagined-1}}

\hypertarget{how-communities-might-redesign-various-types-of-streets}{%
\paragraph{How communities might redesign various types of
streets.}\label{how-communities-might-redesign-various-types-of-streets}}

Residential streets

46th Street in Hell's Kitchen

Mid-block crossing

Recycling and

waste pickup

Social services

Commercial streets

50th Street in Midtown

Two-way protected bike lane

Rideshare and

taxi drop-off

Sidewalk

expansion

Crosstown arterials

125th Street in Harlem

Street vendors

Dedicated bus lanes

Bus stop

Mid-block pedestrian crossing

Residential streets like 46th Street in Hell's Kitchen

Recycling and waste pickup

Social services

Two-way protected bike lane

Commercial streets like 50th Street in Midtown

Taxi and rideshare drop-off

Sidewalk expansion

Street vendors

Crosstown arterials like 125th Street in Harlem

Dedicated bus lanes

Bus stop

Source: Practice for Architecture and Urbanism

It's good urban policy, but it's also a matter of equity and justice.
Chakrabarti often refers to a concept he calls
``\href{https://twitter.com/vishaannyca/status/1180833269367136264}{street
equity}.''

Imagine you'd like to transport 50 people from one end of Manhattan to
the other. If you were to send them by bus, you could stuff everyone in
a single bus car --- taking up around 450 square feet of road space,
about the size of a tiny studio apartment. But if you were going to send
50 people by automobile, you'd need a lot more road. For 50 people, each
driving alone, you'd need 2,750 square feet of space ---~ basically a
McMansion of roadway to transport 50 fat cats.

\hypertarget{what-does-it-take-to-move-50-people}{%
\subsection{What does it take to move 50
people?}\label{what-does-it-take-to-move-50-people}}

\hypertarget{what-does-it-take-to-move-50-people-1}{%
\subsubsection{What does it take to move 50
people?}\label{what-does-it-take-to-move-50-people-1}}

50 cars

55 square feet per person

One bus

9 square feet per person

50 bicycles

15 square feet per person

50 cars

55 sq. ft. per person

50 bicycles

15 sq. ft. per person

One bus

9 sq. ft. per person

Source: Practice for Architecture and Urbanism

And cars take up space even while they're not in use. They need to be
parked, which consumes yet more space on the sides of streets or in
garages. Cars take up a lot of space even when they're
just\href{http://shoup.luskin.ucla.edu/wp-content/uploads/sites/2/2015/02/CruisingForParkingAccess.pdf}{}\href{http://shoup.luskin.ucla.edu/wp-content/uploads/sites/2/2015/02/CruisingForParkingAccess.pdf}{\emph{looking}}\href{http://shoup.luskin.ucla.edu/wp-content/uploads/sites/2/2015/02/CruisingForParkingAccess.pdf}{for
parking}.

Add it all up and you get a huge number: In addition to the 2,450 acres
of roadway in Manhattan, nearly 1,000 more acres --- an area about the
size of Central Park --- is occupied by parking garages, gas stations,
carwashes, car dealerships and auto repair shops. There is three times
more roadway for cars on Manhattan as there is for bikes. There's more
road for cars than there is sidewalk for pedestrians.

\hypertarget{park-avenue}{%
\subsection{Park Avenue}\label{park-avenue}}

Cars have a way of gobbling up urban space.

Look at Park Avenue. When it was constructed in the early 20th century,
it was true to its name --- a large park ran down its center.

Over the years, much of the park was converted to roads for cars. Now
just a small median remains.

A redesigned Park Avenue could reclaim its former glory, with a large
pedestrian promenade winding down the commercial corridor.

Source: Practice for Architecture and Urbanism

The amount of space devoted to cars in Manhattan is not just wasteful,
but, in a deeper sense, also unfair to the millions of New Yorkers who
have no need for cars.

\href{http://blog.tstc.org/2017/04/21/car-free-new-york-city/}{More than
half} of the city's households do not own a car, and of those who do,
most do not use them for commuting. Of the 1.6 million commuters who
come into Manhattan every weekday (or, who did, before the virus), more
than 80 percent make the trip via public transit, mostly trains and
buses, or by walking or biking. Only around 12 percent of daily
commuters get to the island by car.

``It really does feel like there is a silent majority that doesn't get
any real say in how the public space is used,'' Chakrabarti told me.

New York's drivers are essentially being given enormous tracts of land
for their own pleasure and convenience. To add to the overall misery of
the situation, though, even the drivers are not especially happy about
the whole deal, because despite all the roadway they've been given,
they're still stuck in gridlock.

And they most likely will be forever, because cars are not just greedy
for physical space, they're insatiable. There is even a term for the
phenomenon: ``induced demand,'' which holds that
\href{https://www.wired.com/2014/06/wuwt-traffic-induced-demand/}{the
more land you give to cars, the more attractive driving becomes,~leading
to more traffic}, leading to more roads --- an unwinnable cycle that
ends with every inch of our cities paved over.

In that sense, even drivers should have an interest in fostering
alternatives to driving.

``The one thing we know for sure, because we understand geometry, is
that if everyone drives, nobody moves,'' Brent Toderian, the former
chief planner for the city of Vancouver, British Columbia, told me. Even
if you're a committed daily driver, ``it's in your best interest for
walking, biking and public transit to be as attractive as possible for
everyone else --- because that means you're going to be able to drive
easier.''

Indeed, PAU's plan bears this out. Banning private cars on Manhattan
would reduce traffic by as much as 20 percent on routes that start and
end within New York's other boroughs --- that is, in places where cars
would still be allowed ---~according to an analysis by traffic engineers
at Buro Happold, a consulting firm that studied PAU's plan.

\hypertarget{uptown-avenues}{%
\subsection{Uptown avenues}\label{uptown-avenues}}

Currently, wide uptown avenues like Adam Clayton Powell Jr. Boulevard
are mired in traffic.

Eight lanes of traffic and parking take up most of the roadway, with
pedestrians forced to hustle to cross long crosswalks.

In the new plan, community members could vote on how they wanted to use
the space reclaimed from cars. There would be room for curbside vendors,
gathering spaces and civic and social services.

Source: Practice for Architecture and Urbanism

How would people get around in a Manhattan without private cars?

Mostly on foot, by bus or by subway; often on a bicycle, e-bike,
scooter, or some future light, battery-powered ``micromobility'' device
(things like
\href{https://www.wired.com/story/one-wheeled-vehicles-micromobility/}{one-wheeled,
self-balancing skateboards}); and sometimes, in a pinch, in a taxi or
Uber.

Some of these may not sound like your cup of tea. Buses are slow,
bicycles are dangerous, and you wouldn't be caught dead on a scooter,
let alone a one-wheeled skateboard. But that's only because you're
imagining these other ways of getting around as they exist today, in the
world of cars.

Cars make every other form of transportation a little bit terrible. The
absence of cars, then, exerts its own kind of magic ---~take private
cars away, and every other way of getting around gets much better.

Under PAU's plan, road traffic in a car-free Manhattan would fall by
about 60 percent. The absence of cars would allow pedestrians, buses and
bikes to race across New York at unheard-of speeds. Today, a bus trip
from uptown to downtown --- for instance, from Harlem to City Hall ---
takes an hour and 48 minutes. With the sort of rapid bus system PAU
imagines, and without cars in the way, the same trek would take 35
minutes.

\hypertarget{fewer-cars-faster-buses}{%
\subsection{Fewer cars, faster buses}\label{fewer-cars-faster-buses}}

\hypertarget{fewer-cars-faster-buses-1}{%
\subsubsection{Fewer cars, faster
buses}\label{fewer-cars-faster-buses-1}}

\hypertarget{removing-private-cars-would-shorten-bus-commutes-into-and-around-manhattan}{%
\paragraph{Removing private cars would shorten bus commutes into and
around
Manhattan.}\label{removing-private-cars-would-shorten-bus-commutes-into-and-around-manhattan}}

bronx

bronx

▼ 74 min.

Hunts Point to Union Square

▼ 74 min.

Hunts Point to Union Square

▼ 41 min.

Jackson Heights to Union Square

▼ 41 min.

Jackson Heights to Union Square

Queens

Queens

▼ 22 min.

Long Island City to Dumbo

▼ 22 min.

Long Island City to Dumbo

Brookyln

Brookyln

▼ 27 min.

Flatbush to Union Square

▼ 27 min.

Flatbush to Union Square

bronx

bronx

▼ 74 min.

Hunts Point to Union Square

▼ 74 min.

Hunts Point to Union Square

New Jersey

New Jersey

▼ 41 min.

Jackson Heights to Union Square

▼ 41 min.

Jackson Heights to Union Square

▼ 45 min.

Paterson, N.J. to Union Square

▼ 45 min.

Paterson, N.J. to Union Square

Queens

Queens

▼ 22 min.

Long Island City to Dumbo

▼ 22 min.

Long Island City to Dumbo

Brookyln

Brookyln

▼ 27 min.

Flatbush to Union Square

▼ 27 min.

Flatbush to Union Square

Note: Assuming a traffic reduction of 60 percent in Manhattan and 8
percent outside of the borough. Source: Practice for Architecture and
Urbanism, estimates from Buro Happold

The plan wouldn't improve just Manhattan. A ban on private cars on the
island would ripple across the Hudson, altering transportation and
livability across the wider metropolitan region.

\hypertarget{manhattan-bridge}{%
\subsection{Manhattan Bridge}\label{manhattan-bridge}}

Today, cars clog the tunnels and bridges coming into Manhattan.

On the Manhattan Bridge, for example, there are seven lanes for cars.

A new layout would replace four of them with bus lanes, paths for
cyclists and a pedestrian promenade. Three lanes would go to taxis and
ride-share vehicles. The middle lane of traffic would switch direction
depending on demand.

Source: Practice for Architecture and Urbanism

The public health effects would ripple across the region, too. The most
polluted air in New York
\href{https://www.ucsusa.org/sites/default/files/attach/2019/06/Inequitable-Exposure-to-Vehicle-Pollution-NY.pdf}{hangs
over the Bronx and Queens}, in communities largely populated by
immigrants and people of color. New York City has some of the dirtiest
air in the nation, estimated to
\href{https://www1.nyc.gov/assets/doh/downloads/pdf/eode/eode-air-quality-impact.pdf}{cause
3,000 premature deaths} annually.

Among other ailments, long-term exposure to polluted air is thought to
\href{https://www.nytimes3xbfgragh.onion/2020/04/07/climate/air-pollution-coronavirus-covid.html}{increase
the deadliness of Covid-19}. Much of the unhealthy air is caused by
traffic sitting idle on the roads leading to Manhattan. Buro Happold
estimates that PAU's plan would lead to a 50 percent reduction in toxic
air pollution in Manhattan, and a 20 percent reduction in the other
boroughs.

\hypertarget{fdr-drive}{%
\subsection{F.D.R. Drive}\label{fdr-drive}}

It's easy to forget that Manhattan is an island because roads block the
view of the waterfront.

This is especially true on parts of the borough's east side, where
Franklin D. Roosevelt Drive runs along the edge of the water.

An expanded greenway would connect with the one on the island's west
side, making it easier for people to bike, run and walk around
Manhattan's perimeter.

Source: Practice for Architecture and Urbanism

Given how completely automobiles rule most cities, calling for their
outright banishment can sound almost ludicrous. (We can't even get some
people to agree to wear masks to stop the spread of a devastating
pandemic.)

Instead of fighting a war on cars, Toderian told me, urbanists should
fight a war on car \emph{dependency} --- on cities that leave residents
with few choices other than cars. Alleviating car dependency can improve
commutes for everyone in a city.

Chakrabarti acknowledges the political risks of trying to ban private
cars. But Manhattan, he points out, is a special place. With a
population that is already quite used to getting along without cars, the
island is just about the only place in the country where you could even
consider calling for the banishment of cars. Manhattan could be a place
for all of America to witness how reducing an urban area's reliance on
cars can lead to a better life.

At the moment, many of the most intractable challenges faced by
America's urban centers stem from the same cause --- a lack of
accessible physical space. We live in a time of epidemic homelessness.
There's a national housing affordability crisis caused by an extreme
shortage of places to live. And now there's a contagion that thrives on
indoor overcrowding.

Given these threats, how can American cities continue to justify wasting
such enormous tracts of land on death machines?

Animations, illustrations and source material provided by
\href{http://pau.studio/}{Practice for Architecture and Urbanism} with
contributions from Vishaan Chakrabarti, Ruchika Modi, Julia Lewis,
Skylar Bisom-Rapp, Junxi Wu, George Distefano and Mateo Fernández-Muro.
Buro Happold provided additional source material with contributions from
Francesco Cerroni, Alice Shay and Gabriel Warshaw. Satellite imagery
provided by Google.

Produced by Gus Wezerek.

\hypertarget{office-hours-with-farhad-manjoo}{%
\subsection{Office Hours With Farhad
Manjoo}\label{office-hours-with-farhad-manjoo}}

\emph{Farhad wants to}
\href{https://www.nytimes3xbfgragh.onion/2019/05/16/opinion/farhad-office-hours.html?module=inline}{\emph{chat
with readers on the phone}}\emph{. If you're interested in talking to a
New York Times columnist about anything that's on your mind, please fill
out this form. Farhad will select a few readers to call.}

\emph{The Times is committed to publishing}
\href{https://www.nytimes3xbfgragh.onion/2019/01/31/opinion/letters/letters-to-editor-new-york-times-women.html}{\emph{a
diversity of letters}} \emph{to the editor. We'd like to hear what you
think about this or any of our articles. Here are some}
\href{https://help.nytimes3xbfgragh.onion/hc/en-us/articles/115014925288-How-to-submit-a-letter-to-the-editor}{\emph{tips}}\emph{.
And here's our email:}
\href{mailto:letters@NYTimes.com}{\emph{letters@NYTimes.com}}\emph{.}

\emph{Follow The New York Times Opinion section on}
\href{https://www.facebookcorewwwi.onion/nytopinion}{\emph{Facebook}}\emph{,}
\href{http://twitter.com/NYTOpinion}{\emph{Twitter (@NYTopinion)}}
\emph{and}
\href{https://www.instagram.com/nytopinion/}{\emph{Instagram}}\emph{.}

Advertisement

\protect\hyperlink{after-bottom}{Continue reading the main story}

\hypertarget{site-index}{%
\subsection{Site Index}\label{site-index}}

\hypertarget{site-information-navigation}{%
\subsection{Site Information
Navigation}\label{site-information-navigation}}

\begin{itemize}
\tightlist
\item
  \href{https://help.nytimes3xbfgragh.onion/hc/en-us/articles/115014792127-Copyright-notice}{©~2020~The
  New York Times Company}
\end{itemize}

\begin{itemize}
\tightlist
\item
  \href{https://www.nytco.com/}{NYTCo}
\item
  \href{https://help.nytimes3xbfgragh.onion/hc/en-us/articles/115015385887-Contact-Us}{Contact
  Us}
\item
  \href{https://www.nytco.com/careers/}{Work with us}
\item
  \href{https://nytmediakit.com/}{Advertise}
\item
  \href{http://www.tbrandstudio.com/}{T Brand Studio}
\item
  \href{https://www.nytimes3xbfgragh.onion/privacy/cookie-policy\#how-do-i-manage-trackers}{Your
  Ad Choices}
\item
  \href{https://www.nytimes3xbfgragh.onion/privacy}{Privacy}
\item
  \href{https://help.nytimes3xbfgragh.onion/hc/en-us/articles/115014893428-Terms-of-service}{Terms
  of Service}
\item
  \href{https://help.nytimes3xbfgragh.onion/hc/en-us/articles/115014893968-Terms-of-sale}{Terms
  of Sale}
\item
  \href{https://spiderbites.nytimes3xbfgragh.onion}{Site Map}
\item
  \href{https://help.nytimes3xbfgragh.onion/hc/en-us}{Help}
\item
  \href{https://www.nytimes3xbfgragh.onion/subscription?campaignId=37WXW}{Subscriptions}
\end{itemize}
