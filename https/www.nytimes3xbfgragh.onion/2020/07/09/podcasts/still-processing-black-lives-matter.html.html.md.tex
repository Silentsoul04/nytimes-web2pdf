Sections

SEARCH

\protect\hyperlink{site-content}{Skip to
content}\protect\hyperlink{site-index}{Skip to site index}

\href{https://www.nytimes3xbfgragh.onion/spotlight/podcasts}{Podcasts}

\href{https://myaccount.nytimes3xbfgragh.onion/auth/login?response_type=cookie\&client_id=vi}{}

\href{https://www.nytimes3xbfgragh.onion/section/todayspaper}{Today's
Paper}

\href{/spotlight/podcasts}{Podcasts}\textbar{}So Y'all Finally Get It

\url{https://nyti.ms/38Hzu7r}

\begin{itemize}
\item
\item
\item
\item
\item
\end{itemize}

\href{https://www.nytimes3xbfgragh.onion/news-event/george-floyd-protests-minneapolis-new-york-los-angeles?action=click\&pgtype=Article\&state=default\&region=TOP_BANNER\&context=storylines_menu}{Race
and America}

\begin{itemize}
\tightlist
\item
  \href{https://www.nytimes3xbfgragh.onion/interactive/2020/07/03/us/george-floyd-protests-crowd-size.html?action=click\&pgtype=Article\&state=default\&region=TOP_BANNER\&context=storylines_menu}{Black
  Lives Matter Movement}
\item
  \href{https://www.nytimes3xbfgragh.onion/interactive/2020/06/28/us/i-cant-breathe-police-arrest.html?action=click\&pgtype=Article\&state=default\&region=TOP_BANNER\&context=storylines_menu}{History
  of `I Can't Breathe'}
\item
  \href{https://www.nytimes3xbfgragh.onion/interactive/2020/06/10/upshot/black-lives-matter-attitudes.html?action=click\&pgtype=Article\&state=default\&region=TOP_BANNER\&context=storylines_menu}{How
  Public Opinion Shifted}
\item
  \href{https://www.nytimes3xbfgragh.onion/interactive/2020/07/16/us/black-lives-matter-protests-louisville-breonna-taylor.html?action=click\&pgtype=Article\&state=default\&region=TOP_BANNER\&context=storylines_menu}{45
  Days in Louisville}
\end{itemize}

Advertisement

\protect\hyperlink{after-top}{Continue reading the main story}

transcript

Back to Still Processing

bars

0:00/26:29

-26:29

transcript

\hypertarget{so-yall-finally-get-it}{%
\subsection{So Y'all Finally Get It}\label{so-yall-finally-get-it}}

\hypertarget{hosted-by-wesley-morris-and-jenna-wortham-produced-by-hans-buetow-and-edited-by-sara-sarasohn}{%
\subsubsection{Hosted by Wesley Morris and Jenna Wortham. Produced by
Hans Buetow and edited by Sara
Sarasohn.}\label{hosted-by-wesley-morris-and-jenna-wortham-produced-by-hans-buetow-and-edited-by-sara-sarasohn}}

\hypertarget{america-just-wont-let-us-rest}{%
\paragraph{America just won't let us
rest.}\label{america-just-wont-let-us-rest}}

Thursday, July 9th, 2020

\begin{itemize}
\item
  {[}music{]}
\item
  jenna wortham\\
  Look, I know that we said we were on break until the fall. But ---
\item
  {[}laughter{]}
\item
  jenna wortham\\
  --- so much happened, we just had to reunite and talk about it. So the
  first thing we did was a live event for New York Times Events. And we
  did it over video call a few weeks ago. And we wanted to share an
  excerpt of that conversation with our listeners, for those who might
  have missed it.
\item
  wesley morris\\
  Right. And that day we were talking about empathy and hope. And those
  are still things that we're going to continue to talk about, and
  things we've always talked about. And a lot of what we were doing with
  each other was the actual real time working out of our thoughts and
  our feelings. So it's going to sound like we're unplugged, basically,
  and just thinking through this without the --- you know, all of the
  bells and whistles of a regular episode.
\item
  jenna wortham\\
  It's --- we were where we were. And because we're us, we realize
  there's more to do. So good news, I guess? I don't know. We've got two
  more episodes coming in the next two weeks. We dive into our thoughts,
  our feelings, and continue to process everything that's happening in
  America right now.
\item
  wesley morris\\
  Jenna, that is great news. It's great news!
\item
  jenna wortham\\
  (LAUGHS) Yeah. So please enjoy the first part of our midsummer
  trilogy, and stay tuned for more.
\item
  {[}music{]}
\item
  jenna wortham\\
  Well, thank you to everyone for joining us today. I'm Jenna Wortham.
\item
  wesley morris\\
  I'm Wesley Morris. We're two New York Times writers, stuck in our
  houses.
\item
  jenna wortham\\
  And this is ``Still Processing.''
\end{itemize}

{[}laughter{]}

{[}music{]}

\begin{itemize}
\item
  jenna wortham\\
  Well, we're here today --- we're gathered here today ---
\item
  wesley morris\\
  Oh my goodness.
\item
  jenna wortham\\
  --- because this country won't quit. It really won't quit. I mean, I'm
  --- listen. I was in the park today having lunch and with a friend,
  socially distant, and just taking a break offline. One of my
  meditation healer friends --- we were just really taking in the air.
  And, you know, it's in New York now --- it's as common as hearing a
  siren or anything else. You hear the cheers of ``Black lives matter.''
  And it brings me to tears. I was running a little late for this. I
  stopped to watch the procession, and it was actually children. It was
  a a ton right in the middle of Bed-Stuy saying ``Black kids matter.''
  And I was just openly weeping. That symbolism --- I mean, that's not
  symbolism. That's, like, policy in action. You know? That, to me, is a
  political action that's so important. And I'm like, yes, this is what
  it means to be alive right now in this moment in 2020. But I don't
  trust it just because you've taped up a Black power fist onto the
  front or your business or on the front of your home. It's got to look
  more --- it just has to look more dynamic than that. Y'all got to show
  us more. And it needs to be more than this because this is a global
  problem. And it's not just enough to put a banner on it. The banner is
  not a Band-Aid.
\item
  wesley morris\\
  Right. Now, I agree. But OK, so what do you think --- because I
  actually --- I've been thinking a lot about what --- the question is
  always, where do we go from here? There are some of America's great
  thinkers wrestling with this question. And there's no answer. Right? I
  mean, there's no one answer. But I think we should ask for everything.
  There's not a thing we shouldn't be asking for.
\item
  jenna wortham\\
  Yes.
\item
  wesley morris\\
  I mean, shoot the moon? I want to shoot Saturn. This is the moment
  where there is no crazy thing to ask. Because if you know anything
  about what's happened in this country, everything that's led to this
  moment is the crazy thing.
\item
  jenna wortham\\
  Right.
\item
  wesley morris\\
  And asking to undo it is the only sane thing we can do. And I feel
  like asking for, asking for as much as we can get. Because, I've said
  this to you, and I'm just going to --- I'm going to float this out
  here. I don't know. This is not really truly tested material.
\item
  jenna wortham\\
  Oh, OK.
\item
  wesley morris\\
  But it is really important to think about what it means to treat all
  Americans like Black people. And white people, either seeing or
  experiencing what it's like to be at the police --- at the hands of
  the police, at the whims of the police in some cases ---
\item
  jenna wortham\\
  That's right. That's right.
\item
  wesley morris\\
  --- I think that the --- let everybody go to a crappy public school or
  an underfunded public school in this country that is actively,
  actively starved of resources, actively neglected. The problem in this
  entire situation has always been a matter of empathy. It's always been
  not being able to imagine that the people undergoing all this unequal
  treatment were actual people or that they didn't deserve that.
\item
  jenna wortham\\
  Yes.
\item
  wesley morris\\
  Because what was happening with these other shootings before?
\item
  jenna wortham\\
  Right. Right.
\item
  wesley morris\\
  There would always be somebody who would have some extenuating
  circumstances they didn't know anything about. And I think there's ---

  one of the things that's been happening the whole time was the
  extenuating circumstance or extenuating excuse: ``Well, they must have
  done something.''
\item
  jenna wortham\\
  Right. Right.
\item
  wesley morris\\
  Michael Brown must have done something. Eric Garner must have done
  something. But that question of what they did is immediately ---
\item
  jenna wortham\\
  Right.
\item
  wesley morris\\
  --- the cost can't be their life for selling loose cigarettes.
\item
  jenna wortham\\
  That's right. So it's like, the next question is, so what? Right?
\item
  wesley morris\\
  Right.
\item
  jenna wortham\\
  So what? It doesn't give the police free license just to kill people,
  right, just to shoot people at will or all the other things they do.
  But, yeah, I agree. I'm listening.
\item
  wesley morris\\
  So I'm saying that what we've got right now is the closing of an
  empathy gap. And some of that is just pure empirical experience on
  behalf of some of these people who were marching alongside Black
  people.
\item
  jenna wortham\\
  Yes.
\item
  wesley morris\\
  And it's the experience of understanding --- and I don't know ---
  there's no way to quantify this, but I wonder how far out the empathy
  goes. Right? And this is why I'm saying there's no thing that should
  not be demanded right now, in some ways.
\item
  jenna wortham\\
  But my asterisk, to your point, is that there is an empathy gap that's
  being closed. And I think I'm really urging for watchfulness around
  how much empathy there is, and kind of where it's being directed.
  Right? Because the world and the country is outraged about George
  Floyd. And that same vector of outrage and anger is not being directed
  towards what happened to Breonna Taylor, what happened to Tony McDade.
  And so what is interesting to me right now, and what I'm really
  thinking through, is how do we make this moment as intersectional as
  possible and remind everyone that all Black lives matter? Right? Women
  matter. Queer people matter. Trans Black people matter. You know? And
  I think that it's been tough because it's mixed. Right? There is a lot
  of empathy around Mr. Floyd. But is that empathy being shared for Miss
  Taylor? Right? Is that empathy being shared for Miss Dior, who was
  attacked in Minneapolis as well? And so it needs to be inclusive. It
  has to be inclusive. Because, if not, then we're just perpetuating the
  same systemic, misogynistic, patriarchical problems.

  There has to be enough empathy for all of us. It cannot just be about
  cis Black men, right, who are suffering at the hands of state
  violence. It has to be about everybody. So that's kind of where I get
  a little bit tripped up. But I do want to say --- OK, so as you've
  been talking I've been thinking. And something that is meaningful for
  me that I've seen, because I do think that cultural --- cultural
  shifts do shape societal shifts. I mean, absolutely. And so small
  shifts like Legos, right, saying we're not going to do any more police
  officer Legos --- ``COPS'' coming off the air --- although, I think
  that damage has been done. That's a long overdue thing. Yeah. I think
  that it will be really interesting to see, well, how are police
  officers represented on all the ``S.V.U.'' shows going forward? How
  are police officers represented from here on out? I think that's
  actually more interesting, to me. But the fact that those shifts are
  starting to happen feel important. Because, like you say, it's like,
  we do get these --- we get these images about what law and order mean
  imprinted into our minds. And to sort of start having some of that
  untangling happen in real time feels --- I do think that's what I feel
  strongly about.
\item
  wesley morris\\
  I guess the reason I'm exasperated by some of these things is because
  they're not even symbolic. They're merely cosmetic. Right?
\item
  jenna wortham\\
  Right.
\item
  wesley morris\\
  And I think NASCAR vowing to purge its races of Confederate flag
  imagery is an interesting and meaningful symbolic change that I don't
  see being enforced. To the degree that I'm a journalist at all ---
\item
  jenna wortham\\
  (LAUGHS)
\item
  wesley morris\\
  --- there's a little tiny part of me that wants to put on a little
  disguise and just go down and just check things out. You can purge all
  the Confederate flags you want. You know what you can't purge? My fear
  of going into a NASCAR race at Talladega.
\item
  jenna wortham\\
  That very valid fear, my friend. Don't you dare go. I will not let you
  go. But that's the thing. It's like, I don't like how we're being
  asked to gauge the sincerity of these decisions. And it feels like
  unfair labor that I have to do. It doesn't feel valid or sincere at
  all to me.
\item
  wesley morris\\
  But you and I went on the ``CBS Morning Show'' after --- I don't know
  if you remember this.
\item
  jenna wortham\\
  Are you kidding me? Of course. Anyway, go on. Yes, we went on ``CBS
  This Morning'' after the white supremacist rally in Charlottesville.
\item
  wesley morris\\
  Yeah. The question was, what kind of work do we need to do to get
  these things to stop? And we both almost said in unison, not our job.
  We've been doing it.
\item
  jenna wortham\\
  Not on us. Yeah.
\item
  wesley morris\\
  We've done it. But I really believe that. And I think that we have
  given, I mean, bread crumbs, blueprints, instruction manuals. It's all
  already out there. And I think that one of the things that's important
  about this moment that's really interesting to think about is like,
  there are concrete --- the ways in which now is different from five
  years ago is that Black Lives Matter was repeatedly dogged for not
  like, ``What do y'all want? What do you want? What do you stand for?''
\item
  jenna wortham\\
  You're disorganized.
\item
  wesley morris\\
  ``What do you believe?''
\item
  jenna wortham\\
  And I have to say, I think that's why the question of empathy haunts
  me. Right? I don't know what that means. And I've been revisiting
  Angela Davis and Kwame Ture. If you haven't listened to --- I think
  it's his 1966 speech where Kwame Ture, who at the time, was Stokely
  Carmichael, gives this this speech about this idea of Black power,
  right at Berkeley, and is talking about you can't just have empathy
  for Black people you've decided are good or OK. Because that, in
  itself, like, a type of Black life worth fighting for --- is a --- is
  a function of racism. And that's something that I struggle with
  because I think, even being in Brooklyn, what does it mean to see a
  procession of people walking through Fort Greene, Flatbush, Bed-Stuy,
  saying Black life matters, and you're white? Does it matter to you?
  Because, from where I'm sitting, you seem like you have no problem
  {[}INAUDIBLE{]} Black people out of this city, pushing Black people
  into different neighborhoods, pushing them out of their historically
  Black --- I don't know. So I think there is this kind of cognitive
  dissonance to this moment, too, of like, well, what --- and I think
  that's what I keep coming back to. It's like, I can hold the enormity
  and the magnitude and tears. I was crying into the sunset last night,
  just really thinking about this kind of recognition that's happening.
  And at the same time, this sort of deep reckoning and recognition that
  I don't trust is happening on a cellular level, right, because of all
  of these ways in which the history of this country and the history of
  Black people in this world you know have not been treated as
  mattering. And so the idea --- there's a learning curve. There's a
  curve. And so for this question of who --- how do we get empathy,
  right, I think the empathy is there. But I think it's still at a
  comfortable level. And I'm really interested in the discomforting
  empathy. Right? When you're being asked to care about not just Black
  life when it involves our death, but when you're being asked to care
  about Black life at every single level. Right? Like, Black life having
  the same access to housing as you. People having the same job
  opportunities --- I mean, we're seeing this racial reckoning happening
  online right now where all these people are coming out about
  experiencing macro aggressions and racism in different professions and
  fields. From the food industry, to media, to writing, to television,
  to --- and it's unearthing all of these ways in which people have felt
  marginalized, and not in a way that they were called, hey, you're the
  n-word. Right? But like, oh, you're not qualified. And all of these
  subliminal ways that racism manifests to keep Black people from having
  things that white people have {[}INAUDIBLE{]}. You know? I don't know.
  So I, I just don't --- that's the part --- I just want to say that
  that's the part, for me, that I don't think is our work. I think white
  people really need to look inside themselves and think about how they
  really feel about Black people, and face their own anti-Blackness.
  Because, listen, growing up in America I had to face my own
  anti-Blackness. That's work I had to do because I grew up in a culture
  steeped in it. So I had to do that work for myself. So I don't believe
  that you don't have it, too. Last thing I just want to say really
  quickly is that ---
\item
  wesley morris\\
  Last thing you want to say? This is your show! You can say whatever
  you want.
\item
  jenna wortham\\
  I'm sorry. I know! But I feel like you were --- I can see you and I
  know you so well that I know you want to say something. So that's why
  I was just like, trying to rewind ---
\item
  wesley morris\\
  Not anymore. I'm in the choir, baby. Go on.
\item
  jenna wortham\\
  I forgot, actually. It'll come back to me.
\item
  wesley morris\\
  (LAUGHS)
\item
  jenna wortham\\
  Go ahead. After all that, I know. (LAUGHS) I keep lighting my palo
  santo because this conversation is making me so worked up. I'm like, I
  got to cleanse the air while we're talking because it's really heavy.
  But you said you were going to speak, so you speak. And then I'll
  remember.
\item
  wesley morris\\
  Well, no. I mean, what you were talking about was --- you were sort of
  addressing all of the ways in which things would need to change in
  order for this to have achieved all that it has the potential to,
  right, and all of the ways in which Black Americans have been
  historically disserved. But it's funny to talk about history. Because
  history is ---

  it's the present. Right? I mean, George Floyd's death. That is a 21st
  century death. It is a 20th century death. It's a 19th century death.
  It's an 18th century death. He died a death that's been died for
  centuries. And so this idea of history is a really --- I mean, we've
  been talking a lot since the pandemic hit, people have been joking
  about time. And I've been joking about time, knowing what day is what.
  I feel like there's something about what's happening right now that is
  beyond us.
\item
  jenna wortham\\
  Yes. Yes.
\item
  wesley morris\\
  It is cosmic.
\item
  jenna wortham\\
  Yes. Speak.
\item
  wesley morris\\
  And it is like somebody dialed for help in 1619 and somebody finally
  picked up the phone in 2020.
\item
  jenna wortham\\
  Yes.
\item
  wesley morris\\
  And I feel like the cosmic story that's unfolding right now is, a call
  that got placed 401 years ago is finally being answered.
\item
  jenna wortham\\
  That's right.
\item
  wesley morris\\
  And we'll see what resources show up. These things have been ready to
  be toppled the whole time. It just needed more bodies, more force,
  more belief, more time to think about what it means to topple, and
  what you put in its place. And we're in the middle of thinking about
  all of these things and thinking about them collectively, and
  understanding that the stakes are so high that to get it wrong --- I
  don't know what getting it wrong looks like. I guess getting it wrong
  looks like doing nothing. But this is why I'm really a big believer in
  just asking or demanding --- demanding the most.
\item
  jenna wortham\\
  You know and you --- I'm so happy we're processing this right now
  because you do always kind of push my thinking forward. And I feel
  like I do start to knit together the pieces of what you're saying, and
  kind of integrate it into my own thinking while we're talking. So
  thanks for just sharing so freely and vulnerably. What I was going to
  say earlier that I lost my train on was, in terms of the question of
  empathy, like getting white people to understand how deep the
  anti-Blackness goes, that's not my job. I also am not leaving it to
  you guys to fix the world even though you broke it because I don't
  trust that. I don't trust y'all to fix what you broke!
\item
  wesley morris\\
  (LAUGHS)
\item
  jenna wortham\\
  So I do think there is this incredible opportunity. I've seen people
  be like, not my problem. Like, it's y'all's to fix. And I'm like, do
  we trust that, though? Come on. Y'all saw ``Get Out.'' No!
\item
  wesley morris\\
  Yeah. I agree.
\item
  jenna wortham\\
  OK. So I'm just getting to the questions. There is a question from
  somebody named Chase. Thank you for your question, Chase. Where is
  your hope coming from? What helps you access the hope and not just
  feel despair?
\item
  wesley morris\\
  That this is different. This is fundamentally different. Every elder
  that I've spoken to recognizes that it's different. A lot of older
  people feel hope. I mean, you know, John Lewis --- there's a
  documentary about the congressman and civil rights hero, John Lewis.
  And I've been thinking a lot about what --- I'm praying for him every
  day, by the way, and his health. But what must it be like to be a John
  Lewis-level person, a Jesse Jackson-level person, and see what is
  happening right now, and to know that there is something different
  about what's going on. And I think that that is a thing that --- but I
  think hope is important. I think hope is --- we are a people of hope,
  even when it doesn't seem like we should have any or should be hopeful
  at all. We have it. We keep it. It's a little pilot light that keeps
  us going, a lot of us.

  I don't want to be at hope yet. I want to be at observance. I want to
  still be mad. I still want to be in awe of what is happening. I'm not
  --- hope is kind of always there for me. But for people who are
  despairing, there's hope to be found in all that chaos.
\item
  jenna wortham\\
  Well put. Beautifully said.
\item
  wesley morris\\
  What's our next question?
\item
  jenna wortham\\
  Well, there is one last question, which I think we can answer, which
  is, what are you grateful for right now?

  My gosh. I mean, I'm grateful to be alive. I'm grateful to be able to
  use this platform. I'm grateful to hold space. I feel very called to
  this work that you and I do. And I think that it is important and it
  is dire. And I'm grateful that I feel prepared. I feel prepared for
  right now.
\item
  wesley morris\\
  That's --- oh, interesting.
\item
  jenna wortham\\
  I don't think I was ready --- I wasn't ready ---
\item
  wesley morris\\
  That's interesting.
\item
  jenna wortham\\
  --- in 2012, 2013, `14. Pick a year. It was so painful. I didn't have
  the tools to really allow myself to feel the deep grief, and the
  anger, and the rage, and the fatigue. And I've been working on myself
  to really get to a place where I can hold so much emotion and not be
  afraid of it. Because to face the unfairness and cruelty that Black
  people face in this country is to face your own obsolescence, in a
  way. It's like, to really sit with what it means to --- we are living
  under a government that does not value Black life, and will --- has no
  problem showing us over and over again how little it cares.
\item
  wesley morris\\
  It doesn't value anybody's life, really.
\item
  jenna wortham\\
  Yeah. But I'm just going to keep it centered on ---
\item
  wesley morris\\
  I know. I know. I just --- (LAUGHS)
\item
  jenna wortham\\
  I guess what I'm really trying to say, though, is that to be a Black
  person in America is to face existential despair. Because how do you
  grapple with the depths of disregard and lack of care, and still get
  up in the morning? You know? And so I think I'm really grateful that I
  have the tools for myself, and I'm also in a position to share them
  with others. So, I don't know, it's hard. I feel equipped and I feel
  ready.
\item
  wesley morris\\
  I feel like --- that's powerful. Like, the readiness --- the
  readiness. I feel that is part of what this moment is for me, too. I'm
  ready. And I think so many of us are ready.
\item
  jenna wortham\\
  Yeah.
\item
  wesley morris\\
  We've been ready. But this is the moment for all that previous work.
\item
  jenna wortham\\
  I agree.
\item
  wesley morris\\
  We're ready.
\end{itemize}

{[}music{]}

\begin{itemize}
\item
  jenna wortham\\
  That feels like a good note to end on. I mean, thank you to everybody
  who tuned in. Thank you to everybody who follows our work.
\item
  wesley morris\\
  Thanks, everybody. And we'll be back. We're going to get back. We'll
  be recording soon.
\end{itemize}

{[}music{]}

jenna wortham

Ah, Wesley, we were so young just a month ago.

wesley morris

(LAUGHS)

jenna wortham

That was our live event on June 12 with NYT Events. And it's only been a
month, but it actually could have been a decade. So we're going to do a
lot more processing, and we're going to expand on the ideas that we
touched upon in that live event. So buckle up. There are two more
episodes in the pipeline. You heard that right! Two more brand new
``Still Processing'' episodes coming into your feeds. Stay tuned.

{[}music{]}

wesley morris

``Still Processing'' is a product of The New York Times.

jenna wortham

It is produced by Hans Buetow.

wesley morris

And our editors are Sara Sarasohn and Sasha Weiss.

jenna wortham

Our engineer is Jake Gorski.

wesley morris

Our theme song is ``World Restart'' by Kindness, and it's from the album
``Otherness.''

jenna wortham

And you can find all of our various shows and past things at
NYTimes.com/StillProcessing.

wesley morris

Bye everybody. We'll talk to you next week.

jenna wortham

(LAUGHS) See you soon.

{[}music{]}

\href{https://www.nytimes3xbfgragh.onion/column/still-processing-podcast}{\includegraphics{https://static01.graylady3jvrrxbe.onion/images/2019/09/15/podcasts/still-processing-album-art-2/still-processing-album-art-2-square320.jpg}Still
Processing}Subscribe:

\begin{itemize}
\tightlist
\item
  \href{https://itunes.apple.com/us/podcast/id1151436460}{Apple
  Podcasts}
\item
  \href{https://www.google.com/podcasts?feed=aHR0cHM6Ly9yc3MuYXJ0MTkuY29tL255dC1zdGlsbC1wcm9jZXNzaW5n}{Google
  Podcasts}
\end{itemize}

\hypertarget{so-yall-finally-get-it-1}{%
\section{So Y'all Finally Get It}\label{so-yall-finally-get-it-1}}

\hypertarget{america-just-wont-let-us-rest-1}{%
\subsection{America just won't let us
rest.}\label{america-just-wont-let-us-rest-1}}

Hosted by Wesley Morris and Jenna Wortham. Produced by Hans Buetow and
edited by Sara Sarasohn.

Transcript

transcript

Back to Still Processing

bars

0:00/26:29

-0:00

transcript

\hypertarget{so-yall-finally-get-it-2}{%
\subsection{So Y'all Finally Get It}\label{so-yall-finally-get-it-2}}

\hypertarget{hosted-by-wesley-morris-and-jenna-wortham-produced-by-hans-buetow-and-edited-by-sara-sarasohn-1}{%
\subsubsection{Hosted by Wesley Morris and Jenna Wortham. Produced by
Hans Buetow and edited by Sara
Sarasohn.}\label{hosted-by-wesley-morris-and-jenna-wortham-produced-by-hans-buetow-and-edited-by-sara-sarasohn-1}}

\hypertarget{america-just-wont-let-us-rest-2}{%
\paragraph{America just won't let us
rest.}\label{america-just-wont-let-us-rest-2}}

Thursday, July 9th, 2020

\begin{itemize}
\item
  {[}music{]}
\item
  jenna wortham\\
  Look, I know that we said we were on break until the fall. But ---
\item
  {[}laughter{]}
\item
  jenna wortham\\
  --- so much happened, we just had to reunite and talk about it. So the
  first thing we did was a live event for New York Times Events. And we
  did it over video call a few weeks ago. And we wanted to share an
  excerpt of that conversation with our listeners, for those who might
  have missed it.
\item
  wesley morris\\
  Right. And that day we were talking about empathy and hope. And those
  are still things that we're going to continue to talk about, and
  things we've always talked about. And a lot of what we were doing with
  each other was the actual real time working out of our thoughts and
  our feelings. So it's going to sound like we're unplugged, basically,
  and just thinking through this without the --- you know, all of the
  bells and whistles of a regular episode.
\item
  jenna wortham\\
  It's --- we were where we were. And because we're us, we realize
  there's more to do. So good news, I guess? I don't know. We've got two
  more episodes coming in the next two weeks. We dive into our thoughts,
  our feelings, and continue to process everything that's happening in
  America right now.
\item
  wesley morris\\
  Jenna, that is great news. It's great news!
\item
  jenna wortham\\
  (LAUGHS) Yeah. So please enjoy the first part of our midsummer
  trilogy, and stay tuned for more.
\item
  {[}music{]}
\item
  jenna wortham\\
  Well, thank you to everyone for joining us today. I'm Jenna Wortham.
\item
  wesley morris\\
  I'm Wesley Morris. We're two New York Times writers, stuck in our
  houses.
\item
  jenna wortham\\
  And this is ``Still Processing.''
\end{itemize}

{[}laughter{]}

{[}music{]}

\begin{itemize}
\item
  jenna wortham\\
  Well, we're here today --- we're gathered here today ---
\item
  wesley morris\\
  Oh my goodness.
\item
  jenna wortham\\
  --- because this country won't quit. It really won't quit. I mean, I'm
  --- listen. I was in the park today having lunch and with a friend,
  socially distant, and just taking a break offline. One of my
  meditation healer friends --- we were just really taking in the air.
  And, you know, it's in New York now --- it's as common as hearing a
  siren or anything else. You hear the cheers of ``Black lives matter.''
  And it brings me to tears. I was running a little late for this. I
  stopped to watch the procession, and it was actually children. It was
  a a ton right in the middle of Bed-Stuy saying ``Black kids matter.''
  And I was just openly weeping. That symbolism --- I mean, that's not
  symbolism. That's, like, policy in action. You know? That, to me, is a
  political action that's so important. And I'm like, yes, this is what
  it means to be alive right now in this moment in 2020. But I don't
  trust it just because you've taped up a Black power fist onto the
  front or your business or on the front of your home. It's got to look
  more --- it just has to look more dynamic than that. Y'all got to show
  us more. And it needs to be more than this because this is a global
  problem. And it's not just enough to put a banner on it. The banner is
  not a Band-Aid.
\item
  wesley morris\\
  Right. Now, I agree. But OK, so what do you think --- because I
  actually --- I've been thinking a lot about what --- the question is
  always, where do we go from here? There are some of America's great
  thinkers wrestling with this question. And there's no answer. Right? I
  mean, there's no one answer. But I think we should ask for everything.
  There's not a thing we shouldn't be asking for.
\item
  jenna wortham\\
  Yes.
\item
  wesley morris\\
  I mean, shoot the moon? I want to shoot Saturn. This is the moment
  where there is no crazy thing to ask. Because if you know anything
  about what's happened in this country, everything that's led to this
  moment is the crazy thing.
\item
  jenna wortham\\
  Right.
\item
  wesley morris\\
  And asking to undo it is the only sane thing we can do. And I feel
  like asking for, asking for as much as we can get. Because, I've said
  this to you, and I'm just going to --- I'm going to float this out
  here. I don't know. This is not really truly tested material.
\item
  jenna wortham\\
  Oh, OK.
\item
  wesley morris\\
  But it is really important to think about what it means to treat all
  Americans like Black people. And white people, either seeing or
  experiencing what it's like to be at the police --- at the hands of
  the police, at the whims of the police in some cases ---
\item
  jenna wortham\\
  That's right. That's right.
\item
  wesley morris\\
  --- I think that the --- let everybody go to a crappy public school or
  an underfunded public school in this country that is actively,
  actively starved of resources, actively neglected. The problem in this
  entire situation has always been a matter of empathy. It's always been
  not being able to imagine that the people undergoing all this unequal
  treatment were actual people or that they didn't deserve that.
\item
  jenna wortham\\
  Yes.
\item
  wesley morris\\
  Because what was happening with these other shootings before?
\item
  jenna wortham\\
  Right. Right.
\item
  wesley morris\\
  There would always be somebody who would have some extenuating
  circumstances they didn't know anything about. And I think there's ---

  one of the things that's been happening the whole time was the
  extenuating circumstance or extenuating excuse: ``Well, they must have
  done something.''
\item
  jenna wortham\\
  Right. Right.
\item
  wesley morris\\
  Michael Brown must have done something. Eric Garner must have done
  something. But that question of what they did is immediately ---
\item
  jenna wortham\\
  Right.
\item
  wesley morris\\
  --- the cost can't be their life for selling loose cigarettes.
\item
  jenna wortham\\
  That's right. So it's like, the next question is, so what? Right?
\item
  wesley morris\\
  Right.
\item
  jenna wortham\\
  So what? It doesn't give the police free license just to kill people,
  right, just to shoot people at will or all the other things they do.
  But, yeah, I agree. I'm listening.
\item
  wesley morris\\
  So I'm saying that what we've got right now is the closing of an
  empathy gap. And some of that is just pure empirical experience on
  behalf of some of these people who were marching alongside Black
  people.
\item
  jenna wortham\\
  Yes.
\item
  wesley morris\\
  And it's the experience of understanding --- and I don't know ---
  there's no way to quantify this, but I wonder how far out the empathy
  goes. Right? And this is why I'm saying there's no thing that should
  not be demanded right now, in some ways.
\item
  jenna wortham\\
  But my asterisk, to your point, is that there is an empathy gap that's
  being closed. And I think I'm really urging for watchfulness around
  how much empathy there is, and kind of where it's being directed.
  Right? Because the world and the country is outraged about George
  Floyd. And that same vector of outrage and anger is not being directed
  towards what happened to Breonna Taylor, what happened to Tony McDade.
  And so what is interesting to me right now, and what I'm really
  thinking through, is how do we make this moment as intersectional as
  possible and remind everyone that all Black lives matter? Right? Women
  matter. Queer people matter. Trans Black people matter. You know? And
  I think that it's been tough because it's mixed. Right? There is a lot
  of empathy around Mr. Floyd. But is that empathy being shared for Miss
  Taylor? Right? Is that empathy being shared for Miss Dior, who was
  attacked in Minneapolis as well? And so it needs to be inclusive. It
  has to be inclusive. Because, if not, then we're just perpetuating the
  same systemic, misogynistic, patriarchical problems.

  There has to be enough empathy for all of us. It cannot just be about
  cis Black men, right, who are suffering at the hands of state
  violence. It has to be about everybody. So that's kind of where I get
  a little bit tripped up. But I do want to say --- OK, so as you've
  been talking I've been thinking. And something that is meaningful for
  me that I've seen, because I do think that cultural --- cultural
  shifts do shape societal shifts. I mean, absolutely. And so small
  shifts like Legos, right, saying we're not going to do any more police
  officer Legos --- ``COPS'' coming off the air --- although, I think
  that damage has been done. That's a long overdue thing. Yeah. I think
  that it will be really interesting to see, well, how are police
  officers represented on all the ``S.V.U.'' shows going forward? How
  are police officers represented from here on out? I think that's
  actually more interesting, to me. But the fact that those shifts are
  starting to happen feel important. Because, like you say, it's like,
  we do get these --- we get these images about what law and order mean
  imprinted into our minds. And to sort of start having some of that
  untangling happen in real time feels --- I do think that's what I feel
  strongly about.
\item
  wesley morris\\
  I guess the reason I'm exasperated by some of these things is because
  they're not even symbolic. They're merely cosmetic. Right?
\item
  jenna wortham\\
  Right.
\item
  wesley morris\\
  And I think NASCAR vowing to purge its races of Confederate flag
  imagery is an interesting and meaningful symbolic change that I don't
  see being enforced. To the degree that I'm a journalist at all ---
\item
  jenna wortham\\
  (LAUGHS)
\item
  wesley morris\\
  --- there's a little tiny part of me that wants to put on a little
  disguise and just go down and just check things out. You can purge all
  the Confederate flags you want. You know what you can't purge? My fear
  of going into a NASCAR race at Talladega.
\item
  jenna wortham\\
  That very valid fear, my friend. Don't you dare go. I will not let you
  go. But that's the thing. It's like, I don't like how we're being
  asked to gauge the sincerity of these decisions. And it feels like
  unfair labor that I have to do. It doesn't feel valid or sincere at
  all to me.
\item
  wesley morris\\
  But you and I went on the ``CBS Morning Show'' after --- I don't know
  if you remember this.
\item
  jenna wortham\\
  Are you kidding me? Of course. Anyway, go on. Yes, we went on ``CBS
  This Morning'' after the white supremacist rally in Charlottesville.
\item
  wesley morris\\
  Yeah. The question was, what kind of work do we need to do to get
  these things to stop? And we both almost said in unison, not our job.
  We've been doing it.
\item
  jenna wortham\\
  Not on us. Yeah.
\item
  wesley morris\\
  We've done it. But I really believe that. And I think that we have
  given, I mean, bread crumbs, blueprints, instruction manuals. It's all
  already out there. And I think that one of the things that's important
  about this moment that's really interesting to think about is like,
  there are concrete --- the ways in which now is different from five
  years ago is that Black Lives Matter was repeatedly dogged for not
  like, ``What do y'all want? What do you want? What do you stand for?''
\item
  jenna wortham\\
  You're disorganized.
\item
  wesley morris\\
  ``What do you believe?''
\item
  jenna wortham\\
  And I have to say, I think that's why the question of empathy haunts
  me. Right? I don't know what that means. And I've been revisiting
  Angela Davis and Kwame Ture. If you haven't listened to --- I think
  it's his 1966 speech where Kwame Ture, who at the time, was Stokely
  Carmichael, gives this this speech about this idea of Black power,
  right at Berkeley, and is talking about you can't just have empathy
  for Black people you've decided are good or OK. Because that, in
  itself, like, a type of Black life worth fighting for --- is a --- is
  a function of racism. And that's something that I struggle with
  because I think, even being in Brooklyn, what does it mean to see a
  procession of people walking through Fort Greene, Flatbush, Bed-Stuy,
  saying Black life matters, and you're white? Does it matter to you?
  Because, from where I'm sitting, you seem like you have no problem
  {[}INAUDIBLE{]} Black people out of this city, pushing Black people
  into different neighborhoods, pushing them out of their historically
  Black --- I don't know. So I think there is this kind of cognitive
  dissonance to this moment, too, of like, well, what --- and I think
  that's what I keep coming back to. It's like, I can hold the enormity
  and the magnitude and tears. I was crying into the sunset last night,
  just really thinking about this kind of recognition that's happening.
  And at the same time, this sort of deep reckoning and recognition that
  I don't trust is happening on a cellular level, right, because of all
  of these ways in which the history of this country and the history of
  Black people in this world you know have not been treated as
  mattering. And so the idea --- there's a learning curve. There's a
  curve. And so for this question of who --- how do we get empathy,
  right, I think the empathy is there. But I think it's still at a
  comfortable level. And I'm really interested in the discomforting
  empathy. Right? When you're being asked to care about not just Black
  life when it involves our death, but when you're being asked to care
  about Black life at every single level. Right? Like, Black life having
  the same access to housing as you. People having the same job
  opportunities --- I mean, we're seeing this racial reckoning happening
  online right now where all these people are coming out about
  experiencing macro aggressions and racism in different professions and
  fields. From the food industry, to media, to writing, to television,
  to --- and it's unearthing all of these ways in which people have felt
  marginalized, and not in a way that they were called, hey, you're the
  n-word. Right? But like, oh, you're not qualified. And all of these
  subliminal ways that racism manifests to keep Black people from having
  things that white people have {[}INAUDIBLE{]}. You know? I don't know.
  So I, I just don't --- that's the part --- I just want to say that
  that's the part, for me, that I don't think is our work. I think white
  people really need to look inside themselves and think about how they
  really feel about Black people, and face their own anti-Blackness.
  Because, listen, growing up in America I had to face my own
  anti-Blackness. That's work I had to do because I grew up in a culture
  steeped in it. So I had to do that work for myself. So I don't believe
  that you don't have it, too. Last thing I just want to say really
  quickly is that ---
\item
  wesley morris\\
  Last thing you want to say? This is your show! You can say whatever
  you want.
\item
  jenna wortham\\
  I'm sorry. I know! But I feel like you were --- I can see you and I
  know you so well that I know you want to say something. So that's why
  I was just like, trying to rewind ---
\item
  wesley morris\\
  Not anymore. I'm in the choir, baby. Go on.
\item
  jenna wortham\\
  I forgot, actually. It'll come back to me.
\item
  wesley morris\\
  (LAUGHS)
\item
  jenna wortham\\
  Go ahead. After all that, I know. (LAUGHS) I keep lighting my palo
  santo because this conversation is making me so worked up. I'm like, I
  got to cleanse the air while we're talking because it's really heavy.
  But you said you were going to speak, so you speak. And then I'll
  remember.
\item
  wesley morris\\
  Well, no. I mean, what you were talking about was --- you were sort of
  addressing all of the ways in which things would need to change in
  order for this to have achieved all that it has the potential to,
  right, and all of the ways in which Black Americans have been
  historically disserved. But it's funny to talk about history. Because
  history is ---

  it's the present. Right? I mean, George Floyd's death. That is a 21st
  century death. It is a 20th century death. It's a 19th century death.
  It's an 18th century death. He died a death that's been died for
  centuries. And so this idea of history is a really --- I mean, we've
  been talking a lot since the pandemic hit, people have been joking
  about time. And I've been joking about time, knowing what day is what.
  I feel like there's something about what's happening right now that is
  beyond us.
\item
  jenna wortham\\
  Yes. Yes.
\item
  wesley morris\\
  It is cosmic.
\item
  jenna wortham\\
  Yes. Speak.
\item
  wesley morris\\
  And it is like somebody dialed for help in 1619 and somebody finally
  picked up the phone in 2020.
\item
  jenna wortham\\
  Yes.
\item
  wesley morris\\
  And I feel like the cosmic story that's unfolding right now is, a call
  that got placed 401 years ago is finally being answered.
\item
  jenna wortham\\
  That's right.
\item
  wesley morris\\
  And we'll see what resources show up. These things have been ready to
  be toppled the whole time. It just needed more bodies, more force,
  more belief, more time to think about what it means to topple, and
  what you put in its place. And we're in the middle of thinking about
  all of these things and thinking about them collectively, and
  understanding that the stakes are so high that to get it wrong --- I
  don't know what getting it wrong looks like. I guess getting it wrong
  looks like doing nothing. But this is why I'm really a big believer in
  just asking or demanding --- demanding the most.
\item
  jenna wortham\\
  You know and you --- I'm so happy we're processing this right now
  because you do always kind of push my thinking forward. And I feel
  like I do start to knit together the pieces of what you're saying, and
  kind of integrate it into my own thinking while we're talking. So
  thanks for just sharing so freely and vulnerably. What I was going to
  say earlier that I lost my train on was, in terms of the question of
  empathy, like getting white people to understand how deep the
  anti-Blackness goes, that's not my job. I also am not leaving it to
  you guys to fix the world even though you broke it because I don't
  trust that. I don't trust y'all to fix what you broke!
\item
  wesley morris\\
  (LAUGHS)
\item
  jenna wortham\\
  So I do think there is this incredible opportunity. I've seen people
  be like, not my problem. Like, it's y'all's to fix. And I'm like, do
  we trust that, though? Come on. Y'all saw ``Get Out.'' No!
\item
  wesley morris\\
  Yeah. I agree.
\item
  jenna wortham\\
  OK. So I'm just getting to the questions. There is a question from
  somebody named Chase. Thank you for your question, Chase. Where is
  your hope coming from? What helps you access the hope and not just
  feel despair?
\item
  wesley morris\\
  That this is different. This is fundamentally different. Every elder
  that I've spoken to recognizes that it's different. A lot of older
  people feel hope. I mean, you know, John Lewis --- there's a
  documentary about the congressman and civil rights hero, John Lewis.
  And I've been thinking a lot about what --- I'm praying for him every
  day, by the way, and his health. But what must it be like to be a John
  Lewis-level person, a Jesse Jackson-level person, and see what is
  happening right now, and to know that there is something different
  about what's going on. And I think that that is a thing that --- but I
  think hope is important. I think hope is --- we are a people of hope,
  even when it doesn't seem like we should have any or should be hopeful
  at all. We have it. We keep it. It's a little pilot light that keeps
  us going, a lot of us.

  I don't want to be at hope yet. I want to be at observance. I want to
  still be mad. I still want to be in awe of what is happening. I'm not
  --- hope is kind of always there for me. But for people who are
  despairing, there's hope to be found in all that chaos.
\item
  jenna wortham\\
  Well put. Beautifully said.
\item
  wesley morris\\
  What's our next question?
\item
  jenna wortham\\
  Well, there is one last question, which I think we can answer, which
  is, what are you grateful for right now?

  My gosh. I mean, I'm grateful to be alive. I'm grateful to be able to
  use this platform. I'm grateful to hold space. I feel very called to
  this work that you and I do. And I think that it is important and it
  is dire. And I'm grateful that I feel prepared. I feel prepared for
  right now.
\item
  wesley morris\\
  That's --- oh, interesting.
\item
  jenna wortham\\
  I don't think I was ready --- I wasn't ready ---
\item
  wesley morris\\
  That's interesting.
\item
  jenna wortham\\
  --- in 2012, 2013, `14. Pick a year. It was so painful. I didn't have
  the tools to really allow myself to feel the deep grief, and the
  anger, and the rage, and the fatigue. And I've been working on myself
  to really get to a place where I can hold so much emotion and not be
  afraid of it. Because to face the unfairness and cruelty that Black
  people face in this country is to face your own obsolescence, in a
  way. It's like, to really sit with what it means to --- we are living
  under a government that does not value Black life, and will --- has no
  problem showing us over and over again how little it cares.
\item
  wesley morris\\
  It doesn't value anybody's life, really.
\item
  jenna wortham\\
  Yeah. But I'm just going to keep it centered on ---
\item
  wesley morris\\
  I know. I know. I just --- (LAUGHS)
\item
  jenna wortham\\
  I guess what I'm really trying to say, though, is that to be a Black
  person in America is to face existential despair. Because how do you
  grapple with the depths of disregard and lack of care, and still get
  up in the morning? You know? And so I think I'm really grateful that I
  have the tools for myself, and I'm also in a position to share them
  with others. So, I don't know, it's hard. I feel equipped and I feel
  ready.
\item
  wesley morris\\
  I feel like --- that's powerful. Like, the readiness --- the
  readiness. I feel that is part of what this moment is for me, too. I'm
  ready. And I think so many of us are ready.
\item
  jenna wortham\\
  Yeah.
\item
  wesley morris\\
  We've been ready. But this is the moment for all that previous work.
\item
  jenna wortham\\
  I agree.
\item
  wesley morris\\
  We're ready.
\end{itemize}

{[}music{]}

\begin{itemize}
\item
  jenna wortham\\
  That feels like a good note to end on. I mean, thank you to everybody
  who tuned in. Thank you to everybody who follows our work.
\item
  wesley morris\\
  Thanks, everybody. And we'll be back. We're going to get back. We'll
  be recording soon.
\end{itemize}

{[}music{]}

jenna wortham

Ah, Wesley, we were so young just a month ago.

wesley morris

(LAUGHS)

jenna wortham

That was our live event on June 12 with NYT Events. And it's only been a
month, but it actually could have been a decade. So we're going to do a
lot more processing, and we're going to expand on the ideas that we
touched upon in that live event. So buckle up. There are two more
episodes in the pipeline. You heard that right! Two more brand new
``Still Processing'' episodes coming into your feeds. Stay tuned.

{[}music{]}

wesley morris

``Still Processing'' is a product of The New York Times.

jenna wortham

It is produced by Hans Buetow.

wesley morris

And our editors are Sara Sarasohn and Sasha Weiss.

jenna wortham

Our engineer is Jake Gorski.

wesley morris

Our theme song is ``World Restart'' by Kindness, and it's from the album
``Otherness.''

jenna wortham

And you can find all of our various shows and past things at
NYTimes.com/StillProcessing.

wesley morris

Bye everybody. We'll talk to you next week.

jenna wortham

(LAUGHS) See you soon.

{[}music{]}

Previous

More episodes ofStill Processing

\href{https://www.nytimes3xbfgragh.onion/2020/07/23/podcasts/hamilton-ziwe-discomfort.html?action=click\&module=audio-series-bar\&region=header\&pgtype=Article}{\includegraphics{https://static01.graylady3jvrrxbe.onion/images/2020/07/23/multimedia/23stillprocessing-pix/23stillprocessing-pix-thumbLarge.jpg}}

July 23, 2020~~•~ 38:10Ziwe May Destroy Hamilton

\href{https://www.nytimes3xbfgragh.onion/2020/07/16/podcasts/reparations-for-aunt-jemima.html?action=click\&module=audio-series-bar\&region=header\&pgtype=Article}{\includegraphics{https://static01.graylady3jvrrxbe.onion/images/2020/07/18/multimedia/16stillprocessing-pix/16stillprocessing-pix-thumbLarge.jpg}}

July 16, 2020~~•~ 35:35Reparations for Aunt Jemima!

\href{https://www.nytimes3xbfgragh.onion/2020/07/09/podcasts/still-processing-black-lives-matter.html?action=click\&module=audio-series-bar\&region=header\&pgtype=Article}{\includegraphics{https://static01.graylady3jvrrxbe.onion/images/2020/07/12/podcasts/09stillprocessing-image/xx-stillprocessing-thumbLarge.jpg}}

July 9, 2020~~•~ 26:29So Y'all Finally Get It

\href{https://www.nytimes3xbfgragh.onion/2020/05/14/podcasts/still-processing-westworld-hollywood-utopia-dystopia.html?action=click\&module=audio-series-bar\&region=header\&pgtype=Article}{\includegraphics{https://static01.graylady3jvrrxbe.onion/images/2020/05/16/podcasts/14stillprocessing-image/14stillprocessing-image-thumbLarge-v2.jpg}}

May 14, 2020New Loop, America

\href{https://www.nytimes3xbfgragh.onion/2020/05/07/podcasts/still-processing-internet-vulnerability-sondheim-parks-recreation.html?action=click\&module=audio-series-bar\&region=header\&pgtype=Article}{\includegraphics{https://static01.graylady3jvrrxbe.onion/images/2020/04/28/pageoneplus/28sondheimjp-sp/28sondheimjp-sp-thumbLarge-v4.jpg}}

May 7, 2020Does This Phone Make Me Look Human?

\href{https://www.nytimes3xbfgragh.onion/2020/04/30/podcasts/still-processing-fiona-apple-fetch-bolt-cutters.html?action=click\&module=audio-series-bar\&region=header\&pgtype=Article}{\includegraphics{https://static01.graylady3jvrrxbe.onion/images/2020/05/03/multimedia/30stillpro-image/30stillpro-image-thumbLarge.jpg}}

May 1, 2020Fiona Ex Machina

\href{https://www.nytimes3xbfgragh.onion/2020/04/23/podcasts/still-processing-halle-berry-sharon-stone-catwoman-quarantine.html?action=click\&module=audio-series-bar\&region=header\&pgtype=Article}{\includegraphics{https://static01.graylady3jvrrxbe.onion/images/2020/04/25/arts/23stillprocessing/23stillprocessing-thumbLarge-v3.jpg}}

April 23, 2020Halle Berry? Hallelujah.

\href{https://www.nytimes3xbfgragh.onion/2020/04/16/podcasts/still-processing-AIDS-survive-coronavirus.html?action=click\&module=audio-series-bar\&region=header\&pgtype=Article}{\includegraphics{https://static01.graylady3jvrrxbe.onion/images/2020/04/20/us/16stillprocessing/16stillprocessing-thumbLarge-v3.jpg}}

April 16, 2020How to Learn From a Plague

\href{https://www.nytimes3xbfgragh.onion/2020/04/09/podcasts/still-processing-tiger-king.html?action=click\&module=audio-series-bar\&region=header\&pgtype=Article}{\includegraphics{https://static01.graylady3jvrrxbe.onion/images/2020/04/11/podcasts/09stillprocessing-image2/09stillprocessing-image2-thumbLarge-v2.jpg}}

April 9, 2020~~•~ 39:49Frosted Flakes

\href{https://www.nytimes3xbfgragh.onion/2020/04/02/podcasts/high-fidelity-zoe-kravitz.html?action=click\&module=audio-series-bar\&region=header\&pgtype=Article}{\includegraphics{https://static01.graylady3jvrrxbe.onion/images/2020/04/05/arts/02still-processing-highfidelity/13highfidelity-thumbLarge.jpg}}

April 2, 2020~~•~ 40:55Delicious Vinyl

\href{https://www.nytimes3xbfgragh.onion/2020/03/26/podcasts/still-processing-quarantine.html?action=click\&module=audio-series-bar\&region=header\&pgtype=Article}{\includegraphics{https://static01.graylady3jvrrxbe.onion/images/2020/03/29/podcasts/26stillprocessing1/26stillprocessing1-thumbLarge.jpg}}

March 26, 2020~~•~ 30:47A Pod From Both Our Houses

\href{https://www.nytimes3xbfgragh.onion/2019/11/07/podcasts/still-processing-parasite-watchmen-bong-joon-ho.html?action=click\&module=audio-series-bar\&region=header\&pgtype=Article}{\includegraphics{https://static01.graylady3jvrrxbe.onion/images/2019/11/08/arts/07stilpr-parasite/00parasite-1-thumbLarge.jpg}}

November 7, 2019Wake

\href{https://www.nytimes3xbfgragh.onion/column/still-processing-podcast}{See
All Episodes ofStill Processing}

Next

Published July 9, 2020Updated July 12, 2020

\begin{itemize}
\item
\item
\item
\item
\item
\end{itemize}

By \href{https://www.nytimes3xbfgragh.onion/by/wesley-morris}{Wesley
Morris} and
\href{https://www.nytimes3xbfgragh.onion/by/jenna-wortham}{Jenna
Wortham}

Excerpts from our June 12 live event, where we caught up about the
uprisings and resurgence of the Black Lives Matter movement. This is the
first of three special summer episodes.

\includegraphics{https://static01.graylady3jvrrxbe.onion/images/2020/07/12/podcasts/09stillprocessing-image/xx-stillprocessing-articleLarge.jpg?quality=75\&auto=webp\&disable=upscale}

Discussed this week:

\begin{itemize}
\item
  Jenna Wortham and Wesley Morris in a
  \href{https://timesevents.nytimes3xbfgragh.onion/stillprocessing0612}{live
  New York Times event} (June 12, 2020).
\item
  ``\href{https://www.nytimes3xbfgragh.onion/2020/06/09/business/media/cops-canceled-paramount-tv-show.html}{`Cops,'
  Long-Running Reality Show That Glorified Police, Is Canceled}'' (The
  New York Times, June 2020)
\item
  ``\href{https://toybook.com/lego-pulling-back-potentially-sensitive-product-amid-george-floyd-protests/}{LEGO
  Pulls Back Police Playset Affiliate Marketing Amid George Floyd
  Protests}'' (ToyBook, June 2020)
\item
  ``\href{https://www.nytimes3xbfgragh.onion/2020/06/10/sports/autoracing/nascar-confederate-flags.html}{NASCAR
  Says It Will Ban Confederate Flags}'' (The New York Times, June 2020)
\item
  Jenna Wortham and Wesley Morris on
  ``\href{https://www.youtube.com/watch?v=my1FfSsT5-E}{CBS This
  Morning}'' (Aug. 18, 2017, CBS)
\item
  \href{http://americanradioworks.publicradio.org/features/blackspeech/scarmichael.html}{Speech
  by Stokely Carmichael} (a.k.a. Kwame Ture) (Oct. 29, 1966, Berkeley,
  Calif.)
\item
  ``\href{https://www.youtube.com/watch?v=z_oEkOdIXdo}{John Lewis: Good
  Trouble}'' (directed by Dawn Porter, July 2020)
\end{itemize}

``Still Processing'' is produced by Hans Buetow and Sydney Harper, and
edited by Sara Sarasohn and Sasha Weiss, with editorial oversight from
Wendy Dorr and Lisa Tobin. Our engineer is Jake Gorski. Our theme music
is by Kindness. It's called ``World Restart,'' from the album
``Otherness.''

Advertisement

\protect\hyperlink{after-bottom}{Continue reading the main story}

\hypertarget{site-index}{%
\subsection{Site Index}\label{site-index}}

\hypertarget{site-information-navigation}{%
\subsection{Site Information
Navigation}\label{site-information-navigation}}

\begin{itemize}
\tightlist
\item
  \href{https://help.nytimes3xbfgragh.onion/hc/en-us/articles/115014792127-Copyright-notice}{©~2020~The
  New York Times Company}
\end{itemize}

\begin{itemize}
\tightlist
\item
  \href{https://www.nytco.com/}{NYTCo}
\item
  \href{https://help.nytimes3xbfgragh.onion/hc/en-us/articles/115015385887-Contact-Us}{Contact
  Us}
\item
  \href{https://www.nytco.com/careers/}{Work with us}
\item
  \href{https://nytmediakit.com/}{Advertise}
\item
  \href{http://www.tbrandstudio.com/}{T Brand Studio}
\item
  \href{https://www.nytimes3xbfgragh.onion/privacy/cookie-policy\#how-do-i-manage-trackers}{Your
  Ad Choices}
\item
  \href{https://www.nytimes3xbfgragh.onion/privacy}{Privacy}
\item
  \href{https://help.nytimes3xbfgragh.onion/hc/en-us/articles/115014893428-Terms-of-service}{Terms
  of Service}
\item
  \href{https://help.nytimes3xbfgragh.onion/hc/en-us/articles/115014893968-Terms-of-sale}{Terms
  of Sale}
\item
  \href{https://spiderbites.nytimes3xbfgragh.onion}{Site Map}
\item
  \href{https://help.nytimes3xbfgragh.onion/hc/en-us}{Help}
\item
  \href{https://www.nytimes3xbfgragh.onion/subscription?campaignId=37WXW}{Subscriptions}
\end{itemize}
