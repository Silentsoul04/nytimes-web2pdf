Sections

SEARCH

\protect\hyperlink{site-content}{Skip to
content}\protect\hyperlink{site-index}{Skip to site index}

\href{/section/us}{U.S.}\textbar{}`Big Mess' Looms if Schools Don't Get
Billions to Reopen Safely

\url{https://nyti.ms/2W4bV3B}

\begin{itemize}
\item
\item
\item
\item
\item
\end{itemize}

\href{https://www.nytimes3xbfgragh.onion/news-event/coronavirus?action=click\&pgtype=Article\&state=default\&region=TOP_BANNER\&context=storylines_menu}{The
Coronavirus Outbreak}

\begin{itemize}
\tightlist
\item
  live\href{https://www.nytimes3xbfgragh.onion/2020/08/04/world/coronavirus-cases.html?action=click\&pgtype=Article\&state=default\&region=TOP_BANNER\&context=storylines_menu}{Latest
  Updates}
\item
  \href{https://www.nytimes3xbfgragh.onion/interactive/2020/us/coronavirus-us-cases.html?action=click\&pgtype=Article\&state=default\&region=TOP_BANNER\&context=storylines_menu}{Maps
  and Cases}
\item
  \href{https://www.nytimes3xbfgragh.onion/interactive/2020/science/coronavirus-vaccine-tracker.html?action=click\&pgtype=Article\&state=default\&region=TOP_BANNER\&context=storylines_menu}{Vaccine
  Tracker}
\item
  \href{https://www.nytimes3xbfgragh.onion/2020/08/02/us/covid-college-reopening.html?action=click\&pgtype=Article\&state=default\&region=TOP_BANNER\&context=storylines_menu}{College
  Reopening}
\item
  \href{https://www.nytimes3xbfgragh.onion/live/2020/08/04/business/stock-market-today-coronavirus?action=click\&pgtype=Article\&state=default\&region=TOP_BANNER\&context=storylines_menu}{Economy}
\end{itemize}

\includegraphics{https://static01.graylady3jvrrxbe.onion/images/2020/07/08/us/00virus-costs01/merlin_174339027_b56dac6e-5e55-4e28-8acb-df6545647200-articleLarge.jpg?quality=75\&auto=webp\&disable=upscale}

\hypertarget{big-mess-looms-if-schools-dont-get-billions-to-reopen-safely}{%
\section{`Big Mess' Looms if Schools Don't Get Billions to Reopen
Safely}\label{big-mess-looms-if-schools-dont-get-billions-to-reopen-safely}}

As President Trump and others push to get students back in the classroom
this fall, educators say they need more support and clearer guidelines.

A classroom at Marietta High School in Georgia.~The district plans to
spend \$200,000 on desk partitions. Credit...Audra Melton for The New
York Times

Supported by

\protect\hyperlink{after-sponsor}{Continue reading the main story}

\href{https://www.nytimes3xbfgragh.onion/by/dana-goldstein}{\includegraphics{https://static01.graylady3jvrrxbe.onion/images/2018/06/12/multimedia/author-dana-goldstein/author-dana-goldstein-thumbLarge.png}}

By \href{https://www.nytimes3xbfgragh.onion/by/dana-goldstein}{Dana
Goldstein}

\begin{itemize}
\item
  Published July 9, 2020Updated July 29, 2020
\item
  \begin{itemize}
  \item
  \item
  \item
  \item
  \item
  \end{itemize}
\end{itemize}

Bus monitors to screen students for symptoms in Marietta, Ga.:
\$640,000. Protective gear and classroom cleaning equipment for a small
district in rural Michigan: \$100,000. Disinfecting
\href{https://www.nytimes3xbfgragh.onion/2020/07/10/us/politics/trump-schools-reopening.html}{school}
buildings and hiring extra nurses and educators in San Diego: \$90
million.

As the White House, the
\href{https://www.nytimes3xbfgragh.onion/2020/06/30/us/coronavirus-schools-reopening-guidelines-aap.html}{nation's
pediatricians} and many worn-down, economically strapped parents push
for school doors to swing open this fall, local education officials say
they are being crushed by the costs of getting students and
\href{https://www.nytimes3xbfgragh.onion/2020/07/11/us/virus-teachers-classrooms.html}{teachers
back in classrooms safely}.

President
\href{https://www.nytimes3xbfgragh.onion/2020/07/10/us/politics/trump-schools-reopening.html}{Trump}
threatened this week to cut off federal funding to districts that do not
\href{https://www.nytimes3xbfgragh.onion/2020/07/10/us/politics/trump-schools-reopening.html}{reopen},
though he controls only a sliver of money for schools. But
administrators say they are already struggling to cover the
head-spinning logistical and financial challenges of retrofitting
buildings, adding staff members and protective gear, and providing
students with the academic and emotional support that many will need
after a traumatic disruption to their lives.

The federal relief package passed in March dedicated \$13.5 billion to
K-12 education --- less than 1 percent of the total stimulus. But
education groups estimate that schools will need many times that, and
with many local and state budgets already depleted by the economic
impact of the coronavirus, it is unclear where it will come from.

``If Congress doesn't do something in the summer, there is going to be a
big mess,'' said John Lee Evans, president of the San Diego Board of
Education.

Dr. Evans, a psychologist, said his district hoped to physically reopen
five days a week, starting Aug. 31, for families that want their
children to attend in-person classes. But it currently has the money to
do so safely for only half of the academic year, he said, and might need
to revert to online instruction after the winter holidays.

The district's superintendent, Cindy Marten, has been calling colleagues
in political swing states, like Ohio and Colorado, to ask them to lobby
their senators to pass additional education funding. Democratic plans in
Congress call for between
\href{https://blogs.edweek.org/edweek/campaign-k-12/2020/05/heroes-act-education-funding-house-passes.html}{\$58
billion} and
\href{https://www.help.senate.gov/imo/media/doc/CCCERA\%20Fact\%20Sheet\%20062920.pdf}{\$175
billion} for local schools, but those efforts lack significant
Republican support.

``It's incredible to me that the federal government would see the
necessity of bailing out airlines and banks,'' said Adam Goldstein, a
fifth-grade teacher in San Diego, ``and not see the need to do something
similar for the public schools in this country.''

Senator Mitch McConnell of Kentucky, the majority leader, has said he is
open to a ``final'' relief bill that would cover some of the expenses of
opening schools safely. ``We can't get back to normal if the kids are
not back in school,''
\href{https://www.courier-journal.com/story/news/politics/2020/07/08/cdc-guidelines-school-reopenings-mitch-mcconnell-trump-differ/5399417002/}{he
said this week}.

\hypertarget{latest-updates-global-coronavirus-outbreak}{%
\section{\texorpdfstring{\href{https://www.nytimes3xbfgragh.onion/2020/08/04/world/coronavirus-cases.html?action=click\&pgtype=Article\&state=default\&region=MAIN_CONTENT_1\&context=storylines_live_updates}{Latest
Updates: Global Coronavirus
Outbreak}}{Latest Updates: Global Coronavirus Outbreak}}\label{latest-updates-global-coronavirus-outbreak}}

Updated 2020-08-04T21:34:02.738Z

\begin{itemize}
\tightlist
\item
  \href{https://www.nytimes3xbfgragh.onion/2020/08/04/world/coronavirus-cases.html?action=click\&pgtype=Article\&state=default\&region=MAIN_CONTENT_1\&context=storylines_live_updates\#link-2daa96b5}{As
  talks drag on, McConnell signals openness to jobless aid extension
  that Republicans have opposed.}
\item
  \href{https://www.nytimes3xbfgragh.onion/2020/08/04/world/coronavirus-cases.html?action=click\&pgtype=Article\&state=default\&region=MAIN_CONTENT_1\&context=storylines_live_updates\#link-1228a480}{Novavax
  sees encouraging results from two studies of its experimental
  vaccine.}
\item
  \href{https://www.nytimes3xbfgragh.onion/2020/08/04/world/coronavirus-cases.html?action=click\&pgtype=Article\&state=default\&region=MAIN_CONTENT_1\&context=storylines_live_updates\#link-4825b93}{Public
  and private schools in Maryland and elsewhere are divided over
  in-person instruction.}
\end{itemize}

\href{https://www.nytimes3xbfgragh.onion/2020/08/04/world/coronavirus-cases.html?action=click\&pgtype=Article\&state=default\&region=MAIN_CONTENT_1\&context=storylines_live_updates}{See
more updates}

More live coverage:
\href{https://www.nytimes3xbfgragh.onion/live/2020/08/04/business/stock-market-today-coronavirus?action=click\&pgtype=Article\&state=default\&region=MAIN_CONTENT_1\&context=storylines_live_updates}{Markets}

Exactly how much money the nation's
\href{https://www.nytimes3xbfgragh.onion/2020/07/16/upshot/coronavirus-school-reopening-private-public-gap.html}{schools
need to reopen} is a matter of debate, complicated by the conflicting,
sometimes shifting advice and guidelines that administrators have
received from government agencies and medical authorities.

In May, the Centers for Disease Control and Prevention
\href{https://www.cdc.gov/coronavirus/2019-ncov/community/schools-childcare/schools.html}{said
that remote learning} was the safest option, and that students in
classrooms should remain six feet apart ``when feasible.'' A six-foot
requirement would mean that many schools could accommodate half of their
students or fewer at any given time.

Given that, many systems
\href{https://www.nytimes3xbfgragh.onion/2020/06/26/us/coronavirus-schools-reopen-fall.html}{planned
a hybrid approach}, with students splitting their time between
classrooms and remote learning. On Wednesday, Mayor Bill de Blasio of
New York said students in the nation's largest school district would
most
\href{https://www.nytimes3xbfgragh.onion/2020/07/08/nyregion/nyc-schools-reopening-plan.html}{likely
attend only one to three days per week} if schools open in September.

The Trump administration has promoted an alternate set of guidelines
\href{https://www.nytimes3xbfgragh.onion/2020/06/30/us/coronavirus-schools-reopening-guidelines-aap.html}{from
the American Academy of Pediatrics} that suggest three feet of physical
distance may be sufficient in classrooms if students wear masks.

Regardless of which recommendations are followed,
\href{https://www.nytimes3xbfgragh.onion/2020/07/29/magazine/schools-reopening-covid.html}{reopening
schools} will require changes. An average-size district of 3,700
students can expect \$1.8 million in pandemic-related costs for 2020-21,
representing 3 to 4 percent of a typical annual budget, according to an
estimate from AASA, the School Superintendents Association. Districts
say they typically operate on tight budgets, and even more so this year
as state and local tax revenues run low.

But some experts have suggested that much of what schools are planning
for the fall, such as checking students' symptoms before they board
buses or enter schools, is unnecessary, akin to the type of
``\href{https://onezero.medium.com/the-public-is-being-misled-by-pandemic-technology-that-wont-keep-them-safe-1966ed740a87}{security
theater}'' Americans became accustomed to after the terrorist attacks of
Sept. 11, 2001 --- reassuring, but often providing only superficial
protection.

That criticism is driven in part by emerging research suggesting that
children are not only
\href{https://pubmed.ncbi.nlm.nih.gov/32546824/}{less likely to be
infected by the coronavirus}, but also
\href{https://pediatrics.aappublications.org/content/early/2020/05/22/peds.2020-004879}{less
likely to transmit it} to one another or to adults. Some school finance
experts have also suggested that districts need to make tougher choices,
prioritizing core educational functions.

Marguerite Roza, director of the
\href{https://edunomicslab.org/}{Edunomics Lab} at Georgetown
University, suggested that schools could save money by holding core
classes in large spaces like auditoriums or gyms, allowing a single
teacher to work with more students while keeping everyone physically
distanced.

Too few systems, Professor Roza said, were willing to delay planned pay
raises for teachers or furlough unneeded staff members. She also
suggested cutting programs like indoor sports and chorus, which may not
be safe this year because they spread respiratory droplets that can
transmit the coronavirus.

Image

The small school district in Dundee, Mich., has spent over \$100,000 on
masks, gloves, thermometers and ionizing mist machines to disinfect
classrooms.Credit...Sylvia Jarrus for The New York Times

Image

The cross-country team practiced outside of Dundee Community High School
this week. The district reallocated \$350,000 intended for a new
athletics complex.Credit...Sylvia Jarrus for The New York Times

Image

The district in Marietta, Ga., plans to hire bus monitors to screen
symptoms and a staff member to assist with contact
tracing.Credit...Audra Melton for The New York Times

Image

A sign reminding people to practice six feet of social distancing hangs
outside of the cafeteria at Marietta High School.Credit...Audra Melton
for The New York Times

In much of the country, such changes require consideration of teacher
contracts. Like many districts, San Diego is currently in tense
negotiations with its teachers' union regarding return-to-work
conditions.

A key demand of many unions is to protect the jobs of teachers who are
at high risk of serious illness if they contract the virus, or who live
with someone who is high risk. Many of those teachers prefer to continue
to work remotely. Nationwide,
\href{https://nces.ed.gov/pubs2020/2020142.pdf}{more than a quarter} of
the public school teaching force is over the age of 50.

Mr. Goldstein said he was eager to return to his fifth-grade classroom
in San Diego. But like many teachers across the country, he was not yet
convinced that his district had enough money, or the detailed plans in
place, to protect the health of employees and students, especially given
rising coronavirus caseloads
\href{https://www.nytimes3xbfgragh.onion/interactive/2020/us/california-coronavirus-cases.html}{in
California} and many other states.

\href{https://www.nytimes3xbfgragh.onion/news-event/coronavirus?action=click\&pgtype=Article\&state=default\&region=MAIN_CONTENT_3\&context=storylines_faq}{}

\hypertarget{the-coronavirus-outbreak-}{%
\subsubsection{The Coronavirus Outbreak
›}\label{the-coronavirus-outbreak-}}

\hypertarget{frequently-asked-questions}{%
\paragraph{Frequently Asked
Questions}\label{frequently-asked-questions}}

Updated August 4, 2020

\begin{itemize}
\item ~
  \hypertarget{i-have-antibodies-am-i-now-immune}{%
  \paragraph{I have antibodies. Am I now
  immune?}\label{i-have-antibodies-am-i-now-immune}}

  \begin{itemize}
  \tightlist
  \item
    As of right
    now,\href{https://www.nytimes3xbfgragh.onion/2020/07/22/health/covid-antibodies-herd-immunity.html?action=click\&pgtype=Article\&state=default\&region=MAIN_CONTENT_3\&context=storylines_faq}{that
    seems likely, for at least several months.} There have been
    frightening accounts of people suffering what seems to be a second
    bout of Covid-19. But experts say these patients may have a
    drawn-out course of infection, with the virus taking a slow toll
    weeks to months after initial exposure. People infected with the
    coronavirus typically
    \href{https://www.nature.com/articles/s41586-020-2456-9}{produce}
    immune molecules called antibodies, which are
    \href{https://www.nytimes3xbfgragh.onion/2020/05/07/health/coronavirus-antibody-prevalence.html?action=click\&pgtype=Article\&state=default\&region=MAIN_CONTENT_3\&context=storylines_faq}{protective
    proteins made in response to an
    infection}\href{https://www.nytimes3xbfgragh.onion/2020/05/07/health/coronavirus-antibody-prevalence.html?action=click\&pgtype=Article\&state=default\&region=MAIN_CONTENT_3\&context=storylines_faq}{.
    These antibodies may} last in the body
    \href{https://www.nature.com/articles/s41591-020-0965-6}{only two to
    three months}, which may seem worrisome, but that's perfectly normal
    after an acute infection subsides, said Dr. Michael Mina, an
    immunologist at Harvard University. It may be possible to get the
    coronavirus again, but it's highly unlikely that it would be
    possible in a short window of time from initial infection or make
    people sicker the second time.
  \end{itemize}
\item ~
  \hypertarget{im-a-small-business-owner-can-i-get-relief}{%
  \paragraph{I'm a small-business owner. Can I get
  relief?}\label{im-a-small-business-owner-can-i-get-relief}}

  \begin{itemize}
  \tightlist
  \item
    The
    \href{https://www.nytimes3xbfgragh.onion/article/small-business-loans-stimulus-grants-freelancers-coronavirus.html?action=click\&pgtype=Article\&state=default\&region=MAIN_CONTENT_3\&context=storylines_faq}{stimulus
    bills enacted in March} offer help for the millions of American
    small businesses. Those eligible for aid are businesses and
    nonprofit organizations with fewer than 500 workers, including sole
    proprietorships, independent contractors and freelancers. Some
    larger companies in some industries are also eligible. The help
    being offered, which is being managed by the Small Business
    Administration, includes the Paycheck Protection Program and the
    Economic Injury Disaster Loan program. But lots of folks have
    \href{https://www.nytimes3xbfgragh.onion/interactive/2020/05/07/business/small-business-loans-coronavirus.html?action=click\&pgtype=Article\&state=default\&region=MAIN_CONTENT_3\&context=storylines_faq}{not
    yet seen payouts.} Even those who have received help are confused:
    The rules are draconian, and some are stuck sitting on
    \href{https://www.nytimes3xbfgragh.onion/2020/05/02/business/economy/loans-coronavirus-small-business.html?action=click\&pgtype=Article\&state=default\&region=MAIN_CONTENT_3\&context=storylines_faq}{money
    they don't know how to use.} Many small-business owners are getting
    less than they expected or
    \href{https://www.nytimes3xbfgragh.onion/2020/06/10/business/Small-business-loans-ppp.html?action=click\&pgtype=Article\&state=default\&region=MAIN_CONTENT_3\&context=storylines_faq}{not
    hearing anything at all.}
  \end{itemize}
\item ~
  \hypertarget{what-are-my-rights-if-i-am-worried-about-going-back-to-work}{%
  \paragraph{What are my rights if I am worried about going back to
  work?}\label{what-are-my-rights-if-i-am-worried-about-going-back-to-work}}

  \begin{itemize}
  \tightlist
  \item
    Employers have to provide
    \href{https://www.osha.gov/SLTC/covid-19/standards.html}{a safe
    workplace} with policies that protect everyone equally.
    \href{https://www.nytimes3xbfgragh.onion/article/coronavirus-money-unemployment.html?action=click\&pgtype=Article\&state=default\&region=MAIN_CONTENT_3\&context=storylines_faq}{And
    if one of your co-workers tests positive for the coronavirus, the
    C.D.C.} has said that
    \href{https://www.cdc.gov/coronavirus/2019-ncov/community/guidance-business-response.html}{employers
    should tell their employees} -\/- without giving you the sick
    employee's name -\/- that they may have been exposed to the virus.
  \end{itemize}
\item ~
  \hypertarget{should-i-refinance-my-mortgage}{%
  \paragraph{Should I refinance my
  mortgage?}\label{should-i-refinance-my-mortgage}}

  \begin{itemize}
  \tightlist
  \item
    \href{https://www.nytimes3xbfgragh.onion/article/coronavirus-money-unemployment.html?action=click\&pgtype=Article\&state=default\&region=MAIN_CONTENT_3\&context=storylines_faq}{It
    could be a good idea,} because mortgage rates have
    \href{https://www.nytimes3xbfgragh.onion/2020/07/16/business/mortgage-rates-below-3-percent.html?action=click\&pgtype=Article\&state=default\&region=MAIN_CONTENT_3\&context=storylines_faq}{never
    been lower.} Refinancing requests have pushed mortgage applications
    to some of the highest levels since 2008, so be prepared to get in
    line. But defaults are also up, so if you're thinking about buying a
    home, be aware that some lenders have tightened their standards.
  \end{itemize}
\item ~
  \hypertarget{what-is-school-going-to-look-like-in-september}{%
  \paragraph{What is school going to look like in
  September?}\label{what-is-school-going-to-look-like-in-september}}

  \begin{itemize}
  \tightlist
  \item
    It is unlikely that many schools will return to a normal schedule
    this fall, requiring the grind of
    \href{https://www.nytimes3xbfgragh.onion/2020/06/05/us/coronavirus-education-lost-learning.html?action=click\&pgtype=Article\&state=default\&region=MAIN_CONTENT_3\&context=storylines_faq}{online
    learning},
    \href{https://www.nytimes3xbfgragh.onion/2020/05/29/us/coronavirus-child-care-centers.html?action=click\&pgtype=Article\&state=default\&region=MAIN_CONTENT_3\&context=storylines_faq}{makeshift
    child care} and
    \href{https://www.nytimes3xbfgragh.onion/2020/06/03/business/economy/coronavirus-working-women.html?action=click\&pgtype=Article\&state=default\&region=MAIN_CONTENT_3\&context=storylines_faq}{stunted
    workdays} to continue. California's two largest public school
    districts --- Los Angeles and San Diego --- said on July 13, that
    \href{https://www.nytimes3xbfgragh.onion/2020/07/13/us/lausd-san-diego-school-reopening.html?action=click\&pgtype=Article\&state=default\&region=MAIN_CONTENT_3\&context=storylines_faq}{instruction
    will be remote-only in the fall}, citing concerns that surging
    coronavirus infections in their areas pose too dire a risk for
    students and teachers. Together, the two districts enroll some
    825,000 students. They are the largest in the country so far to
    abandon plans for even a partial physical return to classrooms when
    they reopen in August. For other districts, the solution won't be an
    all-or-nothing approach.
    \href{https://bioethics.jhu.edu/research-and-outreach/projects/eschool-initiative/school-policy-tracker/}{Many
    systems}, including the nation's largest, New York City, are
    devising
    \href{https://www.nytimes3xbfgragh.onion/2020/06/26/us/coronavirus-schools-reopen-fall.html?action=click\&pgtype=Article\&state=default\&region=MAIN_CONTENT_3\&context=storylines_faq}{hybrid
    plans} that involve spending some days in classrooms and other days
    online. There's no national policy on this yet, so check with your
    municipal school system regularly to see what is happening in your
    community.
  \end{itemize}
\end{itemize}

He would feel more assured, he said, if teachers could be guaranteed
small class sizes --- he had 35 students on his roster last year --- and
if school staff and students could be regularly tested for the virus,
something the C.D.C.
\href{https://www.cdc.gov/coronavirus/2019-ncov/community/schools-childcare/k-12-testing.html}{said
was unnecessary}.

Administrators forging ahead with plans for full-time physical school,
even as the virus rages, may have been seized by a type of ``wishful
thinking,'' Mr. Goldstein added.

Ms. Marten, the San Diego superintendent, has the job of trying to
balance all of the competing needs and requirements to get students and
teachers back in classrooms.

Closing school buildings in the spring brought the district some savings
--- lower utility bills and fuel costs for vehicles, for example --- but
did not come close to covering the \$30 million it took to transition to
remote learning, she said. The district took on a variety of expenses,
including meal distribution for low-income families and providing laptop
chargers and home internet connections for students who lacked them. It
also offered hazard pay to staff members who delivered those meals and
devices.

The district has not made significant program or staff cuts.

\includegraphics{https://static01.graylady3jvrrxbe.onion/images/2020/07/08/us/00virus-costs06/merlin_174339087_3f840f77-d577-4314-9c40-91c54e9519a0-articleLarge.jpg?quality=75\&auto=webp\&disable=upscale}

When schools shut down in Dundee, Mich., the rural district of 1,700
students an hour from Detroit did reallocate \$350,000 intended for a
new athletics complex, according to the superintendent, Edward Manuszak.
It spent much of that money preserving jobs.

Mr. Manuszak said that one of his priorities was maintaining trust
between the district, its employees and the community during a difficult
time.

Superintendents in districts without collective bargaining have more
flexibility in how they staff their schools under pandemic conditions.
Teachers who work for Marietta City Schools in Georgia can ask to go on
leave if they are uncomfortable returning to physical classrooms, said
Grant Rivera, the superintendent. But they might not receive full pay
and will not have the opportunity to work from home, he said.

Dr. Rivera plans to staff the district's remote learning program, which
will be an option for any family that chooses it, from among those
teachers who are willing to return to school and who have also
demonstrated skill in online instruction.

He faces a daunting to-do list as he prepares to reopen the school
system on Aug. 4. The district plans to spend \$200,000 to install desk
partitions in classrooms where it will be impossible for students and
teachers to stay six feet apart. It will also distribute masks, which
will be required for teachers and students, and hire an employee to
assist with contact tracing.

To pay for it all, the superintendent is looking for philanthropic
support, lobbying his county for \$2.9 million in additional funds and
also hoping for federal aid.

``You can see what they value,'' he said of political leaders, ``by what
they spend their money on.''

Advertisement

\protect\hyperlink{after-bottom}{Continue reading the main story}

\hypertarget{site-index}{%
\subsection{Site Index}\label{site-index}}

\hypertarget{site-information-navigation}{%
\subsection{Site Information
Navigation}\label{site-information-navigation}}

\begin{itemize}
\tightlist
\item
  \href{https://help.nytimes3xbfgragh.onion/hc/en-us/articles/115014792127-Copyright-notice}{©~2020~The
  New York Times Company}
\end{itemize}

\begin{itemize}
\tightlist
\item
  \href{https://www.nytco.com/}{NYTCo}
\item
  \href{https://help.nytimes3xbfgragh.onion/hc/en-us/articles/115015385887-Contact-Us}{Contact
  Us}
\item
  \href{https://www.nytco.com/careers/}{Work with us}
\item
  \href{https://nytmediakit.com/}{Advertise}
\item
  \href{http://www.tbrandstudio.com/}{T Brand Studio}
\item
  \href{https://www.nytimes3xbfgragh.onion/privacy/cookie-policy\#how-do-i-manage-trackers}{Your
  Ad Choices}
\item
  \href{https://www.nytimes3xbfgragh.onion/privacy}{Privacy}
\item
  \href{https://help.nytimes3xbfgragh.onion/hc/en-us/articles/115014893428-Terms-of-service}{Terms
  of Service}
\item
  \href{https://help.nytimes3xbfgragh.onion/hc/en-us/articles/115014893968-Terms-of-sale}{Terms
  of Sale}
\item
  \href{https://spiderbites.nytimes3xbfgragh.onion}{Site Map}
\item
  \href{https://help.nytimes3xbfgragh.onion/hc/en-us}{Help}
\item
  \href{https://www.nytimes3xbfgragh.onion/subscription?campaignId=37WXW}{Subscriptions}
\end{itemize}
