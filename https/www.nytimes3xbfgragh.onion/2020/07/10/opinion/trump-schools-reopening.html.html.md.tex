Sections

SEARCH

\protect\hyperlink{site-content}{Skip to
content}\protect\hyperlink{site-index}{Skip to site index}

\href{https://myaccount.nytimes3xbfgragh.onion/auth/login?response_type=cookie\&client_id=vi}{}

\href{https://www.nytimes3xbfgragh.onion/section/todayspaper}{Today's
Paper}

\href{/section/opinion}{Opinion}\textbar{}Trump Would Like to See You
Now

\href{https://nyti.ms/322C6vj}{https://nyti.ms/322C6vj}

\begin{itemize}
\item
\item
\item
\item
\item
\item
\end{itemize}

Advertisement

\protect\hyperlink{after-top}{Continue reading the main story}

\href{/section/opinion}{Opinion}

Supported by

\protect\hyperlink{after-sponsor}{Continue reading the main story}

\hypertarget{trump-would-like-to-see-you-now}{%
\section{Trump Would Like to See You
Now}\label{trump-would-like-to-see-you-now}}

Why does the president want to open everything up? It's not because it's
safe to do so.

\href{https://www.nytimes3xbfgragh.onion/column/jamelle-bouie}{\includegraphics{https://static01.graylady3jvrrxbe.onion/images/2019/01/24/opinion/jamelle-bouie/jamelle-bouie-thumbLarge-v3.png}}

By
\href{https://www.nytimes3xbfgragh.onion/column/jamelle-bouie}{Jamelle
Bouie}

Opinion Columnist

\begin{itemize}
\item
  July 10, 2020
\item
  \begin{itemize}
  \item
  \item
  \item
  \item
  \item
  \item
  \end{itemize}
\end{itemize}

\includegraphics{https://static01.graylady3jvrrxbe.onion/images/2020/07/13/opinion/13bouie_print_top/merlin_174326484_ccde26ee-a45a-4f4f-a8a1-72668ae07f31-articleLarge.jpg?quality=75\&auto=webp\&disable=upscale}

\hypertarget{listen-to-this-op-ed}{%
\subsubsection{Listen to This Op-Ed}\label{listen-to-this-op-ed}}

Audio Recording by Audm

\emph{To hear more audio stories from publishers like The New York
Times,
download}\href{https://www.audm.com/?utm_source=nytmag\&utm_medium=embed\&utm_campaign=left_behind_draper}{**}\href{https://www.audm.com/?utm_source=nytopinion\&utm_medium=embed\&utm_campaign=trump_see_you}{\emph{Audm
for iPhone or Android}}\emph{.}

Back in March, when Congress was debating pandemic relief, Senator Rick
Scott of Florida
\href{https://twitter.com/thehill/status/1242894562823151616?s=21}{spoke
out} against a Democratic plan to greatly expand federal unemployment
insurance. ``The moment we can go back to work, we cannot create an
incentive for people to say `I don't need to come back to work because I
can do better some place else,' '' Scott said at a news conference in
support of an amendment that would strike the program from the bill.
``These employers are going to need these workers to rebuild this
economy, so we cannot pay people more money on unemployment than what
they would get in their jobs.''

\href{https://thehill.com/homenews/senate/489589-senate-rejects-gop-attempt-to-change-unemployment-benefits-in-stimulus-bill}{Most
Republican senators} voted to remove the unemployment expansion at its
full size, but it survived. Billions of dollars of benefits have gone to
tens of millions of Americans. The increase in aid was so great that, as
The New York Times
\href{https://www.nytimes3xbfgragh.onion/2020/06/21/us/politics/coronavirus-poverty.html\#click=https://t.co/6LfQmIUQ2G}{reported}
last month, the federal poverty rate declined even as the jobless rate
reached incredible heights. And there's also
\href{https://www.cnbc.com/2020/06/25/people-receiving-unemployment-benefits-are-more-likely-to-look-for-jobs.html}{no
evidence} that additional benefits are keeping people who want to work
from working.

But while that is important, I'm less interested in the trajectory of
the Cares Act than I am the nature of Scott's opposition. The Florida
senator (and former governor) wasn't so much concerned with the ability
of people to work as much as he was with the ability of employers to
discipline them. Workers are kept on edge --- and willing to accept
whatever wage is on offer --- by the threat of immiseration. This, for
politicians who back both big business and existing social relations, is
a feature and not a bug of our economic system, since insecurity and
desperation keep power in the hands of capital and its allies. Even
something as modest as expanded unemployment benefits is a threat to
that arrangement, as they give workers the power to say no to work they
do not want.

We should keep all of this in mind as we try to understand the
\href{https://www.nytimes3xbfgragh.onion/2020/07/10/us/politics/trump-schools-reopening.html}{Trump}
administration's response to the Covid-19 economic crisis, whether it is
the hostility to more stimulus, the indifference to the mounting
eviction crisis, the opposition to state budget aid, the drive to reopen
businesses, or the current push to
\href{https://www.nytimes3xbfgragh.onion/2020/07/10/us/politics/trump-schools-reopening.html}{reopen
public schools}, even as the virus rages nearly out of control in huge
sections of the country.

Yes, you can understand the president's approach as an attempt to goose
the economy enough for him to win a second term (Democrats ``don't want
to reopen because they think it will help them on Nov. 3,''
\href{https://twitter.com/joshtpm/status/1281320469921136641?s=21}{Trump
said} on Thursday). But there's a reason his business allies are
\href{https://apnews.com/57f673f31fb343042e3806b9806e8f7d}{committed} to
the same course of action. A forced reopening helps keep the market
afloat; it is what you would do if you were trying to protect capital
from any serious losses. And it is exactly what you would expect from an
administration whose central aim, beyond immigration restrictionism, is
the upward redistribution of wealth to heirs, owners and industry.

Let's turn back to schools. Trump, again, wants them open. ``In Germany,
Denmark, Norway, Sweden and many other countries, SCHOOLS ARE OPEN WITH
NO PROBLEMS,'' the president
\href{https://twitter.com/realdonaldtrump/status/1280853299600789505?s=21}{said
on Twitter on Wednesday}. ``The Dems think it would be bad for them
politically if U.S. schools open before the November Election, but is
important for the children \& families.'' He also threatened to ``cut
off funding if not open.'' His education secretary, Betsy DeVos, said
the same. ``Kids have to continue learning and schools have to open
up,'' she
\href{https://thehill.com/homenews/administration/506427-devos-very-seriously-considering-withholding-funding-from-schools}{told
Fox News}, announcing, like Trump, that she is ``very seriously''
considering withholding federal funding.

But it was Trump's secretary of labor, Eugene Scalia, who made clear why
the administration is so eager to open schools, even as the pandemic
rages out of control. ``One study has suggested that if we closed all
our schools and day care for just a month --- just, hypothetically, if
we did that --- the impact on U.S. productivity would be in the order of
\$50 billion,''
\href{https://www.whitehouse.gov/briefings-statements/press-briefing-vice-president-pence-members-coronavirus-task-force-july-8-2020/}{he
said} at a Wednesday press briefing for the White House Coronavirus Task
Force.

Many parents, for good reason, want to send their children back to
school. Others aren't so sure. The situation is dire. But that has
everything to do with the priorities of an administration that shows no
real interest in fighting the virus and has done everything it can to
prevent additional lockdowns.

And so schools have to open because parents have to work, and parents
have to work because the president opposes any additional aid to
American families, who might stay home and avoid the virus if they had
the choice. This would raise the unemployment rate --- possibly
jeopardizing Mr. Trump's chances for re-election --- and redistribute
power from employers to workers, while also strengthening the case for a
robust and generous social safety net.

Millions of Americans are in the impossible situation of juggling work
and child care while protecting their families from a deadly virus, and
it's because the White House and its allies would rather try to save the
stock market and pursue narrow ideological goals than try to preserve
the fabric of this society.

``The ultimate expression of sovereignty resides, to a large degree, in
the power and capacity to dictate who may live and who must die,'' the
Cameroonian philosopher Achille Mbembe wrote
\href{https://warwick.ac.uk/fac/arts/english/currentstudents/postgraduate/masters/modules/postcol_theory/mbembe_22necropolitics22.pdf}{in
a 2003 essay called ``Necropolitics.''} ``Hence, to kill or allow to
live constitutes the limits of sovereignty, its fundamental
attributes.'' I read this line not long before the pandemic reached
American shores, and I've been thinking about it ever since.

President Trump has lately refused almost any action to control the
virus and largely abdicated his responsibility for helping Americans
weather the economic crisis. But he has pushed meatpackers to go back
into dangerous plants, urged businesses to reopen despite danger to the
public and hindered the production and distribution of protective gear
and other critical materials.

Trump has power. But in the face of Covid-19, he doesn't use it to
facilitate life as much as he does to dictate exposure to death.

\emph{The Times is committed to publishing}
\href{https://www.nytimes3xbfgragh.onion/2019/01/31/opinion/letters/letters-to-editor-new-york-times-women.html}{\emph{a
diversity of letters}} \emph{to the editor. We'd like to hear what you
think about this or any of our articles. Here are some}
\href{https://help.nytimes3xbfgragh.onion/hc/en-us/articles/115014925288-How-to-submit-a-letter-to-the-editor}{\emph{tips}}\emph{.
And here's our email:}
\href{mailto:letters@NYTimes.com}{\emph{letters@NYTimes.com}}\emph{.}

\emph{Follow The New York Times Opinion section on}
\href{https://www.facebookcorewwwi.onion/nytopinion}{\emph{Facebook}}\emph{,}
\href{http://twitter.com/NYTOpinion}{\emph{Twitter (@NYTopinion)}}
\emph{and}
\href{https://www.instagram.com/nytopinion/}{\emph{Instagram}}\emph{.}

Advertisement

\protect\hyperlink{after-bottom}{Continue reading the main story}

\hypertarget{site-index}{%
\subsection{Site Index}\label{site-index}}

\hypertarget{site-information-navigation}{%
\subsection{Site Information
Navigation}\label{site-information-navigation}}

\begin{itemize}
\tightlist
\item
  \href{https://help.nytimes3xbfgragh.onion/hc/en-us/articles/115014792127-Copyright-notice}{©~2020~The
  New York Times Company}
\end{itemize}

\begin{itemize}
\tightlist
\item
  \href{https://www.nytco.com/}{NYTCo}
\item
  \href{https://help.nytimes3xbfgragh.onion/hc/en-us/articles/115015385887-Contact-Us}{Contact
  Us}
\item
  \href{https://www.nytco.com/careers/}{Work with us}
\item
  \href{https://nytmediakit.com/}{Advertise}
\item
  \href{http://www.tbrandstudio.com/}{T Brand Studio}
\item
  \href{https://www.nytimes3xbfgragh.onion/privacy/cookie-policy\#how-do-i-manage-trackers}{Your
  Ad Choices}
\item
  \href{https://www.nytimes3xbfgragh.onion/privacy}{Privacy}
\item
  \href{https://help.nytimes3xbfgragh.onion/hc/en-us/articles/115014893428-Terms-of-service}{Terms
  of Service}
\item
  \href{https://help.nytimes3xbfgragh.onion/hc/en-us/articles/115014893968-Terms-of-sale}{Terms
  of Sale}
\item
  \href{https://spiderbites.nytimes3xbfgragh.onion}{Site Map}
\item
  \href{https://help.nytimes3xbfgragh.onion/hc/en-us}{Help}
\item
  \href{https://www.nytimes3xbfgragh.onion/subscription?campaignId=37WXW}{Subscriptions}
\end{itemize}
