Sections

SEARCH

\protect\hyperlink{site-content}{Skip to
content}\protect\hyperlink{site-index}{Skip to site index}

\href{https://www.nytimes3xbfgragh.onion/section/technology}{Technology}

\href{https://myaccount.nytimes3xbfgragh.onion/auth/login?response_type=cookie\&client_id=vi}{}

\href{https://www.nytimes3xbfgragh.onion/section/todayspaper}{Today's
Paper}

\href{/section/technology}{Technology}\textbar{}Amazon Backtracks From
Demand That Employees Delete TikTok

\url{https://nyti.ms/2OfhrMt}

\begin{itemize}
\item
\item
\item
\item
\item
\item
\end{itemize}

Advertisement

\protect\hyperlink{after-top}{Continue reading the main story}

Supported by

\protect\hyperlink{after-sponsor}{Continue reading the main story}

\hypertarget{amazon-backtracks-from-demand-that-employees-delete-tiktok}{%
\section{Amazon Backtracks From Demand That Employees Delete
TikTok}\label{amazon-backtracks-from-demand-that-employees-delete-tiktok}}

TikTok, owned by the Chinese company ByteDance, has been under scrutiny
as a potential national security threat.

\includegraphics{https://static01.graylady3jvrrxbe.onion/images/2020/07/11/business/08JPamazontiktok-print/merlin_174043584_e9748acd-bdba-4ed4-9c1c-afe9ac329074-articleLarge.jpg?quality=75\&auto=webp\&disable=upscale}

\href{https://www.nytimes3xbfgragh.onion/by/mike-isaac}{\includegraphics{https://static01.graylady3jvrrxbe.onion/images/2018/02/16/multimedia/author-mike-isaac/author-mike-isaac-thumbLarge.jpg}}\href{https://www.nytimes3xbfgragh.onion/by/karen-weise}{\includegraphics{https://static01.graylady3jvrrxbe.onion/images/2019/04/11/multimedia/author-karen-weise/author-karen-weise-thumbLarge.png}}

By \href{https://www.nytimes3xbfgragh.onion/by/mike-isaac}{Mike Isaac}
and \href{https://www.nytimes3xbfgragh.onion/by/karen-weise}{Karen
Weise}

\begin{itemize}
\item
  July 10, 2020
\item
  \begin{itemize}
  \item
  \item
  \item
  \item
  \item
  \item
  \end{itemize}
\end{itemize}

SAN FRANCISCO --- Amazon on Friday asked its employees to delete the
Chinese-owned video app
\href{https://www.nytimes3xbfgragh.onion/2020/08/03/technology/trump-tiktok-microsoft.html}{TikTok}
from their cellphones, putting the tech giant at the center of growing
suspicion and paranoia about the app.

Almost five hours later, Amazon reversed course, saying the email to
workers was sent in error.

In the initial email, which was obtained by The New York Times, Amazon
officials said that because of ``security risks,'' employees must delete
the app from any devices that ``access Amazon email.'' Employees had to
remove the app by Friday to remain able to obtain mobile access to their
Amazon email, the note said.

In a statement sent later on Friday, company spokeswoman Kristin Brown
said, ``There is no change to our policies right now with regard to
TikTok.''

But by then, the initial email had already added to the storm
surrounding TikTok, which has been popular with young audiences in the
United States for its short, fun videos and is owned by the Chinese tech
company ByteDance. Because of its Chinese ownership and heightened
tensions between the United States and China over issues such as trade
and technology dominance, TikTok has come
\href{https://www.nytimes3xbfgragh.onion/2019/11/01/technology/tiktok-national-security-review.html}{under
increasing scrutiny in Washington} over its security.

Mike Pompeo, the secretary of state, said on Monday that the Trump
administration was considering blocking some Chinese apps, which he has
called a threat to national security. Many users who have built
community and business on TikTok
\href{https://www.nytimes3xbfgragh.onion/2020/07/10/style/tiktok-ban-us-users-influencers.html}{are
fearful} of a broad ban. Some Amazon employees publicly shared dismay at
not being able to use the app.

Last year, the Committee on Foreign Investment in the United States, a
federal panel that reviews foreign acquisitions of American firms on
national security grounds,
\href{https://www.nytimes3xbfgragh.onion/2019/11/01/technology/tiktok-national-security-review.html}{opened
a national security review} of ByteDance's acquisition of the American
company, Musical.ly, which eventually became TikTok.

In December, the Defense Department began telling military personnel
\href{https://www.nytimes3xbfgragh.onion/2020/01/04/us/tiktok-pentagon-military-ban.html}{to
delete the app} from government-issued phones. The same month, the
Democratic National Committee warned campaigns, committees and state
parties about TikTok's ``Chinese ties and potentially sending data back
to the Chinese government.''

With TikTok making headlines for similar security concerns, the D.N.C.
reiterated the previous warnings about TikTok in one of its regular
security emails this week.

Other companies are scrutinizing use of the app among employees. Wells
Fargo said it told some workers who had installed TikTok on
company-owned phones to delete the app. ``Due to concerns about TikTok's
privacy and security controls and practices, and because corporate-owned
devices should be used for company business only, we have directed those
employees to remove the app from their devices,'' a Wells Fargo
spokeswoman said in a statement.

ByteDance has made a series of moves in response to the concerns. The
company said that it would separate TikTok from much of its Chinese
operations, and that users' personal data would be stored in the United
States and not in China. In May, ByteDance
\href{https://www.nytimes3xbfgragh.onion/2020/05/18/business/media/tiktok-ceo-kevin-mayer.html}{hired
Kevin Mayer, a former Disney executive}, to be chief executive of TikTok
based in Los Angeles. It has said that managers outside China call the
shots on key aspects of its business, including rules about data.

On Monday, TikTok also said that it would
\href{https://www.nytimes3xbfgragh.onion/2020/07/07/business/hong-kong-security-law-tech.html}{withdraw
from app stores in Hong Kong}, where a new national security law from
China was enacted. The company said it would make the app inoperable to
users there within a few days.

After Amazon's first email on Friday, TikTok said in a statement that
user security was ``of the utmost importance'' and that it was committed
to user privacy. It added, ``While Amazon did not communicate to us
before sending their email, and we still do not understand their
concerns, we welcome a dialogue.''

Before Amazon sent out its second message on Friday, Senator Josh
Hawley, Republican of Missouri, who has called for investigations into
the national security ramifications of Chinese apps, said, ``The whole
federal government should follow suit.''

TikTok has long been a concern of American intelligence officials, who
fear the social networking app is a thinly veiled data collection
service. Over the past six months, security researchers have only
furthered those concerns with a series of discoveries.

Last month, a researcher uncovered that TikTok had the ability to siphon
off anything a user copied to a clipboard on a smartphone --- passwords,
photos and other sensitive data like Social Security numbers, emails and
texts. The researcher began posting the findings on the online message
board Reddit.

The researcher, who goes by the handle Bangorlol, also said that TikTok
was capturing data about a user's phone hardware and data on other apps
installed on the phone. Many of these abilities are found in other apps,
but TikTok's developers had gone out of their way to prevent anyone from
analyzing the app, the researcher said.

``This was very concerning and very rare,'' Oded Vanunu, who leads
research into product vulnerability at the Israeli security firm Check
Point, said about the findings. ``There's been a lot of fear and
speculation about this app, but the recent findings are raising big
questions.''

TikTok's Chinese ownership has also been problematic for other
governments.
\href{https://www.nytimes3xbfgragh.onion/2020/06/29/world/asia/tik-tok-banned-india-china.html}{India
banned nearly 60 Chinese mobile apps} including
\href{https://www.nytimes3xbfgragh.onion/2020/06/30/technology/india-china-tiktok.html}{TikTok}
last month, citing national security concerns. India and China recently
clashed along a disputed border,
\href{https://www.nytimes3xbfgragh.onion/2020/06/16/world/asia/indian-china-border-clash.html}{leaving
20 Indian soldiers dead} and an unknown number of Chinese casualties.

The video app has
\href{https://www.nytimes3xbfgragh.onion/2019/03/10/style/what-is-tik-tok.html}{soared
in popularity} over the past few years, especially among teenagers and
young adults, who use it to make and share short videos. The app has
minted its own influencers, with celebrities such as Reese Witherspoon
also posting their own videos on it. Young people on TikTok have also
recently
\href{https://www.nytimes3xbfgragh.onion/2020/06/21/style/tiktok-trump-rally-tulsa.html}{exercised
their political clout} by claiming to have registered potentially
hundreds of thousands of tickets for one of President Trump's recent
campaign rallies as a prank --- and then not showing up.

The app has been
\href{https://sensortower.com/blog/tiktok-downloads-2-billion}{downloaded
roughly two billion times} worldwide, according to data from analytics
firm Sensor Tower. About 170 million of those downloads were from users
in the United States. It has been installed more than 610 million times
in India.

TikTok has been viewed as
\href{https://www.nytimes3xbfgragh.onion/2019/11/03/technology/tiktok-facebook-youtube.html}{a
competitive threat by some American internet companies}, which are eager
to tap into younger audiences. Mark Zuckerberg, Facebook's chief
executive, has said that he worries about the popularity of TikTok, and
his company has been building competing products like Reels, an app
similar to TikTok that capitalizes on the social video format.

Mike Isaac reported from San Francisco, and Karen Weise from Seattle.
Nicole Perlroth contributed reporting from Palo Alto, Calif., and Nick
Corasiniti from New York.

Advertisement

\protect\hyperlink{after-bottom}{Continue reading the main story}

\hypertarget{site-index}{%
\subsection{Site Index}\label{site-index}}

\hypertarget{site-information-navigation}{%
\subsection{Site Information
Navigation}\label{site-information-navigation}}

\begin{itemize}
\tightlist
\item
  \href{https://help.nytimes3xbfgragh.onion/hc/en-us/articles/115014792127-Copyright-notice}{©~2020~The
  New York Times Company}
\end{itemize}

\begin{itemize}
\tightlist
\item
  \href{https://www.nytco.com/}{NYTCo}
\item
  \href{https://help.nytimes3xbfgragh.onion/hc/en-us/articles/115015385887-Contact-Us}{Contact
  Us}
\item
  \href{https://www.nytco.com/careers/}{Work with us}
\item
  \href{https://nytmediakit.com/}{Advertise}
\item
  \href{http://www.tbrandstudio.com/}{T Brand Studio}
\item
  \href{https://www.nytimes3xbfgragh.onion/privacy/cookie-policy\#how-do-i-manage-trackers}{Your
  Ad Choices}
\item
  \href{https://www.nytimes3xbfgragh.onion/privacy}{Privacy}
\item
  \href{https://help.nytimes3xbfgragh.onion/hc/en-us/articles/115014893428-Terms-of-service}{Terms
  of Service}
\item
  \href{https://help.nytimes3xbfgragh.onion/hc/en-us/articles/115014893968-Terms-of-sale}{Terms
  of Sale}
\item
  \href{https://spiderbites.nytimes3xbfgragh.onion}{Site Map}
\item
  \href{https://help.nytimes3xbfgragh.onion/hc/en-us}{Help}
\item
  \href{https://www.nytimes3xbfgragh.onion/subscription?campaignId=37WXW}{Subscriptions}
\end{itemize}
