\href{/section/us}{U.S.}\textbar{}In Texas Beach City, Out-of-Towners
Drove In an Outbreak

\url{https://nyti.ms/38Mw1ED}

\begin{itemize}
\item
\item
\item
\item
\item
\end{itemize}

\href{https://www.nytimes3xbfgragh.onion/news-event/coronavirus?action=click\&pgtype=Article\&state=default\&region=TOP_BANNER\&context=storylines_menu}{The
Coronavirus Outbreak}

\begin{itemize}
\tightlist
\item
  live\href{https://www.nytimes3xbfgragh.onion/2020/08/04/world/coronavirus-covid-19.html?action=click\&pgtype=Article\&state=default\&region=TOP_BANNER\&context=storylines_menu}{Latest
  Updates}
\item
  \href{https://www.nytimes3xbfgragh.onion/interactive/2020/us/coronavirus-us-cases.html?action=click\&pgtype=Article\&state=default\&region=TOP_BANNER\&context=storylines_menu}{Maps
  and Cases}
\item
  \href{https://www.nytimes3xbfgragh.onion/interactive/2020/science/coronavirus-vaccine-tracker.html?action=click\&pgtype=Article\&state=default\&region=TOP_BANNER\&context=storylines_menu}{Vaccine
  Tracker}
\item
  \href{https://www.nytimes3xbfgragh.onion/2020/08/02/us/covid-college-reopening.html?action=click\&pgtype=Article\&state=default\&region=TOP_BANNER\&context=storylines_menu}{College
  Reopening}
\item
  \href{https://www.nytimes3xbfgragh.onion/live/2020/08/03/business/stock-market-today-coronavirus?action=click\&pgtype=Article\&state=default\&region=TOP_BANNER\&context=storylines_menu}{Economy}
\end{itemize}

\includegraphics{https://static01.graylady3jvrrxbe.onion/images/2020/07/12/us/00virus-corpuschristi1/merlin_174339849_3ca8775b-fdab-4c59-b283-c7c2285dd217-articleLarge.jpg?quality=75\&auto=webp\&disable=upscale}

Sections

\protect\hyperlink{site-content}{Skip to
content}\protect\hyperlink{site-index}{Skip to site index}

\hypertarget{in-texas-beach-city-out-of-towners-drove-in-an-outbreak}{%
\section{In Texas Beach City, Out-of-Towners Drove In an
Outbreak}\label{in-texas-beach-city-out-of-towners-drove-in-an-outbreak}}

A month ago, Corpus Christi had hardly any cases of coronavirus and
business was booming. Now it is struggling to contain one of the state's
fastest growing outbreaks. What happened?

The Thornton family, who drove from Missouri, in Corpus Christi on July
7.Credit...Christopher Lee for The New York Times

Supported by

\protect\hyperlink{after-sponsor}{Continue reading the main story}

\href{https://www.nytimes3xbfgragh.onion/by/j-david-goodman}{\includegraphics{https://static01.graylady3jvrrxbe.onion/images/2018/07/18/nyregion/author-j-david-goodman/author-j-david-goodman-thumbLarge.png}}

By \href{https://www.nytimes3xbfgragh.onion/by/j-david-goodman}{J. David
Goodman}

\begin{itemize}
\item
  July 11, 2020
\item
  \begin{itemize}
  \item
  \item
  \item
  \item
  \item
  \end{itemize}
\end{itemize}

CORPUS CHRISTI, Texas --- As recently as early June, days went by with
hardly anyone testing positive for the
\href{https://www.nytimes3xbfgragh.onion/2020/07/26/us/coronavirus-family-houston-masks.html}{coronavirus}.
A single case one day. Three the next. Then zero. Zero. Zero.

Word spread that Corpus Christi, always a popular beachfront vacation
spot for Texans from around the state, was a safe place to go. They
didn't even require masks indoors. It was an oasis from the virus.

``People in San Antonio, in Houston, Austin, even Dallas, knew that we
had low caseload,'' said Peter Zanoni, the city manager. ``It was a nice
getaway from the rules, the regulations, the doom and gloom.''

It turned out that no place was safe.

Now the city of 325,000 has one of the
\href{https://www.nytimes3xbfgragh.onion/interactive/2020/us/texas-coronavirus-cases.html}{fastest-growing
outbreaks in Texas}, a state where new records for positive cases were
set for four straight days last week, with nearly 11,000 recorded on
Thursday. Corpus Christi has seen more cases per capita than Houston and
a rapidly mounting death toll: of the 38 deaths recorded from the
pandemic, 30 have come in July, including a baby less than 6 months old.

\includegraphics{https://static01.graylady3jvrrxbe.onion/images/2020/07/12/us/00virus-corpuschristi2/merlin_174339864_d9ca688c-71b0-4ba7-aa0d-aa360f101c33-articleLarge.jpg?quality=75\&auto=webp\&disable=upscale}

Local officials have been left scrambling to get ahead of an outbreak
that went into overdrive without warning. As recently as June 15, the
city had tallied 360 cases during the entirety of the outbreak; on
Wednesday alone, there were 445.

The city's two dozen contact tracers are so overwhelmed that they are no
longer able to seek detailed information about each new infection.
Hospital beds have filled at an alarming rate, prompting pleas for
additional staffing.

The surge in cases forced local leaders, businesses and residents to
grapple with the uncomfortable reality that the same out-of-towners who
help the city thrive economically may have caused the outbreak. The
feeling is less one of resentment than of frustration at a seemingly
impossible dilemma.

``I never thought in my wildest dreams that I'd be telling tourists,
`Don't come to our beaches,''' said Mayor Joe McComb, 72.

\hypertarget{latest-updates-global-coronavirus-outbreak}{%
\section{\texorpdfstring{\href{https://www.nytimes3xbfgragh.onion/2020/08/04/world/coronavirus-covid-19.html?action=click\&pgtype=Article\&state=default\&region=MAIN_CONTENT_1\&context=storylines_live_updates}{Latest
Updates: Global Coronavirus
Outbreak}}{Latest Updates: Global Coronavirus Outbreak}}\label{latest-updates-global-coronavirus-outbreak}}

Updated 2020-08-04T10:03:05.885Z

\begin{itemize}
\tightlist
\item
  \href{https://www.nytimes3xbfgragh.onion/2020/08/04/world/coronavirus-covid-19.html?action=click\&pgtype=Article\&state=default\&region=MAIN_CONTENT_1\&context=storylines_live_updates\#link-6b644638}{`Long
  days, long nights': Washington prepares for a prolonged fight over
  virus relief.}
\item
  \href{https://www.nytimes3xbfgragh.onion/2020/08/04/world/coronavirus-covid-19.html?action=click\&pgtype=Article\&state=default\&region=MAIN_CONTENT_1\&context=storylines_live_updates\#link-7af9fca0}{Israel's
  rocky reopening of its schools may be a lesson for the U.S.}
\item
  \href{https://www.nytimes3xbfgragh.onion/2020/08/04/world/coronavirus-covid-19.html?action=click\&pgtype=Article\&state=default\&region=MAIN_CONTENT_1\&context=storylines_live_updates\#link-33bf9168}{Hurricane
  Isaias arrives in North Carolina as officials along the East Coast
  scramble.}
\end{itemize}

\href{https://www.nytimes3xbfgragh.onion/2020/08/04/world/coronavirus-covid-19.html?action=click\&pgtype=Article\&state=default\&region=MAIN_CONTENT_1\&context=storylines_live_updates}{See
more updates}

More live coverage:
\href{https://www.nytimes3xbfgragh.onion/live/2020/08/03/business/stock-market-today-coronavirus?action=click\&pgtype=Article\&state=default\&region=MAIN_CONTENT_1\&context=storylines_live_updates}{Markets}

The speed of the spread is what struck researchers. Other vacation
destinations have seen a rising number of cases, but the increase in
Corpus Christi outstripped even much larger major urban centers, said
Dr. Christopher Bird, a professor at Texas A\&M-Corpus Christi.

``The part that's different here is just how fast we rose in the number
of cases and how fast it spread,'' said Dr. Bird, who has been modeling
the outbreak for officials.

The reason for the rapid spread in Corpus Christi is not certain. Data
gathered from cellphones indicated that movement around the city
returned to pre-pandemic levels by early June, especially at
restaurants. ``When I saw that, I knew it wasn't a good sign,'' Dr. Bird
said.

Many pointed at the visitors from big cities.

``I think they should stay home,'' said Jasmine Rodriguez, 24, a
security guard at a La Michoacana grocery store. ``There should be
checkpoints and a mandatory stay-in-your-city.''

But some locals said they had no one but themselves to blame. People
went to bars. They partied. They did not social distance or wear masks.
The city did not require masks in most retail stores until last week,
days before a statewide order.

Image

``I never thought in my wildest dreams that I'd be telling tourists,
`Don't come to our beaches,''' Mayor Joe McComb
said.Credit...Christopher Lee for The New York Times

``It's us. Yeah, it's us,'' said Marilyn McCaleb, 62, speaking through a
flower-print mask as she went grocery shopping at a local H-E-B store.
``They don't wear their masks --- maybe they do now, because they have
to.''

Whatever the reason, the virus was almost nowhere, and then, seemingly
overnight, it was everywhere. Bars. Restaurants. Graduation
celebrations. A draft party for a local baseball player picked up by a
major league team. A chance encounter on the beach.

``I know because they would say to the contact tracers, oh, I was at the
beach and some girls from San Antonio told us at the end of the night
that they had Covid,'' said Annette Rodriguez, the public health
director for Corpus Christi and the surrounding county. ``And we shared
a bottle.''

The county attorney tested positive, as did many city workers. At one
point, 10 percent of the firefighters in the city were out sick or
quarantining because of possible exposure. At City Hall, staffers who
were back in the office after months of working from home in the spring
were told to return to remote work. Officials instituted a beach curfew
and barred cars from the sands over the July 4 holiday.

The contrast with even a few weeks ago could not be more stark.

At first, city officials had been able to jump on and contain what few
small outbreaks there were: at a meat processing plant, or a halfway
house. Officials tested aggressively and got those who were exposed to
isolate. They felt confident in their approach.

Corpus Christi is a politically split and culturally mixed town, with a
Democratic county leader, a conservative mayor and a population that is
majority Hispanic.

``It's not even purple. It's more like lavender,'' said Barbara Canales,
the top executive for Nueces County, which includes Corpus Christi.
``We're much more interested in our own backyard than in the national
scene.''

Last month, the city stood out as an example of a place that had
suffered economically from pandemic-related shutdowns --- with
unemployment at nearly 16 percent in early June --- without actually
experiencing much of a viral outbreak at all. Few residents knew anyone
who had gotten sick.

Not only was tourism devastated, but another major industry in the city
--- its massive port for oil and gas exports --- suffered from declining
demand and plunging oil prices.

Then, as Texas reopened beginning on May 1, Texans began flocking to
Corpus. It started on Memorial Day weekend and did not stop for weeks.

Image

Customers at JB's German Bakery \& Cafe last week.Credit...Christopher
Lee for The New York Times

``The entire city was completely sold out. Every hotel. Every short-term
rental,'' said Brett Oetting, the head of the Corpus Christi tourism
bureau. ``What happened during the entire month of June: every weekend
was a Memorial Day weekend.''

Hotels, restaurants and bars that had been starved for life surged back.
But some business owners grew wary of the number of people suddenly
flooding into town.

\href{https://www.nytimes3xbfgragh.onion/news-event/coronavirus?action=click\&pgtype=Article\&state=default\&region=MAIN_CONTENT_3\&context=storylines_faq}{}

\hypertarget{the-coronavirus-outbreak-}{%
\subsubsection{The Coronavirus Outbreak
›}\label{the-coronavirus-outbreak-}}

\hypertarget{frequently-asked-questions}{%
\paragraph{Frequently Asked
Questions}\label{frequently-asked-questions}}

Updated August 3, 2020

\begin{itemize}
\item ~
  \hypertarget{im-a-small-business-owner-can-i-get-relief}{%
  \paragraph{I'm a small-business owner. Can I get
  relief?}\label{im-a-small-business-owner-can-i-get-relief}}

  \begin{itemize}
  \tightlist
  \item
    The
    \href{https://www.nytimes3xbfgragh.onion/article/small-business-loans-stimulus-grants-freelancers-coronavirus.html?action=click\&pgtype=Article\&state=default\&region=MAIN_CONTENT_3\&context=storylines_faq}{stimulus
    bills enacted in March} offer help for the millions of American
    small businesses. Those eligible for aid are businesses and
    nonprofit organizations with fewer than 500 workers, including sole
    proprietorships, independent contractors and freelancers. Some
    larger companies in some industries are also eligible. The help
    being offered, which is being managed by the Small Business
    Administration, includes the Paycheck Protection Program and the
    Economic Injury Disaster Loan program. But lots of folks have
    \href{https://www.nytimes3xbfgragh.onion/interactive/2020/05/07/business/small-business-loans-coronavirus.html?action=click\&pgtype=Article\&state=default\&region=MAIN_CONTENT_3\&context=storylines_faq}{not
    yet seen payouts.} Even those who have received help are confused:
    The rules are draconian, and some are stuck sitting on
    \href{https://www.nytimes3xbfgragh.onion/2020/05/02/business/economy/loans-coronavirus-small-business.html?action=click\&pgtype=Article\&state=default\&region=MAIN_CONTENT_3\&context=storylines_faq}{money
    they don't know how to use.} Many small-business owners are getting
    less than they expected or
    \href{https://www.nytimes3xbfgragh.onion/2020/06/10/business/Small-business-loans-ppp.html?action=click\&pgtype=Article\&state=default\&region=MAIN_CONTENT_3\&context=storylines_faq}{not
    hearing anything at all.}
  \end{itemize}
\item ~
  \hypertarget{what-are-my-rights-if-i-am-worried-about-going-back-to-work}{%
  \paragraph{What are my rights if I am worried about going back to
  work?}\label{what-are-my-rights-if-i-am-worried-about-going-back-to-work}}

  \begin{itemize}
  \tightlist
  \item
    Employers have to provide
    \href{https://www.osha.gov/SLTC/covid-19/standards.html}{a safe
    workplace} with policies that protect everyone equally.
    \href{https://www.nytimes3xbfgragh.onion/article/coronavirus-money-unemployment.html?action=click\&pgtype=Article\&state=default\&region=MAIN_CONTENT_3\&context=storylines_faq}{And
    if one of your co-workers tests positive for the coronavirus, the
    C.D.C.} has said that
    \href{https://www.cdc.gov/coronavirus/2019-ncov/community/guidance-business-response.html}{employers
    should tell their employees} -\/- without giving you the sick
    employee's name -\/- that they may have been exposed to the virus.
  \end{itemize}
\item ~
  \hypertarget{should-i-refinance-my-mortgage}{%
  \paragraph{Should I refinance my
  mortgage?}\label{should-i-refinance-my-mortgage}}

  \begin{itemize}
  \tightlist
  \item
    \href{https://www.nytimes3xbfgragh.onion/article/coronavirus-money-unemployment.html?action=click\&pgtype=Article\&state=default\&region=MAIN_CONTENT_3\&context=storylines_faq}{It
    could be a good idea,} because mortgage rates have
    \href{https://www.nytimes3xbfgragh.onion/2020/07/16/business/mortgage-rates-below-3-percent.html?action=click\&pgtype=Article\&state=default\&region=MAIN_CONTENT_3\&context=storylines_faq}{never
    been lower.} Refinancing requests have pushed mortgage applications
    to some of the highest levels since 2008, so be prepared to get in
    line. But defaults are also up, so if you're thinking about buying a
    home, be aware that some lenders have tightened their standards.
  \end{itemize}
\item ~
  \hypertarget{what-is-school-going-to-look-like-in-september}{%
  \paragraph{What is school going to look like in
  September?}\label{what-is-school-going-to-look-like-in-september}}

  \begin{itemize}
  \tightlist
  \item
    It is unlikely that many schools will return to a normal schedule
    this fall, requiring the grind of
    \href{https://www.nytimes3xbfgragh.onion/2020/06/05/us/coronavirus-education-lost-learning.html?action=click\&pgtype=Article\&state=default\&region=MAIN_CONTENT_3\&context=storylines_faq}{online
    learning},
    \href{https://www.nytimes3xbfgragh.onion/2020/05/29/us/coronavirus-child-care-centers.html?action=click\&pgtype=Article\&state=default\&region=MAIN_CONTENT_3\&context=storylines_faq}{makeshift
    child care} and
    \href{https://www.nytimes3xbfgragh.onion/2020/06/03/business/economy/coronavirus-working-women.html?action=click\&pgtype=Article\&state=default\&region=MAIN_CONTENT_3\&context=storylines_faq}{stunted
    workdays} to continue. California's two largest public school
    districts --- Los Angeles and San Diego --- said on July 13, that
    \href{https://www.nytimes3xbfgragh.onion/2020/07/13/us/lausd-san-diego-school-reopening.html?action=click\&pgtype=Article\&state=default\&region=MAIN_CONTENT_3\&context=storylines_faq}{instruction
    will be remote-only in the fall}, citing concerns that surging
    coronavirus infections in their areas pose too dire a risk for
    students and teachers. Together, the two districts enroll some
    825,000 students. They are the largest in the country so far to
    abandon plans for even a partial physical return to classrooms when
    they reopen in August. For other districts, the solution won't be an
    all-or-nothing approach.
    \href{https://bioethics.jhu.edu/research-and-outreach/projects/eschool-initiative/school-policy-tracker/}{Many
    systems}, including the nation's largest, New York City, are
    devising
    \href{https://www.nytimes3xbfgragh.onion/2020/06/26/us/coronavirus-schools-reopen-fall.html?action=click\&pgtype=Article\&state=default\&region=MAIN_CONTENT_3\&context=storylines_faq}{hybrid
    plans} that involve spending some days in classrooms and other days
    online. There's no national policy on this yet, so check with your
    municipal school system regularly to see what is happening in your
    community.
  \end{itemize}
\item ~
  \hypertarget{is-the-coronavirus-airborne}{%
  \paragraph{Is the coronavirus
  airborne?}\label{is-the-coronavirus-airborne}}

  \begin{itemize}
  \tightlist
  \item
    The coronavirus
    \href{https://www.nytimes3xbfgragh.onion/2020/07/04/health/239-experts-with-one-big-claim-the-coronavirus-is-airborne.html?action=click\&pgtype=Article\&state=default\&region=MAIN_CONTENT_3\&context=storylines_faq}{can
    stay aloft for hours in tiny droplets in stagnant air}, infecting
    people as they inhale, mounting scientific evidence suggests. This
    risk is highest in crowded indoor spaces with poor ventilation, and
    may help explain super-spreading events reported in meatpacking
    plants, churches and restaurants.
    \href{https://www.nytimes3xbfgragh.onion/2020/07/06/health/coronavirus-airborne-aerosols.html?action=click\&pgtype=Article\&state=default\&region=MAIN_CONTENT_3\&context=storylines_faq}{It's
    unclear how often the virus is spread} via these tiny droplets, or
    aerosols, compared with larger droplets that are expelled when a
    sick person coughs or sneezes, or transmitted through contact with
    contaminated surfaces, said Linsey Marr, an aerosol expert at
    Virginia Tech. Aerosols are released even when a person without
    symptoms exhales, talks or sings, according to Dr. Marr and more
    than 200 other experts, who
    \href{https://academic.oup.com/cid/article/doi/10.1093/cid/ciaa939/5867798}{have
    outlined the evidence in an open letter to the World Health
    Organization}.
  \end{itemize}
\end{itemize}

``It was horrible --- it was so busy,'' said Brigitte Kazenmayer, 59,
the owner of the popular breakfast spot JB's German Bakery \& Cafe.
``People didn't wear masks. They didn't understand the six feet.''

Ms. Kazenmayer, who immigrated from Germany and fell in love with Corpus
Christi, said that in June the lines would snake out the door and across
the parking lot. ``They came from Houston, Austin, San Antonio --- and I
think, why are you here? You bring it here!'' she said of the virus.
``But they like the beach. That's why I'm here, too.''

The Bait Bucket, a cinder-block box of a store painted bright yellow,
saw so many customers in June that they had to add a second sales person
to deal with the crowds, said Miriam Longoria, 21, who worked behind the
counter.

The store attracts both locals and tourists, and people kept coming, she
said, even after the governor ordered bars to close in late June and
other places in town began slowing down.

``This is the one thing you can do, is fish,'' said Jeff Soward, 56,
holding up a white plastic bag of dead shrimp he had just bought inside.
He said he knew several people who had been infected: a 72-year-old
business partner in Mexico; the children of several friends; his
daughter's boyfriend in Dallas.

``The kids I know that got it, they're fine,'' he observed.

But dealing with the outbreak has strained medical resources in a city
where officials said nearly one in five residents does not have health
insurance. Hospitals have stopped performing elective surgeries and are
paying overtime to keep up.

``The coast is not clear,'' Ms. Canales said during a daily news
briefing this week. ``It is not clear to come to at this time.''

Image

Brittany Thangiah works her shift as a bartender at Water Street Oyster
Bar in Corpus Christi.Credit...Christopher Lee for The New York Times

Still, people come, albeit in smaller numbers. The beaches are open. And
the surf is inviting.

But now it is the visitors from other parts of Texas who are wary of
being around people in Corpus Christi.

At an R.V. campsite just a few steps from the beach, Billy Arocha, 34,
prepared to grill as his three children played in a sprinkler. Mr.
Arocha, from outside San Antonio, said he had considered canceling the
trip when he saw the cases exploding in Corpus Christi. ``I'm scared,''
he said, adding an expletive for emphasis.

When they were planning their stay, Mr. Arocha said he considered going
out to eat in a restaurant, ``but not anymore.'' He said he was not
talking to anyone in town and only going to the beach when it was
uncrowded.

The vacation had been meant as a much-needed break, and a way to
celebrate his wife's birthday. But the mood had darkened over their
festivities.

``We just got a call that her aunt is close to dying from the
coronavirus,'' he said. ``That virus is something else.''

Advertisement

\protect\hyperlink{after-bottom}{Continue reading the main story}

\hypertarget{site-index}{%
\subsection{Site Index}\label{site-index}}

\hypertarget{site-information-navigation}{%
\subsection{Site Information
Navigation}\label{site-information-navigation}}

\begin{itemize}
\tightlist
\item
  \href{https://help.nytimes3xbfgragh.onion/hc/en-us/articles/115014792127-Copyright-notice}{©~2020~The
  New York Times Company}
\end{itemize}

\begin{itemize}
\tightlist
\item
  \href{https://www.nytco.com/}{NYTCo}
\item
  \href{https://help.nytimes3xbfgragh.onion/hc/en-us/articles/115015385887-Contact-Us}{Contact
  Us}
\item
  \href{https://www.nytco.com/careers/}{Work with us}
\item
  \href{https://nytmediakit.com/}{Advertise}
\item
  \href{http://www.tbrandstudio.com/}{T Brand Studio}
\item
  \href{https://www.nytimes3xbfgragh.onion/privacy/cookie-policy\#how-do-i-manage-trackers}{Your
  Ad Choices}
\item
  \href{https://www.nytimes3xbfgragh.onion/privacy}{Privacy}
\item
  \href{https://help.nytimes3xbfgragh.onion/hc/en-us/articles/115014893428-Terms-of-service}{Terms
  of Service}
\item
  \href{https://help.nytimes3xbfgragh.onion/hc/en-us/articles/115014893968-Terms-of-sale}{Terms
  of Sale}
\item
  \href{https://spiderbites.nytimes3xbfgragh.onion}{Site Map}
\item
  \href{https://help.nytimes3xbfgragh.onion/hc/en-us}{Help}
\item
  \href{https://www.nytimes3xbfgragh.onion/subscription?campaignId=37WXW}{Subscriptions}
\end{itemize}
