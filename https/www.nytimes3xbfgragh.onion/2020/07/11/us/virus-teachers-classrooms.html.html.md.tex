Sections

SEARCH

\protect\hyperlink{site-content}{Skip to
content}\protect\hyperlink{site-index}{Skip to site index}

\href{https://www.nytimes3xbfgragh.onion/section/us}{U.S.}

\href{https://myaccount.nytimes3xbfgragh.onion/auth/login?response_type=cookie\&client_id=vi}{}

\href{https://www.nytimes3xbfgragh.onion/section/todayspaper}{Today's
Paper}

\href{/section/us}{U.S.}\textbar{}`I Don't Want to Go Back': Many
Teachers Are Fearful and Angry Over Pressure to Return

\url{https://nyti.ms/2Ofp9q2}

\begin{itemize}
\item
\item
\item
\item
\item
\end{itemize}

\href{https://www.nytimes3xbfgragh.onion/news-event/coronavirus?action=click\&pgtype=Article\&state=default\&region=TOP_BANNER\&context=storylines_menu}{The
Coronavirus Outbreak}

\begin{itemize}
\tightlist
\item
  live\href{https://www.nytimes3xbfgragh.onion/2020/08/04/world/coronavirus-cases.html?action=click\&pgtype=Article\&state=default\&region=TOP_BANNER\&context=storylines_menu}{Latest
  Updates}
\item
  \href{https://www.nytimes3xbfgragh.onion/interactive/2020/us/coronavirus-us-cases.html?action=click\&pgtype=Article\&state=default\&region=TOP_BANNER\&context=storylines_menu}{Maps
  and Cases}
\item
  \href{https://www.nytimes3xbfgragh.onion/interactive/2020/science/coronavirus-vaccine-tracker.html?action=click\&pgtype=Article\&state=default\&region=TOP_BANNER\&context=storylines_menu}{Vaccine
  Tracker}
\item
  \href{https://www.nytimes3xbfgragh.onion/2020/08/02/us/covid-college-reopening.html?action=click\&pgtype=Article\&state=default\&region=TOP_BANNER\&context=storylines_menu}{College
  Reopening}
\item
  \href{https://www.nytimes3xbfgragh.onion/live/2020/08/04/business/stock-market-today-coronavirus?action=click\&pgtype=Article\&state=default\&region=TOP_BANNER\&context=storylines_menu}{Economy}
\end{itemize}

Advertisement

\protect\hyperlink{after-top}{Continue reading the main story}

Supported by

\protect\hyperlink{after-sponsor}{Continue reading the main story}

\hypertarget{i-dont-want-to-go-back-many-teachers-are-fearful-and-angry-over-pressure-to-return}{%
\section{`I Don't Want to Go Back': Many Teachers Are Fearful and Angry
Over Pressure to
Return}\label{i-dont-want-to-go-back-many-teachers-are-fearful-and-angry-over-pressure-to-return}}

Teachers say crucial questions about how schools will stay clean, keep
students physically distanced and prevent further spread of the virus
have not been answered.

\includegraphics{https://static01.graylady3jvrrxbe.onion/images/2020/07/12/us/10VIRUS-TEACHERS/merlin_174437808_89c71b1e-7ff5-4373-b581-511592d37144-articleLarge.jpg?quality=75\&auto=webp\&disable=upscale}

\href{https://www.nytimes3xbfgragh.onion/by/dana-goldstein}{\includegraphics{https://static01.graylady3jvrrxbe.onion/images/2018/06/12/multimedia/author-dana-goldstein/author-dana-goldstein-thumbLarge.png}}\href{https://www.nytimes3xbfgragh.onion/by/eliza-shapiro}{\includegraphics{https://static01.graylady3jvrrxbe.onion/images/2018/12/28/multimedia/author-eliza-shapiro/author-eliza-shapiro-thumbLarge.png}}

By \href{https://www.nytimes3xbfgragh.onion/by/dana-goldstein}{Dana
Goldstein} and
\href{https://www.nytimes3xbfgragh.onion/by/eliza-shapiro}{Eliza
Shapiro}

\begin{itemize}
\item
  Published July 11, 2020Updated July 31, 2020
\item
  \begin{itemize}
  \item
  \item
  \item
  \item
  \item
  \end{itemize}
\end{itemize}

Many of the nation's 3.5 million teachers found themselves feeling under
siege this week as pressure from the White House, pediatricians and some
parents to get back to physical classrooms intensified --- even as the
coronavirus rages across much of the country.

On Friday, the teachers' union in Los Angeles, the nation's
second-largest district,
\href{https://www.utla.net/news/utla-recommends-keeping-school-campuses-closed}{demanded
full-time remote learning} when the academic year begins on Aug. 18, and
called President Trump's push to reopen schools part of a ``dangerous,
anti-science agenda that puts the lives of our members, our students and
our families at risk.''

Teachers say crucial questions about how schools will stay clean, keep
students physically distanced and prevent further spread of the virus
have not been answered. And they feel that their own lives, and those of
the family members they come home to, are at stake.

``I want to serve the students, but it's hard to say you're going to
sacrifice all of the teachers, paraprofessionals, cafeteria workers and
bus drivers,'' said Hannah Wysong, a teacher at the Esperanza Community
School in Tempe, Ariz., where virus cases
\href{https://www.nytimes3xbfgragh.onion/interactive/2020/us/arizona-coronavirus-cases.html}{are
increasing}.

School systems struggling to meet the financial and logistical
challenges of reopening safely will need to carefully weigh teachers'
concerns. A wave of leave requests, early retirements or resignations
driven by health fears could imperil efforts to reach students learning
both in physical classrooms and online.

On social media, teachers across the country promoted the hashtag
\href{https://twitter.com/search?q=\%2314daysnonewcases\&src=typed_query}{\#14daysnonewcases},
with some
\href{https://www.change.org/p/u-s-department-of-education-refuse-to-return-to-campus-until-counties-report-no-new-cases-for-14-days?utm_content=cl_sharecopy_23175660_en-US\%3A0\&recruiter=441547298\&recruited_by_id=44bf9f90-9ac2-11e5-a10a-97d04f3cc37d\&utm_source=share_petition\&utm_medium=copylink\&utm_campaign=psf_combo_share_abi\&utm_term=psf_combo_share_abi}{pledging}
to refuse to enter classrooms until the coronavirus transmission rate in
their counties falls, essentially, to zero.

Now, educators are using some of the same organizing tactics they
employed in walkouts over issues of pay and funding in recent years to
demand that schools remain closed, at least in the short term. It's a
stance that could potentially be divisive, with some district surveys
suggesting that more than half of parents would like their children to
return to classrooms.

Big districts like San Diego and smaller ones, like Marietta, Ga., are
\href{https://www.nytimes3xbfgragh.onion/2020/07/09/us/schools-reopening-trump.html}{forging
ahead} with plans to open schools five days per week. Many other
systems, like those in New York City and Seattle, hope to offer several
days per week of in-person school.

Adding to the confusion, optional guidelines released by the Centers for
Disease Control and Prevention in May set out ambitious safety
precautions for schools. But the president, and many local school system
leaders, have suggested they do not need to be strictly followed,
alarming teachers.

\hypertarget{latest-updates-global-coronavirus-outbreak}{%
\section{\texorpdfstring{\href{https://www.nytimes3xbfgragh.onion/2020/08/04/world/coronavirus-cases.html?action=click\&pgtype=Article\&state=default\&region=MAIN_CONTENT_1\&context=storylines_live_updates}{Latest
Updates: Global Coronavirus
Outbreak}}{Latest Updates: Global Coronavirus Outbreak}}\label{latest-updates-global-coronavirus-outbreak}}

Updated 2020-08-04T21:34:02.738Z

\begin{itemize}
\tightlist
\item
  \href{https://www.nytimes3xbfgragh.onion/2020/08/04/world/coronavirus-cases.html?action=click\&pgtype=Article\&state=default\&region=MAIN_CONTENT_1\&context=storylines_live_updates\#link-2daa96b5}{As
  talks drag on, McConnell signals openness to jobless aid extension
  that Republicans have opposed.}
\item
  \href{https://www.nytimes3xbfgragh.onion/2020/08/04/world/coronavirus-cases.html?action=click\&pgtype=Article\&state=default\&region=MAIN_CONTENT_1\&context=storylines_live_updates\#link-1228a480}{Novavax
  sees encouraging results from two studies of its experimental
  vaccine.}
\item
  \href{https://www.nytimes3xbfgragh.onion/2020/08/04/world/coronavirus-cases.html?action=click\&pgtype=Article\&state=default\&region=MAIN_CONTENT_1\&context=storylines_live_updates\#link-4825b93}{Public
  and private schools in Maryland and elsewhere are divided over
  in-person instruction.}
\end{itemize}

\href{https://www.nytimes3xbfgragh.onion/2020/08/04/world/coronavirus-cases.html?action=click\&pgtype=Article\&state=default\&region=MAIN_CONTENT_1\&context=storylines_live_updates}{See
more updates}

More live coverage:
\href{https://www.nytimes3xbfgragh.onion/live/2020/08/04/business/stock-market-today-coronavirus?action=click\&pgtype=Article\&state=default\&region=MAIN_CONTENT_1\&context=storylines_live_updates}{Markets}

Many doctors, education experts, parents and policymakers have argued
that the social and academic costs of school closures on children need
to be weighed alongside the risks of the virus itself.

The heated national debate about how and whether to bring students back
to classrooms plays upon all the anxieties of the teaching profession.
The comparison between teachers and other essential workers currently
laboring outside their homes rankles some educators. They note that they
are paid much less than doctors --- the average salary nationwide for
teachers is about \$60,000 per year --- but are more highly educated
than delivery people, restaurant workers or most staffers in child care
centers, many of whom are already back at work.

\includegraphics{https://static01.graylady3jvrrxbe.onion/images/2020/07/12/us/10VIRUS-TEACHERS-marietta/merlin_174339066_1efd9a1f-b65d-4c04-9df3-bba548f0d425-articleLarge.jpg?quality=75\&auto=webp\&disable=upscale}

Now, as teachers listen to a national conversation about reopening
schools that many believe elevates the needs of the economy and working
parents above the concerns of the classroom work force, many are fearful
and angry. They point out that so far Congress has dedicated less than 1
percent of federal pandemic stimulus funds to public schools stretching
to meet the costs of reopening safely.

The message to teachers, said Christina Setzer, a preschool educator in
Sacramento, is, ``Yes, you guys are really important and essential and
kids and parents need you. But sorry, we don't have the money.''

Earlier in the shutdown, Mr. Trump acknowledged the health risks to
teachers over the age of 60 and those with underlying conditions,
\href{https://www.whitehouse.gov/briefings-statements/remarks-president-trump-signing-proclamation-honor-national-nurses-day/}{saying
at a White House event in May} that ``they should not be teaching school
for a while, and everybody would understand that fully.''

But this week, as the administration launched a full-throated campaign
to pressure schools to reopen in the fall --- a crucial step for
jump-starting the economy --- it all but ignored the potential risks
teachers face. \href{https://nces.ed.gov/pubs2020/2020142.pdf}{More than
one-quarter} of public schoolteachers are over the age of 50.

Teachers say many of their questions about how schools will operate
safely remain unanswered. They point out that some classrooms have
windows that do not reliably open to promote air circulation, while
school buildings can have aging heating and cooling systems that lack
the filtration features that reduce virus transmission.

Although many districts are
\href{https://www.nytimes3xbfgragh.onion/2020/07/09/us/schools-reopening-trump.html?smid=tw-share}{spending
millions this summer} procuring masks, sanitizers and additional
custodial staff, many teachers say they have little faith that limited
resources will stretch to fill the need.

They also worry about access to tests and contact tracing to confirm
Covid-19 diagnoses and clarify who in a school might need to isolate at
home in the event of a symptomatic student or staff member.

The C.D.C. has
\href{https://www.cdc.gov/coronavirus/2019-ncov/community/schools-childcare/k-12-testing.html}{advised
against} regular testing in K-12 schools, but on Wednesday, Dr. Deborah
L. Birx, the White House coronavirus response coordinator, said the
Trump administration was exploring whether testing being developed for
other vulnerable environments, like nursing homes, could be used in
schools.

Indeed, educators have had to process a head-spinning set of conflicting
health and safety guidelines from Washington, states and medical
experts.

The C.D.C.
\href{https://www.cdc.gov/coronavirus/2019-ncov/community/schools-childcare/schools.html}{has
recommended} that when schools reopen, students remain six feet apart
``when feasible,'' while the American Academy of Pediatrics
\href{https://www.nytimes3xbfgragh.onion/2020/06/30/us/coronavirus-schools-reopening-guidelines-aap.html}{released
guidelines} suggesting that three feet could be enough space if students
wore masks.

But after major pushback from educator groups, who felt there was too
little attention on the health risks for adults who work in schools, the
Academy joined with the two national teachers' unions on Friday to
\href{https://services.aap.org/en/news-room/news-releases/aap/2020/pediatricians-educators-and-superintendents-urge-a-safe-return-to-school-this-fall/}{release
a statement} saying, ``Schools in areas with high levels of Covid-19
community spread should not be compelled to reopen against the judgment
of local experts.''

\href{https://www.nytimes3xbfgragh.onion/news-event/coronavirus?action=click\&pgtype=Article\&state=default\&region=MAIN_CONTENT_3\&context=storylines_faq}{}

\hypertarget{the-coronavirus-outbreak-}{%
\subsubsection{The Coronavirus Outbreak
›}\label{the-coronavirus-outbreak-}}

\hypertarget{frequently-asked-questions}{%
\paragraph{Frequently Asked
Questions}\label{frequently-asked-questions}}

Updated August 4, 2020

\begin{itemize}
\item ~
  \hypertarget{i-have-antibodies-am-i-now-immune}{%
  \paragraph{I have antibodies. Am I now
  immune?}\label{i-have-antibodies-am-i-now-immune}}

  \begin{itemize}
  \tightlist
  \item
    As of right
    now,\href{https://www.nytimes3xbfgragh.onion/2020/07/22/health/covid-antibodies-herd-immunity.html?action=click\&pgtype=Article\&state=default\&region=MAIN_CONTENT_3\&context=storylines_faq}{that
    seems likely, for at least several months.} There have been
    frightening accounts of people suffering what seems to be a second
    bout of Covid-19. But experts say these patients may have a
    drawn-out course of infection, with the virus taking a slow toll
    weeks to months after initial exposure. People infected with the
    coronavirus typically
    \href{https://www.nature.com/articles/s41586-020-2456-9}{produce}
    immune molecules called antibodies, which are
    \href{https://www.nytimes3xbfgragh.onion/2020/05/07/health/coronavirus-antibody-prevalence.html?action=click\&pgtype=Article\&state=default\&region=MAIN_CONTENT_3\&context=storylines_faq}{protective
    proteins made in response to an
    infection}\href{https://www.nytimes3xbfgragh.onion/2020/05/07/health/coronavirus-antibody-prevalence.html?action=click\&pgtype=Article\&state=default\&region=MAIN_CONTENT_3\&context=storylines_faq}{.
    These antibodies may} last in the body
    \href{https://www.nature.com/articles/s41591-020-0965-6}{only two to
    three months}, which may seem worrisome, but that's perfectly normal
    after an acute infection subsides, said Dr. Michael Mina, an
    immunologist at Harvard University. It may be possible to get the
    coronavirus again, but it's highly unlikely that it would be
    possible in a short window of time from initial infection or make
    people sicker the second time.
  \end{itemize}
\item ~
  \hypertarget{im-a-small-business-owner-can-i-get-relief}{%
  \paragraph{I'm a small-business owner. Can I get
  relief?}\label{im-a-small-business-owner-can-i-get-relief}}

  \begin{itemize}
  \tightlist
  \item
    The
    \href{https://www.nytimes3xbfgragh.onion/article/small-business-loans-stimulus-grants-freelancers-coronavirus.html?action=click\&pgtype=Article\&state=default\&region=MAIN_CONTENT_3\&context=storylines_faq}{stimulus
    bills enacted in March} offer help for the millions of American
    small businesses. Those eligible for aid are businesses and
    nonprofit organizations with fewer than 500 workers, including sole
    proprietorships, independent contractors and freelancers. Some
    larger companies in some industries are also eligible. The help
    being offered, which is being managed by the Small Business
    Administration, includes the Paycheck Protection Program and the
    Economic Injury Disaster Loan program. But lots of folks have
    \href{https://www.nytimes3xbfgragh.onion/interactive/2020/05/07/business/small-business-loans-coronavirus.html?action=click\&pgtype=Article\&state=default\&region=MAIN_CONTENT_3\&context=storylines_faq}{not
    yet seen payouts.} Even those who have received help are confused:
    The rules are draconian, and some are stuck sitting on
    \href{https://www.nytimes3xbfgragh.onion/2020/05/02/business/economy/loans-coronavirus-small-business.html?action=click\&pgtype=Article\&state=default\&region=MAIN_CONTENT_3\&context=storylines_faq}{money
    they don't know how to use.} Many small-business owners are getting
    less than they expected or
    \href{https://www.nytimes3xbfgragh.onion/2020/06/10/business/Small-business-loans-ppp.html?action=click\&pgtype=Article\&state=default\&region=MAIN_CONTENT_3\&context=storylines_faq}{not
    hearing anything at all.}
  \end{itemize}
\item ~
  \hypertarget{what-are-my-rights-if-i-am-worried-about-going-back-to-work}{%
  \paragraph{What are my rights if I am worried about going back to
  work?}\label{what-are-my-rights-if-i-am-worried-about-going-back-to-work}}

  \begin{itemize}
  \tightlist
  \item
    Employers have to provide
    \href{https://www.osha.gov/SLTC/covid-19/standards.html}{a safe
    workplace} with policies that protect everyone equally.
    \href{https://www.nytimes3xbfgragh.onion/article/coronavirus-money-unemployment.html?action=click\&pgtype=Article\&state=default\&region=MAIN_CONTENT_3\&context=storylines_faq}{And
    if one of your co-workers tests positive for the coronavirus, the
    C.D.C.} has said that
    \href{https://www.cdc.gov/coronavirus/2019-ncov/community/guidance-business-response.html}{employers
    should tell their employees} -\/- without giving you the sick
    employee's name -\/- that they may have been exposed to the virus.
  \end{itemize}
\item ~
  \hypertarget{should-i-refinance-my-mortgage}{%
  \paragraph{Should I refinance my
  mortgage?}\label{should-i-refinance-my-mortgage}}

  \begin{itemize}
  \tightlist
  \item
    \href{https://www.nytimes3xbfgragh.onion/article/coronavirus-money-unemployment.html?action=click\&pgtype=Article\&state=default\&region=MAIN_CONTENT_3\&context=storylines_faq}{It
    could be a good idea,} because mortgage rates have
    \href{https://www.nytimes3xbfgragh.onion/2020/07/16/business/mortgage-rates-below-3-percent.html?action=click\&pgtype=Article\&state=default\&region=MAIN_CONTENT_3\&context=storylines_faq}{never
    been lower.} Refinancing requests have pushed mortgage applications
    to some of the highest levels since 2008, so be prepared to get in
    line. But defaults are also up, so if you're thinking about buying a
    home, be aware that some lenders have tightened their standards.
  \end{itemize}
\item ~
  \hypertarget{what-is-school-going-to-look-like-in-september}{%
  \paragraph{What is school going to look like in
  September?}\label{what-is-school-going-to-look-like-in-september}}

  \begin{itemize}
  \tightlist
  \item
    It is unlikely that many schools will return to a normal schedule
    this fall, requiring the grind of
    \href{https://www.nytimes3xbfgragh.onion/2020/06/05/us/coronavirus-education-lost-learning.html?action=click\&pgtype=Article\&state=default\&region=MAIN_CONTENT_3\&context=storylines_faq}{online
    learning},
    \href{https://www.nytimes3xbfgragh.onion/2020/05/29/us/coronavirus-child-care-centers.html?action=click\&pgtype=Article\&state=default\&region=MAIN_CONTENT_3\&context=storylines_faq}{makeshift
    child care} and
    \href{https://www.nytimes3xbfgragh.onion/2020/06/03/business/economy/coronavirus-working-women.html?action=click\&pgtype=Article\&state=default\&region=MAIN_CONTENT_3\&context=storylines_faq}{stunted
    workdays} to continue. California's two largest public school
    districts --- Los Angeles and San Diego --- said on July 13, that
    \href{https://www.nytimes3xbfgragh.onion/2020/07/13/us/lausd-san-diego-school-reopening.html?action=click\&pgtype=Article\&state=default\&region=MAIN_CONTENT_3\&context=storylines_faq}{instruction
    will be remote-only in the fall}, citing concerns that surging
    coronavirus infections in their areas pose too dire a risk for
    students and teachers. Together, the two districts enroll some
    825,000 students. They are the largest in the country so far to
    abandon plans for even a partial physical return to classrooms when
    they reopen in August. For other districts, the solution won't be an
    all-or-nothing approach.
    \href{https://bioethics.jhu.edu/research-and-outreach/projects/eschool-initiative/school-policy-tracker/}{Many
    systems}, including the nation's largest, New York City, are
    devising
    \href{https://www.nytimes3xbfgragh.onion/2020/06/26/us/coronavirus-schools-reopen-fall.html?action=click\&pgtype=Article\&state=default\&region=MAIN_CONTENT_3\&context=storylines_faq}{hybrid
    plans} that involve spending some days in classrooms and other days
    online. There's no national policy on this yet, so check with your
    municipal school system regularly to see what is happening in your
    community.
  \end{itemize}
\end{itemize}

In Arizona, Ms. Wysong, 30, said she was willing to return to her Tempe
classroom; she is not in a high-risk category for complications from
Covid-19 and her school caps classes at 15 students. But given the
\href{https://www.nytimes3xbfgragh.onion/2018/05/02/us/arizona-teachers-philippines.html}{long-term
teacher and substitute shortage} in Arizona, which has some of the
\href{https://blogs.edweek.org/teachers/teaching_now/2019/04/which_states_have_the_highest_and_lowest_teacher_salaries.html}{lowest
educator salaries} in the nation, she said she believed the overall
system could not reopen safely with small enough class sizes.

Health and education experts who support reopening schools have
sometimes questioned the need for strict physical distancing, pointing
in recent weeks to emerging research suggesting that children may be not
only \href{https://pubmed.ncbi.nlm.nih.gov/32546824/}{less likely to
contract Covid-19}, but also
\href{https://pediatrics.aappublications.org/content/early/2020/07/08/peds.2020-004879}{less
likely to transmit it to adults}.

In interviews, many teachers said they were unaware of or skeptical of
such studies, arguing that much about the virus remains unknown, and
that even if teachers do not catch coronavirus in large numbers from
children, it could be spread among adults working in a school building,
or during commutes to and from schools via public transit.

The education systems in Germany and Denmark have
\href{https://www.nytimes3xbfgragh.onion/2020/05/10/world/europe/reopen-schools-germany.html}{successfully
reopened}, but generally
\href{https://www.sciencemag.org/news/2020/07/school-openings-across-globe-suggest-ways-keep-coronavirus-bay-despite-outbreaks}{only
after} local virus transmission rates were brought under control.

American schools currently have
\href{https://www.nytimes3xbfgragh.onion/2020/06/26/us/coronavirus-schools-reopen-fall.html}{a
variety of plans} for welcoming students back to campuses, ranging from
regular, five-day schedules with children using desk partitions to stay
distanced, to hybrid approaches that seek to keep students physically
distanced by having them attend school in-person only a few days per
week, and spend the rest of their time learning online from home.

In New York City, Mayor Bill de Blasio
\href{https://www.nytimes3xbfgragh.onion/2020/07/08/nyregion/nyc-schools-reopening-plan.html}{announced
last week} that the nation's largest school system would reopen only
part-time for students this
fall\href{https://www.nytimes3xbfgragh.onion/2020/07/08/nyregion/nyc-schools-reopening-plan.html}{,}
but teachers would most likely be back in classrooms five days a week.

The teachers' union president, Michael Mulgrew, has said he does not
believe schools can reopen at all if the city does not receive
\href{https://www.nytimes3xbfgragh.onion/2020/07/09/us/schools-reopen-fall.html}{additional
federal funding} this summer.

With many teachers reluctant to return to work,
\href{https://www.edweek.org/ew/articles/2020/06/03/most-educators-want-schools-to-stay-closed.html}{according}
to
\href{https://www.usatoday.com/story/news/education/2020/05/26/coronavirus-schools-teachers-poll-ipsos-parents-fall-online/5254729002/}{polls},
staffing will be a major challenge for districts across the country. New
York estimates that about 1 in 5 of its teachers will receive a medical
exemption to teach remotely this fall.

Matthew Landau, a history teacher at Democracy Prep Charter High School
in Harlem, hopes he will be one of them. He survived stage four cancer
several years ago and said he does not feel comfortable going back to
his classroom.

``I feel there's no way to keep immunocompromised teachers safe,'' he
said.

Kevin Kearns, a high school English teacher at the High School of
Fashion Industries in downtown Manhattan, has spent the last few weeks
wrestling with his own dilemma.

Mr. Kearns and his wife became parents in March, and need child care for
their infant son. Their only option is to have Mr. Kearns'
mother-in-law, who is in her 70s, stay with them. Mr. Kearns is
terrified of bringing the virus home.

``I don't want to go back, I don't think it's safe to go back, but I
don't know that I necessarily have a choice,'' he said.

Still, Mr. Kearns said he feels a duty to the mostly low-income, Black
and Latino students he teaches.

``It puts me in a very difficult moral conundrum,'' he said, ``to choose
between supporting my community, students, colleagues and my own
family's safety.''

Erica L. Green contributed reporting.

Advertisement

\protect\hyperlink{after-bottom}{Continue reading the main story}

\hypertarget{site-index}{%
\subsection{Site Index}\label{site-index}}

\hypertarget{site-information-navigation}{%
\subsection{Site Information
Navigation}\label{site-information-navigation}}

\begin{itemize}
\tightlist
\item
  \href{https://help.nytimes3xbfgragh.onion/hc/en-us/articles/115014792127-Copyright-notice}{©~2020~The
  New York Times Company}
\end{itemize}

\begin{itemize}
\tightlist
\item
  \href{https://www.nytco.com/}{NYTCo}
\item
  \href{https://help.nytimes3xbfgragh.onion/hc/en-us/articles/115015385887-Contact-Us}{Contact
  Us}
\item
  \href{https://www.nytco.com/careers/}{Work with us}
\item
  \href{https://nytmediakit.com/}{Advertise}
\item
  \href{http://www.tbrandstudio.com/}{T Brand Studio}
\item
  \href{https://www.nytimes3xbfgragh.onion/privacy/cookie-policy\#how-do-i-manage-trackers}{Your
  Ad Choices}
\item
  \href{https://www.nytimes3xbfgragh.onion/privacy}{Privacy}
\item
  \href{https://help.nytimes3xbfgragh.onion/hc/en-us/articles/115014893428-Terms-of-service}{Terms
  of Service}
\item
  \href{https://help.nytimes3xbfgragh.onion/hc/en-us/articles/115014893968-Terms-of-sale}{Terms
  of Sale}
\item
  \href{https://spiderbites.nytimes3xbfgragh.onion}{Site Map}
\item
  \href{https://help.nytimes3xbfgragh.onion/hc/en-us}{Help}
\item
  \href{https://www.nytimes3xbfgragh.onion/subscription?campaignId=37WXW}{Subscriptions}
\end{itemize}
