Sections

SEARCH

\protect\hyperlink{site-content}{Skip to
content}\protect\hyperlink{site-index}{Skip to site index}

\href{https://myaccount.nytimes3xbfgragh.onion/auth/login?response_type=cookie\&client_id=vi}{}

\href{https://www.nytimes3xbfgragh.onion/section/todayspaper}{Today's
Paper}

\href{/section/opinion}{Opinion}\textbar{}Dear Liberal Arts Students:
Seize This Moment

\url{https://nyti.ms/2Zn3JOc}

\begin{itemize}
\item
\item
\item
\item
\item
\item
\end{itemize}

\href{https://www.nytimes3xbfgragh.onion/spotlight/at-home?action=click\&pgtype=Article\&state=default\&region=TOP_BANNER\&context=at_home_menu}{At
Home}

\begin{itemize}
\tightlist
\item
  \href{https://www.nytimes3xbfgragh.onion/2020/07/28/books/time-for-a-literary-road-trip.html?action=click\&pgtype=Article\&state=default\&region=TOP_BANNER\&context=at_home_menu}{Take:
  A Literary Road Trip}
\item
  \href{https://www.nytimes3xbfgragh.onion/2020/07/29/magazine/bored-with-your-home-cooking-some-smoky-eggplant-will-fix-that.html?action=click\&pgtype=Article\&state=default\&region=TOP_BANNER\&context=at_home_menu}{Cook:
  Smoky Eggplant}
\item
  \href{https://www.nytimes3xbfgragh.onion/2020/07/27/travel/moose-michigan-isle-royale.html?action=click\&pgtype=Article\&state=default\&region=TOP_BANNER\&context=at_home_menu}{Look
  Out: For Moose}
\item
  \href{https://www.nytimes3xbfgragh.onion/interactive/2020/at-home/even-more-reporters-editors-diaries-lists-recommendations.html?action=click\&pgtype=Article\&state=default\&region=TOP_BANNER\&context=at_home_menu}{Explore:
  Reporters' Obsessions}
\end{itemize}

Advertisement

\protect\hyperlink{after-top}{Continue reading the main story}

\href{/section/opinion}{Opinion}

Supported by

\protect\hyperlink{after-sponsor}{Continue reading the main story}

\hypertarget{dear-liberal-arts-students-seize-this-moment}{%
\section{Dear Liberal Arts Students: Seize This
Moment}\label{dear-liberal-arts-students-seize-this-moment}}

The world needs you. Here's your chance.

\href{https://www.nytimes3xbfgragh.onion/by/jennifer-senior}{\includegraphics{https://static01.graylady3jvrrxbe.onion/images/2018/10/26/opinion/jennifer-senior/jennifer-senior-thumbLarge.png}}

By \href{https://www.nytimes3xbfgragh.onion/by/jennifer-senior}{Jennifer
Senior}

Opinion columnist

\begin{itemize}
\item
  July 12, 2020
\item
  \begin{itemize}
  \item
  \item
  \item
  \item
  \item
  \item
  \end{itemize}
\end{itemize}

\includegraphics{https://static01.graylady3jvrrxbe.onion/images/2020/07/12/opinion/12Senior/merlin_173814348_0b379044-ece0-42e2-8c29-840fb84ffb4a-articleLarge.jpg?quality=75\&auto=webp\&disable=upscale}

Over the last few weeks, America's elite colleges and universities have
been
\href{https://www.washingtonpost.com/education/2020/07/06/harvard-reopen-with-fewer-than-half-undergrads-campus-because-coronavirus/}{announcing
their reopening strategies} one by one, weighing explicit health
imperatives against implicit economic ones. Most have landed on a
disappointing-but-realistic conclusion: The fall semester of 2020 will
have to involve an alloy of in-person and remote learning. Many college
students will experience only an attenuated version of campus life ---
if they get to experience it at all.

It's easy to see what they will lose in this situation. The intoxicating
pleasures of independence. The stimulation of late-night conversation
about life, meaning, and the universe. The pure exhilaration (and
relief) of finding your own kind.

But let's face it: It's pretty luxurious to have these experiences in
the first place. Perhaps it's worth discussing what these luckiest of
students might also gain at this moment. Because they do stand to gain.
It just requires that they reimagine what late adolescence is for, and
what it should be about.

The fact is, many colleges --- especially private liberal arts schools,
the ones that have driven much of the discussion in recent weeks --- are
cloistered, passive settings. Students are fed and housed, just as they
were at home; their time and activities are structured, just as it was
when they were still in high school. College may give them wonderful
opportunities to think, form relationships, and self-define. But it
seldom gives them the chance to productively engage with the world.

Yet they are more than capable of doing so. In
``\href{https://www.publishersweekly.com/978-0-674-01508-1}{Huck's
Raft},'' his highly readable history of American childhood, Steven Mintz
points out that Herman Melville worked as a clerk, a teacher, a farmer
laborer and a cabin boy on a whaling ship, all by the age of 20. (``A
whale-ship was my Yale College and my Harvard,'' says Ishmael.) George
Washington became an official surveyor for Culpepper County at 17 and a
commissioned major in the militia at 20.

``Behavior that we would consider precocious,'' Mintz writes, ``was
commonplace.''

It wasn't until the end of World War II that children emerged as a truly
protected class in this country. That was when they finally became, in
the words of the sociologist Viviana Zelizer, ``economically worthless
but emotionally priceless.''

But only the most privileged kids got to be useless right through
college. If you think about it, that's a pretty awkward time to be
useless --- as is adolescence more generally. We may regard teenagers as
unruly and rebellious. But what they may really be is restless, pining
for greater agency and productivity, \emph{utility}.

Well. Now they have their chance.

I called Nancy Darling, a psychology professor at Oberlin College, to
discuss this. She's one of the smartest people I've ever read on the
subject of adolescence and emerging adulthood. Her response was swift
and enthusiastic. ``I keep telling my students: `This is the defining
event of your cohort. It's going to be hard, but it offers unique
opportunities. Rather than doing a crummy internship while you're
learning from home, go do something exciting! What do you want to
\emph{build}?' ''

Students of means can distribute food from food banks. They can mobilize
voters. They can organize social media campaigns for advocacy groups and
child care for essential workers and reading lists for libraries. ``If
you're a volunteer for six months,'' she points out, ``in many places
you can just take over the damn organization.''

They can help remove Donald J. Trump from office. There's an idea.

Darling notes that finding a way to be useful will be especially
valuable (if challenging) to this generation, which hasn't had much
experience in structuring its own time --- many of her students have
been overscheduled since birth --- and often conceives of
identity-building as a process of self-examination, rather than simple
\emph{doing}. They'll also have a chance to discover the importance of
civic engagement at a time when it's in severe decline.

The irony is lovely: While social distancing, they can develop habits
that will ensure they won't spend their adulthood
\href{http://-fabric-political-scientist-renews-his-alarm.html/}{bowling
alone}, to borrow the political scientist Robert Putnam's shorthand for
our disengaged lives.

Of course, most students already know what it means to be useful. A 2018
report from Georgetown University found that
\href{https://www.insidehighered.com/news/2019/11/18/most-college-students-work-and-thats-both-good-and-bad\#:~:text=The\%20Georgetown\%20report\%20found\%20that,time\%20college\%20students\%20are\%20working.}{70
percent} of full-time college students work. Those in community college,
for instance, are generally
\href{https://www.usnews.com/education/community-colleges/articles/2015/02/06/frequently-asked-questions-community-college}{older}
and come from
\href{https://ccrc.tc.columbia.edu/Community-College-FAQs.html}{low-income}
homes. Many take for granted that they'll be organizing their educations
\href{https://iwpr.org/publications/parents-college-numbers/\#:~:text=Today's\%20Student\%20Parent\%20Population,National\%20Postsecondary\%20Student\%20Aid\%20Study.}{around
work and parenting schedules}. One can only hope that asynchronous
learning will to them be a boon. It's much easier to care for your kids
and hold down a day job if you're liberated from the tyranny of a fixed
lecture schedule.

But that assumes they can afford the technology and have internet
access. Many students, at community colleges and elsewhere, now do not.
Others find themselves in households with one or two unemployed family
members, and it's suddenly on them to make ends meet ---which may or may
not mean dropping out. It's a burden that, like so many others right
now,
\href{https://www.insidehighered.com/news/2020/06/17/pandemic-has-worsened-equity-gaps-higher-education-and-work}{is
disproportionately} afflicting African-Americans and Latinos.

Having the chance to be useful --- not to their families, but to the
world --- is a luxury at this moment. Students ought to embrace it. They
may be astonished by what they find.

\emph{The Times is committed to publishing}
\href{https://www.nytimes3xbfgragh.onion/2019/01/31/opinion/letters/letters-to-editor-new-york-times-women.html}{\emph{a
diversity of letters}} \emph{to the editor. We'd like to hear what you
think about this or any of our articles. Here are some}
\href{https://help.nytimes3xbfgragh.onion/hc/en-us/articles/115014925288-How-to-submit-a-letter-to-the-editor}{\emph{tips}}\emph{.
And here's our email:}
\href{mailto:letters@NYTimes.com}{\emph{letters@NYTimes.com}}\emph{.}

\emph{Follow The New York Times Opinion section on}
\href{https://www.facebookcorewwwi.onion/nytopinion}{\emph{Facebook}}\emph{,}
\href{http://twitter.com/NYTOpinion}{\emph{Twitter (@NYTopinion)}}
\emph{and}
\href{https://www.instagram.com/nytopinion/}{\emph{Instagram}}\emph{.}

Advertisement

\protect\hyperlink{after-bottom}{Continue reading the main story}

\hypertarget{site-index}{%
\subsection{Site Index}\label{site-index}}

\hypertarget{site-information-navigation}{%
\subsection{Site Information
Navigation}\label{site-information-navigation}}

\begin{itemize}
\tightlist
\item
  \href{https://help.nytimes3xbfgragh.onion/hc/en-us/articles/115014792127-Copyright-notice}{©~2020~The
  New York Times Company}
\end{itemize}

\begin{itemize}
\tightlist
\item
  \href{https://www.nytco.com/}{NYTCo}
\item
  \href{https://help.nytimes3xbfgragh.onion/hc/en-us/articles/115015385887-Contact-Us}{Contact
  Us}
\item
  \href{https://www.nytco.com/careers/}{Work with us}
\item
  \href{https://nytmediakit.com/}{Advertise}
\item
  \href{http://www.tbrandstudio.com/}{T Brand Studio}
\item
  \href{https://www.nytimes3xbfgragh.onion/privacy/cookie-policy\#how-do-i-manage-trackers}{Your
  Ad Choices}
\item
  \href{https://www.nytimes3xbfgragh.onion/privacy}{Privacy}
\item
  \href{https://help.nytimes3xbfgragh.onion/hc/en-us/articles/115014893428-Terms-of-service}{Terms
  of Service}
\item
  \href{https://help.nytimes3xbfgragh.onion/hc/en-us/articles/115014893968-Terms-of-sale}{Terms
  of Sale}
\item
  \href{https://spiderbites.nytimes3xbfgragh.onion}{Site Map}
\item
  \href{https://help.nytimes3xbfgragh.onion/hc/en-us}{Help}
\item
  \href{https://www.nytimes3xbfgragh.onion/subscription?campaignId=37WXW}{Subscriptions}
\end{itemize}
