Sections

SEARCH

\protect\hyperlink{site-content}{Skip to
content}\protect\hyperlink{site-index}{Skip to site index}

\href{https://www.nytimes3xbfgragh.onion/section/politics}{Politics}

\href{https://myaccount.nytimes3xbfgragh.onion/auth/login?response_type=cookie\&client_id=vi}{}

\href{https://www.nytimes3xbfgragh.onion/section/todayspaper}{Today's
Paper}

\href{/section/politics}{Politics}\textbar{}Millions Have Lost Health
Insurance in Pandemic-Driven Recession

\url{https://nyti.ms/2Oq0qiE}

\begin{itemize}
\item
\item
\item
\item
\item
\end{itemize}

\begin{itemize}
\item
  \href{https://www.nytimes3xbfgragh.onion/2020/07/31/us/elections/biden-vs-trump.html?action=click\&pgtype=Article\&state=default\&region=TOP_BANNER\&context=storylines_menu}{Election
  Updates}
\item
  \href{https://www.nytimes3xbfgragh.onion/article/biden-vice-president-2020.html?action=click\&pgtype=Article\&state=default\&region=TOP_BANNER\&context=storylines_menu}{Biden's
  V.P. Search}
\item
  \href{https://www.nytimes3xbfgragh.onion/interactive/2020/07/24/us/politics/trump-biden-campaign-donors.html?action=click\&pgtype=Article\&state=default\&region=TOP_BANNER\&context=storylines_menu}{Map
  of Donations}
\item
  \href{https://www.nytimes3xbfgragh.onion/interactive/2020/us/elections/delegate-count-primary-results.html?action=click\&pgtype=Article\&state=default\&region=TOP_BANNER\&context=storylines_menu}{Delegate
  Count}
\item
  \href{https://www.nytimes3xbfgragh.onion/interactive/2019/us/politics/2020-presidential-candidates.html?action=click\&pgtype=Article\&state=default\&region=TOP_BANNER\&context=storylines_menu}{The
  Candidates}
\item
  \href{https://www.nytimes3xbfgragh.onion/newsletters/politics?action=click\&pgtype=Article\&state=default\&region=TOP_BANNER\&context=storylines_menu}{Politics
  Newsletter}
\end{itemize}

Advertisement

\protect\hyperlink{after-top}{Continue reading the main story}

Supported by

\protect\hyperlink{after-sponsor}{Continue reading the main story}

\hypertarget{millions-have-lost-health-insurance-in-pandemic-driven-recession}{%
\section{Millions Have Lost Health Insurance in Pandemic-Driven
Recession}\label{millions-have-lost-health-insurance-in-pandemic-driven-recession}}

A new study estimates that more than five million American workers lost
their insurance this spring, a number higher than those in any full year
of insurance losses.

\includegraphics{https://static01.graylady3jvrrxbe.onion/images/2020/07/13/us/politics/13dc-virus-uninsured-sub/merlin_174189681_8d85e4c0-3518-4fb7-8b6b-67283a5e2783-articleLarge.jpg?quality=75\&auto=webp\&disable=upscale}

\href{https://www.nytimes3xbfgragh.onion/by/sheryl-gay-stolberg}{\includegraphics{https://static01.graylady3jvrrxbe.onion/images/2018/11/26/multimedia/author-sheryl-gay-stolberg/author-sheryl-gay-stolberg-thumbLarge.png}}

By
\href{https://www.nytimes3xbfgragh.onion/by/sheryl-gay-stolberg}{Sheryl
Gay Stolberg}

\begin{itemize}
\item
  July 13, 2020
\item
  \begin{itemize}
  \item
  \item
  \item
  \item
  \item
  \end{itemize}
\end{itemize}

WASHINGTON --- The coronavirus pandemic stripped an estimated 5.4
million American workers of their health insurance between February and
May, a stretch in which more adults became uninsured because of job
losses than have ever lost coverage in a single year, according to a new
analysis.

The
\href{https://www.familiesusa.org/resources/the-covid-19-pandemic-and-resulting-economic-crash-have-caused-the-greatest-health-insurance-losses-in-american-history/}{study,}
to be announced Tuesday by the nonpartisan consumer advocacy group
Families USA, found that the estimated increase in uninsured workers
from February to May was nearly 40 percent higher than the highest
previous increase, which occurred during the recession of 2008 and 2009,
when 3.9 million adults lost insurance.

``We knew these numbers would be big,'' said Stan Dorn, who directs the
group's National Center for Coverage Innovation and wrote the study.
``This is the worst economic downturn since World War II. It dwarfs the
Great Recession. So it's not surprising that we would also see the worst
increase in the uninsured.''

Families USA is one of a number of groups trying to estimate the number
of people who have lost insurance during the pandemic; definitive data
will not become available until mid- to late 2021, when the federal
government publishes health insurance estimates for 2020. The analyses
vary, but all reach the same grim conclusion: More people lack insurance
than ever before.

The nonpartisan Kaiser Family Foundation has
\href{https://www.kff.org/coronavirus-covid-19/issue-brief/eligibility-for-aca-health-coverage-following-job-loss/}{estimated}
that 27 million Americans have lost coverage in the pandemic; that study
took into account family members of the insured.
\href{https://www.rwjf.org/en/library/research/2020/07/changes-in-health-insurance-coverage-due-to-the-covid-19-recession--preliminary-estimates-using-microsimulation.html?cid=xem_other_unpd_ini:quickstrike_dte:20200713_des:coverage\%20changes;\%20covid}{Another
analysis,} published Monday by the Urban Institute and the Robert Wood
Johnson Foundation, projected that by the end of 2020, 10.1 million
people will no longer have employer-sponsored health insurance or
coverage that was tied to a job they lost because of the pandemic.

And those losing coverage could face staggering costs if they are struck
by Covid-19, which has sent the seriously ill to hospital intensive care
units for weeks, sometimes months.

The studies come in the thick of the campaign season, when health care
--- and in particular the future of the Affordable Care Act, popularly
known as Obamacare --- will be a major issue. Democrats and their
presumptive presidential nominee, Joseph R. Biden Jr., want to expand
the law. President Trump has asked the Supreme Court to
\href{https://www.nytimes3xbfgragh.onion/2020/06/22/us/politics/republicans-health-care-coronavirus.html}{overturn
it}.

\hypertarget{latest-updates-2020-election}{%
\section{\texorpdfstring{\href{https://www.nytimes3xbfgragh.onion/2020/07/31/us/elections/biden-vs-trump.html?action=click\&pgtype=Article\&state=default\&region=MAIN_CONTENT_1\&context=storylines_live_updates}{Latest
Updates: 2020
Election}}{Latest Updates: 2020 Election}}\label{latest-updates-2020-election}}

Updated 2020-08-01T01:26:45.732Z

\begin{itemize}
\tightlist
\item
  \href{https://www.nytimes3xbfgragh.onion/2020/07/31/us/elections/biden-vs-trump.html?action=click\&pgtype=Article\&state=default\&region=MAIN_CONTENT_1\&context=storylines_live_updates\#link-29fdff45}{Kamala
  Harris, a top vice-presidential contender, confronts double
  standards.}
\item
  \href{https://www.nytimes3xbfgragh.onion/2020/07/31/us/elections/biden-vs-trump.html?action=click\&pgtype=Article\&state=default\&region=MAIN_CONTENT_1\&context=storylines_live_updates\#link-13ec3d9c}{Karen
  Bass and Susan Rice are rising on Biden's vice-presidential
  shortlist.}
\item
  \href{https://www.nytimes3xbfgragh.onion/2020/07/31/us/elections/biden-vs-trump.html?action=click\&pgtype=Article\&state=default\&region=MAIN_CONTENT_1\&context=storylines_live_updates\#link-49e9a016}{Trump
  says Russian bounties to kill U.S. troops `never took place.'}
\end{itemize}

\href{https://www.nytimes3xbfgragh.onion/2020/07/31/us/elections/biden-vs-trump.html?action=click\&pgtype=Article\&state=default\&region=MAIN_CONTENT_1\&context=storylines_live_updates}{See
more updates}

Four of every five people who have lost employer-provided health
insurance during the coronavirus pandemic are eligible for free coverage
through expanded Medicaid programs or
\href{https://www.kff.org/health-reform/issue-brief/explaining-health-care-reform-questions-about-health/}{government-subsidized
private insurance} through the Obama-era health law, according to the
Kaiser Family Foundation.

But experts say that insuring the recently unemployed is a difficult
challenge. Many people cannot afford premiums for coverage through
either the health care law or the program known as
\href{http://www.dol.gov/dol/topic/health-plans/cobra.htm}{COBRA,} for
the Consolidated Omnibus Budget Reconciliation Act. Others might not
know they are eligible for Medicaid.

The White House and Congress have done little to help. The Trump
administration has imposed sharp cuts on the funding for outreach
programs that assist people in signing up for coverage under the health
law. And while House Democrats have passed legislation intended to help
people to keep their health insurance, the bill is stuck in the
Republican-controlled Senate.

Rather than expand access to subsidized insurance under the Affordable
Care Act, Mr. Trump has promised to directly reimburse hospitals for the
care of coronavirus patients who have lost their insurance. But there is
little evidence that has begun.

``Helping people keep their insurance through a public health crisis
surprisingly has not gotten much attention,'' said Larry Levitt,
executive vice president for health policy at the Kaiser Family
Foundation. ``This is the first recession in which the A.C.A. is there
as a safety net, but it's an imperfect safety net.''

The Families USA study is a state-by-state examination of the effects of
the pandemic on laid-off adults younger than 65, the age at which
Americans become eligible for Medicare. It found that nearly half --- 46
percent --- of the coverage losses from the pandemic came in five
states: California, Texas, Florida, New York and North Carolina.

In Texas alone, the number of uninsured jumped from about 4.3 million to
nearly 4.9 million; three out of every 10 Texans are uninsured, the
research found. In the 37 states that expanded Medicaid under the
Affordable Care Act, 23 percent of laid-off workers became uninsured;
the percentage was nearly double that --- 43 percent --- in the
\href{https://www.kff.org/medicaid/issue-brief/status-of-state-medicaid-expansion-decisions-interactive-map/}{13
states that did not expand Medicaid}, which include Texas, Florida and
North Carolina.

Five states have experienced increases in the number of uninsured adults
that exceed 40 percent, the analysis found. In Massachusetts, the number
nearly doubled, rising by 93 percent --- a figure Mr. Dorn attributed to
a large number of people losing employer-based coverage there. Across
the country as a whole, more than one in seven adults --- 16 percent ---
is now uninsured, the analysis found.

To generate the estimates, Mr. Dorn examined the number of laid-off
workers in each state and calculated how many had become uninsured based
on coverage patterns since 2014, when the central provisions of the
Affordable Care Act went into effect. The underlying data for those
patterns comes from work published by the Urban Institute in April.

Although analysts will have a clearer picture of the actual figures next
year, Mr. Dorn said, ``policymakers need to know now what the
approximate magnitude is of insurance losses to decide what they need to
do. So this is our best estimate for what the actual coverage losses
have been.''

Democrats and health care advocacy groups argue that the importance of
insurance coverage extends beyond personal well-being because the
uninsured tend to avoid going to the doctor, and that exposes others to
an infectious disease outbreak like Covid-19.

On Capitol Hill, Senator Patty Murray of Washington, the top Democrat on
the Senate health committee, has been pressing the Trump administration
to do a better job of promoting a provision in the Affordable Care Act
that creates a special enrollment period for people who lose their jobs.

Ms. Murray and other Democrats have also
\href{https://www.help.senate.gov/ranking/newsroom/press/bicameral-democratic-health-leaders-call-for-additional-action-to-ensure-cost-is-not-a-barrier-for-covid-19-treatment}{called
for} the federal government to provide financial assistance to help
workers who are laid off maintain their coverage through COBRA, and to
give states that have refused to expand Medicaid an incentive to do so,
by increasing the federal share of the cost

``Taking steps like these to help people get access to health care
during a pandemic shouldn't be controversial, it should be common
sense,'' Ms. Murray said in an email, ``and we should be doing it right
now instead of waiting for things to get even worse.''

\hypertarget{our-2020-election-guide}{%
\section{Our 2020 Election Guide}\label{our-2020-election-guide}}

Updated July 31, 2020

\begin{itemize}
\item
  \begin{center}\rule{0.5\linewidth}{\linethickness}\end{center}

  \hypertarget{the-latest}{%
  \subsection{The Latest}\label{the-latest}}

  \begin{itemize}
  \tightlist
  \item
    President Trump's assault on the Postal Service is intersecting with
    his attacks on mail-in voting.
    \href{https://www.nytimes3xbfgragh.onion/2020/07/31/us/politics/trump-usps-mail-delays.html?action=click\&pgtype=Article\&state=default\&region=BELOW_MAIN_CONTENT\&context=storylines_guide}{Voting
    rights groups say it is a recipe for disaster.}
  \end{itemize}
\item
  \begin{center}\rule{0.5\linewidth}{\linethickness}\end{center}

  \hypertarget{bidens-vp-search}{%
  \subsection{Biden's V.P. Search}\label{bidens-vp-search}}

  \begin{itemize}
  \tightlist
  \item
    \href{https://www.nytimes3xbfgragh.onion/article/biden-vice-president-2020.html?action=click\&pgtype=Article\&state=default\&region=BELOW_MAIN_CONTENT\&context=storylines_guide}{Here
    are 13 women} who have been under consideration to be Joe Biden's
    running mate, and why each might be chosen --- and might not be.
  \end{itemize}
\item
  \begin{center}\rule{0.5\linewidth}{\linethickness}\end{center}

  \hypertarget{keep-up-with-our-coverage}{%
  \subsection{Keep Up With Our
  Coverage}\label{keep-up-with-our-coverage}}

  \begin{itemize}
  \tightlist
  \item
    Get an
    \href{https://www.nytimes3xbfgragh.onion/newsletters/politics?action=click\&pgtype=Article\&state=default\&region=BELOW_MAIN_CONTENT\&context=storylines_guide}{email}
    recapping the day's news
  \end{itemize}

  \begin{itemize}
  \tightlist
  \item
    Download our mobile app on
    \href{https://apps.apple.com/us/app/nytimes/id284862083?ls=1\&mat_click_id=5c79ae7455014fd1bd66b5610c05b8f2-20191112-16948\&referrer=mat_click_id\%3D5c79ae7455014fd1bd66b5610c05b8f2-20191112-16948\%26link_click_id\%3D722930677036718082}{iOS}
    and
    \href{http://a.localytics.com/android?id=com.nytimes.android\&referrer=utm_source\%3Dother_nyt_mobile_web\%26utm_medium\%3DWeb\%2520page\%26utm_term\%3DGeneral\%2520Mobile\%2520Page\%26utm_campaign\%3DNYT\%2520Mobile\%2520General\%2520Page}{Android}
    and turn on Breaking News and Politics alerts
  \end{itemize}
\end{itemize}

Advertisement

\protect\hyperlink{after-bottom}{Continue reading the main story}

\hypertarget{site-index}{%
\subsection{Site Index}\label{site-index}}

\hypertarget{site-information-navigation}{%
\subsection{Site Information
Navigation}\label{site-information-navigation}}

\begin{itemize}
\tightlist
\item
  \href{https://help.nytimes3xbfgragh.onion/hc/en-us/articles/115014792127-Copyright-notice}{©~2020~The
  New York Times Company}
\end{itemize}

\begin{itemize}
\tightlist
\item
  \href{https://www.nytco.com/}{NYTCo}
\item
  \href{https://help.nytimes3xbfgragh.onion/hc/en-us/articles/115015385887-Contact-Us}{Contact
  Us}
\item
  \href{https://www.nytco.com/careers/}{Work with us}
\item
  \href{https://nytmediakit.com/}{Advertise}
\item
  \href{http://www.tbrandstudio.com/}{T Brand Studio}
\item
  \href{https://www.nytimes3xbfgragh.onion/privacy/cookie-policy\#how-do-i-manage-trackers}{Your
  Ad Choices}
\item
  \href{https://www.nytimes3xbfgragh.onion/privacy}{Privacy}
\item
  \href{https://help.nytimes3xbfgragh.onion/hc/en-us/articles/115014893428-Terms-of-service}{Terms
  of Service}
\item
  \href{https://help.nytimes3xbfgragh.onion/hc/en-us/articles/115014893968-Terms-of-sale}{Terms
  of Sale}
\item
  \href{https://spiderbites.nytimes3xbfgragh.onion}{Site Map}
\item
  \href{https://help.nytimes3xbfgragh.onion/hc/en-us}{Help}
\item
  \href{https://www.nytimes3xbfgragh.onion/subscription?campaignId=37WXW}{Subscriptions}
\end{itemize}
