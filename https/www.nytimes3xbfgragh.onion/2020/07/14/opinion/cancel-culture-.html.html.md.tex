Sections

SEARCH

\protect\hyperlink{site-content}{Skip to
content}\protect\hyperlink{site-index}{Skip to site index}

\href{https://myaccount.nytimes3xbfgragh.onion/auth/login?response_type=cookie\&client_id=vi}{}

\href{https://www.nytimes3xbfgragh.onion/section/todayspaper}{Today's
Paper}

\href{/section/opinion}{Opinion}\textbar{}10 Theses About Cancel Culture

\url{https://nyti.ms/32hDJWe}

\begin{itemize}
\item
\item
\item
\item
\item
\item
\end{itemize}

Advertisement

\protect\hyperlink{after-top}{Continue reading the main story}

\href{/section/opinion}{Opinion}

Supported by

\protect\hyperlink{after-sponsor}{Continue reading the main story}

\hypertarget{10-theses-about-cancel-culture}{%
\section{10 Theses About Cancel
Culture}\label{10-theses-about-cancel-culture}}

What we talk about when we talk about ``cancellation.''

\href{https://www.nytimes3xbfgragh.onion/by/ross-douthat}{\includegraphics{https://static01.graylady3jvrrxbe.onion/images/2018/04/03/opinion/ross-douthat/ross-douthat-thumbLarge.png}}

By \href{https://www.nytimes3xbfgragh.onion/by/ross-douthat}{Ross
Douthat}

Opinion Columnist

\begin{itemize}
\item
  July 14, 2020
\item
  \begin{itemize}
  \item
  \item
  \item
  \item
  \item
  \item
  \end{itemize}
\end{itemize}

\includegraphics{https://static01.graylady3jvrrxbe.onion/images/2020/07/14/opinion/14douthatWeb/14douthatWeb-articleLarge.jpg?quality=75\&auto=webp\&disable=upscale}

Cancel culture is destroying liberalism. No, cancel culture doesn't
exist. No, it has always existed; remember when Brutus and Cassius
canceled Julius Caesar? No, it exists but it's just a bunch of rich
entitled celebrities complaining that people can finally talk back to
them on Twitter. No, it doesn't exist except when it's good and the
canceled deserve it. Actually, it does exist, but --- well, look, I
can't explain it to you until you've read at least four open letters on
the subject.

These are just a few of the answers that you'll get to a simple question
--- ``What is this \emph{cancel culture} thing, anyway?'' --- if you're
foolish enough to toss it, like chum, into the seething waters of the
internet. They're contradictory because the phenomenon is complicated
--- but not complicated enough to deter me from making 10 sweeping
claims about the subject.

So here goes:

\hypertarget{1-cancellation-properly-understood-refers-to-an-attack-on-someones-employment-and-reputation-by-a-determined-collective-of-critics-based-on-an-opinion-or-an-action-that-is-alleged-to-be-disgraceful-and-disqualifying}{%
\subsubsection{1. Cancellation, properly understood, refers to an attack
on someone's employment and reputation by a determined collective of
critics, based on an opinion or an action that is alleged to be
disgraceful and
disqualifying.}\label{1-cancellation-properly-understood-refers-to-an-attack-on-someones-employment-and-reputation-by-a-determined-collective-of-critics-based-on-an-opinion-or-an-action-that-is-alleged-to-be-disgraceful-and-disqualifying}}

``Reputation'' and ``employment'' are key terms here. You are not being
canceled if you are merely being heckled or insulted --- if somebody
describes you as a moron or a fascist or some profane alternative to
``Douthat'' on the internet --- no matter how vivid and threatening the
heckling becomes. You are decidedly at risk of cancellation, however, if
your critics are calling for you to be de-platformed or fired or
\href{https://coloradosun.com/2020/06/29/kindness-yoga-closure-during-black-lives-matter/}{put
out of business}, and especially if the call is coming from inside the
house --- from within your professional community, from co-workers or
employees or potential customers or colleagues, on a professional
message board or Slack or some interest-specific slice of social media.

\hypertarget{2-all-cultures-cancel-the-question-is-for-what-how-widely-and-through-what-means}{%
\subsubsection{2. All cultures cancel; the question is for what, how
widely and through what
means.}\label{2-all-cultures-cancel-the-question-is-for-what-how-widely-and-through-what-means}}

There is no human society where you can say or do anything you like and
expect to keep your reputation and your job. Reputational cancellation
hung over the heads of Edith Wharton's heroines; professional
cancellation shadowed 20th-century figures like Lenny Bruce. Today,
almost all critics of cancel culture have some line they draw, some
figure --- usually a racist or anti-Semite --- that they would cancel,
too. And social conservatives who criticize cancel culture, especially,
have to acknowledge that we're partly just disagreeing with today's list
of cancellation-worthy sins.

\hypertarget{3-cancellation-isnt-exactly-about-free-speech-but-a-liberal-society-should-theoretically-cancel-less-frequently-than-its-rivals}{%
\subsubsection{3. Cancellation isn't exactly about free speech, but a
liberal society should theoretically cancel less frequently than its
rivals.}\label{3-cancellation-isnt-exactly-about-free-speech-but-a-liberal-society-should-theoretically-cancel-less-frequently-than-its-rivals}}

The canceled individual hasn't lost any First Amendment rights, because
there is no constitutional right to a particular job or reputation. At
the same time, under its own self-understanding, liberalism is supposed
to clear a wider space for debate than other political systems and allow
a wider range of personal expression. So you would expect a liberal
society to be slower to cancel, more inclined to separate the personal
and the professional (or the ideological and the artistic), and quicker
to offer opportunities to regain one's reputation and start one's
professional life anew.

``It's a free country,'' runs the American boast, and even if it doesn't
violate the Constitution, cancellation cuts against that promise ---
which is one reason arguments about cancel culture so often become
arguments about liberalism itself.

\includegraphics{https://static01.graylady3jvrrxbe.onion/images/2019/11/19/autossell/cancelling-thumb_01/cancelling-thumb_01-videoSixteenByNine3000.png}

\hypertarget{4-the-internet-has-changed-the-way-we-cancel-and-extended-cancellations-reach}{%
\subsubsection{4. The internet has changed the way we cancel, and
extended cancellation's
reach.}\label{4-the-internet-has-changed-the-way-we-cancel-and-extended-cancellations-reach}}

On the other hand, a skeptic might say that it wasn't liberalism but
space and distance that made America a free country --- the fact that
you could always escape the tyrannies of local conformism by ``lighting
out for the territory,'' in the old Mark Twain phrase. But under the
rule of the internet there's no leaving the village: Everywhere is the
same place, and so is every time. You can be canceled for something you
said in a crowd of complete strangers, if one of them uploads the video,
or for a joke that came out wrong if you happened to make it on social
media, or for something you said or did
\href{https://www.nytimes3xbfgragh.onion/reuters/2020/07/02/business/02reuters-boeing-resignation.html}{a
long time ago} if the internet remembers. And you don't have to be
prominent or political to be
\href{https://www.penguinrandomhouse.com/books/306321/so-youve-been-publicly-shamed-by-jon-ronson/}{publicly
shamed} and
\href{https://www.nytimes3xbfgragh.onion/2018/10/23/nyregion/man-found-dead-in-car-new-york.html}{permanently
marked}: All you need to do is have a particularly bad day, and the
consequences could endure as long as Google.

\hypertarget{5-the-internet-has-also-made-it-harder-to-figure-out-whether-speech-is-getting-freer-or-less-free}{%
\subsubsection{5. The internet has also made it harder to figure out
whether speech is getting freer or less
free.}\label{5-the-internet-has-also-made-it-harder-to-figure-out-whether-speech-is-getting-freer-or-less-free}}

When critics of cancel culture fret about a potential online-era chill
on speech, one rejoinder is that you can find far more ideas --- both
radical and noxious --- swirling on the internet than you could in a
sampling of magazines and daily newspapers circa 1990. It's easier to
encounter ideological extremes on your smartphone than it was in the
beforetime of print media, and easier to encounter hateful speech as
well.

But at the same time the internet has hastened the consolidation of
cultural institutions, so that The New York Times and the Ivy League and
other behemoths loom larger than they did 30 years ago, and it's
arguably increased uniformity across cities and regions and industries
in general. And the battle over norms for cancellation reflects both of
these changes: For would-be cancelers, the chaos of the internet makes
it seem that much more important to establish rigorous new norms, lest
the online racists win \ldots{} but for people under threat of
cancellation, it feels like they're at risk at being shut out of a
journalistic or academic marketplace that's ever more consolidated, or
defying a consensus that's embraced by every boardroom and H.R.
department.

\hypertarget{6-celebrities-are-the-easiest-people-to-target-but-the-hardest-people-to-actually-cancel}{%
\subsubsection{6. Celebrities are the easiest people to target, but the
hardest people to actually
cancel.}\label{6-celebrities-are-the-easiest-people-to-target-but-the-hardest-people-to-actually-cancel}}

One of the ur-examples of cancel culture was the activist Suey Park's
2014 hashtag campaign to \#cancelColbert over a satirical tweet from the
Twitter account of ``The Colbert Report.'' Six years later, Stephen
Colbert is very much uncanceled. So are Dave Chappelle, J.K. Rowling and
a much longer list of prominent pop culture figures who have faced
online mobs and lived to tell, sell and perform.

Their resilience explains why some people dismiss cancellation as just
famous people whining about their critics. If someone has a big enough
name or fan base, the bar for actual cancellation is quite high, and the
celebrity might even have the opportunity --- like a certain
reality-television star on the campaign trail in 2016 --- to use the
hatred of the would-be cancelers to confirm a fandom or cement a
following.

However, not everyone is a celebrity, and \ldots{}

\hypertarget{7-cancel-culture-is-most-effective-against-people-who-are-still-rising-in-their-fields-and-it-influences-many-people-who-dont-actually-get-canceled}{%
\subsubsection{7. Cancel culture is most effective against people who
are still rising in their fields, and it influences many people who
don't actually get
canceled.}\label{7-cancel-culture-is-most-effective-against-people-who-are-still-rising-in-their-fields-and-it-influences-many-people-who-dont-actually-get-canceled}}

The point of cancellation is ultimately to establish norms for the
majority, not to bring the stars back down to earth. So a climate of
cancellation can succeed in changing the way people talk and argue and
behave even if it doesn't succeed in destroying the careers of some of
the famous people that it targets. You don't need to cancel Rowling if
you can cancel the
\href{https://www.insider.com/gillian-philip-childrens-author-sacked-tweeting-support-jk-rowling-2020-7}{lesser-known
novelist} who takes her side; you don't have to take down the famous
academics who signed last week's
\href{https://harpers.org/a-letter-on-justice-and-open-debate/}{Harper's
Magazine letter} attacking cancel culture if you can discourage
\href{https://twitter.com/SwipeWright/status/1281793002986336256}{people
half their age} from saying what they think. The goal isn't to punish
everyone, or even very many someones; it's to shame or scare just enough
people to make the rest conform.

\hypertarget{8-the-right-and-the-left-both-cancel-its-just-that-todays-right-is-too-weak-to-do-it-effectively}{%
\subsubsection{8. The right and the left both cancel; it's just that
today's right is too weak to do it
effectively.}\label{8-the-right-and-the-left-both-cancel-its-just-that-todays-right-is-too-weak-to-do-it-effectively}}

Is it cancel culture when conservatives try to get college professors
disciplined for anti-Americanism, or critics of Israel de-platformed for
anti-Semitism? Sure, in a sense. Was it cancel culture when the Dixie
Chicks --- sorry, the artists
\href{https://www.nytimes3xbfgragh.onion/2020/06/25/arts/music/dixie-chicks-change-name.html}{formerly
known as the Dixie Chicks} --- were dropped by radio stations and tour
venues, or when Bill Maher's ``Politically Incorrect'' was literally
canceled, for falling afoul of patriotic correctness? Absolutely.

But as the latter examples suggest, the last peak of right-wing cultural
power was the patriotically correct climate after Sept. 11, a cultural
eon in the past. Today the people with the most to fear from a
right-wing cancel culture usually work inside Trump-era professional
conservatism. (And even for them there's often a new life awaiting as a
professional NeverTrumper.) Attempted cancellations on the right are
mostly battles for control over diminishing terrain, with occasional
forays against red-state academics and anti-Trump celebrities.
Meanwhile, the left's cancel warriors imagine themselves conquering the
entire non-Fox News map.

\hypertarget{9-the-heat-of-the-cancel-culture-debate-reflects-the-intersection-of-the-internet-as-a-medium-for-cancellation-with-the-increasing-power-of-left-wing-moral-norms-as-a-justification-for-cancellation}{%
\subsubsection{9. The heat of the cancel-culture debate reflects the
intersection of the internet as a medium for cancellation with the
increasing power of left-wing moral norms as a justification for
cancellation.}\label{9-the-heat-of-the-cancel-culture-debate-reflects-the-intersection-of-the-internet-as-a-medium-for-cancellation-with-the-increasing-power-of-left-wing-moral-norms-as-a-justification-for-cancellation}}

It's not just technology or ideology, in other words, it's both. The
emergent, youthful left wants to take current taboos against racism and
anti-Semitism and use them as a model for a wider range of limits ---
with more expansive definitions of what counts as racism and sexism and
homophobia, a more sweeping theory of what sorts of speech and behavior
threaten ``harm'' and a more precise linguistic etiquette for
respectable professionals to follow. And the internet and social media,
both outside institutions and within, are crucial mechanisms for this
push.

It's debatable whether these new left-wing norms would be illiberal or
whether they would simply infuse liberalism with a new morality to
\href{https://www.nytimes3xbfgragh.onion/2020/07/07/opinion/protestant-progressive-reformation.html}{replace}
the old Protestant consensus. It's arguable whether they would expand
the space for previously marginalized voices more than they would
restrict once-mainstream, now ``phobic'' points of view. But there's no
question that people who fall afoul of the emergent norms are more
exposed to cancellation than they would have been 10 or 20 years ago.

\hypertarget{10-if-you-oppose-left-wing-cancel-culture-appeals-to-liberalism-and-free-speech-arent-enough}{%
\subsubsection{10. If you oppose left-wing cancel culture, appeals to
liberalism and free speech aren't
enough.}\label{10-if-you-oppose-left-wing-cancel-culture-appeals-to-liberalism-and-free-speech-arent-enough}}

I said earlier that debates about cancellations are also inevitably
debates about liberalism and its limits. But to defend a liberal
position in these arguments you need more than just a defense of free
speech in the abstract; you need to defend free speech for the sake of
some important, true idea. General principles are well and good, but if
you can't champion controversial ideas on their own merits, no merely
procedural argument for granting them a platform will sustain itself
against a passionate, morally confident attack.

So liberals or centrists who fear the left-wing zeal for cancellation
need a counterargument that doesn't rest on right-to-be-wrong principles
alone. They need to identify the places where they think the new
left-wing norms aren't merely too censorious but simply wrong, and fight
the battle there, on substance as well as liberal principle.

Otherwise their battle for free speech is only likely to win them the
privilege of having their own ideas canceled last of all.

\emph{The Times is committed to publishing}
\href{https://www.nytimes3xbfgragh.onion/2019/01/31/opinion/letters/letters-to-editor-new-york-times-women.html}{\emph{a
diversity of letters}} \emph{to the editor. We'd like to hear what you
think about this or any of our articles. Here are some}
\href{https://help.nytimes3xbfgragh.onion/hc/en-us/articles/115014925288-How-to-submit-a-letter-to-the-editor}{\emph{tips}}\emph{.
And here's our email:}
\href{mailto:letters@NYTimes.com}{\emph{letters@NYTimes.com}}\emph{.}

\emph{Follow The New York Times Opinion section on}
\href{https://www.facebookcorewwwi.onion/nytopinion}{\emph{Facebook}}\emph{,}
\href{http://twitter.com/NYTOpinion}{\emph{Twitter (@NYTOpinion)}}
\emph{and}
\href{https://www.instagram.com/nytopinion/}{\emph{Instagram}}\emph{,
join the Facebook political discussion group,}
\href{https://www.facebookcorewwwi.onion/groups/votingwhilefemale/}{\emph{Voting
While Female}}\emph{.}

Advertisement

\protect\hyperlink{after-bottom}{Continue reading the main story}

\hypertarget{site-index}{%
\subsection{Site Index}\label{site-index}}

\hypertarget{site-information-navigation}{%
\subsection{Site Information
Navigation}\label{site-information-navigation}}

\begin{itemize}
\tightlist
\item
  \href{https://help.nytimes3xbfgragh.onion/hc/en-us/articles/115014792127-Copyright-notice}{©~2020~The
  New York Times Company}
\end{itemize}

\begin{itemize}
\tightlist
\item
  \href{https://www.nytco.com/}{NYTCo}
\item
  \href{https://help.nytimes3xbfgragh.onion/hc/en-us/articles/115015385887-Contact-Us}{Contact
  Us}
\item
  \href{https://www.nytco.com/careers/}{Work with us}
\item
  \href{https://nytmediakit.com/}{Advertise}
\item
  \href{http://www.tbrandstudio.com/}{T Brand Studio}
\item
  \href{https://www.nytimes3xbfgragh.onion/privacy/cookie-policy\#how-do-i-manage-trackers}{Your
  Ad Choices}
\item
  \href{https://www.nytimes3xbfgragh.onion/privacy}{Privacy}
\item
  \href{https://help.nytimes3xbfgragh.onion/hc/en-us/articles/115014893428-Terms-of-service}{Terms
  of Service}
\item
  \href{https://help.nytimes3xbfgragh.onion/hc/en-us/articles/115014893968-Terms-of-sale}{Terms
  of Sale}
\item
  \href{https://spiderbites.nytimes3xbfgragh.onion}{Site Map}
\item
  \href{https://help.nytimes3xbfgragh.onion/hc/en-us}{Help}
\item
  \href{https://www.nytimes3xbfgragh.onion/subscription?campaignId=37WXW}{Subscriptions}
\end{itemize}
