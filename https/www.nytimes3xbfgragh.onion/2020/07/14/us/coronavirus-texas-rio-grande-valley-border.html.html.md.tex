Sections

SEARCH

\protect\hyperlink{site-content}{Skip to
content}\protect\hyperlink{site-index}{Skip to site index}

\href{https://www.nytimes3xbfgragh.onion/section/us}{U.S.}

\href{https://myaccount.nytimes3xbfgragh.onion/auth/login?response_type=cookie\&client_id=vi}{}

\href{https://www.nytimes3xbfgragh.onion/section/todayspaper}{Today's
Paper}

\href{/section/us}{U.S.}\textbar{}I Went Home to Texas to Cover the
Virus. Then My Family Got It.

\url{https://nyti.ms/2OqWuhC}

\begin{itemize}
\item
\item
\item
\item
\item
\item
\end{itemize}

\href{https://www.nytimes3xbfgragh.onion/news-event/coronavirus?action=click\&pgtype=Article\&state=default\&region=TOP_BANNER\&context=storylines_menu}{The
Coronavirus Outbreak}

\begin{itemize}
\tightlist
\item
  live\href{https://www.nytimes3xbfgragh.onion/2020/08/04/world/coronavirus-cases.html?action=click\&pgtype=Article\&state=default\&region=TOP_BANNER\&context=storylines_menu}{Latest
  Updates}
\item
  \href{https://www.nytimes3xbfgragh.onion/interactive/2020/us/coronavirus-us-cases.html?action=click\&pgtype=Article\&state=default\&region=TOP_BANNER\&context=storylines_menu}{Maps
  and Cases}
\item
  \href{https://www.nytimes3xbfgragh.onion/interactive/2020/science/coronavirus-vaccine-tracker.html?action=click\&pgtype=Article\&state=default\&region=TOP_BANNER\&context=storylines_menu}{Vaccine
  Tracker}
\item
  \href{https://www.nytimes3xbfgragh.onion/2020/08/02/us/covid-college-reopening.html?action=click\&pgtype=Article\&state=default\&region=TOP_BANNER\&context=storylines_menu}{College
  Reopening}
\item
  \href{https://www.nytimes3xbfgragh.onion/live/2020/08/04/business/stock-market-today-coronavirus?action=click\&pgtype=Article\&state=default\&region=TOP_BANNER\&context=storylines_menu}{Economy}
\end{itemize}

Advertisement

\protect\hyperlink{after-top}{Continue reading the main story}

Supported by

\protect\hyperlink{after-sponsor}{Continue reading the main story}

\hypertarget{i-went-home-to-texas-to-cover-the-virus-then-my-family-got-it}{%
\section{I Went Home to Texas to Cover the Virus. Then My Family Got
It.}\label{i-went-home-to-texas-to-cover-the-virus-then-my-family-got-it}}

Edgar Sandoval knew he was well prepared to report on the spread of the
coronavirus on the Texas-Mexico border. He was going home. Little did he
know exactly what he would find.

\includegraphics{https://static01.graylady3jvrrxbe.onion/images/2020/07/10/us/00virus-mcallen01/00virus-mcallen01-articleLarge.jpg?quality=75\&auto=webp\&disable=upscale}

By \href{https://www.nytimes3xbfgragh.onion/by/edgar-sandoval}{Edgar
Sandoval}

Mr. Sandoval, a criminal justice reporter for The Times, was born in
California, came of age in Mexico and moved to the Rio Grande Valley as
a teenager.

\begin{itemize}
\item
  Published July 14, 2020Updated Aug. 4, 2020
\item
  \begin{itemize}
  \item
  \item
  \item
  \item
  \item
  \item
  \end{itemize}
\end{itemize}

\href{https://www.nytimes3xbfgragh.onion/es/2020/07/14/espanol/texas-coronavirus-rio-grande-valley.html}{Leer
en español}

McALLEN, Texas --- There were many reasons for a special level of alarm
when the
\href{https://www.nytimes3xbfgragh.onion/2020/08/04/us/texas-coronavirus-rio-grande-valley-starr-county.html}{coronavirus}
swept out of
\href{https://www.nytimes3xbfgragh.onion/2020/08/04/us/texas-coronavirus-rio-grande-valley-starr-county.html}{Texas'}
biggest cities in recent weeks and arrived with force in the Rio Grande
Valley.

The small cities along the border with Mexico are among the poorest in
Texas. The Valley, as local residents call it, is a place of hard labor
and low pay where ``working from home'' is unfeasible. It is dotted with
teeming colonias, orphan communities that are often without paved roads
or sewer connections --- places where the virus, once it arrives, can
thrive.

The worries, it turned out, were justified. More than 8,000 people in
Hidalgo County --- some of whom I know only too well --- have had cases
of the virus confirmed. The county on Thursday surpassed its previous
record with 1,274 cases in a single day; more than 150 people have died.

My family moved to the Rio Grande Valley in the mid-1990s, when I was
16. My parents, two sisters and their offspring all live on the same
block outside of McAllen. As soon as the outbreak reached the border, I
volunteered to report on the story because I was uniquely equipped to
tell it.

After having reported on the virus during its calamitous sweep through
New York City, I knew that the warm, close-knit family culture I had
grown up with in the Valley would make social distancing a challenge ---
and as I called ahead to prepare for my return home, I quickly learned
that my worst fears were coming true.

Seemingly overnight, people's aunts, uncles, grandmas and cousins were
falling victim, one by one, to the highly contagious virus.

``Three months ago very few knew anyone who had contracted the virus,''
Jim Darling, the mayor of McAllen, Hidalgo County's largest city, told
me. ``Now, you can't find people who don't know anyone who isn't
infected. It completely flipped.''

But I never expected that I would be part of this story.

The day before I boarded a plane from New York, my youngest sister sent
me a text message that froze me in place. ``Brother, it looks like all
of the Sandovales have Covid,'' it read in Spanish.

Five in my family, including my mother, Arcelia; my father, Filiberto;
two sisters; and a nephew all had symptoms, she said.

By the time my plane landed the next day, that number had doubled.

\hypertarget{a-culture-of-close-knit-families}{%
\subsection{A culture of close-knit
families}\label{a-culture-of-close-knit-families}}

In recent weeks, public health officials have been imploring Texans to
wear masks and obey social-distancing guidelines. Some restaurants have
begun taking patrons' temperatures, and bars have remained largely
closed since the pandemic made a resurgence.

But one thing that has continued to stymie efforts at keeping people at
a distance in the Valley is its longstanding culture of pachangas, a
colloquial expression for the festive family gatherings where social
distancing is almost nonexistent.

\hypertarget{latest-updates-global-coronavirus-outbreak}{%
\section{\texorpdfstring{\href{https://www.nytimes3xbfgragh.onion/2020/08/04/world/coronavirus-cases.html?action=click\&pgtype=Article\&state=default\&region=MAIN_CONTENT_1\&context=storylines_live_updates}{Latest
Updates: Global Coronavirus
Outbreak}}{Latest Updates: Global Coronavirus Outbreak}}\label{latest-updates-global-coronavirus-outbreak}}

Updated 2020-08-05T07:58:24.076Z

\begin{itemize}
\tightlist
\item
  \href{https://www.nytimes3xbfgragh.onion/2020/08/04/world/coronavirus-cases.html?action=click\&pgtype=Article\&state=default\&region=MAIN_CONTENT_1\&context=storylines_live_updates\#link-762df92}{As
  talks drag on, McConnell signals openness to jobless aid extension,
  and negotiators agree on a deadline.}
\item
  \href{https://www.nytimes3xbfgragh.onion/2020/08/04/world/coronavirus-cases.html?action=click\&pgtype=Article\&state=default\&region=MAIN_CONTENT_1\&context=storylines_live_updates\#link-1228a480}{Novavax
  sees encouraging results from two studies of its experimental
  vaccine.}
\item
  \href{https://www.nytimes3xbfgragh.onion/2020/08/04/world/coronavirus-cases.html?action=click\&pgtype=Article\&state=default\&region=MAIN_CONTENT_1\&context=storylines_live_updates\#link-794484ed}{Mississippians
  must now wear masks in public, governor says.}
\end{itemize}

\href{https://www.nytimes3xbfgragh.onion/2020/08/04/world/coronavirus-cases.html?action=click\&pgtype=Article\&state=default\&region=MAIN_CONTENT_1\&context=storylines_live_updates}{See
more updates}

More live coverage:
\href{https://www.nytimes3xbfgragh.onion/live/2020/08/04/business/stock-market-today-coronavirus?action=click\&pgtype=Article\&state=default\&region=MAIN_CONTENT_1\&context=storylines_live_updates}{Markets}

There has recently been a troublingly high rate of infection within
family clusters, said Eduardo Olivarez, chief administrative officer of
the Hidalgo County Health Department.

``When you have one or two people in the household who may be infected,
the probability of spreading it to others in the household is high,''
said Mr. Olivarez, who is known as Eddie.

Pachangas have been a way of life in the Valley for as long as I can
remember. The one I attended this past February held a special meaning
for my family and me.

My mother had recently overcome an aggressive form of breast cancer, and
we had turned the garage into an impromptu party hall for her 66th
birthday. Relatives sat close to each other on plastic chairs and
savored smoky carne asada. A mariachi band played ``Las Mañanitas'' as
my mother clapped to the sounds of the trumpets and the guitars.

``The day you were born, all the flowers were born, too,'' the mariachi
sang. My mother took turns dancing with almost every guest until her
body gave out. ``I haven't danced like that since I was a teenager,''
she said that night.

As I reported the story, I ran into other families who had gathered in
recent weeks, only to see the coronavirus strike those who were there,
one by one.

\includegraphics{https://static01.graylady3jvrrxbe.onion/images/2020/07/10/us/00virus-mcallen02/merlin_174426264_cf3ca7e1-f47b-46cd-96ba-a4eafed7b2c6-articleLarge.jpg?quality=75\&auto=webp\&disable=upscale}

Cris Flores told me about her grandfather, Ramon Contreras, who had
followed all the rules set out by the state for dealing with the virus.
When he turned 84 at the height of the pandemic in late April, the
family settled for an impersonal Zoom session.

Weeks later, after Texas eased its stay-at-home restrictions, Mr.
Contreras gathered with about 10 family members for a pachanga.

By mid-June, the family patriarch was the first to display severe
symptoms of the coronavirus and later died, Ms. Flores said. By the time
of Mr. Contreras's funeral, nearly 20 relatives had been infected.

``He loved to get his family together,'' Ms. Flores said. ``And that's
what took him.''

My family's encounter with the virus started in the last week of June,
when my 17-year-old nephew, who mistook his virus symptoms for strep,
joined my 66-year-old mother, 69-year-old father, two sisters and a
brother-in-law on a medical road trip to Houston, where my mother had
scheduled a mammogram.

\textbf{¿Hablas español?} \emph{To read more of our stories in
Spanish,}\href{https://www.nytimes3xbfgragh.onion/newsletters/el-times}{\emph{subscribe
to our newsletter El Times.}}

On their way back to the Valley, they visited relatives in Galveston.
After Father's Day, about a dozen relatives who had met one another
during the trip began describing debilitating headaches, body chills,
fever and trouble breathing, all classic Covid-19 symptoms.

The Contreras family was in a similar situation. They gathered for their
festive pachanga on June 1, dancing to mariachi music, sharing family
stories and savoring classic Mexican barbecue.

``Because they kept it small, they thought they were doing the right
thing,'' said Ms. Flores, who stayed home because she worried about the
virus.

It only took a few days for Mr. Contreras to develop a severe
respiratory illness. Two of his sons soon joined him in the hospital
with difficulty breathing. Soon uncles, aunts and cousins also fell ill.

Ms. Flores told me that when she heard her grandfather's brain was
bleeding, she rushed to the hospital and found him unconscious and
connected to several tubes.

``Your güera is here,'' she whispered, using the nickname he had given
her as a child, alluding to her light complexion.

She recalled praying next to his bed and then dialing several family
members who bid him emotional farewells before he took his last breath.

``I am forever grateful to have been given that opportunity not to let
my grandpa die alone,'' she said in tears.

Image

Patients are screened before being tested for the coronavirus at Nuestra
Clinica del Valle in San Juan, Texas.Credit...Ilana Panich-Linsman for
The New York Times

\hypertarget{a-pandemic-hits-close-to-home}{%
\subsection{A pandemic hits close to
home}\label{a-pandemic-hits-close-to-home}}

As I arrived in the Valley on June 27, I learned that most of my sick
family members were weathering Covid-19 in isolation. One of my aunts
had complained of trouble breathing and was taken by ambulance to a
hospital near Galveston. I did not worry much for myself --- I had come
down with the virus earlier in New York, and had antibodies that might
fend it off.

\href{https://www.nytimes3xbfgragh.onion/news-event/coronavirus?action=click\&pgtype=Article\&state=default\&region=MAIN_CONTENT_3\&context=storylines_faq}{}

\hypertarget{the-coronavirus-outbreak-}{%
\subsubsection{The Coronavirus Outbreak
›}\label{the-coronavirus-outbreak-}}

\hypertarget{frequently-asked-questions}{%
\paragraph{Frequently Asked
Questions}\label{frequently-asked-questions}}

Updated August 4, 2020

\begin{itemize}
\item ~
  \hypertarget{i-have-antibodies-am-i-now-immune}{%
  \paragraph{I have antibodies. Am I now
  immune?}\label{i-have-antibodies-am-i-now-immune}}

  \begin{itemize}
  \tightlist
  \item
    As of right
    now,\href{https://www.nytimes3xbfgragh.onion/2020/07/22/health/covid-antibodies-herd-immunity.html?action=click\&pgtype=Article\&state=default\&region=MAIN_CONTENT_3\&context=storylines_faq}{that
    seems likely, for at least several months.} There have been
    frightening accounts of people suffering what seems to be a second
    bout of Covid-19. But experts say these patients may have a
    drawn-out course of infection, with the virus taking a slow toll
    weeks to months after initial exposure. People infected with the
    coronavirus typically
    \href{https://www.nature.com/articles/s41586-020-2456-9}{produce}
    immune molecules called antibodies, which are
    \href{https://www.nytimes3xbfgragh.onion/2020/05/07/health/coronavirus-antibody-prevalence.html?action=click\&pgtype=Article\&state=default\&region=MAIN_CONTENT_3\&context=storylines_faq}{protective
    proteins made in response to an
    infection}\href{https://www.nytimes3xbfgragh.onion/2020/05/07/health/coronavirus-antibody-prevalence.html?action=click\&pgtype=Article\&state=default\&region=MAIN_CONTENT_3\&context=storylines_faq}{.
    These antibodies may} last in the body
    \href{https://www.nature.com/articles/s41591-020-0965-6}{only two to
    three months}, which may seem worrisome, but that's perfectly normal
    after an acute infection subsides, said Dr. Michael Mina, an
    immunologist at Harvard University. It may be possible to get the
    coronavirus again, but it's highly unlikely that it would be
    possible in a short window of time from initial infection or make
    people sicker the second time.
  \end{itemize}
\item ~
  \hypertarget{im-a-small-business-owner-can-i-get-relief}{%
  \paragraph{I'm a small-business owner. Can I get
  relief?}\label{im-a-small-business-owner-can-i-get-relief}}

  \begin{itemize}
  \tightlist
  \item
    The
    \href{https://www.nytimes3xbfgragh.onion/article/small-business-loans-stimulus-grants-freelancers-coronavirus.html?action=click\&pgtype=Article\&state=default\&region=MAIN_CONTENT_3\&context=storylines_faq}{stimulus
    bills enacted in March} offer help for the millions of American
    small businesses. Those eligible for aid are businesses and
    nonprofit organizations with fewer than 500 workers, including sole
    proprietorships, independent contractors and freelancers. Some
    larger companies in some industries are also eligible. The help
    being offered, which is being managed by the Small Business
    Administration, includes the Paycheck Protection Program and the
    Economic Injury Disaster Loan program. But lots of folks have
    \href{https://www.nytimes3xbfgragh.onion/interactive/2020/05/07/business/small-business-loans-coronavirus.html?action=click\&pgtype=Article\&state=default\&region=MAIN_CONTENT_3\&context=storylines_faq}{not
    yet seen payouts.} Even those who have received help are confused:
    The rules are draconian, and some are stuck sitting on
    \href{https://www.nytimes3xbfgragh.onion/2020/05/02/business/economy/loans-coronavirus-small-business.html?action=click\&pgtype=Article\&state=default\&region=MAIN_CONTENT_3\&context=storylines_faq}{money
    they don't know how to use.} Many small-business owners are getting
    less than they expected or
    \href{https://www.nytimes3xbfgragh.onion/2020/06/10/business/Small-business-loans-ppp.html?action=click\&pgtype=Article\&state=default\&region=MAIN_CONTENT_3\&context=storylines_faq}{not
    hearing anything at all.}
  \end{itemize}
\item ~
  \hypertarget{what-are-my-rights-if-i-am-worried-about-going-back-to-work}{%
  \paragraph{What are my rights if I am worried about going back to
  work?}\label{what-are-my-rights-if-i-am-worried-about-going-back-to-work}}

  \begin{itemize}
  \tightlist
  \item
    Employers have to provide
    \href{https://www.osha.gov/SLTC/covid-19/standards.html}{a safe
    workplace} with policies that protect everyone equally.
    \href{https://www.nytimes3xbfgragh.onion/article/coronavirus-money-unemployment.html?action=click\&pgtype=Article\&state=default\&region=MAIN_CONTENT_3\&context=storylines_faq}{And
    if one of your co-workers tests positive for the coronavirus, the
    C.D.C.} has said that
    \href{https://www.cdc.gov/coronavirus/2019-ncov/community/guidance-business-response.html}{employers
    should tell their employees} -\/- without giving you the sick
    employee's name -\/- that they may have been exposed to the virus.
  \end{itemize}
\item ~
  \hypertarget{should-i-refinance-my-mortgage}{%
  \paragraph{Should I refinance my
  mortgage?}\label{should-i-refinance-my-mortgage}}

  \begin{itemize}
  \tightlist
  \item
    \href{https://www.nytimes3xbfgragh.onion/article/coronavirus-money-unemployment.html?action=click\&pgtype=Article\&state=default\&region=MAIN_CONTENT_3\&context=storylines_faq}{It
    could be a good idea,} because mortgage rates have
    \href{https://www.nytimes3xbfgragh.onion/2020/07/16/business/mortgage-rates-below-3-percent.html?action=click\&pgtype=Article\&state=default\&region=MAIN_CONTENT_3\&context=storylines_faq}{never
    been lower.} Refinancing requests have pushed mortgage applications
    to some of the highest levels since 2008, so be prepared to get in
    line. But defaults are also up, so if you're thinking about buying a
    home, be aware that some lenders have tightened their standards.
  \end{itemize}
\item ~
  \hypertarget{what-is-school-going-to-look-like-in-september}{%
  \paragraph{What is school going to look like in
  September?}\label{what-is-school-going-to-look-like-in-september}}

  \begin{itemize}
  \tightlist
  \item
    It is unlikely that many schools will return to a normal schedule
    this fall, requiring the grind of
    \href{https://www.nytimes3xbfgragh.onion/2020/06/05/us/coronavirus-education-lost-learning.html?action=click\&pgtype=Article\&state=default\&region=MAIN_CONTENT_3\&context=storylines_faq}{online
    learning},
    \href{https://www.nytimes3xbfgragh.onion/2020/05/29/us/coronavirus-child-care-centers.html?action=click\&pgtype=Article\&state=default\&region=MAIN_CONTENT_3\&context=storylines_faq}{makeshift
    child care} and
    \href{https://www.nytimes3xbfgragh.onion/2020/06/03/business/economy/coronavirus-working-women.html?action=click\&pgtype=Article\&state=default\&region=MAIN_CONTENT_3\&context=storylines_faq}{stunted
    workdays} to continue. California's two largest public school
    districts --- Los Angeles and San Diego --- said on July 13, that
    \href{https://www.nytimes3xbfgragh.onion/2020/07/13/us/lausd-san-diego-school-reopening.html?action=click\&pgtype=Article\&state=default\&region=MAIN_CONTENT_3\&context=storylines_faq}{instruction
    will be remote-only in the fall}, citing concerns that surging
    coronavirus infections in their areas pose too dire a risk for
    students and teachers. Together, the two districts enroll some
    825,000 students. They are the largest in the country so far to
    abandon plans for even a partial physical return to classrooms when
    they reopen in August. For other districts, the solution won't be an
    all-or-nothing approach.
    \href{https://bioethics.jhu.edu/research-and-outreach/projects/eschool-initiative/school-policy-tracker/}{Many
    systems}, including the nation's largest, New York City, are
    devising
    \href{https://www.nytimes3xbfgragh.onion/2020/06/26/us/coronavirus-schools-reopen-fall.html?action=click\&pgtype=Article\&state=default\&region=MAIN_CONTENT_3\&context=storylines_faq}{hybrid
    plans} that involve spending some days in classrooms and other days
    online. There's no national policy on this yet, so check with your
    municipal school system regularly to see what is happening in your
    community.
  \end{itemize}
\end{itemize}

On July 1, I hurried to my parents' home and found my mother --- I
usually call her ``Ama'' --- in the living room, gasping for air.

I knew she had to go, and quickly, to one of the hospitals, but where?
The few hospitals in the Valley were filling up quickly. By the time my
sister and I got her into the emergency room at Doctors Hospital at
Renaissance in McAllen, her blood oxygen level had reached a paltry 80
percent, and a nurse quickly connected her to an oxygen supply. X-ray
images showed her lungs nearly covered in what resembled pale spider
webs.

``I'm surprised your mother was able to breathe on her own, given the
poor state of her lungs,'' another nurse said.

I nodded my head quietly. This was the woman who had prided herself on
working as a cleaning lady well into her eighth month of pregnancy.
Whenever one of my three sisters complained about the slightest ache,
she was quick to remind them that she had juggled mops, brooms and
cleaning carts, all while carrying a belly the size of a watermelon.

On this night, she said little. Instead she focused on slowly inhaling
and exhaling the dwindling air moving through her lungs. Less than an
hour after a nurse administered a coronavirus test, he announced that
she was positive.

``No surprise there,'' Ama said.

Two attendants arrived with a stretcher to transport her into a Covid
wing at another location, where she would not be allowed to have
visitors, I knew. My throat tightened. The beeping sound of the monitors
echoed around the small room.

The two attendants asked her to place her arms on her stomach and they
wrapped her in a white blanket.

``We're going to make a ** señora ** burrito,'' one of them said, and we
laughed.

Suddenly I panicked. Our family, while close, has never been overly
emotional. Growing up, Ama used to remind us that she might not say ``I
love you'' often, but that she and Apa worked hard to provide us with
food and a roof over our heads. And that's what matters, she would say.
Actions, not words.

I fought the urge to reach for her and say something profound. Should I
say I love you? Was it time for a heartfelt farewell? What if this was
the last time I would see her alive?

I decided that if I said something poignant, she might interpret it as a
final goodbye and give up. Instead, I decided to act as casual as
possible.

``Echele ganas,'' I murmured as the attendants began pushing her away.
``Do your best.'' I waved goodbye.

Ama nodded yes and disappeared down the hallway.

Image

Edgar Sandoval, second from right, with his father, Filiberto; mother,
Arcelia; and sister Mirna.

\hypertarget{the-virus-rolls-on}{%
\subsection{The virus rolls on}\label{the-virus-rolls-on}}

After Mr. Contreras's burial, Ms. Flores developed a dry cough. She
later tested positive for the virus.

Two weeks after she had started feeling sick, all four of her children
were also showing signs.

Looking back, she wished her family had heeded the warnings. Some days
she wonders if she should have pushed back more forcefully. First there
was the pachanga, then the funeral. They had known such gatherings could
be risky, she said, but somehow no one really believed there would be
serious consequences.

``There is a whole mentality of *`*no pasa nada,' you know?'' she said.
``Nothing will happen.''

As the second week of July rolled around, most of my dozen or so family
members who had fallen ill began peeling out of bed. My mother and aunt
remained hospitalized but were showing signs of recovery. Everyone told
stories of excruciating body aches, debilitating chills and burning
fevers.

Apa limped out of his bedroom, the lights from a window stabbing his
eyes. He said he felt as if he had wrestled a monster made out of
burning lava all night. My oldest sister said that every morning after
waking up, she felt as if an invisible hammer was smashing her head.

Some days, Ama managed to text us a selfie, outfitted in her oxygen
mask. Other days she told us that she had slept poorly and that her
breathing became labored when she tried to walk.

All five of us children stared at our phones as if our lives depended on
it, waiting for news.

``They are giving me plasma,'' she would write, and then go silent.

``I want to come home soon,'' she would text days later.

Nearly a week after I had dropped her at the emergency room, her mood
and breathing had significantly improved. She was able to sit upright
and hold a phone conversation for five minutes. We began talking about
preparations for her eventual return home.

I wanted to say I loved her. But again I choked. \emph{Don't make it
sound like you're} \emph{saying goodbye}, I told myself.

After we hung up, I sent her a GIF of a white bunny that shoots hearts
every time it hugs.

``I love you,'' the message flashed, over and over.

Advertisement

\protect\hyperlink{after-bottom}{Continue reading the main story}

\hypertarget{site-index}{%
\subsection{Site Index}\label{site-index}}

\hypertarget{site-information-navigation}{%
\subsection{Site Information
Navigation}\label{site-information-navigation}}

\begin{itemize}
\tightlist
\item
  \href{https://help.nytimes3xbfgragh.onion/hc/en-us/articles/115014792127-Copyright-notice}{©~2020~The
  New York Times Company}
\end{itemize}

\begin{itemize}
\tightlist
\item
  \href{https://www.nytco.com/}{NYTCo}
\item
  \href{https://help.nytimes3xbfgragh.onion/hc/en-us/articles/115015385887-Contact-Us}{Contact
  Us}
\item
  \href{https://www.nytco.com/careers/}{Work with us}
\item
  \href{https://nytmediakit.com/}{Advertise}
\item
  \href{http://www.tbrandstudio.com/}{T Brand Studio}
\item
  \href{https://www.nytimes3xbfgragh.onion/privacy/cookie-policy\#how-do-i-manage-trackers}{Your
  Ad Choices}
\item
  \href{https://www.nytimes3xbfgragh.onion/privacy}{Privacy}
\item
  \href{https://help.nytimes3xbfgragh.onion/hc/en-us/articles/115014893428-Terms-of-service}{Terms
  of Service}
\item
  \href{https://help.nytimes3xbfgragh.onion/hc/en-us/articles/115014893968-Terms-of-sale}{Terms
  of Sale}
\item
  \href{https://spiderbites.nytimes3xbfgragh.onion}{Site Map}
\item
  \href{https://help.nytimes3xbfgragh.onion/hc/en-us}{Help}
\item
  \href{https://www.nytimes3xbfgragh.onion/subscription?campaignId=37WXW}{Subscriptions}
\end{itemize}
