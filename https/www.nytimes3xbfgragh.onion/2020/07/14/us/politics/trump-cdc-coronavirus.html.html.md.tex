Sections

SEARCH

\protect\hyperlink{site-content}{Skip to
content}\protect\hyperlink{site-index}{Skip to site index}

\href{https://www.nytimes3xbfgragh.onion/section/politics}{Politics}

\href{https://myaccount.nytimes3xbfgragh.onion/auth/login?response_type=cookie\&client_id=vi}{}

\href{https://www.nytimes3xbfgragh.onion/section/todayspaper}{Today's
Paper}

\href{/section/politics}{Politics}\textbar{}Trump Administration Strips
C.D.C. of Control of Coronavirus Data

\href{https://nyti.ms/309Xtsi}{https://nyti.ms/309Xtsi}

\begin{itemize}
\item
\item
\item
\item
\item
\end{itemize}

\href{https://www.nytimes3xbfgragh.onion/news-event/coronavirus?action=click\&pgtype=Article\&state=default\&region=TOP_BANNER\&context=storylines_menu}{The
Coronavirus Outbreak}

\begin{itemize}
\tightlist
\item
  live\href{https://www.nytimes3xbfgragh.onion/2020/08/04/world/coronavirus-cases.html?action=click\&pgtype=Article\&state=default\&region=TOP_BANNER\&context=storylines_menu}{Latest
  Updates}
\item
  \href{https://www.nytimes3xbfgragh.onion/interactive/2020/us/coronavirus-us-cases.html?action=click\&pgtype=Article\&state=default\&region=TOP_BANNER\&context=storylines_menu}{Maps
  and Cases}
\item
  \href{https://www.nytimes3xbfgragh.onion/interactive/2020/science/coronavirus-vaccine-tracker.html?action=click\&pgtype=Article\&state=default\&region=TOP_BANNER\&context=storylines_menu}{Vaccine
  Tracker}
\item
  \href{https://www.nytimes3xbfgragh.onion/2020/08/02/us/covid-college-reopening.html?action=click\&pgtype=Article\&state=default\&region=TOP_BANNER\&context=storylines_menu}{College
  Reopening}
\item
  \href{https://www.nytimes3xbfgragh.onion/live/2020/08/04/business/stock-market-today-coronavirus?action=click\&pgtype=Article\&state=default\&region=TOP_BANNER\&context=storylines_menu}{Economy}
\end{itemize}

Advertisement

\protect\hyperlink{after-top}{Continue reading the main story}

Supported by

\protect\hyperlink{after-sponsor}{Continue reading the main story}

\hypertarget{trump-administration-strips-cdc-of-control-of-coronavirus-data}{%
\section{Trump Administration Strips C.D.C. of Control of Coronavirus
Data}\label{trump-administration-strips-cdc-of-control-of-coronavirus-data}}

Hospitals have been ordered to bypass the Centers for Disease Control
and Prevention and send all patient information to a central database in
Washington, raising questions about transparency.

\includegraphics{https://static01.graylady3jvrrxbe.onion/images/2020/07/14/us/politics/14dc-virus-data/merlin_174009087_d81a3118-690c-4a62-9d54-a7d9cee2a6da-articleLarge.jpg?quality=75\&auto=webp\&disable=upscale}

\href{https://www.nytimes3xbfgragh.onion/by/sheryl-gay-stolberg}{\includegraphics{https://static01.graylady3jvrrxbe.onion/images/2018/11/26/multimedia/author-sheryl-gay-stolberg/author-sheryl-gay-stolberg-thumbLarge.png}}

By
\href{https://www.nytimes3xbfgragh.onion/by/sheryl-gay-stolberg}{Sheryl
Gay Stolberg}

\begin{itemize}
\item
  July 14, 2020
\item
  \begin{itemize}
  \item
  \item
  \item
  \item
  \item
  \end{itemize}
\end{itemize}

WASHINGTON --- The Trump administration has ordered hospitals to bypass
the Centers for Disease Control and Prevention and send all Covid-19
patient information to a central database in Washington beginning on
Wednesday. The move has alarmed health experts who fear the data will be
politicized or withheld from the public.

The new instructions were posted recently in a
\href{https://www.hhs.gov/sites/default/files/covid-19-faqs-hospitals-hospital-laboratory-acute-care-facility-data-reporting.pdf}{little-noticed
document} on the Department of Health and Human Services website. From
now on, the department --- not the C.D.C. --- will collect daily reports
about the patients that each hospital is treating, the number of
available beds and ventilators, and other information vital to tracking
the pandemic.

Officials say the change will streamline data gathering and assist the
\href{https://www.nytimes3xbfgragh.onion/2020/07/18/us/politics/trump-coronavirus-response-failure-leadership.html}{White
House coronavirus task force} in allocating scarce supplies like
personal protective gear and
\href{https://www.nytimes3xbfgragh.onion/2020/06/29/health/coronavirus-remdesivir-gilead.html}{remdesivir},
the first drug shown to be effective against the virus. But the Health
and Human Services database that will receive new information is not
open to the public, which could affect the work of scores of
researchers, modelers and health officials who rely on
\href{https://www.cdc.gov/nhsn/covid19/report-patient-impact.html}{C.D.C.
data}to make projections and crucial decisions.

``Historically, C.D.C. has been the place where public health data has
been sent, and this raises questions about not just access for
researchers but access for reporters, access for the public to try to
better understand what is happening with the outbreak," said Jen Kates,
the director of global health and H.I.V. policy with the nonpartisan
Kaiser Family Foundation.

``How will the data be protected?'' she asked. ``Will there be
transparency, will there be access, and what is the role of the C.D.C.
in understanding the data?''

News of the change came as a shock at the
\href{https://www.nytimes3xbfgragh.onion/2020/07/24/health/cdc-schools-coronavirus.html}{C.D.C}.,
according to two officials who spoke on the condition of anonymity
because they were not authorized to discuss the matter. Michael R.
Caputo, a Health and Human Services spokesman, called the C.D.C.'s
system inadequate and said the two systems would be linked. The C.D.C.
would continue to make data public, he said.

``Today, the C.D.C. still has at least a week lag in reporting hospital
data,'' Mr. Caputo said. ``America requires it in real time. The new,
faster and complete data system is what our nation needs to defeat the
coronavirus, and the C.D.C., an operating division of H.H.S., will
certainly participate in this streamlined all-of-government response.
They will simply no longer control it.''

\hypertarget{latest-updates-global-coronavirus-outbreak}{%
\section{\texorpdfstring{\href{https://www.nytimes3xbfgragh.onion/2020/08/04/world/coronavirus-cases.html?action=click\&pgtype=Article\&state=default\&region=MAIN_CONTENT_1\&context=storylines_live_updates}{Latest
Updates: Global Coronavirus
Outbreak}}{Latest Updates: Global Coronavirus Outbreak}}\label{latest-updates-global-coronavirus-outbreak}}

Updated 2020-08-04T21:52:13.695Z

\begin{itemize}
\tightlist
\item
  \href{https://www.nytimes3xbfgragh.onion/2020/08/04/world/coronavirus-cases.html?action=click\&pgtype=Article\&state=default\&region=MAIN_CONTENT_1\&context=storylines_live_updates\#link-2daa96b5}{As
  talks drag on, McConnell signals openness to jobless aid extension
  that Republicans have opposed.}
\item
  \href{https://www.nytimes3xbfgragh.onion/2020/08/04/world/coronavirus-cases.html?action=click\&pgtype=Article\&state=default\&region=MAIN_CONTENT_1\&context=storylines_live_updates\#link-1228a480}{Novavax
  sees encouraging results from two studies of its experimental
  vaccine.}
\item
  \href{https://www.nytimes3xbfgragh.onion/2020/08/04/world/coronavirus-cases.html?action=click\&pgtype=Article\&state=default\&region=MAIN_CONTENT_1\&context=storylines_live_updates\#link-4825b93}{Public
  and private schools in Maryland and elsewhere are divided over
  in-person instruction.}
\end{itemize}

\href{https://www.nytimes3xbfgragh.onion/2020/08/04/world/coronavirus-cases.html?action=click\&pgtype=Article\&state=default\&region=MAIN_CONTENT_1\&context=storylines_live_updates}{See
more updates}

More live coverage:
\href{https://www.nytimes3xbfgragh.onion/live/2020/08/04/business/stock-market-today-coronavirus?action=click\&pgtype=Article\&state=default\&region=MAIN_CONTENT_1\&context=storylines_live_updates}{Markets}

But the instructions to hospitals in the department guidance are
explicit and underscored: ``As of July 15, 2020, hospitals should no
longer report the Covid-19 information in this document to the National
Healthcare Safety Network site,'' the C.D.C.'s system for gathering data
from more than 25,000 medical centers around the country.

Public health experts have long expressed concerns that the Trump
administration is politicizing science and undermining its health
experts, in particular the C.D.C.; four of the agency's former
directors, spanning both Republican and Democratic administrations, said
as much in an
\href{https://www.washingtonpost.com/outlook/2020/07/14/cdc-directors-trump-politics/}{opinion
piece} published Tuesday in The Washington Post. The data collection
shift reinforced those fears.

``Centralizing control of all data under the umbrella of an inherently
political apparatus is dangerous and breeds distrust,'' said Dr. Nicole
Lurie, who served as assistant secretary for preparedness and response
under former President Barack Obama. ``It appears to cut off the ability
of agencies like C.D.C. to do its basic job.''

The shift grew out of a tense conference call several weeks ago between
hospital executives and Dr. Deborah L. Birx, the White House coronavirus
response coordinator. After Dr. Birx said hospitals were not adequately
reporting their data, she convened a working group of government and
hospital officials who devised the new plan, according to Dr. Janis
Orlowski, the chief health care officer of the Association of American
Medical Colleges, who participated in the group's meetings.

While she said she understood Dr. Lurie's concern, Dr. Orlowski said the
administration had pledged in ``a verbal discussion'' to make the data
public --- or at least give hospitals access to it.

``We are comfortable with that as long as they continue to work with us,
as long as they continue to make the information public, and as long as
we're able to continue to advise them and look at the data,'' she said,
calling the switch ``a sincere effort to streamline and improve data
collection.''

The change exposes the vast gaps in the government's ability to collect
and manage health data --- an antiquated system at best, experts say.
The C.D.C. has been collecting coronavirus data through its
\href{https://www.cdc.gov/nhsn/index.html}{National Healthcare Safety
Network}, which was expanded at the outset of the pandemic to track
hospital capacity and patient information specific to Covid-19.

In its new guidance, Health and Human Services said that going forward,
hospitals should report detailed information on a daily basis directly
to the new centralized system, which is managed by TeleTracking, a
health data firm with headquarters in Pittsburgh. However, if hospitals
were already reporting such information to their states, they could
continue to do so if they received a written release saying the state
would handle reporting.

Senator Patty Murray of Washington, the top Democrat on the Senate
health committee, has
\href{https://www.help.senate.gov/ranking/newsroom/press/murray-demands-answers-regarding-non-competitive-multimillion-dollar-contract-for-duplicative-health-data-system-}{raised
questions} about the TeleTracking contract, calling it a
``noncompetitive, multimillion-dollar contract'' for a ``duplicative
health data system.''

Both the C.D.C. network and the TeleTracking system set up by Health and
Human Services rely on so-called push data, meaning hospital employees
must manually enter data, rather than the government tapping into an
electronic system to obtain the information.

``The whole thing needs to be scrapped and started anew,'' said Dr. Dan
Hanfling, an expert in medical and disaster preparedness and a vice
president at In-Q-Tel, a nonprofit strategic investment firm focused on
national security. ``It is laughable that this administration can't find
the wherewithal to bring 21st-century technologies in data management to
the fight.''

Dr. Hanfling and others agree that information does need to be
centralized, but they disagree on how that should happen. Dr. Hanfling
called for a new ``national data coordination center'' that would be
used for ``forecasting, identifying, detecting, tracking and reporting
on emerging diseases.''

\href{https://www.nytimes3xbfgragh.onion/news-event/coronavirus?action=click\&pgtype=Article\&state=default\&region=MAIN_CONTENT_3\&context=storylines_faq}{}

\hypertarget{the-coronavirus-outbreak-}{%
\subsubsection{The Coronavirus Outbreak
›}\label{the-coronavirus-outbreak-}}

\hypertarget{frequently-asked-questions}{%
\paragraph{Frequently Asked
Questions}\label{frequently-asked-questions}}

Updated August 4, 2020

\begin{itemize}
\item ~
  \hypertarget{i-have-antibodies-am-i-now-immune}{%
  \paragraph{I have antibodies. Am I now
  immune?}\label{i-have-antibodies-am-i-now-immune}}

  \begin{itemize}
  \tightlist
  \item
    As of right
    now,\href{https://www.nytimes3xbfgragh.onion/2020/07/22/health/covid-antibodies-herd-immunity.html?action=click\&pgtype=Article\&state=default\&region=MAIN_CONTENT_3\&context=storylines_faq}{that
    seems likely, for at least several months.} There have been
    frightening accounts of people suffering what seems to be a second
    bout of Covid-19. But experts say these patients may have a
    drawn-out course of infection, with the virus taking a slow toll
    weeks to months after initial exposure. People infected with the
    coronavirus typically
    \href{https://www.nature.com/articles/s41586-020-2456-9}{produce}
    immune molecules called antibodies, which are
    \href{https://www.nytimes3xbfgragh.onion/2020/05/07/health/coronavirus-antibody-prevalence.html?action=click\&pgtype=Article\&state=default\&region=MAIN_CONTENT_3\&context=storylines_faq}{protective
    proteins made in response to an
    infection}\href{https://www.nytimes3xbfgragh.onion/2020/05/07/health/coronavirus-antibody-prevalence.html?action=click\&pgtype=Article\&state=default\&region=MAIN_CONTENT_3\&context=storylines_faq}{.
    These antibodies may} last in the body
    \href{https://www.nature.com/articles/s41591-020-0965-6}{only two to
    three months}, which may seem worrisome, but that's perfectly normal
    after an acute infection subsides, said Dr. Michael Mina, an
    immunologist at Harvard University. It may be possible to get the
    coronavirus again, but it's highly unlikely that it would be
    possible in a short window of time from initial infection or make
    people sicker the second time.
  \end{itemize}
\item ~
  \hypertarget{im-a-small-business-owner-can-i-get-relief}{%
  \paragraph{I'm a small-business owner. Can I get
  relief?}\label{im-a-small-business-owner-can-i-get-relief}}

  \begin{itemize}
  \tightlist
  \item
    The
    \href{https://www.nytimes3xbfgragh.onion/article/small-business-loans-stimulus-grants-freelancers-coronavirus.html?action=click\&pgtype=Article\&state=default\&region=MAIN_CONTENT_3\&context=storylines_faq}{stimulus
    bills enacted in March} offer help for the millions of American
    small businesses. Those eligible for aid are businesses and
    nonprofit organizations with fewer than 500 workers, including sole
    proprietorships, independent contractors and freelancers. Some
    larger companies in some industries are also eligible. The help
    being offered, which is being managed by the Small Business
    Administration, includes the Paycheck Protection Program and the
    Economic Injury Disaster Loan program. But lots of folks have
    \href{https://www.nytimes3xbfgragh.onion/interactive/2020/05/07/business/small-business-loans-coronavirus.html?action=click\&pgtype=Article\&state=default\&region=MAIN_CONTENT_3\&context=storylines_faq}{not
    yet seen payouts.} Even those who have received help are confused:
    The rules are draconian, and some are stuck sitting on
    \href{https://www.nytimes3xbfgragh.onion/2020/05/02/business/economy/loans-coronavirus-small-business.html?action=click\&pgtype=Article\&state=default\&region=MAIN_CONTENT_3\&context=storylines_faq}{money
    they don't know how to use.} Many small-business owners are getting
    less than they expected or
    \href{https://www.nytimes3xbfgragh.onion/2020/06/10/business/Small-business-loans-ppp.html?action=click\&pgtype=Article\&state=default\&region=MAIN_CONTENT_3\&context=storylines_faq}{not
    hearing anything at all.}
  \end{itemize}
\item ~
  \hypertarget{what-are-my-rights-if-i-am-worried-about-going-back-to-work}{%
  \paragraph{What are my rights if I am worried about going back to
  work?}\label{what-are-my-rights-if-i-am-worried-about-going-back-to-work}}

  \begin{itemize}
  \tightlist
  \item
    Employers have to provide
    \href{https://www.osha.gov/SLTC/covid-19/standards.html}{a safe
    workplace} with policies that protect everyone equally.
    \href{https://www.nytimes3xbfgragh.onion/article/coronavirus-money-unemployment.html?action=click\&pgtype=Article\&state=default\&region=MAIN_CONTENT_3\&context=storylines_faq}{And
    if one of your co-workers tests positive for the coronavirus, the
    C.D.C.} has said that
    \href{https://www.cdc.gov/coronavirus/2019-ncov/community/guidance-business-response.html}{employers
    should tell their employees} -\/- without giving you the sick
    employee's name -\/- that they may have been exposed to the virus.
  \end{itemize}
\item ~
  \hypertarget{should-i-refinance-my-mortgage}{%
  \paragraph{Should I refinance my
  mortgage?}\label{should-i-refinance-my-mortgage}}

  \begin{itemize}
  \tightlist
  \item
    \href{https://www.nytimes3xbfgragh.onion/article/coronavirus-money-unemployment.html?action=click\&pgtype=Article\&state=default\&region=MAIN_CONTENT_3\&context=storylines_faq}{It
    could be a good idea,} because mortgage rates have
    \href{https://www.nytimes3xbfgragh.onion/2020/07/16/business/mortgage-rates-below-3-percent.html?action=click\&pgtype=Article\&state=default\&region=MAIN_CONTENT_3\&context=storylines_faq}{never
    been lower.} Refinancing requests have pushed mortgage applications
    to some of the highest levels since 2008, so be prepared to get in
    line. But defaults are also up, so if you're thinking about buying a
    home, be aware that some lenders have tightened their standards.
  \end{itemize}
\item ~
  \hypertarget{what-is-school-going-to-look-like-in-september}{%
  \paragraph{What is school going to look like in
  September?}\label{what-is-school-going-to-look-like-in-september}}

  \begin{itemize}
  \tightlist
  \item
    It is unlikely that many schools will return to a normal schedule
    this fall, requiring the grind of
    \href{https://www.nytimes3xbfgragh.onion/2020/06/05/us/coronavirus-education-lost-learning.html?action=click\&pgtype=Article\&state=default\&region=MAIN_CONTENT_3\&context=storylines_faq}{online
    learning},
    \href{https://www.nytimes3xbfgragh.onion/2020/05/29/us/coronavirus-child-care-centers.html?action=click\&pgtype=Article\&state=default\&region=MAIN_CONTENT_3\&context=storylines_faq}{makeshift
    child care} and
    \href{https://www.nytimes3xbfgragh.onion/2020/06/03/business/economy/coronavirus-working-women.html?action=click\&pgtype=Article\&state=default\&region=MAIN_CONTENT_3\&context=storylines_faq}{stunted
    workdays} to continue. California's two largest public school
    districts --- Los Angeles and San Diego --- said on July 13, that
    \href{https://www.nytimes3xbfgragh.onion/2020/07/13/us/lausd-san-diego-school-reopening.html?action=click\&pgtype=Article\&state=default\&region=MAIN_CONTENT_3\&context=storylines_faq}{instruction
    will be remote-only in the fall}, citing concerns that surging
    coronavirus infections in their areas pose too dire a risk for
    students and teachers. Together, the two districts enroll some
    825,000 students. They are the largest in the country so far to
    abandon plans for even a partial physical return to classrooms when
    they reopen in August. For other districts, the solution won't be an
    all-or-nothing approach.
    \href{https://bioethics.jhu.edu/research-and-outreach/projects/eschool-initiative/school-policy-tracker/}{Many
    systems}, including the nation's largest, New York City, are
    devising
    \href{https://www.nytimes3xbfgragh.onion/2020/06/26/us/coronavirus-schools-reopen-fall.html?action=click\&pgtype=Article\&state=default\&region=MAIN_CONTENT_3\&context=storylines_faq}{hybrid
    plans} that involve spending some days in classrooms and other days
    online. There's no national policy on this yet, so check with your
    municipal school system regularly to see what is happening in your
    community.
  \end{itemize}
\end{itemize}

Representative Donna E. Shalala of Florida, who served as health
secretary under former President Bill Clinton, said the C.D.C. was the
proper agency to gather health data. If there were flaws in the C.D.C.'s
systems, she said, they should be fixed.

``Only the C.D.C. has the expertise to collect data,'' Ms. Shalala said.
``I think any move to take responsibility away from the people who have
the expertise is politicizing.''

Hospitals say the previous reporting requirements were cumbersome,
partly because they frequently changed.

``It has been an administrative hassle and confusing to constantly be
shifting gears on reporting while hospitals are on the front lines
during a pandemic,'' Carrie Williams, a spokeswoman for the Texas
Hospital Association, wrote in an email.

At Rush University Medical Center in Chicago as the pandemic raged, the
hospital had four full-time employees reporting coronavirus data to four
different agencies, said Dr. Bala N. Hota, the hospital's chief
analytics officer. Rush collected more than 100 different measures, some
of which determined how much money it would receive under different
federal programs.

But while Dr. Hota said he supported streamlining the process and the
involvement of state and local agencies in reporting, he was also
concerned that months into the pandemic, the United States still did not
have an established system of collecting the kind of information it
needed to seamlessly move patients from a full hospital to one with
available beds.

``The C.D.C. is the right agency to be at the forefront of collecting
the data,'' Dr. Hota said.

The C.D.C. has been criticized for its data collection, however. In May,
the agency
\href{https://www.nytimes3xbfgragh.onion/2020/05/22/us/politics/coronavirus-tests-cdc.html}{acknowledged}
that in tracking the spread of the virus, it had been combining tests
that detect active infection with those that detect recovery from
Covid-19. That system muddied the picture of the pandemic but raised the
percentage of Americans tested as President Trump was boasting about the
number of tests the United States was conducting.

Similar complaints about coronavirus data have bubbled up around the
country.

In Florida, a former data manager for the Health Department
\href{https://www.tampabay.com/news/health/2020/05/22/ousted-manager-was-told-to-manipulate-covid-19-data-before-states-re-opening-she-says/}{accused
one of her superiors} of directing her to ``manipulate'' data used in
the state's plan to lift stay-at-home orders this spring. Ms. Shalala
said the mayor of Miami-Dade County ``was so concerned about the state
data that he has the hospitals reporting their data directly to him as
well.''

And Arizona ended its partnership with a university modeling team whose
projections showed a rising caseload, prompting pushback from Will
Humble, the executive director of the Arizona Public Health Association
and a former director of the state's Health Services Department.

``Trust and accountability and transparency --- all three go together,''
Mr. Humble said. Of the federal government's new system, he said:
``They'd better keep it transparent, or else people are going to think
that it was an ulterior motive.''

Reed Abelson contributed reporting from New York.

Advertisement

\protect\hyperlink{after-bottom}{Continue reading the main story}

\hypertarget{site-index}{%
\subsection{Site Index}\label{site-index}}

\hypertarget{site-information-navigation}{%
\subsection{Site Information
Navigation}\label{site-information-navigation}}

\begin{itemize}
\tightlist
\item
  \href{https://help.nytimes3xbfgragh.onion/hc/en-us/articles/115014792127-Copyright-notice}{©~2020~The
  New York Times Company}
\end{itemize}

\begin{itemize}
\tightlist
\item
  \href{https://www.nytco.com/}{NYTCo}
\item
  \href{https://help.nytimes3xbfgragh.onion/hc/en-us/articles/115015385887-Contact-Us}{Contact
  Us}
\item
  \href{https://www.nytco.com/careers/}{Work with us}
\item
  \href{https://nytmediakit.com/}{Advertise}
\item
  \href{http://www.tbrandstudio.com/}{T Brand Studio}
\item
  \href{https://www.nytimes3xbfgragh.onion/privacy/cookie-policy\#how-do-i-manage-trackers}{Your
  Ad Choices}
\item
  \href{https://www.nytimes3xbfgragh.onion/privacy}{Privacy}
\item
  \href{https://help.nytimes3xbfgragh.onion/hc/en-us/articles/115014893428-Terms-of-service}{Terms
  of Service}
\item
  \href{https://help.nytimes3xbfgragh.onion/hc/en-us/articles/115014893968-Terms-of-sale}{Terms
  of Sale}
\item
  \href{https://spiderbites.nytimes3xbfgragh.onion}{Site Map}
\item
  \href{https://help.nytimes3xbfgragh.onion/hc/en-us}{Help}
\item
  \href{https://www.nytimes3xbfgragh.onion/subscription?campaignId=37WXW}{Subscriptions}
\end{itemize}
