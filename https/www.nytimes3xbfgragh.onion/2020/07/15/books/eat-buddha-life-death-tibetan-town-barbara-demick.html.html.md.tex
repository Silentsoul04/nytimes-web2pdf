Sections

SEARCH

\protect\hyperlink{site-content}{Skip to
content}\protect\hyperlink{site-index}{Skip to site index}

\href{https://www.nytimes3xbfgragh.onion/section/books}{Books}

\href{https://myaccount.nytimes3xbfgragh.onion/auth/login?response_type=cookie\&client_id=vi}{}

\href{https://www.nytimes3xbfgragh.onion/section/todayspaper}{Today's
Paper}

\href{/section/books}{Books}\textbar{}`Eat the Buddha' Reports From the
`World Capital of Self-Immolations'

\href{https://nyti.ms/2CzMccG}{https://nyti.ms/2CzMccG}

\begin{itemize}
\item
\item
\item
\item
\item
\end{itemize}

Advertisement

\protect\hyperlink{after-top}{Continue reading the main story}

Supported by

\protect\hyperlink{after-sponsor}{Continue reading the main story}

\href{/column/books-of-the-times}{Books of The Times}

\hypertarget{eat-the-buddha-reports-from-the-world-capital-of-self-immolations}{%
\section{`Eat the Buddha' Reports From the `World Capital of
Self-Immolations'}\label{eat-the-buddha-reports-from-the-world-capital-of-self-immolations}}

By \href{https://www.nytimes3xbfgragh.onion/by/parul-sehgal}{Parul
Sehgal}

\begin{itemize}
\item
  Published July 15, 2020Updated July 21, 2020
\item
  \begin{itemize}
  \item
  \item
  \item
  \item
  \item
  \end{itemize}
\end{itemize}

\includegraphics{https://static01.graylady3jvrrxbe.onion/images/2020/07/16/books/15BOOKDEMICK1/15BOOKDEMICK1-articleLarge.png?quality=75\&auto=webp\&disable=upscale}

Buy Book ▾

\begin{itemize}
\tightlist
\item
  \href{https://www.amazon.com/gp/search?index=books\&tag=NYTBSREV-20\&field-keywords=Eat+the+Buddha+Barbara+Demick}{Amazon}
\item
  \href{https://du-gae-books-dot-nyt-du-prd.appspot.com/buy?title=Eat+the+Buddha\&author=Barbara+Demick}{Apple
  Books}
\item
  \href{https://www.anrdoezrs.net/click-7990613-11819508?url=https\%3A\%2F\%2Fwww.barnesandnoble.com\%2Fw\%2F\%3Fean\%3D9780812998757}{Barnes
  and Noble}
\item
  \href{https://www.anrdoezrs.net/click-7990613-35140?url=https\%3A\%2F\%2Fwww.booksamillion.com\%2Fp\%2FEat\%2Bthe\%2BBuddha\%2FBarbara\%2BDemick\%2F9780812998757}{Books-A-Million}
\item
  \href{https://bookshop.org/a/3546/9780812998757}{Bookshop}
\item
  \href{https://www.indiebound.org/book/9780812998757?aff=NYT}{Indiebound}
\end{itemize}

When you purchase an independently reviewed book through our site, we
earn an affiliate commission.

In ``The Unwomanly Face of War,'' an oral history of World War II, the
Nobel Prize-winning writer Svetlana Alexievich recounts a strange little
story. A woman leaps into dark water to rescue a drowning man. At the
shore, however, she realizes it is not a man she has hauled from the
water but a gigantic sturgeon. The sturgeon dies.

Censors initially cut the scene from Alexievich's book. \emph{You're not
asking about the right things}, they remonstrated. \emph{Focus on
bravery, on patriotism}. \emph{Let's have less about fear, and less
about hairstyles.} There was no place in the canon for her sort of
wartime stories, Alexievich recalled in an interview with The Paris
Review. There was no place for reality, which comes stuffed with
sturgeons and all manner of misapprehensions and muddle; reality, which
shows notable indifference, if not outright hostility, to plot.

Perhaps an alternative canon exists, in the work of oral historians like
Alexievich, and in the deeply reported narratives of journalists like
Barbara Demick. The method is programmatic openness, deep listening, a
willingness to be waylaid; the effect, a prismatic picture of history as
experienced and understood by individuals in their full amplitude and
idiosyncrasy. Alexievich collects the daydreams of her subjects. In
Demick's impressive account of life in North Korea,
\href{https://www.nytimes3xbfgragh.onion/2010/01/27/books/27book.html}{``Nothing
to Envy,''} she described a society on the brink of starvation, cut off
from the world, lacking even electricity. But she told love stories,
too. Darkness proved to be a surprising boon;
\href{https://www.nytimes3xbfgragh.onion/2010/01/27/books/excerpt-nothing-to-envy.html}{some
North Koreans told her} they grew to need it, as it conferred the only
freedom they knew. Young people fell in love in the dark: ``Wrapped in a
magic cloak of invisibility, you can do what you like without worrying
about the prying eyes of parents, neighbors or secret police.''

``Eat the Buddha'' is Demick's third book, all of them told in rotating
perspectives --- a model inspired by
\href{https://www.nytimes3xbfgragh.onion/1985/08/04/books/beyond-the-brink-of-destruction.html}{John
Hersey's ``Hiroshima,''} and one she has made her own. In ``Logavina
Street,'' she described daily life during the Bosnian War through the
lens of one neighborhood in Sarajevo. ``Nothing to Envy'' followed six
refugees from the port city of Chongjin. The close focus gives her work
its granularity, but it also allows her to crosscheck the stories of her
subjects. ``Good reporting should have the same standard as in a
courtroom --- beyond a reasonable doubt,'' she has said. In her latest,
the masterly ``Eat the Buddha,'' she profiles a group of Tibetans with
roots in Ngaba County, in the Chinese province of Sichuan, which bears
the gory distinction of being the ``undisputed world capital of
self-immolations.''

Despite the Buddhist taboo against suicide, some 156 Tibetans --- at the
time of Demick's writing --- have set themselves on fire in recent
years, protesting China's rule. They have perfected their technique,
wrapping themselves in quilts and wire to prevent rescue, dousing
themselves in gasoline and swallowing it, too, to ensure they will burn
from the inside. Almost a third of these people --- monks, mothers,
ordinary citizens --- have come from Ngaba and the surrounding region.

Why Ngaba? ``Why were so many of its residents willing to destroy their
bodies by one of the most horrific methods imaginable?'' This mystery
hooked Demick, who arrived in China in 2007 as the Beijing bureau chief
of The Los Angeles Times. On the face of it, Ngaba is better off than
many of its counterparts, she observes. The residents are comfortable,
the infrastructure comparatively decent. (The government invested in a
``blitz'' of modernization in the hopes of quelling the uprisings). Some
attribute the protests to the
\href{https://www.nytimes3xbfgragh.onion/2012/06/03/world/asia/wave-of-tibet-self-immolations-challenges-chinese-rule.html}{harsh
and oppressive police presence}. But Demick argues that the roots run
deeper. Ngaba was the site of Tibet's first meeting with Chinese
Communists, in the 1930s. ``The people of this region have a particular
wound causing excessive suffering that spans three generations,'' the
monk Kirti Rinpoche testified before a U.S. congressional commission in
2011. ``This wound is very difficult to forget or heal.''

Image

Barbara Demick, whose new book is ``Eat the Buddha: Life and Death in a
Tibetan Town.''Credit...Madeleine Grant

Fleeing Nationalist forces, the Red Army marauded through monasteries.
They burned holy books and manuscripts, and survived by boiling and
eating the skin of drums and the votive offerings to the Buddha (from
which the book gets its title). Demick traces this first encounter, and
the ensuing violent history, through the testimonies of her cast of
characters: students and teachers, market sellers, the private secretary
to the Dalai Lama, the former princess of the Mei kingdom.

These scenes are narrated as flashes of memory, anchored by the types of
details children remember, giving them an unbearable vividness and
horror. One man recalls hiding himself as a little boy when his house
was invaded by Chinese soldiers. He emerged to find his grandfather gone
and grandmother badly shaken, her scalp bleeding. He remembers
wondering: Where are her pigtails? The former princess remembers being
so curious about the Chinese at first, so delighted to meet them. Her
mother joked that she offered grass to their trucks, the first vehicles
she had ever seen. She thought they were horses.

Those who self-immolate today are the grandchildren of those who
participated in the early uprisings, Demick writes. Having imbibed the
Dalai Lama's teachings of nonviolence, they can only bear to hurt
themselves. They bear the scars of the ``Democratic Reforms'' in eastern
Tibet that began in 1958. ``Tibetans of this generation refer to this
period simply as \emph{ngabgay} --- '58. Like 9/11, it is shorthand for
a catastrophe so overwhelming that words cannot express it, only the
number,'' Demick writes. ``Some will call it \emph{dhulok}, a word that
roughly translates as the `collapse of time,' or, hauntingly, `when the
sky and earth changed places.'''

Tibetans were forced into cooperative living, stripped of their herds
and land. Their yaks --- sources of their food, clothing and light
(candles were made from yak fat) --- were seized and slaughtered,
recalling the American government's devastating policy of culling the
Lakota's bison. Daylong public ``struggle sessions'' were instituted ---
rituals of public humiliation in which those accused of perceived
infractions were forced to admit to crimes and submit to verbal and
physical abuse --- with children forced to observe and cheer along. Some
20 percent of the population was arrested and held in prisons that were
often only pits in the ground filled with hundreds of people. An
estimated 300,000 Tibetans died.

Demick covers an awe-inspiring breadth of history --- from the heyday of
the Tibetan empire, which could compete with those of the Turks and
Arabs, to the present day, as the movement for Tibetan independence has
faltered and transformed into an effort at cultural and spiritual
survival. She charts the creative rebellions of recent years, the
efforts at revitalizing the language and traditions, Tibetans'
attachment to the Dalai Lama (and their criticisms). Above all, Demick
wants to give room for contemporary Tibetans to testify to their
desires. If the spectacular horror of the self-immolations first
attracted her interest, she finds, at least among her sources, demands
that sound surprisingly modest. They want only the rights enjoyed by the
Han Chinese, she writes --- to travel, hold a passport, to study their
own language, to educate their children abroad if they wish.

Her forecast is pessimistic. Only in North Korea has she witnessed such
smothering surveillance and high levels of fear, she writes, accelerated
by technological developments like a social credit system in development
that will prevent ``untrustworthy'' citizens from employment, buying
plane tickets and using credit cards.

In Ngaba, the last Tibetan-language school --- the last one in all of
China --- has switched to teaching primarily in Chinese. Meanwhile,
across the country, Demick notices the same red billboards springing up,
proclaiming the latest propaganda: ``\emph{TOGETHER WE WILL BUILD A
BEAUTIFUL HOME. BEND LOW. LISTEN TO WHAT PEOPLE SAY.''}

Advertisement

\protect\hyperlink{after-bottom}{Continue reading the main story}

\hypertarget{site-index}{%
\subsection{Site Index}\label{site-index}}

\hypertarget{site-information-navigation}{%
\subsection{Site Information
Navigation}\label{site-information-navigation}}

\begin{itemize}
\tightlist
\item
  \href{https://help.nytimes3xbfgragh.onion/hc/en-us/articles/115014792127-Copyright-notice}{©~2020~The
  New York Times Company}
\end{itemize}

\begin{itemize}
\tightlist
\item
  \href{https://www.nytco.com/}{NYTCo}
\item
  \href{https://help.nytimes3xbfgragh.onion/hc/en-us/articles/115015385887-Contact-Us}{Contact
  Us}
\item
  \href{https://www.nytco.com/careers/}{Work with us}
\item
  \href{https://nytmediakit.com/}{Advertise}
\item
  \href{http://www.tbrandstudio.com/}{T Brand Studio}
\item
  \href{https://www.nytimes3xbfgragh.onion/privacy/cookie-policy\#how-do-i-manage-trackers}{Your
  Ad Choices}
\item
  \href{https://www.nytimes3xbfgragh.onion/privacy}{Privacy}
\item
  \href{https://help.nytimes3xbfgragh.onion/hc/en-us/articles/115014893428-Terms-of-service}{Terms
  of Service}
\item
  \href{https://help.nytimes3xbfgragh.onion/hc/en-us/articles/115014893968-Terms-of-sale}{Terms
  of Sale}
\item
  \href{https://spiderbites.nytimes3xbfgragh.onion}{Site Map}
\item
  \href{https://help.nytimes3xbfgragh.onion/hc/en-us}{Help}
\item
  \href{https://www.nytimes3xbfgragh.onion/subscription?campaignId=37WXW}{Subscriptions}
\end{itemize}
