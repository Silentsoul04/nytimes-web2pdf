Sections

SEARCH

\protect\hyperlink{site-content}{Skip to
content}\protect\hyperlink{site-index}{Skip to site index}

\href{https://www.nytimes3xbfgragh.onion/section/business}{Business}

\href{https://myaccount.nytimes3xbfgragh.onion/auth/login?response_type=cookie\&client_id=vi}{}

\href{https://www.nytimes3xbfgragh.onion/section/todayspaper}{Today's
Paper}

\href{/section/business}{Business}\textbar{}Airlines May Learn to Love
the Boeing 737 Max Again

\url{https://nyti.ms/3j5UaL8}

\begin{itemize}
\item
\item
\item
\item
\item
\end{itemize}

\href{https://www.nytimes3xbfgragh.onion/news-event/coronavirus?action=click\&pgtype=Article\&state=default\&region=TOP_BANNER\&context=storylines_menu}{The
Coronavirus Outbreak}

\begin{itemize}
\tightlist
\item
  live\href{https://www.nytimes3xbfgragh.onion/2020/08/03/world/coronavirus-covid-19.html?action=click\&pgtype=Article\&state=default\&region=TOP_BANNER\&context=storylines_menu}{Latest
  Updates}
\item
  \href{https://www.nytimes3xbfgragh.onion/interactive/2020/us/coronavirus-us-cases.html?action=click\&pgtype=Article\&state=default\&region=TOP_BANNER\&context=storylines_menu}{Maps
  and Cases}
\item
  \href{https://www.nytimes3xbfgragh.onion/interactive/2020/science/coronavirus-vaccine-tracker.html?action=click\&pgtype=Article\&state=default\&region=TOP_BANNER\&context=storylines_menu}{Vaccine
  Tracker}
\item
  \href{https://www.nytimes3xbfgragh.onion/2020/08/02/us/covid-college-reopening.html?action=click\&pgtype=Article\&state=default\&region=TOP_BANNER\&context=storylines_menu}{College
  Reopening}
\item
  \href{https://www.nytimes3xbfgragh.onion/live/2020/08/03/business/stock-market-today-coronavirus?action=click\&pgtype=Article\&state=default\&region=TOP_BANNER\&context=storylines_menu}{Economy}
\end{itemize}

Advertisement

\protect\hyperlink{after-top}{Continue reading the main story}

Supported by

\protect\hyperlink{after-sponsor}{Continue reading the main story}

\hypertarget{airlines-may-learn-to-love-the-boeing-737-max-again}{%
\section{Airlines May Learn to Love the Boeing 737 Max
Again}\label{airlines-may-learn-to-love-the-boeing-737-max-again}}

Airlines have canceled orders for hundreds of the troubled jet because
of its safety problems and the pandemic, but others are still eager to
buy them.

\includegraphics{https://static01.graylady3jvrrxbe.onion/images/2020/07/15/business/15BOEING-MAX1/merlin_174539235_ee8ddef5-8f4e-4e0e-bcac-6bc0872c1022-articleLarge.jpg?quality=75\&auto=webp\&disable=upscale}

\href{https://www.nytimes3xbfgragh.onion/by/niraj-chokshi}{\includegraphics{https://static01.graylady3jvrrxbe.onion/images/2018/02/20/multimedia/author-niraj-chokshi/author-niraj-chokshi-thumbLarge.jpg}}

By \href{https://www.nytimes3xbfgragh.onion/by/niraj-chokshi}{Niraj
Chokshi}

\begin{itemize}
\item
  Published July 15, 2020Updated July 16, 2020
\item
  \begin{itemize}
  \item
  \item
  \item
  \item
  \item
  \end{itemize}
\end{itemize}

The first half of the year was not kind to the 737 Max. Boeing froze
production of its beleaguered plane from January through much of May as
customers canceled hundreds of orders, and deals for hundreds more were
put at risk by delays in the plane's return to the skies and the
coronavirus pandemic.

But Boeing is back to work on the Max, and if it passes regulatory
scrutiny, the plane could fly again
\href{https://www.nytimes3xbfgragh.onion/2020/07/02/travel/boeings-737-max-flying.html}{as
soon as the end of the year}. When it does, it will return to an
industry that was hammered by the coronavirus and faces a yearslong
recovery.

The Max crisis has already wrecked Boeing's bottom line. In January, the
company said it expected the grounding
\href{https://www.nytimes3xbfgragh.onion/2020/01/29/business/boeing-737-max-costs.html}{to
cost more than \$18 billion}, which didn't account for the ruinous
effect the pandemic would have on airlines. In April, it announced plans
\href{https://www.nytimes3xbfgragh.onion/2020/04/29/business/boeing-layoffs-coronavirus.html}{to
cut} about 16,000 jobs, or a tenth of its work force, because of the
pandemic's impact.

The aerospace manufacturer said this week that its customers had
canceled 373 Max orders in the first six months of the year. Another 439
are considered at risk, including nearly 100 that Norwegian Air, a
struggling low-cost carrier, recently said it no longer planned to buy.

Boeing still has several thousand pending orders for the Max, but
analysts expect that to shrink somewhat as more customers back out of
deals. And even though the company plans to increase production of the
jet and other 737 variants to 31 planes per month sometime next year,
that is about half the rate Boeing had targeted before the Max was
grounded.

Globally, airlines are losing hundreds of millions of dollars by the
day, and most experts predict it will be two to five years before the
industry sees as many passengers as it did in 2019. After the Sept. 11,
2001, attacks and the financial crisis a decade ago, airlines recovered
before the overall economy, according to Boeing, which expects the
opposite this time around.

\includegraphics{https://static01.graylady3jvrrxbe.onion/images/2020/07/15/business/15BOEING-MAX5/merlin_174539265_22164736-0cd6-4fb6-ba04-5cffffc49d48-articleLarge.jpg?quality=75\&auto=webp\&disable=upscale}

In the United States, a limited recovery in domestic travel has stalled
in recent weeks as virus infections soared and states and cities
reimposed restrictions on travel and business activity. And more than a
third of the world's passenger planes --- over 8,000 aircraft ---
\href{https://www.cirium.com/thoughtcloud/covid-19-navigating-the-flight-plan-to-recovery-daily-update/}{remain
parked} and unused, according to Cirium, an airline data firm.

Yet experts said the 737 Max would survive because many airlines still
saw value in it as they fought for what few passengers remained.

``It's not phenomenal, but I don't think it's all that dire for the Max,
despite Covid and everything else,'' said Sheila Kahyaoglu, an aerospace
and defense analyst with Jefferies, an investment bank.

\hypertarget{latest-updates-economy}{%
\section{\texorpdfstring{\href{https://www.nytimes3xbfgragh.onion/live/2020/08/03/business/stock-market-today-coronavirus?action=click\&pgtype=Article\&state=default\&region=MAIN_CONTENT_1\&context=storylines_live_updates}{Latest
Updates:
Economy}}{Latest Updates: Economy}}\label{latest-updates-economy}}

\href{https://www.nytimes3xbfgragh.onion/live/2020/08/03/business/stock-market-today-coronavirus?action=click\&pgtype=Article\&state=default\&region=MAIN_CONTENT_1\&context=storylines_live_updates\#the-chicago-fed-president-says-its-up-to-congress-to-save-the-economy}{11h
ago}

\href{https://www.nytimes3xbfgragh.onion/live/2020/08/03/business/stock-market-today-coronavirus?action=click\&pgtype=Article\&state=default\&region=MAIN_CONTENT_1\&context=storylines_live_updates\#the-chicago-fed-president-says-its-up-to-congress-to-save-the-economy}{The
Chicago Fed president says it's up to Congress to save the economy.}

\href{https://www.nytimes3xbfgragh.onion/live/2020/08/03/business/stock-market-today-coronavirus?action=click\&pgtype=Article\&state=default\&region=MAIN_CONTENT_1\&context=storylines_live_updates\#faa-says-boeing-has-effectively-mitigated-defects-in-the-737-max}{11h
ago}

\href{https://www.nytimes3xbfgragh.onion/live/2020/08/03/business/stock-market-today-coronavirus?action=click\&pgtype=Article\&state=default\&region=MAIN_CONTENT_1\&context=storylines_live_updates\#faa-says-boeing-has-effectively-mitigated-defects-in-the-737-max}{F.A.A.
says Boeing has `effectively mitigated' defects in the 737 Max.}

\href{https://www.nytimes3xbfgragh.onion/live/2020/08/03/business/stock-market-today-coronavirus?action=click\&pgtype=Article\&state=default\&region=MAIN_CONTENT_1\&context=storylines_live_updates\#small-businesses-got-emergency-loans-but-not-what-they-expected}{14h
ago}

\href{https://www.nytimes3xbfgragh.onion/live/2020/08/03/business/stock-market-today-coronavirus?action=click\&pgtype=Article\&state=default\&region=MAIN_CONTENT_1\&context=storylines_live_updates\#small-businesses-got-emergency-loans-but-not-what-they-expected}{Small
businesses got emergency loans, but not what they expected.}

\href{https://www.nytimes3xbfgragh.onion/live/2020/08/03/business/stock-market-today-coronavirus?action=click\&pgtype=Article\&state=default\&region=MAIN_CONTENT_1\&context=storylines_live_updates}{See
more updates}

More live coverage:
\href{https://www.nytimes3xbfgragh.onion/2020/08/03/world/coronavirus-covid-19.html?action=click\&pgtype=Article\&state=default\&region=MAIN_CONTENT_1\&context=storylines_live_updates}{Global}

It may seem misguided for an airline in the midst of a major crisis to
buy a tarnished jet that costs tens of millions of dollars, but experts
say there is good reason many companies like Southwest Airlines and
American Airlines will stick with the Max. The plane can offer
substantial savings on fuel and maintenance that are even more valuable
in lean times. Other airlines might find it difficult to walk away from
orders they have already placed, and will reluctantly go through with
purchases.

A new plane can last a generation, and the Max's efficiency matters a
lot because fuel can account for about a fifth of an airline's operating
costs. Boeing says the plane uses
\href{https://www.boeing.com/commercial/737max/by-design/\#/improved-aerodynamics}{at
least 14 percent less jet fuel} than its predecessors. That could yield
double-digit increases in profits for airlines, said Vitaly Guzhva, a
professor of aviation finance at Embry Riddle Aeronautical University.
``There's still a pretty strong business case for the Max.''

Southwest, for example, has nearly 750 planes in its fleet, each some
version of the 737. If it had been able to replace part of its fleet
last year with the more than 275 Max jets it hopes to own, Southwest
could have saved more than \$230 million in fuel costs, according to Dr.
Guzhva's math. Boeing says the plane offers fuel savings of more than
\$10 million over its 25- to 30-year life span.

Airlines can also point to fuel savings as an indication of their
environmental stewardship to customers who are increasingly cognizant of
air travel's contribution to climate change. Others might just want to
apply the money saved to lowering the price of tickets to lure business.

The jet could yield big savings on maintenance, too. New planes often
come with warranties, and expensive engine overhauls are typically
needed a few years after those end, said Robert Spingarn, an aerospace
and defense analyst at Credit Suisse. If the timing is right, an airline
might choose to replace a plane in need of major repairs with a Max.

``When you have a brand-new airplane, you don't have to think about that
kind of expense,'' Mr. Spingarn said. ``There's going to be some that
say, `I'm sticking with the Max because the math works better for me
than not taking it.'''

Strapped airlines could also see an opportunity in buying the Max,
selling it to a third party for cash and then immediately leasing it
back. ``They get an upfront 10, 15, maybe even 20 million dollars, which
helps with liquidity,'' Dr. Guzhva said.

Delta Air Lines did just that after passenger traffic bottomed out this
year. Between April and June, the airline raised \$2.8 billion by
selling and leasing back planes, it said this week. Delta is the only
major U.S. airline not to use the Max.

By Boeing's count, thousands of airplanes worldwide are at least 20
years old and may be due for expensive maintenance or replacement soon.
And airlines over the past few months have retired older aircraft,
sometimes years ahead of schedule.

Image

An employee inspecting 737 Max 8 at Renton Municipal
Airport.Credit...Lindsey Wasson for The New York Times

Rather than back away from Boeing, airlines might also try to negotiate
compensation for the plane's grounding and delays in securing the jets.
Customers could demand that Boeing defer deliveries or offer them deep
discounts. After the Sept. 11 terrorist attacks, the low-cost Irish
carrier Ryanair reportedly snapped up 737s
\href{http://news.bbc.co.uk/2/hi/business/1779700.stm}{at a substantial
discount}, for example. When asked the price he paid, the airline's
chief executive, Michael O'Leary,
\href{http://news.bbc.co.uk/2/hi/business/2962698.stm}{demurred}: ``I
wouldn't even tell my priest what discount I got off Boeing.''

Industry trends are also on Boeing's side. For years, airlines have been
shifting away from wide-bodied planes toward narrow-bodied ones like the
Max, which are easier to fill. And the pandemic only seems to be
accelerating the shift. The rebound in air travel, pitiful as it is, is
also being driven by domestic flights, exactly the kind of short trips
for which the Max was designed.

Walking away from the Max may prove difficult for airlines, too.

Contracts are drawn up years in advance of delivery and can be difficult
to break, experts said. Still, terms vary substantially by order, so
some airlines may be better positioned than others, and contracts for
the Max typically provide for some renegotiation rights if deliveries
are delayed by more than a year. Some airlines may also have signed
contracts that require them to forfeit money they've already paid if
they cancel.

The Max has \href{https://www.boeing.com/company/about-bca/}{a list
price} of as much as \$135 million for the latest model, but can sell
for far less: as little as 50 percent of that figure for a large enough
order, according to experts. An airline might pay 1 percent upfront when
it signs a letter of intent and 5 percent more when it signs a contract,
said Eddy Pieniazek, an airline consultant at Ishka, a consulting firm.
The rest is typically paid in the year or two before a plane is
delivered.

Relationships with manufacturers can run deep, with long-term plans
built around an all-Boeing or all-Airbus fleet; the two companies have a
roughly equal share of the commercial plane market. At Southwest, for
example, introducing a new plane would increase maintenance and training
costs.

``There are companies that stick with Boeing, and there are companies
that stick with Airbus; you don't often get people jumping and
changing,'' Mr. Pieniazek said. ``There are people who have bought into
the Max story and will want to fly their airplanes.''

Advertisement

\protect\hyperlink{after-bottom}{Continue reading the main story}

\hypertarget{site-index}{%
\subsection{Site Index}\label{site-index}}

\hypertarget{site-information-navigation}{%
\subsection{Site Information
Navigation}\label{site-information-navigation}}

\begin{itemize}
\tightlist
\item
  \href{https://help.nytimes3xbfgragh.onion/hc/en-us/articles/115014792127-Copyright-notice}{©~2020~The
  New York Times Company}
\end{itemize}

\begin{itemize}
\tightlist
\item
  \href{https://www.nytco.com/}{NYTCo}
\item
  \href{https://help.nytimes3xbfgragh.onion/hc/en-us/articles/115015385887-Contact-Us}{Contact
  Us}
\item
  \href{https://www.nytco.com/careers/}{Work with us}
\item
  \href{https://nytmediakit.com/}{Advertise}
\item
  \href{http://www.tbrandstudio.com/}{T Brand Studio}
\item
  \href{https://www.nytimes3xbfgragh.onion/privacy/cookie-policy\#how-do-i-manage-trackers}{Your
  Ad Choices}
\item
  \href{https://www.nytimes3xbfgragh.onion/privacy}{Privacy}
\item
  \href{https://help.nytimes3xbfgragh.onion/hc/en-us/articles/115014893428-Terms-of-service}{Terms
  of Service}
\item
  \href{https://help.nytimes3xbfgragh.onion/hc/en-us/articles/115014893968-Terms-of-sale}{Terms
  of Sale}
\item
  \href{https://spiderbites.nytimes3xbfgragh.onion}{Site Map}
\item
  \href{https://help.nytimes3xbfgragh.onion/hc/en-us}{Help}
\item
  \href{https://www.nytimes3xbfgragh.onion/subscription?campaignId=37WXW}{Subscriptions}
\end{itemize}
