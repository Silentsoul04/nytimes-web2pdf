Sections

SEARCH

\protect\hyperlink{site-content}{Skip to
content}\protect\hyperlink{site-index}{Skip to site index}

\href{https://myaccount.nytimes3xbfgragh.onion/auth/login?response_type=cookie\&client_id=vi}{}

\href{https://www.nytimes3xbfgragh.onion/section/todayspaper}{Today's
Paper}

\href{/section/opinion}{Opinion}\textbar{}Please Don't Call Them Heroes

\url{https://nyti.ms/2OrIX9H}

\begin{itemize}
\item
\item
\item
\item
\item
\item
\end{itemize}

Advertisement

\protect\hyperlink{after-top}{Continue reading the main story}

\href{/section/opinion}{Opinion}

Supported by

\protect\hyperlink{after-sponsor}{Continue reading the main story}

\hypertarget{please-dont-call-them-heroes}{%
\section{Please Don't Call Them
Heroes}\label{please-dont-call-them-heroes}}

Parents and teachers need a real plan to reopen schools safely.

\href{https://www.nytimes3xbfgragh.onion/by/farhad-manjoo}{\includegraphics{https://static01.graylady3jvrrxbe.onion/images/2019/01/08/opinion/farhad-manjoo-opinion/farhad-manjoo-opinion-thumbLarge.png}}

By \href{https://www.nytimes3xbfgragh.onion/by/farhad-manjoo}{Farhad
Manjoo}

Opinion Columnist

\begin{itemize}
\item
  July 15, 2020
\item
  \begin{itemize}
  \item
  \item
  \item
  \item
  \item
  \item
  \end{itemize}
\end{itemize}

\includegraphics{https://static01.graylady3jvrrxbe.onion/images/2020/07/15/opinion/15manjooWeb/merlin_174339132_8c5b6f24-5725-40ed-ac8b-ddd9ba33c790-articleLarge.jpg?quality=75\&auto=webp\&disable=upscale}

\hypertarget{listen-to-this-op-ed}{%
\subsubsection{Listen to This Op-Ed}\label{listen-to-this-op-ed}}

Audio Recording by Audm

\emph{To hear more audio stories from publishers like The New York
Times,
download}\href{https://www.audm.com/?utm_source=nytmag\&utm_medium=embed\&utm_campaign=left_behind_draper}{**}\href{https://www.audm.com/?utm_source=nytopinion\&utm_medium=embed\&utm_campaign=please_call_heroes}{\emph{Audm
for iPhone or Android}}\emph{.}

In America, you should always get a little suspicious when politicians
suddenly
\href{https://slate.com/news-and-politics/2020/04/coronavirus-humans-vs-heroes.html}{start
calling you a hero}. It's a well-worn trick; they're buttering you up
before sacrificing you to the gods of unconstrained capitalism and
governmental neglect.

A few months ago, it was nurses, doctors and other essential workers who
were hailed as heroes --- a perfectly accurate and heartwarming
sentiment, but also one meant to obscure the sorry reality that the
world's richest country was asking health care workers to treat
coronavirus patients
\href{https://www.nytimes3xbfgragh.onion/2020/03/25/opinion/coronavirus-face-mask.html}{without
providing adequate protective gear}.

``Please don't call me a hero,''
\href{https://slate.com/technology/2020/04/nurse-hero-protest.html}{a
nurse in Brooklyn} wrote on a protest sign at the time. ``I am being
martyred against my will.''

Now, it's America's parents and teachers who are being valorized for
doing a job that really should not require putting their lives on the
line.

At
\href{https://www.whitehouse.gov/briefings-statements/remarks-president-trump-safely-reopening-americas-schools/}{a
White House event} last week to encourage the nation's schools to
reopen, Vice President Mike Pence laid the heroism on thick. Parents and
teachers, he said, were ``two categories of heroes that emerged'' in the
crisis. Since the pandemic is all but over, at least in the magical
thinking of the Trump administration, Pence wants parents and teachers
to again put on their capes and save the day. ``To open up America
again, we got to open up America's schools,'' he said.

I want schools to reopen as much any parent does. My wife and I were
\href{https://www.nytimes3xbfgragh.onion/2020/04/22/opinion/coronavirus-parenting-burnout.html}{driven
to the verge of breakdown}this spring while trying to home-school our
kids while working from home, and I am freaking out about having to do
that again in the fall.

But parents and teachers would be wise to reject any invitation to
unnecessary heroism. I don't want educating my kids to be a heroic act
of American defiance --- I want it to be ordinary. And I'd rather not
sacrifice my children's teachers, either, so that America's economy can
begin humming once more.

Again and again in this crisis, the federal government's callous
incompetence has left Americans with no good options.
\href{https://www.sciencemag.org/news/2020/07/school-openings-across-globe-suggest-ways-keep-coronavirus-bay-despite-outbreaks}{Early
research on school reopening} suggests that classrooms can be safe when
the virus is contained or declining, and so long as schools take
necessary precautions to minimize the chance that classrooms become
superspreaders. But in much of the nation, the virus remains
uncontained, and so we face a grim future. There will likely be danger
and chaos if the schools do not reopen, and there will be
\href{https://www.nytimes3xbfgragh.onion/2020/07/10/us/politics/trump-schools-reopening.html?searchResultPosition=1}{danger
and chaos if they do}.

The needs of children and working parents have long been ignored by
American lawmakers, but I've never felt the government's neglect as
viscerally as in its inability to make school safe again during a
pandemic.

Shouldn't getting our kids back to school have been a primary goal of
the federal government throughout the summer? What possible excuse can
anyone muster for falling down on this job?

I'm reminded of that famous presidential call to sacrifice: Ask not what
your country can do for you \ldots{} because, honestly, it probably
won't do much of anything, and your best bet when facing a crisis is to
\href{https://www.nytimes3xbfgragh.onion/2020/03/18/opinion/coronavirus-trump.html}{just
learn to live with it}. (This sounds like a joke, but it's the actual
plan: ``The virus is with us, but we need to live with it,'' one
anonymous administration official recently
\href{https://www.nbcnews.com/politics/politics-news/we-need-live-it-white-house-readies-new-message-nation-n1232884}{told
NBC News}.)

Experts say there are many ways to
\href{https://www.nytimes3xbfgragh.onion/2020/07/10/opinion/coronavirus-schools-reopening.html}{mitigate
the risks of the coronavirus while reopening the schools}. The most
obvious of these would have been to reduce the spread of the virus, but
you know how well that's going.

The federal government could also have provided
\href{https://ccsso.org/blog/ccsso-submits-cost-estimate-safely-reopening-public-schools-after-covid}{the
hundreds of billions of dollars} that school district officials say is
necessary to remake education during a pandemic. We could have funded
hazard pay for teachers and paid time off for parents, and come up with
a plan to repurpose office buildings or gyms for the space required to
teach students while social distancing.

In May,
\href{https://blogs.edweek.org/edweek/campaign-k-12/2020/05/covid-relief-schools-60-billion-democrats-bill.html}{Democrats
in the House passed} a bill that calls for \$58 billion in new funds for
schools. But the Republican Senate has not taken up the measure, and
President Trump has done little more than post several all-caps tweets
demanding that they reopen. In cable-news interviews this weekend, Betsy
DeVos, the education secretary,
\href{https://www.cnn.com/2020/07/12/politics/betsy-devos-schools-reopening-coronavirus-cnntv/index.html}{refused
to say} if schools should follow the Centers for Disease Control and
Prevention's
\href{https://www.cdc.gov/coronavirus/2019-ncov/community/schools-childcare/schools.html}{health
guidelines}, which calls for strict social distancing, masks and the
installation of physical barriers and improved ventilation to reduce the
spread of the virus.

DeVos's plan, like Trump's, appears to be little more than wishful
thinking: Go to school. Don't worry about it. Things will be fine.
You're a hero!

Forgive me if I feel less like a hero than like a chump. This week
several large school districts, including Los Angeles and San Diego,
announced that it's too dangerous to open for in-person instruction. I
expect that we'll see a wave of others deciding the same, leaving
parents across the country in an impossible bind.

There is a danger that frustrated parents blame teachers for the crisis.
After all, distance learning has been a disaster. It's unfair, and
likely impossible, for kids to learn by themselves off a screen, and in
my experience, remote learning requires a great deal of parental
oversight, which is difficult or impossible for most overburdened
parents.

But as I tried and failed to educate my kids during months under
quarantine, I gained new appreciation for my children's teachers, and
I'm wary of asking more of them. Spending a day teaching kids has got to
be one of the most difficult and most thankless job our society asks
professionals to do. It doesn't strike me as fair to demand that
teachers now risk their lives, too, just because our government couldn't
be bothered to protect them. Teachers shouldn't have to be heroes to do
their jobs; educating our children should be heroism enough.

\hypertarget{office-hours-with-farhad-manjoo}{%
\subsection{Office Hours With Farhad
Manjoo}\label{office-hours-with-farhad-manjoo}}

\emph{Farhad wants to}
\href{https://www.nytimes3xbfgragh.onion/2019/05/16/opinion/farhad-office-hours.html?module=inline}{\emph{chat
with readers on the phone}}\emph{. If you're interested in talking to a
New York Times columnist about anything that's on your mind, please fill
out this form. Farhad will select a few readers to call.}

\emph{The Times is committed to publishing}
\href{https://www.nytimes3xbfgragh.onion/2019/01/31/opinion/letters/letters-to-editor-new-york-times-women.html}{\emph{a
diversity of letters}} \emph{to the editor. We'd like to hear what you
think about this or any of our articles. Here are some}
\href{https://help.nytimes3xbfgragh.onion/hc/en-us/articles/115014925288-How-to-submit-a-letter-to-the-editor}{\emph{tips}}\emph{.
And here's our email:}
\href{mailto:letters@NYTimes.com}{\emph{letters@NYTimes.com}}\emph{.}

\emph{Follow The New York Times Opinion section on}
\href{https://www.facebookcorewwwi.onion/nytopinion}{\emph{Facebook}}\emph{,}
\href{http://twitter.com/NYTOpinion}{\emph{Twitter (@NYTopinion)}}
\emph{and}
\href{https://www.instagram.com/nytopinion/}{\emph{Instagram}}\emph{.}

Advertisement

\protect\hyperlink{after-bottom}{Continue reading the main story}

\hypertarget{site-index}{%
\subsection{Site Index}\label{site-index}}

\hypertarget{site-information-navigation}{%
\subsection{Site Information
Navigation}\label{site-information-navigation}}

\begin{itemize}
\tightlist
\item
  \href{https://help.nytimes3xbfgragh.onion/hc/en-us/articles/115014792127-Copyright-notice}{©~2020~The
  New York Times Company}
\end{itemize}

\begin{itemize}
\tightlist
\item
  \href{https://www.nytco.com/}{NYTCo}
\item
  \href{https://help.nytimes3xbfgragh.onion/hc/en-us/articles/115015385887-Contact-Us}{Contact
  Us}
\item
  \href{https://www.nytco.com/careers/}{Work with us}
\item
  \href{https://nytmediakit.com/}{Advertise}
\item
  \href{http://www.tbrandstudio.com/}{T Brand Studio}
\item
  \href{https://www.nytimes3xbfgragh.onion/privacy/cookie-policy\#how-do-i-manage-trackers}{Your
  Ad Choices}
\item
  \href{https://www.nytimes3xbfgragh.onion/privacy}{Privacy}
\item
  \href{https://help.nytimes3xbfgragh.onion/hc/en-us/articles/115014893428-Terms-of-service}{Terms
  of Service}
\item
  \href{https://help.nytimes3xbfgragh.onion/hc/en-us/articles/115014893968-Terms-of-sale}{Terms
  of Sale}
\item
  \href{https://spiderbites.nytimes3xbfgragh.onion}{Site Map}
\item
  \href{https://help.nytimes3xbfgragh.onion/hc/en-us}{Help}
\item
  \href{https://www.nytimes3xbfgragh.onion/subscription?campaignId=37WXW}{Subscriptions}
\end{itemize}
