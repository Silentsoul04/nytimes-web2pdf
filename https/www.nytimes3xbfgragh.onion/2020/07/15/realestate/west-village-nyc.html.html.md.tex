Sections

SEARCH

\protect\hyperlink{site-content}{Skip to
content}\protect\hyperlink{site-index}{Skip to site index}

\href{https://www.nytimes3xbfgragh.onion/section/realestate}{Real
Estate}

\href{https://myaccount.nytimes3xbfgragh.onion/auth/login?response_type=cookie\&client_id=vi}{}

\href{https://www.nytimes3xbfgragh.onion/section/todayspaper}{Today's
Paper}

\href{/section/realestate}{Real Estate}\textbar{}The West Village: A
Pause, and a Reset, for a Coveted Area

\url{https://nyti.ms/2C6rph0}

\begin{itemize}
\item
\item
\item
\item
\item
\item
\end{itemize}

\href{https://www.nytimes3xbfgragh.onion/spotlight/at-home?action=click\&pgtype=Article\&state=default\&region=TOP_BANNER\&context=at_home_menu}{At
Home}

\begin{itemize}
\tightlist
\item
  \href{https://www.nytimes3xbfgragh.onion/2020/07/28/books/time-for-a-literary-road-trip.html?action=click\&pgtype=Article\&state=default\&region=TOP_BANNER\&context=at_home_menu}{Take:
  A Literary Road Trip}
\item
  \href{https://www.nytimes3xbfgragh.onion/2020/07/29/magazine/bored-with-your-home-cooking-some-smoky-eggplant-will-fix-that.html?action=click\&pgtype=Article\&state=default\&region=TOP_BANNER\&context=at_home_menu}{Cook:
  Smoky Eggplant}
\item
  \href{https://www.nytimes3xbfgragh.onion/2020/07/27/travel/moose-michigan-isle-royale.html?action=click\&pgtype=Article\&state=default\&region=TOP_BANNER\&context=at_home_menu}{Look
  Out: For Moose}
\item
  \href{https://www.nytimes3xbfgragh.onion/interactive/2020/at-home/even-more-reporters-editors-diaries-lists-recommendations.html?action=click\&pgtype=Article\&state=default\&region=TOP_BANNER\&context=at_home_menu}{Explore:
  Reporters' Obsessions}
\end{itemize}

Advertisement

\protect\hyperlink{after-top}{Continue reading the main story}

Supported by

\protect\hyperlink{after-sponsor}{Continue reading the main story}

Living in

\hypertarget{the-west-village-a-pause-and-a-reset-for-a-coveted-area}{%
\section{The West Village: A Pause, and a Reset, for a Coveted
Area}\label{the-west-village-a-pause-and-a-reset-for-a-coveted-area}}

The neighborhood's low-key, small-scale charms are gradually returning
following the lockdown, although residential sales and prices are way
down.

\href{https://www.nytimes3xbfgragh.onion/slideshow/2020/07/15/realestate/living-in-the-west-village.html}{}

\hypertarget{living-in--the-west-village}{%
\subsection{Living In ... The West
Village}\label{living-in--the-west-village}}

17 Photos

View Slide Show ›

\includegraphics{https://static01.graylady3jvrrxbe.onion/images/2020/07/19/realestate/15LIVING-WESTVILLAGE-slide-1WKE/15LIVING-WESTVILLAGE-slide-1WKE-articleLarge.jpg?quality=75\&auto=webp\&disable=upscale}

Chang W. Lee/The New York Times

\href{https://www.nytimes3xbfgragh.onion/by/c-j-hughes}{\includegraphics{https://static01.graylady3jvrrxbe.onion/images/2018/12/05/multimedia/author-c-j-hughes/author-c-j-hughes-thumbLarge.png}}

By \href{https://www.nytimes3xbfgragh.onion/by/c-j-hughes}{C. J. Hughes}

\begin{itemize}
\item
  July 15, 2020
\item
  \begin{itemize}
  \item
  \item
  \item
  \item
  \item
  \item
  \end{itemize}
\end{itemize}

Store owners have barricaded their shops. West Fourth Street is missing
its brunch-hour crowds. And shouting about masks occasionally erupts in
packed Hudson River Park.

But despite a war-torn air from guarding against damage during recent
anti-police-brutality protests and from the lingering effects of the
coronavirus, residents of the West Village, one of New York's oldest and
most-coveted enclaves, say it is too soon to count out their home.

W. 18TH ST.

High Line

Hudson R.

Meatpacking

District

W. 14TH ST.

HORATIO ST.

EIGHTH AVE.

GREENWICH AVE.

Westbeth

Hudson

River

Park

WAVERLY PL.

W. FOURTH ST.

BLEECKER ST.

W.11TH ST.

West Village

CHRISTOPHER ST.

Greenwich

Village~

SEVENTH AVE. S.

St. Luke's

Church

WASHINGTON ST.

GREENWICH ST.

HUDSON ST.

West Village

W. HOUSTON ST.

Manhattan

By The New York Times

``I have hope,'' said Mary Phillips, whose rent-stabilized one-bedroom
rental is her fourth address since she moved to the neighborhood with
her then-husband,
\href{https://www.nytimes3xbfgragh.onion/2013/12/20/nyregion/al-goldstein-pioneering-pornographer-dies-at-77.html}{Al
Goldstein}, the porn-magazine publisher, in 1969. ``There's still no
other place that has the same sense of community.''

Of course, since those days, the neighborhood --- a quieter and quainter
version of next-door Greenwich Village --- has become less bohemian and
much more expensive: The 12 market-rate units in Ms. Phillips's 15-unit
walk-up can each rent for over \$4,000 a month.

There have been small silver linings among the upheaval of the past few
months, said Ms. Phillips, 75, a retired criminology researcher --- less
horn-honking, neighbors introducing themselves, and a seeming resetting
of nature. As Ms. Phillips spoke, a mourning dove alighted on her fire
escape, continuing an uptick in bird visits. If the Covid-19 era
presents opportunities for improvement, some hope for a revived retail
scene along streets like Bleecker, which was hammered by closures even
before the coronavirus hit.

``You feel a little sorry for the stores that have been kicked out
because of ridiculously high rents,'' said Stefano Bonini, 48, an
associate professor of finance at the Stevens Institute of Technology
and a neighborhood resident.

Mr. Bonini lives in a three-bedroom duplex with his wife, Daniela, 45, a
marketing executive for an eyeglasses company, and their three children,
ages 10, 7 and 3. The apartment, which cost \$1.68 million in 2019, is
his fourth home in the neighborhood in 12 years.

\includegraphics{https://static01.graylady3jvrrxbe.onion/images/2020/07/15/realestate/15LIVING-WESTVILLAGE-slide-IYYZ/15LIVING-WESTVILLAGE-slide-IYYZ-articleLarge.jpg?quality=75\&auto=webp\&disable=upscale}

``The vibe here is so completely different,'' he said, fondly recalling
a male neighbor he once had who would occasionally dress up as a female
nurse and hit the town. ``There are just a lot of genuine people.''

Enjoying their company has been easier since outdoor dining resumed in
late June. Mr. Bonini, who had been baking bread at home, decided to let
somebody else handle the cooking by eating dinner at Fiaschetteria
Pistoia, a cozy spot on Christopher Street that in many ways distills
the West Village's low-key, small-scale charms. ``When you go out, you
feel like you are with friends all the time,'' he said.

Skinny tree-lined sidewalks are often thick with tourists, who can be
observed snapping pictures of blooming pear trees on West 11th Street
and a Perry Street townhouse that stood in as the home of Carrie
Bradshaw on the ``Sex and the City.'' (On Google Earth, its facade is
blurred, an apparent bid by the building's owners for privacy.)

Walking in a straight line has been easier since the mid-March lockdown,
although some blocks have seemed ``quite eerie,'' said Resa Tylim, a
retired rehab counselor who owns a three-bedroom, two-bath co-op in a
former pharmaceutical supplies building.

When Ms. Tylim bought the apartment in 1988, for \$420,000, she was
drawn to the area's low density. With a limited number of high-rises,
light and air can seem more abundant. Those qualities could be a saving
grace once the coronavirus crisis ends.

``I think people will realize they shouldn't be living in such close
quarters,'' she said. ``Our new reality will be different.''

Image

150 CHARLES STREET, No. 3DN \textbar{} A three-bedroom,
three-and-a-half-bathroom condo with 2,607 square feet, an open kitchen,
casement windows and two exposures, listed for \$8.92 million.
212-352-5217Credit...Chang W. Lee/The New York Times

\hypertarget{what-youll-find}{%
\subsection{What You'll Find}\label{what-youll-find}}

``The quality of its architecture, the nature of the artistic life
within its boundaries and the feeling of history that permeates its
streets'' give the West Village an unequaled distinction, according to
the city's Landmarks Preservation Commission, which fought to preserve
the trapped-in-time look.

Indeed, it took four years, from 1965 to 1969, to formalize the area's
landmark status, in part because property owners were opposed to limits
on exterior renovation. Other districts followed, and today almost all
of the West Village is a landmark.

``Prewar'' could easily mean pre-Civil War. The oldest house, a
red-brick gem at 77 Bedford Street, was built in 1801. With a waterfront
that hummed with shipping and other industrial activities, the West
Village was once considered gritty. Now conversions have made homes of
warehouses, factories and stables, as well as the 19th-century
Renaissance Revival police station at 133 Charles Street, now Le
Gendarme Apartments. Westbeth, the artists' complex at West and Bethune
Streets, was once Bell Telephone Labs.

Buildings hug streets that boomerang, dogleg and terminate, adding
quirkiness. Six- and even seven-story walk-ups are not uncommon.

Townhouses built for a single family and subdivided years later as
rooming houses have gradually come back to their original function, and
then some. In recent years, moguls have assembled mega-mansions of
side-by-side properties.

Image

ONE MORTON SQUARE, No. 2CW \textbar{} A two-bedroom,
two-and-a-half-bathroom condo with 1,536 square feet and Hudson River
views, listed for \$2.995 million. 212-606-7660Credit...Chang W. Lee/The
New York Times

\hypertarget{what-youll-pay}{%
\subsection{What You'll Pay}\label{what-youll-pay}}

In early July, there were 133 co-ops and condos for sale at an average
price of \$3.8 million, according to StreetEasy. The least expensive was
a studio co-op with a nonworking fireplace, listed for \$449,000; the
priciest was a six-bedroom triplex at shimmering 165 Charles Street, a
rare new addition, for \$53 million.

In addition, 18 townhouses were for sale for an average of \$15.1
million, including a brick-and-brownstone former firehouse with a
four-car garage.

New condos are few and far between, and small. Examples are 111 Leroy
Street, a 13-unit complex with five five-story townhouses, and 90 Morton
Street, a converted printing plant with 35 two- to five-bedrooms.

Pre-pandemic, the market was soft. There were 255 sales of co-ops and
condos in 2019, for an average of \$2.43 million, versus 333 in 2018, at
\$2.88 million, according to StreetEasy. But the bottom seemed to
collapse this spring, when in-person showings were banned. In the second
quarter (based on preliminary data, as not all June closings were yet
recorded), there were 43 sales at an average of \$2.75 million --- a 22
percent plunge in prices from the same period in 2019, when 75 units
sold at an average of \$3.54 million.

Image

100 BANK STREET, No. 4G \textbar{} A studio co-op with a renovated
kitchen, two exposures and three closets in a non-doorman elevator
building, listed for \$625,000. 646-725-7889Credit...Chang W. Lee/The
New York Times

\hypertarget{the-vibe}{%
\subsection{The Vibe}\label{the-vibe}}

Less commercial than Greenwich Village, the West Village still offers a
lively mix of dive bars and upscale restaurants, including along Hudson
Street, home to the 31-year-old Cowgirl restaurant, whose sidewalk
tables have been a hit this summer. In all, there are 237 restaurants in
the neighborhood with outdoor dining, 209 of which serve alcohol,
according to a recent snapshot of the 10014 ZIP code using
\href{https://experience.arcgis.com/experience/ba953db7d541423a8e67ae1cf52bc698}{city
data}.

Hudson River Park, with its shaded lawns, refurbished piers and busy
bike path, is popular at all hours of the day. For a mellower escape,
duck into the lush gardens next to the Church of Saint Luke in the
Fields.

Image

Charles Street, between Washington and Greenwich Streets. The area was
once industrial, but now most factories and stables have been converted
into homes.Credit...Chang W. Lee/The New York Times

\hypertarget{the-schools}{%
\subsection{The Schools}\label{the-schools}}

Zoned public school options include the top-notch P.S. 41, Greenwich
Village School, which offers prekindergarten to fifth grade for 640
students. On state exams in the 2018-19 school year, 75 percent met
standards in English, versus 45 percent statewide; on the math exam, 88
percent met standards, versus 47 percent statewide.

Also highly ranked is P.S. 3, the Charrette School, with 680 students.
On last year's state exams, 75 met standards in English and 79 percent
did in math.

For sixth to eighth grade, there is M.S. 297, a three-year-old school
where 66 percent of students met standards in English last year and 61
percent did in math.

Students are not constrained by neighborhoods for high school, but a
nearby possibility is the Clinton School on East 15th Street, which
awards International Baccalaureate diplomas.

Image

While quieter than it was before the pandemic, Hudson Street has creaked
back to life courtesy of outdoor dining.Credit...Chang W. Lee/The New
York Times

\hypertarget{the-commute}{%
\subsection{The Commute}\label{the-commute}}

Public transportation is abundant. The 1, 2 and 3 subway lines run along
Seventh Avenue South, and all stop at West 14th Street, while the 1
train serves Christopher and West Houston Streets. The A, C and E trains
stop at West 14th Street, at Eighth Avenue, as does the L.

\hypertarget{the-history}{%
\subsection{The History}\label{the-history}}

The High Line, the train-line-turned-park in the next-door meatpacking
district, once continued along Washington Street
\href{https://www.nytimes3xbfgragh.onion/1991/01/15/nyregion/elevated-freight-line-being-razed-amid-protests.html?searchResultPosition=3}{to
St. John's Park Freight Terminal} at West Houston, where the tech giant
Google is now at work on a new headquarters. Much of that track was
razed in the 1960s; more came down in the early 1990s to allow apartment
buildings. A train-size open area on Westbeth's third floor is a
vestige.

For weekly email updates on residential real estate news,
\href{http://www.nytimes3xbfgragh.onion/newsletters/realestate/}{sign up
here}. Follow us on Twitter:
\href{https://twitter.com/nytrealestate}{@nytrealestate}.

Advertisement

\protect\hyperlink{after-bottom}{Continue reading the main story}

\hypertarget{site-index}{%
\subsection{Site Index}\label{site-index}}

\hypertarget{site-information-navigation}{%
\subsection{Site Information
Navigation}\label{site-information-navigation}}

\begin{itemize}
\tightlist
\item
  \href{https://help.nytimes3xbfgragh.onion/hc/en-us/articles/115014792127-Copyright-notice}{©~2020~The
  New York Times Company}
\end{itemize}

\begin{itemize}
\tightlist
\item
  \href{https://www.nytco.com/}{NYTCo}
\item
  \href{https://help.nytimes3xbfgragh.onion/hc/en-us/articles/115015385887-Contact-Us}{Contact
  Us}
\item
  \href{https://www.nytco.com/careers/}{Work with us}
\item
  \href{https://nytmediakit.com/}{Advertise}
\item
  \href{http://www.tbrandstudio.com/}{T Brand Studio}
\item
  \href{https://www.nytimes3xbfgragh.onion/privacy/cookie-policy\#how-do-i-manage-trackers}{Your
  Ad Choices}
\item
  \href{https://www.nytimes3xbfgragh.onion/privacy}{Privacy}
\item
  \href{https://help.nytimes3xbfgragh.onion/hc/en-us/articles/115014893428-Terms-of-service}{Terms
  of Service}
\item
  \href{https://help.nytimes3xbfgragh.onion/hc/en-us/articles/115014893968-Terms-of-sale}{Terms
  of Sale}
\item
  \href{https://spiderbites.nytimes3xbfgragh.onion}{Site Map}
\item
  \href{https://help.nytimes3xbfgragh.onion/hc/en-us}{Help}
\item
  \href{https://www.nytimes3xbfgragh.onion/subscription?campaignId=37WXW}{Subscriptions}
\end{itemize}
