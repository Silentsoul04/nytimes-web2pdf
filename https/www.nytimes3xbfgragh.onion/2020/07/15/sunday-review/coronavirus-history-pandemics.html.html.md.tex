Sections

SEARCH

\protect\hyperlink{site-content}{Skip to
content}\protect\hyperlink{site-index}{Skip to site index}

\href{https://www.nytimes3xbfgragh.onion/section/opinion/sunday}{Sunday
Review}

\href{https://myaccount.nytimes3xbfgragh.onion/auth/login?response_type=cookie\&client_id=vi}{}

\href{https://www.nytimes3xbfgragh.onion/section/todayspaper}{Today's
Paper}

\href{/section/opinion/sunday}{Sunday Review}\textbar{}Your Ancestors
Knew Death in Ways You Never Will

\url{https://nyti.ms/32iMejO}

\begin{itemize}
\item
\item
\item
\item
\item
\end{itemize}

\hypertarget{the-coronavirus-outbreak}{%
\subsubsection{\texorpdfstring{\href{https://www.nytimes3xbfgragh.onion/news-event/coronavirus?name=styln-coronavirus-national\&region=TOP_BANNER\&variant=undefined\&block=storyline_menu_recirc\&action=click\&pgtype=Article\&impression_id=52c561f0-e38e-11ea-81d8-93a7ced789f1}{The
Coronavirus
Outbreak}}{The Coronavirus Outbreak}}\label{the-coronavirus-outbreak}}

\begin{itemize}
\tightlist
\item
  live\href{https://www.nytimes3xbfgragh.onion/2020/08/20/world/coronavirus-covid.html?name=styln-coronavirus-national\&region=TOP_BANNER\&variant=undefined\&block=storyline_menu_recirc\&action=click\&pgtype=Article\&impression_id=52c561f1-e38e-11ea-81d8-93a7ced789f1}{Latest
  Updates}
\item
  \href{https://www.nytimes3xbfgragh.onion/interactive/2020/us/coronavirus-us-cases.html?name=styln-coronavirus-national\&region=TOP_BANNER\&variant=undefined\&block=storyline_menu_recirc\&action=click\&pgtype=Article\&impression_id=52c58900-e38e-11ea-81d8-93a7ced789f1}{Maps
  and Cases}
\item
  \href{https://www.nytimes3xbfgragh.onion/interactive/2020/science/coronavirus-vaccine-tracker.html?name=styln-coronavirus-national\&region=TOP_BANNER\&variant=undefined\&block=storyline_menu_recirc\&action=click\&pgtype=Article\&impression_id=52c58901-e38e-11ea-81d8-93a7ced789f1}{Vaccine
  Tracker}
\item
  \href{https://www.nytimes3xbfgragh.onion/2020/08/19/us/colleges-closing-covid.html?name=styln-coronavirus-national\&region=TOP_BANNER\&variant=undefined\&block=storyline_menu_recirc\&action=click\&pgtype=Article\&impression_id=52c58902-e38e-11ea-81d8-93a7ced789f1}{Colleges
  Closing}
\item
  \href{https://www.nytimes3xbfgragh.onion/live/2020/08/20/business/stock-market-today-coronavirus?name=styln-coronavirus-national\&region=TOP_BANNER\&variant=undefined\&block=storyline_menu_recirc\&action=click\&pgtype=Article\&impression_id=52c58903-e38e-11ea-81d8-93a7ced789f1}{Economy}
\end{itemize}

Advertisement

\protect\hyperlink{after-top}{Continue reading the main story}

Supported by

\protect\hyperlink{after-sponsor}{Continue reading the main story}

news analysis

\hypertarget{your-ancestors-knew-death-in-ways-you-never-will}{%
\section{Your Ancestors Knew Death in Ways You Never
Will}\label{your-ancestors-knew-death-in-ways-you-never-will}}

Some say we're doomed. But science and public spending have saved us
from pandemics worse than this one.

\includegraphics{https://static01.graylady3jvrrxbe.onion/images/2020/07/15/opinion/sunday/15McNeil2/15McNeil2-articleLarge.jpg?quality=75\&auto=webp\&disable=upscale}

\href{https://www.nytimes3xbfgragh.onion/by/donald-g-mcneil-jr}{\includegraphics{https://static01.graylady3jvrrxbe.onion/images/2018/06/13/multimedia/author-donald-g-mcneil-jr/author-donald-g-mcneil-jr-thumbLarge-v4.png}}

By
\href{https://www.nytimes3xbfgragh.onion/by/donald-g-mcneil-jr}{Donald
G. McNeil Jr.}

Mr. McNeil is a science reporter for The New York Times and has covered
epidemics since 2002.

\begin{itemize}
\item
  July 15, 2020
\item
  \begin{itemize}
  \item
  \item
  \item
  \item
  \item
  \end{itemize}
\end{itemize}

Nearly 140,000 Americans have been lost to coronavirus, and many experts
fear that the deaths will only accelerate in the fall as cold weather
forces us indoors. By year's end, half as many Americans may have died
as did in the four years of World War II.

And yet we are still arguing over how to prevent this --- each state
struggling over how much lockdown to impose and what its governor can
make its citizens do.

``You know the five stages of grief --- denial, anger, bargaining,
depression, acceptance?'' asked Dr. Emily Landon, a coronavirus expert
at the University of Chicago medical school. ``I think the American
people are in all five of them --- but different parts of the country
are in different stages.''

As death stalks us, especially our elders, have we simply become inured
to the idea that many of us are doomed?

The stock market appears to have priced in a huge wave of deaths. In the
2008-2009 recession, it fell 50 percent and took four years to recover.
In March it fell only 34 percent and has made up much of that ground
already. Looked at with Wall Street's bloodless arithmetic, that makes
sense: Most of the deaths are among the very elderly and nursing home
residents, who no longer travel or dine out or contribute much to the
economy, and who are a burden on the struggling Medicare and Social
Security Trust Funds.

One can even argue that the acceptance of death as master of us all is
part of the human psyche. But because of modern medicine, we have been
out of touch with our ultimate fate for generations.

We've all heard of the Black Death and perhaps the
\href{https://www.ancient.eu/article/782/justinians-plague-541-542-ce/}{Plague
of Justinian,} events that may have killed up to a third of mankind and
rewrote the fates of empires. They seem lost in the mists of time.

\includegraphics{https://static01.graylady3jvrrxbe.onion/images/2020/07/19/opinion/sunday/15McNeil1/15McNeil1-articleLarge.jpg?quality=75\&auto=webp\&disable=upscale}

But not that long back, our great-great-great-great-grandparents felt
the omnipresence of death in ways we will never know.

There is a chart famous among epidemiologists titled ``The Conquest of
Pestilence in New York City.'' Produced by New York City's health
department, it tracks and explains deaths in the city from the early
1800s to the present day.

Image

Credit...New York City Department of Health and Mental Hygiene

At first glance, it looks innocuous, like the ups and downs of the Dow
Jones index. But the longer you stare at the fine print, the more
horrified you become.

This past March, before coronavirus cases began to mount, the annual
death rate in New York City was about six per 1,000 New Yorkers. The
virus's first wave added about 2.5 more deaths per 1,000 to that
baseline. By contrast, from 1800 into the 1850s, deaths in the city rose
in a relentless series of epidemic spikes, year after year, with only
brief respites in between.

Image

In April, refrigerated trucks were lined up near the offices of the
medical examiner as New York prepared for the large number of deaths
expected from the coronavirus.Credit...Justin Lane/EPA, via Shutterstock

The annual baseline back then was about 25 deaths per 1,000 New Yorkers,
and in some years the toll reached 50 per 1,000. In other words, in bad
years, New Yorkers saw twice as many people around them die as usual.
And they were used to seeing about four times as much death as we now
do.

The sharpest peaks were the cholera epidemics of 1832, 1849 and 1854.
But plagues came in waves, sometimes more than one simultaneously:
yellow fever, smallpox, measles, scarlet fever, diphtheria, typhus and
meningitis.

Other than cholera and typhus, most of those were childhood diseases
that adults were immune to because they had survived them, so the chart
is a parabola of parental grief, each spike another nail in a hundred
small coffins.

\href{https://www.nytimes3xbfgragh.onion/news-event/coronavirus?action=click\&pgtype=Article\&state=default\&region=MAIN_CONTENT_3\&context=storylines_faq}{}

\hypertarget{the-coronavirus-outbreak-}{%
\subsubsection{The Coronavirus Outbreak
›}\label{the-coronavirus-outbreak-}}

\hypertarget{frequently-asked-questions}{%
\paragraph{Frequently Asked
Questions}\label{frequently-asked-questions}}

Updated August 17, 2020

\begin{itemize}
\item ~
  \hypertarget{why-does-standing-six-feet-away-from-others-help}{%
  \paragraph{Why does standing six feet away from others
  help?}\label{why-does-standing-six-feet-away-from-others-help}}

  \begin{itemize}
  \tightlist
  \item
    The coronavirus spreads primarily through droplets from your mouth
    and nose, especially when you cough or sneeze. The C.D.C., one of
    the organizations using that measure,
    \href{https://www.nytimes3xbfgragh.onion/2020/04/14/health/coronavirus-six-feet.html?action=click\&pgtype=Article\&state=default\&region=MAIN_CONTENT_3\&context=storylines_faq}{bases
    its recommendation of six feet} on the idea that most large droplets
    that people expel when they cough or sneeze will fall to the ground
    within six feet. But six feet has never been a magic number that
    guarantees complete protection. Sneezes, for instance, can launch
    droplets a lot farther than six feet,
    \href{https://jamanetwork.com/journals/jama/fullarticle/2763852}{according
    to a recent study}. It's a rule of thumb: You should be safest
    standing six feet apart outside, especially when it's windy. But
    keep a mask on at all times, even when you think you're far enough
    apart.
  \end{itemize}
\item ~
  \hypertarget{i-have-antibodies-am-i-now-immune}{%
  \paragraph{I have antibodies. Am I now
  immune?}\label{i-have-antibodies-am-i-now-immune}}

  \begin{itemize}
  \tightlist
  \item
    As of right
    now,\href{https://www.nytimes3xbfgragh.onion/2020/07/22/health/covid-antibodies-herd-immunity.html?action=click\&pgtype=Article\&state=default\&region=MAIN_CONTENT_3\&context=storylines_faq}{that
    seems likely, for at least several months.} There have been
    frightening accounts of people suffering what seems to be a second
    bout of Covid-19. But experts say these patients may have a
    drawn-out course of infection, with the virus taking a slow toll
    weeks to months after initial exposure. People infected with the
    coronavirus typically
    \href{https://www.nature.com/articles/s41586-020-2456-9}{produce}
    immune molecules called antibodies, which are
    \href{https://www.nytimes3xbfgragh.onion/2020/05/07/health/coronavirus-antibody-prevalence.html?action=click\&pgtype=Article\&state=default\&region=MAIN_CONTENT_3\&context=storylines_faq}{protective
    proteins made in response to an
    infection}\href{https://www.nytimes3xbfgragh.onion/2020/05/07/health/coronavirus-antibody-prevalence.html?action=click\&pgtype=Article\&state=default\&region=MAIN_CONTENT_3\&context=storylines_faq}{.
    These antibodies may} last in the body
    \href{https://www.nature.com/articles/s41591-020-0965-6}{only two to
    three months}, which may seem worrisome, but that's perfectly normal
    after an acute infection subsides, said Dr. Michael Mina, an
    immunologist at Harvard University. It may be possible to get the
    coronavirus again, but it's highly unlikely that it would be
    possible in a short window of time from initial infection or make
    people sicker the second time.
  \end{itemize}
\item ~
  \hypertarget{im-a-small-business-owner-can-i-get-relief}{%
  \paragraph{I'm a small-business owner. Can I get
  relief?}\label{im-a-small-business-owner-can-i-get-relief}}

  \begin{itemize}
  \tightlist
  \item
    The
    \href{https://www.nytimes3xbfgragh.onion/article/small-business-loans-stimulus-grants-freelancers-coronavirus.html?action=click\&pgtype=Article\&state=default\&region=MAIN_CONTENT_3\&context=storylines_faq}{stimulus
    bills enacted in March} offer help for the millions of American
    small businesses. Those eligible for aid are businesses and
    nonprofit organizations with fewer than 500 workers, including sole
    proprietorships, independent contractors and freelancers. Some
    larger companies in some industries are also eligible. The help
    being offered, which is being managed by the Small Business
    Administration, includes the Paycheck Protection Program and the
    Economic Injury Disaster Loan program. But lots of folks have
    \href{https://www.nytimes3xbfgragh.onion/interactive/2020/05/07/business/small-business-loans-coronavirus.html?action=click\&pgtype=Article\&state=default\&region=MAIN_CONTENT_3\&context=storylines_faq}{not
    yet seen payouts.} Even those who have received help are confused:
    The rules are draconian, and some are stuck sitting on
    \href{https://www.nytimes3xbfgragh.onion/2020/05/02/business/economy/loans-coronavirus-small-business.html?action=click\&pgtype=Article\&state=default\&region=MAIN_CONTENT_3\&context=storylines_faq}{money
    they don't know how to use.} Many small-business owners are getting
    less than they expected or
    \href{https://www.nytimes3xbfgragh.onion/2020/06/10/business/Small-business-loans-ppp.html?action=click\&pgtype=Article\&state=default\&region=MAIN_CONTENT_3\&context=storylines_faq}{not
    hearing anything at all.}
  \end{itemize}
\item ~
  \hypertarget{what-are-my-rights-if-i-am-worried-about-going-back-to-work}{%
  \paragraph{What are my rights if I am worried about going back to
  work?}\label{what-are-my-rights-if-i-am-worried-about-going-back-to-work}}

  \begin{itemize}
  \tightlist
  \item
    Employers have to provide
    \href{https://www.osha.gov/SLTC/covid-19/standards.html}{a safe
    workplace} with policies that protect everyone equally.
    \href{https://www.nytimes3xbfgragh.onion/article/coronavirus-money-unemployment.html?action=click\&pgtype=Article\&state=default\&region=MAIN_CONTENT_3\&context=storylines_faq}{And
    if one of your co-workers tests positive for the coronavirus, the
    C.D.C.} has said that
    \href{https://www.cdc.gov/coronavirus/2019-ncov/community/guidance-business-response.html}{employers
    should tell their employees} -\/- without giving you the sick
    employee's name -\/- that they may have been exposed to the virus.
  \end{itemize}
\item ~
  \hypertarget{what-is-school-going-to-look-like-in-september}{%
  \paragraph{What is school going to look like in
  September?}\label{what-is-school-going-to-look-like-in-september}}

  \begin{itemize}
  \tightlist
  \item
    It is unlikely that many schools will return to a normal schedule
    this fall, requiring the grind of
    \href{https://www.nytimes3xbfgragh.onion/2020/06/05/us/coronavirus-education-lost-learning.html?action=click\&pgtype=Article\&state=default\&region=MAIN_CONTENT_3\&context=storylines_faq}{online
    learning},
    \href{https://www.nytimes3xbfgragh.onion/2020/05/29/us/coronavirus-child-care-centers.html?action=click\&pgtype=Article\&state=default\&region=MAIN_CONTENT_3\&context=storylines_faq}{makeshift
    child care} and
    \href{https://www.nytimes3xbfgragh.onion/2020/06/03/business/economy/coronavirus-working-women.html?action=click\&pgtype=Article\&state=default\&region=MAIN_CONTENT_3\&context=storylines_faq}{stunted
    workdays} to continue. California's two largest public school
    districts --- Los Angeles and San Diego --- said on July 13, that
    \href{https://www.nytimes3xbfgragh.onion/2020/07/13/us/lausd-san-diego-school-reopening.html?action=click\&pgtype=Article\&state=default\&region=MAIN_CONTENT_3\&context=storylines_faq}{instruction
    will be remote-only in the fall}, citing concerns that surging
    coronavirus infections in their areas pose too dire a risk for
    students and teachers. Together, the two districts enroll some
    825,000 students. They are the largest in the country so far to
    abandon plans for even a partial physical return to classrooms when
    they reopen in August. For other districts, the solution won't be an
    all-or-nothing approach.
    \href{https://bioethics.jhu.edu/research-and-outreach/projects/eschool-initiative/school-policy-tracker/}{Many
    systems}, including the nation's largest, New York City, are
    devising
    \href{https://www.nytimes3xbfgragh.onion/2020/06/26/us/coronavirus-schools-reopen-fall.html?action=click\&pgtype=Article\&state=default\&region=MAIN_CONTENT_3\&context=storylines_faq}{hybrid
    plans} that involve spending some days in classrooms and other days
    online. There's no national policy on this yet, so check with your
    municipal school system regularly to see what is happening in your
    community.
  \end{itemize}
\end{itemize}

The death rate began dropping after the 1860s. New Yorkers --- both
citizens and doctors --- had finally stopped arguing and reached
consensus on some basic issues.

First of all, most finally accepted the ``germ theory'' of disease,
acknowledging that it was caused by invisible enemies, not by swamps,
trash, manure or the other nuisances that underlay the ``miasma
theory,'' which held that bad smells caused disease. (Only a century
earlier, Americans had given up on the ``humors theory,'' which posited
that disease was caused by imbalances among blood, urine, sweat and bile
that had to be rebalanced by bleeding, sweating or purging.)

They also agreed that whether immigrants had brought some diseases or
simply suffered from them, no one was safe until everyone was safe, so
they made public health universal.

As a result, New Yorkers took certain steps --- sometimes very expensive
and contentious, but all based on science: They dug sewers to pipe filth
into the Hudson and East Rivers instead of letting it pool in the
streets. In 1842, they built the Croton Aqueduct to carry fresh water to
Manhattan. In 1910, they chlorinated its water to kill more germs. In
1912, they began requiring dairies to heat their milk because a
Frenchman named Louis Pasteur had shown that doing so spared children
from tuberculosis. Over time, they made smallpox vaccination mandatory.

Libertarians battled almost every step. Some fought sewers and water
mains being dug through their properties, arguing that they owned
perfectly good wells and cesspools. Some refused smallpox vaccines until
the Supreme Court put an end to that in 1905, in
\href{https://www.nytimes3xbfgragh.onion/2019/04/09/health/measles-outbreak-vaccinations-brooklyn.html}{Jacobson
v. Massachusetts}.

In the Spanish flu epidemic of 1918, many New Yorkers donned masks but
4,000 San Franciscans formed an Anti-Mask League. (The city's mayor,
James Rolph,
\href{https://www.kqed.org/arts/13878268/mindblowing-tales-from-san-franciscos-long-history-with-epidemics}{was
fined \$50} for flouting his own health department's mask order.)
Slowly, science prevailed, and death rates went down.

Today, Americans are facing the same choice our ancestors did: We can
listen to scientists and spend money to save lives, or we can watch our
neighbors die.

``The people who say `Let her rip, let's go for herd immunity' ---
that's just public-health nihilism,'' said Dr. Joia S. Mukherjee, the
chief medical office of Partners in Health, a medical charity fighting
the virus. ``How many deaths do we have to accept to get there?''

A vaccine may be close at hand, and so may treatments like monoclonal
antibodies that will cut our losses.

Till then, we need not accept death as our overlord --- we can simply
hang on and outlast him.

\emph{The Times is committed to publishing}
\href{https://www.nytimes3xbfgragh.onion/2019/01/31/opinion/letters/letters-to-editor-new-york-times-women.html}{\emph{a
diversity of letters}} \emph{to the editor. We'd like to hear what you
think about this or any of our articles. Here are some}
\href{https://help.nytimes3xbfgragh.onion/hc/en-us/articles/115014925288-How-to-submit-a-letter-to-the-editor}{\emph{tips}}\emph{.
And here's our email:}
\href{mailto:letters@NYTimes.com}{\emph{letters@NYTimes.com}}\emph{.}

\emph{Follow The New York Times Opinion section on}
\href{https://www.facebookcorewwwi.onion/nytopinion}{\emph{Facebook}}\emph{,}
\href{http://twitter.com/NYTOpinion}{\emph{Twitter (@NYTopinion)}}
\emph{and}
\href{https://www.instagram.com/nytopinion/}{\emph{Instagram}}\emph{.}

Advertisement

\protect\hyperlink{after-bottom}{Continue reading the main story}

\hypertarget{site-index}{%
\subsection{Site Index}\label{site-index}}

\hypertarget{site-information-navigation}{%
\subsection{Site Information
Navigation}\label{site-information-navigation}}

\begin{itemize}
\tightlist
\item
  \href{https://help.nytimes3xbfgragh.onion/hc/en-us/articles/115014792127-Copyright-notice}{©~2020~The
  New York Times Company}
\end{itemize}

\begin{itemize}
\tightlist
\item
  \href{https://www.nytco.com/}{NYTCo}
\item
  \href{https://help.nytimes3xbfgragh.onion/hc/en-us/articles/115015385887-Contact-Us}{Contact
  Us}
\item
  \href{https://www.nytco.com/careers/}{Work with us}
\item
  \href{https://nytmediakit.com/}{Advertise}
\item
  \href{http://www.tbrandstudio.com/}{T Brand Studio}
\item
  \href{https://www.nytimes3xbfgragh.onion/privacy/cookie-policy\#how-do-i-manage-trackers}{Your
  Ad Choices}
\item
  \href{https://www.nytimes3xbfgragh.onion/privacy}{Privacy}
\item
  \href{https://help.nytimes3xbfgragh.onion/hc/en-us/articles/115014893428-Terms-of-service}{Terms
  of Service}
\item
  \href{https://help.nytimes3xbfgragh.onion/hc/en-us/articles/115014893968-Terms-of-sale}{Terms
  of Sale}
\item
  \href{https://spiderbites.nytimes3xbfgragh.onion}{Site Map}
\item
  \href{https://help.nytimes3xbfgragh.onion/hc/en-us}{Help}
\item
  \href{https://www.nytimes3xbfgragh.onion/subscription?campaignId=37WXW}{Subscriptions}
\end{itemize}
