Sections

SEARCH

\protect\hyperlink{site-content}{Skip to
content}\protect\hyperlink{site-index}{Skip to site index}

\href{https://www.nytimes3xbfgragh.onion/section/technology}{Technology}

\href{https://myaccount.nytimes3xbfgragh.onion/auth/login?response_type=cookie\&client_id=vi}{}

\href{https://www.nytimes3xbfgragh.onion/section/todayspaper}{Today's
Paper}

\href{/section/technology}{Technology}\textbar{}Just Collect Less Data,
Period.

\href{https://nyti.ms/30dq0gn}{https://nyti.ms/30dq0gn}

\begin{itemize}
\item
\item
\item
\item
\item
\end{itemize}

Advertisement

\protect\hyperlink{after-top}{Continue reading the main story}

Supported by

\protect\hyperlink{after-sponsor}{Continue reading the main story}

on tech

\hypertarget{just-collect-less-data-period}{%
\section{Just Collect Less Data,
Period.}\label{just-collect-less-data-period}}

Every company wants the biggest data stockpile possible. We need
unilateral data disarmament.

\includegraphics{https://static01.graylady3jvrrxbe.onion/images/2020/07/15/business/15ontech/15ontech-articleLarge.png?quality=75\&auto=webp\&disable=upscale}

\href{https://www.nytimes3xbfgragh.onion/by/shira-ovide}{\includegraphics{https://static01.graylady3jvrrxbe.onion/images/2020/03/18/reader-center/author-shira-ovide/author-shira-ovide-thumbLarge-v2.png}}

By \href{https://www.nytimes3xbfgragh.onion/by/shira-ovide}{Shira Ovide}

\begin{itemize}
\item
  Published July 15, 2020Updated July 16, 2020
\item
  \begin{itemize}
  \item
  \item
  \item
  \item
  \item
  \end{itemize}
\end{itemize}

\emph{This article is part of the On Tech newsletter. You can}
\href{https://www.nytimes3xbfgragh.onion/newsletters/signup/OT}{\emph{sign
up here}} \emph{to receive it weekdays.}

Let's list some anxieties about digital life: We're being tracked all
the time. Internet superpowers hold sway over what information we see
and what we buy. Our sensitive information keeps getting
\href{https://www.nytimes3xbfgragh.onion/2019/07/22/business/equifax-settlement.html}{hacked}.

There's no simple fix for these complex worries. But there's a step that
works toward addressing them all: Americans deserve a national limit on
the information companies collect about us.

If you want to focus on one broad approach to tackle many of our
internet horribles, remember this motto: Just collect less data, period.

Companies want to harvest as much data about us as possible because ---
well, why wouldn't they? More information could help them target
advertisements at us,
\href{https://www.nytimes3xbfgragh.onion/2019/03/10/business/retail-stores-technology.html}{track
high-traffic areas in stores} or show us more dog videos to keep us on
their site longer.

For the companies, there's no downside to limitless data collection, and
there's little to prevent them from doing so in the United States.

Why should we care? Fair question. (My Opinion section colleagues
\href{https://www.nytimes3xbfgragh.onion/2019/12/26/reader-center/location-tracking-phones-questions.html}{tackled
this} in their
\href{https://www.nytimes3xbfgragh.onion/interactive/2019/opinion/internet-privacy-project.html}{Privacy
Project} series.) Personally, I feel icky knowing that
\href{https://www.theverge.com/2019/7/19/20700866/steve-bannon-location-data-carriers-tmobile-att-verizon}{political
campaigns can buy data showing who attends church,} and that the
Internal Revenue Service bought
\href{https://www.wsj.com/articles/irs-used-cellphone-location-data-to-try-to-find-suspects-11592587815}{bulk
records of people's locations} to (ineffectively) hunt for financial
criminals. When there's an arms race for our personal data, we lose
control over where our information winds up and how it's used.

And I can't shake what my colleague
\href{https://www.nytimes3xbfgragh.onion/by/paul-mozur}{Paul Mozur} said
\href{https://www.nytimes3xbfgragh.onion/2020/06/12/technology/surveillance-protests-hong-kong.html}{about
digital surveillance in Hong Kong.} For people to feel free, he told me,
we need to know that we're not always being watched. On some level, I'd
bet that goes for Facebook and
\href{https://www.washingtonpost.com/technology/2019/09/18/you-watch-tv-your-tv-watches-back/}{our
television sets} just as it does for governments.

Data also consolidates power. If you worry about Google, Facebook and
Amazon having too much influence, you should be aware that at the root
of their power is control over reams of information on where we go, what
we do and what we like.

Competitors then do icky things to play catch up, like buying
information
\href{https://www.nytimes3xbfgragh.onion/interactive/2018/12/10/business/location-data-privacy-apps.html}{from
data-harvesting companies}. The digital economy is a game of data
intrusion one-upmanship.

When we and elected officials try to fight this, our efforts are often
too myopic. Data privacy regulation and legislation has focused on
requiring disclosure
\href{https://www.nytimes3xbfgragh.onion/2017/10/19/us/politics/facebook-google-russia-meddling-disclosure.html}{of
data-targeted political advertisements}, making privacy policies more
clear, or forcing companies to
\href{https://www.nytimes3xbfgragh.onion/2020/01/03/us/ccpa-california-privacy-law.html}{show
their digital dossiers on us}.

Fine, those are pragmatic steps. Better still is to step back to the
underlying problem: All companies collect too much data about us in the
first place.

I recognize that the devil is in the details, and I'm not offering that.
(I'll work on it. My Opinion colleagues
\href{https://www.nytimes3xbfgragh.onion/interactive/2019/12/21/opinion/location-data-privacy-rights.html}{had
suggestions} for lawmakers and regulators.) Senator Sherrod Brown, a
Democrat from Ohio, released a
\href{https://www.brown.senate.gov/newsroom/press/release/brown-proposal-protect-consumers-privacy}{draft
privacy bill} last month that proposed companies collect information
only when
\href{https://www.washingtonpost.com/technology/2020/06/18/data-privacy-law-sherrod-brown/}{it's
``strictly necessary.''} Senator Josh Hawley, a Republican from
Missouri,
\href{https://www.congress.gov/bill/116th-congress/senate-bill/1578/text}{introduced
a relatively similar proposal} last year.

They weren't the first to try for a sweeping federal privacy law, and
they probably won't be the last to fail. But I'm glad they're trying to
coalesce us around a big idea: Unrestricted harvesting of personal data
has gone too far.

\begin{center}\rule{0.5\linewidth}{\linethickness}\end{center}

\hypertarget{in-praise-of-unflashy-technology}{%
\subsection{In praise of unflashy
technology}\label{in-praise-of-unflashy-technology}}

While I was reporting an
\href{https://www.bloomberg.com/news/features/2019-06-07/the-next-big-phones-could-bring-a-billion-people-online?sref=7ooTCNG1}{article}
last year about expanding internet access in parts of Africa, no one
wanted to talk about
\href{https://www.nytimes3xbfgragh.onion/2020/07/07/world/africa/google-loon-balloon-kenya.html}{high-altitude
balloons that transmit the internet}.

Instead, people couldn't stop praising
\href{https://carrier.huawei.com/en/trends-and-insights/emsite/ruralstar}{metal
poles} that made it far easier and cheaper to bring internet connections
to places where conventional cellphone towers weren't a good option.

It was a useful reminder to me that sometimes the biggest innovations
are the ones we never notice.

David M. Perry, a journalist and academic adviser at the University of
Minnesota,
\href{https://www.nytimes3xbfgragh.onion/2020/07/14/style/assistive-technology.html}{wrote}
in The New York Times this week about how spell-check, smartphone voice
control and other technology we take for granted can be life changing
for people with disabilities.

``Disability technology can be so quotidian that nondisabled users don't
even notice,'' Perry wrote.

To be fair, things like voice-activated helpers were marvels when they
were first introduced, and now we've gotten used to them. And we do
still want people to think big. Modern smartphones are an example of a
flashy technology that really did change everything.

But what Perry highlighted is that we sometimes fixate too much on
big-bang technologies that turn out to be impractical ---
\href{https://www.nytimes3xbfgragh.onion/2020/05/13/technology/driverless-cars.html}{driverless
cars}, to pick one example --- or are trying to solve problems that
people don't really have, at the expense of lower-tech ideas that can be
magical.

Perry said, for example, that there are constantly ideas to replace the
white canes used by people with vision impairments, and blind people
think the canes are great as is. (Here are some more
\href{https://www.nytimes3xbfgragh.onion/2020/07/08/technology/internet-infrastructure.html}{boring
but important technologies}.)

Remember that when you get excited about
\href{https://www.nytimes3xbfgragh.onion/2020/06/17/us/google-wing-drones-virginia-books.html}{drones
that deliver library books},
\href{https://arstechnica.com/information-technology/2020/07/amazon-supermarket-will-roll-out-fancy-new-smart-cart-when-it-opens/}{all-seeing
shopping carts} or whatever flashy thing that makes us think, ``Cool!''
Sometimes the stuff that draws the most attention will never work, and
the seemingly simple stuff has the biggest impact.

\begin{center}\rule{0.5\linewidth}{\linethickness}\end{center}

\hypertarget{before-we-go-}{%
\subsection{Before we go \ldots{}}\label{before-we-go-}}

\begin{itemize}
\item
  \textbf{Apple scores a legal (and financial) win:} A European court
  \href{https://www.nytimes3xbfgragh.onion/2020/07/15/business/apple-eu-ireland-tax.html}{overruled
  an order} that would have forced the company to pay \$14.9 billion in
  unpaid taxes, the Times tech reporter
  \href{https://www.nytimes3xbfgragh.onion/by/adam-satariano}{Adam
  Satariano} reported. In 2016, a European Union regulator said Apple
  had made illegal deals with Ireland to keep its tax bill low.
\item
  \textbf{Memes for a cause in Iran:} People in Iran are
  \href{https://www.nytimes3xbfgragh.onion/2020/07/15/world/middleeast/iran-protests-capital-punishment.html}{coalescing
  around the hashtag \#DontExecute and other social media messages}to
  push for an end to government executions based on murky charges, from
  drinking alcohol to political activism. My colleague
  \href{https://www.nytimes3xbfgragh.onion/by/farnaz-fassihi}{Farnaz
  Fassihi} wrote that it was a ``rare moment of solidarity among
  Iranians of varying political views around a single issue'' in a
  country where the government has brutally crushed other forms of
  dissent.
\item
  \textbf{Why does that textbook cost \$24 million?} Fortune
  \href{https://fortune.com/2020/07/14/wayfair-cabinet-conspiracy-algorithm-amazon-pricing-ecommerce/}{digs
  into the oddities of computerized pricing} to explain why a pillow
  might list for more than \$10,000 on Amazon. Among the culprits are
  computers programmed to respond to price changes of competing products
  --- with sometimes illogical results --- and merchants setting
  artificially high prices so you won't try to buy something.
\end{itemize}

\hypertarget{hugs-to-this}{%
\subsubsection{Hugs to this}\label{hugs-to-this}}

A simple but beautiful idea: \href{https://window-swap.com}{This
website} drops you in on a view from someone else's window.

\begin{center}\rule{0.5\linewidth}{\linethickness}\end{center}

\emph{We want to hear from you. Tell us what you think of this
newsletter and what else you'd like us to explore. You can reach us at}
\href{mailto:ontech@NYTimes.com?subject=On\%20Tech\%20Feedback}{\emph{ontech@NYTimes.com.}}
**

\emph{If you don't already get this newsletter in your inbox,}
\href{https://www.nytimes3xbfgragh.onion/newsletters/signup/OT}{\emph{please
sign up here}}\emph{.}

Advertisement

\protect\hyperlink{after-bottom}{Continue reading the main story}

\hypertarget{site-index}{%
\subsection{Site Index}\label{site-index}}

\hypertarget{site-information-navigation}{%
\subsection{Site Information
Navigation}\label{site-information-navigation}}

\begin{itemize}
\tightlist
\item
  \href{https://help.nytimes3xbfgragh.onion/hc/en-us/articles/115014792127-Copyright-notice}{©~2020~The
  New York Times Company}
\end{itemize}

\begin{itemize}
\tightlist
\item
  \href{https://www.nytco.com/}{NYTCo}
\item
  \href{https://help.nytimes3xbfgragh.onion/hc/en-us/articles/115015385887-Contact-Us}{Contact
  Us}
\item
  \href{https://www.nytco.com/careers/}{Work with us}
\item
  \href{https://nytmediakit.com/}{Advertise}
\item
  \href{http://www.tbrandstudio.com/}{T Brand Studio}
\item
  \href{https://www.nytimes3xbfgragh.onion/privacy/cookie-policy\#how-do-i-manage-trackers}{Your
  Ad Choices}
\item
  \href{https://www.nytimes3xbfgragh.onion/privacy}{Privacy}
\item
  \href{https://help.nytimes3xbfgragh.onion/hc/en-us/articles/115014893428-Terms-of-service}{Terms
  of Service}
\item
  \href{https://help.nytimes3xbfgragh.onion/hc/en-us/articles/115014893968-Terms-of-sale}{Terms
  of Sale}
\item
  \href{https://spiderbites.nytimes3xbfgragh.onion}{Site Map}
\item
  \href{https://help.nytimes3xbfgragh.onion/hc/en-us}{Help}
\item
  \href{https://www.nytimes3xbfgragh.onion/subscription?campaignId=37WXW}{Subscriptions}
\end{itemize}
