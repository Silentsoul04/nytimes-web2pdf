Sections

SEARCH

\protect\hyperlink{site-content}{Skip to
content}\protect\hyperlink{site-index}{Skip to site index}

\href{https://www.nytimes3xbfgragh.onion/section/technology/personaltech}{Personal
Tech}

\href{https://myaccount.nytimes3xbfgragh.onion/auth/login?response_type=cookie\&client_id=vi}{}

\href{https://www.nytimes3xbfgragh.onion/section/todayspaper}{Today's
Paper}

\href{/section/technology/personaltech}{Personal Tech}\textbar{}You're
Doomscrolling Again. Here's How to Snap Out of It.

\url{https://nyti.ms/3j0HMfH}

\begin{itemize}
\item
\item
\item
\item
\item
\item
\end{itemize}

\href{https://www.nytimes3xbfgragh.onion/spotlight/at-home?action=click\&pgtype=Article\&state=default\&region=TOP_BANNER\&context=at_home_menu}{At
Home}

\begin{itemize}
\tightlist
\item
  \href{https://www.nytimes3xbfgragh.onion/2020/07/28/books/time-for-a-literary-road-trip.html?action=click\&pgtype=Article\&state=default\&region=TOP_BANNER\&context=at_home_menu}{Take:
  A Literary Road Trip}
\item
  \href{https://www.nytimes3xbfgragh.onion/2020/07/29/magazine/bored-with-your-home-cooking-some-smoky-eggplant-will-fix-that.html?action=click\&pgtype=Article\&state=default\&region=TOP_BANNER\&context=at_home_menu}{Cook:
  Smoky Eggplant}
\item
  \href{https://www.nytimes3xbfgragh.onion/2020/07/27/travel/moose-michigan-isle-royale.html?action=click\&pgtype=Article\&state=default\&region=TOP_BANNER\&context=at_home_menu}{Look
  Out: For Moose}
\item
  \href{https://www.nytimes3xbfgragh.onion/interactive/2020/at-home/even-more-reporters-editors-diaries-lists-recommendations.html?action=click\&pgtype=Article\&state=default\&region=TOP_BANNER\&context=at_home_menu}{Explore:
  Reporters' Obsessions}
\end{itemize}

Advertisement

\protect\hyperlink{after-top}{Continue reading the main story}

Supported by

\protect\hyperlink{after-sponsor}{Continue reading the main story}

Tech Fix

\hypertarget{youre-doomscrolling-again-heres-how-to-snap-out-of-it}{%
\section{You're Doomscrolling Again. Here's How to Snap Out of
It.}\label{youre-doomscrolling-again-heres-how-to-snap-out-of-it}}

In a pandemic that forces us to stay home, bingeing on doom-and-gloom
news feels irresistible. These health experts offer ways to break the
habit.

\includegraphics{https://static01.graylady3jvrrxbe.onion/images/2020/07/16/business/16Techfix-illo/15Techfix-illo-articleLarge.gif?quality=75\&auto=webp\&disable=upscale}

\href{https://www.nytimes3xbfgragh.onion/by/brian-x-chen}{\includegraphics{https://static01.graylady3jvrrxbe.onion/images/2018/02/16/multimedia/author-brian-x-chen/author-brian-x-chen-thumbLarge.jpg}}

By \href{https://www.nytimes3xbfgragh.onion/by/brian-x-chen}{Brian X.
Chen}

\begin{itemize}
\item
  July 15, 2020
\item
  \begin{itemize}
  \item
  \item
  \item
  \item
  \item
  \item
  \end{itemize}
\end{itemize}

Your phone alarm goes off at 6 in the morning. You check some news sites
and Facebook. It's bad news after bad news. Coronavirus cases keep
climbing, and so do deaths. Children can't go back to school. Your
favorite restaurant and barbershop are still closed. People are losing
their jobs.

Everything is awful. The world as we remember it has ended. Next thing
you know, it's 9 a.m. You haven't climbed out of your pit of despair yet
to even shower. You repeat this masochistic exercise during your lunch
break --- and again while getting ready for bed.

This experience of sinking into emotional quicksand while bingeing on
doom-and-gloom news is so common that there's now internet lingo for it:
``\href{https://www.merriam-webster.com/words-at-play/doomsurfing-doomscrolling-words-were-watching}{doomscrolling.}''
Exacerbating this behavior, shelter-in-place orders leave us with little
to do other than to look at our screens; by some measures, our
\href{https://www.axios.com/kids-screen-time-coronavirus-562073f6-0638-47f2-8ea3-4f8781d6b31b.html}{screen
time has jumped at least 50 percent}.

We're not alone, exactly, with so many of us going through this. Yet
doomscrolling, combined with screen addiction, could take a significant
toll on our mental and physical well-being, according to health experts.
The activity can make us angry, anxious, depressed, unproductive and
less connected with our loved ones and ourselves.

``It's the path of least resistance to keep consuming passively through
social media,'' said Dr. Vivek Murthy, the former surgeon general, who
has written extensively about the impact of loneliness on personal
health. ``You have to pull yourself out of that. It's not just
disengaging but also dealing with the impact that has on your mind-set,
which can often last for hours.''

Fret not: We aren't doomed just yet, and there are approaches to
modifying our behavior. We can create structure in our lives, for one,
and practice meditation techniques, for another. Here's what the health
and wellness experts say.

\hypertarget{create-a-plan-to-control-your-time}{%
\subsection{Create a plan to control your
time}\label{create-a-plan-to-control-your-time}}

People are, by nature, information consumers, and the news is like
digital candy being dispensed 24 hours a day. To resist information
bingeing, we can create a plan to control how much we consume, similar
to how people can create a dieting plan to lose weight, said Adam
Gazzaley, a neuroscientist and co-author of the book
``\href{https://mitpress.mit.edu/books/distracted-mind}{The Distracted
Mind: Ancient Brains in a High-Tech World}.''

Step 1 is to acknowledge the burden that doomscrolling creates for our
health, Dr. Gazzaley said. ``You have to realize you don't want to live
your life in a hamster wheel of complete news consumption,'' he said.
``It'll take a toll on you in the way that stops becoming valuable, and
being an informed person is a diminishing return.''

Step 2 is to create a realistic plan that you can stick with and repeat
until it forms a habit.

Creating a schedule is an effective approach. Start by making calendar
appointments for everything from mundane activities, like taking a walk
outside, to business matters, like videoconferencing meetings.

Set aside certain times of the day to read the news, if you must --- and
if it helps, set a 10-minute timer to remind you to stop scrolling.
Another trick is to wear a rubber band around your hand while you are
reading the news, and when you believe you are succumbing to
doomscrolling, snap the rubber band against your wrist, Dr. Murthy said.

It's also important to rethink breaks. Before the pandemic, one of our
typical lunch breaks involved browsing Facebook. With nowhere to go out
for lunch under shelter-in-place orders, browsing the web has become the
default work break, an obvious trap that could lead to doomscrolling.

Instead of staying glued to a screen, take a stroll around the block,
hop on the exercise bike, prepare your favorite snack. And, yes, set
calendar appointments even for your breaks, Dr. Gazzaley said.

\hypertarget{practice-meditation}{%
\subsection{Practice meditation}\label{practice-meditation}}

Exercises in mindfulness can help us break the cycle of information
bingeing or prevent us from sinking into a dark place altogether.

Sharon Salzberg, a meditation teacher and author of the book
``\href{https://www.sharonsalzberg.com/realchange/}{Real Change:
Mindfulness to Heal Ourselves and the World,}'' recommended this
exercise to feel more connected with others in a time when we can't see
many people:

\begin{itemize}
\item
  Take some breaths and think about the people who have helped you in
  the past. This could be your friends, colleagues and even the
  restaurant workers bagging your takeout food.
\item
  While imagining these people, give them positive wishes. For example:
  ``May you be happy. May you be peaceful. May you be safe. May you be
  healthy.''
\end{itemize}

``You're gift-giving,'' Ms. Salzberg said. ``It's a different way of
relating and not feeling isolation.''

\hypertarget{connect-with-others}{%
\subsection{Connect with others}\label{connect-with-others}}

Dr. Murthy's book
``\href{http://www.harperwave.com/book/9780062913296/Together-Vivek-H.-Murthy-MD/}{Together:
The Healing Power of Human Connection in a Sometimes Lonely World}''
underlined the importance of spending 15 minutes a day connecting with
the people we care about most. That can help us feel less alone and
resist doomscrolling.

But how can we connect with people when we can't easily see them? In the
beginning of the pandemic, many of us turned to videoconferencing apps
to virtually connect with friends, colleagues and loved ones. Now, more
than four months into the pandemic, many are experiencing
``\href{https://www.nytimes3xbfgragh.onion/2020/05/20/smarter-living/coronavirus-zoom-facetime-fatigue.html}{Zoom
fatigue.}''

Dr. Murthy said he, too, was getting tired of the neck strain from
constant video calls and had begun shifting many work and personal calls
to the phone while taking a walk, which lifts his energy and helps him
stay focused.

Dr. Murthy also recommended that people try to form a ``moai,'' a
Japanese word for a social support group. This could be a small group of
friends who regularly convene --- on the phone, in video chat or in
person at a safe distance --- and agree to look out for one another. He
and two friends formed a moai, and, once a month, they spend two hours
catching up in a frank conversation about personal issues related to
health, relationships and finances.

Changing behavior can be tough to do on your own. So you could even tell
your moai that you want to stop doomscrolling, and they could hold you
accountable. Dr. Murthy said that his moai conversation with his friends
was coming up, and that he planned to talk about having a cleaner
relationship with social media --- because he, too, occasionally gives
in to doomscrolling.

``The idea of carving time out for people you care about, whether it's
15 minutes or more, is all the more important in a world where the lines
between day and night, weekday and weekend, have been erased,'' he said.

Advertisement

\protect\hyperlink{after-bottom}{Continue reading the main story}

\hypertarget{site-index}{%
\subsection{Site Index}\label{site-index}}

\hypertarget{site-information-navigation}{%
\subsection{Site Information
Navigation}\label{site-information-navigation}}

\begin{itemize}
\tightlist
\item
  \href{https://help.nytimes3xbfgragh.onion/hc/en-us/articles/115014792127-Copyright-notice}{©~2020~The
  New York Times Company}
\end{itemize}

\begin{itemize}
\tightlist
\item
  \href{https://www.nytco.com/}{NYTCo}
\item
  \href{https://help.nytimes3xbfgragh.onion/hc/en-us/articles/115015385887-Contact-Us}{Contact
  Us}
\item
  \href{https://www.nytco.com/careers/}{Work with us}
\item
  \href{https://nytmediakit.com/}{Advertise}
\item
  \href{http://www.tbrandstudio.com/}{T Brand Studio}
\item
  \href{https://www.nytimes3xbfgragh.onion/privacy/cookie-policy\#how-do-i-manage-trackers}{Your
  Ad Choices}
\item
  \href{https://www.nytimes3xbfgragh.onion/privacy}{Privacy}
\item
  \href{https://help.nytimes3xbfgragh.onion/hc/en-us/articles/115014893428-Terms-of-service}{Terms
  of Service}
\item
  \href{https://help.nytimes3xbfgragh.onion/hc/en-us/articles/115014893968-Terms-of-sale}{Terms
  of Sale}
\item
  \href{https://spiderbites.nytimes3xbfgragh.onion}{Site Map}
\item
  \href{https://help.nytimes3xbfgragh.onion/hc/en-us}{Help}
\item
  \href{https://www.nytimes3xbfgragh.onion/subscription?campaignId=37WXW}{Subscriptions}
\end{itemize}
