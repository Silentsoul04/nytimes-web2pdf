Sections

SEARCH

\protect\hyperlink{site-content}{Skip to
content}\protect\hyperlink{site-index}{Skip to site index}

\href{https://www.nytimes3xbfgragh.onion/section/business}{Business}

\href{https://myaccount.nytimes3xbfgragh.onion/auth/login?response_type=cookie\&client_id=vi}{}

\href{https://www.nytimes3xbfgragh.onion/section/todayspaper}{Today's
Paper}

\href{/section/business}{Business}\textbar{}Letting Your Insurer Ride
Shotgun, for a Discounted Rate

\url{https://nyti.ms/32miIto}

\begin{itemize}
\item
\item
\item
\item
\item
\item
\end{itemize}

Advertisement

\protect\hyperlink{after-top}{Continue reading the main story}

Supported by

\protect\hyperlink{after-sponsor}{Continue reading the main story}

Wheels

\hypertarget{letting-your-insurer-ride-shotgun-for-a-discounted-rate}{%
\section{Letting Your Insurer Ride Shotgun, for a Discounted
Rate}\label{letting-your-insurer-ride-shotgun-for-a-discounted-rate}}

Most big car insurers offer apps that monitor your driving, and one
start-up requires it. The trade-off in privacy is a premium that could
be substantially cheaper for safe drivers.

\includegraphics{https://static01.graylady3jvrrxbe.onion/images/2020/07/17/business/17WHEELS1/merlin_146064276_97aea028-0fa5-44a4-8ad0-7a642ae3199a-articleLarge.jpg?quality=75\&auto=webp\&disable=upscale}

By Paul Stenquist

\begin{itemize}
\item
  July 16, 2020
\item
  \begin{itemize}
  \item
  \item
  \item
  \item
  \item
  \item
  \end{itemize}
\end{itemize}

It's like the Elf on the Shelf, but for car insurance. Call it the mole
on the console.

Drivers willing to give up a little privacy are becoming familiar with a
multitude of apps that track their behavior behind the wheel, and in
exchange are getting decent discounts on their premiums --- as long as
they lay off the gas and don't brake too hard too often.

In a
\href{https://www.ispot.tv/ad/ZAGx/state-farm-floor-it-featuring-aaron-rodgers-david-haydn-jones}{television
commercial} for one offering, the State Farm pitchman and pro
quarterback Aaron Rodgers exclaims, ``Don't mess with my discount,''
while a fictional sports agent tries to goad him into driving recklessly
and ``Breaking the Law'' by Judas Priest blares from the car's speakers.
Mr. Rodgers's insurance agent --- his ``other agent'' --- had previously
explained that the safer he drove, the more money he would save, with
the Drive Safe \& Save app.

Rival usage-based programs, offered where regulations allow, include
Signal from Farmers, Snapshot from Progressive, Drivewise from Allstate,
RightTrack from Liberty Mutual, DriveEasy from GEICO and SmartRide from
Nationwide. Root Insurance, a recent start-up, does not offer
traditional policies --- its drivers must be willing to be monitored.
Roughly 10 to 20 percent of customers choose these programs; the figures
vary by state.

The data recorded by the apps can include braking, acceleration, speed,
miles driven and cellphone use while driving. The companies vary only
slightly in determining what makes a safe driver. Proponents of this
insurance play up the potential savings, the financial motivation to
drive with care and the emissions reductions resulting from fewer miles
driven. Critics of such programs call the apps ``the spy in your car.''

Most of the programs use a smartphone's global positioning sensor,
accelerometer, gyroscope and magnetometer --- essentially a compass.
State Farm uses a Bluetooth device to activate the app when the car door
is opened.

Other programs use a tracking device plugged into the car's OBD-II
diagnostic port. Some offer a choice of using your smartphone or
installing an OBD-II device. Still others gather data through General
Motors' OnStar telemetric system.

Most programs provide a 10 percent discount against the price of a
standard policy on sign-up and reward good driving with additional
discounts. Several states limit the initial discount to 5 percent or
less.

Image

During a 14-mile test of Farmers' Signal app, a light turned yellow as
the driver approached an intersection, earning a demerit for hard
braking.

Some programs will rescind the initial discount and may even raise rates
if drivers can't keep their foot off the throttle or rack up a lot of
mileage, use their cellphone while driving, frequently brake hard or
commit other violations. But for most drivers, this usage-based
insurance is less expensive than a conventional policy, and for cautious
drivers who don't often drive, it can be considerably less expensive.

Real savings vary depending on driver performance and the specifics of
each program, and they range from that initial 10 percent discount to
about 50 percent. Allstate reports that 50 to 60 percent of all drivers
in its Drivewise program earn safe-driving benefits beyond the initial
discount.

Scott Bruns, a State Farm director, said: ``The average discount is
between 10 and 15 percent, but we regularly see customers receiving 20
to 30 percent discounts. The maximum is 50 percent.''

Insurers indicated that the discounts were supported by an improvement
in what is known as loss ratio --- claims paid and adjustment expenses
vs. premiums earned --- thanks to more cautious driving by people who
use the apps.

``When customers use SmartRide, they earn an average discount of 21
percent,'' said Teresa Scharn, an associate vice president at
Nationwide. ``That's directly correlated to the loss ratio improvement
of 21 percent we see. We're giving the 21 percent back to the
customer.''

Root Insurance's policies work somewhat differently. Root sets a rate
after an app records driving data over a 30-day trial period, then
maintains it --- generally for the life of the relationship. About 15
percent of drivers who complete the 30-day trial are denied coverage.
The rest are billed considerably less than what a conventional policy
might cost.

``For the best drivers, we can reduce their insurance rate by 52
percent,'' said Dan Manges, a co-founder of Root and its chief
technology officer.

These discounts are not available in the state perhaps best known for
its commutes and traffic: California, whose insurance department bars
the use of driving for setting rates. Privacy is a chief concern.

For now, most of these programs are optional, but the day may come when
insurance companies will be watching every driver on the road and basing
their billing on those observations.

``Like it or not, the world is going to that,'' said Jeffrey Lake, a
Farmers agent in Grand Rapids, Mich. He said he expected that most auto
insurance policies would be usage-based in 10 to 15 years.

Ginger Purgatorio, an Allstate senior vice president, agreed in
principle. Noting that Allstate's Drivewise is nearly 10 years old, she
said the company would continue urging customers to participate.

Progressive first offered usage-based insurance pricing in 1998. At that
time, it rewarded only safe drivers. Subsequently, it began using the
data to raise the rates of those who drove aggressively. Most other
companies say they do not raise rates based on the data but can rescind
the original discount.

Drivers who might be considered high risk when applying for a
conventional policy, including teenagers, those with a less-than-perfect
driving record and those who live in neighborhoods considered high risk,
can save substantially if they demonstrate they're safe. Owners of
high-performance cars can mute the penalty that generally accompanies
insuring an ultrapowerful automobile if they drive with a featherweight
foot. On the other hand, if they pound the throttle, they could end up
paying substantially more than they might with a standard policy.

Image

Allstate's Drivewise app. Most insurers' apps will display your last
trip and reveal your safe-driving score.

Mr. Lake, asked if usage-based policies punished drivers of performance
cars who drove aggressively from time to time, said, ``Why should my
elderly dad subsidize your high-performance car?''

The way the programs analyze data can lead to faulty conclusions. For
example, the software considers hard braking a negative and counts it
against the driver's record, even when the stop is the result of an
urgent situation, such as a ball in the street or a stoplight.

The insurers claim hard braking is a good predictor of collisions, as it
is indicative of tailgating. Mr. Bruns said drivers were not penalized
for occasional hard-braking incidents. It's a pattern the companies are
looking for.

As for the privacy considerations. most of the programs report vehicle
location, so your insurer will know what you're up to. Those who don't
like to be monitored might be uneasy knowing their insurance company is
watching their every move behind the wheel. Of course, we've pretty much
sacrificed privacy in the age of the cellphone anyway.

Advertisement

\protect\hyperlink{after-bottom}{Continue reading the main story}

\hypertarget{site-index}{%
\subsection{Site Index}\label{site-index}}

\hypertarget{site-information-navigation}{%
\subsection{Site Information
Navigation}\label{site-information-navigation}}

\begin{itemize}
\tightlist
\item
  \href{https://help.nytimes3xbfgragh.onion/hc/en-us/articles/115014792127-Copyright-notice}{©~2020~The
  New York Times Company}
\end{itemize}

\begin{itemize}
\tightlist
\item
  \href{https://www.nytco.com/}{NYTCo}
\item
  \href{https://help.nytimes3xbfgragh.onion/hc/en-us/articles/115015385887-Contact-Us}{Contact
  Us}
\item
  \href{https://www.nytco.com/careers/}{Work with us}
\item
  \href{https://nytmediakit.com/}{Advertise}
\item
  \href{http://www.tbrandstudio.com/}{T Brand Studio}
\item
  \href{https://www.nytimes3xbfgragh.onion/privacy/cookie-policy\#how-do-i-manage-trackers}{Your
  Ad Choices}
\item
  \href{https://www.nytimes3xbfgragh.onion/privacy}{Privacy}
\item
  \href{https://help.nytimes3xbfgragh.onion/hc/en-us/articles/115014893428-Terms-of-service}{Terms
  of Service}
\item
  \href{https://help.nytimes3xbfgragh.onion/hc/en-us/articles/115014893968-Terms-of-sale}{Terms
  of Sale}
\item
  \href{https://spiderbites.nytimes3xbfgragh.onion}{Site Map}
\item
  \href{https://help.nytimes3xbfgragh.onion/hc/en-us}{Help}
\item
  \href{https://www.nytimes3xbfgragh.onion/subscription?campaignId=37WXW}{Subscriptions}
\end{itemize}
