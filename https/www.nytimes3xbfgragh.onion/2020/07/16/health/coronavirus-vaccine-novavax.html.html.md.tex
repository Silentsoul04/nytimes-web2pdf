Sections

SEARCH

\protect\hyperlink{site-content}{Skip to
content}\protect\hyperlink{site-index}{Skip to site index}

\href{https://www.nytimes3xbfgragh.onion/section/health}{Health}

\href{https://myaccount.nytimes3xbfgragh.onion/auth/login?response_type=cookie\&client_id=vi}{}

\href{https://www.nytimes3xbfgragh.onion/section/todayspaper}{Today's
Paper}

\href{/section/health}{Health}\textbar{}How a Struggling Company Won
\$1.6 Billion to Make a Coronavirus Vaccine

\url{https://nyti.ms/2ChPu4q}

\begin{itemize}
\item
\item
\item
\item
\item
\item
\end{itemize}

\hypertarget{the-coronavirus-outbreak}{%
\subsubsection{\texorpdfstring{\href{https://www.nytimes3xbfgragh.onion/news-event/coronavirus?name=styln-coronavirus-national\&region=TOP_BANNER\&variant=undefined\&block=storyline_menu_recirc\&action=click\&pgtype=Article\&impression_id=d84f7cc0-e39d-11ea-a4ee-ebb6d55f184c}{The
Coronavirus
Outbreak}}{The Coronavirus Outbreak}}\label{the-coronavirus-outbreak}}

\begin{itemize}
\tightlist
\item
  live\href{https://www.nytimes3xbfgragh.onion/2020/08/21/world/covid-19-coronavirus.html?name=styln-coronavirus-national\&region=TOP_BANNER\&variant=undefined\&block=storyline_menu_recirc\&action=click\&pgtype=Article\&impression_id=d84f7cc1-e39d-11ea-a4ee-ebb6d55f184c}{Latest
  Updates}
\item
  \href{https://www.nytimes3xbfgragh.onion/interactive/2020/us/coronavirus-us-cases.html?name=styln-coronavirus-national\&region=TOP_BANNER\&variant=undefined\&block=storyline_menu_recirc\&action=click\&pgtype=Article\&impression_id=d84fa3d0-e39d-11ea-a4ee-ebb6d55f184c}{Maps
  and Cases}
\item
  \href{https://www.nytimes3xbfgragh.onion/interactive/2020/science/coronavirus-vaccine-tracker.html?name=styln-coronavirus-national\&region=TOP_BANNER\&variant=undefined\&block=storyline_menu_recirc\&action=click\&pgtype=Article\&impression_id=d84fa3d1-e39d-11ea-a4ee-ebb6d55f184c}{Vaccine
  Tracker}
\item
  \href{https://www.nytimes3xbfgragh.onion/2020/08/19/us/colleges-closing-covid.html?name=styln-coronavirus-national\&region=TOP_BANNER\&variant=undefined\&block=storyline_menu_recirc\&action=click\&pgtype=Article\&impression_id=d84fa3d2-e39d-11ea-a4ee-ebb6d55f184c}{Colleges
  Closing}
\item
  \href{https://www.nytimes3xbfgragh.onion/live/2020/08/20/business/stock-market-today-coronavirus?name=styln-coronavirus-national\&region=TOP_BANNER\&variant=undefined\&block=storyline_menu_recirc\&action=click\&pgtype=Article\&impression_id=d84fa3d3-e39d-11ea-a4ee-ebb6d55f184c}{Economy}
\end{itemize}

Advertisement

\protect\hyperlink{after-top}{Continue reading the main story}

Supported by

\protect\hyperlink{after-sponsor}{Continue reading the main story}

\hypertarget{how-a-struggling-company-won-16-billion-to-make-a-coronavirus-vaccine}{%
\section{How a Struggling Company Won \$1.6 Billion to Make a
Coronavirus
Vaccine}\label{how-a-struggling-company-won-16-billion-to-make-a-coronavirus-vaccine}}

Novavax just received the Trump administration's largest vaccine
contract. In the Maryland company's 33-year history, it has never
brought a vaccine to market.

\includegraphics{https://static01.graylady3jvrrxbe.onion/images/2020/07/15/science/15VIRUS-NOVAVAX1/15VIRUS-NOVAVAX1-articleLarge.jpg?quality=75\&auto=webp\&disable=upscale}

By \href{https://www.nytimes3xbfgragh.onion/by/katie-thomas}{Katie
Thomas} and
\href{https://www.nytimes3xbfgragh.onion/by/megan-twohey}{Megan Twohey}

\begin{itemize}
\item
  July 16, 2020
\item
  \begin{itemize}
  \item
  \item
  \item
  \item
  \item
  \item
  \end{itemize}
\end{itemize}

In late February, as the coronavirus spread around the world, Dr.
Richard Hatchett, the head of an international nonprofit that gives
money to vaccine developers, got on an important call to discuss vaccine
candidates after his plane touched down at London's Heathrow Airport.

Executives from the Bill \& Melinda Gates Foundation, which helped found
and finance the nonprofit, were on the line, enthusiastic about Novavax,
a small biotech company they thought had the potential to develop a
vaccine against the virus ---~fast.

Although the company, based in Gaithersburg, Md., had never brought a
vaccine to market in its 33-year history, these experts were optimistic
about its technology, which uses moth cells to pump out crucial
molecules at a much faster rate than typical vaccines --- a major
advantage in a pandemic.

Dr. Hatchett's organization, the Coalition for Epidemic Preparedness
Innovations, would go on to invest \$388 million in the company's
coronavirus vaccine. With that powerful backing, Novavax made an
aggressive push to the U.S. government. The company's effort paid off
last week when Operation Warp Speed, the Trump administration's effort
to hurry coronavirus vaccines to the market,
\href{https://www.nytimes3xbfgragh.onion/2020/07/07/health/novavax-coronavirus-vaccine-warp-speed.html}{gave
Novavax \$1.6 billion}, the largest award to date. The company's stock
surged 30 percent.

It was a dramatic turnaround for a little-known company that, just one
year earlier, had been on the verge of collapse. One of its leading
vaccine candidates --- to prevent a deadly virus in infants --- had
failed for the second time in three years. The company's stock was
trading so low that it risked being removed from the Nasdaq. Looking for
cash, it sold its manufacturing facilities. Word spread around the small
world of Maryland biotech that Novavax might be closing soon.

Novavax's good fortune may appear puzzling, given its track record and
the air of secrecy surrounding Operation Warp Speed. But for those in
the insular biotech world where connections matter, it is far less
surprising. In the face of a deadly pandemic that is devastating the
economy, the government is placing huge bets on vaccines and treatments
that could enable a return to some semblance of normal life.

The Trump administration has said it wants to invest in a variety of
vaccine technologies, and Novavax ---
\href{https://www.nytimes3xbfgragh.onion/interactive/2020/05/20/science/coronavirus-vaccine-development.html}{which
uses coronavirus proteins to provoke an immune response} --- offers an
approach that is distinct from those of other companies that have
already received major federal backing. Its method's potential to
quickly manufacture millions of doses was also attractive to the federal
government and Dr. Hatchett's organization. The success this spring of a
clinical trial of Novavax's flu vaccine boosted confidence in the
company.

``When the need is great, you have to be willing to take financial
risks,'' said Dr. Hatchett.

But skeptics see Novavax as a classic example of a second-tier player
that has survived by limping from crisis to crisis, boosting its stock
by promising vaccines for new outbreaks, yet never delivering. In its
three decades in business, with a mix of public and private investment,
it has developed experimental vaccines for viruses like SARS, MERS and
Ebola that never made it past early safety studies. It's telling,
critics say, that even as it has received growing amounts of government
and philanthropic support, the company's coronavirus vaccine effort has
not attracted any deals with major drug makers.

``The market wants to believe in fairy tales,'' said David Maris, the
managing partner of Phalanx Investment Partners and a longtime analyst
covering the pharmaceutical industry. He said investors wanted to
believe that --- like Cinderella --- the companies that couldn't go to
the ball would eventually win the prince.

``It sometimes happens,'' he said. ``Usually it doesn't.''

\includegraphics{https://static01.graylady3jvrrxbe.onion/images/2020/07/15/science/15VIRUS-NOVAVAX2/15VIRUS-NOVAVAX2-articleLarge.jpg?quality=75\&auto=webp\&disable=upscale}

So far, the federal government has promised nearly \$4 billion
\href{https://medicalcountermeasures.gov/app/barda/coronavirus/COVID19.aspx}{to
six vaccine projects}, but many aspects of the deals are confidential.
The Trump administration has only released heavily redacted copies of
**** its **** contracts with these companies.

When asked this week why Novavax has received more than anyone else, a
Trump administration official said **** that smaller companies needed
more federal investment in manufacturing compared to large
pharmaceutical firms, which have an established track record for
mass-producing vaccines. The \$1.6 billion comes from the Department of
Health and Human Services and the Defense Department, and will be used
to help develop and manufacture Novavax's vaccine.

In pursuing its contracts, Novavax drew on influential ties it has
cultivated in the federal government and close-knit global health
community, according to interviews with current and former company
executives, federal and global health officials, vaccine experts and
investment analysts.

\hypertarget{latest-updates-the-coronavirus-outbreak}{%
\section{\texorpdfstring{\href{https://www.nytimes3xbfgragh.onion/2020/08/21/world/covid-19-coronavirus.html?action=click\&pgtype=Article\&state=default\&region=MAIN_CONTENT_1\&context=storylines_live_updates}{Latest
Updates: The Coronavirus
Outbreak}}{Latest Updates: The Coronavirus Outbreak}}\label{latest-updates-the-coronavirus-outbreak}}

Updated 2020-08-21T11:01:03.794Z

\begin{itemize}
\tightlist
\item
  \href{https://www.nytimes3xbfgragh.onion/2020/08/21/world/covid-19-coronavirus.html?action=click\&pgtype=Article\&state=default\&region=MAIN_CONTENT_1\&context=storylines_live_updates\#link-4690b6aa}{Shutdowns,
  warnings and scoldings follow gatherings on college campuses.}
\item
  \href{https://www.nytimes3xbfgragh.onion/2020/08/21/world/covid-19-coronavirus.html?action=click\&pgtype=Article\&state=default\&region=MAIN_CONTENT_1\&context=storylines_live_updates\#link-324af071}{As
  he accepts the Democratic nomination, Biden knocks Trump's pandemic
  response.}
\item
  \href{https://www.nytimes3xbfgragh.onion/2020/08/21/world/covid-19-coronavirus.html?action=click\&pgtype=Article\&state=default\&region=MAIN_CONTENT_1\&context=storylines_live_updates\#link-35890b73}{Hundreds
  of doctors in Kenya go on strike over their pay and protective gear.}
\end{itemize}

\href{https://www.nytimes3xbfgragh.onion/2020/08/21/world/covid-19-coronavirus.html?action=click\&pgtype=Article\&state=default\&region=MAIN_CONTENT_1\&context=storylines_live_updates}{See
more updates}

More live coverage:
\href{https://www.nytimes3xbfgragh.onion/live/2020/08/20/business/stock-market-today-coronavirus?action=click\&pgtype=Article\&state=default\&region=MAIN_CONTENT_1\&context=storylines_live_updates}{Markets}

The Biomedical Advanced Research and Development Authority, or BARDA,
which makes deals with drug manufacturers during public health
emergencies and is one of the federal agencies carrying out Operation
Warp Speed, has been headed by two former Novavax executives. One of
them would later complain that the company crossed ethical lines when it
approached him about receiving funding this spring.

Novavax also tapped into a longstanding relationship with the Gates
Foundation, which had previously provided it with funding and is one of
the most powerful global players in the vaccine world.

John J. Trizzino, Novavax's chief business and financial officer, said
the company did nothing inappropriate, but acknowledged that it used its
connections to help win the deals. ``This doesn't happen by itself,'' he
said. ``This happens through years and years of working within the
industry, building solid relationships, having worked with many of these
partners.''

If Novavax does succeed, it will represent a major success story for a
company that has struggled for years. Founded in 1987, the company has
operated on the outskirts of the industry, far from the biotech hubs of
Boston and San Diego. Although vaccines have been its main focus,
Novavax has over the years dabbled in other businesses, like prenatal
vitamins and
\href{https://www.washingtonpost.com/archive/business/2003/10/11/novavaxs-estrogen-lotion-is-approved/53bc53d2-de8e-4c95-a1f0-8b05b3bcd6d0/}{estrogen
lotion}.

In 2016, the company suffered a major setback when its late-stage
clinical trial to treat respiratory syncytial virus, or R.S.V.,
\href{https://ir.novavax.com/news-releases/news-release-details/novavax-announces-topline-rsv-f-vaccine-data-two-clinical-trials}{in
older people failed}, and the
\href{https://endpts.com/crushed-by-a-phiii-flop-novavax-slashes-jobs-as-it-looks-to-chop-out-up-to-100m-in-costs/}{company
laid off} one-third of its staff.

\href{https://www.glassdoor.com.hk/Reviews/Novavax-let-go-Reviews-EI_IE5710.0,7_KH8,14.htm}{A
review in 2017} from an employee on the website Glassdoor summed up the
atmosphere. ``Bowling on Fridays, unlimited sick days,'' the person
wrote under ``pros.'' Under ``cons,'' the person wrote: ``The management
rushed clinical trials for R.S.V., clinical trials failed, and layoffs
insued {[}sic{]}.''

Image

Novavax headquarters in Gaithersburg, Md. Although vaccines have been
its main focus, the company has over the years dabbled in prenatal
vitamins and estrogen lotion.Credit...Jim Lo Scalzo/EPA, via
Shutterstock

But Novavax was able to pursue a second clinical trial of the R.S.V.
vaccine with assistance from the Gates Foundation, which
\href{https://ir.novavax.com/news-releases/news-release-details/novavax-announces-grant-89-million-support-development-rsv-f}{granted
the company up to \$89 million}. That study tested whether giving the
vaccine to pregnant women would pass the immunity to their newborns, who
can become very sick from the virus.

But that
\href{https://ir.novavax.com/news-releases/news-release-details/novavax-announces-topline-results-phase-3-preparetm-trial}{trial
failed} too, and the company again found itself in a financial crisis.
It initiated a reverse stock split to lift its share price and
\href{https://www.bizjournals.com/washington/news/2019/04/15/novavax-faces-delisting-threat-following-failed.html}{avoid
delisting from the Nasdaq}, and it
\href{https://endpts.com/beefing-up-its-new-gene-therapy-unit-catalent-inks-18m-deal-to-snap-up-novavax-facilities/}{sold
its manufacturing facilities} to another company, Catalent, for \$18
million. The deal included 100 workers, or about one-third of its work
force at the time.

Mr. Trizzino said it was all part of the business.

``It's biotech, and biotech can be a bit of a roller-coaster ride,'' he
said. ``We're innovators, and we're looking for opportunities that other
companies haven't developed.''

He said working with the Gates Foundation on the R.S.V. vaccine cemented
that relationship. ``They became very familiar with our technology,''
Mr. Trizzino said. ``So when coronavirus reared its head, they were
supportive.''

The foundation declined to discuss details of vaccine candidates, but in
a statement, Emilio Emini, its H.I.V. program director, said, ``We see
promise in a range of Covid-19 vaccine candidates, including Novavax's
approach.''

Like dozens of other companies, Novavax began working on a coronavirus
vaccine in January, when the virus's genome was first made public, using
the same technology as it had for its R.S.V. and flu vaccines.

It makes vaccines by turning moth cells into tiny factories that pump
out proteins of the coronavirus --- a quicker way to make large
quantities than using cells from hamsters and other mammals.

``I like the company. I like the technology,'' said Dr. Luciana Borio,
who oversaw public health preparedness for the National Security Council
under President Trump and was the acting chief scientist at the Food and
Drug Administration under President Obama. The French drug maker Sanofi
is developing a coronavirus vaccine that uses an insect technology
similar to Novavax's, but has not entered clinical trials.

By February, Dr. Hatchett's nonprofit, CEPI, was flooded with proposals
for vaccine development efforts. Eager to move quickly, the organization
evaluated candidates based on whether the vaccines could be developed
rapidly and manufactured in large enough quantities to be distributed
across the world.

Like the Gates Foundation, Dr. Hatchett was already familiar with
Novavax's work. He had worked at BARDA
\href{https://www.prnewswire.com/news-releases/novavax-awarded-hhs-barda-contract-valued-at-up-to-179-million-to-develop-pandemic-and-seasonal-influenza-vaccines-for-us-government-using-recombinant-vlp-technology-117145058.html}{when
the agency awarded a \$179 million contract} to Novavax in 2011 to
develop its flu vaccine, which could allow for a rapid national response
to a pandemic flu.

When looking for investments in coronavirus vaccines, ``they were a
natural consideration,'' Dr. Hatchett said.

Image

Novavax uses moth cells to pump out crucial molecules at a much faster
rate than typical vaccines --- a major advantage in a
pandemic.Credit...Andrew Caballero-Reynolds/Agence France-Presse ---
Getty Images

CEPI will not release any of its contracts. Dr. Hatchett said its
funding decisions are based on independent external reviews, a
scientific advisory committee, and financial vetting by the accounting
firm KPMG.

\href{https://www.nytimes3xbfgragh.onion/news-event/coronavirus?action=click\&pgtype=Article\&state=default\&region=MAIN_CONTENT_3\&context=storylines_faq}{}

\hypertarget{the-coronavirus-outbreak-}{%
\subsubsection{The Coronavirus Outbreak
›}\label{the-coronavirus-outbreak-}}

\hypertarget{frequently-asked-questions}{%
\paragraph{Frequently Asked
Questions}\label{frequently-asked-questions}}

Updated August 17, 2020

\begin{itemize}
\item ~
  \hypertarget{why-does-standing-six-feet-away-from-others-help}{%
  \paragraph{Why does standing six feet away from others
  help?}\label{why-does-standing-six-feet-away-from-others-help}}

  \begin{itemize}
  \tightlist
  \item
    The coronavirus spreads primarily through droplets from your mouth
    and nose, especially when you cough or sneeze. The C.D.C., one of
    the organizations using that measure,
    \href{https://www.nytimes3xbfgragh.onion/2020/04/14/health/coronavirus-six-feet.html?action=click\&pgtype=Article\&state=default\&region=MAIN_CONTENT_3\&context=storylines_faq}{bases
    its recommendation of six feet} on the idea that most large droplets
    that people expel when they cough or sneeze will fall to the ground
    within six feet. But six feet has never been a magic number that
    guarantees complete protection. Sneezes, for instance, can launch
    droplets a lot farther than six feet,
    \href{https://jamanetwork.com/journals/jama/fullarticle/2763852}{according
    to a recent study}. It's a rule of thumb: You should be safest
    standing six feet apart outside, especially when it's windy. But
    keep a mask on at all times, even when you think you're far enough
    apart.
  \end{itemize}
\item ~
  \hypertarget{i-have-antibodies-am-i-now-immune}{%
  \paragraph{I have antibodies. Am I now
  immune?}\label{i-have-antibodies-am-i-now-immune}}

  \begin{itemize}
  \tightlist
  \item
    As of right
    now,\href{https://www.nytimes3xbfgragh.onion/2020/07/22/health/covid-antibodies-herd-immunity.html?action=click\&pgtype=Article\&state=default\&region=MAIN_CONTENT_3\&context=storylines_faq}{that
    seems likely, for at least several months.} There have been
    frightening accounts of people suffering what seems to be a second
    bout of Covid-19. But experts say these patients may have a
    drawn-out course of infection, with the virus taking a slow toll
    weeks to months after initial exposure. People infected with the
    coronavirus typically
    \href{https://www.nature.com/articles/s41586-020-2456-9}{produce}
    immune molecules called antibodies, which are
    \href{https://www.nytimes3xbfgragh.onion/2020/05/07/health/coronavirus-antibody-prevalence.html?action=click\&pgtype=Article\&state=default\&region=MAIN_CONTENT_3\&context=storylines_faq}{protective
    proteins made in response to an
    infection}\href{https://www.nytimes3xbfgragh.onion/2020/05/07/health/coronavirus-antibody-prevalence.html?action=click\&pgtype=Article\&state=default\&region=MAIN_CONTENT_3\&context=storylines_faq}{.
    These antibodies may} last in the body
    \href{https://www.nature.com/articles/s41591-020-0965-6}{only two to
    three months}, which may seem worrisome, but that's perfectly normal
    after an acute infection subsides, said Dr. Michael Mina, an
    immunologist at Harvard University. It may be possible to get the
    coronavirus again, but it's highly unlikely that it would be
    possible in a short window of time from initial infection or make
    people sicker the second time.
  \end{itemize}
\item ~
  \hypertarget{im-a-small-business-owner-can-i-get-relief}{%
  \paragraph{I'm a small-business owner. Can I get
  relief?}\label{im-a-small-business-owner-can-i-get-relief}}

  \begin{itemize}
  \tightlist
  \item
    The
    \href{https://www.nytimes3xbfgragh.onion/article/small-business-loans-stimulus-grants-freelancers-coronavirus.html?action=click\&pgtype=Article\&state=default\&region=MAIN_CONTENT_3\&context=storylines_faq}{stimulus
    bills enacted in March} offer help for the millions of American
    small businesses. Those eligible for aid are businesses and
    nonprofit organizations with fewer than 500 workers, including sole
    proprietorships, independent contractors and freelancers. Some
    larger companies in some industries are also eligible. The help
    being offered, which is being managed by the Small Business
    Administration, includes the Paycheck Protection Program and the
    Economic Injury Disaster Loan program. But lots of folks have
    \href{https://www.nytimes3xbfgragh.onion/interactive/2020/05/07/business/small-business-loans-coronavirus.html?action=click\&pgtype=Article\&state=default\&region=MAIN_CONTENT_3\&context=storylines_faq}{not
    yet seen payouts.} Even those who have received help are confused:
    The rules are draconian, and some are stuck sitting on
    \href{https://www.nytimes3xbfgragh.onion/2020/05/02/business/economy/loans-coronavirus-small-business.html?action=click\&pgtype=Article\&state=default\&region=MAIN_CONTENT_3\&context=storylines_faq}{money
    they don't know how to use.} Many small-business owners are getting
    less than they expected or
    \href{https://www.nytimes3xbfgragh.onion/2020/06/10/business/Small-business-loans-ppp.html?action=click\&pgtype=Article\&state=default\&region=MAIN_CONTENT_3\&context=storylines_faq}{not
    hearing anything at all.}
  \end{itemize}
\item ~
  \hypertarget{what-are-my-rights-if-i-am-worried-about-going-back-to-work}{%
  \paragraph{What are my rights if I am worried about going back to
  work?}\label{what-are-my-rights-if-i-am-worried-about-going-back-to-work}}

  \begin{itemize}
  \tightlist
  \item
    Employers have to provide
    \href{https://www.osha.gov/SLTC/covid-19/standards.html}{a safe
    workplace} with policies that protect everyone equally.
    \href{https://www.nytimes3xbfgragh.onion/article/coronavirus-money-unemployment.html?action=click\&pgtype=Article\&state=default\&region=MAIN_CONTENT_3\&context=storylines_faq}{And
    if one of your co-workers tests positive for the coronavirus, the
    C.D.C.} has said that
    \href{https://www.cdc.gov/coronavirus/2019-ncov/community/guidance-business-response.html}{employers
    should tell their employees} -\/- without giving you the sick
    employee's name -\/- that they may have been exposed to the virus.
  \end{itemize}
\item ~
  \hypertarget{what-is-school-going-to-look-like-in-september}{%
  \paragraph{What is school going to look like in
  September?}\label{what-is-school-going-to-look-like-in-september}}

  \begin{itemize}
  \tightlist
  \item
    It is unlikely that many schools will return to a normal schedule
    this fall, requiring the grind of
    \href{https://www.nytimes3xbfgragh.onion/2020/06/05/us/coronavirus-education-lost-learning.html?action=click\&pgtype=Article\&state=default\&region=MAIN_CONTENT_3\&context=storylines_faq}{online
    learning},
    \href{https://www.nytimes3xbfgragh.onion/2020/05/29/us/coronavirus-child-care-centers.html?action=click\&pgtype=Article\&state=default\&region=MAIN_CONTENT_3\&context=storylines_faq}{makeshift
    child care} and
    \href{https://www.nytimes3xbfgragh.onion/2020/06/03/business/economy/coronavirus-working-women.html?action=click\&pgtype=Article\&state=default\&region=MAIN_CONTENT_3\&context=storylines_faq}{stunted
    workdays} to continue. California's two largest public school
    districts --- Los Angeles and San Diego --- said on July 13, that
    \href{https://www.nytimes3xbfgragh.onion/2020/07/13/us/lausd-san-diego-school-reopening.html?action=click\&pgtype=Article\&state=default\&region=MAIN_CONTENT_3\&context=storylines_faq}{instruction
    will be remote-only in the fall}, citing concerns that surging
    coronavirus infections in their areas pose too dire a risk for
    students and teachers. Together, the two districts enroll some
    825,000 students. They are the largest in the country so far to
    abandon plans for even a partial physical return to classrooms when
    they reopen in August. For other districts, the solution won't be an
    all-or-nothing approach.
    \href{https://bioethics.jhu.edu/research-and-outreach/projects/eschool-initiative/school-policy-tracker/}{Many
    systems}, including the nation's largest, New York City, are
    devising
    \href{https://www.nytimes3xbfgragh.onion/2020/06/26/us/coronavirus-schools-reopen-fall.html?action=click\&pgtype=Article\&state=default\&region=MAIN_CONTENT_3\&context=storylines_faq}{hybrid
    plans} that involve spending some days in classrooms and other days
    online. There's no national policy on this yet, so check with your
    municipal school system regularly to see what is happening in your
    community.
  \end{itemize}
\end{itemize}

He said his organization took note in March when, weeks after
\href{https://ir.novavax.com/news-releases/news-release-details/novavax-awarded-funding-cepi-covid-19-vaccine-development}{it
awarded Novavax} its initial \$4 million contract, the company
\href{https://ir.novavax.com/news-releases/news-release-details/novavax-nanoflu-achieves-all-primary-endpoints-phase-3-clinical}{announced
that its flu vaccine} had succeeded in a late-stage clinical trial ---
its first major success, and an important validation of its underlying
vaccine technology.

``We were very relieved when that positive result came back,'' Dr.
Hatchett said. In May, his organization ramped up its support with an
additional deal awarding Novavax up to \$384 million.

For Novavax, the contracts with CEPI proved crucial. Until then, the
company had been aggressively seeking funding from BARDA, without much
luck.

In April, eager to pitch their vaccine, Novavax's chief executive asked
to speak with Rick Bright, the former director of BARDA, who served as
\href{https://ir.novavax.com/news-releases/news-release-details/novavax-appoints-new-vice-president-vaccine-research}{head
of vaccine research} at Novavax from 2006 to 2008,
\href{https://www.cnn.com/2020/05/05/politics/rick-bright-full-complaint/index.html}{according
to a whistle-blower complaint}that Dr. Bright later filed.

Dr. Bright said in the complaint that he declined the meeting with the
chief executive, Stanley C. Erck, because discussing the vaccine while
the company's application was being considered would violate federal
law, given that it could influence what is supposed to be a purely
scientific review.

But three days later, the company pursued a meeting instead with Dr.
Bright's boss, Dr. Robert Kadlec, H.H.S.'s assistant secretary for
preparedness and response. Dr. Kadlec initially wrote that he was
``looking forward'' to the meeting, according to emails obtained by The
New York Times, but a H.H.S. spokeswoman said that he did not meet with
the company.

Dr. Bright was removed from his BARDA post in April, and filed the
complaint
\href{https://www.nytimes3xbfgragh.onion/2020/05/09/us/politics/whistle-blower-trump-coronavirus.html}{after
he said he had been protesting} ``cronyism'' and contract abuse for
years.

Novavax's contact with H.H.S. raised alarms with Steven L. Schooner, a
law professor at George Washington University Law School who is an
expert in federal procurement.

``When you're dealing with something as important as a vaccine for a
pandemic, you want that review to be made on the scientific merits, not
based on who knew who or who is willing to pay a bribe or who applied
leverage during the evaluation process,'' he said.

Mr. Trizzino said the company did nothing wrong. ``We did what we
thought was prudent and reasonable under the circumstances of a pandemic
and the need to move very quickly,'' he said.

Image

Dr. Rick Bright,~the former director of BARDA, testifying before a House
subcommittee in May.~Credit...Pool photo by Shawn Thew

When the conversation with BARDA never materialized, the company shifted
its attention to the Defense Department.

With the second CEPI contract in hand, Novavax was finally able to ``get
more and more traction,'' Mr. Trizzino said. In June, the department
\href{https://ir.novavax.com/news-releases/news-release-details/novavax-awarded-department-defense-contract-covid-19-vaccine}{awarded
the company \$60 million}. ``And ultimately, Operation Warp Speed took
over from there,'' he said.

``They have shown very encouraging and promising results,'' said Robin
Robinson, who was the head of Novavax's vaccine division until 2004,
when he left to become the head of BARDA's influenza division, and then
its director.

Dr. Robinson helped develop an earlier version of the company's vaccine
technology and consulted for Novavax on their flu vaccine. ``I do expect
the vaccine to be one of the ones in the winner's circle next year.''

With two major contracts, Novavax must now balance two powerful --- and
potentially competing --- investors.
\href{https://www.nytimes3xbfgragh.onion/interactive/2020/science/coronavirus-vaccine-tracker.html}{The
company's vaccine is now in safety trials}, and results are expected
this month. It plans to begin so-called Phase 3 efficacy trials by the
fall, and could release data by the end of the year. If the vaccine is
successful, the company has promised to supply the United States with
100 million doses --- or enough to immunize at least 50 million U.S.
residents. And through its deal with CEPI, it has pledged an unspecified
number of doses to low-income countries.

Novavax has said that it can accomplish both by simultaneously
manufacturing the vaccine in the United States, Europe and Asia. The
company used an outside manufacturer, the Maryland-based Emergent
BioSolutions, to make initial doses for the clinical trials, but said
that they had not yet selected a company to do large-scale manufacturing
in the United States. It recently acquired a factory in the Czech
Republic and will hire other manufacturers to supply the rest of the
world. With those new factory workers, the company said it now employs
about 360 people.

``It's very well coordinated, and we know what we're doing,'' Mr.
Trizzino said.

But because vaccine development is so unpredictable, and with these
deals largely unfolding in private, it's impossible to know how far the
company will get.

``The U.S. darling of the moment is Novavax,'' said Kate Elder, a senior
vaccines policy adviser for Doctors Without Borders. ``But I see this as
just a further diversification of the U.S.'s risky bets with public
money and little transparency.''

Advertisement

\protect\hyperlink{after-bottom}{Continue reading the main story}

\hypertarget{site-index}{%
\subsection{Site Index}\label{site-index}}

\hypertarget{site-information-navigation}{%
\subsection{Site Information
Navigation}\label{site-information-navigation}}

\begin{itemize}
\tightlist
\item
  \href{https://help.nytimes3xbfgragh.onion/hc/en-us/articles/115014792127-Copyright-notice}{©~2020~The
  New York Times Company}
\end{itemize}

\begin{itemize}
\tightlist
\item
  \href{https://www.nytco.com/}{NYTCo}
\item
  \href{https://help.nytimes3xbfgragh.onion/hc/en-us/articles/115015385887-Contact-Us}{Contact
  Us}
\item
  \href{https://www.nytco.com/careers/}{Work with us}
\item
  \href{https://nytmediakit.com/}{Advertise}
\item
  \href{http://www.tbrandstudio.com/}{T Brand Studio}
\item
  \href{https://www.nytimes3xbfgragh.onion/privacy/cookie-policy\#how-do-i-manage-trackers}{Your
  Ad Choices}
\item
  \href{https://www.nytimes3xbfgragh.onion/privacy}{Privacy}
\item
  \href{https://help.nytimes3xbfgragh.onion/hc/en-us/articles/115014893428-Terms-of-service}{Terms
  of Service}
\item
  \href{https://help.nytimes3xbfgragh.onion/hc/en-us/articles/115014893968-Terms-of-sale}{Terms
  of Sale}
\item
  \href{https://spiderbites.nytimes3xbfgragh.onion}{Site Map}
\item
  \href{https://help.nytimes3xbfgragh.onion/hc/en-us}{Help}
\item
  \href{https://www.nytimes3xbfgragh.onion/subscription?campaignId=37WXW}{Subscriptions}
\end{itemize}
