Sections

SEARCH

\protect\hyperlink{site-content}{Skip to
content}\protect\hyperlink{site-index}{Skip to site index}

\href{https://myaccount.nytimes3xbfgragh.onion/auth/login?response_type=cookie\&client_id=vi}{}

\href{https://www.nytimes3xbfgragh.onion/section/todayspaper}{Today's
Paper}

\href{/section/opinion}{Opinion}\textbar{}The Next Disaster Is Just a
Few Days Away

\href{https://nyti.ms/2OyD4aU}{https://nyti.ms/2OyD4aU}

\begin{itemize}
\item
\item
\item
\item
\item
\item
\end{itemize}

Advertisement

\protect\hyperlink{after-top}{Continue reading the main story}

\href{/section/opinion}{Opinion}

Supported by

\protect\hyperlink{after-sponsor}{Continue reading the main story}

\hypertarget{the-next-disaster-is-just-a-few-days-away}{%
\section{The Next Disaster Is Just a Few Days
Away}\label{the-next-disaster-is-just-a-few-days-away}}

Millions of unemployed Americans face imminent catastrophe.

\href{https://www.nytimes3xbfgragh.onion/by/paul-krugman}{\includegraphics{https://static01.graylady3jvrrxbe.onion/images/2018/04/02/opinion/paul-krugman/paul-krugman-thumbLarge.png}}

By \href{https://www.nytimes3xbfgragh.onion/by/paul-krugman}{Paul
Krugman}

Opinion Columnist

\begin{itemize}
\item
  July 16, 2020
\item
  \begin{itemize}
  \item
  \item
  \item
  \item
  \item
  \item
  \end{itemize}
\end{itemize}

\includegraphics{https://static01.graylady3jvrrxbe.onion/images/2020/07/16/opinion/16krugmanWeb/merlin_173028825_fdf86537-fb18-4527-873f-051d0579b13c-articleLarge.jpg?quality=75\&auto=webp\&disable=upscale}

Some of us
\href{https://krugman.blogs.nytimes3xbfgragh.onion/2016/11/11/the-long-haul/}{knew
from the beginning} that Donald Trump wasn't up to the job of being
president, that he wouldn't be able to deal with a crisis that wasn't of
his own making. Still, the magnitude of America's coronavirus failure
has shocked even the cynics.

At this point Florida alone has an
\href{https://covidtracking.com/data/charts/all-metrics-per-state}{average
daily death toll} roughly equal to that of the whole
\href{https://ourworldindata.org/coronavirus-data-explorer?zoomToSelection=true\&deathsMetric=true\&dailyFreq=true\&smoothing=7\&country=USA~EuropeanUnion\&pickerMetric=location\&pickerSort=asc}{European
Union}, which has 20 times its population.

How did this happen? One key element in our deadly debacle has been
extreme shortsightedness: At
\href{https://twitter.com/TBPInvictus/status/1281256803788107776}{every
stage of the crisis} Trump and his allies refused to acknowledge or get
ahead of disasters everyone paying attention clearly saw coming.

Blithe denials that Covid-19 posed a threat gave way to blithe denials
that rapid reopening would lead to a new surge in infections; now that
the surge is upon us, Republican governors are responding sluggishly and
grudgingly, while the White House is doing nothing at all.

And now another disaster --- this time economic rather than
epidemiological --- is just days away.

To understand the cliff we're about to plunge over, you need to know
that while America's overall handling of Covid-19 was catastrophically
bad, one piece --- the economic response --- was actually better than
many of us expected. The
\href{https://www.nytimes3xbfgragh.onion/2020/03/26/us/coronavirus-senate-stimulus-package.html}{CARES
Act}, largely devised by Democrats but enacted by a bipartisan majority
late in March, had flaws in both design and implementation, yet it did a
lot both to alleviate hardship and to limit the economic fallout from
the pandemic.

In particular, the act provided vastly increased aid to workers idled by
lockdowns imposed to curb the spread of the coronavirus. U.S.
unemployment insurance is normally a weak protection against adversity:
Many workers aren't covered, and even those who are usually receive only
a small fraction of their previous wages. But the CARES Act both
expanded coverage, for example to gig workers, and sharply increased
benefits, adding \$600 to every recipient's weekly check.

These enhanced benefits did double duty. They meant that there was far
less misery than one might otherwise have expected from a crisis that
temporarily eliminated
\href{https://fred.stlouisfed.org/series/PAYEMS}{22 million jobs}; by
some measures poverty
\href{https://harris.uchicago.edu/news-events/news/new-poverty-measure-confirms-coronavirus-driven-federal-stimulus-measures-were}{actually
declined}.

They also helped sustain those parts of the economy that \emph{weren't}
locked down. Without those emergency benefits, laid-off workers would
have been forced to slash spending across the board. This would have
generated a whole second round of job loss and economic contraction, as
well as creating a huge wave of missed rental payments and evictions.

So enhanced unemployment benefits have been a crucial lifeline to tens
of millions of Americans. Unfortunately, all of those beneficiaries are
now just a few days from being thrown overboard.

For that \$600 weekly supplement --- which accounts for
\href{https://www.bea.gov/system/files/2020-06/Effects-of-Selected-Federal-Pandemic-Response-Programs-on-Personal-Income-May-2020.pdf}{most}
of the expansion of benefits --- applies only to benefit weeks that end
``\href{https://www.natlawreview.com/article/additional-600-unemployment-payment-expiration-date-looms-near}{on
or before July 31}.'' July 31 is a Friday. State unemployment benefit
weeks typically end on Saturday or Sunday. So the supplement will end,
in most places, on
\href{https://www.cnbc.com/2020/06/26/the-extra-600-unemployment-benefits-will-end-before-july-31.html}{July
25 or 26}, and millions of workers will see their incomes plunge 60
percent or more just a few days from now.

Two months have gone by since the House passed a
\href{https://www.congress.gov/bill/116th-congress/house-bill/6800}{relief
measure} that would, among other things, extend enhanced benefits
through the rest of the year. But neither Senate Republicans nor the
White House has shown any sense of urgency about the looming crisis.
Why?

Part of the answer is that Trump and his officials are, as always, far
behind the coronavirus curve. They're still talking about a rapid,
\href{https://twitter.com/JStein_WaPo/status/1282349502284865536}{V-shape
recovery} that will bring us quickly back to full employment, making
special aid to the unemployed unnecessary; they're apparently oblivious
to what everyone else sees --- an economy that is
\href{https://www.nytimes3xbfgragh.onion/2020/07/15/business/economy/economic-recovery-coronavirus-resurgence.html}{stumbling
again} as the coronavirus surges back.

Delusions about the state of the economic recovery, in turn, allow
conservatives to indulge in one of their favorite
\href{https://wwnorton.com/books/9781324005018}{zombie ideas} --- that
helping the unemployed in a depressed economy hurts job creation, by
discouraging people from taking jobs.

Worrying about employment incentives in the midst of a pandemic is even
crazier than worrying about those incentives in the
\href{https://www.nytimes3xbfgragh.onion/2010/07/05/opinion/05krugman.html}{aftermath
of a financial crisis}, but it seems to be at the core of White House
\href{https://twitter.com/paulkrugman/status/1282737752828764161}{thinking}
(or maybe that's ``thinking'') about economic policy right now.

One last thing: My sense is that Republicans have a delusional view of
their own bargaining position. They don't seem to realize that they, not
the Democrats, will be blamed if millions are plunged into penury
because relief is delayed; to the extent that they're willing to act at
all, they still imagine that they can extract concessions like a blanket
exemption of businesses from
\href{https://www.cnn.com/2020/07/13/politics/covid-aid-congress-mitch-mcconnell/index.html}{pandemic
liability}.

Maybe the prospect of catastrophe will concentrate Republican minds, but
it seems more likely that we're heading for weeks if not months of
extreme financial distress for millions of Americans, distress that will
hobble the economy as a whole. This disaster didn't need to happen; but
you can say the same thing about most of what has gone wrong in this
country lately.

\emph{The Times is committed to publishing}
\href{https://www.nytimes3xbfgragh.onion/2019/01/31/opinion/letters/letters-to-editor-new-york-times-women.html}{\emph{a
diversity of letters}} \emph{to the editor. We'd like to hear what you
think about this or any of our articles. Here are some}
\href{https://help.nytimes3xbfgragh.onion/hc/en-us/articles/115014925288-How-to-submit-a-letter-to-the-editor}{\emph{tips}}\emph{.
And here's our email:}
\href{mailto:letters@NYTimes.com}{\emph{letters@NYTimes.com}}\emph{.}

\emph{Follow The New York Times Opinion section on}
\href{https://www.facebookcorewwwi.onion/nytopinion}{\emph{Facebook}}\emph{,}
\href{http://twitter.com/NYTOpinion}{\emph{Twitter (@NYTopinion)}}
\emph{and}
\href{https://www.instagram.com/nytopinion/}{\emph{Instagram}}\emph{.}

Advertisement

\protect\hyperlink{after-bottom}{Continue reading the main story}

\hypertarget{site-index}{%
\subsection{Site Index}\label{site-index}}

\hypertarget{site-information-navigation}{%
\subsection{Site Information
Navigation}\label{site-information-navigation}}

\begin{itemize}
\tightlist
\item
  \href{https://help.nytimes3xbfgragh.onion/hc/en-us/articles/115014792127-Copyright-notice}{©~2020~The
  New York Times Company}
\end{itemize}

\begin{itemize}
\tightlist
\item
  \href{https://www.nytco.com/}{NYTCo}
\item
  \href{https://help.nytimes3xbfgragh.onion/hc/en-us/articles/115015385887-Contact-Us}{Contact
  Us}
\item
  \href{https://www.nytco.com/careers/}{Work with us}
\item
  \href{https://nytmediakit.com/}{Advertise}
\item
  \href{http://www.tbrandstudio.com/}{T Brand Studio}
\item
  \href{https://www.nytimes3xbfgragh.onion/privacy/cookie-policy\#how-do-i-manage-trackers}{Your
  Ad Choices}
\item
  \href{https://www.nytimes3xbfgragh.onion/privacy}{Privacy}
\item
  \href{https://help.nytimes3xbfgragh.onion/hc/en-us/articles/115014893428-Terms-of-service}{Terms
  of Service}
\item
  \href{https://help.nytimes3xbfgragh.onion/hc/en-us/articles/115014893968-Terms-of-sale}{Terms
  of Sale}
\item
  \href{https://spiderbites.nytimes3xbfgragh.onion}{Site Map}
\item
  \href{https://help.nytimes3xbfgragh.onion/hc/en-us}{Help}
\item
  \href{https://www.nytimes3xbfgragh.onion/subscription?campaignId=37WXW}{Subscriptions}
\end{itemize}
