Sections

SEARCH

\protect\hyperlink{site-content}{Skip to
content}\protect\hyperlink{site-index}{Skip to site index}

\href{https://www.nytimes3xbfgragh.onion/spotlight/podcasts}{Podcasts}

\href{https://myaccount.nytimes3xbfgragh.onion/auth/login?response_type=cookie\&client_id=vi}{}

\href{https://www.nytimes3xbfgragh.onion/section/todayspaper}{Today's
Paper}

\href{/spotlight/podcasts}{Podcasts}\textbar{}Reparations for Aunt
Jemima!

\href{https://nyti.ms/2B69YMR}{https://nyti.ms/2B69YMR}

\begin{itemize}
\item
\item
\item
\item
\item
\end{itemize}

Advertisement

\protect\hyperlink{after-top}{Continue reading the main story}

transcript

Back to Still Processing

bars

0:00/35:35

-35:35

transcript

\hypertarget{reparations-for-aunt-jemima}{%
\subsection{Reparations for Aunt
Jemima!}\label{reparations-for-aunt-jemima}}

\hypertarget{hosted-by-wesley-morris-and-jenna-wortham-produced-by-hans-buetow}{%
\subsubsection{Hosted by Wesley Morris and Jenna Wortham. Produced by
Hans
Buetow.}\label{hosted-by-wesley-morris-and-jenna-wortham-produced-by-hans-buetow}}

\hypertarget{we-explore-the-erasure-of-aunt-jemima-and-the-reconciliation-of-that-icon}{%
\paragraph{We explore the erasure of Aunt Jemima and the reconciliation
of that
icon.}\label{we-explore-the-erasure-of-aunt-jemima-and-the-reconciliation-of-that-icon}}

Thursday, July 16th, 2020

\begin{itemize}
\item
  wesley morris\\
  Jenna.
\item
  jenna wortham\\
  Yes, baby.
\item
  wesley morris\\
  It has been some summer.
\item
  jenna wortham\\
  Mm.
\item
  wesley morris\\
  George Floyd was killed in Minneapolis, Minn., on Memorial Day.
\item
  jenna wortham\\
  Right.
\item
  wesley morris\\
  We are now in July. And a lot of things have happened in the wake of
  his death.
\item
  jenna wortham\\
  Mm. Mm-hm, mm-hm.
\item
  wesley morris\\
  They are mind-blowing to think about.
\item
  jenna wortham\\
  Right.
\item
  wesley morris\\
  Is your seat belt on?
\item
  jenna wortham\\
  I'm buckled up, baby. Let's go.
\item
  {[}music{]}
\item
  wesley morris\\
  Monuments, statues being toppled and brought down.
\item
  jenna wortham\\
  Mm.
\item
  wesley morris\\
  There is giant Black Lives Matter murals going along the streets all
  over the country.
\item
  jenna wortham\\
  Mm.
\item
  wesley morris\\
  Kente scarves being worn by our Congress people.
\item
  jenna wortham\\
  Rude. Stop reminding me.
\item
  wesley morris\\
  We now work at a news organization that is capitalizing the identity
  Black. No more LEGO police sets for you kids. ``Cops'' has been
  canceled, blackface episodes of ``30 Rock'' and ``Community'' taken
  down. The N.F.L., league full of Black people, realizes that Black
  lives do matter, but never says that Colin Kaepernick does. Aunt
  Jemima, Uncle Ben retired. Lady Antebellum and Dixie Chicks are now
  Lady A and the Chicks?
\item
  jenna wortham\\
  Don't forget also, the Washington football team finally --- finally
  --- going to have a new name.
\item
  wesley morris\\
  Finally.
\item
  jenna wortham\\
  I mean, we could talk about each of those items for the rest of our
  lives. And we probably will. But I just want to dial in on one ---
\item
  wesley morris\\
  Sure.
\item
  jenna wortham\\
  --- which is Aunt Jemima, right?
\item
  wesley morris\\
  OK.
\item
  jenna wortham\\
  I don't want to wallpaper over that past because I think that lets the
  people responsible for that character and that caricature off way too
  easy. If we just disappear Aunt Jemima, then we're effectively erasing
  her, which does not heal that wound. It actually exacerbates it. It
  just covers it up with a Band-Aid, no bacitracin, you know?
\item
  wesley morris\\
  Oh, yeah. Please, we should definitely talk about Aunt Jemima and the
  pancake mix. Let's crack some eggs. We'll get out some water, and
  we'll just --- we're going to get right into some stirring.
\item
  {[}music - kindness, world restart{]}
\item
  jenna wortham\\
  I'm Jenna Wortham.
\item
  wesley morris\\
  I'm Wesley Morris. We're two culture writers at The New York Times,
  and recording in our living rooms today during a pandemic, a movement,
  and apparently a rainstorm.
\item
  jenna wortham\\
  (LAUGHS)
\item
  wesley morris\\
  Y'all gon' hear it.
\item
  jenna wortham\\
  This is ``Still Processing.''
\item
  {[}music{]}
\item
  jenna wortham\\
  So Wesley, in mid-June, Quaker makes this announcement --- we are
  going to retire Aunt Jemima. And the two people that are closest to me
  in my life both send me a text at around the same time, near identical
  text messages by the way. They contain an image of the pancake box and
  a text message basically that says, oh, S-H-I-T. And I think they
  wanted to know, what do I do with the box? And it's worth noting
  neither of them are white. But my immediate response back to both of
  them was, well, how full are the boxes?
\item
  wesley morris\\
  (LAUGHS)
\item
  jenna wortham\\
  Right? I mean, at this point, you might as well finish them. But when
  you do, can you save me the box?
\item
  wesley morris\\
  (LAUGHS) Why do you want the box?
\item
  jenna wortham\\
  You know, it's a part of history, right?
\item
  wesley morris\\
  Mm-hm.
\item
  jenna wortham\\
  And it's uncomfortable. But it feels really important to remember how
  America talked about Black people. And I know that you've done so much
  research into the history and the origins of minstrelsy in this
  country for this mysterious-ass book you're always working on it,
  which is why we can't hang out on the weekends, not that we'd be
  hanging out that much anyway these days. But it's helpful. You know a
  lot.
\item
  wesley morris\\
  Yes, that is true. I have been working on a book about, among many
  other things, the history of Black entertainment for the last 200
  years, it feels like at this point.
\item
  jenna wortham\\
  Well, Mr. Morris, I also did some research into the origins of this
  caricature because I was really interested in where the idea to use
  this Black woman to sell pancake mix came from. And what I found is
  fascinating. Are you ready?
\item
  wesley morris\\
  Oh, I'm ready.
\item
  jenna wortham\\
  So in 1889, this dude named Christopher Rutt and his friend Charles G.
  Underwood, they decide to buy a flour mill in St. Louis, Mo., which if
  you look at a map, is actually not that far from Minneapolis, Minn.,
  just saying. So in order to revive sales, they start making an instant
  pancake mix that all you have to do is add water. And they put it in
  these paper bags, and they sell it. And they realize that they need
  something to kind of give the brand a little zhuzh and distinguish it
  from other companies. So our old friend Chris Rutt --- and this is the
  part where the tea gets piping hot, honey --- this man edits a
  newspaper called The St. Joseph Gazette. So think about that. So this
  is someone who fully understands the power of narrative, who fully
  understands how images work, how media works. Apparently, he decides
  that the image they're going to use is going to be an image that he
  saw from an old, yeah, ``minstrel vaudeville'' --- I'm using that in
  quotes --- poster, where there was a figure that they lifted and
  transformed into what we call Aunt Jemima. There's also a part of the
  lore that goes, there was a song called ``Old Aunt Jemima,'' a
  minstrel song. And that served as inspiration as well.
\item
  wesley morris\\
  ``Old Aunt Jemima'' isn't just any minstrel song. A, it was a huge hit
  for Billy Kersands, who wrote and performed it like thousands of times
  all over the country. He might even have gone to England and done it a
  couple times, too. Written by a Black man who performed as a blackface
  minstrel because that's how it worked. It's the only way Black people
  could really get a foothold in the entertainment industry at any point
  in the 19th century. But the song is different from other minstrel
  songs in that it's about an old Black woman who is waiting for her
  mistress to die. But this bitch won't die. (LAUGHS)
\item
  jenna wortham\\
  (LAUGHS) Listen.
\item
  wesley morris\\
  She keeps not dying. And old Aunt Jemima is just like, when --- you
  said you were going to die and set me free. You said you were going to
  die. We had a deal. You just keep living, and I keep being enslaved.
  Why?
\item
  jenna wortham\\
  Well, so Chris Rutt, he decides he's going to use an image that he's
  pulled from maybe the song, maybe a poster. He's going to do it with
  the help of, you know, a Black woman who is in this stereotypical role
  as a, quote, ``mammy,'' right? That is essentially how you get Aunt
  Jemima on a pancake box.
\item
  wesley morris\\
  That Aunt Jemima looked like a classic minstrel figure. Big lipped.
  She's got these little wooden teeth, these --- they look like corn
  cobs. And these big sunken eyes that seem to have no human expression
  at all. And atop her head is this kerchief. And in some ways, this is
  the classic mammy figure that you would have gone to the theater to
  see a white man perform on stage. And so this image has basically been
  set in the minds of American popular culture for a long time by 1889.
  So it was kind of a no-brainer because these images are also, it's
  worth pointing out, pro-slavery images. They are pictures of domestic
  servitude that the servant herself seems more than happy to provide.
\item
  jenna wortham\\
  Yeah.
\item
  wesley morris\\
  But really, that was a common way to draw Black people at that point
  in time.
\item
  {[}rain{]}
\item
  wesley morris\\
  Here comes the rain again.
\item
  jenna wortham\\
  Oh, yeah, we're in the middle of it. It's juicy. I'm not mad. My
  plants aren't mad, either. OK, so this character has been created to
  sell the pancake mix. And a year later in 1890, an entrepreneur named
  R.T. Davis buys the whole company and decides, you know what, we're
  actually going to hire Black women to play the role of Aunt Jemima to
  further sell this product. And one of the places they do it is at the
  World's Fair in Chicago, Ill., in 1893. And a formerly enslaved woman
  named Nancy Green is hired to walk around the fair, do these pancake
  demos, presumably cook the pancakes, and feed them to people.
\item
  wesley morris\\
  She cooked the pancakes, yes.
\item
  jenna wortham\\
  And she's dressed like the character of Aunt Jemima, with the
  bandanna, and the apron, the whole nine.
\item
  wesley morris\\
  Yeah, I mean at that same World's Fair, Frederick Douglass and Ida B.
  Wells were walking around handing out pamphlets ---
\item
  jenna wortham\\
  Wow.
\item
  wesley morris\\
  --- basically criticizing the fair organizers for including no Black
  people in the planning of the fair itself.
\item
  jenna wortham\\
  Wow.
\item
  wesley morris\\
  Talk about the tension between being represented and being respected.
  It was all happening at that festival.
\item
  jenna wortham\\
  Wow.
\item
  wesley morris\\
  It was a really crazy time. I mean, you've got two of the world's most
  important Black people in Frederick Douglass and Ida B. Wells doing
  the work to try to bring attention to the sort of inherent unfairness
  of who gets to decide how Black people are represented at this fair.
\item
  jenna wortham\\
  Yes, absolutely.
\item
  wesley morris\\
  But then you have all --- you have like huge Black attendance coming
  to see all of the spectacles at the fair, including Nancy Green. The
  Pancake Queen, by the way, is what they called her.
\item
  jenna wortham\\
  Wow. Well, I'm really glad you brought that into the room, Wesley,
  because it's really important to note. So Nancy is one of the first of
  many women to play this role over the years, and including up until
  and after Quaker buys the brand. And many of the descendants,
  including the descendants of a woman named Lillian Richard, who was
  also hired to portray her, they talk about that inherited legacy as
  being a source of pride, because Black people and Black women never
  had public-facing roles. They never had jobs where they were meant to
  be seen and meant to be looked at. And so it's very complicated that
  that was her job. So it's not as simple as we might like to think,
  looking backwards. But you know, Wesley, one of the things they were
  asking these women to do, these actors, as they were portraying Aunt
  Jemima, they wanted them to reinforce the idea that Black people were
  happy in these subservient, domestic roles.
\item
  wesley morris\\
  Mm-hm.
\item
  archived recording 1\\
  Smiling, happy Aunt Jemima, famous for her secret recipe pancakes,
  waffles, and buckwheat.
\item
  archived recording 2\\
  And now, Aunt Jemima, one of your old plantation sings, if you will.
\item
  archived recording (aunt jemima)\\
  It's a pleasure, folks, to remind y'all that the Lord meant for the
  sky to be blue. But if the day is blue, it's probably our fault.
\item
  archived recording 2\\
  True, Aunt Jemima, true.
\end{itemize}

wesley morris

This is straight out of minstrelsy. She was a minstrel figure,
essentially. And let's just talk about the way that those women were
depicted on those boxes. And into this you can throw a caricature of a
person like Uncle Ben, or Rastus, the face of the Cream of Wheat, the
racistly named Rastus on the box of Cream of Wheat. Each of these people
is depicted how, Jenna? What are they doing with their faces on all
those boxes?

jenna wortham

I mean, they have a maniacal grin on their faces, you know?

wesley morris

Yeah.

jenna wortham

I mean, they look like a character from the movie ``Get Out.'' Like,
they look like they're being held hostage and being forced to enjoy it.

wesley morris

Yes. And that smile is the smile that has been haunting us for almost
200 years.

jenna wortham

Yeah.

wesley morris

It is the smile of servitude. It is the smile of complicity. It is the
smile that says, these people like the harm that we are doing to them
every day.

jenna wortham

They enjoy it.

wesley morris

We've been doing it for centuries.

jenna wortham

They like it.

wesley morris

But you see this smile? This smile means that it's OK that that harm is
being done. It is the smile that Black people know or associate with a
kind of complicity in their own victimization. It is a smile that
basically says, I'm not going to resist. It's associated with whatever
we mean when we say Uncle Tom, because it implies --- that sort of
smiling Black person implies that you're going along with this racist
system.

\begin{itemize}
\item
  archived recording (aunt jemima)\\
  (LAUGHS) Greetings, folks, greetings. This is your old friend, Aunt
  Jemima.
\item
  archived recording\\
  And pleased as punch with yourself this morning, aren't you, Aunt
  Jemima?
\item
  archived recording (aunt jemima)\\
  (LAUGHS) `Course I am, Mr. Lyon.
\item
  archived recording\\
  Smiling, happy Aunt Jemima. Smiling, happy Aunt Jemima, famous for her
  secret recipe pancakes, waffles, and buckwheat. What's the good word,
  Aunt Jemima?
\item
  archived recording (aunt jemima)\\
  Well, Mr. Lyon, folks says there's nothing so pretty as a happy face
  and nothing so worthwhile as a happy life.
\item
  archived recording\\
  Yes, Aunt Jemima, that is true.
\end{itemize}

wesley morris

And that's some powerful stuff. And it --- like what it does in terms of
the image that gets reinforced, it's very hard to see around that.

jenna wortham

Yeah.

wesley morris

This is generations of kids of all races raised on these images.

jenna wortham

It's funny to me, Wesley, how much of this I'm really only learning
because we're making an episode about it. You know, America works
really, really, really hard to hide this history from us. But it impacts
us, whether or not we talk about it or not, right?

wesley morris

Mm-hm.

jenna wortham

So for Quaker to say, ``Oh my gosh, you know what, we don't want to have
this image anymore, we realize it's totally racist. You guys have been
telling us for years. We finally agree. Black Pancake Lives Matter!''
And they decide to vaporize it. In doing that, you can't wave a magic
wand over centuries of subjugation and oppression. And in trying to do
that, I think you end up ripping open a bigger wound than you're able to
heal.

wesley morris

Yes, yes, yes. Part of what is happening right now in this country is
people becoming aware of how deep the roots go. And I think being forced
to think about --- I mean, listen, I'm as guilty of this as anybody,
right? I mean, for as much time as I spend thinking about, researching,
writing about, emoting about all of this history, some things just slip
through my cracks. And Lady Antebellum and the Dixie Chicks are two such
things. I will sit here and tell you that when they changed the names of
their bands.

jenna wortham

Uh-huh.

wesley morris

I was really like, oh, yeah. I guess that does make sense.

jenna wortham

Oh, my god.

wesley morris

And it's not that I don't know what Antebellum is probably referring to.
And it's not like I don't know what Dixie is. Lady Antebellum, I guess I
didn't think about it because ``Need You Now'' is just one of the great
American songs, period. It just hit me. I'm like, maybe Dixie isn't a
thing you want in your band name right now, or ever.

jenna wortham

But that's how it works, though, Wesley. And I really love when our
geographical Black starts showing. Like, that is such a Northern thing
to say because what's interesting about this intersection of our
conversation is that it's revealing the ways in which these ideas and
these terms are normalized, because either, A, they don't register as a
red flag in your case, and then, B, in my case, if I were to say
something, I would be gaslit. People would be like, oh, you're
sensitive. You make everything about race. I fully drove down I-95 in
the summer blasting Lady Antebellum as well, and fully being
uncomfortable, and then trying to parse it out in my mind, and being
like, yo, but what does that mean? Like, what is --- what's the deal,
and not really feeling like I could talk about it openly. And if I did,
I'd be shamed for it. So it's --- there are just the --- this is ---
it's really an exposition of about how race and racism works in this
country. It's that we're told it doesn't matter. We're told everything's
neutral. We're told not to pay attention to it when, in fact, it's all,
it's all so important.

wesley morris

Yeah.

jenna wortham

You know, I remember growing up and going to a yard sale in deep
Southern Virginia and there just being a Confederate flag, like that's
normal, right? I remember friends inviting me to come with them to see a
NASCAR race. And me being like, I can't go there, and them being like,
why not? Like the ways in which intuitively I understood these things to
be dangerous, but the popular culture kept insisting that they weren't.

wesley morris

Yeah. I've got to tell you, I was in love with Bo and Luke Duke for a
long time.

jenna wortham

(LAUGHS)

wesley morris

And for anybody who wasn't around for ``The Dukes of Hazard'' real show,
like in the 1980s, the car that Bo and Luke Duke drove was called the
General Lee. And emblazoned on the hood of the car was this Confederate
flag. And, you know, Bo and Luke Duke were always sliding across the
hood of the car and, therefore, that Confederate flag on the General
Lee. And I mean, the thing that was so sexy about them was those
tight-ass jeans they wore. And those jeans were flying back and forth
over that flag in so many episodes.

jenna wortham

Wild.

wesley morris

I don't recall a single Black person being on it. Therefore --- I guess
therefore, I don't recall there being a single racist incident. I think
the racist incident is the existence of the show itself in some ways.
And something about that transmitted itself from C.B.S. through the
airwaves, to my living room, to my psyche. And it was just like the
wrongness of these sexy men and this sexy Daisy Duke. All the sex on
that show, or sexiness, was tinged with a wrongness, you know? But the
thing about what we're talking about when we're talking about
normalization is a separate thing from really thinking about what it
would mean to then pretend that this thing that had always been normal
never even existed.

jenna wortham

Right.

wesley morris

And that is the thing that is sort of annoying me about this Aunt Jemima
situation, which is that we are not even going to do the work that's
required, I think, to understand what the problem even was in the first
place.

jenna wortham

Yeah, definitely.

wesley morris

Because there is a way to receive this Aunt Jemima news and just be
like, well, I guess we shouldn't put Black people on the cover of
things.

jenna wortham

No!

wesley morris

I guess we shouldn't put Black people on --- we shouldn't have Black
people advertising food.

jenna wortham

No!

wesley morris

Yeah, well, then we need to talk about why Aunt Jemima is an actual
problem.

jenna wortham

I mean, Wesley, you know, you and I are both kind of circling around
this idea of transformative justice, which is how do you repair harm
that's been done without creating more violence? That's just a very
basic way to talk about it. But what we're looking for is a framework to
move forward. And just hitting the delete button won't get us there. And
that's, I think, the thing that keeps getting stuck for both of us with
Aunt Jemima, right? And it's like all these other shifts that are
happening, they've been problematic for years. But, you know, the
Chicks, as they're calling themselves now, even in their statement, they
said, we want to respond to the moment. What they did not say, though,
was like, A, we've done some deep soul-searching. We realize the word
Dixie is racist and problematic, or it has this historical weight that
we no longer feel is appropriate. No, they literally just said, we have
FOMO. And listen, you and I both have professed our deep love for the
Dixie Chicks on this show. And, you know, that can be there, and also I
can lovingly hold them accountable and ask for more, which is to say
that I have a hard time believing that that's a new awareness for them.
And to sort of insist that exacerbates the problem that they're trying
to eradicate. So I don't know, Dixie Chicks. Take it from us. Don't do
that.

wesley morris

Also, if you're Lady Antebellum, by the --- I mean, just to --- while
we're in country music trio name changes, it seems like they hadn't
checked in with the Black blues singer who'd been going by the name Lady
A for like 20 years. So now they worked something out, it seemed. But
now Lady A took a second to think about what they worked out and was
like, mm, uh, mm, I don't think so. Now, there's more legal wrangling
between Lady A and Lady A. We'll see how that goes. But the larger
question is, changing something cosmetic like the name of your band is
just like the first step in the reckoning of something deeper with
respect to reconciliation and reparations or reconciliation being part
of reparations. And it brings up this other larger question of how do
you begin to right the harms done, both to consumers of these products
and to the people who are not being acknowledged in their sale and
proliferation? And so what does it mean for the faces of these products
to not be adequately or equally or justly compensated?

{[}cymbals{]}

wesley morris

So you and I should just take a quick break, get our breath. And then
when we come back, we're going to talk about reconciliation and
reparations.

{[}music - kindness, world restart{]}

jenna wortham

Speaking of reparations, which is a sentence I love to say, speaking of
reparations and Aunt Jemima, she was played by a number of Black women
over the years up until, what, the mid-`60s?

wesley morris

Mm-hm.

jenna wortham

So there are a lot of relatives of these actresses who have sought
restitution from Quaker. And most famously, there was a case in 2015,
where some of the descendants of a woman named Anna Short Harrington,
who played Aunt Jemima in the late `30s, they filed a lawsuit. They were
just like, in 1937, you created a trademark based on this imagery that
we believe is based on our relative. And you made a lot of money off of
her, money that none of us ever saw. And one of the bases for the claims
was that because the trademark was established in the `30s, there had
been billions made ---

wesley morris

Oh, wow.

jenna wortham

--- off of this likeness ---

wesley morris

Damn!

jenna wortham

--- because Quaker was bought by Pepsi, right?

wesley morris

Yes. Yes. Yes.

jenna wortham

At some point. So what these men wanted was \$2 billion in cash. And
they wanted a share of sales revenue. They wanted stock. They wanted all
the things that would be rightfully theirs if someone were to license a
trademark in your image and then make money off of it. But ultimately,
the judge on the case threw it out because he felt there wasn't enough
documentation that the men making the complaint were her heirs or that
there was an estate. So when you think about how difficult it is to find
historical Black records, I mean, I cannot reliably find a birth
certificate for my father. What our family says is the story of his
birth is very different from what the government has as a record of his
birth. So I bring that up to say, just because they were not able to
reliably prove her employment or their relation to her has --- it means
nothing. But that's not how it's seen in a court of law. And that's
definitely not how the media reported on this lawsuit in 2015.

wesley morris

Oh, yeah, I'm sure. I feel like one of the things that those descendants
of Miss Harrington are entitled to is to be able to tell a story. I
think that one of the things that has to happen in this moment of
erasure and correction is the story of what needs to be repaired. On the
one hand, there is this thing called reparations. And some very smart,
very educated people have done the work on what that would mean to
actually even talk about, let alone distribute. Ta-Nehisi Coates has
written ``The Case for Reparations.'' Nikole Hannah-Jones has written a
story as recently as last month called ``What Is Owed.'' What you and I
are talking about right now is another part of the reparation process
that makes the money make sense in some ways, at least to me. It's more
cultural, and psychic, and emotional than that, because we're not
talking about money. I'm talking about something much closer to truth
and reconciliation.

jenna wortham

Yes.

wesley morris

The process by which you lay out the harms that this country has
committed against any number of people, be it Indigenous Americans, be
it Black Americans, be it the Chinese and the Japanese and the Mexicans.
And this is why Aunt Jemima is such an amazing figure through which to
think about this. This is a woman who money is still being made off of
in 2020, whose fame --- I mean, this is the only word for it is fame ---
began in the Postbellum South and was connected to 60 years before that,
before the Civil War, to an art form that started in the 1830s. And so I
don't know how the descendents of those women aren't owed something for
the exploitative labor practices committed against them in some way. I
don't know if Nancy Green --- did she go to meetings? Did she have a say
in how her image was used and was proliferated?

jenna wortham

Definitely not.

wesley morris

She was probably proud of the work she did.

jenna wortham

Of course.

wesley morris

It was a big deal! But that representation should not have been free.
And there was a cost associated with all that pancake mixing because we
are still bearing that cost today. I, as a Black person, I live with the
awareness that one of the side effects of all this racist advertising,
and just the racist imagery that has been with us for all of these
centuries, is this kind of self-consciousness about my relationship to
food. And this is --- I am --- I mean, Jenna, I know you've got a little
bit of this, too.

jenna wortham

Mm-hm.

wesley morris

Many Black people in my life have this awareness of just not feeling
entirely comfortable, for instance, eating a watermelon in mixed
company. Fried chicken, for a long time, was a thing that I've really
had to sit and think about whether I wanted to eat fried chicken with
white people. I know it's insane.

jenna wortham

It's not.

wesley morris

But this is the direct result of a trauma that has been given to us that
proliferates through these racist images. And how do I know as like an
11- or 12-year-old that I don't want to be eating a watermelon in front
of white people? Why do I know that? Where did that come from? TV told
me. The ads told me. And so I'm thinking through the story of something
that needs to be repaired, right? And I don't think that changing the
face on the box of a pancake mix is really fixing anything.

jenna wortham

You know, one of the suggestions that's come from one of the descendants
of one of the actors that played Aunt Jemima --- it's so funny how murky
it is. But one of the things they've suggested is coming out with a
commemorative box that recognizes all the women who've portrayed Aunt
Jemima over the years. The back of the box could list their names. They
could spotlight different women. They could turn them into collector's
items. Don't get rid of the image of Aunt Jemima, but also show the
women. There are ideas out there. Like, people just aren't walking
around saying, you owe us money. People have ideas for what this process
of reconciliation can look like. You know, the women who played Aunt
Jemima weren't given the choice to shape how their image was being used.
And right now, there are the descendants of these women, and Black
consumers, who have real actionable ideas about how to preserve that
legacy, pain, dignity, all of it.

wesley morris

Yep.

jenna wortham

And we have the opportunity to be thoughtful. They have the opportunity
to listen and to make those choices. And look, if Quaker can take Aunt
Jemima on a road show all around the country and overseas, too. You
know, they went to the Paris World Fair as well, and Disneyland itself
had an entire restaurant about Aunt Jemima and devoted to pancakes.

wesley morris

Sure did.

jenna wortham

If they can do all that in the past, then they can certainly take the
steps to push it forward, and take the legacy of Aunt Jemima to both the
Blacksonian and Disneyland with the real story. Just commemorate her,
and unearth the history, and bring it into the future so that everybody
can learn from those mistakes.

wesley morris

Mm, yes.

jenna wortham

You know, there's a meme that's going around that I've really been
holding close throughout all of this. And it's a slide show. And the
first slide is the news announcing that non-white actors will no longer
voice characters on ``The Simpsons.'' There is a slide, ``The Office''
is taking down episodes featuring blackface. Realtors in some state are
no longer going to call the master bedroom the master bedroom anymore,
right? It kind of gets more ridiculous as they go on. And then the last
slide is a quote from Malcolm X. And it says, in very typical Malcolm X
fashion, but this has really been sticking with me and I'm going to hold
it close: ``The white man will try to satisfy us with symbolic
victories, rather than economic equity and real justice.'' And I really
think that sums up everything about this moment, that we have to be
really careful not to get bogged down in the symbolism of it all. And we
have to ask for real things that both look like that economic
renumeration, but also that cultural acknowledgment that we were here.
We had lives, and they mattered, you know? That's really all anybody
wants, and it's really not too much.

{[}music - kindness, world restart{]}

wesley morris

That's our show. ``Still Processing'' is a product of The New York
Times. And it was recorded in our living rooms.

jenna wortham

It is produced by Hans Buetow.

wesley morris

Our editors are Sara Sarasohn, Sasha Weiss, Wendy Dorr, and Lisa Tobin.

jenna wortham

Our engineer is Jake Gorski.

wesley morris

And our theme music's by Kindness. It's called ``World Restart,'' from
the album ``Otherness.''

jenna wortham

And per ushe, all of our episodes and various things are at
NYTimes.com/stillprocessing.

wesley morris

Thanks for listening, everybody. Be well. Be safe. Wear your mask.

jenna wortham

And sunscreen. Bye.

\href{https://www.nytimes3xbfgragh.onion/column/still-processing-podcast}{\includegraphics{https://static01.graylady3jvrrxbe.onion/images/2019/09/15/podcasts/still-processing-album-art-2/still-processing-album-art-2-square320.jpg}Still
Processing}Subscribe:

\begin{itemize}
\tightlist
\item
  \href{https://itunes.apple.com/us/podcast/id1151436460}{Apple
  Podcasts}
\item
  \href{https://www.google.com/podcasts?feed=aHR0cHM6Ly9yc3MuYXJ0MTkuY29tL255dC1zdGlsbC1wcm9jZXNzaW5n}{Google
  Podcasts}
\end{itemize}

\hypertarget{reparations-for-aunt-jemima-1}{%
\section{Reparations for Aunt
Jemima!}\label{reparations-for-aunt-jemima-1}}

\hypertarget{we-explore-the-erasure-of-aunt-jemima-and-the-reconciliation-of-that-icon-1}{%
\subsection{We explore the erasure of Aunt Jemima and the reconciliation
of that
icon.}\label{we-explore-the-erasure-of-aunt-jemima-and-the-reconciliation-of-that-icon-1}}

Hosted by Wesley Morris and Jenna Wortham. Produced by Hans Buetow.

Transcript

transcript

Back to Still Processing

bars

0:00/35:35

-0:00

transcript

\hypertarget{reparations-for-aunt-jemima-2}{%
\subsection{Reparations for Aunt
Jemima!}\label{reparations-for-aunt-jemima-2}}

\hypertarget{hosted-by-wesley-morris-and-jenna-wortham-produced-by-hans-buetow-1}{%
\subsubsection{Hosted by Wesley Morris and Jenna Wortham. Produced by
Hans
Buetow.}\label{hosted-by-wesley-morris-and-jenna-wortham-produced-by-hans-buetow-1}}

\hypertarget{we-explore-the-erasure-of-aunt-jemima-and-the-reconciliation-of-that-icon-2}{%
\paragraph{We explore the erasure of Aunt Jemima and the reconciliation
of that
icon.}\label{we-explore-the-erasure-of-aunt-jemima-and-the-reconciliation-of-that-icon-2}}

Thursday, July 16th, 2020

\begin{itemize}
\item
  wesley morris\\
  Jenna.
\item
  jenna wortham\\
  Yes, baby.
\item
  wesley morris\\
  It has been some summer.
\item
  jenna wortham\\
  Mm.
\item
  wesley morris\\
  George Floyd was killed in Minneapolis, Minn., on Memorial Day.
\item
  jenna wortham\\
  Right.
\item
  wesley morris\\
  We are now in July. And a lot of things have happened in the wake of
  his death.
\item
  jenna wortham\\
  Mm. Mm-hm, mm-hm.
\item
  wesley morris\\
  They are mind-blowing to think about.
\item
  jenna wortham\\
  Right.
\item
  wesley morris\\
  Is your seat belt on?
\item
  jenna wortham\\
  I'm buckled up, baby. Let's go.
\item
  {[}music{]}
\item
  wesley morris\\
  Monuments, statues being toppled and brought down.
\item
  jenna wortham\\
  Mm.
\item
  wesley morris\\
  There is giant Black Lives Matter murals going along the streets all
  over the country.
\item
  jenna wortham\\
  Mm.
\item
  wesley morris\\
  Kente scarves being worn by our Congress people.
\item
  jenna wortham\\
  Rude. Stop reminding me.
\item
  wesley morris\\
  We now work at a news organization that is capitalizing the identity
  Black. No more LEGO police sets for you kids. ``Cops'' has been
  canceled, blackface episodes of ``30 Rock'' and ``Community'' taken
  down. The N.F.L., league full of Black people, realizes that Black
  lives do matter, but never says that Colin Kaepernick does. Aunt
  Jemima, Uncle Ben retired. Lady Antebellum and Dixie Chicks are now
  Lady A and the Chicks?
\item
  jenna wortham\\
  Don't forget also, the Washington football team finally --- finally
  --- going to have a new name.
\item
  wesley morris\\
  Finally.
\item
  jenna wortham\\
  I mean, we could talk about each of those items for the rest of our
  lives. And we probably will. But I just want to dial in on one ---
\item
  wesley morris\\
  Sure.
\item
  jenna wortham\\
  --- which is Aunt Jemima, right?
\item
  wesley morris\\
  OK.
\item
  jenna wortham\\
  I don't want to wallpaper over that past because I think that lets the
  people responsible for that character and that caricature off way too
  easy. If we just disappear Aunt Jemima, then we're effectively erasing
  her, which does not heal that wound. It actually exacerbates it. It
  just covers it up with a Band-Aid, no bacitracin, you know?
\item
  wesley morris\\
  Oh, yeah. Please, we should definitely talk about Aunt Jemima and the
  pancake mix. Let's crack some eggs. We'll get out some water, and
  we'll just --- we're going to get right into some stirring.
\item
  {[}music - kindness, world restart{]}
\item
  jenna wortham\\
  I'm Jenna Wortham.
\item
  wesley morris\\
  I'm Wesley Morris. We're two culture writers at The New York Times,
  and recording in our living rooms today during a pandemic, a movement,
  and apparently a rainstorm.
\item
  jenna wortham\\
  (LAUGHS)
\item
  wesley morris\\
  Y'all gon' hear it.
\item
  jenna wortham\\
  This is ``Still Processing.''
\item
  {[}music{]}
\item
  jenna wortham\\
  So Wesley, in mid-June, Quaker makes this announcement --- we are
  going to retire Aunt Jemima. And the two people that are closest to me
  in my life both send me a text at around the same time, near identical
  text messages by the way. They contain an image of the pancake box and
  a text message basically that says, oh, S-H-I-T. And I think they
  wanted to know, what do I do with the box? And it's worth noting
  neither of them are white. But my immediate response back to both of
  them was, well, how full are the boxes?
\item
  wesley morris\\
  (LAUGHS)
\item
  jenna wortham\\
  Right? I mean, at this point, you might as well finish them. But when
  you do, can you save me the box?
\item
  wesley morris\\
  (LAUGHS) Why do you want the box?
\item
  jenna wortham\\
  You know, it's a part of history, right?
\item
  wesley morris\\
  Mm-hm.
\item
  jenna wortham\\
  And it's uncomfortable. But it feels really important to remember how
  America talked about Black people. And I know that you've done so much
  research into the history and the origins of minstrelsy in this
  country for this mysterious-ass book you're always working on it,
  which is why we can't hang out on the weekends, not that we'd be
  hanging out that much anyway these days. But it's helpful. You know a
  lot.
\item
  wesley morris\\
  Yes, that is true. I have been working on a book about, among many
  other things, the history of Black entertainment for the last 200
  years, it feels like at this point.
\item
  jenna wortham\\
  Well, Mr. Morris, I also did some research into the origins of this
  caricature because I was really interested in where the idea to use
  this Black woman to sell pancake mix came from. And what I found is
  fascinating. Are you ready?
\item
  wesley morris\\
  Oh, I'm ready.
\item
  jenna wortham\\
  So in 1889, this dude named Christopher Rutt and his friend Charles G.
  Underwood, they decide to buy a flour mill in St. Louis, Mo., which if
  you look at a map, is actually not that far from Minneapolis, Minn.,
  just saying. So in order to revive sales, they start making an instant
  pancake mix that all you have to do is add water. And they put it in
  these paper bags, and they sell it. And they realize that they need
  something to kind of give the brand a little zhuzh and distinguish it
  from other companies. So our old friend Chris Rutt --- and this is the
  part where the tea gets piping hot, honey --- this man edits a
  newspaper called The St. Joseph Gazette. So think about that. So this
  is someone who fully understands the power of narrative, who fully
  understands how images work, how media works. Apparently, he decides
  that the image they're going to use is going to be an image that he
  saw from an old, yeah, ``minstrel vaudeville'' --- I'm using that in
  quotes --- poster, where there was a figure that they lifted and
  transformed into what we call Aunt Jemima. There's also a part of the
  lore that goes, there was a song called ``Old Aunt Jemima,'' a
  minstrel song. And that served as inspiration as well.
\item
  wesley morris\\
  ``Old Aunt Jemima'' isn't just any minstrel song. A, it was a huge hit
  for Billy Kersands, who wrote and performed it like thousands of times
  all over the country. He might even have gone to England and done it a
  couple times, too. Written by a Black man who performed as a blackface
  minstrel because that's how it worked. It's the only way Black people
  could really get a foothold in the entertainment industry at any point
  in the 19th century. But the song is different from other minstrel
  songs in that it's about an old Black woman who is waiting for her
  mistress to die. But this bitch won't die. (LAUGHS)
\item
  jenna wortham\\
  (LAUGHS) Listen.
\item
  wesley morris\\
  She keeps not dying. And old Aunt Jemima is just like, when --- you
  said you were going to die and set me free. You said you were going to
  die. We had a deal. You just keep living, and I keep being enslaved.
  Why?
\item
  jenna wortham\\
  Well, so Chris Rutt, he decides he's going to use an image that he's
  pulled from maybe the song, maybe a poster. He's going to do it with
  the help of, you know, a Black woman who is in this stereotypical role
  as a, quote, ``mammy,'' right? That is essentially how you get Aunt
  Jemima on a pancake box.
\item
  wesley morris\\
  That Aunt Jemima looked like a classic minstrel figure. Big lipped.
  She's got these little wooden teeth, these --- they look like corn
  cobs. And these big sunken eyes that seem to have no human expression
  at all. And atop her head is this kerchief. And in some ways, this is
  the classic mammy figure that you would have gone to the theater to
  see a white man perform on stage. And so this image has basically been
  set in the minds of American popular culture for a long time by 1889.
  So it was kind of a no-brainer because these images are also, it's
  worth pointing out, pro-slavery images. They are pictures of domestic
  servitude that the servant herself seems more than happy to provide.
\item
  jenna wortham\\
  Yeah.
\item
  wesley morris\\
  But really, that was a common way to draw Black people at that point
  in time.
\item
  {[}rain{]}
\item
  wesley morris\\
  Here comes the rain again.
\item
  jenna wortham\\
  Oh, yeah, we're in the middle of it. It's juicy. I'm not mad. My
  plants aren't mad, either. OK, so this character has been created to
  sell the pancake mix. And a year later in 1890, an entrepreneur named
  R.T. Davis buys the whole company and decides, you know what, we're
  actually going to hire Black women to play the role of Aunt Jemima to
  further sell this product. And one of the places they do it is at the
  World's Fair in Chicago, Ill., in 1893. And a formerly enslaved woman
  named Nancy Green is hired to walk around the fair, do these pancake
  demos, presumably cook the pancakes, and feed them to people.
\item
  wesley morris\\
  She cooked the pancakes, yes.
\item
  jenna wortham\\
  And she's dressed like the character of Aunt Jemima, with the
  bandanna, and the apron, the whole nine.
\item
  wesley morris\\
  Yeah, I mean at that same World's Fair, Frederick Douglass and Ida B.
  Wells were walking around handing out pamphlets ---
\item
  jenna wortham\\
  Wow.
\item
  wesley morris\\
  --- basically criticizing the fair organizers for including no Black
  people in the planning of the fair itself.
\item
  jenna wortham\\
  Wow.
\item
  wesley morris\\
  Talk about the tension between being represented and being respected.
  It was all happening at that festival.
\item
  jenna wortham\\
  Wow.
\item
  wesley morris\\
  It was a really crazy time. I mean, you've got two of the world's most
  important Black people in Frederick Douglass and Ida B. Wells doing
  the work to try to bring attention to the sort of inherent unfairness
  of who gets to decide how Black people are represented at this fair.
\item
  jenna wortham\\
  Yes, absolutely.
\item
  wesley morris\\
  But then you have all --- you have like huge Black attendance coming
  to see all of the spectacles at the fair, including Nancy Green. The
  Pancake Queen, by the way, is what they called her.
\item
  jenna wortham\\
  Wow. Well, I'm really glad you brought that into the room, Wesley,
  because it's really important to note. So Nancy is one of the first of
  many women to play this role over the years, and including up until
  and after Quaker buys the brand. And many of the descendants,
  including the descendants of a woman named Lillian Richard, who was
  also hired to portray her, they talk about that inherited legacy as
  being a source of pride, because Black people and Black women never
  had public-facing roles. They never had jobs where they were meant to
  be seen and meant to be looked at. And so it's very complicated that
  that was her job. So it's not as simple as we might like to think,
  looking backwards. But you know, Wesley, one of the things they were
  asking these women to do, these actors, as they were portraying Aunt
  Jemima, they wanted them to reinforce the idea that Black people were
  happy in these subservient, domestic roles.
\item
  wesley morris\\
  Mm-hm.
\item
  archived recording 1\\
  Smiling, happy Aunt Jemima, famous for her secret recipe pancakes,
  waffles, and buckwheat.
\item
  archived recording 2\\
  And now, Aunt Jemima, one of your old plantation sings, if you will.
\item
  archived recording (aunt jemima)\\
  It's a pleasure, folks, to remind y'all that the Lord meant for the
  sky to be blue. But if the day is blue, it's probably our fault.
\item
  archived recording 2\\
  True, Aunt Jemima, true.
\end{itemize}

wesley morris

This is straight out of minstrelsy. She was a minstrel figure,
essentially. And let's just talk about the way that those women were
depicted on those boxes. And into this you can throw a caricature of a
person like Uncle Ben, or Rastus, the face of the Cream of Wheat, the
racistly named Rastus on the box of Cream of Wheat. Each of these people
is depicted how, Jenna? What are they doing with their faces on all
those boxes?

jenna wortham

I mean, they have a maniacal grin on their faces, you know?

wesley morris

Yeah.

jenna wortham

I mean, they look like a character from the movie ``Get Out.'' Like,
they look like they're being held hostage and being forced to enjoy it.

wesley morris

Yes. And that smile is the smile that has been haunting us for almost
200 years.

jenna wortham

Yeah.

wesley morris

It is the smile of servitude. It is the smile of complicity. It is the
smile that says, these people like the harm that we are doing to them
every day.

jenna wortham

They enjoy it.

wesley morris

We've been doing it for centuries.

jenna wortham

They like it.

wesley morris

But you see this smile? This smile means that it's OK that that harm is
being done. It is the smile that Black people know or associate with a
kind of complicity in their own victimization. It is a smile that
basically says, I'm not going to resist. It's associated with whatever
we mean when we say Uncle Tom, because it implies --- that sort of
smiling Black person implies that you're going along with this racist
system.

\begin{itemize}
\item
  archived recording (aunt jemima)\\
  (LAUGHS) Greetings, folks, greetings. This is your old friend, Aunt
  Jemima.
\item
  archived recording\\
  And pleased as punch with yourself this morning, aren't you, Aunt
  Jemima?
\item
  archived recording (aunt jemima)\\
  (LAUGHS) `Course I am, Mr. Lyon.
\item
  archived recording\\
  Smiling, happy Aunt Jemima. Smiling, happy Aunt Jemima, famous for her
  secret recipe pancakes, waffles, and buckwheat. What's the good word,
  Aunt Jemima?
\item
  archived recording (aunt jemima)\\
  Well, Mr. Lyon, folks says there's nothing so pretty as a happy face
  and nothing so worthwhile as a happy life.
\item
  archived recording\\
  Yes, Aunt Jemima, that is true.
\end{itemize}

wesley morris

And that's some powerful stuff. And it --- like what it does in terms of
the image that gets reinforced, it's very hard to see around that.

jenna wortham

Yeah.

wesley morris

This is generations of kids of all races raised on these images.

jenna wortham

It's funny to me, Wesley, how much of this I'm really only learning
because we're making an episode about it. You know, America works
really, really, really hard to hide this history from us. But it impacts
us, whether or not we talk about it or not, right?

wesley morris

Mm-hm.

jenna wortham

So for Quaker to say, ``Oh my gosh, you know what, we don't want to have
this image anymore, we realize it's totally racist. You guys have been
telling us for years. We finally agree. Black Pancake Lives Matter!''
And they decide to vaporize it. In doing that, you can't wave a magic
wand over centuries of subjugation and oppression. And in trying to do
that, I think you end up ripping open a bigger wound than you're able to
heal.

wesley morris

Yes, yes, yes. Part of what is happening right now in this country is
people becoming aware of how deep the roots go. And I think being forced
to think about --- I mean, listen, I'm as guilty of this as anybody,
right? I mean, for as much time as I spend thinking about, researching,
writing about, emoting about all of this history, some things just slip
through my cracks. And Lady Antebellum and the Dixie Chicks are two such
things. I will sit here and tell you that when they changed the names of
their bands.

jenna wortham

Uh-huh.

wesley morris

I was really like, oh, yeah. I guess that does make sense.

jenna wortham

Oh, my god.

wesley morris

And it's not that I don't know what Antebellum is probably referring to.
And it's not like I don't know what Dixie is. Lady Antebellum, I guess I
didn't think about it because ``Need You Now'' is just one of the great
American songs, period. It just hit me. I'm like, maybe Dixie isn't a
thing you want in your band name right now, or ever.

jenna wortham

But that's how it works, though, Wesley. And I really love when our
geographical Black starts showing. Like, that is such a Northern thing
to say because what's interesting about this intersection of our
conversation is that it's revealing the ways in which these ideas and
these terms are normalized, because either, A, they don't register as a
red flag in your case, and then, B, in my case, if I were to say
something, I would be gaslit. People would be like, oh, you're
sensitive. You make everything about race. I fully drove down I-95 in
the summer blasting Lady Antebellum as well, and fully being
uncomfortable, and then trying to parse it out in my mind, and being
like, yo, but what does that mean? Like, what is --- what's the deal,
and not really feeling like I could talk about it openly. And if I did,
I'd be shamed for it. So it's --- there are just the --- this is ---
it's really an exposition of about how race and racism works in this
country. It's that we're told it doesn't matter. We're told everything's
neutral. We're told not to pay attention to it when, in fact, it's all,
it's all so important.

wesley morris

Yeah.

jenna wortham

You know, I remember growing up and going to a yard sale in deep
Southern Virginia and there just being a Confederate flag, like that's
normal, right? I remember friends inviting me to come with them to see a
NASCAR race. And me being like, I can't go there, and them being like,
why not? Like the ways in which intuitively I understood these things to
be dangerous, but the popular culture kept insisting that they weren't.

wesley morris

Yeah. I've got to tell you, I was in love with Bo and Luke Duke for a
long time.

jenna wortham

(LAUGHS)

wesley morris

And for anybody who wasn't around for ``The Dukes of Hazard'' real show,
like in the 1980s, the car that Bo and Luke Duke drove was called the
General Lee. And emblazoned on the hood of the car was this Confederate
flag. And, you know, Bo and Luke Duke were always sliding across the
hood of the car and, therefore, that Confederate flag on the General
Lee. And I mean, the thing that was so sexy about them was those
tight-ass jeans they wore. And those jeans were flying back and forth
over that flag in so many episodes.

jenna wortham

Wild.

wesley morris

I don't recall a single Black person being on it. Therefore --- I guess
therefore, I don't recall there being a single racist incident. I think
the racist incident is the existence of the show itself in some ways.
And something about that transmitted itself from C.B.S. through the
airwaves, to my living room, to my psyche. And it was just like the
wrongness of these sexy men and this sexy Daisy Duke. All the sex on
that show, or sexiness, was tinged with a wrongness, you know? But the
thing about what we're talking about when we're talking about
normalization is a separate thing from really thinking about what it
would mean to then pretend that this thing that had always been normal
never even existed.

jenna wortham

Right.

wesley morris

And that is the thing that is sort of annoying me about this Aunt Jemima
situation, which is that we are not even going to do the work that's
required, I think, to understand what the problem even was in the first
place.

jenna wortham

Yeah, definitely.

wesley morris

Because there is a way to receive this Aunt Jemima news and just be
like, well, I guess we shouldn't put Black people on the cover of
things.

jenna wortham

No!

wesley morris

I guess we shouldn't put Black people on --- we shouldn't have Black
people advertising food.

jenna wortham

No!

wesley morris

Yeah, well, then we need to talk about why Aunt Jemima is an actual
problem.

jenna wortham

I mean, Wesley, you know, you and I are both kind of circling around
this idea of transformative justice, which is how do you repair harm
that's been done without creating more violence? That's just a very
basic way to talk about it. But what we're looking for is a framework to
move forward. And just hitting the delete button won't get us there. And
that's, I think, the thing that keeps getting stuck for both of us with
Aunt Jemima, right? And it's like all these other shifts that are
happening, they've been problematic for years. But, you know, the
Chicks, as they're calling themselves now, even in their statement, they
said, we want to respond to the moment. What they did not say, though,
was like, A, we've done some deep soul-searching. We realize the word
Dixie is racist and problematic, or it has this historical weight that
we no longer feel is appropriate. No, they literally just said, we have
FOMO. And listen, you and I both have professed our deep love for the
Dixie Chicks on this show. And, you know, that can be there, and also I
can lovingly hold them accountable and ask for more, which is to say
that I have a hard time believing that that's a new awareness for them.
And to sort of insist that exacerbates the problem that they're trying
to eradicate. So I don't know, Dixie Chicks. Take it from us. Don't do
that.

wesley morris

Also, if you're Lady Antebellum, by the --- I mean, just to --- while
we're in country music trio name changes, it seems like they hadn't
checked in with the Black blues singer who'd been going by the name Lady
A for like 20 years. So now they worked something out, it seemed. But
now Lady A took a second to think about what they worked out and was
like, mm, uh, mm, I don't think so. Now, there's more legal wrangling
between Lady A and Lady A. We'll see how that goes. But the larger
question is, changing something cosmetic like the name of your band is
just like the first step in the reckoning of something deeper with
respect to reconciliation and reparations or reconciliation being part
of reparations. And it brings up this other larger question of how do
you begin to right the harms done, both to consumers of these products
and to the people who are not being acknowledged in their sale and
proliferation? And so what does it mean for the faces of these products
to not be adequately or equally or justly compensated?

{[}cymbals{]}

wesley morris

So you and I should just take a quick break, get our breath. And then
when we come back, we're going to talk about reconciliation and
reparations.

{[}music - kindness, world restart{]}

jenna wortham

Speaking of reparations, which is a sentence I love to say, speaking of
reparations and Aunt Jemima, she was played by a number of Black women
over the years up until, what, the mid-`60s?

wesley morris

Mm-hm.

jenna wortham

So there are a lot of relatives of these actresses who have sought
restitution from Quaker. And most famously, there was a case in 2015,
where some of the descendants of a woman named Anna Short Harrington,
who played Aunt Jemima in the late `30s, they filed a lawsuit. They were
just like, in 1937, you created a trademark based on this imagery that
we believe is based on our relative. And you made a lot of money off of
her, money that none of us ever saw. And one of the bases for the claims
was that because the trademark was established in the `30s, there had
been billions made ---

wesley morris

Oh, wow.

jenna wortham

--- off of this likeness ---

wesley morris

Damn!

jenna wortham

--- because Quaker was bought by Pepsi, right?

wesley morris

Yes. Yes. Yes.

jenna wortham

At some point. So what these men wanted was \$2 billion in cash. And
they wanted a share of sales revenue. They wanted stock. They wanted all
the things that would be rightfully theirs if someone were to license a
trademark in your image and then make money off of it. But ultimately,
the judge on the case threw it out because he felt there wasn't enough
documentation that the men making the complaint were her heirs or that
there was an estate. So when you think about how difficult it is to find
historical Black records, I mean, I cannot reliably find a birth
certificate for my father. What our family says is the story of his
birth is very different from what the government has as a record of his
birth. So I bring that up to say, just because they were not able to
reliably prove her employment or their relation to her has --- it means
nothing. But that's not how it's seen in a court of law. And that's
definitely not how the media reported on this lawsuit in 2015.

wesley morris

Oh, yeah, I'm sure. I feel like one of the things that those descendants
of Miss Harrington are entitled to is to be able to tell a story. I
think that one of the things that has to happen in this moment of
erasure and correction is the story of what needs to be repaired. On the
one hand, there is this thing called reparations. And some very smart,
very educated people have done the work on what that would mean to
actually even talk about, let alone distribute. Ta-Nehisi Coates has
written ``The Case for Reparations.'' Nikole Hannah-Jones has written a
story as recently as last month called ``What Is Owed.'' What you and I
are talking about right now is another part of the reparation process
that makes the money make sense in some ways, at least to me. It's more
cultural, and psychic, and emotional than that, because we're not
talking about money. I'm talking about something much closer to truth
and reconciliation.

jenna wortham

Yes.

wesley morris

The process by which you lay out the harms that this country has
committed against any number of people, be it Indigenous Americans, be
it Black Americans, be it the Chinese and the Japanese and the Mexicans.
And this is why Aunt Jemima is such an amazing figure through which to
think about this. This is a woman who money is still being made off of
in 2020, whose fame --- I mean, this is the only word for it is fame ---
began in the Postbellum South and was connected to 60 years before that,
before the Civil War, to an art form that started in the 1830s. And so I
don't know how the descendents of those women aren't owed something for
the exploitative labor practices committed against them in some way. I
don't know if Nancy Green --- did she go to meetings? Did she have a say
in how her image was used and was proliferated?

jenna wortham

Definitely not.

wesley morris

She was probably proud of the work she did.

jenna wortham

Of course.

wesley morris

It was a big deal! But that representation should not have been free.
And there was a cost associated with all that pancake mixing because we
are still bearing that cost today. I, as a Black person, I live with the
awareness that one of the side effects of all this racist advertising,
and just the racist imagery that has been with us for all of these
centuries, is this kind of self-consciousness about my relationship to
food. And this is --- I am --- I mean, Jenna, I know you've got a little
bit of this, too.

jenna wortham

Mm-hm.

wesley morris

Many Black people in my life have this awareness of just not feeling
entirely comfortable, for instance, eating a watermelon in mixed
company. Fried chicken, for a long time, was a thing that I've really
had to sit and think about whether I wanted to eat fried chicken with
white people. I know it's insane.

jenna wortham

It's not.

wesley morris

But this is the direct result of a trauma that has been given to us that
proliferates through these racist images. And how do I know as like an
11- or 12-year-old that I don't want to be eating a watermelon in front
of white people? Why do I know that? Where did that come from? TV told
me. The ads told me. And so I'm thinking through the story of something
that needs to be repaired, right? And I don't think that changing the
face on the box of a pancake mix is really fixing anything.

jenna wortham

You know, one of the suggestions that's come from one of the descendants
of one of the actors that played Aunt Jemima --- it's so funny how murky
it is. But one of the things they've suggested is coming out with a
commemorative box that recognizes all the women who've portrayed Aunt
Jemima over the years. The back of the box could list their names. They
could spotlight different women. They could turn them into collector's
items. Don't get rid of the image of Aunt Jemima, but also show the
women. There are ideas out there. Like, people just aren't walking
around saying, you owe us money. People have ideas for what this process
of reconciliation can look like. You know, the women who played Aunt
Jemima weren't given the choice to shape how their image was being used.
And right now, there are the descendants of these women, and Black
consumers, who have real actionable ideas about how to preserve that
legacy, pain, dignity, all of it.

wesley morris

Yep.

jenna wortham

And we have the opportunity to be thoughtful. They have the opportunity
to listen and to make those choices. And look, if Quaker can take Aunt
Jemima on a road show all around the country and overseas, too. You
know, they went to the Paris World Fair as well, and Disneyland itself
had an entire restaurant about Aunt Jemima and devoted to pancakes.

wesley morris

Sure did.

jenna wortham

If they can do all that in the past, then they can certainly take the
steps to push it forward, and take the legacy of Aunt Jemima to both the
Blacksonian and Disneyland with the real story. Just commemorate her,
and unearth the history, and bring it into the future so that everybody
can learn from those mistakes.

wesley morris

Mm, yes.

jenna wortham

You know, there's a meme that's going around that I've really been
holding close throughout all of this. And it's a slide show. And the
first slide is the news announcing that non-white actors will no longer
voice characters on ``The Simpsons.'' There is a slide, ``The Office''
is taking down episodes featuring blackface. Realtors in some state are
no longer going to call the master bedroom the master bedroom anymore,
right? It kind of gets more ridiculous as they go on. And then the last
slide is a quote from Malcolm X. And it says, in very typical Malcolm X
fashion, but this has really been sticking with me and I'm going to hold
it close: ``The white man will try to satisfy us with symbolic
victories, rather than economic equity and real justice.'' And I really
think that sums up everything about this moment, that we have to be
really careful not to get bogged down in the symbolism of it all. And we
have to ask for real things that both look like that economic
renumeration, but also that cultural acknowledgment that we were here.
We had lives, and they mattered, you know? That's really all anybody
wants, and it's really not too much.

{[}music - kindness, world restart{]}

wesley morris

That's our show. ``Still Processing'' is a product of The New York
Times. And it was recorded in our living rooms.

jenna wortham

It is produced by Hans Buetow.

wesley morris

Our editors are Sara Sarasohn, Sasha Weiss, Wendy Dorr, and Lisa Tobin.

jenna wortham

Our engineer is Jake Gorski.

wesley morris

And our theme music's by Kindness. It's called ``World Restart,'' from
the album ``Otherness.''

jenna wortham

And per ushe, all of our episodes and various things are at
NYTimes.com/stillprocessing.

wesley morris

Thanks for listening, everybody. Be well. Be safe. Wear your mask.

jenna wortham

And sunscreen. Bye.

Previous

More episodes ofStill Processing

\href{https://www.nytimes3xbfgragh.onion/2020/07/23/podcasts/hamilton-ziwe-discomfort.html?action=click\&module=audio-series-bar\&region=header\&pgtype=Article}{\includegraphics{https://static01.graylady3jvrrxbe.onion/images/2020/07/23/multimedia/23stillprocessing-pix/23stillprocessing-pix-thumbLarge.jpg}}

July 23, 2020~~•~ 38:10Ziwe May Destroy Hamilton

\href{https://www.nytimes3xbfgragh.onion/2020/07/16/podcasts/reparations-for-aunt-jemima.html?action=click\&module=audio-series-bar\&region=header\&pgtype=Article}{\includegraphics{https://static01.graylady3jvrrxbe.onion/images/2020/07/18/multimedia/16stillprocessing-pix/16stillprocessing-pix-thumbLarge.jpg}}

July 16, 2020~~•~ 35:35Reparations for Aunt Jemima!

\href{https://www.nytimes3xbfgragh.onion/2020/07/09/podcasts/still-processing-black-lives-matter.html?action=click\&module=audio-series-bar\&region=header\&pgtype=Article}{\includegraphics{https://static01.graylady3jvrrxbe.onion/images/2020/07/12/podcasts/09stillprocessing-image/xx-stillprocessing-thumbLarge.jpg}}

July 9, 2020~~•~ 26:29So Y'all Finally Get It

\href{https://www.nytimes3xbfgragh.onion/2020/05/14/podcasts/still-processing-westworld-hollywood-utopia-dystopia.html?action=click\&module=audio-series-bar\&region=header\&pgtype=Article}{\includegraphics{https://static01.graylady3jvrrxbe.onion/images/2020/05/16/podcasts/14stillprocessing-image/14stillprocessing-image-thumbLarge-v2.jpg}}

May 14, 2020New Loop, America

\href{https://www.nytimes3xbfgragh.onion/2020/05/07/podcasts/still-processing-internet-vulnerability-sondheim-parks-recreation.html?action=click\&module=audio-series-bar\&region=header\&pgtype=Article}{\includegraphics{https://static01.graylady3jvrrxbe.onion/images/2020/04/28/pageoneplus/28sondheimjp-sp/28sondheimjp-sp-thumbLarge-v4.jpg}}

May 7, 2020Does This Phone Make Me Look Human?

\href{https://www.nytimes3xbfgragh.onion/2020/04/30/podcasts/still-processing-fiona-apple-fetch-bolt-cutters.html?action=click\&module=audio-series-bar\&region=header\&pgtype=Article}{\includegraphics{https://static01.graylady3jvrrxbe.onion/images/2020/05/03/multimedia/30stillpro-image/30stillpro-image-thumbLarge.jpg}}

May 1, 2020Fiona Ex Machina

\href{https://www.nytimes3xbfgragh.onion/2020/04/23/podcasts/still-processing-halle-berry-sharon-stone-catwoman-quarantine.html?action=click\&module=audio-series-bar\&region=header\&pgtype=Article}{\includegraphics{https://static01.graylady3jvrrxbe.onion/images/2020/04/25/arts/23stillprocessing/23stillprocessing-thumbLarge-v3.jpg}}

April 23, 2020Halle Berry? Hallelujah.

\href{https://www.nytimes3xbfgragh.onion/2020/04/16/podcasts/still-processing-AIDS-survive-coronavirus.html?action=click\&module=audio-series-bar\&region=header\&pgtype=Article}{\includegraphics{https://static01.graylady3jvrrxbe.onion/images/2020/04/20/us/16stillprocessing/16stillprocessing-thumbLarge-v3.jpg}}

April 16, 2020How to Learn From a Plague

\href{https://www.nytimes3xbfgragh.onion/2020/04/09/podcasts/still-processing-tiger-king.html?action=click\&module=audio-series-bar\&region=header\&pgtype=Article}{\includegraphics{https://static01.graylady3jvrrxbe.onion/images/2020/04/11/podcasts/09stillprocessing-image2/09stillprocessing-image2-thumbLarge-v2.jpg}}

April 9, 2020~~•~ 39:49Frosted Flakes

\href{https://www.nytimes3xbfgragh.onion/2020/04/02/podcasts/high-fidelity-zoe-kravitz.html?action=click\&module=audio-series-bar\&region=header\&pgtype=Article}{\includegraphics{https://static01.graylady3jvrrxbe.onion/images/2020/04/05/arts/02still-processing-highfidelity/13highfidelity-thumbLarge.jpg}}

April 2, 2020~~•~ 40:55Delicious Vinyl

\href{https://www.nytimes3xbfgragh.onion/2020/03/26/podcasts/still-processing-quarantine.html?action=click\&module=audio-series-bar\&region=header\&pgtype=Article}{\includegraphics{https://static01.graylady3jvrrxbe.onion/images/2020/03/29/podcasts/26stillprocessing1/26stillprocessing1-thumbLarge.jpg}}

March 26, 2020~~•~ 30:47A Pod From Both Our Houses

\href{https://www.nytimes3xbfgragh.onion/2019/11/07/podcasts/still-processing-parasite-watchmen-bong-joon-ho.html?action=click\&module=audio-series-bar\&region=header\&pgtype=Article}{\includegraphics{https://static01.graylady3jvrrxbe.onion/images/2019/11/08/arts/07stilpr-parasite/00parasite-1-thumbLarge.jpg}}

November 7, 2019Wake

\href{https://www.nytimes3xbfgragh.onion/column/still-processing-podcast}{See
All Episodes ofStill Processing}

Next

July 16, 2020

\begin{itemize}
\item
\item
\item
\item
\item
\end{itemize}

By \href{https://www.nytimes3xbfgragh.onion/by/wesley-morris}{Wesley
Morris} and
\href{https://www.nytimes3xbfgragh.onion/by/jenna-wortham}{Jenna
Wortham}

When Quaker decided to take Aunt Jemima off the red pancake box after
131 years, did it also try to scrub the legacy she represents? And what
sort of compensation is appropriate --- and to whom --- from a brand
that maintained that image in public for so long?

\includegraphics{https://static01.graylady3jvrrxbe.onion/images/2020/07/18/multimedia/16stillprocessing-pix/merlin_17630971_012c740c-3579-4869-8c3c-1119ba004dab-articleLarge.jpg?quality=75\&auto=webp\&disable=upscale}

Discussed this week:

\begin{itemize}
\item
  ``\href{https://www.nytimes3xbfgragh.onion/2020/06/17/business/media/aunt-jemima-racial-stereotype.html}{Aunt
  Jemima Brand to Change Name and Image Over `Racial Stereotype'}''
  (Tiffany Hsu, The New York Times, June 2020)
\item
  ``\href{https://youtu.be/3ipamH6EEwI}{Aunt Jemima: I'se in Town,
  Honey}'' (Jim Crow Museum of Racist Memorabilia at Ferris State
  University)
\item
  ``\href{https://www.nytimes3xbfgragh.onion/2020/06/25/arts/music/dixie-chicks-change-name.html}{The
  Dixie Chicks Change Their Name, Dropping the `Dixie'}'' (Ben Sisario,
  The New York Times, June 2020)
\item
  ``\href{https://www.nytimes3xbfgragh.onion/2020/07/08/arts/music/lady-antebellum-lady-a-lawsuit.html}{Lady
  Antebellum Sues the Singer Lady A Over Name Change}'' (Joe Coscarelli,
  The New York Times, July 2020)
\item
  ``\href{https://www.thewrap.com/aunt-jemima-3-billion-lawsuit-against-pepsi-quaker-oats-tossed-by-judge/}{Aunt
  Jemima's Heirs' \$3 Billion Lawsuit Against Pepsi, Quaker Oats Tossed
  by Judge}'' (Tim Kenneally, The Wrap, February 2015)
\item
  ``\href{https://www.nytimes3xbfgragh.onion/interactive/2020/06/24/magazine/reparations-slavery.html}{What
  Is Owed}'' (Nikole Hannah-Jones, The New York Times, June 2020)
\item
  ``\href{https://www.theatlantic.com/magazine/archive/2014/06/the-case-for-reparations/361631/?gclid=EAIaIQobChMIprGKz6bQ6gIVStbACh14Vgv2EAAYASAAEgJ6gPD_BwE}{The
  Case for Reparations}'' (Ta-Nehisi Coates, The Atlantic, June 2014)
\end{itemize}

``Still Processing'' is produced by Hans Buetow and edited by Sara
Sarasohn and Sasha Weiss, with editorial oversight from Wendy Dorr and
Lisa Tobin. Our engineer is Jake Gorski. Our theme music is by Kindness.
It's called ``World Restart,'' from the album ``Otherness.''

Advertisement

\protect\hyperlink{after-bottom}{Continue reading the main story}

\hypertarget{site-index}{%
\subsection{Site Index}\label{site-index}}

\hypertarget{site-information-navigation}{%
\subsection{Site Information
Navigation}\label{site-information-navigation}}

\begin{itemize}
\tightlist
\item
  \href{https://help.nytimes3xbfgragh.onion/hc/en-us/articles/115014792127-Copyright-notice}{©~2020~The
  New York Times Company}
\end{itemize}

\begin{itemize}
\tightlist
\item
  \href{https://www.nytco.com/}{NYTCo}
\item
  \href{https://help.nytimes3xbfgragh.onion/hc/en-us/articles/115015385887-Contact-Us}{Contact
  Us}
\item
  \href{https://www.nytco.com/careers/}{Work with us}
\item
  \href{https://nytmediakit.com/}{Advertise}
\item
  \href{http://www.tbrandstudio.com/}{T Brand Studio}
\item
  \href{https://www.nytimes3xbfgragh.onion/privacy/cookie-policy\#how-do-i-manage-trackers}{Your
  Ad Choices}
\item
  \href{https://www.nytimes3xbfgragh.onion/privacy}{Privacy}
\item
  \href{https://help.nytimes3xbfgragh.onion/hc/en-us/articles/115014893428-Terms-of-service}{Terms
  of Service}
\item
  \href{https://help.nytimes3xbfgragh.onion/hc/en-us/articles/115014893968-Terms-of-sale}{Terms
  of Sale}
\item
  \href{https://spiderbites.nytimes3xbfgragh.onion}{Site Map}
\item
  \href{https://help.nytimes3xbfgragh.onion/hc/en-us}{Help}
\item
  \href{https://www.nytimes3xbfgragh.onion/subscription?campaignId=37WXW}{Subscriptions}
\end{itemize}
