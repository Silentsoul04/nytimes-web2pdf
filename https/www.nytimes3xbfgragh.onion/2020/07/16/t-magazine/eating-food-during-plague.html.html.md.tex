Sections

SEARCH

\protect\hyperlink{site-content}{Skip to
content}\protect\hyperlink{site-index}{Skip to site index}

\href{https://myaccount.nytimes3xbfgragh.onion/auth/login?response_type=cookie\&client_id=vi}{}

\href{https://www.nytimes3xbfgragh.onion/section/todayspaper}{Today's
Paper}

What We Eat During a Plague

\url{https://nyti.ms/3jaEW7E}

\begin{itemize}
\item
\item
\item
\item
\item
\end{itemize}

\href{https://www.nytimes3xbfgragh.onion/spotlight/at-home?action=click\&pgtype=Article\&state=default\&region=TOP_BANNER\&context=at_home_menu}{At
Home}

\begin{itemize}
\tightlist
\item
  \href{https://www.nytimes3xbfgragh.onion/2020/07/28/books/time-for-a-literary-road-trip.html?action=click\&pgtype=Article\&state=default\&region=TOP_BANNER\&context=at_home_menu}{Take:
  A Literary Road Trip}
\item
  \href{https://www.nytimes3xbfgragh.onion/2020/07/29/magazine/bored-with-your-home-cooking-some-smoky-eggplant-will-fix-that.html?action=click\&pgtype=Article\&state=default\&region=TOP_BANNER\&context=at_home_menu}{Cook:
  Smoky Eggplant}
\item
  \href{https://www.nytimes3xbfgragh.onion/2020/07/27/travel/moose-michigan-isle-royale.html?action=click\&pgtype=Article\&state=default\&region=TOP_BANNER\&context=at_home_menu}{Look
  Out: For Moose}
\item
  \href{https://www.nytimes3xbfgragh.onion/interactive/2020/at-home/even-more-reporters-editors-diaries-lists-recommendations.html?action=click\&pgtype=Article\&state=default\&region=TOP_BANNER\&context=at_home_menu}{Explore:
  Reporters' Obsessions}
\end{itemize}

Advertisement

\protect\hyperlink{after-top}{Continue reading the main story}

Supported by

\protect\hyperlink{after-sponsor}{Continue reading the main story}

\hypertarget{what-we-eat-during-a-plague}{%
\section{What We Eat During a
Plague}\label{what-we-eat-during-a-plague}}

Over the past months, Americans have embraced comfort food with a
renewed fervor. But this isn't the first time culinary habits have
shifted during a pandemic.

\includegraphics{https://static01.graylady3jvrrxbe.onion/images/2020/07/14/t-magazine/14tmag-plague/14tmag-plague-articleLarge.jpg?quality=75\&auto=webp\&disable=upscale}

By Michael Snyder

\begin{itemize}
\item
  Published July 16, 2020Updated July 17, 2020
\item
  \begin{itemize}
  \item
  \item
  \item
  \item
  \item
  \end{itemize}
\end{itemize}

\href{https://www.nytimes3xbfgragh.onion/es/2020/07/23/t-magazine/comida-pandemia.html}{Leer
en español}

AT THE OUTSET of
``\href{https://www.nytimes3xbfgragh.onion/interactive/2020/07/07/magazine/decameron-project-short-story-collection.html}{The
Decameron},'' the 14th-century story collection by the Italian writer
Giovanni Boccaccio, a group of 10 young nobles ---~seven women and three
men --- flee ``the death-dealing pestilence'' sweeping through Florence
and make their way to a country repast in the Tuscan hills. ``Using very
temperately of the most delicate viands and the finest wines and
eschewing all incontinence,'' Boccaccio writes, as translated by John
Payne, of their carefree, 10-day idyll, ``they abode with music and such
other diversions as they might have, never suffering themselves to speak
with any nor choosing to hear any news from without of death or sick
folk.'' They dined off ``tables laid with the whitest of cloths and
beakers that seemed of silver,'' sustaining themselves according to the
common medical wisdom of the day, which held that a cheerful disposition
was as necessary to keep the plague at bay as the right kind of food.

Boccaccio never describes these feasts in detail, but it's easy to guess
what his nobles might have eaten: rich banquets of wild birds and veal
spiced with pepper, cinnamon and nutmeg imported at great expense from
Asia --- and white bread, sliced and crustless, the only kind considered
suitable for the wealthy. Vegetables, deemed lowly and unwholesome, and
thus fit for the laity, might have been missing from the table. Diets at
the time, for rich and poor alike, were based off the humoral science of
the ancient Greeks, which held that unevenness between the body's four
humors --- blood, phlegm, choler (yellow bile) and melancholy (black
bile) --- caused every kind of ailment. Once consumed, food was thought
to become blood and then flesh, having the potential to recalibrate the
body's humoral balance, which could affect, or indeed transform, a
person's constitution. Every food possessed fixed humoral qualities ---
fennel was hot and dry, cucumber was cold and wet --- and assigned a
place in a rigid cosmic hierarchy. While peasants ate foods like
cabbages and turnips that grew near the ground, along with whole-grain
breads and thick, stodgy gruels, aristocrats feasted on airborne fowl,
sometimes **** dressed, says
\href{https://www.pacific.edu/academics/schools-and-colleges/college-of-the-pacific/academics/departments-and-programs/history/faculty-directory/ken-albala.html}{Ken
Albala}, a historian at the University of the Pacific, ``in completely
whimsical, shocking disguises'' --- tinged with dye, suspended in aspic
(a medieval invention) or stitched together into fantastical creatures.
Those underlying principles didn't change at the height of the Black
Death, which arrived in Europe around 1347, but dietary recommendations
did ``become less daring,'' Albala adds, with physicians at the time
suggesting ``mild foods unlikely to corrupt into melancholy or upset the
system in any way, which is, coincidentally, what people do
psychologically in any time of stress.'' Even centuries ago, times of
crisis induced
\href{https://www.nytimes3xbfgragh.onion/2020/04/07/business/coronavirus-processed-foods.html}{a
return to the familiar}.

\includegraphics{https://static01.graylady3jvrrxbe.onion/images/2020/07/14/t-magazine/14tmag-plague-02/14tmag-plague-02-articleLarge.jpg?quality=75\&auto=webp\&disable=upscale}

SINCE MARCH, NEWSPAPERS, magazines, lifestyle websites and, of course,
social media feeds have bloated with images of
\href{https://www.nytimes3xbfgragh.onion/2020/04/24/dining/focaccia-bread.html}{focaccia}
and
\href{https://www.nytimes3xbfgragh.onion/2020/04/11/science/sourdough-bread-starter-library.html}{sourdough},
\href{https://www.nytimes3xbfgragh.onion/2020/03/22/business/coronavirus-beans-sales.html}{beans}
and
\href{https://www.nytimes3xbfgragh.onion/2020/04/07/realestate/home-farming-tips-coronavirus.html}{ferments},
glossy-skinned
\href{https://www.nytimes3xbfgragh.onion/2020/07/03/dining/the-most-delicious-chicken.html}{chickens}
and fat-slicked
\href{https://cooking.nytimes3xbfgragh.onion/recipes/1020845-slow-cooker-pot-roast}{pot
roasts}: rich, flavorful dishes that, for the most part, Boccaccio might
have recognized. Following the recent shift toward plant-based cooking
and the boom in boutique dietary restrictions --- sales of gluten-free
products, for instance, have grown enormously over the last decade,
while the past several years have seen immense investments in
tech-fueled meat replacements --- these images are striking in their
apparent indifference to the dogmas of so-called ``clean'' eating.
Indeed, in their flagrant carnality, the comfort foods of the novel
coronavirus crisis can seem practically medieval, particularly in their
flouting of health trends in favor of comfort.

Such comfort foods, according to the dominant paradigm of Anglo-American
food culture, are almost invariably bad for us --- balms for the soul
but never what the body \emph{needs,} at least not nutritionally. But
there's a paradox therein: In medieval Europe, as in many of the world's
food cultures today, comfort and health were inseparable; pleasure and
familiarity were among the guideposts to maintaining the body's
equilibrium, a notion that persisted in popular thought even as medical
science transformed over the centuries.

When Spanish invaders brought a catastrophic outbreak of smallpox and
measles to the Americas in the 16th century, for instance, some
colonizers blamed the unfathomable crisis that ensued not on disease,
but on the same unfamiliar meats and wines introduced from Europe that
they claimed would ``civilize'' native populations (deaths among their
own kind, meanwhile, were attributed to local ingredients like corn and
chilies). To the Spaniards, eating unfamiliar foods could either
transform you or kill you. By the end of the 18th century, the
Enlightenment idea that all bodies --- at least all white male bodies
--- were fundamentally the same made humoral medicine seem largely
outdated, yet outside a small medical elite, food remained a principal
tool for treating disease. In the antebellum South, says
\href{https://afamstudies.yale.edu/people/carolyn-roberts}{Carolyn
Roberts}, a Yale historian focused on medicine and the slave trade,
enslaved Black healers remained **** a first line of defense against
disease in their communities, combining food-based medical knowledge
with local botanicals to blend healing traditions from Africa and the
Americas, even after hospitals became more common. **** In his ``An
Account of the Bilious Remitting Yellow Fever, as it Appeared in the
City of Philadelphia, in the Year 1793,'' the physician Benjamin Rush, a
proponent of modern medicine, nevertheless
\href{https://collections.nlm.nih.gov/catalog/nlm:nlmuid-2569009R-bk}{prescribes}
``lemonade, tamarind, jelly, and raw apple water, toast and water
\ldots{} and camomile tea,'' along with mercury-based treatments, during
the early stages of the illness and, as healing progressed, a course of
``rich broths, the flesh of poultry, oysters, thick gruel, mush and milk
and chocolate.'' Recommended diets during the 1918 influenza pandemic
were practically identical, including meat broths and citrus juices to
stave off fever and oatmeal, potato soups, custards and toast as the
patient recovered. Even **** the folk adage to ``feed a cold, starve a
fever'' contains vestiges of that humoral sensibility.

But what \emph{did} change was the way that many Europeans and Americans
related to their bodies outside of illness. The same Enlightenment
ideals that yielded political revolutions and, on the flip side,
justified colonialism on the basis of European superiority as a supposed
biological imperative, later remade how the aristocracy dined: Coursed
meals, where every diner ate the same thing at the same time, replaced
vast banquets, where everyone chose the food that best suited his or her
constitution. Later, in the 19th century, breakthroughs in chemistry and
the discovery of germs as a vector for disease broke humans down into
agglomerations of fats and proteins. ``You no longer had a right to have
views on what your body required: what is required is a scientific
fact,'' says
\href{https://warwick.ac.uk/fac/arts/history/people/staff_index/earle/}{Rebecca
Earle}, a food historian at the University of Warwick. ``And your
appetite is just a problem as far as nutritional science is concerned.''

That same authoritative attitude persisted in the 20th century in the
form of diet culture, which still treats having the ``wrong'' body as a
sign of moral sickness. In the early days of the H.I.V./AIDS epidemic,
the assimilationist wing of the gay community relied on a similar
philosophy, recalls the Oakland-based food writer
\href{https://www.john-birdsall.com/}{John Birdsall}, the argument being
that, \emph{if you eat well then that will stave off the infection}.
Hedonism, the wider culture insisted, had brought this plague down upon
gay people; austerity, in the form of fatless macrobiotic diets and
then-nascent American vegetarianism, could prevent it.

At the same time, the more radical side of the queer movement insisted
that gustatory pleasure could save queer bodies, just as claiming a
right to sexual pleasure had saved their souls. In his 1990s-era cooking
column ``Get Fat, Don't Die!'' published for nearly a decade in the
darkly humorous San Francisco zine
``\href{https://calisphere.org/collections/22661/?fbclid=IwAR2XcihgRMuSlZePJfzGuwyfZhaKWUutxnHqZQworbMUDdoOj0wpYYNf-5s}{Diseased
Pariah News},'' the activist Beowulf Thorne, writing under the pseudonym
Biffy Mae, prescribed meals of cereal with cream or half-and-half,
gingerbread puddings and Thai curries with the same enthusiasm that Rush
reserved for meat broths and weak tea. As Jonathan Kauffman wrote in
\href{https://hazlitt.net/longreads/get-fat-dont-die}{his recent
article} for Hazlitt, Thorne ``mocked nutritional supplements marketed
to people with AIDS, and leaned into Bisquick, his tastes alternately
cosmopolitan and straight-from-the-box comforting.''

Image

An anonymous miniature entitled ``The Meal'' from Giovanni Boccaccio's
``The Decameron'' (1432) at the Bibliothèque Nationale de
France.Credit...Album/Alamy Stock Photo

Image

A woodcut by Leonhard Thurneysser (1531-1596), circa the 16th century,
depicting the four elements of ancient Greek humoralism --- blood,
phlegm, choler (yellow bile) and melancholy (black bile) --- on which
diets were based in the 14th century.Credit...Interfoto/Alamy Stock
Photo

``ALTERNATELY COSMOPOLITAN AND straight-from-the-box comforting'' more
or less encapsulates the core of
\href{https://www.nytimes3xbfgragh.onion/article/recipes-cooking-tips-coronavirus.html}{today's
quarantine home-cooking}. The foods that have come to dominate social
media --- from lasagna to congee, Tamil omelet curry to huevos
rancheros, sourdough pancakes to kimchi jjigae (with homemade kimchi, of
course) --- combine the limited ingredients available at grocery stores
with the one commodity that is still in ample supply: time. Birdsall,
after a few weeks of elaborate meals, has returned in recent months to
thrift and simplicity, imbuing his back-to-basics dinners of braised
greens and perfectly seared hamburger patties with a monastic attention
to detail that, he says, ``creates a halo around these limited
ingredients.''
\href{https://www.wildfermentation.com/who-is-sandorkraut/}{Sandor Ellix
Katz}, whose books ``Wild Fermentation'' (2003) and ``The Art of
Fermentation'' (2012) ** helped drive the fermentation revival of the
last 15 years --- and who came to fermentation himself in the early
1990s --- says his online sourdough **** classes now draw up to 1,000
students each session. In this time of disease and uncertainty, the
making of artisanal foodstuffs that many people would previously have
left to professionals --- buying their bread at a bakery, their pickles
from a deli, their kimchi from a Korean grocer --- has replaced physical
fitness as one sign of aspirational care. Corporeal pleasure has once
again become a signal, if not of physical health, then at least of
mental health, as fundamental to surviving this plague as it was to
surviving the Black Death.

But while cooking has brought comfort and meaning into countless homes,
it has also highlighted stark global disparities.
\href{https://sph.umich.edu/news/2020posts/coronavirus-pandemic-worsens-food-insecurity-for-low-income-adults.html}{One
recent study} out of the University of Michigan's School of Public
Health found that 44 percent of 1,500 low-income American households
surveyed in late March were already experiencing food insecurity. In
\href{https://www.nytimes3xbfgragh.onion/interactive/2020/world/americas/mexico-coronavirus-cases.html}{Mexico},
where a nominally leftist president has suggested that eating healthy
foods instead of junk could prevent contagion,
\href{https://www.washingtonpost.com/world/2020/06/21/coronavirus-mexico-city-centro-abasto-market/?arc404=true}{dozens
of merchants have died} in Latin America's largest wholesale produce
market. In
\href{https://www.nytimes3xbfgragh.onion/interactive/2020/world/asia/india-coronavirus-cases.html}{India},
millions starve as they flee cities for villages, even while the
government stockpiles
\href{https://www.bloomberg.com/news/articles/2020-03-24/india-has-enough-food-to-feed-poor-amid-prolonged-shutdown-fears}{unprecedented
quantities of grain}. Like previous pandemics, Covid-19 has killed the
poor quicker and in greater numbers. If the foods we crave and cook have
come to resemble a medieval feast, maybe it's because our society has
been medieval all along.

Still, the bubonic plague, for all its horror, was no apocalypse --- and
Europe's Middle Ages were not actually a time of darkness or stagnation.
The tragic deaths of tens of millions in Europe generated a labor
shortage that, over the course of more than a century, allowed the labor
class to demand higher wages, accumulate modest family wealth and even
change their diets, incorporating meat that had previously been
accessible only to the aristocracy. The 15th century heralded the
proliferation of Europe's first published cookbooks as people of
middling ranks sought to emulate the cooking of the aristocracy,
complete with spices such as clove, galangal and long pepper that were
previously beyond their reach. The innovations often associated with the
Renaissance emerged out of revolutions in politics, education, art and
philosophy set in motion centuries before, often inspired and fed by the
same commercial and cultural exchanges that facilitated the spread of
disease in the first place.

Our generation's pandemic has come with a revolution of its own, one
that has spread even faster than the virus. Calls for justice and
political change replaced the loving pictures of split sourdough loaves,
dan dan noodles gleaming with chile oil and turmeric-stained bowls of
khichdi, the rice-and-lentil porridge served in countless variations
across South Asia as both an iconic comfort food and, in times of
illness, a **** tonic. In the last couple months, we've witnessed the
toppling of bastions of
\href{https://www.nytimes3xbfgragh.onion/2020/06/29/dining/john-t-edge-southern-foodways-alliance.html}{white
food culture} along with monuments that commemorate a shameful history
of racism and colonization, a movement --- led by people of color ---
that demands, yet again, the sort of political equality that the
Enlightenment failed to deliver. It also seems to require a return to a
much older understanding of our bodies as fluid and changeable, each
with its own way of healing, its own individual kind of comfort.
Restricted as they were by class and access, perhaps the foods that
unfurled across Instagram for all those months were a glimpse of a food
culture that matches a new society, one that relies not on self-denial
or appropriation or facile notions of unity, but looks instead like a
medieval banquet refracted through Thorne's proposed comfort and
cosmopolitanism: an endless table, a fantastical bounty, with space for
every kind of body and every kind of desire.

Advertisement

\protect\hyperlink{after-bottom}{Continue reading the main story}

\hypertarget{site-index}{%
\subsection{Site Index}\label{site-index}}

\hypertarget{site-information-navigation}{%
\subsection{Site Information
Navigation}\label{site-information-navigation}}

\begin{itemize}
\tightlist
\item
  \href{https://help.nytimes3xbfgragh.onion/hc/en-us/articles/115014792127-Copyright-notice}{©~2020~The
  New York Times Company}
\end{itemize}

\begin{itemize}
\tightlist
\item
  \href{https://www.nytco.com/}{NYTCo}
\item
  \href{https://help.nytimes3xbfgragh.onion/hc/en-us/articles/115015385887-Contact-Us}{Contact
  Us}
\item
  \href{https://www.nytco.com/careers/}{Work with us}
\item
  \href{https://nytmediakit.com/}{Advertise}
\item
  \href{http://www.tbrandstudio.com/}{T Brand Studio}
\item
  \href{https://www.nytimes3xbfgragh.onion/privacy/cookie-policy\#how-do-i-manage-trackers}{Your
  Ad Choices}
\item
  \href{https://www.nytimes3xbfgragh.onion/privacy}{Privacy}
\item
  \href{https://help.nytimes3xbfgragh.onion/hc/en-us/articles/115014893428-Terms-of-service}{Terms
  of Service}
\item
  \href{https://help.nytimes3xbfgragh.onion/hc/en-us/articles/115014893968-Terms-of-sale}{Terms
  of Sale}
\item
  \href{https://spiderbites.nytimes3xbfgragh.onion}{Site Map}
\item
  \href{https://help.nytimes3xbfgragh.onion/hc/en-us}{Help}
\item
  \href{https://www.nytimes3xbfgragh.onion/subscription?campaignId=37WXW}{Subscriptions}
\end{itemize}
