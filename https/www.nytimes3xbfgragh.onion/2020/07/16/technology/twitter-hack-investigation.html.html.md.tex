Sections

SEARCH

\protect\hyperlink{site-content}{Skip to
content}\protect\hyperlink{site-index}{Skip to site index}

\href{https://www.nytimes3xbfgragh.onion/section/technology}{Technology}

\href{https://myaccount.nytimes3xbfgragh.onion/auth/login?response_type=cookie\&client_id=vi}{}

\href{https://www.nytimes3xbfgragh.onion/section/todayspaper}{Today's
Paper}

\href{/section/technology}{Technology}\textbar{}Twitter Struggles to
Unpack a Hack Within Its Walls

\href{https://nyti.ms/2Zzh0D9}{https://nyti.ms/2Zzh0D9}

\begin{itemize}
\item
\item
\item
\item
\item
\end{itemize}

Advertisement

\protect\hyperlink{after-top}{Continue reading the main story}

Supported by

\protect\hyperlink{after-sponsor}{Continue reading the main story}

\hypertarget{twitter-struggles-to-unpack-a-hack-within-its-walls}{%
\section{Twitter Struggles to Unpack a Hack Within Its
Walls}\label{twitter-struggles-to-unpack-a-hack-within-its-walls}}

Even some basic questions about how an array of Twitter's most popular
accounts were taken over remain unanswered.

\includegraphics{https://static01.graylady3jvrrxbe.onion/images/2020/07/16/business/16twitter/merlin_170383824_eb9a7dad-a1da-4ef7-8b36-6be12f299f4b-articleLarge.jpg?quality=75\&auto=webp\&disable=upscale}

By \href{https://www.nytimes3xbfgragh.onion/by/mike-isaac}{Mike Isaac},
\href{https://www.nytimes3xbfgragh.onion/by/sheera-frenkel}{Sheera
Frenkel} and
\href{https://www.nytimes3xbfgragh.onion/by/kate-conger}{Kate Conger}

\begin{itemize}
\item
  Published July 16, 2020Updated July 31, 2020
\item
  \begin{itemize}
  \item
  \item
  \item
  \item
  \item
  \end{itemize}
\end{itemize}

As forensic investigators at Twitter hurried on Wednesday to discover
the origin of one of the
\href{https://www.nytimes3xbfgragh.onion/2020/07/17/technology/twitter-hackers-interview.html}{worst
hacks in the company's history}, the team came to a startling
conclusion: The hack was coming from an account inside the house.

But even by Thursday afternoon, 24 hours after hackers pushed a Bitcoin
scam from the accounts of political leaders like Joseph R. Biden Jr. and
industry titans like Elon Musk, the company's researchers were still
struggling to nail down many other basic aspects of the breach,
including whether an employee had been willfully complicit. The company
was also still sorting out how many accounts were affected, and whether
the attackers had gained access to details within the accounts, such as
private messages.

A few things were certain. Investigators know that at least one
employee's account and credentials were taken over and used to gain
access to an internal dashboard, allowing the infiltrator to control
most Twitter accounts, according to two people briefed on the company's
investigation. They would speak only anonymously because the
investigation was still underway. Yet many of the details remained
unclear, the people said. Investigators were still trying to determine
if the hackers tricked the employee into handing over login information.
Twitter suggested on Wednesday that the hackers had used ``social
engineering,'' a strategy to gain passwords or other personal
information by posing as a trusted person like a company representative.
But another line of inquiry includes whether a Twitter employee was
bribed for their credentials, something one person who claimed
responsibility for the hack told the technology site
\href{https://www.vice.com/en_us/article/jgxd3d/twitter-insider-access-panel-account-hacks-biden-uber-bezos}{Motherboard}.

The Federal Bureau of Investigation said it was looking into the hack.
``At this time, the accounts appear to have been compromised in order to
perpetuate cryptocurrency fraud,'' the agency said in a statement. ``We
advise the public not to fall victim to this scam by sending
cryptocurrency or money in relation to this incident.''

Twitter said in a statement, ``We've taken steps to further secure our
systems and will continue to share what we learn through our
investigation.''

On Thursday evening, Twitter said 130 accounts had been targeted in the
incident. The attackers were able to send tweets from a smaller subset
of accounts, the company said. It was still unclear whether private data
like direct messages had been accessed.

The hack, and the company's inability to quickly figure out what
happened, is a major embarrassment for Twitter. Over the past year, in
response to damaging revelations that disinformation spread widely on
the service during the 2016 presidential election, Jack Dorsey, the
chief executive, put a priority on promoting healthy and trustworthy
tweets. The hack of high-profile accounts to share a scam showed that
Twitter remained unprepared for the security threats it faces.

The attack also raised questions about election security, especially
since political leaders were among those attacked. If the messages sent
by hackers were political in nature instead of a financial scam ---
perhaps about closed polling sites on Election Day --- that could
manipulate turnout.

President Trump's account was not affected by the breach, Kayleigh
McEnany, the White House press secretary, said on Thursday. Mr. Trump's
account got extra protection after past incidents, according to a senior
administration official and a Twitter employee, who would speak only
anonymously because the security measures were private.

The Senate Select Committee on Intelligence said it would request
information from Twitter about the hack. ``The ability of bad actors to
take over prominent accounts, even fleetingly, signals a worrisome
vulnerability in this media environment, exploitable not just for scams
but for more impactful efforts to cause confusion, havoc and political
mischief,'' said Senator Mark Warner, Democrat of Virginia, the vice
chairman of the committee.

The attack on Wednesday came in waves. First, attackers used their
access to Twitter's internal tools to take over accounts with
\href{https://medium.com/@lucky225/the-twitter-hack-what-exactly-happened-d8740d33c1c}{distinctive
user names like @6}, an account that once belonged to the security
researcher and hacker Adrian Lamo. Then the attack hit the Twitter
accounts of prominent cryptocurrency leaders and companies. The next
wave included many of the most popular accounts, including those
belonging to political leaders, industry titans and top entertainers.

The messages were a version of a long-running scam in which hackers pose
as public figures on Twitter and promise to match or even triple any
funds that are sent to their Bitcoin wallets. But the scam on Wednesday
was the first to use the real accounts of public figures.

The hackers received \$120,000 worth of Bitcoin in 518 transactions from
around the world, according to Chainalysis, a research company that
tracks the movement of cryptocurrencies. Most of the victims had Bitcoin
wallets associated with Asia, but about a quarter came from the United
States, according to another cryptocurrency research firm, Elliptic.

Soon after the money came into their wallet, the hackers began moving
the money in a complicated pattern of transactions that will help
obscure the source and make it harder to track, Chainalysis found.

``It looks like someone who has some computer skills, but not someone
who is using the most sophisticated ways to launder the coins,'' said
Jonathan Levin, the chief strategy officer at Chainalysis.

Twitter quickly removed many of the messages, but in some cases similar
tweets were sent again from the same accounts. The company eventually
disabled broad swaths of its service for hours.

``Tough day for us at Twitter,'' Mr. Dorsey
\href{https://twitter.com/jack/status/1283571658339397632}{tweeted}
Wednesday night. ``We all feel terrible this happened.''

On Thursday, there were lingering questions about what the attackers did
with their access. Area 1 Security, a cybersecurity company, documented
an increase in spear-phishing emails sent out from accounts
impersonating the same people targeted on Twitter, such as the
billionaire Bill Gates. The emails asked for people to send money to the
same Bitcoin wallet cited in the Twitter attack.

The breach raises significant questions about how Twitter's internal
systems function, and how taking over one employee's internal access
could give an outside attacker carte blanche control over some of the
world's highest-profile and most popular accounts.

In a
\href{https://medium.com/@lucky225/the-twitter-hack-what-exactly-happened-d8740d33c1c}{blog
post} on Thursday, a security expert who saw the hack take over an
account that the expert administers detailed how someone with access to
administrative tools could effectively force their way into most Twitter
accounts using a password reset function. The method was used in the
account takeovers on Wednesday, according to two people familiar with
the attack.

Security researchers also questioned why Twitter did not have better
safeguards to monitor suspicious activity on employee accounts. Many
companies have systems that alert them if an employee is getting into
sensitive data, or changing passwords and emails on accounts multiple
times within a short period, said Rachel Tobac, a hacker and the chief
executive of SocialProof Security, who works with companies to train and
test on social engineering to keep companies safe.

The company is still rushing to figure out the extent of the damage, and
whether or not there is more to come. Twitter representatives said the
company would update the public as it discovered more about the attack.
But experts believe that depending on the length of time the hackers had
administrative access, more fallout could be in store.

``What you saw on Wednesday was probably not the end of the incident,''
said Alon Gal, chief technology officer of Hudson Rock, a cybersecurity
intelligence firm that has been investigating the hack. ``If they got
access to direct messages, this isn't over.''

Nathaniel Popper contributed reporting.

Advertisement

\protect\hyperlink{after-bottom}{Continue reading the main story}

\hypertarget{site-index}{%
\subsection{Site Index}\label{site-index}}

\hypertarget{site-information-navigation}{%
\subsection{Site Information
Navigation}\label{site-information-navigation}}

\begin{itemize}
\tightlist
\item
  \href{https://help.nytimes3xbfgragh.onion/hc/en-us/articles/115014792127-Copyright-notice}{©~2020~The
  New York Times Company}
\end{itemize}

\begin{itemize}
\tightlist
\item
  \href{https://www.nytco.com/}{NYTCo}
\item
  \href{https://help.nytimes3xbfgragh.onion/hc/en-us/articles/115015385887-Contact-Us}{Contact
  Us}
\item
  \href{https://www.nytco.com/careers/}{Work with us}
\item
  \href{https://nytmediakit.com/}{Advertise}
\item
  \href{http://www.tbrandstudio.com/}{T Brand Studio}
\item
  \href{https://www.nytimes3xbfgragh.onion/privacy/cookie-policy\#how-do-i-manage-trackers}{Your
  Ad Choices}
\item
  \href{https://www.nytimes3xbfgragh.onion/privacy}{Privacy}
\item
  \href{https://help.nytimes3xbfgragh.onion/hc/en-us/articles/115014893428-Terms-of-service}{Terms
  of Service}
\item
  \href{https://help.nytimes3xbfgragh.onion/hc/en-us/articles/115014893968-Terms-of-sale}{Terms
  of Sale}
\item
  \href{https://spiderbites.nytimes3xbfgragh.onion}{Site Map}
\item
  \href{https://help.nytimes3xbfgragh.onion/hc/en-us}{Help}
\item
  \href{https://www.nytimes3xbfgragh.onion/subscription?campaignId=37WXW}{Subscriptions}
\end{itemize}
