Sections

SEARCH

\protect\hyperlink{site-content}{Skip to
content}\protect\hyperlink{site-index}{Skip to site index}

\href{https://myaccount.nytimes3xbfgragh.onion/auth/login?response_type=cookie\&client_id=vi}{}

\href{https://www.nytimes3xbfgragh.onion/section/todayspaper}{Today's
Paper}

\href{/section/upshot}{The Upshot}\textbar{}In the Same Towns, Private
Schools Are Reopening While Public Schools Are Not

\url{https://nyti.ms/3h1BJFN}

\begin{itemize}
\item
\item
\item
\item
\item
\item
\end{itemize}

\href{https://www.nytimes3xbfgragh.onion/news-event/coronavirus?action=click\&pgtype=Article\&state=default\&region=TOP_BANNER\&context=storylines_menu}{The
Coronavirus Outbreak}

\begin{itemize}
\tightlist
\item
  live\href{https://www.nytimes3xbfgragh.onion/2020/08/04/world/coronavirus-cases.html?action=click\&pgtype=Article\&state=default\&region=TOP_BANNER\&context=storylines_menu}{Latest
  Updates}
\item
  \href{https://www.nytimes3xbfgragh.onion/interactive/2020/us/coronavirus-us-cases.html?action=click\&pgtype=Article\&state=default\&region=TOP_BANNER\&context=storylines_menu}{Maps
  and Cases}
\item
  \href{https://www.nytimes3xbfgragh.onion/interactive/2020/science/coronavirus-vaccine-tracker.html?action=click\&pgtype=Article\&state=default\&region=TOP_BANNER\&context=storylines_menu}{Vaccine
  Tracker}
\item
  \href{https://www.nytimes3xbfgragh.onion/2020/08/02/us/covid-college-reopening.html?action=click\&pgtype=Article\&state=default\&region=TOP_BANNER\&context=storylines_menu}{College
  Reopening}
\item
  \href{https://www.nytimes3xbfgragh.onion/live/2020/08/04/business/stock-market-today-coronavirus?action=click\&pgtype=Article\&state=default\&region=TOP_BANNER\&context=storylines_menu}{Economy}
\end{itemize}

Advertisement

\protect\hyperlink{after-top}{Continue reading the main story}

Upshot

Supported by

\protect\hyperlink{after-sponsor}{Continue reading the main story}

\hypertarget{in-the-same-towns-private-schools-are-reopening-while-public-schools-are-not}{%
\section{In the Same Towns, Private Schools Are Reopening While Public
Schools Are
Not}\label{in-the-same-towns-private-schools-are-reopening-while-public-schools-are-not}}

Private schools have always had more flexibility, and usually more
money, but never has that disparity made a bigger difference than now.

\includegraphics{https://static01.graylady3jvrrxbe.onion/images/2020/07/15/upshot/00up-virus-school2/merlin_17395133_1149fb0e-99ea-43f7-a4a6-959b77ea1e69-articleLarge.jpg?quality=75\&auto=webp\&disable=upscale}

\href{https://www.nytimes3xbfgragh.onion/by/claire-cain-miller}{\includegraphics{https://static01.graylady3jvrrxbe.onion/images/2018/06/13/multimedia/author-claire-cain-miller/author-claire-cain-miller-thumbLarge.jpg}}

By
\href{https://www.nytimes3xbfgragh.onion/by/claire-cain-miller}{Claire
Cain Miller}

\begin{itemize}
\item
  Published July 16, 2020Updated Aug. 3, 2020
\item
  \begin{itemize}
  \item
  \item
  \item
  \item
  \item
  \item
  \end{itemize}
\end{itemize}

In Honolulu, nearly all public schools are planning to allow students to
return for just part of the week. But at Punahou, a private school for
grades kindergarten through 12, school will open full time for everyone.

The school has an epidemiologist on staff and is installing thermal
scanners in the hallways to take people's temperatures as they walk by.
It has a new commons area and design lab as well as an 80-acre campus
that students can use to spread out. There were already two teachers for
25 children, so it will be easy to cut classes in half to meet public
health requirements for small, consistent groups.

The same thing is happening in communities across the country:
\href{https://www.nytimes3xbfgragh.onion/2020/08/03/us/school-closing-coronavirus.html}{Public
schools plan to open} not at all or just a
\href{https://www.nytimes3xbfgragh.onion/2020/07/08/nyregion/nyc-schools-reopening-plan.html}{few
days a week}, while many neighboring private schools are opening full
time.

Private schools may reverse course if there are outbreaks in their
communities, and governors could still shut down all schools if they
determine that local infection rates call for it. Some families and
teachers won't feel comfortable returning. But the ways in which private
\href{https://www.nytimes3xbfgragh.onion/interactive/2020/07/31/us/coronavirus-school-reopening-risk.html}{schools
are reopening} show it can be done with creative ideas --- and the money
to carry them out.

Public schools, which serve roughly 90 percent of American children,
tend to have less money, larger class sizes and less flexibility to make
changes to things like the curriculum, facilities or work force.

``The virus is this huge stress test on our education system,'' said
Robert Pianta, dean of the school of education at the University of
Virginia. ``It has exposed a great deal of inequity, and we are going to
see this only exacerbated in the coming months, not years. Certain kids
in certain systems, depending on the resources, are going to get much
closer to what looks like a typical high-quality education than
others.''

Gretchen Hoff Varner, a lawyer in Alameda, Calif., and the mother of two
elementary schoolers who will go to school part time this fall, said
public school educators had done a heroic job with what they have. But
the fact that they cannot fully open while independent schools can, she
said, represents ``a failure of political will and resources.''

``If we were a country interested in saving schools the same way we've
saved airlines and banks, then this is a problem we could solve,'' said
Ms. Hoff Varner, who was the P.T.A. president at her children's school
last year.

Some public districts have developed plans to open full time for most
students. They include smaller, wealthier suburban districts as well as
urban ones like those in Durham, N.C., and Charlottesville, Va.

\hypertarget{latest-updates-global-coronavirus-outbreak}{%
\section{\texorpdfstring{\href{https://www.nytimes3xbfgragh.onion/2020/08/04/world/coronavirus-cases.html?action=click\&pgtype=Article\&state=default\&region=MAIN_CONTENT_1\&context=storylines_live_updates}{Latest
Updates: Global Coronavirus
Outbreak}}{Latest Updates: Global Coronavirus Outbreak}}\label{latest-updates-global-coronavirus-outbreak}}

Updated 2020-08-04T18:55:17.683Z

\begin{itemize}
\tightlist
\item
  \href{https://www.nytimes3xbfgragh.onion/2020/08/04/world/coronavirus-cases.html?action=click\&pgtype=Article\&state=default\&region=MAIN_CONTENT_1\&context=storylines_live_updates\#link-4825b93}{Public
  and private schools in Maryland and elsewhere are divided over
  in-person instruction.}
\item
  \href{https://www.nytimes3xbfgragh.onion/2020/08/04/world/coronavirus-cases.html?action=click\&pgtype=Article\&state=default\&region=MAIN_CONTENT_1\&context=storylines_live_updates\#link-4d1eafa8}{N.Y.C.'s
  health commissioner resigns after clashing with the mayor over the
  virus.}
\item
  \href{https://www.nytimes3xbfgragh.onion/2020/08/04/world/coronavirus-cases.html?action=click\&pgtype=Article\&state=default\&region=MAIN_CONTENT_1\&context=storylines_live_updates\#link-6b644638}{`Long
  days, long nights': Washington prepares for a prolonged fight over
  virus relief.}
\end{itemize}

\href{https://www.nytimes3xbfgragh.onion/2020/08/04/world/coronavirus-cases.html?action=click\&pgtype=Article\&state=default\&region=MAIN_CONTENT_1\&context=storylines_live_updates}{See
more updates}

More live coverage:
\href{https://www.nytimes3xbfgragh.onion/live/2020/08/04/business/stock-market-today-coronavirus?action=click\&pgtype=Article\&state=default\&region=MAIN_CONTENT_1\&context=storylines_live_updates}{Markets}

But over all, fall reopening plans are just another way the pandemic has
widened gaps in education. Private schools were able to offer
\href{https://www.nytimes3xbfgragh.onion/2020/05/09/us/coronavirus-public-private-school.html}{much
more robust online learning} last spring, and research suggests that
\href{https://www.nytimes3xbfgragh.onion/2020/07/29/health/covid-school-reopening.html}{school
closures} have
\href{https://www.nytimes3xbfgragh.onion/2020/06/05/us/coronavirus-education-lost-learning.html}{widened
achievement gaps}. Now, as private schools move forward with reopening
plans, it's the children who most need to attend in-person school ---
those lacking the necessary technology for online learning, or with
parents unequipped to oversee it --- who will tend to be the least
likely to do so.

The biggest challenge for schools is how to maintain physical distance,
as required by guidelines from state governments and the Centers for
Disease Control and Prevention. Most countries where schools have opened
after reducing infection levels and imposing distancing measures
\href{https://www.nytimes3xbfgragh.onion/2020/07/11/health/coronavirus-schools-reopen.html}{have
not had outbreaks}.

It generally means capping classes at around a dozen. Public school
buildings in the United States are often old, with small classrooms,
cramped hallways and outdated ventilation systems. Independent schools
(private schools not run by a for-profit company or religious
organization) are more likely to have smaller class sizes to begin with,
and money to hire additional teachers.

Public schools faced a funding crisis even before the pandemic. K-12
schools received \$13.5 billion from the federal coronavirus relief
package in March (though Education Secretary Betsy DeVos has required
that it be
\href{https://www.nytimes3xbfgragh.onion/2020/05/15/us/politics/betsy-devos-coronavirus-religious-schools.html}{shared
with private schools}). School officials and education policy
researchers say that the money was not nearly enough, and that because
states are facing budget shortfalls because of lockdowns, schools would
\href{https://www.nytimes3xbfgragh.onion/2020/07/09/us/schools-reopening-trump.html}{need
a huge federal infusion of cash} to reopen for all students. An average
district with 3,700 students and eight buildings would need to spend an
additional \$1.8 million on health and safety measures, a report
\href{https://www.asumag.com/covid-19/article/21133640/reopening-means-an-additional-18-million-in-costs-for-averagesized-school-district-administrators-estimate}{estimated}.

``There's a giant hole in state and local tax revenue due to not
bringing in money over the past months,'' said Sarah Cohodes, an
associate professor of economics and education at Teachers College at
Columbia. ``Schools need more money. The money needs to come tomorrow.
There's no way anything can happen without the money.''

Punahou has spent \$3 million on health, technology and hiring in
response to the pandemic, and has increased its financial aid by 50
percent, to \$12 million, so families who are suffering financially
because of the pandemic can still enroll.

``We've been very fortunate in this respect that our donors and our
alumni have been able to give us additional money to make that
possible,'' said Mike Latham, Punahou's president.

Size makes a difference, too. When Justin Guerra, a sixth-grade English
teacher at the Athenian School in Danville, Calif., taught public
school, he had three times as many students as he has now. He is
comfortable returning to the classroom when Athenian opens, given the
new safety measures in place, but said he was sympathetic to public
schoolteachers'
\href{https://www.nytimes3xbfgragh.onion/2020/07/11/us/virus-teachers-classrooms.html}{fears
about returning}.

``It's such painful mixed emotions,'' he said. ``Remote learning is
freaking hard --- the amount of work that goes into it for the return
you get is painful. You want to be with the kids; it's the reason to do
this job. At the same time, safety is our priority, and I don't know the
guilt we could carry if someone in our community got sick or died.''

Another key difference for private schools, said Mike Walker, the head
of school at San Francisco Day, is flexibility. Independent schools
don't have all the same regulations for the curriculum or facilities
that public schools have, and teachers generally aren't unionized. They
also have smaller student bodies, with less diverse needs.

The school decided to focus on bringing younger students back full time
--- because distance learning doesn't work as well for them and it's a
crucial period for learning social skills --- and junior high students
part time. Though some teachers are anxious about returning, Mr. Walker
said, they are trying to meet teacher's individual needs.

\href{https://www.nytimes3xbfgragh.onion/news-event/coronavirus?action=click\&pgtype=Article\&state=default\&region=MAIN_CONTENT_3\&context=storylines_faq}{}

\hypertarget{the-coronavirus-outbreak-}{%
\subsubsection{The Coronavirus Outbreak
›}\label{the-coronavirus-outbreak-}}

\hypertarget{frequently-asked-questions}{%
\paragraph{Frequently Asked
Questions}\label{frequently-asked-questions}}

Updated August 4, 2020

\begin{itemize}
\item ~
  \hypertarget{i-have-antibodies-am-i-now-immune}{%
  \paragraph{I have antibodies. Am I now
  immune?}\label{i-have-antibodies-am-i-now-immune}}

  \begin{itemize}
  \tightlist
  \item
    As of right
    now,\href{https://www.nytimes3xbfgragh.onion/2020/07/22/health/covid-antibodies-herd-immunity.html?action=click\&pgtype=Article\&state=default\&region=MAIN_CONTENT_3\&context=storylines_faq}{that
    seems likely, for at least several months.} There have been
    frightening accounts of people suffering what seems to be a second
    bout of Covid-19. But experts say these patients may have a
    drawn-out course of infection, with the virus taking a slow toll
    weeks to months after initial exposure. People infected with the
    coronavirus typically
    \href{https://www.nature.com/articles/s41586-020-2456-9}{produce}
    immune molecules called antibodies, which are
    \href{https://www.nytimes3xbfgragh.onion/2020/05/07/health/coronavirus-antibody-prevalence.html?action=click\&pgtype=Article\&state=default\&region=MAIN_CONTENT_3\&context=storylines_faq}{protective
    proteins made in response to an
    infection}\href{https://www.nytimes3xbfgragh.onion/2020/05/07/health/coronavirus-antibody-prevalence.html?action=click\&pgtype=Article\&state=default\&region=MAIN_CONTENT_3\&context=storylines_faq}{.
    These antibodies may} last in the body
    \href{https://www.nature.com/articles/s41591-020-0965-6}{only two to
    three months}, which may seem worrisome, but that's perfectly normal
    after an acute infection subsides, said Dr. Michael Mina, an
    immunologist at Harvard University. It may be possible to get the
    coronavirus again, but it's highly unlikely that it would be
    possible in a short window of time from initial infection or make
    people sicker the second time.
  \end{itemize}
\item ~
  \hypertarget{im-a-small-business-owner-can-i-get-relief}{%
  \paragraph{I'm a small-business owner. Can I get
  relief?}\label{im-a-small-business-owner-can-i-get-relief}}

  \begin{itemize}
  \tightlist
  \item
    The
    \href{https://www.nytimes3xbfgragh.onion/article/small-business-loans-stimulus-grants-freelancers-coronavirus.html?action=click\&pgtype=Article\&state=default\&region=MAIN_CONTENT_3\&context=storylines_faq}{stimulus
    bills enacted in March} offer help for the millions of American
    small businesses. Those eligible for aid are businesses and
    nonprofit organizations with fewer than 500 workers, including sole
    proprietorships, independent contractors and freelancers. Some
    larger companies in some industries are also eligible. The help
    being offered, which is being managed by the Small Business
    Administration, includes the Paycheck Protection Program and the
    Economic Injury Disaster Loan program. But lots of folks have
    \href{https://www.nytimes3xbfgragh.onion/interactive/2020/05/07/business/small-business-loans-coronavirus.html?action=click\&pgtype=Article\&state=default\&region=MAIN_CONTENT_3\&context=storylines_faq}{not
    yet seen payouts.} Even those who have received help are confused:
    The rules are draconian, and some are stuck sitting on
    \href{https://www.nytimes3xbfgragh.onion/2020/05/02/business/economy/loans-coronavirus-small-business.html?action=click\&pgtype=Article\&state=default\&region=MAIN_CONTENT_3\&context=storylines_faq}{money
    they don't know how to use.} Many small-business owners are getting
    less than they expected or
    \href{https://www.nytimes3xbfgragh.onion/2020/06/10/business/Small-business-loans-ppp.html?action=click\&pgtype=Article\&state=default\&region=MAIN_CONTENT_3\&context=storylines_faq}{not
    hearing anything at all.}
  \end{itemize}
\item ~
  \hypertarget{what-are-my-rights-if-i-am-worried-about-going-back-to-work}{%
  \paragraph{What are my rights if I am worried about going back to
  work?}\label{what-are-my-rights-if-i-am-worried-about-going-back-to-work}}

  \begin{itemize}
  \tightlist
  \item
    Employers have to provide
    \href{https://www.osha.gov/SLTC/covid-19/standards.html}{a safe
    workplace} with policies that protect everyone equally.
    \href{https://www.nytimes3xbfgragh.onion/article/coronavirus-money-unemployment.html?action=click\&pgtype=Article\&state=default\&region=MAIN_CONTENT_3\&context=storylines_faq}{And
    if one of your co-workers tests positive for the coronavirus, the
    C.D.C.} has said that
    \href{https://www.cdc.gov/coronavirus/2019-ncov/community/guidance-business-response.html}{employers
    should tell their employees} -\/- without giving you the sick
    employee's name -\/- that they may have been exposed to the virus.
  \end{itemize}
\item ~
  \hypertarget{should-i-refinance-my-mortgage}{%
  \paragraph{Should I refinance my
  mortgage?}\label{should-i-refinance-my-mortgage}}

  \begin{itemize}
  \tightlist
  \item
    \href{https://www.nytimes3xbfgragh.onion/article/coronavirus-money-unemployment.html?action=click\&pgtype=Article\&state=default\&region=MAIN_CONTENT_3\&context=storylines_faq}{It
    could be a good idea,} because mortgage rates have
    \href{https://www.nytimes3xbfgragh.onion/2020/07/16/business/mortgage-rates-below-3-percent.html?action=click\&pgtype=Article\&state=default\&region=MAIN_CONTENT_3\&context=storylines_faq}{never
    been lower.} Refinancing requests have pushed mortgage applications
    to some of the highest levels since 2008, so be prepared to get in
    line. But defaults are also up, so if you're thinking about buying a
    home, be aware that some lenders have tightened their standards.
  \end{itemize}
\item ~
  \hypertarget{what-is-school-going-to-look-like-in-september}{%
  \paragraph{What is school going to look like in
  September?}\label{what-is-school-going-to-look-like-in-september}}

  \begin{itemize}
  \tightlist
  \item
    It is unlikely that many schools will return to a normal schedule
    this fall, requiring the grind of
    \href{https://www.nytimes3xbfgragh.onion/2020/06/05/us/coronavirus-education-lost-learning.html?action=click\&pgtype=Article\&state=default\&region=MAIN_CONTENT_3\&context=storylines_faq}{online
    learning},
    \href{https://www.nytimes3xbfgragh.onion/2020/05/29/us/coronavirus-child-care-centers.html?action=click\&pgtype=Article\&state=default\&region=MAIN_CONTENT_3\&context=storylines_faq}{makeshift
    child care} and
    \href{https://www.nytimes3xbfgragh.onion/2020/06/03/business/economy/coronavirus-working-women.html?action=click\&pgtype=Article\&state=default\&region=MAIN_CONTENT_3\&context=storylines_faq}{stunted
    workdays} to continue. California's two largest public school
    districts --- Los Angeles and San Diego --- said on July 13, that
    \href{https://www.nytimes3xbfgragh.onion/2020/07/13/us/lausd-san-diego-school-reopening.html?action=click\&pgtype=Article\&state=default\&region=MAIN_CONTENT_3\&context=storylines_faq}{instruction
    will be remote-only in the fall}, citing concerns that surging
    coronavirus infections in their areas pose too dire a risk for
    students and teachers. Together, the two districts enroll some
    825,000 students. They are the largest in the country so far to
    abandon plans for even a partial physical return to classrooms when
    they reopen in August. For other districts, the solution won't be an
    all-or-nothing approach.
    \href{https://bioethics.jhu.edu/research-and-outreach/projects/eschool-initiative/school-policy-tracker/}{Many
    systems}, including the nation's largest, New York City, are
    devising
    \href{https://www.nytimes3xbfgragh.onion/2020/06/26/us/coronavirus-schools-reopen-fall.html?action=click\&pgtype=Article\&state=default\&region=MAIN_CONTENT_3\&context=storylines_faq}{hybrid
    plans} that involve spending some days in classrooms and other days
    online. There's no national policy on this yet, so check with your
    municipal school system regularly to see what is happening in your
    community.
  \end{itemize}
\end{itemize}

``We're nonunionized and really want to stay that way,'' he said.
``There's a different ethos, a different culture. I chose to work in a
smaller system because I think we can make decisions more quickly.''

When the San Francisco Unified School District said this month that it
would probably be unable to open schools in the fall, ``it raised really
considerable issues of equity,'' he said. ``It breaks our heart.''

Public schools typically don't have the resources that private schools
do, and certain things, like the size of public districts, will always
be different. But other changes are within policymakers' or
administrators' control, researchers said.

Schools
\href{https://www.theatlantic.com/ideas/archive/2020/07/better-fall-possible/613882/}{could
consider} using libraries, community centers, empty office buildings,
churches or parks, Professor Cohodes said. Teachers may need to teach a
different grade or subject for a year. Unemployed young people could
tutor or monitor recess or small group activities.

And, she said, communities could make a public health push, conducting
lockdowns, testing and contact tracing to bring the virus under control
before schools opened.

Andrew Saultz, director of the education and leadership program at
Pacific University, said, ``I think the way to do it politically is to
make it about equity: This is good for low-socioeconomic students and
students of color.''

\includegraphics{https://static01.graylady3jvrrxbe.onion/images/2020/07/15/upshot/00up-virus-schools3/merlin_171010044_8fe5a313-4873-42c6-b61e-6dc9dc39d740-articleLarge.jpg?quality=75\&auto=webp\&disable=upscale}

The Durham public school system in North Carolina has made plans to open
full time for pre-K through eighth grade. High schoolers would be remote
unless they had special learning needs. It's a large, urban district,
with 33,000 students in 53 schools, 60 percent of whom receive free or
reduced-price lunch.

The district, which began planning for fall in April, would use the
empty high schools for extra space, and reassign some teachers. Families
could choose to attend online, and teachers could teach remotely if they
had medical issues. The district used money from the federal coronavirus
relief act for laptops and a new remote curriculum, and has largely
relied on local philanthropy from high-tech businesses and universities
in the area.

``We recognize that working families depend on the benefits of in-person
instruction, social-emotional support and food,'' said William Sudderth
III, known as Chip, the chief communications officer for the district.
``This is our effort to provide the best possible education for our
students in a situation that is far from the best possible.''

The reopening is not assured, though: This week,
\href{https://daenc.com/2020/07/13/statement-on-school-re-open-plans/}{a
Durham teachers group requested} that schools stay closed while
coronavirus is still spreading.

Advertisement

\protect\hyperlink{after-bottom}{Continue reading the main story}

\hypertarget{site-index}{%
\subsection{Site Index}\label{site-index}}

\hypertarget{site-information-navigation}{%
\subsection{Site Information
Navigation}\label{site-information-navigation}}

\begin{itemize}
\tightlist
\item
  \href{https://help.nytimes3xbfgragh.onion/hc/en-us/articles/115014792127-Copyright-notice}{©~2020~The
  New York Times Company}
\end{itemize}

\begin{itemize}
\tightlist
\item
  \href{https://www.nytco.com/}{NYTCo}
\item
  \href{https://help.nytimes3xbfgragh.onion/hc/en-us/articles/115015385887-Contact-Us}{Contact
  Us}
\item
  \href{https://www.nytco.com/careers/}{Work with us}
\item
  \href{https://nytmediakit.com/}{Advertise}
\item
  \href{http://www.tbrandstudio.com/}{T Brand Studio}
\item
  \href{https://www.nytimes3xbfgragh.onion/privacy/cookie-policy\#how-do-i-manage-trackers}{Your
  Ad Choices}
\item
  \href{https://www.nytimes3xbfgragh.onion/privacy}{Privacy}
\item
  \href{https://help.nytimes3xbfgragh.onion/hc/en-us/articles/115014893428-Terms-of-service}{Terms
  of Service}
\item
  \href{https://help.nytimes3xbfgragh.onion/hc/en-us/articles/115014893968-Terms-of-sale}{Terms
  of Sale}
\item
  \href{https://spiderbites.nytimes3xbfgragh.onion}{Site Map}
\item
  \href{https://help.nytimes3xbfgragh.onion/hc/en-us}{Help}
\item
  \href{https://www.nytimes3xbfgragh.onion/subscription?campaignId=37WXW}{Subscriptions}
\end{itemize}
