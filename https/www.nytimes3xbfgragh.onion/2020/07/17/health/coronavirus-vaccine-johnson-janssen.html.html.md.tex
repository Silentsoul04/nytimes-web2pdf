Sections

SEARCH

\protect\hyperlink{site-content}{Skip to
content}\protect\hyperlink{site-index}{Skip to site index}

\href{/section/health}{Health}\textbar{}Inside Johnson \& Johnson's
Nonstop Hunt for a Coronavirus Vaccine

\url{https://nyti.ms/3jemueE}

\begin{itemize}
\item
\item
\item
\item
\item
\end{itemize}

\hypertarget{the-coronavirus-outbreak}{%
\subsubsection{\texorpdfstring{\href{https://www.nytimes3xbfgragh.onion/news-event/coronavirus?name=styln-coronavirus-national\&region=TOP_BANNER\&block=storyline_menu_recirc\&action=click\&pgtype=Article\&impression_id=dab10050-efba-11ea-8004-9385f1c1fc5b\&variant=undefined}{The
Coronavirus
Outbreak}}{The Coronavirus Outbreak}}\label{the-coronavirus-outbreak}}

\begin{itemize}
\tightlist
\item
  live\href{https://www.nytimes3xbfgragh.onion/2020/09/05/world/coronavirus-covid.html?name=styln-coronavirus-national\&region=TOP_BANNER\&block=storyline_menu_recirc\&action=click\&pgtype=Article\&impression_id=dab12760-efba-11ea-8004-9385f1c1fc5b\&variant=undefined}{Latest
  Updates}
\item
  \href{https://www.nytimes3xbfgragh.onion/interactive/2020/us/coronavirus-us-cases.html?name=styln-coronavirus-national\&region=TOP_BANNER\&block=storyline_menu_recirc\&action=click\&pgtype=Article\&impression_id=dab12761-efba-11ea-8004-9385f1c1fc5b\&variant=undefined}{Maps
  and Cases}
\item
  \href{https://www.nytimes3xbfgragh.onion/interactive/2020/science/coronavirus-vaccine-tracker.html?name=styln-coronavirus-national\&region=TOP_BANNER\&block=storyline_menu_recirc\&action=click\&pgtype=Article\&impression_id=dab12762-efba-11ea-8004-9385f1c1fc5b\&variant=undefined}{Vaccine
  Tracker}
\item
  \href{https://www.nytimes3xbfgragh.onion/2020/09/02/your-money/eviction-moratorium-covid.html?name=styln-coronavirus-national\&region=TOP_BANNER\&block=storyline_menu_recirc\&action=click\&pgtype=Article\&impression_id=dab12763-efba-11ea-8004-9385f1c1fc5b\&variant=undefined}{Eviction
  Moratorium}
\item
  \href{https://www.nytimes3xbfgragh.onion/interactive/2020/09/02/magazine/food-insecurity-hunger-us.html?name=styln-coronavirus-national\&region=TOP_BANNER\&block=storyline_menu_recirc\&action=click\&pgtype=Article\&impression_id=dab12764-efba-11ea-8004-9385f1c1fc5b\&variant=undefined}{American
  Hunger}
\end{itemize}

\includegraphics{https://static01.graylady3jvrrxbe.onion/images/2020/07/21/science/00VIRUS-JNJ1/00VIRUS-JNJ1-articleLarge.jpg?quality=75\&auto=webp\&disable=upscale}

\hypertarget{inside-johnson--johnsons-nonstop-hunt-for-a-coronavirus-vaccine}{%
\section{Inside Johnson \& Johnson's Nonstop Hunt for a Coronavirus
Vaccine}\label{inside-johnson--johnsons-nonstop-hunt-for-a-coronavirus-vaccine}}

In Boston and in the Netherlands, scientists are racing to build a
vaccine against the virus strangling the world.

Noe Mercado, a scientist at the Center for Virology and Vaccine Research
in Boston, which is developing a coronavirus vaccine with Johnson \&
Johnson.Credit...Tony Luong for The New York Times

Supported by

\protect\hyperlink{after-sponsor}{Continue reading the main story}

\href{https://www.nytimes3xbfgragh.onion/by/carl-zimmer}{\includegraphics{https://static01.graylady3jvrrxbe.onion/images/2018/06/12/multimedia/author-carl-zimmer/author-carl-zimmer-thumbLarge.png}}

By \href{https://www.nytimes3xbfgragh.onion/by/carl-zimmer}{Carl Zimmer}

\begin{itemize}
\item
  July 17, 2020
\item
  \begin{itemize}
  \item
  \item
  \item
  \item
  \item
  \end{itemize}
\end{itemize}

Each workday morning in March, Noe Mercado drove through the desolate
streets of Boston to a tall glass building on Blackfan Circle, in the
heart of the city's biotech hub. Most residents had gone into hiding
from the coronavirus, but Mr. Mercado had an essential job: searching
for a
\href{https://www.nytimes3xbfgragh.onion/2020/07/20/world/covid-coronavirus-vaccine.html}{vaccine}
against this new, devastating pathogen.

Parking in the underground lot, he put on a mask and rode the empty
elevator to the tenth floor, joining a skeleton crew at the Center for
Virology and Vaccine Research at Beth Israel Deaconess Medical Center.
Day after day, Mr. Mercado sat at his lab bench, searching for signs of
the virus in nasal swabs taken from dozens of monkeys.

The animals had been injected with experimental vaccines Mr. Mercado had
helped create. The monkeys then had been exposed to the coronavirus, and
now Mr. Mercado was finding out whether any vaccine had protected them.
One morning, after he loaded all the data into a software program, a
single telling graph set his heart beating: Some of the vaccines, it
appeared,
\href{https://www.nytimes3xbfgragh.onion/2020/05/20/health/coronavirus-vaccine-harvard.html}{had
worked}.

Mr. Mercado hurried around the lab to share the news. Given the times,
there were no hugs, no high-fives. And he did not bask in glory for
long. Making a vaccine demands patience, attention to detail --- and a
tolerance for bitter failure.

``Yeah, I'm excited, but I'm also thinking about the next step,'' Mr.
Mercado later recalled. ``What if it doesn't pan out?''

The coronavirus has
\href{https://www.nytimes3xbfgragh.onion/interactive/2020/world/coronavirus-maps.html}{now
infected about 13.8 million people worldwide} and killed at least
590,000. Millions more may die. The only hope for a long-term
protection, literally the only shot at a return to normal life, is an
effective vaccine.

In January, researchers at the vaccine center dropped everything they
were doing to find one. The man heading up the effort is Mr. Mercado's
boss, Dr. Dan Barouch, the director of the center and one of the world's
leading vaccine-makers.

Now they are about to take a major step forward. Janssen Pharmaceutica,
a division of Johnson \& Johnson, has been collaborating with the Beth
Israel team to craft a coronavirus vaccine based on a design pioneered
by Dr. Barouch and his colleagues ten years ago.

Next week, clinical trials of the vaccine will begin in Belgium. Dr.
Barouch's team will soon start up a trial in Boston.

The past six months have been a blur of long weeks and late nights, of
strict safety measures and scarce lab supplies. ``Everything has been
orders of magnitude more challenging than in the pre-pandemic era,'' Dr.
Barouch said.

Researchers around the world have been making vaccines of their own,
some with dead viruses, others with protein fragments and strings of
DNA. As of July, there are
\href{https://www.nytimes3xbfgragh.onion/interactive/2020/science/coronavirus-vaccine-tracker.html}{over
135 vaccines in preclinical tests, and another 30 in clinical trials} on
people. Never have so many vaccines moved so quickly into trials for one
disease.

Since January, Dr. Barouch's team in Boston has run experiments in cells
and monkeys, while Janssen's researchers in the Netherlands have raced
to find a recipe for producing the new vaccine in huge quantities.
Already they have started producing a batch for the clinical trials.

If the vaccine proves safe in initial tests, a trial for efficacy will
launch in September. If that experiment is successful, Johnson \&
Johnson will manufacture hundreds of millions of doses for emergency use
in January. Over the course of next year, the company plans to produce
up to a billion doses.

While Johnson \& Johnson is one of the world's biggest companies, with a
market capitalization over \$370 billion, it's a fairly small player in
the vaccine market. On July 1, its Ebola vaccine received approval from
the European Commission. The company's vaccines for other diseases are
still in clinical trials.

Even so, the United States government has given \$456 million to Johnson
\& Johnson, funding from
\href{https://www.hhs.gov/about/news/2020/06/16/fact-sheet-explaining-operation-warp-speed.html}{the
Trump administration's Operation Warp Speed}; the company has invested
another \$500 million in the coronavirus vaccine project.

Dr. Barouch and his colleagues are now finishing up tests of the final
formulation in monkeys. In the next few months, they will begin to see
how people respond to the injection.

It is a monumental task to develop a vaccine so quickly against a
pathogen that no one had heard of before this year. But, Dr. Barouch
said, ``I'm even more optimistic now than I was several months ago.''

\hypertarget{latest-updates-the-coronavirus-outbreak}{%
\section{\texorpdfstring{\href{https://www.nytimes3xbfgragh.onion/2020/09/04/world/covid-19-coronavirus.html?action=click\&pgtype=Article\&state=default\&region=MAIN_CONTENT_1\&context=storylines_live_updates}{Latest
Updates: The Coronavirus
Outbreak}}{Latest Updates: The Coronavirus Outbreak}}\label{latest-updates-the-coronavirus-outbreak}}

Updated 2020-09-05T12:05:40.998Z

\begin{itemize}
\tightlist
\item
  \href{https://www.nytimes3xbfgragh.onion/2020/09/04/world/covid-19-coronavirus.html?action=click\&pgtype=Article\&state=default\&region=MAIN_CONTENT_1\&context=storylines_live_updates\#link-1654f6ad}{Research
  connects vaping to a higher chance of catching the virus --- and
  suffering its worst effects.}
\item
  \href{https://www.nytimes3xbfgragh.onion/2020/09/04/world/covid-19-coronavirus.html?action=click\&pgtype=Article\&state=default\&region=MAIN_CONTENT_1\&context=storylines_live_updates\#link-52e4198a}{Another
  college football game won't be played as planned.}
\item
  \href{https://www.nytimes3xbfgragh.onion/2020/09/04/world/covid-19-coronavirus.html?action=click\&pgtype=Article\&state=default\&region=MAIN_CONTENT_1\&context=storylines_live_updates\#link-181cef0}{Pharmaceutical
  companies plan a joint pledge on safety standards as they move
  vaccines to the marketplace.}
\end{itemize}

\href{https://www.nytimes3xbfgragh.onion/2020/09/04/world/covid-19-coronavirus.html?action=click\&pgtype=Article\&state=default\&region=MAIN_CONTENT_1\&context=storylines_live_updates}{See
more updates}

More live coverage:
\href{https://www.nytimes3xbfgragh.onion/live/2020/09/04/business/stock-market-today-coronavirus?action=click\&pgtype=Article\&state=default\&region=MAIN_CONTENT_1\&context=storylines_live_updates}{Markets}

\hypertarget{forty-one-cases}{%
\subsection{Forty-One Cases}\label{forty-one-cases}}

\includegraphics{https://static01.graylady3jvrrxbe.onion/images/2020/07/21/science/00VIRUS-JNJ2/00VIRUS-JNJ2-articleLarge.jpg?quality=75\&auto=webp\&disable=upscale}

Late afternoon on Jan. 10, the temperature in Boston was in the low 50s,
almost 20 degrees above normal. Dr. Barouch had spent the day hosting
the lab's annual retreat on the top floor of Boston's Museum of Science.

Out the tall windows, the scientists could see cars streaming across the
Charles River. During breaks between presentations, they crowded
together for group photos, with big, unworried smiles.

At the end of the meeting, they discussed news of a mysterious cluster
of 41 pneumonia cases in Wuhan, China. ``Forty-one cases seemed like a
lot at that point,'' Dr. Barouch said.

The new cases reminded them of SARS, a disease caused by a coronavirus,
which had appeared in China in 2002 and had spread to 29 countries,
striking 8,096 people and killing 773, before it was halted. Chinese
scientists had just reported that another coronavirus was on the loose.

``We thought, maybe we should make a vaccine for that,'' recalled Jinyan
Liu, a staff scientist at the center. But without more information about
the new virus, there was nothing they could do.

Everything changed that night. At 9:41 p.m., Dr. Kathryn Stephenson, the
director of the center's clinical trial unit, sent Dr. Barouch a short
email from her iPhone: ``This was released today --- saw someone link to
it on Twitter.''

The link led to an open-access virology site where scientists based in
China had posted a file
\href{https://virological.org/t/novel-2019-coronavirus-genome/319}{containing
the entire genetic sequence of the new coronavirus}. ``Please feel free
to download, share, use, and analyze this data,'' wrote Yong-Zhen Zhang,
a professor at Fudan University in Shanghai and the leader of the
consortium.

Five minutes later, Dr. Barouch emailed Dr. Liu, Mr. Mercado and
Zhenfeng Li, a research assistant at the center: ``Can one of you
extract the new coronavirus sequence from this file?''

Soon the four scientists were poring over the sequence, a series of
30,000 genetic letters that no one had seen arranged in exactly this
order before. ``We worked Friday, Saturday and Sunday, day and night,''
Dr. Liu said.

By the end of the weekend, they had a good idea of what they were up
against, and how to defeat it potentially. On Monday, the scientists
returned to the lab, ready to start on the most ambitious endeavor any
of them had ever undertaken.

But the researchers would not have to create a vaccine from scratch.
They would be working from a playbook that Dr. Barouch had been writing
for 20 years.

By 2004, when Dr. Barouch opened his first lab at Harvard Medical
School, he had gained a reputation as an ambitious young researcher. He
immediately set a suitably daunting goal: a vaccine against H.I.V., the
virus that causes AIDS.

The virus had been found in 1983, but two decades of vaccine work had
led to one disappointment after another. The standard ways to train the
immune system to recognize a virus failed when it came to H.I.V.

Dr. Barouch decided to try something different: a vaccine made from
another virus. They chose adenovirus serotype 26 --- Ad26, for short ---
a relatively rare virus that causes mild colds but is very effective at
invading human cells.

To create the vaccine, they collaborated with Crucell, a Dutch company
that was bought by Johnson \& Johnson in 2011. The researchers disabled
the Ad26 virus so that it could only invade cells but not multiply in
them.

Image

Researchers are working from a vaccine playbook that Dr. Barouch has
been writing for 20 years.Credit...Tony Luong for The New York Times

Image

The vaccine would prime the immune system to attack the so-called spike
proteins that cover the surface of the new coronavirus.Credit...Tony
Luong for The New York Times

Then they added a gene from H.I.V. Cells infected with Ad26 would make
H.I.V. proteins that drifted in the bloodstream, priming the immune
system.

In experiments on monkeys, the vaccine offered protection against H.I.V.
In trials on people, the vaccine was safe and triggered a strong immune
response against the virus. But the trials to see if it effectively
protects against the virus are still underway.

In 2016, amid the Zika epidemic, Dr. Barouch and his colleagues quickly
retooled their Ad26 vaccine to make Zika viral proteins. They got as far
as trials that showed the vaccine was safe in people and generated a
long-lasting immune response, but shelved the project when the Zika
epidemic retreated.

As the new coronavirus began to spread in January, the lab already knew
how to make a vaccine for a sudden outbreak. What they needed now was a
way to target the new virus.

Previous research on SARS and other coronaviruses made the choice clear.
They would prime the immune system to attack the so-called spike
proteins that cover the surface of the new coronavirus.

\hypertarget{a-war-we-could-win}{%
\subsection{`A War We Could Win'}\label{a-war-we-could-win}}

As January wore on, Dr. Barouch realized that Covid-19 was going to be
far graver threat than SARS.

``We would not be able to stop this virus by traditional public health
measures,'' he said. ``It was absolutely clear that we needed a
vaccine.''

He emailed to Johan Van Hoof, the head of vaccines at Janssen. ``I am
writing today because the coronavirus outbreak in China is looking
bad,'' Dr. Barouch wrote. ``Are you interested in making a rapid Ad
based vaccine like we did for Zika in 2016-2017?''

Two minutes later, Dr. Van Hoof replied: ``Would a call work now?'' And
four days after the call, they signed an agreement to collaborate.

The Center for Virology and Vaccine Research has a staff of dozens of
researchers, ranging from medical doctors and senior scientists to
postdoctoral researchers, grad students and assistants just out of
college. Dr. Barouch's team turned away from projects on H.I.V. and
other diseases, and divided up the work to make a coronavirus vaccine.

Mr. Mercado and his colleagues fashioned copies of the coronavirus gene
that directs production of its spike protein. They came up with ten
variations to see which would produce the best immune response.

Meanwhile, Katherine McMahan, a research assistant at the center, worked
on the team building a test for spike antibodies in the animals that
would receive the vaccine. Creating it took up most of her waking life.
On some days, she didn't get around to eating lunch till nighttime.

In late February, researchers injected the spike genes into mice and
then sent Ms. McMahan blood from the animals. Ms. McMahan's test
confirmed that they were making coronavirus antibodies.

Ms. McMahan was near tears: ``It began to feel like a war that we could
win.''

Outside the lab, though, there was no sense that a war was coming. She
urged family and friends to stock up on food and other supplies, without
much luck.

``Many of us were having a Chicken Little experience,'' she said.
``You're saying, `Look, you've got to take this seriously,' and getting
blown off.''

Image

At the Center for Virology and Vaccine Research, scientists work nights
and weekends.Credit...Tony Luong for The New York Times

Image

Dr. Jinyan Liu, one of the staff scientists who pored over the
coronavirus's 30,000-letter genetic sequence.Credit...Tony Luong for The
New York Times

Soon enough, people desperately ill with Covid-19 flooded into Boston's
hospitals, and the city began to shut down. In labs high above Boston's
empty streets, Dr. Barouch's team shifted from studies on mice to
monkeys.

The nasal swabs that Mr. Mercado examined revealed that some versions of
the vaccines only partially protected the monkey, but others worked much
better. As the investigators reported in the journal Science, they
couldn't detect the virus at all in eight of the 25 monkeys who got
experimental vaccines.

\href{https://www.nytimes3xbfgragh.onion/news-event/coronavirus?action=click\&pgtype=Article\&state=default\&region=MAIN_CONTENT_3\&context=storylines_faq}{}

\hypertarget{the-coronavirus-outbreak-}{%
\subsubsection{The Coronavirus Outbreak
›}\label{the-coronavirus-outbreak-}}

\hypertarget{frequently-asked-questions}{%
\paragraph{Frequently Asked
Questions}\label{frequently-asked-questions}}

Updated September 4, 2020

\begin{itemize}
\item ~
  \hypertarget{what-are-the-symptoms-of-coronavirus}{%
  \paragraph{What are the symptoms of
  coronavirus?}\label{what-are-the-symptoms-of-coronavirus}}

  \begin{itemize}
  \tightlist
  \item
    In the beginning, the coronavirus
    \href{https://www.nytimes3xbfgragh.onion/article/coronavirus-facts-history.html?action=click\&pgtype=Article\&state=default\&region=MAIN_CONTENT_3\&context=storylines_faq\#link-6817bab5}{seemed
    like it was primarily a respiratory illness}~--- many patients had
    fever and chills, were weak and tired, and coughed a lot, though
    some people don't show many symptoms at all. Those who seemed
    sickest had pneumonia or acute respiratory distress syndrome and
    received supplemental oxygen. By now, doctors have identified many
    more symptoms and syndromes. In April,
    \href{https://www.nytimes3xbfgragh.onion/2020/04/27/health/coronavirus-symptoms-cdc.html?action=click\&pgtype=Article\&state=default\&region=MAIN_CONTENT_3\&context=storylines_faq}{the
    C.D.C. added to the list of early signs}~sore throat, fever, chills
    and muscle aches. Gastrointestinal upset, such as diarrhea and
    nausea, has also been observed. Another telltale sign of infection
    may be a sudden, profound diminution of one's
    \href{https://www.nytimes3xbfgragh.onion/2020/03/22/health/coronavirus-symptoms-smell-taste.html?action=click\&pgtype=Article\&state=default\&region=MAIN_CONTENT_3\&context=storylines_faq}{sense
    of smell and taste.}~Teenagers and young adults in some cases have
    developed painful red and purple lesions on their fingers and toes
    --- nicknamed ``Covid toe'' --- but few other serious symptoms.
  \end{itemize}
\item ~
  \hypertarget{why-is-it-safer-to-spend-time-together-outside}{%
  \paragraph{Why is it safer to spend time together
  outside?}\label{why-is-it-safer-to-spend-time-together-outside}}

  \begin{itemize}
  \tightlist
  \item
    \href{https://www.nytimes3xbfgragh.onion/2020/05/15/us/coronavirus-what-to-do-outside.html?action=click\&pgtype=Article\&state=default\&region=MAIN_CONTENT_3\&context=storylines_faq}{Outdoor
    gatherings}~lower risk because wind disperses viral droplets, and
    sunlight can kill some of the virus. Open spaces prevent the virus
    from building up in concentrated amounts and being inhaled, which
    can happen when infected people exhale in a confined space for long
    stretches of time, said Dr. Julian W. Tang, a virologist at the
    University of Leicester.
  \end{itemize}
\item ~
  \hypertarget{why-does-standing-six-feet-away-from-others-help}{%
  \paragraph{Why does standing six feet away from others
  help?}\label{why-does-standing-six-feet-away-from-others-help}}

  \begin{itemize}
  \tightlist
  \item
    The coronavirus spreads primarily through droplets from your mouth
    and nose, especially when you cough or sneeze. The C.D.C., one of
    the organizations using that measure,
    \href{https://www.nytimes3xbfgragh.onion/2020/04/14/health/coronavirus-six-feet.html?action=click\&pgtype=Article\&state=default\&region=MAIN_CONTENT_3\&context=storylines_faq}{bases
    its recommendation of six feet}~on the idea that most large droplets
    that people expel when they cough or sneeze will fall to the ground
    within six feet. But six feet has never been a magic number that
    guarantees complete protection. Sneezes, for instance, can launch
    droplets a lot farther than six feet,
    \href{https://jamanetwork.com/journals/jama/fullarticle/2763852}{according
    to a recent study}. It's a rule of thumb: You should be safest
    standing six feet apart outside, especially when it's windy. But
    keep a mask on at all times, even when you think you're far enough
    apart.
  \end{itemize}
\item ~
  \hypertarget{i-have-antibodies-am-i-now-immune}{%
  \paragraph{I have antibodies. Am I now
  immune?}\label{i-have-antibodies-am-i-now-immune}}

  \begin{itemize}
  \tightlist
  \item
    As of right
    now,\href{https://www.nytimes3xbfgragh.onion/2020/07/22/health/covid-antibodies-herd-immunity.html?action=click\&pgtype=Article\&state=default\&region=MAIN_CONTENT_3\&context=storylines_faq}{~that
    seems likely, for at least several months.}~There have been
    frightening accounts of people suffering what seems to be a second
    bout of Covid-19. But experts say these patients may have a
    drawn-out course of infection, with the virus taking a slow toll
    weeks to months after initial exposure.~People infected with the
    coronavirus typically
    \href{https://www.nature.com/articles/s41586-020-2456-9}{produce}~immune
    molecules called antibodies, which are
    \href{https://www.nytimes3xbfgragh.onion/2020/05/07/health/coronavirus-antibody-prevalence.html?action=click\&pgtype=Article\&state=default\&region=MAIN_CONTENT_3\&context=storylines_faq}{protective
    proteins made in response to an
    infection}\href{https://www.nytimes3xbfgragh.onion/2020/05/07/health/coronavirus-antibody-prevalence.html?action=click\&pgtype=Article\&state=default\&region=MAIN_CONTENT_3\&context=storylines_faq}{.
    These antibodies may}~last in the body
    \href{https://www.nature.com/articles/s41591-020-0965-6}{only two to
    three months}, which may seem worrisome, but that's~perfectly normal
    after an acute infection subsides, said Dr. Michael Mina, an
    immunologist at Harvard University. It may be possible to get the
    coronavirus again, but it's highly unlikely that it would be
    possible in a short window of time from initial infection or make
    people sicker the second time.
  \end{itemize}
\item ~
  \hypertarget{what-are-my-rights-if-i-am-worried-about-going-back-to-work}{%
  \paragraph{What are my rights if I am worried about going back to
  work?}\label{what-are-my-rights-if-i-am-worried-about-going-back-to-work}}

  \begin{itemize}
  \tightlist
  \item
    Employers have to provide
    \href{https://www.osha.gov/SLTC/covid-19/standards.html}{a safe
    workplace}~with policies that protect everyone equally.
    \href{https://www.nytimes3xbfgragh.onion/article/coronavirus-money-unemployment.html?action=click\&pgtype=Article\&state=default\&region=MAIN_CONTENT_3\&context=storylines_faq}{And
    if one of your co-workers tests positive for the coronavirus, the
    C.D.C.}~has said that
    \href{https://www.cdc.gov/coronavirus/2019-ncov/community/guidance-business-response.html}{employers
    should tell their employees}~-\/- without giving you the sick
    employee's name -\/- that they may have been exposed to the virus.
  \end{itemize}
\end{itemize}

The results gave Dr. Barouch hope that one of his team's vaccines --- or
one of those developed by another group --- might work. ``It's the real
deal,'' he said.

More monkeys were injected with the Ad26 virus, now equipped to produce
the spike gene. Dr. Barouch predicts that this vaccine will induce
higher levels of antibodies than the prototypes did.

The experiment will also provide crucial clues about how the immune
system responds to the Ad26 vaccine. Some vaccines confer protection
mostly by triggering the body to make antibodies that attack a virus.
But others can stir virus-hunting immune cells to join the attack.

The results of the latest round of experiments will be published within
a few weeks.

For all the progress made by Dr. Barouch's team, the Ad26 vaccine has
its skeptics. John Moore, a virologist at Weill Cornell Medical College,
said other types of vaccines tested in animals have produced higher
levels of antibodies. These vaccines, made of viral proteins, would be
his choice for a weapon against the coronavirus.

Six companies have already launched human safety trials of their protein
vaccines. ``That's what I'd be doing,'' said Dr. Moore. ``It's freaking
obvious.''

One drawback of viral-protein vaccines is that they take more time to
produce in huge quantities. Other vaccines, like Johnson \& Johnson's
Ad26, will come more quickly, and Dr. Moore acknowledged that they may
work well enough to provide protection.

If so, there may not be a need for a better but slower vaccine. ``If
Plan A works, then you don't need a Plan B,'' Dr. Moore said.

\hypertarget{a-virus-seed}{%
\subsection{A Virus Seed}\label{a-virus-seed}}

While Dr. Barouch and his colleagues were testing the vaccines on
animals in the United States, a team of Johnson \& Johnson researchers
was gearing up to manufacture them in the Netherlands. Scientists there
took advantage of their years of experience with Ad26, which they have
used to make vaccines for H.I.V., Ebola and other viruses.

Making an Ad26 vaccine requires remodeling an adenovirus and then
creating vast quantities of the new version. But Ad26 cannot multiply in
ordinary cells. It must infect specially engineered ones.

Johnson \& Johnson's technicians produce batches of these cells in huge
vats filled with a nutrient-rich broth kept at a constant temperature
and stirred to pull in oxygen.

``It's to make the cells feel happy and comfortable, to make product,''
said Paul Ives, the senior director of drug development at Janssen.

Once a batch of these nurturing cells has grown sufficiently, Dr. Ives
and his colleagues infect them with the modified Ad26 viruses. Each cell
churns out thousands of new viruses, which are removed and purified so
that they can be used as vaccines.

Dr. Ives and his colleagues have been measuring how quickly various
versions of the revamped Ad26 cell can multiply. Some reproduce more
easily than others, the scientists have found.

Even a slightly slower reproduction rate could leave Johnson \& Johnson
with a huge shortfall in vaccine doses. ``It can mean you have 300
million vaccines or 30 million,'' said Paul Stoffels, the chief
scientific officer at Johnson \& Johnson.

Image

Dr. Barouch and his colleagues are getting ready to inject their vaccine
into hundreds of volunteers in Boston in late July. Credit...Tony Luong
for The New York Times

Image

If those trials produce promising results, Johnson \& Johnson will run a
much larger one in the fall to see if the vaccine is
effective.Credit...Tony Luong for The New York Times

Dr. Ives and his colleagues recently chose the best virus for the
vaccine and turned it into their ``master virus seed.'' They created
gallons of frozen virus stock. A batch of this seed will become the
vaccine used in the clinical trials.

And if those trials show that the vaccine is effective, the factory will
use the same master virus seed to manufacture an emergency supply that
would be distributed at the start of 2021. ``We can theoretically
produce 300 million vaccines,'' Dr. Stoffels said.

The company has formed a partnership with an American vaccine maker and
is also setting up two more plants in Asia and Europe, ``so that we can
come to a manufacturing capacity north of a billion vaccines,'' Dr.
Stoffels said.

Florian Krammer, a virologist at the Icahn School of Medicine at Mount
Sinai, wonders if Johnson \& Johnson can live up to that promise, given
that it has never made Ad26 at anywhere close to this scale.

``Making a couple of million doses over several years for clinical
trials is very different than producing hundreds of millions of doses
within months for the market,'' he said.

Johnson \& Johnson has said it will distribute the vaccine on a
not-for-profit basis. Speaking in March to the Belgian newspaper De
Tijd, Dr. Stoffels
\href{https://www.tijd.be/ondernemen/farma-biotech/we-rekenen-op-een-vaccin-van-10-euro-tegen-coronavirus/10217795.html}{calculated
a cost of ten dollars per vaccine}. In a follow-up interview, he said
that the price would not be set until the company finished making an
initial supply.

Amid a pandemic, critics say Johnson \& Johnson should not be allowed to
set the terms. ``If we get a vaccine, it should be free and available to
everybody,'' said the Rev. Dr. William J. Barber II, the president of
the North Carolina N.A.A.C.P. and a critic of Johnson \& Johnson's drug
pricing.

``How do you get these big, massive awards to produce a vaccine without
any rider on the money saying it must be used in a way that it's
affordable to everybody?'' he asked.

For now, no one knows if the vaccine will actually work. Dr. Barouch and
his colleagues are getting ready to inject the Ad26 vaccine into
hundreds of volunteers in Boston in late July. Researchers will not only
observe whether the vaccine is safe but also look at the antibodies it
prompts the volunteers to make. If those trials produce promising
results, Johnson \& Johnson will run a much larger one in the fall to
see if the vaccine is effective.

At the same time, Dr. Barouch and his colleagues are planning a third
round of experiments on monkeys. They want to inject the animals with
antibodies against the coronavirus and then infect them. By giving
different monkeys varying doses, the investigators hope to figure out
what level of antibodies in the human body is required to prevent
Covid-19.

And so, even as Boston is starting to reopen, Dr. Barouch and others at
the vaccine center continue to work nights and weekends.

``I keep a series of Post-it notes at my desk, which I update each day
with the number of lives lost to Covid,'' said Ms. McMahan. ``When I'm
feeling drained, I look at that number.''

Image

Dr. Barouch on a stroll through the lab.Credit...Tony Luong for The New
York Times

Advertisement

\protect\hyperlink{after-bottom}{Continue reading the main story}

\hypertarget{site-index}{%
\subsection{Site Index}\label{site-index}}

\hypertarget{site-information-navigation}{%
\subsection{Site Information
Navigation}\label{site-information-navigation}}

\begin{itemize}
\tightlist
\item
  \href{https://help.nytimes3xbfgragh.onion/hc/en-us/articles/115014792127-Copyright-notice}{©~2020~The
  New York Times Company}
\end{itemize}

\begin{itemize}
\tightlist
\item
  \href{https://www.nytco.com/}{NYTCo}
\item
  \href{https://help.nytimes3xbfgragh.onion/hc/en-us/articles/115015385887-Contact-Us}{Contact
  Us}
\item
  \href{https://www.nytco.com/careers/}{Work with us}
\item
  \href{https://nytmediakit.com/}{Advertise}
\item
  \href{http://www.tbrandstudio.com/}{T Brand Studio}
\item
  \href{https://www.nytimes3xbfgragh.onion/privacy/cookie-policy\#how-do-i-manage-trackers}{Your
  Ad Choices}
\item
  \href{https://www.nytimes3xbfgragh.onion/privacy}{Privacy}
\item
  \href{https://help.nytimes3xbfgragh.onion/hc/en-us/articles/115014893428-Terms-of-service}{Terms
  of Service}
\item
  \href{https://help.nytimes3xbfgragh.onion/hc/en-us/articles/115014893968-Terms-of-sale}{Terms
  of Sale}
\item
  \href{https://spiderbites.nytimes3xbfgragh.onion}{Site Map}
\item
  \href{https://help.nytimes3xbfgragh.onion/hc/en-us}{Help}
\item
  \href{https://www.nytimes3xbfgragh.onion/subscription?campaignId=37WXW}{Subscriptions}
\end{itemize}
