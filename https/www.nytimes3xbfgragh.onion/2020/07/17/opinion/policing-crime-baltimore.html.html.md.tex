Sections

SEARCH

\protect\hyperlink{site-content}{Skip to
content}\protect\hyperlink{site-index}{Skip to site index}

\href{https://myaccount.nytimes3xbfgragh.onion/auth/login?response_type=cookie\&client_id=vi}{}

\href{https://www.nytimes3xbfgragh.onion/section/todayspaper}{Today's
Paper}

\href{/section/opinion}{Opinion}\textbar{}American Crime and the
Baltimore Model

\url{https://nyti.ms/2CdMXZ7}

\begin{itemize}
\item
\item
\item
\item
\item
\item
\end{itemize}

Advertisement

\protect\hyperlink{after-top}{Continue reading the main story}

\href{/section/opinion}{Opinion}

Supported by

\protect\hyperlink{after-sponsor}{Continue reading the main story}

\hypertarget{american-crime-and-the-baltimore-model}{%
\section{American Crime and the Baltimore
Model}\label{american-crime-and-the-baltimore-model}}

Cities across the country are in danger of making the same mistakes.

\href{https://www.nytimes3xbfgragh.onion/by/bret-stephens}{\includegraphics{https://static01.graylady3jvrrxbe.onion/images/2017/08/27/insider/bretstephens/bretstephens-thumbLarge-v6.png}}

By \href{https://www.nytimes3xbfgragh.onion/by/bret-stephens}{Bret
Stephens}

Opinion Columnist

\begin{itemize}
\item
  July 17, 2020
\item
  \begin{itemize}
  \item
  \item
  \item
  \item
  \item
  \item
  \end{itemize}
\end{itemize}

\includegraphics{https://static01.graylady3jvrrxbe.onion/images/2020/07/17/opinion/17stephens3/merlin_94989593_e0600611-d4a5-4dc2-9103-1532bc5581bb-articleLarge.jpg?quality=75\&auto=webp\&disable=upscale}

In 2014, there were 211 homicides in the city of Baltimore. The
following year, there were 342, an astonishing increase of 62 percent.
The murder rate has barely budged since.

What happened? On April 12, 2015, Freddie Gray sustained a fatal injury
in the back of a police van. Peaceful protests and then violence ensued.
A demoralized, under-resourced and sometimes corrupt police force
stopped doing its job properly. Nearly 30,000 residents have since fled
the city, whose
\href{https://www.baltimoresun.com/maryland/baltimore-city/bs-md-ci-population-estimates-20200326-nebck2k2anbwrcfsbknphsfgwi-story.html}{population}
is now the lowest it's been in a century.

The story of Baltimore's unraveling was best told by the journalist Alec
MacGillis in
\href{https://www.nytimes3xbfgragh.onion/2019/03/12/magazine/baltimore-tragedy-crime.html}{a
searing account} last year in The Times Magazine. It should be read
again today, against a backdrop of sudden surges in crime that are
mainly devastating minority communities. In New York, shootings during
the first three weeks in June
\href{https://www.nytimes3xbfgragh.onion/2020/06/23/nyregion/nyc-shootings-surge.html?searchResultPosition=6}{more
than doubled} over the same period last year. In Minneapolis, the
homicide rate is
\href{https://www.startribune.com/as-gun-violence-continues-minneapolis-marchers-plead-for-peace/571713702/}{double
what it was this time last year}.

\href{https://www.nytimes3xbfgragh.onion/2020/07/06/upshot/murders-rising-crime-coronavirus.html}{Murder
rates are similarly rising} in some of America's largest cities. Why is
it all happening now? Alexandria Ocasio-Cortez
\href{https://www.nydailynews.com/news/politics/ny-ocasio-cortez-crime-nyc-shoplifting-bread-hungry-20200712-4ga4tm6vzze7ricmk2o2er4iwa-story.html}{had
an idea}: Out-of-work parents, reeling from the recession, are
shoplifting to feed their kids. The New York congresswoman later
defended her remarks by insisting on the link between poverty and
violent crime.

The theory, however, is doubtful --- not to mention insulting to poor
people. As the social scientist James Q. Wilson pointed out, crime rose
dramatically in the 1960s, an era of steady employment and strong
economic growth. But it dropped across the board
\href{https://www.city-journal.org/html/crime-and-great-recession-13399.html}{during
the Great Recession} (including in rates of property theft) when the
unemployment rate abruptly doubled. The causes of crime may be
reasonably debated, but the Jean Valjean theory belongs to the pages of
Victor Hugo.

More instructive is the Baltimore example. ``The national progressive
story of Baltimore during this era of criminal-justice reform has been
the story of the police excesses that led to Gray's death and the
uprising, not the surge of violence that has overtaken the city ever
since,'' MacGillis writes. ``As a result, Baltimore has been left mostly
on its own to contend with what has been happening, which has amounted
to nothing less than a failure of order and governance the likes of
which few American cities have seen in years.''

Image

Protesters and police clashed after Freddie Gray's funeral.Credit...Matt
Roth for The New York Times

Image

A boy looked out the bus window at a line of National Guards in
Baltimore after news of charges against the policemen involved in the
death of Freddie Gray was announced.Credit...Gabriella Demczuk for The
New York Times

The causes of this are several, racism among them. But MacGillis leaves
no doubt that the quality of policing is at the center. Until 2011,
Baltimore had become safer thanks to smart policing that targeted
criminal hot spots while making fewer arrests, albeit with a rise in
police-involved shootings.

That changed after a new commissioner arrived touting the virtues of
police restraint and improved community relations. In the protests and
violence that followed Gray's death, the police were urged to hold back
until they came under attack: 130 officers were injured and the National
Guard was called in.

Toxic relations between the police and the city's political leadership
made matters worse. A federal consent decree showed little understanding
of how effective policing works, further hamstringing law enforcement.
Expanded definitions of ``use of force'' made cops especially reluctant
to intervene in situations where there was a chance of a physical
altercation. The police force shrank. A new mayor touted the benefits of
after-school programs and social mediators to treat the root causes of
crime. But, as MacGillis acidly notes, the mayor's plan ``risked
overlooking the most immediate dilemma: People inclined toward
lawbreaking increasingly thought they could do so with impunity.''

The result is a comprehensive urban tragedy that can't be blamed on long
lockdowns, hot summer weather, the coronavirus or the state of the
economy.

It's also a cautionary tale. With all the usual good intentions, cities
across America risk emulating the same catastrophic mistakes made in
Baltimore. New York has disbanded its plainclothes crime-fighting unit
and may criminalize
\href{https://www.nydailynews.com/news/politics/ny-new-york-state-police-union-pull-out-of-city-reforms-20200715-uwcgecxovbfpxo3kxjej3otrby-story.html}{the
use of holds} that, while prone to abuse, many cops consider essential
for dealing with violent suspects. Milwaukee is looking at a
\href{https://www.jsonline.com/story/news/solutions/2020/07/14/mpd-chief-details-cuts-might-required-defund-police/5421894002/}{10
percent cut in police funding}. Minneapolis may disband its police force
entirely, at least if its City Council gets its way.

\includegraphics{https://static01.graylady3jvrrxbe.onion/images/2020/07/17/opinion/17stephens2/17stephens2-articleLarge.jpg?quality=75\&auto=webp\&disable=upscale}

Idealists may hope these changes will eliminate police brutality as
communities find better ways to prevent crime than deterrence and force.
But on the hunch that human nature hasn't changed, that isn't going to
happen. Criminals, fearing less, will continue to prey on others.
Police, fearing more, will hold back from doing their jobs. Those with
means to leave their neighborhoods, will. Those without the means will
suffer.

For those under the age of, say, 35, who fail to appreciate what it
means to live in places where cops are more fearful than robbers,
there's no need to look at the historical crime data. Just put Baltimore
on your itinerary to see where certain progressive fantasies lead.

\emph{The Times is committed to publishing}
\href{https://www.nytimes3xbfgragh.onion/2019/01/31/opinion/letters/letters-to-editor-new-york-times-women.html}{\emph{a
diversity of letters}} \emph{to the editor. We'd like to hear what you
think about this or any of our articles. Here are some}
\href{https://help.nytimes3xbfgragh.onion/hc/en-us/articles/115014925288-How-to-submit-a-letter-to-the-editor}{\emph{tips}}\emph{.
And here's our email:}
\href{mailto:letters@NYTimes.com}{\emph{letters@NYTimes.com}}\emph{.}

\emph{Follow The New York Times Opinion section on}
\href{https://www.facebookcorewwwi.onion/nytopinion}{\emph{Facebook}}\emph{,}
\href{http://twitter.com/NYTOpinion}{\emph{Twitter (@NYTopinion)}}
\emph{and}
\href{https://www.instagram.com/nytopinion/}{\emph{Instagram}}\emph{.}

Advertisement

\protect\hyperlink{after-bottom}{Continue reading the main story}

\hypertarget{site-index}{%
\subsection{Site Index}\label{site-index}}

\hypertarget{site-information-navigation}{%
\subsection{Site Information
Navigation}\label{site-information-navigation}}

\begin{itemize}
\tightlist
\item
  \href{https://help.nytimes3xbfgragh.onion/hc/en-us/articles/115014792127-Copyright-notice}{©~2020~The
  New York Times Company}
\end{itemize}

\begin{itemize}
\tightlist
\item
  \href{https://www.nytco.com/}{NYTCo}
\item
  \href{https://help.nytimes3xbfgragh.onion/hc/en-us/articles/115015385887-Contact-Us}{Contact
  Us}
\item
  \href{https://www.nytco.com/careers/}{Work with us}
\item
  \href{https://nytmediakit.com/}{Advertise}
\item
  \href{http://www.tbrandstudio.com/}{T Brand Studio}
\item
  \href{https://www.nytimes3xbfgragh.onion/privacy/cookie-policy\#how-do-i-manage-trackers}{Your
  Ad Choices}
\item
  \href{https://www.nytimes3xbfgragh.onion/privacy}{Privacy}
\item
  \href{https://help.nytimes3xbfgragh.onion/hc/en-us/articles/115014893428-Terms-of-service}{Terms
  of Service}
\item
  \href{https://help.nytimes3xbfgragh.onion/hc/en-us/articles/115014893968-Terms-of-sale}{Terms
  of Sale}
\item
  \href{https://spiderbites.nytimes3xbfgragh.onion}{Site Map}
\item
  \href{https://help.nytimes3xbfgragh.onion/hc/en-us}{Help}
\item
  \href{https://www.nytimes3xbfgragh.onion/subscription?campaignId=37WXW}{Subscriptions}
\end{itemize}
