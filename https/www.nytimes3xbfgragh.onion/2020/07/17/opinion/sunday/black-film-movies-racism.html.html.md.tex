Sections

SEARCH

\protect\hyperlink{site-content}{Skip to
content}\protect\hyperlink{site-index}{Skip to site index}

\href{https://www.nytimes3xbfgragh.onion/section/opinion/sunday}{Sunday
Review}

\href{https://myaccount.nytimes3xbfgragh.onion/auth/login?response_type=cookie\&client_id=vi}{}

\href{https://www.nytimes3xbfgragh.onion/section/todayspaper}{Today's
Paper}

\href{/section/opinion/sunday}{Sunday Review}\textbar{}The Problem With
`Anti-Racist' Movie Lists

\url{https://nyti.ms/2DNVfqP}

\begin{itemize}
\item
\item
\item
\item
\item
\end{itemize}

Advertisement

\protect\hyperlink{after-top}{Continue reading the main story}

\href{/section/opinion}{Opinion}

Supported by

\protect\hyperlink{after-sponsor}{Continue reading the main story}

\hypertarget{the-problem-with-anti-racist-movie-lists}{%
\section{The Problem With `Anti-Racist' Movie
Lists}\label{the-problem-with-anti-racist-movie-lists}}

Black films should be valued for more than what they can teach white
viewers about race.

By Racquel Gates

Ms. Gates is a film scholar.

\begin{itemize}
\item
  July 17, 2020
\item
  \begin{itemize}
  \item
  \item
  \item
  \item
  \item
  \end{itemize}
\end{itemize}

\includegraphics{https://static01.graylady3jvrrxbe.onion/images/2020/07/17/opinion/17Gates2/17Gates2-articleLarge.jpg?quality=75\&auto=webp\&disable=upscale}

Watching and discussing films and TV shows has always been a source of
connection within my family. I remember watching ``Gone With the Wind''
with my parents as a child, and recall my mother shaking her head in
disapproval at Hattie McDaniel's stereotypical Mammy character. My
father countered by noting the strength of her performance and how she
dominated every scene in which she appeared.

When I was a college freshman, my father sent me a care package that
included stacks of VHS tapes on which he had lovingly and painstakingly
recorded episodes of my favorite television show, ``The Golden Girls.''
Like ``Gone With the Wind,'' it's something that we had often watched
together at home, a source of enjoyment as well as a prompt for talking
about the various issues that the episodes raised.

\includegraphics{https://static01.graylady3jvrrxbe.onion/images/2020/07/17/opinion/17Gates7/17Gates7-articleLarge.jpg?quality=75\&auto=webp\&disable=upscale}

Our experiences take place at the intersection of the personal and
political. Both of these titles were formative in my life as a media
scholar, and both are among the many cultural products that are now
being reconsidered in light of the George Floyd protests and ongoing
discussions about systemic racism. Last month ``Gone With the Wind'' was
temporarily removed from HBO Max so that a
\href{https://www.youtube.com/watch?v=0DF2FKRToiQ\&feature=emb_title}{newly
recorded introduction} providing historical context could be added. A
couple of weeks later, an episode of ``The Golden Girls'' was pulled
from Hulu altogether because of an ``accidental blackface'' sight gag
involving two characters in mud masks.

HBO Max's strategy of grappling with ``Gone With the Wind'' while
contextualizing its production and reception is a meaningful one,
requiring thoughtfulness and additional labor. Yet Hulu and other
companies have decided to do the equivalent of a ``dirty delete'':
erasing the evidence of their racist practices, rather than addressing
how they came to be in the first place. Like Hulu, many businesses and
institutions are now quick to make seemingly anti-racist overtures,
while neglecting the nuanced reflection the moment requires.

In this long overdue rush to acknowledge America's deeply entrenched
anti-blackness, white people's perspectives have remained prioritized.
\href{https://www.esquire.com/entertainment/g32799997/black-movies-shows-on-netflix/}{Countless}\href{https://www.vanityfair.com/hollywood/2020/06/the-help-netflix-watch-list}{articles}\href{https://www.digitaltrends.com/movies/best-black-movies-on-netflix/}{have}
offered variations on the same theme --- ``Concerned About Racism? Here
are 19 Anti-Racist Movies and TV Shows You Can Stream Right Now'' ---
followed by the same handful of titles. (``13th.'' ``Dear White
People.'' ``Malcolm X.'') The articles may be well-intentioned and these
works \emph{are} worth viewing. But such lists reduce Black art to a
hastily constructed manual to understanding oppression, always with
white people as the implied audience.

Image

Teyonah Parris in the 2014 film "Dear White People," directed by Justin
Simien.Credit...Ashley Nguyen/Lionsgate and Roadside Attractions

The idea that a singular film, or even a collection of films, can serve
as a guide to the history of Black oppression is simplistic. In her 1975
speech ``A Humanist View,'' Toni Morrison identified the function of
racism as one of ``distraction,'' a device that ``keeps you explaining,
over and over again, your reason for being.'' Indeed, the very idea that
Black film's greatest purpose is to be an educational primer on race in
America is a notion that we need to lay to rest.

During this reflection on blackness and media, we must focus on the
complexity and brilliance of Black film on its own merits. Now more than
ever, we should return to Black narratives that decenter whiteness or
ignore it altogether, films that connect audiences with the pathos, joy
and even treachery of the Black characters and lives they depict, the
films that recognize their complex humanity.

Image

A scene from the 1898 short film, ``Something Good Negro
Kiss.''Credit...Something Good Negro Kiss

Such films have existed since the beginning of American cinema. At first
glance, the recently discovered short film
``\href{https://www.youtube.com/watch?v=LIKU5kncg8U}{Something Good
Negro Kiss,}'' from 1898, appears to be a joyous rendering of Black
people in love, but as the film historian Allyson Field argues, the
Black performers actually seem to be satirizing Thomas Edison's
``\href{https://www.youtube.com/watch?v=IUyTcpvTPu0}{The Kiss}'' from
two years prior.

The 1913 film
``\href{https://silentfilm.org/bert-williams-lime-kiln-club-field-day/}{Lime
Kiln Club Field Day}'' features the Black minstrel performer Bert
Williams wearing blackface makeup, and complicates our understanding of
the art form. His presence alongside non-blackfaced Black actors draws
focus to the performative, rather than racist, aspects of blackface. It
demonstrates how Black performers sometimes used the makeup as a mask to
differentiate between cinematic tropes of blackness and ``real'' Black
people, a practice that indicates a keen awareness --- yes, even then
--- of how cinema functions in relation to representation.

Though we keep insisting on the importance of ``authenticity'' in Black
representations (a fraught and contested term), these early examples
suggest Black modalities that were wry and subversive, and which
understood the difference between performance and reality. How different
might our understanding of the history of film be today, had these been
the starting point for what we imagine Black film was, is and can be
yet?

Image

Richard Romain in the 1982 film ``Cane River.''Credit...Oscilloscope,
via Everett Collection

The recently rediscovered films
``\href{https://www.nytimes3xbfgragh.onion/2016/04/17/movies/losing-ground-meditates-on-art-as-it-examines-a-marriage-in-peril.html}{Losing
Ground},'' directed by Kathleen Collins, and
``\href{https://www.nytimes3xbfgragh.onion/2020/02/06/movies/cane-river-review.html}{Cane
River},'' directed by Horace Jenkins, offer loving and nuanced
explorations of Black characters. Both films were released in 1982 ---
the same year as the Nick Nolte/Eddie Murphy buddy cop film ``48 Hours''
--- yet have remained largely unknown and unseen until recently.
``Losing Ground'' focuses on a Black woman professor navigating the
tensions in her work and in her marriage, while ``Cane River'' is a love
story that intersects with class and skin color issues.

Layered, thoughtful and depicting aspects of Black experience that exist
outside of Hollywood tropes, neither film received theatrical release
until decades after their production. Their rediscovery inspires a
feeling of melancholy for me as much as excitement. How many more Black
films languish on the verge of disappearance, films that may not have
been deemed ``important'' because they cared more to focus on the lovely
intricacies of Black life rather than delivering Black pain for white
consumption?

Image

Marlon Riggs and Essex Hemphill, as seen in "Tongues Untied," a
documentary directed by Mr. Riggs.Credit...Signifyin' Works and
Frameline Distribution

Even as filmmakers like Ryan Coogler, Ava DuVernay and Barry Jenkins
have found recent success in telling these kinds of stories, Black film
is still too often assessed for its didactic value, with artistic and
intellectual contributions deemed secondary. We need to emphasize the
works of Zeinabu Irene Davis, Yvonne Welbon, Garrett Bradley, Marlon
Riggs, Dee Rees, Cheryl Dunye and other filmmakers who tap into themes
on Black peoples' experiences as individuals, and how those experiences
are shaped by race, sexuality, class and countless other social
realities. These names, among so many others, need to be at the center
of a discussion about the potential of film to connect audiences with
the intimate contours of Black life.

Like monuments, state flags and pancake mix, film and television have
always been contested and negotiated endeavors --- histories that
society is doomed to repeat because white people refuse to sit with the
discomfort and complexity of its past. The recent performative nods to
blackness through screening lists and the erasure of racist elements in
TV shows, do more to assuage white guilt than to offer recompense to
Black people, who have always had to deal with contradictions in the
very art that gives them pleasure.

These moves are shallow and condescending, meant to gaslight us into
thinking that there was a time when production companies, networks,
performers and audiences --- when ``we'' --- simply didn't know any
better. That time never existed.

Racquel Gates (\href{https://twitter.com/racquelgates}{@racquelgates})
is an associate professor of cinema and media studies at The College of
Staten Island and the author of ``Double Negative: The Black Image and
Popular Culture.''

\emph{The Times is committed to publishing}
\href{https://www.nytimes3xbfgragh.onion/2019/01/31/opinion/letters/letters-to-editor-new-york-times-women.html}{\emph{a
diversity of letters}} \emph{to the editor. We'd like to hear what you
think about this or any of our articles. Here are some}
\href{https://help.nytimes3xbfgragh.onion/hc/en-us/articles/115014925288-How-to-submit-a-letter-to-the-editor}{\emph{tips}}\emph{.
And here's our email:}
\href{mailto:letters@NYTimes.com}{\emph{letters@NYTimes.com}}\emph{.}

\emph{Follow The New York Times Opinion section on}
\href{https://www.facebookcorewwwi.onion/nytopinion}{\emph{Facebook}}\emph{,}
\href{http://twitter.com/NYTOpinion}{\emph{Twitter (@NYTopinion)}}
\emph{and}
\href{https://www.instagram.com/nytopinion/}{\emph{Instagram}}\emph{.}

Advertisement

\protect\hyperlink{after-bottom}{Continue reading the main story}

\hypertarget{site-index}{%
\subsection{Site Index}\label{site-index}}

\hypertarget{site-information-navigation}{%
\subsection{Site Information
Navigation}\label{site-information-navigation}}

\begin{itemize}
\tightlist
\item
  \href{https://help.nytimes3xbfgragh.onion/hc/en-us/articles/115014792127-Copyright-notice}{©~2020~The
  New York Times Company}
\end{itemize}

\begin{itemize}
\tightlist
\item
  \href{https://www.nytco.com/}{NYTCo}
\item
  \href{https://help.nytimes3xbfgragh.onion/hc/en-us/articles/115015385887-Contact-Us}{Contact
  Us}
\item
  \href{https://www.nytco.com/careers/}{Work with us}
\item
  \href{https://nytmediakit.com/}{Advertise}
\item
  \href{http://www.tbrandstudio.com/}{T Brand Studio}
\item
  \href{https://www.nytimes3xbfgragh.onion/privacy/cookie-policy\#how-do-i-manage-trackers}{Your
  Ad Choices}
\item
  \href{https://www.nytimes3xbfgragh.onion/privacy}{Privacy}
\item
  \href{https://help.nytimes3xbfgragh.onion/hc/en-us/articles/115014893428-Terms-of-service}{Terms
  of Service}
\item
  \href{https://help.nytimes3xbfgragh.onion/hc/en-us/articles/115014893968-Terms-of-sale}{Terms
  of Sale}
\item
  \href{https://spiderbites.nytimes3xbfgragh.onion}{Site Map}
\item
  \href{https://help.nytimes3xbfgragh.onion/hc/en-us}{Help}
\item
  \href{https://www.nytimes3xbfgragh.onion/subscription?campaignId=37WXW}{Subscriptions}
\end{itemize}
