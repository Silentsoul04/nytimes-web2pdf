Sections

SEARCH

\protect\hyperlink{site-content}{Skip to
content}\protect\hyperlink{site-index}{Skip to site index}

\href{https://myaccount.nytimes3xbfgragh.onion/auth/login?response_type=cookie\&client_id=vi}{}

\href{https://www.nytimes3xbfgragh.onion/section/todayspaper}{Today's
Paper}

\href{/section/opinion}{Opinion}\textbar{}TikTok Is Wonderful. I Still
Don't Want It on My Phone.

\url{https://nyti.ms/30icsjT}

\begin{itemize}
\item
\item
\item
\item
\item
\item
\end{itemize}

Advertisement

\protect\hyperlink{after-top}{Continue reading the main story}

\href{/section/opinion}{Opinion}

Supported by

\protect\hyperlink{after-sponsor}{Continue reading the main story}

\hypertarget{tiktok-is-wonderful-i-still-dont-want-it-on-my-phone}{%
\section{TikTok Is Wonderful. I Still Don't Want It on My
Phone.}\label{tiktok-is-wonderful-i-still-dont-want-it-on-my-phone}}

There's good reason to remain skeptical about the Chinese-owned company,
but for now it's one of the best social media platforms.

\includegraphics{https://static01.graylady3jvrrxbe.onion/images/2018/08/02/opinion/02swisher/02swisher-thumbLarge.png}

By Kara Swisher

Ms. Swisher is a contributing Opinion writer.

\begin{itemize}
\item
  July 17, 2020
\item
  \begin{itemize}
  \item
  \item
  \item
  \item
  \item
  \item
  \end{itemize}
\end{itemize}

\includegraphics{https://static01.graylady3jvrrxbe.onion/images/2020/07/16/opinion/16Swisher/16Swisher-articleLarge-v2.jpg?quality=75\&auto=webp\&disable=upscale}

\href{https://cn.nytimes3xbfgragh.onion/opinion/20200720/tiktok-ban-china/}{阅读简体中文版}\href{https://cn.nytimes3xbfgragh.onion/opinion/20200720/tiktok-ban-china/zh-hant/}{閱讀繁體中文版}

I confess: I have a TikTok-designated burner phone.

The short-video sharing service sucked me in immediately with its
algorithmically perfect cascade of videos that has been designed to
entertain me for forever.

I marvel at how smoothly it works. I am even more gobsmacked by the
creativity on display --- by many kinds of people, both professional
entertainers and just regular folks (and not just teenagers).

It is perhaps the best social media --- with an emphasis on media ---
network to come along in quite a while. Simply put, it is a wonderful
tech product.

Just recently: The shoe-jumping people. The superhero suit-switching
guy. Fabulous lesbian pairs like Madison Bailey and Mariah Linney. The
chai-making-judging stylings of Kevin Wilson, he of the perpetually
raised eyebrow. And of course, the comic Sarah Cooper's epic and
eviscerating Trump translating.

TikTok, of course, has many of the problems that plague other social
media platforms: the haters, the liars, the toxic posters. But it is ---
for now --- one of the better places to spend time online, compared with
other similar services.

\includegraphics{https://static01.graylady3jvrrxbe.onion/images/2020/07/16/opinion/16Swisher2/merlin_171745821_cc5731ed-e82e-47bb-a740-8814d8681e26-articleLarge.jpg?quality=75\&auto=webp\&disable=upscale}

Still, I don't want TikTok on my main iPhone. The reason is obvious to
most people who have been
\href{https://www.nytimes3xbfgragh.onion/2020/07/15/technology/tiktok-washington-lobbyist.html}{paying
attention to the news} of late: I worry about security and surveillance
because TikTok is owned by a Chinese company (called ByteDance).

In some ways, I feel stupid revving up my Google Pixel to use one app
since there is no proof of nefarious behavior by TikTok. And many
American tech companies whose services live on my mobile phone are,
let's be honest, data thieves themselves.

But while the efforts by U.S. companies to suck up personal data and
turn it into revenue are ongoing, the many tech firms operating in China
can be influenced and even controlled by the Chinese government. With
quite a lot of confidential information on my phone, I have become extra
wary.

Let me be clear: I am not faux indignant with a
\href{https://www.nytimes3xbfgragh.onion/reuters/2020/07/07/business/07reuters-usa-tiktok-china-pompeo.html}{Mike
Pompeo-level of alarm} about TikTok. The secretary of state recently
made one of this administration's typically disingenuous threats that it
might move to ban the app for security reasons, while also referencing
actions taken against other Chinese companies like Huawei and ZTE.

TikTok pushed back, noting after the Pompeo barb that the company has an
American chief executive and hundreds of key leaders based in the United
States.

``We have no higher priority than promoting a safe and secure app
experience for our users,'' a TikTok representative said in a statement.
``We have never provided user data to the Chinese government, nor would
we do so if asked.'' Though small, with only 35 people on staff, TikTok
has also leaned into significantly upgrading its lobbying forces in
Washington.

That makes sense since the anti-TikTok sentiment is gaining steam in the
capital. Aside from Mr. Pompeo's bluster, there is a
\href{https://www.nytimes3xbfgragh.onion/reuters/2020/07/15/technology/15reuters-usa-legislation-tiktok.html}{Republican
bill in Congress pushed by Senator Josh Hawley} to ban the app on
federal phones, and many companies have also expressed worries about its
use on corporate phones. There is also a previous and continuing effort
spearheaded by Senator Marco Rubio for the Committee on Foreign
Investment in the United States to look more closely at the 2017
acquisition by ByteDance of Musical.ly, which later became TikTok.

Democrats have also begun to pile on, backing an effort to look into
further violations by TikTok of the Children's Online Privacy Protection
Act, asking
\href{https://www.nytimes3xbfgragh.onion/reuters/2020/07/07/business/07reuters-tiktok-privacy-children-exclusive.html}{regulators
to investigate}whether the company has since breached
\href{https://www.ftc.gov/news-events/press-releases/2019/02/video-social-networking-app-musically-agrees-settle-ftc}{a
previous \$5.7 million settlement}with the Federal Trade Commission over
the illegal collection of children's personal data.

TikTok is also getting flack abroad, such as a recent move by India in a
sweeping banning of 59 Chinese apps, noting they posed threats to the
country's ``sovereignty and security.''

Image

A pediatrician in Shaoxing, China, recording a TikTok video to provide
information about various diseases.Credit...Costfoto/Barcroft Media, via
Getty Images

The obvious problem for TikTok is a backdrop of heinous digital
surveillance practices by the Chinese government, in its country and
elsewhere, sometimes to quash internal dissent and control its
population, and sometimes to get a leg up on technology by swiping it.
There's no doubt the Chinese government can control tech firms in China
and has been active in disinformation campaigns on social networks.

But, aside from the data collection that resulted in the F.T.C.
settlement (a relatively common type of violation), it is still unproven
that TikTok is doing what its detractors are alleging. The company has
been adamant that it stores U.S. user data in the United States, with
back up in Singapore. And then there are complicated tech issues related
to mobile phones, which were best expressed by the tech analyst Ben
Thompson in a recent post on his Stratchery blog: ``Banning TikTok
because it is surreptitiously stealing your email doesn't make technical
sense.''

But he also pointed out that TikTok notes in its privacy policy that
``we may share your information with a parent, subsidiary, or other
affiliate of our corporate group.''

In other words, just because TikTok doesn't grab your personal
information now does not mean it will not or might not be compelled to
do so in the future. (TikTok says such legal language is standard.)

As for China's increasingly aggressive application of its
\href{https://www.nytimes3xbfgragh.onion/2020/06/28/world/asia/china-hong-kong-national-security-law.html?searchResultPosition=1}{unconscionable
national intelligence law}, sources with knowledge of the company tell
me TikTok is not bound to it due to its unusual corporate structure. The
app, by the way, is unavailable in China itself.

There are also worries about censorship, since many things have been
disallowed on the app. And, of course, Chinese influence over users.
``The point, though, is not just censorship, but its inverse:
propaganda,'' Mr. Thompson wrote. ``TikTok's algorithm, unmoored from
the constraints of your social network or professional content creators,
is free to promote whatever videos it likes, without anyone knowing the
difference.''

He's right about the concerns and how they will only grow, which is why
TikTok needs to take care, as its popularity explodes, to not fritter
away a good thing.

There's one way to make the company's claims of safety stick. TikTok's
recent hiring of a well-regarded former Disney executive, Kevin Mayer,
as chief executive is a flashing signal that the company is likely to
spin off of ByteDance as a U.S. company. Going public in America could
mean definitively breaking its chains to the Chinese government --- and
perhaps be the best possible answer to the company's critics. That would
also give TikTok its best shot at becoming a formidable competitor to
Facebook.

That would be a good thing, removing the shadow cast by the Chinese
government and reassuring users like me who wait in anticipation of
Sarah Cooper's next Trumptastic video.

But until that day, I'm keeping her, the Donald and TikTok on my burner
phone.

\emph{The Times is committed to publishing}
\href{https://www.nytimes3xbfgragh.onion/2019/01/31/opinion/letters/letters-to-editor-new-york-times-women.html}{\emph{a
diversity of letters}} \emph{to the editor. We'd like to hear what you
think about this or any of our articles. Here are some}
\href{https://help.nytimes3xbfgragh.onion/hc/en-us/articles/115014925288-How-to-submit-a-letter-to-the-editor}{\emph{tips}}\emph{.
And here's our email:}
\href{mailto:letters@NYTimes.com}{\emph{letters@NYTimes.com}}\emph{.}

\emph{Follow The New York Times Opinion section on}
\href{https://www.facebookcorewwwi.onion/nytopinion}{\emph{Facebook}}\emph{,}
\href{http://twitter.com/NYTOpinion}{\emph{Twitter (@NYTopinion)}}
\emph{and}
\href{https://www.instagram.com/nytopinion/}{\emph{Instagram}}\emph{,
and sign up for the}
\href{http://www.nytimes3xbfgragh.onion/newsletters/opiniontoday/}{\emph{Opinion
Today newsletter}}\emph{.}

Advertisement

\protect\hyperlink{after-bottom}{Continue reading the main story}

\hypertarget{site-index}{%
\subsection{Site Index}\label{site-index}}

\hypertarget{site-information-navigation}{%
\subsection{Site Information
Navigation}\label{site-information-navigation}}

\begin{itemize}
\tightlist
\item
  \href{https://help.nytimes3xbfgragh.onion/hc/en-us/articles/115014792127-Copyright-notice}{©~2020~The
  New York Times Company}
\end{itemize}

\begin{itemize}
\tightlist
\item
  \href{https://www.nytco.com/}{NYTCo}
\item
  \href{https://help.nytimes3xbfgragh.onion/hc/en-us/articles/115015385887-Contact-Us}{Contact
  Us}
\item
  \href{https://www.nytco.com/careers/}{Work with us}
\item
  \href{https://nytmediakit.com/}{Advertise}
\item
  \href{http://www.tbrandstudio.com/}{T Brand Studio}
\item
  \href{https://www.nytimes3xbfgragh.onion/privacy/cookie-policy\#how-do-i-manage-trackers}{Your
  Ad Choices}
\item
  \href{https://www.nytimes3xbfgragh.onion/privacy}{Privacy}
\item
  \href{https://help.nytimes3xbfgragh.onion/hc/en-us/articles/115014893428-Terms-of-service}{Terms
  of Service}
\item
  \href{https://help.nytimes3xbfgragh.onion/hc/en-us/articles/115014893968-Terms-of-sale}{Terms
  of Sale}
\item
  \href{https://spiderbites.nytimes3xbfgragh.onion}{Site Map}
\item
  \href{https://help.nytimes3xbfgragh.onion/hc/en-us}{Help}
\item
  \href{https://www.nytimes3xbfgragh.onion/subscription?campaignId=37WXW}{Subscriptions}
\end{itemize}
