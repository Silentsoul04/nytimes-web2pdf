Sections

SEARCH

\protect\hyperlink{site-content}{Skip to
content}\protect\hyperlink{site-index}{Skip to site index}

\href{https://www.nytimes3xbfgragh.onion/section/style}{Style}

\href{https://myaccount.nytimes3xbfgragh.onion/auth/login?response_type=cookie\&client_id=vi}{}

\href{https://www.nytimes3xbfgragh.onion/section/todayspaper}{Today's
Paper}

\href{/section/style}{Style}\textbar{}Trying to Feel Love-Worthy (While
Working for a Dating App)

\url{https://nyti.ms/3h91KTK}

\begin{itemize}
\item
\item
\item
\item
\item
\end{itemize}

Advertisement

\protect\hyperlink{after-top}{Continue reading the main story}

Supported by

\protect\hyperlink{after-sponsor}{Continue reading the main story}

Modern Love

\hypertarget{trying-to-feel-love-worthy-while-working-for-a-dating-app}{%
\section{Trying to Feel Love-Worthy (While Working for a Dating
App)}\label{trying-to-feel-love-worthy-while-working-for-a-dating-app}}

Being awash in romantic complaints has left me --- a Black woman who's
had heartache --- feeling dismayed but hopeful.

\includegraphics{https://static01.graylady3jvrrxbe.onion/images/2020/07/19/fashion/19MODERN-DATINGAPP/19MODERN-DATINGAPP-articleLarge.jpg?quality=75\&auto=webp\&disable=upscale}

By Loré Yessuff

\begin{itemize}
\item
  July 17, 2020
\item
  \begin{itemize}
  \item
  \item
  \item
  \item
  \item
  \end{itemize}
\end{itemize}

As fascinating as it may sound, working in customer service for a dating
app tends to be repetitive and mundane. During each eight-hour shift, I
often feel like some sort of robot-cheerleader as I attempt to answer
the complaints and mollify the anxieties of digital daters around the
world.

My official title when hired --- community experience associate --- made
me think I would be engaged in interesting conversations about love and
relationships. In reality, the vast majority of ``community experience''
I end up dealing with involves questions about refunds, forgotten
passwords and duplicate accounts. I try to respond in more personal ways
to each user, but in most cases, for efficiency, I end up copy-pasting
replies.

``Hi, there! Thanks for reaching out. Let's take a look at this issue.''

``Hello, we're so sorry you're having a negative experience.''

``Hey! Thank you for bringing this to our attention. We'd love to
help!''

Copy, paste, copy, paste, copy, paste. Until I've reached or exceeded my
quota of responses per hour.

Certain inquiries break the pattern. I have messaged with users who fear
their partner is cheating; transgender people who wish to change their
gender setting; and men who feel bereft and confused after being
repeatedly ghosted. These exchanges bring a sense of humanity that
changes the pace and reminds me of my job's potential impact. Amid all
the anger and callousness is the tenderness of romantic yearning, the
universal desire to be loved.

This job opportunity arose as a silver lining during a very downhearted
summer. I was a recent college graduate recovering from a breakup,
longing to be with someone who didn't want to commit to me.

It was my first breakup, but as a Black woman, I was not new to
heartache; this pain felt familiar. The symptoms are easy to decipher.
It always starts in the throat, hums in the chest, drops to the lowest
point of the belly. Sharp, thick, burning.

\emph{{[}}\href{https://www.nytimes3xbfgragh.onion/newsletters/love-letter}{\emph{Sign
up for Love Letter, our weekly email about Modern Love, weddings and
relationships.}}\emph{{]}}

The first time I felt it was in second grade, on the school bus, when
two white boys screeched at me, saying I was ugly while tugging on my
newly plaited braids. I was so shocked that I froze, waiting for the
mockery to end.

The ache reappeared in high school when a friend told me I would be
pretty if my skin were lighter. I tried to avoid the sun that summer and
summers after. Needless to say, it didn't change anything.

Far too many Black women are taught that romantic fantasies do not
belong to us, that we are never someone's first choice or second or even
third, and that we should feel lucky if we are wanted, which really
means that we should feel suspicious.

In movies and TV, we're typically brought in as the reliable buddy or
for comic relief. Recent shows such as HBO's ``I May Destroy You'' and
``Insecure'' reflect more nuanced takes on our varied experiences, and
although I'm grateful for these stories, I'm struck by how long it has
taken for them to exist in mainstream media.

Romantic desire is complicated for everyone, but for us it is so often
political. Nearly every Black girl I know has a story about being
blatantly rejected for her Blackness --- if not rejected outright, then
fetishized or dismissed in some other racially charged way.

During my adolescence, I consumed messages that equated my worth to my
relationship status. My mom and aunties stressed the importance of being
a good girl so I could one day be a good wife. My evangelical Christian
mentors idolized sexual purity and marriage. By age 16, I understood
that the validation of straight boys should be my priority.

However, the boys I liked would accept my friendship but disregard the
possibility of dating. My friends and crushes openly told me that they
didn't date Black women, confessing this stinging truth as easily as if
they were stating a preference for pizza.

Given all that, imagine how strange it then felt for me, a recently
dumped, insecure Black girl, to accept a job at a dating app. The first
few weeks were brutal, as hundreds of anxiety-ridden pleas flooded the
inboxes, with some people sending rapid-fire messages with unsettling
urgency: ``I'm not getting any matches!'' ``Why doesn't anyone respond
to my messages???'' ``Am I ugly?'' ``I haven't been on a date in
months!'' ``This app is a scam!''

At first, I didn't know how to defuse people's anger while also
validating it. With time, I learned to look for the humanity in each
message. There was something so vulnerable about someone telling a total
**** stranger they ``just want to be loved.''

I would scroll through the daily outpouring, trying to not grow numb,
translating messages like ``There's nothing wrong with me. I'm
OK-looking, smart. There's something wrong with your app!'' into the
kind of questions we all ask: ``Am I attractive enough? Am I smart
enough? Is there something wrong with me?''

Sometimes I would reply with the words I most needed to read myself. My
supervisors had instructed me to address people with kindness and
caution. Despite the clichés I sent, the sentiment was authentic.
``Dating is really difficult,'' I would type. ``But I believe you
deserve a meaningful connection. Often it just takes time to find it.
I'm rooting for you!''

My corny encouragement usually broke down people's walls. ``Thank you,
that means a lot,'' they would reply, or ``Yes, dating is so hard. I
hope I meet someone soon, crossing my fingers tight!''

Although I was practicing empathy, I didn't kid myself that I was
learning to do this better than anyone else. At dinner one night, a
friend asked if my job was helping me master the art of dating.

I spat out my drink. ``No, not at all! I'm just as confused as the
people I talk to.''

Of course, I was on the apps too. I had learned all the tricks to
creating a promising profile: portraits that show off your personality,
bios that end with an engaging question, a verification checkmark to
show you're real. I could help others, but I still felt clueless about
improving my own digital likability.

And I knew the odds were against me: Some
\href{http://www.karen-levy.net/wp-content/uploads/2018/09/Debiasing_Desire_published.pdf}{research
has shown} that Black women are among those who receive the least
attention of any category on dating apps. Knowing that, it's hard to
have faith. A white friend once showed me her dating profile and said,
``I know exactly why these boys swiped right on me.''

How would it feel to \emph{know} you are naturally someone's type or
even a lot of people's type? How would it feel to \emph{know} you are
desired? I kept wondering these things until my wonder hardened in the
back of my throat --- sharp, thick, burning.

I became so accustomed to unrequited love and being the cheerleader for
my non-Black friends finding love that I started to believe there wasn't
anyone for me.

As soon as I began to develop feelings for someone, I would fight it
off, bracing against the looming disappointment. If a guy did express
interest, I would overthink it to the point of self-sabotage. Even when
I dated my first boyfriend, I spent most of our relationship doubting
the authenticity of his affection. I didn't know how to be desired
because I didn't believe I was.

More recently I have become better at embracing the radiance of my
Blackness, and it has become easier to feel secure in my identity. To
not just accept myself but to celebrate and admire the woman I am.

But I know enough to realize that self-love, for all its benefits, can't
kiss me on the forehead, can't cheek to cheek, can't heart-eye stare in
the middle of a room. And though I finally believe I'm worthy, I still
sometimes doubt others will be able to reach beyond their social
conditioning to believe I'm worthy too.

On Valentine's Day this year, I worked the night shift and had to laugh
at the absurdity of my circumstances. Instead of holding hands with
someone I adored, I spent the night typing messages to others racing to
find hands to hold. I felt pathetic and alone, isolated from the very
thing I was helping people find.

As the evening progressed, a Black woman messaged simply to express her
gratitude. Through the app, she said, she had found her now longtime
boyfriend --- something she never thought would happen for her.

I smiled at the attached photos of her and her partner, brown and
shining in their love. It felt like some sort of cosmic reassurance. I
patted my chest as I began to write another cliché response, but all I
wanted to say was: ``I hope to find this kind of love someday too. Thank
you, thank you.''

\href{https://loreyessuff.com/}{Loré Yessuff} is a writer in Austin,
Texas.

Modern Love can be reached at
\href{mailto:modernlove@NYTimes.com}{\nolinkurl{modernlove@NYTimes.com}}.

Want more from Modern Love? Watch the
\href{https://www.nytimes3xbfgragh.onion/2019/09/12/style/modern-love-tv-show-trailer.html}{TV
series}; sign up for the
\href{https://www.nytimes3xbfgragh.onion/newsletters/love-letter}{newsletter};
or listen to the
\href{https://www.nytimes3xbfgragh.onion/column/modern-love-podcast}{podcast}
on
\href{https://itunes.apple.com/us/podcast/modern-love/id1065559535?mt=2\&version=meter+at+0\&module=meter-Links\&pgtype=article\&contentId=\&mediaId=\&referrer=\&priority=true\&action=click\&contentCollection=meter-links-click}{iTunes},
\href{https://open.spotify.com/show/03Er7mSPq9IEewOgbPD3vO}{Spotify} or
\href{https://play.google.com/music/listen?u=0\#/ps/Iktqjbkz7bychbnofblw32dik64}{Google
Play}. We also have swag at
\href{https://store.nytimes3xbfgragh.onion/collections/modern-love}{the
NYT Store} and a book,
``\href{https://www.penguinrandomhouse.com/books/623036/modern-love-revised-and-updated-by-edited-by-daniel-jones-with-contributions-by-andrew-rannells-ayelet-waldman-amy-krouse-rosenthal-veronica-chambers-and-more/}{Modern
Love: True Stories of Love, Loss, and Redemption}.''

Advertisement

\protect\hyperlink{after-bottom}{Continue reading the main story}

\hypertarget{site-index}{%
\subsection{Site Index}\label{site-index}}

\hypertarget{site-information-navigation}{%
\subsection{Site Information
Navigation}\label{site-information-navigation}}

\begin{itemize}
\tightlist
\item
  \href{https://help.nytimes3xbfgragh.onion/hc/en-us/articles/115014792127-Copyright-notice}{©~2020~The
  New York Times Company}
\end{itemize}

\begin{itemize}
\tightlist
\item
  \href{https://www.nytco.com/}{NYTCo}
\item
  \href{https://help.nytimes3xbfgragh.onion/hc/en-us/articles/115015385887-Contact-Us}{Contact
  Us}
\item
  \href{https://www.nytco.com/careers/}{Work with us}
\item
  \href{https://nytmediakit.com/}{Advertise}
\item
  \href{http://www.tbrandstudio.com/}{T Brand Studio}
\item
  \href{https://www.nytimes3xbfgragh.onion/privacy/cookie-policy\#how-do-i-manage-trackers}{Your
  Ad Choices}
\item
  \href{https://www.nytimes3xbfgragh.onion/privacy}{Privacy}
\item
  \href{https://help.nytimes3xbfgragh.onion/hc/en-us/articles/115014893428-Terms-of-service}{Terms
  of Service}
\item
  \href{https://help.nytimes3xbfgragh.onion/hc/en-us/articles/115014893968-Terms-of-sale}{Terms
  of Sale}
\item
  \href{https://spiderbites.nytimes3xbfgragh.onion}{Site Map}
\item
  \href{https://help.nytimes3xbfgragh.onion/hc/en-us}{Help}
\item
  \href{https://www.nytimes3xbfgragh.onion/subscription?campaignId=37WXW}{Subscriptions}
\end{itemize}
