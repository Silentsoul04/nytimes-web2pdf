Sections

SEARCH

\protect\hyperlink{site-content}{Skip to
content}\protect\hyperlink{site-index}{Skip to site index}

\href{https://myaccount.nytimes3xbfgragh.onion/auth/login?response_type=cookie\&client_id=vi}{}

\href{https://www.nytimes3xbfgragh.onion/section/todayspaper}{Today's
Paper}

A Portuguese Artist's Chilled Tomato Soup

\url{https://nyti.ms/2BaiqLd}

\begin{itemize}
\item
\item
\item
\item
\item
\end{itemize}

\href{https://www.nytimes3xbfgragh.onion/spotlight/at-home?action=click\&pgtype=Article\&state=default\&region=TOP_BANNER\&context=at_home_menu}{At
Home}

\begin{itemize}
\tightlist
\item
  \href{https://www.nytimes3xbfgragh.onion/2020/08/03/well/family/the-benefits-of-talking-to-strangers.html?action=click\&pgtype=Article\&state=default\&region=TOP_BANNER\&context=at_home_menu}{Talk:
  To Strangers}
\item
  \href{https://www.nytimes3xbfgragh.onion/2020/08/01/at-home/coronavirus-make-pizza-on-a-grill.html?action=click\&pgtype=Article\&state=default\&region=TOP_BANNER\&context=at_home_menu}{Make:
  Grilled Pizza}
\item
  \href{https://www.nytimes3xbfgragh.onion/2020/07/31/arts/television/goldbergs-abc-stream.html?action=click\&pgtype=Article\&state=default\&region=TOP_BANNER\&context=at_home_menu}{Watch:
  'The Goldbergs'}
\item
  \href{https://www.nytimes3xbfgragh.onion/interactive/2020/at-home/even-more-reporters-editors-diaries-lists-recommendations.html?action=click\&pgtype=Article\&state=default\&region=TOP_BANNER\&context=at_home_menu}{Explore:
  Reporters' Google Docs}
\end{itemize}

Advertisement

\protect\hyperlink{after-top}{Continue reading the main story}

Supported by

\protect\hyperlink{after-sponsor}{Continue reading the main story}

One Good Meal

\hypertarget{a-portuguese-artists-chilled-tomato-soup}{%
\section{A Portuguese Artist's Chilled Tomato
Soup}\label{a-portuguese-artists-chilled-tomato-soup}}

For a simple lunch or dinner, Vanessa Barragão often makes arjamolho,
which is healthy, flavorful and perfect for summer.

\includegraphics{https://static01.graylady3jvrrxbe.onion/images/2020/07/15/t-magazine/15tmag-barragao-03/15tmag-barragao-03-articleLarge.jpg?quality=75\&auto=webp\&disable=upscale}

By Nick Marino

\begin{itemize}
\item
  July 17, 2020
\item
  \begin{itemize}
  \item
  \item
  \item
  \item
  \item
  \end{itemize}
\end{itemize}

\emph{In
``}\href{https://www.nytimes3xbfgragh.onion/column/one-good-meal?module=inline}{\emph{One
Good Meal}}\emph{,'' we ask cooking-inclined creative people to share
the story behind a favorite dish they actually make and eat at home on a
regular basis --- and not just when they're trying to impress.}

When summer hits the Iberian Peninsula, everyone starts eating cold
tomato soup. Spaniards, of course, prepare gazpacho --- the pinkish
purée that's often served smooth enough to be drunk like a shot.
Meanwhile, across the border, the Portuguese make \emph{arjamolho},
which is essentially gazpacho's fraternal twin. The key difference is
the texture, arjamolho's being similar to a that of a chunky salsa; it's
a rustic dish often served alongside other local staples such as grilled
sardines. But for the 28-year-old textile artist
\href{https://www.instagram.com/vanessabarragao_work/?hl=en}{Vanessa
Barragão}, a native of the seaside Algarve region of southern Portugal,
arjamolho is the main course. ``It's like a soup,'' she says, ``but at
the same time a salad.''

\includegraphics{https://static01.graylady3jvrrxbe.onion/images/2020/07/15/t-magazine/15tmag-barragao/15tmag-barragao-articleLarge.jpg?quality=75\&auto=webp\&disable=upscale}

Over the past four years, Barragão --- a trained fashion designer who
realized, while receiving her master's degree in fashion design at
Lisbon University, that she didn't want to make garments after all ---
has become one of her country's most in-demand young artists. ``I
decided, `O.K., this is not for clothes but mostly to put on the
wall,''' she recalls. **** Her work since then has focused on tapestries
and rugs, which she makes using materials discarded from
textile-industry factories. For her tapestries, she works atop a jute
canvas, applying the threads via latch hook, needle felt, macramé and
crochet until they resemble the vibrant, varied landscape of a coral
reef. Her
\href{https://www.thisiscolossal.com/2019/07/vanessa-barragao-world-tapestry/}{signature
piece}thus far is also a map of the world; it hangs, at more than six
feet high and nearly 20 feet long, in London's Heathrow Airport and took
Barragão five months to complete. She's currently finishing another
large tapestry for the 3HB Hotel, soon to open in the Portuguese city of
Faro, just down the shore from Albufeira, where she was born and
recently returned. ``I had 10 years living out of here,'' she says from
her light-filled studio near the coast. ``I lived in Lisbon and then I
decided to go to Porto and then thought, `I'm done.''' Her reasons for
going back were simple: Less traffic. More time to work. Sunny weather.
``I can walk everywhere,'' she says. ``I have the beach. It's a better
life.''

She also has her family; her mother and grandmothers have worked
alongside her on artworks, as has her younger sister Telma, who manages
Barragão's studio. (Telma also grows tomatoes, which proves especially
useful during arjamolho season.) Barragão has realized that she doesn't
need the big city in order to do her work --- ``Instagram is the main
way people find me,'' she says --- and with a schedule booked more than
a year in advance, she's hardly hurting for clients. In a moment of
worldwide hunkering down, she has found what so many of us seek: success
and tranquillity, close to home.

Image

Like most of the best summer dishes, this one has just a few
ingredients.~Credit...Pedro Guimarães

\hypertarget{vanessas-arjamolho}{%
\subsubsection{Vanessa's Arjamolho}\label{vanessas-arjamolho}}

\begin{itemize}
\item
  5 ripe tomatoes, peeled and diced
\item
  1 onion, minced
\item
  2 garlic cloves, minced
\item
  Cucumber, minced (optional)
\item
  3 tablespoons olive oil
\item
  3 tablespoons vinegar
\item
  1 tablespoon salt
\item
  Oregano (to taste)
\item
  1 loaf stale sourdough bread, cut into one-inch cubes
\end{itemize}

1. Place cut tomatoes in a large bowl, followed by the onion, garlic and
(if using) cucumber. Lightly mash together with your hand. Season with
oil, vinegar, oregano and salt. Taste and adjust if needed.

2. Pour about four cups cold water into the bowl until it has the
consistency of a light soup. Stir. Taste again to check the seasoning.

3. Serve with the bread cubes, stirring them in like croutons. The goal
is to have a chunky texture without letting the bread get too soggy.

Advertisement

\protect\hyperlink{after-bottom}{Continue reading the main story}

\hypertarget{site-index}{%
\subsection{Site Index}\label{site-index}}

\hypertarget{site-information-navigation}{%
\subsection{Site Information
Navigation}\label{site-information-navigation}}

\begin{itemize}
\tightlist
\item
  \href{https://help.nytimes3xbfgragh.onion/hc/en-us/articles/115014792127-Copyright-notice}{©~2020~The
  New York Times Company}
\end{itemize}

\begin{itemize}
\tightlist
\item
  \href{https://www.nytco.com/}{NYTCo}
\item
  \href{https://help.nytimes3xbfgragh.onion/hc/en-us/articles/115015385887-Contact-Us}{Contact
  Us}
\item
  \href{https://www.nytco.com/careers/}{Work with us}
\item
  \href{https://nytmediakit.com/}{Advertise}
\item
  \href{http://www.tbrandstudio.com/}{T Brand Studio}
\item
  \href{https://www.nytimes3xbfgragh.onion/privacy/cookie-policy\#how-do-i-manage-trackers}{Your
  Ad Choices}
\item
  \href{https://www.nytimes3xbfgragh.onion/privacy}{Privacy}
\item
  \href{https://help.nytimes3xbfgragh.onion/hc/en-us/articles/115014893428-Terms-of-service}{Terms
  of Service}
\item
  \href{https://help.nytimes3xbfgragh.onion/hc/en-us/articles/115014893968-Terms-of-sale}{Terms
  of Sale}
\item
  \href{https://spiderbites.nytimes3xbfgragh.onion}{Site Map}
\item
  \href{https://help.nytimes3xbfgragh.onion/hc/en-us}{Help}
\item
  \href{https://www.nytimes3xbfgragh.onion/subscription?campaignId=37WXW}{Subscriptions}
\end{itemize}
