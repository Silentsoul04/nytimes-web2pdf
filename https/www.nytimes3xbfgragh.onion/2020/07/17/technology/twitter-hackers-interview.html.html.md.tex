Sections

SEARCH

\protect\hyperlink{site-content}{Skip to
content}\protect\hyperlink{site-index}{Skip to site index}

\href{https://www.nytimes3xbfgragh.onion/section/technology}{Technology}

\href{https://myaccount.nytimes3xbfgragh.onion/auth/login?response_type=cookie\&client_id=vi}{}

\href{https://www.nytimes3xbfgragh.onion/section/todayspaper}{Today's
Paper}

\href{/section/technology}{Technology}\textbar{}Hackers Tell the Story
of the Twitter Attack From the Inside

\url{https://nyti.ms/2ZFjKPm}

\begin{itemize}
\item
\item
\item
\item
\item
\end{itemize}

Advertisement

\protect\hyperlink{after-top}{Continue reading the main story}

Supported by

\protect\hyperlink{after-sponsor}{Continue reading the main story}

\hypertarget{hackers-tell-the-story-of-the-twitter-attack-from-the-inside}{%
\section{Hackers Tell the Story of the Twitter Attack From the
Inside}\label{hackers-tell-the-story-of-the-twitter-attack-from-the-inside}}

Several people involved in the events that took down Twitter this week
spoke with The Times, giving the first account of what happened as a
pursuit of Bitcoin spun out of control.

\includegraphics{https://static01.graylady3jvrrxbe.onion/images/2020/07/17/business/17twitter/merlin_161161320_11215742-9ce6-4748-a87a-d905b1324c4b-articleLarge.jpg?quality=75\&auto=webp\&disable=upscale}

By
\href{https://www.nytimes3xbfgragh.onion/by/nathaniel-popper}{Nathaniel
Popper} and
\href{https://www.nytimes3xbfgragh.onion/by/kate-conger}{Kate Conger}

\begin{itemize}
\item
  Published July 17, 2020Updated July 31, 2020
\item
  \begin{itemize}
  \item
  \item
  \item
  \item
  \item
  \end{itemize}
\end{itemize}

OAKLAND, Calif. --- A Twitter hacking scheme that targeted political,
corporate and cultural elites this week began with a teasing message
between two hackers late Tuesday on the online messaging platform
Discord.

``yoo bro,'' wrote a user named ``Kirk,'' according to a screenshot of
the conversation shared with The New York Times. ``i work at twitter /
don't show this to anyone / seriously.''

He then demonstrated that he could take control of valuable Twitter
accounts --- the sort of thing that would require insider access to the
company's computer network.

The hacker who received the message, using the screen name ``lol,''
decided over the next 24 hours that Kirk did not actually work for
Twitter because he was too willing to damage the company. But Kirk did
have access to Twitter's most sensitive tools, which allowed him to take
control of almost any Twitter account, including those of former
President Barack Obama, Joseph R. Biden Jr., Elon Musk and many other
celebrities.

Despite global attention on the intrusion, which has shaken confidence
in Twitter and the
\href{https://www.nytimes3xbfgragh.onion/2020/07/16/us/politics/twitter-hack.html}{security
provided by other technology companies}, the basic details of who were
responsible, and how they did it, have been a mystery. Officials are
still in the early stages of their investigation.

But four people who participated in the scheme spoke with The Times and
shared numerous logs and screen shots of the conversations they had on
Tuesday and Wednesday, demonstrating their involvement both before and
after the hack became public.

The interviews indicate that the attack was not the work of a single
country like Russia or a sophisticated group of hackers. Instead, it was
done by a group of young people --- one of whom says he lives at home
with his mother --- who got to know one another because of their
obsession with owning early or unusual screen names, particularly one
letter or number, like @y or @6.

The Times verified that the four people were connected to the hack by
matching their social media and cryptocurrency accounts to accounts that
were involved with the events on Wednesday. They also presented
corroborating evidence of their involvement, like the logs from their
conversations on Discord, a messaging platform popular with gamers and
hackers, and Twitter.

Playing a central role in the attack was Kirk, who was taking money in
and out of the same Bitcoin address as the day went on, according to an
analysis of the Bitcoin transactions by The Times, with assistance from
the research firm Chainalysis.

But the identity of Kirk, his motivation and whether he shared his
access to Twitter with anyone else remain a mystery even to the people
who worked with him. It is still unclear how much Kirk used his access
to the accounts of people like Mr. Biden and Mr. Musk to gain more
privileged information, like their private conversations on Twitter.

The hacker ``lol'' and another one he worked with, who went by the
screen name ``ever so anxious,'' told The Times that they wanted to talk
about their work with Kirk in order to prove that they had only
facilitated the purchases and takeovers of lesser-known Twitter
addresses early in the day. They said they had not continued to work
with Kirk once he began more high-profile attacks around 3:30 p.m.
Eastern time on Wednesday.

``I just wanted to tell you my story because i think you might be able
to clear some thing up about me and ever so anxious,'' ``lol'' said in a
chat on Discord, where he shared all the logs of his conversation with
Kirk and proved his ownership of the cryptocurrency accounts he used to
transact with Kirk.

``lol'' did not confirm his real-world identity, but said he lived on
the West Coast and was in his 20s. ``ever so anxious'' said he was 19
and lived in the south of England with his mother.

Investigators looking into the attacks said several of the details given
by the hackers lined up with what they have learned so far, including
Kirk's involvement both in the big hacks later in the day and the
lower-profile attacks early on Wednesday.

The Times was initially put in touch with the hackers by a security
researcher in California, Haseeb Awan, who was communicating with them
because, he said, a number of them had previously targeted him and a
Bitcoin-related company he once owned. They also unsuccessfully targeted
his current company, \href{https://www.efani.com/}{Efani}, a secure
phone provider.

The user known as Kirk did not have much of a reputation in hacker
circles before Wednesday. His profile on Discord had been created only
on July 7.

But ``lol'' and ``ever so anxious'' were well known on the website
OGusers.com, where hackers have met for years to buy and sell valuable
social media screen names, security experts said.

For online gamers, Twitter users and hackers, so-called O.G. user names
--- usually a short word or even a number --- are hotly desired. These
eye-catching handles are often snapped up by early adopters of a new
online platform, the ``original gangsters'' of a fresh app.

Users who arrive on the platform later often crave the credibility of an
O.G. user name, and will pay thousands of dollars to hackers who steal
them from their original owners.

Image

A conversation between ``ever so anxious'' and Kirk regarding Twitter
accounts for sale. A cryptocurrency account address has been redacted
from the screenshot.

Kirk connected with ``lol'' late Tuesday and then ``ever so anxious'' on
Discord early on Wednesday, and asked if they wanted to be his
middlemen, selling Twitter accounts to the online underworld where they
were known. They would take a cut from each transaction.

In one of the first transactions, ``lol'' brokered a deal for someone
who was willing to pay \$1,500, in Bitcoin, for the Twitter user name
@y. The money went to the same Bitcoin wallet that Kirk used later in
the day when he got payments from hacking the Twitter accounts of
celebrities, the public ledger of Bitcoin transactions shows.

The group posted an ad on OGusers.com, offering Twitter handles in
exchange for Bitcoin. ``ever so anxious'' took the screen name @anxious,
which he had long coveted. (His personalized details still sit atop the
suspended account.)

``i just kinda found it cool having a username that other people would
want,'' ``ever so anxious'' said in a chat with The Times.

As the morning went on, customers poured in and the prices that Kirk
demanded went up. He also demonstrated how much access he had to
Twitter's systems. He was able to quickly change the most fundamental
security settings on any user name and sent out pictures of Twitter's
internal dashboards as proof that he had taken control of the requested
accounts.

The group handed over @dark, @w, @l, @50 and @vague, among many others.

\includegraphics{https://static01.graylady3jvrrxbe.onion/images/2020/07/17/business/17twitter2/merlin_174683604_e451ec1d-d8ba-4c25-9d35-f960495228c1-articleLarge.jpg?quality=75\&auto=webp\&disable=upscale}

One of their customers was another well-known figure among hackers
dealing in user names --- a young man known as ``PlugWalkJoe.'' On
Thursday, PlugWalkJoe was the
\href{https://krebsonsecurity.com/2020/07/whos-behind-wednesdays-epic-twitter-hack/}{subject
of an article} by the security journalist Brian Krebs, who identified
the hacker as a key player in the Twitter intrusion.

Discord logs show that while PlugWalkJoe acquired the Twitter account @6
through ``ever so anxious,'' and briefly personalized it, he was not
otherwise involved in the conversation. PlugWalkJoe, who said his real
name is Joseph O'Connor, added in an interview with The Times that he
had been getting a massage near his current home in Spain as the events
occurred.

``I don't care,'' said Mr. O'Connor, who said he was 21 and British.
``They can come arrest me. I would laugh at them. I haven't done
anything.''

Mr. O'Connor said other hackers had informed him that Kirk got access to
the Twitter credentials when he found a way into Twitter's internal
Slack messaging channel and saw them posted there, along with a service
that gave him access to the company's servers. People investigating the
case said that was consistent with what they had learned so far. A
Twitter spokesman declined to comment, citing the active investigation.

All of the transactions involving ``lol'' and ``ever so anxious'' took
place before the world knew what was going on. But shortly before 3:30
p.m., tweets from the biggest cryptocurrency companies, like Coinbase,
started asking for Bitcoin donations to the site cryptoforhealth.com.

``we just hit cb,'' an abbreviation for Coinbase, Kirk wrote to ``lol''
on Discord a minute after taking over the company's Twitter account.

The public ledger of Bitcoin transactions shows that the Bitcoin wallet
that paid to set up cryptoforhealth.com was the wallet that Kirk had
been using all morning, according to three investigators, who said they
could not speak on the record because of the open investigation.

In several messages on Wednesday morning, ``ever so anxious'' talked
about his need to get some sleep, given that it was later in the day in
England. Shortly before the big hacks began, he sent a phone message to
his girlfriend saying, ``nap time nap time,'' and he disappeared from
the Discord logs.

Kirk quickly escalated his efforts, posting a message from accounts
belonging to celebrities like Kanye West and tech titans like Jeff
Bezos: Send Bitcoin to a specific account and your money would be sent
back, doubled.

Shortly after 6 p.m., Twitter seemed to catch up with the attacker, and
the messages stopped. But the company had to turn off access for broad
swaths of users, and days later, the company was still piecing together
what had happened.

Twitter said in a
\href{https://blog.twitter.com/en_us/topics/company/2020/an-update-on-our-security-incident.html}{blog
post} that the attackers had targeted 130 accounts, gaining access and
tweeting from 45 of that set. They were able to download data from eight
of the accounts, the company added.

``We're acutely aware of our responsibilities to the people who use our
service and to society more generally,'' the blog post read. ``We're
embarrassed, we're disappointed, and more than anything, we're sorry.''

When ``ever so anxious'' woke up just after 2:30 a.m. in Britain, he
looked online, saw what had happened and sent a disappointed message to
his fellow middleman, ``lol.''

``i'm not sad more just annoyed. i mean he only made 20 btc,'' he said,
referring to Kirk's Bitcoin profits from the scam, which translated to
about \$180,000.

Kirk, whoever he was, had stopped responding to his middlemen and had
disappeared.

Advertisement

\protect\hyperlink{after-bottom}{Continue reading the main story}

\hypertarget{site-index}{%
\subsection{Site Index}\label{site-index}}

\hypertarget{site-information-navigation}{%
\subsection{Site Information
Navigation}\label{site-information-navigation}}

\begin{itemize}
\tightlist
\item
  \href{https://help.nytimes3xbfgragh.onion/hc/en-us/articles/115014792127-Copyright-notice}{©~2020~The
  New York Times Company}
\end{itemize}

\begin{itemize}
\tightlist
\item
  \href{https://www.nytco.com/}{NYTCo}
\item
  \href{https://help.nytimes3xbfgragh.onion/hc/en-us/articles/115015385887-Contact-Us}{Contact
  Us}
\item
  \href{https://www.nytco.com/careers/}{Work with us}
\item
  \href{https://nytmediakit.com/}{Advertise}
\item
  \href{http://www.tbrandstudio.com/}{T Brand Studio}
\item
  \href{https://www.nytimes3xbfgragh.onion/privacy/cookie-policy\#how-do-i-manage-trackers}{Your
  Ad Choices}
\item
  \href{https://www.nytimes3xbfgragh.onion/privacy}{Privacy}
\item
  \href{https://help.nytimes3xbfgragh.onion/hc/en-us/articles/115014893428-Terms-of-service}{Terms
  of Service}
\item
  \href{https://help.nytimes3xbfgragh.onion/hc/en-us/articles/115014893968-Terms-of-sale}{Terms
  of Sale}
\item
  \href{https://spiderbites.nytimes3xbfgragh.onion}{Site Map}
\item
  \href{https://help.nytimes3xbfgragh.onion/hc/en-us}{Help}
\item
  \href{https://www.nytimes3xbfgragh.onion/subscription?campaignId=37WXW}{Subscriptions}
\end{itemize}
