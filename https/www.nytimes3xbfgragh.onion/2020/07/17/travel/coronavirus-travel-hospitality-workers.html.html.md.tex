Sections

SEARCH

\protect\hyperlink{site-content}{Skip to
content}\protect\hyperlink{site-index}{Skip to site index}

\href{https://www.nytimes3xbfgragh.onion/section/travel}{Travel}

\href{https://myaccount.nytimes3xbfgragh.onion/auth/login?response_type=cookie\&client_id=vi}{}

\href{https://www.nytimes3xbfgragh.onion/section/todayspaper}{Today's
Paper}

\href{/section/travel}{Travel}\textbar{}Voices From Travel's Front
Lines: `We Walk in Fear, We Work in Fear'

\url{https://nyti.ms/2DSzBSv}

\begin{itemize}
\item
\item
\item
\item
\item
\end{itemize}

\href{https://www.nytimes3xbfgragh.onion/news-event/coronavirus?action=click\&pgtype=Article\&state=default\&region=TOP_BANNER\&context=storylines_menu}{The
Coronavirus Outbreak}

\begin{itemize}
\tightlist
\item
  live\href{https://www.nytimes3xbfgragh.onion/2020/08/04/world/coronavirus-covid-19.html?action=click\&pgtype=Article\&state=default\&region=TOP_BANNER\&context=storylines_menu}{Latest
  Updates}
\item
  \href{https://www.nytimes3xbfgragh.onion/interactive/2020/us/coronavirus-us-cases.html?action=click\&pgtype=Article\&state=default\&region=TOP_BANNER\&context=storylines_menu}{Maps
  and Cases}
\item
  \href{https://www.nytimes3xbfgragh.onion/interactive/2020/science/coronavirus-vaccine-tracker.html?action=click\&pgtype=Article\&state=default\&region=TOP_BANNER\&context=storylines_menu}{Vaccine
  Tracker}
\item
  \href{https://www.nytimes3xbfgragh.onion/2020/08/02/us/covid-college-reopening.html?action=click\&pgtype=Article\&state=default\&region=TOP_BANNER\&context=storylines_menu}{College
  Reopening}
\item
  \href{https://www.nytimes3xbfgragh.onion/live/2020/08/03/business/stock-market-today-coronavirus?action=click\&pgtype=Article\&state=default\&region=TOP_BANNER\&context=storylines_menu}{Economy}
\end{itemize}

Advertisement

\protect\hyperlink{after-top}{Continue reading the main story}

Supported by

\protect\hyperlink{after-sponsor}{Continue reading the main story}

\hypertarget{voices-from-travels-front-lines-we-walk-in-fear-we-work-in-fear}{%
\section{Voices From Travel's Front Lines: `We Walk in Fear, We Work in
Fear'}\label{voices-from-travels-front-lines-we-walk-in-fear-we-work-in-fear}}

Six travel workers from Alaska to the Maldives talk about returning to
the job. They are in a new world, with restrictions and health measures,
uncertainty and new procedures.

\includegraphics{https://static01.graylady3jvrrxbe.onion/images/2020/07/18/travel/17travel-top/17travel-top-articleLarge.jpg?quality=75\&auto=webp\&disable=upscale}

By \href{https://www.nytimes3xbfgragh.onion/by/emily-palmer}{Emily
Palmer} and
\href{https://www.nytimes3xbfgragh.onion/by/tariro-mzezewa}{Tariro
Mzezewa}

\begin{itemize}
\item
  Published July 17, 2020Updated July 21, 2020
\item
  \begin{itemize}
  \item
  \item
  \item
  \item
  \item
  \end{itemize}
\end{itemize}

Over the past six weeks, countries across the globe have
\href{https://www.nytimes3xbfgragh.onion/2020/06/12/travel/reopening-europe-caribbean-virus.html}{gradually
begun to reopen} to travelers, both domestic and foreign. But the travel
world, still mired in an
\href{https://www.nytimes3xbfgragh.onion/news-event/coronavirus?action=click\&pgtype=Article\&state=default\&module=styln-coronavirus-national\&variant=show\&region=TOP_BANNER\&context=storylines_menu}{unyielding
pandemic}, is drastically altered. This is especially so for workers in
the tourism industry.

Their workplaces are now filled with restrictions and health measures,
uncertainty and new procedures. With new coronavirus cases rising in
many regions, those just returning to work wonder, yet again, how long
they will be employed and if they are safe.

We spoke with six travel workers, from Alaska to the Maldives, on
returning to the job. They shared their stories in English and Spanish.
Like the workers we
\href{https://www.nytimes3xbfgragh.onion/2020/03/25/travel/coronavirus-travel-hospitality-workers.html}{interviewed}
when international borders first closed, they are embarking on a new
journey, one of caution and reserve.

Among some of them is relief and a renewed sense of hope, while others
seek to answer this question: With all the new obstacles, how can
visitors feel welcome again?

\emph{The interviews have been edited and condensed for clarity.}

\includegraphics{https://static01.graylady3jvrrxbe.onion/images/2020/07/17/travel/17voices-henry/17voices-henry-articleLarge.jpg?quality=75\&auto=webp\&disable=upscale}

The Maldives

\hypertarget{henry-jordan}{%
\subsection{Henry Jordan}\label{henry-jordan}}

\emph{\textbf{The executive chef at}}
\textbf{\href{https://www.luxresorts.com/en/maldives/hotel/luxnorthmaleatoll}{\emph{LUX*
North Male Atoll Resort \& Villas}}}

We're one of the few resorts in the Maldives that never officially
closed. We tried to get other guests from other islands who were
displaced. Toward the end, we only had one family staying with us, a
Russian family who couldn't get home.

We cooked for the three guests we had --- two plus one small child. In
the end, they Googled some Russian recipes.

I think it will be a
\href{http://www.tourism.gov.mv/en/news/maldives_welcomes_back_first_tourists}{slow
transition}. In the last few days we've had more guest bookings, we're
at right about 30 percent capacity for next week. We weren't even
expecting that, so that's a good sign. When people see other people
traveling that's when we'll see it really pick up.

The owner of the resort lives on the island and from the get-go he made
sure that nobody was worried about the future: Whatever was going to
come, we would get through it. Nobody lost their job.

Now, all team members wear masks. The service team wears gloves. We have
daily temperature checks, sanitizer in all the restaurants, a six-foot
distance between the tables.

\hypertarget{latest-updates-global-coronavirus-outbreak}{%
\section{\texorpdfstring{\href{https://www.nytimes3xbfgragh.onion/2020/08/04/world/coronavirus-covid-19.html?action=click\&pgtype=Article\&state=default\&region=MAIN_CONTENT_1\&context=storylines_live_updates}{Latest
Updates: Global Coronavirus
Outbreak}}{Latest Updates: Global Coronavirus Outbreak}}\label{latest-updates-global-coronavirus-outbreak}}

Updated 2020-08-04T09:15:14.275Z

\begin{itemize}
\tightlist
\item
  \href{https://www.nytimes3xbfgragh.onion/2020/08/04/world/coronavirus-covid-19.html?action=click\&pgtype=Article\&state=default\&region=MAIN_CONTENT_1\&context=storylines_live_updates\#link-6b644638}{`Long
  days, long nights': Washington prepares for a prolonged fight over
  virus relief.}
\item
  \href{https://www.nytimes3xbfgragh.onion/2020/08/04/world/coronavirus-covid-19.html?action=click\&pgtype=Article\&state=default\&region=MAIN_CONTENT_1\&context=storylines_live_updates\#link-7af9fca0}{Israel's
  rocky reopening of its schools may be a lesson for the U.S.}
\item
  \href{https://www.nytimes3xbfgragh.onion/2020/08/04/world/coronavirus-covid-19.html?action=click\&pgtype=Article\&state=default\&region=MAIN_CONTENT_1\&context=storylines_live_updates\#link-33bf9168}{Hurricane
  Isaias arrives in North Carolina as officials along the East Coast
  scramble.}
\end{itemize}

\href{https://www.nytimes3xbfgragh.onion/2020/08/04/world/coronavirus-covid-19.html?action=click\&pgtype=Article\&state=default\&region=MAIN_CONTENT_1\&context=storylines_live_updates}{See
more updates}

More live coverage:
\href{https://www.nytimes3xbfgragh.onion/live/2020/08/03/business/stock-market-today-coronavirus?action=click\&pgtype=Article\&state=default\&region=MAIN_CONTENT_1\&context=storylines_live_updates}{Markets}

Normally you keep the table set as an inviting feel, but now the table
has to stay bare, so there's a bit to get used to visually. A lot of the
little luxury touches associated with a five-star hotel, we're having to
tweak.

Our kitchens are well-air conditioned. The New York City basement
kitchen doesn't exist here. The hotel has a custom-made cloth mask, so
that may be better than the hot disposable masks. We'll figure it out
soon.

Image

U.S. Virgin Islands

\hypertarget{jalayne-jones}{%
\subsection{Jalayne Jones}\label{jalayne-jones}}

\emph{\textbf{A bartender at}}
\textbf{\href{http://www.skinnylegsvi.com/}{\emph{Skinny Legs Bar and
Grill}}} \emph{\textbf{on St. John}}

I didn't expect it to be this busy. I'd say we're doing like 50 to 60
percent. We allow 50 people in a restaurant at a time. And we've had
maximum capacity a couple of times, with people outside, waiting to get
in.

I think tourists are just happy to be here. The employees are more on
edge. There are people --- mostly tourists --- who won't wear their
masks and follow the rules. I say, ``Go get your masks,'' and they get
all huffy and leave. We're just trying to make a buck, and if masks will
keep us from blowing up, we'd like to make our money while we're open.

When we shut down, we were blessed to have only 11 cases across all
three islands. We've seen spikes in California, spikes in Florida ---
and people are coming here from those states.
\href{https://www.covid19usvi.com/}{Our numbers are rising}, and it
doesn't make sense: We had the benefit of being away from the mainland.

Thirty to 40 percent of our population is elderly and as an island with
a majority Black population, we have to consider that this virus is
affecting us disproportionally. I feel like opening back up is
incredibly detrimental to the community, but you know: tourism.

It's the way we've become so dependent on tourism, people thinking this
is the ``American Paradise,'' and the entitlement of tourists. So now
people are starting to say, ``We've got to fix that.''

This could wake people up. Our jobs are in this industry that, at the
same time, is killing us.

\emph{Though the U.S.V.I. and Skinny Legs reopened to tourists in June,
the restaurant is now closed for several weeks after Ms. Jones and other
employees learned that they had been exposed to the coronavirus.}

Image

Florida

\hypertarget{beatrice-menendez}{%
\subsection{Beatrice Menendez}\label{beatrice-menendez}}

\emph{\textbf{A room attendant at the luxury hotel}}
\textbf{\href{https://www.fontainebleau.com/}{\emph{Fontainebleau Miami
Beach}}} \emph{\textbf{for 34 years}}

I didn't find out we were closing through the company. The news came
from my co-workers. It felt like I wasn't important --- that they
wouldn't call and talk to us about it.

And then the famous card saying we'd been laid off arrived in the mail
--- 15 days after the hotel closed. They said they'd call us when they
reopened and that was it.

I was surprised to hear that they were reopening and the way I found out
about it was through the union reaching out to me. Right now there's a
super-low occupancy level, about 11 percent.

Everything has changed. We walk in fear, we work in fear, we don't have
the same compassion with each other because it's just scary to be in
there. It doesn't seem like a hotel, it seems like a hospital.

You don't know who has the virus and who doesn't. We are front line
workers who every day interact with guests --- who may or may not have
their masks on --- and that's something that's very scary, having that
constant contact with so many people.

Me going back to work on June 1 was something that I was forced to do.
The hotel is not taking into consideration that we could get sick. I'm a
person who suffers from respiratory problems. There are already
co-workers who have caught Covid-19 after returning to work.

\emph{During the closure of the hotel, the owners of the Fontainebleau}
\href{https://www.miamiherald.com/news/business/tourism-cruises/article243908132.html}{\emph{decided
to stop paying for worker health insurance}}\emph{. Almost 1,000
staffers, including Ms. Menendez, lost their insurance earlier this
month as a result.}

Image

Rome

\hypertarget{manuela-guzzi}{%
\subsection{Manuela Guzzi}\label{manuela-guzzi}}

\emph{\textbf{A flight attendant for the Italian airline Alitalia}}

I haven't been at work since March, when things got bad with the
coronavirus. I'm on standby on Friday for my first flight since then. I
don't know where I'll be going or if I'll be going, but I'll definitely
be back on a flight to Budapest next week. I'm so nervous and excited.

I've been a flight attendant since I was 20. I am 49 now. I've been with
Alitalia since 1995 and have never experienced something like this.
During lockdown I was a little worried I wouldn't go back to work this
year, but instead, I focused on my family and cooking good food. I'm
lucky I have my salary, but it's a base. The more I work, the more I
earn, and I wasn't working for four months.

\href{https://www.nytimes3xbfgragh.onion/news-event/coronavirus?action=click\&pgtype=Article\&state=default\&region=MAIN_CONTENT_3\&context=storylines_faq}{}

\hypertarget{the-coronavirus-outbreak-}{%
\subsubsection{The Coronavirus Outbreak
›}\label{the-coronavirus-outbreak-}}

\hypertarget{frequently-asked-questions}{%
\paragraph{Frequently Asked
Questions}\label{frequently-asked-questions}}

Updated August 3, 2020

\begin{itemize}
\item ~
  \hypertarget{im-a-small-business-owner-can-i-get-relief}{%
  \paragraph{I'm a small-business owner. Can I get
  relief?}\label{im-a-small-business-owner-can-i-get-relief}}

  \begin{itemize}
  \tightlist
  \item
    The
    \href{https://www.nytimes3xbfgragh.onion/article/small-business-loans-stimulus-grants-freelancers-coronavirus.html?action=click\&pgtype=Article\&state=default\&region=MAIN_CONTENT_3\&context=storylines_faq}{stimulus
    bills enacted in March} offer help for the millions of American
    small businesses. Those eligible for aid are businesses and
    nonprofit organizations with fewer than 500 workers, including sole
    proprietorships, independent contractors and freelancers. Some
    larger companies in some industries are also eligible. The help
    being offered, which is being managed by the Small Business
    Administration, includes the Paycheck Protection Program and the
    Economic Injury Disaster Loan program. But lots of folks have
    \href{https://www.nytimes3xbfgragh.onion/interactive/2020/05/07/business/small-business-loans-coronavirus.html?action=click\&pgtype=Article\&state=default\&region=MAIN_CONTENT_3\&context=storylines_faq}{not
    yet seen payouts.} Even those who have received help are confused:
    The rules are draconian, and some are stuck sitting on
    \href{https://www.nytimes3xbfgragh.onion/2020/05/02/business/economy/loans-coronavirus-small-business.html?action=click\&pgtype=Article\&state=default\&region=MAIN_CONTENT_3\&context=storylines_faq}{money
    they don't know how to use.} Many small-business owners are getting
    less than they expected or
    \href{https://www.nytimes3xbfgragh.onion/2020/06/10/business/Small-business-loans-ppp.html?action=click\&pgtype=Article\&state=default\&region=MAIN_CONTENT_3\&context=storylines_faq}{not
    hearing anything at all.}
  \end{itemize}
\item ~
  \hypertarget{what-are-my-rights-if-i-am-worried-about-going-back-to-work}{%
  \paragraph{What are my rights if I am worried about going back to
  work?}\label{what-are-my-rights-if-i-am-worried-about-going-back-to-work}}

  \begin{itemize}
  \tightlist
  \item
    Employers have to provide
    \href{https://www.osha.gov/SLTC/covid-19/standards.html}{a safe
    workplace} with policies that protect everyone equally.
    \href{https://www.nytimes3xbfgragh.onion/article/coronavirus-money-unemployment.html?action=click\&pgtype=Article\&state=default\&region=MAIN_CONTENT_3\&context=storylines_faq}{And
    if one of your co-workers tests positive for the coronavirus, the
    C.D.C.} has said that
    \href{https://www.cdc.gov/coronavirus/2019-ncov/community/guidance-business-response.html}{employers
    should tell their employees} -\/- without giving you the sick
    employee's name -\/- that they may have been exposed to the virus.
  \end{itemize}
\item ~
  \hypertarget{should-i-refinance-my-mortgage}{%
  \paragraph{Should I refinance my
  mortgage?}\label{should-i-refinance-my-mortgage}}

  \begin{itemize}
  \tightlist
  \item
    \href{https://www.nytimes3xbfgragh.onion/article/coronavirus-money-unemployment.html?action=click\&pgtype=Article\&state=default\&region=MAIN_CONTENT_3\&context=storylines_faq}{It
    could be a good idea,} because mortgage rates have
    \href{https://www.nytimes3xbfgragh.onion/2020/07/16/business/mortgage-rates-below-3-percent.html?action=click\&pgtype=Article\&state=default\&region=MAIN_CONTENT_3\&context=storylines_faq}{never
    been lower.} Refinancing requests have pushed mortgage applications
    to some of the highest levels since 2008, so be prepared to get in
    line. But defaults are also up, so if you're thinking about buying a
    home, be aware that some lenders have tightened their standards.
  \end{itemize}
\item ~
  \hypertarget{what-is-school-going-to-look-like-in-september}{%
  \paragraph{What is school going to look like in
  September?}\label{what-is-school-going-to-look-like-in-september}}

  \begin{itemize}
  \tightlist
  \item
    It is unlikely that many schools will return to a normal schedule
    this fall, requiring the grind of
    \href{https://www.nytimes3xbfgragh.onion/2020/06/05/us/coronavirus-education-lost-learning.html?action=click\&pgtype=Article\&state=default\&region=MAIN_CONTENT_3\&context=storylines_faq}{online
    learning},
    \href{https://www.nytimes3xbfgragh.onion/2020/05/29/us/coronavirus-child-care-centers.html?action=click\&pgtype=Article\&state=default\&region=MAIN_CONTENT_3\&context=storylines_faq}{makeshift
    child care} and
    \href{https://www.nytimes3xbfgragh.onion/2020/06/03/business/economy/coronavirus-working-women.html?action=click\&pgtype=Article\&state=default\&region=MAIN_CONTENT_3\&context=storylines_faq}{stunted
    workdays} to continue. California's two largest public school
    districts --- Los Angeles and San Diego --- said on July 13, that
    \href{https://www.nytimes3xbfgragh.onion/2020/07/13/us/lausd-san-diego-school-reopening.html?action=click\&pgtype=Article\&state=default\&region=MAIN_CONTENT_3\&context=storylines_faq}{instruction
    will be remote-only in the fall}, citing concerns that surging
    coronavirus infections in their areas pose too dire a risk for
    students and teachers. Together, the two districts enroll some
    825,000 students. They are the largest in the country so far to
    abandon plans for even a partial physical return to classrooms when
    they reopen in August. For other districts, the solution won't be an
    all-or-nothing approach.
    \href{https://bioethics.jhu.edu/research-and-outreach/projects/eschool-initiative/school-policy-tracker/}{Many
    systems}, including the nation's largest, New York City, are
    devising
    \href{https://www.nytimes3xbfgragh.onion/2020/06/26/us/coronavirus-schools-reopen-fall.html?action=click\&pgtype=Article\&state=default\&region=MAIN_CONTENT_3\&context=storylines_faq}{hybrid
    plans} that involve spending some days in classrooms and other days
    online. There's no national policy on this yet, so check with your
    municipal school system regularly to see what is happening in your
    community.
  \end{itemize}
\item ~
  \hypertarget{is-the-coronavirus-airborne}{%
  \paragraph{Is the coronavirus
  airborne?}\label{is-the-coronavirus-airborne}}

  \begin{itemize}
  \tightlist
  \item
    The coronavirus
    \href{https://www.nytimes3xbfgragh.onion/2020/07/04/health/239-experts-with-one-big-claim-the-coronavirus-is-airborne.html?action=click\&pgtype=Article\&state=default\&region=MAIN_CONTENT_3\&context=storylines_faq}{can
    stay aloft for hours in tiny droplets in stagnant air}, infecting
    people as they inhale, mounting scientific evidence suggests. This
    risk is highest in crowded indoor spaces with poor ventilation, and
    may help explain super-spreading events reported in meatpacking
    plants, churches and restaurants.
    \href{https://www.nytimes3xbfgragh.onion/2020/07/06/health/coronavirus-airborne-aerosols.html?action=click\&pgtype=Article\&state=default\&region=MAIN_CONTENT_3\&context=storylines_faq}{It's
    unclear how often the virus is spread} via these tiny droplets, or
    aerosols, compared with larger droplets that are expelled when a
    sick person coughs or sneezes, or transmitted through contact with
    contaminated surfaces, said Linsey Marr, an aerosol expert at
    Virginia Tech. Aerosols are released even when a person without
    symptoms exhales, talks or sings, according to Dr. Marr and more
    than 200 other experts, who
    \href{https://academic.oup.com/cid/article/doi/10.1093/cid/ciaa939/5867798}{have
    outlined the evidence in an open letter to the World Health
    Organization}.
  \end{itemize}
\end{itemize}

I live in Rome with my husband and two teenage daughters, so you see why
I am eager to go back to work. I'm joking. I miss the people I work with
and I miss the people I meet when I travel. I'm tired of cooking
everyday. My skin is so nice now from all this rest and being behind a
mask, so I'm really ready to start again. I'm not worried about people
not following the rules. They don't have a choice but to follow the
rules. They have to wear masks. I'll have to wear the mask for 12 hours.

I've been talking to my colleagues and they, like me, were enjoying all
the family time at the beginning, and having a moment to relax, but now
want to start working again. There are a lot of sleepless nights in this
job and we didn't miss them at the start of lockdown, but now we do.
It's a hard job being a flight attendant, but it's like a drug.

Image

St. Lucia

\hypertarget{charlene-mohammed}{%
\subsection{Charlene Mohammed}\label{charlene-mohammed}}

\emph{\textbf{A front desk agent and entertainer at}}
\textbf{\href{https://www.baygardensresorts.com/}{\emph{Bay Gardens
Resorts}}\emph{, a collection of family-owned hotels}}

How do I feel about going back to work? Thrilled, ecstatic, so happy to
know I'm going to welcome guests back, but also to know I get to see my
co-workers again. We are all excited to get back. It feels good to know
that we are reopening, that the island is open although people are not
coming now. But when they do, it's going to be like, ``Oh my gosh.
Welcome back!''

In March, things were a bit slow then everyone had to be sent home ---
the guests and the staff. It was sad and shocking at once. I've been in
the tourism industry for eight years and I have never experienced a
total shutdown. It hit everybody hard.

We've had to learn new protocols. Our managers were working hard on
training and sending us all the information we'd need to eventually come
back to work and be safe. We had lots of emails and Zoom calls. We did
some training on washing hands --- imagine training on how to wash your
hands --- how to handle food, how to handle takeaway orders,
housekeeping training, how to handle rooms and so on.

\href{https://www.nytimes3xbfgragh.onion/interactive/2020/05/06/travel/coronavirus-travel-questions.html}{}

\includegraphics{https://static01.graylady3jvrrxbe.onion/images/2020/05/06/multimedia/06FOT-coverairplanes/06FOT-coverairplanes-articleLarge-v2.jpg}

\hypertarget{the-future-of-travel}{%
\subsection{The Future of Travel}\label{the-future-of-travel}}

Perhaps no industry has been as hard hit by the pandemic as tourism. As
restrictions on companies and travelers ease, what will the new world
look like?

I was thinking about how I'm a big hugger. Normally, when I see my
returning guests pop up in the lobby, I run to them with open arms. I
need to come up with a way to make them feel welcome again. I know we
can't touch and there are no more handshakes. The guests can't see our
smile, but we have to learn to smile with our eyes, and let them see
that we still have effective communication with them. They'll know we
are smiling under our masks.

Image

ALASKA

\hypertarget{kristin-platt}{%
\subsection{Kristin Platt}\label{kristin-platt}}

\emph{\textbf{A reservations manager for}}
\textbf{\href{https://www.kingoftheriver.com/}{\emph{King of the River
Fishing}}\emph{, a tour company in Kenai}}

I'm the one who picks up the phones and helps people plan their fishing
trips with us. I usually spend the winter doing that, and the summer
making sure everything is ready to go.

We have a small team of three. Me, my boss, Dean, who has been leading
these trips for more than 30 years, and Jason, another guide. Things got
really quiet in March, which in other years is when the phones are
typically ringing the most, with people from all over the world wanting
to come here and go on fishing trips. Some people want to go deep-sea
fishing, others want more remote trips that involve planes.

The height of the pandemic was filled with uncertainty and questioning
about whether things would reopen and if travel would return this
summer. We typically start fishing in mid-May, and Alaska reopened fully
on Memorial Day, so that eased some of our worries.

Alaska had a 14-day quarantine that really prevented people from coming.
That was a good thing, and in the last month things started picking up
again. When the mandatory quarantine period stopped, mandatory Covid
testing at the airport started. To come here now you have to present a
negative test at the airport or get a test here. We are not checking any
of that stuff. We trust that the people who are coming here are
following the rules and being honest. I've been surprised by how many
people are willing to take a Covid test just so they can come fishing.

One of the biggest changes this year is that everything is happening at
the last minute. Instead of calling in the winter to book for the
summer, people are calling now and coming in a few weeks. That's made my
job a bit crazy, but we are still at 25 percent capacity. We're excited
to be back at work and welcoming people back.

\emph{\textbf{Follow New York Times Travel}}
\emph{on}\href{https://www.instagram.com/nytimestravel/}{\emph{Instagram}}\emph{,}\href{https://twitter.com/nytimestravel}{\emph{Twitter}}
\emph{and}\href{https://www.facebookcorewwwi.onion/nytimestravel/}{\emph{Facebook}}\emph{.
And}\href{https://www.nytimes3xbfgragh.onion/newsletters/traveldispatch}{\emph{sign
up for our weekly Travel Dispatch newsletter}} \emph{to receive expert
tips on traveling smarter and inspiration for your next vacation}

Advertisement

\protect\hyperlink{after-bottom}{Continue reading the main story}

\hypertarget{site-index}{%
\subsection{Site Index}\label{site-index}}

\hypertarget{site-information-navigation}{%
\subsection{Site Information
Navigation}\label{site-information-navigation}}

\begin{itemize}
\tightlist
\item
  \href{https://help.nytimes3xbfgragh.onion/hc/en-us/articles/115014792127-Copyright-notice}{©~2020~The
  New York Times Company}
\end{itemize}

\begin{itemize}
\tightlist
\item
  \href{https://www.nytco.com/}{NYTCo}
\item
  \href{https://help.nytimes3xbfgragh.onion/hc/en-us/articles/115015385887-Contact-Us}{Contact
  Us}
\item
  \href{https://www.nytco.com/careers/}{Work with us}
\item
  \href{https://nytmediakit.com/}{Advertise}
\item
  \href{http://www.tbrandstudio.com/}{T Brand Studio}
\item
  \href{https://www.nytimes3xbfgragh.onion/privacy/cookie-policy\#how-do-i-manage-trackers}{Your
  Ad Choices}
\item
  \href{https://www.nytimes3xbfgragh.onion/privacy}{Privacy}
\item
  \href{https://help.nytimes3xbfgragh.onion/hc/en-us/articles/115014893428-Terms-of-service}{Terms
  of Service}
\item
  \href{https://help.nytimes3xbfgragh.onion/hc/en-us/articles/115014893968-Terms-of-sale}{Terms
  of Sale}
\item
  \href{https://spiderbites.nytimes3xbfgragh.onion}{Site Map}
\item
  \href{https://help.nytimes3xbfgragh.onion/hc/en-us}{Help}
\item
  \href{https://www.nytimes3xbfgragh.onion/subscription?campaignId=37WXW}{Subscriptions}
\end{itemize}
