Sections

SEARCH

\protect\hyperlink{site-content}{Skip to
content}\protect\hyperlink{site-index}{Skip to site index}

\href{https://www.nytimes3xbfgragh.onion/section/us}{U.S.}

\href{https://myaccount.nytimes3xbfgragh.onion/auth/login?response_type=cookie\&client_id=vi}{}

\href{https://www.nytimes3xbfgragh.onion/section/todayspaper}{Today's
Paper}

\href{/section/us}{U.S.}\textbar{}Newsom Order Would Keep Most
California Schools Online

\href{https://nyti.ms/32uOzYU}{https://nyti.ms/32uOzYU}

\begin{itemize}
\item
\item
\item
\item
\item
\end{itemize}

\href{https://www.nytimes3xbfgragh.onion/news-event/coronavirus?action=click\&pgtype=Article\&state=default\&region=TOP_BANNER\&context=storylines_menu}{The
Coronavirus Outbreak}

\begin{itemize}
\tightlist
\item
  live\href{https://www.nytimes3xbfgragh.onion/2020/08/04/world/coronavirus-cases.html?action=click\&pgtype=Article\&state=default\&region=TOP_BANNER\&context=storylines_menu}{Latest
  Updates}
\item
  \href{https://www.nytimes3xbfgragh.onion/interactive/2020/us/coronavirus-us-cases.html?action=click\&pgtype=Article\&state=default\&region=TOP_BANNER\&context=storylines_menu}{Maps
  and Cases}
\item
  \href{https://www.nytimes3xbfgragh.onion/interactive/2020/science/coronavirus-vaccine-tracker.html?action=click\&pgtype=Article\&state=default\&region=TOP_BANNER\&context=storylines_menu}{Vaccine
  Tracker}
\item
  \href{https://www.nytimes3xbfgragh.onion/2020/08/02/us/covid-college-reopening.html?action=click\&pgtype=Article\&state=default\&region=TOP_BANNER\&context=storylines_menu}{College
  Reopening}
\item
  \href{https://www.nytimes3xbfgragh.onion/live/2020/08/04/business/stock-market-today-coronavirus?action=click\&pgtype=Article\&state=default\&region=TOP_BANNER\&context=storylines_menu}{Economy}
\end{itemize}

Advertisement

\protect\hyperlink{after-top}{Continue reading the main story}

Supported by

\protect\hyperlink{after-sponsor}{Continue reading the main story}

\hypertarget{newsom-order-would-keep-most-california-schools-online}{%
\section{Newsom Order Would Keep Most California Schools
Online}\label{newsom-order-would-keep-most-california-schools-online}}

Districts in counties with rising caseloads will be required to teach
remotely until conditions improve.

\includegraphics{https://static01.graylady3jvrrxbe.onion/images/2020/07/17/us/17virus-calschools/17virus-calschools-articleLarge.jpg?quality=75\&auto=webp\&disable=upscale}

\href{https://www.nytimes3xbfgragh.onion/by/jill-cowan}{\includegraphics{https://static01.graylady3jvrrxbe.onion/images/2018/12/10/multimedia/author-jill-cowan/author-jill-cowan-thumbLarge.png}}

By \href{https://www.nytimes3xbfgragh.onion/by/jill-cowan}{Jill Cowan}

\begin{itemize}
\item
  Published July 17, 2020Updated July 24, 2020
\item
  \begin{itemize}
  \item
  \item
  \item
  \item
  \item
  \end{itemize}
\end{itemize}

LOS ANGELES --- Responding to
\href{https://www.nytimes3xbfgragh.onion/interactive/2020/us/california-coronavirus-cases.html}{soaring
coronavirus infections} and growing concern from teachers, Gov. Gavin
Newsom of
\href{https://www.nytimes3xbfgragh.onion/2020/07/24/us/ca-schools-reopening.html}{California}
announced new rules on Friday that would force many of the state's
districts to teach remotely when school starts next month.

More than 80 percent of the state's population lives in counties that
would currently not qualify for
\href{https://www.nytimes3xbfgragh.onion/2020/07/24/us/ca-schools-reopening.html}{schools}
to reopen based on their surging caseloads and hospitalization rates.
The rules would also require most students to wear masks in classrooms
and force schools that do hold in-person classes to shut down if enough
students or employees test positive for the virus.

``We all prefer in-classroom instruction for all the obvious reasons,''
Mr. Newsom said, ``but only if it can be done safely.''

The announcement comes at the end of a week in which many school
districts across the state and the country, including California's two
largest,
\href{https://www.nytimes3xbfgragh.onion/2020/07/13/us/lausd-san-diego-school-reopening.html}{Los
Angeles and San Diego}, abandoned plans for in-person instruction,
saying they would start the school year remotely, and in which
California announced
\href{https://www.nytimes3xbfgragh.onion/2020/07/14/us/california-counties-reopening.html}{a
sweeping rollback} of plans to reopen businesses.

\emph{{[}Sign up}
\href{https://www.nytimes3xbfgragh.onion/newsletters/california-today}{\emph{for
California Today}}\emph{, our newsletter from the Golden State.{]}}

Education leaders in Houston, Atlanta, Nashville, Arlington, Va., and
Broward County, Fla., also said this week that they planned to open the
academic year online, despite
\href{https://www.nytimes3xbfgragh.onion/2020/07/14/us/coronavirus-schools-fall.html}{pressure
from President Trump} and some Republican governors who want students in
their classrooms five days a week.

In Texas, where state officials had previously put limits on online
schooling, \href{https://pbs.twimg.com/media/EdI6_hZWkAAG24Y.jpg}{new
guidelines} were issued Friday that would allow as many as eight weeks
of online-only instruction when schools return next month.

And leaders of Chicago's public school system, the nation's
third-largest district after New York and Los Angeles, said on Friday
that they were planning for a mix of in-person and online classes. But
they stressed that the announcement was a tentative framework, with a
final plan expected in August. New York City schools are also planning
an in-person and online mix.

The California rules announced on Friday would require schools in
counties that the state has put on a ``watchlist'' ---
\href{https://www.cdph.ca.gov/Programs/CID/DCDC/Pages/COVID-19/COVID19CountyDataTable.aspx}{based
on indicators} that include new infections per capita, the test
positivity rate and the hospitalization rate --- to teach online until
conditions improve. Currently, 33 of the state's 58 counties, including
many of the most populated,
\href{https://www.cdph.ca.gov/Programs/CID/DCDC/Pages/COVID-19/CountyMonitoringDataStep2.aspx}{are
on the list}.

Counties would have to be off the list for at least two weeks before
their classrooms would be allowed to reopen, but the decision would
still be up to local officials on whether to resume in-person classes,
the governor said.

\hypertarget{latest-updates-global-coronavirus-outbreak}{%
\section{\texorpdfstring{\href{https://www.nytimes3xbfgragh.onion/2020/08/04/world/coronavirus-cases.html?action=click\&pgtype=Article\&state=default\&region=MAIN_CONTENT_1\&context=storylines_live_updates}{Latest
Updates: Global Coronavirus
Outbreak}}{Latest Updates: Global Coronavirus Outbreak}}\label{latest-updates-global-coronavirus-outbreak}}

Updated 2020-08-04T22:07:26.192Z

\begin{itemize}
\tightlist
\item
  \href{https://www.nytimes3xbfgragh.onion/2020/08/04/world/coronavirus-cases.html?action=click\&pgtype=Article\&state=default\&region=MAIN_CONTENT_1\&context=storylines_live_updates\#link-2daa96b5}{As
  talks drag on, McConnell signals openness to jobless aid extension
  that Republicans have opposed.}
\item
  \href{https://www.nytimes3xbfgragh.onion/2020/08/04/world/coronavirus-cases.html?action=click\&pgtype=Article\&state=default\&region=MAIN_CONTENT_1\&context=storylines_live_updates\#link-1228a480}{Novavax
  sees encouraging results from two studies of its experimental
  vaccine.}
\item
  \href{https://www.nytimes3xbfgragh.onion/2020/08/04/world/coronavirus-cases.html?action=click\&pgtype=Article\&state=default\&region=MAIN_CONTENT_1\&context=storylines_live_updates\#link-4825b93}{Public
  and private schools in Maryland and elsewhere are divided over
  in-person instruction.}
\end{itemize}

\href{https://www.nytimes3xbfgragh.onion/2020/08/04/world/coronavirus-cases.html?action=click\&pgtype=Article\&state=default\&region=MAIN_CONTENT_1\&context=storylines_live_updates}{See
more updates}

More live coverage:
\href{https://www.nytimes3xbfgragh.onion/live/2020/08/04/business/stock-market-today-coronavirus?action=click\&pgtype=Article\&state=default\&region=MAIN_CONTENT_1\&context=storylines_live_updates}{Markets}

Many districts in those counties had already said they would start the
academic year online, including San Francisco, Oakland, Sacramento, Long
Beach, Santa Ana and San Bernardino. But at least a few districts that
had planned to hold in-person classes would be required to switch.

``Quite frankly, we are going to have to pivot, which is my new least
favorite word,'' said Eimear O'Farrell, the superintendent of Clovis
Unified School District in Fresno County, which is on the watchlist.

On Wednesday, the
\href{https://www.cusd.com/CUSDAt-A-Glance.aspx}{Central Valley
district} with about 43,000 students
\href{https://www.fresnobee.com/news/coronavirus/article244300057.html}{had
rolled out} what Ms. O'Farrell described as a painstakingly developed
plan to welcome students back to classrooms in August.

``It is a challenge and a source of frustration for us,'' she said in a
virtual news conference on Friday afternoon. ``Every superintendent has
been aiming to get this information out in time.''

The statewide rules would also require teachers and staff members in
schools that are allowed to reopen to maintain six feet of physical
distance with one another and children, and mandate masks for students
in third grade and up. Younger children would be encouraged but not
required to wear face coverings, and all children would be encouraged to
maintain six feet of distance.

The guidelines recommend that school employees be tested regularly for
the coronavirus, something teachers across the country have been pushing
for, although the federal Centers for Disease Control and Prevention has
said doing so
\href{https://www.cdc.gov/coronavirus/2019-ncov/community/schools-childcare/k-12-testing.html}{is
not necessary}, and scaling up testing has been a challenge.

Mr. Newsom's unveiling of a statewide framework for schools marked a
contrast to his usual reluctance to apply one set of policies for all of
California, which he has often described as a kind of
\href{https://www.nytimes3xbfgragh.onion/2020/04/14/us/california-coronavirus-newsom-nation-state.html}{nation-state
unto itself}, with roughly 40 million residents spread across a vast and
varied geography.

By and large,
\href{https://www.nytimes3xbfgragh.onion/2020/06/04/us/california-newsom-coronavirus.html}{Californians
have said they are supportive} of restrictions aimed at curbing the
spread of the virus. But Mr. Newsom has been hesitant to impose
statewide orders, instead repeating that ``localism is determinative''
and allowing counties to drive much of their own reopening.

The results have been at times confusing for residents attempting to
navigate
\href{https://www.nytimes3xbfgragh.onion/article/coronavirus-california-reopening-phases.html}{a
patchwork of restrictions} --- though on Monday, in response to climbing
numbers of new cases and shrinking hospital capacity in some areas, he
announced the most
\href{https://www.nytimes3xbfgragh.onion/2020/07/14/us/california-counties-reopening.html}{sweeping
statewide rollback} yet of efforts to reopen many businesses.

Still, the state's efforts have met pushback from conservative
officials, largely in more rural parts of the state where the virus has
hit less hard, but also in some populous counties.

In Orange County, where local officials have taken public stances
\href{https://www.nytimes3xbfgragh.onion/aponline/2020/07/14/us/ap-us-virus-outbreak-school-politics.html}{against
many of the state's restrictions}, the
\href{https://www.ocregister.com/2020/07/15/wagner-chau-back-away-from-controversial-school-board-recommendation/}{Board
of Education voted this week to recommend} that schools reopen without
requiring students to wear masks --- a decision that sparked widespread
outcry. But the board is largely advisory and does not carry authority
over the county's more than two dozen districts, and many education
leaders there have said they will adhere to state and county public
health recommendations.

\href{https://www.nytimes3xbfgragh.onion/news-event/coronavirus?action=click\&pgtype=Article\&state=default\&region=MAIN_CONTENT_3\&context=storylines_faq}{}

\hypertarget{the-coronavirus-outbreak-}{%
\subsubsection{The Coronavirus Outbreak
›}\label{the-coronavirus-outbreak-}}

\hypertarget{frequently-asked-questions}{%
\paragraph{Frequently Asked
Questions}\label{frequently-asked-questions}}

Updated August 4, 2020

\begin{itemize}
\item ~
  \hypertarget{i-have-antibodies-am-i-now-immune}{%
  \paragraph{I have antibodies. Am I now
  immune?}\label{i-have-antibodies-am-i-now-immune}}

  \begin{itemize}
  \tightlist
  \item
    As of right
    now,\href{https://www.nytimes3xbfgragh.onion/2020/07/22/health/covid-antibodies-herd-immunity.html?action=click\&pgtype=Article\&state=default\&region=MAIN_CONTENT_3\&context=storylines_faq}{that
    seems likely, for at least several months.} There have been
    frightening accounts of people suffering what seems to be a second
    bout of Covid-19. But experts say these patients may have a
    drawn-out course of infection, with the virus taking a slow toll
    weeks to months after initial exposure. People infected with the
    coronavirus typically
    \href{https://www.nature.com/articles/s41586-020-2456-9}{produce}
    immune molecules called antibodies, which are
    \href{https://www.nytimes3xbfgragh.onion/2020/05/07/health/coronavirus-antibody-prevalence.html?action=click\&pgtype=Article\&state=default\&region=MAIN_CONTENT_3\&context=storylines_faq}{protective
    proteins made in response to an
    infection}\href{https://www.nytimes3xbfgragh.onion/2020/05/07/health/coronavirus-antibody-prevalence.html?action=click\&pgtype=Article\&state=default\&region=MAIN_CONTENT_3\&context=storylines_faq}{.
    These antibodies may} last in the body
    \href{https://www.nature.com/articles/s41591-020-0965-6}{only two to
    three months}, which may seem worrisome, but that's perfectly normal
    after an acute infection subsides, said Dr. Michael Mina, an
    immunologist at Harvard University. It may be possible to get the
    coronavirus again, but it's highly unlikely that it would be
    possible in a short window of time from initial infection or make
    people sicker the second time.
  \end{itemize}
\item ~
  \hypertarget{im-a-small-business-owner-can-i-get-relief}{%
  \paragraph{I'm a small-business owner. Can I get
  relief?}\label{im-a-small-business-owner-can-i-get-relief}}

  \begin{itemize}
  \tightlist
  \item
    The
    \href{https://www.nytimes3xbfgragh.onion/article/small-business-loans-stimulus-grants-freelancers-coronavirus.html?action=click\&pgtype=Article\&state=default\&region=MAIN_CONTENT_3\&context=storylines_faq}{stimulus
    bills enacted in March} offer help for the millions of American
    small businesses. Those eligible for aid are businesses and
    nonprofit organizations with fewer than 500 workers, including sole
    proprietorships, independent contractors and freelancers. Some
    larger companies in some industries are also eligible. The help
    being offered, which is being managed by the Small Business
    Administration, includes the Paycheck Protection Program and the
    Economic Injury Disaster Loan program. But lots of folks have
    \href{https://www.nytimes3xbfgragh.onion/interactive/2020/05/07/business/small-business-loans-coronavirus.html?action=click\&pgtype=Article\&state=default\&region=MAIN_CONTENT_3\&context=storylines_faq}{not
    yet seen payouts.} Even those who have received help are confused:
    The rules are draconian, and some are stuck sitting on
    \href{https://www.nytimes3xbfgragh.onion/2020/05/02/business/economy/loans-coronavirus-small-business.html?action=click\&pgtype=Article\&state=default\&region=MAIN_CONTENT_3\&context=storylines_faq}{money
    they don't know how to use.} Many small-business owners are getting
    less than they expected or
    \href{https://www.nytimes3xbfgragh.onion/2020/06/10/business/Small-business-loans-ppp.html?action=click\&pgtype=Article\&state=default\&region=MAIN_CONTENT_3\&context=storylines_faq}{not
    hearing anything at all.}
  \end{itemize}
\item ~
  \hypertarget{what-are-my-rights-if-i-am-worried-about-going-back-to-work}{%
  \paragraph{What are my rights if I am worried about going back to
  work?}\label{what-are-my-rights-if-i-am-worried-about-going-back-to-work}}

  \begin{itemize}
  \tightlist
  \item
    Employers have to provide
    \href{https://www.osha.gov/SLTC/covid-19/standards.html}{a safe
    workplace} with policies that protect everyone equally.
    \href{https://www.nytimes3xbfgragh.onion/article/coronavirus-money-unemployment.html?action=click\&pgtype=Article\&state=default\&region=MAIN_CONTENT_3\&context=storylines_faq}{And
    if one of your co-workers tests positive for the coronavirus, the
    C.D.C.} has said that
    \href{https://www.cdc.gov/coronavirus/2019-ncov/community/guidance-business-response.html}{employers
    should tell their employees} -\/- without giving you the sick
    employee's name -\/- that they may have been exposed to the virus.
  \end{itemize}
\item ~
  \hypertarget{should-i-refinance-my-mortgage}{%
  \paragraph{Should I refinance my
  mortgage?}\label{should-i-refinance-my-mortgage}}

  \begin{itemize}
  \tightlist
  \item
    \href{https://www.nytimes3xbfgragh.onion/article/coronavirus-money-unemployment.html?action=click\&pgtype=Article\&state=default\&region=MAIN_CONTENT_3\&context=storylines_faq}{It
    could be a good idea,} because mortgage rates have
    \href{https://www.nytimes3xbfgragh.onion/2020/07/16/business/mortgage-rates-below-3-percent.html?action=click\&pgtype=Article\&state=default\&region=MAIN_CONTENT_3\&context=storylines_faq}{never
    been lower.} Refinancing requests have pushed mortgage applications
    to some of the highest levels since 2008, so be prepared to get in
    line. But defaults are also up, so if you're thinking about buying a
    home, be aware that some lenders have tightened their standards.
  \end{itemize}
\item ~
  \hypertarget{what-is-school-going-to-look-like-in-september}{%
  \paragraph{What is school going to look like in
  September?}\label{what-is-school-going-to-look-like-in-september}}

  \begin{itemize}
  \tightlist
  \item
    It is unlikely that many schools will return to a normal schedule
    this fall, requiring the grind of
    \href{https://www.nytimes3xbfgragh.onion/2020/06/05/us/coronavirus-education-lost-learning.html?action=click\&pgtype=Article\&state=default\&region=MAIN_CONTENT_3\&context=storylines_faq}{online
    learning},
    \href{https://www.nytimes3xbfgragh.onion/2020/05/29/us/coronavirus-child-care-centers.html?action=click\&pgtype=Article\&state=default\&region=MAIN_CONTENT_3\&context=storylines_faq}{makeshift
    child care} and
    \href{https://www.nytimes3xbfgragh.onion/2020/06/03/business/economy/coronavirus-working-women.html?action=click\&pgtype=Article\&state=default\&region=MAIN_CONTENT_3\&context=storylines_faq}{stunted
    workdays} to continue. California's two largest public school
    districts --- Los Angeles and San Diego --- said on July 13, that
    \href{https://www.nytimes3xbfgragh.onion/2020/07/13/us/lausd-san-diego-school-reopening.html?action=click\&pgtype=Article\&state=default\&region=MAIN_CONTENT_3\&context=storylines_faq}{instruction
    will be remote-only in the fall}, citing concerns that surging
    coronavirus infections in their areas pose too dire a risk for
    students and teachers. Together, the two districts enroll some
    825,000 students. They are the largest in the country so far to
    abandon plans for even a partial physical return to classrooms when
    they reopen in August. For other districts, the solution won't be an
    all-or-nothing approach.
    \href{https://bioethics.jhu.edu/research-and-outreach/projects/eschool-initiative/school-policy-tracker/}{Many
    systems}, including the nation's largest, New York City, are
    devising
    \href{https://www.nytimes3xbfgragh.onion/2020/06/26/us/coronavirus-schools-reopen-fall.html?action=click\&pgtype=Article\&state=default\&region=MAIN_CONTENT_3\&context=storylines_faq}{hybrid
    plans} that involve spending some days in classrooms and other days
    online. There's no national policy on this yet, so check with your
    municipal school system regularly to see what is happening in your
    community.
  \end{itemize}
\end{itemize}

That difficulty is mirrored at the federal level, where guidance on the
safety of reopening schools has been unsteady.

On Friday, the C.D.C. postponed releasing new guidance amid a clash with
Mr. Trump, who this month criticized the agency's proposed school
reopening guidelines as
\href{https://www.nytimes3xbfgragh.onion/2020/07/08/us/politics/trump-schools-reopening.html}{``very
tough and expensive.''} A copy of the draft rules to which Mr. Trump
apparently objected, outlined in a
\href{https://www.nytimes3xbfgragh.onion/2020/07/10/us/politics/trump-schools-reopening.html}{document
obtained by The New York Times} and marked ``For Internal Use Only,''
warned that fully reopening schools remained ``the highest risk'' for
spreading the virus.

But without being able to send their children to school, parents must
shoulder huge burdens, which affects their ability to work. Tens of
millions of school children are
\href{https://www.nytimes3xbfgragh.onion/2020/06/05/us/coronavirus-education-lost-learning.html}{falling
behind academically}, and the trends will widen existing socioeconomic
gaps.

Some educators
\href{https://www.nytimes3xbfgragh.onion/2020/07/11/us/virus-teachers-classrooms.html}{are
fearful and angry about the prospect of returning to schools} with
unanswered questions. The 310,000-member California Teachers Association
\href{https://aarjb2jw4n53e35fhbquj418-wpengine.netdna-ssl.com/wp-content/uploads/2020/07/CTA-Elected-Leaders-COVID-Letter-7.8.20.pdf}{urged
state leaders} this month to require districts to start the year ``under
robust distance learning protocols'' and described even a hybrid model
as ``high-risk.''

On Friday, the president of the California Federation of Teachers, which
represents 120,000 school employees, said the state's new guidelines did
not go far enough to protect them. He objected to the state's
willingness to allow waivers for some schools in counties on the
watchlist, as well as the lack of mandatory social distancing and
testing for all students.

``This updated guidance still allows for an `acceptable amount of harm'
to come to our students, their families and our education
professionals,'' the president, Jeff Freitas, said in a statement,
demanding that schools across California continue to provide remote
learning until ``the governor can ensure adequate safety measures
throughout the state.''

Erin Springer, who teaches fifth grade at a charter school in Inglewood,
said that starting the new year with remote learning has felt inevitable
since the beginning of summer. She said she wished the call had been
made sooner, so that she and her colleagues could have prepared.

``Almost every teacher I know felt like we were just wasting time,'' Ms.
Springer said.

Mr. Newsom emphasized that education officials across the state would be
working hard to ensure that all students would have access to ``rigorous
distance learning,'' including access to devices and connectivity and
live interactions every day with teachers and other students.

He added that leaders would pay specific attention to students who are
homeless, in the foster care system, or who are English language
learners. The state, he said, has put \$5.3 billion in additional money
toward those efforts.

``We want to create some sense of equivalency,'' Mr. Newsom said.

In Marin County, a Bay Area suburb that is on the state's watchlist,
Brooke Palizi, a mother of two elementary school students, said the
district recently announced that parents would be able to choose between
letting their children attend socially distant in-person classes, remote
learning, and a virtual academy akin to home schooling.

The new guidelines, she said, at least take the guesswork out of her
children's back-to-school plans.

``A decision's been made, instead of it being up in the air, which is
annoying,'' she said.

Dan Levin contributed reporting from New York.

Advertisement

\protect\hyperlink{after-bottom}{Continue reading the main story}

\hypertarget{site-index}{%
\subsection{Site Index}\label{site-index}}

\hypertarget{site-information-navigation}{%
\subsection{Site Information
Navigation}\label{site-information-navigation}}

\begin{itemize}
\tightlist
\item
  \href{https://help.nytimes3xbfgragh.onion/hc/en-us/articles/115014792127-Copyright-notice}{©~2020~The
  New York Times Company}
\end{itemize}

\begin{itemize}
\tightlist
\item
  \href{https://www.nytco.com/}{NYTCo}
\item
  \href{https://help.nytimes3xbfgragh.onion/hc/en-us/articles/115015385887-Contact-Us}{Contact
  Us}
\item
  \href{https://www.nytco.com/careers/}{Work with us}
\item
  \href{https://nytmediakit.com/}{Advertise}
\item
  \href{http://www.tbrandstudio.com/}{T Brand Studio}
\item
  \href{https://www.nytimes3xbfgragh.onion/privacy/cookie-policy\#how-do-i-manage-trackers}{Your
  Ad Choices}
\item
  \href{https://www.nytimes3xbfgragh.onion/privacy}{Privacy}
\item
  \href{https://help.nytimes3xbfgragh.onion/hc/en-us/articles/115014893428-Terms-of-service}{Terms
  of Service}
\item
  \href{https://help.nytimes3xbfgragh.onion/hc/en-us/articles/115014893968-Terms-of-sale}{Terms
  of Sale}
\item
  \href{https://spiderbites.nytimes3xbfgragh.onion}{Site Map}
\item
  \href{https://help.nytimes3xbfgragh.onion/hc/en-us}{Help}
\item
  \href{https://www.nytimes3xbfgragh.onion/subscription?campaignId=37WXW}{Subscriptions}
\end{itemize}
