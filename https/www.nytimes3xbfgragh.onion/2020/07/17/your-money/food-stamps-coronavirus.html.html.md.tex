Sections

SEARCH

\protect\hyperlink{site-content}{Skip to
content}\protect\hyperlink{site-index}{Skip to site index}

\href{https://www.nytimes3xbfgragh.onion/section/your-money}{Your Money}

\href{https://myaccount.nytimes3xbfgragh.onion/auth/login?response_type=cookie\&client_id=vi}{}

\href{https://www.nytimes3xbfgragh.onion/section/todayspaper}{Today's
Paper}

\href{/section/your-money}{Your Money}\textbar{}Are You Eligible for
Food Stamps Now? Maybe, but It's Complex

\url{https://nyti.ms/3hdoQIy}

\begin{itemize}
\item
\item
\item
\item
\item
\end{itemize}

\hypertarget{the-coronavirus-outbreak}{%
\subsubsection{\texorpdfstring{\href{https://www.nytimes3xbfgragh.onion/news-event/coronavirus?name=styln-coronavirus-national\&region=TOP_BANNER\&variant=undefined\&block=storyline_menu_recirc\&action=click\&pgtype=Article\&impression_id=34c72290-e386-11ea-9791-4976ace6e451}{The
Coronavirus
Outbreak}}{The Coronavirus Outbreak}}\label{the-coronavirus-outbreak}}

\begin{itemize}
\tightlist
\item
  live\href{https://www.nytimes3xbfgragh.onion/2020/08/20/world/coronavirus-covid.html?name=styln-coronavirus-national\&region=TOP_BANNER\&variant=undefined\&block=storyline_menu_recirc\&action=click\&pgtype=Article\&impression_id=34c749a0-e386-11ea-9791-4976ace6e451}{Latest
  Updates}
\item
  \href{https://www.nytimes3xbfgragh.onion/interactive/2020/us/coronavirus-us-cases.html?name=styln-coronavirus-national\&region=TOP_BANNER\&variant=undefined\&block=storyline_menu_recirc\&action=click\&pgtype=Article\&impression_id=34c749a1-e386-11ea-9791-4976ace6e451}{Maps
  and Cases}
\item
  \href{https://www.nytimes3xbfgragh.onion/interactive/2020/science/coronavirus-vaccine-tracker.html?name=styln-coronavirus-national\&region=TOP_BANNER\&variant=undefined\&block=storyline_menu_recirc\&action=click\&pgtype=Article\&impression_id=34c749a2-e386-11ea-9791-4976ace6e451}{Vaccine
  Tracker}
\item
  \href{https://www.nytimes3xbfgragh.onion/2020/08/19/us/colleges-closing-covid.html?name=styln-coronavirus-national\&region=TOP_BANNER\&variant=undefined\&block=storyline_menu_recirc\&action=click\&pgtype=Article\&impression_id=34c749a3-e386-11ea-9791-4976ace6e451}{Colleges
  Closing}
\item
  \href{https://www.nytimes3xbfgragh.onion/live/2020/08/20/business/stock-market-today-coronavirus?name=styln-coronavirus-national\&region=TOP_BANNER\&variant=undefined\&block=storyline_menu_recirc\&action=click\&pgtype=Article\&impression_id=34c749a4-e386-11ea-9791-4976ace6e451}{Economy}
\end{itemize}

Advertisement

\protect\hyperlink{after-top}{Continue reading the main story}

Supported by

\protect\hyperlink{after-sponsor}{Continue reading the main story}

Your Money

\hypertarget{are-you-eligible-for-food-stamps-now-maybe-but-its-complex}{%
\section{Are You Eligible for Food Stamps Now? Maybe, but It's
Complex}\label{are-you-eligible-for-food-stamps-now-maybe-but-its-complex}}

States run SNAP, and many students and older people don't realize
they're eligible. Without more federal support, millions more may
qualify.

\includegraphics{https://static01.graylady3jvrrxbe.onion/images/2020/07/17/business/17money/merlin_174674154_12a41568-9815-4fab-a63d-46d8e49ce948-articleLarge.jpg?quality=75\&auto=webp\&disable=upscale}

\href{https://www.nytimes3xbfgragh.onion/by/ron-lieber}{\includegraphics{https://static01.graylady3jvrrxbe.onion/images/2018/10/22/multimedia/author-ron-lieber/author-ron-lieber-thumbLarge.png}}

By \href{https://www.nytimes3xbfgragh.onion/by/ron-lieber}{Ron Lieber}

\begin{itemize}
\item
  July 17, 2020
\item
  \begin{itemize}
  \item
  \item
  \item
  \item
  \item
  \end{itemize}
\end{itemize}

The safety net is starting to unravel.

At the end of the month, struggling Americans could lose the extra \$600
per week they've been receiving in unemployment insurance. Some
\href{https://www.nytimes3xbfgragh.onion/2020/07/11/your-money/coronavirus-eviction-prevention-renters-landlord.html}{eviction
protections} are already expiring.

And as people scramble to afford basic needs, hunger looms.

Tens of millions of Americans are in danger. According to Census Bureau
\href{https://www.census.gov/householdpulsedata}{Pulse Survey} data
released this week,
\href{https://www.census.gov/data-tools/demo/hhp/\#/?measures=FIR}{10.8
percent} of American adults are experiencing some level of food
insecurity. Louisiana, Nevada and Ohio had the highest rates: 17 to 18
percent. Food lines have been a feature of newspaper front pages and
home pages for months now.

And yet there is a program that may be able to help millions of
struggling Americans. One that was underused even before the coronavirus
crisis: food stamps, or as they are known in most places now, the
Supplemental Nutrition Assistance Program.

Policy experts and social services administrators are hoping that
everyone whose income has gone to zero or close to it will at least ask.
``If you've never accessed these benefits before, it may be because of
the way that SNAP in particular has been portrayed or vilified,'' said
Carlos M. Rodriguez, president and chief executive of the Community
FoodBank of New Jersey, which \href{https://cfbnj.org/findfood/}{helps
people} sign up for SNAP. ``People do not understand that this program
is here for them at this exact time.''

SNAP is overseen by the Department of Agriculture, which lays out the
rules. States handle applications and administration, and they have some
leeway with the federal regulations. (And with the terms:
\href{https://mydss.mo.gov/food-assistance/food-stamp-program}{Missouri}
still uses the older ``food stamp'' phrasing.)

As a result, it's possible to offer some general guidelines for
understanding how the program works, but your state has the final word.
The rules are numerous and complicated, but there are exceptions and
waivers that might apply to you --- so don't be deterred.

\hypertarget{latest-updates-the-coronavirus-outbreak}{%
\section{\texorpdfstring{\href{https://www.nytimes3xbfgragh.onion/2020/08/20/world/coronavirus-covid.html?action=click\&pgtype=Article\&state=default\&region=MAIN_CONTENT_1\&context=storylines_live_updates}{Latest
Updates: The Coronavirus
Outbreak}}{Latest Updates: The Coronavirus Outbreak}}\label{latest-updates-the-coronavirus-outbreak}}

Updated 2020-08-21T07:46:15.883Z

\begin{itemize}
\tightlist
\item
  \href{https://www.nytimes3xbfgragh.onion/2020/08/20/world/coronavirus-covid.html?action=click\&pgtype=Article\&state=default\&region=MAIN_CONTENT_1\&context=storylines_live_updates\#link-68774d88}{Shutdowns,
  warnings and scoldings follow alarming incidents on college campuses.}
\item
  \href{https://www.nytimes3xbfgragh.onion/2020/08/20/world/coronavirus-covid.html?action=click\&pgtype=Article\&state=default\&region=MAIN_CONTENT_1\&context=storylines_live_updates\#link-26b58724}{Biden
  knocks Trump's pandemic response, and outlines a national strategy.}
\item
  \href{https://www.nytimes3xbfgragh.onion/2020/08/20/world/coronavirus-covid.html?action=click\&pgtype=Article\&state=default\&region=MAIN_CONTENT_1\&context=storylines_live_updates\#link-4e542da3}{U.S.
  health agencies announce moves to confront the flu season and
  plummeting child vaccination rates.}
\end{itemize}

\href{https://www.nytimes3xbfgragh.onion/2020/08/20/world/coronavirus-covid.html?action=click\&pgtype=Article\&state=default\&region=MAIN_CONTENT_1\&context=storylines_live_updates}{See
more updates}

More live coverage:
\href{https://www.nytimes3xbfgragh.onion/live/2020/08/20/business/stock-market-today-coronavirus?action=click\&pgtype=Article\&state=default\&region=MAIN_CONTENT_1\&context=storylines_live_updates}{Markets}

\hypertarget{am-i-eligible}{%
\subsection{Am I eligible?}\label{am-i-eligible}}

In the 2018 fiscal year,
\href{https://fns-prod.azureedge.net/sites/default/files/resource-files/Characteristics2018-Summary.pdf}{39.7
million people} qualified in an average month. To do so, they usually
had to pass both income tests and asset tests, though households with
elderly or disabled people may face less strict rules.

In most places, someone living alone can have a gross monthly income of
no more than \$1,354 and a net income of \$1,041. For a family of four,
the gross income limit is \$2,790 while the net income limit is \$2,146.
The Food and Nutrition Service of the Department of Agriculture lists
these limits and many other rules on its website via a
\href{https://www.fns.usda.gov/snap/recipient/eligibility}{SNAP
frequently asked questions page}.

Net income figures account for deductions that the program allows. Those
deductions include allowances for earnings (to encourage work),
dependent care, certain medical expenses and unusually large housing
costs. Applicants generally have to provide documentation.

Money you receive from unemployment payments may reduce or eliminate
your SNAP eligibility. Still, if unemployment is your only income and
you have few assets, it's worth applying for SNAP to see if you qualify.

The cap on assets is \$2,250, or \$3,500 if a household has someone 60
or older or someone with a disability. Homes and most retirement plan
balances don't count. Vehicles can count, though states have leeway to
set those rules.

\hypertarget{is-there-a-work-requirement}{%
\subsection{Is there a work
requirement?}\label{is-there-a-work-requirement}}

Yes, \href{https://www.fns.usda.gov/snap/work-requirements}{two of
them}.

First, if you're between the ages of 16 and 59, you're supposed to
enroll in relevant state training programs, accept suitable offers of
employment and not quit voluntarily or choose to work less than 30 hours
per week. But there are exceptions, including for people caring for
children under 6 years old or incapacitated adults, and those who have a
physical or mental limitation or are participating regularly in a drug
or alcohol treatment program.

There's another set of rules for people between the ages of 18 and 49
who are both able bodied and have no dependents, including working or
participating in a work program at least 80 hours per month. You can
read more about them on the Department of Agriculture's
\href{https://www.fns.usda.gov/snap/work-requirements-policies}{website}.

Waivers sometimes apply to work rules as well, which is why it's
important to apply for SNAP if you're not sure how your own work
situation applies, instead of just assuming that you're ineligible.

\hypertarget{how-does-the-application-work}{%
\subsection{How does the application
work?}\label{how-does-the-application-work}}

You apply through your state. The Department of Agriculture has a
\href{https://www.fns.usda.gov/snap/state-directory}{map-based
directory} on its website, and the Center on Budget and Policy
Priorities
\href{https://www.cbpp.org/research/food-assistance/snap-state-by-state-data-fact-sheets-and-resources}{has
collected} additional state-by-state information.

For people with no internet access, SNAP's phone number is
1-800-221-5689. There or via the 211 phone service in many areas, you
can likely find a state program's phone number.

Most states have online applications and calculators that screen for
eligibility. The application process usually includes an interview,
which can often happen over the phone. The process is supposed to take
no more than 30 days, and it could take less than a week if your income
or assets are particularly low.

To gain access to benefits, you'll use an electronic benefit transfer
card that works like a debit card in grocery stores. You'll need to be
ready to recertify eligibility from time to time, which can be a major
obstacle for struggling individuals who may also be trying to navigate
uncertain unemployment schedules or commute without a reliable vehicle.

\href{https://www.nytimes3xbfgragh.onion/news-event/coronavirus?action=click\&pgtype=Article\&state=default\&region=MAIN_CONTENT_3\&context=storylines_faq}{}

\hypertarget{the-coronavirus-outbreak-}{%
\subsubsection{The Coronavirus Outbreak
›}\label{the-coronavirus-outbreak-}}

\hypertarget{frequently-asked-questions}{%
\paragraph{Frequently Asked
Questions}\label{frequently-asked-questions}}

Updated August 17, 2020

\begin{itemize}
\item ~
  \hypertarget{why-does-standing-six-feet-away-from-others-help}{%
  \paragraph{Why does standing six feet away from others
  help?}\label{why-does-standing-six-feet-away-from-others-help}}

  \begin{itemize}
  \tightlist
  \item
    The coronavirus spreads primarily through droplets from your mouth
    and nose, especially when you cough or sneeze. The C.D.C., one of
    the organizations using that measure,
    \href{https://www.nytimes3xbfgragh.onion/2020/04/14/health/coronavirus-six-feet.html?action=click\&pgtype=Article\&state=default\&region=MAIN_CONTENT_3\&context=storylines_faq}{bases
    its recommendation of six feet} on the idea that most large droplets
    that people expel when they cough or sneeze will fall to the ground
    within six feet. But six feet has never been a magic number that
    guarantees complete protection. Sneezes, for instance, can launch
    droplets a lot farther than six feet,
    \href{https://jamanetwork.com/journals/jama/fullarticle/2763852}{according
    to a recent study}. It's a rule of thumb: You should be safest
    standing six feet apart outside, especially when it's windy. But
    keep a mask on at all times, even when you think you're far enough
    apart.
  \end{itemize}
\item ~
  \hypertarget{i-have-antibodies-am-i-now-immune}{%
  \paragraph{I have antibodies. Am I now
  immune?}\label{i-have-antibodies-am-i-now-immune}}

  \begin{itemize}
  \tightlist
  \item
    As of right
    now,\href{https://www.nytimes3xbfgragh.onion/2020/07/22/health/covid-antibodies-herd-immunity.html?action=click\&pgtype=Article\&state=default\&region=MAIN_CONTENT_3\&context=storylines_faq}{that
    seems likely, for at least several months.} There have been
    frightening accounts of people suffering what seems to be a second
    bout of Covid-19. But experts say these patients may have a
    drawn-out course of infection, with the virus taking a slow toll
    weeks to months after initial exposure. People infected with the
    coronavirus typically
    \href{https://www.nature.com/articles/s41586-020-2456-9}{produce}
    immune molecules called antibodies, which are
    \href{https://www.nytimes3xbfgragh.onion/2020/05/07/health/coronavirus-antibody-prevalence.html?action=click\&pgtype=Article\&state=default\&region=MAIN_CONTENT_3\&context=storylines_faq}{protective
    proteins made in response to an
    infection}\href{https://www.nytimes3xbfgragh.onion/2020/05/07/health/coronavirus-antibody-prevalence.html?action=click\&pgtype=Article\&state=default\&region=MAIN_CONTENT_3\&context=storylines_faq}{.
    These antibodies may} last in the body
    \href{https://www.nature.com/articles/s41591-020-0965-6}{only two to
    three months}, which may seem worrisome, but that's perfectly normal
    after an acute infection subsides, said Dr. Michael Mina, an
    immunologist at Harvard University. It may be possible to get the
    coronavirus again, but it's highly unlikely that it would be
    possible in a short window of time from initial infection or make
    people sicker the second time.
  \end{itemize}
\item ~
  \hypertarget{im-a-small-business-owner-can-i-get-relief}{%
  \paragraph{I'm a small-business owner. Can I get
  relief?}\label{im-a-small-business-owner-can-i-get-relief}}

  \begin{itemize}
  \tightlist
  \item
    The
    \href{https://www.nytimes3xbfgragh.onion/article/small-business-loans-stimulus-grants-freelancers-coronavirus.html?action=click\&pgtype=Article\&state=default\&region=MAIN_CONTENT_3\&context=storylines_faq}{stimulus
    bills enacted in March} offer help for the millions of American
    small businesses. Those eligible for aid are businesses and
    nonprofit organizations with fewer than 500 workers, including sole
    proprietorships, independent contractors and freelancers. Some
    larger companies in some industries are also eligible. The help
    being offered, which is being managed by the Small Business
    Administration, includes the Paycheck Protection Program and the
    Economic Injury Disaster Loan program. But lots of folks have
    \href{https://www.nytimes3xbfgragh.onion/interactive/2020/05/07/business/small-business-loans-coronavirus.html?action=click\&pgtype=Article\&state=default\&region=MAIN_CONTENT_3\&context=storylines_faq}{not
    yet seen payouts.} Even those who have received help are confused:
    The rules are draconian, and some are stuck sitting on
    \href{https://www.nytimes3xbfgragh.onion/2020/05/02/business/economy/loans-coronavirus-small-business.html?action=click\&pgtype=Article\&state=default\&region=MAIN_CONTENT_3\&context=storylines_faq}{money
    they don't know how to use.} Many small-business owners are getting
    less than they expected or
    \href{https://www.nytimes3xbfgragh.onion/2020/06/10/business/Small-business-loans-ppp.html?action=click\&pgtype=Article\&state=default\&region=MAIN_CONTENT_3\&context=storylines_faq}{not
    hearing anything at all.}
  \end{itemize}
\item ~
  \hypertarget{what-are-my-rights-if-i-am-worried-about-going-back-to-work}{%
  \paragraph{What are my rights if I am worried about going back to
  work?}\label{what-are-my-rights-if-i-am-worried-about-going-back-to-work}}

  \begin{itemize}
  \tightlist
  \item
    Employers have to provide
    \href{https://www.osha.gov/SLTC/covid-19/standards.html}{a safe
    workplace} with policies that protect everyone equally.
    \href{https://www.nytimes3xbfgragh.onion/article/coronavirus-money-unemployment.html?action=click\&pgtype=Article\&state=default\&region=MAIN_CONTENT_3\&context=storylines_faq}{And
    if one of your co-workers tests positive for the coronavirus, the
    C.D.C.} has said that
    \href{https://www.cdc.gov/coronavirus/2019-ncov/community/guidance-business-response.html}{employers
    should tell their employees} -\/- without giving you the sick
    employee's name -\/- that they may have been exposed to the virus.
  \end{itemize}
\item ~
  \hypertarget{what-is-school-going-to-look-like-in-september}{%
  \paragraph{What is school going to look like in
  September?}\label{what-is-school-going-to-look-like-in-september}}

  \begin{itemize}
  \tightlist
  \item
    It is unlikely that many schools will return to a normal schedule
    this fall, requiring the grind of
    \href{https://www.nytimes3xbfgragh.onion/2020/06/05/us/coronavirus-education-lost-learning.html?action=click\&pgtype=Article\&state=default\&region=MAIN_CONTENT_3\&context=storylines_faq}{online
    learning},
    \href{https://www.nytimes3xbfgragh.onion/2020/05/29/us/coronavirus-child-care-centers.html?action=click\&pgtype=Article\&state=default\&region=MAIN_CONTENT_3\&context=storylines_faq}{makeshift
    child care} and
    \href{https://www.nytimes3xbfgragh.onion/2020/06/03/business/economy/coronavirus-working-women.html?action=click\&pgtype=Article\&state=default\&region=MAIN_CONTENT_3\&context=storylines_faq}{stunted
    workdays} to continue. California's two largest public school
    districts --- Los Angeles and San Diego --- said on July 13, that
    \href{https://www.nytimes3xbfgragh.onion/2020/07/13/us/lausd-san-diego-school-reopening.html?action=click\&pgtype=Article\&state=default\&region=MAIN_CONTENT_3\&context=storylines_faq}{instruction
    will be remote-only in the fall}, citing concerns that surging
    coronavirus infections in their areas pose too dire a risk for
    students and teachers. Together, the two districts enroll some
    825,000 students. They are the largest in the country so far to
    abandon plans for even a partial physical return to classrooms when
    they reopen in August. For other districts, the solution won't be an
    all-or-nothing approach.
    \href{https://bioethics.jhu.edu/research-and-outreach/projects/eschool-initiative/school-policy-tracker/}{Many
    systems}, including the nation's largest, New York City, are
    devising
    \href{https://www.nytimes3xbfgragh.onion/2020/06/26/us/coronavirus-schools-reopen-fall.html?action=click\&pgtype=Article\&state=default\&region=MAIN_CONTENT_3\&context=storylines_faq}{hybrid
    plans} that involve spending some days in classrooms and other days
    online. There's no national policy on this yet, so check with your
    municipal school system regularly to see what is happening in your
    community.
  \end{itemize}
\end{itemize}

``A lot of people roll off at that point,'' said Pamela Herd, a
Georgetown University professor and an expert on the
``\href{https://www.russellsage.org/publications/administrative-burden}{administrative
burdens}'' that keep otherwise eligible people from getting access to
many public programs.

\hypertarget{how-much-money-might-i-get}{%
\subsection{How much money might I
get?}\label{how-much-money-might-i-get}}

People who have less get more, but there are limits and they depend on
your family size.

The \href{https://www.fns.usda.gov/snap/recipient/eligibility}{maximum
monthly allotment} for a one-person household is \$194. For a family of
four, the cap is \$646. Cost-of-living adjustments may change those
amounts in Alaska, Hawaii, Guam and the Virgin Islands.

\hypertarget{are-college-and-graduate-students-eligible}{%
\subsection{Are college and graduate students
eligible?}\label{are-college-and-graduate-students-eligible}}

Sometimes, yes. A 2018 Government Accountability Office
\href{https://www.gao.gov/assets/700/696254.pdf}{report} found that 57
percent of low-income students who seemed potentially eligible for SNAP
(and had at least one other additional factor that suggested they were
food insecure) did not report receiving SNAP benefits. That was about
1.8 million people.

Moreover, investigators found that state SNAP employees and some federal
officials admitted confusion about student eligibility rules.

SNAP rules generally keep students whose parents are supporting them (or
those on a meal plan) from getting benefits. Others who have little
income or assets should consult the Agriculture Department's
\href{https://www.fns.usda.gov/snap/students}{bare-bones guidance} and
inquire further with their state if they think they might qualify. The
\href{https://hope4college.com/about-the-hope-center/}{Hope Center for
College, Community and Justice} at Temple University has
\href{https://hope4college.com/wp-content/uploads/2019/04/Beyond-the-Food-Pantry-Student-Access-to-SNAP.pdf}{a
guide} for colleges and universities that want to help students.

\hypertarget{what-about-social-security-recipients}{%
\subsection{What about Social Security
recipients?}\label{what-about-social-security-recipients}}

It depends. If you're receiving Supplemental Security Income benefits,
you should definitely apply for SNAP. In many instances, someone from a
Social Security office may be able to help.

Some people receiving Social Security retirement benefits may be
eligible for SNAP, too, but as of 2015,
\href{https://www.fns.usda.gov/pressrelease/2015/020215}{fewer than
half} of eligible older Americans were receiving benefits. The
Department of Agriculture has a
\href{https://www.fns.usda.gov/snap/eligibility/elderly-disabled-special-rules}{separate
section} of its website laying out the different eligibility rules for
elderly and disabled people.

\hypertarget{who-can-help-me-sign-up}{%
\subsection{Who can help me sign up?}\label{who-can-help-me-sign-up}}

Carrie R. Welton, director of policy at the Hope Center, a research and
advocacy group, said your first stop should still be the state agency
that determines eligibility. Caseworkers can be both helpful and
empathetic: Ms. Welton recalled her own time on public assistance, when
the person on the other side of the desk started to cry when she
realized that Ms. Welton would need to stop attending college full time
if she hoped to maintain her benefits.

Other organizations may be able to help. Part of Ms. Welton's work
involves translating federal and state policy to help students who may
be eligible for SNAP and other benefits. College financial aid offices
may be able to assist students, too.

Help may also be available at your local food bank (several hundred
colleges and universities have food banks as well). You can find a food
bank near you using the ZIP code tool on
\href{https://www.feedingamerica.org/find-your-local-foodbank}{Feeding
America's website}.

``We're pursuing the initiative to feed the people in the lines but help
shorten them as well,'' said Mr. Rodriguez, the Community FoodBank of
New Jersey president. ``SNAP puts dollars in people's hands to shop the
way you and I do.''

Advertisement

\protect\hyperlink{after-bottom}{Continue reading the main story}

\hypertarget{site-index}{%
\subsection{Site Index}\label{site-index}}

\hypertarget{site-information-navigation}{%
\subsection{Site Information
Navigation}\label{site-information-navigation}}

\begin{itemize}
\tightlist
\item
  \href{https://help.nytimes3xbfgragh.onion/hc/en-us/articles/115014792127-Copyright-notice}{©~2020~The
  New York Times Company}
\end{itemize}

\begin{itemize}
\tightlist
\item
  \href{https://www.nytco.com/}{NYTCo}
\item
  \href{https://help.nytimes3xbfgragh.onion/hc/en-us/articles/115015385887-Contact-Us}{Contact
  Us}
\item
  \href{https://www.nytco.com/careers/}{Work with us}
\item
  \href{https://nytmediakit.com/}{Advertise}
\item
  \href{http://www.tbrandstudio.com/}{T Brand Studio}
\item
  \href{https://www.nytimes3xbfgragh.onion/privacy/cookie-policy\#how-do-i-manage-trackers}{Your
  Ad Choices}
\item
  \href{https://www.nytimes3xbfgragh.onion/privacy}{Privacy}
\item
  \href{https://help.nytimes3xbfgragh.onion/hc/en-us/articles/115014893428-Terms-of-service}{Terms
  of Service}
\item
  \href{https://help.nytimes3xbfgragh.onion/hc/en-us/articles/115014893968-Terms-of-sale}{Terms
  of Sale}
\item
  \href{https://spiderbites.nytimes3xbfgragh.onion}{Site Map}
\item
  \href{https://help.nytimes3xbfgragh.onion/hc/en-us}{Help}
\item
  \href{https://www.nytimes3xbfgragh.onion/subscription?campaignId=37WXW}{Subscriptions}
\end{itemize}
