Sections

SEARCH

\protect\hyperlink{site-content}{Skip to
content}\protect\hyperlink{site-index}{Skip to site index}

\href{https://www.nytimes3xbfgragh.onion/section/your-money}{Your Money}

\href{https://myaccount.nytimes3xbfgragh.onion/auth/login?response_type=cookie\&client_id=vi}{}

\href{https://www.nytimes3xbfgragh.onion/section/todayspaper}{Today's
Paper}

\href{/section/your-money}{Your Money}\textbar{}New Tools for Home
Buyers as the Pandemic Upends Real Estate

\url{https://nyti.ms/32rn9TS}

\begin{itemize}
\item
\item
\item
\item
\item
\end{itemize}

\href{https://www.nytimes3xbfgragh.onion/news-event/coronavirus?action=click\&pgtype=Article\&state=default\&region=TOP_BANNER\&context=storylines_menu}{The
Coronavirus Outbreak}

\begin{itemize}
\tightlist
\item
  live\href{https://www.nytimes3xbfgragh.onion/2020/08/01/world/coronavirus-covid-19.html?action=click\&pgtype=Article\&state=default\&region=TOP_BANNER\&context=storylines_menu}{Latest
  Updates}
\item
  \href{https://www.nytimes3xbfgragh.onion/interactive/2020/us/coronavirus-us-cases.html?action=click\&pgtype=Article\&state=default\&region=TOP_BANNER\&context=storylines_menu}{Maps
  and Cases}
\item
  \href{https://www.nytimes3xbfgragh.onion/interactive/2020/science/coronavirus-vaccine-tracker.html?action=click\&pgtype=Article\&state=default\&region=TOP_BANNER\&context=storylines_menu}{Vaccine
  Tracker}
\item
  \href{https://www.nytimes3xbfgragh.onion/interactive/2020/07/29/us/schools-reopening-coronavirus.html?action=click\&pgtype=Article\&state=default\&region=TOP_BANNER\&context=storylines_menu}{What
  School May Look Like}
\item
  \href{https://www.nytimes3xbfgragh.onion/live/2020/07/31/business/stock-market-today-coronavirus?action=click\&pgtype=Article\&state=default\&region=TOP_BANNER\&context=storylines_menu}{Economy}
\end{itemize}

Advertisement

\protect\hyperlink{after-top}{Continue reading the main story}

Supported by

\protect\hyperlink{after-sponsor}{Continue reading the main story}

Wealth Matters

\hypertarget{new-tools-for-home-buyers-as-the-pandemic-upends-real-estate}{%
\section{New Tools for Home Buyers as the Pandemic Upends Real
Estate}\label{new-tools-for-home-buyers-as-the-pandemic-upends-real-estate}}

As homeowners rethink their surroundings and rush to relocate, companies
are offering creative ways to help ease a fraught process.

\includegraphics{https://static01.graylady3jvrrxbe.onion/images/2020/07/17/business/17Wealth-01/merlin_174612078_80263407-18e1-4b2e-8281-3d389ec05ddc-articleLarge.jpg?quality=75\&auto=webp\&disable=upscale}

By \href{https://www.nytimes3xbfgragh.onion/by/paul-sullivan}{Paul
Sullivan}

\begin{itemize}
\item
  July 17, 2020
\item
  \begin{itemize}
  \item
  \item
  \item
  \item
  \item
  \end{itemize}
\end{itemize}

\href{https://tmagazine.blogs.nytimes3xbfgragh.onion/2012/03/27/a-bold-move/}{Dan
Feder}, a graphic designer, and his husband, Don Bacigalupi, a museum
director, have lived in the Hancock Park neighborhood of Los Angeles for
two years. But they started looking to move closer to their teenage
son's school when pandemic lockdowns showed they could avoid a commute
across town by working from home.

Yet having moved from Chicago two years ago, they still felt that they
had an outsider's view of Southern California real estate. They worried,
too, about making a mistake in an uncertain economy.

``With so many unknown variables in the world, we wanted as much
information as we could get before moving forward with such a large
financial decision,'' Mr. Feder said. ``As an outsider, it's difficult
to gauge that.''

Buying a home under any circumstances can be stressful. But with some
buyers looking to improve their lives during the pandemic and others
trying to flee congested cities for greater space in the suburbs, the
rush to move could cause unexpected problems. Companies are coming up
with creative ways to help ease the process.

The financial aspect alone is daunting in this economy. Down payments on
a home can be 25 percent to 30 percent of the purchase price, money that
becomes largely untouchable once it's handed over. But there are other
factors to consider. Carrying costs are one, but so is the nature of a
home as an asset: It may be worth millions of dollars on Zillow, but
selling it for that price can be difficult.

The pandemic has created interrelated dynamics that make the process
more complex. Some buyers want out of their city apartments, having
already been cooped up in them for months this winter. If they have
children, they may be acting quickly to get settled before the school
year starts.

Existing-home sales in the United States
\href{https://www.nar.realtor/newsroom/existing-home-sales-fall-9-7-in-may-while-nar-expects-strong-rebound-in-coming-months}{fell
26.6 percent in May} from a year earlier, to 3.9 million, according to
the National Association of Realtors, but the pandemic has been a
catalyst for many to move, with suburban areas drawing more interest,
according to
\href{https://www.nar.realtor/sites/default/files/documents/2020-market-recovery-survey-07-09-2020.pdf}{a
separate report} from the Realtor group. For example, the median sale
price of a home in New Canaan, Conn., which is within commuting distance
of Manhattan,
\href{https://newcanaanite.com/new-canaan-home-sales-median-price-up-29-in-may-3046583}{rose
29 percent in May} from the year before.

The national average interest rate on 30-year fixed mortgages
\href{https://www.nytimes3xbfgragh.onion/2020/07/16/business/mortgage-rates-below-3-percent.html}{fell
below 3 percent} for the first time on record this week, which could
spur more people to buy a home.

There are deals in cities for those with cash and a strong stomach. Some
of those purchases, costing millions of dollars, have been made through
online scrolling and Zoom walk-throughs. But when purchases are driven
by video, what catches the eye sells, and flaws can be missed.

``While various video technologies are changing how we travel and what
we need to see live, I would not advise nor endorse buying any real
estate sight unseen,'' said Bill Nimmo, who heads the real estate asset
management group at Wells Fargo Private Bank. ``It can make sense to put
a property under contract before visually seeing it, but not to close on
it without seeing it live.''

\hypertarget{latest-updates-global-coronavirus-outbreak}{%
\section{\texorpdfstring{\href{https://www.nytimes3xbfgragh.onion/2020/08/01/world/coronavirus-covid-19.html?action=click\&pgtype=Article\&state=default\&region=MAIN_CONTENT_1\&context=storylines_live_updates}{Latest
Updates: Global Coronavirus
Outbreak}}{Latest Updates: Global Coronavirus Outbreak}}\label{latest-updates-global-coronavirus-outbreak}}

Updated 2020-08-02T06:58:18.835Z

\begin{itemize}
\tightlist
\item
  \href{https://www.nytimes3xbfgragh.onion/2020/08/01/world/coronavirus-covid-19.html?action=click\&pgtype=Article\&state=default\&region=MAIN_CONTENT_1\&context=storylines_live_updates\#link-34047410}{The
  U.S. reels as July cases more than double the total of any other
  month.}
\item
  \href{https://www.nytimes3xbfgragh.onion/2020/08/01/world/coronavirus-covid-19.html?action=click\&pgtype=Article\&state=default\&region=MAIN_CONTENT_1\&context=storylines_live_updates\#link-780ec966}{Top
  U.S. officials work to break an impasse over the federal jobless
  benefit.}
\item
  \href{https://www.nytimes3xbfgragh.onion/2020/08/01/world/coronavirus-covid-19.html?action=click\&pgtype=Article\&state=default\&region=MAIN_CONTENT_1\&context=storylines_live_updates\#link-2bc8948}{Its
  outbreak untamed, Melbourne goes into even greater lockdown.}
\end{itemize}

\href{https://www.nytimes3xbfgragh.onion/2020/08/01/world/coronavirus-covid-19.html?action=click\&pgtype=Article\&state=default\&region=MAIN_CONTENT_1\&context=storylines_live_updates}{See
more updates}

More live coverage:
\href{https://www.nytimes3xbfgragh.onion/live/2020/07/31/business/stock-market-today-coronavirus?action=click\&pgtype=Article\&state=default\&region=MAIN_CONTENT_1\&context=storylines_live_updates}{Markets}

He pointed to the ineffable feeling of home, from the light and flow to
the neighbors. ``These are subjective things that most buyers should see
and experience to make a major investment,'' Mr. Nimmo said.

\includegraphics{https://static01.graylady3jvrrxbe.onion/images/2020/07/17/business/17Wealth-sub/17Wealth-sub-articleLarge.jpg?quality=75\&auto=webp\&disable=upscale}

The risk of buying anything sight unseen is unavoidable with a project
under development. During the stay-at-home order in New York, 10 buyers
purchased luxury condos for \$2 million to \$6 million at an unfinished
development, \href{https://130william.com/}{130 William}, in the
financial district.

Buyers usually get to peruse a high-end show room, to touch finishes and
see how kitchens and bathrooms will be laid out. But in this case, they
relied on online material, created for each apartment, down to the view
from the windows.

``The shutdown worked to our advantage,'' said Scott J. Avram, a senior
vice president at Lightstone, the developer of 130 William. ``It turned
out to be a good bet to push more advertising. Buyers were almost a
captive audience.''

Some were persuaded to buy because they could put down a 10 percent
deposit, lock in a low-interest mortgage but not have to close until the
fall, when the building will be finished.

How people are doing their due diligence in this stressful time is as
distinct as this moment in history.

Some are trying to do traditional due diligence through new services.
This was the case with Mr. Feder. Last year, he fell in love with a
neighborhood on the west side of Los Angeles only to watch flames engulf
it in last season's wildfires.

``We watched half the neighborhood burn up,'' he said. ``We have a large
art collection, and we didn't want to put that at risk.''

So in the pandemic, he turned to a new product offered by PURE Insurance
called Home Spotlight. The report rates a home's risk for natural
disasters like flooding, hurricanes, earthquakes and wildfires, and it
is meant to show buyers whether PURE would insure the home. Not getting
home insurance through a commercial carrier like PURE, AIG or Chubb
substantially increases the annual premiums.

But the report also details any insurance claims, construction permits
and work done on the home that could point to other problems that are
not apparent at first glance --- and certainly not through a video tour.

\href{https://www.nytimes3xbfgragh.onion/news-event/coronavirus?action=click\&pgtype=Article\&state=default\&region=MAIN_CONTENT_3\&context=storylines_faq}{}

\hypertarget{the-coronavirus-outbreak-}{%
\subsubsection{The Coronavirus Outbreak
›}\label{the-coronavirus-outbreak-}}

\hypertarget{frequently-asked-questions}{%
\paragraph{Frequently Asked
Questions}\label{frequently-asked-questions}}

Updated July 27, 2020

\begin{itemize}
\item ~
  \hypertarget{should-i-refinance-my-mortgage}{%
  \paragraph{Should I refinance my
  mortgage?}\label{should-i-refinance-my-mortgage}}

  \begin{itemize}
  \tightlist
  \item
    \href{https://www.nytimes3xbfgragh.onion/article/coronavirus-money-unemployment.html?action=click\&pgtype=Article\&state=default\&region=MAIN_CONTENT_3\&context=storylines_faq}{It
    could be a good idea,} because mortgage rates have
    \href{https://www.nytimes3xbfgragh.onion/2020/07/16/business/mortgage-rates-below-3-percent.html?action=click\&pgtype=Article\&state=default\&region=MAIN_CONTENT_3\&context=storylines_faq}{never
    been lower.} Refinancing requests have pushed mortgage applications
    to some of the highest levels since 2008, so be prepared to get in
    line. But defaults are also up, so if you're thinking about buying a
    home, be aware that some lenders have tightened their standards.
  \end{itemize}
\item ~
  \hypertarget{what-is-school-going-to-look-like-in-september}{%
  \paragraph{What is school going to look like in
  September?}\label{what-is-school-going-to-look-like-in-september}}

  \begin{itemize}
  \tightlist
  \item
    It is unlikely that many schools will return to a normal schedule
    this fall, requiring the grind of
    \href{https://www.nytimes3xbfgragh.onion/2020/06/05/us/coronavirus-education-lost-learning.html?action=click\&pgtype=Article\&state=default\&region=MAIN_CONTENT_3\&context=storylines_faq}{online
    learning},
    \href{https://www.nytimes3xbfgragh.onion/2020/05/29/us/coronavirus-child-care-centers.html?action=click\&pgtype=Article\&state=default\&region=MAIN_CONTENT_3\&context=storylines_faq}{makeshift
    child care} and
    \href{https://www.nytimes3xbfgragh.onion/2020/06/03/business/economy/coronavirus-working-women.html?action=click\&pgtype=Article\&state=default\&region=MAIN_CONTENT_3\&context=storylines_faq}{stunted
    workdays} to continue. California's two largest public school
    districts --- Los Angeles and San Diego --- said on July 13, that
    \href{https://www.nytimes3xbfgragh.onion/2020/07/13/us/lausd-san-diego-school-reopening.html?action=click\&pgtype=Article\&state=default\&region=MAIN_CONTENT_3\&context=storylines_faq}{instruction
    will be remote-only in the fall}, citing concerns that surging
    coronavirus infections in their areas pose too dire a risk for
    students and teachers. Together, the two districts enroll some
    825,000 students. They are the largest in the country so far to
    abandon plans for even a partial physical return to classrooms when
    they reopen in August. For other districts, the solution won't be an
    all-or-nothing approach.
    \href{https://bioethics.jhu.edu/research-and-outreach/projects/eschool-initiative/school-policy-tracker/}{Many
    systems}, including the nation's largest, New York City, are
    devising
    \href{https://www.nytimes3xbfgragh.onion/2020/06/26/us/coronavirus-schools-reopen-fall.html?action=click\&pgtype=Article\&state=default\&region=MAIN_CONTENT_3\&context=storylines_faq}{hybrid
    plans} that involve spending some days in classrooms and other days
    online. There's no national policy on this yet, so check with your
    municipal school system regularly to see what is happening in your
    community.
  \end{itemize}
\item ~
  \hypertarget{is-the-coronavirus-airborne}{%
  \paragraph{Is the coronavirus
  airborne?}\label{is-the-coronavirus-airborne}}

  \begin{itemize}
  \tightlist
  \item
    The coronavirus
    \href{https://www.nytimes3xbfgragh.onion/2020/07/04/health/239-experts-with-one-big-claim-the-coronavirus-is-airborne.html?action=click\&pgtype=Article\&state=default\&region=MAIN_CONTENT_3\&context=storylines_faq}{can
    stay aloft for hours in tiny droplets in stagnant air}, infecting
    people as they inhale, mounting scientific evidence suggests. This
    risk is highest in crowded indoor spaces with poor ventilation, and
    may help explain super-spreading events reported in meatpacking
    plants, churches and restaurants.
    \href{https://www.nytimes3xbfgragh.onion/2020/07/06/health/coronavirus-airborne-aerosols.html?action=click\&pgtype=Article\&state=default\&region=MAIN_CONTENT_3\&context=storylines_faq}{It's
    unclear how often the virus is spread} via these tiny droplets, or
    aerosols, compared with larger droplets that are expelled when a
    sick person coughs or sneezes, or transmitted through contact with
    contaminated surfaces, said Linsey Marr, an aerosol expert at
    Virginia Tech. Aerosols are released even when a person without
    symptoms exhales, talks or sings, according to Dr. Marr and more
    than 200 other experts, who
    \href{https://academic.oup.com/cid/article/doi/10.1093/cid/ciaa939/5867798}{have
    outlined the evidence in an open letter to the World Health
    Organization}.
  \end{itemize}
\item ~
  \hypertarget{what-are-the-symptoms-of-coronavirus}{%
  \paragraph{What are the symptoms of
  coronavirus?}\label{what-are-the-symptoms-of-coronavirus}}

  \begin{itemize}
  \tightlist
  \item
    Common symptoms
    \href{https://www.nytimes3xbfgragh.onion/article/symptoms-coronavirus.html?action=click\&pgtype=Article\&state=default\&region=MAIN_CONTENT_3\&context=storylines_faq}{include
    fever, a dry cough, fatigue and difficulty breathing or shortness of
    breath.} Some of these symptoms overlap with those of the flu,
    making detection difficult, but runny noses and stuffy sinuses are
    less common.
    \href{https://www.nytimes3xbfgragh.onion/2020/04/27/health/coronavirus-symptoms-cdc.html?action=click\&pgtype=Article\&state=default\&region=MAIN_CONTENT_3\&context=storylines_faq}{The
    C.D.C. has also} added chills, muscle pain, sore throat, headache
    and a new loss of the sense of taste or smell as symptoms to look
    out for. Most people fall ill five to seven days after exposure, but
    symptoms may appear in as few as two days or as many as 14 days.
  \end{itemize}
\item ~
  \hypertarget{does-asymptomatic-transmission-of-covid-19-happen}{%
  \paragraph{Does asymptomatic transmission of Covid-19
  happen?}\label{does-asymptomatic-transmission-of-covid-19-happen}}

  \begin{itemize}
  \tightlist
  \item
    So far, the evidence seems to show it does. A widely cited
    \href{https://www.nature.com/articles/s41591-020-0869-5}{paper}
    published in April suggests that people are most infectious about
    two days before the onset of coronavirus symptoms and estimated that
    44 percent of new infections were a result of transmission from
    people who were not yet showing symptoms. Recently, a top expert at
    the World Health Organization stated that transmission of the
    coronavirus by people who did not have symptoms was ``very rare,''
    \href{https://www.nytimes3xbfgragh.onion/2020/06/09/world/coronavirus-updates.html?action=click\&pgtype=Article\&state=default\&region=MAIN_CONTENT_3\&context=storylines_faq\#link-1f302e21}{but
    she later walked back that statement.}
  \end{itemize}
\end{itemize}

``If the buyers haven't been given full disclosure, which is the law in
California, then it causes other questions to come up,'' said Steven
Brown, chief executive of Hoffman Brown, an insurance brokerage firm.
``If you didn't tell me about the toilet that overflowed and cost
\$100,000 to fix, what else didn't you tell us about?''

The report keeps the seller honest; it also slows down a buyer who may
be acting quickly without full information.

A different type of fear --- fear of missing out --- is driving buyers
who see opportunities as city dwellers flee to the suburbs. Christopher
Rim, founder and chief executive of Command Education, an education
consultant in New York, said he had started getting unsolicited offers
for the TriBeCa loft that he bought less than a year ago. The offers
were for more than the \$3 million he paid.

``I figured if I could score a deal on another apartment, it would be a
good trade,'' he said.

He began looking for a bigger condo in a less-desirable neighborhood
where prices were cheaper, figuring he would be working from home for
the foreseeable future. But when his lowball offer of \$3.5 million
offer on a condo listed for \$4.8 million was accepted, Mr. Rim paused.
He worried about getting swept up in the moment.

Now his plan is to be more strategic and wait until the fall, when he is
betting people who have left the city will discount their apartments
even further. ``There's no way school is happening this fall,'' he said.
``They're going to want to offload some inventory.''

In addition to viewing real estate, closing on a property has become
more complicated and time consuming when social distancing is required.
No longer are a dozen people crammed into a law office conference room
to watch a buyer sign a stack of legal documents. But remote closings
actually require greater vigilance and time to ensure mistakes are not
made.

``When you're face to face with a client, you're able to go over all the
numbers with them and make sure what they're signing at the closing is
what they negotiated,'' said Sal Strazzullo, a real estate and estate
planning lawyer at Strazzullo Law. ``Being on the phone with someone and
having them sign the documents when you read through them --- you have
to step up your game.''

He said closings that normally took an hour and a half could stretch on
for 48 hours or more. Recently, he had to amend a closing when money
wired to buy a condo arrived four days late.

Mr. Feder has not found another house in Los Angeles yet, but he and his
husband have bought a weekend home in Palm Springs. They used their
insurer's home spotlight to assess the renovations that had been done to
it.

``It stood out for its cleanliness, but we saw permits that were pulled
for work on the home,'' he said.

In this case, the renovations were purely for aesthetic reasons, but Mr.
Feder felt comfort having the additional background. ``This is a very
uncertain age, and there's a part of you that says, `Are we crazy for
making a real estate purchase now?''' he said. ``But if you can be
confident and move forward, that's better.''

Advertisement

\protect\hyperlink{after-bottom}{Continue reading the main story}

\hypertarget{site-index}{%
\subsection{Site Index}\label{site-index}}

\hypertarget{site-information-navigation}{%
\subsection{Site Information
Navigation}\label{site-information-navigation}}

\begin{itemize}
\tightlist
\item
  \href{https://help.nytimes3xbfgragh.onion/hc/en-us/articles/115014792127-Copyright-notice}{©~2020~The
  New York Times Company}
\end{itemize}

\begin{itemize}
\tightlist
\item
  \href{https://www.nytco.com/}{NYTCo}
\item
  \href{https://help.nytimes3xbfgragh.onion/hc/en-us/articles/115015385887-Contact-Us}{Contact
  Us}
\item
  \href{https://www.nytco.com/careers/}{Work with us}
\item
  \href{https://nytmediakit.com/}{Advertise}
\item
  \href{http://www.tbrandstudio.com/}{T Brand Studio}
\item
  \href{https://www.nytimes3xbfgragh.onion/privacy/cookie-policy\#how-do-i-manage-trackers}{Your
  Ad Choices}
\item
  \href{https://www.nytimes3xbfgragh.onion/privacy}{Privacy}
\item
  \href{https://help.nytimes3xbfgragh.onion/hc/en-us/articles/115014893428-Terms-of-service}{Terms
  of Service}
\item
  \href{https://help.nytimes3xbfgragh.onion/hc/en-us/articles/115014893968-Terms-of-sale}{Terms
  of Sale}
\item
  \href{https://spiderbites.nytimes3xbfgragh.onion}{Site Map}
\item
  \href{https://help.nytimes3xbfgragh.onion/hc/en-us}{Help}
\item
  \href{https://www.nytimes3xbfgragh.onion/subscription?campaignId=37WXW}{Subscriptions}
\end{itemize}
