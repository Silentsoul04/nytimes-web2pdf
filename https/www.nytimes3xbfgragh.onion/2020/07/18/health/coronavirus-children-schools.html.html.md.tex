Sections

SEARCH

\protect\hyperlink{site-content}{Skip to
content}\protect\hyperlink{site-index}{Skip to site index}

\href{https://www.nytimes3xbfgragh.onion/section/health}{Health}

\href{https://myaccount.nytimes3xbfgragh.onion/auth/login?response_type=cookie\&client_id=vi}{}

\href{https://www.nytimes3xbfgragh.onion/section/todayspaper}{Today's
Paper}

\href{/section/health}{Health}\textbar{}Older Children Spread the
Coronavirus Just as Much as Adults, Large Study Finds

\url{https://nyti.ms/3jpd7cb}

\begin{itemize}
\item
\item
\item
\item
\item
\item
\end{itemize}

\hypertarget{schools-reopening}{%
\subsubsection{\texorpdfstring{\href{https://www.nytimes3xbfgragh.onion/spotlight/schools-reopening?name=styln-coronavirus-schools-reopening\&region=TOP_BANNER\&variant=undefined\&block=storyline_menu_recirc\&action=click\&pgtype=Article\&impression_id=93cdfbf0-e3b4-11ea-96b6-7d25c67b750c}{Schools
Reopening}}{Schools Reopening}}\label{schools-reopening}}

\begin{itemize}
\tightlist
\item
  \href{https://www.nytimes3xbfgragh.onion/2020/08/19/us/colleges-closing-covid.html?name=styln-coronavirus-schools-reopening\&region=TOP_BANNER\&variant=undefined\&block=storyline_menu_recirc\&action=click\&pgtype=Article\&impression_id=93ce2300-e3b4-11ea-96b6-7d25c67b750c}{Colleges
  Closing}
\item
  \href{https://www.nytimes3xbfgragh.onion/2020/08/20/us/schools-reopening-nurses-covid.html?name=styln-coronavirus-schools-reopening\&region=TOP_BANNER\&variant=undefined\&block=storyline_menu_recirc\&action=click\&pgtype=Article\&impression_id=93ce2301-e3b4-11ea-96b6-7d25c67b750c}{Missing
  School Nurses}
\item
  \href{https://www.nytimes3xbfgragh.onion/2020/08/18/parenting/homeschool-families.html?name=styln-coronavirus-schools-reopening\&region=TOP_BANNER\&variant=undefined\&block=storyline_menu_recirc\&action=click\&pgtype=Article\&impression_id=93ce2302-e3b4-11ea-96b6-7d25c67b750c}{Home-Schooling
  Families}
\item
  \href{https://www.nytimes3xbfgragh.onion/2020/08/05/parenting/parents-distance-learning.html?name=styln-coronavirus-schools-reopening\&region=TOP_BANNER\&variant=undefined\&block=storyline_menu_recirc\&action=click\&pgtype=Article\&impression_id=93ce2303-e3b4-11ea-96b6-7d25c67b750c}{Prepare
  for Distance Learning}
\end{itemize}

Advertisement

\protect\hyperlink{after-top}{Continue reading the main story}

Supported by

\protect\hyperlink{after-sponsor}{Continue reading the main story}

\hypertarget{older-children-spread-the-coronavirus-just-as-much-as-adults-large-study-finds}{%
\section{Older Children Spread the Coronavirus Just as Much as Adults,
Large Study
Finds}\label{older-children-spread-the-coronavirus-just-as-much-as-adults-large-study-finds}}

The study of nearly 65,000 people in South Korea suggests that school
reopenings will trigger more outbreaks.

\includegraphics{https://static01.graylady3jvrrxbe.onion/images/2020/07/18/science/18virus--kids/merlin_174135054_aed03bad-4e4b-4a8f-8ee7-b89113da5478-articleLarge.jpg?quality=75\&auto=webp\&disable=upscale}

By
\href{https://www.nytimes3xbfgragh.onion/by/apoorva-mandavilli}{Apoorva
Mandavilli}

\begin{itemize}
\item
  Published July 18, 2020Updated July 30, 2020
\item
  \begin{itemize}
  \item
  \item
  \item
  \item
  \item
  \item
  \end{itemize}
\end{itemize}

In the heated debate over reopening schools, one burning question has
been whether and
\href{https://www.nytimes3xbfgragh.onion/2020/07/30/health/coronavirus-children.html}{how
efficiently children can spread the virus} to others.

A \href{https://wwwnc.cdc.gov/eid/article/26/10/20-1315_article}{large
new study} from South Korea offers an answer: Children younger than 10
transmit to others much less often than adults do, but the risk is not
zero. And those between the ages of 10 and 19 can spread the virus at
least as well as adults do.

The findings suggest that as schools reopen, communities will see
clusters of infection take root that include children of all ages,
several experts cautioned.

``I fear that there has been this sense that kids just won't get
infected or don't get infected in the same way as adults and that,
therefore, they're almost like a bubbled population,'' said Michael
Osterholm, an infectious diseases expert at the University of Minnesota.

``There will be transmission,'' Dr. Osterholm said. ``What we have to do
is accept that now and include that in our plans.''

Several studies from Europe and Asia have suggested that young children
are less likely to get infected and to spread the virus. But most of
those studies were small and flawed, said Dr. Ashish Jha, director of
the Harvard Global Health Institute.

The new study ``is very carefully done, it's systematic and looks at a
very large population,'' Dr. Jha said. ``It's one of the best studies
we've had to date on this issue.''

Other experts also praised the scale and rigor of the analysis. South
Korean researchers identified 5,706 people who were the first to report
Covid-19 symptoms in their households between Jan. 20 and March 27, when
schools were closed, and then traced the 59,073 contacts of these
``index cases.'' They tested all of the household contacts of each
patient, regardless of symptoms, but only tested symptomatic contacts
outside the household.

The first person in a household to develop symptoms is not necessarily
the first to have been infected, and the researchers acknowledged this
limitation. Children are also less likely than adults to show symptoms,
so the study may have underestimated the number of children who set off
the chain of transmission within their households.

Still, experts said the approach was reasonable. ``It is also from a
place with great contact tracing, done at the point interventions were
being put in place,'' said Bill Hanage, an epidemiologist at the Harvard
T.H. Chan School of Public Health.

Children under 10 were roughly half as likely as adults to spread the
virus to others, consistent with other studies. That may be because
children generally exhale less air --- and therefore less virus-laden
air --- or because they exhale that air closer to the ground, making it
less likely that adults would breathe it in.

Even so, the number of new infections seeded by children may rise when
schools reopen, the study authors cautioned. ``Young children may show
higher attack rates when the school closure ends, contributing to
community transmission of Covid-19,'' they wrote. Other studies have
also suggested that the large number of contacts for schoolchildren, who
interact with dozens of others for a good part of the day, may
\href{https://www.nytimes3xbfgragh.onion/2020/05/05/health/coronavirus-children-transmission-school.html}{cancel
out their smaller risk} of infecting others.

The researchers traced the contacts only of children who felt ill, so
it's still unclear how efficiently asymptomatic children spread the
virus, said Caitlin Rivers, an epidemiologist at the Johns Hopkins
Bloomberg School of Public Health.

``I think it was always going to be the case that symptomatic children
are infectious,'' she said. ``The questions about the role of children
are more around whether children who don't have symptoms are
infectious.''

Dr. Rivers was a member of a scientific panel that on Wednesday
\href{https://www.nytimes3xbfgragh.onion/2020/07/15/health/coronavirus-schools-reopening.html}{recommended
reopening schools} wherever possible for disabled children and for those
in elementary schools, because those groups have the most trouble
learning online. She said the new study does not alter that
recommendation.

\textbf{\emph{{[}}\href{http://on.fb.me/1paTQ1h}{\emph{Like the Science
Times page on Facebook.}}} ****** \emph{\textbar{} Sign up for the}
\textbf{\href{http://nyti.ms/1MbHaRU}{\emph{Science Times
newsletter.}}\emph{{]}}}

The study is more worrisome for children in middle and high school. This
group was even more likely to infect others than adults were, the study
found. But some experts said that finding may be a fluke or may stem
from the children's behaviors.

These older children are frequently as big as adults, and yet may have
some of the same unhygienic habits as young children do. They may also
have been more likely than the younger children to socialize with their
peers within the high-rise complexes in South Korea.

\href{https://www.nytimes3xbfgragh.onion/spotlight/schools-reopening?action=click\&pgtype=Article\&state=default\&region=MAIN_CONTENT_3\&context=storylines_keepup}{}

\hypertarget{schools-reopening-}{%
\subsubsection{Schools Reopening ›}\label{schools-reopening-}}

\hypertarget{back-to-school}{%
\paragraph{Back to School}\label{back-to-school}}

Updated Aug. 21, 2020

The latest on how schools are reopening amid the pandemic.

\begin{itemize}
\item
  \begin{itemize}
  \tightlist
  \item
    With courses moved online but football games still on,
    \href{https://www.nytimes3xbfgragh.onion/2020/08/18/sports/ncaafootball/unc-football-acc-online-classes.html?action=click\&pgtype=Article\&state=default\&region=MAIN_CONTENT_3\&context=storylines_keepup}{the
    University of North Carolina is a test case} for college sports.
  \item
    The federal government has announced measures
    \href{https://www.nytimes3xbfgragh.onion/2020/08/20/health/coronavirus-flu-vaccine.html?action=click\&pgtype=Article\&state=default\&region=MAIN_CONTENT_3\&context=storylines_keepup}{intended
    to boost childhood vaccination rates}, which are sagging during the
    pandemic.
  \item
    Much more will be
    \href{https://www.nytimes3xbfgragh.onion/2020/08/20/us/schools-reopening-nurses-covid.html?action=click\&pgtype=Article\&state=default\&region=MAIN_CONTENT_3\&context=storylines_keepup}{expected
    of America's school nurses} this year, but the reality is: Many
    schools don't have one.
  \item
    We want to hear from teachers making difficult choices. How are you
    thinking about the start of the school year?
    \href{https://www.nytimes3xbfgragh.onion/2020/08/19/us/teachers-school-reopenings.html?action=click\&pgtype=Article\&state=default\&region=MAIN_CONTENT_3\&context=storylines_keepup}{Tell
    us here}.
  \end{itemize}
\end{itemize}

``We can speculate all day about this, but we just don't know,'' Dr.
Osterholm said. ``The bottom line message is: There's going to be
transmission.''

He and other experts said schools will need to prepare for infections to
pop up. Apart from implementing physical distancing, hand hygiene and
masks, schools should also decide when and how to test students and
staff --- including, for example, bus drivers --- when and how long to
require people to quarantine, and when to decide to close and reopen
schools.

But they face a monumental challenge because the evidence on
transmission within schools has been far from conclusive so far, experts
said. Some countries like Denmark and Finland have successfully reopened
schools, but others, like China, Israel and South Korea, have had to
close them down again.

``People, depending on their ideology on school opening, are choosing
which evidence to present --- and that needs to be avoided,'' said
Jeffrey Shaman, an epidemiologist at Columbia University's Mailman
School of Public Health in New York.

Although the new study does not offer definitive answers, he said, it
does indicate that schools can increase virus levels within a community.

``So long as children are not just a complete dead end --- incapable of
passing the virus on, which does not seem to be the case --- putting
them together in schools, having them mix with teachers and other
students will provide additional opportunities for the virus to move
from person to person,'' he said.

At the same time, Dr. Shaman said, it's important for children not to
miss out on critical years in education and socialization, and school
districts have the unenviable task of choosing between those options:
``It's hard trying to find the right balance.''

Advertisement

\protect\hyperlink{after-bottom}{Continue reading the main story}

\hypertarget{site-index}{%
\subsection{Site Index}\label{site-index}}

\hypertarget{site-information-navigation}{%
\subsection{Site Information
Navigation}\label{site-information-navigation}}

\begin{itemize}
\tightlist
\item
  \href{https://help.nytimes3xbfgragh.onion/hc/en-us/articles/115014792127-Copyright-notice}{©~2020~The
  New York Times Company}
\end{itemize}

\begin{itemize}
\tightlist
\item
  \href{https://www.nytco.com/}{NYTCo}
\item
  \href{https://help.nytimes3xbfgragh.onion/hc/en-us/articles/115015385887-Contact-Us}{Contact
  Us}
\item
  \href{https://www.nytco.com/careers/}{Work with us}
\item
  \href{https://nytmediakit.com/}{Advertise}
\item
  \href{http://www.tbrandstudio.com/}{T Brand Studio}
\item
  \href{https://www.nytimes3xbfgragh.onion/privacy/cookie-policy\#how-do-i-manage-trackers}{Your
  Ad Choices}
\item
  \href{https://www.nytimes3xbfgragh.onion/privacy}{Privacy}
\item
  \href{https://help.nytimes3xbfgragh.onion/hc/en-us/articles/115014893428-Terms-of-service}{Terms
  of Service}
\item
  \href{https://help.nytimes3xbfgragh.onion/hc/en-us/articles/115014893968-Terms-of-sale}{Terms
  of Sale}
\item
  \href{https://spiderbites.nytimes3xbfgragh.onion}{Site Map}
\item
  \href{https://help.nytimes3xbfgragh.onion/hc/en-us}{Help}
\item
  \href{https://www.nytimes3xbfgragh.onion/subscription?campaignId=37WXW}{Subscriptions}
\end{itemize}
