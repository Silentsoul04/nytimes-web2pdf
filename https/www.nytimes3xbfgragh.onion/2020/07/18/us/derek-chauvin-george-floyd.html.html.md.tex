Sections

SEARCH

\protect\hyperlink{site-content}{Skip to
content}\protect\hyperlink{site-index}{Skip to site index}

\href{/section/us}{U.S.}\textbar{}Officer Who Pressed His Knee on George
Floyd's Neck Drew Scrutiny Long Before

\url{https://nyti.ms/2DR0TZf}

\begin{itemize}
\item
\item
\item
\item
\item
\end{itemize}

\href{https://www.nytimes3xbfgragh.onion/news-event/george-floyd-protests-minneapolis-new-york-los-angeles?action=click\&pgtype=Article\&state=default\&region=TOP_BANNER\&context=storylines_menu}{Race
and America}

\begin{itemize}
\tightlist
\item
  \href{https://www.nytimes3xbfgragh.onion/2020/07/26/us/protests-portland-seattle-trump.html?action=click\&pgtype=Article\&state=default\&region=TOP_BANNER\&context=storylines_menu}{Protesters
  Return to Other Cities}
\item
  \href{https://www.nytimes3xbfgragh.onion/2020/07/24/us/portland-oregon-protests-white-race.html?action=click\&pgtype=Article\&state=default\&region=TOP_BANNER\&context=storylines_menu}{Portland
  at the Center}
\item
  \href{https://www.nytimes3xbfgragh.onion/2020/07/23/podcasts/the-daily/portland-protests.html?action=click\&pgtype=Article\&state=default\&region=TOP_BANNER\&context=storylines_menu}{Podcast:
  Showdown in Portland}
\item
  \href{https://www.nytimes3xbfgragh.onion/interactive/2020/07/16/us/black-lives-matter-protests-louisville-breonna-taylor.html?action=click\&pgtype=Article\&state=default\&region=TOP_BANNER\&context=storylines_menu}{45
  Days in Louisville}
\end{itemize}

\includegraphics{https://static01.graylady3jvrrxbe.onion/images/2020/07/17/us/00UNREST-CHAUVIN-station3/merlin_174688452_c3b1a004-2130-49ea-8ead-64b6fc028b86-articleLarge.jpg?quality=75\&auto=webp\&disable=upscale}

\hypertarget{officer-who-pressed-his-knee-on-george-floyds-neck-drew-scrutiny-long-before}{%
\section{Officer Who Pressed His Knee on George Floyd's Neck Drew
Scrutiny Long
Before}\label{officer-who-pressed-his-knee-on-george-floyds-neck-drew-scrutiny-long-before}}

In more than 19 years on the Minneapolis police force, Mr. Chauvin had a
reputation as a rigid workaholic with few friends. He sometimes made
other officers uncomfortable.

The Minneapolis Police Department's Third Precinct station remained
blocked off and empty on Friday. It was set on fire by protesters after
George Floyd was killed in police custody.Credit...Caroline Yang for The
New York Times

Supported by

\protect\hyperlink{after-sponsor}{Continue reading the main story}

By \href{https://www.nytimes3xbfgragh.onion/by/kim-barker}{Kim Barker}
and \href{https://www.nytimes3xbfgragh.onion/by/serge-f-kovaleski}{Serge
F. Kovaleski}

\begin{itemize}
\item
  Published July 18, 2020Updated July 22, 2020
\item
  \begin{itemize}
  \item
  \item
  \item
  \item
  \item
  \end{itemize}
\end{itemize}

MINNEAPOLIS --- The four teenagers drove around playing a game of Nerf
Gun Assassin on a May evening before graduation in 2013. One of them
randomly fired an orange dart out the window.

It was a stupid teenager move. What happened next was deadly serious:
Two
\href{https://www.nytimes3xbfgragh.onion/2020/07/29/us/george-floyd-memorial.html}{Minneapolis}
police officers pulled up, pointed their guns at the teenagers and
shouted orders laced with expletives, two of them later recalled.

Kristofer Bergh, then 17, said he kept telling himself not to move
suddenly or give the police any reason to shoot him. The youth who had
fired the dart was steered into their cruiser for what seemed like an
hour, and the officers seized everyone's Nerf guns. One officer made a
lasting impression; in fact, Mr. Bergh and another passenger said they
would never forget him, nor what he said as he gave them back their
guns.

``Most of you will be 18 by the end of the year,'' the officer said,
before letting them go. ``That means you'll be old enough for `big boy
jail.'''

It was
\href{https://www.nytimes3xbfgragh.onion/2020/07/22/us/derek-chauvin-tax-fraud.html}{Derek
Chauvin}, the white police officer who, seven years later, would become
known around the world for putting his knee on the neck of a Black man
named
\href{https://www.nytimes3xbfgragh.onion/2020/07/22/us/derek-chauvin-tax-fraud.html}{George
Floyd} during an arrest and holding it there for
\href{https://www.nytimes3xbfgragh.onion/2020/05/31/us/george-floyd-investigation.html}{more
than eight minutes}, until he no longer had a pulse.

Mr. Floyd's death sparked protests across the country. But even as the
Minneapolis police chief called Mr. Floyd's death
``\href{https://www.cnn.com/2020/06/24/us/minneapolis-police-chief-comment-george-floyd-trnd/index.html}{murder}''
and claimed that Mr. Chauvin ``knew what he was doing,'' little has
emerged about the 44-year-old officer, now charged with second-degree
murder and second-degree manslaughter, or what in his career might have
led up to an arrest so chilling in its quiet ferocity.

\includegraphics{https://static01.graylady3jvrrxbe.onion/images/2020/07/17/us/00UNREST-CHAUVIN-mug/00UNREST-CHAUVIN-mug-articleLarge.jpg?quality=75\&auto=webp\&disable=upscale}

The roadside encounter with the four teenagers led to a complaint
against Mr. Chauvin, and it reflected what both co-workers and citizens
told The New York Times about encountering the officer over his 19 years
with the Minneapolis Police Department: Mr. Chauvin did his job as if he
were playing a role --- a tough Dirty Harry on the lookout for bad guys.

``He was overly aggressive and not understanding that we were just
kids,'' recalled Noah McGurran-Hanson, who was in the car with Mr. Bergh
and the two others, all of whom are white. ``He was treating us like we
had been tried and convicted.''

Mr. Chauvin, his lawyer and family members have declined to talk to The
Times. Yet dozens of interviews with acquaintances depict a police
officer who seemed to operate at an emotional distance from those around
him. Mr. Chauvin was a quiet and rigid workaholic with poor people
skills and a tendency to overreact --- with intoxicated people,
especially --- when a less aggressive stance might have led to a better
outcome, interviews show.

He was awkward. Other officers often didn't like him or didn't know him.
He didn't go to parties and didn't seem to have many friends. Some
neighbors knew so little about him that they didn't even know he was a
police officer until after his arrest. Even his wife of 10 years, a
Hmong refugee and real estate agent, ended up estranged: Days after Mr.
Floyd's death, she filed for divorce and asked to change her last name.

Mr. Chauvin always wanted action. He continued to pound the streets in
one of Minneapolis's busiest precincts on its hardest shift, 4 p.m. to 2
a.m., long after many others his age moved to desk jobs or the day
shift.

That earned him kudos. He received two medals of commendation, for
tackling an armed suspect and arresting an armed gang member. He also
was awarded two medals of valor, after shooting a man wielding a
sawed-off shotgun and subduing a domestic-violence suspect --- whom he
shot and wounded in the process.

But his performance also led to at least 22 complaints or internal
investigations. Only one resulted in discipline. (Mr. Bergh said his
complaint was shrugged off by a sergeant who apologized for any
``negative interaction.'')

That is a high number compared with other officers, said Dave Bicking, a
board member of Communities United Against Police Brutality, based in
the Twin Cities. ``His numbers should have definitely raised alarm with
the department and triggered a review,'' said Mr. Bicking, adding that
most officers might get one or two complaints in seven years.

On his off nights, such as they were, Mr. Chauvin often worked security
at a nightclub.

Even on the police force, Mr. Chauvin was an outsider. He often
partnered with a rookie he was training, exacting in his expectations.
That was fine with veteran colleagues, who did not necessarily want to
ride alongside him.

``Occasionally, he would seem a little cocky,'' said Lucy Gerold, a
retired police commander who knew Mr. Chauvin. He was, she said, ``the
guy not everybody liked or wanted to work with.''

Image

Signs on Friday at the site where Mr. Chauvin put his knee on Mr.
Floyd's neck for more than eight minutes.Credit...Caroline Yang for The
New York Times

\hypertarget{a-face-in-the-crowd}{%
\subsection{`A Face in the Crowd'}\label{a-face-in-the-crowd}}

Mr. Chauvin spent his early years in the Twin Cities suburb of West St.
Paul, Minn., with a stay-at-home mother and a father who earned about
\$1,000 a month as a certified public accountant, barely enough for
their small family. When Derek was 7, his mother filed for divorce,
asking for the family home and child support for Derek and his baby
sister.

His father soon asked for a paternity test of Derek's baby sister; a
blood test showed he was not the father. His father ended up with the
family home and shared custody of Derek. His mother married her lover.
And Derek attended four elementary schools in five years.

Derek did not play sports in school --- at least, not that anyone
remembers. He did not have a yearbook photo for his junior or senior
years. One classmate from Park High School in Cottage Grove, another
Twin Cities suburb, remembered him as the student in R.O.T.C. who never
talked but always held the flag. Another classmate, Scott Swanson, said
Derek flew under the radar.

``I don't think he was an outcast or anything like that,'' said Mr.
Swanson, who said he had talked to fellow classmates in recent weeks who
also barely recalled him. ``He was just a face in the crowd.''

Weeks after graduation, Mr. Chauvin started as a prep cook at Tinucci's,
a restaurant 10 minutes from home. He enrolled that fall at the local
technical college to study ``quantity food preparation.''

But Mr. Chauvin decided he wanted a uniform.

He studied law enforcement at a community college; eventually, he would
also earn a university degree in law enforcement. After joining the
military police, he was deployed to a U.S. Army base in Germany, where
he studied for the Minnesota police exam in his spare time. He did not
socialize much or drink alcohol.

``He volunteered to be a designated driver for the guys who wanted to go
into town at night and have a few beers,'' said Jerry Obieglo, a platoon
sergeant who supervised Mr. Chauvin.

Back home, in September 2000, at age 24, he applied to the Minneapolis
police.

From the beginning, Mr. Chauvin stood out as gung-ho. When he reported
for training after the police academy, he showed up in a new white Crown
Victoria outfitted to resemble a police car, recalled one officer,
speaking on condition of anonymity because talking to the news media
could get him fired.

Leaving work, most officers dressed casually. But Mr. Chauvin, who stood
ramrod straight like he was still in the military, left in full uniform,
his pants pulled higher than most people wore them, his boots polished.

``In a group setting he would never connect and stand there like a small
child,'' the officer said. He added: ``I was put off by his lack of
communication skills. You never felt like he was present.''

Mr. Chauvin landed in the Third Precinct, one of the city's busiest.

The biggest call of his young career came when he was 30, in 2006:
Shortly after midnight, he and five other officers pursued a car driven
by a man suspected of stabbing two friends. The man soon pointed a
sawed-off shotgun at officers, the police said. They shot the man,
fatally. Mr. Chauvin received a medal of valor.

Mr. Chauvin soon earned two letters of reprimand for his behavior on
another call --- his only formal discipline.

In August 2007, Melissa Borton was heading home from grocery shopping
when Mr. Chauvin and a fellow officer pulled her over. Mr. Chauvin
reached into the open window of Ms. Borton's minivan, unlocked her door,
undid her seatbelt and started pulling her out, without any explanation,
she recalled. Her baby and dog were left in the vehicle.

She said the officers put her in their cruiser and told her that they
were looking for a vehicle resembling hers that had been involved in a
crime. Eventually they told Ms. Borton, who was by then quite upset,
that she could leave.

``When I got out, they noticed that my shirt was wet, which was from
being a breastfeeding mother,'' Ms. Borton recalled. She could not tell
who taunted her as she returned to her car. ``Chauvin or the other
officer rudely said, `You probably have postpartum depression, and you
need help.'''

Image

Protesters gathered near Mr. Chauvin's home in Oakdale, Minn., in late
May.Credit...Jenn Ackerman for The New York Times

\hypertarget{a-wife-and-a-home}{%
\subsection{A Wife and a Home}\label{a-wife-and-a-home}}

Until he was 27, Mr. Chauvin's home address was his grandmother's
suburban house in Inver Grove Heights.

But about the time he pulled over Ms. Borton, Mr. Chauvin was becoming
serious with his girlfriend, Kellie Xiong.

Ms. Xiong was a survivor. Her father had been a Hmong soldier fighting
Communists in Laos before the family fled in the late 1970s, he later
told a newspaper. After more than a year in a Thai refugee camp, the
family moved to Wisconsin, sponsored by a church in Eau Claire.

Ms. Xiong married another Hmong refugee in 1991 in what
\href{https://www.twincities.com/2019/01/06/first-hmong-mrs-minnesota-america-readies-for-the-national-pageant-stage/https://www.twincities.com/2018/06/02/refugee-who-was-shamed-for-her-looks-as-a-child-is-vying-to-be-the-first-hmong-mrs-minnesota/}{she
later told The Pioneer Press} was an arranged marriage. She was 16. By
19, she had given birth to two sons.

She later left her husband, whom she described as abusive, and moved to
the Minneapolis area to work as a radiologic technician at Hennepin
County Medical Center. There, she met Mr. Chauvin, who had brought
someone in for a health check before an arrest, she later told The
Pioneer Press. He soon asked her out.

By 2008, they were planning their lives. Two weeks after Ms. Xiong filed
for divorce from her first husband, Mr. Chauvin bought a new house in a
new subdivision for \$441,000. It was fit for a family, with four
bedrooms, four bathrooms and a three-car garage.

The couple married in June 2010. From the beginning, they spread their
money thin. Not only did Mr. Chauvin hold on to a townhouse he had
bought in 2003, but the couple also bought a vacation home near Disney
World in Florida in 2011.

Mr. Chauvin soon fell behind on fees for his townhouse. On a delinquency
notice for \$280 in 2013, Mr. Chauvin responded that he had paid
everything and added, ``So no payment is actually owed!'' He faxed the
response at 3:17 a.m., after finishing his shift in the Third Precinct.

By July 2014, the small debt had snowballed into a judgment of almost
\$8,000 because Mr. Chauvin never came to court.

Meanwhile, the Chauvins downsized. They sold their large house for
almost \$60,000 less than its purchase price. They bought a home a few
blocks away, almost half the size.

In 2015, they appeared to toy with moving to Florida. They sold the home
they had just bought. Ms. Chauvin got her radiologic technology license
in Florida. Mr. Chauvin registered to vote there.

But they stayed in Minnesota, where Ms. Chauvin got her real estate
license in 2016. In her spare time, Ms. Chauvin continued with one
passion --- rescuing dogs, often caring for four at once --- and found
another, in beauty pageants.

Before one pageant, she described her husband as a ``softy'' who always
opened doors for her.

But there were some awkward moments at the Mrs. Minnesota America
contest in June 2018 when the husbands joined the show. A host asked Mr.
Chauvin, wearing an ill-fitting tuxedo and bow tie, what additional
competition the women should perform. He suggested a rock-climbing wall
--- for the husbands.

``Well, you're not competing, I'm talking about your wife here,'' the
host replied.

During a quiz segment, each contestant wrote down something about her
husband, and the men had to guess which one described him. Mr. Chauvin
failed miserably, even as other husbands correctly recognized their
wives' responses. Initially, Mr. Chauvin thought he was the one whose
wife said he liked to tell stories. But he wasn't.

A bit later, a host gave another clue: ``Whoever you are, you do
upside-down hanging crunches. You can do 100 at a time.''

No one stepped forward.

``Uh, Derek Chauvin?'' the host said.

Image

Mr. Floyd and Mr. Chauvin worked security at El Nuevo Rodeo, a Latin
club in Minneapolis. The club was burned during protests, and the
building has since been torn down.Credit...Caroline Yang for The New
York Times

\hypertarget{that-is-protocol}{%
\subsection{`That Is Protocol'}\label{that-is-protocol}}

The Chauvins often seemed to live on separate tracks.

When Ms. Chauvin took trips to help dogs --- including one she rescued
from Florida and named Marley --- she often brought a female friend for
company.

On most weekends for 17 years, Mr. Chauvin worked an off-duty police gig
outside the El Nuevo Rodeo nightclub, earning \$55 an hour. Maya
Santamaria, who once owned the club, said the Third Precinct decided
which officers were assigned.

Mr. Chauvin often overreacted when he saw something that bothered him,
like unruly behavior around the club, including drunk patrons
congregating on the street --- especially on ``urban nights,'' when the
clientele was largely Black, Ms. Santamaria said.

He often resorted to using pepper spray, she said. When she complained,
she said, she usually got the same response.

``That is protocol,'' Mr. Chauvin told her.

Mr. Floyd, by coincidence, also did
\href{https://www.nytimes3xbfgragh.onion/2020/05/29/us/derek-chauvin-george-floyd-worked-together.html}{security
at the club}, but Ms. Santamaria said she does not recall seeing them
together since Mr. Floyd worked inside.

Their one known encounter came on the evening of May 25, after a corner
store employee reported that Mr. Floyd had tried to pass a counterfeit
\$20 bill. Two rookie officers,
\href{https://www.nytimes3xbfgragh.onion/2020/06/27/us/minneapolis-police-officer-kueng.html}{J.
Alexander Kueng} and
\href{https://www.nytimes3xbfgragh.onion/2020/07/08/us/george-floyd-body-camera-transcripts.html}{Thomas
Lane}, responded.

The two failed to get Mr. Floyd into their cruiser. Mr. Chauvin and
another officer, Tou Thao, arrived. Mr. Chauvin had been Mr. Kueng's
main training officer; Mr. Lane had relied on him for advice. (The three
other officers, who were fired alongside Mr. Chauvin, have been charged
with
\href{https://www.nytimes3xbfgragh.onion/2020/07/08/us/george-floyd-body-camera-transcripts.html}{aiding
and abetting} in Mr. Floyd's death.)

At Mr. Chauvin's suggestion, the officers got Mr. Floyd, agitated and
struggling, on the ground. Mr. Chauvin jammed his knee in the back of
Mr. Floyd's neck. The rookies held his back and legs.

Body camera footage shows what unfolded:

As Mr. Floyd said he could not breathe and asked for his mother, Mr.
Chauvin uttered another tough-cop line. ``You're under arrest, guy,'' he
said. ``That's why you're going to jail.''

Mr. Chauvin asked if Mr. Floyd was high; Mr. Lane said he assumed so.
Toxicology results would later show that
\href{https://www.nytimes3xbfgragh.onion/article/george-floyd-autopsy-michael-baden.html}{Mr.
Floyd was on fentanyl}.

``They're going to kill me, man,'' Mr. Floyd said a few moments later.

``Takes a heck of a lot of oxygen to say that,'' Mr. Chauvin replied
nonchalantly.

After the remark, Mr. Floyd said he could not breathe four times and
``please'' three times, and then nothing. Mr. Lane, who had called for
an ambulance because Mr. Floyd's mouth was bleeding, asked Mr. Chauvin
whether he wanted Mr. Floyd on his side.

``No, leave him,'' Mr. Chauvin said. He said an ambulance was coming.

In the middle of this --- of a man dying, under his knee --- Mr. Chauvin
checked his rookies. ``You guys all right, though?'' Mr. Chauvin asked.

Mr. Lane asked again if they should roll Mr. Floyd on his side.
Onlookers asked if he had a pulse. ``You got one?'' Mr. Lane asked.

``I can't find one,'' Mr. Kueng said.

``Uh-huh,'' Mr. Chauvin replied.

Mr. Kueng tried again, and again said he could not find a pulse. Still,
Mr. Chauvin kept his knee on Mr. Floyd's neck for more than two minutes.
He ignored the crowd, the pleas for Mr. Floyd's life, the jeers. He
waited for an ambulance that showed up far too late. And only then did
Mr. Chauvin stand up.

This time, not quite as straight.

Kim Barker reported from Minneapolis and New York, and Serge Kovaleski
from New York. Contributing reporting were Mike Baker from Seattle, Dan
Levin from New York, Matt Furber from Minneapolis,
\href{https://www.nytimes3xbfgragh.onion/by/frances-robles}{Frances
Robles} from Key West, Fla., and Amaris Castillo from Windermere, Fla.
Susan Beachy contributed research.

Advertisement

\protect\hyperlink{after-bottom}{Continue reading the main story}

\hypertarget{site-index}{%
\subsection{Site Index}\label{site-index}}

\hypertarget{site-information-navigation}{%
\subsection{Site Information
Navigation}\label{site-information-navigation}}

\begin{itemize}
\tightlist
\item
  \href{https://help.nytimes3xbfgragh.onion/hc/en-us/articles/115014792127-Copyright-notice}{©~2020~The
  New York Times Company}
\end{itemize}

\begin{itemize}
\tightlist
\item
  \href{https://www.nytco.com/}{NYTCo}
\item
  \href{https://help.nytimes3xbfgragh.onion/hc/en-us/articles/115015385887-Contact-Us}{Contact
  Us}
\item
  \href{https://www.nytco.com/careers/}{Work with us}
\item
  \href{https://nytmediakit.com/}{Advertise}
\item
  \href{http://www.tbrandstudio.com/}{T Brand Studio}
\item
  \href{https://www.nytimes3xbfgragh.onion/privacy/cookie-policy\#how-do-i-manage-trackers}{Your
  Ad Choices}
\item
  \href{https://www.nytimes3xbfgragh.onion/privacy}{Privacy}
\item
  \href{https://help.nytimes3xbfgragh.onion/hc/en-us/articles/115014893428-Terms-of-service}{Terms
  of Service}
\item
  \href{https://help.nytimes3xbfgragh.onion/hc/en-us/articles/115014893968-Terms-of-sale}{Terms
  of Sale}
\item
  \href{https://spiderbites.nytimes3xbfgragh.onion}{Site Map}
\item
  \href{https://help.nytimes3xbfgragh.onion/hc/en-us}{Help}
\item
  \href{https://www.nytimes3xbfgragh.onion/subscription?campaignId=37WXW}{Subscriptions}
\end{itemize}
