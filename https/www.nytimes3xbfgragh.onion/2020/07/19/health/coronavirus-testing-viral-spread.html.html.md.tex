Sections

SEARCH

\protect\hyperlink{site-content}{Skip to
content}\protect\hyperlink{site-index}{Skip to site index}

\href{https://www.nytimes3xbfgragh.onion/section/health}{Health}

\href{https://myaccount.nytimes3xbfgragh.onion/auth/login?response_type=cookie\&client_id=vi}{}

\href{https://www.nytimes3xbfgragh.onion/section/todayspaper}{Today's
Paper}

\href{/section/health}{Health}\textbar{}Testing Backlogs May Cloud the
True Spread of the Coronavirus

\url{https://nyti.ms/3fEyT9n}

\begin{itemize}
\item
\item
\item
\item
\item
\end{itemize}

\hypertarget{the-coronavirus-outbreak}{%
\subsubsection{\texorpdfstring{\href{https://www.nytimes3xbfgragh.onion/news-event/coronavirus?name=styln-coronavirus-national\&region=TOP_BANNER\&variant=undefined\&block=storyline_menu_recirc\&action=click\&pgtype=Article\&impression_id=036d8260-e387-11ea-8c8a-b7fbae24c18c}{The
Coronavirus
Outbreak}}{The Coronavirus Outbreak}}\label{the-coronavirus-outbreak}}

\begin{itemize}
\tightlist
\item
  live\href{https://www.nytimes3xbfgragh.onion/2020/08/20/world/coronavirus-covid.html?name=styln-coronavirus-national\&region=TOP_BANNER\&variant=undefined\&block=storyline_menu_recirc\&action=click\&pgtype=Article\&impression_id=036da970-e387-11ea-8c8a-b7fbae24c18c}{Latest
  Updates}
\item
  \href{https://www.nytimes3xbfgragh.onion/interactive/2020/us/coronavirus-us-cases.html?name=styln-coronavirus-national\&region=TOP_BANNER\&variant=undefined\&block=storyline_menu_recirc\&action=click\&pgtype=Article\&impression_id=036da971-e387-11ea-8c8a-b7fbae24c18c}{Maps
  and Cases}
\item
  \href{https://www.nytimes3xbfgragh.onion/interactive/2020/science/coronavirus-vaccine-tracker.html?name=styln-coronavirus-national\&region=TOP_BANNER\&variant=undefined\&block=storyline_menu_recirc\&action=click\&pgtype=Article\&impression_id=036da972-e387-11ea-8c8a-b7fbae24c18c}{Vaccine
  Tracker}
\item
  \href{https://www.nytimes3xbfgragh.onion/2020/08/19/us/colleges-closing-covid.html?name=styln-coronavirus-national\&region=TOP_BANNER\&variant=undefined\&block=storyline_menu_recirc\&action=click\&pgtype=Article\&impression_id=036da973-e387-11ea-8c8a-b7fbae24c18c}{Colleges
  Closing}
\item
  \href{https://www.nytimes3xbfgragh.onion/live/2020/08/20/business/stock-market-today-coronavirus?name=styln-coronavirus-national\&region=TOP_BANNER\&variant=undefined\&block=storyline_menu_recirc\&action=click\&pgtype=Article\&impression_id=036da974-e387-11ea-8c8a-b7fbae24c18c}{Economy}
\end{itemize}

Advertisement

\protect\hyperlink{after-top}{Continue reading the main story}

Supported by

\protect\hyperlink{after-sponsor}{Continue reading the main story}

\hypertarget{testing-backlogs-may-cloud-the-true-spread-of-the-coronavirus}{%
\section{Testing Backlogs May Cloud the True Spread of the
Coronavirus}\label{testing-backlogs-may-cloud-the-true-spread-of-the-coronavirus}}

Public health experts say delays in testing continue to hinder attempts
to track and contain the spread of disease.

\includegraphics{https://static01.graylady3jvrrxbe.onion/images/2020/07/19/science/19virus-health2/19virus-health2-articleLarge.jpg?quality=75\&auto=webp\&disable=upscale}

\href{https://www.nytimes3xbfgragh.onion/by/katherine-j--wu}{\includegraphics{https://static01.graylady3jvrrxbe.onion/images/2020/08/11/reader-center/author-katherine-j-wu/author-katherine-j-wu-thumbLarge.png}}

By
\href{https://www.nytimes3xbfgragh.onion/by/katherine-j--wu}{Katherine
J. Wu}

\begin{itemize}
\item
  Published July 19, 2020Updated Aug. 15, 2020
\item
  \begin{itemize}
  \item
  \item
  \item
  \item
  \item
  \end{itemize}
\end{itemize}

As demand for
\href{https://www.nytimes3xbfgragh.onion/2020/08/15/us/coronavirus-testing-decrease.html}{coronavirus
testing} surges around the nation, laboratories that process samples are
again experiencing backlogs that have left anxious patients and their
doctors waiting days ---~sometimes a week or more~---~for results.

At the city and state levels, testing delays could mask persistent
\href{https://www.nytimes3xbfgragh.onion/2020/07/21/health/coronavirus-infections-us.html}{rises
in case numbers} and could cloud ways to combat the coronavirus, as
health officials continue to find themselves one step behind the virus's
rapid and often silent spread, experts said.

Dr. Francis Collins, the director of the National Institutes of Health,
acknowledged the dangers associated with such delays in
\href{https://www.nbcnews.com/meet-the-press/video/full-nih-director-interview-all-americans-need-to-recognize-it-s-up-to-us-87958597515}{an
interview} on NBC's ``Meet the Press'' that aired on Sunday.

``The average test delay is too long,'' Dr. Collins said. ``That really
undercuts the value of the testing, because you do the testing to find
out who's carrying the virus, and then quickly get them isolated so they
don't spread it around. And it's very hard to make that work when
there's a long delay built in.''

Though the coronavirus testing landscape continues to expand, most
patient samples must still be routed through laboratories for
processing, and the demand is once again straining supplies, equipment
and trained technicians and causing shortages.

``It's very important for people to be able to get the results in time,
so they don't continue infecting people,'' said Pamela Martinez, an
expert in disease dynamics at the Harvard T.H. Chan School of Public
Health.

That has become increasingly essential, Dr. Martinez added, as mounting
evidence has indicated that the virus can spread from people who don't
have symptoms. ``Maybe if I take a test, but I don't have many symptoms,
I'm not going to take the same precautions,'' she said.

\hypertarget{latest-updates-the-coronavirus-outbreak}{%
\section{\texorpdfstring{\href{https://www.nytimes3xbfgragh.onion/2020/08/20/world/coronavirus-covid.html?action=click\&pgtype=Article\&state=default\&region=MAIN_CONTENT_1\&context=storylines_live_updates}{Latest
Updates: The Coronavirus
Outbreak}}{Latest Updates: The Coronavirus Outbreak}}\label{latest-updates-the-coronavirus-outbreak}}

Updated 2020-08-21T07:46:15.883Z

\begin{itemize}
\tightlist
\item
  \href{https://www.nytimes3xbfgragh.onion/2020/08/20/world/coronavirus-covid.html?action=click\&pgtype=Article\&state=default\&region=MAIN_CONTENT_1\&context=storylines_live_updates\#link-68774d88}{Shutdowns,
  warnings and scoldings follow alarming incidents on college campuses.}
\item
  \href{https://www.nytimes3xbfgragh.onion/2020/08/20/world/coronavirus-covid.html?action=click\&pgtype=Article\&state=default\&region=MAIN_CONTENT_1\&context=storylines_live_updates\#link-26b58724}{Biden
  knocks Trump's pandemic response, and outlines a national strategy.}
\item
  \href{https://www.nytimes3xbfgragh.onion/2020/08/20/world/coronavirus-covid.html?action=click\&pgtype=Article\&state=default\&region=MAIN_CONTENT_1\&context=storylines_live_updates\#link-4e542da3}{U.S.
  health agencies announce moves to confront the flu season and
  plummeting child vaccination rates.}
\end{itemize}

\href{https://www.nytimes3xbfgragh.onion/2020/08/20/world/coronavirus-covid.html?action=click\&pgtype=Article\&state=default\&region=MAIN_CONTENT_1\&context=storylines_live_updates}{See
more updates}

More live coverage:
\href{https://www.nytimes3xbfgragh.onion/live/2020/08/20/business/stock-market-today-coronavirus?action=click\&pgtype=Article\&state=default\&region=MAIN_CONTENT_1\&context=storylines_live_updates}{Markets}

Health workers typically advise their patients to quarantine at home
while they await their test results, out of an abundance of caution. To
the extent that one can, ``The best thing to do is to act as if you've
been infected'' in this interim period, said Olivia Prosper, an
infectious disease modeler at the University of Tennessee, Knoxville.
But the longer people are forced to wait, the more difficult that advice
is to follow ---~and the larger toll their absence from work or family
responsibilities can take.

Additionally, negative results can be of little use if they are
delivered after too long of a delay. Diagnostic testing, which searches
for bits of the coronavirus's genetic material, can only assess a
person's health status from the time the sample was taken, and can't
account for any subsequent exposures to the virus.

Some have held out hope that new, confirmed coronavirus cases could soon
peak in certain states, after which parts of the nation might experience
a much-needed respite in infections ---~as, perhaps, some regions did
for \href{https://covidtracking.com/data/us-daily/}{much of May}. But
the duration of that apex, which might actually manifest more like a
plateau,~can't be definitively forecast. With many laboratories
stretched to or past their limits, a leveling-off in confirmed cases
could indicate a slowing in the coronavirus's spread. Or it could simply
reflect a regional ceiling in testing capacity.

These scenarios can be difficult to tease apart, Dr. Prosper said.
Disease transmission dynamics also undergo some natural flux within
populations, as people change their socialization patterns or spend more
or less time outdoors. As such, putting too much stock in day-to-day
trends can be risky, and possibly misleading.

``We need to be looking at long-term trends,'' Dr. Prosper said. ``Even
if there is a string of days where case counts are lower, that's not
necessarily reflective of a downward trend in this epidemic.''

Researchers also believe that current reported case counts of the
coronavirus are
\href{https://www.nytimes3xbfgragh.onion/2020/06/27/health/coronavirus-antibodies-asymptomatic.html}{vastly
undercounting the actual number of infections} --- a knowledge gap that
has made it difficult to quash the virus for good. ``Most people will
seek a test if they're showing symptoms,'' Dr. Prosper said. But those
who feel well often have less incentive to do so.

In an ideal world, she said, far more community testing would occur to
catch some of these more silent cases. ``That would give us more of an
idea of how this disease is actually playing out,'' she said.

\href{https://www.nytimes3xbfgragh.onion/interactive/2020/us/coronavirus-us-cases.html}{}

\includegraphics{https://static01.graylady3jvrrxbe.onion/images/2020/03/03/us/coronavirus-us-cases-map-promo-1583277425489/coronavirus-us-cases-map-promo-1583277425489-articleLarge-v680.png}

\hypertarget{coronavirus-in-the-us-latest-map-and-case-count}{%
\subsection{Coronavirus in the U.S.: Latest Map and Case
Count}\label{coronavirus-in-the-us-latest-map-and-case-count}}

A detailed county map shows the extent of the coronavirus outbreak, with
tables of the number of cases by county.

But laboratories that are hitting their breaking points may be unlikely
candidates to bridge that divide.

To speed turnaround times, Dr. Collins said, health officials are
pushing for more
\href{https://www.nytimes3xbfgragh.onion/2020/07/06/health/fast-coronavirus-tests.html}{point-of-care
testing} ---~``on the spot'' tests designed to be done rapidly and
easily, without the need for specialized laboratory equipment or
personnel.

\href{https://www.nytimes3xbfgragh.onion/news-event/coronavirus?action=click\&pgtype=Article\&state=default\&region=MAIN_CONTENT_3\&context=storylines_faq}{}

\hypertarget{the-coronavirus-outbreak-}{%
\subsubsection{The Coronavirus Outbreak
›}\label{the-coronavirus-outbreak-}}

\hypertarget{frequently-asked-questions}{%
\paragraph{Frequently Asked
Questions}\label{frequently-asked-questions}}

Updated August 17, 2020

\begin{itemize}
\item ~
  \hypertarget{why-does-standing-six-feet-away-from-others-help}{%
  \paragraph{Why does standing six feet away from others
  help?}\label{why-does-standing-six-feet-away-from-others-help}}

  \begin{itemize}
  \tightlist
  \item
    The coronavirus spreads primarily through droplets from your mouth
    and nose, especially when you cough or sneeze. The C.D.C., one of
    the organizations using that measure,
    \href{https://www.nytimes3xbfgragh.onion/2020/04/14/health/coronavirus-six-feet.html?action=click\&pgtype=Article\&state=default\&region=MAIN_CONTENT_3\&context=storylines_faq}{bases
    its recommendation of six feet} on the idea that most large droplets
    that people expel when they cough or sneeze will fall to the ground
    within six feet. But six feet has never been a magic number that
    guarantees complete protection. Sneezes, for instance, can launch
    droplets a lot farther than six feet,
    \href{https://jamanetwork.com/journals/jama/fullarticle/2763852}{according
    to a recent study}. It's a rule of thumb: You should be safest
    standing six feet apart outside, especially when it's windy. But
    keep a mask on at all times, even when you think you're far enough
    apart.
  \end{itemize}
\item ~
  \hypertarget{i-have-antibodies-am-i-now-immune}{%
  \paragraph{I have antibodies. Am I now
  immune?}\label{i-have-antibodies-am-i-now-immune}}

  \begin{itemize}
  \tightlist
  \item
    As of right
    now,\href{https://www.nytimes3xbfgragh.onion/2020/07/22/health/covid-antibodies-herd-immunity.html?action=click\&pgtype=Article\&state=default\&region=MAIN_CONTENT_3\&context=storylines_faq}{that
    seems likely, for at least several months.} There have been
    frightening accounts of people suffering what seems to be a second
    bout of Covid-19. But experts say these patients may have a
    drawn-out course of infection, with the virus taking a slow toll
    weeks to months after initial exposure. People infected with the
    coronavirus typically
    \href{https://www.nature.com/articles/s41586-020-2456-9}{produce}
    immune molecules called antibodies, which are
    \href{https://www.nytimes3xbfgragh.onion/2020/05/07/health/coronavirus-antibody-prevalence.html?action=click\&pgtype=Article\&state=default\&region=MAIN_CONTENT_3\&context=storylines_faq}{protective
    proteins made in response to an
    infection}\href{https://www.nytimes3xbfgragh.onion/2020/05/07/health/coronavirus-antibody-prevalence.html?action=click\&pgtype=Article\&state=default\&region=MAIN_CONTENT_3\&context=storylines_faq}{.
    These antibodies may} last in the body
    \href{https://www.nature.com/articles/s41591-020-0965-6}{only two to
    three months}, which may seem worrisome, but that's perfectly normal
    after an acute infection subsides, said Dr. Michael Mina, an
    immunologist at Harvard University. It may be possible to get the
    coronavirus again, but it's highly unlikely that it would be
    possible in a short window of time from initial infection or make
    people sicker the second time.
  \end{itemize}
\item ~
  \hypertarget{im-a-small-business-owner-can-i-get-relief}{%
  \paragraph{I'm a small-business owner. Can I get
  relief?}\label{im-a-small-business-owner-can-i-get-relief}}

  \begin{itemize}
  \tightlist
  \item
    The
    \href{https://www.nytimes3xbfgragh.onion/article/small-business-loans-stimulus-grants-freelancers-coronavirus.html?action=click\&pgtype=Article\&state=default\&region=MAIN_CONTENT_3\&context=storylines_faq}{stimulus
    bills enacted in March} offer help for the millions of American
    small businesses. Those eligible for aid are businesses and
    nonprofit organizations with fewer than 500 workers, including sole
    proprietorships, independent contractors and freelancers. Some
    larger companies in some industries are also eligible. The help
    being offered, which is being managed by the Small Business
    Administration, includes the Paycheck Protection Program and the
    Economic Injury Disaster Loan program. But lots of folks have
    \href{https://www.nytimes3xbfgragh.onion/interactive/2020/05/07/business/small-business-loans-coronavirus.html?action=click\&pgtype=Article\&state=default\&region=MAIN_CONTENT_3\&context=storylines_faq}{not
    yet seen payouts.} Even those who have received help are confused:
    The rules are draconian, and some are stuck sitting on
    \href{https://www.nytimes3xbfgragh.onion/2020/05/02/business/economy/loans-coronavirus-small-business.html?action=click\&pgtype=Article\&state=default\&region=MAIN_CONTENT_3\&context=storylines_faq}{money
    they don't know how to use.} Many small-business owners are getting
    less than they expected or
    \href{https://www.nytimes3xbfgragh.onion/2020/06/10/business/Small-business-loans-ppp.html?action=click\&pgtype=Article\&state=default\&region=MAIN_CONTENT_3\&context=storylines_faq}{not
    hearing anything at all.}
  \end{itemize}
\item ~
  \hypertarget{what-are-my-rights-if-i-am-worried-about-going-back-to-work}{%
  \paragraph{What are my rights if I am worried about going back to
  work?}\label{what-are-my-rights-if-i-am-worried-about-going-back-to-work}}

  \begin{itemize}
  \tightlist
  \item
    Employers have to provide
    \href{https://www.osha.gov/SLTC/covid-19/standards.html}{a safe
    workplace} with policies that protect everyone equally.
    \href{https://www.nytimes3xbfgragh.onion/article/coronavirus-money-unemployment.html?action=click\&pgtype=Article\&state=default\&region=MAIN_CONTENT_3\&context=storylines_faq}{And
    if one of your co-workers tests positive for the coronavirus, the
    C.D.C.} has said that
    \href{https://www.cdc.gov/coronavirus/2019-ncov/community/guidance-business-response.html}{employers
    should tell their employees} -\/- without giving you the sick
    employee's name -\/- that they may have been exposed to the virus.
  \end{itemize}
\item ~
  \hypertarget{what-is-school-going-to-look-like-in-september}{%
  \paragraph{What is school going to look like in
  September?}\label{what-is-school-going-to-look-like-in-september}}

  \begin{itemize}
  \tightlist
  \item
    It is unlikely that many schools will return to a normal schedule
    this fall, requiring the grind of
    \href{https://www.nytimes3xbfgragh.onion/2020/06/05/us/coronavirus-education-lost-learning.html?action=click\&pgtype=Article\&state=default\&region=MAIN_CONTENT_3\&context=storylines_faq}{online
    learning},
    \href{https://www.nytimes3xbfgragh.onion/2020/05/29/us/coronavirus-child-care-centers.html?action=click\&pgtype=Article\&state=default\&region=MAIN_CONTENT_3\&context=storylines_faq}{makeshift
    child care} and
    \href{https://www.nytimes3xbfgragh.onion/2020/06/03/business/economy/coronavirus-working-women.html?action=click\&pgtype=Article\&state=default\&region=MAIN_CONTENT_3\&context=storylines_faq}{stunted
    workdays} to continue. California's two largest public school
    districts --- Los Angeles and San Diego --- said on July 13, that
    \href{https://www.nytimes3xbfgragh.onion/2020/07/13/us/lausd-san-diego-school-reopening.html?action=click\&pgtype=Article\&state=default\&region=MAIN_CONTENT_3\&context=storylines_faq}{instruction
    will be remote-only in the fall}, citing concerns that surging
    coronavirus infections in their areas pose too dire a risk for
    students and teachers. Together, the two districts enroll some
    825,000 students. They are the largest in the country so far to
    abandon plans for even a partial physical return to classrooms when
    they reopen in August. For other districts, the solution won't be an
    all-or-nothing approach.
    \href{https://bioethics.jhu.edu/research-and-outreach/projects/eschool-initiative/school-policy-tracker/}{Many
    systems}, including the nation's largest, New York City, are
    devising
    \href{https://www.nytimes3xbfgragh.onion/2020/06/26/us/coronavirus-schools-reopen-fall.html?action=click\&pgtype=Article\&state=default\&region=MAIN_CONTENT_3\&context=storylines_faq}{hybrid
    plans} that involve spending some days in classrooms and other days
    online. There's no national policy on this yet, so check with your
    municipal school system regularly to see what is happening in your
    community.
  \end{itemize}
\end{itemize}

Some of these tests could be completed in a doctor's office, or perhaps
even at home, in under an hour. Simple, speedy tests could prove to be a
boon for institutions and communities that care for large numbers of
vulnerable people, such as nursing homes. They could also help health
workers bring testing supplies to populations that have often been
denied access to testing and reliable health care, including those
marginalized by race, ethnicity or socioeconomic status.

A handful of point-of-care tests have been
\href{https://www.fda.gov/medical-devices/coronavirus-disease-2019-covid-19-emergency-use-authorizations-medical-devices/vitro-diagnostics-euas\#imft1}{greenlighted
for emergency use} by the Food and Drug Administration.

``We need to invest a lot of money, and the government is willing to do
so, in scaling those up,'' Dr. Collins said on Sunday. ``That's the kind
of thing that I personally, along with many others in other parts of the
government, are working on night and day to try to do a better job of.''

But Dr. Prosper pointed out that speed often comes at the price of
accuracy ---~an issue that has plagued some point-of-care tests in the
past. Though rapid testing can still play a substantial role in
mitigating the spread of the coronavirus, researchers will need to
remain wary of these trade-offs, she said.

As testing efforts continue to ramp up, Dr. Martinez cautioned that the
nation will need to maintain its vigilance for some time yet. ``The
effects of social distancing are reversible,'' she said. If people give
up on those strategies too soon, ``It's likely that we will observe a
third or a fourth peak. And that could have big implications.''

In
\href{https://www.cbsnews.com/news/transcript-scott-gottlieb-discusses-coronavirus-on-face-the-nation-july-19-2020/}{an
interview} on Sunday with CBS's ``Face the Nation,'' Dr. Scott Gottlieb,
the former commissioner of the Food and Drug Administration, reiterated
the potentially devastating consequences of failing to rein in the
virus, noting spikes in cases in states like California, Texas, Arizona
and Florida. He warned that other states, like Georgia, Tennessee and
Kentucky, could follow similar patterns.

``We're seeing record numbers of cases, rising hospitalizations and
really a shifting of the center of the epidemic potentially in the
United States,'' Dr. Gottlieb said. ``This just portends more trouble
for the fall and the winter, that we're going to be taking a lot of
infection into the fall, that we're never going to really be able to
come down.''

Advertisement

\protect\hyperlink{after-bottom}{Continue reading the main story}

\hypertarget{site-index}{%
\subsection{Site Index}\label{site-index}}

\hypertarget{site-information-navigation}{%
\subsection{Site Information
Navigation}\label{site-information-navigation}}

\begin{itemize}
\tightlist
\item
  \href{https://help.nytimes3xbfgragh.onion/hc/en-us/articles/115014792127-Copyright-notice}{©~2020~The
  New York Times Company}
\end{itemize}

\begin{itemize}
\tightlist
\item
  \href{https://www.nytco.com/}{NYTCo}
\item
  \href{https://help.nytimes3xbfgragh.onion/hc/en-us/articles/115015385887-Contact-Us}{Contact
  Us}
\item
  \href{https://www.nytco.com/careers/}{Work with us}
\item
  \href{https://nytmediakit.com/}{Advertise}
\item
  \href{http://www.tbrandstudio.com/}{T Brand Studio}
\item
  \href{https://www.nytimes3xbfgragh.onion/privacy/cookie-policy\#how-do-i-manage-trackers}{Your
  Ad Choices}
\item
  \href{https://www.nytimes3xbfgragh.onion/privacy}{Privacy}
\item
  \href{https://help.nytimes3xbfgragh.onion/hc/en-us/articles/115014893428-Terms-of-service}{Terms
  of Service}
\item
  \href{https://help.nytimes3xbfgragh.onion/hc/en-us/articles/115014893968-Terms-of-sale}{Terms
  of Sale}
\item
  \href{https://spiderbites.nytimes3xbfgragh.onion}{Site Map}
\item
  \href{https://help.nytimes3xbfgragh.onion/hc/en-us}{Help}
\item
  \href{https://www.nytimes3xbfgragh.onion/subscription?campaignId=37WXW}{Subscriptions}
\end{itemize}
