Sections

SEARCH

\protect\hyperlink{site-content}{Skip to
content}\protect\hyperlink{site-index}{Skip to site index}

\href{https://www.nytimes3xbfgragh.onion/podcasts/the-daily}{The Daily}

\href{https://myaccount.nytimes3xbfgragh.onion/auth/login?response_type=cookie\&client_id=vi}{}

\href{https://www.nytimes3xbfgragh.onion/section/todayspaper}{Today's
Paper}

\href{/podcasts/the-daily}{The Daily}\textbar{}The Life and Legacy of
John Lewis

\url{https://nyti.ms/32BXCr2}

\begin{itemize}
\item
\item
\item
\item
\item
\item
\end{itemize}

Advertisement

\protect\hyperlink{after-top}{Continue reading the main story}

transcript

Back to The Daily

bars

0:00/38:56

-38:56

transcript

\hypertarget{the-life-and-legacy-of-john-lewis}{%
\subsection{The Life and Legacy of John
Lewis}\label{the-life-and-legacy-of-john-lewis}}

\hypertarget{hosted-by-michael-barbaro-produced-by-lynsea-garrison-and-clare-toeniskoetter-and-edited-by-lisa-tobin}{%
\subsubsection{Hosted by Michael Barbaro, produced by Lynsea Garrison
and Clare Toeniskoetter, and edited by Lisa
Tobin}\label{hosted-by-michael-barbaro-produced-by-lynsea-garrison-and-clare-toeniskoetter-and-edited-by-lisa-tobin}}

\hypertarget{a-look-at-the-extraordinary-life-of-the-civil-rights-icon}{%
\paragraph{A look at the extraordinary life of the civil rights
icon.}\label{a-look-at-the-extraordinary-life-of-the-civil-rights-icon}}

Monday, July 20th, 2020

\begin{itemize}
\item
  michael barbaro\\
  From The New York Times, I'm Michael Barbaro. This is ``The Daily.''
\item
  {[}music{]}
\item
  michael barbaro\\
  Today: The life, lessons and legacy of John Lewis. I spoke with my
  colleague, Times editorial board member, Brent Staples.

  It's Monday, July 20.

  Brent, I want to start by going back with you to the time when John
  Lewis and others began engaging in nonviolent protests as part of the
  civil rights movement. Where were you during that period?
\item
  brent staples\\
  Like everyone else at the time in the middle `60s, I was sitting with
  my parents, watching television ---
\item
  archived recording 1\\
  {[}MUSIC{]} It's 11:00 p.m., and time for the reporters and the news.
\item
  archived recording 2\\
  Good evening. Bad news for Alabama today. Some school desegregation
  strategy has backfired.
\item
  archived recording 3\\
  Policemen occasionally clubbed demonstrators and used a variety of
  other tactics designed to break their spirit.
\end{itemize}

brent staples

--- and the nightly news of the scenes of people being ravaged by the
police in the South in the streets. And I became attuned to the
revolution that was unfolding in the South, where people had put
themselves in harm's way to highlight the injustice of Southern
apartheid.

\begin{itemize}
\tightlist
\item
  archived recording\\
  The Southerners, white and Negro, spattered with ketchup and mustard,
  sugar, salt and pepper, were carted off, unprotesting to jail.
\end{itemize}

brent staples

But as I became more politically active, I had difficulty --- probably
natural difficulty, understanding how one would put oneself on the line
to be actually beaten and bloodied, and what the utility of that was.
And by that time, the movement in the South had evolved some way.

\begin{itemize}
\tightlist
\item
  archived recording\\
  (SINGING) No more brothers in jail. Off the pigs! The pigs are gonna
  catch hell.
\end{itemize}

brent staples

By `66, you began to get the Black Power slogans.

\begin{itemize}
\tightlist
\item
  archived recording\\
  They don't want us to use Black Power. I got news for them.
  {[}LAUGHTER{]}
\end{itemize}

brent staples

So I came into consciousness in my teens as a Black Power figure.

\begin{itemize}
\tightlist
\item
  archived recording\\
  This is not a riot, it is a rebellion.
\end{itemize}

brent staples

It was empowering.

\begin{itemize}
\tightlist
\item
  archived recording\\
  Number two, you are to be proud of your Black brothers and sisters at
  Fifth, because a honky cop touched one of them, and they told them,
  you've got to touch all of us.
\end{itemize}

brent staples

But it took some time into my late teens to begin to understand what had
happened coming up to that. And that is when I become aware of what had
passed before when I was a younger kid.

michael barbaro

In other words, it took you some time to understand why the nonviolent
figures had taken the approach that they had taken.

brent staples

Yes.

michael barbaro

So I want you to take us through John Lewis's life and how this
philosophy you just described was shaped and how it evolved. So where
does that story begin?

brent staples

John Lewis, he grows up in rural Alabama near Troy, Alabama. His parents
initially were sharecroppers. And you know, sharecropping was a
successor form of slavery. So John was born into that. And his parents,
they were lucky they saved enough money to buy a farm.

But he graduated high school, segregated high school, and wanted to go
to Troy State College, which didn't admit Black people. And he applied
and sent in his information. And he never heard back. And he wrote a
letter at one point to Martin Luther King. And I presume he wanted some
help in desegregating Troy State.

And Martin Luther King sent him a roundtrip bus ticket. And he went and
he met King, and they formed a relationship. The substance of that, I
don't know, but it was extremely influential for John. And he left Troy
not long after that and moved to Nashville.

{[}music{]}

brent staples

He went to seminary in Nashville. And there, he met some of the early
civil rights figures. He met James Lawson --- Reverend James Lawson ---
who was kind of a philosopher of nonviolent resistance. Lawson had
studied Gandhi's nonviolent movement and the strategies that Gandhi had
deployed against Britain during the colonial period. And he'd come back
with a deep sense of what the philosophy was and how powerful
nonviolence could be.

michael barbaro

And what exactly is the philosophy?

brent staples

I have John's memoir here. In 1958, I think it is, Jim Lawson mentions
to him the idea of, quote, ``redemptive suffering.'' And he explains
that it affects not only ourselves, but it touches and changes those
around us as well. It opens us and those around us to a force beyond
ourselves. A force that is right and moral. The force of righteous truth
that is the basis of human conscious. Suffering puts us and those around
us in touch with our consciences. It open and touches our hearts. It
makes us feel compassion where we need to and guilt if we must. So this
idea, to him, this redemptive suffering, it is at the heart of the
philosophy of nonviolent protest. At the very heart of it. This is a
good paragraph from the book:

``One method of practicing this approach, when faced with a hateful,
angry, aggressive, even despicable person, is to imagine that person ---
actually visualize him or her --- as an infant, as a baby. If you can
see this full-grown attacker who faces you as a pure, innocent child
that he or she once was, it is not hard to find compassion in your
heart.''

But then it wasn't just a tactic. It was a way of life. It was embracing
the biblical prescription that one must love one's enemies. That's a
biblical prescription. And it's the hardest thing in the world to carry
out.

michael barbaro

Well, so how do we start to see this get carried out among Lewis and
these seminary students in Nashville?

brent staples

Well, Nashville was itself at the time another southern town, where if
you went into a restaurant and sat down the people would just look at
you and the waitresses would say, sorry, this place doesn't serve
{[}EXPLETIVE{]}. And that would generally be the end of it. But these
students came in that place and sat down and asked to be fed. And when
they were told that they were not served, they stayed.

And they took a lot of abuse from it. And people spat upon them, beat
them, battered them, and poured condiments over their heads. All kinds
of things. And I remember our friend, David Halberstam, our former
colleague, was working at a newspaper there at the time. He was working
at The Nashville Tennesseean. And this is one of the things he wrote:

``The protest had been conducted with exceptional dignity, and
gradually, one image had come into prevail --- that of elegant,
courteous, young Black people, holding to their Gandhian principles,
seeking the most elemental of rights, while being assaulted by young
white hoodlums who beat them up and on occasion extinguished cigarettes
on their bodies.''

So you see John Lewis and others being carried away in these, really,
suits and ties and crisp white shirt. And basically refusing to walk
themselves, being completely passive and nonresistant. And this worked
out extremely --- I mean, in a very short period of time, it worked out
extremely well in Nashville. After three months of sit-ins, the city
basically caved and became the first major Southern city to begin
desegregating public facilities.

michael barbaro

So very early on, the protest that John Lewis is beginning to
participate in, after he meets Martin Luther King and begins to
understand the strategy, they are starting to show real signs of
effectiveness.

brent staples

Oh, yeah.

michael barbaro

These peaceful, nonviolent protests.

brent staples

Exactly. Yeah. This is astonishing thing, to me, to this day, to
practice the non-violent approach to life. To really embrace it, one
needs to understand that the person who was extinguishing a cigarette in
your throat because you want to sit down at a luncheon counter is as
much a victim as you are. What John was saying --- you, in pursuit of
justice, you cannot let violence win your heart. That if you do that,
you're surrendering, really, to the dark force that you're trying to
defeat.

michael barbaro

Brent, you're reading from a memoir that is written in the later years
of John Lewis's life. But my understanding is that this philosophy that
he embraced and that he practiced, it was not entirely a foregone
conclusion that this would be the way that it went, and that the March
on Washington is an example of a moment where we see a young John Lewis
grappling with which path he's going to take. Can you tell us the story
of that speech?

brent staples

Well, you see, what I finally figured out is that by the time John
Lewis, at the age of 23, gets to the march on Washington --- this is the
most important public gathering of Black people in the century --- he
has already been on the Freedom Rides. Integrated groups have taken
buses into the deep south to test laws that forbid segregation on
interstate transport. He's been arrested on those trips for going into
white-only bathrooms. He's been beaten just for being on buses with
white people. In the end, John ended up being arrested, like, 40 times.

michael barbaro

Right.

brent staples

And if you look at some of the pictures of the mugshots when he's
arrested, you can see him smiling because he's basically saying, you
think you're afflicting me, but you're playing into what I want to do.
But he's still 23 years old. That's all he is.

And he basically comes into Washington with a speech. And this one
somehow found its way into public. And one of the striking things about
it, he tells people to get into the street and stay in the street until
the revolution is finished. And he names the sort of racist
segregationist senators by name. And state governors, too.

michael barbaro

Wow.

brent staples

I'm going to read from it.

``We won't stop now. All the forces of Eastland, Barnett, Wallace,
Thurman won't stop this revolution. The time will come when we will not
confine our marching to Washington. We will march through the South,
through the heart of Dixie, the way Sherman did. We shall pursue our own
scorched earth policy and burn Jim Crow to the ground nonviolently.''

michael barbaro

Wow. And to explain that reference, he's referring to the Union General
who literally burns large sections of the South during the Civil War.

brent staples

Yes, he's referring to William Tecumseh Sherman. ``We shall pursue our
own scorched earth policy and burn Jim Crow to the ground nonviolently.
We shall fragment the south into a thousand pieces and put them back
together in the image of democracy. We will make the action of the past
few months look petty. And I say to you, wake up, America.''

michael barbaro

Brent, that language doesn't sound, even as it invokes the word,
nonviolent. So what do you think that he meant by those words in that
draft?

brent staples

Well, I've come to figure out what he meant. Now, understand, as I was
saying earlier, by the time he renders this speech, he has become
steeped in the nonviolent impulse. But his frame, he was portraying it
as a forceful measure that could be as powerful and changing, as was the
sweep of Sherman through South Carolina and Georgia. What you have here
is, John is working at a very high concept here, right? He's working at
a high concept. He's saying, we can be --- he was essentially arguing
that nonviolent protests could be transformative. As transformative and
as disruptive as war as carried out by the most feared general in the
Union Army. That is itself a very powerful metaphor. And it's a
testament to his beliefs and what his approach could do. And he was
calling upon hundreds of thousands of people to come out into the street
and make that a reality. But that's a high concept. And on the evening
news, you can imagine you'd end up with a snippet of a scorched earth
Sherman burning Atlanta again.

michael barbaro

Right, it might get lost in translation. It might actually undermine the
very thing he's trying to promote.

brent staples

Exactly. So basically, A. Philip Randolph, Martin Luther King, those
guys prevailed on him to make some changes in it. They were talking
about, let's not do anything to just not give them a sound bite that's
going to give us trouble. I can hear him saying it. I don't have the
tape, but I can hear them saying to John, let's not give them a sound
bite that's going to give us trouble. You can say it's a revolution. You
can call people on the street. You can even call them the Black masses,
if you want to, even though that sounds like communism, right? You could
say those things. But let's leave off Sherman for next time. Right?
{[}LAUGHS{]} So Sherman goes out.

\begin{itemize}
\tightlist
\item
  archived recording (john lewis)\\
  We march today for jobs and freedom. But we have nothing to be proud
  of. For hundreds of thousands of our brothers are not here.
\end{itemize}

brent staples

He talks about marching.

\begin{itemize}
\tightlist
\item
  archived recording (john lewis)\\
  The time will come. We will not confine our marching to Washington. We
  will march through the South, through the streets of Jackson ---
\end{itemize}

brent staples

We will march through the street of Jackson, through the streets of
Danville, through the streets of Cambridge, through the streets of
Birmingham.

\begin{itemize}
\item
  archived recording (john lewis)\\
  But we will march with the spirit of love and with the spirit of
  dignity that we have shown here today. {[}APPLAUSE{]}

  By the forces of our demand, our determination and our numbers, we
  shall splinter the segregated South into a thousand pieces and put
  them together in the image of God and democracy. We must say, Wake up,
  America! Wake up!'' For we cannot stop, and we will not and cannot be
  patient. {[}APPLAUSE{]}
\end{itemize}

brent staples

So this is pretty much the same. But Sherman is missing.

michael barbaro

So in terms of thinking about the speech that John Lewis drafted versus
the one he delivered, on that historic day, you're saying it's not that
the earlier version of the speech shows John Lewis questioning the
nonviolent approach. It's that he believed in the nonviolent approach,
but that the language he contemplated using --- his belief that the
power of that approach could be as powerful as burning --- that was
determined to be potentially counterproductive to the nonviolent
approach he believed in.

brent staples

Exactly. But he talks about going back to his hotel room after that
first conversation and just being livid because, of course, it's a
23-year-old man's speech. And a 23-year-old man who had been beaten to
an inch of his life while fighting for dignity for Black people. I'm
sure he felt entitled to say any damn thing he wanted to. Because he had
the credibility of the streets behind him and the people in the Douth
beginning to know who he was. And they were going to really know who he
was, come two years later at the voting rights march at Selma.

{[}music{]}

michael barbaro

We'll be right back.

{[}music{]}

michael barbaro

Brent, tell me about that. I mean, let's talk about what happens after
the March on Washington, after John Lewis's rhetorical wings are ever so
slightly clipped, but he does deliver the essential message. How do we
see this concept of the power of nonviolence actually play out over the
next couple of years?

brent staples

The next big data point becomes the voting rights struggle in Selma,
Alabama. Now it's important, I think, to dilate for just a second for
the modern listener.

The modern listener needs to understand that in voting arrangements in
the South, before the Voting Rights Act, local registrars had complete
authority to do whatever they wanted with people who came in to register
to vote. They could give you a test and then say you failed it so you
can't register. And they did that all day long, all day, every day, to
Black people.

In Virginia, a college-educated woman, Black woman, who I believe was a
teacher, went into register at one point and filled the application. And
the registrar handed her a literacy test. You know what it consisted of?
A blank sheet of paper. He asked her, what does this say? And she looked
at it and handed it back, and she said nothing. He said, you're wrong.
You fail. You can't register.

Elsewhere in the South, they might ask you if you came in to register to
vote, how many bubbles are there in a bar of soap?

Elsewhere, a famous example in the film ``Selma,'' where a woman had
come in to register one year, and they asked her, how many judges are
there in the state of Alabama? And she didn't know. They said, well, you
fail. And she came back the second year. And he said, how many county
judges are there in Alabama? And I think it was 67. And she said 67. And
he said now, before you register, you have to name them.

michael barbaro

Aw.

brent staples

So this was what life was like for Black people seeking to vote in the
South. Now John Lewis's organization, the Student Nonviolent
Coordinating Committee, had already set up a voting rights project in
Selma and had been working on that. But it came to fruition in 1965,
where people had been fed up. And so they staged a march from Selma to
Montgomery to protest in favor of voting rights for Black people.

And then that fateful day, on the Edmund Pettus Bridge, they were
walking. State troopers came out and said, this is unlawful.

\begin{itemize}
\tightlist
\item
  archived recording\\
  It would be detrimental to your safety to continue this march. And I'm
  saying that this is an unlawful assembly. You have to disperse. You
  are ordered to disperse. Go home, or go to your church. This march
  will not continue.
\end{itemize}

brent staples

You cannot march.

\begin{itemize}
\item
  archived recording\\
  Troopers, here, advance toward the gate.

  See that they disperse.
\end{itemize}

brent staples

And then John said to the sheriff. He said, can I have a word? Because
you know he's in the front with his little off-white trenchcoat on. And
now there's a thing about this trenchcoat, right? It's very light, so
you can see him standing out from everyone else. And the other thing
about it is, once blood gets on it, you can really see it. So I'm sure
that that was premeditated. So he comes out front, and he says, can I
have a word? No, you can't have a word. And the troopers begin to
advance. And they beat holy hell out of those people. They sent 58
people to the hospital.

John Lewis suffers a fractured skull. And by the time the film is flown
back to New York to be shown on the air --- and it's really one of those
films where you see these people running. You see tear gas and these
billy clubs just going up and down, just beating the shit out of people.
And because John was in front, you could see him holding his head where
he'd been hit. And it was on the ground.

\begin{itemize}
\tightlist
\item
  archived recording\\
  Can we have somebody take somebody to a doctor? {[}CROWD CLAMORING{]}
\end{itemize}

brent staples

That, in my opinion --- I'm not the historian here --- but in my
opinion, that was the ultimate triumph of the nonviolent approach and
the suffering approach as he was saying.

michael barbaro

Why triumphant?

brent staples

Well, it was triumphant because even people who had tried to look away
from what was happening in the South were forced to see the long arm of
the law persecuting people publicly. Not just perscute, trying to kill
them publicly. And also, the Voting Rights Act was pending at that time.
And after this happened, Lyndon Johnson addressed a joint session of
both Houses of Congress, I believe, and said ---

\begin{itemize}
\tightlist
\item
  archived recording (lyndon b. johnson)\\
  It was more than 100 years ago that Abraham Lincoln, the great
  president of another party, signed the Emancipation Proclamation. But
  emancipation is a proclamation, and not a fact. A century has passed,
  more than 100 years since equality was promised.
\end{itemize}

brent staples

We cannot delay any longer.

\begin{itemize}
\item
  archived recording (lyndon b. johnson)\\
  A century has passed since the day of promise. And the promise is
  unkept. The time of justice has now come.

  We shall overcome.
\end{itemize}

{[}music{]}

brent staples

So the Voting Rights Act was signed later that summer in August. It
didn't take long. So when people come in to register to vote, you can no
longer ask them how many bubbles are in a bar of soap, or to name every
judge in the state. This is a big leap in our time.

michael barbaro

So quite literally, there's a straight line between the scene of what
happened on that bridge, and something John Lewis knew would be so
powerful --- the concept of nonviolent suffering --- and the legislative
remedy back in Washington that resulted.

brent staples

Yes.

michael barbaro

Because the world had seen this happen.

brent staples

Right. Yeah, I think so. But also, you begin to see the sort of apex of
this message really is `65, `66. And at some point, then, John is
replaced in the Student Nonviolent Coordinating Committee presidency by
Stokely Carmichael, a fiery orator and one of the primary enunciaters of
the Black Power movement, that was more consistent with the emerging
radicalism of the time.

michael barbaro

And that was the movement that you felt a part of.

brent staples

Yes. That's where we came into the story, you know?

michael barbaro

Brent, on some level, you and your cohort must have thought that this
approach, the John Lewis approach, had limitations. Given that by the
time you were a teenager, or maybe even entering your early 20s, there
was this new philosophy taking hold of a more elbows out, less
restrained approach. So how do you think about that?

brent staples

Well, it's interesting. And I do think about it. What had happened,
really, is every generation, until it educates itself, thinks its
experience is unique. So we thought we were unique. My cousins and I,
and we had our big meetings, and we had our press conferences. And we
had a different rhetorical stance.

But in the end, the tools were exactly the same. The tools were the
sit-in of the administration building. The tool was the sit-in in the
street that ran through campus. The tool was the building takeover.
These were the same tools, man. I mean, I had bigger hair, right? Right?
And I'll send you a picture. I'll have {[}INAUDIBLE{]}.

michael barbaro

Please do.

brent staples

I have good hair, though. But if you look back on it, the tools were the
same.

michael barbaro

And it was a foundation that had already been built, even if you didn't
see it that way at the time.

brent staples

And even if you didn't know it. You know what I'm saying? Even if you
didn't know. I keep going back to this point earlier in this story, when
they were doing the Freedom Rides in 1961. They had a big Chinese dinner
in Washington, that people were going off on these Freedom Rides. And a
lot of people wrote their wills, because they thought that they'd be
killed --- there's a chance they'd be killed and never come back. And
they referred to the meal at the Chinese restaurant as ``the last
supper.''

So these people were willing to put their lives on the line, were
willing to accept the possibility that they would be killed in the
pursuit of justice, and that their dead bodies laying out in public
would be part of a sacrifice that would advance the cause of justice.
That's profound. No?

michael barbaro

Mm-hmm. So with all that in mind, Brent, how are you thinking about John
Lewis's legacy at this moment, as we talk in the middle of yet another
critical moment in this movement, and when the work is still understood
to be very unfinished.

brent staples

You know, John Lewis, in the waning days of his life, was heartened and
overjoyed to see the global protests that unfolded after the killing of
Mr. Floyd. He talked about it as part of the extension of his work. And
one of the things he said, he essentially --- I'm paraphrasing --- he
said, the thing's out of the box now. He said, there's no going back
from this.

michael barbaro

Hm. And what about the principles of his life? How are you thinking
about those in this moment?

brent staples

Well, I think that, as you see, his point of view was borne out.

\begin{itemize}
\tightlist
\item
  archived recording\\
  {[}SHOUTING{]} Don't touch me! Don't touch me! Don't touch me.
\end{itemize}

brent staples

The other day, The New York Times had a story in which it had 64
examples, video examples of police brutalizing peaceful demonstrators.

michael barbaro

Mm-hmm, I saw it.

brent staples

Right? Now, what is that?

What that is, is what John was talking about. He was talking about this
kind of injustice perpetrated on people who did not deserve it, did not
warrant that kind of treatment. And also, we've been seeing in this
unfolding of the Floyd protests in --- it's a repeated theme in the news
stories --- white suburbanites, middle class white people, who supported
the police unquestioningly, right? They have changed their minds.

michael barbaro

Hm.

\begin{itemize}
\tightlist
\item
  archived recording\\
  {[}CROWD CLAMORING{]}
\end{itemize}

brent staples

The real persuasive thing is seeing people walking around in the street
with signs, unarmed, not doing anything untoward, and be brutalized.
That turns out to be the most persuasive thing for the society and for
the people to whom it has happening.

\begin{itemize}
\item
  archived recording 1\\
  Don't resist! Don't resist! Don't resist! Don't resist, bro.
\item
  archived recording 2\\
  I'm not!
\item
  archived recording 3\\
  Don't resist, bro.
\end{itemize}

michael barbaro

Hm. In other words, we are again seeing this idea of the beloved
community playing out, the Gandhian philosophy, this biblical approach
that you described.

brent staples

Yes.

michael barbaro

It's working. It's painfully working again.

brent staples

Yes, it's painful. Lord knows it is. It's painful. But abuse of the
people in public --- people's constitutional rights --- through violence
by police organizations, has broad, rippling consequences. It's having
broad, rippling consequences. It's beyond the people who you beat up who
now don't have confidence in the police. And John saw all that.

\begin{itemize}
\tightlist
\item
  archived recording\\
  {[}CROWD CLAMORING{]}
\end{itemize}

michael barbaro

Brent, is there anything from Lewis's memoir that you haven't already
shared that you want to leave us with?

brent staples

Well, I don't know if it fits. But perhaps we should just put that aside
and read from one of John Lewis's favorite poems. It's ``Invictus'' by
William Ernest Henley:

``Out of the night that covers me, black as the pit from pole to pole, I
thank whatever gods may be for my unconquerable soul. In the fell clutch
of circumstance, I have not winced nor cried aloud. Under the
bludgeonings of chance, my head is bloody but unbowed. Beyond this place
of wrath and tears looms but the horror of the shade. And yet the menace
of the years finds and shall find me unafraid. It matters not how
straight the gate, how charged with punishments the scroll. I am the
master of my fate. I am the captain of my soul.''

{[}music{]}

michael barbaro

Brent, thank you very much. We really appreciate your time today.

brent staples

Well, good to be with you.

{[}music{]}

\begin{itemize}
\item
  archived recording\\
  Boy, Congressman Lewis joins us now for his first network TV interview
  since the protests over the death of George Floyd began. Congressman
  John Lewis, it's so good to see you. I can't tell you, you are such a
  sight for sore eyes today. It's really good to see you. What would you
  tell, Congressman, young people, and people, quite frankly, who are
  not so young, about the best way to seek justice? You know, there's
  been a lot of controversy, a lot of talk about the looting. And we
  should stress that most of the protests were very peaceful. But there
  was some looting. There was some disruption. What would you say to
  people about the best way to achieve justice?
\item
  archived recording (john lewis)\\
  It was very moving. It was very moving to see hundreds and thousands
  of people from all over America and around the world taking to the
  streets, to the roadways, to stand up, to speak up, to speak out, to
  do what I call ``getting in trouble.''

  During the `60s, the great majority of us accepted the way of peace,
  the way of love, the philosophy and discipline of nonviolence as a way
  of life, as a way of living. There's something cleansing, something
  wholesome, about being peaceful and orderly, to stand up and with a
  sense of dignity, and a sense of pride, and never hate.

  Dr. King said over and over again, ``Hate is too heavy a burden to
  bear.'' The way of love is the much better way. And that's what we
  did. We were arrested. Yes, I was beaten, left bloody and unconscious.
  But I never became bitter or hostile, never gave up. I believed that
  somehow and some way, if it becomes necessary to use our bodies to
  have redeem the soul of a nation, then we must do it. Create a society
  at peace with yourself.
\end{itemize}

{[}music{]}

michael barbaro

That's it for ``The Daily.'' I'm Michael Barbaro. See you tomorrow.

\href{https://www.nytimes3xbfgragh.onion/column/the-daily}{\includegraphics{https://static01.graylady3jvrrxbe.onion/images/2017/01/29/podcasts/the-daily-album-art/the-daily-album-art-square320-v4.png}The
Daily}Subscribe:

\begin{itemize}
\tightlist
\item
  \href{https://itunes.apple.com/us/podcast/id1200361736}{Apple
  Podcasts}
\item
  \href{https://www.google.com/podcasts?feed=aHR0cHM6Ly9yc3MuYXJ0MTkuY29tL3RoZS1kYWlseQ\%3D\%3D}{Google
  Podcasts}
\end{itemize}

\hypertarget{the-life-and-legacy-of-john-lewis-1}{%
\section{The Life and Legacy of John
Lewis}\label{the-life-and-legacy-of-john-lewis-1}}

\hypertarget{a-look-at-the-extraordinary-life-of-the-civil-rights-icon-1}{%
\subsection{A look at the extraordinary life of the civil rights
icon.}\label{a-look-at-the-extraordinary-life-of-the-civil-rights-icon-1}}

Hosted by Michael Barbaro, produced by Lynsea Garrison and Clare
Toeniskoetter, and edited by Lisa Tobin

Transcript

transcript

Back to The Daily

bars

0:00/38:56

-0:00

transcript

\hypertarget{the-life-and-legacy-of-john-lewis-2}{%
\subsection{The Life and Legacy of John
Lewis}\label{the-life-and-legacy-of-john-lewis-2}}

\hypertarget{hosted-by-michael-barbaro-produced-by-lynsea-garrison-and-clare-toeniskoetter-and-edited-by-lisa-tobin-1}{%
\subsubsection{Hosted by Michael Barbaro, produced by Lynsea Garrison
and Clare Toeniskoetter, and edited by Lisa
Tobin}\label{hosted-by-michael-barbaro-produced-by-lynsea-garrison-and-clare-toeniskoetter-and-edited-by-lisa-tobin-1}}

\hypertarget{a-look-at-the-extraordinary-life-of-the-civil-rights-icon-2}{%
\paragraph{A look at the extraordinary life of the civil rights
icon.}\label{a-look-at-the-extraordinary-life-of-the-civil-rights-icon-2}}

Monday, July 20th, 2020

\begin{itemize}
\item
  michael barbaro\\
  From The New York Times, I'm Michael Barbaro. This is ``The Daily.''
\item
  {[}music{]}
\item
  michael barbaro\\
  Today: The life, lessons and legacy of John Lewis. I spoke with my
  colleague, Times editorial board member, Brent Staples.

  It's Monday, July 20.

  Brent, I want to start by going back with you to the time when John
  Lewis and others began engaging in nonviolent protests as part of the
  civil rights movement. Where were you during that period?
\item
  brent staples\\
  Like everyone else at the time in the middle `60s, I was sitting with
  my parents, watching television ---
\item
  archived recording 1\\
  {[}MUSIC{]} It's 11:00 p.m., and time for the reporters and the news.
\item
  archived recording 2\\
  Good evening. Bad news for Alabama today. Some school desegregation
  strategy has backfired.
\item
  archived recording 3\\
  Policemen occasionally clubbed demonstrators and used a variety of
  other tactics designed to break their spirit.
\end{itemize}

brent staples

--- and the nightly news of the scenes of people being ravaged by the
police in the South in the streets. And I became attuned to the
revolution that was unfolding in the South, where people had put
themselves in harm's way to highlight the injustice of Southern
apartheid.

\begin{itemize}
\tightlist
\item
  archived recording\\
  The Southerners, white and Negro, spattered with ketchup and mustard,
  sugar, salt and pepper, were carted off, unprotesting to jail.
\end{itemize}

brent staples

But as I became more politically active, I had difficulty --- probably
natural difficulty, understanding how one would put oneself on the line
to be actually beaten and bloodied, and what the utility of that was.
And by that time, the movement in the South had evolved some way.

\begin{itemize}
\tightlist
\item
  archived recording\\
  (SINGING) No more brothers in jail. Off the pigs! The pigs are gonna
  catch hell.
\end{itemize}

brent staples

By `66, you began to get the Black Power slogans.

\begin{itemize}
\tightlist
\item
  archived recording\\
  They don't want us to use Black Power. I got news for them.
  {[}LAUGHTER{]}
\end{itemize}

brent staples

So I came into consciousness in my teens as a Black Power figure.

\begin{itemize}
\tightlist
\item
  archived recording\\
  This is not a riot, it is a rebellion.
\end{itemize}

brent staples

It was empowering.

\begin{itemize}
\tightlist
\item
  archived recording\\
  Number two, you are to be proud of your Black brothers and sisters at
  Fifth, because a honky cop touched one of them, and they told them,
  you've got to touch all of us.
\end{itemize}

brent staples

But it took some time into my late teens to begin to understand what had
happened coming up to that. And that is when I become aware of what had
passed before when I was a younger kid.

michael barbaro

In other words, it took you some time to understand why the nonviolent
figures had taken the approach that they had taken.

brent staples

Yes.

michael barbaro

So I want you to take us through John Lewis's life and how this
philosophy you just described was shaped and how it evolved. So where
does that story begin?

brent staples

John Lewis, he grows up in rural Alabama near Troy, Alabama. His parents
initially were sharecroppers. And you know, sharecropping was a
successor form of slavery. So John was born into that. And his parents,
they were lucky they saved enough money to buy a farm.

But he graduated high school, segregated high school, and wanted to go
to Troy State College, which didn't admit Black people. And he applied
and sent in his information. And he never heard back. And he wrote a
letter at one point to Martin Luther King. And I presume he wanted some
help in desegregating Troy State.

And Martin Luther King sent him a roundtrip bus ticket. And he went and
he met King, and they formed a relationship. The substance of that, I
don't know, but it was extremely influential for John. And he left Troy
not long after that and moved to Nashville.

{[}music{]}

brent staples

He went to seminary in Nashville. And there, he met some of the early
civil rights figures. He met James Lawson --- Reverend James Lawson ---
who was kind of a philosopher of nonviolent resistance. Lawson had
studied Gandhi's nonviolent movement and the strategies that Gandhi had
deployed against Britain during the colonial period. And he'd come back
with a deep sense of what the philosophy was and how powerful
nonviolence could be.

michael barbaro

And what exactly is the philosophy?

brent staples

I have John's memoir here. In 1958, I think it is, Jim Lawson mentions
to him the idea of, quote, ``redemptive suffering.'' And he explains
that it affects not only ourselves, but it touches and changes those
around us as well. It opens us and those around us to a force beyond
ourselves. A force that is right and moral. The force of righteous truth
that is the basis of human conscious. Suffering puts us and those around
us in touch with our consciences. It open and touches our hearts. It
makes us feel compassion where we need to and guilt if we must. So this
idea, to him, this redemptive suffering, it is at the heart of the
philosophy of nonviolent protest. At the very heart of it. This is a
good paragraph from the book:

``One method of practicing this approach, when faced with a hateful,
angry, aggressive, even despicable person, is to imagine that person ---
actually visualize him or her --- as an infant, as a baby. If you can
see this full-grown attacker who faces you as a pure, innocent child
that he or she once was, it is not hard to find compassion in your
heart.''

But then it wasn't just a tactic. It was a way of life. It was embracing
the biblical prescription that one must love one's enemies. That's a
biblical prescription. And it's the hardest thing in the world to carry
out.

michael barbaro

Well, so how do we start to see this get carried out among Lewis and
these seminary students in Nashville?

brent staples

Well, Nashville was itself at the time another southern town, where if
you went into a restaurant and sat down the people would just look at
you and the waitresses would say, sorry, this place doesn't serve
{[}EXPLETIVE{]}. And that would generally be the end of it. But these
students came in that place and sat down and asked to be fed. And when
they were told that they were not served, they stayed.

And they took a lot of abuse from it. And people spat upon them, beat
them, battered them, and poured condiments over their heads. All kinds
of things. And I remember our friend, David Halberstam, our former
colleague, was working at a newspaper there at the time. He was working
at The Nashville Tennesseean. And this is one of the things he wrote:

``The protest had been conducted with exceptional dignity, and
gradually, one image had come into prevail --- that of elegant,
courteous, young Black people, holding to their Gandhian principles,
seeking the most elemental of rights, while being assaulted by young
white hoodlums who beat them up and on occasion extinguished cigarettes
on their bodies.''

So you see John Lewis and others being carried away in these, really,
suits and ties and crisp white shirt. And basically refusing to walk
themselves, being completely passive and nonresistant. And this worked
out extremely --- I mean, in a very short period of time, it worked out
extremely well in Nashville. After three months of sit-ins, the city
basically caved and became the first major Southern city to begin
desegregating public facilities.

michael barbaro

So very early on, the protest that John Lewis is beginning to
participate in, after he meets Martin Luther King and begins to
understand the strategy, they are starting to show real signs of
effectiveness.

brent staples

Oh, yeah.

michael barbaro

These peaceful, nonviolent protests.

brent staples

Exactly. Yeah. This is astonishing thing, to me, to this day, to
practice the non-violent approach to life. To really embrace it, one
needs to understand that the person who was extinguishing a cigarette in
your throat because you want to sit down at a luncheon counter is as
much a victim as you are. What John was saying --- you, in pursuit of
justice, you cannot let violence win your heart. That if you do that,
you're surrendering, really, to the dark force that you're trying to
defeat.

michael barbaro

Brent, you're reading from a memoir that is written in the later years
of John Lewis's life. But my understanding is that this philosophy that
he embraced and that he practiced, it was not entirely a foregone
conclusion that this would be the way that it went, and that the March
on Washington is an example of a moment where we see a young John Lewis
grappling with which path he's going to take. Can you tell us the story
of that speech?

brent staples

Well, you see, what I finally figured out is that by the time John
Lewis, at the age of 23, gets to the march on Washington --- this is the
most important public gathering of Black people in the century --- he
has already been on the Freedom Rides. Integrated groups have taken
buses into the deep south to test laws that forbid segregation on
interstate transport. He's been arrested on those trips for going into
white-only bathrooms. He's been beaten just for being on buses with
white people. In the end, John ended up being arrested, like, 40 times.

michael barbaro

Right.

brent staples

And if you look at some of the pictures of the mugshots when he's
arrested, you can see him smiling because he's basically saying, you
think you're afflicting me, but you're playing into what I want to do.
But he's still 23 years old. That's all he is.

And he basically comes into Washington with a speech. And this one
somehow found its way into public. And one of the striking things about
it, he tells people to get into the street and stay in the street until
the revolution is finished. And he names the sort of racist
segregationist senators by name. And state governors, too.

michael barbaro

Wow.

brent staples

I'm going to read from it.

``We won't stop now. All the forces of Eastland, Barnett, Wallace,
Thurman won't stop this revolution. The time will come when we will not
confine our marching to Washington. We will march through the South,
through the heart of Dixie, the way Sherman did. We shall pursue our own
scorched earth policy and burn Jim Crow to the ground nonviolently.''

michael barbaro

Wow. And to explain that reference, he's referring to the Union General
who literally burns large sections of the South during the Civil War.

brent staples

Yes, he's referring to William Tecumseh Sherman. ``We shall pursue our
own scorched earth policy and burn Jim Crow to the ground nonviolently.
We shall fragment the south into a thousand pieces and put them back
together in the image of democracy. We will make the action of the past
few months look petty. And I say to you, wake up, America.''

michael barbaro

Brent, that language doesn't sound, even as it invokes the word,
nonviolent. So what do you think that he meant by those words in that
draft?

brent staples

Well, I've come to figure out what he meant. Now, understand, as I was
saying earlier, by the time he renders this speech, he has become
steeped in the nonviolent impulse. But his frame, he was portraying it
as a forceful measure that could be as powerful and changing, as was the
sweep of Sherman through South Carolina and Georgia. What you have here
is, John is working at a very high concept here, right? He's working at
a high concept. He's saying, we can be --- he was essentially arguing
that nonviolent protests could be transformative. As transformative and
as disruptive as war as carried out by the most feared general in the
Union Army. That is itself a very powerful metaphor. And it's a
testament to his beliefs and what his approach could do. And he was
calling upon hundreds of thousands of people to come out into the street
and make that a reality. But that's a high concept. And on the evening
news, you can imagine you'd end up with a snippet of a scorched earth
Sherman burning Atlanta again.

michael barbaro

Right, it might get lost in translation. It might actually undermine the
very thing he's trying to promote.

brent staples

Exactly. So basically, A. Philip Randolph, Martin Luther King, those
guys prevailed on him to make some changes in it. They were talking
about, let's not do anything to just not give them a sound bite that's
going to give us trouble. I can hear him saying it. I don't have the
tape, but I can hear them saying to John, let's not give them a sound
bite that's going to give us trouble. You can say it's a revolution. You
can call people on the street. You can even call them the Black masses,
if you want to, even though that sounds like communism, right? You could
say those things. But let's leave off Sherman for next time. Right?
{[}LAUGHS{]} So Sherman goes out.

\begin{itemize}
\tightlist
\item
  archived recording (john lewis)\\
  We march today for jobs and freedom. But we have nothing to be proud
  of. For hundreds of thousands of our brothers are not here.
\end{itemize}

brent staples

He talks about marching.

\begin{itemize}
\tightlist
\item
  archived recording (john lewis)\\
  The time will come. We will not confine our marching to Washington. We
  will march through the South, through the streets of Jackson ---
\end{itemize}

brent staples

We will march through the street of Jackson, through the streets of
Danville, through the streets of Cambridge, through the streets of
Birmingham.

\begin{itemize}
\item
  archived recording (john lewis)\\
  But we will march with the spirit of love and with the spirit of
  dignity that we have shown here today. {[}APPLAUSE{]}

  By the forces of our demand, our determination and our numbers, we
  shall splinter the segregated South into a thousand pieces and put
  them together in the image of God and democracy. We must say, Wake up,
  America! Wake up!'' For we cannot stop, and we will not and cannot be
  patient. {[}APPLAUSE{]}
\end{itemize}

brent staples

So this is pretty much the same. But Sherman is missing.

michael barbaro

So in terms of thinking about the speech that John Lewis drafted versus
the one he delivered, on that historic day, you're saying it's not that
the earlier version of the speech shows John Lewis questioning the
nonviolent approach. It's that he believed in the nonviolent approach,
but that the language he contemplated using --- his belief that the
power of that approach could be as powerful as burning --- that was
determined to be potentially counterproductive to the nonviolent
approach he believed in.

brent staples

Exactly. But he talks about going back to his hotel room after that
first conversation and just being livid because, of course, it's a
23-year-old man's speech. And a 23-year-old man who had been beaten to
an inch of his life while fighting for dignity for Black people. I'm
sure he felt entitled to say any damn thing he wanted to. Because he had
the credibility of the streets behind him and the people in the Douth
beginning to know who he was. And they were going to really know who he
was, come two years later at the voting rights march at Selma.

{[}music{]}

michael barbaro

We'll be right back.

{[}music{]}

michael barbaro

Brent, tell me about that. I mean, let's talk about what happens after
the March on Washington, after John Lewis's rhetorical wings are ever so
slightly clipped, but he does deliver the essential message. How do we
see this concept of the power of nonviolence actually play out over the
next couple of years?

brent staples

The next big data point becomes the voting rights struggle in Selma,
Alabama. Now it's important, I think, to dilate for just a second for
the modern listener.

The modern listener needs to understand that in voting arrangements in
the South, before the Voting Rights Act, local registrars had complete
authority to do whatever they wanted with people who came in to register
to vote. They could give you a test and then say you failed it so you
can't register. And they did that all day long, all day, every day, to
Black people.

In Virginia, a college-educated woman, Black woman, who I believe was a
teacher, went into register at one point and filled the application. And
the registrar handed her a literacy test. You know what it consisted of?
A blank sheet of paper. He asked her, what does this say? And she looked
at it and handed it back, and she said nothing. He said, you're wrong.
You fail. You can't register.

Elsewhere in the South, they might ask you if you came in to register to
vote, how many bubbles are there in a bar of soap?

Elsewhere, a famous example in the film ``Selma,'' where a woman had
come in to register one year, and they asked her, how many judges are
there in the state of Alabama? And she didn't know. They said, well, you
fail. And she came back the second year. And he said, how many county
judges are there in Alabama? And I think it was 67. And she said 67. And
he said now, before you register, you have to name them.

michael barbaro

Aw.

brent staples

So this was what life was like for Black people seeking to vote in the
South. Now John Lewis's organization, the Student Nonviolent
Coordinating Committee, had already set up a voting rights project in
Selma and had been working on that. But it came to fruition in 1965,
where people had been fed up. And so they staged a march from Selma to
Montgomery to protest in favor of voting rights for Black people.

And then that fateful day, on the Edmund Pettus Bridge, they were
walking. State troopers came out and said, this is unlawful.

\begin{itemize}
\tightlist
\item
  archived recording\\
  It would be detrimental to your safety to continue this march. And I'm
  saying that this is an unlawful assembly. You have to disperse. You
  are ordered to disperse. Go home, or go to your church. This march
  will not continue.
\end{itemize}

brent staples

You cannot march.

\begin{itemize}
\item
  archived recording\\
  Troopers, here, advance toward the gate.

  See that they disperse.
\end{itemize}

brent staples

And then John said to the sheriff. He said, can I have a word? Because
you know he's in the front with his little off-white trenchcoat on. And
now there's a thing about this trenchcoat, right? It's very light, so
you can see him standing out from everyone else. And the other thing
about it is, once blood gets on it, you can really see it. So I'm sure
that that was premeditated. So he comes out front, and he says, can I
have a word? No, you can't have a word. And the troopers begin to
advance. And they beat holy hell out of those people. They sent 58
people to the hospital.

John Lewis suffers a fractured skull. And by the time the film is flown
back to New York to be shown on the air --- and it's really one of those
films where you see these people running. You see tear gas and these
billy clubs just going up and down, just beating the shit out of people.
And because John was in front, you could see him holding his head where
he'd been hit. And it was on the ground.

\begin{itemize}
\tightlist
\item
  archived recording\\
  Can we have somebody take somebody to a doctor? {[}CROWD CLAMORING{]}
\end{itemize}

brent staples

That, in my opinion --- I'm not the historian here --- but in my
opinion, that was the ultimate triumph of the nonviolent approach and
the suffering approach as he was saying.

michael barbaro

Why triumphant?

brent staples

Well, it was triumphant because even people who had tried to look away
from what was happening in the South were forced to see the long arm of
the law persecuting people publicly. Not just perscute, trying to kill
them publicly. And also, the Voting Rights Act was pending at that time.
And after this happened, Lyndon Johnson addressed a joint session of
both Houses of Congress, I believe, and said ---

\begin{itemize}
\tightlist
\item
  archived recording (lyndon b. johnson)\\
  It was more than 100 years ago that Abraham Lincoln, the great
  president of another party, signed the Emancipation Proclamation. But
  emancipation is a proclamation, and not a fact. A century has passed,
  more than 100 years since equality was promised.
\end{itemize}

brent staples

We cannot delay any longer.

\begin{itemize}
\item
  archived recording (lyndon b. johnson)\\
  A century has passed since the day of promise. And the promise is
  unkept. The time of justice has now come.

  We shall overcome.
\end{itemize}

{[}music{]}

brent staples

So the Voting Rights Act was signed later that summer in August. It
didn't take long. So when people come in to register to vote, you can no
longer ask them how many bubbles are in a bar of soap, or to name every
judge in the state. This is a big leap in our time.

michael barbaro

So quite literally, there's a straight line between the scene of what
happened on that bridge, and something John Lewis knew would be so
powerful --- the concept of nonviolent suffering --- and the legislative
remedy back in Washington that resulted.

brent staples

Yes.

michael barbaro

Because the world had seen this happen.

brent staples

Right. Yeah, I think so. But also, you begin to see the sort of apex of
this message really is `65, `66. And at some point, then, John is
replaced in the Student Nonviolent Coordinating Committee presidency by
Stokely Carmichael, a fiery orator and one of the primary enunciaters of
the Black Power movement, that was more consistent with the emerging
radicalism of the time.

michael barbaro

And that was the movement that you felt a part of.

brent staples

Yes. That's where we came into the story, you know?

michael barbaro

Brent, on some level, you and your cohort must have thought that this
approach, the John Lewis approach, had limitations. Given that by the
time you were a teenager, or maybe even entering your early 20s, there
was this new philosophy taking hold of a more elbows out, less
restrained approach. So how do you think about that?

brent staples

Well, it's interesting. And I do think about it. What had happened,
really, is every generation, until it educates itself, thinks its
experience is unique. So we thought we were unique. My cousins and I,
and we had our big meetings, and we had our press conferences. And we
had a different rhetorical stance.

But in the end, the tools were exactly the same. The tools were the
sit-in of the administration building. The tool was the sit-in in the
street that ran through campus. The tool was the building takeover.
These were the same tools, man. I mean, I had bigger hair, right? Right?
And I'll send you a picture. I'll have {[}INAUDIBLE{]}.

michael barbaro

Please do.

brent staples

I have good hair, though. But if you look back on it, the tools were the
same.

michael barbaro

And it was a foundation that had already been built, even if you didn't
see it that way at the time.

brent staples

And even if you didn't know it. You know what I'm saying? Even if you
didn't know. I keep going back to this point earlier in this story, when
they were doing the Freedom Rides in 1961. They had a big Chinese dinner
in Washington, that people were going off on these Freedom Rides. And a
lot of people wrote their wills, because they thought that they'd be
killed --- there's a chance they'd be killed and never come back. And
they referred to the meal at the Chinese restaurant as ``the last
supper.''

So these people were willing to put their lives on the line, were
willing to accept the possibility that they would be killed in the
pursuit of justice, and that their dead bodies laying out in public
would be part of a sacrifice that would advance the cause of justice.
That's profound. No?

michael barbaro

Mm-hmm. So with all that in mind, Brent, how are you thinking about John
Lewis's legacy at this moment, as we talk in the middle of yet another
critical moment in this movement, and when the work is still understood
to be very unfinished.

brent staples

You know, John Lewis, in the waning days of his life, was heartened and
overjoyed to see the global protests that unfolded after the killing of
Mr. Floyd. He talked about it as part of the extension of his work. And
one of the things he said, he essentially --- I'm paraphrasing --- he
said, the thing's out of the box now. He said, there's no going back
from this.

michael barbaro

Hm. And what about the principles of his life? How are you thinking
about those in this moment?

brent staples

Well, I think that, as you see, his point of view was borne out.

\begin{itemize}
\tightlist
\item
  archived recording\\
  {[}SHOUTING{]} Don't touch me! Don't touch me! Don't touch me.
\end{itemize}

brent staples

The other day, The New York Times had a story in which it had 64
examples, video examples of police brutalizing peaceful demonstrators.

michael barbaro

Mm-hmm, I saw it.

brent staples

Right? Now, what is that?

What that is, is what John was talking about. He was talking about this
kind of injustice perpetrated on people who did not deserve it, did not
warrant that kind of treatment. And also, we've been seeing in this
unfolding of the Floyd protests in --- it's a repeated theme in the news
stories --- white suburbanites, middle class white people, who supported
the police unquestioningly, right? They have changed their minds.

michael barbaro

Hm.

\begin{itemize}
\tightlist
\item
  archived recording\\
  {[}CROWD CLAMORING{]}
\end{itemize}

brent staples

The real persuasive thing is seeing people walking around in the street
with signs, unarmed, not doing anything untoward, and be brutalized.
That turns out to be the most persuasive thing for the society and for
the people to whom it has happening.

\begin{itemize}
\item
  archived recording 1\\
  Don't resist! Don't resist! Don't resist! Don't resist, bro.
\item
  archived recording 2\\
  I'm not!
\item
  archived recording 3\\
  Don't resist, bro.
\end{itemize}

michael barbaro

Hm. In other words, we are again seeing this idea of the beloved
community playing out, the Gandhian philosophy, this biblical approach
that you described.

brent staples

Yes.

michael barbaro

It's working. It's painfully working again.

brent staples

Yes, it's painful. Lord knows it is. It's painful. But abuse of the
people in public --- people's constitutional rights --- through violence
by police organizations, has broad, rippling consequences. It's having
broad, rippling consequences. It's beyond the people who you beat up who
now don't have confidence in the police. And John saw all that.

\begin{itemize}
\tightlist
\item
  archived recording\\
  {[}CROWD CLAMORING{]}
\end{itemize}

michael barbaro

Brent, is there anything from Lewis's memoir that you haven't already
shared that you want to leave us with?

brent staples

Well, I don't know if it fits. But perhaps we should just put that aside
and read from one of John Lewis's favorite poems. It's ``Invictus'' by
William Ernest Henley:

``Out of the night that covers me, black as the pit from pole to pole, I
thank whatever gods may be for my unconquerable soul. In the fell clutch
of circumstance, I have not winced nor cried aloud. Under the
bludgeonings of chance, my head is bloody but unbowed. Beyond this place
of wrath and tears looms but the horror of the shade. And yet the menace
of the years finds and shall find me unafraid. It matters not how
straight the gate, how charged with punishments the scroll. I am the
master of my fate. I am the captain of my soul.''

{[}music{]}

michael barbaro

Brent, thank you very much. We really appreciate your time today.

brent staples

Well, good to be with you.

{[}music{]}

\begin{itemize}
\item
  archived recording\\
  Boy, Congressman Lewis joins us now for his first network TV interview
  since the protests over the death of George Floyd began. Congressman
  John Lewis, it's so good to see you. I can't tell you, you are such a
  sight for sore eyes today. It's really good to see you. What would you
  tell, Congressman, young people, and people, quite frankly, who are
  not so young, about the best way to seek justice? You know, there's
  been a lot of controversy, a lot of talk about the looting. And we
  should stress that most of the protests were very peaceful. But there
  was some looting. There was some disruption. What would you say to
  people about the best way to achieve justice?
\item
  archived recording (john lewis)\\
  It was very moving. It was very moving to see hundreds and thousands
  of people from all over America and around the world taking to the
  streets, to the roadways, to stand up, to speak up, to speak out, to
  do what I call ``getting in trouble.''

  During the `60s, the great majority of us accepted the way of peace,
  the way of love, the philosophy and discipline of nonviolence as a way
  of life, as a way of living. There's something cleansing, something
  wholesome, about being peaceful and orderly, to stand up and with a
  sense of dignity, and a sense of pride, and never hate.

  Dr. King said over and over again, ``Hate is too heavy a burden to
  bear.'' The way of love is the much better way. And that's what we
  did. We were arrested. Yes, I was beaten, left bloody and unconscious.
  But I never became bitter or hostile, never gave up. I believed that
  somehow and some way, if it becomes necessary to use our bodies to
  have redeem the soul of a nation, then we must do it. Create a society
  at peace with yourself.
\end{itemize}

{[}music{]}

michael barbaro

That's it for ``The Daily.'' I'm Michael Barbaro. See you tomorrow.

Previous

More episodes ofThe Daily

\href{https://www.nytimes3xbfgragh.onion/2020/07/31/podcasts/the-daily/vanessa-guillen-military-metoo.html?action=click\&module=audio-series-bar\&region=header\&pgtype=Article}{\includegraphics{https://static01.graylady3jvrrxbe.onion/images/2020/07/12/us/politics/31daily/00dc-army-metoo-thumbLarge.jpg}}

July 31, 2020A \#MeToo Moment in the Military

\href{https://www.nytimes3xbfgragh.onion/2020/07/30/podcasts/the-daily/congress-facebook-amazon-google-apple.html?action=click\&module=audio-series-bar\&region=header\&pgtype=Article}{\includegraphics{https://static01.graylady3jvrrxbe.onion/images/2020/07/30/reader-center/30daily/merlin_175077825_5ebc931b-baa1-489a-960c-34e4d845e997-thumbLarge.jpg}}

July 30, 2020The Big Tech Hearing

\href{https://www.nytimes3xbfgragh.onion/2020/07/29/podcasts/the-daily/china-trump-foreign-policy.html?action=click\&module=audio-series-bar\&region=header\&pgtype=Article}{\includegraphics{https://static01.graylady3jvrrxbe.onion/images/2020/07/26/world/29daily/00china-us-clash1-thumbLarge.jpg}}

July 29, 2020~~•~ 28:40Confronting China

\href{https://www.nytimes3xbfgragh.onion/2020/07/28/podcasts/the-daily/unemployment-benefits-coronavirus.html?action=click\&module=audio-series-bar\&region=header\&pgtype=Article}{\includegraphics{https://static01.graylady3jvrrxbe.onion/images/2020/07/23/business/28daily/23virus-uiexplain1-thumbLarge.jpg}}

July 28, 2020~~•~ 26:13Why \$600 Checks Are Tearing Republicans Apart

\href{https://www.nytimes3xbfgragh.onion/2020/07/27/podcasts/the-daily/new-york-hospitals-covid.html?action=click\&module=audio-series-bar\&region=header\&pgtype=Article}{\includegraphics{https://static01.graylady3jvrrxbe.onion/images/2020/07/27/world/27daily-hospitals/27daily-hospitals-thumbLarge.jpg}}

July 27, 2020~~•~ 33:28The Mistakes New York Made

\href{https://www.nytimes3xbfgragh.onion/2020/07/26/podcasts/the-daily/the-accusation-the-sunday-read.html?action=click\&module=audio-series-bar\&region=header\&pgtype=Article}{\includegraphics{https://static01.graylady3jvrrxbe.onion/images/2020/03/22/magazine/26audm-2/22mag-titleix-thumbLarge.jpg}}

July 26, 2020The Sunday Read: `The Accusation'

\href{https://www.nytimes3xbfgragh.onion/2020/07/24/podcasts/the-daily/mlb-baseball-season-coronavirus.html?action=click\&module=audio-series-bar\&region=header\&pgtype=Article}{\includegraphics{https://static01.graylady3jvrrxbe.onion/images/2020/07/22/sports/24daily/22mlb-previewlede1-thumbLarge.jpg}}

July 24, 2020~~•~ 45:34The Battle for a Baseball Season

\href{https://www.nytimes3xbfgragh.onion/2020/07/23/podcasts/the-daily/portland-protests.html?action=click\&module=audio-series-bar\&region=header\&pgtype=Article}{\includegraphics{https://static01.graylady3jvrrxbe.onion/images/2020/07/22/us/23daily-image/22portland-tactics02-thumbLarge.jpg}}

July 23, 2020~~•~ 30:04The Showdown in Portland

\href{https://www.nytimes3xbfgragh.onion/2020/07/22/podcasts/the-daily/school-reopenings-coronavirus.html?action=click\&module=audio-series-bar\&region=header\&pgtype=Article}{\includegraphics{https://static01.graylady3jvrrxbe.onion/images/2020/07/12/science/22daily/00virus-schools-reopen01-thumbLarge.jpg}}

July 22, 2020~~•~ 27:24The Science of School Reopenings

\href{https://www.nytimes3xbfgragh.onion/2020/07/21/podcasts/the-daily/coronavirus-vaccine.html?action=click\&module=audio-series-bar\&region=header\&pgtype=Article}{\includegraphics{https://static01.graylady3jvrrxbe.onion/images/2020/07/19/science/21daily/00VIRUS-VAX-DOUBTS1-thumbLarge.jpg}}

July 21, 2020~~•~ 29:14The Vaccine Trust Problem

\href{https://www.nytimes3xbfgragh.onion/2020/07/20/podcasts/the-daily/john-lewis.html?action=click\&module=audio-series-bar\&region=header\&pgtype=Article}{\includegraphics{https://static01.graylady3jvrrxbe.onion/images/2020/01/07/obituaries/20thedaily_lewis/00Lewis-John13-thumbLarge.jpg}}

July 20, 2020~~•~ 38:56The Life and Legacy of John Lewis

\href{https://www.nytimes3xbfgragh.onion/2020/07/19/podcasts/the-daily/lottery-winner-scam.html?action=click\&module=audio-series-bar\&region=header\&pgtype=Article}{\includegraphics{https://static01.graylady3jvrrxbe.onion/images/2018/05/05/magazine/31audm-image/05mag-lottery-image1-thumbLarge-v4.png}}

July 19, 2020~~•~ 45:27The Sunday Read: `The Man Who Cracked the
Lottery'

\href{https://www.nytimes3xbfgragh.onion/column/the-daily}{See All
Episodes ofThe Daily}

Next

July 20, 2020

\begin{itemize}
\item
\item
\item
\item
\item
\item
\end{itemize}

\emph{\textbf{Listen and subscribe to our podcast from your mobile
device:}}\\
\textbf{\href{https://itunes.apple.com/us/podcast/the-daily/id1200361736?mt=2}{\emph{Via
Apple Podcasts}}} \emph{\textbf{\textbar{}}}
\textbf{\href{https://open.spotify.com/show/3IM0lmZxpFAY7CwMuv9H4g?si=SfuMSC55R1qprFsRZU3_zw}{\emph{Via
Spotify}}} \emph{\textbf{\textbar{}}}
\textbf{\href{http://www.stitcher.com/podcast/the-new-york-times/the-daily-10}{\emph{Via
Stitcher}}}

\emph{This episode includes disturbing language, including racial
slurs.}

Representative John Lewis, a stalwart of the civil rights era, died on
Friday. We take a look at his life, lessons and legacy.

\textbf{On today's episode:}

\begin{itemize}
\tightlist
\item
  \href{https://www.nytimes3xbfgragh.onion/by/brent-staples}{Brent
  Staples}, a member of the Times editorial board
\end{itemize}

\includegraphics{https://static01.graylady3jvrrxbe.onion/images/2020/01/07/obituaries/20thedaily_lewis/merlin_166560165_9904ae50-e859-46cc-8c1b-892d204bbb17-articleLarge.jpg?quality=75\&auto=webp\&disable=upscale}

\textbf{Background reading:}

\begin{itemize}
\item
  Mr. Lewis, a son of sharecroppers and an apostle of nonviolence who
  was bloodied at Selma, Ala., and across the Jim Crow South in the
  historic struggle for racial equality, and who then carried a mantle
  of moral authority into Congress,
  \href{https://www.nytimes3xbfgragh.onion/2020/07/17/us/john-lewis-dead.html}{died
  on Friday}. He was 80.
\item
  Bipartisan
  \href{https://www.nytimes3xbfgragh.onion/2020/07/18/us/politics/john-lewis-dies-reaction.html}{praise
  poured in for the civil rights leader}, as friends, colleagues and
  admirers reached for the appropriate superlatives to sum up an
  extraordinary life.
\item
  Mr. Lewis risked his life for justice,
  \href{https://www.nytimes3xbfgragh.onion/2020/07/17/opinion/john-lewis.html}{The
  Times's editorial board wrote}.
\end{itemize}

\emph{Tune in, and tell us what you think. Email us at}
\href{mailto:thedaily@NYTimes.com}{\emph{thedaily@NYTimes.com}}\emph{.
Follow Michael Barbaro on Twitter:}
\href{https://twitter.com/mikiebarb}{\emph{@mikiebarb}}\emph{. And if
you're interested in advertising with ``The Daily,'' write to us at}
\href{mailto:thedaily-ads@NYTimes.com}{\emph{thedaily-ads@NYTimes.com}}\emph{.}

Brent Staples contributed reporting.

``The Daily'' is made by Theo Balcomb, Andy Mills, Lisa Tobin, Rachel
Quester, Lynsea Garrison, Annie Brown, Clare Toeniskoetter, Paige
Cowett, Michael Simon Johnson, Brad Fisher, Larissa Anderson, Wendy
Dorr, Chris Wood, Jessica Cheung, Stella Tan, Alexandra Leigh Young,
Jonathan Wolfe, Lisa Chow, Eric Krupke, Marc Georges, Luke Vander Ploeg,
Adizah Eghan, Kelly Prime, Julia Longoria, Sindhu Gnanasambandan, M.J.
Davis Lin, Austin Mitchell, Sayre Quevedo, Neena Pathak, Dan Powell,
Dave Shaw, Sydney Harper, Daniel Guillemette, Hans Buetow, Robert
Jimison, Mike Benoist, Bianca Giaever and Asthaa Chaturvedi. Our theme
music is by Jim Brunberg and Ben Landsverk of Wonderly. Special thanks
to Sam Dolnick, Mikayla Bouchard, Lauren Jackson, Julia Simon, Mahima
Chablani and Nora Keller.

Advertisement

\protect\hyperlink{after-bottom}{Continue reading the main story}

\hypertarget{site-index}{%
\subsection{Site Index}\label{site-index}}

\hypertarget{site-information-navigation}{%
\subsection{Site Information
Navigation}\label{site-information-navigation}}

\begin{itemize}
\tightlist
\item
  \href{https://help.nytimes3xbfgragh.onion/hc/en-us/articles/115014792127-Copyright-notice}{©~2020~The
  New York Times Company}
\end{itemize}

\begin{itemize}
\tightlist
\item
  \href{https://www.nytco.com/}{NYTCo}
\item
  \href{https://help.nytimes3xbfgragh.onion/hc/en-us/articles/115015385887-Contact-Us}{Contact
  Us}
\item
  \href{https://www.nytco.com/careers/}{Work with us}
\item
  \href{https://nytmediakit.com/}{Advertise}
\item
  \href{http://www.tbrandstudio.com/}{T Brand Studio}
\item
  \href{https://www.nytimes3xbfgragh.onion/privacy/cookie-policy\#how-do-i-manage-trackers}{Your
  Ad Choices}
\item
  \href{https://www.nytimes3xbfgragh.onion/privacy}{Privacy}
\item
  \href{https://help.nytimes3xbfgragh.onion/hc/en-us/articles/115014893428-Terms-of-service}{Terms
  of Service}
\item
  \href{https://help.nytimes3xbfgragh.onion/hc/en-us/articles/115014893968-Terms-of-sale}{Terms
  of Sale}
\item
  \href{https://spiderbites.nytimes3xbfgragh.onion}{Site Map}
\item
  \href{https://help.nytimes3xbfgragh.onion/hc/en-us}{Help}
\item
  \href{https://www.nytimes3xbfgragh.onion/subscription?campaignId=37WXW}{Subscriptions}
\end{itemize}
