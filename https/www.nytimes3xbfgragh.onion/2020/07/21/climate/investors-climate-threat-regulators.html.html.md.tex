Sections

SEARCH

\protect\hyperlink{site-content}{Skip to
content}\protect\hyperlink{site-index}{Skip to site index}

\href{https://www.nytimes3xbfgragh.onion/section/climate}{Climate}

\href{https://myaccount.nytimes3xbfgragh.onion/auth/login?response_type=cookie\&client_id=vi}{}

\href{https://www.nytimes3xbfgragh.onion/section/todayspaper}{Today's
Paper}

\href{/section/climate}{Climate}\textbar{}Climate Change Poses `Systemic
Threat' to the Economy, Big Investors Warn

\url{https://nyti.ms/32EujUN}

\begin{itemize}
\item
\item
\item
\item
\item
\end{itemize}

\href{https://www.nytimes3xbfgragh.onion/section/climate?action=click\&pgtype=Article\&state=default\&region=TOP_BANNER\&context=storylines_menu}{Climate
and Environment}

\begin{itemize}
\tightlist
\item
  \href{https://www.nytimes3xbfgragh.onion/2020/07/30/climate/sea-level-inland-floods.html?action=click\&pgtype=Article\&state=default\&region=TOP_BANNER\&context=storylines_menu}{Rising
  Seas}
\item
  \href{https://www.nytimes3xbfgragh.onion/interactive/2020/climate/trump-environment-rollbacks.html?action=click\&pgtype=Article\&state=default\&region=TOP_BANNER\&context=storylines_menu}{Trump's
  Changes}
\item
  \href{https://www.nytimes3xbfgragh.onion/interactive/2020/04/19/climate/climate-crash-course-1.html?action=click\&pgtype=Article\&state=default\&region=TOP_BANNER\&context=storylines_menu}{Climate
  101}
\item
  \href{https://www.nytimes3xbfgragh.onion/interactive/2018/08/30/climate/how-much-hotter-is-your-hometown.html?action=click\&pgtype=Article\&state=default\&region=TOP_BANNER\&context=storylines_menu}{Is
  Your Hometown Hotter?}
\item
  \href{https://www.nytimes3xbfgragh.onion/newsletters/climate-change?action=click\&pgtype=Article\&state=default\&region=TOP_BANNER\&context=storylines_menu}{Newsletter}
\end{itemize}

Advertisement

\protect\hyperlink{after-top}{Continue reading the main story}

Supported by

\protect\hyperlink{after-sponsor}{Continue reading the main story}

\hypertarget{climate-change-poses-systemic-threat-to-the-economy-big-investors-warn}{%
\section{Climate Change Poses `Systemic Threat' to the Economy, Big
Investors
Warn}\label{climate-change-poses-systemic-threat-to-the-economy-big-investors-warn}}

Financial regulators should act to avoid economic disaster, according to
a letter from pension funds and other investors representing almost \$1
trillion in assets.

\includegraphics{https://static01.graylady3jvrrxbe.onion/images/2020/07/21/climate/21CLI-REGULATORS/21CLI-REGULATORS-articleLarge.jpg?quality=75\&auto=webp\&disable=upscale}

\href{https://www.nytimes3xbfgragh.onion/by/christopher-flavelle}{\includegraphics{https://static01.graylady3jvrrxbe.onion/images/2019/06/28/climate/author-chris-flavelle/author-chris-flavelle-thumbLarge-v3.png}}

By
\href{https://www.nytimes3xbfgragh.onion/by/christopher-flavelle}{Christopher
Flavelle}

\begin{itemize}
\item
  July 21, 2020
\item
  \begin{itemize}
  \item
  \item
  \item
  \item
  \item
  \end{itemize}
\end{itemize}

WASHINGTON --- Climate change threatens to create turmoil in the
financial markets, and the Federal Reserve and other regulators must act
to avoid an economic disaster, according to a letter sent on Tuesday by
a group of large investors.

``The climate crisis poses a systemic threat to financial markets and
the real economy, with significant disruptive consequences on asset
valuations and our nation's economic stability,'' reads the letter,
which was signed by more than three dozen pension plans, fund managers
and other financial institutions that together manage almost \$1
trillion in assets.

That financial threat, combined with the physical risks posed by climate
change, may create ``disastrous impacts the likes of which we haven't
seen before,'' the letter says. It urges the Fed, the Securities and
Exchange Commission and other agencies to ``explicitly integrate climate
change across your mandates.''

Investors worry that if regulators do not act, climate change may cause
the price of some companies to fall suddenly, the effects of which may
ricochet through the economy. Providing more information about that risk
--- for example, by requiring companies to disclose more about their
greenhouse gas emissions, or which of their facilities are at risk from
rising seas --- could help investors make better decisions.

That, in turn, might encourage companies to lower their emissions, or
risk losing access to investment or affordable insurance coverage.
``Every medium and large business has bank loans and has insurance,''
said Steven Rothstein, managing director of the Ceres Accelerator for
Sustainable Capital Markets, a group that works with investors and which
organized the letter.

The letter calls on regulators to adopt the steps Ceres outlined last
month in a report that makes 51 recommendations to eight federal
agencies. At its core are two demands: that the agencies treat climate
change as a systemic risk, and that the S.E.C. ensures mandatory and
consistent disclosure of climate threats facing companies.

According to Ceres, regulators can adopt each of its recommendations
without new legislation from Congress. Still, during the Trump
administration, even agencies that are meant to have a degree of
independence from the White House have been reluctant to address climate
change. President Trump has called global warming
\href{https://twitter.com/realDonaldTrump/status/265895292191248385?ref_src=twsrc\%5Etfw}{a
hoax}, and he has reversed
\href{https://www.nytimes3xbfgragh.onion/interactive/2020/climate/trump-environment-rollbacks.html}{nearly
70} environmental rules, with another 30 in progress.

\href{https://www.nytimes3xbfgragh.onion/section/climate?action=click\&pgtype=Article\&state=default\&region=MAIN_CONTENT_1\&context=storylines_keepup}{}

\hypertarget{climate-and-environment-}{%
\subsubsection{Climate and Environment
›}\label{climate-and-environment-}}

\hypertarget{keep-up-on-the-latest-climate-news}{%
\paragraph{Keep Up on the Latest Climate
News}\label{keep-up-on-the-latest-climate-news}}

Updated July 30, 2020

Here's what you need to know about the latest climate change news this
week:

\begin{itemize}
\item
  \begin{itemize}
  \tightlist
  \item
    \href{https://www.nytimes3xbfgragh.onion/2020/07/30/climate/bangladesh-floods.html?action=click\&pgtype=Article\&state=default\&region=MAIN_CONTENT_1\&context=storylines_keepup}{Floods
    in}\href{https://www.nytimes3xbfgragh.onion/2020/07/30/climate/bangladesh-floods.html?action=click\&pgtype=Article\&state=default\&region=MAIN_CONTENT_1\&context=storylines_keepup}{Bangladesh}
    are punishing the people least responsible for climate change.
  \item
    As climate change raises sea levels,
    \href{https://www.nytimes3xbfgragh.onion/2020/07/30/climate/sea-level-inland-floods.html?action=click\&pgtype=Article\&state=default\&region=MAIN_CONTENT_1\&context=storylines_keepup}{storm
    surges and high tides} are likely to push farther inland.
  \item
    The E.P.A. inspector general plans to investigate whether a rollback
    of fuel efficiency standards
    \href{https://www.nytimes3xbfgragh.onion/2020/07/27/climate/trump-fuel-efficiency-rule.html?action=click\&pgtype=Article\&state=default\&region=MAIN_CONTENT_1\&context=storylines_keepup}{violated
    government rules}.
  \end{itemize}
\end{itemize}

Nevertheless, Ceres's recommendations offer a blueprint for how a
Democratic administration might begin to tackle climate change, should
former Vice President Joseph R. Biden Jr. win the presidency in
November. Last month, Democrats on the House Select Committee on the
Climate Crisis released a
\href{https://climatecrisis.house.gov/news/press-releases/climate-plan-press-release}{report}
that echoed some of the recommendations from Ceres, particularly ones
regarding the disclosure of financial risks.

The letter on Tuesday suggests that those recommendations have
significant support among investors as well.

The letter was signed by some of the largest pension funds in the
country, including the California State Teachers' Retirement System, or
CalSTRS, which manages \$246 billion; the New York City Comptroller's
Office, which oversees pension funds worth \$206 billion; and the New
York State Comptroller's Office, which manages the state's \$211 billion
retirement fund.

Liz Gordon, executive director of corporate governance for New York
State's fund, said that even large institutional investors with skilled
researchers could not protect their holdings against climate risk. ``We
do a lot of engagement with companies individually,'' Ms. Gordon said.
``But that's not going to solve the broader problem.''

She said the S.E.C., which regulates the stock market and requires
publicly traded companies to regularly disclose information about a
range of perils they face, should also require those companies to better
disclose the financial risks they confront from climate change.

Other asset managers warned that climate change would increasingly
disrupt businesses.

Julie Gorte, senior vice president for sustainable investing at Impax
Asset Management, which manages \$23 billion, said the S.E.C. should
force companies to disclose the location of their physical assets, such
as factories and other facilities. That way, investors can gauge the
risks facing those facilities from wildfires, hurricanes or flooding,
and push companies to address them. Investors would then be able to
choose whether to invest based on that information.

``Regulators actually have the power to make the risks smaller,'' she
said. ``That will help all investors.''

Another useful change, Mr. Rothstein said, would be for the Fed to
require banks to examine the climate vulnerability of the companies they
lend money to. Banks already do those tests for other types of financial
risk, through a process that regulators and investors call ``stress
tests.''

Banks could then use the information from those climate-related stress
tests to increase the amount of money they hold in reserve, to help them
stay solvent if some of those companies defaulted. After the 2008
financial crisis and the collapse of the United States housing market,
``we looked at stress tests for banks, focusing on housing,'' Mr.
Rothstein said. Now, he said, ``think about the climate risks.''

Sarah Bloom Raskin, a former Federal Reserve governor and deputy
secretary of the Treasury who wrote the foreword to Ceres's
\href{https://www.ceres.org/resources/reports/addressing-climate-systemic-risk}{list
of recommendations}, said that regulators in the United States were
falling behind their counterparts in other countries, which have already
begun imposing stress tests for climate change as well as other steps.

``You see very credible central banks, like the Bank of England and the
European Central Bank, taking the risk of a climate calamity into their
mission in a very disciplined and structured way,'' Ms. Raskin said.
``These aren't fringe ideas.''

While the changes don't require congressional approval, the objections
of some Republican lawmakers to acting on climate change have had a
chilling effect on regulators, said former Representative Carlos
Curbelo, Republican of Florida, who signed the letter.

``Some civil servants logically fear that certain legislators, certain
committees would come after them or attack them,'' Mr. Curbelo said.
``By and large, regulators try to stay out of controversy.''

Still, Mr. Curbelo said the need to act was clear. ``The risks are
real,'' he said, ``and those of us who live here in South Florida
observe them on a daily basis.''

Advertisement

\protect\hyperlink{after-bottom}{Continue reading the main story}

\hypertarget{site-index}{%
\subsection{Site Index}\label{site-index}}

\hypertarget{site-information-navigation}{%
\subsection{Site Information
Navigation}\label{site-information-navigation}}

\begin{itemize}
\tightlist
\item
  \href{https://help.nytimes3xbfgragh.onion/hc/en-us/articles/115014792127-Copyright-notice}{©~2020~The
  New York Times Company}
\end{itemize}

\begin{itemize}
\tightlist
\item
  \href{https://www.nytco.com/}{NYTCo}
\item
  \href{https://help.nytimes3xbfgragh.onion/hc/en-us/articles/115015385887-Contact-Us}{Contact
  Us}
\item
  \href{https://www.nytco.com/careers/}{Work with us}
\item
  \href{https://nytmediakit.com/}{Advertise}
\item
  \href{http://www.tbrandstudio.com/}{T Brand Studio}
\item
  \href{https://www.nytimes3xbfgragh.onion/privacy/cookie-policy\#how-do-i-manage-trackers}{Your
  Ad Choices}
\item
  \href{https://www.nytimes3xbfgragh.onion/privacy}{Privacy}
\item
  \href{https://help.nytimes3xbfgragh.onion/hc/en-us/articles/115014893428-Terms-of-service}{Terms
  of Service}
\item
  \href{https://help.nytimes3xbfgragh.onion/hc/en-us/articles/115014893968-Terms-of-sale}{Terms
  of Sale}
\item
  \href{https://spiderbites.nytimes3xbfgragh.onion}{Site Map}
\item
  \href{https://help.nytimes3xbfgragh.onion/hc/en-us}{Help}
\item
  \href{https://www.nytimes3xbfgragh.onion/subscription?campaignId=37WXW}{Subscriptions}
\end{itemize}
