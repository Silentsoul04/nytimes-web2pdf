Sections

SEARCH

\protect\hyperlink{site-content}{Skip to
content}\protect\hyperlink{site-index}{Skip to site index}

\href{https://www.nytimes3xbfgragh.onion/section/food}{Food}

\href{https://myaccount.nytimes3xbfgragh.onion/auth/login?response_type=cookie\&client_id=vi}{}

\href{https://www.nytimes3xbfgragh.onion/section/todayspaper}{Today's
Paper}

\href{/section/food}{Food}\textbar{}Zooming In on Bill Buford's Latest
Obsession

\url{https://nyti.ms/30x2GKX}

\begin{itemize}
\item
\item
\item
\item
\item
\item
\end{itemize}

\href{https://www.nytimes3xbfgragh.onion/spotlight/at-home?action=click\&pgtype=Article\&state=default\&region=TOP_BANNER\&context=at_home_menu}{At
Home}

\begin{itemize}
\tightlist
\item
  \href{https://www.nytimes3xbfgragh.onion/2020/08/03/well/family/the-benefits-of-talking-to-strangers.html?action=click\&pgtype=Article\&state=default\&region=TOP_BANNER\&context=at_home_menu}{Talk:
  To Strangers}
\item
  \href{https://www.nytimes3xbfgragh.onion/2020/08/01/at-home/coronavirus-make-pizza-on-a-grill.html?action=click\&pgtype=Article\&state=default\&region=TOP_BANNER\&context=at_home_menu}{Make:
  Grilled Pizza}
\item
  \href{https://www.nytimes3xbfgragh.onion/2020/07/31/arts/television/goldbergs-abc-stream.html?action=click\&pgtype=Article\&state=default\&region=TOP_BANNER\&context=at_home_menu}{Watch:
  'The Goldbergs'}
\item
  \href{https://www.nytimes3xbfgragh.onion/interactive/2020/at-home/even-more-reporters-editors-diaries-lists-recommendations.html?action=click\&pgtype=Article\&state=default\&region=TOP_BANNER\&context=at_home_menu}{Explore:
  Reporters' Google Docs}
\end{itemize}

Advertisement

\protect\hyperlink{after-top}{Continue reading the main story}

Supported by

\protect\hyperlink{after-sponsor}{Continue reading the main story}

\hypertarget{zooming-in-on-bill-bufords-latest-obsession}{%
\section{Zooming In on Bill Buford's Latest
Obsession}\label{zooming-in-on-bill-bufords-latest-obsession}}

The author spent more than a decade seeking the heart of French cuisine
for his new book, ``Dirt.'' But in quarantine, he just wants to make the
perfect chicken.

\includegraphics{https://static01.graylady3jvrrxbe.onion/images/2020/07/22/dining/21Buford1/21Buford1-articleLarge.jpg?quality=75\&auto=webp\&disable=upscale}

By \href{https://www.nytimes3xbfgragh.onion/by/pete-wells}{Pete Wells}

\begin{itemize}
\item
  July 21, 2020
\item
  \begin{itemize}
  \item
  \item
  \item
  \item
  \item
  \item
  \end{itemize}
\end{itemize}

Last month, the writer Bill Buford attempted to teach me, using Zoom,
how to bone, stuff, tie and poach a chicken. It was something like one
of those MasterClasses that everybody is watching these days, except
that those run for 20 minutes, and my lesson lasted more than six hours.
It may not have been the longest Zoom call in history, but I suspect it
set a record for Zoom calls with a strong focus on boneless poultry.

It came about because for the past few months Mr. Buford has been on
what he calls ``a chicken kick.'' He tends to get preoccupied with a
recipe or technique or ingredient and keep coming back to it, over and
over, until a new obsession takes hold. Recently he went on a mayonnaise
kick, whisking grapeseed oil into egg yolks a drop at a time, sometimes
adding the zest of a couple of lemons, a trick he learned when he worked
for the chef
\href{https://www.nytimes3xbfgragh.onion/2016/08/16/dining/michel-richard-dead.html}{Michel
Richard}.

Several years ago, Mr. Buford went on a bread kick. He made the dough
with flour he brought back from his last trip to France, a flour that he
said was ``magical in the way that Bob's flour was magical'' --- Bob
being what everybody called Yves Richard, a baker in Lyon who gave Mr.
Buford his first apprenticeship after he moved to that city with his
family in 2008. He said that as soon as the magical French flour ran
out, ``I thought, `There's no point in making bread.'''

The roots of all three kicks are recounted in Mr. Buford's latest book,
``\href{https://www.nytimes3xbfgragh.onion/2020/05/25/books/review-dirt-bill-buford.html?searchResultPosition=1}{Dirt:
Adventures in Lyon as a Chef in Training, Father, and Sleuth Looking for
the Secret of French Cooking}.'' As in his previous two books, Mr.
Buford's reportorial method is to embed himself with his subjects so
completely that he becomes almost indistinguishable from them.

In ``Among the Thugs,'' he got so caught up in the culture of violent
English soccer fans that he was beaten up by the police. (``It's
surprisingly painful to be beaten,'' he said.) Pursuing his interest in
Italian cuisine, the subject of ``Heat,'' he spent so much time with the
chef Mario Batali that Mr. Batali seemed to forget that he was being
watched by a very sharp and observant journalist who was taking notes on
his crude patter and sexualized behavior years before
\href{https://www.nytimes3xbfgragh.onion/2019/03/06/dining/mario-batali-bastianich-restaurants.html}{multiple
women accused him} of assault, abuse and harassment.

``Dirt'' is the story of how Mr. Buford tried to understand the food of
France by learning to cook it. For this, he moved to Lyon with his wife,
\href{https://twitter.com/jessicahgreen}{Jessica Green}, and their twin
sons; took classes at the
\href{https://en.institutpaulbocuse.com/}{Institut Paul Bocuse}, just
outside the city; and eventually persuaded the chef
\href{https://www.instagram.com/mathieuviannay/}{Mathieu Viannay} to let
him volunteer in the kitchen of \href{https://lamerebrazier.fr/en/}{La
Mère Brazier}.

\includegraphics{https://static01.graylady3jvrrxbe.onion/images/2020/07/22/dining/21Buford2/21Buford2-articleLarge.jpg?quality=75\&auto=webp\&disable=upscale}

The idea was to stay nine months. In Mr. Buford's world, though,
departure dates, deadlines and other end points in time are more or less
hypothetical. Not flight times --- they turned out to be real, as he
learned when he and the boys missed the plane to Lyon that was supposed
to begin their new life there, where Ms. Green was already waiting. He
writes that his call to her from the airport was ``the nadir of our
married life.'' Once they arrived, though, the family did not move back
to New York for almost five years.

In January, I asked Mr. Buford if we could cook together, making
something he had learned in Lyon. At first he was going to have a dinner
party. We were still talking in March when dinner parties began to seem
unwise and then were banned. We finally settled on a video call.

Whenever we talked or emailed, he would suggest different elaborations
on the basic theme of poached chicken, and after a while just talking
about chicken with Mr. Buford seemed at least as interesting as cooking
it.

Perhaps he would slide truffle slices under the skin for poulet en
demi-deuil, the rarely seen chicken in half-mourning made famous by
Eugènie Brazier, the chef who founded La Mère Brazier. I wanted to see
that. But I was also intrigued by another dish that had become a
signature of
\href{https://www.nytimes3xbfgragh.onion/2018/01/20/obituaries/paul-bocuse-dead.html?searchResultPosition=1}{Paul
Bocuse}, who had passed through the kitchen of La Mère Brazier in the
1950s: poulet en vessie, a whole bird steamed inside a pig's bladder.

Image

Paul Bocuse was nearly as towering a figure in Lyon as his
toque.Credit...Jeff Pachoud/Agence France-Presse --- Getty Images

In ``Dirt,'' Mr. Buford described the appearance of the bladder during
cooking with a sense of wonder that stuck with me: ``a beautifully
golden, nearly translucent beach ball that some maniac is still
insisting on pumping more air into. Also, you can see the chicken!''

One day Mr. Buford called to say that a parcel of bladders had arrived
(``they look like socks''), but that he had also been talking to a
producer
\href{https://www.npr.org/2020/05/08/852496954/in-dirt-bill-buford-is-able-to-offer-an-authentic-adventure-in-french-cooking}{at
NPR about making chicken en vessie}. Would I mind if we cooked something
else?

I did not mind. We had never exactly settled on poulet en vessie, and
the French have so many other ways to cook a chicken.

Image

Paul Dench-Layton of Violet Hill Farm was a major supplier of Mr.
Buford's quarantine chickens.Credit...Brittainy Newman for The New York
Times

There would be more discussions about chickens before we got together on
Zoom, but finally, one day at 2:30 p.m., I clicked a link and Mr.
Buford's face appeared on my screen.

He was wearing an apron. He had also finished his prep work. I found an
apron, tied it on and chopped piles of mushrooms, parsley and bread
crumbs as fast as I could. Mr. Buford seemed to be in no hurry.

``We split our duties, and Jessica usually takes over the school nights,
because she tends to get food on the table more punctually than I do,''
he said. It was a Friday, though, one of the days when he is allowed to
cook, and there was an understanding in the house that dinner might be
served on the later side.

Almost an hour had passed before we made our first approach to the
chickens. Mr. Buford had me nick out the wishbone first. Off went the
wing tips, and then he showed me how to slip the point of the knife into
the small socket where the wing joins the carcass, running the point of
the blade around the hollow place. Without any slicing of meat or sawing
of bone, the joint swung free.

Image

Mr. Buford showing the reporter his low-temperature method for poaching
a chicken.Credit...Pete Wells/The New York Times

Then we began exploring the interior of the chicken leg, which becomes a
confusing place once you start taking out the bones. I angled my
laptop's camera so he had a view of my cutting board. What was lying
there looked less and less like a bird.

``It helps to reassemble the chicken every so often so you know where
you are,'' he said.

His patient manner was nothing like the rushed, barking tone he
describes inside La Mère Brazier. ``Dirt'' depicts the sadism and
bullying Mr. Buford experienced in the kitchen, and the sexism and
racism shown to other workers. A Malaysian culinary student is taken on
as a stagiaire; everybody calls him Jackie Chan. Even Mr. Buford did it
until he learned the student's name, Chern Hwei Gan. A female cook,
perhaps the first since the days of Eugènie Brazier, was addressed as
``Mademoiselle'' for two weeks. After that, all pretense of respect was
dropped, and the chef de partie would pretend to ``mount her from
behind'' each time she passed.

For the epilogue, Mr. Buford tracked them down to see how the
mistreatment had affected their careers. Mr. Chern went on to open what
Mr. Buford calls a \href{https://www.parapluie-dijon.com/}{``witty and
anarchic'' restaurant in Dijon}. The female cook, though, gave up on
restaurant kitchens and went into the fashion business. ``Her cooking
spirit had been crushed,'' he writes.

``Dirt'' is more explicit about the damage done by masculine aggression
and volatility in the kitchen than ``Heat,'' published in 2006. In that
book, Mr. Batali is seen acting coarsely, making sexual suggestions
toward at least one female employee. The author's tone, though, is
nonjudgmental. I asked Mr. Buford if he regretted that now.

``I think I felt pretty confident in the weight of the observations,''
he said. ``A lot of the observations were not loaded in their
presentation but loaded in their content. And I think that's what
works.''

The book, which for years remained one of the only pieces of journalism
that showed the uglier side of Mr. Batali's bacchanalian drive, caused a
rift between the two men. ``Mario hated it, and it took him a long time
to get over it,'' he said. ``He described it as `standing naked in front
of a mirror for 24 hours.'''

By 4:30, with Mr. Buford's video guidance, I had successfully taken out
all the chicken bones that seemed to matter. I'd spread the chicken out
like an open paperback and covered it with parsley, mushrooms and bread.
Mr. Buford showed me how to tie the bird into a neat bundle with a
wrap-and-twist motion that I've seen butchers make but have never been
able to copy.

``Now we poach them,'' he said. The trick of this was to keep the stock
between 150 and 160 degrees, well below boiling, barely hot enough to
decorate the surface of the liquid with slow fingers of steam. The
steady, moderate heat of the stock mimics the sous vide method. In fact,
chicken breasts are now cooked sous vide with an immersion circulator at
La Mère Brazier.

Mr. Buford prefers the older, low-tech method because he gets two
delicious things out of one recipe: very tender chicken breasts and
double-strength chicken stock. During the lockdown, he began taking it
one step beyond, boning several chickens, poaching them and then cooking
the stock a second time, with the carcasses. He called it ``chicken
squared.''

Earlier, I had asked Mr. Buford how the restaurants of a city once
famous for its female chefs --- Brazier was one of many mères in Lyon
--- had come to be dominated by men such as Bocuse.

He pointed to the generation that followed the mères, including Bocuse's
father and Fernand Point of \href{https://www.lapyramide.com/en/}{La
Pyramide} in Vienne, who went off to train in expensive hotels before
coming back to take over the family business.

``In hotels then, you would be taught the brigade system, which would be
\href{https://www.nytimes3xbfgragh.onion/1961/04/16/archives/chef-of-chefs-he-was-escoffier-the-most-renowned-man-ever-to-wear-a.html}{Escoffier},''
he said. ``You're learning an expertise that the mères didn't have.
Also, if you have that training, you're not doing home cooking. You're
being taught to do grand cuisine.''

The brigade system became standard in ambitious French restaurants, as
did a more stylized form of cooking that took pains to set itself apart
from what the mères had achieved.

It would take several generations for French women to regain what had
been taken from them. Now, Mr. Buford said, ``There's a whole generation
of female chefs in France who are better at the men's game than the men.
They're brilliant and tough and they all had to force their way through
the heaps of abuse they faced entering those kitchens.''

By 5:15, both birds were in the pot. Clearly, a glass of wine was called
for.

Image

Mr. Buford and his wife, Jessica Green, shown here in 2017, once worked
together in the fiction department of The New Yorker.Credit...Patrick
McMullan/Getty Images

Mr. Buford called Ms. Green to join our Zoom cocktail hour. About 20
years ago, when I was writing news releases for The New Yorker, she had
worked in the fiction department and Mr. Buford had been the fiction
editor. The three of us never had a drink together, though.

I had opened a Sardinian vermentino. Mr. Buford and Ms. Green had a long
colloquy about whether they had a vermentino in the house from Sardinia
or anywhere else. No. A riesling, perhaps? They settled on a Chablis,
taking it out to a small patio behind their apartment.

Ms. Green is a wine writer and educator. While the couple lived in
France, she earned a diploma from the Wine \& Spirit Education Trust in
Mâcon. In the first year, though, she took care of 3-year-old twins
while Mr. Buford learned to cook.

``They all the time said no to me,'' Ms. Green said. ``They walked home
from the library barefoot one day because I got tired of fighting with
them about putting on their shoes. And you were never there.''

This was addressed to Mr. Buford, who nodded.

``This was not a job I would have applied for,'' she said. ``And I did
that all over the earth. I spent most of that first year taking them on
trips.''

The boys are 14 now. The New Yorker has hired them to make videos of
their father's cooking kicks.

``Frederick is the bossy-pants director and George is the genuinely
bossy-pants cinematographer,'' Mr. Buford said. ``They're not getting a
lot and they've got no rights whatsoever, but they're already talking
about what they're going to do with the money. And they're quietly
expressing disappointment that I haven't turned in another piece.''

Once a few glasses of wine had been put down, Mr. Buford and I went back
to our kitchens. His chicken was nearly cooked. Mine was not. He began
to whisking up a variation on sauce suprême that would incorporate
sherry vinegar and mustard made with grape must.

``It's one of the secret weapons I picked up from Mère Brazier
kitchen,'' he said. ``It looks like caviar. I love it.''

In theory, we were going to use the poaching stock for our sauces and
for rice pilaf. In reality, there wasn't enough for both things, and we
used quarts of stock we had bought from
\href{https://www.cascunfarm.com/}{Cascun Farm} just in case.

The conversation turned back to his family's time in Lyon. Mr. Buford
said that they would have become eligible for French citizenship if they
had stayed a few more months. I asked him what happened.

``We ran out of money,'' he said. ``I wasn't earning money writing a
book that took much, much longer than I thought it was going to take.''

The writing of ``Dirt,'' which he sold to Knopf in 2008, consumed 12
years. By May, when it was published, Mr. Richard had died --- both Mr.
Richards, the chef and the bread-baking Mr. Richard known as Bob. So had
other subjects and characters in the book. So had its editor, Sonny
Mehta. This is all recounted in an epilogue called ``Just About
Everybody Dies.''

Image

This chicken breast in cream uses~a stove-top approximation of a sous
vide technique.Credit...Andrew Purcell for The New York Times. Food
Stylist: Carrie Purcell.

At 7:30, Mr. Buford told his wife that dinner would be ready in half an
hour. He brushed the poached chicken with orange juice and butter to
help it brown, and put it into the oven along with a casserole of rice
pilaf. I asked if he'd used stock for the rice.

``I used it for the rice and I used other stock for the sauce, but I'm
not using it for the zucchini I'm about to do,'' he said.

Wait, zucchini?

It was about 8:15 when Mr. Buford told Ms. Green that it was time to
call the boys for dinner.

After Mr. Buford and I hung up, I tried to picture them all sitting down
together. When they grew up, the boys would almost certainly remember
when they didn't have to go to school for months and their father kept
coming up with new ways to poach a chicken.

A few days later, according to Mr. Buford, Frederick looked in the
refrigerator. Once again, it was full of birds.

``So many?'' he asked. ``Why are we always eating chicken? I hate
chicken.''

Recipes:
\textbf{\href{https://cooking.nytimes3xbfgragh.onion/recipes/1021177-supreme-de-volaille-fermiere-a-la-creme-chicken-breast-in-cream}{Suprême
de Volaille Fermière à la Crème (Chicken Breast in Cream)}} \textbar{}
\textbf{\href{https://cooking.nytimes3xbfgragh.onion/recipes/1021178-cervelle-de-canut-herbed-cheese-spread}{Cervelle
de Canut (Herbed Cheese Spread)}}

\emph{Follow} \href{https://twitter.com/nytfood}{\emph{NYT Food on
Twitter}} \emph{and}
\href{https://www.instagram.com/nytcooking/}{\emph{NYT Cooking on
Instagram}}\emph{,}
\href{https://www.facebookcorewwwi.onion/nytcooking/}{\emph{Facebook}}\emph{,}
\href{https://www.youtube.com/nytcooking}{\emph{YouTube}} \emph{and}
\href{https://www.pinterest.com/nytcooking/}{\emph{Pinterest}}\emph{.}
\href{https://www.nytimes3xbfgragh.onion/newsletters/cooking}{\emph{Get
regular updates from NYT Cooking, with recipe suggestions, cooking tips
and shopping advice}}\emph{.}

Advertisement

\protect\hyperlink{after-bottom}{Continue reading the main story}

\hypertarget{site-index}{%
\subsection{Site Index}\label{site-index}}

\hypertarget{site-information-navigation}{%
\subsection{Site Information
Navigation}\label{site-information-navigation}}

\begin{itemize}
\tightlist
\item
  \href{https://help.nytimes3xbfgragh.onion/hc/en-us/articles/115014792127-Copyright-notice}{©~2020~The
  New York Times Company}
\end{itemize}

\begin{itemize}
\tightlist
\item
  \href{https://www.nytco.com/}{NYTCo}
\item
  \href{https://help.nytimes3xbfgragh.onion/hc/en-us/articles/115015385887-Contact-Us}{Contact
  Us}
\item
  \href{https://www.nytco.com/careers/}{Work with us}
\item
  \href{https://nytmediakit.com/}{Advertise}
\item
  \href{http://www.tbrandstudio.com/}{T Brand Studio}
\item
  \href{https://www.nytimes3xbfgragh.onion/privacy/cookie-policy\#how-do-i-manage-trackers}{Your
  Ad Choices}
\item
  \href{https://www.nytimes3xbfgragh.onion/privacy}{Privacy}
\item
  \href{https://help.nytimes3xbfgragh.onion/hc/en-us/articles/115014893428-Terms-of-service}{Terms
  of Service}
\item
  \href{https://help.nytimes3xbfgragh.onion/hc/en-us/articles/115014893968-Terms-of-sale}{Terms
  of Sale}
\item
  \href{https://spiderbites.nytimes3xbfgragh.onion}{Site Map}
\item
  \href{https://help.nytimes3xbfgragh.onion/hc/en-us}{Help}
\item
  \href{https://www.nytimes3xbfgragh.onion/subscription?campaignId=37WXW}{Subscriptions}
\end{itemize}
