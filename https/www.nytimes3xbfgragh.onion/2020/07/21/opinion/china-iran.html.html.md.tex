Sections

SEARCH

\protect\hyperlink{site-content}{Skip to
content}\protect\hyperlink{site-index}{Skip to site index}

\href{https://myaccount.nytimes3xbfgragh.onion/auth/login?response_type=cookie\&client_id=vi}{}

\href{https://www.nytimes3xbfgragh.onion/section/todayspaper}{Today's
Paper}

\href{/section/opinion}{Opinion}\textbar{}When China Met Iran

\url{https://nyti.ms/3fTSItC}

\begin{itemize}
\item
\item
\item
\item
\item
\end{itemize}

Advertisement

\protect\hyperlink{after-top}{Continue reading the main story}

\href{/section/opinion}{Opinion}

Supported by

\protect\hyperlink{after-sponsor}{Continue reading the main story}

\hypertarget{when-china-met-iran}{%
\section{When China Met Iran}\label{when-china-met-iran}}

A growing partnership between America's main Middle East adversary and
Asia's rising superpower bears careful watching in Washington.

By Michael Singh

Mr. Singh served as a former senior director for Middle East affairs at
the National Security Council.

\begin{itemize}
\item
  July 21, 2020
\item
  \begin{itemize}
  \item
  \item
  \item
  \item
  \item
  \end{itemize}
\end{itemize}

\includegraphics{https://static01.graylady3jvrrxbe.onion/images/2020/07/21/opinion/21singh1/merlin_103791600_11e6e2b6-b8f2-44f0-8c50-b184b76e732d-articleLarge.jpg?quality=75\&auto=webp\&disable=upscale}

Leaked news this month that China and Iran had come to the verge of
signing a 25-year trade and military partnership agreement struck like a
geopolitical storm in Washington --- a rising rival of America and a
longtime foe joining forces to threaten the United States's predominant
position in the Middle East.

The agreement ambitiously promises to bring a huge Chinese presence into
Iran's economic development, in exchange for a regular supply of heavily
discounted Iranian oil.

Yet in Iran and China themselves, the reaction was hardly ebullient.
Critics of Iran's beleaguered president, Hassan Rouhani, called the deal
a new Treaty of Turkmenchay, after the notorious 1828 accord under which
a weakened Persia ceded much of the South Caucasus to the Russian
Empire. In Beijing, a government spokesman who was asked about the deal
dodged rather than criticize Washington, insisting blandly that Iran is
merely one of many countries with which China is ``developing normal
friendly relations,'' and claiming not to have further information about
the reported deal.

Tehran's and Beijing's ambivalence hardly suggests a loving embrace
between the two adversaries of America; rather, it reveals the conundrum
each faces in pursuing closer ties with the other --- conundrums that
the United States can turn to our advantage.

In recent years, as the United States has been bogged down in
unrewarding conflicts in the Middle East, China has been quietly
expanding its economic, diplomatic and even military activities in the
region. Beijing's motives are straightforward but varied: It seeks to
advance its interests, such as a pressing need for energy imports and
for destinations for surplus capital and labor. In practice, it tries to
advance President Xi Jinping's signature Belt and Road Initiative, which
is aimed at reshaping regional economic topographies in China's favor
and counters what Beijing sees as an American effort to contain it. In
short, China seeks to establish itself in the eyes of the world --- and
its own people --- as a great power capable of contending with the
United States.

Yet Chinese leaders are aware that few of the great powers have emerged
unscathed from Middle East adventures and that they face a particularly
formidable set of obstacles. The region's deep political fissures make
it difficult for outsiders to avoid taking sides in its many conflicts.
This poses a risk for a China that aims to be a friend of everyone ---
Iran, Israel and Arab states alike --- to maximize the benefits of its
regional engagement while minimizing its commitments. Consistent with
this balancing act, the leak of the China-Iran agreement roughly
coincided with the biannual China-Arab States Cooperation Forum early
this month, which prompted numerous proclamations of Chinese friendship
with Iran's Arab rivals.

Even more daunting is the outsize role played by the United States in
the region, one that ironically benefits Beijing as it expands its
economic ventures without being able to provide adequate security for
its capital and citizens deployed to the region. For all its ``wolf
warrior'' bluster, China continues to pick its battles with Washington.
So far, it largely has chosen not to wage them in the Middle East ---
for example, doing little beyond diplomatic finger-wagging to shelter
Iran from American secondary sanctions.

So even as China seeks to make strategic gains in the Middle East, it
does so sotto voce --- bolstering its own role and challenging
Washington's without fanfare.

Iran, China's foremost regional partner, faces a conundrum of its own.
Squeezed by America's ``maximum pressure'' campaign, Tehran needs
whatever friends it can find. And China, which beside being rich and
powerful shares Iran's revisionist inclination to challenge the United
States's role in the international order, would seem to be a perfect
match.

Yet Iran's history has made it suspicious of external powers, and
China's ``help'' in recent years --- buying Iranian oil in small
quantities at a steep discount and crowding out Iranian domestic
producers with low-cost imports --- seems not to have engendered
affection among the Iranian populace. What Iranians seem to desire is to
be no one's junior partner, but to be self-sufficient and stand among
the likes of Russia and China as equals.

What amount to challenges for China and Iran in further developing their
ties constitute opportunities for an America worried about the
partnership between them. The United States should, at every step, aim
to exacerbate the conundrums each faces. It should, for example,
emphasize that Iranian dependence on China --- besides being costly ---
is a policy choice and that the door remains open to the rest of the
international community if Tehran is willing to compromise on its
nuclear ambitions and regional policies. And Washington should enlist
regional partners --- who may otherwise prefer to enjoy the benefits of
good relations with Beijing while leaving it up to the United States to
confront China when it empowers Iran through arms sales or investment.

Make no mistake: The China-Iran relationship has long been important for
both countries, contributing for example to Iran's nuclear and missile
advancements. And whether in the form of formal partnership agreements
or simply ad hoc cooperation, those relations are very likely to grow
closer yet in coming years, as China tries to project power westward and
Iran seeks to insulate itself from the debilitating effects of American
power and enhance its own regional influence.

But while the deepening of the Iran-China relationship may be
inevitable, the United States shouldn't let it be easy for either Tehran
or Beijing.

Michael Singh
(\href{https://twitter.com/MichaelSinghDC}{@MichaelSinghDC}), a former
senior director for Middle East affairs at the National Security
Council, is the managing director of the Washington Institute for Near
East Policy.

\emph{The Times is committed to publishing}
\href{https://www.nytimes3xbfgragh.onion/2019/01/31/opinion/letters/letters-to-editor-new-york-times-women.html}{\emph{a
diversity of letters}} \emph{to the editor. We'd like to hear what you
think about this or any of our articles. Here are some}
\href{https://help.nytimes3xbfgragh.onion/hc/en-us/articles/115014925288-How-to-submit-a-letter-to-the-editor}{\emph{tips}}\emph{.
And here's our email:}
\href{mailto:letters@NYTimes.com}{\emph{letters@NYTimes.com}}\emph{.}

\emph{Follow The New York Times Opinion section on}
\href{https://www.facebookcorewwwi.onion/nytopinion}{\emph{Facebook}}\emph{,}
\href{http://twitter.com/NYTOpinion}{\emph{Twitter (@NYTopinion)}}
\emph{and}
\href{https://www.instagram.com/nytopinion/}{\emph{Instagram}}\emph{.}

Advertisement

\protect\hyperlink{after-bottom}{Continue reading the main story}

\hypertarget{site-index}{%
\subsection{Site Index}\label{site-index}}

\hypertarget{site-information-navigation}{%
\subsection{Site Information
Navigation}\label{site-information-navigation}}

\begin{itemize}
\tightlist
\item
  \href{https://help.nytimes3xbfgragh.onion/hc/en-us/articles/115014792127-Copyright-notice}{©~2020~The
  New York Times Company}
\end{itemize}

\begin{itemize}
\tightlist
\item
  \href{https://www.nytco.com/}{NYTCo}
\item
  \href{https://help.nytimes3xbfgragh.onion/hc/en-us/articles/115015385887-Contact-Us}{Contact
  Us}
\item
  \href{https://www.nytco.com/careers/}{Work with us}
\item
  \href{https://nytmediakit.com/}{Advertise}
\item
  \href{http://www.tbrandstudio.com/}{T Brand Studio}
\item
  \href{https://www.nytimes3xbfgragh.onion/privacy/cookie-policy\#how-do-i-manage-trackers}{Your
  Ad Choices}
\item
  \href{https://www.nytimes3xbfgragh.onion/privacy}{Privacy}
\item
  \href{https://help.nytimes3xbfgragh.onion/hc/en-us/articles/115014893428-Terms-of-service}{Terms
  of Service}
\item
  \href{https://help.nytimes3xbfgragh.onion/hc/en-us/articles/115014893968-Terms-of-sale}{Terms
  of Sale}
\item
  \href{https://spiderbites.nytimes3xbfgragh.onion}{Site Map}
\item
  \href{https://help.nytimes3xbfgragh.onion/hc/en-us}{Help}
\item
  \href{https://www.nytimes3xbfgragh.onion/subscription?campaignId=37WXW}{Subscriptions}
\end{itemize}
