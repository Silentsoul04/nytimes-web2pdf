Sections

SEARCH

\protect\hyperlink{site-content}{Skip to
content}\protect\hyperlink{site-index}{Skip to site index}

\href{https://myaccount.nytimes3xbfgragh.onion/auth/login?response_type=cookie\&client_id=vi}{}

\href{https://www.nytimes3xbfgragh.onion/section/todayspaper}{Today's
Paper}

\href{/section/opinion}{Opinion}\textbar{}The Border War in Portland

\href{https://nyti.ms/3hkPKyl}{https://nyti.ms/3hkPKyl}

\begin{itemize}
\item
\item
\item
\item
\item
\item
\end{itemize}

Advertisement

\protect\hyperlink{after-top}{Continue reading the main story}

\href{/section/opinion}{Opinion}

Supported by

\protect\hyperlink{after-sponsor}{Continue reading the main story}

\hypertarget{the-border-war-in-portland}{%
\section{The Border War in Portland}\label{the-border-war-in-portland}}

How can this be a job for Homeland Security?

\href{https://www.nytimes3xbfgragh.onion/column/jamelle-bouie}{\includegraphics{https://static01.graylady3jvrrxbe.onion/images/2019/01/24/opinion/jamelle-bouie/jamelle-bouie-thumbLarge-v3.png}}

By
\href{https://www.nytimes3xbfgragh.onion/column/jamelle-bouie}{Jamelle
Bouie}

Opinion Columnist

\begin{itemize}
\item
  July 21, 2020
\item
  \begin{itemize}
  \item
  \item
  \item
  \item
  \item
  \item
  \end{itemize}
\end{itemize}

\includegraphics{https://static01.graylady3jvrrxbe.onion/images/2020/07/21/opinion/21bouie4/merlin_174758502_f22dd3f8-6060-4927-a471-8ddf34c7db58-articleLarge.jpg?quality=75\&auto=webp\&disable=upscale}

\hypertarget{listen-to-this-op-ed}{%
\subsubsection{Listen to This Op-Ed}\label{listen-to-this-op-ed}}

Audio Recording by Audm

\emph{To hear more audio stories from publishers like The New York
Times,
download}\href{https://www.audm.com/?utm_source=nytmag\&utm_medium=embed\&utm_campaign=left_behind_draper}{**}\href{https://www.audm.com/?utm_source=nytopinion\&utm_medium=embed\&utm_campaign=border_war_portland}{\emph{Audm
for iPhone or Android}}\emph{.}

Something dangerous is taking shape within the Department of Homeland
Security.

We got our first glimpse of it last week in Oregon, when unidentified
federal agents clad in camouflage and tactical gear descended on
Portland, beat and tear-gassed protesters and pulled others into
unmarked vehicles for
\href{https://www.washingtonpost.com/nation/2020/07/17/portland-protests-federal-arrests/}{arrest
and questioning}.

Apparently cobbled together using personnel from Customs and Border
Protection, Immigration and Customs Enforcement, the Transportation
Security Administration and the Coast Guard, these
``\href{https://abcnews.go.com/Politics/dhs-launches-rapid-deployment-teams-federal-monuments-july/story?id=71552901}{rapid
deployment teams}'' are formally tasked with securing federal buildings
from graffiti and vandalism in tandem with the Federal Protective
Service, which is ordinarily responsible for the job. But they're being
used to suppress protests in what appears to be an election year gambit
by the Trump administration to create images of disorder and chaos on
which the president can then campaign. ``This political theater from
President Trump has nothing to do with public safety,''
\href{https://www.wweek.com/news/2020/07/16/oregon-gov-kate-brown-says-president-trump-is-invading-portland-as-an-election-stunt/}{Kate
Brown, the Democratic governor of Oregon, said last week}. ``Trump is
looking for a confrontation in Oregon in the hopes of winning political
points in Ohio or Iowa.''

The official tasked with coordinating all this action, the acting
secretary of the Department of Homeland Security, Chad Wolf, is an
enthusiastic participant, casting protesters as ``violent anarchists and
extremists'' in order to justify what's been done to them. ``The city of
Portland has been under siege for 47 straight days by a violent mob
while local political leaders refuse to restore order to protect their
city,''
\href{https://www.dhs.gov/news/2020/07/16/acting-secretary-wolf-condemns-rampant-long-lasting-violence-portland}{Wolf
said}. ``This siege can end if state and local officials decide to take
appropriate action instead of refusing to enforce the law.''

On Sunday, Wolf's deputy, Ken Cuccinelli (whose
\href{https://www.uscis.gov/about-us/leadership/kenneth-t-ken-cuccinelli-senior-official-performing-the-duties-of-the-director-us-citizenship-and\#:~:text=\%3B\%20Director\%20(vacant)-,Kenneth\%20T.,Mr.}{official
title} is ``Senior Official Performing the Duties of the Deputy
Secretary for the Department of Homeland Security''),
\href{https://talkingpointsmemo.com/edblog/dhs-under-boss-were-taking-this-national}{told
NPR} that Homeland Security would be taking these tactics nationwide.
Wolf affirmed this, telling Fox News that his agency can act with or
without local cooperation. ``I don't need invitations by the state,
state mayors or state governors to do our job,'' he said. ``We're going
to do that, whether they like us there or not.'' President Trump
likewise
\href{https://www.foxnews.com/politics/trump-vows-to-send-federal-agents-dhs-chicago}{vowed}
to send federal law enforcement agents to several more cities, amid
reports that a Portland-like force was
\href{https://www.chicagotribune.com/news/criminal-justice/ct-chicago-police-dhs-deployment-20200720-dftu5ychwbcxtg4ltarh5qnwma-story.html}{headed
to Chicago}.

There's more. In addition to its rapid deployment teams, the Department
of Homeland Security has also authorized domestic surveillance of
Americans on the basis of the president's June executive order on the
protection of statues and monuments. Writing for
\href{https://www.lawfareblog.com/dhs-authorizes-domestic-surveillance-protect-statues-and-monuments}{the
Lawfare blog}, the legal scholars Steve Vladeck and Benjamin Wittes
explain that the ``animating premise'' of the new rules ``is that the
threat to monuments and statues is a homeland security threat warranting
intelligence analysis and collection by federal officials.'' The
administration, they continue, is using the ``cover of minor property
damage'' to ``justify intelligence gathering against ordinary
Americans'' for ``peacefully protesting their government.''

The United States is no stranger to the use of military or
quasi-military force against protesters. During the Whiskey Rebellion, a
tax revolt of farmers and distillers in western Pennsylvania that
culminated in 1794, President George Washington raised a federal militia
to meet insurgents in the field. To break the Pullman Strike of 1894,
during which workers shut down rail traffic in much of the country,
President Grover Cleveland deployed federal troops to Chicago, sparking
a confrontation that ended in the deaths of 30 workers. And in 1932,
under orders from President Herbert Hoover, Gen. Douglas MacArthur
confronted the Bonus Army --- a group of World War I veterans who camped
out in Washington, D.C., petitioning the government for their promised
bonuses for military service --- with infantry, cavalry and tanks.

The difference lies less in the acts themselves than in the ways these
events developed. Use of military force against strikers and protesters
is certainly controversial, but for the most part it unfolds along clear
lines of responsibility and involves powers expressly granted to the
president. As the example of Washington and the Whiskey rebellion
demonstrates, it was part of the constitutional design. President
Trump's internal security force was, by contrast, created out of public
view, using loopholes and expansive interpretations of the law. The
reason Customs and Border Protection can be used to police a protest in
Portland is, for example, because the Department of Homeland Security
can supplement law enforcement from one agency with personnel from
another.

There are other elements beyond the fact of its existence that make the
emergence of an internal security force extremely troubling. As a
candidate, Trump actively cultivated both the leadership and the
rank-and-file of the border police and ICE. In turn, they gave him his
support --- unions for both
\href{https://www.npr.org/2016/03/30/472420387/border-patrol-union-endorses-donald-trump}{agencies}
\href{https://www.politico.com/story/2016/09/immigration-customs-enforcement-union-endorses-trump-228664}{endorsed}
Trump for president. Under his leadership, these agencies have shown
themselves to be deeply simpatico with the administration's draconian
approach to immigration at the southern border, with aggressive action
against migrants, asylum-seekers and unauthorized immigrants.

\includegraphics{https://static01.graylady3jvrrxbe.onion/images/2020/07/21/opinion/21bouie3/merlin_174758886_b77b5742-fa5c-4c0c-811e-e73d0d778de9-articleLarge.jpg?quality=75\&auto=webp\&disable=upscale}

A secretive, nationwide police force --- created without congressional
input or authorization, formed from highly politicized agencies, tasked
with rooting out vague threats and answerable only to the president ---
is a nightmare out of the fever dreams of the founding generation,
federalists and antifederalists alike. It's something Americans continue
to fear and for good reason. It is a power that cannot and should not
exist in a democracy, lest it undermine and destroy the entire project.

Democrats, thankfully, seem to recognize this. ``We live in a democracy,
not a banana republic. We will not tolerate the use of Oregonians,
Washingtonians --- or any other Americans --- as props in President
Trump's political games,'' said House Speaker Nancy Pelosi on Saturday,
in \href{https://www.speaker.gov/newsroom/71820}{a joint statement} with
Representative Earl Blumenauer of Oregon. ``The House is committed to
moving swiftly to curb these egregious abuses of power immediately.''

But rhetoric isn't enough. The House must act and act now. In addition
to holding hearings and investigations --- including eliciting testimony
from Wolf and other officials --- Democrats should condition final
passage
\href{https://appropriations.house.gov/news/press-releases/appropriations-committee-approves-fiscal-year-2021-homeland-security-funding}{of
its Homeland Security appropriations bill} on a complete halt to
operations in Portland and other cities and the dissolution of the
response force. Should Democrats find themselves in control of both
legislative branches and the White House next year, they should also use
the opportunity to
\href{https://slate.com/news-and-politics/2019/04/trump-acting-secretaries-dhs-fvra-senate-reform.html}{amend}
the relatively obscure Federal Vacancies Reform Act, which Trump has
used to install loyalists in high-level positions without Senate
confirmation.

There's also the issue of the Department of Homeland Security itself.
Since its creation in the wake of the Sept. 11 attacks, the department
has been criticized for its size, scope and waste. ``It goes without
saying that I observed up-close the dysfunction, turf battles, and
inherent limitations in an entity that does so much,'' Matt Mayer, a
Homeland Security official under George W. Bush,
\href{http://reason.com/archives/2015/06/23/president-bush-was-right-before-he-was-w}{wrote}
in 2015. Report after report --- from
\href{https://www.hsgac.senate.gov/media/minority-media/final-coburn-oversight-report-finds-major-problems-in-dhs}{congressional
oversight committees}, from
\href{http://www.washingtonpost.com/wp-dyn/content/article/2008/09/16/AR2008091603200_pf.html}{the
Government Accountability Office} --- show an agency practically defined
by waste and dysfunction. And if the Trump years have shown anything, it
is that the agencies within D.H.S., and especially ICE and C.B.P., are
in desperate need of root-and-branch reform or some other fundamental
change.

Should President Trump fail to win re-election, perhaps the way to
prevent a replay of the abuse in Portland is to dismantle the
institution behind it. Just as local communities do not need militarized
police officers, the federal government does not need an alphabet soup
of militarized law enforcement agencies, as well as the cultures of
\href{https://www.propublica.org/article/secret-border-patrol-facebook-group-agents-joke-about-migrant-deaths-post-sexist-memes}{prejudice}
and
\href{https://www.aclu.org/issues/immigrants-rights/ice-and-border-patrol-abuses}{brutality}
that have gone along with them. If and when we close the book on Trump,
perhaps we should use the opportunity to close the book on Homeland
Security too.

\emph{The Times is committed to publishing}
\href{https://www.nytimes3xbfgragh.onion/2019/01/31/opinion/letters/letters-to-editor-new-york-times-women.html}{\emph{a
diversity of letters}} \emph{to the editor. We'd like to hear what you
think about this or any of our articles. Here are some}
\href{https://help.nytimes3xbfgragh.onion/hc/en-us/articles/115014925288-How-to-submit-a-letter-to-the-editor}{\emph{tips}}\emph{.
And here's our email:}
\href{mailto:letters@NYTimes.com}{\emph{letters@NYTimes.com}}\emph{.}

\emph{Follow The New York Times Opinion section on}
\href{https://www.facebookcorewwwi.onion/nytopinion}{\emph{Facebook}}\emph{,}
\href{http://twitter.com/NYTOpinion}{\emph{Twitter (@NYTopinion)}}
\emph{and}
\href{https://www.instagram.com/nytopinion/}{\emph{Instagram}}\emph{.}

Advertisement

\protect\hyperlink{after-bottom}{Continue reading the main story}

\hypertarget{site-index}{%
\subsection{Site Index}\label{site-index}}

\hypertarget{site-information-navigation}{%
\subsection{Site Information
Navigation}\label{site-information-navigation}}

\begin{itemize}
\tightlist
\item
  \href{https://help.nytimes3xbfgragh.onion/hc/en-us/articles/115014792127-Copyright-notice}{©~2020~The
  New York Times Company}
\end{itemize}

\begin{itemize}
\tightlist
\item
  \href{https://www.nytco.com/}{NYTCo}
\item
  \href{https://help.nytimes3xbfgragh.onion/hc/en-us/articles/115015385887-Contact-Us}{Contact
  Us}
\item
  \href{https://www.nytco.com/careers/}{Work with us}
\item
  \href{https://nytmediakit.com/}{Advertise}
\item
  \href{http://www.tbrandstudio.com/}{T Brand Studio}
\item
  \href{https://www.nytimes3xbfgragh.onion/privacy/cookie-policy\#how-do-i-manage-trackers}{Your
  Ad Choices}
\item
  \href{https://www.nytimes3xbfgragh.onion/privacy}{Privacy}
\item
  \href{https://help.nytimes3xbfgragh.onion/hc/en-us/articles/115014893428-Terms-of-service}{Terms
  of Service}
\item
  \href{https://help.nytimes3xbfgragh.onion/hc/en-us/articles/115014893968-Terms-of-sale}{Terms
  of Sale}
\item
  \href{https://spiderbites.nytimes3xbfgragh.onion}{Site Map}
\item
  \href{https://help.nytimes3xbfgragh.onion/hc/en-us}{Help}
\item
  \href{https://www.nytimes3xbfgragh.onion/subscription?campaignId=37WXW}{Subscriptions}
\end{itemize}
