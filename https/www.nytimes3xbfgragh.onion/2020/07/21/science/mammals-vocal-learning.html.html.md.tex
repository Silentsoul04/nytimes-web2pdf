Sections

SEARCH

\protect\hyperlink{site-content}{Skip to
content}\protect\hyperlink{site-index}{Skip to site index}

\href{https://www.nytimes3xbfgragh.onion/section/science}{Science}

\href{https://myaccount.nytimes3xbfgragh.onion/auth/login?response_type=cookie\&client_id=vi}{}

\href{https://www.nytimes3xbfgragh.onion/section/todayspaper}{Today's
Paper}

\href{/section/science}{Science}\textbar{}Nature's Noisiest Liars Carry
Secrets in Their Calls

\url{https://nyti.ms/2CTFgHv}

\begin{itemize}
\item
\item
\item
\item
\item
\item
\end{itemize}

Advertisement

\protect\hyperlink{after-top}{Continue reading the main story}

Supported by

\protect\hyperlink{after-sponsor}{Continue reading the main story}

Trilobites

\hypertarget{natures-noisiest-liars-carry-secrets-in-their-calls}{%
\section{Nature's Noisiest Liars Carry Secrets in Their
Calls}\label{natures-noisiest-liars-carry-secrets-in-their-calls}}

Many mammals that have loud calls to deceive other animals seem to have
a particular learning style in common.

\includegraphics{https://static01.graylady3jvrrxbe.onion/images/2020/08/04/science/19TB-VOCAL/19TB-VOCAL-articleLarge.jpg?quality=75\&auto=webp\&disable=upscale}

By Joshua Sokol

\begin{itemize}
\item
  July 21, 2020
\item
  \begin{itemize}
  \item
  \item
  \item
  \item
  \item
  \item
  \end{itemize}
\end{itemize}

A beast calls in the distance. Hearing a low rumble, you might imagine
the source will be an unholy cross between a wild boar and a chain saw.
The message is unmistakable: I'm here, I'm huge and you can either come
mate with me or stay out of my way.

Surprise! It's just a cuddly little koala.

Like online dating, the soundscape of the animal world is rife with
exaggerations about size, which animals use to scare off rivals and
attract mates. Gazelles,
\href{https://www.nytimes3xbfgragh.onion/2015/10/23/science/howler-monkey-species-deep-voice-testicle-size.html}{howler
monkeys,} bats and
\href{https://www.nature.com/articles/ncomms12739}{many more creatures}
have evolved to create calls with deep sonic frequencies that sound as
if they come from a much larger animal.

Now scientists have proposed this same underlying pressure to exaggerate
size might be linked to an even deeper mystery. It could have spurred
mammals toward developing the ability to make a wider array of possible
calls, to mimic sounds after hearing them and maybe even speech, what
scientists call vocal learning.

``We are offering one possible way for vocal learning to have evolved,''
says Maxime Garcia, a biologist at the University of Zurich in
Switzerland who suggested the relationship with his colleague, Andrea
Ravignani,
\href{https://royalsocietypublishing.org/doi/full/10.1098/rsbl.2020.0081?af=R}{in
the journal Biology Letters} this month*.*

Their idea builds off previous studies on vocal learning in humans.
Beyond just opera singers,
\href{https://www.nytimes3xbfgragh.onion/2018/11/07/science/beatboxing-mri-scanner.html}{beatboxers}
and Michael Winslow from the ``Police Academy'' movies, we all have some
level of control over the frequencies of our voices.

``I can tell you to lower your pitch or try to sound big, and you can
\emph{soound like thissss},'' said Katarzyna Pisanski at the University
of Lyon in France, affecting a deep voice.

But this power to voluntarily deceive seems to make us unique among our
closest relatives. Even captive apes seem to have only very limited
control over their voices. ``You can't tell another primate to do
this,'' she says.

Traditionally, scientists have reasoned that the vocal control of Homo
sapiens was a rung on the ladder of our evolution of speech. But in
2016, biologist David Reby and others, including Dr. Pisanski, argued
that, instead, humans became acoustic ``cheaters'' because of
evolutionary pressures to sound bigger and more masculine.

Now Dr. Garcia and Dr. Ravignani have applied the same idea to the
animal world.

Among many species that cheat by willfully modulating their voices,
previous experiments show a correlation: Many can also imitate sounds.
Elephants, for example, can make sounds through their mouths or their
long, stretchy trunks.

``Going through one way or the other, the characteristics will change
drastically, and change the impression of body size,'' Dr. Garcia says.
``And it happens they are vocal learners, too.''

One of the best examples may be Hoover, a harbor seal who spent much of
his life heckling visitors at the New England Aquarium in a thick,
impossibly human Maine accent. Hoover died in 1985. But scientists
confirmed
\href{https://www.sciencedirect.com/science/article/pii/S0960982219306852}{last
year} that seals --- even much less talented ones --- really could learn
to change the same sonic frequencies that also hint at their size.

At least 164 non-primate mammal species are either confirmed vocal
learners or related to one and can produce frequencies far from what you
might expect based on their body size, Dr. Garcia and Dr. Ravignani
found. Now they hope to see if screening for animals that exaggerate
size in their calls might be a good way to pick which animals to test
next for vocal learning abilities.

The idea of a deeper connection will be hard to prove or disprove, says
Dr. Pisanski, who didn't participate in the current study.

``The biggest impact this article will have is by looking at the role of
vocal control outside of primates, and comparing a large number of
mammals,'' she said. ``This is something we really need to do.''

Advertisement

\protect\hyperlink{after-bottom}{Continue reading the main story}

\hypertarget{site-index}{%
\subsection{Site Index}\label{site-index}}

\hypertarget{site-information-navigation}{%
\subsection{Site Information
Navigation}\label{site-information-navigation}}

\begin{itemize}
\tightlist
\item
  \href{https://help.nytimes3xbfgragh.onion/hc/en-us/articles/115014792127-Copyright-notice}{©~2020~The
  New York Times Company}
\end{itemize}

\begin{itemize}
\tightlist
\item
  \href{https://www.nytco.com/}{NYTCo}
\item
  \href{https://help.nytimes3xbfgragh.onion/hc/en-us/articles/115015385887-Contact-Us}{Contact
  Us}
\item
  \href{https://www.nytco.com/careers/}{Work with us}
\item
  \href{https://nytmediakit.com/}{Advertise}
\item
  \href{http://www.tbrandstudio.com/}{T Brand Studio}
\item
  \href{https://www.nytimes3xbfgragh.onion/privacy/cookie-policy\#how-do-i-manage-trackers}{Your
  Ad Choices}
\item
  \href{https://www.nytimes3xbfgragh.onion/privacy}{Privacy}
\item
  \href{https://help.nytimes3xbfgragh.onion/hc/en-us/articles/115014893428-Terms-of-service}{Terms
  of Service}
\item
  \href{https://help.nytimes3xbfgragh.onion/hc/en-us/articles/115014893968-Terms-of-sale}{Terms
  of Sale}
\item
  \href{https://spiderbites.nytimes3xbfgragh.onion}{Site Map}
\item
  \href{https://help.nytimes3xbfgragh.onion/hc/en-us}{Help}
\item
  \href{https://www.nytimes3xbfgragh.onion/subscription?campaignId=37WXW}{Subscriptions}
\end{itemize}
