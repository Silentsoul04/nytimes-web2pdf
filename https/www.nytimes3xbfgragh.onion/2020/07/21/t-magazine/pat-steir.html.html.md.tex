Sections

SEARCH

\protect\hyperlink{site-content}{Skip to
content}\protect\hyperlink{site-index}{Skip to site index}

\href{https://myaccount.nytimes3xbfgragh.onion/auth/login?response_type=cookie\&client_id=vi}{}

\href{https://www.nytimes3xbfgragh.onion/section/todayspaper}{Today's
Paper}

The Rembrandt Self-Portrait That Has Long Captivated Pat Steir

\url{https://nyti.ms/2CYkAxK}

\begin{itemize}
\item
\item
\item
\item
\item
\item
\end{itemize}

Advertisement

\protect\hyperlink{after-top}{Continue reading the main story}

Supported by

\protect\hyperlink{after-sponsor}{Continue reading the main story}

True Believers

\hypertarget{the-rembrandt-self-portrait-that-has-long-captivated-pat-steir}{%
\section{The Rembrandt Self-Portrait That Has Long Captivated Pat
Steir}\label{the-rembrandt-self-portrait-that-has-long-captivated-pat-steir}}

When she lived in Amsterdam, the New York-based artist went to see this
painting at least once a week.

\includegraphics{https://static01.graylady3jvrrxbe.onion/images/2020/07/10/t-magazine/art/steir-slide-1K6L/steir-slide-1K6L-videoSixteenByNine3000.jpg}

By Julia Felsenthal

\begin{itemize}
\item
  July 21, 2020
\item
  \begin{itemize}
  \item
  \item
  \item
  \item
  \item
  \item
  \end{itemize}
\end{itemize}

\href{https://www.nytimes3xbfgragh.onion/2019/02/27/magazine/rembrandt-jan-six.html}{Rembrandt
van Rijn} painted and etched his own image dozens of times over the
course of his life, but it's one of his earliest self-portraits, from
around 1628, that haunts the New York artist
\href{https://www.nytimes3xbfgragh.onion/2019/01/18/arts/design/pat-steir-barnes-foundation-waterfall-kiki-smith-feminist.html}{Pat
Steir}. In this work, simply titled ``Self-Portrait,'' the 22-year-old
Dutch master in the making renders himself in heavy chiaroscuro, his
gaze obscured and light bouncing off his jaw to illuminate an earlobe
and the edges of a helmet of frizzy hair. It's an oddly dispassionate
depiction. ``He painted it the way he would paint a model, without
personifying it,'' observes Steir. ``It's letting go of imparting wisdom
to the figure. Brilliance, scale, beauty --- letting go of all that.
That's what's hard to do.''

\href{https://www.nytimes3xbfgragh.onion/issue/t-magazine/2020/07/02/true-believers-art-issue}{\includegraphics{https://static01.graylady3jvrrxbe.onion/newsgraphics/2020/06/29/tmag-art-embeds-new/assets/images/art_issue_gif_special_editon.gif}}

Image

Rembrandt's ``Self-Portrait'' (circa 1628) at the Rijksmuseum in
Amsterdam.Credit...Courtesy of the Rijksmuseum

Steir should know: Now in her early 80s, she's made a career-long quest
of divesting her work of just that kind of painterly ego --- even if she
has in the past few years received the kind of recognition that might
give someone less grounded a big head. That includes two major
site-specific museum installations of her ethereal and monumentally
scaled
``\href{https://www.metmuseum.org/art/collection/search/497082}{Waterfall}''
paintings, the first at the
\href{https://www.barnesfoundation.org/whats-on/pat-steir-silent-secret-waterfalls}{Barnes
Foundation} in Philadelphia in early 2019, the next in an ongoing
(though temporarily shuttered) exhibition at the
\href{https://hirshhorn.si.edu/exhibitions/pat-steir/}{Hirshhorn Museum}
in Washington, D.C., where she has lined a circular gallery with 30
vibrant canvases that make up a three-dimensional color wheel. In 2018,
her painting
``\href{https://www.phillips.com/detail/pat-steir/NY010318/3}{Elective
Affinity Waterfall}'' (1992) sold at auction for a career-record \$2.3
million. And the filmmaker Veronica Gonzalez Peña just released
\href{https://patsteirthemovie.com/}{a sensitive documentary} about
Steir's life and unusual way of making art.

Steir paints using a carefully choreographed aleatory process inspired
by her late friend, the modernist composer
\href{https://www.nytimes3xbfgragh.onion/topic/person/john-cage}{John
Cage}, and informed by her interest in Zen meditation and Japanese and
Chinese art-making traditions. Standing on a cherry picker to reach the
top of her canvas, which often measures upward of seven feet tall, Steir
pours, flings or swabs a line of paint, then patiently waits for it to
trickle down in rivulets. She controls the color --- building it up on
the surface in dense layers --- the weight and viscosity of her pigments
and the force of the flick of her wrist. Gravity and atmospheric
conditions **** do the rest, creating a dance with chance that she finds
liberating (when things go well or awry, nature gets the credit). Her
drippy mark, her athletic way of working --- ``painting is also my
gym,'' she says --- and her outsize scale call to mind the Abstract
Expressionists, but Steir's work is a deliberate rejection of their
angsty, macho schtick. She describes her art not as abstract but as
nonobjective: Her lush, dramatic paintings resemble cascades of water,
but they are \emph{actually} cascades of oil paint, not landscape
paintings but literal landscapes. ``Like an open door, not a window,''
she explains.

Steir has been honing this process for the past 30 years, which may
account for why she finds Rembrandt's precocity so touching. Born Iris
Patricia Sukoneck in Newark, N.J., (she took the name Steir from her
flash-in-the-pan first marriage), she turned down a scholarship at Smith
College for the Pratt Institute, and entered the New York art world at a
moment when the prevailing wisdom held, she remembers, that ``a woman
couldn't be a good painter: She didn't have the balls.'' In her early
years, Steir made psychologically fraught figurative work, but by the
late '60s, she was beginning to wrestle with questions of
representation, producing paintings throughout the 1970s that resembled
pictograms, topographical charts and book mechanicals, using a personal
lexicon of childlike marks, cryptic symbols and scribbled words (she was
shaped by two influential relationships --- a friendship with
\href{https://www.nytimes3xbfgragh.onion/2004/12/17/arts/design/agnes-martin-abstract-painter-dies-at-92.html}{Agnes
Martin} and a romance turned lifelong friendship with
\href{https://www.nytimes3xbfgragh.onion/2007/04/09/arts/design/09lewitt.html}{Sol
LeWitt}, one of her collaborators in founding the **** downtown art-book
shop \href{https://www.printedmatter.org/}{Printed Matter}).

\includegraphics{https://static01.graylady3jvrrxbe.onion/images/2020/07/10/t-magazine/art/steir-slide-657V/steir-slide-657V-articleLarge.jpg?quality=75\&auto=webp\&disable=upscale}

In the '80s, inspired by the architectural discourse around
postmodernism and the rage for appropriation art, she turned to ****
what she called quotation painting. Steir's thing was to borrow the
hand, not the imagery, of her forebears. She produced a series of
self-portraits in the manner of painters like Courbet,
\href{https://www.nytimes3xbfgragh.onion/topic/person/henri-matisse}{Matisse}
--- and, yes, Rembrandt: By this point, she was living part-time near
his onetime Amsterdam home with her second husband, the Dutch book
publisher
\href{https://www.nytimes3xbfgragh.onion/2019/10/28/style/the-secret-language-of-birthdays.html}{Joost
Elffers}. **** In her most epic undertaking, Steir reproduced a floral
still life by the early 17th-century Flemish painter Jan Brueghel the
Elder at massive scale in 64 panels, each painted in the style of a
different artist
(\href{https://www.nytimes3xbfgragh.onion/topic/person/edouard-manet}{Manet},
\href{https://www.nytimes3xbfgragh.onion/topic/person/vincent-van-gogh}{Van
Gogh}, Rothko) or artistic period.
``\href{https://www.brooklynmuseum.org/opencollection/exhibitions/998}{The
Brueghel Series (A Vanitas of Style)}'' (1982-84) was a grand mash-up of
Western art movements that slyly inserted its author into the canon ---
two panels are in the manner of Steir herself. But it was also a way of
relinquishing control, using other artists' gestures to mediate her
relationship with her own.

``Such a relief,'' she remembers fondly, describing her journey as a
painter as first trying to express herself, then trying to express
something outside of herself, then arriving, at last, at ``not trying to
express anything.'' With her mature practice, ``I'm not trying to do
something to you. I'm not trying to make you see yourself, or make you
see even a waterfall. I'm giving you the opportunity to stand there and
become part of the painting. But if you want to walk by, there's nothing
that says you can't.''

\hypertarget{true-believers-art-issue}{%
\subsubsection{\texorpdfstring{\href{https://www.nytimes3xbfgragh.onion/issue/t-magazine/2020/07/02/true-believers-art-issue}{True
Believers Art
Issue}}{True Believers Art Issue}}\label{true-believers-art-issue}}

Advertisement

\protect\hyperlink{after-bottom}{Continue reading the main story}

\hypertarget{site-index}{%
\subsection{Site Index}\label{site-index}}

\hypertarget{site-information-navigation}{%
\subsection{Site Information
Navigation}\label{site-information-navigation}}

\begin{itemize}
\tightlist
\item
  \href{https://help.nytimes3xbfgragh.onion/hc/en-us/articles/115014792127-Copyright-notice}{©~2020~The
  New York Times Company}
\end{itemize}

\begin{itemize}
\tightlist
\item
  \href{https://www.nytco.com/}{NYTCo}
\item
  \href{https://help.nytimes3xbfgragh.onion/hc/en-us/articles/115015385887-Contact-Us}{Contact
  Us}
\item
  \href{https://www.nytco.com/careers/}{Work with us}
\item
  \href{https://nytmediakit.com/}{Advertise}
\item
  \href{http://www.tbrandstudio.com/}{T Brand Studio}
\item
  \href{https://www.nytimes3xbfgragh.onion/privacy/cookie-policy\#how-do-i-manage-trackers}{Your
  Ad Choices}
\item
  \href{https://www.nytimes3xbfgragh.onion/privacy}{Privacy}
\item
  \href{https://help.nytimes3xbfgragh.onion/hc/en-us/articles/115014893428-Terms-of-service}{Terms
  of Service}
\item
  \href{https://help.nytimes3xbfgragh.onion/hc/en-us/articles/115014893968-Terms-of-sale}{Terms
  of Sale}
\item
  \href{https://spiderbites.nytimes3xbfgragh.onion}{Site Map}
\item
  \href{https://help.nytimes3xbfgragh.onion/hc/en-us}{Help}
\item
  \href{https://www.nytimes3xbfgragh.onion/subscription?campaignId=37WXW}{Subscriptions}
\end{itemize}
