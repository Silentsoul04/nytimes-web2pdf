Sections

SEARCH

\protect\hyperlink{site-content}{Skip to
content}\protect\hyperlink{site-index}{Skip to site index}

\href{https://www.nytimes3xbfgragh.onion/section/us}{U.S.}

\href{https://myaccount.nytimes3xbfgragh.onion/auth/login?response_type=cookie\&client_id=vi}{}

\href{https://www.nytimes3xbfgragh.onion/section/todayspaper}{Today's
Paper}

\href{/section/us}{U.S.}\textbar{}Chaotic Scenes in Portland as Backlash
to Federal Deployment Grows

\url{https://nyti.ms/3hjjfAL}

\begin{itemize}
\item
\item
\item
\item
\item
\item
\end{itemize}

\href{https://www.nytimes3xbfgragh.onion/news-event/george-floyd-protests-minneapolis-new-york-los-angeles?action=click\&pgtype=Article\&state=default\&region=TOP_BANNER\&context=storylines_menu}{Race
and America}

\begin{itemize}
\tightlist
\item
  \href{https://www.nytimes3xbfgragh.onion/2020/07/26/us/protests-portland-seattle-trump.html?action=click\&pgtype=Article\&state=default\&region=TOP_BANNER\&context=storylines_menu}{Protesters
  Return to Other Cities}
\item
  \href{https://www.nytimes3xbfgragh.onion/2020/07/24/us/portland-oregon-protests-white-race.html?action=click\&pgtype=Article\&state=default\&region=TOP_BANNER\&context=storylines_menu}{Portland
  at the Center}
\item
  \href{https://www.nytimes3xbfgragh.onion/2020/07/23/podcasts/the-daily/portland-protests.html?action=click\&pgtype=Article\&state=default\&region=TOP_BANNER\&context=storylines_menu}{Podcast:
  Showdown in Portland}
\item
  \href{https://www.nytimes3xbfgragh.onion/interactive/2020/07/16/us/black-lives-matter-protests-louisville-breonna-taylor.html?action=click\&pgtype=Article\&state=default\&region=TOP_BANNER\&context=storylines_menu}{45
  Days in Louisville}
\end{itemize}

Advertisement

\protect\hyperlink{after-top}{Continue reading the main story}

Supported by

\protect\hyperlink{after-sponsor}{Continue reading the main story}

\hypertarget{chaotic-scenes-in-portland-as-backlash-to-federal-deployment-grows}{%
\section{Chaotic Scenes in Portland as Backlash to Federal Deployment
Grows}\label{chaotic-scenes-in-portland-as-backlash-to-federal-deployment-grows}}

The street demonstrations that have shuddered through Portland for 54
consecutive nights have drawn out a complicated mix of emotions and
grievances.

\includegraphics{https://static01.graylady3jvrrxbe.onion/images/2020/07/21/us/21unrest-portland01/21unrest-portland01-videoSixteenByNine3000-v2.jpg}

\href{https://www.nytimes3xbfgragh.onion/by/mike-baker}{\includegraphics{https://static01.graylady3jvrrxbe.onion/images/2020/05/19/reader-center/author-mike-baker/author-mike-baker-thumbLarge.png}}

By \href{https://www.nytimes3xbfgragh.onion/by/mike-baker}{Mike Baker}

\begin{itemize}
\item
  Published July 21, 2020Updated July 28, 2020
\item
  \begin{itemize}
  \item
  \item
  \item
  \item
  \item
  \item
  \end{itemize}
\end{itemize}

PORTLAND, Ore. --- With a ski helmet and goggles on her head, Allison
Hyder recounted how she had told relatives that she planned to stand at
the rear of protests in downtown
\href{https://www.nytimes3xbfgragh.onion/2020/07/28/us/portland-protests-fact-check.html}{Portland}.
But, in the early hours of Tuesday, the grandmother of five found
herself right up front, locking arms with other mothers dressed in
yellow.

Standing with a pack of other
\href{https://www.nytimes3xbfgragh.onion/interactive/2020/07/22/us/portland-protests.html}{protesters},
she chanted in front of the boarded-up entrance to the federal
courthouse. She remained resolute even as some in the crowd began prying
at the wood affixed to the building, leaving Ms. Hyder uneasy about
where things were headed.

``I am the face of `anarchy,''' Ms. Hyder declared. ``The people of the
U.S. need to know that moms, grandmas and nurses are out here in the
middle of the night demanding rights for everybody.''

The demonstrations that have shuddered through Portland for 54
consecutive nights have drawn out a complicated mix of grievances, with
a wide array of people expressing them using a multitude of tactics to
make sure they get heard.

In Oregon, a state with a deep history of racism that included
racial-exclusion laws that extended into the 20th century, the Portland
protests have persisted since George Floyd's killing even as Black Lives
Matter demonstrations have waned in many other parts of the country.

But some leaders in the Black community, grateful for a widespread
discussion on race, worry that what should be a moment for racial
justice in Portland could be squandered by violence. Business owners
supportive of change have been left demoralized by the mayhem the
protests have brought. The city's mayor, Ted Wheeler, despised by many
of those in the streets, has now been fighting to have federal officers
leave them alone.

Amid the Gordian knot of frustrations and escalations, most everyone
seems to agree about one thing:
\href{https://www.nytimes3xbfgragh.onion/2020/07/17/us/portland-protests.html}{The
combative deployment of camouflaged federal agents} has only made things
worse.

\includegraphics{https://static01.graylady3jvrrxbe.onion/images/2020/07/21/us/portland-vid/portland-vid-videoSixteenByNine3000-v2.jpg}

President Trump, in pushing a law-and-order message for his re-election
campaign, has embraced a dark vision of Portland as a lawless place
filled with ``anarchists'' who ``hate our country.'' His
administration's crackdown has brought armed officers from a variety of
federal agencies to the streets,
\href{https://www.nytimes3xbfgragh.onion/2020/07/18/us/portland-protests.html}{including
tactical units typically suited for handling drug smuggling}. They have
been firing tear gas and pulling protesters into unmarked vans.

The president's portrayal of Portland and the crackdown he has unleashed
have infuriated protesters who believe that Mr. Trump is trying to use
the city's unrest as political theater during an election year. He has
forced a federal police presence on a city that does not want it --- a
city with such a rich tradition of protest that an aide to another
Republican president, George H.W. Bush, reportedly referred to it as
Little Beirut.

Chad F. Wolf, the acting secretary of homeland security, said at a news
conference on Tuesday that he had tried to collaborate with Mayor
Wheeler and Gov. Kate Brown, asking them this month whether they were
overwhelmed and wanted resources, including personnel.

He said that when he asked what they needed to ``bring the violence to a
close and still allow the peaceful protesters to protest each and every
night,'' they told him to stay out of the city.

``I asked the mayor and governor, How long do you plan on having this
continue?'' Mr. Wolf said. ``Is it 82 nights or 92 nights?''

While the protests have consumed parts of downtown after dark, much of
the city has been left untouched. By day, boaters putter up the
Willamette River while joggers run down the trail alongside it. On
Monday evening, large groups of diners were eating on outdoor patios a
few blocks away from the county's Justice Center, where protesters were
amassing for the night.

To the protesters, the president's unusual deployment of federal power
has provided yet more compelling evidence that their fears about rising
fascism in the United States are justified.

Image

Federal agents clashed with protesters early Tuesday
morning.Credit...Mason Trinca for The New York Times

Image

Tear gas filled the air near the courthouse.Credit...Mason Trinca for
The New York Times

\includegraphics{https://static01.graylady3jvrrxbe.onion/images/2020/07/21/us/21unrest-portland04/merlin_174794304_4c7a018e-d416-4fbf-9eb8-a5a4de5cc4f6-articleLarge.jpg?quality=75\&auto=webp\&disable=upscale}

Image

Protesters gathered tear canisters and munitions cartridges left over
from the clash.Credit...Mason Trinca for The New York Times

Image

A protester who took precaution against tear gas but still felt its
effects.Credit...Mason Trinca for The New York Times

In the Portland area, activists aligned with the loosely organized group
known as antifa have long denounced police militarization and a
punishing criminal justice system, and have clashed with the police in
recent years. Some of the activists operating under the antifa umbrella,
wearing all black,
\href{https://www.nytimes3xbfgragh.onion/article/what-antifa-trump.html}{have
embraced anarchist-style tactics}, while others have shown up to
demonstrate peacefully.

The protests of the last seven weeks developed a near-nightly cycle of
conflict between protesters and the authorities, with officers reacting
to objects being thrown by protesters and protesters expressing alarm by
the use of tear gas that wafted over peaceful people.

Reflecting an oft-heard refrain, Angel Almanza, 44, said a government
could only suppress people so much before they would respond in kind.
``This has been an act of self-defense,'' Mr. Almanza said.

The strife on the streets escalated with the arrival of federal forces,
which have relied heavily on tear gas, munitions fired from
paintball-style guns and batons.

Among the others concerned by the federal crackdown was Joey Gibson, a
far-right activist who has long battled with Portland's antifa
demonstrators and was charged with a felony last year for his role in a
street brawl with activists. He said he found it somewhat frightening to
see video of
\href{https://www.nytimes3xbfgragh.onion/2020/07/20/us/portland-protests-navy-christopher-david.html}{one
officer whacking a Navy veteran with a baton}, and he worried that the
Trump administration was setting a precedent that would encourage other
presidents to embrace a more expansive use of federal forces.

``It is very concerning,'' Mr. Gibson said.

On Tuesday morning, dozens of agents moved around the streets and at
times threw people to the ground to detain them. From behind a wood
facade of the federal courthouse, officers at times used small tactical
holes to shoot less-lethal weapons or pepper spray. As federal officers
appeared to try detaining one person, others in the crowd rushed to free
the person.

The Portland authorities have cited continuing troubles with the
protesters, and on Tuesday the police said a jewelry store had been
looted. Protesters tried to light fires against the federal courthouse,
drawing officers back out to disperse them.

The nightly protests have also alarmed the owners of downtown
businesses, who were first hit with widespread looting in the aftermath
of Mr. Floyd's death in Minneapolis on Memorial Day and have struggled
to navigate the weeks since. Stacey Gibson, who owns a Subway sandwich
shop downtown, said the store's windows have been boarded up much of the
time, which has contributed to shrinking sales already limited by the
coronavirus and related lockdowns.

Ms. Gibson said she was frustrated that the city's Police Department did
not seem to have the resources to respond when there was a problem, and
she was upset that city leaders have been unable to find a resolution.
She is not optimistic about the future of her store, either.

``At this point, I'm just trying to figure out how to get out and to
minimize the damage,'' Ms. Gibson said.

Image

The Portland protests have persisted since George Floyd's killing even
as Black Lives Matter demonstrations have waned in parts of the
country.Credit...Mason Trinca for The New York Times

The protests after Mr. Floyd's death drew thousands to the streets in
Portland, creating powerful images of crowds lying facedown on the
Burnside Bridge in honor of Mr. Floyd.

While the numbers dwindled over the subsequent weeks and Governor Brown
expressed a belief that things were beginning to cool off, the crowds
have surged back in recent days, with protesters chanting ``Feds go
home'' and focusing much of their ire on the federal courthouse.

The demonstrations have continued to have a strong component of calls
for racial justice, including on Monday night, when thousands chanted
``Black Lives Matter'' and young Black activists led the predominantly
white crowd in speeches and song.

But some Black leaders have grown wary of the persistent unrest, fearing
that it is distracting from the goals of racial justice. Antoinette
Edwards, the former leader of the city's Office of Youth Violence
Prevention, called for peace and said she wanted to meet with
protesters.

``I will be there to have a conversation with folks: What can we do to
move this forward, for peace, for policy, for all of us?'' she said.

The mayor and the governor have both called on federal forces to leave
in order to ease tensions and give the city space to resolve the
differences.

Sergio Olmos contributed reporting from Portland and Zolan Kanno-Youngs
from Washington.

Advertisement

\protect\hyperlink{after-bottom}{Continue reading the main story}

\hypertarget{site-index}{%
\subsection{Site Index}\label{site-index}}

\hypertarget{site-information-navigation}{%
\subsection{Site Information
Navigation}\label{site-information-navigation}}

\begin{itemize}
\tightlist
\item
  \href{https://help.nytimes3xbfgragh.onion/hc/en-us/articles/115014792127-Copyright-notice}{©~2020~The
  New York Times Company}
\end{itemize}

\begin{itemize}
\tightlist
\item
  \href{https://www.nytco.com/}{NYTCo}
\item
  \href{https://help.nytimes3xbfgragh.onion/hc/en-us/articles/115015385887-Contact-Us}{Contact
  Us}
\item
  \href{https://www.nytco.com/careers/}{Work with us}
\item
  \href{https://nytmediakit.com/}{Advertise}
\item
  \href{http://www.tbrandstudio.com/}{T Brand Studio}
\item
  \href{https://www.nytimes3xbfgragh.onion/privacy/cookie-policy\#how-do-i-manage-trackers}{Your
  Ad Choices}
\item
  \href{https://www.nytimes3xbfgragh.onion/privacy}{Privacy}
\item
  \href{https://help.nytimes3xbfgragh.onion/hc/en-us/articles/115014893428-Terms-of-service}{Terms
  of Service}
\item
  \href{https://help.nytimes3xbfgragh.onion/hc/en-us/articles/115014893968-Terms-of-sale}{Terms
  of Sale}
\item
  \href{https://spiderbites.nytimes3xbfgragh.onion}{Site Map}
\item
  \href{https://help.nytimes3xbfgragh.onion/hc/en-us}{Help}
\item
  \href{https://www.nytimes3xbfgragh.onion/subscription?campaignId=37WXW}{Subscriptions}
\end{itemize}
