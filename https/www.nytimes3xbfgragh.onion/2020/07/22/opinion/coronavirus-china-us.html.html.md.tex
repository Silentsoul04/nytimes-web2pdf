Sections

SEARCH

\protect\hyperlink{site-content}{Skip to
content}\protect\hyperlink{site-index}{Skip to site index}

\href{https://myaccount.nytimes3xbfgragh.onion/auth/login?response_type=cookie\&client_id=vi}{}

\href{https://www.nytimes3xbfgragh.onion/section/todayspaper}{Today's
Paper}

\href{/section/opinion}{Opinion}\textbar{}My Relatives in Wuhan
Survived. My Uncle in New York Did Not.

\url{https://nyti.ms/3jtEqC8}

\begin{itemize}
\item
\item
\item
\item
\item
\end{itemize}

Advertisement

\protect\hyperlink{after-top}{Continue reading the main story}

\href{/section/opinion}{Opinion}

Supported by

\protect\hyperlink{after-sponsor}{Continue reading the main story}

\hypertarget{my-relatives-in-wuhan-survived-my-uncle-in-new-york-did-not}{%
\section{My Relatives in Wuhan Survived. My Uncle in New York Did
Not.}\label{my-relatives-in-wuhan-survived-my-uncle-in-new-york-did-not}}

My father, a Chinese pulmonologist, believes his brother could have been
saved.

By Yi Rao

Dr. Rao is a molecular neurobiologist in China.

\begin{itemize}
\item
  July 22, 2020
\item
  \begin{itemize}
  \item
  \item
  \item
  \item
  \item
  \end{itemize}
\end{itemize}

\includegraphics{https://static01.graylady3jvrrxbe.onion/images/2020/07/20/opinion/20YiRao/20YiRao-articleLarge.jpg?quality=75\&auto=webp\&disable=upscale}

\href{https://cn.nytimes3xbfgragh.onion/opinion/20200723/coronavirus-china-us/}{阅读简体中文版}\href{https://cn.nytimes3xbfgragh.onion/opinion/20200723/coronavirus-china-us/zh-hant/}{閱讀繁體中文版}

BEIJING --- Eight is thought to be a lucky number in China because in
Chinese it sounds like the word for ``fortune''; 444 is a bad number
because it rings like ``death''; 520 sounds like ``I love you.''

Having always disliked superstition, I was dismayed to receive a message
by WeChat at 4:44 p.m. on May 20, Beijing time, informing me that my
Uncle Eric, who lived in New York, had died from Covid-19. He was 74.

Uncle Eric was a pharmacist, so presumably he contracted the virus from
a patient who had visited his shop in Queens. Infected in March, he was
sick for more than two months. He was kept on a ventilator until his
last 10 days: By then, he was deemed incurable and the ventilator was
redirected to other patients who might be saved.

The medical trade runs in my family. I now preside over a medical
university in Beijing with 19 affiliated hospitals. I studied medicine
because my father was a doctor, a pulmonary physician. He decided to
study medicine after losing his mother to a minor infection when he was
13. My father did not expect to lose a brother 15 years his junior to a
disease in his own specialty: the respiratory system.

My father (Weihua) and Eric (Houhua) were first separated in 1947. My
father, then 17, stayed behind in Nanchang, the capital of Jiangxi
Province, in central-southern China, to finish his education, while
Eric, age 2, and other brothers and a sister sailed to Taiwan with their
parents. With the end of World War II, Taiwan had been returned to China
after five decades of Japanese occupation, and there were job
opportunities there.

The family did not anticipate what would happen in 1949: The Communist
takeover of mainland China --- and, for them, the beginning of another
kind of, and very long, separation.

My father completed his medical education in Nanchang and had graduate
training with one of the top respiratory physicians in Shanghai, but in
the 1960s the
\href{https://www.nytimes3xbfgragh.onion/2016/05/15/world/asia/china-cultural-revolution-explainer.html}{Cultural
Revolution} then took him to a small town and after that to a village,
where he was the sole doctor. He moved back to a major hospital in
Nanchang in 1972.

In the mid-1970s, my grandfather sent him --- by way of Fiji --- a
letter at a previous address, and miraculously it arrived.

Soon, Uncle Eric became their emissary.

Uncle Eric was the first member of my family to become an American
citizen. He arrived in San Francisco in the late 1970s, drawn to an
economic powerhouse of a country, so starkly different from what he had
grown up with in Taiwan.

It was 35 years before the brothers met again, in 1982. My father was a
visiting scholar for a year at the Cardiovascular Research Institute at
the University of California, San Francisco, where he conducted research
on pulmonary edema, and he received a few months of clinical training in
the intensive care unit at what is now called the Zuckerberg San
Francisco General Hospital and Trauma Center.

In the early 1980s, the gap between China and the United States was
gigantic. And my father has always been grateful for the education he
received at U.C.S.F. and the kindness and generosity of the Americans he
met.

He brought his American training back to Nanchang to establish the first
I.C.U. in Jiangxi Province and one of the first I.C.U.s in China. He
also established one of the first --- if not the very first ---
institute of molecular medicine in China.

In 1985, I followed in his footsteps and in those of my uncles --- Uncle
Tim (Xinghua) had immigrated to California as well: I went to San
Francisco to study for my Ph.D., also at U.C.S.F. My younger brother
moved to the United States a few years later.

In the 1990s, with the collapse of the Soviet model, America seemed to
be the only other exemplar left. Having studied in the United States and
with plans to work and live there for the long haul, I applied for
American citizenship and obtained it in 2000. My children were born in
the United States.

But then 9/11 happened, and this axis of evil emerged: Dick Cheney (vice
president); Paul Wolfowitz (deputy secretary of defense); David
Addington (counsel to the vice president); John Yoo (Justice Department
lawyer and author of the
``\href{https://archive.nytimes3xbfgragh.onion/www.nytimes3xbfgragh.onion/ref/international/24MEMO-GUIDE.html?}{Torture
Memos}''). These men were ready to do anything to advance their agenda,
imposing their own law --- meaning, really, no proper laws and no rule
of law --- in Iraq, at Guantánamo and elsewhere. And too many Americans
went along. That period proved to me that America was not the democratic
beacon many of us had thought it to be.

I first started looking into how to renounce my U.S. citizenship while I
lived in Chicago and then again after moving back to China in 2007. I
completed the process in 2011 --- a decision that has been validated
since by the advent of President Trump and Trumpism, which are a natural
expansion of what was put in motion after 9/11.

Uncle Eric never returned to mainland China.

By the time my father retired in 2005, at 75, he had treated countless
respiratory and I.C.U. patients in China. He had worked through the SARS
epidemic in 2002-3, issuing dark predictions that the virus, or
something like it, would come back. He and I debate whether the new
coronavirus proves his prediction right.

As Covid-19 began to spread earlier this year, my father, now 90 and
long retired, would send me advice about how to treat the disease so
that I could relay it to other doctors, including the one leading
response efforts in the city of Wuhan, the pandemic's epicenter early
on.

Our family has 12 members in Wuhan, mostly on my mother's side, and six
in New York, mostly on my father's side. All my relatives in Wuhan are
safe. Uncle Eric died in New York after the pandemic had moved to the
United States --- the world's strongest country militarily, the richest
economically and the most advanced medically.

The United States had two months or more to learn from China's
experience with this coronavirus, and it could have done much more to
lower infection rates and fatalities. My father is struggling to accept
his brother's death partly, too, because he believes that he could have
treated Uncle Eric --- that in China Uncle Eric would have been saved.

As the pandemic rages on in the United States and throughout the world,
with some smaller outbreaks in China, the United States and China are
not collaborating, but competing, in the search for a successful vaccine
for the virus and treatment measures for the disease.

My father's family has been divided for most of his life, separated
mostly by the decisions of political leaders. For a long time, the
United States seemed like the better place to live --- for those lucky
enough to have such a choice.

Now, my father and Uncle Eric have been separated once again. This time
that outcome doesn't speak well of America.

Yi Rao is the president of Capital Medical University, a chair professor
at Peking University and the director of the Chinese Institute for Brain
Research, in Beijing.

\emph{The Times is committed to publishing}
\href{https://www.nytimes3xbfgragh.onion/2019/01/31/opinion/letters/letters-to-editor-new-york-times-women.html}{\emph{a
diversity of letters}} \emph{to the editor. We'd like to hear what you
think about this or any of our articles. Here are some}
\href{https://help.nytimes3xbfgragh.onion/hc/en-us/articles/115014925288-How-to-submit-a-letter-to-the-editor}{\emph{tips}}\emph{.
And here's our email:}
\href{mailto:letters@NYTimes.com}{\emph{letters@NYTimes.com}}\emph{.}

\emph{Follow The New York Times Opinion section on}
\href{https://www.facebookcorewwwi.onion/nytopinion}{\emph{Facebook}}\emph{,}
\href{http://twitter.com/NYTOpinion}{\emph{Twitter (@NYTopinion)}}
\emph{and}
\href{https://www.instagram.com/nytopinion/}{\emph{Instagram}}\emph{.}

Advertisement

\protect\hyperlink{after-bottom}{Continue reading the main story}

\hypertarget{site-index}{%
\subsection{Site Index}\label{site-index}}

\hypertarget{site-information-navigation}{%
\subsection{Site Information
Navigation}\label{site-information-navigation}}

\begin{itemize}
\tightlist
\item
  \href{https://help.nytimes3xbfgragh.onion/hc/en-us/articles/115014792127-Copyright-notice}{©~2020~The
  New York Times Company}
\end{itemize}

\begin{itemize}
\tightlist
\item
  \href{https://www.nytco.com/}{NYTCo}
\item
  \href{https://help.nytimes3xbfgragh.onion/hc/en-us/articles/115015385887-Contact-Us}{Contact
  Us}
\item
  \href{https://www.nytco.com/careers/}{Work with us}
\item
  \href{https://nytmediakit.com/}{Advertise}
\item
  \href{http://www.tbrandstudio.com/}{T Brand Studio}
\item
  \href{https://www.nytimes3xbfgragh.onion/privacy/cookie-policy\#how-do-i-manage-trackers}{Your
  Ad Choices}
\item
  \href{https://www.nytimes3xbfgragh.onion/privacy}{Privacy}
\item
  \href{https://help.nytimes3xbfgragh.onion/hc/en-us/articles/115014893428-Terms-of-service}{Terms
  of Service}
\item
  \href{https://help.nytimes3xbfgragh.onion/hc/en-us/articles/115014893968-Terms-of-sale}{Terms
  of Sale}
\item
  \href{https://spiderbites.nytimes3xbfgragh.onion}{Site Map}
\item
  \href{https://help.nytimes3xbfgragh.onion/hc/en-us}{Help}
\item
  \href{https://www.nytimes3xbfgragh.onion/subscription?campaignId=37WXW}{Subscriptions}
\end{itemize}
