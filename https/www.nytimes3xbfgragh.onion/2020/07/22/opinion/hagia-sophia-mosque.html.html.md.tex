\href{/section/opinion}{Opinion}\textbar{}The Hagia Sophia Was a
Cathedral, a Mosque and a Museum. It's Converting Again.

\url{https://nyti.ms/2OKBEtW}

\begin{itemize}
\item
\item
\item
\item
\item
\item
\end{itemize}

\includegraphics{https://static01.graylady3jvrrxbe.onion/images/2020/07/23/opinion/23hagia_print/merlin_159497028_90a3bcb5-f68f-4262-99cb-474ecf102154-articleLarge.jpg?quality=75\&auto=webp\&disable=upscale}

Sections

\protect\hyperlink{site-content}{Skip to
content}\protect\hyperlink{site-index}{Skip to site index}

\href{/section/opinion}{Opinion}

\hypertarget{the-hagia-sophia-was-a-cathedral-a-mosque-and-a-museum-its-converting-again}{%
\section{The Hagia Sophia Was a Cathedral, a Mosque and a Museum. It's
Converting
Again.}\label{the-hagia-sophia-was-a-cathedral-a-mosque-and-a-museum-its-converting-again}}

Changing the secular space back into a religious one is a risk for the
World Heritage Site.

In 2019, the Hagia-Sophia was the most-visited museum in
Turkey.Credit...Piotr Redlinski for The New York Times

Supported by

\protect\hyperlink{after-sponsor}{Continue reading the main story}

By
\href{https://www.nytimes3xbfgragh.onion/interactive/opinion/editorialboard.html}{The
Editorial Board}

The editorial board is a group of opinion journalists whose views are
informed by expertise, research, debate and certain longstanding ****
\href{https://www.nytimes3xbfgragh.onion/interactive/2018/opinion/editorialboard.html?module=inline}{values}.
It is separate from the newsroom.

\begin{itemize}
\item
  July 22, 2020
\item
  \begin{itemize}
  \item
  \item
  \item
  \item
  \item
  \item
  \end{itemize}
\end{itemize}

On Friday, after 86 years as a museum, the great
\href{https://www.nytimes3xbfgragh.onion/2020/07/24/world/europe/turkey-hagia-sophia-mosque-prayers.html}{Hagia
Sophia in Istanbul} will once again echo with Muslim prayers. To Turkish
Islamists, the
\href{https://www.nytimes3xbfgragh.onion/2020/07/10/world/europe/hagia-sophia-erdogan.html}{conversion}
marks the fulfillment of a long-held dream of restoring a symbol of
Ottoman grandeur. For many others around the world, the change is a
dismaying setback for one of the world's greatest architectural and
cultural landmarks.

Grandly arrayed on a hilltop over the Bosporus where it divides Europe
and Asia, the Hagia Sophia's 15-century history is suffused with events,
myths and symbols important to both East and West. Built in the sixth
century by a Byzantine emperor, Justinian I, as the premier cathedral of
the Roman Empire and dedicated to ``Holy Wisdom,'' it was for almost
1,000 years the largest church in the world, a temple so majestic that
upon its dedication the emperor is said to have proclaimed, ``Solomon, I
have surpassed thee!'' Its influence on history and architecture and
religion, Christian and Islamic, is profound.

When Constantinople fell to Ottoman forces in 1453, Mehmed II the
Conqueror converted it to a mosque, the Great Mosque of Ayasofya, and
with time the Byzantine mosaics were covered over or destroyed and four
great minarets were raised around the structure. It remained a mosque
until 1934, when Mustafa Kemal Ataturk, the founder of the secular,
modern republic of Turkey, transformed the Hagia Sophia into a museum,
exposing long-concealed mosaics and marble floor decorations, in what
was seen as a bid to free the monument, and the nation, from myths of
sacred conquest.

\includegraphics{https://static01.graylady3jvrrxbe.onion/images/2020/07/22/opinion/22hagia2/merlin_174145167_9d64c08a-933c-4f77-990d-995fc7d3ef09-articleLarge.jpg?quality=75\&auto=webp\&disable=upscale}

It became the
\href{http://en.istanbul.gov.tr/the-most-visited-museums-of-turkey-hagia-sophia-museum}{most-visited
museum} in Turkey, attracting about 3.7 million visitors in 2019. It was
designated a World Heritage Site by UNESCO, identified as a landmark of
exceptional cultural significance to all humanity, worthy of
conservation.

Why President Recep Tayyip Erdogan chose to reverse Ataturk's decision
is a matter of some conjecture. A product of an Islamist political
tradition, he said he was unable to sleep on the night he issued the
presidential decree making the change. Only a year earlier he had argued
against the conversion. What is clear is that despite the great powers
Mr. Erdogan has seized over 17 years in power as prime minister and
president, his current political standing is shaky, and he needs to feed
his nationalist base.

In his
\href{https://www.nytimes3xbfgragh.onion/2020/07/10/world/europe/hagia-sophia-erdogan.html}{address
to the nation on July 10} announcing the conversion, Mr. Erdogan made no
mention of Ataturk. There was no need --- his speech was preceded by a
ruling of the Council of State, the highest administrative court of the
country, nullifying Ataturk's decree. And in his speech, Mr. Erdogan
extensively quoted Sultan Mehmed's will, calling down frightful curses
on anyone who would change the Hagia Sophia's status.

Image

President Recep Tayyip Erdogan visiting the Hagia Sophia on July
19.Credit...Turkish President Office

Reversing Ataturk's secular legacy plays well among Turkish
nationalists, for whom the museum inside the Hagia Sophia long
represented a humiliating foreign imposition and a blot on the Ottoman
past they glorify. And evidently not only nationalists. The conversion
of the museum has drawn little criticism within Turkey and among Muslims
outside, and all political parties save one applauded the change.

The reaction from Christian leaders has been relatively muted, perhaps
for fear of fomenting sectarian strife. Pope Francis said only that he
was ``pained,'' while the Eastern Orthodox patriarch of Constantinople,
Bartholomew, who as a resident of Turkey needs to be cautious in his
pronouncements, expressed regret that the Hagia Sophia would cease being
``a place and symbol of encounter, dialogue and peaceful coexistence of
peoples and cultures.''

UNESCO was more direct. A
\href{https://en.unesco.org/news/unesco-statement-hagia-sophia-istanbul}{statement}
from the organization said it ``deeply regrets'' that the decision was
made without any prior discussion, adding: ``Hagia Sophia is an
architectural masterpiece and a unique testimony to interactions between
Europe and Asia over the centuries. Its status as a museum reflects the
universal nature of its heritage, and makes it a powerful symbol for
dialogue.'' The statement also warned that alterations to physical
structures or changes to accessibility of the site could violate the
\href{https://whc.unesco.org/en/statesparties/}{1972 World Heritage
Convention}, to which Turkey was a signatory.

Mr. Erdogan, for his part, has sought to reassure the world that when
not being used for prayer, the Hagia Sophia would remain open to the
public, and that Christian frescoes would remain on display, though
covered with curtains during Muslim prayers.

It is critical that at least on these matters, he be held to his word.
It is a sad reflection on the state of Turkey's democracy that a
monument of such global importance and value should become an
authoritarian leader's political tool. But what's done is done; there is
no chance that Mr. Erdogan would reverse his decree, even if he could,
without firing the fury of his base.

Image

The Hagia Sophia is~one of the world's greatest architectural and
cultural landmarks.Credit...Umit Bektas/Reuters

But the Hagia Sophia remains a World Heritage Site in the most profound
sense of the designation, a structure of surpassing beauty with a deep
overlay of the histories of East and West, Christianity and Islam. That
need not preclude prayer; nor should it preclude Turks from feeling a
powerful connection to a monument that has been the pride of their
nation for centuries. But like the damaged Notre-Dame in Paris, or the
Acropolis in Athens, that must not undermine its calling as a place of
exceptional significance to all humanity.

In converting the Hagia Sophia to a mosque, Mr. Erdogan has assumed the
weighty responsibility of a custodian of one of the world's cultural
landmarks. He ought not be allowed to forget that.

\emph{The Times is committed to publishing}
\href{https://www.nytimes3xbfgragh.onion/2019/01/31/opinion/letters/letters-to-editor-new-york-times-women.html}{\emph{a
diversity of letters}} \emph{to the editor. We'd like to hear what you
think about this or any of our articles. Here are some}
\href{https://help.nytimes3xbfgragh.onion/hc/en-us/articles/115014925288-How-to-submit-a-letter-to-the-editor}{\emph{tips}}\emph{.
And here's our email:}
\href{mailto:letters@NYTimes.com}{\emph{letters@NYTimes.com}}\emph{.}

\emph{Follow The New York Times Opinion section on}
\href{https://www.facebookcorewwwi.onion/nytopinion}{\emph{Facebook}}\emph{,}
\href{http://twitter.com/NYTOpinion}{\emph{Twitter (@NYTopinion)}}
\emph{and}
\href{https://www.instagram.com/nytopinion/}{\emph{Instagram}}\emph{.}

Advertisement

\protect\hyperlink{after-bottom}{Continue reading the main story}

\hypertarget{site-index}{%
\subsection{Site Index}\label{site-index}}

\hypertarget{site-information-navigation}{%
\subsection{Site Information
Navigation}\label{site-information-navigation}}

\begin{itemize}
\tightlist
\item
  \href{https://help.nytimes3xbfgragh.onion/hc/en-us/articles/115014792127-Copyright-notice}{©~2020~The
  New York Times Company}
\end{itemize}

\begin{itemize}
\tightlist
\item
  \href{https://www.nytco.com/}{NYTCo}
\item
  \href{https://help.nytimes3xbfgragh.onion/hc/en-us/articles/115015385887-Contact-Us}{Contact
  Us}
\item
  \href{https://www.nytco.com/careers/}{Work with us}
\item
  \href{https://nytmediakit.com/}{Advertise}
\item
  \href{http://www.tbrandstudio.com/}{T Brand Studio}
\item
  \href{https://www.nytimes3xbfgragh.onion/privacy/cookie-policy\#how-do-i-manage-trackers}{Your
  Ad Choices}
\item
  \href{https://www.nytimes3xbfgragh.onion/privacy}{Privacy}
\item
  \href{https://help.nytimes3xbfgragh.onion/hc/en-us/articles/115014893428-Terms-of-service}{Terms
  of Service}
\item
  \href{https://help.nytimes3xbfgragh.onion/hc/en-us/articles/115014893968-Terms-of-sale}{Terms
  of Sale}
\item
  \href{https://spiderbites.nytimes3xbfgragh.onion}{Site Map}
\item
  \href{https://help.nytimes3xbfgragh.onion/hc/en-us}{Help}
\item
  \href{https://www.nytimes3xbfgragh.onion/subscription?campaignId=37WXW}{Subscriptions}
\end{itemize}
