Sections

SEARCH

\protect\hyperlink{site-content}{Skip to
content}\protect\hyperlink{site-index}{Skip to site index}

\href{https://myaccount.nytimes3xbfgragh.onion/auth/login?response_type=cookie\&client_id=vi}{}

\href{https://www.nytimes3xbfgragh.onion/section/todayspaper}{Today's
Paper}

The Man Who Brian Eno Called `the Daddy of Us All'

\url{https://nyti.ms/2ZQQ16s}

\begin{itemize}
\item
\item
\item
\item
\item
\item
\end{itemize}

Advertisement

\protect\hyperlink{after-top}{Continue reading the main story}

Supported by

\protect\hyperlink{after-sponsor}{Continue reading the main story}

True Believers

\hypertarget{the-man-who-brian-eno-called-the-daddy-of-us-all}{%
\section{The Man Who Brian Eno Called `the Daddy of Us
All'}\label{the-man-who-brian-eno-called-the-daddy-of-us-all}}

La Monte Young, the composer who quietly shaped much of contemporary
Western music, reaches his last act.

\includegraphics{https://static01.graylady3jvrrxbe.onion/images/2020/07/13/t-magazine/13tmag-young-slide-M0J6/13tmag-young-slide-M0J6-articleLarge.jpg?quality=75\&auto=webp\&disable=upscale}

By \href{https://www.nytimes3xbfgragh.onion/by/m-h-miller}{M.H. Miller}

\begin{itemize}
\item
  Published July 22, 2020Updated July 23, 2020
\item
  \begin{itemize}
  \item
  \item
  \item
  \item
  \item
  \item
  \end{itemize}
\end{itemize}

THE COMPOSER
\href{https://www.nytimes3xbfgragh.onion/2015/08/23/arts/music/la-monte-young-is-still-patiently-working-on-a-glacial-scale.html}{LA
MONTE YOUNG} held his 84th birthday party on a drab, rainy Sunday
afternoon last October inside the Church Street building where he has
lived with his wife, the 80-year-old artist
\href{https://www.nytimes3xbfgragh.onion/2015/04/03/arts/design/young-and-zazeelas-dream-house-is-getting-a-new-lease-at-dia.html}{Marian
Zazeela}, since 1963. That people like Young and Zazeela can still be
found in TriBeCa in 2020 is nothing short of remarkable. Young is
certainly the lone musician alive today who can claim to be no more than
two degrees removed from three of the greatest icons of 20th-century
modern music:
\href{https://www.nytimes3xbfgragh.onion/topic/person/igor-stravinsky}{Igor
Stravinsky} (who taught
\href{https://newsroom.ucla.edu/faculty-bulletin-board/in-memoriam--leading-music-scholar-robert-m--stevenson}{Robert
Stevenson}, the musicologist who was Young's mentor in college),
\href{https://www.nytimes3xbfgragh.onion/topic/person/charlie-parker}{Charlie
Parker} (who played with the saxophonist
\href{https://www.nytimes3xbfgragh.onion/2020/04/16/arts/music/lee-konitz-dead-coronavirus.html}{Lee
Konitz}, with whom Young himself played as a session musician in Los
Angeles) and
\href{https://www.nytimes3xbfgragh.onion/topic/person/lou-reed}{Lou
Reed} (who co-founded the
\href{https://www.nytimes3xbfgragh.onion/topic/organization/the-velvet-underground}{Velvet
Underground} with one of Young's bandmates,
\href{https://www.nytimes3xbfgragh.onion/2018/10/11/arts/music/velvet-underground-experience-john-cale.html}{John
Cale}). He is often considered the founder of minimalist music, having
stripped composition down to its bare elements of rhythm, sound and
tempo as early as the late '50s. His compositions can alternate between
abrasive (the 30-minute-long ``Day of Niagara,'' recorded in 1965 with
Zazeela, along with Cale and sometime Velvet Underground collaborators
\href{https://www.nytimes3xbfgragh.onion/2016/04/10/arts/artsspecial/tony-conrad-experimental-filmmaker-and-musician-dies-at-76.html}{Tony
Conrad} and
\href{https://www.nytimes3xbfgragh.onion/2011/05/06/arts/music/angus-maclise-of-velvet-underground-in-dreamweapon.html}{Angus
MacLise}, is a loud score that sounds not unlike an airplane engine and
predated the noise rock of Reed's ``Metal Machine Music'' and other
staples of early punk by about a decade) and hauntingly beautiful (such
as 1958's ``Trio for Strings,'' a three-hour piece that effectively
created the language of minimalist music through the use of sustained
tones punctuated by brief gasps of silence). In the 1960s, he introduced
generations of musicians to the concepts of drones and sustained
repetition and began incorporating Indian classical music into his work,
about five years before
\href{https://www.nytimes3xbfgragh.onion/2001/12/01/arts/george-harrison-quiet-beatle-and-lead-guitarist-dies-at-58.html}{George
Harrison} had ever heard of the sitar.

\href{https://www.nytimes3xbfgragh.onion/issue/t-magazine/2020/07/02/true-believers-art-issue}{\includegraphics{https://static01.graylady3jvrrxbe.onion/newsgraphics/2020/06/29/tmag-art-embeds-new/assets/images/art_issue_gif_special_editon.gif}}

I arrived to his party trailing two deliverymen carrying plastic bags of
Indian takeout to feed the two dozen or so guests. The banality of this
entrance, compared to what I was entering into, was dislocating. On the
second floor of the building, where a small crowd had encircled him,
Young was seated in what might be called his living room, though it was
really more of a small clearing amid the collected detritus of 60 or so
years. There were layers of boxes containing records and papers and home
recordings, all of which I suspected to be invaluable in some way to
understanding his musical output. Young has recorded every performance
and rehearsal he's ever done, though he rarely releases anything to the
public. ``I'm only interested in releasing masterpieces,'' he'd tell me
later.

\includegraphics{https://static01.graylady3jvrrxbe.onion/images/2020/07/13/t-magazine/13tmag-young-slide-EQC4/13tmag-young-slide-EQC4-articleLarge.jpg?quality=75\&auto=webp\&disable=upscale}

As the years go on and Young gets older, it becomes only clearer how
impossible it would be to conceive of most postwar musical innovations,
from distortion to electronic music to sampling, without him. He is
prone to statements --- that he delivers with such deadpan
matter-of-factness that it is hard not to take him at his word --- such
as, ``I don't want to be overbearing, but I think I'm one of the
greatest musicians who ever lived.'' Since so little of his music has
ever been available to listen to, much of the evidence for this rests
mostly on Young himself.

Later, the image that would stay with me from Young's house was of an
old cathode-ray tube TV, stacked atop an older, worse TV, with a Post-it
stuck to its screen noting the channel for CNN; the TV was, in fact,
turned to CNN, but the signal kept going out, the picture ****
alternating between the blurred image of the 24-hour news cycle and a
dead screen the color of a clear morning sky. This acted as an
occasional reminder of the reality that existed just outside his door,
but each time the picture cut out, it was as if the force of Young
himself --- his very presence in the room --- was overpowering it.

Young was wearing small sunglasses, leather gloves and a torn jean vest
with biker chains. The vest --- under which he was shirtless, though the
front of his body was obscured by his beard, a big, white scraggly tuft
of hair --- appeared as if it hadn't been removed in several decades. He
looked simultaneously like a dignified spiritual leader and a Hells
Angel who never left his post at the Altamont Speedway in December 1969.
Across from him, dressed entirely in purple and sitting in a wheelchair,
was Zazeela, and seated on the floor, at his feet, was a tabla player
named Naren Budhkar as well as \href{http://jungheechoi.com/}{Jung Hee
Choi}, an artist and musician who has studied with Young and Zazeela for
the last 20 years, and whom Young describes as his ``senior disciple.''
Like Young, she was holding a microphone. In a chair behind and to their
left was \href{https://www.hansfordrowe.com/}{Hansford Rowe}, the bass
player, a wiry man with short cropped hair, and to the right was an
empty chair. The guitarist was late. ``Should we have paid him more?''
Choi asked Young jokingly. ``I think we should pay him less,'' he
responded in a more ambiguous tone. When the guitar player, Jon Catler,
did arrive, his entrance was hardly acknowledged beyond the fact that
everyone began to play.

\hypertarget{la-monte-youngs-trio-for-strings-original-full-length-version-195819982015}{%
\subsubsection{La Monte Young's ``Trio for Strings Original Full Length
Version''
(1958--1998--2015)}\label{la-monte-youngs-trio-for-strings-original-full-length-version-195819982015}}

An excerpt from the composer's 2015 performance at the ``Dia 15 VI 13
545 West 22 Street Dream House'' in New York City.

I RETURNED TO the Church Street space the following week. The building
has four floors, and Young and Zazeela live on the second. On the top
floor is storage space, where Young keeps one of his pianos, and between
the two is the ``Dream House,'' Young and Zazeela's long-running sound
and light installation. This one dates from 1993, though Young and
Zazeela have made some 40 iterations of the piece in locations across
the world. The ``Dream Houses'' are rooms that Zazeela has lit, usually
in magenta or purple, through the use of carefully positioned colored
lights, and inside of which Young's music (or, in this case, his and
Choi's music) plays, either live or through speakers, or both at the
same time. Visitors sit or stand or lie down in the carpeted space for
as long as they like, and the ``Dream House" in the Church Street
building is open to the public, as long as someone is there to buzz you
in. It's a simple concept, and yet this harmony between sound and light
was one of the earliest cohabitations of contemporary music and art.

The two conceptualized the ``Dream House'' around the same time they
first met in the early '60s. Zazeela was then a painter who used
calligraphy --- her work, Young offhandedly mentions, was ``better than
Picasso'' --- and was experimenting with colored light and shadow. (A
selection of Zazeela's works on paper is currently
\href{https://www.diaart.org/exhibition/exhibitions-projects/marian-zazeela-exhibition}{on
view} at Dia: Beacon in upstate New York.) The first fully realized one
was completed in 1969, inside Galerie Heiner Friedrich in Munich. The
idea was both spiritual --- an environment in which musicians could
perform for ``24 hours and going to eternity,'' as Choi put it --- but
also practical: If Young had a dedicated space to play that he enjoyed
being inside of, his music could evolve naturally, and he wouldn't have
to bother with the distraction of setting up equipment and breaking it
down each time he performed. This was crucial because Young has always
done everything slowly. (Choi told me that when she first met Young 20
years ago, it took him up to six hours just to wash his hair, even
though he was already going bald; to this, Young said, with such
sincerity that my heart broke a little bit, ``I just find the warm-water
experience to be very seductive,'' which is such a great way to describe
a shower.) From early on, Young's music was built around the ideas of
endurance and duration, more so than conventional musical ideas like
melody and lyric. The 1987 recording of his most famous piece, ``The
Well-Tuned Piano'' (reissued in 2018, though he first wrote it in 1964),
runs for nearly six-and-a-half hours.

Image

Young and Marian Zazeela performing at the artist Larry Poons's loft in
1966.Credit...Peter Moore © 2020 Barbara Moore/licensed by VAGA at
Artists Rights Society (ARS), N.Y., courtesy of Paula Cooper Gallery,
N.Y.

Image

Young's ``Piano Piece for Terry Riley \#1'' (1960).Credit...© The Museum
of Modern Art/Licensed by SCALA/Art Resource, N.Y.

In person, Young is an imposing but friendly presence. He, Choi and
Zazeela have worked together for the last two decades, an arrangement
that has its roots in the ancient Indian musical practice of a master
musician who trains a small group of disciples. For more than 25 years,
beginning in 1970, Young and Zazeela were followers of Pandit Pran Nath,
one of the greatest singers of \emph{Kirana gharana}, an Indian
classical vocal style sung over a drone. After Nath became their guru,
Young and Zazeela traveled to India several times, and soon brought him
back to America, helping arrange his green card and setting him up with
a teaching job, in addition to putting him up in their house. The idea
behind the guru discipleship is for the students to do whatever the
master says. ``You're supposed to make his life very fantastic, so he
can do his greatest creations,'' Young said. When Nath
\href{https://www.nytimes3xbfgragh.onion/1996/06/17/arts/pandit-pran-nath-77-singer-in-classical-hindustani-style.html}{died}
in 1996, Young and Zazeela, who often performed with her husband as a
singer, were heartbroken. They hadn't sung in three years when they met
Choi. Every time they tried, they just began crying. When they told Choi
this, she said, ``Your teacher would be very unhappy if you stopped
singing.'' And with that, she became their disciple.

The disciple's primary responsibility is to carry on their guru's
musical tradition. Or, as Young puts it, a disciple is supposed to be
like a pair of sandals for the guru to step into and walk off in. In
this way, Choi, though she **** still does her own compositions, has
been integrated into Young and Zazeela's work in the same way that Young
and Zazeela were once integrated into Nath's. But Choi's discipleship
resembles less indentured servitude and more the relationship between a
daughter and her parents. (Young and Zazeela have no children, though at
age 50, Choi could theoretically be their child.) The two serve as
backing musicians in Choi's own band, an arrangement that would have
never existed between Nath and Young. Choi told a story from the early
days of this relationship, when Choi often carried a first-generation
digital camera, which she used to document her work with Young and
Zazeela. It was heavy, housed in a large aluminum box, and Young always
insisted on carrying it for her.
\href{https://www.nytimes3xbfgragh.onion/2019/12/19/arts/music/terry-riley.html}{Terry
Riley}, another influential minimalist composer who had been a disciple
of Nath, once saw Young carrying his disciple's bag for her. ``Things
have really changed,'' he said.

Image

Pandit Pran Nath (singing, at center) with his students Young and
Zazeela playing tamburas,~accompanied by K. Paramjyoti on tabla (far
left) in 1982.Credit...Photo: John Cliett © Pandit Pran Nath, 1987.
Courtesy of the Pandit Pran Nath Musical Composition Trust.

THOUGH HE'S HAD a great influence on it, Young has never been a fan of
popular music.
\href{https://www.nytimes3xbfgragh.onion/2020/04/29/arts/music/brian-eno-ambient-songs.html}{Brian
Eno}, who did as much as any single figure to introduce electronic music
to Top 40 radio, has called Young ``the daddy of us all.'' It's true:
One could trace a lineage of adventurous composition directly back to
Young from multiple directions. In classical music, his experiments with
volume and repetition would birth an entirely new style of playing, the
influence of which could later be found in the works of
\href{https://www.nytimes3xbfgragh.onion/2016/10/02/arts/music/steve-reich-at-80-still-plugged-in-still-plugging-away.html}{Steve
Reich} and
\href{https://www.nytimes3xbfgragh.onion/topic/person/philip-glass}{Philip
Glass}. Young also taught John Cale how to play a drone on an amplified
viola, which would define the sound of the early Velvet Underground
records and would forever connect rock to the avant-garde, inspiring
innumerable musicians, from
\href{https://www.nytimes3xbfgragh.onion/topic/person/david-bowie}{David
Bowie} to
\href{https://www.nytimes3xbfgragh.onion/2019/09/14/style/kim-gordons-other-life.html}{Sonic
Youth} to
\href{https://www.nytimes3xbfgragh.onion/2019/08/28/arts/music/lana-del-rey-norman-rockwell-album.html}{Lana
Del Rey}. (Young also introduced Cale to marijuana.)

Choi attempted to further explain his lasting influence to me by
mentioning that last year, Katia and Marielle Labèque's ``Minimalist
Dream House'' tour (in part a tribute to Young) featured new
compositions by
\href{https://www.nytimes3xbfgragh.onion/interactive/2019/10/28/magazine/thom-yorke-radiohead-interview.html}{Thom
Yorke}, the singer for
\href{https://www.nytimes3xbfgragh.onion/topic/organization/radiohead}{Radiohead}.

``Say it again?'' Young asked.

``Radiohead, like the most popular group,'' Choi said.

``Radio what?'' Young asked, and Choi moved on.

It is difficult to square Young's influence with the humbleness of
Young's upbringing. He was born into a conservative Mormon family,
inside a log cabin in Bern, Idaho, a town that at the time had a
population of about 145. His father was a sheep herder who moved the
family to Los Angeles in 1940 to work as an experimental machinist for
the Douglas Aircraft Company when Young was about 5. But at heart,
Young's father was a cowboy, a country hill jack with a temper. A doctor
once told Young's father that his son was very smart, and upon hearing
this news, Young's father took him outside and beat him. ``That's what
my life was like,'' Young said, sadly. He describes his mother as
beautiful but long-suffering. In Idaho, when his father would go up into
the foothills to herd sheep, she would get on a horse and take food up
to him. ``The cowboy life was very hard on her,'' Young said.

Image

The log cabin in Bern, Idaho, where Young was born.Credit...© Marian
Zazeela, 1994

But despite all odds, the family was also musical. Young's father sung
cowboy songs to him from birth, and around age 2, Young began receiving
music lessons from his Aunt Norma, a singer who performed at rodeos. By
age 3, he and his older sister La Juana were tap dancing and singing
onstage in Montpelier, the town across the railroad tracks from Bern.
His father bought him his first saxophone around age 6, and his Uncle
Thornton, who had been in a swing band in the '30s, gave him Jimmy
Dorsey sheet music.

The idea of La Monte Young playing Jimmy Dorsey is likely inconceivable
to anyone who has heard Young's compositions, with their long,
improvisational drones and outré chord changes, but this grounding in
familiar conventions is what separates Young from other minimalists. His
music at times can sound like a pastiche of musical touchstones, with a
sudden blues riff or jazzy scale or folky chord change amid the chaos:
Just when you think you can't take it anymore, he offers up something
familiar and comforting.

It was in Los Angeles that Young went from a promising child prodigy to
a serious working musician, and the people who guided him in this
transition are central figures in his impressive mythology. By the time
he arrived at Los Angeles City College as a music student, he already
had a reasonably successful career. A slightly older student there,
\href{https://www.nytimes3xbfgragh.onion/2014/05/28/arts/music/a-new-focus-on-eric-dolphy-in-washington-and-montclair.html}{Eric
Dolphy}, who later became one of the most revered avant-garde saxophone
players of the 20th century as a sideman for
\href{https://www.nytimes3xbfgragh.onion/topic/person/john-coltrane}{John
Coltrane},
\href{https://www.nytimes3xbfgragh.onion/1979/01/09/archives/charles-mingus-56-bass-player-bandleader-and-composer-dead-an.html}{Charles
Mingus} and
\href{https://www.nytimes3xbfgragh.onion/2015/06/12/arts/music/ornette-coleman-jazz-saxophonist-dies-at-85-obituary.html}{Ornette
Coleman}, would set Young up with gigs when he was too busy to take
them. At City College, Young studied with Leonard Stein, who had been
the longtime assistant of Arnold Schoenberg, before transferring to the
University of California, Los Angeles. He'd go on to graduate school at
Berkeley, where he began moving away from jazz and developing his unique
style of piano playing, which he describes as more percussive than
melodic, and experimenting with sounds and the idea of continuous noise.

In 1960, Young moved to New York and placed himself at the center of the
avant-garde. Not long after his arrival, the composers
\href{https://www.nytimes3xbfgragh.onion/topic/person/john-cage}{John
Cage} and
\href{https://www.nytimes3xbfgragh.onion/1996/08/15/arts/david-tudor-70-electronic-composer-dies.html}{David
Tudor}, to whom Young had mailed a few of his scores, presented Young's
music at the Living Theater. He then became a teaching assistant for
Richard Maxfield, the electronic music composer, and was, briefly, the
saxophone player in Warhol's first rock band, the Druds, which included
\href{https://www.nytimes3xbfgragh.onion/2013/07/27/arts/design/walter-de-maria-artist-on-grand-scale-dies-at-77.html}{Walter
De Maria} on drums and lyrics by
\href{https://www.nytimes3xbfgragh.onion/2019/02/18/t-magazine/jasper-johns.html}{Jasper
Johns}. (Young quit after one rehearsal because nobody seemed to be
taking the music seriously.) Yoko Ono let him host concerts at her loft
on Chambers Street, where he refined his ideas about endurance and
performance by introducing a method of instructions for performers, some
literally achievable (``draw a straight line and follow it'') and some
less so (``bring a bale of hay and a bucket of water onto the stage for
the piano to eat and drink'').

Throughout all of this, his family remained unimpressed. Young now owns
the cabin he grew up in and has been wondering what to do with it. He
was going to visit the place --- Choi has never seen it --- but there
are too many unknown factors to consider. For one thing, he's
\href{https://charity.gofundme.com/o/en/campaign/save-the-dream-house-keep-our-dream-alive}{behind
on his rent} in New York (he never managed to buy the Church Street
building), and he owes his landlord more than \$100,000. He's worried
about being evicted. His health is also poor, as is Zazeela's, and even
before the pandemic, he wasn't sure they could manage plane travel. To
get to Bern, they'd have to fly to Logan-Cache Airport in Utah, and then
drive through Logan Canyon, through the Bear River Mountains and across
the state line into Idaho. Logan Canyon was the first place Young saw
colored lights, which he often asked his parents about, though they
would never answer him. Years later, he would return with Zazeela, and
they would discover the lights were from a resort where people could
drink alcohol, which was outlawed by his parents' religion. He's been
thinking of installing a sign in front of the cabin --- ``La Monte Young
Was Born Here'' --- but his sister has demurred: She was born there,
too, she says. Why shouldn't \emph{she} get a sign as well?

Image

Young playing sopranino saxophone in May 1963 at a happening on the
artist George Segal's farm in South Brunswick, N.J.Credit...Peter Moore
© 2020 Barbara Moore/Licensed by VAGA at Artists Rights Society (ARS),
N.Y., courtesy Paula Cooper Gallery, N.Y.

I'VE PUT OFF writing about what it was like to watch Young perform,
because I fear words will be insufficient. I can describe what happened
literally: The assembled players performed one of Choi's compositions.
Young and Choi sang in the style that Young had learned from his guru
(Zazeela, who has lost most of her hearing, simply listened), chanting
into their microphones at different pitches for over an hour without
stopping, their voices harmonizing with one another and the drone, a
1982 recording of Young and Zazeela on tambouras, over which they were
singing. Despite Young's age, his voice was powerful and unwaveringly on
key. This was expected, given that he's spent the last 50 years
perfecting this style of singing, but still astonishing to witness. The
tabla player kept a steady rhythm, while the guitarist and bassist
played a repetitive figure that in some ways resembled blues. There were
notes that sounded like the typical flat thirds and minor pentatonic
scales that define the form, but it had been reordered and twisted into
nonrecognition --- the music was based on a scale that Choi invented
herself. This instrumentation seemed to grow in intensity as the
performance went on, as if the music was working itself up into some
kind of frenzy, but it never became the center of focus: That remained
Young's voice, strong and balanced, like something that was simply
affixed to the wall and had always been there and --- while the
performance was happening, at least --- seemed like it always would be.

It's harder for me to say how I felt having witnessed this. My first
thought was how was it possible that I could watch such a performance,
amid the hyper-gentrified luxury condos of TriBeCa of all places, a
feeling of disbelief that intensified once I re-emerged onto Church
Street and my stunned silence was disrupted by a crowd of about 40
yuppies led by a man with a bullhorn who was shouting about how they
were a ``meet-up dance party.'' Much later, my thoughts drifted back to
what Young had told me about the guru and the disciple --- that music is
greater than its performers, that it can be taught and passed on. It's
always difficult to imagine a future, and some days it is near
impossible, but I tried to envision myself as an old man about the age
Young is now, and how if I make it that long, I'll do my small part in
passing on the musical tradition, by telling people about the one rainy
Sunday afternoon a long time ago when I went to La Monte Young's house
and heard him sing.

\hypertarget{true-believers-art-issue}{%
\subsubsection{\texorpdfstring{\href{https://www.nytimes3xbfgragh.onion/issue/t-magazine/2020/07/02/true-believers-art-issue}{True
Believers Art
Issue}}{True Believers Art Issue}}\label{true-believers-art-issue}}

Advertisement

\protect\hyperlink{after-bottom}{Continue reading the main story}

\hypertarget{site-index}{%
\subsection{Site Index}\label{site-index}}

\hypertarget{site-information-navigation}{%
\subsection{Site Information
Navigation}\label{site-information-navigation}}

\begin{itemize}
\tightlist
\item
  \href{https://help.nytimes3xbfgragh.onion/hc/en-us/articles/115014792127-Copyright-notice}{©~2020~The
  New York Times Company}
\end{itemize}

\begin{itemize}
\tightlist
\item
  \href{https://www.nytco.com/}{NYTCo}
\item
  \href{https://help.nytimes3xbfgragh.onion/hc/en-us/articles/115015385887-Contact-Us}{Contact
  Us}
\item
  \href{https://www.nytco.com/careers/}{Work with us}
\item
  \href{https://nytmediakit.com/}{Advertise}
\item
  \href{http://www.tbrandstudio.com/}{T Brand Studio}
\item
  \href{https://www.nytimes3xbfgragh.onion/privacy/cookie-policy\#how-do-i-manage-trackers}{Your
  Ad Choices}
\item
  \href{https://www.nytimes3xbfgragh.onion/privacy}{Privacy}
\item
  \href{https://help.nytimes3xbfgragh.onion/hc/en-us/articles/115014893428-Terms-of-service}{Terms
  of Service}
\item
  \href{https://help.nytimes3xbfgragh.onion/hc/en-us/articles/115014893968-Terms-of-sale}{Terms
  of Sale}
\item
  \href{https://spiderbites.nytimes3xbfgragh.onion}{Site Map}
\item
  \href{https://help.nytimes3xbfgragh.onion/hc/en-us}{Help}
\item
  \href{https://www.nytimes3xbfgragh.onion/subscription?campaignId=37WXW}{Subscriptions}
\end{itemize}
