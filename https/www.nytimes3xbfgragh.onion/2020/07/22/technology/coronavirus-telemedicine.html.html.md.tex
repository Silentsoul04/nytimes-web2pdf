Sections

SEARCH

\protect\hyperlink{site-content}{Skip to
content}\protect\hyperlink{site-index}{Skip to site index}

\href{https://www.nytimes3xbfgragh.onion/section/technology}{Technology}

\href{https://myaccount.nytimes3xbfgragh.onion/auth/login?response_type=cookie\&client_id=vi}{}

\href{https://www.nytimes3xbfgragh.onion/section/todayspaper}{Today's
Paper}

\href{/section/technology}{Technology}\textbar{}Health Care Comes to Us

\href{https://nyti.ms/2BqiYwC}{https://nyti.ms/2BqiYwC}

\begin{itemize}
\item
\item
\item
\item
\item
\end{itemize}

\href{https://www.nytimes3xbfgragh.onion/news-event/coronavirus?action=click\&pgtype=Article\&state=default\&region=TOP_BANNER\&context=storylines_menu}{The
Coronavirus Outbreak}

\begin{itemize}
\tightlist
\item
  live\href{https://www.nytimes3xbfgragh.onion/2020/08/04/world/coronavirus-covid-19.html?action=click\&pgtype=Article\&state=default\&region=TOP_BANNER\&context=storylines_menu}{Latest
  Updates}
\item
  \href{https://www.nytimes3xbfgragh.onion/interactive/2020/us/coronavirus-us-cases.html?action=click\&pgtype=Article\&state=default\&region=TOP_BANNER\&context=storylines_menu}{Maps
  and Cases}
\item
  \href{https://www.nytimes3xbfgragh.onion/interactive/2020/science/coronavirus-vaccine-tracker.html?action=click\&pgtype=Article\&state=default\&region=TOP_BANNER\&context=storylines_menu}{Vaccine
  Tracker}
\item
  \href{https://www.nytimes3xbfgragh.onion/2020/08/02/us/covid-college-reopening.html?action=click\&pgtype=Article\&state=default\&region=TOP_BANNER\&context=storylines_menu}{College
  Reopening}
\item
  \href{https://www.nytimes3xbfgragh.onion/live/2020/08/03/business/stock-market-today-coronavirus?action=click\&pgtype=Article\&state=default\&region=TOP_BANNER\&context=storylines_menu}{Economy}
\end{itemize}

Advertisement

\protect\hyperlink{after-top}{Continue reading the main story}

Supported by

\protect\hyperlink{after-sponsor}{Continue reading the main story}

on tech

\hypertarget{health-care-comes-to-us}{%
\section{Health Care Comes to Us}\label{health-care-comes-to-us}}

Technology doesn't have to cure the coronavirus to be an enabler for
good.

\includegraphics{https://static01.graylady3jvrrxbe.onion/images/2020/07/22/business/22ontech/22ontech-articleLarge.jpg?quality=75\&auto=webp\&disable=upscale}

\href{https://www.nytimes3xbfgragh.onion/by/shira-ovide}{\includegraphics{https://static01.graylady3jvrrxbe.onion/images/2020/03/18/reader-center/author-shira-ovide/author-shira-ovide-thumbLarge-v2.png}}

By \href{https://www.nytimes3xbfgragh.onion/by/shira-ovide}{Shira Ovide}

\begin{itemize}
\item
  July 22, 2020
\item
  \begin{itemize}
  \item
  \item
  \item
  \item
  \item
  \end{itemize}
\end{itemize}

\emph{This article is part of the On Tech newsletter. You can}
\href{https://www.nytimes3xbfgragh.onion/newsletters/signup/OT}{\emph{sign
up here}} \emph{to receive it weekdays.}

The pandemic, an unemployment surge and unrest over racial inequality
have made more Americans
\href{https://www.nytimes3xbfgragh.onion/2020/04/09/opinion/covid-anxiety.html}{feel}
isolated,
\href{https://www.kff.org/statedata/mental-health-and-substance-use-state-fact-sheets/}{anxious
or depressed}. Psychological distress
\href{https://www.nytimes3xbfgragh.onion/2020/06/21/health/coronavirus-mental-health-anxiety.html}{could
prove temporary,} but the hurt and the
\href{https://www.nytimes3xbfgragh.onion/interactive/2020/07/15/upshot/drug-overdose-deaths.html}{ripple
effects are serious} nevertheless.

Now here's some good news. Benjamin F. Miller, a psychologist and chief
strategy officer for Well Being Trust, a national foundation focusing on
mental and spiritual health, told me something hopeful: In part because
of technology, this moment in history contains the makings of more
accessible and effective mental health care for everyone.

``Probably one of the most profound impacts that technology had in the
pandemic is that the care now comes to the patients,'' Dr. Miller said.

He's talking about the many physicians, therapists and clinicians
shifting to seeing patients by web video or over the telephone. Not
everyone loves health care through a computer screen, but Dr. Miller
said it has removed barriers that prevented many people from accessing
mental health services.

Care can now be just a FaceTime call away, and U.S. insurers quickly
made changes that allowed more people to get help on their terms.

I've been thinking about how peripheral technology has felt these last
few months. Sure, we've relied on technology for work, school and
staying in touch, but brave essential workers, capable political and
public health leaders and effective institutions matter more than
anything else.

Dr. Miller reminded me that technology doesn't have to cure the
coronavirus to be an enabler for good. He said he believed that
technology has an important role to play in what he hoped would become a
larger restructuring of American health care.

\hypertarget{latest-updates-economy}{%
\section{\texorpdfstring{\href{https://www.nytimes3xbfgragh.onion/live/2020/08/03/business/stock-market-today-coronavirus?action=click\&pgtype=Article\&state=default\&region=MAIN_CONTENT_1\&context=storylines_live_updates}{Latest
Updates:
Economy}}{Latest Updates: Economy}}\label{latest-updates-economy}}

\href{https://www.nytimes3xbfgragh.onion/live/2020/08/03/business/stock-market-today-coronavirus?action=click\&pgtype=Article\&state=default\&region=MAIN_CONTENT_1\&context=storylines_live_updates\#the-chicago-fed-president-says-its-up-to-congress-to-save-the-economy}{13h
ago}

\href{https://www.nytimes3xbfgragh.onion/live/2020/08/03/business/stock-market-today-coronavirus?action=click\&pgtype=Article\&state=default\&region=MAIN_CONTENT_1\&context=storylines_live_updates\#the-chicago-fed-president-says-its-up-to-congress-to-save-the-economy}{The
Chicago Fed president says it's up to Congress to save the economy.}

\href{https://www.nytimes3xbfgragh.onion/live/2020/08/03/business/stock-market-today-coronavirus?action=click\&pgtype=Article\&state=default\&region=MAIN_CONTENT_1\&context=storylines_live_updates\#faa-says-boeing-has-effectively-mitigated-defects-in-the-737-max}{14h
ago}

\href{https://www.nytimes3xbfgragh.onion/live/2020/08/03/business/stock-market-today-coronavirus?action=click\&pgtype=Article\&state=default\&region=MAIN_CONTENT_1\&context=storylines_live_updates\#faa-says-boeing-has-effectively-mitigated-defects-in-the-737-max}{F.A.A.
says Boeing has `effectively mitigated' defects in the 737 Max.}

\href{https://www.nytimes3xbfgragh.onion/live/2020/08/03/business/stock-market-today-coronavirus?action=click\&pgtype=Article\&state=default\&region=MAIN_CONTENT_1\&context=storylines_live_updates\#small-businesses-got-emergency-loans-but-not-what-they-expected}{16h
ago}

\href{https://www.nytimes3xbfgragh.onion/live/2020/08/03/business/stock-market-today-coronavirus?action=click\&pgtype=Article\&state=default\&region=MAIN_CONTENT_1\&context=storylines_live_updates\#small-businesses-got-emergency-loans-but-not-what-they-expected}{Small
businesses got emergency loans, but not what they expected.}

\href{https://www.nytimes3xbfgragh.onion/live/2020/08/03/business/stock-market-today-coronavirus?action=click\&pgtype=Article\&state=default\&region=MAIN_CONTENT_1\&context=storylines_live_updates}{See
more updates}

More live coverage:
\href{https://www.nytimes3xbfgragh.onion/2020/08/04/world/coronavirus-covid-19.html?action=click\&pgtype=Article\&state=default\&region=MAIN_CONTENT_1\&context=storylines_live_updates}{Global}

But first, some capable people and institutions had to cut red tape to
let technology in.

Since the start of the pandemic, Medicare and many private health
insurers have changed policies to reimburse practitioners for patient
visits by phone or web video at somewhere close to the payment rate of
in-person visits.

Privacy rules were relaxed to let people use familiar web video services
like Skype and not only medical-specific video sites. (Yes, this comes
with a possible risk to patient information.)

Telemedicine for all types of health care remains a tiny fraction of
patient care, but many more people and providers
\href{https://www.nytimes3xbfgragh.onion/2020/07/09/well/mind/teletherapy-mental-health-coronavirus.html}{have
tried and liked it}. Nearly every major mental health organization is
pressing policymakers to make those temporary changes permanent, Dr.
Miller said.

Technology is not a panacea, Dr. Miller stressed. (Reader: May you
remember this sentence always, about everything in tech.) Lack of
internet access or discomfort with technology still holds some people
back from telemedicine, Dr. Miller said. And tech doesn't resolve the
stigma that can be associated with mental health services or close gaps
in health insurance coverage.

But Dr. Miller said technology's role in mental health during the
pandemic is a gift that he hoped would be the start of work to better
structure mental health services, integrate them into the rest of health
care and ensure they get enough resources to help everyone.

Dr. Miller's essential message wasn't about technology at all. Because
so many of us have felt stress and isolation recently, he hoped that we
can now talk openly about the importance of healthy minds and bodies,
and better understand people who live with mental distress.

``Now that we know how hard this is, I hope we have empathy,'' he said.

\emph{If you don't already get this newsletter in your inbox,}
\href{https://www.nytimes3xbfgragh.onion/newsletters/signup/OT}{\emph{please
sign up here}}\emph{.}

\begin{center}\rule{0.5\linewidth}{\linethickness}\end{center}

\hypertarget{tv-commercials-and-the-maturation-of-tech}{%
\subsection{TV commercials and the maturation of
tech}\label{tv-commercials-and-the-maturation-of-tech}}

If you want one statistic that shows technology companies' maturation
\href{https://www.nytimes3xbfgragh.onion/2020/06/22/technology/big-tech-underdogs.html}{from
iconoclastic underdogs to the mainstream}, look at advertising.

Amazon now spends more money on promoting itself in television
commercials, internet ads and other spots than any company in the United
States, according to an
\href{https://s3-prod.adage.com/s3fs-public/2020-07/lnafp_aa_20200713_locked.pdf}{analysis}
of 2019 advertising trends by the publication AdAge. Google was No. 6.
(I first read about this in the
\href{https://www.axios.com/newsletters/axios-media-trends-d14d7aa7-b160-4449-b01d-447b83fa18e5.html}{Axios
Media Trends newsletter}.)

Companies that make physical devices, like Apple, used advertising for
years to shape our perceptions. But until quite recently, Amazon and
many of America's upstart internet companies thought advertising was
kinda tacky.

``Advertising is the price you pay for having an unremarkable product or
service,'' the Amazon chief executive Jeff Bezos
\href{https://blog.seattlepi.com/amazon/2009/05/28/amazons-jeff-bezos-on-kindle-advertising-and-being-green/}{quipped}
more than 10 years ago. Last year, Bezos said that
\href{https://www.cnbc.com/2019/02/02/jeff-bezos-says-hes-had-a-change-of-heart-on-advertising--now-amazon-is-the-fifth-biggest-ad-spender-in-the-us-.html}{he
had changed his mind}.

Why the change? Well, technology is becoming just like every other
product. There's not much difference between a Ford and Toyota pickup
truck, so those companies know they must persuade you to feel warm and
fuzzy about their model. Picking an app or an online shopping company
likewise has become a lot about picking one that makes you feel good.

And as tech companies wanted us to turn over more of our habits and
lives to them, they needed to pitch themselves harder. Amazon, for
example, spends a lot of money advertising its movies, internet TV
gadgets and voice assistants to turn our homes into all-Amazon zones.

There's a similar pattern to tech companies' spending on policy
persuasion. They used to consider lobbying unseemly or unimportant, and
now America's tech powers are
\href{https://www.washingtonpost.com/technology/2020/01/22/amazon-facebook-google-lobbying-2019/}{among
the country's biggest lobbyists}.

There you go. Tech is not a special species anymore. It is big and
everywhere, and that means the industry's leading lights spend a lot of
money to stay on top.

\emph{Our newsletter cousins at DealBook are hosting a reader conference
call featuring David E. Sanger, The New York Times's national security
correspondent, discussing the tug of war over technology between the
United States and China. To hear from David and ask him your questions,
you can}
\href{https://timesevents.nytimes3xbfgragh.onion/dealbook0723}{\emph{R.S.V.P.
here}}\emph{. The call is tomorrow (July 23) at 11 a.m. Eastern.}

\begin{center}\rule{0.5\linewidth}{\linethickness}\end{center}

\hypertarget{before-we-go-}{%
\subsection{Before we go \ldots{}}\label{before-we-go-}}

\begin{itemize}
\item
  \textbf{Tackling a dangerous conspiracy:} Twitter announced a series
  of sweeping actions intended to remove or hide more accounts and
  material related to QAnon, a movement promoting baseless conspiracies
  that has proliferated on Facebook, YouTube and Twitter, my colleague
  Kate Conger
  \href{https://www.nytimes3xbfgragh.onion/2020/07/21/technology/twitter-bans-qanon-accounts.html}{reported}.
  People who believe in QAnon's intricate and false theories have
  \href{https://www.nytimes3xbfgragh.onion/2020/02/09/us/politics/qanon-trump-conspiracy-theory.html}{committed
  violence} and harassed people online, and internet companies have been
  under pressure to do more to combat the spread of this and other
  harmful material. Facebook is also preparing to take similar steps to
  limit the reach of QAnon content, Kate wrote.
\item
  \textbf{The criticism is coming from inside the house:} The Times ****
  tech reporter Karen Weise writes about Tim Bray, a respected
  technologist and Amazon executive who recently quit the company and
  \href{https://www.nytimes3xbfgragh.onion/2020/07/22/technology/amazon-critic-tim-bray.html}{became
  one of its highest-profile critics}. Bray is using the mind-set and
  tools of Amazon --- including the intense, six-page internal memos
  called PRFAQs --- to articulate how and why he believes Amazon hurts
  competition and should be broken apart.
\item
  \textbf{Big tech versus the big scourge of climate change:} Somini
  Sengupta and Veronica Penney of The New York Times
  \href{https://www.nytimes3xbfgragh.onion/2020/07/21/climate/apple-emissions-pledge.html}{walk
  through} what Apple, Microsoft and other large tech companies are
  doing to combat planet-warming carbon emissions, and where their
  rhetoric might fall short of their actions. (I'll have more in
  tomorrow's newsletter about technology and climate change.)
\end{itemize}

\hypertarget{hugs-to-this}{%
\subsubsection{Hugs to this}\label{hugs-to-this}}

These six
\href{https://twitter.com/BirdCentralPark/status/1282425858368638977?s=09}{ducklings
bobbing in the water} are giving me joy\emph{We want to hear from you.
Tell us what you think of this newsletter and what else you'd like us to
explore. You can reach us at}
\href{mailto:ontech@NYTimes.com?subject=On\%20Tech\%20Feedback}{\emph{ontech@NYTimes.com.}}
**

\emph{If you don't already get this newsletter in your inbox,}
\href{https://www.nytimes3xbfgragh.onion/newsletters/signup/OT}{\emph{please
sign up here}}\emph{.}

Advertisement

\protect\hyperlink{after-bottom}{Continue reading the main story}

\hypertarget{site-index}{%
\subsection{Site Index}\label{site-index}}

\hypertarget{site-information-navigation}{%
\subsection{Site Information
Navigation}\label{site-information-navigation}}

\begin{itemize}
\tightlist
\item
  \href{https://help.nytimes3xbfgragh.onion/hc/en-us/articles/115014792127-Copyright-notice}{©~2020~The
  New York Times Company}
\end{itemize}

\begin{itemize}
\tightlist
\item
  \href{https://www.nytco.com/}{NYTCo}
\item
  \href{https://help.nytimes3xbfgragh.onion/hc/en-us/articles/115015385887-Contact-Us}{Contact
  Us}
\item
  \href{https://www.nytco.com/careers/}{Work with us}
\item
  \href{https://nytmediakit.com/}{Advertise}
\item
  \href{http://www.tbrandstudio.com/}{T Brand Studio}
\item
  \href{https://www.nytimes3xbfgragh.onion/privacy/cookie-policy\#how-do-i-manage-trackers}{Your
  Ad Choices}
\item
  \href{https://www.nytimes3xbfgragh.onion/privacy}{Privacy}
\item
  \href{https://help.nytimes3xbfgragh.onion/hc/en-us/articles/115014893428-Terms-of-service}{Terms
  of Service}
\item
  \href{https://help.nytimes3xbfgragh.onion/hc/en-us/articles/115014893968-Terms-of-sale}{Terms
  of Sale}
\item
  \href{https://spiderbites.nytimes3xbfgragh.onion}{Site Map}
\item
  \href{https://help.nytimes3xbfgragh.onion/hc/en-us}{Help}
\item
  \href{https://www.nytimes3xbfgragh.onion/subscription?campaignId=37WXW}{Subscriptions}
\end{itemize}
