Sections

SEARCH

\protect\hyperlink{site-content}{Skip to
content}\protect\hyperlink{site-index}{Skip to site index}

\href{https://myaccount.nytimes3xbfgragh.onion/auth/login?response_type=cookie\&client_id=vi}{}

\href{https://www.nytimes3xbfgragh.onion/section/todayspaper}{Today's
Paper}

\href{/section/upshot}{The Upshot}\textbar{}The U.S. Commits to Buying
Millions of Vaccine Doses. Why That's Unusual.

\url{https://nyti.ms/2ON7hTI}

\begin{itemize}
\item
\item
\item
\item
\item
\item
\end{itemize}

\href{https://www.nytimes3xbfgragh.onion/news-event/coronavirus?action=click\&pgtype=Article\&state=default\&region=TOP_BANNER\&context=storylines_menu}{The
Coronavirus Outbreak}

\begin{itemize}
\tightlist
\item
  live\href{https://www.nytimes3xbfgragh.onion/2020/08/01/world/coronavirus-covid-19.html?action=click\&pgtype=Article\&state=default\&region=TOP_BANNER\&context=storylines_menu}{Latest
  Updates}
\item
  \href{https://www.nytimes3xbfgragh.onion/interactive/2020/us/coronavirus-us-cases.html?action=click\&pgtype=Article\&state=default\&region=TOP_BANNER\&context=storylines_menu}{Maps
  and Cases}
\item
  \href{https://www.nytimes3xbfgragh.onion/interactive/2020/science/coronavirus-vaccine-tracker.html?action=click\&pgtype=Article\&state=default\&region=TOP_BANNER\&context=storylines_menu}{Vaccine
  Tracker}
\item
  \href{https://www.nytimes3xbfgragh.onion/interactive/2020/07/29/us/schools-reopening-coronavirus.html?action=click\&pgtype=Article\&state=default\&region=TOP_BANNER\&context=storylines_menu}{What
  School May Look Like}
\item
  \href{https://www.nytimes3xbfgragh.onion/live/2020/07/31/business/stock-market-today-coronavirus?action=click\&pgtype=Article\&state=default\&region=TOP_BANNER\&context=storylines_menu}{Economy}
\end{itemize}

Advertisement

\protect\hyperlink{after-top}{Continue reading the main story}

Upshot

Supported by

\protect\hyperlink{after-sponsor}{Continue reading the main story}

\hypertarget{the-us-commits-to-buying-millions-of-vaccine-doses-why-thats-unusual}{%
\section{The U.S. Commits to Buying Millions of Vaccine Doses. Why
That's
Unusual.}\label{the-us-commits-to-buying-millions-of-vaccine-doses-why-thats-unusual}}

Besides the fact that there's no coronavirus vaccine yet, the government
usually lets the private sector handle it.

By \href{https://www.nytimes3xbfgragh.onion/by/sarah-kliff}{Sarah Kliff}

\begin{itemize}
\item
  July 22, 2020
\item
  \begin{itemize}
  \item
  \item
  \item
  \item
  \item
  \item
  \end{itemize}
\end{itemize}

Image

A Pfizer laboratory in Pearl River, N.Y. Pfizer, along with a smaller
drugmaker, has entered into a contract with the U.S. government to
provide millions of doses of a coronavirus vaccine still under
development.Credit...Pfizer, via Reuters

The Trump administration's
\href{https://www.nytimes3xbfgragh.onion/2020/07/22/us/politics/pfizer-coronavirus-vaccine.html}{commitment}
on Wednesday to purchase 100 million doses of a not-yet-finished vaccine
is unusual in two ways. The private sector buys most vaccines in the
United States, not the government.

The drug industry has lobbied to keep things that way; private payers
usually pay more for vaccines than the government does.

And when the government does buy vaccines --- typically on behalf of
low-income children --- it is almost always vaccines that have already
received safety and efficacy approval from the Food and Drug
Administration.

``This is not at all typical, but I think it's a great thing,'' said
Claire Hannan, executive director of the Association of Immunization
Managers. ``We're in a public health crisis, and it's great the
government is going to be providing a vaccine for Americans.''

Widespread government purchasing of coronavirus vaccines may ultimately
lower their price in the United States. That's what happens in other
countries, like Canada and many in Europe, where large, national health
systems routinely buy vaccines, prescription drugs and many other
medical services on behalf of citizens.

Experts in vaccine financing policy said the contract, with the
pharmaceutical giant Pfizer and a smaller German biotechnology company,
was risky because the drug could fail future trials. The Trump
administration included a safeguard, and will not send the money until
the vaccine proves to be safe and effective.

``It's a gamble, but a reasonable one,'' said Dr. Walter Orenstein, an
epidemiologist at Emory University who worked for the Clinton
administration as director of the United States immunization program.
``What the government is doing here, which I think is a good thing,
they're helping gear up production so that if this vaccine is licensed,
it will be available in large quantities.''

\hypertarget{latest-updates-global-coronavirus-outbreak}{%
\section{\texorpdfstring{\href{https://www.nytimes3xbfgragh.onion/2020/08/01/world/coronavirus-covid-19.html?action=click\&pgtype=Article\&state=default\&region=MAIN_CONTENT_1\&context=storylines_live_updates}{Latest
Updates: Global Coronavirus
Outbreak}}{Latest Updates: Global Coronavirus Outbreak}}\label{latest-updates-global-coronavirus-outbreak}}

Updated 2020-08-01T19:54:00.494Z

\begin{itemize}
\tightlist
\item
  \href{https://www.nytimes3xbfgragh.onion/2020/08/01/world/coronavirus-covid-19.html?action=click\&pgtype=Article\&state=default\&region=MAIN_CONTENT_1\&context=storylines_live_updates\#link-3ac56579}{Top
  officials work to break impasse over jobless benefit.}
\item
  \href{https://www.nytimes3xbfgragh.onion/2020/08/01/world/coronavirus-covid-19.html?action=click\&pgtype=Article\&state=default\&region=MAIN_CONTENT_1\&context=storylines_live_updates\#link-8796723}{The
  virus picks up dangerous speed in the Midwest, and in areas that had
  seen success.}
\item
  \href{https://www.nytimes3xbfgragh.onion/2020/08/01/world/coronavirus-covid-19.html?action=click\&pgtype=Article\&state=default\&region=MAIN_CONTENT_1\&context=storylines_live_updates\#link-25930521}{Thousands
  in Berlin protest Germany's coronavirus measures.}
\end{itemize}

\href{https://www.nytimes3xbfgragh.onion/2020/08/01/world/coronavirus-covid-19.html?action=click\&pgtype=Article\&state=default\&region=MAIN_CONTENT_1\&context=storylines_live_updates}{See
more updates}

More live coverage:
\href{https://www.nytimes3xbfgragh.onion/live/2020/07/31/business/stock-market-today-coronavirus?action=click\&pgtype=Article\&state=default\&region=MAIN_CONTENT_1\&context=storylines_live_updates}{Markets}

The Obama administration made a similar bet during the H1N1 (swine flu)
pandemic, when it awarded the French drugmaker Sanofi
\href{http://www.news.sanofi.us/press-releases?item=137044}{\$190
million} in May 2009 to develop a vaccine against the nascent disease.
The drug successfully finished the clinical trial process six months
later.

In the United States, where medical prices are set in negotiations
between providers and insurers, charges are typically higher and vary
significantly from one place to another. A flu shot, for example, can
cost between \$28 and \$80 depending on who delivers it.

The high prices that the American health care system allows
---~sometimes
\href{https://www.nytimes3xbfgragh.onion/2020/06/29/upshot/coronavirus-tests-unpredictable-prices.html}{thousands
of dollars for a simple test} ---~have long been a focus of ``Medicare
for all'' advocates, who argue that the country would be better served
by a national health system that regulates health prices.

In moments of crisis, with H1N1 and now with the coronavirus, the
government sometimes takes small steps in that direction, such as by
buying medical goods that its citizens need but may not be able to
afford or negotiate for on their own.

``We're trying to make sure access for all Americans is equal,'' said
L.J. Tan, chief strategy officer at the Immunization Action Coalition.
``The best way to do that is to make vaccines free for everybody.''

The U.S. government has also recently declared that insured Americans
wouldn't face fees for coronavirus testing, and it dedicated a new fund
to cover the testing and treatment costs of the uninsured.

In these ways, the United States is acting more like Canada or European
countries. Medical prices are typically lower when the government buys
in bulk, for example. ``The advantage of the government buying is the
potential for a better price,'' Dr. Orenstein said. ``It can remove
financial barriers to access.''

Government purchasing of vaccines has a contentious history in the
United States. In the early 1990s, the Clinton administration tried to
create a public program that would buy vaccines for all American
children under 2.

\href{https://www.nytimes3xbfgragh.onion/news-event/coronavirus?action=click\&pgtype=Article\&state=default\&region=MAIN_CONTENT_3\&context=storylines_faq}{}

\hypertarget{the-coronavirus-outbreak-}{%
\subsubsection{The Coronavirus Outbreak
›}\label{the-coronavirus-outbreak-}}

\hypertarget{frequently-asked-questions}{%
\paragraph{Frequently Asked
Questions}\label{frequently-asked-questions}}

Updated July 27, 2020

\begin{itemize}
\item ~
  \hypertarget{should-i-refinance-my-mortgage}{%
  \paragraph{Should I refinance my
  mortgage?}\label{should-i-refinance-my-mortgage}}

  \begin{itemize}
  \tightlist
  \item
    \href{https://www.nytimes3xbfgragh.onion/article/coronavirus-money-unemployment.html?action=click\&pgtype=Article\&state=default\&region=MAIN_CONTENT_3\&context=storylines_faq}{It
    could be a good idea,} because mortgage rates have
    \href{https://www.nytimes3xbfgragh.onion/2020/07/16/business/mortgage-rates-below-3-percent.html?action=click\&pgtype=Article\&state=default\&region=MAIN_CONTENT_3\&context=storylines_faq}{never
    been lower.} Refinancing requests have pushed mortgage applications
    to some of the highest levels since 2008, so be prepared to get in
    line. But defaults are also up, so if you're thinking about buying a
    home, be aware that some lenders have tightened their standards.
  \end{itemize}
\item ~
  \hypertarget{what-is-school-going-to-look-like-in-september}{%
  \paragraph{What is school going to look like in
  September?}\label{what-is-school-going-to-look-like-in-september}}

  \begin{itemize}
  \tightlist
  \item
    It is unlikely that many schools will return to a normal schedule
    this fall, requiring the grind of
    \href{https://www.nytimes3xbfgragh.onion/2020/06/05/us/coronavirus-education-lost-learning.html?action=click\&pgtype=Article\&state=default\&region=MAIN_CONTENT_3\&context=storylines_faq}{online
    learning},
    \href{https://www.nytimes3xbfgragh.onion/2020/05/29/us/coronavirus-child-care-centers.html?action=click\&pgtype=Article\&state=default\&region=MAIN_CONTENT_3\&context=storylines_faq}{makeshift
    child care} and
    \href{https://www.nytimes3xbfgragh.onion/2020/06/03/business/economy/coronavirus-working-women.html?action=click\&pgtype=Article\&state=default\&region=MAIN_CONTENT_3\&context=storylines_faq}{stunted
    workdays} to continue. California's two largest public school
    districts --- Los Angeles and San Diego --- said on July 13, that
    \href{https://www.nytimes3xbfgragh.onion/2020/07/13/us/lausd-san-diego-school-reopening.html?action=click\&pgtype=Article\&state=default\&region=MAIN_CONTENT_3\&context=storylines_faq}{instruction
    will be remote-only in the fall}, citing concerns that surging
    coronavirus infections in their areas pose too dire a risk for
    students and teachers. Together, the two districts enroll some
    825,000 students. They are the largest in the country so far to
    abandon plans for even a partial physical return to classrooms when
    they reopen in August. For other districts, the solution won't be an
    all-or-nothing approach.
    \href{https://bioethics.jhu.edu/research-and-outreach/projects/eschool-initiative/school-policy-tracker/}{Many
    systems}, including the nation's largest, New York City, are
    devising
    \href{https://www.nytimes3xbfgragh.onion/2020/06/26/us/coronavirus-schools-reopen-fall.html?action=click\&pgtype=Article\&state=default\&region=MAIN_CONTENT_3\&context=storylines_faq}{hybrid
    plans} that involve spending some days in classrooms and other days
    online. There's no national policy on this yet, so check with your
    municipal school system regularly to see what is happening in your
    community.
  \end{itemize}
\item ~
  \hypertarget{is-the-coronavirus-airborne}{%
  \paragraph{Is the coronavirus
  airborne?}\label{is-the-coronavirus-airborne}}

  \begin{itemize}
  \tightlist
  \item
    The coronavirus
    \href{https://www.nytimes3xbfgragh.onion/2020/07/04/health/239-experts-with-one-big-claim-the-coronavirus-is-airborne.html?action=click\&pgtype=Article\&state=default\&region=MAIN_CONTENT_3\&context=storylines_faq}{can
    stay aloft for hours in tiny droplets in stagnant air}, infecting
    people as they inhale, mounting scientific evidence suggests. This
    risk is highest in crowded indoor spaces with poor ventilation, and
    may help explain super-spreading events reported in meatpacking
    plants, churches and restaurants.
    \href{https://www.nytimes3xbfgragh.onion/2020/07/06/health/coronavirus-airborne-aerosols.html?action=click\&pgtype=Article\&state=default\&region=MAIN_CONTENT_3\&context=storylines_faq}{It's
    unclear how often the virus is spread} via these tiny droplets, or
    aerosols, compared with larger droplets that are expelled when a
    sick person coughs or sneezes, or transmitted through contact with
    contaminated surfaces, said Linsey Marr, an aerosol expert at
    Virginia Tech. Aerosols are released even when a person without
    symptoms exhales, talks or sings, according to Dr. Marr and more
    than 200 other experts, who
    \href{https://academic.oup.com/cid/article/doi/10.1093/cid/ciaa939/5867798}{have
    outlined the evidence in an open letter to the World Health
    Organization}.
  \end{itemize}
\item ~
  \hypertarget{what-are-the-symptoms-of-coronavirus}{%
  \paragraph{What are the symptoms of
  coronavirus?}\label{what-are-the-symptoms-of-coronavirus}}

  \begin{itemize}
  \tightlist
  \item
    Common symptoms
    \href{https://www.nytimes3xbfgragh.onion/article/symptoms-coronavirus.html?action=click\&pgtype=Article\&state=default\&region=MAIN_CONTENT_3\&context=storylines_faq}{include
    fever, a dry cough, fatigue and difficulty breathing or shortness of
    breath.} Some of these symptoms overlap with those of the flu,
    making detection difficult, but runny noses and stuffy sinuses are
    less common.
    \href{https://www.nytimes3xbfgragh.onion/2020/04/27/health/coronavirus-symptoms-cdc.html?action=click\&pgtype=Article\&state=default\&region=MAIN_CONTENT_3\&context=storylines_faq}{The
    C.D.C. has also} added chills, muscle pain, sore throat, headache
    and a new loss of the sense of taste or smell as symptoms to look
    out for. Most people fall ill five to seven days after exposure, but
    symptoms may appear in as few as two days or as many as 14 days.
  \end{itemize}
\item ~
  \hypertarget{does-asymptomatic-transmission-of-covid-19-happen}{%
  \paragraph{Does asymptomatic transmission of Covid-19
  happen?}\label{does-asymptomatic-transmission-of-covid-19-happen}}

  \begin{itemize}
  \tightlist
  \item
    So far, the evidence seems to show it does. A widely cited
    \href{https://www.nature.com/articles/s41591-020-0869-5}{paper}
    published in April suggests that people are most infectious about
    two days before the onset of coronavirus symptoms and estimated that
    44 percent of new infections were a result of transmission from
    people who were not yet showing symptoms. Recently, a top expert at
    the World Health Organization stated that transmission of the
    coronavirus by people who did not have symptoms was ``very rare,''
    \href{https://www.nytimes3xbfgragh.onion/2020/06/09/world/coronavirus-updates.html?action=click\&pgtype=Article\&state=default\&region=MAIN_CONTENT_3\&context=storylines_faq\#link-1f302e21}{but
    she later walked back that statement.}
  \end{itemize}
\end{itemize}

The program met opposition from drugmakers, who expressed concern that
the lower government prices would reduce innovation. One drugmaker
\href{https://timesmachine.nytimes3xbfgragh.onion/timesmachine/1993/04/02/175993.html?pageNumber=20}{told}
The New York Times that it ``may be unwilling to consider further
development of vaccines'' if the government didn't set prices high
enough to cover development costs.

Some legislators derided the program as a new and expensive entitlement.
``The government is taking over the vaccine distribution system,'' John
Danforth, a Republican senator from Missouri,
\href{https://timesmachine.nytimes3xbfgragh.onion/timesmachine/1994/05/30/242993.html?pageNumber=1}{said
at the time}. ``That's crazy.''

Mr. Danforth criticized the program as ``a new entitlement unrelated to
financial need.''

Congress ultimately landed on a compromise: the Vaccines for Children
Program, which buys vaccines for about half of Americans under 18. It
serves children who are uninsured or those who still face high vaccine
costs after their insurers pay, as well as Native American and Alaska
Native children.

Data from the Vaccines for Children Program shows that the government
typically gets better deals on vaccines. For example: It pays \$21.71
for a common measles, mumps and rubella vaccine. Private insurers pay an
average of \$78.69 for the same drug.

The price for the new Pfizer vaccine --- which could work out to
somewhere between \$20 and \$40 depending on how many doses are needed
--- is in line with what the government pays for many childhood
vaccines. (Americans would receive the vaccine free.)

``Forty dollars for a two-dose treatment is, in my opinion, very
reasonable,'' Dr. Tan said.

\href{https://www.cdc.gov/mmwr/preview/mmwrhtml/mm6316a4.htm}{A 2014
study} showed that vaccination rates among children had improved rapidly
since the start of the Vaccines for Children Program. The authors
concluded that the program's ability to ``remove financial and
logistical barriers hindering vaccination for low-income children likely
played a significant role in obtaining high coverage.''

In using the federal government's purchasing power for a coronavirus
vaccine, the Trump administration is taking an approach similar to one
suggested by the Democratic presidential candidate Joe Biden. Mr.
Biden's coronavirus response plan calls for the nation to ** ``ramp up
the large-scale manufacturing of as many vaccine candidates as
necessary'' and to ``ensure everyone, not just the wealthy and well
connected,'' has access to new therapies.

Advertisement

\protect\hyperlink{after-bottom}{Continue reading the main story}

\hypertarget{site-index}{%
\subsection{Site Index}\label{site-index}}

\hypertarget{site-information-navigation}{%
\subsection{Site Information
Navigation}\label{site-information-navigation}}

\begin{itemize}
\tightlist
\item
  \href{https://help.nytimes3xbfgragh.onion/hc/en-us/articles/115014792127-Copyright-notice}{©~2020~The
  New York Times Company}
\end{itemize}

\begin{itemize}
\tightlist
\item
  \href{https://www.nytco.com/}{NYTCo}
\item
  \href{https://help.nytimes3xbfgragh.onion/hc/en-us/articles/115015385887-Contact-Us}{Contact
  Us}
\item
  \href{https://www.nytco.com/careers/}{Work with us}
\item
  \href{https://nytmediakit.com/}{Advertise}
\item
  \href{http://www.tbrandstudio.com/}{T Brand Studio}
\item
  \href{https://www.nytimes3xbfgragh.onion/privacy/cookie-policy\#how-do-i-manage-trackers}{Your
  Ad Choices}
\item
  \href{https://www.nytimes3xbfgragh.onion/privacy}{Privacy}
\item
  \href{https://help.nytimes3xbfgragh.onion/hc/en-us/articles/115014893428-Terms-of-service}{Terms
  of Service}
\item
  \href{https://help.nytimes3xbfgragh.onion/hc/en-us/articles/115014893968-Terms-of-sale}{Terms
  of Sale}
\item
  \href{https://spiderbites.nytimes3xbfgragh.onion}{Site Map}
\item
  \href{https://help.nytimes3xbfgragh.onion/hc/en-us}{Help}
\item
  \href{https://www.nytimes3xbfgragh.onion/subscription?campaignId=37WXW}{Subscriptions}
\end{itemize}
