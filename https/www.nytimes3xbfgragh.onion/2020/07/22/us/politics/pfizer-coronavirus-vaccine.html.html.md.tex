Sections

SEARCH

\protect\hyperlink{site-content}{Skip to
content}\protect\hyperlink{site-index}{Skip to site index}

\href{https://www.nytimes3xbfgragh.onion/section/politics}{Politics}

\href{https://myaccount.nytimes3xbfgragh.onion/auth/login?response_type=cookie\&client_id=vi}{}

\href{https://www.nytimes3xbfgragh.onion/section/todayspaper}{Today's
Paper}

\href{/section/politics}{Politics}\textbar{}Pfizer Gets \$1.95 Billion
to Produce Coronavirus Vaccine by Year's End

\url{https://nyti.ms/2OOem6k}

\begin{itemize}
\item
\item
\item
\item
\item
\item
\end{itemize}

\hypertarget{the-coronavirus-outbreak}{%
\subsubsection{\texorpdfstring{\href{https://www.nytimes3xbfgragh.onion/news-event/coronavirus?name=styln-coronavirus-national\&region=TOP_BANNER\&block=storyline_menu_recirc\&action=click\&pgtype=Article\&impression_id=1a28d2e0-efba-11ea-8539-d55bbc3b7401\&variant=undefined}{The
Coronavirus
Outbreak}}{The Coronavirus Outbreak}}\label{the-coronavirus-outbreak}}

\begin{itemize}
\tightlist
\item
  live\href{https://www.nytimes3xbfgragh.onion/2020/09/05/world/coronavirus-covid.html?name=styln-coronavirus-national\&region=TOP_BANNER\&block=storyline_menu_recirc\&action=click\&pgtype=Article\&impression_id=1a28d2e1-efba-11ea-8539-d55bbc3b7401\&variant=undefined}{Latest
  Updates}
\item
  \href{https://www.nytimes3xbfgragh.onion/interactive/2020/us/coronavirus-us-cases.html?name=styln-coronavirus-national\&region=TOP_BANNER\&block=storyline_menu_recirc\&action=click\&pgtype=Article\&impression_id=1a28f9f0-efba-11ea-8539-d55bbc3b7401\&variant=undefined}{Maps
  and Cases}
\item
  \href{https://www.nytimes3xbfgragh.onion/interactive/2020/science/coronavirus-vaccine-tracker.html?name=styln-coronavirus-national\&region=TOP_BANNER\&block=storyline_menu_recirc\&action=click\&pgtype=Article\&impression_id=1a28f9f1-efba-11ea-8539-d55bbc3b7401\&variant=undefined}{Vaccine
  Tracker}
\item
  \href{https://www.nytimes3xbfgragh.onion/2020/09/02/your-money/eviction-moratorium-covid.html?name=styln-coronavirus-national\&region=TOP_BANNER\&block=storyline_menu_recirc\&action=click\&pgtype=Article\&impression_id=1a28f9f2-efba-11ea-8539-d55bbc3b7401\&variant=undefined}{Eviction
  Moratorium}
\item
  \href{https://www.nytimes3xbfgragh.onion/interactive/2020/09/02/magazine/food-insecurity-hunger-us.html?name=styln-coronavirus-national\&region=TOP_BANNER\&block=storyline_menu_recirc\&action=click\&pgtype=Article\&impression_id=1a28f9f3-efba-11ea-8539-d55bbc3b7401\&variant=undefined}{American
  Hunger}
\end{itemize}

Advertisement

\protect\hyperlink{after-top}{Continue reading the main story}

Supported by

\protect\hyperlink{after-sponsor}{Continue reading the main story}

\hypertarget{pfizer-gets-195-billion-to-produce-coronavirus-vaccine-by-years-end}{%
\section{Pfizer Gets \$1.95 Billion to Produce Coronavirus Vaccine by
Year's
End}\label{pfizer-gets-195-billion-to-produce-coronavirus-vaccine-by-years-end}}

Two pharmaceutical companies announced a nearly \$2 billion contract for
600 million doses of a vaccine, with the first 100 million promised
before the end of the year.

\includegraphics{https://static01.graylady3jvrrxbe.onion/images/2020/07/22/us/politics/22dc-virus-vaccine/merlin_174113568_f0395436-1ab7-4df9-86d6-fd1b39b08578-articleLarge.jpg?quality=75\&auto=webp\&disable=upscale}

By \href{https://www.nytimes3xbfgragh.onion/by/noah-weiland}{Noah
Weiland},
\href{https://www.nytimes3xbfgragh.onion/by/denise-grady}{Denise Grady}
and \href{https://www.nytimes3xbfgragh.onion/by/david-e-sanger}{David E.
Sanger}

\begin{itemize}
\item
  Published July 22, 2020Updated July 30, 2020
\item
  \begin{itemize}
  \item
  \item
  \item
  \item
  \item
  \item
  \end{itemize}
\end{itemize}

WASHINGTON --- As nations around the world race to lock up
\href{https://www.nytimes3xbfgragh.onion/2020/07/27/health/moderna-vaccine-covid.html}{coronavirus
vaccines} even before they are ready, the Trump administration on
Wednesday made one of the largest investments yet, announcing a nearly
\$2 billion contract with
\href{https://www.nytimes3xbfgragh.onion/2020/07/27/health/moderna-vaccine-covid.html}{Pfizer}
and a German biotechnology company for 100 million doses by December.

The contract is part of what the White House calls the Warp Speed
project, an effort to drastically shorten the time it would take to
manufacture and distribute a working
\href{https://www.nytimes3xbfgragh.onion/2020/07/22/upshot/vaccine-coronavirus-government-purchase.html}{vaccine}.
So far, the United States has put money into more than a half dozen
efforts, hoping to build manufacturing ability for an eventual
breakthrough.

Europe has a parallel effort underway. Germany recently took a 23
percent stake in a German firm, CureVac, that
\href{https://www.nytimes3xbfgragh.onion/2020/03/15/world/europe/cornonavirus-vaccine-us-germany.html}{President
Trump once tried to lure} to American shores in hopes that its vaccine,
if successful, would be distributed in the United States first. A
\href{https://www.nytimes3xbfgragh.onion/2020/05/04/world/europe/eu-coronavirus-vaccine.html}{European-led
fund-raising effort in May} brought \$8 billion in pledges from the
world's governments, philanthropists and leaders for coronavirus vaccine
research, even with the United States sitting out the conference.

China, meantime, has militarized the effort: Researchers associated with
the Academy of Military Medical Sciences have developed one of China's
leading vaccine candidates, and another Chinese firm, Sinopharm Group,
announced in June that it was beginning Phase 3 trials in the United
Arab Emirates.

The Pfizer contract, an agreement to ensure the pharmaceutical giant has
a market for its work, is the biggest splash yet by the Americans. No
vaccine has yet been developed, and it is not clear whether the Pfizer
version will work. But if the vaccine being produced by Pfizer and
BioNTech, the German firm, proves to be safe and effective in clinical
trials, the companies say they could manufacture those first 100 million
doses by the end of the year.

Under
\href{https://www.businesswire.com/news/home/20200722005438/en/Pfizer-BioNTech-Announce-Agreement-U.S.-Government-600}{the
arrangement}, the federal government would obtain that first batch for
\$1.95 billion, or about \$20 a dose, with the rights to acquire up to
500 million more, or 600 million total. Americans would receive the
vaccine for free. Before it could be distributed, it would need
emergency approval by the Food and Drug Administration. But the U.S.
government does not pay the nearly \$2 billion until the drug is
approved and the first 100 million doses are delivered.

Pfizer said that large-scale safety and efficacy trials were to begin
this month, with regulatory review set for as early as October, although
nothing was guaranteed.

``Depending on success in clinical trials, today's agreement will enable
the delivery of approximately 100 million doses of vaccine being
developed by Pfizer and BioNTech,'' Alex M. Azar II, the health
secretary, said in a statement announcing the deal.

On Monday, Pfizer and
\href{https://www.nytimes3xbfgragh.onion/2020/05/21/health/coronavirus-vaccine-astrazeneca.html}{AstraZeneca},
a British-Swedish drug company developing a potential
\href{https://www.nytimes3xbfgragh.onion/interactive/2020/06/09/magazine/covid-vaccine.html}{vaccine}
with Oxford University,
\href{https://www.nytimes3xbfgragh.onion/2020/07/20/world/covid-coronavirus-vaccine.html}{released
data} suggesting that their vaccines could stimulate strong immune
responses with only minor side effects.

But unlike AstraZeneca, which has also obtained funding from the U.S.
government, Pfizer did not receive a contract for its earlier research
and development efforts ---~only for the doses and their distribution.

\hypertarget{latest-updates-the-coronavirus-outbreak}{%
\section{\texorpdfstring{\href{https://www.nytimes3xbfgragh.onion/2020/09/04/world/covid-19-coronavirus.html?action=click\&pgtype=Article\&state=default\&region=MAIN_CONTENT_1\&context=storylines_live_updates}{Latest
Updates: The Coronavirus
Outbreak}}{Latest Updates: The Coronavirus Outbreak}}\label{latest-updates-the-coronavirus-outbreak}}

Updated 2020-09-05T12:05:40.998Z

\begin{itemize}
\tightlist
\item
  \href{https://www.nytimes3xbfgragh.onion/2020/09/04/world/covid-19-coronavirus.html?action=click\&pgtype=Article\&state=default\&region=MAIN_CONTENT_1\&context=storylines_live_updates\#link-1654f6ad}{Research
  connects vaping to a higher chance of catching the virus --- and
  suffering its worst effects.}
\item
  \href{https://www.nytimes3xbfgragh.onion/2020/09/04/world/covid-19-coronavirus.html?action=click\&pgtype=Article\&state=default\&region=MAIN_CONTENT_1\&context=storylines_live_updates\#link-52e4198a}{Another
  college football game won't be played as planned.}
\item
  \href{https://www.nytimes3xbfgragh.onion/2020/09/04/world/covid-19-coronavirus.html?action=click\&pgtype=Article\&state=default\&region=MAIN_CONTENT_1\&context=storylines_live_updates\#link-181cef0}{Pharmaceutical
  companies plan a joint pledge on safety standards as they move
  vaccines to the marketplace.}
\end{itemize}

\href{https://www.nytimes3xbfgragh.onion/2020/09/04/world/covid-19-coronavirus.html?action=click\&pgtype=Article\&state=default\&region=MAIN_CONTENT_1\&context=storylines_live_updates}{See
more updates}

More live coverage:
\href{https://www.nytimes3xbfgragh.onion/live/2020/09/04/business/stock-market-today-coronavirus?action=click\&pgtype=Article\&state=default\&region=MAIN_CONTENT_1\&context=storylines_live_updates}{Markets}

By refusing funding up until now, Pfizer was able to avoid drawn-out
contractual negotiations and get its vaccine to trials, company
officials say.

``We didn't accept the federal government funding solely for the reason
that we wanted to be able to move as quickly as possible with our
vaccine candidate into the clinic,'' John Young, Pfizer's chief business
officer, said on Tuesday at
\href{https://www.nytimes3xbfgragh.onion/2020/07/21/health/covid-19-vaccine-coronavirus-moderna-pfizer.html}{a
congressional hearing} with executives from five vaccine manufacturers.

Pfizer and BioNTech are developing a
\href{https://www.nytimes3xbfgragh.onion/2020/05/15/us/politics/coronavirus-vaccine-timeline.html}{vaccine}
candidate that uses genetic material from the virus,
\href{https://www.nytimes3xbfgragh.onion/2020/05/05/health/pfizer-vaccine-coronavirus.html}{known
as messenger RNA}, to stimulate the immune system without making the
recipient sick. The technology can create a vaccine quickly, but has not
yet produced one that has been approved and marketed.

\href{https://www.nytimes3xbfgragh.onion/2020/07/27/health/moderna-vaccine-covid.html}{Moderna},
a Massachusetts biotech company, received \$483 million from the U.S.
government for its vaccine development and is also using mRNA
technology. By putting the might of an industry giant behind it, Pfizer
is making the technology mainstream.

The lack of a track record has prompted some skepticism about this
approach, but Dr. Kathrin Jansen, a senior vice president and the head
of vaccine research and development at Pfizer, dismissed the criticism.

``That's not a scientific mind-set --- that just because it's new, it
will fail,'' she said in an interview.

Earlier in her career, Dr. Jansen worked for Merck, where she led its
development of a vaccine to prevent cervical cancer, which is caused by
a virus. The vaccine, Gardasil, has been successful. It, too, used a
technology that was new at the time and faced considerable skepticism.

Dr. Jansen said Pfizer had placed its bet on messenger RNA not just
because the technology could produce a vaccine quickly, but also because
its review of previous work by BioNTech on experimental cancer vaccines
suggested the approach could cause a powerful immune response. Before
the coronavirus pandemic, the two companies had been collaborating on
flu vaccines.

Vaccines using mRNA consist of genetic material from part of the virus,
encased in tiny particles made of fat that help it get into human cells.
The messenger RNA then prompts the cells to churn out a tiny piece of
the virus, causing the immune system to attack the real virus if the
person is exposed. In essence, the patient's cells become factories for
a harmless fragment of the virus.

These vaccines set off several different kinds of immune responses, Dr.
Jansen said, which is important because scientists do not know yet which
type will be most potent against the coronavirus.

Dr. Jansen described making such a vaccine as a clean, fast process that
required a relatively small footprint to produce many doses.

\href{https://www.nytimes3xbfgragh.onion/news-event/coronavirus?action=click\&pgtype=Article\&state=default\&region=MAIN_CONTENT_3\&context=storylines_faq}{}

\hypertarget{the-coronavirus-outbreak-}{%
\subsubsection{The Coronavirus Outbreak
›}\label{the-coronavirus-outbreak-}}

\hypertarget{frequently-asked-questions}{%
\paragraph{Frequently Asked
Questions}\label{frequently-asked-questions}}

Updated September 4, 2020

\begin{itemize}
\item ~
  \hypertarget{what-are-the-symptoms-of-coronavirus}{%
  \paragraph{What are the symptoms of
  coronavirus?}\label{what-are-the-symptoms-of-coronavirus}}

  \begin{itemize}
  \tightlist
  \item
    In the beginning, the coronavirus
    \href{https://www.nytimes3xbfgragh.onion/article/coronavirus-facts-history.html?action=click\&pgtype=Article\&state=default\&region=MAIN_CONTENT_3\&context=storylines_faq\#link-6817bab5}{seemed
    like it was primarily a respiratory illness}~--- many patients had
    fever and chills, were weak and tired, and coughed a lot, though
    some people don't show many symptoms at all. Those who seemed
    sickest had pneumonia or acute respiratory distress syndrome and
    received supplemental oxygen. By now, doctors have identified many
    more symptoms and syndromes. In April,
    \href{https://www.nytimes3xbfgragh.onion/2020/04/27/health/coronavirus-symptoms-cdc.html?action=click\&pgtype=Article\&state=default\&region=MAIN_CONTENT_3\&context=storylines_faq}{the
    C.D.C. added to the list of early signs}~sore throat, fever, chills
    and muscle aches. Gastrointestinal upset, such as diarrhea and
    nausea, has also been observed. Another telltale sign of infection
    may be a sudden, profound diminution of one's
    \href{https://www.nytimes3xbfgragh.onion/2020/03/22/health/coronavirus-symptoms-smell-taste.html?action=click\&pgtype=Article\&state=default\&region=MAIN_CONTENT_3\&context=storylines_faq}{sense
    of smell and taste.}~Teenagers and young adults in some cases have
    developed painful red and purple lesions on their fingers and toes
    --- nicknamed ``Covid toe'' --- but few other serious symptoms.
  \end{itemize}
\item ~
  \hypertarget{why-is-it-safer-to-spend-time-together-outside}{%
  \paragraph{Why is it safer to spend time together
  outside?}\label{why-is-it-safer-to-spend-time-together-outside}}

  \begin{itemize}
  \tightlist
  \item
    \href{https://www.nytimes3xbfgragh.onion/2020/05/15/us/coronavirus-what-to-do-outside.html?action=click\&pgtype=Article\&state=default\&region=MAIN_CONTENT_3\&context=storylines_faq}{Outdoor
    gatherings}~lower risk because wind disperses viral droplets, and
    sunlight can kill some of the virus. Open spaces prevent the virus
    from building up in concentrated amounts and being inhaled, which
    can happen when infected people exhale in a confined space for long
    stretches of time, said Dr. Julian W. Tang, a virologist at the
    University of Leicester.
  \end{itemize}
\item ~
  \hypertarget{why-does-standing-six-feet-away-from-others-help}{%
  \paragraph{Why does standing six feet away from others
  help?}\label{why-does-standing-six-feet-away-from-others-help}}

  \begin{itemize}
  \tightlist
  \item
    The coronavirus spreads primarily through droplets from your mouth
    and nose, especially when you cough or sneeze. The C.D.C., one of
    the organizations using that measure,
    \href{https://www.nytimes3xbfgragh.onion/2020/04/14/health/coronavirus-six-feet.html?action=click\&pgtype=Article\&state=default\&region=MAIN_CONTENT_3\&context=storylines_faq}{bases
    its recommendation of six feet}~on the idea that most large droplets
    that people expel when they cough or sneeze will fall to the ground
    within six feet. But six feet has never been a magic number that
    guarantees complete protection. Sneezes, for instance, can launch
    droplets a lot farther than six feet,
    \href{https://jamanetwork.com/journals/jama/fullarticle/2763852}{according
    to a recent study}. It's a rule of thumb: You should be safest
    standing six feet apart outside, especially when it's windy. But
    keep a mask on at all times, even when you think you're far enough
    apart.
  \end{itemize}
\item ~
  \hypertarget{i-have-antibodies-am-i-now-immune}{%
  \paragraph{I have antibodies. Am I now
  immune?}\label{i-have-antibodies-am-i-now-immune}}

  \begin{itemize}
  \tightlist
  \item
    As of right
    now,\href{https://www.nytimes3xbfgragh.onion/2020/07/22/health/covid-antibodies-herd-immunity.html?action=click\&pgtype=Article\&state=default\&region=MAIN_CONTENT_3\&context=storylines_faq}{~that
    seems likely, for at least several months.}~There have been
    frightening accounts of people suffering what seems to be a second
    bout of Covid-19. But experts say these patients may have a
    drawn-out course of infection, with the virus taking a slow toll
    weeks to months after initial exposure.~People infected with the
    coronavirus typically
    \href{https://www.nature.com/articles/s41586-020-2456-9}{produce}~immune
    molecules called antibodies, which are
    \href{https://www.nytimes3xbfgragh.onion/2020/05/07/health/coronavirus-antibody-prevalence.html?action=click\&pgtype=Article\&state=default\&region=MAIN_CONTENT_3\&context=storylines_faq}{protective
    proteins made in response to an
    infection}\href{https://www.nytimes3xbfgragh.onion/2020/05/07/health/coronavirus-antibody-prevalence.html?action=click\&pgtype=Article\&state=default\&region=MAIN_CONTENT_3\&context=storylines_faq}{.
    These antibodies may}~last in the body
    \href{https://www.nature.com/articles/s41591-020-0965-6}{only two to
    three months}, which may seem worrisome, but that's~perfectly normal
    after an acute infection subsides, said Dr. Michael Mina, an
    immunologist at Harvard University. It may be possible to get the
    coronavirus again, but it's highly unlikely that it would be
    possible in a short window of time from initial infection or make
    people sicker the second time.
  \end{itemize}
\item ~
  \hypertarget{what-are-my-rights-if-i-am-worried-about-going-back-to-work}{%
  \paragraph{What are my rights if I am worried about going back to
  work?}\label{what-are-my-rights-if-i-am-worried-about-going-back-to-work}}

  \begin{itemize}
  \tightlist
  \item
    Employers have to provide
    \href{https://www.osha.gov/SLTC/covid-19/standards.html}{a safe
    workplace}~with policies that protect everyone equally.
    \href{https://www.nytimes3xbfgragh.onion/article/coronavirus-money-unemployment.html?action=click\&pgtype=Article\&state=default\&region=MAIN_CONTENT_3\&context=storylines_faq}{And
    if one of your co-workers tests positive for the coronavirus, the
    C.D.C.}~has said that
    \href{https://www.cdc.gov/coronavirus/2019-ncov/community/guidance-business-response.html}{employers
    should tell their employees}~-\/- without giving you the sick
    employee's name -\/- that they may have been exposed to the virus.
  \end{itemize}
\end{itemize}

She added that it ``has the potential to be fast to produce a product
that is very well defined and very pure.''

Several other companies are also making such vaccines, and each has its
own formulation of the genetic material and types of fat used to encase
it.

The large vaccine studies set to begin this month will each include
30,000 people, with some getting placebo shots. The
\href{https://www.fda.gov/regulatory-information/search-fda-guidance-documents/development-and-licensure-vaccines-prevent-covid-19}{Food
and Drug Administration has said} that to be considered effective, a
coronavirus vaccine should protect 50 percent of the people who receive
it.

Companies hope to show proof of effectiveness by the fall, but that will
depend on enrolling enough volunteers in areas where the infection rate
is high enough to see a significant difference between the vaccinated
people and the placebo group.

``We think we will see the end points, given that the infection rates
are going up, up, up,'' Dr. Jansen said. ``If the stars are aligned, it
could be next fall. But everything has to be right.''

Dr. Amesh Adalja, an infectious disease physician and senior scholar at
the Johns Hopkins University Center for Health Security, said that
Pfizer, unlike some smaller pharmaceutical companies that the government
had contracted with, did not need research money because it was likely
to have the infrastructure and early data it needed to speed its vaccine
to trials without federal assistance.

``Pfizer is a company that has a lot of expertise in making vaccines,''
he said. ``They knew that any negotiation with the government could have
delayed the start'' of trials, which he said the company knew how to set
up rapidly.

He added that the \$1.95 billion agreement was a way to guarantee a
market for the vaccine at the end of production, since prominent
drugmakers have historically been hesitant to spend on infectious
disease outbreaks.

``Advance purchase agreements have been one way we've been able to
acquire vaccines and countermeasures against certain threats that
pharmaceutical companies have traditionally stayed away from,'' he said.

The agreement with Pfizer, which the company and the Department of
Health and Human Services announced Wednesday morning, is the largest
one yet for Operation Warp Speed. The federal government announced this
month that it would pay the Maryland-based company
\href{https://www.nytimes3xbfgragh.onion/2020/07/16/health/coronavirus-vaccine-novavax.html}{Novavax}
\$1.6 billion to expedite the development of a
\href{https://www.nytimes3xbfgragh.onion/interactive/2020/science/coronavirus-vaccine-tracker.html}{coronavirus
vaccine}.

``We've been committed to making the impossible possible by working
tirelessly to develop and produce in record time a safe and effective
vaccine to help bring an end to this global health crisis,'' Dr. Albert
Bourla, Pfizer's chairman and chief executive officer, said in a news
release.

Advertisement

\protect\hyperlink{after-bottom}{Continue reading the main story}

\hypertarget{site-index}{%
\subsection{Site Index}\label{site-index}}

\hypertarget{site-information-navigation}{%
\subsection{Site Information
Navigation}\label{site-information-navigation}}

\begin{itemize}
\tightlist
\item
  \href{https://help.nytimes3xbfgragh.onion/hc/en-us/articles/115014792127-Copyright-notice}{©~2020~The
  New York Times Company}
\end{itemize}

\begin{itemize}
\tightlist
\item
  \href{https://www.nytco.com/}{NYTCo}
\item
  \href{https://help.nytimes3xbfgragh.onion/hc/en-us/articles/115015385887-Contact-Us}{Contact
  Us}
\item
  \href{https://www.nytco.com/careers/}{Work with us}
\item
  \href{https://nytmediakit.com/}{Advertise}
\item
  \href{http://www.tbrandstudio.com/}{T Brand Studio}
\item
  \href{https://www.nytimes3xbfgragh.onion/privacy/cookie-policy\#how-do-i-manage-trackers}{Your
  Ad Choices}
\item
  \href{https://www.nytimes3xbfgragh.onion/privacy}{Privacy}
\item
  \href{https://help.nytimes3xbfgragh.onion/hc/en-us/articles/115014893428-Terms-of-service}{Terms
  of Service}
\item
  \href{https://help.nytimes3xbfgragh.onion/hc/en-us/articles/115014893968-Terms-of-sale}{Terms
  of Sale}
\item
  \href{https://spiderbites.nytimes3xbfgragh.onion}{Site Map}
\item
  \href{https://help.nytimes3xbfgragh.onion/hc/en-us}{Help}
\item
  \href{https://www.nytimes3xbfgragh.onion/subscription?campaignId=37WXW}{Subscriptions}
\end{itemize}
