Sections

SEARCH

\protect\hyperlink{site-content}{Skip to
content}\protect\hyperlink{site-index}{Skip to site index}

\href{https://www.nytimes3xbfgragh.onion/section/business}{Business}

\href{https://myaccount.nytimes3xbfgragh.onion/auth/login?response_type=cookie\&client_id=vi}{}

\href{https://www.nytimes3xbfgragh.onion/section/todayspaper}{Today's
Paper}

\href{/section/business}{Business}\textbar{}Landlords Jump the Gun as
Eviction Moratorium Wanes

\href{https://nyti.ms/3g51ftq}{https://nyti.ms/3g51ftq}

\begin{itemize}
\item
\item
\item
\item
\item
\item
\end{itemize}

\href{https://www.nytimes3xbfgragh.onion/news-event/coronavirus?action=click\&pgtype=Article\&state=default\&region=TOP_BANNER\&context=storylines_menu}{The
Coronavirus Outbreak}

\begin{itemize}
\tightlist
\item
  live\href{https://www.nytimes3xbfgragh.onion/2020/08/04/world/coronavirus-cases.html?action=click\&pgtype=Article\&state=default\&region=TOP_BANNER\&context=storylines_menu}{Latest
  Updates}
\item
  \href{https://www.nytimes3xbfgragh.onion/interactive/2020/us/coronavirus-us-cases.html?action=click\&pgtype=Article\&state=default\&region=TOP_BANNER\&context=storylines_menu}{Maps
  and Cases}
\item
  \href{https://www.nytimes3xbfgragh.onion/interactive/2020/science/coronavirus-vaccine-tracker.html?action=click\&pgtype=Article\&state=default\&region=TOP_BANNER\&context=storylines_menu}{Vaccine
  Tracker}
\item
  \href{https://www.nytimes3xbfgragh.onion/2020/08/02/us/covid-college-reopening.html?action=click\&pgtype=Article\&state=default\&region=TOP_BANNER\&context=storylines_menu}{College
  Reopening}
\item
  \href{https://www.nytimes3xbfgragh.onion/live/2020/08/04/business/stock-market-today-coronavirus?action=click\&pgtype=Article\&state=default\&region=TOP_BANNER\&context=storylines_menu}{Economy}
\end{itemize}

Advertisement

\protect\hyperlink{after-top}{Continue reading the main story}

Supported by

\protect\hyperlink{after-sponsor}{Continue reading the main story}

\hypertarget{landlords-jump-the-gun-as-eviction-moratorium-wanes}{%
\section{Landlords Jump the Gun as Eviction Moratorium
Wanes}\label{landlords-jump-the-gun-as-eviction-moratorium-wanes}}

The CARES Act temporarily protects millions of renters from being kicked
out of their homes for nonpayment. Filings aren't supposed to resume
until after Friday.

\includegraphics{https://static01.graylady3jvrrxbe.onion/images/2020/07/22/business/22evictions2/merlin_174815295_11edc9e6-4c9a-49c7-a279-7256ef3c6958-articleLarge.jpg?quality=75\&auto=webp\&disable=upscale}

\href{https://www.nytimes3xbfgragh.onion/by/matthew-goldstein}{\includegraphics{https://static01.graylady3jvrrxbe.onion/images/2018/11/06/multimedia/author-matthew-goldstein/author-matthew-goldstein-thumbLarge.png}}

By
\href{https://www.nytimes3xbfgragh.onion/by/matthew-goldstein}{Matthew
Goldstein}

\begin{itemize}
\item
  July 23, 2020
\item
  \begin{itemize}
  \item
  \item
  \item
  \item
  \item
  \item
  \end{itemize}
\end{itemize}

The four-month pause that has protected millions of Americans from
eviction cases is set to expire at the end of this week. But that hasn't
stopped landlords across the country from trying to get a head start
forcing renters out.

Landlords in Tucson, Ariz., filed dozens of eviction cases last month
despite the federal moratorium, which was put in place because of the
coronavirus crisis. Legal aid lawyers had to go to court to stop the
eviction of a San Antonio renter who had lost her job during a citywide
stay-at-home order. And in
\href{https://www.nhlp.org/wp-content/uploads/Douglas-County-Order-of-Dismissal.pdf}{Omaha},
a court found that a struggling renter's attempted eviction had violated
the emergency law.

As the number of Covid-19 cases has
\href{https://www.nytimes3xbfgragh.onion/interactive/2020/us/coronavirus-us-cases.html}{surged
across the country}, a disturbing trend has emerged: landlords
commencing eviction proceedings even though the CARES Act relief law
currently protects about 12 million tenants living in qualifying
properties.

Yolanda Jackson, a special-education paraprofessional in the DeKalb
County schools outside of Atlanta, lost her job in March when the
schools shut down. Ms. Jackson, a mother of two, has yet to receive an
unemployment check, despite confirmation that she was approved, and
hasn't been able to pay her rent. A charitable organization agreed to
cover her missed payments, but so far the manager of her complex,
LaVista Crossing Apartments, hasn't sent the necessary documentation to
accept it.

``I have tried everything in my power not to get to this point,'' Ms.
Jackson said. ``I've been here seven years, and they will not work with
me. I am just stressed out and trying to hold it together.''

She received an eviction notice in late June, and the manager said in a
court filing that the property wasn't covered by the federal moratorium.
But on Tuesday, lawyers for Legal Aid in Atlanta decided to take her
case after finding that the complex is in fact listed as having a
federally backed mortgage --- making it covered by the CARES Act
moratorium.

\includegraphics{https://static01.graylady3jvrrxbe.onion/images/2020/07/22/business/22evictions1a/merlin_174815346_7fe88791-46f4-4fdf-92d5-1337fecc011b-articleLarge.jpg?quality=75\&auto=webp\&disable=upscale}

Lawyers for LaVista Crossing did not respond to messages seeking
comment.

At least two other residents of the apartment complex have been served
with eviction notices for nonpayment, said Lindsey Siegel with Atlanta
Legal Aid. ``Many Legal Aid clients are facing evictions simply because
their unemployment benefits haven't come through,'' she said.

State and local governments have also issued eviction moratoriums, but
the CARES Act is the furthest reaching, covering as many as 12.3 million
renters living in an apartment complex or single-family home financed
with a federally backed mortgage. But like other moratoriums, it's about
to expire: After Friday, landlords can begin filing eviction notices for
failure to pay rent. It will be at least 30 days after that before any
tenants are kicked out.

The moratorium has been a lifeline for millions of unemployed people,
allowing renters waiting on slow-to-arrive aid to stay in their homes
and make up the payments later.

\hypertarget{latest-updates-economy}{%
\section{\texorpdfstring{\href{https://www.nytimes3xbfgragh.onion/live/2020/08/04/business/stock-market-today-coronavirus?action=click\&pgtype=Article\&state=default\&region=MAIN_CONTENT_1\&context=storylines_live_updates}{Latest
Updates:
Economy}}{Latest Updates: Economy}}\label{latest-updates-economy}}

\href{https://www.nytimes3xbfgragh.onion/live/2020/08/04/business/stock-market-today-coronavirus?action=click\&pgtype=Article\&state=default\&region=MAIN_CONTENT_1\&context=storylines_live_updates\#the-ad-giant-publicis-has-parted-ways-with-an-executive-over-his-virus-tweets}{1m
ago}

\href{https://www.nytimes3xbfgragh.onion/live/2020/08/04/business/stock-market-today-coronavirus?action=click\&pgtype=Article\&state=default\&region=MAIN_CONTENT_1\&context=storylines_live_updates\#the-ad-giant-publicis-has-parted-ways-with-an-executive-over-his-virus-tweets}{The
ad giant Publicis has `parted ways' with an executive over his virus
tweets.}

\href{https://www.nytimes3xbfgragh.onion/live/2020/08/04/business/stock-market-today-coronavirus?action=click\&pgtype=Article\&state=default\&region=MAIN_CONTENT_1\&context=storylines_live_updates\#nbcuniversal-to-cut-about-10-percent-of-its-work-force}{1h
ago}

\href{https://www.nytimes3xbfgragh.onion/live/2020/08/04/business/stock-market-today-coronavirus?action=click\&pgtype=Article\&state=default\&region=MAIN_CONTENT_1\&context=storylines_live_updates\#nbcuniversal-to-cut-about-10-percent-of-its-work-force}{NBCUniversal
to cut about 10 percent of its work force.}

\href{https://www.nytimes3xbfgragh.onion/live/2020/08/04/business/stock-market-today-coronavirus?action=click\&pgtype=Article\&state=default\&region=MAIN_CONTENT_1\&context=storylines_live_updates\#loans-are-harder-to-get-even-as-interest-rates-are-low}{3h
ago}

\href{https://www.nytimes3xbfgragh.onion/live/2020/08/04/business/stock-market-today-coronavirus?action=click\&pgtype=Article\&state=default\&region=MAIN_CONTENT_1\&context=storylines_live_updates\#loans-are-harder-to-get-even-as-interest-rates-are-low}{Loans
are harder to get, even as interest rates are low.}

\href{https://www.nytimes3xbfgragh.onion/live/2020/08/04/business/stock-market-today-coronavirus?action=click\&pgtype=Article\&state=default\&region=MAIN_CONTENT_1\&context=storylines_live_updates}{See
more updates}

More live coverage:
\href{https://www.nytimes3xbfgragh.onion/2020/08/04/world/coronavirus-cases.html?action=click\&pgtype=Article\&state=default\&region=MAIN_CONTENT_1\&context=storylines_live_updates}{Global}

But the far-ranging and hastily assembled CARES Act --- which, among
things, had provisions for direct relief payments, a temporary expansion
of unemployment insurance and hundreds of billions of dollars in
small-business aid --- does not penalize landlords who violate the
moratorium.

Paula Cino, a vice president for policy and government affairs at the
National Multifamily Housing Council, a landlord group, said there had
been some legitimate confusion at the outset with the federal moratorium
and local and state eviction pauses.

``That said, I wouldn't minimize the fact that there is the potential
for bad actors in this space,'' she said. ``Even if they weren't
initially taking advantage of the system, they have the responsibility
to better understand.''

Once an eviction case enters the legal system, it can have lasting
consequences: Even a wrongfully filed action can be difficult to remove
from court records and keep turning up
\href{https://www.nytimes3xbfgragh.onion/2020/05/28/business/renters-background-checks.html}{when
renters go through background checks}.

``An eviction judgment stays on a tenant's credit report for seven
years, is grounds for wage garnishment and makes it more difficult for a
tenant to find future housing,'' said Stacy Butler, a law professor at
the University of Arizona who has been tracking violations of the CARES
Act.

Image

Even with a moratorium in place, landlords have been serving eviction
notices in places across the country, housing advocates
say.Credit...Melissa Golden for The New York Times

Image

The moratorium bars the start of evictions for nonpayment for about 12
million renters in properties that have federally backed
mortgages.Credit...Melissa Golden for The New York Times

The scope of the problem is elusive. Wrongly evicted renters might not
bother trying to challenge their landlords, sometimes
\href{https://www.nytimes3xbfgragh.onion/2020/07/04/us/coronavirus-evictions-renters-immigrants.html}{because
of their immigration status}, or because they do not know they have the
right.

But wrongful evictions have been reported across the country. The
Private Equity Stakeholder Project, a consumer advocacy group, found
more than 100 filings in apparent violation of the CARES Act in Arizona,
Texas, Florida and Massachusetts.

And in a survey of 100 legal aid lawyers in 38 states, by the National
Housing Law Project, all but nine said they knew of attempts at illegal
evictions in their cities. The problem prompted the group to create
\href{https://www.nhlp.org/campaign/protecting-renter-and-homeowner-rights-during-our-national-health-crisis-2/}{a
draft complaint} to challenge a violation of the CARES Act moratorium.

Judges have been troubled, too.
\href{http://www.txcourts.gov/media/1449329/209086.pdf}{The Texas
Supreme Court} issued a statewide order on Tuesday requiring landlords
to certify whether the CARES Act applies to an eviction case, and
\href{http://www.azcourts.gov/Portals/22/admorder/Orders20/2020-105Final.pdf?ver=2020-07-07-153304-167}{Arizona's
Supreme Court} took a similar action earlier this month.

Lawmakers in Washington are debating another relief law --- including
possible stimulus payments, aid for governments and schools, and a
decision on what to do about the
\href{https://www.nytimes3xbfgragh.onion/2020/07/21/business/economy/coronavirus-unemployment-benefits.html}{extra
\$600 weekly unemployment benefit} --- and housing advocates want it to
have more help for renters.

The landlord group is in favor of help for tenants, too. The National
Multifamily Housing Council said it favored the creation of an emergency
rental assistance program of up to \$100 billion. But the organization
opposes a ``protracted extension of a federal eviction moratorium.''

If the moratorium is extended in another relief bill --- it is part of
t\href{https://www.congress.gov/bill/116th-congress/house-bill/6800/text}{he
\$3 trillion package passed by House Democrats} --- there are calls from
housing advocates to give it enough teeth to keep landlords from trying
to skirt the rules.

``There should also be clearly delineated enforcement mechanisms and
steep penalties for landlords who flout the law,'' said Diane Yentel,
president of the National Low Income Housing Coalition, which has set up
\href{https://nlihc.org/federal-moratoriums}{a webpage} to help tenants
determine if their rental is covered by the CARES Act.

Image

With some forms of aid slow to arrive, the eviction moratorium has
allowed struggling tenants to stay in their homes.Credit...Melissa
Golden for The New York Times

Nelson Mock, an attorney with Texas RioGrande Legal Aid, said lawyers
across Texas had seen ``landlords trying to sidestep the issue.''

Juanita Herrera DeLeon, 57, who lost her job in March during San
Antonio's stay-at-home order, had to fend off an eviction attempt
despite the CARES Act moratorium.

Soon after Ms. DeLeon lost her job, the manager of her apartment
complex, the Olmos Club Apartments, tried to lock her out by installing
a device on her doorknob. It was removed after she complained to the
police, but she said the complex had tried other tactics to get her to
leave, like posting on her front door a three-day notice to vacate the
premises.

That was when she sought help from RioGrande Legal Aid. In a statement
filed with her lawsuit, she said the property manager ``did not leave me
anything in writing about locking me out'' before the first attempt.

The suit was recently settled; Mr. Mock said he was not permitted to
discuss the terms.

Jason Adelstein, a lawyer for the Olmos Club Apartments, said, ``The
dispute was settled between the parties, my client denies any
wrongdoing, and due to the terms of the settlement agreement between the
parties there can be no further comment.''

The issue of CARES Act violations may be worst in Arizona.

In June alone, at least 80 eviction proceedings that were started in the
local courts in Pima County appeared to violate the CARES Act, according
to research by a team that included Ms. Butler, the law professor in
Tucson. Many were filed by small landlords, and it's hard to know
whether the filings were intentional or a mistake, she said.

One property owner, however, was responsible for filing more than a
dozen cases against residents of the
\href{https://www.equilibriumprops.com/cordova-village-apartments}{Cordova
Village} apartment complex on Tucson's south side.

The landlord, Equilibrium Properties, which operates several apartment
buildings in Tucson and Washington, D.C., said in an emailed statement
that the eviction filings had been made in error. The company, which
\href{https://projects.propublica.org/coronavirus/bailouts/loans/equilibrium-realty-llc-af99191e3bb8f54ef7b7046b9a4fe5be}{received
at least \$150,000 under the Paycheck Protection Program} established by
the CARES Act, said it had moved to vacate the proceedings and was
``rescinding all notices for nonpayment that have been given to
tenants.''

``Moving forward,'' the company said, ``we will take every effort to
comply with the CARES Act.''

Advertisement

\protect\hyperlink{after-bottom}{Continue reading the main story}

\hypertarget{site-index}{%
\subsection{Site Index}\label{site-index}}

\hypertarget{site-information-navigation}{%
\subsection{Site Information
Navigation}\label{site-information-navigation}}

\begin{itemize}
\tightlist
\item
  \href{https://help.nytimes3xbfgragh.onion/hc/en-us/articles/115014792127-Copyright-notice}{©~2020~The
  New York Times Company}
\end{itemize}

\begin{itemize}
\tightlist
\item
  \href{https://www.nytco.com/}{NYTCo}
\item
  \href{https://help.nytimes3xbfgragh.onion/hc/en-us/articles/115015385887-Contact-Us}{Contact
  Us}
\item
  \href{https://www.nytco.com/careers/}{Work with us}
\item
  \href{https://nytmediakit.com/}{Advertise}
\item
  \href{http://www.tbrandstudio.com/}{T Brand Studio}
\item
  \href{https://www.nytimes3xbfgragh.onion/privacy/cookie-policy\#how-do-i-manage-trackers}{Your
  Ad Choices}
\item
  \href{https://www.nytimes3xbfgragh.onion/privacy}{Privacy}
\item
  \href{https://help.nytimes3xbfgragh.onion/hc/en-us/articles/115014893428-Terms-of-service}{Terms
  of Service}
\item
  \href{https://help.nytimes3xbfgragh.onion/hc/en-us/articles/115014893968-Terms-of-sale}{Terms
  of Sale}
\item
  \href{https://spiderbites.nytimes3xbfgragh.onion}{Site Map}
\item
  \href{https://help.nytimes3xbfgragh.onion/hc/en-us}{Help}
\item
  \href{https://www.nytimes3xbfgragh.onion/subscription?campaignId=37WXW}{Subscriptions}
\end{itemize}
