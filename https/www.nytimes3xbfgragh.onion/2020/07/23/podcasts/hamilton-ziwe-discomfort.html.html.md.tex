Sections

SEARCH

\protect\hyperlink{site-content}{Skip to
content}\protect\hyperlink{site-index}{Skip to site index}

\href{https://www.nytimes3xbfgragh.onion/spotlight/podcasts}{Podcasts}

\href{https://myaccount.nytimes3xbfgragh.onion/auth/login?response_type=cookie\&client_id=vi}{}

\href{https://www.nytimes3xbfgragh.onion/section/todayspaper}{Today's
Paper}

\href{/spotlight/podcasts}{Podcasts}\textbar{}Ziwe May Destroy Hamilton

\href{https://nyti.ms/2OSCYee}{https://nyti.ms/2OSCYee}

\begin{itemize}
\item
\item
\item
\item
\item
\end{itemize}

Advertisement

\protect\hyperlink{after-top}{Continue reading the main story}

transcript

Back to Still Processing

bars

0:00/38:10

-38:10

transcript

\hypertarget{ziwe-may-destroy-hamilton}{%
\subsection{Ziwe May Destroy Hamilton}\label{ziwe-may-destroy-hamilton}}

\hypertarget{hosted-by-wesley-morris-and-jenna-wortham-produced-by-hans-buetow}{%
\subsubsection{Hosted by Wesley Morris and Jenna Wortham. Produced by
Hans
Buetow.}\label{hosted-by-wesley-morris-and-jenna-wortham-produced-by-hans-buetow}}

\hypertarget{welcome-to-the-age-of-discomfort}{%
\paragraph{Welcome to the Age of
Discomfort}\label{welcome-to-the-age-of-discomfort}}

Thursday, July 23rd, 2020

\begin{itemize}
\item
  wesley morris\\
  We are rolling.
\item
  jenna wortham\\
  I'm so ready. Yo, put me in the cabinet battle. I'm ready. We got
  things to discuss.
\item
  {[}music{]}
\item
  wesley morris\\
  Well, we should talk about how you were blowing up my phone last
  night.
\item
  jenna wortham\\
  OK, I'm starting this thing later than I planned because I watched the
  sunset, and I made dinner, and I'm already a little bit embarrassed.
  Who is this Negro in colonial get-up?
\item
  archived recording (leslie odom jr)\\
  I'm the damn fool that shot him.
\end{itemize}

{[}music - kindness, world restart{]}

wesley morris

I'm Wesley Morris.

jenna wortham

And I'm Jenna Wortham. We are two culture writers for The New York
Times, and I decided to watch ``Hamilton,'' of course. I've successfully
blocked out so much around ``Hamilton,'' there's a Hamilton-shaped hole
in my brain, and I'm fine with it. But Disney+ put the production on its
streaming service a few weeks ago. I felt like the internet was talking
about it. I didn't know exactly what people were referencing. I had to
take the bait, and that's what we did.

wesley morris

So I mean, let's just talk about those texts you sent me and what you
saw.

jenna wortham

OK.

{[}music{]}

\begin{itemize}
\tightlist
\item
  archived recording (chorus)\\
  1, 2, 3, 4, 5, 6, 7, 8, 9.
\end{itemize}

wesley morris

Let's go. Text number one, ``Wait --- ''

jenna wortham

Oh, god.

wesley morris

'' --- why are people clapping so much for George III?'' Text number
two, ``How did anyone watch this without a phone to Google shit?''

jenna wortham

Who is George III? I'm like, who ---

wesley morris

(LAGUHS)

jenna wortham

--- who dis? I was like, why are y'all so excited for this man? Who is
this man? What has he done or not done to us? I really was confused, and
so I had to pause. I had to look some things up.

wesley morris

(LAUGHS) Text number three, ``June 2016, dot, dot, dot, those innocent
creatures.''

jenna wortham

It was a different time. It was a different time,

wesley morris

Text number five is just a parenthetical that is obviously a lie. ''(I
won't do this all night. I promise. But I have to a little.)'' Lies,
damn lies.

Number 12. All caps, ``NO, OH, WHITE GLOVES.''

jenna wortham

Wesley, you're forgetting the voice notes that I --- I had to stop
typing and start sending you voice notes, because I was losing my
ability to type clearly.

\begin{itemize}
\tightlist
\item
  jenna wortham\\
  Sir, wait, this Thomas Jefferson dance, why didn't we see this
  together in person? I would have died. I can't believe you went to see
  this without me. I don't forgive you. I don't forgive you.
\end{itemize}

jenna wortham

I feel that so hard, I put that in our Slack this morning. I will never
forgive you for seeing this without me! And then I got self-conscious
about sending you so many messages, so I --- I reeled it in.

wesley morris

You don't ever have to keep yourself self-conscious about sending
messages to me. Anyway, white gloves.

\begin{itemize}
\tightlist
\item
  archived recording (daveed diggs)\\
  So what did I miss?
\end{itemize}

jenna wortham

Yo, my G, it's not just that Thomas Jefferson comes out and starts
shuckin' and jivin', played by Daveed Diggs, who is so incandescent in
this role, like lit from within.

\begin{itemize}
\tightlist
\item
  archived recording (daveed diggs)\\
  I guess I basically missed the late `80s. I traveled the wide, wide
  world and came back to this.
\end{itemize}

jenna wortham

But Thomas Jefferson's back from France talking about, what did I miss
out here, doing this little kicky dance! And then in the background, all
the dancers, who are Black and beautiful, have on white gloves. I was
like, Wesley, we already saw ``Get Out.'' We saw ``Us'' together. What
horror show is this? I was like, oh my god, why weren't we in this
together? My hands would have been digging into your thighs, and I would
have been crying with tears. Like, what out of body transportation
experience is this? And everybody in the audience is going wild.

\begin{itemize}
\tightlist
\item
  archived recording (daveed diggs)\\
  What did I miss?
\end{itemize}

wesley morris

The first time I saw it, it was just mayhem in the audience.

jenna wortham

My god, I bet.

wesley morris

It was like the late summer of 2015. So I don't know how long --- the
show had barely been opened.

jenna wortham

It just had --- yeah, it had just opened.

wesley morris

And it was like --- I mean, I don't even know what to say. It was just
electric. It was like after every song, people were losing their minds.
I think in 2015, the thing that was so exciting about ``Hamilton'' was
its mere existence. I mean, it wasn't just white people who had taken
out their bowls and spoons and were ready to eat some, you know,
non-white people on Broadway. It was young, Black, and Latino people,
who were just dying to maybe be able to see this thing appear on a
Broadway stage. You definitely felt that something rare was happening.
There was a real belief, and it was symbolized by the Obamas and Marian
Robinson being in the White House together. And not just that, but you
had Eric Holder.

jenna wortham

Right.

wesley morris

This was about non-white people being in the government, right?

jenna wortham

Yes, yes, I understand. I can see that.

wesley morris

There is a real mirror between the presidential administration of 2015
and the goings-on in this new country being formed at the end of the
18th century.

jenna wortham

But, Wesley, I'm really curious to hear your perspective, too. Because
in my recollection, there were some academics who were coming out, and
there was definitely a little bit of discomfort around the way
``Hamilton'' was treating history. And what does it mean to put Black
people into colonial get-up and also say, what does it mean to have
Thomas Jefferson played by a Black person, who's talking around slavery
and not really about slavery, but all this wealth that it creates within
the play? What's your sense?

wesley morris

Well, I mean, here's what I would say. I'd say at the time, criticisms
began, and where they were most robust in 2015, were entirely --- almost
entirely around the question of the depiction of the history. And what
the show was responsible for perpetuating or not addressing with respect
to Alexander ``Hamilton,'' and Thomas Jefferson, and Hercules Mulligan,
and Aaron Burr, and George Washington, and, and, and, and, and. These
were actual legitimate criticisms of Ron Chernow's book, which came out
in 2004, and is the source material that Lin-Manuel Miranda used when he
was writing ``Hamilton.'' But those criticisms were very apparent during
the original run of the show. The thing was that there were a lot of
people who were like, shhhh ---

jenna wortham

Yes, yes.

wesley morris

--- just shh, shh, shh!

jenna wortham

I remember this.

wesley morris

``Stop it, stop it! We've got this nice thing, and you cannot take it
from us. Just stop it.''

jenna wortham

Right.

wesley morris

``He just got to Broadway, y'all. Just let him win his Tonys, and then
we will have this fight.''

jenna wortham

Have those conversations over there.

wesley morris

Yeah. But even in 2015, it was a little uncomfortable, sitting there
watching, knowing what you were watching in some ways, that you had
these white men being played by these non-white people.

jenna wortham

And frankly, ``Hamilton'' really pushes you to grapple with the idea
that the white gaze ain't always the only gaze that can be problematic,
right? Like, Black people, brown people, other kinds of people can have
problematic gazes. But that's not a conversation we were comfortable
having I would say five years ago. It's a conversation we struggle with.
We still struggle with it.

wesley morris

Still struggling.

jenna wortham

Still struggling. I mean, a big part of that is what you were alluding
to earlier, right? Like, nobody wants to be the person coming after all
these successful Black and brown men. And it gets really complicated to
talk about how Black people can be anti-Black as well. Hello, Kanye?
We're still starting to have that conversation. It's not easy.

wesley morris

Yeah, yeah.

jenna wortham

Right?

wesley morris

Yeah, yeah, yeah.

jenna wortham

And that's something that we really --- we really don't like talking
about.

wesley morris

Yup.

jenna wortham

At all.

wesley morris

Nope.

jenna wortham

Anyone who's not, quote, the majority, so rarely gets these
opportunities. Like, it doesn't feel right to poke holes in it or pull
the rug underneath them. But that's where we're at also right now.

wesley morris

So I think that what you're identifying, though, is a major shift,
right?

jenna wortham

Mm-hmm.

wesley morris

There has been a shift in the culture between 2015 ---

jenna wortham

Yes.

wesley morris

--- and 2020. In 2015, our cultural priorities were really about being
seen. ``I can't believe that this person looks like me. I am so happy to
see a person who looks like me. Looks like me, looks like me, looks like
me.''

jenna wortham

Right, right, right.

wesley morris

I think during the Obama era, it was about being seen. And I think in
the Trump era, it is about being heard.

jenna wortham

Mm-hmm.

wesley morris

And in the same way that it was right on time in 2015, it is right on
time for 2020.

jenna wortham

I think the conversation in 2020 looks so different because the stakes
feel lower. ``Hamilton's'' already had its success. There's no way to
dim the star shine of ``Hamilton'' anymore. It had its run. People were
tweeting things like, it's problematic to glorify Hamilton, Washington,
Jefferson, and Madison. People were tweeting, there is a total erasure
of slavery in this show. Given every single thing that's happening right
now, are you really going to not say what you really think about
``Hamilton?'' Is that where you draw the line? No, all bets are off. And
so I think ``Hamilton'' bears the brunt of that. And rightfully so.

wesley morris

These aren't just any old white men, right? These are men whose statues
are in the middle of being defaced, torn down, debated about whether
they should have ever gone up in the first place. Like, what are we
honoring when we honor these men? And here is a show that essentially is
being peopled by the people who oppressed them, more or less. And I
think not only do we get to watch and celebrate this thing that on many
levels is fun, I think we also have to sit in the discomfort, if you
were a person who championed this thing five years ago, what it means to
sit in the discomfort now.

jenna wortham

Yes, yes.

wesley morris

And the discomfort of being made to see something that you just couldn't
see before, not necessarily because you didn't want to see it, but
because the time wasn't right for you to see it, or no one had
sufficiently forced you to look at what you were enjoying so much.

jenna wortham

Mm-hmm.

wesley morris

But if we're talking about reckonings and topplings, I think there is a
deeper part of that which entails us as people who listen, and read, and
watch, and visit, and attend, and write to think about our relationship
to some of these problems, too. I mean, I'm down for discomfort. I'm
down to not feel stable. Because we shouldn't be right now. That's not
what this moment is actually about.

{[}music{]}

wesley morris

Jenna.

jenna wortham

Yes.

wesley morris

I'm an American. I live with discomfort all the time.

jenna wortham

Mm.

wesley morris

I'm aware, for instance, that when I wake up in the morning, I have
awoken on somebody else's land.

jenna wortham

Yes, in fact, the land of the Canarsie and the Munsee Lenape, which I
love this app called Native Land. You can plug into any address anywhere
in the world, and it will tell you who the original stewards of the land
are, and it really has been opening my eyes to just the hidden
indigeneity that's all around us.

wesley morris

That is a great app.

jenna wortham

Mm-hmm.

wesley morris

And so I'm also aware that I am a traceable descendant of enslaved
people. And that leads into a pretty natural awareness of all kinds of
other things. Like, who is making my food? Who is growing the food? Who
is slaughtering the animals that I'm going to eat as a non-vegeterian,
you know? I just go through life assuming that there is discomfort out
there, and that it's built into all aspects of living in the 21st
century. And I'm comfortable with that discomfort. And not the sort of
discomfort that a person like Robin DiAngelo is trying to get people to
feel. Robin DiAngelo is, of course, the author of the best selling again
book, ``White Fragility.'' I just find that project nauseating. It's
like she is telling white people, ``Don't talk, shut up. Things are too
racist for you to speak, and you are the source of the racism. And you
can't say anything about racism, either, because Black people will not
like you. They will not be able to handle it.'' I don't like her
pitching white people into that place of discomfort, because what she is
offering as discomfort for her white readership is really work for me
and you.

jenna wortham

Absolutely.

wesley morris

Right? Also, I'm not getting a paycheck to do that work that she's
setting me up to do. I don't know about you.

jenna wortham

I mean, it's one thing to sit in a realm of discomfort. Like a lot of my
talk therapy practice is rooted in this idea of what we do and don't
face, and then the childhood roots behind that. And so there's a lot of
discussion of sitting with the discomfort and looking at it, and being
willing to tangle with it, and then try to root out what are the
feelings, what are they masking, why do I feel so uncomfortable, and
then what am I going to do about it? There's a type of, I think, passive
discomfort that you're talking about versus something that is actually
quite an active, emotional, psychological, intellectual state.

wesley morris

I'm here for a little discomfort. I'm used to feeling uncomfortable. I'm
used to having discomfort induced in me.

jenna wortham

Yes.

wesley morris

And that is part of the pleasure of watching this show ``Baited'' that
you got me to watch, this show by Ziwe, Ziwe Fumudoh is the comedian who
goes by the first name Ziwe on the show, which is now an Instagram Live
experience between her and her white guests.

\begin{itemize}
\item
  archived recording (ziwe fumudoh)\\
  Hi.
\item
  archived recording (woman)\\
  Hi.
\item
  archived recording (ziwe fumudoh)\\
  How are you doing?
\item
  archived recording (woman)\\
  Honestly, I'm really nervous.
\item
  archived recording (ziwe fumudoh)\\
  You're really nervous?
\item
  archived recording (woman)\\
  Yeah.
\item
  archived recording (ziwe fumudoh)\\
  On a scale of Black person minding their business in their house to
  Black person minding their business at the movie theater, how nervous
  are you?
\item
  archived recording (woman)\\
  Like a zero. Like, cannot compute that. So it's not registering on
  that scale for sure.
\end{itemize}

jenna wortham

It's worth noting that this project of Ziwe's has been going on for some
time. She had this show on YouTube, and then it transitioned to
Instagram Live. And historically, the types of people she brings on the
show are friends, other stand-up comedians, colleagues, who I think she
feels haven't really thought out their relationship to their own
whiteness, or their relationship to identity, and the various
intersections of their identity. There's a really hilarious convo, which
you can put in the show notes. She interviews a white cis gay male
colleague of hers, who keeps insisting that because he's gay he
understands what it means to be marginalized. More recently on her
Instagram Lives, Ziwe's been inviting white people onto the show who
have had some sort of embarrassingly public moment dealing with race. So
not just her immediate circle, but people who've so-called been canceled
via Twitter. So the, I guess author, social media scam artist Caroline
Calloway, and then Alison Roman, who is a chef, and a cookbook author,
and a columnist for The New York Times. And she just wants to get white
people who've had some sort of social media kerfuffle on the show. But
it seems like the kinds of people who say yes are the kinds of people
who feel like going on that show is doing the work.

\begin{itemize}
\item
  archived recording (caroline calloway)\\
  This is a really stressful time for Black people, and I'm really glad
  that you can have this emotional rest by having me on the show.
\item
  archived recording (ziwe fumudoh)\\
  Totally. And when you say Black people, do you capitalize the B?
\item
  archived recording (caroline calloway)\\
  Absolutely, because otherwise it's a color.
\item
  archived recording (ziwe fumudoh)\\
  Performative. Now, first question for you Caroline is, I saw --- I
  watched your Cambridge Union interview today, and you said that,
  famously, you discovered racism in 2018. What were you doing for the
  first 25 years of your life?
\end{itemize}

jenna wortham

I don't think Ziwe's calling anybody a racist, but I think she's
acknowledging that if you are a person, especially a white person and
you grew up in America, it is very difficult to unlearn the ways in
which you have been prioritized, and privileged, and given advantages,
and taught to think about your relationship to every other person in
this country, you know? And I think that's what she's trying to tease
out in some of those similarities in the way that her white friends, and
then the white people she encounters operate.

\begin{itemize}
\item
  archived recording (ziwe fumudoh)\\
  How many Black friends do you have, Alison Roman?
\item
  archived recording (alison roman)\\
  Do you define friend like someone who would pick me up from the
  airport, or like people I follow or people that I know? Because I have
  like, I would say, four to five Black friends that would pick me up at
  the airport.
\item
  archived recording (ziwe fumudoh)\\
  Four to five? You are the third person to say they have four to five
  Black friends in the last week to me. Caroline Callaway, Nick
  Ciarelli, and yourself.
\item
  archived recording (alison roman)\\
  Oh.
\item
  archived recording (ziwe fumudoh)\\
  That's an interesting statistic.
\item
  archived recording (alison roman)\\
  I actually didn't see that part of them. I mean, yeah, in the grand
  total of actual friends I have that would pick me up at the airport is
  probably like 12 to 13.
\end{itemize}

jenna wortham

The joke becomes in the repetition that no one bothers to do their
research. When you start to watch all the ways in which these white
women mostly think they're doing themselves a favor, right, by going on
Ziwe's show. But then they're all being --- they've done no research.
They're just popping in to say hey, and are totally unprepared and
totally disheveled. And just imagine if you or I showed up for any kind
of interview, or went on Anderson Cooper, and didn't know who he was. Or
imagine going on Jimmy Fallon, and thinking it was Jimmy Kimmel, right?
It just starts to become in itself this larger arc of the way whiteness
is allowed to function. And when you start to interrogate the larger
questions about why these women continue to treat Ziwe this way, it's
sinister. It's ``Get Out'' Part 3.

wesley morris

And these aren't like an obviously racist cop or Jeff Sessions. This is
the racism that regular people feel on a daily basis, right?

jenna wortham

Yes, yes.

wesley morris

This is the racism of social intimacy or social acquaintance, and how do
those people who practice that kind of ambient racism, institutional
racism, who benefit from it, how do they respond when asked, how many
Black authors have you read? How many Black people do you know?

\begin{itemize}
\item
  archived recording (ziwe fumudoh)\\
  Now, what do you qualitatively like about Black people?
\item
  archived recording (alison roman)\\
  Qualitatively that ---

  I mean, I can say what I like qualitatively about my Black friends, or
  you mean generally speaking?
\item
  archived recording (ziwe fumudoh)\\
  However you interpret the question.
\item
  archived recording (alison roman)\\
  Qualitatively, I love that their food almost always tastes better than
  mine.
\item
  archived recording (ziwe fumudoh)\\
  OK, we're gonna stop you right there.
\item
  archived recording (alison roman)\\
  They're way better dancers.
\item
  archived recording (ziwe fumudoh)\\
  I'll stop you right there.
\item
  archived recording (alison roman)\\
  OK.
\item
  archived recording (ziwe fumudoh)\\
  We're gonna stop you --- I'm doing you a favor (LAUGHS). OK?
\end{itemize}

wesley morris

But I would love --- and part of the reason that I'm interested in this
project is because I would like to see it evolve into an actual
discourse and not a performance, right?

jenna wortham

Yes, absolutely. I think at the beginning of the Instagram Lives, what
was entertaining was, it almost started to feel like an exploration of a
kink, of like a humiliation, submissive, kind of B.D.S.M. role play.

wesley morris

Welcome to racism.

jenna wortham

Welcome to racism, right? It was really fascinating. I do think that
what's fascinating about the ``Baited'' series is that even though white
people are the subject, they are completely not the center. They're not
centered as a subject. It's like the performance of it is to allow her
Black viewers, and I would say mostly Black women who are watching the
show, to really feel validated in some ways about all the
microaggressions, and the gaslighting, and the really uncomfortable
experiences that you've had. You have this duplicitousness of people who
are openly saying, yes, Black Lives Matter, support Black chefs, support
Black authors, but then can't name them when they're asked. That's real.
You're playing with this outward performance and how that performance
shows up in your personal politics. That's really interesting.

\begin{itemize}
\item
  archived recording (ziwe fumudoh)\\
  Now, I saw on your Instagram that you are promoting Black authors like
  Wesley Lowery, who wrote ``They Can't Kill Us,'' and ``The New Jim
  Crow'' by Michelle Alexander.
\item
  archived recording (caroline calloway)\\
  Austin Channing. Let's fucking go!
\item
  archived recording (ziwe fumudoh)\\
  Totally.
\item
  archived recording (caroline calloway)\\
  Layla Safad (sic). Let's go!
\item
  archived recording (ziwe fumudoh)\\
  Exactly. Now, you're a vociferous reader. How many of these books have
  you read?
\item
  archived recording (caroline calloway)\\
  Honestly, of the nine books that I recommended on my Instagram, I've
  read four.
\item
  archived recording (ziwe fumudoh)\\
  Wow.
\item
  archived recording (caroline calloway)\\
  But I've ordered the other five from Black book shops, so I would like
  my ally cookie now.
\item
  archived recording (ziwe fumudoh)\\
  There are no cookies in this game.
\end{itemize}

jenna wortham

I think there's something really profound about it happening on
Instagram as well, which is a space where so much of those performances
are also happening right now, and this feeling of fatigue around, stop
telling me my life matters, and tell me what you're doing about it, and
what are we actually going to do about it besides talk about it? How
does what you share online show up in your offline life? And I think
there is this kind of ---

wesley morris

That's ---

jenna wortham

It's very conscious, and very in the air, and it's also very
subconscious, too, because my discomfort in watching the show is not
about my relationship to race and to whiteness in the way of the guests,
right? But it's about, what does it mean for me to be participating in
these types of evisceration. That's what makes me uncomfortable. And I
really value that. I really value that. Because I'm not someone who
loves a pile-on. That's not my Twitter M.O. or Instagram M.O., and yet
here I am, every Thursday at 8.

wesley morris

Yeah, yeah, yeah, yeah. So, Jenna, we're talking about another form of
discomfort that's different from the ``Hamilton'' discomfort of
realizing that a thing that you love has a lot of problems in it.

jenna wortham

Mm-hmm.

wesley morris

I mean, we're talking about Ziwe Fumudoh's project, but I'm also
thinking about ``I May Destroy You,'' this show that H.B.O. is putting
on from England. And it's been conceived by and stars Michaela Coel, who
is an indescribably chameleonic performer, and a very astute observer of
human relationships and micro-everything.

jenna wortham

Yes.

wesley morris

Like micro-reactions, micro-emotions, micro-shifts in interpersonal
dynamics, the micro-shifts in time, and space, and memory. She's so
smart about these small things that escape the notice of so many
creators. I mean, this show, which is essentially about what happens to
this woman who Michaela Coel is playing, her name is Arabella. And she
has been sexually assaulted more than one time. And you are trying to
figure out what happened to her.

\begin{itemize}
\item
  archived recording (woman)\\
  And the assault you recall?
\item
  archived recording (michaela coel)\\
  The thing in my head?
\item
  archived recording (woman)\\
  Yes.
\item
  archived recording (michaela coel)\\
  Yeah, I wouldn't --- because now you're calling it something that I
  never --- I never said that.
\item
  archived recording (woman)\\
  Do you see anyone else?
\item
  archived recording (michaela coel)\\
  Where?
\item
  archived recording (woman)\\
  In this memory.
\item
  archived recording (michaela coel)\\
  You can't call it a memory.
\item
  archived recording (woman)\\
  OK, other than the man in the ---
\item
  archived recording (michaela coel)\\
  In my head. He may not even be real, because I'm the person that can
  actually see it, and I'm not sure, so I should probably pay attention
  to that.
\end{itemize}

wesley morris

And one of my favorite things about this show is that it manages to
feature discomfort working simultaneously on several different levels.

jenna wortham

Mm-hmm.

wesley morris

I'm thinking specifically about episode 6, which is essentially a
flashback episode. And all the main characters are younger versions of
themselves, and they're being played by different actors, kids. And they
are in the middle of sleuthing an alleged sexual assault that's occurred
among them. And they're standing in the courtyard, the playground,
having this conversation about what they think happened or will happen
to one of them, while in the background there is a girl being beaten up.
She's just being rolled on by these other kids, and none of these
characters in the foreground, these characters that we have been
watching for five previous episodes as adults, are noticing anything
happening behind them. They do not hear, notice, see anything happening
to this girl who's on the ground.

jenna wortham

Who's also Black.

wesley morris

Getting the --- who's also Black, getting the life kicked out of her.
And this search for justice in this one situation, meanwhile there is
actually something ---

jenna wortham

Right, right.

wesley morris

--- demonstrably something happening behind these kids, and they don't
care about it at all. They don't even see it.

jenna wortham

Yeah, the way that scene is working is, obviously it's a metaphor, but
it's a metaphor for the state of Black women in general all around the
world. We are at a moment where we're witnessing collectively the way
the world will rally around the murder of George Floyd but not Breonna
Taylor, right? The way the world pays attention to what happens to cis
Black folks and not Black trans folks. I mean, there's all of this
nuance that feels inconvenient and difficult to discuss, that that
episode is perfectly encapsulating by what's happening in the foreground
and what's happening in the background when it comes to side by side
assaults, and side by side attention, and side by side caretaking.

wesley morris

It's just interesting that these things are coming from --- are being
told by starring, driven by Black women. And like a group, by the way,
that has never been given its comedic due.

jenna wortham

Well, they're all practicing comedians. These are all women who have
been working through storytelling in very different ways over a number
of years, and there's just this really divine timing of all of these
timelines converging at the same time. You have all these entities
touching on the neglect of the condition of Black women around the
world, from all these different intersections. And the way it's
surfacing right now, to tell this collective, untold really gruesome
story about even while we're in a global reckoning around racism, that
misogyny still exists. I mean, it's really fascinating. But like
anything else worth reckoning with in this country, it's in plain sight.
You just have to be willing to look. No one's going to spell it out for
you.

wesley morris

I think that is one of the things about culture in this moment is that
we've been doing it wrong all this time. We've been looking for biopics
and stories about American heroes.

jenna wortham

Right.

wesley morris

We have never really focused on actual Black life in this country.

jenna wortham

Oof. Go ahead.

wesley morris

Never really focused on it. It's always people you would put on a
pedestal, which is great. But I'm more interested in the people who
would go stand in front of the pedestals and go look at these Harriet
Tubmans, or Sojourner Truths, and Martin Luther Kings, and John Lewises.
That's what I want to see. We just had a conversation last week about
the exploitation of Aunt Jemima, her creation, proliferation, none of
the women who played her had any control over how her image was used,
how their images were used in the world. And this, to me, seems like a
new frontier, where these women have been the face of a particular kind
of either entertainment, or comfort, or servitude, are now not
interested in serving anybody but themselves.

jenna wortham

Yes.

wesley morris

That's never happened before in mass entertainment. I mean, in
literature there's a long history of it, but never anywhere else, with
respect to TV and movies. And I can't believe that in this moment right
now, when we are talking about Black life mattering and statues, that we
have very little examples of Black life as it is actually lived by
particularly Black women, you have these shows that are really
interested in the psychologies and states of mind of Black women. And
the thing that I'm realizing in this moment is that the harder work, the
more uncomfortable work to be done is to start looking at the discomfort
in your own home, not the people who live next door. And I feel like
that's what's happening with ``Hamilton,'' with this reexamination of
what's happening with ``Hamilton'' now versus what was happening in 2015
and `16. Like, we were happy to have a nice house in the neighborhood.
I'm not going to sit here and interrogate that. But thinking about what
is happening in your own house for Michaela Coel and to some extent with
Ziwe, there is this interest in and a willingness to confront the idea
that our problems are not entirely to do with white people. Sometimes
they're just human problems that have to do with us, each other. And
we've been focused on the neighbors, and I've been finding it really,
really thrilling to watch art by people who are focused on the house,
their own house.

jenna wortham

Mm.

wesley morris

And the power of that work is that it is not about making anybody feel
good or feel comfortable. It's about destabilizing us enough so that we
can sit with what we couldn't previously see, or what we had denied, to
sit with the actual discomfort of if not our behavior, then the behavior
around us, our choices around us that we just sort of --- just kind of
ignored. It's destabilizing. And I appreciate, admire, and would love
more of that destabilization. Because it's about changing things and
shaking them up. And so I don't know, like where --- what do we do with
this discomfort, right? We can't sit in it forever the way the white
fragilicist would want us to.

jenna wortham

OK, but that white fragilicist, which I love that phrase, aside, why
not? These are issues that we have spent centuries mired in, and we've
only spent about half that time with the freedom to actually start to
process it or do anything with it. I mean, it is not an understatement
to say that most people right now are really concerned with the basic
tenants of survival. And so even having the luxury to start to grapple
with what is so uncomfortable in these spaces, and within our own
communities and the way we focus on some things and avert our eyes from
others, like the scene you point out in ``I May Destroy You,'' I mean,
these are not easy issues. And the process to face them is not going to
be easy, either. My therapist is really an advocate of trying to think
through why some things feel intolerable, why some things feel
insurmountable. And when you avoid something, that is a huge red flag
that you need to pay attention to it. And so I think being able to
tolerate the discomfort, even noticing it is actually the first step to
changing it. I like the discomfort. I mean, I really enjoyed the
discomfort of 2 and 1/2 hours of ``Hamilton.'' OK, I fast forwarded
through some of it, I'm not going to lie. But I enjoyed really tangling
with why I felt so uncomfortable. And the reason it was exciting is
because I had the tools, and I had the language to really talk to
myself, I guess, about it, and then come and talk to you about it.
Because I want to think about why I feel uncomfortable when a group of
young Black kids are ignoring someone getting beaten up behind them,
even as they're trying exonerate someone else. Why do I feel weird when
I'm watching Ziwe basically roast white people over the internet's fire?
That's the work. That is actually the work that ends up getting us
somewhere different.

{[}music - kindness, world restart{]}

wesley morris

That's our show. And the reason I sound sad to say that is because we're
taking time. We're taking some time. We don't really know how long. It
won't be like before, though, where it's like we're gone for a year and
nobody knows where we went. Anyway, as usual, ``Still Processing'' is a
product of The New York Times.

jenna wortham

It was recorded safely and remotely in our living rooms.

wesley morris

I don't know about safely, but definitely remotely. (LAUGHS)

jenna wortham

What is going on today over there?

wesley morris

I mean, I had to turn the A.C. off. Hans Buetow as always is our
producer.

jenna wortham

Our editors are Sarah Sarasohn, Sasha Weiss, Wendy Dorr, and Lisa Tobin.

wesley morris

Our engineer is Jake Gorski.

jenna wortham

Our theme music is by Kindness. It is called ``World Restart'' from the
album ``Otherness.''

wesley morris

And you can find all our other old episodes and everything else we've
ever done in our whole lives, except --- just old episodes, at
NYTimes.com/stillprocessing.

jenna wortham

See you soon. Bye.

wesley morris

But not soon enough, you guys, though really, honestly. I'm sad, Jenna.

jenna wortham

It's OK, baby. Come over and barbecue. It'll be all right. Bring your
mask.

wesley morris

Oh, yeah. I'll bring a mask and barbecue. That's right.

\href{https://www.nytimes3xbfgragh.onion/column/still-processing-podcast}{\includegraphics{https://static01.graylady3jvrrxbe.onion/images/2019/09/15/podcasts/still-processing-album-art-2/still-processing-album-art-2-square320.jpg}Still
Processing}Subscribe:

\begin{itemize}
\tightlist
\item
  \href{https://itunes.apple.com/us/podcast/id1151436460}{Apple
  Podcasts}
\item
  \href{https://www.google.com/podcasts?feed=aHR0cHM6Ly9yc3MuYXJ0MTkuY29tL255dC1zdGlsbC1wcm9jZXNzaW5n}{Google
  Podcasts}
\end{itemize}

\hypertarget{ziwe-may-destroy-hamilton-1}{%
\section{Ziwe May Destroy Hamilton}\label{ziwe-may-destroy-hamilton-1}}

\hypertarget{welcome-to-the-age-of-discomfort-1}{%
\subsection{Welcome to the Age of
Discomfort}\label{welcome-to-the-age-of-discomfort-1}}

Hosted by Wesley Morris and Jenna Wortham. Produced by Hans Buetow.

Transcript

transcript

Back to Still Processing

bars

0:00/38:10

-0:00

transcript

\hypertarget{ziwe-may-destroy-hamilton-2}{%
\subsection{Ziwe May Destroy
Hamilton}\label{ziwe-may-destroy-hamilton-2}}

\hypertarget{hosted-by-wesley-morris-and-jenna-wortham-produced-by-hans-buetow-1}{%
\subsubsection{Hosted by Wesley Morris and Jenna Wortham. Produced by
Hans
Buetow.}\label{hosted-by-wesley-morris-and-jenna-wortham-produced-by-hans-buetow-1}}

\hypertarget{welcome-to-the-age-of-discomfort-2}{%
\paragraph{Welcome to the Age of
Discomfort}\label{welcome-to-the-age-of-discomfort-2}}

Thursday, July 23rd, 2020

\begin{itemize}
\item
  wesley morris\\
  We are rolling.
\item
  jenna wortham\\
  I'm so ready. Yo, put me in the cabinet battle. I'm ready. We got
  things to discuss.
\item
  {[}music{]}
\item
  wesley morris\\
  Well, we should talk about how you were blowing up my phone last
  night.
\item
  jenna wortham\\
  OK, I'm starting this thing later than I planned because I watched the
  sunset, and I made dinner, and I'm already a little bit embarrassed.
  Who is this Negro in colonial get-up?
\item
  archived recording (leslie odom jr)\\
  I'm the damn fool that shot him.
\end{itemize}

{[}music - kindness, world restart{]}

wesley morris

I'm Wesley Morris.

jenna wortham

And I'm Jenna Wortham. We are two culture writers for The New York
Times, and I decided to watch ``Hamilton,'' of course. I've successfully
blocked out so much around ``Hamilton,'' there's a Hamilton-shaped hole
in my brain, and I'm fine with it. But Disney+ put the production on its
streaming service a few weeks ago. I felt like the internet was talking
about it. I didn't know exactly what people were referencing. I had to
take the bait, and that's what we did.

wesley morris

So I mean, let's just talk about those texts you sent me and what you
saw.

jenna wortham

OK.

{[}music{]}

\begin{itemize}
\tightlist
\item
  archived recording (chorus)\\
  1, 2, 3, 4, 5, 6, 7, 8, 9.
\end{itemize}

wesley morris

Let's go. Text number one, ``Wait --- ''

jenna wortham

Oh, god.

wesley morris

'' --- why are people clapping so much for George III?'' Text number
two, ``How did anyone watch this without a phone to Google shit?''

jenna wortham

Who is George III? I'm like, who ---

wesley morris

(LAGUHS)

jenna wortham

--- who dis? I was like, why are y'all so excited for this man? Who is
this man? What has he done or not done to us? I really was confused, and
so I had to pause. I had to look some things up.

wesley morris

(LAUGHS) Text number three, ``June 2016, dot, dot, dot, those innocent
creatures.''

jenna wortham

It was a different time. It was a different time,

wesley morris

Text number five is just a parenthetical that is obviously a lie. ''(I
won't do this all night. I promise. But I have to a little.)'' Lies,
damn lies.

Number 12. All caps, ``NO, OH, WHITE GLOVES.''

jenna wortham

Wesley, you're forgetting the voice notes that I --- I had to stop
typing and start sending you voice notes, because I was losing my
ability to type clearly.

\begin{itemize}
\tightlist
\item
  jenna wortham\\
  Sir, wait, this Thomas Jefferson dance, why didn't we see this
  together in person? I would have died. I can't believe you went to see
  this without me. I don't forgive you. I don't forgive you.
\end{itemize}

jenna wortham

I feel that so hard, I put that in our Slack this morning. I will never
forgive you for seeing this without me! And then I got self-conscious
about sending you so many messages, so I --- I reeled it in.

wesley morris

You don't ever have to keep yourself self-conscious about sending
messages to me. Anyway, white gloves.

\begin{itemize}
\tightlist
\item
  archived recording (daveed diggs)\\
  So what did I miss?
\end{itemize}

jenna wortham

Yo, my G, it's not just that Thomas Jefferson comes out and starts
shuckin' and jivin', played by Daveed Diggs, who is so incandescent in
this role, like lit from within.

\begin{itemize}
\tightlist
\item
  archived recording (daveed diggs)\\
  I guess I basically missed the late `80s. I traveled the wide, wide
  world and came back to this.
\end{itemize}

jenna wortham

But Thomas Jefferson's back from France talking about, what did I miss
out here, doing this little kicky dance! And then in the background, all
the dancers, who are Black and beautiful, have on white gloves. I was
like, Wesley, we already saw ``Get Out.'' We saw ``Us'' together. What
horror show is this? I was like, oh my god, why weren't we in this
together? My hands would have been digging into your thighs, and I would
have been crying with tears. Like, what out of body transportation
experience is this? And everybody in the audience is going wild.

\begin{itemize}
\tightlist
\item
  archived recording (daveed diggs)\\
  What did I miss?
\end{itemize}

wesley morris

The first time I saw it, it was just mayhem in the audience.

jenna wortham

My god, I bet.

wesley morris

It was like the late summer of 2015. So I don't know how long --- the
show had barely been opened.

jenna wortham

It just had --- yeah, it had just opened.

wesley morris

And it was like --- I mean, I don't even know what to say. It was just
electric. It was like after every song, people were losing their minds.
I think in 2015, the thing that was so exciting about ``Hamilton'' was
its mere existence. I mean, it wasn't just white people who had taken
out their bowls and spoons and were ready to eat some, you know,
non-white people on Broadway. It was young, Black, and Latino people,
who were just dying to maybe be able to see this thing appear on a
Broadway stage. You definitely felt that something rare was happening.
There was a real belief, and it was symbolized by the Obamas and Marian
Robinson being in the White House together. And not just that, but you
had Eric Holder.

jenna wortham

Right.

wesley morris

This was about non-white people being in the government, right?

jenna wortham

Yes, yes, I understand. I can see that.

wesley morris

There is a real mirror between the presidential administration of 2015
and the goings-on in this new country being formed at the end of the
18th century.

jenna wortham

But, Wesley, I'm really curious to hear your perspective, too. Because
in my recollection, there were some academics who were coming out, and
there was definitely a little bit of discomfort around the way
``Hamilton'' was treating history. And what does it mean to put Black
people into colonial get-up and also say, what does it mean to have
Thomas Jefferson played by a Black person, who's talking around slavery
and not really about slavery, but all this wealth that it creates within
the play? What's your sense?

wesley morris

Well, I mean, here's what I would say. I'd say at the time, criticisms
began, and where they were most robust in 2015, were entirely --- almost
entirely around the question of the depiction of the history. And what
the show was responsible for perpetuating or not addressing with respect
to Alexander ``Hamilton,'' and Thomas Jefferson, and Hercules Mulligan,
and Aaron Burr, and George Washington, and, and, and, and, and. These
were actual legitimate criticisms of Ron Chernow's book, which came out
in 2004, and is the source material that Lin-Manuel Miranda used when he
was writing ``Hamilton.'' But those criticisms were very apparent during
the original run of the show. The thing was that there were a lot of
people who were like, shhhh ---

jenna wortham

Yes, yes.

wesley morris

--- just shh, shh, shh!

jenna wortham

I remember this.

wesley morris

``Stop it, stop it! We've got this nice thing, and you cannot take it
from us. Just stop it.''

jenna wortham

Right.

wesley morris

``He just got to Broadway, y'all. Just let him win his Tonys, and then
we will have this fight.''

jenna wortham

Have those conversations over there.

wesley morris

Yeah. But even in 2015, it was a little uncomfortable, sitting there
watching, knowing what you were watching in some ways, that you had
these white men being played by these non-white people.

jenna wortham

And frankly, ``Hamilton'' really pushes you to grapple with the idea
that the white gaze ain't always the only gaze that can be problematic,
right? Like, Black people, brown people, other kinds of people can have
problematic gazes. But that's not a conversation we were comfortable
having I would say five years ago. It's a conversation we struggle with.
We still struggle with it.

wesley morris

Still struggling.

jenna wortham

Still struggling. I mean, a big part of that is what you were alluding
to earlier, right? Like, nobody wants to be the person coming after all
these successful Black and brown men. And it gets really complicated to
talk about how Black people can be anti-Black as well. Hello, Kanye?
We're still starting to have that conversation. It's not easy.

wesley morris

Yeah, yeah.

jenna wortham

Right?

wesley morris

Yeah, yeah, yeah.

jenna wortham

And that's something that we really --- we really don't like talking
about.

wesley morris

Yup.

jenna wortham

At all.

wesley morris

Nope.

jenna wortham

Anyone who's not, quote, the majority, so rarely gets these
opportunities. Like, it doesn't feel right to poke holes in it or pull
the rug underneath them. But that's where we're at also right now.

wesley morris

So I think that what you're identifying, though, is a major shift,
right?

jenna wortham

Mm-hmm.

wesley morris

There has been a shift in the culture between 2015 ---

jenna wortham

Yes.

wesley morris

--- and 2020. In 2015, our cultural priorities were really about being
seen. ``I can't believe that this person looks like me. I am so happy to
see a person who looks like me. Looks like me, looks like me, looks like
me.''

jenna wortham

Right, right, right.

wesley morris

I think during the Obama era, it was about being seen. And I think in
the Trump era, it is about being heard.

jenna wortham

Mm-hmm.

wesley morris

And in the same way that it was right on time in 2015, it is right on
time for 2020.

jenna wortham

I think the conversation in 2020 looks so different because the stakes
feel lower. ``Hamilton's'' already had its success. There's no way to
dim the star shine of ``Hamilton'' anymore. It had its run. People were
tweeting things like, it's problematic to glorify Hamilton, Washington,
Jefferson, and Madison. People were tweeting, there is a total erasure
of slavery in this show. Given every single thing that's happening right
now, are you really going to not say what you really think about
``Hamilton?'' Is that where you draw the line? No, all bets are off. And
so I think ``Hamilton'' bears the brunt of that. And rightfully so.

wesley morris

These aren't just any old white men, right? These are men whose statues
are in the middle of being defaced, torn down, debated about whether
they should have ever gone up in the first place. Like, what are we
honoring when we honor these men? And here is a show that essentially is
being peopled by the people who oppressed them, more or less. And I
think not only do we get to watch and celebrate this thing that on many
levels is fun, I think we also have to sit in the discomfort, if you
were a person who championed this thing five years ago, what it means to
sit in the discomfort now.

jenna wortham

Yes, yes.

wesley morris

And the discomfort of being made to see something that you just couldn't
see before, not necessarily because you didn't want to see it, but
because the time wasn't right for you to see it, or no one had
sufficiently forced you to look at what you were enjoying so much.

jenna wortham

Mm-hmm.

wesley morris

But if we're talking about reckonings and topplings, I think there is a
deeper part of that which entails us as people who listen, and read, and
watch, and visit, and attend, and write to think about our relationship
to some of these problems, too. I mean, I'm down for discomfort. I'm
down to not feel stable. Because we shouldn't be right now. That's not
what this moment is actually about.

{[}music{]}

wesley morris

Jenna.

jenna wortham

Yes.

wesley morris

I'm an American. I live with discomfort all the time.

jenna wortham

Mm.

wesley morris

I'm aware, for instance, that when I wake up in the morning, I have
awoken on somebody else's land.

jenna wortham

Yes, in fact, the land of the Canarsie and the Munsee Lenape, which I
love this app called Native Land. You can plug into any address anywhere
in the world, and it will tell you who the original stewards of the land
are, and it really has been opening my eyes to just the hidden
indigeneity that's all around us.

wesley morris

That is a great app.

jenna wortham

Mm-hmm.

wesley morris

And so I'm also aware that I am a traceable descendant of enslaved
people. And that leads into a pretty natural awareness of all kinds of
other things. Like, who is making my food? Who is growing the food? Who
is slaughtering the animals that I'm going to eat as a non-vegeterian,
you know? I just go through life assuming that there is discomfort out
there, and that it's built into all aspects of living in the 21st
century. And I'm comfortable with that discomfort. And not the sort of
discomfort that a person like Robin DiAngelo is trying to get people to
feel. Robin DiAngelo is, of course, the author of the best selling again
book, ``White Fragility.'' I just find that project nauseating. It's
like she is telling white people, ``Don't talk, shut up. Things are too
racist for you to speak, and you are the source of the racism. And you
can't say anything about racism, either, because Black people will not
like you. They will not be able to handle it.'' I don't like her
pitching white people into that place of discomfort, because what she is
offering as discomfort for her white readership is really work for me
and you.

jenna wortham

Absolutely.

wesley morris

Right? Also, I'm not getting a paycheck to do that work that she's
setting me up to do. I don't know about you.

jenna wortham

I mean, it's one thing to sit in a realm of discomfort. Like a lot of my
talk therapy practice is rooted in this idea of what we do and don't
face, and then the childhood roots behind that. And so there's a lot of
discussion of sitting with the discomfort and looking at it, and being
willing to tangle with it, and then try to root out what are the
feelings, what are they masking, why do I feel so uncomfortable, and
then what am I going to do about it? There's a type of, I think, passive
discomfort that you're talking about versus something that is actually
quite an active, emotional, psychological, intellectual state.

wesley morris

I'm here for a little discomfort. I'm used to feeling uncomfortable. I'm
used to having discomfort induced in me.

jenna wortham

Yes.

wesley morris

And that is part of the pleasure of watching this show ``Baited'' that
you got me to watch, this show by Ziwe, Ziwe Fumudoh is the comedian who
goes by the first name Ziwe on the show, which is now an Instagram Live
experience between her and her white guests.

\begin{itemize}
\item
  archived recording (ziwe fumudoh)\\
  Hi.
\item
  archived recording (woman)\\
  Hi.
\item
  archived recording (ziwe fumudoh)\\
  How are you doing?
\item
  archived recording (woman)\\
  Honestly, I'm really nervous.
\item
  archived recording (ziwe fumudoh)\\
  You're really nervous?
\item
  archived recording (woman)\\
  Yeah.
\item
  archived recording (ziwe fumudoh)\\
  On a scale of Black person minding their business in their house to
  Black person minding their business at the movie theater, how nervous
  are you?
\item
  archived recording (woman)\\
  Like a zero. Like, cannot compute that. So it's not registering on
  that scale for sure.
\end{itemize}

jenna wortham

It's worth noting that this project of Ziwe's has been going on for some
time. She had this show on YouTube, and then it transitioned to
Instagram Live. And historically, the types of people she brings on the
show are friends, other stand-up comedians, colleagues, who I think she
feels haven't really thought out their relationship to their own
whiteness, or their relationship to identity, and the various
intersections of their identity. There's a really hilarious convo, which
you can put in the show notes. She interviews a white cis gay male
colleague of hers, who keeps insisting that because he's gay he
understands what it means to be marginalized. More recently on her
Instagram Lives, Ziwe's been inviting white people onto the show who
have had some sort of embarrassingly public moment dealing with race. So
not just her immediate circle, but people who've so-called been canceled
via Twitter. So the, I guess author, social media scam artist Caroline
Calloway, and then Alison Roman, who is a chef, and a cookbook author,
and a columnist for The New York Times. And she just wants to get white
people who've had some sort of social media kerfuffle on the show. But
it seems like the kinds of people who say yes are the kinds of people
who feel like going on that show is doing the work.

\begin{itemize}
\item
  archived recording (caroline calloway)\\
  This is a really stressful time for Black people, and I'm really glad
  that you can have this emotional rest by having me on the show.
\item
  archived recording (ziwe fumudoh)\\
  Totally. And when you say Black people, do you capitalize the B?
\item
  archived recording (caroline calloway)\\
  Absolutely, because otherwise it's a color.
\item
  archived recording (ziwe fumudoh)\\
  Performative. Now, first question for you Caroline is, I saw --- I
  watched your Cambridge Union interview today, and you said that,
  famously, you discovered racism in 2018. What were you doing for the
  first 25 years of your life?
\end{itemize}

jenna wortham

I don't think Ziwe's calling anybody a racist, but I think she's
acknowledging that if you are a person, especially a white person and
you grew up in America, it is very difficult to unlearn the ways in
which you have been prioritized, and privileged, and given advantages,
and taught to think about your relationship to every other person in
this country, you know? And I think that's what she's trying to tease
out in some of those similarities in the way that her white friends, and
then the white people she encounters operate.

\begin{itemize}
\item
  archived recording (ziwe fumudoh)\\
  How many Black friends do you have, Alison Roman?
\item
  archived recording (alison roman)\\
  Do you define friend like someone who would pick me up from the
  airport, or like people I follow or people that I know? Because I have
  like, I would say, four to five Black friends that would pick me up at
  the airport.
\item
  archived recording (ziwe fumudoh)\\
  Four to five? You are the third person to say they have four to five
  Black friends in the last week to me. Caroline Callaway, Nick
  Ciarelli, and yourself.
\item
  archived recording (alison roman)\\
  Oh.
\item
  archived recording (ziwe fumudoh)\\
  That's an interesting statistic.
\item
  archived recording (alison roman)\\
  I actually didn't see that part of them. I mean, yeah, in the grand
  total of actual friends I have that would pick me up at the airport is
  probably like 12 to 13.
\end{itemize}

jenna wortham

The joke becomes in the repetition that no one bothers to do their
research. When you start to watch all the ways in which these white
women mostly think they're doing themselves a favor, right, by going on
Ziwe's show. But then they're all being --- they've done no research.
They're just popping in to say hey, and are totally unprepared and
totally disheveled. And just imagine if you or I showed up for any kind
of interview, or went on Anderson Cooper, and didn't know who he was. Or
imagine going on Jimmy Fallon, and thinking it was Jimmy Kimmel, right?
It just starts to become in itself this larger arc of the way whiteness
is allowed to function. And when you start to interrogate the larger
questions about why these women continue to treat Ziwe this way, it's
sinister. It's ``Get Out'' Part 3.

wesley morris

And these aren't like an obviously racist cop or Jeff Sessions. This is
the racism that regular people feel on a daily basis, right?

jenna wortham

Yes, yes.

wesley morris

This is the racism of social intimacy or social acquaintance, and how do
those people who practice that kind of ambient racism, institutional
racism, who benefit from it, how do they respond when asked, how many
Black authors have you read? How many Black people do you know?

\begin{itemize}
\item
  archived recording (ziwe fumudoh)\\
  Now, what do you qualitatively like about Black people?
\item
  archived recording (alison roman)\\
  Qualitatively that ---

  I mean, I can say what I like qualitatively about my Black friends, or
  you mean generally speaking?
\item
  archived recording (ziwe fumudoh)\\
  However you interpret the question.
\item
  archived recording (alison roman)\\
  Qualitatively, I love that their food almost always tastes better than
  mine.
\item
  archived recording (ziwe fumudoh)\\
  OK, we're gonna stop you right there.
\item
  archived recording (alison roman)\\
  They're way better dancers.
\item
  archived recording (ziwe fumudoh)\\
  I'll stop you right there.
\item
  archived recording (alison roman)\\
  OK.
\item
  archived recording (ziwe fumudoh)\\
  We're gonna stop you --- I'm doing you a favor (LAUGHS). OK?
\end{itemize}

wesley morris

But I would love --- and part of the reason that I'm interested in this
project is because I would like to see it evolve into an actual
discourse and not a performance, right?

jenna wortham

Yes, absolutely. I think at the beginning of the Instagram Lives, what
was entertaining was, it almost started to feel like an exploration of a
kink, of like a humiliation, submissive, kind of B.D.S.M. role play.

wesley morris

Welcome to racism.

jenna wortham

Welcome to racism, right? It was really fascinating. I do think that
what's fascinating about the ``Baited'' series is that even though white
people are the subject, they are completely not the center. They're not
centered as a subject. It's like the performance of it is to allow her
Black viewers, and I would say mostly Black women who are watching the
show, to really feel validated in some ways about all the
microaggressions, and the gaslighting, and the really uncomfortable
experiences that you've had. You have this duplicitousness of people who
are openly saying, yes, Black Lives Matter, support Black chefs, support
Black authors, but then can't name them when they're asked. That's real.
You're playing with this outward performance and how that performance
shows up in your personal politics. That's really interesting.

\begin{itemize}
\item
  archived recording (ziwe fumudoh)\\
  Now, I saw on your Instagram that you are promoting Black authors like
  Wesley Lowery, who wrote ``They Can't Kill Us,'' and ``The New Jim
  Crow'' by Michelle Alexander.
\item
  archived recording (caroline calloway)\\
  Austin Channing. Let's fucking go!
\item
  archived recording (ziwe fumudoh)\\
  Totally.
\item
  archived recording (caroline calloway)\\
  Layla Safad (sic). Let's go!
\item
  archived recording (ziwe fumudoh)\\
  Exactly. Now, you're a vociferous reader. How many of these books have
  you read?
\item
  archived recording (caroline calloway)\\
  Honestly, of the nine books that I recommended on my Instagram, I've
  read four.
\item
  archived recording (ziwe fumudoh)\\
  Wow.
\item
  archived recording (caroline calloway)\\
  But I've ordered the other five from Black book shops, so I would like
  my ally cookie now.
\item
  archived recording (ziwe fumudoh)\\
  There are no cookies in this game.
\end{itemize}

jenna wortham

I think there's something really profound about it happening on
Instagram as well, which is a space where so much of those performances
are also happening right now, and this feeling of fatigue around, stop
telling me my life matters, and tell me what you're doing about it, and
what are we actually going to do about it besides talk about it? How
does what you share online show up in your offline life? And I think
there is this kind of ---

wesley morris

That's ---

jenna wortham

It's very conscious, and very in the air, and it's also very
subconscious, too, because my discomfort in watching the show is not
about my relationship to race and to whiteness in the way of the guests,
right? But it's about, what does it mean for me to be participating in
these types of evisceration. That's what makes me uncomfortable. And I
really value that. I really value that. Because I'm not someone who
loves a pile-on. That's not my Twitter M.O. or Instagram M.O., and yet
here I am, every Thursday at 8.

wesley morris

Yeah, yeah, yeah, yeah. So, Jenna, we're talking about another form of
discomfort that's different from the ``Hamilton'' discomfort of
realizing that a thing that you love has a lot of problems in it.

jenna wortham

Mm-hmm.

wesley morris

I mean, we're talking about Ziwe Fumudoh's project, but I'm also
thinking about ``I May Destroy You,'' this show that H.B.O. is putting
on from England. And it's been conceived by and stars Michaela Coel, who
is an indescribably chameleonic performer, and a very astute observer of
human relationships and micro-everything.

jenna wortham

Yes.

wesley morris

Like micro-reactions, micro-emotions, micro-shifts in interpersonal
dynamics, the micro-shifts in time, and space, and memory. She's so
smart about these small things that escape the notice of so many
creators. I mean, this show, which is essentially about what happens to
this woman who Michaela Coel is playing, her name is Arabella. And she
has been sexually assaulted more than one time. And you are trying to
figure out what happened to her.

\begin{itemize}
\item
  archived recording (woman)\\
  And the assault you recall?
\item
  archived recording (michaela coel)\\
  The thing in my head?
\item
  archived recording (woman)\\
  Yes.
\item
  archived recording (michaela coel)\\
  Yeah, I wouldn't --- because now you're calling it something that I
  never --- I never said that.
\item
  archived recording (woman)\\
  Do you see anyone else?
\item
  archived recording (michaela coel)\\
  Where?
\item
  archived recording (woman)\\
  In this memory.
\item
  archived recording (michaela coel)\\
  You can't call it a memory.
\item
  archived recording (woman)\\
  OK, other than the man in the ---
\item
  archived recording (michaela coel)\\
  In my head. He may not even be real, because I'm the person that can
  actually see it, and I'm not sure, so I should probably pay attention
  to that.
\end{itemize}

wesley morris

And one of my favorite things about this show is that it manages to
feature discomfort working simultaneously on several different levels.

jenna wortham

Mm-hmm.

wesley morris

I'm thinking specifically about episode 6, which is essentially a
flashback episode. And all the main characters are younger versions of
themselves, and they're being played by different actors, kids. And they
are in the middle of sleuthing an alleged sexual assault that's occurred
among them. And they're standing in the courtyard, the playground,
having this conversation about what they think happened or will happen
to one of them, while in the background there is a girl being beaten up.
She's just being rolled on by these other kids, and none of these
characters in the foreground, these characters that we have been
watching for five previous episodes as adults, are noticing anything
happening behind them. They do not hear, notice, see anything happening
to this girl who's on the ground.

jenna wortham

Who's also Black.

wesley morris

Getting the --- who's also Black, getting the life kicked out of her.
And this search for justice in this one situation, meanwhile there is
actually something ---

jenna wortham

Right, right.

wesley morris

--- demonstrably something happening behind these kids, and they don't
care about it at all. They don't even see it.

jenna wortham

Yeah, the way that scene is working is, obviously it's a metaphor, but
it's a metaphor for the state of Black women in general all around the
world. We are at a moment where we're witnessing collectively the way
the world will rally around the murder of George Floyd but not Breonna
Taylor, right? The way the world pays attention to what happens to cis
Black folks and not Black trans folks. I mean, there's all of this
nuance that feels inconvenient and difficult to discuss, that that
episode is perfectly encapsulating by what's happening in the foreground
and what's happening in the background when it comes to side by side
assaults, and side by side attention, and side by side caretaking.

wesley morris

It's just interesting that these things are coming from --- are being
told by starring, driven by Black women. And like a group, by the way,
that has never been given its comedic due.

jenna wortham

Well, they're all practicing comedians. These are all women who have
been working through storytelling in very different ways over a number
of years, and there's just this really divine timing of all of these
timelines converging at the same time. You have all these entities
touching on the neglect of the condition of Black women around the
world, from all these different intersections. And the way it's
surfacing right now, to tell this collective, untold really gruesome
story about even while we're in a global reckoning around racism, that
misogyny still exists. I mean, it's really fascinating. But like
anything else worth reckoning with in this country, it's in plain sight.
You just have to be willing to look. No one's going to spell it out for
you.

wesley morris

I think that is one of the things about culture in this moment is that
we've been doing it wrong all this time. We've been looking for biopics
and stories about American heroes.

jenna wortham

Right.

wesley morris

We have never really focused on actual Black life in this country.

jenna wortham

Oof. Go ahead.

wesley morris

Never really focused on it. It's always people you would put on a
pedestal, which is great. But I'm more interested in the people who
would go stand in front of the pedestals and go look at these Harriet
Tubmans, or Sojourner Truths, and Martin Luther Kings, and John Lewises.
That's what I want to see. We just had a conversation last week about
the exploitation of Aunt Jemima, her creation, proliferation, none of
the women who played her had any control over how her image was used,
how their images were used in the world. And this, to me, seems like a
new frontier, where these women have been the face of a particular kind
of either entertainment, or comfort, or servitude, are now not
interested in serving anybody but themselves.

jenna wortham

Yes.

wesley morris

That's never happened before in mass entertainment. I mean, in
literature there's a long history of it, but never anywhere else, with
respect to TV and movies. And I can't believe that in this moment right
now, when we are talking about Black life mattering and statues, that we
have very little examples of Black life as it is actually lived by
particularly Black women, you have these shows that are really
interested in the psychologies and states of mind of Black women. And
the thing that I'm realizing in this moment is that the harder work, the
more uncomfortable work to be done is to start looking at the discomfort
in your own home, not the people who live next door. And I feel like
that's what's happening with ``Hamilton,'' with this reexamination of
what's happening with ``Hamilton'' now versus what was happening in 2015
and `16. Like, we were happy to have a nice house in the neighborhood.
I'm not going to sit here and interrogate that. But thinking about what
is happening in your own house for Michaela Coel and to some extent with
Ziwe, there is this interest in and a willingness to confront the idea
that our problems are not entirely to do with white people. Sometimes
they're just human problems that have to do with us, each other. And
we've been focused on the neighbors, and I've been finding it really,
really thrilling to watch art by people who are focused on the house,
their own house.

jenna wortham

Mm.

wesley morris

And the power of that work is that it is not about making anybody feel
good or feel comfortable. It's about destabilizing us enough so that we
can sit with what we couldn't previously see, or what we had denied, to
sit with the actual discomfort of if not our behavior, then the behavior
around us, our choices around us that we just sort of --- just kind of
ignored. It's destabilizing. And I appreciate, admire, and would love
more of that destabilization. Because it's about changing things and
shaking them up. And so I don't know, like where --- what do we do with
this discomfort, right? We can't sit in it forever the way the white
fragilicist would want us to.

jenna wortham

OK, but that white fragilicist, which I love that phrase, aside, why
not? These are issues that we have spent centuries mired in, and we've
only spent about half that time with the freedom to actually start to
process it or do anything with it. I mean, it is not an understatement
to say that most people right now are really concerned with the basic
tenants of survival. And so even having the luxury to start to grapple
with what is so uncomfortable in these spaces, and within our own
communities and the way we focus on some things and avert our eyes from
others, like the scene you point out in ``I May Destroy You,'' I mean,
these are not easy issues. And the process to face them is not going to
be easy, either. My therapist is really an advocate of trying to think
through why some things feel intolerable, why some things feel
insurmountable. And when you avoid something, that is a huge red flag
that you need to pay attention to it. And so I think being able to
tolerate the discomfort, even noticing it is actually the first step to
changing it. I like the discomfort. I mean, I really enjoyed the
discomfort of 2 and 1/2 hours of ``Hamilton.'' OK, I fast forwarded
through some of it, I'm not going to lie. But I enjoyed really tangling
with why I felt so uncomfortable. And the reason it was exciting is
because I had the tools, and I had the language to really talk to
myself, I guess, about it, and then come and talk to you about it.
Because I want to think about why I feel uncomfortable when a group of
young Black kids are ignoring someone getting beaten up behind them,
even as they're trying exonerate someone else. Why do I feel weird when
I'm watching Ziwe basically roast white people over the internet's fire?
That's the work. That is actually the work that ends up getting us
somewhere different.

{[}music - kindness, world restart{]}

wesley morris

That's our show. And the reason I sound sad to say that is because we're
taking time. We're taking some time. We don't really know how long. It
won't be like before, though, where it's like we're gone for a year and
nobody knows where we went. Anyway, as usual, ``Still Processing'' is a
product of The New York Times.

jenna wortham

It was recorded safely and remotely in our living rooms.

wesley morris

I don't know about safely, but definitely remotely. (LAUGHS)

jenna wortham

What is going on today over there?

wesley morris

I mean, I had to turn the A.C. off. Hans Buetow as always is our
producer.

jenna wortham

Our editors are Sarah Sarasohn, Sasha Weiss, Wendy Dorr, and Lisa Tobin.

wesley morris

Our engineer is Jake Gorski.

jenna wortham

Our theme music is by Kindness. It is called ``World Restart'' from the
album ``Otherness.''

wesley morris

And you can find all our other old episodes and everything else we've
ever done in our whole lives, except --- just old episodes, at
NYTimes.com/stillprocessing.

jenna wortham

See you soon. Bye.

wesley morris

But not soon enough, you guys, though really, honestly. I'm sad, Jenna.

jenna wortham

It's OK, baby. Come over and barbecue. It'll be all right. Bring your
mask.

wesley morris

Oh, yeah. I'll bring a mask and barbecue. That's right.

Previous

More episodes ofStill Processing

\href{https://www.nytimes3xbfgragh.onion/2020/07/23/podcasts/hamilton-ziwe-discomfort.html?action=click\&module=audio-series-bar\&region=header\&pgtype=Article}{\includegraphics{https://static01.graylady3jvrrxbe.onion/images/2020/07/23/multimedia/23stillprocessing-pix/23stillprocessing-pix-thumbLarge.jpg}}

July 23, 2020~~•~ 38:10Ziwe May Destroy Hamilton

\href{https://www.nytimes3xbfgragh.onion/2020/07/16/podcasts/reparations-for-aunt-jemima.html?action=click\&module=audio-series-bar\&region=header\&pgtype=Article}{\includegraphics{https://static01.graylady3jvrrxbe.onion/images/2020/07/18/multimedia/16stillprocessing-pix/16stillprocessing-pix-thumbLarge.jpg}}

July 16, 2020~~•~ 35:35Reparations for Aunt Jemima!

\href{https://www.nytimes3xbfgragh.onion/2020/07/09/podcasts/still-processing-black-lives-matter.html?action=click\&module=audio-series-bar\&region=header\&pgtype=Article}{\includegraphics{https://static01.graylady3jvrrxbe.onion/images/2020/07/12/podcasts/09stillprocessing-image/xx-stillprocessing-thumbLarge.jpg}}

July 9, 2020~~•~ 26:29So Y'all Finally Get It

\href{https://www.nytimes3xbfgragh.onion/2020/05/14/podcasts/still-processing-westworld-hollywood-utopia-dystopia.html?action=click\&module=audio-series-bar\&region=header\&pgtype=Article}{\includegraphics{https://static01.graylady3jvrrxbe.onion/images/2020/05/16/podcasts/14stillprocessing-image/14stillprocessing-image-thumbLarge-v2.jpg}}

May 14, 2020New Loop, America

\href{https://www.nytimes3xbfgragh.onion/2020/05/07/podcasts/still-processing-internet-vulnerability-sondheim-parks-recreation.html?action=click\&module=audio-series-bar\&region=header\&pgtype=Article}{\includegraphics{https://static01.graylady3jvrrxbe.onion/images/2020/04/28/pageoneplus/28sondheimjp-sp/28sondheimjp-sp-thumbLarge-v4.jpg}}

May 7, 2020Does This Phone Make Me Look Human?

\href{https://www.nytimes3xbfgragh.onion/2020/04/30/podcasts/still-processing-fiona-apple-fetch-bolt-cutters.html?action=click\&module=audio-series-bar\&region=header\&pgtype=Article}{\includegraphics{https://static01.graylady3jvrrxbe.onion/images/2020/05/03/multimedia/30stillpro-image/30stillpro-image-thumbLarge.jpg}}

May 1, 2020Fiona Ex Machina

\href{https://www.nytimes3xbfgragh.onion/2020/04/23/podcasts/still-processing-halle-berry-sharon-stone-catwoman-quarantine.html?action=click\&module=audio-series-bar\&region=header\&pgtype=Article}{\includegraphics{https://static01.graylady3jvrrxbe.onion/images/2020/04/25/arts/23stillprocessing/23stillprocessing-thumbLarge-v3.jpg}}

April 23, 2020Halle Berry? Hallelujah.

\href{https://www.nytimes3xbfgragh.onion/2020/04/16/podcasts/still-processing-AIDS-survive-coronavirus.html?action=click\&module=audio-series-bar\&region=header\&pgtype=Article}{\includegraphics{https://static01.graylady3jvrrxbe.onion/images/2020/04/20/us/16stillprocessing/16stillprocessing-thumbLarge-v3.jpg}}

April 16, 2020How to Learn From a Plague

\href{https://www.nytimes3xbfgragh.onion/2020/04/09/podcasts/still-processing-tiger-king.html?action=click\&module=audio-series-bar\&region=header\&pgtype=Article}{\includegraphics{https://static01.graylady3jvrrxbe.onion/images/2020/04/11/podcasts/09stillprocessing-image2/09stillprocessing-image2-thumbLarge-v2.jpg}}

April 9, 2020~~•~ 39:49Frosted Flakes

\href{https://www.nytimes3xbfgragh.onion/2020/04/02/podcasts/high-fidelity-zoe-kravitz.html?action=click\&module=audio-series-bar\&region=header\&pgtype=Article}{\includegraphics{https://static01.graylady3jvrrxbe.onion/images/2020/04/05/arts/02still-processing-highfidelity/13highfidelity-thumbLarge.jpg}}

April 2, 2020~~•~ 40:55Delicious Vinyl

\href{https://www.nytimes3xbfgragh.onion/2020/03/26/podcasts/still-processing-quarantine.html?action=click\&module=audio-series-bar\&region=header\&pgtype=Article}{\includegraphics{https://static01.graylady3jvrrxbe.onion/images/2020/03/29/podcasts/26stillprocessing1/26stillprocessing1-thumbLarge.jpg}}

March 26, 2020~~•~ 30:47A Pod From Both Our Houses

\href{https://www.nytimes3xbfgragh.onion/2019/11/07/podcasts/still-processing-parasite-watchmen-bong-joon-ho.html?action=click\&module=audio-series-bar\&region=header\&pgtype=Article}{\includegraphics{https://static01.graylady3jvrrxbe.onion/images/2019/11/08/arts/07stilpr-parasite/00parasite-1-thumbLarge.jpg}}

November 7, 2019Wake

\href{https://www.nytimes3xbfgragh.onion/column/still-processing-podcast}{See
All Episodes ofStill Processing}

Next

July 23, 2020

\begin{itemize}
\item
\item
\item
\item
\item
\end{itemize}

By \href{https://www.nytimes3xbfgragh.onion/by/wesley-morris}{Wesley
Morris} and
\href{https://www.nytimes3xbfgragh.onion/by/jenna-wortham}{Jenna
Wortham}

``Hamilton'' is back in the mix, but the flavor has changed from beloved
historical blockbuster to ``wait, \emph{that's} what this is?''
Elsewhere, in new works like ``Baited,'' on Instagram Live, and ``I May
Destroy You,'' on HBO, Black women are getting personal in ways that are
expanding our palates for discomfort.

\includegraphics{https://static01.graylady3jvrrxbe.onion/images/2020/07/23/multimedia/23stillprocessing-pix/merlin_150574872_1f6b8cfd-15b9-4c39-b505-25357651e99d-articleLarge.jpg?quality=75\&auto=webp\&disable=upscale}

Discussed this week:

\begin{itemize}
\item
  ``\href{https://disneyplusoriginals.disney.com/movie/hamilton}{Hamilton}''
  (written by Lin-Manuel Miranda, Disney+, 2020)
\item
  ``\href{https://www.instagram.com/ziwef/channel/?hl=en}{Baited}'' on
  Instagram Live
\item
  \href{https://ziwefumudoh.com/work}{Ziwe Fumudoh}
\item
  ``\href{https://www.hbo.com/i-may-destroy-you}{I May Destroy You}''
  (written by Michaela Coel, HBO, 2020)
\item
  \href{https://native-land.ca/}{Native Land} project
\item
  ``\href{https://www.youtube.com/watch?v=HrOFpaB-PQI}{White
  Fragility}'' lecture (by Robin DiAngelo, 2019)
\end{itemize}

``Still Processing'' is produced by Hans Buetow and edited by Sara
Sarasohn and Sasha Weiss, with editorial oversight from Wendy Dorr and
Lisa Tobin. Our engineer is Jake Gorski. Our theme music is by Kindness.
It's called ``World Restart,'' from the album ``Otherness.''

Advertisement

\protect\hyperlink{after-bottom}{Continue reading the main story}

\hypertarget{site-index}{%
\subsection{Site Index}\label{site-index}}

\hypertarget{site-information-navigation}{%
\subsection{Site Information
Navigation}\label{site-information-navigation}}

\begin{itemize}
\tightlist
\item
  \href{https://help.nytimes3xbfgragh.onion/hc/en-us/articles/115014792127-Copyright-notice}{©~2020~The
  New York Times Company}
\end{itemize}

\begin{itemize}
\tightlist
\item
  \href{https://www.nytco.com/}{NYTCo}
\item
  \href{https://help.nytimes3xbfgragh.onion/hc/en-us/articles/115015385887-Contact-Us}{Contact
  Us}
\item
  \href{https://www.nytco.com/careers/}{Work with us}
\item
  \href{https://nytmediakit.com/}{Advertise}
\item
  \href{http://www.tbrandstudio.com/}{T Brand Studio}
\item
  \href{https://www.nytimes3xbfgragh.onion/privacy/cookie-policy\#how-do-i-manage-trackers}{Your
  Ad Choices}
\item
  \href{https://www.nytimes3xbfgragh.onion/privacy}{Privacy}
\item
  \href{https://help.nytimes3xbfgragh.onion/hc/en-us/articles/115014893428-Terms-of-service}{Terms
  of Service}
\item
  \href{https://help.nytimes3xbfgragh.onion/hc/en-us/articles/115014893968-Terms-of-sale}{Terms
  of Sale}
\item
  \href{https://spiderbites.nytimes3xbfgragh.onion}{Site Map}
\item
  \href{https://help.nytimes3xbfgragh.onion/hc/en-us}{Help}
\item
  \href{https://www.nytimes3xbfgragh.onion/subscription?campaignId=37WXW}{Subscriptions}
\end{itemize}
