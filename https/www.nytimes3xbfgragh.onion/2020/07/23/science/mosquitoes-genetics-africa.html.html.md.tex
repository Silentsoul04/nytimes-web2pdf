Sections

SEARCH

\protect\hyperlink{site-content}{Skip to
content}\protect\hyperlink{site-index}{Skip to site index}

\href{https://www.nytimes3xbfgragh.onion/section/science}{Science}

\href{https://myaccount.nytimes3xbfgragh.onion/auth/login?response_type=cookie\&client_id=vi}{}

\href{https://www.nytimes3xbfgragh.onion/section/todayspaper}{Today's
Paper}

\href{/section/science}{Science}\textbar{}Why Some Mosquitoes Prefer
Humans

\url{https://nyti.ms/30EoPa5}

\begin{itemize}
\item
\item
\item
\item
\item
\item
\end{itemize}

Advertisement

\protect\hyperlink{after-top}{Continue reading the main story}

Supported by

\protect\hyperlink{after-sponsor}{Continue reading the main story}

\hypertarget{why-some-mosquitoes-prefer-humans}{%
\section{Why Some Mosquitoes Prefer
Humans}\label{why-some-mosquitoes-prefer-humans}}

A few species find us irresistible; the rest are unimpressed. A
three-year-long experiment offers an explanation for the difference in
taste.

\includegraphics{https://static01.graylady3jvrrxbe.onion/images/2020/07/23/science/23MOSQUITO1/23MOSQUITO1-articleLarge.jpg?quality=75\&auto=webp\&disable=upscale}

By Angela Chen

\begin{itemize}
\item
  July 23, 2020
\item
  \begin{itemize}
  \item
  \item
  \item
  \item
  \item
  \item
  \end{itemize}
\end{itemize}

Mosquitoes have been called the
\href{https://www.gatesnotes.com/health/most-lethal-animal-mosquito-week}{deadliest
animal in the world}: tiny creatures so dangerous that
\href{https://www.nytimes3xbfgragh.onion/2020/01/08/magazine/gene-drive-mosquitoes.html}{genetic
engineering}may be necessary to win the battle against them. But not all
mosquitoes are equally responsible for devastating the human population
by spreading disease. Out of thousands of species, only a few like to
bite humans --- and even within the same species, mosquitoes from
different places can have different preferences. Why do some find us
irresistible, while others remain unimpressed?

To answer that question, a team of Princeton researchers, working with a
large network of local collaborators, spent three years driving around
sub-Saharan Africa collecting the eggs of Aedes aegypti mosquitoes,
which are responsible for Zika, yellow fever and dengue.

There are two subspecies of Aedes aegypti: one that prefers humans and
one that prefers animals; most populations are a genetic mix. After
sending the eggs to New Jersey to grow new colonies, and then tempting
the insects with the sweet smells of human and of rodent, the
researchers found that the more human-loving mosquitoes tended to come
from areas with a dry climate and dense human population. That, in turn,
is because humans provide the water mosquitoes need to breed.

``There had been quite a bit of speculation in the literature that the
original reason this species evolved to be a human specialist had to do
with its use of human water,'' said
\href{https://eeb.princeton.edu/people/lindy-mcbride}{Lindy McBride}, a
Princeton neuroscientist and an author on the study. ``It's easy to come
up with hypotheses, but what was incredibly surprising was that you
could actually see evidence for that.''

Like all mosquitoes, Aedes aegypti lays its eggs on water, so the
project began by setting out thousands of ovitraps, little plastic cups
lined with seed paper and filled with water and dirty leaves to simulate
the ideal breeding environment. (For cups, the team employed the kind
that casinos give out to hold poker chips.)

The ovitraps were placed in big cities and in rural areas, in an effort
to span environmentally diverse locations, said
\href{https://www.noahhrose.com/}{Noah H. Rose}, a postdoctoral fellow
at Princeton and co-author of the study published on Thursday in
\href{https://www.cell.com/current-biology/fulltext/S0960-9822(20)30978-7}{Current
Biology}. A few days later, someone came back and checked for eggs.

Not all of the expeditions were successful. ``Sometimes you'd spend
weeks in a place and just didn't get any eggs,'' Dr. Rose said. But in
all, the team collected eggs from 27 locations. Once dried, the eggs
were akin to seeds; they could lie dormant for six months or a year
before being hatched, and so were brought back from all across Africa to
the Princeton lab to be bred.

After new colonies were established, the next step was figuring out why
some populations evolved to become generalists and some to become
so-called human specialists. This required deploying an olfactometer: a
big plastic box full of mosquitoes, with two removable tubes in it, one
containing a guinea pig (or, occasionally, a quail ordered from a farm)
and the other holding part of a human.

``I was just sitting with my arm in the tube doing this trial over and
over again,'' Dr. Rose said. He spent ``a couple months of my life'' as
mosquito bait, repeating the experiment hundreds of times while
listening to audiobooks. (A favorite was Anna Burns'
``\href{https://www.nytimes3xbfgragh.onion/2018/12/03/books/review-milkman-anna-burns-man-booker-prize-winner.html}{Milkman}\emph{,''}
about The Troubles in Ireland. Screens kept him and the guinea pig from
actually being bitten.)

Within minutes, mosquitoes, attracted to either the human or the
nonhuman scent, would pick a tube and enter it. Later, the tubes were
removed to count the mosquitoes and figure out how many preferred Dr.
Rose.

The resulting data revealed that mosquitoes that originally came from
very dense areas --- more than 5,000 people per square mile --- liked
humans more. (They also had more ancestry from the human-preferring
subspecies.) A bigger factor, however, was climate. Specifically,
mosquitoes that came from places that had a rainy season followed by a
long, hot, dry season greatly preferred humans.

Why? The scientists proposed an explanation that
\href{https://entomology.cals.cornell.edu/people/brian-lazzaro/}{Brian
Lazzaro}, a professor of entomology at Cornell University who was not
involved with the study, called ``pretty convincing.'' Mosquitoes
flourish during the rainy season, but then must find a way to survive
the dry season. Standing water, critical for mosquitoes to breed, is
hard to come by in extremely arid environments. But it can be found
around humans, who store water to live, and so mosquito populations from
arid regions evolved to take advantage of the situation.

Dr. Lazzaro also praised the team for sequencing the mosquitoes. That
procedure revealed that the human-loving mosquitoes were genetically
distinct from the animal-loving ones, and found that the preference for
humans developed at one location and then spread across Africa. ``They
really see a single origin of these human-feeding mosquitoes,'' he said.
``That is a little surprising to me,'' he added, because there plausibly
could have been multiple instances of genetic adaptation.

The Current Biology paper focused on evolutionary history, but its
findings might have implications for public health. The results,
combined with climate and population data from the United Nations,
suggest that there will be more human-biting mosquitoes in sub-Saharan
Africa by 2050, caused mostly by urbanization.

``I think it's counterintuitive, because people know the climate is
changing rapidly, so ** that ** should be the driving force,'' Dr.
McBride said. ``But the features of the climate that we found to be
important for this mosquito aren't predicted to change in strong and
clear ways that would affect the mosquito.''

Urbanization, in contrast, is occurring very quickly. ``You could easily
imagine that having an effect on disease transmission in big cities,''
Dr. McBride said.

The new paper is a ``major achievement,'' said
\href{https://www.paras.uzh.ch/dam/jcr:1b78cbda-d985-448d-b6c2-73a74083e7dd/Verhulst\%20Curriculum\%20vitae\%20UZH\%20website\%202018\%20(1).pdf}{Niels
O. Verhulst}, an entomologist at the University of Zurich in Switzerland
who was not involved in the study. In 2003, Dr. Verhulst gathered for
review many different papers on mosquito host preference; he quickly
found that they all used different methodologies that made them hard to
compare. That the current study investigated so many different sites was
therefore impressive, he said. And it underscored how important it is
for cities to proactively remove possible mosquito breeding sites.

Dr. Rose said that the team planned to conduct follow-ups in other sites
in Africa, and to study the brains of the human-specialist mosquitoes to
figure out the specific mechanisms that make them love our odor so much.
When it comes to mosquitoes, there's much more to learn. ``Their history
is intertwined with our history,'' he said. ``And mosquitoes are one of
the most interesting ways to understand how human and nature are linked
together in the contemporary world.''

\textbf{\emph{{[}}\href{http://on.fb.me/1paTQ1h}{\emph{Like the Science
Times page on Facebook.}}} ****** \emph{\textbar{} Sign up for the}
\textbf{\href{http://nyti.ms/1MbHaRU}{\emph{Science Times
newsletter.}}\emph{{]}}}

Advertisement

\protect\hyperlink{after-bottom}{Continue reading the main story}

\hypertarget{site-index}{%
\subsection{Site Index}\label{site-index}}

\hypertarget{site-information-navigation}{%
\subsection{Site Information
Navigation}\label{site-information-navigation}}

\begin{itemize}
\tightlist
\item
  \href{https://help.nytimes3xbfgragh.onion/hc/en-us/articles/115014792127-Copyright-notice}{©~2020~The
  New York Times Company}
\end{itemize}

\begin{itemize}
\tightlist
\item
  \href{https://www.nytco.com/}{NYTCo}
\item
  \href{https://help.nytimes3xbfgragh.onion/hc/en-us/articles/115015385887-Contact-Us}{Contact
  Us}
\item
  \href{https://www.nytco.com/careers/}{Work with us}
\item
  \href{https://nytmediakit.com/}{Advertise}
\item
  \href{http://www.tbrandstudio.com/}{T Brand Studio}
\item
  \href{https://www.nytimes3xbfgragh.onion/privacy/cookie-policy\#how-do-i-manage-trackers}{Your
  Ad Choices}
\item
  \href{https://www.nytimes3xbfgragh.onion/privacy}{Privacy}
\item
  \href{https://help.nytimes3xbfgragh.onion/hc/en-us/articles/115014893428-Terms-of-service}{Terms
  of Service}
\item
  \href{https://help.nytimes3xbfgragh.onion/hc/en-us/articles/115014893968-Terms-of-sale}{Terms
  of Sale}
\item
  \href{https://spiderbites.nytimes3xbfgragh.onion}{Site Map}
\item
  \href{https://help.nytimes3xbfgragh.onion/hc/en-us}{Help}
\item
  \href{https://www.nytimes3xbfgragh.onion/subscription?campaignId=37WXW}{Subscriptions}
\end{itemize}
