Sections

SEARCH

\protect\hyperlink{site-content}{Skip to
content}\protect\hyperlink{site-index}{Skip to site index}

\href{https://www.nytimes3xbfgragh.onion/section/style}{Style}

\href{https://myaccount.nytimes3xbfgragh.onion/auth/login?response_type=cookie\&client_id=vi}{}

\href{https://www.nytimes3xbfgragh.onion/section/todayspaper}{Today's
Paper}

\href{/section/style}{Style}\textbar{}House Calls for Everything!

\href{https://nyti.ms/39iL9Kr}{https://nyti.ms/39iL9Kr}

\begin{itemize}
\item
\item
\item
\item
\item
\end{itemize}

\href{https://www.nytimes3xbfgragh.onion/spotlight/at-home?action=click\&pgtype=Article\&state=default\&region=TOP_BANNER\&context=at_home_menu}{At
Home}

\begin{itemize}
\tightlist
\item
  \href{https://www.nytimes3xbfgragh.onion/2020/07/28/books/time-for-a-literary-road-trip.html?action=click\&pgtype=Article\&state=default\&region=TOP_BANNER\&context=at_home_menu}{Take:
  A Literary Road Trip}
\item
  \href{https://www.nytimes3xbfgragh.onion/2020/07/29/magazine/bored-with-your-home-cooking-some-smoky-eggplant-will-fix-that.html?action=click\&pgtype=Article\&state=default\&region=TOP_BANNER\&context=at_home_menu}{Cook:
  Smoky Eggplant}
\item
  \href{https://www.nytimes3xbfgragh.onion/2020/07/27/travel/moose-michigan-isle-royale.html?action=click\&pgtype=Article\&state=default\&region=TOP_BANNER\&context=at_home_menu}{Look
  Out: For Moose}
\item
  \href{https://www.nytimes3xbfgragh.onion/interactive/2020/at-home/even-more-reporters-editors-diaries-lists-recommendations.html?action=click\&pgtype=Article\&state=default\&region=TOP_BANNER\&context=at_home_menu}{Explore:
  Reporters' Obsessions}
\end{itemize}

Advertisement

\protect\hyperlink{after-top}{Continue reading the main story}

Supported by

\protect\hyperlink{after-sponsor}{Continue reading the main story}

\hypertarget{house-calls-for-everything}{%
\section{House Calls for Everything!}\label{house-calls-for-everything}}

Scared to go out for that manicure, dog grooming or tattoo renewal?
These businesses will come to you.

\includegraphics{https://static01.graylady3jvrrxbe.onion/images/2020/07/23/fashion/23HOUSECALLS1/merlin_174782814_91cacad2-62b2-4fe2-879f-f95a28b5c5aa-articleLarge.jpg?quality=75\&auto=webp\&disable=upscale}

\href{https://www.nytimes3xbfgragh.onion/by/ronda-kaysen}{\includegraphics{https://static01.graylady3jvrrxbe.onion/images/2018/07/16/multimedia/author-ronda-kaysen/author-ronda-kaysen-thumbLarge-v2.png}}

By \href{https://www.nytimes3xbfgragh.onion/by/ronda-kaysen}{Ronda
Kaysen}

\begin{itemize}
\item
  July 23, 2020
\item
  \begin{itemize}
  \item
  \item
  \item
  \item
  \item
  \end{itemize}
\end{itemize}

In the middle of May, Ashley Barton sipped a mimosa in her best friend's
apartment in Whitestone, Queens, while she enjoyed her first
professional manicure and pedicure since New York enacted stay-at-home
orders.

The in-house experience, with candles burning and soft music playing,
was a game changer. Before the pandemic, Ms. Barton, a 33-year-old
publicist, would drive from her apartment in Long Island City to a salon
near her parents' house on Long Island to get a mani-pedi. But not
anymore.

``There is something about the comfort of doing this in your own home,
to have someone holding your foot like it is the most amazing thing,''
Ms. Barton said. ``I won't forget when I heard the clippers come out. It
sounded like someone was popping bottles of Dom Pérignon.''

Though New York City hair and nail salons reopened on July 6, Ms. Barton
doubts she'll return to them anytime soon. She has since become a
regular client of \href{http://www.greenspaonthego.com/}{Green Spa on
the Go}, a manicurist from Forest Hills that offers in-home manicures
and pedicures that range from \$140 to \$300, depending on location. Ms.
Barton now gets them done with her parents at their home every two
weeks.

``I'm not ready to sit in a nail salon, even with this protective gear.
I'd rather just be in my parents' backyard doing my nails and toes,''
she said. ``I can control the environment.''

As the country undergoes
\href{https://www.nytimes3xbfgragh.onion/interactive/2020/us/states-reopen-map-coronavirus.html}{a
Sisyphean-seeming reopening} with
\href{https://www.nytimes3xbfgragh.onion/interactive/2020/us/coronavirus-us-cases.html?action=click\&pgtype=Article\&state=default\&module=styln-coronavirus\&region=TOP_BANNER\&context=storylines_menu\#states}{infection
rates rising in many regions}, many Americans are still wary of
venturing out. To accommodate skittish clients who have neglected their
\href{https://www.nytimes3xbfgragh.onion/2020/04/21/smarter-living/maybe-consider-shaving-that-pandemic-beard.html}{basic
grooming for months}, a cadre of service providers --- personal
trainers, hair stylists, tattoo artists, pet groomers and spiritual
advisers --- have been making house calls, or plan to start doing so
soon. They have been fielding calls from clients eager to receive
services in their living rooms, backyards and balconies.

For providers who always had an in-home component to their business,
this period of seclusion has proved to be a boon, giving them an edge at
an anxious time. They've attracted new clients who never considered
house calls before, but have since discovered that they like private
pampering.

``We believe this is a long term shift in consumer behavior,'' said Amy
Shecter, the chief executive of
\href{https://www.glamsquad.com/}{Glamsquad}, an in-home beauty company
headquartered in New York City that offers services including blowouts,
manicures, pedicures and makeup application.

Glamsquad has resumed its usual services and introduced haircuts in New
York, Florida, Boston and Washington, D.C. In Los Angeles, the company
moved services outdoors after the state rolled back its reopening plans.
(Its San Francisco market remains closed.) Workers undergo safety
training to reduce the chance of coronavirus transmission and wear
personal protective gear during visits.

``This is our moment,'' said Ms. Schecter. ``People are going to shift
to doing this service in-home and they're going to shift to providers
like us.''

\includegraphics{https://static01.graylady3jvrrxbe.onion/images/2020/07/23/fashion/23HOUSECALLS2/23HOUSECALLS2-articleLarge.jpg?quality=75\&auto=webp\&disable=upscale}

Marianella Aguirre, the founder of Green Spa on the Go, which
specializes in house calls in New York, Connecticut and the Hamptons,
also sees her business model as one that could thrive in the current
environment, with anxious clients worried about squeezing into crowded
nail salons.

``It's incredible,'' she said. ``It's like we were set up for this.''

Before the pandemic, Ms. Aguirre received about five calls a week from
new clients. Now she receives as many a day. Nearly all of them request
that the service be provided in a backyard. ``They are desperate and
they don't want to wait anymore,'' she said.

At-home services have always been a niche offering, and there is no data
to track if the industry has grown during state-mandated stay-at-home
orders, since such services that occurred during that time would have
likely been clandestine.

But interest does not appear to be limited to the beauty industry. At
\href{https://www.groomit.me/}{Groomit}, which provides in-house pet
grooming services in New York, New Jersey and Connecticut, business is
up 30 percent since the company started offering services again
according to a founder, Sohel Kapadia. ``Even though half of New York
City is empty,'' said Mr. Kapadia, referring to the residents who fled
the city as it shut down, ``the new customers are making up for the
losses of the old customers.''

After months spent huddling at home, many Americans feel safer on their
sofa than they do almost anywhere else. But a living room is not
necessarily a safer location for a blowout than a salon. With good
ventilation, adequate physical distancing and enough personal protective
equipment, a salon may actually be less risky than a small, cramped
apartment, even if more people pass through the salon. Grooming services
may not be safe in any setting in areas where cases are rising, like
South Florida.

``We still don't know that much about transmission even though we're
bombarded with information about what we do know,'' said
\href{https://www.publichealth.columbia.edu/research/center-infection-and-immunity/angela-rasmussen-phd}{Dr.
Angela Rasmussen}, a virologist at Columbia University Irving Medical
Center. ``Is it safer to have more people in a space, but in a space
that is bigger and has better ventilation and has better masking? We
can't really quantify what the differences are.''

Image

Woof, this is convenient! Christopher Perkin of Groomit grooms Gus at
his owner's apartment in the Financial District.Credit...Nina Westervelt
for The New York Times

Joey Chavez, a tattoo artist in Corona, Calif., was able to reopen his
tattoo parlor Trusted Tattoo in mid-June, but had been offering services
to clients out of a mobile tattoo bus, the
\href{https://www.bodyartbus.com/}{Body Art Bus}, which he would drive
to clients' homes in the weeks before the shop reopened. He sanitized
the vehicle between clients, used disposable equipment and limited the
number of people in the space to only himself and client.

``It is a lot safer for us to come to their homes,'' said Mr. Chavez,
who charges a \$250 fee for the service on top of the \$150-per-hour fee
for a tattoo. ``We bring a fully disinfectable mobile tattoo parlor.''

Before the pandemic, the Body Art Bus was primarily used for social
occasions like bachelorette and birthday parties, and once the tattoo
parlor reopened, he began bringing the bus to smaller house parties
again. The bus, which he has operated for 10 years, has attracted
celebrity clients too, including
\href{https://twitter.com/BodyArtBus/status/1134131165697257477}{Katy
Perry}, who rented it for an event, and Tech N9ne, who brought the bus
along on a tour. But now, Mr. Chavez sees the bus as a way to showcase
how safe body art can be during a pandemic. ``There are a lot of people
who are afraid'' to go back to a parlor, he said.

Samuel Sanchez is one of them. Mr. Sanchez, 49, a substitute teacher
who's studying speech pathology at California State University San
Marcos, had a March appointment at Mr. Chavez's studio to restore a
25-year-old dragon tattoo etched across his back. But as news of
coronavirus percolated through California, Mr. Sanchez's wife told him
to cancel. ``I felt like the universe was saying maybe you shouldn't do
this,'' he said.

Mr. Sanchez, who lives in Menifee, Calif., was devastated. Two years
earlier, he had been in a serious car accident, with injuries and
subsequent surgeries that left deep scarring across his back, severing
his tattoo. He saw his window to get the tattoo restored closing.

But after Mr. Chavez showed him the bus and the equipment, Mr. Sanchez
was convinced that it was the safest way to get inked. ``The whole thing
is pampered. You're sitting on killer chairs. He sterilized everything.
You sign a waiver. You walk into the bus and only the tattoo artist and
you,'' Mr. Sanchez said. ``I felt like the bus was the most sterile
thing I could do.''

In mid-June, Mr. Sanchez had the first of four sessions to restore the
tattoo. During that appointment, which took about four hours, he
discovered another upside to a bus: He could avoid the prying eyes of
other customers at a crowded tattoo parlor. ``It takes all the machismo
out of it,'' he said. ``You don't got all these bystanders and people
watching to see if you're squirming or not. It's you and your guy.''

Reopening a business that makes house calls during a pandemic is not
without its risks. On June 27, Mr. Chavez had to shut down Trusted
Tattoo and the bus for two weeks after a client tested positive for
coronavirus soon after a Body Art Bus tattoo session. Mr. Chavez and
another tattoo artist who had also been on the bus quarantined for two
weeks. Mr. Chavez reopened his businesses on July 13, only for local
health department officials to shut them down almost immediately, as the
state tightened its quarantine restrictions.

``It's impossible to plan. I had to cancel 14 days of clients,'' Mr.
Chavez said. ``It makes it hard to do anything.''

Image

Nini Grace, a spiritual life coach and medium, ``cleansed the energy''
at The Barn at 70 in Manchester Township, New Jersey on Friday, July 17,
2020.Credit...Calla Kessler for The New York Times

But for those who can receive them, house calls can be a salve for
people isolated in a difficult time. In May, clients started calling
\href{http://theenlightenmentcntr.com/nini}{Nini Grace}, a spiritual
life coach and medium in Avon-by-the-Sea, N.J., requesting house calls.
By June, she had begun visiting clients in their homes again, holding
sessions mostly in people's backyards and charging \$90 per person for
the visits. Ms. Grace said her business has tripled, and she's now
booked through October for home visits.

Ms. Grace wears a mask when she greets the clients, but once everyone is
settled outdoors and at a safe distance, she usually removes it to start
the session.

``This is such a desperate time with so many unknowns for folks,'' she
said. ``They just needed some comfort.''

Advertisement

\protect\hyperlink{after-bottom}{Continue reading the main story}

\hypertarget{site-index}{%
\subsection{Site Index}\label{site-index}}

\hypertarget{site-information-navigation}{%
\subsection{Site Information
Navigation}\label{site-information-navigation}}

\begin{itemize}
\tightlist
\item
  \href{https://help.nytimes3xbfgragh.onion/hc/en-us/articles/115014792127-Copyright-notice}{©~2020~The
  New York Times Company}
\end{itemize}

\begin{itemize}
\tightlist
\item
  \href{https://www.nytco.com/}{NYTCo}
\item
  \href{https://help.nytimes3xbfgragh.onion/hc/en-us/articles/115015385887-Contact-Us}{Contact
  Us}
\item
  \href{https://www.nytco.com/careers/}{Work with us}
\item
  \href{https://nytmediakit.com/}{Advertise}
\item
  \href{http://www.tbrandstudio.com/}{T Brand Studio}
\item
  \href{https://www.nytimes3xbfgragh.onion/privacy/cookie-policy\#how-do-i-manage-trackers}{Your
  Ad Choices}
\item
  \href{https://www.nytimes3xbfgragh.onion/privacy}{Privacy}
\item
  \href{https://help.nytimes3xbfgragh.onion/hc/en-us/articles/115014893428-Terms-of-service}{Terms
  of Service}
\item
  \href{https://help.nytimes3xbfgragh.onion/hc/en-us/articles/115014893968-Terms-of-sale}{Terms
  of Sale}
\item
  \href{https://spiderbites.nytimes3xbfgragh.onion}{Site Map}
\item
  \href{https://help.nytimes3xbfgragh.onion/hc/en-us}{Help}
\item
  \href{https://www.nytimes3xbfgragh.onion/subscription?campaignId=37WXW}{Subscriptions}
\end{itemize}
