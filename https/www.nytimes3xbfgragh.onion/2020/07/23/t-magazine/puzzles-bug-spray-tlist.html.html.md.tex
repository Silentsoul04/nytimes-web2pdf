Sections

SEARCH

\protect\hyperlink{site-content}{Skip to
content}\protect\hyperlink{site-index}{Skip to site index}

\href{https://myaccount.nytimes3xbfgragh.onion/auth/login?response_type=cookie\&client_id=vi}{}

\href{https://www.nytimes3xbfgragh.onion/section/todayspaper}{Today's
Paper}

The T List: Five Things We Recommend This Week

\url{https://nyti.ms/2Cwoa2D}

\begin{itemize}
\item
\item
\item
\item
\item
\end{itemize}

Advertisement

\protect\hyperlink{after-top}{Continue reading the main story}

Supported by

\protect\hyperlink{after-sponsor}{Continue reading the main story}

\hypertarget{the-t-list-five-things-we-recommend-this-week}{%
\section{The T List: Five Things We Recommend This
Week}\label{the-t-list-five-things-we-recommend-this-week}}

Well-designed puzzles, natural bug sprays, Paul McCarthy --- and more.

July 23, 2020

\begin{itemize}
\item
\item
\item
\item
\item
\end{itemize}

\emph{Welcome to the T List, a newsletter from the editors of T
Magazine. Each week, we're sharing things we're eating, wearing,
listening to or coveting now.}
\textbf{\href{https://www.nytimes3xbfgragh.onion/newsletters/t-list?module=inline}{\emph{Sign
up here}}} \emph{\textbf{to find us in your inbox every Wednesday.}}
\emph{You can always reach us at}
\href{mailto:tlist@NYTimes.com}{\emph{tlist@NYTimes.com}}\emph{.}

\begin{center}\rule{0.5\linewidth}{\linethickness}\end{center}

Unbox This

\hypertarget{relaxing-puzzles-with-a-sharp-design-sensibility}{%
\subsection{Relaxing Puzzles With a Sharp Design
Sensibility}\label{relaxing-puzzles-with-a-sharp-design-sensibility}}

\includegraphics{https://static01.graylady3jvrrxbe.onion/images/2020/07/22/t-magazine/22tmag-tlist-slide-KKWD/22tmag-tlist-slide-KKWD-articleLarge.jpg?quality=75\&auto=webp\&disable=upscale}

By Samuel Rutter

Just over a year ago --- well before Covid-19 ushered in a renewed
interest in indoor activities --- Echo Hopkins found herself taking a
few minutes out of each workday to complete a puzzle as a small way of
escaping screens and refocusing her attention. This eventually led her,
together with her mother, Teresa Hopkins, to co-found Ordinary Habit,
under which the pair have now launched a limited-edition series of
jigsaw puzzles that marries the mindfulness of play with thoughtful
design. ``We wanted to encourage a return to doing things that are a
little more tactile,'' said Echo, who is based in Brooklyn, N.Y.
``There's such a simple joy to finding one piece that fits in a
puzzle.'' Meanwhile, to engage the eyes, they have commissioned female
illustrators based around the world, including
\href{https://www.instagram.com/holly_jolley_/?hl=en}{Holly Jolley} in
Chile, \href{https://www.instagram.com/bodiljane/?hl=en}{Bodil Jane} in
Amsterdam and the New York City-based artist
\href{https://www.instagram.com/shawnax/?hl=en}{Shawna X}, whose work
adorns the exterior of the inclusive performance space House of Yes in
Bushwick. There are six 500-piece options in all, each of which is made
of recycled materials and comes in a box with a side resembling the
spine of a novel that fits seamlessly on a bookshelf. A portion of the
profits will be donated monthly to
\href{https://thelovelandfoundation.org/}{the Loveland Foundation},
which provides financial assistance for therapy and other mental-health
services for Black women and girls. \emph{\$40,}
\href{http://ordinaryhabit.com/}{\emph{ordinaryhabit.com}}\emph{.}

\begin{center}\rule{0.5\linewidth}{\linethickness}\end{center}

Covet This

\hypertarget{10-vessels-from-master-potter-lucie-rie}{%
\subsection{10 Vessels From Master Potter Lucie
Rie}\label{10-vessels-from-master-potter-lucie-rie}}

Image

Ceramics by Lucie Rie to be featured in the Phillips Design Auction on
July 29, 2020.Credit...Courtesy of Phillips

By Thessaly La Force

In 1938, the potter Lucie Rie, then 36, fled Vienna with her wheel and a
suitcase of her work. She found refuge in London, where she would live
for the remainder of her life, making artful ceramics and becoming one
of Europe's most celebrated talents. Unlike other master English potters
such as Bernard Leach, who was inspired by the Arts and Crafts movement,
Rie drew her influences from both Modernism (in particular, the spare
sensibility of the midcentury Austrian architect and designer Josef
Hoffmann) as well as much older references, both European and Asian. As
Rie told the American journalist Claire Frankel in June 1990 in The
International Herald Tribune: ``I was not so much influenced by the art
school {[}School of Decorative Arts{]} as by a small country museum on
the border of Hungary where there are Roman pots in the museum and maybe
five Chinese pots that influenced me. My teaching in Vienna was `look at
those beautiful glazes. You will never be able to do that.' It was a
great incentive. And I did it.'' Rie experimented with various colors
and textures, often applying the glaze directly onto the surface of her
stoneware or porcelain works before firing them only once (other potters
often use a more complicated process of bisque firing, applying glaze
and then refiring the object again). She limited her decoration to
sgraffito (using a needle to scratch on lines), subtle spirals, lips or
inlay --- and her pots are beautiful in their simplicity. Ten of them
(from Frankel's collection) are on view by appointment at the Phillips
auction house as part of its Design Auction this week. \emph{Phillips,
450 Park Ave, New York, N.Y., 10022,}
\href{https://www.phillips.com/artist/711/lucie-rie}{\emph{phillips.com}}\emph{.}

\begin{center}\rule{0.5\linewidth}{\linethickness}\end{center}

Buy This

\hypertarget{four-deet-free-moisturizing-bug-sprays}{%
\subsection{Four DEET-Free, Moisturizing Bug
Sprays}\label{four-deet-free-moisturizing-bug-sprays}}

Image

Clockwise from left: Jao's Patio Oil, \$30,
\href{https://jaobrand.com/collections/apothecary/products/patio-oil}{jaobrand.com}.
Kinfield's Golden Hour, \$22,
\href{https://kinfield.com/products/golden-hour?utm_source=tmag\&utm_medium=pr\&utm_campaign=tlistjune2020}{kinfield.com}.
Alba Botanica's Anti-Bug Spray, \$16,
\href{https://www.walmart.com/ip/Alba-Botanica-Anti-Bug-Deet-free-Insect-Repellent-Spray-4-Oz/349243024}{walmart.com}.
Beekman 1802's Bye Bye Bugs Herbal Bug Repellent Bars, \$12,
\href{https://beekman1802.com/products/bug-repellent-soap}{beekman1802.com}.Credit...Courtesy
of the brands

By Caitie Kelly

As temperatures rise this July, so does the presence of pesky
mosquitoes. And with the outdoors being the only safe place to socialize
(from a distance) at the moment, a good bug spray is essential. These
four DEET-free insect repellents promise not only to deter insects but
also to moisturize and refresh tired summer skin. The Brooklyn-based
personal wellness brand Kinfield created its
\href{https://kinfield.com/products/golden-hour}{Golden Hour} (\$22)
spray using a strain of citronella sourced from Indonesia. Its
fast-drying formula also contains lemongrass and clove, offering a more
pleasing scent than your typical bug spray, as well as lauric acid for
hydration. Alba Botanica's
\href{https://www.walmart.com/ip/Alba-Botanica-Anti-Bug-Deet-free-Insect-Repellent-Spray-4-Oz/349243024}{Anti-Bug
Spray} (\$18) uses lemongrass and citronella as well, plus peppermint
oil, which provides a cooling sensation sure to alleviate any existing
bites.
\href{https://jaobrand.com/collections/apothecary/products/patio-oil}{Patio
Oil} (\$30), a super hydrating balm from the Pennsylvania-based brand
Jao, soothes skin with jojoba butter and vitamin E while warding off
bugs with lemon eucalyptus oil. Beekman 1802, which crafts its
goat-milk-rich products in upstate New York, carries a versatile bug
repellent bar soap called
\href{https://beekman1802.com/products/bug-repellent-soap}{Bye Bye Bugs
Herbal Bug Repellent Bars}, that can be used in the shower (the body
releases the fragrance --- citronella, lavender and lemon peel --- into
the air later) or simply rubbed on clothing or pulse points for shorter
exposures. The bars are cut into small squares ideal for throwing in a
weekend bag or having on hand for hikes.

\begin{center}\rule{0.5\linewidth}{\linethickness}\end{center}

Read This

\hypertarget{a-poets-visual-memoir-from-the-70s}{%
\subsection{A Poet's Visual Memoir From the
'70s}\label{a-poets-visual-memoir-from-the-70s}}

Image

A grid from ``Memory'' (2020) by Bernadette Mayer. Credit...Courtesy of
the Bernadette Mayer Papers, Special Collections \& Archives, University
of California, San Diego

By Megan O'Grady

In July **** 1971, the avant-garde poet and photographer Bernadette
Mayer embarked on an ``emotional science project,'' documenting each day
of the month with a roll of Kodachrome film and a journal entry. Mayer
created a seven-hour audio recording of the text, and the result ---
which was shown by gallerist Holly Solomon in 1972 and then not again
until 2016 (at Chicago's Poetry Foundation) --- is now being published
in book form by Siglio Press. Who, really, can resist New York City in
the 1970s --- the elongated yellow taxis, shop signs and hot-dog
vendors, the World Trade Center rising? Mayer captured the Big Apple
before **** the gloss, before Starbucks and bank branches colonized the
streets and artists and writers fled, first for the other boroughs, then
for other cities entirely. But ``Memory'' is first and foremost a deeply
personal exercise in observation, its pages filled with shopping lists,
friends, interiors of diners, evidence of trips upstate, breakfasts,
trees, a shaggy-haired lover. (There's even an analog selfie.) Seen in
another light, the project seems to anticipate the way we think about
representing life today, whether we're sharing snippets of our days on
Instagram or unpolished fragments of thought on Twitter. Mayer, who
became the director of the St. Mark's Poetry Project in the early 1980s,
was a rebel of form who refused to see life as a continuous, unspooling
narrative filled with straightforward meanings. In her thoughts and
images, we find an immersion in quotidian minutiae, synecdoche for a
lost era that feels almost eerily contemporary. \emph{\$45,}
\href{http://sigliopress.com/book/memory/}{\emph{sigliopress.com}}\emph{.}

\begin{center}\rule{0.5\linewidth}{\linethickness}\end{center}

View This

\hypertarget{paul-mccarthys-alpine-sketches}{%
\subsection{Paul McCarthy's Alpine
Sketches}\label{paul-mccarthys-alpine-sketches}}

Image

Left: Paul McCarthy's ``A\&E, EVA, Santa Anita session'' (2020). Right:
the artist's performance still ``A\&E Drawing Session, Santa Anita''
(2020).Credit...Left: photo by~Damon McCarthy. All images: © Paul
McCarthy.~Courtesy of the artist and Hauser \& Wirth

By M.H. Miller

The artist Paul McCarthy is currently the subject of two shows, one
online at Hauser \& Wirth, and another at the exhibition space Tarmak 22
in Gstaad, the resort town in the Swiss Alps. These are, in many ways,
ideal locations for this artist, who has always had a heightened
awareness of detail and context. (On a recent Zoom call to discuss these
shows, McCarthy sat in a mostly unadorned room, the only remarkable
feature of which was a wooden cutout of Santa Claus, propped up against
the wall behind him; McCarthy, with his big white beard, shared a
certain resemblance.) As an early adopter of video-based art, he seemed
to predict the random, frenetic qualities of a mind poisoned by the
internet several decades before the fact, and McCarthy has often darkly
satirized Alpine culture and German fairy tales, Hollywood's
appropriation of both and the fascist tendencies of all three. The
online show, in particular, functions a little like a summary of
McCarthy's stylistic quirks. Nominally an exhibition of drawings, its
contents **** in fact resulted from a series of improvisational
performances McCarthy began filming with the actress Lilith Stangenberg
in early 2020: ``drawing sessions,'' as the artist refers to them, while
the two were in character as versions of Adolf Hitler and Eva Braun for
**** McCarthy's new film project, called simply ``A\&E.'' The results
are messy, uncomfortable, confrontational and ultimately fascinating ---
convincing evidence that, even in the supposedly simple gesture of
making a mark on a piece of paper, McCarthy is never as simple or
straightforward as he might seem.
\href{https://www.hauserwirth.com/hauser-wirth-exhibitions/28962-paul-mccarthy-alpine-stories-dystopias}{\emph{hauserwirth.com}}\emph{.}

\begin{center}\rule{0.5\linewidth}{\linethickness}\end{center}

From T's Instagram

\hypertarget{tartissue-the-story-of-ruth-asawa}{%
\subsection{\#TArtIssue: The Story of Ruth
Asawa}\label{tartissue-the-story-of-ruth-asawa}}

Advertisement

\protect\hyperlink{after-bottom}{Continue reading the main story}

\hypertarget{site-index}{%
\subsection{Site Index}\label{site-index}}

\hypertarget{site-information-navigation}{%
\subsection{Site Information
Navigation}\label{site-information-navigation}}

\begin{itemize}
\tightlist
\item
  \href{https://help.nytimes3xbfgragh.onion/hc/en-us/articles/115014792127-Copyright-notice}{©~2020~The
  New York Times Company}
\end{itemize}

\begin{itemize}
\tightlist
\item
  \href{https://www.nytco.com/}{NYTCo}
\item
  \href{https://help.nytimes3xbfgragh.onion/hc/en-us/articles/115015385887-Contact-Us}{Contact
  Us}
\item
  \href{https://www.nytco.com/careers/}{Work with us}
\item
  \href{https://nytmediakit.com/}{Advertise}
\item
  \href{http://www.tbrandstudio.com/}{T Brand Studio}
\item
  \href{https://www.nytimes3xbfgragh.onion/privacy/cookie-policy\#how-do-i-manage-trackers}{Your
  Ad Choices}
\item
  \href{https://www.nytimes3xbfgragh.onion/privacy}{Privacy}
\item
  \href{https://help.nytimes3xbfgragh.onion/hc/en-us/articles/115014893428-Terms-of-service}{Terms
  of Service}
\item
  \href{https://help.nytimes3xbfgragh.onion/hc/en-us/articles/115014893968-Terms-of-sale}{Terms
  of Sale}
\item
  \href{https://spiderbites.nytimes3xbfgragh.onion}{Site Map}
\item
  \href{https://help.nytimes3xbfgragh.onion/hc/en-us}{Help}
\item
  \href{https://www.nytimes3xbfgragh.onion/subscription?campaignId=37WXW}{Subscriptions}
\end{itemize}
