Sections

SEARCH

\protect\hyperlink{site-content}{Skip to
content}\protect\hyperlink{site-index}{Skip to site index}

\href{https://www.nytimes3xbfgragh.onion/section/arts/music}{Music}

\href{https://myaccount.nytimes3xbfgragh.onion/auth/login?response_type=cookie\&client_id=vi}{}

\href{https://www.nytimes3xbfgragh.onion/section/todayspaper}{Today's
Paper}

\href{/section/arts/music}{Music}\textbar{}Barenboim and Elgar: A
Musical Love Story Continues

\url{https://nyti.ms/30Mmp9D}

\begin{itemize}
\item
\item
\item
\item
\item
\item
\end{itemize}

Advertisement

\protect\hyperlink{after-top}{Continue reading the main story}

Supported by

\protect\hyperlink{after-sponsor}{Continue reading the main story}

\hypertarget{barenboim-and-elgar-a-musical-love-story-continues}{%
\section{Barenboim and Elgar: A Musical Love Story
Continues}\label{barenboim-and-elgar-a-musical-love-story-continues}}

A fifth album in Daniel Barenboim's Elgar cycle with the Staatskapelle
Berlin comes out on Friday.

\includegraphics{https://static01.graylady3jvrrxbe.onion/images/2020/07/25/arts/24barenboim-1/24barenboim-1-articleLarge-v2.jpg?quality=75\&auto=webp\&disable=upscale}

By David Allen

\begin{itemize}
\item
  July 24, 2020
\item
  \begin{itemize}
  \item
  \item
  \item
  \item
  \item
  \item
  \end{itemize}
\end{itemize}

Daniel Barenboim and Edward Elgar have made for one of classical music's
most unusual love affairs in recent years.

Outside England, the music of Elgar (1857-1934) still has a crusty,
flag-waving reputation, despite the efforts of
\href{https://jpehs.co.uk/introduction/publications/elgar-an-extraordinary-life/}{musicologists}
and the \href{https://www.youtube.com/watch?v=Aftq_Yrudv8}{advocacy} of
\href{https://www.nytimes3xbfgragh.onion/2017/05/09/arts/music/a-rare-performance-of-elgars-gigantic-work.html}{musicians}.
But over the past eight years, Mr. Barenboim, 77, and his Staatskapelle
Berlin have released accounts of Elgar's two symphonies, the oratorio
``The Dream of Gerontius'' and the
\href{https://www.nytimes3xbfgragh.onion/2012/10/28/arts/music/elgar-carter-cello-concertos.html}{Cello
Concerto}, with Alisa Weilerstein.

It's a connection of long standing: Mr. Barenboim's first wife, the
\href{https://www.youtube.com/watch?v=qJn86UJPmEo}{cellist Jacqueline du
Pré}, collaborated with the
\href{https://www.youtube.com/watch?v=UgtJZysa11A}{conductor} John
Barbirolli on a classic recording of the Cello Concerto in 1965, and she
and Barbirolli in turn inspired the young Argentine-born Mr. Barenboim
to
\href{https://www.nytimes3xbfgragh.onion/1975/05/16/archives/barenboim-is-fervent-in-elgars-first.html}{learn}
and \href{https://www.youtube.com/watch?v=Dzj9EXwVyz8}{record}
\href{https://www.youtube.com/watch?v=vBuDLLTCnDA}{much}
\href{https://www.youtube.com/watch?v=WYbMsJWoguU}{of}
\href{https://www.youtube.com/watch?v=cswb_WvD28k}{Elgar}'s work with
the London Philharmonic.

A fifth album in the Berlin cycle is coming out on Friday, featuring
``Sea Pictures'' (five songs, sung by Elina Garanca) and ``Falstaff,''
an ambitious, often rambunctious symphonic poem. Mr. Barenboim, whose
contract with the Staatskapelle and the Berlin State Opera
\href{https://www.nytimes3xbfgragh.onion/2019/06/04/arts/music/daniel-barenboim-staatsoper-berlin.html}{was
extended last year} amid accusations of bullying, spoke by phone from
Spain about Elgar and his music. These are edited excerpts from the
conversation.

\textbf{Why do you love this music so much?}

It's a difficult question to answer, because one has to admit that,
historically, Elgar is not so important. If Elgar had not come through
this earth, the development of music would have been the same. One also
has to forget that he was somewhat anachronistic, when you think what
else was being written at the time --- Schoenberg, Stravinsky, etc.

\includegraphics{https://static01.graylady3jvrrxbe.onion/images/2020/07/25/arts/24barenboim-2/24barenboim-2-articleLarge.jpg?quality=75\&auto=webp\&disable=upscale}

But there is a unique quality in his music which appeals to me
tremendously: something emotional, in the best sense of the word. Not
outward, but something very, very deep and sincere, which has to do, I
suppose, with the modulations --- with the harmonic language, which is
unlike that of many other composers. The closest is Strauss.

\textbf{Should we then think of Elgar not as a radical, like Schoenberg
or Stravinsky, but as a progressive, like Strauss or Mahler?}

I think so. ``Falstaff'' is a special work in Elgar's output. It has
things that connect it to his symphonies, but if the symphonies are
close to Strauss's ``Don Juan'' and ``Ein Heldenleben,'' ``Falstaff'' **
is close to ``Till Eulenspiegel*.*''

\textbf{Even in England, ``Falstaff''} ****** \textbf{is not that often
played compared with some of Elgar's works, and if music lovers know the
``Falstaff'' story, it's primarily through Verdi.}

Verdi, of course. But you know, I take very slight objection to the fact
that Elgar's nationality is always mentioned in relation to his music,
as if it was not to be expected that one could be English and be a great
composer. Nobody talks about the nationality of other composers as much
as they talk about Elgar being English; of course, there is a certain
Englishness about it, but it's not the most important element.

\textbf{What is the most important element?}

The harmonic language, the orchestration, is remarkable, if the
conductor balances the orchestra properly and the orchestra has
familiarity with the music, which is very rarely the case, because Elgar
is not played that often. The English saying ``familiarity breeds
contempt'' is totally out of place; we forget that orchestras and
publics alike need familiarity with music in order to love it.

\textbf{One of the things that you seem to be saying is that Elgar was
part of a European --- not just an English --- tradition.}

This is a very dangerous statement you are making now in view of Brexit,
of course. I think he is very much a European composer, don't you?

\textbf{Absolutely. Wasn't that the point you were trying to make when
you} \textbf{\href{https://www.bbc.co.uk/programmes/p058y4np}{played}}
\textbf{his ``Land of Hope and Glory'' at the BBC Proms with the
Staatskapelle,}
\textbf{\href{https://www.theguardian.com/music/2017/jul/17/daniel-barenboim-calls-for-european-unity-in-proms-speech}{the
year after the Brexit vote}?}

``Land of Hope and Glory'' at the Proms had nothing to do with a
political thing; it was totally misinterpreted. We played both
symphonies at the Proms, and I wanted to show that you don't have to be
English to play this music well.

I am a firm believer in the European idea, and I am a firm believer that
a lot of the problem with the European Union is that many people forget
that it was not only a financial or economic idea. Let us not forget
that whether it is France, Germany, Italy, England or Spain, culture is
the greatest contribution, historically, of the continent. It is a
different contribution from the other continents, and therefore culture
--- European culture --- is a very important point for today's world,
too.

\textbf{That raises the issue that Elgar is usually thought of as a
quintessentially English composer because of his association with the
British Empire.}

Yes, but do you think that Elgar's connection to the English part of it
is more important than, shall we say, Debussy's to France? No.

\textbf{But as someone who loves Elgar's music, I still have trouble
with it historically, as I love and still have trouble with Wagner's
music.}

Yes, but your problem with Wagner's music, I imagine, has to do with his
profile as a person, as a human being, which is not the case with Elgar.

\textbf{Elgar still wrote works like
``\href{https://www.youtube.com/watch?v=ah2cG8cjD7A}{The Crown of
India}''} ****** \textbf{and the
``\href{https://www.youtube.com/watch?v=zik9vQsti5k}{Imperial March},''
though. So how do you think about performing him today, during a global
reckoning with racism, slavery and empire? Should we ignore that part of
Elgar? Should we confront it?}

No, I think we have to place it in context. Let's be a little bit more
neutral in our remarks. We realized a long time ago that slavery was a
horrific thing, and we did away with it, but at the time that it was
there, it was there. The English Empire quality is only a part of some
moments of Elgar's pieces. Let's not dwell on the ``Pomp and
Circumstance Marches,'' because that's a ``pièce d'occasion,'' like the
ballet in ``Aida,'' but in the serious works --- ``The Dream of
Gerontius,'' the symphonies, ``Falstaff,'' the Cello Concerto, the ``Sea
Pictures'' --- that element is only a part of it.

\textbf{So we can play him today by accepting that part and moving on?
Is that what you are saying?}

Yes, I don't think we have to play Elgar and pay special attention, as
it were, not to forget that there was a British Empire and that that was
the expression of it. That is part of the whole.

\textbf{Are there particular moments of ``Falstaff'' that you think show
Elgar at his best?}

The interlude in the center, the small interlude with the violin solo,
is very touching, because it is juxtaposed against very rhythmical,
boisterous music. And of course the end. Falstaff's death is an absolute
masterpiece of composition.

\textbf{Elgar had a gift for endings, like}
\textbf{\href{https://youtu.be/5yi2db0ZLg8?list=OLAK5uy_lsUCoQUQ9CuUbQVq1nGW6irUbtDjfIs0c\&t=657}{the
end of the Second Symphony}}.

Yes, and they are very difficult to conduct. If you look at the score of
the end of ``Falstaff,'' it is so constructed --- I wouldn't say
calculated, because that smells of something not natural. Then, when
it's finished, it's finished; it doesn't end on a sentimental note. He
dies, and then there is a very little coda, which seems to say death is
part of life. And that's it.

Advertisement

\protect\hyperlink{after-bottom}{Continue reading the main story}

\hypertarget{site-index}{%
\subsection{Site Index}\label{site-index}}

\hypertarget{site-information-navigation}{%
\subsection{Site Information
Navigation}\label{site-information-navigation}}

\begin{itemize}
\tightlist
\item
  \href{https://help.nytimes3xbfgragh.onion/hc/en-us/articles/115014792127-Copyright-notice}{©~2020~The
  New York Times Company}
\end{itemize}

\begin{itemize}
\tightlist
\item
  \href{https://www.nytco.com/}{NYTCo}
\item
  \href{https://help.nytimes3xbfgragh.onion/hc/en-us/articles/115015385887-Contact-Us}{Contact
  Us}
\item
  \href{https://www.nytco.com/careers/}{Work with us}
\item
  \href{https://nytmediakit.com/}{Advertise}
\item
  \href{http://www.tbrandstudio.com/}{T Brand Studio}
\item
  \href{https://www.nytimes3xbfgragh.onion/privacy/cookie-policy\#how-do-i-manage-trackers}{Your
  Ad Choices}
\item
  \href{https://www.nytimes3xbfgragh.onion/privacy}{Privacy}
\item
  \href{https://help.nytimes3xbfgragh.onion/hc/en-us/articles/115014893428-Terms-of-service}{Terms
  of Service}
\item
  \href{https://help.nytimes3xbfgragh.onion/hc/en-us/articles/115014893968-Terms-of-sale}{Terms
  of Sale}
\item
  \href{https://spiderbites.nytimes3xbfgragh.onion}{Site Map}
\item
  \href{https://help.nytimes3xbfgragh.onion/hc/en-us}{Help}
\item
  \href{https://www.nytimes3xbfgragh.onion/subscription?campaignId=37WXW}{Subscriptions}
\end{itemize}
