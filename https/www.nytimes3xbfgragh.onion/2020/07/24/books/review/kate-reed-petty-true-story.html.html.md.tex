Sections

SEARCH

\protect\hyperlink{site-content}{Skip to
content}\protect\hyperlink{site-index}{Skip to site index}

\href{https://www.nytimes3xbfgragh.onion/section/books/review}{Book
Review}

\href{https://myaccount.nytimes3xbfgragh.onion/auth/login?response_type=cookie\&client_id=vi}{}

\href{https://www.nytimes3xbfgragh.onion/section/todayspaper}{Today's
Paper}

\href{/section/books/review}{Book Review}\textbar{}A Spellbinding Debut
Leaps Across Genres to Recreate the Confusion of Trauma

\url{https://nyti.ms/3jzdtgc}

\begin{itemize}
\item
\item
\item
\item
\item
\end{itemize}

Advertisement

\protect\hyperlink{after-top}{Continue reading the main story}

Supported by

\protect\hyperlink{after-sponsor}{Continue reading the main story}

Fiction

\hypertarget{a-spellbinding-debut-leaps-across-genres-to-recreate-the-confusion-of-trauma}{%
\section{A Spellbinding Debut Leaps Across Genres to Recreate the
Confusion of
Trauma}\label{a-spellbinding-debut-leaps-across-genres-to-recreate-the-confusion-of-trauma}}

\includegraphics{https://static01.graylady3jvrrxbe.onion/images/2020/07/26/books/review/26Abbott/26Abbott-articleLarge.jpg?quality=75\&auto=webp\&disable=upscale}

Buy Book ▾

\begin{itemize}
\tightlist
\item
  \href{https://www.amazon.com/gp/search?index=books\&tag=NYTBSREV-20\&field-keywords=True+Story+Kate+Reed+Petty}{Amazon}
\item
  \href{https://du-gae-books-dot-nyt-du-prd.appspot.com/buy?title=True+Story\&author=Kate+Reed+Petty}{Apple
  Books}
\item
  \href{https://www.anrdoezrs.net/click-7990613-11819508?url=https\%3A\%2F\%2Fwww.barnesandnoble.com\%2Fw\%2F\%3Fean\%3D9781984877680}{Barnes
  and Noble}
\item
  \href{https://www.anrdoezrs.net/click-7990613-35140?url=https\%3A\%2F\%2Fwww.booksamillion.com\%2Fp\%2FTrue\%2BStory\%2FKate\%2BReed\%2BPetty\%2F9781984877680}{Books-A-Million}
\item
  \href{https://bookshop.org/a/3546/9781984877680}{Bookshop}
\item
  \href{https://www.indiebound.org/book/9781984877680?aff=NYT}{Indiebound}
\end{itemize}

When you purchase an independently reviewed book through our site, we
earn an affiliate commission.

By Megan Abbott

\begin{itemize}
\item
  July 24, 2020
\item
  \begin{itemize}
  \item
  \item
  \item
  \item
  \item
  \end{itemize}
\end{itemize}

Four years ago, in her
\href{https://www.buzzfeednews.com/article/katiejmbaker/heres-the-powerful-letter-the-stanford-victim-read-to-her-ra}{powerful
and galvanizing victim impact statement}, Chanel Miller recalled the
horror she had felt upon hearing the testimony offered by Brock Turner,
the Stanford swimmer who had sexually assaulted her while she was
unconscious. Initially, Turner had portrayed the evening as an anonymous
drunken hookup. But at trial, a startling narrative emerged, one in
which the events unfolded like a ``poorly written young adult novel with
kissing and dancing and hand holding and lovingly tumbling onto the
ground,'' and culminated in consent.

For Miller, the story's fabrications were outrageous but so was its
form, framed as a tale of young romance --- and a poorly rendered one at
that. In her astonishing memoir,
``\href{https://www.nytimes3xbfgragh.onion/2019/09/24/books/review/chanel-miller-know-my-name.html}{Know
My Name,}'' she remained concerned with the unique shaping power of
story, of narrative, of genre. ``This is not the ultimate truth,'' she
wrote, ``but it is mine, told to the best of my ability.''

Miller's experience and others like it reverberate throughout Kate Reed
Petty's spellbinding debut novel, ``True Story,'' which focuses on the
rippling impact of an alleged assault following a raucous party in the
late 1990s.

The opening set piece brings us to a suburban Denny's as members of a
high school lacrosse team assemble for a post-mortem on the evening's
boozy mayhem. Two players arrive late with a story to tell about
offering a ride to a drunken ``private school girl.'' It's a story
Petty's characters will keep telling and retelling and reframing for the
next 16 years.

The dominant voice in the novel is that of Alice Lovett, who we soon
learn was the ``private school girl'' in question. Because she can't
remember what happened, she has only the crude tale the lacrosse players
shared --- one that spread throughout the community --- and remains
haunted by both the trauma of that night and the taunting she endured
after.

Now in her early 30s, Alice works as a ghostwriter, telling other
people's stories for a living. Her only connection to her nightmarish
high school experience is her old friend Haley Moreland, with whom she
shared a teenage love of horror movies. Haley urges Alice to finally go
public with her story. But nothing about Alice's story, or this novel,
is so simple.

``True Story'' unwinds through a variety of ``found'' and fashioned
narratives spanning nearly two decades that become a bricolage we
assemble ourselves. We read Alice and Haley's eighth-grade horror
screenplays, drafts of Alice's college admissions essays, complete with
vapid tutor comments (``I'm surprised you love horror movies. \ldots{} I
want to know \emph{why}. Can this essay go deeper?''), Alice's emails to
Haley after fleeing an abusive relationship. A pair of bookending
letters from Alice to Haley do a lot of the work to help us assemble
what happened that night, to get at the ``true story,'' even as the term
itself feels increasingly useless, deceptive.

The most incisive sections, however, are more traditional narratives
devoted to Nick Brothers, a teammate of the two lacrosse players who
took Alice home that night. When we meet him, he's a callow jock
enjoying the last gasps of high school sports prestige. At first,
whatever his teammates did to the ``private school girl'' is merely a
great, dirty story to him. When authorities fleetingly intervene, it
becomes a ``scare'' that soon evaporates as a tide of
\href{https://www.nytimes3xbfgragh.onion/2018/09/26/opinion/brett-kavanaugh-hearing-himpathy.html}{himpathy}
rises for the accused players. ``Things had turned out all right,'' Nick
tells us after the investigation is dropped. ``We had been through
something together, we agreed, and it had made us stronger.''

As the years pass, Nick's precarious masculinity erodes. The intense
partying of his teenage years has curdled into alcoholism and an overall
failure to launch. In one tour de force section, we join 26-year-old
Nick as he makes his way to a cabin in the woods for a ``lost weekend''
of marathon drinking. Evoking one of Kenneth Lonergan's broken and
arrested white men, he mourns his failures as he drives, savoring the
comfort of a \$70 bottle of bourbon between his legs to ``remind himself
of the reward on its way.'' As he approaches the cabin, however, we swap
genres, entering swiftly into the sinister foreboding of a thriller
before moving into body horror as Nick, over the course of two days,
undoes himself with drink and confusion.

Horror, suspense, confessional, epistolary tale, recovery memoir,
cautionary tale, even, late in the novel, paranoiac noir --- Petty leaps
from genre to genre with dizzying velocity. At first, it's jolting, but
slowly we begin to see how she's using shifting genres to show the way
trauma works on us, how it shapes our lived experience and the way we
frame that experience for others and for our own survival. It's a shell
game, and in a sense that's the way the novel operates, tantalizing us
with the ``truth'' about what really happened to Alice that night. We
look for clues embedded in the correspondence and confessions, in
Alice's cryptic emails, in Nick's muddled brain. We await revelations,
or at least a dark confirmation.

But despite its puzzle-box structure, ``True Story'' is not a mystery,
either --- at least not a traditional one with a gasp-inducing final
revelation. Initially, I found the resolution intellectually impressive
rather than narratively or emotionally satisfying. But after a day or
two, the book continued to work on me, spurring me to question my own
expectations of genre, and even story itself, and their capacity to get
at stickier truths about trauma and its reverberations and what we
expect from narratives dealing with sexual assault. What is
ideologically sound is not always narratively exciting, but is that a
failure of execution or a failure of genre conventions?

Ultimately, the novel's true twist is less about what unfurled that
fateful night than it is about form, voice, authorship. Alice's
experience was long ago erased --- by the young men who drove her home
that night and the teammates and school administrators who protected
them. They imposed a new narrative about an unstable young woman who
drinks herself into unconsciousness and threatens to ruin the futures of
promising young men. But it is not just the men. Alice also must fend
off her friend Haley, who, as a documentary filmmaker, attempts to
extract her story from her and weaponize it, insisting Alice tell it the
way victims' stories are supposed to be told to have a larger use.
``Victims exist in a society that tells us our purpose is to be an
inspiring story,'' Chanel Miller wrote. And that is one genre Alice
resists to its core. Her story is messy, full of horror and isolation,
unsent messages and stifled expression, manipulation and erasure. Alice
refuses to make the story useful. Instead, she makes it hers.

Advertisement

\protect\hyperlink{after-bottom}{Continue reading the main story}

\hypertarget{site-index}{%
\subsection{Site Index}\label{site-index}}

\hypertarget{site-information-navigation}{%
\subsection{Site Information
Navigation}\label{site-information-navigation}}

\begin{itemize}
\tightlist
\item
  \href{https://help.nytimes3xbfgragh.onion/hc/en-us/articles/115014792127-Copyright-notice}{©~2020~The
  New York Times Company}
\end{itemize}

\begin{itemize}
\tightlist
\item
  \href{https://www.nytco.com/}{NYTCo}
\item
  \href{https://help.nytimes3xbfgragh.onion/hc/en-us/articles/115015385887-Contact-Us}{Contact
  Us}
\item
  \href{https://www.nytco.com/careers/}{Work with us}
\item
  \href{https://nytmediakit.com/}{Advertise}
\item
  \href{http://www.tbrandstudio.com/}{T Brand Studio}
\item
  \href{https://www.nytimes3xbfgragh.onion/privacy/cookie-policy\#how-do-i-manage-trackers}{Your
  Ad Choices}
\item
  \href{https://www.nytimes3xbfgragh.onion/privacy}{Privacy}
\item
  \href{https://help.nytimes3xbfgragh.onion/hc/en-us/articles/115014893428-Terms-of-service}{Terms
  of Service}
\item
  \href{https://help.nytimes3xbfgragh.onion/hc/en-us/articles/115014893968-Terms-of-sale}{Terms
  of Sale}
\item
  \href{https://spiderbites.nytimes3xbfgragh.onion}{Site Map}
\item
  \href{https://help.nytimes3xbfgragh.onion/hc/en-us}{Help}
\item
  \href{https://www.nytimes3xbfgragh.onion/subscription?campaignId=37WXW}{Subscriptions}
\end{itemize}
