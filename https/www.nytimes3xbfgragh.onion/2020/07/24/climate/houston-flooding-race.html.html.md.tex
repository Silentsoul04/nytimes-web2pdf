Sections

SEARCH

\protect\hyperlink{site-content}{Skip to
content}\protect\hyperlink{site-index}{Skip to site index}

\href{https://www.nytimes3xbfgragh.onion/section/climate}{Climate}

\href{https://myaccount.nytimes3xbfgragh.onion/auth/login?response_type=cookie\&client_id=vi}{}

\href{https://www.nytimes3xbfgragh.onion/section/todayspaper}{Today's
Paper}

\href{/section/climate}{Climate}\textbar{}A Climate Plan in Texas
Focuses on Minorities. Not Everyone Likes It.

\href{https://nyti.ms/3fXXQN0}{https://nyti.ms/3fXXQN0}

\begin{itemize}
\item
\item
\item
\item
\item
\item
\end{itemize}

\href{https://www.nytimes3xbfgragh.onion/section/climate?action=click\&pgtype=Article\&state=default\&region=TOP_BANNER\&context=storylines_menu}{Climate
and Environment}

\begin{itemize}
\tightlist
\item
  \href{https://www.nytimes3xbfgragh.onion/2020/07/30/climate/sea-level-inland-floods.html?action=click\&pgtype=Article\&state=default\&region=TOP_BANNER\&context=storylines_menu}{Rising
  Seas}
\item
  \href{https://www.nytimes3xbfgragh.onion/interactive/2020/climate/trump-environment-rollbacks.html?action=click\&pgtype=Article\&state=default\&region=TOP_BANNER\&context=storylines_menu}{Trump's
  Changes}
\item
  \href{https://www.nytimes3xbfgragh.onion/interactive/2020/04/19/climate/climate-crash-course-1.html?action=click\&pgtype=Article\&state=default\&region=TOP_BANNER\&context=storylines_menu}{Climate
  101}
\item
  \href{https://www.nytimes3xbfgragh.onion/interactive/2018/08/30/climate/how-much-hotter-is-your-hometown.html?action=click\&pgtype=Article\&state=default\&region=TOP_BANNER\&context=storylines_menu}{Is
  Your Hometown Hotter?}
\item
  \href{https://www.nytimes3xbfgragh.onion/newsletters/climate-change?action=click\&pgtype=Article\&state=default\&region=TOP_BANNER\&context=storylines_menu}{Newsletter}
\end{itemize}

Advertisement

\protect\hyperlink{after-top}{Continue reading the main story}

Supported by

\protect\hyperlink{after-sponsor}{Continue reading the main story}

\hypertarget{a-climate-plan-in-texas-focuses-on-minorities-not-everyone-likes-it}{%
\section{A Climate Plan in Texas Focuses on Minorities. Not Everyone
Likes
It.}\label{a-climate-plan-in-texas-focuses-on-minorities-not-everyone-likes-it}}

For years, money for flood protection in the Houston area went mostly to
richer neighborhoods. A new approach prioritizes minority communities,
and it's stirring up resentments.

\includegraphics{https://static01.graylady3jvrrxbe.onion/images/2020/07/17/climate/00CLI-HOUSTON1/merlin_174606915_c1a2be97-673c-4b8e-9d53-642643a969fa-articleLarge.jpg?quality=75\&auto=webp\&disable=upscale}

\href{https://www.nytimes3xbfgragh.onion/by/christopher-flavelle}{\includegraphics{https://static01.graylady3jvrrxbe.onion/images/2019/06/28/climate/author-chris-flavelle/author-chris-flavelle-thumbLarge-v3.png}}

By
\href{https://www.nytimes3xbfgragh.onion/by/christopher-flavelle}{Christopher
Flavelle}

\begin{itemize}
\item
  July 24, 2020
\item
  \begin{itemize}
  \item
  \item
  \item
  \item
  \item
  \item
  \end{itemize}
\end{itemize}

Pleasantville, a few square miles of bungalows and industrial sites
stuck between Houston's railways and freeways, resembles a shallow bowl
--- quick to flood, like much of the city. But like other neighborhoods
with large Black and Hispanic populations and low property values, it
never qualified for the pricey flood-control projects that protect
wealthier parts of Houston.

Projects here ``would be put on a list, and that's where they would go
to die,'' said Bridgette Murray, who is president of the Pleasantville
neighborhood association and whose house got five feet of water during
Hurricane Harvey.

Faced with countless complaints like these, officials in Harris County,
which manages flood control in and around Houston, threw out their old
approach for spending billions of dollars on flood defenses after
Harvey. Instead of prioritizing spending to protect the most valuable
property, which benefited wealthier and whiter areas, they decided to
instead prioritize disadvantaged neighborhoods that would have the
hardest time recovering, including communities of color.

Opponents have criticized the program as social engineering. Advocates
have lauded it as long overdue. And for flood-prone cities nationwide,
the controversial plan has become a test case for grappling with the
overlapping challenges of racial inequity and climate change.

Governments have long used a simple concept, cost-benefit analysis, to
decide where to focus money on flood protection: Spend it where property
values are higher, for the best return on investment. However, that puts
poorer minority areas at a disadvantage. And it feeds a cycle of decline
as flooding returns again and again.

``The status quo wasn't working,'' said Lina Hidalgo, who in 2018 became
the first Latina to win the position of county judge, Harris County's
top elected official.

But the county's year-old experiment is more than a possible model for
other communities. It is also a warning about the political resistance
that can result.

\includegraphics{https://static01.graylady3jvrrxbe.onion/images/2020/07/22/climate/00CLI-HOUSTONFLOODING9/merlin_174817902_c5a32bd8-e4d2-4e19-8f1b-fd92bd19f2a1-articleLarge.jpg?quality=75\&auto=webp\&disable=upscale}

Image

Lina Hidalgo, the first Latina elected county judge for Harris County,
Texas. ``We know the status quo isn't working,'' she
said.Credit...Sergio Flores for The New York Times

Democrats' plan thrust into plain view the wealth and racial divides
that long influenced which communities received flood assistance.

Community groups supporting the changes call them both necessary and
humane. Opponents see the new program simply as a way for Democrats to
channel public funds to Democratic leaning voters, rather than using an
approach they prefer, called ``worst first'' --- the idea that the first
priority for spending should be places facing the worst flood risk.

\href{https://www.nytimes3xbfgragh.onion/section/climate?action=click\&pgtype=Article\&state=default\&region=MAIN_CONTENT_1\&context=storylines_keepup}{}

\hypertarget{climate-and-environment-}{%
\subsubsection{Climate and Environment
›}\label{climate-and-environment-}}

\hypertarget{keep-up-on-the-latest-climate-news}{%
\paragraph{Keep Up on the Latest Climate
News}\label{keep-up-on-the-latest-climate-news}}

Updated July 30, 2020

Here's what you need to know about the latest climate change news this
week:

\begin{itemize}
\item
  \begin{itemize}
  \tightlist
  \item
    \href{https://www.nytimes3xbfgragh.onion/2020/07/30/climate/bangladesh-floods.html?action=click\&pgtype=Article\&state=default\&region=MAIN_CONTENT_1\&context=storylines_keepup}{Floods
    in}\href{https://www.nytimes3xbfgragh.onion/2020/07/30/climate/bangladesh-floods.html?action=click\&pgtype=Article\&state=default\&region=MAIN_CONTENT_1\&context=storylines_keepup}{Bangladesh}
    are punishing the people least responsible for climate change.
  \item
    As climate change raises sea levels,
    \href{https://www.nytimes3xbfgragh.onion/2020/07/30/climate/sea-level-inland-floods.html?action=click\&pgtype=Article\&state=default\&region=MAIN_CONTENT_1\&context=storylines_keepup}{storm
    surges and high tides} are likely to push farther inland.
  \item
    The E.P.A. inspector general plans to investigate whether a rollback
    of fuel efficiency standards
    \href{https://www.nytimes3xbfgragh.onion/2020/07/27/climate/trump-fuel-efficiency-rule.html?action=click\&pgtype=Article\&state=default\&region=MAIN_CONTENT_1\&context=storylines_keepup}{violated
    government rules}.
  \end{itemize}
\end{itemize}

``They want the money for their neighborhoods. They don't care about
ours,'' said Dave Martin, who is Houston's mayor pro tem and represents
the wealthy community of Kingwood on the City Council. ``Using any
mechanism other than worst first is ludicrous.''

In Harris County, like elsewhere, the emphasis on prioritizing
higher-value property led to different levels of protection not just
from big federally funded flood projects, but also basic infrastructure.

Wealthy neighborhoods got sidewalks and curbs with gutters and
underground drainage, while poorer areas still rely on open ditches in
front of their houses, according to Tracy Stephens, a former project
inspector for Houston's public works department who now works with Ms.
Murray at a community development group called Achieving Community Tasks
Successfully.

Mr. Stephens recalled crews being sent to upgrade gutters in upscale
neighborhoods simply to accommodate the extra runoff as more people
installed backyard swimming pools. ``That money could have been spent
doing jobs everywhere else,'' he said.

Environmental policy experts say it makes no sense to decide which
people get protection based on which property is more valuable. That
approach reinforces historical discrimination, which contributed to
minority neighborhoods having lower property values in the first place.
And it doesn't address the deeper question of who needs the most help,
or why.

``The benefit-cost approach has a false transparency, a false rigor,''
said Earthea Nance, an associate professor of urban planning and
environmental policy at Texas Southern University. That approach has a
similar effect to redlining, she said, referring to the practice in
decades past whereby governments and banks would deny mortgages to Black
home buyers. ``Is that really what we want?''

A sequence of unlikely events pushed Harris County to reconsider its
approach. First, in 2017, Hurricane Harvey dropped more rain than any
storm in U.S. history, flooding more than 166,000 homes countywide. The
following summer, voters approved a \$2.5 billion bond to fund more than
500 flood-control projects over several years, the largest such
initiative in the county's history.

Image

Chrishelle Palay with her daughters Faith and Skye. She helped bring
about the recent changes to how the county prioritizes its
flood-protection money.Credit...Sergio Flores for The New York Times

Image

Rainwater collected this month near apartments in the Kashmere Gardens
neighborhood of Houston.Credit...Sergio Flores for The New York Times

A few months later, in November 2018, came a third surprise. For the
first time in three decades, Democrats, buoyed by the county's changing
demographics and their party's midterm wave,
\href{https://www.houstonchronicle.com/news/houston-texas/houston/article/Emmett-threatened-by-Democratic-rout-in-Harris-13369367.php}{won
control} of the Harris County commission, called the Commissioners
Court, and with it, the chance to decide just how that \$2.5 billion
would be spent.

They vowed the focus would be on fairness. The problem was, nobody knew
exactly what that meant.

Rodney Ellis, who until the election of 2018 had been the sole Democrat
on the county's five-member commission, had said the bond measure must
include a commitment to the ``equitable expenditure'' of the money in
return for his support. At the time, he avoided talking about how that
provision should be interpreted, for fear Republicans would reject it.

``Everybody's for equity, until they're against it,'' Mr. Ellis said.
``Everybody's for fairness, until they find out everybody won't get what
they want.''

After Democrats took control of the commission, they eventually decided
to rank projects based in part on the ``social vulnerability'' of the
communities they protected --- an
\href{https://www.ncbi.nlm.nih.gov/pmc/articles/PMC7179070/}{index}
created by the U.S. Centers for Disease Control and Prevention that
reflects what share of residents are minorities, can't speak English,
lack a job, are older, live in mobile homes, don't have cars or face
other challenges.

The goal, according to Ms. Hidalgo, was to reflect how hard it would be
for a neighborhood to recover from the next disaster, and prioritize
flood-control projects in those areas --- what she described as a more
comprehensive version of the worst-first approach. ``That means
elevating some of the communities that had gone overlooked,'' she said.

The commission passed that new approach along party lines, which in
Harris County also means racial lines. The three Democrats who voted in
favor are African-American or Hispanic, while the two Republicans who
voted against it are white.

Jack Cagle, one of the Republican commissioners who voted against the
measure, praised the county's flood-control department for working
quickly on all the bond-funded projects over the past year, blunting the
effect of the new prioritization. But he said his voters feel misled,
after supporting a bond that they thought would focus on physical risk.

``If you voted on a premise of worst first, and now you're being told,
look, go to the end of the line, you could be unworthy --- you're going
to get some pushback from that,'' Mr. Cagle said.

Twenty-five miles north of Pleasantville, in the wealthy neighborhood of
Kingwood at the edge of Lake Houston, Beth Guide's house flooded last
year. When the county said it would prioritize flood-control projects
based in part on social vulnerability, she objected. The only criterion,
she said, should be who faces the greatest flood risk.

``I don't care if your house is a million-dollar house or a \$30,000
hovel in the middle of nowhere,'' said Ms. Guide, who runs a
digital-marketing agency. ``This literally should be, `Whose life is in
the most danger?'''

She rejected the idea that priority should go to people who would have
the hardest time bouncing back from a disaster. ``The fact that you
decide that you want to have a Netflix account versus whether you want
to pay for your flood insurance is not my problem,'' Ms. Guide said.

Image

Beth Guide, a resident of Kingwood in northeast Houston whose house
flooded last year, opposes prioritizing flood-control projects based on
social vulnerability.Credit...Sergio Flores for The New York Times

Image

The Buffalo Bayou, part of the Briar Forest neighborhood of
Houston.Credit...Sergio Flores for The New York Times

Greg Travis is a Republican member of Houston's city council for a
wealthy district that straddles Buffalo Bayou, a waterway whose
flood-control projects have mostly been pushed to the end of the queue
under the new system. He said the Democrats' approach endangers the
region's tax revenues by letting flooding continue to threaten the
values of his constituents' homes.

``We are the goose that lays the golden egg,'' Mr. Travis said. ``If
those falter, then the city falters, because there's no other district
that can pick up the slack. We pay for most of the social programs in
our city.''

When some community groups listened to those arguments, they heard
something other than competing visions of the public good. They heard
complaints about priority going to people with the wrong color skin.

``This is the same public investment that's been going to whiter and
more prominent areas for decades,'' said Chrishelle Palay, who leads the
Houston Organizing Movement for Equity, a group that sought the change.
``They just call it their `tax dollars hard at work.'''

Some supporters and opponents of the new approach agree on one thing: If
Harris County voters had known that the bond money would be prioritized
this way, they might not have passed it.

``It would have failed miserably,'' said Mr. Martin, Houston's mayor pro
tem.

``I actually agree'' with Mr. Martin, said Ms. Palay. ``His constituents
who supported this measure would likely vote differently if they knew
the current equity framework.''

Dr. Nance, of Texas Southern University, said that until recently, the
white and wealthier residents of Harris County had what she called ``the
privilege of not knowing'' about the disparity of flood protection. Now
that the flood bond has forced people to confront that disparity, and
the benefits they derive from it, she said pushback was inevitable.
``When they have to face that,'' Dr. Nance added, ``they're not going to
be hugging you.''

Image

A spillway in Kashmere Gardens.Credit...Sergio Flores for The New York
Times

Advertisement

\protect\hyperlink{after-bottom}{Continue reading the main story}

\hypertarget{site-index}{%
\subsection{Site Index}\label{site-index}}

\hypertarget{site-information-navigation}{%
\subsection{Site Information
Navigation}\label{site-information-navigation}}

\begin{itemize}
\tightlist
\item
  \href{https://help.nytimes3xbfgragh.onion/hc/en-us/articles/115014792127-Copyright-notice}{©~2020~The
  New York Times Company}
\end{itemize}

\begin{itemize}
\tightlist
\item
  \href{https://www.nytco.com/}{NYTCo}
\item
  \href{https://help.nytimes3xbfgragh.onion/hc/en-us/articles/115015385887-Contact-Us}{Contact
  Us}
\item
  \href{https://www.nytco.com/careers/}{Work with us}
\item
  \href{https://nytmediakit.com/}{Advertise}
\item
  \href{http://www.tbrandstudio.com/}{T Brand Studio}
\item
  \href{https://www.nytimes3xbfgragh.onion/privacy/cookie-policy\#how-do-i-manage-trackers}{Your
  Ad Choices}
\item
  \href{https://www.nytimes3xbfgragh.onion/privacy}{Privacy}
\item
  \href{https://help.nytimes3xbfgragh.onion/hc/en-us/articles/115014893428-Terms-of-service}{Terms
  of Service}
\item
  \href{https://help.nytimes3xbfgragh.onion/hc/en-us/articles/115014893968-Terms-of-sale}{Terms
  of Sale}
\item
  \href{https://spiderbites.nytimes3xbfgragh.onion}{Site Map}
\item
  \href{https://help.nytimes3xbfgragh.onion/hc/en-us}{Help}
\item
  \href{https://www.nytimes3xbfgragh.onion/subscription?campaignId=37WXW}{Subscriptions}
\end{itemize}
