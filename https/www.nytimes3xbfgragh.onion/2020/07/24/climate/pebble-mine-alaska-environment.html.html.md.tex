Sections

SEARCH

\protect\hyperlink{site-content}{Skip to
content}\protect\hyperlink{site-index}{Skip to site index}

\href{/section/climate}{Climate}\textbar{}Gold vs. Salmon: An Alaska
Mine Project Just Got a Boost

\href{https://nyti.ms/2CA00UV}{https://nyti.ms/2CA00UV}

\begin{itemize}
\item
\item
\item
\item
\item
\item
\end{itemize}

\href{https://www.nytimes3xbfgragh.onion/section/climate?action=click\&pgtype=Article\&state=default\&region=TOP_BANNER\&context=storylines_menu}{Climate
and Environment}

\begin{itemize}
\tightlist
\item
  \href{https://www.nytimes3xbfgragh.onion/2020/07/30/climate/sea-level-inland-floods.html?action=click\&pgtype=Article\&state=default\&region=TOP_BANNER\&context=storylines_menu}{Rising
  Seas}
\item
  \href{https://www.nytimes3xbfgragh.onion/interactive/2020/climate/trump-environment-rollbacks.html?action=click\&pgtype=Article\&state=default\&region=TOP_BANNER\&context=storylines_menu}{Trump's
  Changes}
\item
  \href{https://www.nytimes3xbfgragh.onion/interactive/2020/04/19/climate/climate-crash-course-1.html?action=click\&pgtype=Article\&state=default\&region=TOP_BANNER\&context=storylines_menu}{Climate
  101}
\item
  \href{https://www.nytimes3xbfgragh.onion/interactive/2018/08/30/climate/how-much-hotter-is-your-hometown.html?action=click\&pgtype=Article\&state=default\&region=TOP_BANNER\&context=storylines_menu}{Is
  Your Hometown Hotter?}
\item
  \href{https://www.nytimes3xbfgragh.onion/newsletters/climate-change?action=click\&pgtype=Article\&state=default\&region=TOP_BANNER\&context=storylines_menu}{Newsletter}
\end{itemize}

\includegraphics{https://static01.graylady3jvrrxbe.onion/images/2020/07/24/climate/24cli-pebblemine-12/24cli-pebblemine-12-articleLarge.jpg?quality=75\&auto=webp\&disable=upscale}

\hypertarget{gold-vs-salmon-an-alaska-mine-project-just-got-a-boost}{%
\section{Gold vs. Salmon: An Alaska Mine Project Just Got a
Boost}\label{gold-vs-salmon-an-alaska-mine-project-just-got-a-boost}}

The Trump Administration, rejecting concerns over the risks to Alaska's
fishery, cleared the way on Friday for the Pebble Mine.

A work camp on the tundra near the site of the proposed Pebble Mine in
Southwest Alaska.Credit...

Supported by

\protect\hyperlink{after-sponsor}{Continue reading the main story}

By \href{https://www.nytimes3xbfgragh.onion/by/henry-fountain}{Henry
Fountain}

Photographs by Acacia Johnson

\begin{itemize}
\item
  Published July 24, 2020Updated July 27, 2020
\item
  \begin{itemize}
  \item
  \item
  \item
  \item
  \item
  \item
  \end{itemize}
\end{itemize}

From the air it looks like just another tract of Alaska's endless,
roadless tundra, pockmarked with lakes and ponds, with a scattering of
some of the state's craggy mountains.

But this swath of land, home to foraging bears and spawning salmon about
200 miles southwest of Anchorage, has been a battleground for years.

Image

A brown bear catching salmon in Katmai National Park and Preserve, 30
miles from Bristol Bay.

Image

Hundreds of thousands of sockeye salmon head upstream in the park to
spawn.

The fight is over what lies just below the surface: one of the richest
deposits of copper, gold and other valuable metals in the world. It sets
two of the state's most important industries, mining and fishing,
against each other.

A mining company plans to dig a pit, more than a mile square and a third
of a mile deep, over two decades to obtain the metals, estimated to be
worth at least \$300 billion.

Supporters say the project, known as the
\href{https://pebblepartnership.com/project-overview}{Pebble Mine},
would be an economic boost for a remote region that has missed out on
the North Slope oil boom and other resource-extraction development in
the state over the past half century. It would employ nearly 1,000
people, and the Canada-based company,
\href{https://www.northerndynastyminerals.com/}{Northern Dynasty
Minerals}, would pay for infrastructure improvements in some Alaska
Native villages and provide cash dividends totaling at least \$3 million
to people in the area.

CANADA

ALASKA

Anchorage

ALASKA

Lake ClarK

National Park

and Preserve

Detail

area

Juneau

Bristol

Bay

Gulf of Alaska

200 mileS

ROAD

AND PIPELINE

ROUTE

Nondalton

ILIAMNA

VOLCANO

PROPOSED PEBBLE MINE SITE

Iliamna

Pedro

Bay

DIAMOND

POINT

Newhalen

GAS PIPELINE

ROUTE

Iliamna Lake

Cook Inlet

Dillingham

KATMAI

National Park

and Preserve

20 miles

CANADA

ALASKA

Anchorage

ALASKA

Lake ClarK

National Park

and Preserve

Detail

area

Juneau

Bristol

Bay

Gulf of Alaska

200 mileS

ROAD

AND PIPELINE

ROUTE

Nondalton

ILIAMNA

VOLCANO

PROPOSED PEBBLE MINE SITE

Iliamna

Pedro

Bay

DIAMOND

POINT

Newhalen

GAS PIPELINE

ROUTE

Iliamna Lake

Cook Inlet

Dillingham

KATMAI

National Park

and Preserve

20 miles

By The New York Times

But opposition has long been widespread,
\href{https://www.juneauempire.com/news/pebble-mine-opposition-wrong-mine-for-the-wrong-place/}{both
in the region and statewide,} with concerns about environmental damage
and the potential for harming another critical resource: salmon. The
fish is the main traditional subsistence food for many of the Alaska
Natives in the region and the basis of both a thriving sport-fishing
industry and, in nearby Bristol Bay, one of the largest commercial
\href{https://www.bbrsda.com/}{wild salmon fisheries} in the world.

The mine will be located in two watersheds that feed fish-spawning
rivers. Opponents say tailings left from the mining operation pose risks
if heavy metals or other contaminants from them leach into groundwater
or if dams holding back the tailings fail in an earthquake.

Tom Collier, the chief executive of Pebble Partnership, the Northern
Dynasty subsidiary developing the project, said the mine was designed to
minimize those and other risks.

\includegraphics{https://static01.graylady3jvrrxbe.onion/images/2020/07/24/climate/24cli-pebblemine-05/24cli-pebblemine-05-articleLarge.jpg?quality=75\&auto=webp\&disable=upscale}

Image

Percy Urban comes from Miami to fish near Dillingham in the summer.

Image

Dillingham, population 2,400, is only reachable by air or water.

The deposit was discovered in the late 1980s, and planning for a mine
began in earnest about 15 years ago. It drew opposition from leaders in
both parties from the start, as battle lines between mining and fishing
were established.
\href{https://www.nytimes3xbfgragh.onion/2008/10/22/us/politics/22mining.html}{But
the project was aided by the pro-mining stance of the governor at the
time, Sarah Palin.}

Under President Barack Obama, the project was blocked in 2014 by the
Environmental Protection Agency, largely over concerns about the risks
to salmon.

Image

Rick Delkittie Sr., an Alaska Native, heads to his fishing camp near
Nondalton, a village of 130 people that is the closest one to the mine
site. Mr. Delkittie is opposed to the project.

But the Pebble Mine gained new momentum under President Trump's more
industry-friendly policies. While at first continuing its criticism of
the project, the Environmental Protection Agency eventually reversed the
Obama-era decision blocking it.

On Friday, \href{https://pebbleprojecteis.com/documents/finaleis}{the
Army Corps of Engineers issued a final environmental impact statement},
or E.I.S., for the project. Under normal operations, the Corps wrote,
the project would not result in ``long-term changes in the health of the
commercial fisheries in Bristol Bay.''

\href{https://www.nytimes3xbfgragh.onion/section/climate?action=click\&pgtype=Article\&state=default\&region=MAIN_CONTENT_1\&context=storylines_keepup}{}

\hypertarget{climate-and-environment-}{%
\subsubsection{Climate and Environment
›}\label{climate-and-environment-}}

\hypertarget{keep-up-on-the-latest-climate-news}{%
\paragraph{Keep Up on the Latest Climate
News}\label{keep-up-on-the-latest-climate-news}}

Updated July 30, 2020

Here's what you need to know about the latest climate change news this
week:

\begin{itemize}
\item
  \begin{itemize}
  \tightlist
  \item
    \href{https://www.nytimes3xbfgragh.onion/2020/07/30/climate/bangladesh-floods.html?action=click\&pgtype=Article\&state=default\&region=MAIN_CONTENT_1\&context=storylines_keepup}{Floods
    in}\href{https://www.nytimes3xbfgragh.onion/2020/07/30/climate/bangladesh-floods.html?action=click\&pgtype=Article\&state=default\&region=MAIN_CONTENT_1\&context=storylines_keepup}{Bangladesh}
    are punishing the people least responsible for climate change.
  \item
    As climate change raises sea levels,
    \href{https://www.nytimes3xbfgragh.onion/2020/07/30/climate/sea-level-inland-floods.html?action=click\&pgtype=Article\&state=default\&region=MAIN_CONTENT_1\&context=storylines_keepup}{storm
    surges and high tides} are likely to push farther inland.
  \item
    The E.P.A. inspector general plans to investigate whether a rollback
    of fuel efficiency standards
    \href{https://www.nytimes3xbfgragh.onion/2020/07/27/climate/trump-fuel-efficiency-rule.html?action=click\&pgtype=Article\&state=default\&region=MAIN_CONTENT_1\&context=storylines_keepup}{violated
    government rules}.
  \end{itemize}
\end{itemize}

In addition to the open-pit mine, the plan would include large dammed
ponds for the tailings, some of them toxic, that result from mining and
concentrating the metals, 80 miles of road and pipeline to carry the
concentrate to a new port on Cook Inlet, and a 165-mile natural gas
pipeline for a generating plant to power the operation.

In an interview this week, Mr. Collier described the release of the
final impact statement as ``the most significant day in the 15-odd-year
history of the Pebble project.''

Image

Diamond Point, where a port would be built to ship metals from the mine,
80 miles away.

``It's really for the first time that a federal agency has conducted a
rigorous scientific review of the specific project Pebble wants to
build,'' he said. The conclusion that the mine was not going to damage
the salmon fishery would be ``unequivocal,'' he added.

But in public comments on a draft of the environmental impact statement
last year, opponents suggested that the review was not so rigorous. They
pointed to numerous hazardous risks, including the potential for a
tailings dam failure that could contaminate waterways used by spawning
fish and harm the Bristol Bay fishery, which employs about 15,000
people.

Alaska is the most seismically active state in the nation, and critics
said the Corps of Engineers had not taken sufficient account of the risk
of earthquakes or volcanic activity, and that its analysis of the dam
designs was inadequate. Some of the dams would be hundreds of feet high.

Tailing dam failures can unleash a sudden flood of contaminated slurry
with disastrous effects. A 2019 failure at an iron mine in Brazil, for
example, killed more than 250 people. Given the Pebble Mine's remote
location, the risk to people might be low, but the heavy metals and
other contaminants could make nearby rivers toxic to fish.

Image

Mount Iliamna, about 80 miles east of the mine site, is one of many
volcanoes in Southwest Alaska.

This year, after the Corps sent a preliminary version of the final
impact statement to federal and state agencies and other groups, the
critiques continued, according to documents obtained by opponents of the
project. U.S. Fish and Wildlife Service scientists, for example, wrote
that the version failed to acknowledge that habitat destruction from
development of the mine ``would erode the portfolio of habitat diversity
and associated life history diversity that stabilize annual salmon
returns to the Bristol Bay region.''

At a news briefing this week, David Hobbie, chief of the Corps' Alaska
district regulatory division, said, ``We've done our best to address all
the comments we've received.''

In a month or perhaps longer, the Corps will make a final decision on
whether to allow the project to proceed. Approval is expected.

That will almost certainly not be the end of the story, however.

Even after the Corps' latest review, ``the E.I.S. is so lacking and
thoroughly inadequate, I anticipate legal challenges,'' said Brian
Litmans, legal director of Trustees for Alaska, a nonprofit public
interest law firm.

The project will require more permits, mostly from the state, which
could take three years to obtain. And should President Trump lose
re-election, a Democratic administration could move to block the project
once again.

In Alaska, statewide public opinion polls have consistently shown more
opposition than support, and locally the anti-mine feelings are even
stronger. ``Opposition is overwhelming throughout the bay,'' Mr. Litmans
said.

Opponents are focusing on an 11th-hour change to one aspect of the
project. In May, the Corps announced it had changed its determination of
what is called the ``least environmentally damaging practicable
alternative'' for the transportation route between the mine and Cook
Inlet.

The company and the Corps had both favored a route that included a ferry
crossing of Iliamna Lake, one of the largest in the United States. But
after hearing concerns about the potential impact on winter travel and
seal hunting on the lake, the Corps now says a land-only route, along
the northern edge of the lake, is the preferred one, although it could
destroy several thousand acres of wetlands.

Image

Brown bear trails create intricate mazes through the sedge grass, an
abundant food source for the bears, in Lake Clark National Park and
Preserve, Alaska.

Image

A brown bear and her three spring cubs in the salt marsh of Chinitna
Bay.

The Bristol Bay Native Corporation, one of 13 regional corporations
established in the 1970s in the settlement of Native claims to Alaska's
lands, owns subsurface rights on land that the route would cross.

``We believe the subsurface will be impacted'' by construction of a road
and pipelines, said Daniel Cheyette, the corporation's vice president
for land and resources. ``We've not given Pebble permission to utilize
those or impact those.''

As to whether the corporation might negotiate on the issue, Mr. Cheyette
said that while he could not speak for his board of directors, ``I
believe that is nonnegotiable.''

``We've been fighting this for a long time and will continue to fight
it,'' he said.

Other Alaska Native groups, including the village corporation of Pedro
Bay on Iliamna Lake, also plan to withhold access to their lands.

But not all Native groups are opposed to the project.

Image

Sue Anelon, left, and Lisa Reimers in Anchorage. The two sisters are
involved with the Native corporation of the village of Iliamna, which
supports the mine project.

Image

Iliamna Lake, with the village of Iliamna, population 150. The village
corporation has come to an agreement with the mine's developer to allow
a transportation route across its land.

A consortium of five village corporations in the area expects to become
a transportation contractor for the mine. And the village corporation
for Iliamna, about 20 miles from the mine site, has already negotiated
with the developer to allow access to 68,000 acres of land it owns.

``We don't see Pebble damaging the area like everybody claims,'' said
Lisa Reimers, a board member of the corporation, Iliamna Natives Ltd.
``Pebble has to do it right because there are so many people watching
them.''

Ms. Reimers was raised in Iliamna, in a house that had no running water
or electricity until she was 12. Her parents, she said, ``wanted the
best for their family and for their grandkids today.''

``They didn't see Iliamna surviving without a project like Pebble.''

Advertisement

\protect\hyperlink{after-bottom}{Continue reading the main story}

\hypertarget{site-index}{%
\subsection{Site Index}\label{site-index}}

\hypertarget{site-information-navigation}{%
\subsection{Site Information
Navigation}\label{site-information-navigation}}

\begin{itemize}
\tightlist
\item
  \href{https://help.nytimes3xbfgragh.onion/hc/en-us/articles/115014792127-Copyright-notice}{©~2020~The
  New York Times Company}
\end{itemize}

\begin{itemize}
\tightlist
\item
  \href{https://www.nytco.com/}{NYTCo}
\item
  \href{https://help.nytimes3xbfgragh.onion/hc/en-us/articles/115015385887-Contact-Us}{Contact
  Us}
\item
  \href{https://www.nytco.com/careers/}{Work with us}
\item
  \href{https://nytmediakit.com/}{Advertise}
\item
  \href{http://www.tbrandstudio.com/}{T Brand Studio}
\item
  \href{https://www.nytimes3xbfgragh.onion/privacy/cookie-policy\#how-do-i-manage-trackers}{Your
  Ad Choices}
\item
  \href{https://www.nytimes3xbfgragh.onion/privacy}{Privacy}
\item
  \href{https://help.nytimes3xbfgragh.onion/hc/en-us/articles/115014893428-Terms-of-service}{Terms
  of Service}
\item
  \href{https://help.nytimes3xbfgragh.onion/hc/en-us/articles/115014893968-Terms-of-sale}{Terms
  of Sale}
\item
  \href{https://spiderbites.nytimes3xbfgragh.onion}{Site Map}
\item
  \href{https://help.nytimes3xbfgragh.onion/hc/en-us}{Help}
\item
  \href{https://www.nytimes3xbfgragh.onion/subscription?campaignId=37WXW}{Subscriptions}
\end{itemize}
