Sections

SEARCH

\protect\hyperlink{site-content}{Skip to
content}\protect\hyperlink{site-index}{Skip to site index}

At the Olympics in Bombed-Out London, She Forever Changed Women's Sports

\url{https://nyti.ms/2Bq3vwo}

\begin{itemize}
\item
\item
\item
\item
\item
\item
\end{itemize}

\includegraphics{https://static01.graylady3jvrrxbe.onion/images/2020/07/24/multimedia/24ww2-1948olympics-08/24ww2-1948olympics-08-articleLarge.jpg?quality=75\&auto=webp\&disable=upscale}

Beyond the World War II We know

\hypertarget{at-the-olympics-in-bombed-out-london-she-forever-changed-womens-sports}{%
\section{At the Olympics in Bombed-Out London, She Forever Changed
Women's
Sports}\label{at-the-olympics-in-bombed-out-london-she-forever-changed-womens-sports}}

The 1948 Summer Games, the first held after the war, were a celebration
of improvisation, renewal and change, embodied in a Dutch track star
named Fanny Blankers-Koen.

Fanny Blankers-Koen crossing the finish line in 11.9 seconds to win the
Women's 100-meter final in the Summer Olympic Games in
1948.Credit...Associated Press

Supported by

\protect\hyperlink{after-sponsor}{Continue reading the main story}

By \href{https://www.nytimes3xbfgragh.onion/by/jere-longman}{Jeré
Longman}

\begin{itemize}
\item
  July 24, 2020
\item
  \begin{itemize}
  \item
  \item
  \item
  \item
  \item
  \item
  \end{itemize}
\end{itemize}

\emph{\emph{\emph{The latest article from
``}\href{https://www.nytimes3xbfgragh.onion/spotlight/beyond-wwii}{\emph{Beyond
the World War II We Know}}},'' a series from The Times that documents
lesser-known stories from the war, recounts the Austerity Olympics, the
first Games to take place in 12 years and the track-and-field star who
changed the perception of women's athletics.}**

When the Dutch track star Fanny Blankers-Koen appeared at the 1948
London Olympics, soon to become the first woman to win
\href{https://www.youtube.com/watch?v=_Y_G0H-b3QE}{four gold medals} at
a single Games, she was not the only welcomed and urgent arrival from
the Netherlands.

A
\href{https://www.olympic.org/news/london-1948-only-two-years-to-prepare-and-huge-challenges}{hundred
tons of fruit} and vegetables were also sent from the Low Countries to
help feed Dutch and other athletes in a still-battered city during the
first Summer Olympics held after World War II. Finland provided timber
for the basketball court. Switzerland donated gymnastics equipment.
Canada felled two Douglas firs to make diving boards.

The Austerity Olympics, they were nicknamed. They represented a renewal
of the world's biggest sporting event following the wartime cancellation
of the Winter and Summer Games of 1940 and 1944 --- a disruption
deadlier and longer than a yearlong postponement of the 2020 Tokyo
Olympics because of the coronavirus pandemic.

``It was a liberation of spirit to be there in London,'' the great Czech
distance runner
\href{https://www.lrb.co.uk/the-paper/v38/n19/jon-day/hang-up-your-running-shoes}{Emil
Zatopek}, who won four cumulative gold medals at the 1948 and 1952
Olympics, later said of the moment. ``After those dark days of the war,
the bombing, the killing and the starvation, the revival of the Olympics
was as if the sun had come out.''

Still, much of London remained devastated by the Blitz. Some critics saw
the Olympics as an obscene waste in a nearly bankrupt Britain. But the
government lent its support to signal postwar rejuvenation and to secure
the desperate lifeline of hard currency from foreign tourists.

Image

A sign was erected where a department store once stood in the Holborn
section of London in 1946.Credit...The New York Times

Image

Shoppers lining up outside a butcher's shop selling horse meat, which
was exempt from postwar rationing in London, circa
1947.Credit...Popperfoto, via Getty Images

\includegraphics{https://static01.graylady3jvrrxbe.onion/images/2020/07/24/multimedia/24ww2-1948olympics-10/24ww2-1948olympics-10-articleLarge.jpg?quality=75\&auto=webp\&disable=upscale}

There was no money to build arenas or housing. Running events were held
on a greyhound track covered with tons of cinders from fireplaces.
Athletes were housed in military camps, college dormitories and schools.
Many competitors made their own uniforms.

Food rationing was still in effect, and while Olympians received more
rations than the average person --- the same amount as dockworkers and
coal miners --- some British athletes supplemented their diets with the
protein of whale meat.

This spirit of improvisation, rebirth and remarkable change was most
notably embodied by the star of the Games, Blankers-Koen, whose success
undermined stereotypes about women competing vigorously in sports and
inspired track and field's world governing body to name her its greatest
female athlete of the 20th century.

``Almost single-handedly she transformed women's athletics from a
sideshow into a central feature of future Olympics,'' Janie Hampton
wrote in ``The Austerity Olympics,'' a history of the 1948 London Games.

Image

Blankers-Koen as she won the 80-meter hurdles final in 1948.Credit...The
New York Times

In an era of amateurism, Blankers-Koen was a rarity. It was difficult
for any athlete to sustain an Olympic career across multiple Games when
the ability to earn money from sport was prohibited. But she persevered
through a gap of 12 years as the world went to war. Fanny Koen
(pronounced COON), unmarried at the time, competed in the 1936 Berlin
Olympics as an 18-year-old, finishing tied for sixth in the high jump
and fifth with the Netherlands' 4x100-meter relay team. She met the
great African-American sprinter Jesse Owens, who subverted Hitler's
notion of Aryan supremacy. In awe of his four gold medals, she asked for
his autograph and had a drink with him, she told me in
an\href{https://archive.nytimes3xbfgragh.onion/www.nytimes3xbfgragh.onion/library/sports/olympics/071100oly-longman-column.html}{interview}
in 2000.

``I thought it must be nice to have just one medal,'' she said at the
time.

Eventually, she would match Owens's haul of four gold medals, but not
before an interruption of more than a decade. In May 1940, Germany
invaded the Netherlands. Although the country was occupied, some
domestic sports competitions continued. Koen trained intermittently but
still set a handful of world records and married her coach, Jan
Blankers, who had competed in the triple jump at the 1928 Amsterdam
Olympics. They had a son, Jan Jr., and, when food became scarce, they
survived on potatoes and watery milk from an uncle who had a farm.

``People were being taken away, and friends of mine in the underground
were shot, and people were hungry and were in the streets begging for
food,'' she told
\href{https://www.nytimes3xbfgragh.onion/1982/10/02/sports/players-olympian-ahead-of-her-time.html}{The
Times} in 1982.

At war's end, Blankers-Koen had a daughter, also named Fanny. For many
women of that era, one child, much less two, would have meant the end of
their athletic careers. But Blankers-Koen persisted, consulting her
doctor, who told her, ``You are breastfeeding, but try it.''

She did and, she told me, ``I had more and more food for the girl than
before.''

As her nickname ``The Flying Housewife'' suggested, Blankers-Koen
accommodated her training to her domestic responsibilities, working out
twice a week, for two hours at a time, and only on Saturday afternoons
during the winter. She was said to have pedaled to practice with her two
children in a bicycle basket. While she ran and jumped, they played in
the sand of the long-jump pit.

Blankers-Koen arrived at the 1948 London Olympics at age 30. By some
accounts, she was also
\href{https://theolympians.co/tag/fanny-blankers-koen/}{three months
pregnant}. Of the nine track-and-field events for women, she won four:
the 100 meters, the 200 meters, the 80-meter hurdles and the 4x100-meter
relay. She might have won five or six gold medals if athletes had not
been restricted to three individual events. The winning distance in the
long jump, for instance, fell nearly two feet short of her world record.

Image

Blankers-Koen's success inspired track and field's world governing body
to name her its greatest female athlete of the 20th century.Credit...The
New York Times

Image

Blankers-Koen, the winner, and the silver and bronze medalists in the
women's 200 meters, Audrey Williamson of Britain and Audrey Patterson of
the United States, in 1948.Credit...Bettmann/Getty Images

Despite her unmatched accomplishments, many questioned Blankers-Koen's
presence in London. She received letters, she told me, saying that she
was too old and ``that it was not good to be running if you had a baby,
that you should do the housekeeping rather than running in shorts in a
stadium.''

Even after women's track and field was introduced at the 1928 Amsterdam
Olympics, cultural dismissiveness and pseudoscience persisted for
decades. Women were considered too fragile. Sweaty exertions were too
masculine, could prevent women from becoming mothers or might even cause
the uterus to fall out. Not until 1984 in Los Angeles was a women's
marathon included in the Olympics.

But in London --- the inaugural Olympics broadcast on home television
and the first filmed for theaters in Technicolor --- here came fierce
repudiation of obstructive convention in the insistent stride of
Blankers-Koen, a 30-year-old mother of two wearing a short-sleeved white
top and homemade orange shorts, her blond hair swept back like a
windsock by her unrivaled speed.

``They were not used to it,'' Blankers-Koen told me of her detractors,
``but I was not listening to them. I did what I wanted to do.''

Of the 4,073 participants in the 1948 London Olympics, only 393 were
women. But they confirmed the capability of women who had taken the jobs
of men absent during the war and reflected enormous societal change to
come.
\href{https://www.nytimes3xbfgragh.onion/2014/07/15/sports/alice-coachman-90-dies-groundbreaking-medalist.html}{Alice
Coachman} of the United States, winner of the high jump, became the
first Black woman to win a gold medal. And Blankers-Koen appeared in
bold headlines, remembers the journalist Roger Robinson, then a boy of 9
living in London who attended some of the track events.

``It was always `the Dutch Housewife and mother of two,''' said
Robinson, the author of ``When Running Made History.'' ``I suppose it
was stereotyping, but she was praised and admired. She showed that a
woman and a mother could be an Olympic superstar.''

Blankers-Koen won the 100 meters on a muddy track, launching herself
with a device that was new to the Olympics, the starting block. In the
80-meter hurdles, she clipped the fifth of eight barriers, then leaned
so desperately at the finish that the tape cut her neck and left
\href{https://www.theguardian.com/sport/blog/2012/jan/18/fanny-blankers-koen-olympic-moments}{droplets
of blood} on her shirt.

The stadium band began playing ``God Save the King,'' and Blankers-Koen
thought she had lost the event to her British rival, Maureen Gardner.
But the band was serenading the entrance of the British royal family,
not Gardner, and Blankers-Koen was declared the victor in a photo
finish.

Image

Fanny Blankers-Koen (right) and Maureen Gardner of Britain reaching the
tape in the 80-meter hurdle final.Credit...Associated Press

Before a preliminary round of the 200 meters, Blankers-Koen was
exhausted from the concentration required to run so many races. And she
missed her children, who had remained in the Netherlands. In a tearful
conversation with Blankers, her coach and husband, she said she wanted
to withdraw from the Games.

As Blankers-Koen recalled to me, he told her: ``It is not necessary for
me that you are running. But if you don't run, I'm sure you will be
sorry for it later on.''

Feeling renewed, she won the 200 meters by seven-tenths of a second ---
still the widest margin at any Olympics --- and reached the finish line
with her head back, so relaxed that her eyes seemed closed.

One event remained, the 4x100-meter relay. Blankers-Koen nearly missed
the race, having gone shopping for a raincoat. Running the anchor leg,
she took the baton in fourth place, five yards behind the leader, but
prevailed at the tape.

In an oral history of the Games, Blankers-Koen said disparaging comments
by Jack Crump, the manager of the British track and field team who
dismissed Blankers-Koen as ``too old to make the grade,'' had angered
and motivated her. ``Too old was I?'' she said. ``I would show them.''

When Blankers-Koen arrived back in Amsterdam, she rode through the
streets in a carriage pulled by four horses. Her neighbors gave her a
bicycle, David Wallechinsky wrote in ``The Complete Book of the
Olympics,'' so she wouldn't ``have to run so much.''

She participated in a third Olympics, the 1952 Helsinki Games, but,
bothered by painful boils, she stopped running the hurdles race after
striking the first two barriers. It was her last major international
competition.

In 2003, a half-century after Blankers-Koen retired, the journalist Kees
Koman published a biography --- the Dutch title translates to
``\href{https://www.independent.co.uk/sport/olympics/blankers-jnr-my-mother-only-enjoyed-herself-when-she-was-being-worshipped-2319488.html}{A
Queen With Man's Legs}'' --- which presented a more complicated portrait
of the Olympic star as distant, insecure and consumed with success. Her
daughter was quoted as saying, ``I think my mother never loved herself
and, the other way around, she could not give love and friendship
herself to other people.'' And: ``My mother only enjoyed herself when
she was being worshiped.''

She died a year later, on Jan. 25, 2004, at age 85 from
\href{https://www.nytimes3xbfgragh.onion/2004/01/26/sports/fanny-blankers-koen-star-of-48-olympics-dies-at-85.html}{heart
problems and Alzheimer's disease}. Blankers-Koen is not well known
today, but three-quarters of a century after her triumphs in London, she
remains the only female track and field athlete to win four gold medals
in a single Olympics.

``I find it difficult to think of anybody who made a bigger contribution
to the development of women's athletics,'' said Sebastian Coe of
Britain, the president of World Athletics, track's governing body, and a
two-time Olympic champion in the metric mile. ``Or who made a broader
impact on societal change, particularly around gender equity and
diversity, both on and off the track.''

\textbf{Jeré Longman}, a sports reporter for The Times, has covered 13
Olympics and has reported on track and field for more than three
decades.

Advertisement

\protect\hyperlink{after-bottom}{Continue reading the main story}

\hypertarget{site-index}{%
\subsection{Site Index}\label{site-index}}

\hypertarget{site-information-navigation}{%
\subsection{Site Information
Navigation}\label{site-information-navigation}}

\begin{itemize}
\tightlist
\item
  \href{https://help.nytimes3xbfgragh.onion/hc/en-us/articles/115014792127-Copyright-notice}{©~2020~The
  New York Times Company}
\end{itemize}

\begin{itemize}
\tightlist
\item
  \href{https://www.nytco.com/}{NYTCo}
\item
  \href{https://help.nytimes3xbfgragh.onion/hc/en-us/articles/115015385887-Contact-Us}{Contact
  Us}
\item
  \href{https://www.nytco.com/careers/}{Work with us}
\item
  \href{https://nytmediakit.com/}{Advertise}
\item
  \href{http://www.tbrandstudio.com/}{T Brand Studio}
\item
  \href{https://www.nytimes3xbfgragh.onion/privacy/cookie-policy\#how-do-i-manage-trackers}{Your
  Ad Choices}
\item
  \href{https://www.nytimes3xbfgragh.onion/privacy}{Privacy}
\item
  \href{https://help.nytimes3xbfgragh.onion/hc/en-us/articles/115014893428-Terms-of-service}{Terms
  of Service}
\item
  \href{https://help.nytimes3xbfgragh.onion/hc/en-us/articles/115014893968-Terms-of-sale}{Terms
  of Sale}
\item
  \href{https://spiderbites.nytimes3xbfgragh.onion}{Site Map}
\item
  \href{https://help.nytimes3xbfgragh.onion/hc/en-us}{Help}
\item
  \href{https://www.nytimes3xbfgragh.onion/subscription?campaignId=37WXW}{Subscriptions}
\end{itemize}
