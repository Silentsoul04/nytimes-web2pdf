Sections

SEARCH

\protect\hyperlink{site-content}{Skip to
content}\protect\hyperlink{site-index}{Skip to site index}

\href{https://myaccount.nytimes3xbfgragh.onion/auth/login?response_type=cookie\&client_id=vi}{}

\href{https://www.nytimes3xbfgragh.onion/section/todayspaper}{Today's
Paper}

\href{/section/opinion}{Opinion}\textbar{}China Is Harvesting the DNA of
Its People. Is This the Future of Policing?

\url{https://nyti.ms/30H5HZ8}

\begin{itemize}
\item
\item
\item
\item
\item
\end{itemize}

Advertisement

\protect\hyperlink{after-top}{Continue reading the main story}

\href{/section/opinion}{Opinion}

Supported by

\protect\hyperlink{after-sponsor}{Continue reading the main story}

\hypertarget{china-is-harvesting-the-dna-of-its-people-is-this-the-future-of-policing}{%
\section{China Is Harvesting the DNA of Its People. Is This the Future
of
Policing?}\label{china-is-harvesting-the-dna-of-its-people-is-this-the-future-of-policing}}

Authorities are systematically gathering genomic data from tens of
millions of people.

By Emile Dirks and James Leibold

Mr. Dirks is a Ph.D. candidate. Mr. Leibold is an expert on ethnic
issues in modern China.

\begin{itemize}
\item
  July 24, 2020
\item
  \begin{itemize}
  \item
  \item
  \item
  \item
  \item
  \end{itemize}
\end{itemize}

\includegraphics{https://static01.graylady3jvrrxbe.onion/images/2020/07/24/opinion/24leibold1/merlin_173558154_14145d71-d38b-4c26-ad1e-76ead5f32fcc-articleLarge.jpg?quality=75\&auto=webp\&disable=upscale}

\href{https://cn.nytimes3xbfgragh.onion/opinion/20200728/china-dna-police/}{阅读简体中文版}\href{https://cn.nytimes3xbfgragh.onion/opinion/20200728/china-dna-police/zh-hant/}{閱讀繁體中文版}

For several years now, the police and other authorities in China have
been collecting across the country DNA samples from millions of men and
boys who aren't suspected of having committed any crime.

In \href{https://www.aspi.org.au/report/genomic-surveillance}{a report
published by the Australian Strategic Policy Institute} last month, we
exposed the extent of the Chinese government's program of genetic
surveillance: It no longer is limited to Xinjiang, Tibet and other areas
mostly populated by ethnic minorities the government represses; DNA
collection --- serving no apparent immediate need --- has spread across
the entire country. We estimate that the authorities' goal is to gather
the DNA samples of 35 million to 70 million Chinese males.

Matched against official family records,
\href{https://www.wsj.com/articles/twelve-days-in-xinjiang-how-chinas-surveillance-state-overwhelms-daily-life-1513700355}{surveillance
footage} or witness statements in police reports, these samples will
become a powerful tool for the Chinese authorities to track down a man
or boy --- or, failing that, a relative of his --- for whatever reason
they deem fit.

The Chinese government
\href{https://www.globaltimes.cn/content/1193942.shtml}{denies} the
existence of any such program, but since our study's publication, we
have continued to uncover online scattered evidence revealing the
program's enormous scale, including government reports and official
procurement orders for DNA kits and testing services.

DNA is being harvested across the country: in the southwestern provinces
of \href{https://archive.is/9JEPP}{Yunnan} and
\href{http://archive.vn/uUHW8}{Guizhou}; in central-southern
\href{https://archive.is/0I3zd}{Hunan}; in
\href{http://archive.vn/zGN8C}{Shandong} and
\href{http://archive.vn/p8BLE}{Jiangsu}, in the east; and up north, in
the autonomous region of \href{https://archive.is/idqI0}{Inner
Mongolia}.

We have continued to find photographic evidence that the police
\href{https://archive.is/E7xQ9}{are collecting blood from children},
pinpricking their fingers at school --- a clear violation of China's
responsibilities under
\href{https://treaties.un.org/doc/Treaties/1990/09/19900902\%2003-14\%20AM/Ch_IV_11p.pdf}{the
U.N. Convention on the Rights of the Child}.

And we have found fresh proof, including official documents, showing
that DNA samples are also being gathered in major urban centers. (For a
time, the focus seemed to be largely on rural communities.)

An official report dated June 16,
\href{http://www.ccgp-sichuan.gov.cn/cms/newscontent/contentupload/file/2020-07/c59d5ba5_0b21_4bc2_a0aa_3226885510ef.pdf}{available
on a website of the government of Sichuan Province}, details the
creation of a DNA database by the Public Security Bureau of the city of
Chengdu, the province's capital, and seeks expert opinion on the
creation of a ``male ancestry investigation system.''

It documents how 17 public security offices have collected DNA samples
from nearly 600,000 male residents across the city --- that's about 7
percent of Chengdu's male population (assuming that roughly half of the
city's total population of about 16.6 million is male).

The Chengdu procurement report states that building a massive genetic
database about local residents will help the police ``maintain public
order and stability as well as meet the needs of daily case work.'' This
is of no comfort.

In China, securing the public order essentially means maintaining the
uncontested rule of the Communist Party. Dissent is a crime, and police
operations are a key part of the state's apparatus of repression.

The Chinese police are not doing this work alone. Evidence continues to
accumulate that private companies, both Chinese and foreign, are
complicit in this extraordinarily vast, and ominous, assault on the
privacy of Chinese citizens.

In Hunan Province, Huangrui Scientific Instruments Ltd. --- a company
based in the provincial capital that produces a range of medical,
chemical and scientific products --- has sold to the ** Public Security
Bureau of the city of Liuyang some 140,000 DNA testing kits produced by
\href{https://www.thermofisher.com/hk/en/home.html}{Thermo Fisher
Scientific}, a U.S.-based
\href{https://fortune.com/fortune500/2019/thermo-fisher-scientific/}{Fortune
500 company}. That's enough equipment to test roughly one in five men in
the community.

In Fujian Province, \href{http://fgidna.com/lxwm/index_29.aspx}{Forensic
Genomics International}, a subsidiary of
\href{https://en.genomics.cn/}{BGI Group}--- a Chinese gene-sequencing
and biomedical company that
\href{https://en.genomics.cn/en-about.html}{describes itself} as ``one
of the world's leading life science and genomics organizations'' --- won
a contract to analyze 16,000 blood samples collected by one district in
the province's capital as part of the authorities' effort to build a
``male ancestry investigation system.'' The estimated total male
population of the district is 43,500.

The Australian Strategic Policy Institute has contacted Thermo Fisher
and Forensic Genomics International asking for comments on our report;
neither company replied.

In a statement issued to The New York Times for
\href{https://www.nytimes3xbfgragh.onion/2020/06/17/world/asia/China-DNA-surveillance.html}{a
news story} last month related to the report, a representative of Thermo
Fisher said that the company was ``proud to be a part of the many
positive ways in which DNA identification has been applied, from
tracking down criminals to stopping human trafficking and freeing the
unjustly accused.''

The note, titled ``Statement on Xinjiang,'' did not address the concerns
we raised about the potential for widespread abuse of genomic data by
the Chinese police throughout the country.

Thermo Fisher had previously been criticized --- by
\href{https://www.hrw.org/news/2017/12/13/china-minority-region-collects-dna-millions}{human
rights organizations}and
\href{https://www.nature.com/articles/d41586-019-03687-x}{scholars} ---
for supplying DNA collection and analysis equipment to the Chinese
authorities in support of their campaign of repression against ethnic
minorities in Xinjiang. Already in February 2019, the company had
\href{https://www.thermofisher.com/hk/en/home.html}{vowed to cease any
such sales} in the region.

Earlier this week, the U.S. Commerce Department
\href{https://www.nytimes3xbfgragh.onion/2020/07/21/business/china-us-trade-banned.html}{added
to its list of sanctioned companies} two other subsidiaries of BGI Group
--- the Chinese parent company of Forensic Genomics International ---
for
``\href{https://www.commerce.gov/news/press-releases/2020/07/commerce-department-adds-eleven-chinese-entities-implicated-human}{conducting
genetic analyses used to further the repression of Uighurs and other
Muslim minorities}'' in Xinjiang.

BGI Group
\href{https://www.nytimes3xbfgragh.onion/2020/07/21/business/china-us-trade-banned.html}{has
rejected the accusations}. Yet the company is also reported to have
agreed to build
``\href{https://www.axios.com/chinese-coronavirus-test-maker-agreed-to-build-a-xinjiang-gene-bank-f82b6918-d6c5-45f9-90b8-dad3341d6a6e.html}{a
gene bank}'' in the region.

BGI Group has also been producing
\href{https://www.prnewswire.com/news-releases/bgi-group-helping-over-80-countries-for-timely-covid-19-detection-and-intervention-301043895.html}{tens
of millions of Covid-19 test kits} for distribution to more than 80
countries --- raising concerns in places like
\href{https://www.theguardian.com/australia-news/2020/jul/09/australian-experts-raise-security-concerns-about-chinese-maker-of-andrew-forrest-covid-19-tests}{Australia}
and
\href{https://www.washingtonpost.com/business/2020/07/02/china-bgi-california-testing/}{California}
that any DNA data collected in the process might then be misused.

For now, China appears to be the only country in the world where police
are harvesting en masse DNA samples outside the scope of criminal
investigations. But how much longer before others follow suit?

In other countries, including the United States, law enforcement
authorities are also pushing the ethical boundaries of genetic data
collection.

The police in New York routinely collect DNA samples from people they
arrest or simply question ---
\href{https://www.nytimes3xbfgragh.onion/2019/08/15/nyregion/nypd-dna-database.html}{sometimes
without saying}. Across the United States, police officers search
private ancestry sites like \href{https://www.gedmatch.com/}{GEDmatch}
scouring genetic data looking for potential leads in cold cases --- also
\href{https://www.sciencemag.org/news/2019/11/judge-said-police-can-search-dna-millions-americans-without-their-consent-what-s-next}{without
the knowledge or the consent} of the people who uploaded their personal
information.

Earlier this year, the Trump administration put in place a program
requiring Immigration and Customs Enforcement agents to
\href{https://www.dhs.gov/sites/default/files/publications/privacy-pia-dhs080-detaineedna-january2020.pdf}{collect
with mouth swabs DNA samples from people in their custody} and add that
information to the FBI's DNA database.

Global norms around how to handle genomic data are unsettled, and
against that shifting background, the actions of superpowers like China
and the United States are likely to set dangerous precedents for other
states.

The high courts of some countries, like
\href{https://www.hrw.org/news/2017/10/17/kuwait-court-strikes-down-draconian-dna-law}{Kuwait}
and
\href{https://www.reuters.com/article/us-kenya-rights/court-orders-safeguards-for-kenyan-digital-ids-bans-dna-collecting-idUSKBN1ZU23D}{Kenya},
have either banned as unconstitutional or restricted the mass collection
of DNA by state authorities.

Other states are trying to forge ahead.

In India, the government of Prime Minister Narendra Modi introduced
\href{https://pib.gov.in/Pressreleaseshare.aspx?PRID=1577738}{a bill}
last year aimed at ``expanding the application of DNA-based forensic
technologies to support and strengthen the justice delivery system of
the country.'' But advocacy groups have identified
\href{http://www.genewatch.org/uploads/f03c6d66a9b354535738483c1c3d49e4/comments-on-india-dna-bill-2019.pdf}{the
risk of potential misuses}.

A senior judge had previously
\href{https://www.thehindu.com/news/national/can-dna-be-taken-for-aadhaar-asks-judge/article23447019.ece}{warned}
that legislation covering India's vast system of biometric
identification, Aadhaar, might be interpreted in the future to justify
the collection, not only of people's fingerprints and iris scans, but of
DNA samples as well.

Last month, civil rights groups in
\href{https://www.khaosodenglish.com/politics/2020/06/01/coronavirus-returnees-ordered-to-give-up-dna-samples/}{Thailand}
\href{https://www.khaosodenglish.com/politics/2020/06/01/coronavirus-returnees-ordered-to-give-up-dna-samples/}{raised
concerns} that Thai border authorities, including soldiers, had, without
explanation, forcibly taken DNA samples from
\href{https://www.codastory.com/authoritarian-tech/surveillance-muslims-thailand/}{minority
Muslim Thai citizens} returning from Malaysia.

Malaysia, for its part, is mulling plans to create a national
registration system that would link biometric and DNA data to existing
ID documents --- this
\href{https://www.thestar.com.my/news/nation/2019/09/23/pbs-nrd-must-fast-track-setting-up-of-dna-database}{purportedly}
to keep ineligible foreigners from fraudulently being added to the
country's citizenship rolls.

The battle over biometric privacy will be one of the defining civil
liberty issues of the 21st century. And grimly, on this front, too,
China seems to be leading the way.

Emile Dirks is a Ph.D. candidate in the department of Political Science
at the University of Toronto. James Leibold is a senior fellow at the
Australian Strategic Policy Institute and the head of the department of
Politics, Media and Philosophy at La Trobe University, in Melbourne.

\emph{The Times is committed to publishing}
\href{https://www.nytimes3xbfgragh.onion/2019/01/31/opinion/letters/letters-to-editor-new-york-times-women.html}{\emph{a
diversity of letters}} \emph{to the editor. We'd like to hear what you
think about this or any of our articles. Here are some}
\href{https://help.nytimes3xbfgragh.onion/hc/en-us/articles/115014925288-How-to-submit-a-letter-to-the-editor}{\emph{tips}}\emph{.
And here's our email:}
\href{mailto:letters@NYTimes.com}{\emph{letters@NYTimes.com}}\emph{.}

\emph{Follow The New York Times Opinion section on}
\href{https://www.facebookcorewwwi.onion/nytopinion}{\emph{Facebook}}\emph{,}
\href{http://twitter.com/NYTOpinion}{\emph{Twitter (@NYTopinion)}}
\emph{and}
\href{https://www.instagram.com/nytopinion/}{\emph{Instagram}}\emph{.}

Advertisement

\protect\hyperlink{after-bottom}{Continue reading the main story}

\hypertarget{site-index}{%
\subsection{Site Index}\label{site-index}}

\hypertarget{site-information-navigation}{%
\subsection{Site Information
Navigation}\label{site-information-navigation}}

\begin{itemize}
\tightlist
\item
  \href{https://help.nytimes3xbfgragh.onion/hc/en-us/articles/115014792127-Copyright-notice}{©~2020~The
  New York Times Company}
\end{itemize}

\begin{itemize}
\tightlist
\item
  \href{https://www.nytco.com/}{NYTCo}
\item
  \href{https://help.nytimes3xbfgragh.onion/hc/en-us/articles/115015385887-Contact-Us}{Contact
  Us}
\item
  \href{https://www.nytco.com/careers/}{Work with us}
\item
  \href{https://nytmediakit.com/}{Advertise}
\item
  \href{http://www.tbrandstudio.com/}{T Brand Studio}
\item
  \href{https://www.nytimes3xbfgragh.onion/privacy/cookie-policy\#how-do-i-manage-trackers}{Your
  Ad Choices}
\item
  \href{https://www.nytimes3xbfgragh.onion/privacy}{Privacy}
\item
  \href{https://help.nytimes3xbfgragh.onion/hc/en-us/articles/115014893428-Terms-of-service}{Terms
  of Service}
\item
  \href{https://help.nytimes3xbfgragh.onion/hc/en-us/articles/115014893968-Terms-of-sale}{Terms
  of Sale}
\item
  \href{https://spiderbites.nytimes3xbfgragh.onion}{Site Map}
\item
  \href{https://help.nytimes3xbfgragh.onion/hc/en-us}{Help}
\item
  \href{https://www.nytimes3xbfgragh.onion/subscription?campaignId=37WXW}{Subscriptions}
\end{itemize}
