Sections

SEARCH

\protect\hyperlink{site-content}{Skip to
content}\protect\hyperlink{site-index}{Skip to site index}

\href{https://myaccount.nytimes3xbfgragh.onion/auth/login?response_type=cookie\&client_id=vi}{}

\href{https://www.nytimes3xbfgragh.onion/section/todayspaper}{Today's
Paper}

\href{/section/opinion}{Opinion}\textbar{}America Is Still No. 1. Here's
How It Can Stay There.

\url{https://nyti.ms/30FfcrP}

\begin{itemize}
\item
\item
\item
\item
\item
\end{itemize}

Advertisement

\protect\hyperlink{after-top}{Continue reading the main story}

\href{/section/opinion}{Opinion}

Supported by

\protect\hyperlink{after-sponsor}{Continue reading the main story}

\hypertarget{america-is-still-no-1-heres-how-it-can-stay-there}{%
\section{America Is Still No. 1. Here's How It Can Stay
There.}\label{america-is-still-no-1-heres-how-it-can-stay-there}}

The country now finds itself on a treacherous new stage. ``America
First'' won't cut it.

By Josef Joffe

Mr. Joffe is a fellow at Stanford University's Hoover Institution.

\begin{itemize}
\item
  July 24, 2020
\item
  \begin{itemize}
  \item
  \item
  \item
  \item
  \item
  \end{itemize}
\end{itemize}

\includegraphics{https://static01.graylady3jvrrxbe.onion/images/2020/07/24/opinion/24joffe1/merlin_165410829_69c0a15f-5464-451d-943b-36d5d69cbfe2-articleLarge.jpg?quality=75\&auto=webp\&disable=upscale}

A renowned scholar of American diplomacy once slapped his 4-year-old in
a fit. Coolly, she responded, ``You shouldn't do this.''

``Why not?'' he asked.

``You don't have that many friends.''

That's the best take on President Trump's foreign policy, especially
when it comes to America's oldest friends. Mr. Trump's favorite target
is Angela Merkel, Germany's eternal chancellor. But he is an
equal-opportunity ruffian, who has bullied the leaders of Britain,
France and more.

He has routinely clobbered them with threats of tariffs and sanctions.
As if running a protection racket, he has told ``obsolete'' NATO: Pay
up, or we pull out. Now he is withdrawing 9,500 American troops from
Germany, the linchpin of the American-built European order. Possibly
next in line is South Korea, where the United States expended over
36,000 lives to repel Mao Zedong's armies.

In his latest assault on Europe, Mr. Trump is going after Russian gas
and Chinese mobile technology. To be fair, he has a point. The Nord
Stream 2 gas pipeline, slated to pump around two trillion cubic feet
annually into Germany, will ensure Russian energy dominance for decades.
And Huawei's bargain-priced 5G equipment might well be a Trojan horse
that could enable the Chinese state to extract vital intelligence and
intellectual property.

Nor is Mr. Trump the first Europe-basher. All American presidents have
tried to strong-arm the Europeans over trade and defense. And after the
Cold War, Europeans have been much better at slashing military spending
than boosting it to counter Russian rearmament. President Barack Obama
also inveighed against
``\href{https://www.nytimes3xbfgragh.onion/2016/03/10/world/middleeast/obama-criticizes-the-free-riders-among-americas-allies.html}{free
riders}.''

But there is a blatant difference between now and then. Before, there
was never any doubt that the United States and its friends would hang
together. Now this historic relationship is at a tipping point. For all
of Europe's backsliding, the main blame must lie with Mr. Trump.

The United States now finds itself on a treacherous new stage: Call it
the ``two-and-a-half-power world.'' America it is still on top. But
China, the No. 2 power, is arming and extending its reach around the
globe to try to dethrone the United States. An economic waif, nuclear
Russia is only a semi-great. Still, President Vladimir Putin is superbly
playing a weak hand in Europe, North Africa and the Middle East.

America should get with the program. To stay No. 1, it should follow
four rules.

Rule 1: Keep your friends and recruit more. They may be free riders, as
smaller players always are, but they add muscle, influence and cachet.
``America First'' does not. Bullies steal your lunch; they are never
elected class president.

Mr. Trump confuses coercion with clout. For him, ``Make America Great
Again'' does not mean leading, but exploiting the show. It isn't
``win-win,'' but ``I win if you lose.'' Europeans want Nord Stream 2?
Then forget about doing business in America. If Berlin goes with Huawei,
say goodbye to the U.S. intelligence-sharing that has foiled many a
terror plot in Germany. This is not smart diplomacy.

Rule 2: Keep and increase authority by providing essential services like
common security, free trade and navigation --- and by upholding a
rules-based order that favors mutually beneficial cooperation.

``America First'' means sacrificing the future for short-term gain. The
history of sanctions, Mr. Trump's favorite sport after golf, offers a
nasty lesson. Victims may buckle, but they will soon sever the tie that
binds. Sanctions devalue themselves by forcing nations into
self-sufficiency. Ciao, America.

Rule 3: Promote your own interests by taking care of others'. What's
good for them is good for America because a supply-side foreign policy
legitimizes U.S. leadership.

Yet Mr. Trump sees multilateralism as a plot against America. ``America
Alone'' is worse. Though hungry for gas and 5G, the Europeans are a
second-order headache compared with Chinese and Russian expansionism. In
this two-and-a-half-power world, plus lesser upstarts like Recep Tayyip
Erdogan's Turkey and Kim Jong-un's North Korea, America needs more, not
fewer, friends.

Finally, rule 4: Always harness the largest-possible coalition. Coax,
don't compel. Return to the diplomacy America has so wisely practiced in
the past.

Remind Ms. Merkel that European 5G suppliers --- Nokia and Ericsson ---
can also deliver. The somewhat higher price will be mitigated by the
gain in common security. And if Ms. Merkel panics over losing the
lucrative Chinese market to Beijing's retaliation? Then assure her that
America is not out to destroy Euro-Chinese trade as such but to keep
strategic industries out of President Xi Jinping's hands. America's
allies will nod in assent.

So, what are the chances?

``Europe should view Trump as an anomaly,'' John Bolton, Mr. Trump's
defrocked national security adviser,
\href{https://www.spiegel.de/international/world/interview-with-john-bolton-trump-is-capable-of-almost-anything-a-90c5452e-53f3-4e72-a788-fd7ac74317a1}{told}
the German magazine Der Spiegel last week. ``It is not going to be that
hard to get back to normal.'' Amen.

``Home alone'' has not been the American way, certainly not since 1945,
when the United States took on responsibility for the liberal world
order. Even Mr. Trump's base would rather be at the helm than hunker
down in the hold.

Josef Joffe is a member of the editorial council of the German weekly
newspaper Die Zeit and a fellow at Stanford University's Hoover
Institution.

\emph{The Times is committed to publishing}
\href{https://www.nytimes3xbfgragh.onion/2019/01/31/opinion/letters/letters-to-editor-new-york-times-women.html}{\emph{a
diversity of letters}} \emph{to the editor. We'd like to hear what you
think about this or any of our articles. Here are some}
\href{https://help.nytimes3xbfgragh.onion/hc/en-us/articles/115014925288-How-to-submit-a-letter-to-the-editor}{\emph{tips}}\emph{.
And here's our email:}
\href{mailto:letters@NYTimes.com}{\emph{letters@NYTimes.com}}\emph{.}

\emph{Follow The New York Times Opinion section on}
\href{https://www.facebookcorewwwi.onion/nytopinion}{\emph{Facebook}}\emph{,}
\href{http://twitter.com/NYTOpinion}{\emph{Twitter (@NYTopinion)}}
\emph{and}
\href{https://www.instagram.com/nytopinion/}{\emph{Instagram}}\emph{.}

Advertisement

\protect\hyperlink{after-bottom}{Continue reading the main story}

\hypertarget{site-index}{%
\subsection{Site Index}\label{site-index}}

\hypertarget{site-information-navigation}{%
\subsection{Site Information
Navigation}\label{site-information-navigation}}

\begin{itemize}
\tightlist
\item
  \href{https://help.nytimes3xbfgragh.onion/hc/en-us/articles/115014792127-Copyright-notice}{©~2020~The
  New York Times Company}
\end{itemize}

\begin{itemize}
\tightlist
\item
  \href{https://www.nytco.com/}{NYTCo}
\item
  \href{https://help.nytimes3xbfgragh.onion/hc/en-us/articles/115015385887-Contact-Us}{Contact
  Us}
\item
  \href{https://www.nytco.com/careers/}{Work with us}
\item
  \href{https://nytmediakit.com/}{Advertise}
\item
  \href{http://www.tbrandstudio.com/}{T Brand Studio}
\item
  \href{https://www.nytimes3xbfgragh.onion/privacy/cookie-policy\#how-do-i-manage-trackers}{Your
  Ad Choices}
\item
  \href{https://www.nytimes3xbfgragh.onion/privacy}{Privacy}
\item
  \href{https://help.nytimes3xbfgragh.onion/hc/en-us/articles/115014893428-Terms-of-service}{Terms
  of Service}
\item
  \href{https://help.nytimes3xbfgragh.onion/hc/en-us/articles/115014893968-Terms-of-sale}{Terms
  of Sale}
\item
  \href{https://spiderbites.nytimes3xbfgragh.onion}{Site Map}
\item
  \href{https://help.nytimes3xbfgragh.onion/hc/en-us}{Help}
\item
  \href{https://www.nytimes3xbfgragh.onion/subscription?campaignId=37WXW}{Subscriptions}
\end{itemize}
