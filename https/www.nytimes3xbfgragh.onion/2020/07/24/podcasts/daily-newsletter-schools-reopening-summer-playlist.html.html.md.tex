Sections

SEARCH

\protect\hyperlink{site-content}{Skip to
content}\protect\hyperlink{site-index}{Skip to site index}

\href{https://www.nytimes3xbfgragh.onion/spotlight/podcasts}{Podcasts}

\href{https://myaccount.nytimes3xbfgragh.onion/auth/login?response_type=cookie\&client_id=vi}{}

\href{https://www.nytimes3xbfgragh.onion/section/todayspaper}{Today's
Paper}

\href{/spotlight/podcasts}{Podcasts}\textbar{}Answering Your Questions
About School Reopenings

\url{https://nyti.ms/2EiMbL1}

\begin{itemize}
\item
\item
\item
\item
\item
\item
\end{itemize}

Advertisement

\protect\hyperlink{after-top}{Continue reading the main story}

Supported by

\protect\hyperlink{after-sponsor}{Continue reading the main story}

the daily newsletter

\hypertarget{answering-your-questions-about-school-reopenings}{%
\section{Answering Your Questions About School
Reopenings}\label{answering-your-questions-about-school-reopenings}}

And a summer mixtape. Because school hasn't started just yet.

\href{https://www.nytimes3xbfgragh.onion/by/pam-belluck}{\includegraphics{https://static01.graylady3jvrrxbe.onion/images/2018/02/16/multimedia/author-pam-belluck/author-pam-belluck-thumbLarge-v2.png}}

By \href{https://www.nytimes3xbfgragh.onion/by/pam-belluck}{Pam Belluck}

\begin{itemize}
\item
  July 24, 2020
\item
  \begin{itemize}
  \item
  \item
  \item
  \item
  \item
  \item
  \end{itemize}
\end{itemize}

\includegraphics{https://static01.graylady3jvrrxbe.onion/images/2020/07/10/science/24daily-newsletter-image1/merlin_174095748_6b11e6ba-a408-471e-8a73-9bf945b568d9-articleLarge.jpg?quality=75\&auto=webp\&disable=upscale}

\emph{On}
\href{https://www.nytimes3xbfgragh.onion/2020/07/22/podcasts/the-daily/school-reopenings-coronavirus.html?}{\emph{Wednesday's
show}}\emph{, we spoke to Pam Belluck about the safety behind reopening
schools in the U.S. As we enter the last month of the summer, the
question of reopening schools continues to be front-of-mind for people
around the world. So we asked Pam a couple of follow-up questions based
on those we got from you.}

\textbf{In the episode, while explaining a line of thinking about why it
may be safer to reopen elementary schools before middle and high
schools, you mentioned the growing evidence that children under 10 are
less likely to spread the virus than older children. Why is that?}

Covid-19 seems to affect little kids differently than most other
respiratory viruses, like the flu or cytomegalovirus, which are commonly
spread by young children. Scientists are exploring whether young
children are less likely to become infected with this new virus or
simply less likely to get sick when they are infected. One theory is
that younger children have fewer of the proteins that allow Covid-19 to
enter lung cells and replicate. Another is that children are protected
from developing serious symptoms because of differences in their
maturing immune systems or in the cells that line the blood vessels.

Milder symptoms may help explain why young children don't spread the
virus as much. If they aren't coughing or breathing hard, they might not
be propelling viral droplets far enough to reach another person.

\textbf{Schools bring together a lot of adults besides teachers. How are
communities in the U.S. thinking about adult-to-adult interaction and
transmission if schools reopen?}

Over half of the employees of public school systems are not full-time
teachers --- they are bus drivers, custodians, cafeteria workers, aides,
substitute teachers and others who fill roles that are vital to making
school reopening possible. They serve in jobs that often pay lower wages
and provide fewer employee benefits than teachers receive. And some will
be asked to take on additional responsibilities when schools open.
Custodians will be implementing procedures for extra cleaning and
sanitizing; drivers in some districts may be asked to assess the health
of children before allowing them to board the bus.

Many school districts are trying to develop plans to protect these
workers. Some intend to install plexiglass dividers to protect bus
drivers, give them medical-grade face masks and limit the number of
children on buses so the children are separated by six feet (steps that
serve to protect children as well). Lunchtime will look different in
many schools: Some will have individualized box lunches and children
eating in their classrooms, so cafeteria workers are exposed to fewer
people.

Many of these precautions will take additional resources. And, since
these are jobs that can't be done online, districts will need
contingency plans or substitute workers for staff members who get sick
or are uncomfortable coming to work.

\textbf{You mentioned Israel as a country that's had less success in
reopening its schools. Why did it relax its class size restrictions? Did
it see a parallel outbreak in its communities when it relaxed class
sizes?}

Israel's Covid-19 case counts have gone up in recent weeks, and it is
experiencing a second wave of the pandemic after having largely gotten
the first wave under control. It relaxed other pandemic restrictions
around the same time as it reopened schools, so it's not clear exactly
what role the schools played and how much of the surge in cases was
driven by social gatherings at weddings, parties and bars.

It's important to note that, while the U.S. can learn from other
countries, none of them exactly mirror American circumstances. In
Israel, for example, classes are typically large, so when they were
divided into small-group capsules, each class still had about 18
students. Also, for a few days in May, the temperature was so hot that
the Israeli government waived the mask requirement, and it's unclear if
that spurred disease transmission.

Still, some public health experts in Israel say they recommended against
relaxing the restrictions on class size, and they believe that the
guidelines were eased because of political pressure and eagerness for
the economy to fully reopen. Israeli citizens may also have felt that
there wasn't a great risk to lifting the restrictions, given that the
country had successfully tamped down its infection rate in April and
May. That's a cautionary message for communities in the U.S. and
elsewhere not to become complacent once they've managed to keep the
virus in check.

\emph{Talk to Pam on Twitter:}
\href{https://twitter.com/pambelluck?lang=en}{\emph{@PamBelluck}}\emph{.}

\begin{center}\rule{0.5\linewidth}{\linethickness}\end{center}

\hypertarget{our-summer-2020-mixtape}{%
\subsection{\texorpdfstring{\href{https://open.spotify.com/playlist/0GvVwghTzyY4lR722dCeM0?si=xQezj1zNTeK1bSZriRY3zg}{Our
summer 2020
mixtape}}{Our summer 2020 mixtape}}\label{our-summer-2020-mixtape}}

Image

``Come now, come dry your eyes, you know you a star, you can touch the
sky.'' This summer calls for a lot of Lizzo.Credit...Amy Lombard for The
New York Times

It's been a strange summer, to say the least. This week, when we asked
each member of our team to share a song from their pandemic summer
soundtrack, our producer Hans Buetow wrote back: ``I am devoid of music
or even joy in my life. SUCH IS THE WORLD RIGHT NOW, I CAN ONLY LISTEN
TO THE WHISTLING WINDS THAT BLOW THROUGH THE EMPTY CHASM OF WHAT ONCE
HELD MY SOUL \ldots{} Oh, but also I do listen to a lot of Whitney
Houston.''

Hans went on to add 25 songs to our team's summer playlist (we only
asked for one).

You can listen to the playlist below, or
\href{https://open.spotify.com/playlist/0GvVwghTzyY4lR722dCeM0?si=SLnkYCqgRLaF-nUDcTHCFA}{follow
the full version on Spotify}, which includes Hans's 25 favorites, as
well as a \href{https://www.youtube.com/watch?v=fE0uc92JlCc}{dubstep
version} of the ``Jurassic Park'' theme song, plus a few Taylor Swift
bangers in honor of
her\href{https://www.nytimes3xbfgragh.onion/aponline/2020/07/23/us/ap-us-music-taylor-swift.html}{surprise
album}. From our team to yours, Happy Listening.

\begin{quote}
\textbf{Alexandra Leigh Young:} 2NE1,
``\href{https://www.youtube.com/watch?v=j7_lSP8Vc3o}{I Am The Best}''

\textbf{Andy Mills}: RAC (feat. Louis the Child),
``\href{https://www.youtube.com/watch?v=csKxedk-KpA}{Passion}''

\textbf{Annie Brown:} Khruangbin,
``\href{https://www.youtube.com/watch?v=oc50wHexbwg}{Time (You and I)}''

\textbf{Asthaa Chaturvedi:} Stylo G, The Fanatix, Nicki Minaj and Vybz
Kartel, ``\href{https://www.youtube.com/watch?v=459bLvt7f9c}{Touch Down
- Remix}''

\textbf{Austin Mitchell:} OutKast,
``\href{https://www.youtube.com/watch?v=M4fwLA1_ARM}{Aquemini}''

\textbf{Bianca Giaever}: Eddy Arnold,
``\href{https://www.youtube.com/watch?v=_-UFTpiqUts}{Cattle Call}''

\textbf{Brad Fisher:} Kacey Musgraves,
``\href{https://www.youtube.com/watch?v=Zr3gscRpAhA}{Lonely Weekend}''

\textbf{Clare Toeniskoetter}: Empress of,
``\href{https://www.youtube.com/watch?v=ln3jvZ5B9o8}{Not the One}''

\textbf{Dan Powell:} Blue Magic,
``\href{https://www.youtube.com/watch?v=bEL7b9zTZHE}{Just Don't Want to
Be Lonely}''

\textbf{Daniel Guillemette:} Karol Conká,
``\href{https://www.youtube.com/watch?v=hU3soPwOLDI}{Boa Noite}''

\textbf{Dave Shaw}: U2,
``\href{https://www.youtube.com/watch?v=q4Gr8Lf2Bzo}{Staring at the
Sun}''

\textbf{Hans Buetow:} Brenton Wood,
``\href{https://www.youtube.com/watch?v=JbZ413-78ok}{Oogum Boogum
Song}''

\textbf{Jake Lucas:} Reyna Tropical,
``\href{https://www.youtube.com/watch?v=FX1fFjh0h8U}{No Me Quieres}''

\textbf{Julia Simon:} Haim,
``\href{https://www.youtube.com/watch?v=ZjuA_o6Jzyo}{Summer Girl}''

\textbf{Kelly Prime:} Salt Cathedral,
``\href{https://www.youtube.com/watch?v=hcCbKwi5fdE}{Te Quiero
Olvidar}''

\textbf{Larissa Anderson:} Marvin Gaye and Tammi Terrell,
``\href{https://www.youtube.com/watch?v=-C_3eYj-pOM}{Ain't No Mountain
High Enough}''

\textbf{Lauren Jackson:} Still Woozy,
``\href{https://www.youtube.com/watch?v=tiXfzapc2Js}{Window}''

\textbf{Lisa Chow:} Justin Bieber,
``\href{https://www.youtube.com/watch?v=3AyMjyHu1bA}{Intentions}''

\textbf{Luke Vander Ploeg}: Lisa Hannigan,
``\href{https://www.youtube.com/watch?v=WSaPbVjcrp4}{I Don't Know}''

\textbf{Mahima Chablani:} Scatman John,
``\href{https://www.youtube.com/watch?v=Hy8kmNEo1i8}{Scatman
(ski-ba-bop-ba-dop-bop)}''

\textbf{Marc Georges:} Lizzo,
``\href{https://www.youtube.com/watch?v=XaCrQL_8eMY}{Juice}''

\textbf{Michael Barbaro:} Frank Sinatra (with Harry James and His
Orchestra), ``\href{https://www.youtube.com/watch?v=dr23NWHEsbU}{Moon
Love}''

\textbf{Michael Simon Johnson:} Lianne La Havas,
``\href{https://www.youtube.com/watch?v=AHeW8McMBS8}{Weird Fishes}''

\textbf{Mike Benoist:} The Black Crowes,
``\href{https://www.youtube.com/watch?v=YtX6ZgEG1Jg\&pbjreload=101}{Soul
Singing}''

\textbf{M.J. Davis Lin}: Fugees,
``\href{https://www.youtube.com/watch?v=B8LSTgs6Vy0}{How Many Mics}''

\textbf{Neena Pathak}: Ric Wilson,
``\href{https://www.youtube.com/watch?v=Dgxh9Rs6D8c}{Fight Like Ida B \&
Marsha P}''

\textbf{Nora Keller:} Molchat Doma,
``\href{https://www.youtube.com/watch?v=HR5zpFs7YpY}{Судно (Sudno)}

\textbf{Rachel Quester}: Angus \& Julia Stone,
``\href{https://www.youtube.com/watch?v=yFTvbcNhEgc}{Big Jet Plane}''

\textbf{Rachelle Bonja:} L'Impératrice,
``\href{https://www.youtube.com/watch?v=PAIllq_srAk}{Voodoo?}''

\textbf{Robert Jimison}: Diana Ross,
``\href{https://www.youtube.com/watch?v=QO93p3ZZ8Xw}{It's My House}''

\textbf{Sindhu Gnanasambandan:} Betty \& Oswald,
``\href{https://www.youtube.com/watch?v=J1JHVOfzVnc}{King of Bohemia}''

\textbf{Sofia Milan:} Bomba Estéreo,
``\href{https://www.youtube.com/watch?v=jgjRPk4E6Eg}{Mar (Lo Que
Siento)}''

\textbf{Stella Tan:} Erykah Badu and Common,
``\href{https://www.youtube.com/watch?v=l6mFqvcknNM}{Love Of My Life (An
Ode to Hip Hop)}''

\textbf{Sydney Harper:} Dua Lipa,
``\href{https://www.youtube.com/watch?v=ylzhMn6MlVc}{Pretty Please}''

\textbf{Wendy Dorr}: Loudon Wainwright III,
``\href{https://www.youtube.com/watch?v=fuz5TKzaJoE}{The Swimming
Song}''
\end{quote}

\hypertarget{on-the-daily-this-week}{%
\subsection{On The Daily this week}\label{on-the-daily-this-week}}

\textbf{Monday:} Brent Staples, a member of The Times editorial board,
\href{https://www.nytimes3xbfgragh.onion/2020/07/20/podcasts/the-daily/john-lewis.html}{honors
the extraordinary life} of the civil rights icon John Lewis.

\textbf{Tuesday:} Jan Hoffman explains why developing a coronavirus
vaccine may be easier than
\href{https://www.nytimes3xbfgragh.onion/2020/07/21/podcasts/the-daily/coronavirus-vaccine.html}{persuading
people to take it}.

\textbf{Wednesday:} Restarting the economy requires reopening schools.
Pam Belluck shares why doing so safely
\href{https://www.nytimes3xbfgragh.onion/2020/07/22/podcasts/the-daily/school-reopenings-coronavirus.html}{remains
one of our most daunting challenges}.

\textbf{Thursday:} With Zolan Kanno-Youngs and Mike Baker,
\href{https://www.nytimes3xbfgragh.onion/2020/07/23/podcasts/the-daily/portland-protests.html}{we
go behind protest lines in Portland} to ask: Why have militarized
federal forces been deployed to an American city?

\textbf{Friday:} Mike Schmidt speaks with the commissioner for Major
League Baseball
\href{https://www.nytimes3xbfgragh.onion/2020/07/24/podcasts/the-daily/mlb-baseball-season-coronavirus.html?action=click\&module=Briefings\&pgtype=Homepage}{about
the path forward for the sport}.

\emph{Have thoughts about the show? Tell us what you think at}
\href{mailto:thedaily@NYTimes.com?subject=The\%20Daily\%20newsletter}{\emph{thedaily@NYTimes.com}}\emph{.}

\emph{Were you forwarded this newsletter?}
\href{https://www.nytimes3xbfgragh.onion/newsletters/the-daily?te=1\&nl=the-daily\&emc=edit_dy_20190627}{\emph{Subscribe
here}} \emph{to get it delivered to your inbox.}

\emph{Love podcasts? Join The New York Times Podcast Club on}
\href{https://www.facebookcorewwwi.onion/groups/nytpodcastclub}{\emph{Facebook}}\emph{.}

Advertisement

\protect\hyperlink{after-bottom}{Continue reading the main story}

\hypertarget{site-index}{%
\subsection{Site Index}\label{site-index}}

\hypertarget{site-information-navigation}{%
\subsection{Site Information
Navigation}\label{site-information-navigation}}

\begin{itemize}
\tightlist
\item
  \href{https://help.nytimes3xbfgragh.onion/hc/en-us/articles/115014792127-Copyright-notice}{©~2020~The
  New York Times Company}
\end{itemize}

\begin{itemize}
\tightlist
\item
  \href{https://www.nytco.com/}{NYTCo}
\item
  \href{https://help.nytimes3xbfgragh.onion/hc/en-us/articles/115015385887-Contact-Us}{Contact
  Us}
\item
  \href{https://www.nytco.com/careers/}{Work with us}
\item
  \href{https://nytmediakit.com/}{Advertise}
\item
  \href{http://www.tbrandstudio.com/}{T Brand Studio}
\item
  \href{https://www.nytimes3xbfgragh.onion/privacy/cookie-policy\#how-do-i-manage-trackers}{Your
  Ad Choices}
\item
  \href{https://www.nytimes3xbfgragh.onion/privacy}{Privacy}
\item
  \href{https://help.nytimes3xbfgragh.onion/hc/en-us/articles/115014893428-Terms-of-service}{Terms
  of Service}
\item
  \href{https://help.nytimes3xbfgragh.onion/hc/en-us/articles/115014893968-Terms-of-sale}{Terms
  of Sale}
\item
  \href{https://spiderbites.nytimes3xbfgragh.onion}{Site Map}
\item
  \href{https://help.nytimes3xbfgragh.onion/hc/en-us}{Help}
\item
  \href{https://www.nytimes3xbfgragh.onion/subscription?campaignId=37WXW}{Subscriptions}
\end{itemize}
