Sections

SEARCH

\protect\hyperlink{site-content}{Skip to
content}\protect\hyperlink{site-index}{Skip to site index}

\href{https://www.nytimes3xbfgragh.onion/podcasts/the-daily}{The Daily}

\href{https://myaccount.nytimes3xbfgragh.onion/auth/login?response_type=cookie\&client_id=vi}{}

\href{https://www.nytimes3xbfgragh.onion/section/todayspaper}{Today's
Paper}

\href{/podcasts/the-daily}{The Daily}\textbar{}The Battle for a Baseball
Season

\url{https://nyti.ms/3jCJ4O0}

\begin{itemize}
\item
\item
\item
\item
\item
\item
\end{itemize}

Advertisement

\protect\hyperlink{after-top}{Continue reading the main story}

transcript

Back to The Daily

bars

0:00/45:34

-45:34

transcript

\hypertarget{the-battle-for-a-baseball-season}{%
\subsection{The Battle for a Baseball
Season}\label{the-battle-for-a-baseball-season}}

\hypertarget{hosted-by-michael-barbaro-produced-by-daniel-guillemette-clare-toeniskoetter-and-sydney-harper-and-edited-by-lisa-tobin-and-dave-shaw}{%
\subsubsection{Hosted by Michael Barbaro, produced by Daniel
Guillemette, Clare Toeniskoetter and Sydney Harper, and edited by Lisa
Tobin and Dave
Shaw}\label{hosted-by-michael-barbaro-produced-by-daniel-guillemette-clare-toeniskoetter-and-sydney-harper-and-edited-by-lisa-tobin-and-dave-shaw}}

\hypertarget{a-conversation-with-the-commissioner-of-major-league-baseball-about-the-path-to-this-seasons-start}{%
\paragraph{A conversation with the commissioner of Major League Baseball
about the path to this season's
start.}\label{a-conversation-with-the-commissioner-of-major-league-baseball-about-the-path-to-this-seasons-start}}

Friday, July 24th, 2020

\begin{itemize}
\tightlist
\item
  archived recording\\
  {[}PRESIDENT TRUMP AND SPORTSCASTER SPEAKING SIMULTANEOUSLY ON
  SEPARATE BROADCASTS{]}
\end{itemize}

mike schmidt

It's March 11, and Major League Baseball's commissioner, Rob Manfred, is
sitting at his home in Jupiter, Florida. It's 15 days before what will
likely be the most challenging season of his entire time as the sport's
commissioner. He's just been crushed in the press for how he's handled a
cheating scandal by the Houston Astros. Interest in this sport is
waning, and attendance is down. But that night, at home, in his living
room, he can see that the coronavirus is starting to engulf the country.

\begin{itemize}
\tightlist
\item
  archived recording (donald trump)\\
  From the beginning of time, nations and people have faced unforeseen
  challenges, including large-scale and very dangerous health threats.
\end{itemize}

{[}music{]}

mike schmidt

On his television, Donald Trump is addressing the country from the Oval
Office. And then Manfred stares down at his iPad.

\begin{itemize}
\tightlist
\item
  archived recording\\
  Completely uncharted territory. And we are going to get a little more
  information now from Adrian Wojnarowski, which is, he is with ---
\end{itemize}

mike schmidt

And he can see that ESPN has just moved a breaking news alert.

\begin{itemize}
\tightlist
\item
  archived recording\\
  Right now, the N.B.A. has made the decision. They have just announced
  that they are suspending play.
\end{itemize}

mike schmidt

--- that the NBA has suspended its season. And Manfred realizes, oh shit
---

\begin{itemize}
\tightlist
\item
  archived recording\\
  And then the league is going to use that hiatus to decide their next
  steps, how they'll go forward.
\end{itemize}

mike schmidt

--- my sport is supposed to begin its season, and the entire country is
shutting down around me. And I have to figure out how to get my sport
back on the field amid all this mess.

\begin{itemize}
\tightlist
\item
  archived recording\\
  And now a player has tested positive for it, the ripple effects that
  has on his own team, on other teams. And right now ---
\end{itemize}

michael barbaro

From The New York Times, I'm Michael Barbaro. This is ``The Daily.''
Today: My colleague Mike Schmidt on the fraught weeks that led up to
last night's opening game of the 2020 baseball season from the
perspective of the commissioner of Major League Baseball. It's Friday,
July 24.

Everybody thinks I've, like, never been to a baseball game.

mike schmidt

No, I was actually told that before we started.

michael barbaro

That's not true. I grew up in the United States of America.

mike schmidt

There's like a pre-briefing now for you, and they're like, yeah, he's
never been to a baseball game.

michael barbaro

So let me get this straight, somebody stands in the middle and throws a
ball.

speaker

Yes, sounds good. Let's do it there.

michael barbaro

OK. OK, we're going to start.

mike schmidt

All right.

michael barbaro

Mike, for the past three years, we have talked to you about the
president, Russia, the F.B.I., national security. We have not talked to
you about baseball. But here we are in this Hangout for you to tell us a
story about baseball. So just explain that.

mike schmidt

During the three years covering the Russia investigation, Mueller,
whether the president obstructed justice, to understand the story, I
focused on the characters who I thought drove it --- the former F.B.I.
director Jim Comey, the former White House counsel Don McGahn. But
earlier my career, in my first beat, when I covered baseball, the Comey
and McGahn of that story for my reporting was a little-known labor
lawyer named Rob Manfred.

\begin{itemize}
\tightlist
\item
  archived recording\\
  Waiting the start of the first of two days of hearings on the use of
  steroids, performance-enhancing drugs, in baseball.
\end{itemize}

mike schmidt

He was the official in the commissioner's office who had to deal with
the steroids scandal that was engulfing the sport.

\begin{itemize}
\item
  archived recording\\
  Mr. Robert Manfred, the executive vice president for labor and human
  resources of Major League Baseball.
\item
  archived recording (rob manfred)\\
  Thank you. Mr. Chairman, Ranking Member, Committee Members, I
  especially appreciate the opportunity to speak with you.
\end{itemize}

mike schmidt

Manfred and I had a pretty rough start to our relationship. I was an
overaggressive young reporter.

michael barbaro

(FEIGNING DISBELIEF) No.

mike schmidt

And he was this pugnacious, in-your-face, takes-no-prisoners lawyer who
had the sport's biggest problem on his desk. And we would just get on
the phone. We'd get on speakerphone. He'd put me on speakerphone. And he
would scream at me. And I would push back at him. And after a while, a
few years of this, I think we both sort of looked down at our hands and
realized that our hands were sort of bloodied, but we hadn't really
gotten anything out of it. And we started to build a much more
constructive relationship.

michael barbaro

Did you finally get off speakerphone?

mike schmidt

No. Rob's the kind of guy that will put you on speakerphone and tell you
what he's really thinking. Rob is not someone who waxes poetic about
baseball. He's someone that ended up working at baseball and is going to
do everything in his power to support and defend that sport. So I went
on to do other things. And he became the head of the sport. But as the
coronavirus was engulfing the country, I said to myself, there's a lot
of weighty, really important things going on right now. But the idea of
Rob sitting at home by himself trying to figure out how to save his
sports season, how to save the summer, was pretty appealing to me. So I
reached out to him, and I said, look, you are going to be spending the
rest of your career explaining how you dealt with this season. And you
and your sport both face an existential threat. If you don't have a
season, your sport's going to lose even more money. It's going to be
your legacy. Let's get on the phone and talk about what that's like.

\begin{itemize}
\item
  mike schmidt\\
  Rob?
\item
  rob manfred\\
  Yeah, just give me one second.
\end{itemize}

mike schmidt

And he agreed.

\begin{itemize}
\tightlist
\item
  rob manfred\\
  All right, I got to do this call with Schmidt. I'll call you in one
  minute, OK? {[}INAUDIBLE{]}
\end{itemize}

mike schmidt

So the first time I call him is on May 20.

\begin{itemize}
\tightlist
\item
  rob manfred\\
  All right, Michael, what do you want to talk about this morning?
\end{itemize}

mike schmidt

He's still at home in Florida. Like many people working from home, he's
taking Zoom calls.

\begin{itemize}
\tightlist
\item
  rob manfred\\
  Yeah, and literally what I'm doing is I got a regular series of calls
  to get feedback on ---
\end{itemize}

mike schmidt

And it's clear that he's immersed and knee-deep in the question, how do
you take a sport that's normally played in stadiums in front of
thousands and thousands of people, players are right up against each
other on the field, and every few days, teams, like a traveling
roadshow, go to another city to play another team that's coming from
another part of the country? So how do you do that in the age of Covid?

\begin{itemize}
\tightlist
\item
  rob manfred\\
  How do we get back to playing? One of the things that floated up from
  one of the experts is, gee whiz, a way that you can do this is to
  quarantine the players, right? And then ---
\end{itemize}

mike schmidt

He explains to me that there was initially an idea to quarantine all the
players in a bubble. Essentially the players would go to a location and
be cut off from the rest of society as they played the season.

\begin{itemize}
\item
  rob manfred\\
  And then you're going to start a four and a half month season. And
  your life is going to be hotel to ballpark, back to hotel, room
  service, not see your family.
\item
  mike schmidt\\
  You can't see your families. You can't be with your families.
\item
  rob manfred\\
  Yeah, I mean, that's one of the --- I mean, look, one of the
  quarantine --- you know, so then we realized, gee, that's pretty
  tough. So then we started talking about including families. And then
  you realize as you move into that phase that you get into quarantine
  numbers that are insane.
\end{itemize}

mike schmidt

So he says there was another plan, that baseball was essentially going
to play in three hubs --- Arizona, Texas and Florida.

\begin{itemize}
\tightlist
\item
  rob manfred\\
  Arizona for the West Coast teams, Texas for the Central teams,
  somewhere in Florida for the East Coast teams. That makes sense
  because those states seem to be more receptive to letting us play.
\end{itemize}

mike schmidt

The three parts of the country that had not really been hit heavily by
the virus.

michael barbaro

Mm-hmm. Not at that point.

mike schmidt

Right. But as they are weighing this plan, the country starts to open
up. So baseball again shifts its plan, and says, OK, the teams will play
in their stadiums and we will have a game, but it will have many, many
new restrictions that me, as a fan, and many fans, never could have
fathomed.

\begin{itemize}
\item
  mike schmidt\\
  What would a game look like now, as things are in place, based on what
  you have? You know, the Yankees and the Red Sox are playing tomorrow.
  What would that look like?
\item
  rob manfred\\
  Look, it's 67 pages of stuff. I mean, it's really thorough in terms of
  what people can --- you know, no high-fives, no spitting, hands
  sanitizing in between innings mandatory. No exchange of lineup cards
  at home plate. It's done via an app. Players who are not likely to
  play in the game are outside the dugout in the first couple of rows of
  the stands. The players ---
\end{itemize}

michael barbaro

So a pretty different version of baseball than we're used to.

mike schmidt

Totally. And there's an economic issue. The owners and the players know
that if they return to the field, it will almost certainly be a shorter
season and there will be less money to go around. And Manfred, as the
representative of the owners, thinks that he has an understanding with
the players about how that issue will be resolved.

\begin{itemize}
\tightlist
\item
  rob manfred\\
  We advanced them about \$170 million of salary. But they agreed in
  return that they would only get paid their salaries based on a
  prorated number of games. So in other words, if we only played 81
  games ---
\end{itemize}

mike schmidt

So it appears like the only thing standing in the way of baseball
returning to the field is the virus. And Manfred, as confident as an
executive as I've ever had to deal with, sounds confident about this,
and says, we're going to make this work.

\begin{itemize}
\item
  rob manfred\\
  Hey, Michael, I got to run for today. I'm happy to pick up the next
  time ---
\item
  mike schmidt\\
  That's fine. Let's do that. That's fine.
\item
  rob manfred\\
  OK. All right. Good to talk.
\item
  mike schmidt\\
  I appreciate it. OK.
\end{itemize}

mike schmidt

But by the next time we got on the phone, on June 11, everything had
changed.

{[}music{]}

\begin{itemize}
\item
  archived recording 1\\
  Let's kick it off with Major League Baseball and what's going on.
\item
  archived recording 2\\
  The players thought they had a deal for 100 percent prorated salaries.
  And the owners are saying, nah, you misunderstood.
\item
  archived recording 3\\
  We're not asking for our full salaries. We're just asking, whatever
  games we play, we'd like to get our game check for that game.
\item
  archived recording 4\\
  Wait a second. You're telling me you're not going to go to work to
  play a game we would all kill to play?
\item
  archived recording 5\\
  Bro, play for the love of the game, man. What's wrong with you, bro?
  Money should not be a thing.
\item
  archived recording 6\\
  Bro, I'm risking my life.
\item
  archived recording 7\\
  I don't believe that the players are going to look good when you've
  got 33-plus million people that have already filed for unemployment.
\item
  archived recording 8\\
  The subject comes up when, oh, greedy players, they make millions. And
  it's pointed out that the owners make billions. Like a lot of people,
  they got all worked up. (MOCKINGLY) There's not going to be any
  baseball. Look at this. They're so far apart. And I'm like, this is
  what they call negotiation. Am I right or wrong?
\item
  archived recording 9\\
  100 percent, 100 percent.
\item
  mike schmidt\\
  What's going on today?
\item
  rob manfred\\
  Well, you tell me. So it's the afternoon of June ---
\end{itemize}

mike schmidt

It's clear that the season is in doubt. But now it's not because of the
virus. It's all about that deal with the players.

michael barbaro

And Mike, what's the crux of this labor issue that the commissioner is
suddenly encountering?

mike schmidt

So at this point in the pandemic, it is clear to the owners and anyone
else paying attention that the last thing to come back is going to be
mass gatherings. And that means no fans in the stadium for any of the
season. Typically, at a game, you have anywhere from 20-, 30-, 40-,
50,000 people there. All those seats will be empty. And all that revenue
will no longer be there either. So the owners want to negotiate new
terms for what the players are going to be paid per game. Because the
owners say, we're going to be making even less money than we thought
because there will be no fans in the stands.

\begin{itemize}
\item
  mike schmidt\\
  What was the lowest moment of the past week?
\item
  rob manfred\\
  {[}SIGHS{]} Oh, you know, I think the union's last proposal, when they
  stayed at 100 --- You know, their failure to move, in response to what
  we thought was a pretty good proposal, was disappointing.
\end{itemize}

michael barbaro

And what's the response from the players?

mike schmidt

The players say, we're already taking a huge pay cut. A shorter season
means fewer games. We're paid per game. And this is a lot less money. So
now, even though we're having a shorter season, you want us to take more
of a pay cut? The players say, look, we only have a couple of years in
which we're in the league. The average career is five years. And you're
asking us to give up more money? What about the owners, who will be
there for many, many more years and are worth billions and billions of
dollars?

\begin{itemize}
\item
  mike schmidt\\
  Has there been a point in this where you sort of said to yourself,
  like, gosh, this is worse than I thought it would be? Because you're
  thinking, I'm going to go down as --- well, because it's an
  existential threat to the sport, right?
\item
  rob manfred\\
  Right.
\item
  mike schmidt\\
  It's an existential threat to you.
\item
  rob manfred\\
  Right.

  Yes, the outcome of no games is a massive threat to the good of the
  game.
\end{itemize}

mike schmidt

Remember, the sport has these other problems. Basketball has more of a
cultural following. Football has better ratings. There was the Astros
cheating scandal. There is the decline in attendance. And if baseball
doesn't come back amid a pandemic, at the same time that other sports
are, because players, many who are millionaires, and owners, many who
are billionaires, are having a fight over money, it could have a
devastating long-term impact on the sport.

{[}music{]}

And of course, looming in the back of Manfred's head, and everyone else
in baseball, is the fact that, in the sport's recent history, they did
lose a season because of labor issues. And baseball paid enormously for
it.

michael barbaro

We'll be right back.

mike schmidt

So as all this is going on with Manfred, I'm thinking of 1994.

michael barbaro

And what happened in 1994?

\begin{itemize}
\tightlist
\item
  archived recording\\
  It's Opening Day `94. Huge crowd. And oh, boy, the weather could not
  have been more cooperative. {[}SPORTS BROADCAST MUSIC{]}
\end{itemize}

mike schmidt

I'm 11 years old. And I am into baseball more than I've ever been.

\begin{itemize}
\tightlist
\item
  archived recording\\
  Yes, 11 games being played this afternoon in Major League Baseball on
  Opening Day 1994, including President Clinton at Jacobs Field in
  downtown Cleveland.
\end{itemize}

mike schmidt

I feel like I know nearly every player on every team.

\begin{itemize}
\tightlist
\item
  archived recording (announcer)\\
  The Yankees on top. And Mike Stanley to lead things off.
\end{itemize}

mike schmidt

I'm looking at the box scores every day. I'm watching SportsCenter in
the morning.

\begin{itemize}
\tightlist
\item
  archived recording (announcer)\\
  Look at this. {[}INAUDIBLE{]} Now, wait a minute.
\end{itemize}

mike schmidt

We just got a computer in the house. I'm printing out pictures of Yankee
players, and pasting them onto cardboard and putting them up in my
bedroom.

\begin{itemize}
\tightlist
\item
  archived recording\\
  And here is perhaps the most popular Padre of all time, Tony Gwynn,
  stepping in.
\end{itemize}

mike schmidt

It's also a magical season.

\begin{itemize}
\tightlist
\item
  archived recording\\
  --- is the right-hander. A ground ball, through the middle into center
  field. That's a base hit.
\end{itemize}

mike schmidt

It looks like the all-star Tony Gwynn is going to hit .400.

\begin{itemize}
\tightlist
\item
  archived recording\\
  And it's a home run! Three home runs in a row!
\end{itemize}

mike schmidt

Looks like the home run record may be broken.

\begin{itemize}
\tightlist
\item
  archived recording\\
  Uh oh, America likes that. It's gone. And the Yanks ride the home run
  out of the park with a 5-3 win.
\end{itemize}

mike schmidt

The Yankees are back. They're great again. I'm a Yankee fan.

\begin{itemize}
\item
  archived recording\\
  And the Expos have won 13 of their last 14. And now they're on a pace
  to win a 110 games. And in this, when I'm literally sitting at the
  edge of my seat as a fan, more engrossed in the game than I've ever
  been before ---

  The Expos leave the field in first place, yet wondering if their best
  season ever is in jeopardy. Even so, they're prepared to sit it out
  for as long as it takes.
\end{itemize}

mike schmidt

--- the season is stopped.

\begin{itemize}
\tightlist
\item
  archived recording (expos player)\\
  We didn't want to strike. We didn't want this to happen. But we have
  --- we didn't have no other choice but to go out and take care of
  ourselves and the game of baseball.
\end{itemize}

mike schmidt

In the middle of the summer, the players go on strike in a dispute with
the owners about money. And then ---

\begin{itemize}
\tightlist
\item
  archived recording (bud selig)\\
  But I'll say what I've said to many of you, either independently or
  collectively.
\end{itemize}

mike schmidt

The baseball commissioner at the time comes out ---

\begin{itemize}
\tightlist
\item
  archived recording (bud selig)\\
  Like a lot of things in life, you anticipate something and fear that
  it's coming, hope that it isn't.
\end{itemize}

mike schmidt

--- and announces that the World Series will be canceled.

\begin{itemize}
\tightlist
\item
  archived recording (bud selig)\\
  And when the day is here, there is an incredible amount of sadness.
\end{itemize}

mike schmidt

There will be no more baseball in 1994.

michael barbaro

And how is young, baseball-crazed Mike feeling as he hears that the
World Series has been canceled, the season is officially over?

mike schmidt

I was crushed. I was crushed. And there was nothing really to compare it
to. What I saw in that moment as a kid is something that I understand
better now after covering it, which is that there's two sides to
baseball. There's the romantic side. But there's the other side of it,
which is that it is a business. And baseball runs into problems when
business rears its head. And it rips the romanticism right out of it.
And what's interesting is that the commissioner at the time, in 1994,
the guy who actually had to cancel the season, was Bud Selig.

michael barbaro

What do you mean?

mike schmidt

Different from Manfred, he's a baseball romantic. If you get on the
phone with Selig, you always have to listen to him regale and tell
stories about baseball history. Just a deep-seated love of the game. And
he was the person that had to cancel the 1994 season. He was the one
that had to put his name on the statement that came out and said, there
will be no World Series.

{[}music{]}

{[}phone ringing{]}

mike schmidt

So as this season --- the 2020 season --- looked in doubt, I thought,
the person to call is Bud Selig.

\begin{itemize}
\item
  bud selig\\
  Hello.
\item
  mike schmidt\\
  Commissioner.
\item
  bud selig\\
  Can you hear me all right?
\item
  mike schmidt\\
  I can hear you very good.
\item
  bud selig\\
  Good.
\end{itemize}

mike schmidt

Because he understands more than anyone else the situation that Manfred
finds himself in. So we're looking at the whole question of a baseball
season.

\begin{itemize}
\item
  bud selig\\
  Right.
\item
  mike schmidt\\
  Take us back to 1994. Tell us that story. And tell us why a baseball
  season is so important.
\item
  bud selig\\
  Well let me, as I always do, Mike, give you a little history.
\end{itemize}

mike schmidt

So like I said, Selig starts with some baseball history. He goes back to
World War II. Hundreds of players were sent off to fight. But even as
they were at war ---

\begin{itemize}
\tightlist
\item
  bud selig\\
  FDR had written a letter in December of `42, 1942, urging baseball to
  continue.
\end{itemize}

mike schmidt

The season was not in question.

\begin{itemize}
\item
  bud selig\\
  And so, through this unbelievable Second World War, they did play
  baseball. So we go to 1994. I guess it was about the 16th of
  September. If my memory serves me well, and I think it does, I was in
  County Stadium, and we were going to have to announce that there would
  be no season. And that night I came home, and I sat upstairs in the
  den, and I replayed every World Series from 1945.
\item
  archived recording (don dunphy)\\
  Good afternoon, everyone. This is Don Dunphy speaking for Bill Coram
  and Bill Slater.
\item
  bud selig\\
  My first recollection was the Cardinal St. Louis Browns.
\item
  archived recording\\
  Here it comes. He swings on it, hits it over Marion's head in left
  field for a hit.
\item
  bud selig\\
  So I just sat there very quietly, in deep thought.
\item
  archived recording\\
  Man on first, one away. Third baseman Mark Christman.
\item
  bud selig\\
  Replayed all that. And I was heartsick. It's one of the low, low
  moments of my career and in my life. Because a World War couldn't
  eliminate the World Series.

  It was really sad.
\item
  mike schmidt\\
  When you're sitting there that night in the den, what else is going
  through your head?
\item
  bud selig\\
  Well, one thing I don't think any of us ever really understood was how
  much it was hurting the sport. Fans were angry. And we dragged them
  through the mud. I mean, it's now 26 years ago, and the pain, as I sit
  here and talk to you, comes back to me. And I was worried. It scared
  me.
\end{itemize}

{[}music{]}

\begin{itemize}
\item
  archived recording 1\\
  How do you feel about a baseball strike?
\item
  archived recording 2\\
  I think it's stupid. I think the players are just being selfish.
\item
  archived recording 3\\
  So do I.
\item
  archived recording 4\\
  You think the players are being selfish?
\item
  archived recording 5\\
  Yeah. They make enough money anyways.
\item
  archived recording 6\\
  It would be pretty boring without baseball in the summer. It just
  ruins the game the way they're striking.
\item
  archived recording 7\\
  I don't know. I think everyone's money hungry.
\item
  archived recording 8\\
  I wish they'd sign an agreement and keep on playing.
\item
  archived recording 9\\
  Should they play for free?
\item
  archived recording 10\\
  Yeah.
\item
  archived recording 11\\
  Why?
\item
  archived recording 12\\
  Because it's just a game. It doesn't matter how much money you get.
\item
  archived recording 13\\
  And the owners, they say they don't make enough money. Well, the
  question is, what is enough money?
\item
  archived recording 14\\
  So what are you going to do?
\item
  archived recording 15\\
  I don't know. I guess just watch minor league baseball.
\item
  archived recording 16\\
  I already know what I'm going to start doing. I'm going to start
  rereading Dante's ``Inferno,'' because that's where I think they
  should send the whole lot of them.
\end{itemize}

michael barbaro

So as you probably suspected, it wasn't just 11-year-old Mike who was
mad at baseball. It was ---

mike schmidt

No. And as bad as `94 was, `95, `96 and `97 were just as bad because
fans were so upset by the strike that the sport had to rebuild itself
and rebuild its credibility with the fans.

michael barbaro

Mike, how much do you think 1994 is on the mind of Manfred?

mike schmidt

Enormously, huge, front of mind. Because that labor lawyer who I met
back when I was covering drugs in baseball, he was just starting out in
baseball in 1994. And he was a junior lawyer who was deeply involved in
the strike and trying to help the owners win the labor fight.

\begin{itemize}
\tightlist
\item
  bud selig\\
  And that's when I met him. And the more I saw Rob, the more I liked
  him. And he and I worked well together. Somebody asked me the other
  day, how often did you talk to Rob back then? Maybe 10 times a day,
  even. More than he wanted, I may add. {[}CHUCKLES{]}
\end{itemize}

mike schmidt

So after Manfred became commissioner, Selig tried to give him some space
and some distance. He didn't want to look like the father telling the
son how to run the sport. And a bit of distance grew between the two of
them. But ---

\begin{itemize}
\tightlist
\item
  bud selig\\
  One thing we'll say to each other, Mike ---
\end{itemize}

mike schmidt

--- as Manfred found himself in this situation, he began to lean back on
Selig.

\begin{itemize}
\tightlist
\item
  bud selig\\
  I'm the only other guy on the face of the earth that understands
  exactly what the pressure is and what the situation is.
\end{itemize}

mike schmidt

And they talked more than they had at any other point since Manfred had
become commissioner.

\begin{itemize}
\item
  bud selig\\
  Since I've done that job for 22 and a half years. And I'm the only one
  who understands what he's gone through. So yeah, there's no question
  about it.
\item
  mike schmidt\\
  I really appreciate it.
\item
  bud selig\\
  Well, great. Well, I hope you enjoyed it. It was a pleasure to do it.
  And we'll talk soon.
\item
  mike schmidt\\
  Thanks.
\item
  bud selig\\
  Bye.
\end{itemize}

michael barbaro

So what happens next in the story, Mike?

mike schmidt

So at this point, it's been more than a month since my first
conversation with Rob. And things are getting really nasty between the
owners and the players.

{[}music{]}

\begin{itemize}
\item
  archived recording 1\\
  --- the middle of June. And boy, what Astros fans would give to be
  sitting in the stands at Minute Maid Park right about now.
\item
  archived recording 2\\
  It's true. And we were hopeful.
\end{itemize}

{[}music{]}

\begin{itemize}
\item
  archived recording 3\\
  Baseball, negotiations grinding to a halt.
\item
  archived recording 4\\
  Like, we're almost to July now, and there's still nothing. I mean,
  we're still in the same position that we were in at March.
\item
  archived recording 5\\
  And if they don't do something about it, the sport is going to fade
  even more.
\item
  archived recording 6\\
  How are you a commissioner --- like, baseball should have been back a
  month ago. They should be basically saying, here's our opportunity to
  recapture an audience.
\item
  archived recording 7\\
  It's just it's unfortunate that it's been so public. I think fans have
  been turned away a little bit.
\item
  archived recording 8\\
  There's a reason Major League Baseball's executive office is filled
  with labor lawyers. Because there's a labor fight every 12 years in
  this league.
\end{itemize}

mike schmidt

So Manfred becomes so frustrated that he decides to go out on television
and say ---

\begin{itemize}
\tightlist
\item
  archived recording (rob manfred)\\
  Well, I know the owners are 100 percent committed to getting baseball
  back on the field. Unfortunately, I can't tell you that I'm 100
  percent certain that it's going to happen.
\end{itemize}

mike schmidt

You know, I said there was going to be a season. But actually, now, I'm
not sure.

michael barbaro

Hmm.

mike schmidt

And he's trying any attempt to restart the negotiations, get the players
back to the table and move forward. That ultimately doesn't really work.
And he has to go out on his own and announces that a 60-game season will
start on July 23.

michael barbaro

So Mike, he can do that, just call a 60-game season without a deal
between the players and the owners?

mike schmidt

It wasn't his first choice. He wanted both sides to buy in. He wanted a
better deal on how much the players would take a reduction in salary.
But without any other choice and his deep desire to have a season at
pretty much any cost, that was his only option.

michael barbaro

So is this seen as a win for the players or for the owners?

mike schmidt

Both and neither. The players are going to get paid their full salary
for the games that they play. Manfred is going to get his season. But
neither of them are walking away feeling good about their relationship.
And in the two months that they've taken to resolve this labor issue,
they're now back to a health and safety problem. Because in that time,
Covid has exploded and spread to new states. And actually, today, as I
was preparing to talk to Rob for the final time, news broke that the
all-star player on the Nationals, Juan Soto, had tested positive for
Covid. So here's Manfred, on the cusp of having the season he fought so
hard for, learning just hours before the first pitch, that one of the
star players may have the virus.

michael barbaro

Not ideal.

mike schmidt

Another ``oh shit'' moment in a long season.

{[}music{]}

\begin{itemize}
\item
  mike schmidt\\
  Commissioner.
\item
  rob manfred\\
  Hey, Schmidt. How are you?
\item
  mike schmidt\\
  Hell of a day.
\item
  rob manfred\\
  Well, we're going to make it to the starting line. {[}CHUCKLES{]}
  Everybody seems excited, like we've done something. All we did was get
  out of the gate, you know what I mean? {[}CHUCKLES{]} The hard part is
  playing 60 games, you know? Anyways, I'm glad we are where we are. You
  know, I feel pretty good about it.
\item
  mike schmidt\\
  Where are you right now?
\item
  rob manfred\\
  I'm in Washington. I'm at Nationals Park.
\item
  mike schmidt\\
  What's the feeling in the air? You're at opening day with no fans and
  just the members of the staff. What does that feel like?
\item
  rob manfred\\
  Like no opening day I've ever seen, I'll tell you that. {[}CHUCKLES{]}
  It's really different, Mike. I mean, it's very stark right now. It's
  early still, but it's very stark right now.
\item
  mike schmidt\\
  When you heard today that Juan Soto contracted Covid --- we're talking
  about the all-star on the world championship team --- are you like, oh
  no, we can't do this? What's your mindset?
\item
  rob manfred\\
  I mean, look, my initial reaction is I can't believe this is happening
  on opening day. But then I dropped back and I thought about, we knew
  we were going to have positives. It's unfortunate that it was opening
  day and that it was Juan Soto. But the protocols were built to deal
  with this. The whole point is you've got to build a system that's
  flexible enough to deal with what's coming. We knew it was coming.
\item
  mike schmidt\\
  I'm mindful that today is July 23. The first time we spoke was May 20.
  It's been two months. What we've got now is pretty much the plan that
  you had back then for the virus. But in this period of time, you went
  through this whole tumultuous thing. If you could go back, would you
  have done anything differently? And is there any mistakes you made in
  the process?
\item
  rob manfred\\
  Well if I could go back, I'd love an opportunity to replay that hand.
  I really would, Michael. I think that one thing I can certainly point
  to, the whole, from the very beginning, the back-and-forth in the
  press and all that. I just --- I tried to avoid it. I didn't manage to
  do it. I'd love to have had a chance to go back and do it over again
  and be better at it.
\item
  mike schmidt\\
  Do you think there's long-term damage from it?
\item
  rob manfred\\
  You know, I think that --- I do think it was unsightly and we should
  not have allowed it to happen. I think we sort of have a debt to our
  fans.
\item
  mike schmidt\\
  Let me ask it this way. The two months of nasty, public,
  back-and-forth negotiations between the owners and the players, do you
  think that will have long-term damage to the sport going forward,
  similar to `94?
\item
  rob manfred\\
  I just --- I don't know what to say to that one, Michael. I just don't
  know.
\item
  mike schmidt\\
  So tonight, on the field, will be exactly what you mapped out. The
  players will be distanced, no high-fives, no spitting. What is your
  hope for how this game feels to the fans watching at home.
\item
  rob manfred\\
  Honestly, I hope that at the end of the night, what fans are thinking
  is, you know what, it's not everything that we're used to and love
  about the game, but you know what, it's great to have baseball.
\item
  archived recording\\
  In an empty stadium, and no acknowledgment from the fans, and no
  acknowledgment to the fans from those there.
\item
  mike schmidt\\
  We're done. That's it.
\item
  rob manfred\\
  Thanks, Schmidt. I'll talk to you soon, huh?
\item
  mike schmidt\\
  Bye.
\item
  archived recording 1\\
  --- we get a real good feel for tonight. And now one of the more
  well-known Washington National fans, Dr. Anthony Fauci, to throw out
  the first pitch.
\end{itemize}

{[}music{]}

\begin{itemize}
\tightlist
\item
  archived recording 2\\
  Dr. Anthony Fauci!
\end{itemize}

michael barbaro

So Mike, I am talking to you at 9:30 p.m. on Thursday evening. The game
is currently underway. And the first pitch was thrown out by Dr. Anthony
Fauci, and it was pretty wobbly. I assume you've been watching the game.

mike schmidt

Yes. And we're now in the middle of a rain delay.

michael barbaro

Right. Which is perhaps why you're talking to me. I know you didn't set
out to be philosophical about baseball with this assignment, but if we
could get philosophical for just a moment I wonder how you're thinking
about baseball and this game right now.

mike schmidt

Look, I told you about my first realization when I was 11 years old,
that baseball walks a fine line between being a game and a business.

When I became a sports reporter, I covered the darkest underbelly of the
sport, and saw it in probably a nastier light than most fans could ever
dream of. And in the years after that, I had a hard time falling back
into the romantic fandom of baseball. I had just seen too much. But last
year, almost a decade after I left sports, I got my fandom back. I
caught the bug again. And as 2020 started, I was ready to continue that
and to try and be that fan again. These negotiations brought back those
feelings of the two-headed monster of baseball, and the business head
becoming too big. But at the end of the day, there's going to be a
season. And it's going to look weird and feel very, very different. But
it's a season. And it's baseball.

michael barbaro

In other words, you're still a little bit of a romantic.

mike schmidt

Look, I'll take it. I'll take it for now.

{[}music{]}

\begin{itemize}
\tightlist
\item
  archived recording\\
  --- games last season, a couple of very difficult I.L. stints really
  prevented him from accomplishing much of anything.
\end{itemize}

michael barbaro

Mike, thank you very much. Enjoy the game. Enjoy the season.

mike schmidt

Thanks for having me.

\begin{itemize}
\item
  archived recording 1\\
  Swung on and hit high in the air to left center. That ball is high, it
  is far, it is gone. Way back in the left-center field seats.
\item
  archived recording 2\\
  Had there been fans in the ballpark, it was a guy that bought the
  worst seat that would have gotten that souvenir.
\item
  archived recording 3\\
  Oh, what a shot by Stanton. It's a two-run dinger. And the Yankees
  immediately take a 2-0 lead.
\item
  archived recording 4\\
  He turned around a 96-mile-an-hour fastball 459 feet. What's the
  saying we always have? The harder it comes in, the harder it goes out.
\item
  archived recording 5\\
  And that's the MVP swing that the New York Yankees acquired from the
  Marlins.
\item
  archived recording 6\\
  {[}CHUCKLES{]} As the old line goes, the ball went so far they should
  serve a meal on it. So the Yanks take a 2-0 lead. That was Stanton's
  21st career ---
\end{itemize}

michael barbaro

We'll be right back.

{[}music{]}

Here's what else you need to know today. On Thursday, the United States
reached a new milestone in the pandemic, with 4 million known
infections. Infections are now on the rise in 39 different states;
Washington, D.C.; Puerto Rico; and the U.S. Virgin Islands.

\begin{itemize}
\tightlist
\item
  archived recording (donald trump)\\
  Everything was going well, a tremendous list of speakers, thousands of
  people wanting to be there, and I mean, in some cases, desperately be
  there. They wanted to attend.
\end{itemize}

michael barbaro

At the White House, President Trump said he would cancel the public
portion of the Republican National Convention scheduled for August. To
avoid strict social distancing rules the president had moved the events
from North Carolina to Florida, which now has the highest infection rate
in the country.

\href{https://www.nytimes3xbfgragh.onion/column/the-daily}{\includegraphics{https://static01.graylady3jvrrxbe.onion/images/2017/01/29/podcasts/the-daily-album-art/the-daily-album-art-square320-v4.png}The
Daily}Subscribe:

\begin{itemize}
\tightlist
\item
  \href{https://itunes.apple.com/us/podcast/id1200361736}{Apple
  Podcasts}
\item
  \href{https://www.google.com/podcasts?feed=aHR0cHM6Ly9yc3MuYXJ0MTkuY29tL3RoZS1kYWlseQ\%3D\%3D}{Google
  Podcasts}
\end{itemize}

\hypertarget{the-battle-for-a-baseball-season-1}{%
\section{The Battle for a Baseball
Season}\label{the-battle-for-a-baseball-season-1}}

\hypertarget{a-conversation-with-the-commissioner-of-major-league-baseball-about-the-path-to-this-seasons-start-1}{%
\subsection{A conversation with the commissioner of Major League
Baseball about the path to this season's
start.}\label{a-conversation-with-the-commissioner-of-major-league-baseball-about-the-path-to-this-seasons-start-1}}

Hosted by Michael Barbaro, produced by Daniel Guillemette, Clare
Toeniskoetter and Sydney Harper, and edited by Lisa Tobin and Dave Shaw

Transcript

transcript

Back to The Daily

bars

0:00/45:34

-0:00

transcript

\hypertarget{the-battle-for-a-baseball-season-2}{%
\subsection{The Battle for a Baseball
Season}\label{the-battle-for-a-baseball-season-2}}

\hypertarget{hosted-by-michael-barbaro-produced-by-daniel-guillemette-clare-toeniskoetter-and-sydney-harper-and-edited-by-lisa-tobin-and-dave-shaw-1}{%
\subsubsection{Hosted by Michael Barbaro, produced by Daniel
Guillemette, Clare Toeniskoetter and Sydney Harper, and edited by Lisa
Tobin and Dave
Shaw}\label{hosted-by-michael-barbaro-produced-by-daniel-guillemette-clare-toeniskoetter-and-sydney-harper-and-edited-by-lisa-tobin-and-dave-shaw-1}}

\hypertarget{a-conversation-with-the-commissioner-of-major-league-baseball-about-the-path-to-this-seasons-start-2}{%
\paragraph{A conversation with the commissioner of Major League Baseball
about the path to this season's
start.}\label{a-conversation-with-the-commissioner-of-major-league-baseball-about-the-path-to-this-seasons-start-2}}

Friday, July 24th, 2020

\begin{itemize}
\tightlist
\item
  archived recording\\
  {[}PRESIDENT TRUMP AND SPORTSCASTER SPEAKING SIMULTANEOUSLY ON
  SEPARATE BROADCASTS{]}
\end{itemize}

mike schmidt

It's March 11, and Major League Baseball's commissioner, Rob Manfred, is
sitting at his home in Jupiter, Florida. It's 15 days before what will
likely be the most challenging season of his entire time as the sport's
commissioner. He's just been crushed in the press for how he's handled a
cheating scandal by the Houston Astros. Interest in this sport is
waning, and attendance is down. But that night, at home, in his living
room, he can see that the coronavirus is starting to engulf the country.

\begin{itemize}
\tightlist
\item
  archived recording (donald trump)\\
  From the beginning of time, nations and people have faced unforeseen
  challenges, including large-scale and very dangerous health threats.
\end{itemize}

{[}music{]}

mike schmidt

On his television, Donald Trump is addressing the country from the Oval
Office. And then Manfred stares down at his iPad.

\begin{itemize}
\tightlist
\item
  archived recording\\
  Completely uncharted territory. And we are going to get a little more
  information now from Adrian Wojnarowski, which is, he is with ---
\end{itemize}

mike schmidt

And he can see that ESPN has just moved a breaking news alert.

\begin{itemize}
\tightlist
\item
  archived recording\\
  Right now, the N.B.A. has made the decision. They have just announced
  that they are suspending play.
\end{itemize}

mike schmidt

--- that the NBA has suspended its season. And Manfred realizes, oh shit
---

\begin{itemize}
\tightlist
\item
  archived recording\\
  And then the league is going to use that hiatus to decide their next
  steps, how they'll go forward.
\end{itemize}

mike schmidt

--- my sport is supposed to begin its season, and the entire country is
shutting down around me. And I have to figure out how to get my sport
back on the field amid all this mess.

\begin{itemize}
\tightlist
\item
  archived recording\\
  And now a player has tested positive for it, the ripple effects that
  has on his own team, on other teams. And right now ---
\end{itemize}

michael barbaro

From The New York Times, I'm Michael Barbaro. This is ``The Daily.''
Today: My colleague Mike Schmidt on the fraught weeks that led up to
last night's opening game of the 2020 baseball season from the
perspective of the commissioner of Major League Baseball. It's Friday,
July 24.

Everybody thinks I've, like, never been to a baseball game.

mike schmidt

No, I was actually told that before we started.

michael barbaro

That's not true. I grew up in the United States of America.

mike schmidt

There's like a pre-briefing now for you, and they're like, yeah, he's
never been to a baseball game.

michael barbaro

So let me get this straight, somebody stands in the middle and throws a
ball.

speaker

Yes, sounds good. Let's do it there.

michael barbaro

OK. OK, we're going to start.

mike schmidt

All right.

michael barbaro

Mike, for the past three years, we have talked to you about the
president, Russia, the F.B.I., national security. We have not talked to
you about baseball. But here we are in this Hangout for you to tell us a
story about baseball. So just explain that.

mike schmidt

During the three years covering the Russia investigation, Mueller,
whether the president obstructed justice, to understand the story, I
focused on the characters who I thought drove it --- the former F.B.I.
director Jim Comey, the former White House counsel Don McGahn. But
earlier my career, in my first beat, when I covered baseball, the Comey
and McGahn of that story for my reporting was a little-known labor
lawyer named Rob Manfred.

\begin{itemize}
\tightlist
\item
  archived recording\\
  Waiting the start of the first of two days of hearings on the use of
  steroids, performance-enhancing drugs, in baseball.
\end{itemize}

mike schmidt

He was the official in the commissioner's office who had to deal with
the steroids scandal that was engulfing the sport.

\begin{itemize}
\item
  archived recording\\
  Mr. Robert Manfred, the executive vice president for labor and human
  resources of Major League Baseball.
\item
  archived recording (rob manfred)\\
  Thank you. Mr. Chairman, Ranking Member, Committee Members, I
  especially appreciate the opportunity to speak with you.
\end{itemize}

mike schmidt

Manfred and I had a pretty rough start to our relationship. I was an
overaggressive young reporter.

michael barbaro

(FEIGNING DISBELIEF) No.

mike schmidt

And he was this pugnacious, in-your-face, takes-no-prisoners lawyer who
had the sport's biggest problem on his desk. And we would just get on
the phone. We'd get on speakerphone. He'd put me on speakerphone. And he
would scream at me. And I would push back at him. And after a while, a
few years of this, I think we both sort of looked down at our hands and
realized that our hands were sort of bloodied, but we hadn't really
gotten anything out of it. And we started to build a much more
constructive relationship.

michael barbaro

Did you finally get off speakerphone?

mike schmidt

No. Rob's the kind of guy that will put you on speakerphone and tell you
what he's really thinking. Rob is not someone who waxes poetic about
baseball. He's someone that ended up working at baseball and is going to
do everything in his power to support and defend that sport. So I went
on to do other things. And he became the head of the sport. But as the
coronavirus was engulfing the country, I said to myself, there's a lot
of weighty, really important things going on right now. But the idea of
Rob sitting at home by himself trying to figure out how to save his
sports season, how to save the summer, was pretty appealing to me. So I
reached out to him, and I said, look, you are going to be spending the
rest of your career explaining how you dealt with this season. And you
and your sport both face an existential threat. If you don't have a
season, your sport's going to lose even more money. It's going to be
your legacy. Let's get on the phone and talk about what that's like.

\begin{itemize}
\item
  mike schmidt\\
  Rob?
\item
  rob manfred\\
  Yeah, just give me one second.
\end{itemize}

mike schmidt

And he agreed.

\begin{itemize}
\tightlist
\item
  rob manfred\\
  All right, I got to do this call with Schmidt. I'll call you in one
  minute, OK? {[}INAUDIBLE{]}
\end{itemize}

mike schmidt

So the first time I call him is on May 20.

\begin{itemize}
\tightlist
\item
  rob manfred\\
  All right, Michael, what do you want to talk about this morning?
\end{itemize}

mike schmidt

He's still at home in Florida. Like many people working from home, he's
taking Zoom calls.

\begin{itemize}
\tightlist
\item
  rob manfred\\
  Yeah, and literally what I'm doing is I got a regular series of calls
  to get feedback on ---
\end{itemize}

mike schmidt

And it's clear that he's immersed and knee-deep in the question, how do
you take a sport that's normally played in stadiums in front of
thousands and thousands of people, players are right up against each
other on the field, and every few days, teams, like a traveling
roadshow, go to another city to play another team that's coming from
another part of the country? So how do you do that in the age of Covid?

\begin{itemize}
\tightlist
\item
  rob manfred\\
  How do we get back to playing? One of the things that floated up from
  one of the experts is, gee whiz, a way that you can do this is to
  quarantine the players, right? And then ---
\end{itemize}

mike schmidt

He explains to me that there was initially an idea to quarantine all the
players in a bubble. Essentially the players would go to a location and
be cut off from the rest of society as they played the season.

\begin{itemize}
\item
  rob manfred\\
  And then you're going to start a four and a half month season. And
  your life is going to be hotel to ballpark, back to hotel, room
  service, not see your family.
\item
  mike schmidt\\
  You can't see your families. You can't be with your families.
\item
  rob manfred\\
  Yeah, I mean, that's one of the --- I mean, look, one of the
  quarantine --- you know, so then we realized, gee, that's pretty
  tough. So then we started talking about including families. And then
  you realize as you move into that phase that you get into quarantine
  numbers that are insane.
\end{itemize}

mike schmidt

So he says there was another plan, that baseball was essentially going
to play in three hubs --- Arizona, Texas and Florida.

\begin{itemize}
\tightlist
\item
  rob manfred\\
  Arizona for the West Coast teams, Texas for the Central teams,
  somewhere in Florida for the East Coast teams. That makes sense
  because those states seem to be more receptive to letting us play.
\end{itemize}

mike schmidt

The three parts of the country that had not really been hit heavily by
the virus.

michael barbaro

Mm-hmm. Not at that point.

mike schmidt

Right. But as they are weighing this plan, the country starts to open
up. So baseball again shifts its plan, and says, OK, the teams will play
in their stadiums and we will have a game, but it will have many, many
new restrictions that me, as a fan, and many fans, never could have
fathomed.

\begin{itemize}
\item
  mike schmidt\\
  What would a game look like now, as things are in place, based on what
  you have? You know, the Yankees and the Red Sox are playing tomorrow.
  What would that look like?
\item
  rob manfred\\
  Look, it's 67 pages of stuff. I mean, it's really thorough in terms of
  what people can --- you know, no high-fives, no spitting, hands
  sanitizing in between innings mandatory. No exchange of lineup cards
  at home plate. It's done via an app. Players who are not likely to
  play in the game are outside the dugout in the first couple of rows of
  the stands. The players ---
\end{itemize}

michael barbaro

So a pretty different version of baseball than we're used to.

mike schmidt

Totally. And there's an economic issue. The owners and the players know
that if they return to the field, it will almost certainly be a shorter
season and there will be less money to go around. And Manfred, as the
representative of the owners, thinks that he has an understanding with
the players about how that issue will be resolved.

\begin{itemize}
\tightlist
\item
  rob manfred\\
  We advanced them about \$170 million of salary. But they agreed in
  return that they would only get paid their salaries based on a
  prorated number of games. So in other words, if we only played 81
  games ---
\end{itemize}

mike schmidt

So it appears like the only thing standing in the way of baseball
returning to the field is the virus. And Manfred, as confident as an
executive as I've ever had to deal with, sounds confident about this,
and says, we're going to make this work.

\begin{itemize}
\item
  rob manfred\\
  Hey, Michael, I got to run for today. I'm happy to pick up the next
  time ---
\item
  mike schmidt\\
  That's fine. Let's do that. That's fine.
\item
  rob manfred\\
  OK. All right. Good to talk.
\item
  mike schmidt\\
  I appreciate it. OK.
\end{itemize}

mike schmidt

But by the next time we got on the phone, on June 11, everything had
changed.

{[}music{]}

\begin{itemize}
\item
  archived recording 1\\
  Let's kick it off with Major League Baseball and what's going on.
\item
  archived recording 2\\
  The players thought they had a deal for 100 percent prorated salaries.
  And the owners are saying, nah, you misunderstood.
\item
  archived recording 3\\
  We're not asking for our full salaries. We're just asking, whatever
  games we play, we'd like to get our game check for that game.
\item
  archived recording 4\\
  Wait a second. You're telling me you're not going to go to work to
  play a game we would all kill to play?
\item
  archived recording 5\\
  Bro, play for the love of the game, man. What's wrong with you, bro?
  Money should not be a thing.
\item
  archived recording 6\\
  Bro, I'm risking my life.
\item
  archived recording 7\\
  I don't believe that the players are going to look good when you've
  got 33-plus million people that have already filed for unemployment.
\item
  archived recording 8\\
  The subject comes up when, oh, greedy players, they make millions. And
  it's pointed out that the owners make billions. Like a lot of people,
  they got all worked up. (MOCKINGLY) There's not going to be any
  baseball. Look at this. They're so far apart. And I'm like, this is
  what they call negotiation. Am I right or wrong?
\item
  archived recording 9\\
  100 percent, 100 percent.
\item
  mike schmidt\\
  What's going on today?
\item
  rob manfred\\
  Well, you tell me. So it's the afternoon of June ---
\end{itemize}

mike schmidt

It's clear that the season is in doubt. But now it's not because of the
virus. It's all about that deal with the players.

michael barbaro

And Mike, what's the crux of this labor issue that the commissioner is
suddenly encountering?

mike schmidt

So at this point in the pandemic, it is clear to the owners and anyone
else paying attention that the last thing to come back is going to be
mass gatherings. And that means no fans in the stadium for any of the
season. Typically, at a game, you have anywhere from 20-, 30-, 40-,
50,000 people there. All those seats will be empty. And all that revenue
will no longer be there either. So the owners want to negotiate new
terms for what the players are going to be paid per game. Because the
owners say, we're going to be making even less money than we thought
because there will be no fans in the stands.

\begin{itemize}
\item
  mike schmidt\\
  What was the lowest moment of the past week?
\item
  rob manfred\\
  {[}SIGHS{]} Oh, you know, I think the union's last proposal, when they
  stayed at 100 --- You know, their failure to move, in response to what
  we thought was a pretty good proposal, was disappointing.
\end{itemize}

michael barbaro

And what's the response from the players?

mike schmidt

The players say, we're already taking a huge pay cut. A shorter season
means fewer games. We're paid per game. And this is a lot less money. So
now, even though we're having a shorter season, you want us to take more
of a pay cut? The players say, look, we only have a couple of years in
which we're in the league. The average career is five years. And you're
asking us to give up more money? What about the owners, who will be
there for many, many more years and are worth billions and billions of
dollars?

\begin{itemize}
\item
  mike schmidt\\
  Has there been a point in this where you sort of said to yourself,
  like, gosh, this is worse than I thought it would be? Because you're
  thinking, I'm going to go down as --- well, because it's an
  existential threat to the sport, right?
\item
  rob manfred\\
  Right.
\item
  mike schmidt\\
  It's an existential threat to you.
\item
  rob manfred\\
  Right.

  Yes, the outcome of no games is a massive threat to the good of the
  game.
\end{itemize}

mike schmidt

Remember, the sport has these other problems. Basketball has more of a
cultural following. Football has better ratings. There was the Astros
cheating scandal. There is the decline in attendance. And if baseball
doesn't come back amid a pandemic, at the same time that other sports
are, because players, many who are millionaires, and owners, many who
are billionaires, are having a fight over money, it could have a
devastating long-term impact on the sport.

{[}music{]}

And of course, looming in the back of Manfred's head, and everyone else
in baseball, is the fact that, in the sport's recent history, they did
lose a season because of labor issues. And baseball paid enormously for
it.

michael barbaro

We'll be right back.

mike schmidt

So as all this is going on with Manfred, I'm thinking of 1994.

michael barbaro

And what happened in 1994?

\begin{itemize}
\tightlist
\item
  archived recording\\
  It's Opening Day `94. Huge crowd. And oh, boy, the weather could not
  have been more cooperative. {[}SPORTS BROADCAST MUSIC{]}
\end{itemize}

mike schmidt

I'm 11 years old. And I am into baseball more than I've ever been.

\begin{itemize}
\tightlist
\item
  archived recording\\
  Yes, 11 games being played this afternoon in Major League Baseball on
  Opening Day 1994, including President Clinton at Jacobs Field in
  downtown Cleveland.
\end{itemize}

mike schmidt

I feel like I know nearly every player on every team.

\begin{itemize}
\tightlist
\item
  archived recording (announcer)\\
  The Yankees on top. And Mike Stanley to lead things off.
\end{itemize}

mike schmidt

I'm looking at the box scores every day. I'm watching SportsCenter in
the morning.

\begin{itemize}
\tightlist
\item
  archived recording (announcer)\\
  Look at this. {[}INAUDIBLE{]} Now, wait a minute.
\end{itemize}

mike schmidt

We just got a computer in the house. I'm printing out pictures of Yankee
players, and pasting them onto cardboard and putting them up in my
bedroom.

\begin{itemize}
\tightlist
\item
  archived recording\\
  And here is perhaps the most popular Padre of all time, Tony Gwynn,
  stepping in.
\end{itemize}

mike schmidt

It's also a magical season.

\begin{itemize}
\tightlist
\item
  archived recording\\
  --- is the right-hander. A ground ball, through the middle into center
  field. That's a base hit.
\end{itemize}

mike schmidt

It looks like the all-star Tony Gwynn is going to hit .400.

\begin{itemize}
\tightlist
\item
  archived recording\\
  And it's a home run! Three home runs in a row!
\end{itemize}

mike schmidt

Looks like the home run record may be broken.

\begin{itemize}
\tightlist
\item
  archived recording\\
  Uh oh, America likes that. It's gone. And the Yanks ride the home run
  out of the park with a 5-3 win.
\end{itemize}

mike schmidt

The Yankees are back. They're great again. I'm a Yankee fan.

\begin{itemize}
\item
  archived recording\\
  And the Expos have won 13 of their last 14. And now they're on a pace
  to win a 110 games. And in this, when I'm literally sitting at the
  edge of my seat as a fan, more engrossed in the game than I've ever
  been before ---

  The Expos leave the field in first place, yet wondering if their best
  season ever is in jeopardy. Even so, they're prepared to sit it out
  for as long as it takes.
\end{itemize}

mike schmidt

--- the season is stopped.

\begin{itemize}
\tightlist
\item
  archived recording (expos player)\\
  We didn't want to strike. We didn't want this to happen. But we have
  --- we didn't have no other choice but to go out and take care of
  ourselves and the game of baseball.
\end{itemize}

mike schmidt

In the middle of the summer, the players go on strike in a dispute with
the owners about money. And then ---

\begin{itemize}
\tightlist
\item
  archived recording (bud selig)\\
  But I'll say what I've said to many of you, either independently or
  collectively.
\end{itemize}

mike schmidt

The baseball commissioner at the time comes out ---

\begin{itemize}
\tightlist
\item
  archived recording (bud selig)\\
  Like a lot of things in life, you anticipate something and fear that
  it's coming, hope that it isn't.
\end{itemize}

mike schmidt

--- and announces that the World Series will be canceled.

\begin{itemize}
\tightlist
\item
  archived recording (bud selig)\\
  And when the day is here, there is an incredible amount of sadness.
\end{itemize}

mike schmidt

There will be no more baseball in 1994.

michael barbaro

And how is young, baseball-crazed Mike feeling as he hears that the
World Series has been canceled, the season is officially over?

mike schmidt

I was crushed. I was crushed. And there was nothing really to compare it
to. What I saw in that moment as a kid is something that I understand
better now after covering it, which is that there's two sides to
baseball. There's the romantic side. But there's the other side of it,
which is that it is a business. And baseball runs into problems when
business rears its head. And it rips the romanticism right out of it.
And what's interesting is that the commissioner at the time, in 1994,
the guy who actually had to cancel the season, was Bud Selig.

michael barbaro

What do you mean?

mike schmidt

Different from Manfred, he's a baseball romantic. If you get on the
phone with Selig, you always have to listen to him regale and tell
stories about baseball history. Just a deep-seated love of the game. And
he was the person that had to cancel the 1994 season. He was the one
that had to put his name on the statement that came out and said, there
will be no World Series.

{[}music{]}

{[}phone ringing{]}

mike schmidt

So as this season --- the 2020 season --- looked in doubt, I thought,
the person to call is Bud Selig.

\begin{itemize}
\item
  bud selig\\
  Hello.
\item
  mike schmidt\\
  Commissioner.
\item
  bud selig\\
  Can you hear me all right?
\item
  mike schmidt\\
  I can hear you very good.
\item
  bud selig\\
  Good.
\end{itemize}

mike schmidt

Because he understands more than anyone else the situation that Manfred
finds himself in. So we're looking at the whole question of a baseball
season.

\begin{itemize}
\item
  bud selig\\
  Right.
\item
  mike schmidt\\
  Take us back to 1994. Tell us that story. And tell us why a baseball
  season is so important.
\item
  bud selig\\
  Well let me, as I always do, Mike, give you a little history.
\end{itemize}

mike schmidt

So like I said, Selig starts with some baseball history. He goes back to
World War II. Hundreds of players were sent off to fight. But even as
they were at war ---

\begin{itemize}
\tightlist
\item
  bud selig\\
  FDR had written a letter in December of `42, 1942, urging baseball to
  continue.
\end{itemize}

mike schmidt

The season was not in question.

\begin{itemize}
\item
  bud selig\\
  And so, through this unbelievable Second World War, they did play
  baseball. So we go to 1994. I guess it was about the 16th of
  September. If my memory serves me well, and I think it does, I was in
  County Stadium, and we were going to have to announce that there would
  be no season. And that night I came home, and I sat upstairs in the
  den, and I replayed every World Series from 1945.
\item
  archived recording (don dunphy)\\
  Good afternoon, everyone. This is Don Dunphy speaking for Bill Coram
  and Bill Slater.
\item
  bud selig\\
  My first recollection was the Cardinal St. Louis Browns.
\item
  archived recording\\
  Here it comes. He swings on it, hits it over Marion's head in left
  field for a hit.
\item
  bud selig\\
  So I just sat there very quietly, in deep thought.
\item
  archived recording\\
  Man on first, one away. Third baseman Mark Christman.
\item
  bud selig\\
  Replayed all that. And I was heartsick. It's one of the low, low
  moments of my career and in my life. Because a World War couldn't
  eliminate the World Series.

  It was really sad.
\item
  mike schmidt\\
  When you're sitting there that night in the den, what else is going
  through your head?
\item
  bud selig\\
  Well, one thing I don't think any of us ever really understood was how
  much it was hurting the sport. Fans were angry. And we dragged them
  through the mud. I mean, it's now 26 years ago, and the pain, as I sit
  here and talk to you, comes back to me. And I was worried. It scared
  me.
\end{itemize}

{[}music{]}

\begin{itemize}
\item
  archived recording 1\\
  How do you feel about a baseball strike?
\item
  archived recording 2\\
  I think it's stupid. I think the players are just being selfish.
\item
  archived recording 3\\
  So do I.
\item
  archived recording 4\\
  You think the players are being selfish?
\item
  archived recording 5\\
  Yeah. They make enough money anyways.
\item
  archived recording 6\\
  It would be pretty boring without baseball in the summer. It just
  ruins the game the way they're striking.
\item
  archived recording 7\\
  I don't know. I think everyone's money hungry.
\item
  archived recording 8\\
  I wish they'd sign an agreement and keep on playing.
\item
  archived recording 9\\
  Should they play for free?
\item
  archived recording 10\\
  Yeah.
\item
  archived recording 11\\
  Why?
\item
  archived recording 12\\
  Because it's just a game. It doesn't matter how much money you get.
\item
  archived recording 13\\
  And the owners, they say they don't make enough money. Well, the
  question is, what is enough money?
\item
  archived recording 14\\
  So what are you going to do?
\item
  archived recording 15\\
  I don't know. I guess just watch minor league baseball.
\item
  archived recording 16\\
  I already know what I'm going to start doing. I'm going to start
  rereading Dante's ``Inferno,'' because that's where I think they
  should send the whole lot of them.
\end{itemize}

michael barbaro

So as you probably suspected, it wasn't just 11-year-old Mike who was
mad at baseball. It was ---

mike schmidt

No. And as bad as `94 was, `95, `96 and `97 were just as bad because
fans were so upset by the strike that the sport had to rebuild itself
and rebuild its credibility with the fans.

michael barbaro

Mike, how much do you think 1994 is on the mind of Manfred?

mike schmidt

Enormously, huge, front of mind. Because that labor lawyer who I met
back when I was covering drugs in baseball, he was just starting out in
baseball in 1994. And he was a junior lawyer who was deeply involved in
the strike and trying to help the owners win the labor fight.

\begin{itemize}
\tightlist
\item
  bud selig\\
  And that's when I met him. And the more I saw Rob, the more I liked
  him. And he and I worked well together. Somebody asked me the other
  day, how often did you talk to Rob back then? Maybe 10 times a day,
  even. More than he wanted, I may add. {[}CHUCKLES{]}
\end{itemize}

mike schmidt

So after Manfred became commissioner, Selig tried to give him some space
and some distance. He didn't want to look like the father telling the
son how to run the sport. And a bit of distance grew between the two of
them. But ---

\begin{itemize}
\tightlist
\item
  bud selig\\
  One thing we'll say to each other, Mike ---
\end{itemize}

mike schmidt

--- as Manfred found himself in this situation, he began to lean back on
Selig.

\begin{itemize}
\tightlist
\item
  bud selig\\
  I'm the only other guy on the face of the earth that understands
  exactly what the pressure is and what the situation is.
\end{itemize}

mike schmidt

And they talked more than they had at any other point since Manfred had
become commissioner.

\begin{itemize}
\item
  bud selig\\
  Since I've done that job for 22 and a half years. And I'm the only one
  who understands what he's gone through. So yeah, there's no question
  about it.
\item
  mike schmidt\\
  I really appreciate it.
\item
  bud selig\\
  Well, great. Well, I hope you enjoyed it. It was a pleasure to do it.
  And we'll talk soon.
\item
  mike schmidt\\
  Thanks.
\item
  bud selig\\
  Bye.
\end{itemize}

michael barbaro

So what happens next in the story, Mike?

mike schmidt

So at this point, it's been more than a month since my first
conversation with Rob. And things are getting really nasty between the
owners and the players.

{[}music{]}

\begin{itemize}
\item
  archived recording 1\\
  --- the middle of June. And boy, what Astros fans would give to be
  sitting in the stands at Minute Maid Park right about now.
\item
  archived recording 2\\
  It's true. And we were hopeful.
\end{itemize}

{[}music{]}

\begin{itemize}
\item
  archived recording 3\\
  Baseball, negotiations grinding to a halt.
\item
  archived recording 4\\
  Like, we're almost to July now, and there's still nothing. I mean,
  we're still in the same position that we were in at March.
\item
  archived recording 5\\
  And if they don't do something about it, the sport is going to fade
  even more.
\item
  archived recording 6\\
  How are you a commissioner --- like, baseball should have been back a
  month ago. They should be basically saying, here's our opportunity to
  recapture an audience.
\item
  archived recording 7\\
  It's just it's unfortunate that it's been so public. I think fans have
  been turned away a little bit.
\item
  archived recording 8\\
  There's a reason Major League Baseball's executive office is filled
  with labor lawyers. Because there's a labor fight every 12 years in
  this league.
\end{itemize}

mike schmidt

So Manfred becomes so frustrated that he decides to go out on television
and say ---

\begin{itemize}
\tightlist
\item
  archived recording (rob manfred)\\
  Well, I know the owners are 100 percent committed to getting baseball
  back on the field. Unfortunately, I can't tell you that I'm 100
  percent certain that it's going to happen.
\end{itemize}

mike schmidt

You know, I said there was going to be a season. But actually, now, I'm
not sure.

michael barbaro

Hmm.

mike schmidt

And he's trying any attempt to restart the negotiations, get the players
back to the table and move forward. That ultimately doesn't really work.
And he has to go out on his own and announces that a 60-game season will
start on July 23.

michael barbaro

So Mike, he can do that, just call a 60-game season without a deal
between the players and the owners?

mike schmidt

It wasn't his first choice. He wanted both sides to buy in. He wanted a
better deal on how much the players would take a reduction in salary.
But without any other choice and his deep desire to have a season at
pretty much any cost, that was his only option.

michael barbaro

So is this seen as a win for the players or for the owners?

mike schmidt

Both and neither. The players are going to get paid their full salary
for the games that they play. Manfred is going to get his season. But
neither of them are walking away feeling good about their relationship.
And in the two months that they've taken to resolve this labor issue,
they're now back to a health and safety problem. Because in that time,
Covid has exploded and spread to new states. And actually, today, as I
was preparing to talk to Rob for the final time, news broke that the
all-star player on the Nationals, Juan Soto, had tested positive for
Covid. So here's Manfred, on the cusp of having the season he fought so
hard for, learning just hours before the first pitch, that one of the
star players may have the virus.

michael barbaro

Not ideal.

mike schmidt

Another ``oh shit'' moment in a long season.

{[}music{]}

\begin{itemize}
\item
  mike schmidt\\
  Commissioner.
\item
  rob manfred\\
  Hey, Schmidt. How are you?
\item
  mike schmidt\\
  Hell of a day.
\item
  rob manfred\\
  Well, we're going to make it to the starting line. {[}CHUCKLES{]}
  Everybody seems excited, like we've done something. All we did was get
  out of the gate, you know what I mean? {[}CHUCKLES{]} The hard part is
  playing 60 games, you know? Anyways, I'm glad we are where we are. You
  know, I feel pretty good about it.
\item
  mike schmidt\\
  Where are you right now?
\item
  rob manfred\\
  I'm in Washington. I'm at Nationals Park.
\item
  mike schmidt\\
  What's the feeling in the air? You're at opening day with no fans and
  just the members of the staff. What does that feel like?
\item
  rob manfred\\
  Like no opening day I've ever seen, I'll tell you that. {[}CHUCKLES{]}
  It's really different, Mike. I mean, it's very stark right now. It's
  early still, but it's very stark right now.
\item
  mike schmidt\\
  When you heard today that Juan Soto contracted Covid --- we're talking
  about the all-star on the world championship team --- are you like, oh
  no, we can't do this? What's your mindset?
\item
  rob manfred\\
  I mean, look, my initial reaction is I can't believe this is happening
  on opening day. But then I dropped back and I thought about, we knew
  we were going to have positives. It's unfortunate that it was opening
  day and that it was Juan Soto. But the protocols were built to deal
  with this. The whole point is you've got to build a system that's
  flexible enough to deal with what's coming. We knew it was coming.
\item
  mike schmidt\\
  I'm mindful that today is July 23. The first time we spoke was May 20.
  It's been two months. What we've got now is pretty much the plan that
  you had back then for the virus. But in this period of time, you went
  through this whole tumultuous thing. If you could go back, would you
  have done anything differently? And is there any mistakes you made in
  the process?
\item
  rob manfred\\
  Well if I could go back, I'd love an opportunity to replay that hand.
  I really would, Michael. I think that one thing I can certainly point
  to, the whole, from the very beginning, the back-and-forth in the
  press and all that. I just --- I tried to avoid it. I didn't manage to
  do it. I'd love to have had a chance to go back and do it over again
  and be better at it.
\item
  mike schmidt\\
  Do you think there's long-term damage from it?
\item
  rob manfred\\
  You know, I think that --- I do think it was unsightly and we should
  not have allowed it to happen. I think we sort of have a debt to our
  fans.
\item
  mike schmidt\\
  Let me ask it this way. The two months of nasty, public,
  back-and-forth negotiations between the owners and the players, do you
  think that will have long-term damage to the sport going forward,
  similar to `94?
\item
  rob manfred\\
  I just --- I don't know what to say to that one, Michael. I just don't
  know.
\item
  mike schmidt\\
  So tonight, on the field, will be exactly what you mapped out. The
  players will be distanced, no high-fives, no spitting. What is your
  hope for how this game feels to the fans watching at home.
\item
  rob manfred\\
  Honestly, I hope that at the end of the night, what fans are thinking
  is, you know what, it's not everything that we're used to and love
  about the game, but you know what, it's great to have baseball.
\item
  archived recording\\
  In an empty stadium, and no acknowledgment from the fans, and no
  acknowledgment to the fans from those there.
\item
  mike schmidt\\
  We're done. That's it.
\item
  rob manfred\\
  Thanks, Schmidt. I'll talk to you soon, huh?
\item
  mike schmidt\\
  Bye.
\item
  archived recording 1\\
  --- we get a real good feel for tonight. And now one of the more
  well-known Washington National fans, Dr. Anthony Fauci, to throw out
  the first pitch.
\end{itemize}

{[}music{]}

\begin{itemize}
\tightlist
\item
  archived recording 2\\
  Dr. Anthony Fauci!
\end{itemize}

michael barbaro

So Mike, I am talking to you at 9:30 p.m. on Thursday evening. The game
is currently underway. And the first pitch was thrown out by Dr. Anthony
Fauci, and it was pretty wobbly. I assume you've been watching the game.

mike schmidt

Yes. And we're now in the middle of a rain delay.

michael barbaro

Right. Which is perhaps why you're talking to me. I know you didn't set
out to be philosophical about baseball with this assignment, but if we
could get philosophical for just a moment I wonder how you're thinking
about baseball and this game right now.

mike schmidt

Look, I told you about my first realization when I was 11 years old,
that baseball walks a fine line between being a game and a business.

When I became a sports reporter, I covered the darkest underbelly of the
sport, and saw it in probably a nastier light than most fans could ever
dream of. And in the years after that, I had a hard time falling back
into the romantic fandom of baseball. I had just seen too much. But last
year, almost a decade after I left sports, I got my fandom back. I
caught the bug again. And as 2020 started, I was ready to continue that
and to try and be that fan again. These negotiations brought back those
feelings of the two-headed monster of baseball, and the business head
becoming too big. But at the end of the day, there's going to be a
season. And it's going to look weird and feel very, very different. But
it's a season. And it's baseball.

michael barbaro

In other words, you're still a little bit of a romantic.

mike schmidt

Look, I'll take it. I'll take it for now.

{[}music{]}

\begin{itemize}
\tightlist
\item
  archived recording\\
  --- games last season, a couple of very difficult I.L. stints really
  prevented him from accomplishing much of anything.
\end{itemize}

michael barbaro

Mike, thank you very much. Enjoy the game. Enjoy the season.

mike schmidt

Thanks for having me.

\begin{itemize}
\item
  archived recording 1\\
  Swung on and hit high in the air to left center. That ball is high, it
  is far, it is gone. Way back in the left-center field seats.
\item
  archived recording 2\\
  Had there been fans in the ballpark, it was a guy that bought the
  worst seat that would have gotten that souvenir.
\item
  archived recording 3\\
  Oh, what a shot by Stanton. It's a two-run dinger. And the Yankees
  immediately take a 2-0 lead.
\item
  archived recording 4\\
  He turned around a 96-mile-an-hour fastball 459 feet. What's the
  saying we always have? The harder it comes in, the harder it goes out.
\item
  archived recording 5\\
  And that's the MVP swing that the New York Yankees acquired from the
  Marlins.
\item
  archived recording 6\\
  {[}CHUCKLES{]} As the old line goes, the ball went so far they should
  serve a meal on it. So the Yanks take a 2-0 lead. That was Stanton's
  21st career ---
\end{itemize}

michael barbaro

We'll be right back.

{[}music{]}

Here's what else you need to know today. On Thursday, the United States
reached a new milestone in the pandemic, with 4 million known
infections. Infections are now on the rise in 39 different states;
Washington, D.C.; Puerto Rico; and the U.S. Virgin Islands.

\begin{itemize}
\tightlist
\item
  archived recording (donald trump)\\
  Everything was going well, a tremendous list of speakers, thousands of
  people wanting to be there, and I mean, in some cases, desperately be
  there. They wanted to attend.
\end{itemize}

michael barbaro

At the White House, President Trump said he would cancel the public
portion of the Republican National Convention scheduled for August. To
avoid strict social distancing rules the president had moved the events
from North Carolina to Florida, which now has the highest infection rate
in the country.

Previous

More episodes ofThe Daily

\href{https://www.nytimes3xbfgragh.onion/2020/07/31/podcasts/the-daily/vanessa-guillen-military-metoo.html?action=click\&module=audio-series-bar\&region=header\&pgtype=Article}{\includegraphics{https://static01.graylady3jvrrxbe.onion/images/2020/07/12/us/politics/31daily/00dc-army-metoo-thumbLarge.jpg}}

July 31, 2020A \#MeToo Moment in the Military

\href{https://www.nytimes3xbfgragh.onion/2020/07/30/podcasts/the-daily/congress-facebook-amazon-google-apple.html?action=click\&module=audio-series-bar\&region=header\&pgtype=Article}{\includegraphics{https://static01.graylady3jvrrxbe.onion/images/2020/07/30/reader-center/30daily/merlin_175077825_5ebc931b-baa1-489a-960c-34e4d845e997-thumbLarge.jpg}}

July 30, 2020The Big Tech Hearing

\href{https://www.nytimes3xbfgragh.onion/2020/07/29/podcasts/the-daily/china-trump-foreign-policy.html?action=click\&module=audio-series-bar\&region=header\&pgtype=Article}{\includegraphics{https://static01.graylady3jvrrxbe.onion/images/2020/07/26/world/29daily/00china-us-clash1-thumbLarge.jpg}}

July 29, 2020~~•~ 28:40Confronting China

\href{https://www.nytimes3xbfgragh.onion/2020/07/28/podcasts/the-daily/unemployment-benefits-coronavirus.html?action=click\&module=audio-series-bar\&region=header\&pgtype=Article}{\includegraphics{https://static01.graylady3jvrrxbe.onion/images/2020/07/23/business/28daily/23virus-uiexplain1-thumbLarge.jpg}}

July 28, 2020~~•~ 26:13Why \$600 Checks Are Tearing Republicans Apart

\href{https://www.nytimes3xbfgragh.onion/2020/07/27/podcasts/the-daily/new-york-hospitals-covid.html?action=click\&module=audio-series-bar\&region=header\&pgtype=Article}{\includegraphics{https://static01.graylady3jvrrxbe.onion/images/2020/07/27/world/27daily-hospitals/27daily-hospitals-thumbLarge.jpg}}

July 27, 2020~~•~ 33:28The Mistakes New York Made

\href{https://www.nytimes3xbfgragh.onion/2020/07/26/podcasts/the-daily/the-accusation-the-sunday-read.html?action=click\&module=audio-series-bar\&region=header\&pgtype=Article}{\includegraphics{https://static01.graylady3jvrrxbe.onion/images/2020/03/22/magazine/26audm-2/22mag-titleix-thumbLarge.jpg}}

July 26, 2020The Sunday Read: `The Accusation'

\href{https://www.nytimes3xbfgragh.onion/2020/07/24/podcasts/the-daily/mlb-baseball-season-coronavirus.html?action=click\&module=audio-series-bar\&region=header\&pgtype=Article}{\includegraphics{https://static01.graylady3jvrrxbe.onion/images/2020/07/22/sports/24daily/22mlb-previewlede1-thumbLarge.jpg}}

July 24, 2020~~•~ 45:34The Battle for a Baseball Season

\href{https://www.nytimes3xbfgragh.onion/2020/07/23/podcasts/the-daily/portland-protests.html?action=click\&module=audio-series-bar\&region=header\&pgtype=Article}{\includegraphics{https://static01.graylady3jvrrxbe.onion/images/2020/07/22/us/23daily-image/22portland-tactics02-thumbLarge.jpg}}

July 23, 2020~~•~ 30:04The Showdown in Portland

\href{https://www.nytimes3xbfgragh.onion/2020/07/22/podcasts/the-daily/school-reopenings-coronavirus.html?action=click\&module=audio-series-bar\&region=header\&pgtype=Article}{\includegraphics{https://static01.graylady3jvrrxbe.onion/images/2020/07/12/science/22daily/00virus-schools-reopen01-thumbLarge.jpg}}

July 22, 2020~~•~ 27:24The Science of School Reopenings

\href{https://www.nytimes3xbfgragh.onion/2020/07/21/podcasts/the-daily/coronavirus-vaccine.html?action=click\&module=audio-series-bar\&region=header\&pgtype=Article}{\includegraphics{https://static01.graylady3jvrrxbe.onion/images/2020/07/19/science/21daily/00VIRUS-VAX-DOUBTS1-thumbLarge.jpg}}

July 21, 2020~~•~ 29:14The Vaccine Trust Problem

\href{https://www.nytimes3xbfgragh.onion/2020/07/20/podcasts/the-daily/john-lewis.html?action=click\&module=audio-series-bar\&region=header\&pgtype=Article}{\includegraphics{https://static01.graylady3jvrrxbe.onion/images/2020/01/07/obituaries/20thedaily_lewis/00Lewis-John13-thumbLarge.jpg}}

July 20, 2020~~•~ 38:56The Life and Legacy of John Lewis

\href{https://www.nytimes3xbfgragh.onion/2020/07/19/podcasts/the-daily/lottery-winner-scam.html?action=click\&module=audio-series-bar\&region=header\&pgtype=Article}{\includegraphics{https://static01.graylady3jvrrxbe.onion/images/2018/05/05/magazine/31audm-image/05mag-lottery-image1-thumbLarge-v4.png}}

July 19, 2020~~•~ 45:27The Sunday Read: `The Man Who Cracked the
Lottery'

\href{https://www.nytimes3xbfgragh.onion/column/the-daily}{See All
Episodes ofThe Daily}

Next

July 24, 2020

\begin{itemize}
\item
\item
\item
\item
\item
\item
\end{itemize}

\emph{\textbf{Listen and subscribe to our podcast from your mobile
device:}}\\
\textbf{\href{https://itunes.apple.com/us/podcast/the-daily/id1200361736?mt=2}{\emph{Via
Apple Podcasts}}} \emph{\textbf{\textbar{}}}
\textbf{\href{https://open.spotify.com/show/3IM0lmZxpFAY7CwMuv9H4g?si=SfuMSC55R1qprFsRZU3_zw}{\emph{Via
Spotify}}} \emph{\textbf{\textbar{}}}
\textbf{\href{http://www.stitcher.com/podcast/the-new-york-times/the-daily-10}{\emph{Via
Stitcher}}}

\emph{This episode contains strong language.}

Today, we go inside the fraught weeks that led up to the opening game of
the 2020 professional baseball season --- from the perspective of the
commissioner of Major League Baseball.

\textbf{On today's episode:}

\begin{itemize}
\tightlist
\item
  \href{https://www.nytimes3xbfgragh.onion/by/michael-s-schmidt?smid=pc-thedaily}{Michael
  S. Schmidt}, who covers national security for The New York Times,
  spoke with Rob Manfred, the commissioner of Major League Baseball.
\end{itemize}

\includegraphics{https://static01.graylady3jvrrxbe.onion/images/2020/07/22/sports/24daily/merlin_174785379_4055717e-72ea-458c-bb74-6f70c40f7b11-articleLarge.jpg?quality=75\&auto=webp\&disable=upscale}

\textbf{Background reading:}

\begin{itemize}
\tightlist
\item
  The schedule is short. The stadiums will be empty. This is what our
  baseball writer
  \href{https://www.nytimes3xbfgragh.onion/2020/07/23/sports/baseball/mlb-season-coronavirus.html}{thinks
  the season might look like} this year.
\end{itemize}

\emph{Tune in, and tell us what you think. Email us at}
\href{mailto:thedaily@NYTimes.com}{\emph{thedaily@NYTimes.com}}\emph{.
Follow Michael Barbaro on Twitter:}
\href{https://twitter.com/mikiebarb}{\emph{@mikiebarb}}\emph{. And if
you're interested in advertising with ``The Daily,'' write to us at}
\href{mailto:thedaily-ads@NYTimes.com}{\emph{thedaily-ads@NYTimes.com}}\emph{.}

``The Daily'' is made by Theo Balcomb, Andy Mills, Lisa Tobin, Rachel
Quester, Lynsea Garrison, Annie Brown, Clare Toeniskoetter, Paige
Cowett, Michael Simon Johnson, Brad Fisher, Larissa Anderson, Wendy
Dorr, Chris Wood, Jessica Cheung, Stella Tan, Alexandra Leigh Young,
Jonathan Wolfe, Lisa Chow, Eric Krupke, Marc Georges, Luke Vander Ploeg,
Adizah Eghan, Kelly Prime, Julia Longoria, Sindhu Gnanasambandan, M.J.
Davis Lin, Austin Mitchell, Sayre Quevedo, Neena Pathak, Dan Powell,
Dave Shaw, Sydney Harper, Daniel Guillemette, Hans Buetow, Robert
Jimison, Mike Benoist, Bianca Giaever and Asthaa Chaturvedi. Our theme
music is by Jim Brunberg and Ben Landsverk of Wonderly. Special thanks
to Sam Dolnick, Mikayla Bouchard, Lauren Jackson, Julia Simon, Mahima
Chablani and Nora Keller.

Advertisement

\protect\hyperlink{after-bottom}{Continue reading the main story}

\hypertarget{site-index}{%
\subsection{Site Index}\label{site-index}}

\hypertarget{site-information-navigation}{%
\subsection{Site Information
Navigation}\label{site-information-navigation}}

\begin{itemize}
\tightlist
\item
  \href{https://help.nytimes3xbfgragh.onion/hc/en-us/articles/115014792127-Copyright-notice}{©~2020~The
  New York Times Company}
\end{itemize}

\begin{itemize}
\tightlist
\item
  \href{https://www.nytco.com/}{NYTCo}
\item
  \href{https://help.nytimes3xbfgragh.onion/hc/en-us/articles/115015385887-Contact-Us}{Contact
  Us}
\item
  \href{https://www.nytco.com/careers/}{Work with us}
\item
  \href{https://nytmediakit.com/}{Advertise}
\item
  \href{http://www.tbrandstudio.com/}{T Brand Studio}
\item
  \href{https://www.nytimes3xbfgragh.onion/privacy/cookie-policy\#how-do-i-manage-trackers}{Your
  Ad Choices}
\item
  \href{https://www.nytimes3xbfgragh.onion/privacy}{Privacy}
\item
  \href{https://help.nytimes3xbfgragh.onion/hc/en-us/articles/115014893428-Terms-of-service}{Terms
  of Service}
\item
  \href{https://help.nytimes3xbfgragh.onion/hc/en-us/articles/115014893968-Terms-of-sale}{Terms
  of Sale}
\item
  \href{https://spiderbites.nytimes3xbfgragh.onion}{Site Map}
\item
  \href{https://help.nytimes3xbfgragh.onion/hc/en-us}{Help}
\item
  \href{https://www.nytimes3xbfgragh.onion/subscription?campaignId=37WXW}{Subscriptions}
\end{itemize}
