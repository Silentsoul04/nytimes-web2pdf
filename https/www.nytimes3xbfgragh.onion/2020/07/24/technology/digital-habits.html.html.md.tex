Sections

SEARCH

\protect\hyperlink{site-content}{Skip to
content}\protect\hyperlink{site-index}{Skip to site index}

\href{https://www.nytimes3xbfgragh.onion/section/technology}{Technology}

\href{https://myaccount.nytimes3xbfgragh.onion/auth/login?response_type=cookie\&client_id=vi}{}

\href{https://www.nytimes3xbfgragh.onion/section/todayspaper}{Today's
Paper}

\href{/section/technology}{Technology}\textbar{}Digital Habits Are Hard
to Break

\href{https://nyti.ms/30HEuFH}{https://nyti.ms/30HEuFH}

\begin{itemize}
\item
\item
\item
\item
\item
\end{itemize}

Advertisement

\protect\hyperlink{after-top}{Continue reading the main story}

Supported by

\protect\hyperlink{after-sponsor}{Continue reading the main story}

on tech

\hypertarget{digital-habits-are-hard-to-break}{%
\section{Digital Habits Are Hard to
Break}\label{digital-habits-are-hard-to-break}}

Technology was supposed to be all about welcoming newcomers. But is it?

\includegraphics{https://static01.graylady3jvrrxbe.onion/images/2020/07/24/business/24ontech/24ontech-articleLarge.jpg?quality=75\&auto=webp\&disable=upscale}

\href{https://www.nytimes3xbfgragh.onion/by/shira-ovide}{\includegraphics{https://static01.graylady3jvrrxbe.onion/images/2020/03/18/reader-center/author-shira-ovide/author-shira-ovide-thumbLarge-v2.png}}

By \href{https://www.nytimes3xbfgragh.onion/by/shira-ovide}{Shira Ovide}

\begin{itemize}
\item
  July 24, 2020
\item
  \begin{itemize}
  \item
  \item
  \item
  \item
  \item
  \end{itemize}
\end{itemize}

\emph{This article is part of the On Tech newsletter. You can}
\href{https://www.nytimes3xbfgragh.onion/newsletters/signup/OT}{\emph{sign
up here}} \emph{to receive it weekdays.}

Digital success can be as flimsy as tissue paper.
\href{https://www.forbes.com/forbes/2010/0830/entrepreneurs-groupon-facebook-twitter-next-web-phenom.html\#568f5cee4c2e}{Remember
Groupon}? BlackBerry went from the king of our pockets to nada in a hot
minute. Heck, it seemed like
\href{https://www.nbcnews.com/pop-culture/pop-culture-news/internet-went-crazy-over-cake-going-mainstream-can-end-trend-n1234588}{we
got bored of those internet cake videos} in a week.

Even in technology, though, some habits can prove tenacious.

No one has been able to get large numbers of Americans to use something
other than Google for all our burning questions. The world has settled
into only two flavors of smartphones: iPhones and Androids. And in the
United States, it's tough to crack Amazon's lock on online shopping.

It's not necessarily because these products or services are better than
the alternatives. They might be, but there are also strategic arts that
explain why some companies endure. And there's the power of inertia.
Sometimes we do what we do because that's what we do.

There's nothing necessarily wrong with these habits. But we have long
thought of technology as more dynamic and open to newcomers. And yet, is
it?

Let's focus on online shopping. In the United States, Amazon has at
least seven times the online business of Walmart, Target, eBay or anyone
else.

Amazon is really good at what it does. It sells just about every product
imaginable --- for good or
\href{https://www.nytimes3xbfgragh.onion/2020/07/09/technology/facebook-crisis.html}{for
ill}, buying is easy and stuff typically arrives reliably and fast.
Prices often aren't the cheapest, and Amazon's website feels like it was
made by 1990s robots rather than by humans with souls \ldots{} but no
matter.

And also there's the power of habit that Amazon cleverly reinforces.
We're on Amazon because we're used to it, and
\href{https://www.nytimes3xbfgragh.onion/2020/04/27/technology/no-the-best-doesnt-win.html}{it
just works}. Merchants focus their attention on Amazon because we're all
shopping there. And the Prime shopping club is essentially an incentive
to never shop anywhere else.

My colleague Dai Wakabayashi chronicled this week Google's
\href{https://www.nytimes3xbfgragh.onion/2020/07/23/technology/google-ecommerce-amazon.html}{repeated,
mostly failed efforts} to make it as easy as possible for merchants to
sell us stuff through Google instead. Dai told me that Google is now
letting merchants list many products without paying sales commissions,
and it's making it easy for them to port over information directly from
their Amazon product listings. Google is trying so hard!

Google can be a scatterbrained mess, but it's also rich and attracts
billions of eyeballs every day. If it can't persuade Americans to shop
somewhere other than Amazon, that shows us something both about Amazon's
strengths and about how tough it can be to persuade us to try something
different. (Worth noting: Amazon rules online but a vast majority of our
consumer spending happens in stores.)

Everyone in Silicon Valley knows the history of technology winners
becoming losers in a flash, so many successful tech companies live in
fear of losing it all.

One question for those of us who use technology, and for governments
concerned about keeping competition healthy, is whether there's
something different that makes today's tech powers more immovable than
yesterday's. This is at the heart of the upcoming congressional
\href{https://www.nytimes3xbfgragh.onion/2020/07/01/technology/amazon-apple-alphabet-facebook-congress-antitrust.html}{antitrust
hearings} involving four of America's digital superpowers.

The bottom line is internet users like us benefit if lots of companies
are afraid for their future and fighting hard for our attention and
dollars. But in some corners of technology, that's not really happening.

\begin{center}\rule{0.5\linewidth}{\linethickness}\end{center}

\hypertarget{every-fight-is-about-data}{%
\subsection{Every fight is about data}\label{every-fight-is-about-data}}

I'm constantly struck that lots of problems about our digital lives boil
down to data: who has it, who doesn't and how it's interpreted and kept
secret.

Let me give you one example:
\href{https://www.wsj.com/articles/facebook-creates-teams-to-study-racial-bias-on-its-platforms-11595362939}{The
Wall Street Journal} and
\href{https://www.nbcnews.com/tech/tech-news/facebook-management-ignored-internal-research-showing-racial-bias-current-former-n1234746}{NBC
News} had details this week about Facebook previously shelving internal
studies of possible racial bias on its site --- including research that
dug into why Black people appeared far more likely to have their
accounts disabled for perceived violations of hate speech rules.

Facebook said in part that it worried these research projects relied on
faulty data. Facebook doesn't know if you're Black, but it makes
inferences about race from the information you engage with. Those
inferences can be wrong, and Facebook said it didn't want to rely on bad
data.

Mind you, Facebook uses this same data to target advertising for
companies who want to sell to Black people. The data was good enough for
Facebook's paying customers. (And, a former Facebook researcher
\href{https://twitter.com/dantley/status/1286519635748114432}{tweeted}
that those probing possible bias didn't rely only on Facebook's
inferences on race.)

The reason we know about this fight inside Facebook is that the
company's employees see data that we never will, and some of them were
uncomfortable with how their bosses used or suppressed the information.
There are similar tales
\href{https://www.theguardian.com/technology/2018/feb/02/youtube-algorithm-election-clinton-trump-guillaume-chaslot}{at
YouTube} and at just about every internet superpower.

There are two crucial lessons here: First, we often think data is
somehow pure and untainted by human bias, but that's wrong. Information
is gathered and interpreted by humans --- or by computers programmed by
humans --- and is therefore subject to our whims and bias.

And second, we are hopelessly incapable of understanding the inner
workings of the world's biggest information machines because they see
every morsel of information happening inside their walls and we see only
what they choose to tell us. Data is power, and we have little of both.

\begin{center}\rule{0.5\linewidth}{\linethickness}\end{center}

\hypertarget{before-we-go-}{%
\subsection{Before we go \ldots{}}\label{before-we-go-}}

\begin{itemize}
\item
  \textbf{Trying to influence the influential:} George Mason
  University's Global Antitrust Institute has pushed a message of
  restraint in antitrust enforcement to hundreds of overseas regulators
  and judges at lavish all-expense paid conferences in Hawaii, Tokyo and
  Portugal. My colleague Dai found that Google, Amazon, Qualcomm and
  other big tech companies helped pay for these events, which critics
  said
  \href{https://www.nytimes3xbfgragh.onion/2020/07/24/technology/global-antitrust-institute-google-amazon-qualcomm.html}{presented
  a one-sided view of corporate regulation} intended to benefit the big
  companies.
\item
  \textbf{Everything is data, part deux:} Another way tech companies
  consolidate and keep power is by harnessing data to learn about
  competitors and countermove against them. The Wall Street Journal
  wrote about Amazon appearing to use its interactions with business
  partners or potential ones to
  \href{https://www.wsj.com/articles/amazon-tech-startup-echo-bezos-alexa-investment-fund-11595520249?mod=searchresults\&page=1\&pos=1}{help
  develop competing products}. And the tech news publication the
  Information wrote about Google using data from Android phones to
  \href{https://www.theinformation.com/articles/internal-google-program-taps-data-on-rival-android-apps}{learn
  about how people use rival apps}.
\item
  \textbf{My new favorite couple:} I loved
  \href{https://www.nytimes3xbfgragh.onion/2020/07/24/world/asia/taiwan-octogenarian-couple-instagram-laundry.html}{this
  New York Times article} about the owners of a laundry shop in Taiwan
  who have become Instagram stars for posing in garments that people
  abandoned. ``I can tell they're elated,'' said the unofficial stylist
  and grandson of the couple, who are in their 80s. (Check out
  \href{https://www.instagram.com/wantshowasyoung/}{their account for
  yourself}. These two have got style.)
\end{itemize}

\hypertarget{hugs-to-this}{%
\subsubsection{Hugs to this}\label{hugs-to-this}}

May we all have the calm and grace of
\href{https://twitter.com/k_thompson212/status/1285240819608424448}{this
bear sitting on patio furniture}. (Thanks to my colleague Charlie Warzel
for spotting this gem.)

\begin{center}\rule{0.5\linewidth}{\linethickness}\end{center}

\emph{We want to hear from you. Tell us what you think of this
newsletter and what else you'd like us to explore. You can reach us at}
\href{mailto:ontech@NYTimes.com?subject=On\%20Tech\%20Feedback}{\emph{ontech@NYTimes.com.}}
**

\emph{If you don't already get this newsletter in your inbox,}
\href{https://www.nytimes3xbfgragh.onion/newsletters/signup/OT}{\emph{please
sign up here}}\emph{.}

Advertisement

\protect\hyperlink{after-bottom}{Continue reading the main story}

\hypertarget{site-index}{%
\subsection{Site Index}\label{site-index}}

\hypertarget{site-information-navigation}{%
\subsection{Site Information
Navigation}\label{site-information-navigation}}

\begin{itemize}
\tightlist
\item
  \href{https://help.nytimes3xbfgragh.onion/hc/en-us/articles/115014792127-Copyright-notice}{©~2020~The
  New York Times Company}
\end{itemize}

\begin{itemize}
\tightlist
\item
  \href{https://www.nytco.com/}{NYTCo}
\item
  \href{https://help.nytimes3xbfgragh.onion/hc/en-us/articles/115015385887-Contact-Us}{Contact
  Us}
\item
  \href{https://www.nytco.com/careers/}{Work with us}
\item
  \href{https://nytmediakit.com/}{Advertise}
\item
  \href{http://www.tbrandstudio.com/}{T Brand Studio}
\item
  \href{https://www.nytimes3xbfgragh.onion/privacy/cookie-policy\#how-do-i-manage-trackers}{Your
  Ad Choices}
\item
  \href{https://www.nytimes3xbfgragh.onion/privacy}{Privacy}
\item
  \href{https://help.nytimes3xbfgragh.onion/hc/en-us/articles/115014893428-Terms-of-service}{Terms
  of Service}
\item
  \href{https://help.nytimes3xbfgragh.onion/hc/en-us/articles/115014893968-Terms-of-sale}{Terms
  of Sale}
\item
  \href{https://spiderbites.nytimes3xbfgragh.onion}{Site Map}
\item
  \href{https://help.nytimes3xbfgragh.onion/hc/en-us}{Help}
\item
  \href{https://www.nytimes3xbfgragh.onion/subscription?campaignId=37WXW}{Subscriptions}
\end{itemize}
