Sections

SEARCH

\protect\hyperlink{site-content}{Skip to
content}\protect\hyperlink{site-index}{Skip to site index}

\href{https://www.nytimes3xbfgragh.onion/section/us}{U.S.}

\href{https://myaccount.nytimes3xbfgragh.onion/auth/login?response_type=cookie\&client_id=vi}{}

\href{https://www.nytimes3xbfgragh.onion/section/todayspaper}{Today's
Paper}

\href{/section/us}{U.S.}\textbar{}Nadine Taub, Early Leader in Women's
Rights Law, Dies at 77

\url{https://nyti.ms/30Faj1Q}

\begin{itemize}
\item
\item
\item
\item
\item
\end{itemize}

Advertisement

\protect\hyperlink{after-top}{Continue reading the main story}

Supported by

\protect\hyperlink{after-sponsor}{Continue reading the main story}

\hypertarget{nadine-taub-early-leader-in-womens-rights-law-dies-at-77}{%
\section{Nadine Taub, Early Leader in Women's Rights Law, Dies at
77}\label{nadine-taub-early-leader-in-womens-rights-law-dies-at-77}}

Along with other feminist lawyers like Ruth Bader Ginsburg, Ms. Taub
made legal history in cases that argued that the Constitution protected
women's rights.

\includegraphics{https://static01.graylady3jvrrxbe.onion/images/2020/07/28/obituaries/00Taub1/00Taub1-articleLarge.jpg?quality=75\&auto=webp\&disable=upscale}

\href{https://www.nytimes3xbfgragh.onion/by/penelope-green}{\includegraphics{https://static01.graylady3jvrrxbe.onion/images/2018/07/18/multimedia/author-penelope-green/author-penelope-green-thumbLarge-v3.png}}

By \href{https://www.nytimes3xbfgragh.onion/by/penelope-green}{Penelope
Green}

\begin{itemize}
\item
  July 24, 2020
\item
  \begin{itemize}
  \item
  \item
  \item
  \item
  \item
  \end{itemize}
\end{itemize}

In the early 1970s, Nadine Taub was one of a cadre of young female
lawyers breaking new ground by fighting gender discrimination. Along
with Ruth Bader Ginsburg, Nancy Stearns and others, she made legal
history in cases that successfully argued that equal rights for women
were protected under the Constitution. She litigated cases for rape
victims, for women seeking access to abortion and for employees battling
workplace discrimination and sexual harassment.

``There weren't many of us, and the field of women's rights law was only
just developing,'' said Ms. Stearns, who as a lawyer with the Center for
Constitutional Rights was instrumental in the struggle to legalize
abortion. ``We all knew each other. We were among the young feminist
progressive lawyers of our day, and it was a wonderful thing to have
sisters doing what we were doing and believing what we believed.''

Ms. Taub,
\href{https://law.rutgers.edu/news/professor-emerita-nadine-taub-77-dies-after-long-illness}{a
professor emerita at Rutgers Law School}, died on June 16 at her home in
Manhattan. She was 77. She had for decades struggled with Langerhans
cell histiocytosis, a rare autoimmune disease, her husband, Olof
Widlund, said in confirming her death.

In 1974, Ms. Taub represented a woman who had reported being raped and
who was then held overnight in a Newark jail as a material witness in
her own assault because the police believed that she was a prostitute.
As she told the journalist Christine VanDeVelde for an article in
\href{https://www.nytimes3xbfgragh.onion/2013/09/05/business/media/judith-daniels-74-editor-of-savvy-magazine-dies.html}{Savvy
magazine} in 1988, she was rattled by the depth of her response to her
client's experience.

``The notion that this woman would be locked up and kept in a cell
overnight after she had been raped was incredible,'' Ms. Taub was quoted
as saying. ``In working on that case, how I felt physically was a
revelation to me. I didn't want my husband to touch me. There was an
element of trust that was gone, and that feeling took a few days to
dissipate.''

\href{https://www.nytimes3xbfgragh.onion/1976/11/23/archives/court-moves-to-curb-witnesslaw-abuses-ruling-in-case-of-rape-victim.html}{Ms.
Taub won the case} for her client, with a court order that the Newark
Police Department end abuses of what was known as the material witness
statute.

She also successfully represented the American Civil Liberties Union
\href{https://www.nytimes3xbfgragh.onion/1974/06/08/archives/hearings-wind-up-in-suit-asking-hospital-abortions-rights-violation.html}{in
a landmark lawsuit against three private hospitals in New Jersey} that
were denying women access to abortions. Ms. Taub argued that the
hospitals were the only adequate health care providers in the area, and
that by refusing to offer abortions, they were depriving the plaintiffs
of their constitutional right to terminate a pregnancy.

``The '70s was the dawn of women's constitutional rights in the U.S.,''
said Nan Hunter, a feminist legal theorist and professor at Georgetown
Law who worked for the A.C.L.U. in New York in the 1980s and alongside
Ms. Taub on a few cases. ``Nadine was very much a part of that wave. Her
legacy extends across precedent-setting cases, legal scholarship and
legal education.''

\includegraphics{https://static01.graylady3jvrrxbe.onion/images/2020/07/21/obituaries/21Taub2/21Taub2-articleLarge.jpg?quality=75\&auto=webp\&disable=upscale}

Ms. Taub was the founder and director of the Women's Rights Litigation
Clinic at Rutgers. In the early 1970s, legal clinics like hers were both
a new source of legal representation and an innovative educational tool,
allowing students to work on real cases.

One such case was brought by a Princeton student, Sally Frank. In the
spring of 1978, when she was a sophomore, Ms. Frank applied for
membership --- ``bickered,'' in Princeton parlance --- in the all-male
eating clubs there. After being denied twice, she asked the Rutgers
clinic to take on her case, and she and Ms. Taub became a legal team.

The case centered on whether the clubs had a symbiotic relationship with
the university and were therefore places of ``public accommodation,'' in
which case they would be covered by New Jersey's anti-discrimination
law. Ms. Frank
\href{https://law.justia.com/cases/federal/appellate-courts/F2/943/270/86129/}{won,
but it took over a decade}to do so.

Ms. Frank recalled attending a Princeton class reunion in the 1980s and
seeing eating club members selling T-shirts with a picture of her face
on them (but given a mustache) along with the slogan ``Better Dead Than
Coed.''

She bought one for herself and Ms. Taub. (It was not until the early
'90s that the last of Princeton's men-only undergraduate clubs were
ordered by the courts to admit women.)

``Nadine was a role model and an incredible feminist,'' Ms. Frank said
of Ms. Taub in a phone interview. ``She was a strategic thinker and a
sharp litigator, and working with someone of that depth and commitment
was a remarkable experience. She taught me how to be a lawyer.''

In 2017, when Rutgers honored Ms. Taub by creating a scholar's position
in her name, her former colleague Jonathan Hyman, who had taught in
Rutgers's constitutional litigation clinic, wrote how ``her thrilling
victories energized us all.''

``Remember the days when some Princeton University eating clubs excluded
women?'' he continued. ``Gone, despite powerful opposition, thanks to
Nadine's work over many years. Is a hostile work environment illegal sex
discrimination? Judge Herbert Stern of the U.S. District Court didn't
think so. Nor, initially, did other courts. But Nadine got Judge Stern's
decision reversed and made the hostile work environment a principal part
of sex discrimination law.''

Nadine Taub was born on Jan. 21, 1943, in Princeton, N.J. Her father,
Abraham Haskell Taub, was a professor of mathematics who had taught at
several universities and was working temporarily in Princeton at the
time. Her mother, Cecilia (Vaslow) Taub, was a homemaker.

In addition to her husband, Mr. Widlund, who is also a mathematician ---
mathematicians often marry the daughters of mathematicians, he said ---
Ms. Taub is survived by her sister, Mara Taub, and her brother, Haskell.

Ms. Taub earned a B.A. in economics from Swarthmore College in 1964, and
during the summers worked as a Head Start teacher in Mississippi. After
graduating from Yale Law School in 1968, she provided legal services for
the poor in the Bronx and then for the A.C.L.U., working out of a
storefront in Newark. She joined the faculty of Rutgers Law School in
1973 and retired in 2000.

Ms. Taub was the co-author of several books and publications on women's
rights and gender discrimination, including ``Sex Discrimination and the
Law: History, Practice and Theory'' (1988).

She said her feminist thinking crystallized early in her career while
she was working on a challenge to a statute restricting abortion. ``In
the process of putting on paper why control over reproduction was
crucial for women,'' she said in the Savvy magazine article, ``I really
began to perceive all the ways women were confined and punished because
of their reproductive functions.''

Not all of Ms. Taub's clients were women. In 1977, she was one of three
lawyers who won a discrimination case for Leon Goldfarb, a widower who
had sued to receive his deceased wife's Social Security benefits. In the
brief for the suit, Ms. Taub was joined by
\href{https://www.aclu.org/files/womensrights/tribute/2.html}{Kathleen
Peratis} and the future Supreme Court justice Ruth Bader Ginsburg, both
from the Women's Rights Project at the A.C.L.U. (Justice Ginsburg was
then a professor at Columbia Law School.)

As
\href{https://www.nytimes3xbfgragh.onion/1977/03/03/archives/leon-goldfarb-doubly-happy-with-decision.html}{Mr.
Goldfarb told Anna Quindlen of The New York Times}, ``These ladies
presented their cases beautifully.''

Advertisement

\protect\hyperlink{after-bottom}{Continue reading the main story}

\hypertarget{site-index}{%
\subsection{Site Index}\label{site-index}}

\hypertarget{site-information-navigation}{%
\subsection{Site Information
Navigation}\label{site-information-navigation}}

\begin{itemize}
\tightlist
\item
  \href{https://help.nytimes3xbfgragh.onion/hc/en-us/articles/115014792127-Copyright-notice}{©~2020~The
  New York Times Company}
\end{itemize}

\begin{itemize}
\tightlist
\item
  \href{https://www.nytco.com/}{NYTCo}
\item
  \href{https://help.nytimes3xbfgragh.onion/hc/en-us/articles/115015385887-Contact-Us}{Contact
  Us}
\item
  \href{https://www.nytco.com/careers/}{Work with us}
\item
  \href{https://nytmediakit.com/}{Advertise}
\item
  \href{http://www.tbrandstudio.com/}{T Brand Studio}
\item
  \href{https://www.nytimes3xbfgragh.onion/privacy/cookie-policy\#how-do-i-manage-trackers}{Your
  Ad Choices}
\item
  \href{https://www.nytimes3xbfgragh.onion/privacy}{Privacy}
\item
  \href{https://help.nytimes3xbfgragh.onion/hc/en-us/articles/115014893428-Terms-of-service}{Terms
  of Service}
\item
  \href{https://help.nytimes3xbfgragh.onion/hc/en-us/articles/115014893968-Terms-of-sale}{Terms
  of Sale}
\item
  \href{https://spiderbites.nytimes3xbfgragh.onion}{Site Map}
\item
  \href{https://help.nytimes3xbfgragh.onion/hc/en-us}{Help}
\item
  \href{https://www.nytimes3xbfgragh.onion/subscription?campaignId=37WXW}{Subscriptions}
\end{itemize}
