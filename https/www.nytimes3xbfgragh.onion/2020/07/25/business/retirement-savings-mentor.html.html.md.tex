Sections

SEARCH

\protect\hyperlink{site-content}{Skip to
content}\protect\hyperlink{site-index}{Skip to site index}

\href{https://www.nytimes3xbfgragh.onion/section/business}{Business}

\href{https://myaccount.nytimes3xbfgragh.onion/auth/login?response_type=cookie\&client_id=vi}{}

\href{https://www.nytimes3xbfgragh.onion/section/todayspaper}{Today's
Paper}

\href{/section/business}{Business}\textbar{}The Retirement Savings Coach
You Already Know

\href{https://nyti.ms/2CLOcPo}{https://nyti.ms/2CLOcPo}

\begin{itemize}
\item
\item
\item
\item
\item
\end{itemize}

Advertisement

\protect\hyperlink{after-top}{Continue reading the main story}

Supported by

\protect\hyperlink{after-sponsor}{Continue reading the main story}

retiring

\hypertarget{the-retirement-savings-coach-you-already-know}{%
\section{The Retirement Savings Coach You Already
Know}\label{the-retirement-savings-coach-you-already-know}}

Workers who have learned how to manage their finances and sock away some
money are teaching colleagues and friends.

\includegraphics{https://static01.graylady3jvrrxbe.onion/images/2020/07/26/business/24RETIRING-01/24RETIRING-01-articleLarge.jpg?quality=75\&auto=webp\&disable=upscale}

By Elizabeth Harris

\begin{itemize}
\item
  July 25, 2020
\item
  \begin{itemize}
  \item
  \item
  \item
  \item
  \item
  \end{itemize}
\end{itemize}

When the pandemic was declared in March and the U.S. stock market
plunged, Dr. Bryan T. Whitlow was surprisingly sanguine about his
retirement investments. Unlike many Americans, he didn't radically cut
his spending, didn't check his balances constantly and didn't yank money
from the market.

``I have not even looked at my I.R.A. since this happened,'' said Dr.
Whitlow, a fellow in pain medicine at the University of Kentucky.

But that kind of detachment didn't happen by accident. Dr. Whitlow, 29,
had talked with his father --- and with a colleague who has become
something of a financial mentor to him. They reassured him that he had
little cause for concern about his retirement investments losing value
in the short term. (An emergency savings fund helped, too.)

At a time when companies are devoting significant attention to helping
their employees plan for retirement, some workers are turning to their
peers and colleagues for more personal guidance. Seeing the choices made
by a friend or co-worker can help demystify investment options and offer
an example to follow amid turbulent markets and a contracting economy.

Dr. Whitlow was benefiting from a project that his instructor, Dr. Ty L.
Bullard, had begun at the University of North Carolina School of
Medicine to encourage residents to start thinking about and planning for
retirement early in their careers. Each year, Dr. Bullard schedules
lectures and quarterly panel discussions for doctors to share some of
their financial decisions, divulge mistakes and discuss investing.

Dr. Bullard has taken it on himself to help educate younger doctors by
sharing what he has learned and start a longer-term financial
conversation --- he was available for residents' questions, for example,
when they received their \$1,200 stimulus check and wondered what to do
with it.

``Our residents now have a place they feel they can get unbiased,
non-conflicted information and they feel comfortable asking those kinds
of questions,'' he said.

The very act of selecting a financial mentor can be motivating,
according to Sarah Newcomb, a director of behavioral science at
Morningstar, who has studied financial mentors. Her findings suggest
that people who admire and emulate someone's choices or habits --- such
as frugality or financial security --- rather than focus on a specific
monetary goal, can actually be more successful in reaching their own
targets.

``An aspirational comparison doesn't ask, `Do I measure up to them now?'
but asks, `Can I measure up to them in the future and can I follow in
their footsteps?''' Dr. Newcomb said.

Alannah NicPhaidin, a consultant in international trade compliance who
lives in Aurora, Colo., has become an accidental adviser to her friends.
She loves crunching numbers and reviewing investments, and eagerly
shares her experience. ``It made me realize the importance of financial
security, particularly for women --- particularly making sure you have
that protection for yourself,'' she said.

Questions from friends tend to crop up when Ms. NicPhaidin, 32, declines
invitations to expensive outings --- if she can't afford them, she isn't
shy about saying so, which often prompts a discussion about saving and
investing. If her friends are still interested, she helps them scour
their financial statements, showing them the kinds of accounts she
herself has set up for automatic investing.

Ms. NicPhaidin tries to make the conversation more palatable by serving
wine or chocolates before getting into the spreadsheets. ``If you're
excited about providing them with their financial security, and you're
able to show how you needed it and what it could do for them in the
future, then they get pretty excited too,'' Ms. NicPhaidin said. ``And
it doesn't just have to be some boring thing of some old guy telling
you, `You have to budget better.'''

Helping her friends feels personal. When she was 9, Ms. NicPhaidin said,
she saw her mother struggle as a single parent after the two emigrated
from Ireland with nothing more than one suitcase of clothes and one
suitcase of teddy bears to start a new life in Florida. She remembers
the hardships they encountered from her mother's irregular income and
how her paltry paycheck sometimes meant eating food discarded by their
local supermarket.

Ms. NicPhaidin said she never wanted to experience that kind of
financial insecurity again. Now, she saves 65 percent of her income,
fueled by a frugality that began after she got her first job as a
grocery bagger at Winn Dixie at age 14.

``I would get my paycheck and most kids would be like, `I'm going to the
mall, I'm going to do this,''' she said. ``And every paycheck was cashed
and put into an envelope and not looked at again, and then if I was
feeling stressed I would count it and go, `Oh I have a little bit more,'
so it was a kind of way to make sure that there was security.''

Ms. NicPhaidin's enthusiasm for saving and investing was hard to resist
for Juliet Swanson, a landscape designer in her mid-20s from Falls
Church, Va., who says Ms. NicPhaidin helped her turn her finances
around.

When the two met in 2018, Ms. Swanson was reeling from the news that her
roommate had collected her rent money, but never paid their landlord,
leading to their eviction, she said. Ms. NicPhaidin not only took her
friend in, but also combed through Ms. Swanson's financial documents and
shared her own.

``I would tell her how I handle money and she would kind of say, `OK,
this is just my experience, and why don't we get together with a cup of
tea and have a quick chat,' and saying it all with her lovely Irish
accent and I would say `Of course,'' Ms. Swanson said. ``She would go
over the books and she would give me a good plan.''

Choosing a peer role model whose goals match your own is crucial, Dr.
Newcomb said. First, make sure your prospective mentor has knowledge and
skill --- and values that align with yours. As social creatures, she
said, people tend to compare themselves to others --- especially a
friend and family. She found a trend: If a person compared herself to
someone who seemed to be doing better, the person tended to report lower
savings and higher spending. But people who compared themselves to a
financial role model broke the negative pattern and tended to save more.

``Regardless of income, in one sense who you compare yourself to may
really have a strong impact on your emotional well-being,'' Dr. Newcomb
said.

While focusing on her medical-school studies and clocking long hours in
her residency at Johns Hopkins Hospital, Dr. Margaret Kott, 30, had
little time to worry about retirement. It wasn't until a friend and
co-resident asked her if she had started saving that she realized she
had fallen behind.

``At 30, I realized how little I had saved, and I kind of panicked,''
Dr. Kott said. At her friend's urging, she opened her first Roth I.R.A.
and has started contributing to a retirement fund within the last year.
She has now become an advocate, helping her friends and colleagues.

``I'm doing what I can to spread this knowledge amongst my co-residents,
who know little of retirement accounts and don't usually think twice
about how to pay for retirement because they're too focused on paying
back student loans,'' Dr. Kott said. ``It's unbelievable that this was
never taught to us in any school.''

As chief resident in her department at Johns Hopkins, Dr. Kott has
helped arrange lectures over the last year and began booking financial
advisers for talks, sometimes pivoting to Zoom sessions as the pandemic
hit.

Dr. Kott said she realized that she was in a position uncommon for most
residents: She has avoided student loan debt as much as she could.
First, she picked the college that gave her a full scholarship for her
undergraduate degree. Later, with her husband, she focused on paying
down the medical school debt. She recently moved to Boston for a new
role as a fellow in pain medicine at Massachusetts General Hospital.

\includegraphics{https://static01.graylady3jvrrxbe.onion/images/2020/07/24/business/24RETIRING-02/merlin_170884872_0921fee2-fd1a-4bae-b10c-02aec96baf34-articleLarge.jpg?quality=75\&auto=webp\&disable=upscale}

Dr. Bullard, 43, sees a strong need for doctors to get solid financial
advice sooner rather than later, because they usually start their
careers later than most and typically carry high student loan debt. And
they can be vulnerable to bad advice: It's only in retrospect that he
sees how he and his wife mistakenly thought their best investment choice
was a high-commission, permanent life insurance policy. But that misstep
served as a warning to fellow doctors and residents about the pitfalls
of not focusing on their financial acumen.

To that end, Dr. Bullard has developed and helps oversee a financial
curriculum for students at the University of North Carolina, Chapel
Hill. He is the division chief of ambulatory anesthesia at UNC Hospitals
and instructs residents, but he also schedules regular money talks given
by fellow doctors. The intention: to help residents and their peers
start saving for retirement while protecting them from being pressured
into buying high-commission products that might be wrong for their
needs.

``One of our main objectives is, they don't have to worry someone is
looking at them like an open wallet,'' he said.

Hearing from more experienced friends or colleagues in anesthesiology
made financial planning more meaningful for Dr. Whitlow.

``When you hear it from someone who's in the same situation,'' he said,
``I think it's more impactful because you know they just came from where
you were.''

Advertisement

\protect\hyperlink{after-bottom}{Continue reading the main story}

\hypertarget{site-index}{%
\subsection{Site Index}\label{site-index}}

\hypertarget{site-information-navigation}{%
\subsection{Site Information
Navigation}\label{site-information-navigation}}

\begin{itemize}
\tightlist
\item
  \href{https://help.nytimes3xbfgragh.onion/hc/en-us/articles/115014792127-Copyright-notice}{©~2020~The
  New York Times Company}
\end{itemize}

\begin{itemize}
\tightlist
\item
  \href{https://www.nytco.com/}{NYTCo}
\item
  \href{https://help.nytimes3xbfgragh.onion/hc/en-us/articles/115015385887-Contact-Us}{Contact
  Us}
\item
  \href{https://www.nytco.com/careers/}{Work with us}
\item
  \href{https://nytmediakit.com/}{Advertise}
\item
  \href{http://www.tbrandstudio.com/}{T Brand Studio}
\item
  \href{https://www.nytimes3xbfgragh.onion/privacy/cookie-policy\#how-do-i-manage-trackers}{Your
  Ad Choices}
\item
  \href{https://www.nytimes3xbfgragh.onion/privacy}{Privacy}
\item
  \href{https://help.nytimes3xbfgragh.onion/hc/en-us/articles/115014893428-Terms-of-service}{Terms
  of Service}
\item
  \href{https://help.nytimes3xbfgragh.onion/hc/en-us/articles/115014893968-Terms-of-sale}{Terms
  of Sale}
\item
  \href{https://spiderbites.nytimes3xbfgragh.onion}{Site Map}
\item
  \href{https://help.nytimes3xbfgragh.onion/hc/en-us}{Help}
\item
  \href{https://www.nytimes3xbfgragh.onion/subscription?campaignId=37WXW}{Subscriptions}
\end{itemize}
