Sections

SEARCH

\protect\hyperlink{site-content}{Skip to
content}\protect\hyperlink{site-index}{Skip to site index}

\href{https://www.nytimes3xbfgragh.onion/section/travel}{Travel}

\href{https://myaccount.nytimes3xbfgragh.onion/auth/login?response_type=cookie\&client_id=vi}{}

\href{https://www.nytimes3xbfgragh.onion/section/todayspaper}{Today's
Paper}

\href{/section/travel}{Travel}\textbar{}Travel the World Through These
Dance Tutorials

\url{https://nyti.ms/2EjLlO3}

\begin{itemize}
\item
\item
\item
\item
\item
\end{itemize}

\href{https://www.nytimes3xbfgragh.onion/spotlight/at-home?action=click\&pgtype=Article\&state=default\&region=TOP_BANNER\&context=at_home_menu}{At
Home}

\begin{itemize}
\tightlist
\item
  \href{https://www.nytimes3xbfgragh.onion/2020/07/28/books/time-for-a-literary-road-trip.html?action=click\&pgtype=Article\&state=default\&region=TOP_BANNER\&context=at_home_menu}{Take:
  A Literary Road Trip}
\item
  \href{https://www.nytimes3xbfgragh.onion/2020/07/29/magazine/bored-with-your-home-cooking-some-smoky-eggplant-will-fix-that.html?action=click\&pgtype=Article\&state=default\&region=TOP_BANNER\&context=at_home_menu}{Cook:
  Smoky Eggplant}
\item
  \href{https://www.nytimes3xbfgragh.onion/2020/07/27/travel/moose-michigan-isle-royale.html?action=click\&pgtype=Article\&state=default\&region=TOP_BANNER\&context=at_home_menu}{Look
  Out: For Moose}
\item
  \href{https://www.nytimes3xbfgragh.onion/interactive/2020/at-home/even-more-reporters-editors-diaries-lists-recommendations.html?action=click\&pgtype=Article\&state=default\&region=TOP_BANNER\&context=at_home_menu}{Explore:
  Reporters' Obsessions}
\end{itemize}

Advertisement

\protect\hyperlink{after-top}{Continue reading the main story}

Supported by

\protect\hyperlink{after-sponsor}{Continue reading the main story}

\hypertarget{travel-the-world-through-these-dance-tutorials}{%
\section{Travel the World Through These Dance
Tutorials}\label{travel-the-world-through-these-dance-tutorials}}

Here are eight cultural dances that you can learn at home through online
tutorials or mobile dance apps.

\includegraphics{https://static01.graylady3jvrrxbe.onion/images/2020/07/26/multimedia/26ah-dance2/26ah-dance2-articleLarge.jpg?quality=75\&auto=webp\&disable=upscale}

By Caterina Hrysomallis

\begin{itemize}
\item
  July 25, 2020
\item
  \begin{itemize}
  \item
  \item
  \item
  \item
  \item
  \end{itemize}
\end{itemize}

Do you have itchy feet after all these months of lockdown? Now could be
the time to do a little homework on the world of dance. And dance does
span the globe, with genres found all over that are historically
integral to their communities, cultures and regions.

Of course, being culturally engaged is not the only advantage of
learning to dance. Moving your body can provide great comforts, both
emotionally and physically. Not only does dancing provide health
benefits, improving everything from cardiovascular strength to bone
density to brain activity, moving your body has also been proven to
reduce stress and anxiety --- which everyone could benefit from right
now.

Here are eight cultural dances that you can learn at home through online
tutorials or mobile dance apps. Take a look (or a spin) and you might be
well equipped when the world is ready for globe-trotting, street parties
and celebrations with no limits.

\hypertarget{belly-dancing}{%
\subsection{Belly Dancing}\label{belly-dancing}}

\textbf{ORIGIN} Egypt. Also commonly danced in Turkey, the Middle East
and India.

When Egyptian women gathered to socialize in the 18th century, they
would belly dance. It was a celebration of the feminine, a dance form
that distinguishes itself through sharp hip movements, belly rolls and
tricky flutters, especially when paired with shimmies, torso isolations
and wavelike movements of the hands and arms.

Nowadays, it's danced at celebrations of all sorts, including weddings
and birthdays, and is synonymous with colorful, intricate outfits and
coin-laden hip scarves that chime to typically high-tempo music. Belly
dancing may take a long time to master, but the dancer Iana
Komarnytska's
\href{https://www.youtube.com/channel/UCT2GoXkVIWza-7ngoklxdaw}{YouTube
channel} is full of drills and tutorials that help whether you're a
beginner or more acquainted with the style.

\hypertarget{salsa}{%
\subsection{Salsa}\label{salsa}}

\textbf{ORIGIN} Cuba. Also commonly danced throughout West Africa and
the Americas.

At a comfortable tempo between not too slow and not too fast, salsa is
one of the most popular Latin dance styles worldwide. It's commonly
performed with a partner, but it can certainly be danced solo, too.

Salsa found its popularity in Cuba's old casinos and community halls.
It's a joyous dance, full of dynamism. It has also spawned many
subgenres thanks to the Cuban diaspora, including Cali style (Colombian
salsa, which is more up-tempo, with faster turns) and New York-style (a
little smoother, influenced by the mambo). Salsa can be sensual or
family-friendly --- it simply depends on whom you're dancing with. Learn
everything from basics to more complicated sequences with the
\href{https://www.addicted2salsa.com/pocketsalsa/}{Pocket Salsa app}.

\hypertarget{break-dancing}{%
\subsection{Break Dancing}\label{break-dancing}}

\includegraphics{https://static01.graylady3jvrrxbe.onion/images/2020/07/26/multimedia/26ah-dance1/merlin_174832122_c5caccc1-01f9-478b-8a32-ad63a19dd9c3-articleLarge.jpg?quality=75\&auto=webp\&disable=upscale}

\textbf{ORIGIN} New York City. Popular around the world.

Born in the borough of the Bronx, break dancing was created during the
1960s by street gangs who modified martial arts moves that were
originally learned for defensive purposes.

Its popularity quickly spread across the United States, and break
dancing is now an integral part of hip-hop culture. Start by stepping
back in time and learning the fundamentals of old-school break dancing
with moves such as the baby freeze, the running man, the smurf and the
wop. If you head down the self-taught break dancing route, kindly do so
with caution, but there are plenty of online studios dedicated to
teaching the style, including
\href{https://www.youtube.com/channel/UCo7SFDRvz0R7XDzM9pN1JCg}{Free
Focus Dance on YouTube.}

\hypertarget{kizomba}{%
\subsection{Kizomba}\label{kizomba}}

\textbf{ORIGIN} Angola. Also commonly danced throughout Africa, the
Caribbean, Central America and South America.

Slow and mesmerizingly romantic, Kizomba is performed in an extremely
close embrace by two people in a loose ballroom hold. The rib cages of
dancers touch for a majority of sequences, as they slowly travel in all
directions via small steps, incorporating turns and body rolls in
between. Kizomba's relaxed nature makes it ideal for older dancers.

Created in Angola, the style --- particularly in its song lyrics --- is
heavily influenced by the country's long history with Portugal. Kizomba
is traditionally danced in clubs and at parties held on the street or in
someone's home. While popular in Portugal among the Angolan community,
the style also has a strong following in France, Spain, Poland, Belgium
and Denmark. The \href{https://www.worldkizombaproject.com/}{World
Kizomba Project}, found online, offers five free beginner lessons and
paid courses of higher difficulty.

\hypertarget{champeta}{%
\subsection{Champeta}\label{champeta}}

\textbf{ORIGIN} Colombia. Also commonly danced throughout Central and
South America.

Champeta emerged in the 1970s in Colombia's coastal barrios, and made
its way into the mainstream thanks to its addictive drumbeat rhythms and
the celebratory nature of its moves. Its succinct steps are performed
with bent knees, in a series of jumps and hops involving a quick
transference of weight from side to side --- you'll need to be extremely
quick on your feet for this one.

After the singer Shakira performed the dance during the halftime show at
the 2020 Super Bowl, the ``Champeta Challenge'' swept social media
globally. Learn champeta in preparation for music festivals, club
gatherings or street parties --- both impromptu and organized --- like
in
\href{https://theculturetrip.com/south-america/colombia/articles/how-to-spend-3-days-in-cartagena/}{Cartagena}
and Barranquilla. You can dance it alone, in a group or with a partner.
The Colombian choreographer and dancer Jey Colon has a wide range of
champeta tutorials on his
\href{https://www.youtube.com/c/JeyColon/featured}{YouTube channel.}

\hypertarget{country-line-dancing}{%
\subsection{Country Line Dancing}\label{country-line-dancing}}

Image

Credit...Michael Shavel for The New York Times

\textbf{ORIGIN} Europe and the United States. Danced widely throughout
the United States.

Country line dancing takes its inspiration from English folk dancing and
music, and evolved into its current form in the United States following
the English settlement there.

Throughout the United States, you'll find people in bars and clubs and
at hoedowns arranging themselves in rows and dancing (and sometimes
singing) to the same sequence of steps. Start by learning the cha-cha
slide, tango with the sheriff and the cowboy hustle steps. The Idaho
studio Dirt Road Dancing has a wide library of country-style line dances
lessons available on its
\href{https://www.youtube.com/c/DirtRoadDancing/featured}{YouTube}
channel.

\hypertarget{bhangra}{%
\subsection{Bhangra}\label{bhangra}}

\textbf{ORIGIN} Punjab, India. Popular throughout India.

Bhangra makes up much of what you see in the dance scenes of Bollywood
movies. Traditionally performed to the melodies of Indian folk
instruments --- and to lyrics of love, strength and patriotism ---
Bhangra is an expression of jubilation.

This style is danced in a group, commonly at weddings and cultural
festivals, and combines jumps, hops, shoulder shakes and
larger-than-life extensions of the limbs. It requires a high level of
stamina and athleticism, and is a cardio workout that can leave you
breathless. Look to \href{https://learnbhangra.com/}{Learn Bhangra}, a
group that runs free classes on its social media channels, to get
started.

\hypertarget{polka}{%
\subsection{Polka}\label{polka}}

Image

Credit...Carl T. Gossett Jr./The New York Times

\textbf{ORIGIN} The Czech Republic. Commonly danced in Central and
Eastern Europe, and the United States.

Polka is a series of lively, fast-traveling ``half'' jumps danced by
couples, or a series of couples to form a group. It is a favorite in the
beer halls of Central and Eastern Europe, and at weddings.

First danced by the lower classes of old Bohemia, polka eventually made
its way into the grand ballrooms of Prague, the arenas of the country's
high society. The popularity of polka in the United States is the result
of two-way travel --- European migration to the United States and
American soldiers serving in Europe during World War II.

Polka is an eeasy dance to learn. To give it a whirl, visit watch a
\href{https://www.youtube.com/watch?v=y846w6PUmCw}{two-minute polka
tutorial} on the how-to channel Howcast on YouTube.

\emph{\textbf{Follow New York Times Travel}}
\emph{on}\href{https://www.instagram.com/nytimestravel/}{\emph{Instagram}}\emph{,}\href{https://twitter.com/nytimestravel}{\emph{Twitter}}
\emph{and}\href{https://www.facebookcorewwwi.onion/nytimestravel/}{\emph{Facebook}}\emph{.
And}\href{https://www.nytimes3xbfgragh.onion/newsletters/traveldispatch}{\emph{sign
up for our weekly Travel Dispatch newsletter}} \emph{to receive expert
tips on traveling smarter and inspiration for your next vacation.}

Advertisement

\protect\hyperlink{after-bottom}{Continue reading the main story}

\hypertarget{site-index}{%
\subsection{Site Index}\label{site-index}}

\hypertarget{site-information-navigation}{%
\subsection{Site Information
Navigation}\label{site-information-navigation}}

\begin{itemize}
\tightlist
\item
  \href{https://help.nytimes3xbfgragh.onion/hc/en-us/articles/115014792127-Copyright-notice}{©~2020~The
  New York Times Company}
\end{itemize}

\begin{itemize}
\tightlist
\item
  \href{https://www.nytco.com/}{NYTCo}
\item
  \href{https://help.nytimes3xbfgragh.onion/hc/en-us/articles/115015385887-Contact-Us}{Contact
  Us}
\item
  \href{https://www.nytco.com/careers/}{Work with us}
\item
  \href{https://nytmediakit.com/}{Advertise}
\item
  \href{http://www.tbrandstudio.com/}{T Brand Studio}
\item
  \href{https://www.nytimes3xbfgragh.onion/privacy/cookie-policy\#how-do-i-manage-trackers}{Your
  Ad Choices}
\item
  \href{https://www.nytimes3xbfgragh.onion/privacy}{Privacy}
\item
  \href{https://help.nytimes3xbfgragh.onion/hc/en-us/articles/115014893428-Terms-of-service}{Terms
  of Service}
\item
  \href{https://help.nytimes3xbfgragh.onion/hc/en-us/articles/115014893968-Terms-of-sale}{Terms
  of Sale}
\item
  \href{https://spiderbites.nytimes3xbfgragh.onion}{Site Map}
\item
  \href{https://help.nytimes3xbfgragh.onion/hc/en-us}{Help}
\item
  \href{https://www.nytimes3xbfgragh.onion/subscription?campaignId=37WXW}{Subscriptions}
\end{itemize}
