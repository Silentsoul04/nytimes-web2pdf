Sections

SEARCH

\protect\hyperlink{site-content}{Skip to
content}\protect\hyperlink{site-index}{Skip to site index}

\href{https://www.nytimes3xbfgragh.onion/section/world/asia}{Asia
Pacific}

\href{https://myaccount.nytimes3xbfgragh.onion/auth/login?response_type=cookie\&client_id=vi}{}

\href{https://www.nytimes3xbfgragh.onion/section/todayspaper}{Today's
Paper}

\href{/section/world/asia}{Asia Pacific}\textbar{}As the World Gets
Tougher on China, Japan Tries to Thread a Needle

\url{https://nyti.ms/2CE6SAH}

\begin{itemize}
\item
\item
\item
\item
\item
\end{itemize}

Advertisement

\protect\hyperlink{after-top}{Continue reading the main story}

Supported by

\protect\hyperlink{after-sponsor}{Continue reading the main story}

\hypertarget{as-the-world-gets-tougher-on-china-japan-tries-to-thread-a-needle}{%
\section{As the World Gets Tougher on China, Japan Tries to Thread a
Needle}\label{as-the-world-gets-tougher-on-china-japan-tries-to-thread-a-needle}}

Tokyo has not confronted Beijing as the United States and other allies
have, mindful of its neighbor's economic might and its own limited
military options.

\includegraphics{https://static01.graylady3jvrrxbe.onion/images/2020/07/22/world/22japan-china1/merlin_166326492_243a5831-46c5-4ad8-b991-6ec8fa131658-articleLarge.jpg?quality=75\&auto=webp\&disable=upscale}

By \href{https://www.nytimes3xbfgragh.onion/by/motoko-rich}{Motoko Rich}
and Makiko Inoue

\begin{itemize}
\item
  July 25, 2020
\item
  \begin{itemize}
  \item
  \item
  \item
  \item
  \item
  \end{itemize}
\end{itemize}

TOKYO --- Earlier this year, as it became clear that the coronavirus
pandemic was not going to pass quickly, the Japanese government delayed
plans for what would be the first state visit by a Chinese leader to
Tokyo since 2008.

Now, with Chinese military aggression rising in the region and Beijing
\href{https://www.nytimes3xbfgragh.onion/2020/06/29/world/asia/china-hong-kong-security-law-rules.html}{cracking
down} on Hong Kong, Japan is considering canceling Xi Jinping's visit
altogether --- but very gingerly.

``We are not in the phase of arranging a concrete schedule now'' was how
Toshimitsu Motegi, the foreign minister, put it this month.

While its top allies have taken a harder line on China --- especially
the United States, which dramatically escalated tensions this past week
by
\href{https://www.nytimes3xbfgragh.onion/2020/07/22/world/asia/us-china-houston-consulate.html?action=click\&module=Top\%20Stories\&pgtype=Homepage}{closing
the Chinese Consulate} in Houston --- Japan has pursued a delicate
balancing act, mindful of the economic might of its largest trading
partner and its own limited military options.

So as Chinese ships have engaged in the longest series of incursions in
or near Japanese waters in several years, Japan has offered a restrained
response, vowing to be firm but ``calm.'' It did not join several
Western nations in an initial statement criticizing the
\href{https://www.nytimes3xbfgragh.onion/2020/06/29/world/asia/china-hong-kong-security-law-rules.html?searchResultPosition=6}{draconian
security law} that Beijing imposed on Hong Kong.

It has abandoned plans to purchase an American missile defense system,
which in part had been considered a shield against China. And the
government has continued to tiptoe around the issue of the state visit
by Mr. Xi, even as polls show that most Japanese believe it should be
scrapped.

``Certainly Japan is in a dilemma,'' said Narushige Michishita, director
of the Security and International Studies Program at the National
Graduate Institute for Policy Studies in Tokyo. ``We understand the fact
that Japan is basically competing with China while cooperating with it.
We are playing those two games at the same time.''

For other world powers, this kind of middle ground on China, in the face
of its growing authoritarianism and heightened bellicosity, has become
less and less tenable.

\includegraphics{https://static01.graylady3jvrrxbe.onion/images/2020/07/22/world/22japan-china2/merlin_174102576_f8696fc0-34f4-4616-a8fe-321bfad4b6e2-articleLarge.jpg?quality=75\&auto=webp\&disable=upscale}

The United States has approved
\href{https://www.nytimes3xbfgragh.onion/2020/06/29/business/economy/us-halts-high-tech-exports-hong-kong.html}{sanctions}
against Chinese companies and officials and
\href{https://www.nytimes3xbfgragh.onion/2020/07/13/world/asia/south-china-sea-pompeo.html}{pushed
back} on China's broad territorial claims in the
\href{https://www.nytimes3xbfgragh.onion/2020/06/26/international-home/china-military-india-taiwan.html?searchResultPosition=2}{region's
seas}. Australia
\href{https://www.nytimes3xbfgragh.onion/2020/05/11/world/australia/coronavirus-china-inquiry.html}{led
the call for a global inquiry} into the origins of the pandemic, which
began in China, and announced an investment of
\href{https://www.nytimes3xbfgragh.onion/2020/06/30/world/australia/cyber-defense-china-hacking.html}{nearly
\$1 billion in cyberweapons and defenses} to counter Beijing.
\href{https://www.nytimes3xbfgragh.onion/2020/07/20/world/asia/extradition-treaty-hong-kong.html}{Britain}
and Canada, along with Australia, have suspended extradition agreements
with Hong Kong because of the national security law.

China has responded by curbing Australian imports and threatening an
array of retaliatory actions against any countries that move to punish
it. On Friday, China responded to the closure of its Houston consulate
by
\href{https://www.nytimes3xbfgragh.onion/2020/07/24/world/asia/china-us-consulate-chengdu.html?action=click\&module=Top\%20Stories\&pgtype=Homepage}{ordering
the United States to shut its consulate in the southwestern city of
Chengdu}.

To some extent, Japan's mild-mannered response to China echoes its
broader approach to foreign policy, in which it tends to avoid direct
conflict or public rebukes of other nations. It has also sometimes
sought a mediating role, as when
\href{https://www.nytimes3xbfgragh.onion/2019/12/20/world/asia/japan-iran-rouhani-abe.html?searchResultPosition=1}{Prime
Minister Shinzo Abe met last December with Iran's president, Hassan
Rouhani}, to try to ease tensions in the Middle East.

Not so long ago, China and Japan --- the world's second- and
third-largest economies --- were engaged in a diplomatic thaw as a
\href{https://www.nytimes3xbfgragh.onion/2018/10/24/world/asia/china-japan-shinzo-abe.html?rref=collection\%2Fsectioncollection\%2Fasia\&action=click\&contentCollection=asia\&region=stream\&module=stream_unit\&version=latest\&contentPlacement=9\&pgtype=sectionfront}{hedge}
against an unpredictable Trump administration. In 2018, Mr. Abe became
\href{https://www.nytimes3xbfgragh.onion/2018/10/26/world/asia/shinzo-abe-china-japan.html?searchResultPosition=2}{the
first Japanese leader to visit China} in seven years, and the two
leaders pledged deeper economic and political cooperation. The
invitation to Mr. Xi to visit Japan followed soon afterward.

Now, given China's muscle-flexing as the world is preoccupied with the
pandemic, some have expressed disappointment that Japan has not rebuffed
its neighbor more vigorously, such as by definitively canceling Mr. Xi's
visit. In recent weeks, China has engaged in
\href{https://www.nytimes3xbfgragh.onion/2020/06/16/world/asia/indian-china-border-clash.html}{deadly
clashes on its border with India} in the Himalayas, and it has sent
ships for 100 straight days --- the longest period in years of such
incursions --- to patrol waters around the Senkakus, islands
administered by Japan but contested by China.

Japan ``should just say `we cannot have him if China continues with this
sort of behavior,''' said Jeffrey Hornung, an analyst at the RAND
Corporation, referring to Mr. Xi. But Mr. Hornung acknowledged that
Tokyo would not want to draw China's full ire, either.

``If you look at what China is doing with India or Hong Kong, Japan
doesn't want to be at the tip of China's spear right now,'' Mr. Hornung
said. ``They know what they could do around the Senkakus in terms of
swarming it with ships.''

On Hong Kong, Japan did not join the United States, Australia, Canada
and Britain in an initial statement criticizing the national security
law.

Image

A Japanese Self-Defense Forces plane over the disputed Senkaku islands
in 2011.Credit...Kyodo News, via Associated Press

It subsequently led an effort to draft a
\href{https://www.mofa.go.jp/press/release/press4e_002844.html}{statement}
by the foreign ministers of the Group of 7 countries expressing ``grave
concern'' about the law, and Japan's governing Liberal Democratic Party
passed a resolution this month saying it could not ``just remain on the
sidelines seeing the situation'' in Hong Kong.

In response to the Chinese incursions in the East China Sea, Japan has
mobilized Self-Defense Forces fighter jets to patrol the area. It
continues, though, to use mild language in its protests to the Chinese
government.

Yoshihide Suga, chief cabinet secretary to Mr. Abe, told reporters that
the Japanese government had ``strongly requested'' that Chinese ships
``stop approaching Japanese fishing boats and quickly leave Japanese
territory.'' He added, ``We would like to continue responding firmly in
a calm manner.''

Parts of the Japanese government have highlighted China's growing
hostility. Earlier this month, the defense ministry warned that China
was trying to ``alter the status quo in the East China Sea and the South
China Sea,'' and it ranked China as a more serious long-term threat than
North Korea.

Yet Japan's recent decision to abandon its plan to buy an American
missile defense system, known as Aegis Ashore, led some to wonder if it
would now be more exposed to potential attacks from both North Korea and
China.

The decision may have looked to some like a genuflection to Beijing. But
soon afterward, the defense committee of the governing party discussed
whether Japan could acquire weapons that would enable it to strike a
missile launch site, if it detected signs of an imminent attack by a
neighbor.

Those discussions are in the early stages, and they would require
extensive examination by constitutional experts to determine if such a
capability would violate the
\href{https://www.nytimes3xbfgragh.onion/2017/05/03/world/asia/japan-constitution-shinzo-abe-military.html?searchResultPosition=1}{pacifist
clause in Japan's Constitution}.

``While the cancellation of Aegis Ashore might put Japan in a more
vulnerable position, if Japan uses this opportunity to pivot to
acquisition of other capabilities, then the result could be even more
worrying for China,'' said Kristi Govella, an assistant professor in the
department of Asian studies at the University of Hawaii at Manoa.

Image

Chinese tourists in the Ginza shopping district of Tokyo in
January.Credit...Tomohiro Ohsumi/Getty Images

One area where Japan has taken steps against China is the economy.
Earlier this year, it passed a law restricting foreign investment in
industries that the government designates as important to national
security, a move that many viewed as targeting China. It has also
offered financial incentives to companies --- especially those in
crucial sectors --- to move operations out of China and into Japan or
Southeast Asia.

``The Chinese economy is recovering while other countries are still
deteriorating,'' said Takahide Kiuchi, an economist at Nomura Research
Institute, a think tank. ``Now China is in a good position to purchase
companies in other countries, so the government is cautious about
critical industries related to the military and national security.''

Still, Japan does not want to push too hard.

In addition to being Japan's largest trading partner, China sent more
tourists to Japan than any other nation before the pandemic shut
borders. Last year, close to 115,000 Chinese students were studying at
Japanese universities. The government, which has imposed entry bans on
nearly 150 countries during the pandemic, is now discussing admitting
travelers from several Asian countries, including China.

``A couple of years ago, it seemed like there was space for Japan to be
seen as a mediator because relations between the U.S. and China had
become so bad,'' Ms. Govella said. But with China's increasing
aggression, it ``really is an actor that has different values and
dubious intentions in the region,'' she said.

As China pursues this more belligerent policy, Japanese analysts say
they hope that Beijing might learn from Japan's own history and not try
to expand its power too far, particularly by repressive means.

Image

A parade last year in Yokohama, Japan, celebrating the 70th anniversary
of the founding of the People's Republic of China.Credit...Tomohiro
Ohsumi/Getty Images

China's efforts to dominate the South China Sea, for example, are ``one
step toward kicking out the Western elements from their sphere of
influence, which they have been dreaming of for the past century and a
half,'' said Kunihiko Miyake, a former Japanese diplomat who is now
teaching at Ritsumeikan University in Kyoto.

``Their nationalistic ambition will not end,'' he said. ``I am very
concerned, and nobody can stop it, as they couldn't stop us in Manchuria
in the 1930s,'' Mr. Miyake said, referring to Japan's invasion of that
region of eastern China.

``At that time, the more pressure we had, the more adamant and arrogant
and self-assertive we became, because we were too nationalistic and too
undemocratic, and that was our destiny,'' Mr. Miyake said. ``China is
following the same path.''

Advertisement

\protect\hyperlink{after-bottom}{Continue reading the main story}

\hypertarget{site-index}{%
\subsection{Site Index}\label{site-index}}

\hypertarget{site-information-navigation}{%
\subsection{Site Information
Navigation}\label{site-information-navigation}}

\begin{itemize}
\tightlist
\item
  \href{https://help.nytimes3xbfgragh.onion/hc/en-us/articles/115014792127-Copyright-notice}{©~2020~The
  New York Times Company}
\end{itemize}

\begin{itemize}
\tightlist
\item
  \href{https://www.nytco.com/}{NYTCo}
\item
  \href{https://help.nytimes3xbfgragh.onion/hc/en-us/articles/115015385887-Contact-Us}{Contact
  Us}
\item
  \href{https://www.nytco.com/careers/}{Work with us}
\item
  \href{https://nytmediakit.com/}{Advertise}
\item
  \href{http://www.tbrandstudio.com/}{T Brand Studio}
\item
  \href{https://www.nytimes3xbfgragh.onion/privacy/cookie-policy\#how-do-i-manage-trackers}{Your
  Ad Choices}
\item
  \href{https://www.nytimes3xbfgragh.onion/privacy}{Privacy}
\item
  \href{https://help.nytimes3xbfgragh.onion/hc/en-us/articles/115014893428-Terms-of-service}{Terms
  of Service}
\item
  \href{https://help.nytimes3xbfgragh.onion/hc/en-us/articles/115014893968-Terms-of-sale}{Terms
  of Sale}
\item
  \href{https://spiderbites.nytimes3xbfgragh.onion}{Site Map}
\item
  \href{https://help.nytimes3xbfgragh.onion/hc/en-us}{Help}
\item
  \href{https://www.nytimes3xbfgragh.onion/subscription?campaignId=37WXW}{Subscriptions}
\end{itemize}
