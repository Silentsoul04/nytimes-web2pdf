Sections

SEARCH

\protect\hyperlink{site-content}{Skip to
content}\protect\hyperlink{site-index}{Skip to site index}

\href{https://www.nytimes3xbfgragh.onion/section/technology}{Technology}

\href{https://myaccount.nytimes3xbfgragh.onion/auth/login?response_type=cookie\&client_id=vi}{}

\href{https://www.nytimes3xbfgragh.onion/section/todayspaper}{Today's
Paper}

\href{/section/technology}{Technology}\textbar{}Don't Ban TikTok. Make
an Example of It.

\url{https://nyti.ms/2EoqB81}

\begin{itemize}
\item
\item
\item
\item
\item
\end{itemize}

Advertisement

\protect\hyperlink{after-top}{Continue reading the main story}

Supported by

\protect\hyperlink{after-sponsor}{Continue reading the main story}

The Shift

\hypertarget{dont-ban-tiktok-make-an-example-of-it}{%
\section{Don't Ban TikTok. Make an Example of
It.}\label{dont-ban-tiktok-make-an-example-of-it}}

There is plenty the U.S. government could do to ensure that TikTok acts
responsibly without getting rid of it altogether.

\includegraphics{https://static01.graylady3jvrrxbe.onion/images/2020/07/25/business/25roose1/merlin_162742050_8f2e22da-5532-46ee-9cec-e5cccb476376-articleLarge.jpg?quality=75\&auto=webp\&disable=upscale}

\href{https://www.nytimes3xbfgragh.onion/by/kevin-roose}{\includegraphics{https://static01.graylady3jvrrxbe.onion/images/2018/02/20/multimedia/author-kevin-roose/author-kevin-roose-thumbLarge.jpg}}

By \href{https://www.nytimes3xbfgragh.onion/by/kevin-roose}{Kevin Roose}

\begin{itemize}
\item
  Published July 26, 2020Updated Aug. 3, 2020
\item
  \begin{itemize}
  \item
  \item
  \item
  \item
  \item
  \end{itemize}
\end{itemize}

\href{https://cn.nytimes3xbfgragh.onion/technology/20200728/tiktok-china-ban-model/}{阅读简体中文版}\href{https://cn.nytimes3xbfgragh.onion/technology/20200728/tiktok-china-ban-model/zh-hant/}{閱讀繁體中文版}

For a while, it seemed that
\href{https://www.nytimes3xbfgragh.onion/2020/08/01/technology/tiktok-trump-microsoft-bytedance-china-ban.html}{TikTok}
might dodge the techlash. After all, what could be problematic about a
short-form video app featuring a bunch of teenagers and 20-somethings
doing
\href{https://www.nytimes3xbfgragh.onion/2020/02/13/style/the-original-renegade.html}{choreographed
dances},
\href{https://www.buzzfeednews.com/article/laurenstrapagiel/roller-skating-is-back-tiktok-ana-coto}{roller
skating}, hanging out in
\href{https://www.nytimes3xbfgragh.onion/2020/01/03/style/hype-house-los-angeles-tik-tok.html}{influencer
mansions} and cutting into
\href{https://www.nytimes3xbfgragh.onion/2020/07/14/style/what-is-the-cake-meme.html}{photorealistic
cakes}?

The answer turns out to be: Plenty.

In the past year, as it has become one of the most popular apps in the
world, TikTok has accumulated many of the same problems that other
large-scale social networks have. In addition to all the harmless Gen Z
fun, there are
\href{https://www.nytimes3xbfgragh.onion/2020/06/29/technology/pizzagate-tiktok.html}{TikTok
conspiracy theories},
\href{https://www.rollingstone.com/culture/culture-features/tiktok-conspiracy-theories-bill-gates-microchip-vaccine-996394/}{TikTok
misinformation} and
\href{https://www.bbc.com/news/technology-53269361}{TikTok extremism}.
There are even activists using TikTok to influence our elections,
including a network of teenagers and K-pop fans who
\href{https://www.nytimes3xbfgragh.onion/2020/06/21/style/tiktok-trump-rally-tulsa.html}{claimed
they used the app} to sabotage President Trump's rally in Tulsa, Okla.,
last month by registering for tickets under false identities.

All of this might have been overlooked or forgiven, except for one fact.
\href{https://www.nytimes3xbfgragh.onion/2020/08/03/technology/tiktok-trump-sale-microsoft.html}{TikTok
is owned by ByteDance}, one of the largest tech companies in China.

TikTok's Chinese ownership has become a subject of intense scrutiny by
lawmakers, regulators and privacy activists in recent weeks. Mr. Trump
is
\href{https://www.nytimes3xbfgragh.onion/reuters/2020/07/20/technology/20reuters-usa-election-trump-tiktok.html}{considering}
taking steps to ban the app in the United States. Companies including
Wells Fargo, and government agencies including the Transportation
Security Administration, have
\href{https://www.theverge.com/2020/7/11/21320935/wells-fargo-bans-tiktok-devices-amazon-pompeo}{instructed
their employees} to delete TikTok from their work phones because of
concerns that it could be used for surveillance or espionage.

In response to the mounting pressures, TikTok is wrapping itself in the
American flag. The company has
\href{https://www.nytimes3xbfgragh.onion/2020/07/15/technology/tiktok-washington-lobbyist.html}{hired
a small army} of lobbyists in Washington, has
\href{https://www.nytimes3xbfgragh.onion/2020/05/18/business/media/tiktok-ceo-kevin-mayer.html}{brought
in} an American chief executive (the former Disney executive Kevin
Mayer) and
\href{https://www.nytimes3xbfgragh.onion/2020/07/23/business/dealbook/tiktok-bytedance-investors-trump.html}{is
reportedly exploring} selling a majority stake in the company to
American investors.

Jamie Favazza, a TikTok spokeswoman, said in a statement that in
addition to the chief executive, the social network had an American as
its chief information security officer and another as its head of
safety.

``We've tripled the number of employees in the U.S. since the start of
2020,'' she said, ``with plans to hire 10,000 more people over the next
three years in places like Texas, New York and Florida.''

There are legitimate concerns about a Chinese-owned company capturing
the attention and data of millions of Americans --- especially one like
ByteDance, which
\href{https://www.nytimes3xbfgragh.onion/2019/11/05/business/tiktok-china-bytedance.html}{has
a history} of bending the knee to the country's ruling regime. Like all
Chinese tech companies, ByteDance is required to abide by Chinese
censorship laws, and it could be forced to give user data to the Chinese
government under the country's national security law. Lawmakers have
also raised concerns that TikTok could be used to promote pro-China
propaganda to young Americans, or censor politically sensitive content.

Ms. Favazza said TikTok stored American user data in Virginia and
Singapore. She added that the company's content moderation efforts were
led by U.S.-based teams and not influenced by any foreign government,
and that TikTok had not and would not give data to the Chinese
government.

There are also reasons to be skeptical of the motives of TikTok's
biggest critics. Many conservative politicians, including Mr. Trump,
appear to care more about appearing tough on China than preventing
potential harm to TikTok users. And Silicon Valley tech companies like
Facebook, whose executives
\href{https://www.cnbc.com/2019/10/17/facebook-ceo-zuckerberg-calls-out-tiktok-censorship-in-china.html}{have
warned} of the dangers of a Chinese tech takeover, would surely like to
see regulators kneecap one of their major competitors.

I'll be honest: I don't buy the argument that TikTok is an urgent threat
to America's national security. Or, to put it more precisely, I am not
convinced that TikTok is inherently more threatening to Americans than
any other Chinese-owned app that collects data from Americans. If TikTok
is a threat, so are WeChat, Alibaba and League of Legends, the popular
video game, whose maker, Riot Games, is owned by China's Tencent.

And since banning every Chinese-owned tech company from operating in
America wouldn't be possible without erecting our own version of China's
Great Firewall --- a drastic step that would raise concerns about
censorship and authoritarian control --- we need to figure out a way for
Chinese apps and American democracy to coexist.

\includegraphics{https://static01.graylady3jvrrxbe.onion/images/2020/07/24/business/24roose/merlin_174342396_6dddd7a0-f307-40a8-81af-9bdd1120658d-articleLarge.jpg?quality=75\&auto=webp\&disable=upscale}

Here's an idea: Instead of banning TikTok, or forcing ByteDance to sell
it to Americans, why not make an example of it by turning it into the
most transparent, privacy-protecting, ethically governed tech platform
in existence?

As a foreign-owned app, TikTok is, in some ways, easier to regulate than
an American tech platform would be. (One way of regulating it, a
national security review by the Committee on Foreign Investment in the
United States of ByteDance's 2017 acquisition of Musical.ly, TikTok's
predecessor app, is
\href{https://www.reuters.com/article/us-tiktok-cfius-exclusive/exclusive-u-s-opens-national-security-investigation-into-tiktok-sources-idUSKBN1XB4IL}{already
reportedly underway}.) And there is plenty more the U.S. government
could do to ensure that TikTok plays a responsible role in our
information ecosystem without getting rid of it altogether. It could
require the company to open-source key parts of its software, including
the machine-learning algorithms that determine which posts users are
shown. It could pressure TikTok to submit to regular audits of its
data-collection practices, and open up its internal content moderation
guidelines for public comment. As Kevin Xu, the author of
Interconnected, a blog about United States-China relations, points out,
ByteDance could
\href{https://interconnected.blog/can-bytedance-build-trust/}{impose
strict internal controls} to prevent its Chinese employees from
accessing any of TikTok's systems, and open-source those controls so
that outsiders could verify the separation.

Samm Sacks, a cyberpolicy fellow at the centrist think tank New America,
told me that some of the solutions being proposed for TikTok --- such as
selling itself to American investors --- wouldn't address the core
problems. An American-owned TikTok could still legally sell data to
third-party data brokers, for example, which could then feed it back to
the Chinese authorities.

Instead, Ms. Sacks said, the American government should enact a strong
federal privacy law that could protect TikTok users' data without
banning the app altogether.

``Let's solve for the problems at hand,'' she said. ``If the concern is
data security, the best way to secure the data is to put TikTok under
the microscope, and put in place really robust and enforceable rules
about how they're using and retaining data.''

Forcing TikTok to operate in a radically transparent way would go a long
way toward assuaging Americans' fears. And it could become a test case
for a new model of tech regulation that could improve the accountability
and responsibility of not just Chinese-owned tech companies but American
ones, too.

At its core, a lot of the TikTok fear factor comes down to a lack of
information. In March, TikTok
\href{https://newsroom.tiktok.com/en-us/tiktok-to-launch-transparency-center-for-moderation-and-data-practices}{announced}
that it would open ``transparency centers'' where independent auditors
could examine its content moderation practices. The company has also
begun
\href{https://newsroom.tiktok.com/en-us/our-h-2-2019-transparency-report}{releasing}
``transparency reports,'' similar to those issued by Facebook and
Twitter, outlining the various takedown requests it gets from
governments around the world.

But we still don't know how TikTok's algorithms are programmed, or why
they're showing which videos to which users. We don't know how it's
using the data it's collecting, or how it makes and enforces its rules.
We should know these things --- not just about TikTok, but about
American social media apps, too.

After all, Facebook, Instagram, YouTube, Twitter and Snapchat are
playing a huge role in the lives of millions of Americans, and for
years, they have operated with a degree of secrecy that few other
companies of their importance have been allowed. What little we
understand about these platforms' inner workings is often learned years
after the fact, gleaned from insider leaks or repentant former
employees.

Image

Alex Stamos, former chief security officer at Facebook, sees an
opportunity to make TikTok ``a thoughtful model of how to regulate
companies that operate in both the U.S. and China.''Credit...Steve
Marcus/Reuters

Some experts see TikTok's current predicament as a chance to change
that.

``I think TikTok is a bit of a red herring,'' Alex Stamos, Facebook's
former chief security officer and a professor at Stanford University,
told me in an interview. Ultimately, Mr. Stamos said, the question of
what to do about TikTok is secondary to the question of how
multinational tech giants in general should be treated.

``This is a chance to come up with a thoughtful model of how to regulate
companies that operate in both the U.S. and China, no matter their
ownership,'' he said.

The debate over TikTok's fate, in other words, should really be a debate
about how all of the big tech companies that entertain, inform and
influence billions of people should operate, and what should be required
of them, whether they're based in China or Copenhagen or California.

If we can figure out how to handle TikTok --- an app with a
\href{https://www.nytimes3xbfgragh.onion/2020/07/10/style/tiktok-ban-us-users-influencers-taylor-lorenz.html}{genuinely
creative culture}, and millions of American young people who love it ---
we'll have done a lot more than preserving a world-class time-waster.
We'll have figured out a model for getting big tech platforms under
control, after years of letting them run amok.

Advertisement

\protect\hyperlink{after-bottom}{Continue reading the main story}

\hypertarget{site-index}{%
\subsection{Site Index}\label{site-index}}

\hypertarget{site-information-navigation}{%
\subsection{Site Information
Navigation}\label{site-information-navigation}}

\begin{itemize}
\tightlist
\item
  \href{https://help.nytimes3xbfgragh.onion/hc/en-us/articles/115014792127-Copyright-notice}{©~2020~The
  New York Times Company}
\end{itemize}

\begin{itemize}
\tightlist
\item
  \href{https://www.nytco.com/}{NYTCo}
\item
  \href{https://help.nytimes3xbfgragh.onion/hc/en-us/articles/115015385887-Contact-Us}{Contact
  Us}
\item
  \href{https://www.nytco.com/careers/}{Work with us}
\item
  \href{https://nytmediakit.com/}{Advertise}
\item
  \href{http://www.tbrandstudio.com/}{T Brand Studio}
\item
  \href{https://www.nytimes3xbfgragh.onion/privacy/cookie-policy\#how-do-i-manage-trackers}{Your
  Ad Choices}
\item
  \href{https://www.nytimes3xbfgragh.onion/privacy}{Privacy}
\item
  \href{https://help.nytimes3xbfgragh.onion/hc/en-us/articles/115014893428-Terms-of-service}{Terms
  of Service}
\item
  \href{https://help.nytimes3xbfgragh.onion/hc/en-us/articles/115014893968-Terms-of-sale}{Terms
  of Sale}
\item
  \href{https://spiderbites.nytimes3xbfgragh.onion}{Site Map}
\item
  \href{https://help.nytimes3xbfgragh.onion/hc/en-us}{Help}
\item
  \href{https://www.nytimes3xbfgragh.onion/subscription?campaignId=37WXW}{Subscriptions}
\end{itemize}
