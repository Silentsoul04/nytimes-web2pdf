Sections

SEARCH

\protect\hyperlink{site-content}{Skip to
content}\protect\hyperlink{site-index}{Skip to site index}

\href{/section/us}{U.S.}\textbar{}Cities in Bind as Turmoil Spreads Far
Beyond Portland

\url{https://nyti.ms/2WXDV9H}

\begin{itemize}
\item
\item
\item
\item
\item
\item
\end{itemize}

\href{https://www.nytimes3xbfgragh.onion/news-event/george-floyd-protests-minneapolis-new-york-los-angeles?action=click\&pgtype=Article\&state=default\&region=TOP_BANNER\&context=storylines_menu}{Race
and America}

\begin{itemize}
\tightlist
\item
  \href{https://www.nytimes3xbfgragh.onion/2020/07/26/us/protests-portland-seattle-trump.html?action=click\&pgtype=Article\&state=default\&region=TOP_BANNER\&context=storylines_menu}{Protesters
  Return to Other Cities}
\item
  \href{https://www.nytimes3xbfgragh.onion/2020/07/24/us/portland-oregon-protests-white-race.html?action=click\&pgtype=Article\&state=default\&region=TOP_BANNER\&context=storylines_menu}{Portland
  at the Center}
\item
  \href{https://www.nytimes3xbfgragh.onion/2020/07/23/podcasts/the-daily/portland-protests.html?action=click\&pgtype=Article\&state=default\&region=TOP_BANNER\&context=storylines_menu}{Podcast:
  Showdown in Portland}
\item
  \href{https://www.nytimes3xbfgragh.onion/interactive/2020/07/16/us/black-lives-matter-protests-louisville-breonna-taylor.html?action=click\&pgtype=Article\&state=default\&region=TOP_BANNER\&context=storylines_menu}{45
  Days in Louisville}
\end{itemize}

\includegraphics{https://static01.graylady3jvrrxbe.onion/images/2020/07/26/us/26PROTESTS-7/26PROTESTS-7-articleLarge.jpg?quality=75\&auto=webp\&disable=upscale}

\hypertarget{cities-in-bind-as-turmoil-spreads-far-beyond-portland}{%
\section{Cities in Bind as Turmoil Spreads Far Beyond
Portland}\label{cities-in-bind-as-turmoil-spreads-far-beyond-portland}}

Galvanized in part by the deployment of federal agents in Portland,
Ore., protesters have returned to the streets in Oakland, Seattle and
elsewhere.

Police attempt to arrest protesters in Seattle on
Saturday.Credit...Grant Hindsley for The New York Times

Supported by

\protect\hyperlink{after-sponsor}{Continue reading the main story}

\href{https://www.nytimes3xbfgragh.onion/by/mike-baker}{\includegraphics{https://static01.graylady3jvrrxbe.onion/images/2020/05/19/reader-center/author-mike-baker/author-mike-baker-thumbLarge.png}}\href{https://www.nytimes3xbfgragh.onion/by/thomas-fuller}{\includegraphics{https://static01.graylady3jvrrxbe.onion/images/2018/06/12/multimedia/author-thomas-fuller/author-thomas-fuller-thumbLarge.png}}\href{https://www.nytimes3xbfgragh.onion/by/shane-goldmacher}{\includegraphics{https://static01.graylady3jvrrxbe.onion/images/2018/07/27/multimedia/author-shane-goldmacher/author-shane-goldmacher-thumbLarge.png}}

By \href{https://www.nytimes3xbfgragh.onion/by/mike-baker}{Mike Baker},
\href{https://www.nytimes3xbfgragh.onion/by/thomas-fuller}{Thomas
Fuller} and
\href{https://www.nytimes3xbfgragh.onion/by/shane-goldmacher}{Shane
Goldmacher}

\begin{itemize}
\item
  Published July 26, 2020Updated July 29, 2020
\item
  \begin{itemize}
  \item
  \item
  \item
  \item
  \item
  \item
  \end{itemize}
\end{itemize}

SEATTLE --- A series of strident new protests over police misconduct
rattled cities across the country over the weekend, creating a new
dilemma for state and local leaders who had succeeded in easing some of
the turbulence in their streets until a showdown over the use of federal
agents in Oregon stirred fresh outrage.

With some demonstrators embracing destructive protest methods and police
often using aggressive tactics to subdue both them and others who are
demonstrating peacefully,
\href{https://www.nytimes3xbfgragh.onion/2020/07/25/us/protests-seattle-portland.html}{the
scenes on Saturday night} in places like Seattle, Oakland, Calif., and
Los Angeles recalled the volatile early days of the protests after the
death of George Floyd at the end of May.

The latest catalyst was the deployment of federal law enforcement agents
in
\href{https://www.nytimes3xbfgragh.onion/2020/07/28/us/portland-protests-fact-check.html}{Portland},
Ore., whose
\href{https://www.nytimes3xbfgragh.onion/2020/07/17/us/portland-protests.html}{militarized
efforts to subdue protests} around the federal courthouse have sparked
mass demonstrations and nightly clashes there. They have also inspired
new protests of solidarity in other cities, where people have expressed
deep concern about the federal government
\href{https://www.nytimes3xbfgragh.onion/2020/07/25/us/portland-federal-legal-jurisdiction-courts.html}{exercising
such extensive authority} in a city that has made it clear it opposes
the presence of federal agents.

\includegraphics{https://static01.graylady3jvrrxbe.onion/images/2020/07/22/autossell/Portland-Still_01/Portland-Still_01-videoSixteenByNineJumbo1600-v4.jpg}

President Trump has seized on the scenes of national unrest --- statues
toppled and windows smashed --- to build a law-and-order message for his
re-election campaign, spending more than \$26 million on television ads
depicting a lawless dystopia of
\href{https://www.youtube.com/watch?v=moZOrq0qL3Q}{empty police stations
and 911 answering services} that he argues might be left in a nation
headed by his Democratic rival, Joseph R. Biden Jr.

Mr. Biden insisted last week that the president's pledge to inject a
federal law-and-order presence into the already volatile issue of
policing shows that he is ``determined to sow chaos and division. To
make matters worse instead of better.''

The situation has left city leaders, now watching the backlash unfold on
their streets, outraged and caught in the middle. Mayor Jenny Durkan of
Seattle said in an interview Sunday that the city is in the middle of a
self-fulfilling prophecy, with protesters infuriated by the federal
presence in Portland smashing windows and setting fires, the very images
of ``anarchy'' that the president has warned about.

Image

Protesters demonstrate against the presence of federal agents outside
the Mark O. Hatfield Courthouse in Portland on Saturday.Credit...Octavio
Jones for The New York Times

Image

Federal agents work to clear protesters in downtown Portland on
Saturday.Credit...Octavio Jones for The New York Times

``There is no question that the actions in Portland have escalated
things, not just in Seattle, but nationwide,'' Ms. Durkan said.

At the same time, a new round of street unrest could intensify
differences among local officials over how best to address the
complaints of demonstrators and respond to vandalism and violence.

In June, amid a wide-ranging police reform movement, the Seattle City
Council banned the use of tear gas and other crowd control tactics,
including pepper spray. The police chief has objected, and the United
States Department of Justice intervened with a lawsuit, winning a
temporary restraining order on Friday blocking implementation of that
ban. Over the weekend, Seattle officers used pepper spray and
flash-grenades to disperse protesters.

Over the weekend, dozens of people were arrested in Seattle. Protesters
in Los Angeles clashed with officers in front of the city's federal
courthouse downtown. Police also made arrests at protests in smaller
cities, such as Omaha, Neb., and Richmond, Va.

In Oakland, what had been a peaceful protest led in part by a group of
mothers proclaiming ``Cops And Feds Off Our Streets'' devolved after
dark as another set of protesters smashed windows at the county
courthouse and lit a fire inside.

\href{https://www.nytimes3xbfgragh.onion/2020/07/26/us/austin-shooting-texas-protests.html}{An
armed protester was shot and killed} in Austin, Texas, by a motorist
whose car, according to witnesses and the police, had been aimed toward
a group of demonstrators also protesting the federal presence in
Portland.

The protests continued in Portland on Sunday as the city neared its 60th
straight day of demonstrations. As crowds gathered once again near the
federal courthouse, the Portland Police Bureau said that a shooting had
occurred about a block away. Officers took two people into custody, and
one person showed up at a hospital with an apparent gunshot wound that
was not life-threatening, the bureau said in a statement.

In Seattle on Sunday night, hundreds of protesters returned to the area
around a police station in the Capitol Hill neighborhood. The gathering
remained peaceful.

Some cities had welcomed Mr. Trump's offer to send additional federal
law enforcement agents in to help combat escalating gang violence and
drug crime, but insisted they would brook no federal agents on their
streets arresting and tear gassing protesters.

\includegraphics{https://static01.graylady3jvrrxbe.onion/images/2020/07/26/us/26PROTESTS-4/merlin_174955527_d825d543-9316-449d-afe8-794a45efa9fb-articleLarge.jpg?quality=75\&auto=webp\&disable=upscale}

Democratic city and state leaders pushed back against the new federal
presence, but also expressed frustration that some on the streets were
going too far and playing into the president's gambit.

``I'm furious that Oakland may have played right into Donald Trump's
twisted campaign strategy,'' Oakland's mayor, Libby Schaaf, said in an
interview on Sunday. ``Images of a vandalized downtown is exactly what
he wants to whip up his base and to potentially justify sending in
federal troops that will only incite more unrest.''

In Chicago, Mayor Lori Lightfoot, a Democrat who has repeatedly clashed
with Mr. Trump, had said she welcomed the president's intervention on
enacting gun control and investing in community programs.

``Any other form of militarized assistance within our borders that would
not be within our control or within the direct command of the Chicago
Police Department would spell disaster,'' she said in a letter to the
president last week.

Gov. Michelle Lujan Grisham of New Mexico on Sunday called the
administration's decision to send law enforcement agents to the state
--- a measure the administration has sought to distinguish from the
agents sent to guard federal property in Portland --- as ``a bit
suspect.''

``They have not provided the federal funding that was promised to
Albuquerque for police and crime interventions,'' she said in an
appearance on ABC's ``This Week.'' She said that the state would work
with federal agents if they folded into existing efforts to address
violent crime. But, she said, ``If we're going to incentivize unrest,
then that's something altogether different.''

Portland has been the epicenter of the most recent protests. After the
initial mass demonstrations in the aftermath of Mr. Floyd's death,
protests in the city continued each night, although in smaller numbers.
Police have said there was persistent vandalism, people pointing lasers
at law enforcement agents, and protesters who threw objects such as
commercial-grade fireworks at officers, including those protecting the
federal courthouse.

Mr. Trump's campaign has sought to capitalize on the unrest to reassure
voters that he will bring an end to the turbulence. ``If there is a
danger for Democrats generically, it is if the Republicans are able to
define them as being on the side of the anarchists in Portland,'' said
Scott Jennings, a veteran Republican strategist. But he said that Mr.
Trump's heated and broad-brush rhetoric has made the Republican cause
harder. ``The bottom line is it's a situation that requires nuance and
it's a presidency that has not engaged in a lot of nuance.''

Democratic strategists and Biden officials expressed confidence that Mr.
Trump's attacks posed little immediate political risk even as street
protests escalated. For one, they said, the president's warning of a
dark Democratic-run future is in stark dissonance with the reality that
the unrest is happening under his own administration. They said the
police issue was being treated by many voters as a distraction by Mr.
Trump from his faltering coronavirus pandemic response and the
struggling economy.

``No matter how many troops Donald Trump sends into American cities,
it's not going to distract them from their primary concern, which is the
coronavirus and their health,'' said Jared Leopold, a Democratic
strategist.

For city officials, the challenge is more immediate than the November
election --- it is bringing an end to nights of clashes on their
streets.

Mr. Floyd's death in Minneapolis at the end of May drew out millions in
an unusual show, across a broad range of race and class, of support for
racial justice.

City leaders had managed to calm the most intense protests in most
cities after soul-searching debates at the community level over funding
for the police and the use of tear gas on protesters. Crowds have waned
in most parts of the country, though pockets of resistance have
continued.

Image

The ``Wall of Moms'' gets tear gassed in Portland on
Friday.Credit...Mason Trinca for The New York Times

That includes Seattle, where protesters in June laid claim to several
city blocks, pressuring the police to temporarily abandon a police
station in the city's Capitol Hill neighborhood and establishing a free
protest zone that endured for several weeks. When the city authorities
finally reclaimed the zone, there was little resistance.

And things have continued to calm. The Seattle City Council began
embracing plans to make substantial cuts to the police department
budget, restrict the use of tear gas and put the city on a path to
reimagining policing.

Mayor Durkan and the council have at times disagreed on how far to go on
issues such as cuts to the police department's budget. She has proposed
trimming just under 20 percent, in part by moving duties to other
agencies, while council members have endorsed cutting the department
budget in half.

Protests in Portland had continued, with Mayor Ted Wheeler decrying
those who continued to use destructive tactics and business urging a
return to a less disruptive life downtown.

But any possibility of that happening soon appeared to be turned on its
head earlier this month when federal officers deployed to the city. Mr.
Wheeler has objected, asking them to leave because their presence seemed
to only reignite tensions.

Image

Protesters moved past a police precinct in Seattle on
Saturday.Credit...Grant Hindsley for The New York Times

Protester numbers swelled into the thousands, with nightly clashes. On
Sunday morning, protesters knocked down a fence surrounding the
courthouse and federal agents deployed tear gas to disperse them. The
Portland Police Bureau also got involved, declaring a riot.

The federal presence in Portland has attracted new attendees to the
protests, many of whom contended that the influx of federal forces was
inappropriate and potentially illegal, as the state Attorney General
argued in a lawsuit.

The focus on the federal agents in Portland has frustrated some
activists who see the pushback against their presence as a distraction
from the racial injustices that had been the focus of protests in May
and June.

In Portland on Saturday night, as protesters marched to a hotel where
they believed federal agents could be staying, some participants urged
the marchers not to forget earlier protests against local police.

``It's complicated, it's chaotic, and it's a little hard for us to stay
focused. We need to stay focused. We cannot forget this is also about
the Portland Police Bureau,'' Kinsey Smyth told the crowd. ``This is not
about destruction, this is about rebuilding.''

Mike Baker reported from Seattle, Thomas Fuller from Oakland, Calif.,
and Shane Goldmacher from New York. Reporting was contributed by Kate
Conger, from Portland, Ore., Julie Bosman from Chicago, Rebecca Halleck,
JMaggie Haberman and Nick Corasaniti from New York and Melina Delkic
from Beacon, N.Y.

Advertisement

\protect\hyperlink{after-bottom}{Continue reading the main story}

\hypertarget{site-index}{%
\subsection{Site Index}\label{site-index}}

\hypertarget{site-information-navigation}{%
\subsection{Site Information
Navigation}\label{site-information-navigation}}

\begin{itemize}
\tightlist
\item
  \href{https://help.nytimes3xbfgragh.onion/hc/en-us/articles/115014792127-Copyright-notice}{©~2020~The
  New York Times Company}
\end{itemize}

\begin{itemize}
\tightlist
\item
  \href{https://www.nytco.com/}{NYTCo}
\item
  \href{https://help.nytimes3xbfgragh.onion/hc/en-us/articles/115015385887-Contact-Us}{Contact
  Us}
\item
  \href{https://www.nytco.com/careers/}{Work with us}
\item
  \href{https://nytmediakit.com/}{Advertise}
\item
  \href{http://www.tbrandstudio.com/}{T Brand Studio}
\item
  \href{https://www.nytimes3xbfgragh.onion/privacy/cookie-policy\#how-do-i-manage-trackers}{Your
  Ad Choices}
\item
  \href{https://www.nytimes3xbfgragh.onion/privacy}{Privacy}
\item
  \href{https://help.nytimes3xbfgragh.onion/hc/en-us/articles/115014893428-Terms-of-service}{Terms
  of Service}
\item
  \href{https://help.nytimes3xbfgragh.onion/hc/en-us/articles/115014893968-Terms-of-sale}{Terms
  of Sale}
\item
  \href{https://spiderbites.nytimes3xbfgragh.onion}{Site Map}
\item
  \href{https://help.nytimes3xbfgragh.onion/hc/en-us}{Help}
\item
  \href{https://www.nytimes3xbfgragh.onion/subscription?campaignId=37WXW}{Subscriptions}
\end{itemize}
