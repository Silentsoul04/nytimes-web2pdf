Sections

SEARCH

\protect\hyperlink{site-content}{Skip to
content}\protect\hyperlink{site-index}{Skip to site index}

\href{https://www.nytimes3xbfgragh.onion/section/business}{Business}

\href{https://myaccount.nytimes3xbfgragh.onion/auth/login?response_type=cookie\&client_id=vi}{}

\href{https://www.nytimes3xbfgragh.onion/section/todayspaper}{Today's
Paper}

\href{/section/business}{Business}\textbar{}They Crossed Oceans to Lift
Their Families Out of Poverty. Now, They Need Help.

\url{https://nyti.ms/2Efeytk}

\begin{itemize}
\item
\item
\item
\item
\item
\item
\end{itemize}

\href{https://www.nytimes3xbfgragh.onion/news-event/coronavirus?action=click\&pgtype=Article\&state=default\&region=TOP_BANNER\&context=storylines_menu}{The
Coronavirus Outbreak}

\begin{itemize}
\tightlist
\item
  live\href{https://www.nytimes3xbfgragh.onion/2020/08/01/world/coronavirus-covid-19.html?action=click\&pgtype=Article\&state=default\&region=TOP_BANNER\&context=storylines_menu}{Latest
  Updates}
\item
  \href{https://www.nytimes3xbfgragh.onion/interactive/2020/us/coronavirus-us-cases.html?action=click\&pgtype=Article\&state=default\&region=TOP_BANNER\&context=storylines_menu}{Maps
  and Cases}
\item
  \href{https://www.nytimes3xbfgragh.onion/interactive/2020/science/coronavirus-vaccine-tracker.html?action=click\&pgtype=Article\&state=default\&region=TOP_BANNER\&context=storylines_menu}{Vaccine
  Tracker}
\item
  \href{https://www.nytimes3xbfgragh.onion/interactive/2020/07/29/us/schools-reopening-coronavirus.html?action=click\&pgtype=Article\&state=default\&region=TOP_BANNER\&context=storylines_menu}{What
  School May Look Like}
\item
  \href{https://www.nytimes3xbfgragh.onion/live/2020/07/31/business/stock-market-today-coronavirus?action=click\&pgtype=Article\&state=default\&region=TOP_BANNER\&context=storylines_menu}{Economy}
\end{itemize}

Advertisement

\protect\hyperlink{after-top}{Continue reading the main story}

Supported by

\protect\hyperlink{after-sponsor}{Continue reading the main story}

The Great Read

\hypertarget{they-crossed-oceans-to-lift-their-families-out-of-poverty-now-they-need-help}{%
\section{They Crossed Oceans to Lift Their Families Out of Poverty. Now,
They Need
Help.}\label{they-crossed-oceans-to-lift-their-families-out-of-poverty-now-they-need-help}}

As the pandemic destroys paychecks, migrant workers are sending less
money home, threatening an increase in poverty from South Asia and
sub-Saharan Africa to Eastern Europe and Latin America.

\includegraphics{https://static01.graylady3jvrrxbe.onion/images/2020/07/24/business/00Remittances-1/00Remittances-1-articleLarge.jpg?quality=75\&auto=webp\&disable=upscale}

\href{https://www.nytimes3xbfgragh.onion/by/peter-s-goodman}{\includegraphics{https://static01.graylady3jvrrxbe.onion/images/2018/02/16/multimedia/author-peter-s-goodman/author-peter-s-goodman-thumbLarge-v2.png}}

By \href{https://www.nytimes3xbfgragh.onion/by/peter-s-goodman}{Peter S.
Goodman}

\begin{itemize}
\item
  Published July 27, 2020Updated July 28, 2020
\item
  \begin{itemize}
  \item
  \item
  \item
  \item
  \item
  \item
  \end{itemize}
\end{itemize}

\href{https://www.nytimes3xbfgragh.onion/es/2020/07/28/espanol/mundo/remesas-coronavirus.html}{Leer
en español}

LONDON --- For more than a decade, Flavius Tudor has shared the money he
has made in England with his mother in Romania, regularly sending home
cash that enabled her to buy medicine.

Last month, the flow reversed. His 82-year-old mother sent him money so
he could pay his bills.

Suffering a high fever and a persistent cough amid the
\href{https://www.nytimes3xbfgragh.onion/2020/07/28/us/politics/coronavirus-hunger-poverty.html}{coronavirus
pandemic}, Mr. Tudor, 52, could no longer enter the nursing home where
he worked as a caregiver. So his mother reached into her pension, earned
from a lifetime as a librarian in one of Europe's poorest countries, and
sent cash to her son in one of the wealthiest lands on earth.

``It's very tough times,'' he said. ``I'm lost.''

Around the globe, the pandemic has jeopardized a vital artery of finance
supporting hundreds of millions of families --- so-called remittances
sent home from wealthy countries by migrant workers. As the coronavirus
has sent economies into lockdown, sowing joblessness, people accustomed
to taking care of relatives at home have lost their paychecks, forcing
some to depend on those who have depended on them.

Last year, migrant workers sent home a record \$554 billion, more than
three times the amount of development aid dispensed by wealthy
countries, according to the
\href{https://www.worldbank.org/en/news/press-release/2020/04/22/world-bank-predicts-sharpest-decline-of-remittances-in-recent-history}{World
Bank}. But those remittances are likely to plunge by one-fifth this
year, representing the most severe contraction in history.

The drop amounts to a catastrophe, heightening the near-certainty that
the pandemic will produce the first global increase in poverty since the
Asian financial crisis of 1998. Some 40 million to 60 million people are
expected this year to fall into
\href{https://www.worldbank.org/en/topic/poverty/overview}{extreme
poverty}, which the World Bank defines as living on \$1.90 a day or
less.

Diminishing remittances are both an outgrowth of the crisis gripping the
world and a portent of more trouble ahead. Developing countries account
for 60 percent of the world economy on the basis of purchasing power,
according to the International Monetary Fund. Less spending in poorer
nations spells less economic growth for the world.

Like the pandemic that has delivered it, the slide in remittances is
global. Europe and Central Asia are expected to suffer a fall of nearly
28 percent in the wages sent home from other countries, while
sub-Saharan Africa sees a drop of 23 percent. South Asia appears set for
a 22 percent decline, while the Middle East, North Africa, and Latin
America and the Caribbean could absorb a reduction of more than 19
percent.

Overall, the pandemic has damaged the earning power of 164 million
migrant workers who support at least 800 million relatives in less
affluent countries, according to an estimate from the
\href{https://migrationnetwork.un.org/sites/default/files/policy_brief-_remittances_in_the_time_of_covid-19.pdf}{United
Nations Network on Migration}.

``We are talking about a staggering number of people who are benefiting
from these remittances,'' said Dilip Ratha, lead economist on migration
and remittances at the World Bank in Washington.

Venturing overseas for work is laced with danger, exposing migrant
workers to dishonest recruitment agents, exploitative employers, and the
physical perils of manual labor. It is also a singularly effective means
of upward mobility.

\includegraphics{https://static01.graylady3jvrrxbe.onion/images/2020/07/24/business/00Remittances-6/merlin_152944797_91edcdcc-e7ca-429d-ae78-d7c9b8f20b5b-articleLarge.jpg?quality=75\&auto=webp\&disable=upscale}

Households receiving remittances eat better, and are more likely to
continue their children's education rather than pressing them into the
work force. Babies born into homes receiving remittances tend to be
higher in birth weight.

In some countries, migrant workers can tap into unemployment insurance
and other government programs --- especially Eastern Europeans from
European Union nations who have labored in other member states. But in
many countries, migrants operate in gray areas, unprotected by
government relief and especially vulnerable to hard times.

\hypertarget{latest-updates-economy}{%
\section{\texorpdfstring{\href{https://www.nytimes3xbfgragh.onion/live/2020/07/31/business/stock-market-today-coronavirus?action=click\&pgtype=Article\&state=default\&region=MAIN_CONTENT_1\&context=storylines_live_updates}{Latest
Updates:
Economy}}{Latest Updates: Economy}}\label{latest-updates-economy}}

\href{https://www.nytimes3xbfgragh.onion/live/2020/07/31/business/stock-market-today-coronavirus?action=click\&pgtype=Article\&state=default\&region=MAIN_CONTENT_1\&context=storylines_live_updates\#kodaks-chief-executive-was-given-stock-options-then-the-share-price-spiked-1000-percent}{20h
ago}

\href{https://www.nytimes3xbfgragh.onion/live/2020/07/31/business/stock-market-today-coronavirus?action=click\&pgtype=Article\&state=default\&region=MAIN_CONTENT_1\&context=storylines_live_updates\#kodaks-chief-executive-was-given-stock-options-then-the-share-price-spiked-1000-percent}{Kodak's
chief executive was given stock options. Then the share price spiked
1,000 percent.}

\href{https://www.nytimes3xbfgragh.onion/live/2020/07/31/business/stock-market-today-coronavirus?action=click\&pgtype=Article\&state=default\&region=MAIN_CONTENT_1\&context=storylines_live_updates\#fitch-ratings-downgrades-its-outlook-on-us-debt}{23h
ago}

\href{https://www.nytimes3xbfgragh.onion/live/2020/07/31/business/stock-market-today-coronavirus?action=click\&pgtype=Article\&state=default\&region=MAIN_CONTENT_1\&context=storylines_live_updates\#fitch-ratings-downgrades-its-outlook-on-us-debt}{Fitch
Ratings downgrades its outlook on U.S. debt.}

\href{https://www.nytimes3xbfgragh.onion/live/2020/07/31/business/stock-market-today-coronavirus?action=click\&pgtype=Article\&state=default\&region=MAIN_CONTENT_1\&context=storylines_live_updates\#us-sanctions-more-chinese-officials-over-human-rights-violations-as-tensions-flare}{29h
ago}

\href{https://www.nytimes3xbfgragh.onion/live/2020/07/31/business/stock-market-today-coronavirus?action=click\&pgtype=Article\&state=default\&region=MAIN_CONTENT_1\&context=storylines_live_updates\#us-sanctions-more-chinese-officials-over-human-rights-violations-as-tensions-flare}{U.S.
sanctions more Chinese officials over human rights violations as
tensions flare}

\href{https://www.nytimes3xbfgragh.onion/live/2020/07/31/business/stock-market-today-coronavirus?action=click\&pgtype=Article\&state=default\&region=MAIN_CONTENT_1\&context=storylines_live_updates}{See
more updates}

More live coverage:
\href{https://www.nytimes3xbfgragh.onion/2020/08/01/world/coronavirus-covid-19.html?action=click\&pgtype=Article\&state=default\&region=MAIN_CONTENT_1\&context=storylines_live_updates}{Global}

``Some people, either naïvely or with good intentions, say this Covid-19
democratizes us all, and we are all exposed to it equally,'' said
Mahmoud Mohieldin, an Egyptian economist who serves as a United Nations
special envoy on financing sustainable development. ``This is not true.
The impacts are very much disproportionate.''

For families in poor countries, sending a relative abroad to earn money
tends to be a collective undertaking. People pool their cash to finance
journeys in what amounts to the largest investment they will make in
their lives.

The pandemic has turned such ventures into disasters.

Three years ago, Mahammed Heron left his village outside Dhaka,
Bangladesh, for work in the energy-rich nation of Qatar, tracing a route
pursued by tens of millions of South Asian migrants.

He borrowed 400,000 Bangladeshi taka (about \$4,700) from relatives and
engaged a local recruitment agent that bought him a plane ticket,
secured a work visa and promised him a job. This was a monumental amount
of money in Bangladesh, more than twice the national income per capita
(about \$1,855). His wife, Monowara Begum, was terrified. Her first
husband --- Mr. Heron's older brother --- had been killed by a drunken
driver more than a decade earlier in Saudi Arabia, where he had been
working as a hospital janitor.

Image

Monowara Begum lives in a shack in a village outside Dhaka, Bangladesh,
that is made of corrugated aluminum and vulnerable to the torrential
rains of the monsoon. There is no running water.Credit...Salahuddin
Ahmed for The New York Times

But if the prospect of her husband venturing to the Persian Gulf was
frightening, staying put seemed riskier still.

Her family lived in a shack made of corrugated aluminum that was
vulnerable to the torrential rains of the monsoon. They had no running
water. Mr. Heron earned perhaps 300 taka (about \$3.50) per day working
in the surrounding rice paddies. They could rarely afford meat or fish,
subsisting on rice and potatoes. Her oldest son had a heart condition
that required medicine.

The only way out of poverty was to invest in her children's education,
but tuition payments reached 6,000 taka (more than \$70) per year.

``Our financial situation was never good,'' Ms. Begum explained in an
interview via a video link, as birds chirped loudly in the village. She
reluctantly agreed to the plan.

Image

Ms. Begum said the prospect of her husband venturing to the Persian Gulf
was frightening, but staying put seemed riskier still. ``Our financial
situation was never good.''Credit...Salahuddin Ahmed for The New York
Times

When Mr. Heron landed in Doha in September 2018, the furnace-like heat
was not the only shock: The recruitment agency had failed to line up a
job. ``I was cheated,'' he said in an interview by video.

He looked frantically for work, eventually securing a position at a
staffing agency that sent him on a variety of assignments --- cleaning
offices, landscaping and digging into the sandy earth to lay fiber
optics cable.

Mr. Heron was paid a monthly salary of 900 Qatari rial (about \$250) and
assigned a bunk inside a dormitory room he shared with 15 other men, all
Bangladeshis.

Every two or three months, he sent home about 30,000 taka (about \$350),
but it all went toward his debt --- still only one-fourth repaid.

Then, in May, with the coronavirus shutting down much of life in Doha,
the agency stopped paying the workers, Mr. Heron said. He suffered an
asthma flare-up that required hospitalization, absorbing all his cash.
He stopped sending money home.

For Bangladesh overall, remittances received from other countries
plunged by 23 percent in April compared with a year earlier, and were
down by 13 percent in May, according to the
\href{https://www.bb.org.bd/econdata/wageremitance.php}{nation's central
bank}, though June saw an increase.

Schools remain shut in Bangladesh, but whenever they open, Ms. Begum
sees no way to afford sending her 16-year-old son, Hasan.

Image

Ms. Begum with her daughter and sons. Schools remain shut in Bangladesh,
but whenever they open, Ms. Begum sees no way to afford sending her
oldest son, Hasan.Credit...Salahuddin Ahmed for The New York Times

She has been urging Hasan to find work --- perhaps in construction,
maybe at an auto repair shop. He has been resisting, preferring to stay
at home and read textbooks.

``I want to continue my studies,'' he said. He imagines a life as a
software engineer. His face lights up as he describes this --- a slender
teenager, standing shirtless in front of his shack as roosters crow,
envisioning himself in a shiny office, leaning over a computer.

Every few days, he and his mother use a smartphone app and a prepaid
internet card to talk to Mr. Heron, stranded in the dormitory in Qatar.
He is too ill to work, he said, but lacks money to fly home. After
another year, the staffing company is contractually obligated to pay for
his return flight. He bides his time, hoping his health improves, hoping
his pay resumes, hoping his own children escape his fate.

``I dream that my sons will do something in their life,'' he said.

Image

Migrant workers playing cricket on a patch of wasteland in Doha last
year. Mahammed Heron, who arrived there in 2018, is now too ill to work,
he said, but lacks money to fly home to Bangladesh.Credit...Petr David
Josek/Associated Press

In the town of Patzún, Guatemala, Edgar Tzirin's family used the money
he made in his job as a prep cook at a New York soup and sandwich
restaurant to erect a new house. Mr. Tzirin earned about \$2,000 a
month. Every two weeks, he dutifully sent home \$500 to \$700.

This money proved vital when the coronavirus threw his three sisters out
of work. When his mother landed in the hospital --- maybe with the
coronavirus --- he paid for her care.

But in April, with New York in lockdown, Mr. Tzirin lost his job. When
his grandfather died the following month, he was unable to send money
home for the funeral --- a deep wound. He used to speak to his family
every two to three days, but he can no longer bear it, receding into
isolation and loneliness. He has not told them that he lost his job.

``My family needs me,'' he said.

Mr. Tzirin gets up at 5:30 every morning and goes out looking for
construction work or odd jobs as a day laborer but usually returns home
empty-handed. ``There's nothing,'' he said.

He is three months behind on his rent. He contemplates returning to
Guatemala for the first time in a decade, but what can he do there?

``It's a hard experience,'' Mr. Tzirin said. ``People are getting
desperate.''

Many migrant workers are now contending with two emergencies at once ---
a loss of income combined with the menace of the virus itself.

Mr. Tudor, the Romanian immigrant living in Britain, left his home
region of Transylvania when he was in his early 20s. Abandoning a
perilous life as a coal miner, he landed first in Spain, where he worked
in security. As the global financial crisis plunged the country into a
veritable depression in 2009, he moved to Britain, settling in
Weston-super-Mare, a seaside town of 76,000 people, about 150 miles west
of London.

He took care of older people through stints arranged by staffing
companies. His most recent job was at a for-profit nursing home called
The Heathers. He was making 848 pounds (about \$1,070) a week. His wife
was cleaning rooms at a hotel, bringing home £1,200 (\$1,536) a month.

As the coronavirus emerged, his wife saw her hours reduced. Hospitals
began shifting older patients stricken with the virus to nursing homes.

Image

Edgar Tzirin's family used the money he made in his job as a prep cook
in New York to build a house in Guatemala. But in April, he lost his job
and is now three months behind on his rent.Credit...Justin French for
The New York Times

According to Mr. Tudor, by early March, 23 of the 30 rooms at The
Heathers were full of coronavirus patients. Within a week, nine were
dead, he said. He and his colleagues were supplied only with disposable
surgical masks. One colleague demanded more protective gear and was
fired. He said during his last week at the facility, the manager placed
a woman with dementia who did not have the virus in the same room as
someone who did.

``It was horrible,'' Mr. Tudor said. ``It's only about business. It's
about money.''

Reached by phone, a part-owner of The Heathers, Bipin Patel, declined to
answer questions. ``We're not making any comments,'' he said.

Mr. Tudor soon came down with a fever and a cough, forcing him to stop
going to work. He twice tested negative for the coronavirus, but has
been unable to secure another job.

In recent years, Britain has
\href{https://www.nytimes3xbfgragh.onion/2018/05/28/world/europe/uk-austerity-poverty.html}{sharply
reduced government support programs} for the jobless and those
struggling to pay their bills, folding them into a lump sum scheme known
as universal credit.

Mr. Flavius has traded his paycheck for a £1,000 (\$1,280) monthly
universal credit payment, cutting his income roughly in half. His
eyeglasses have broken, but he can't afford to replace them. When the
rent came due last month, he paid it only with the help of his mother,
back in Romania.

``The world doesn't know where it's going,'' he said. ``No society can
handle this situation.''

Nic Wirtz contributed reporting from Antigua, Guatemala, Hari Kumar from
New Delhi and Geneva Abdul contributed research from London.

Advertisement

\protect\hyperlink{after-bottom}{Continue reading the main story}

\hypertarget{site-index}{%
\subsection{Site Index}\label{site-index}}

\hypertarget{site-information-navigation}{%
\subsection{Site Information
Navigation}\label{site-information-navigation}}

\begin{itemize}
\tightlist
\item
  \href{https://help.nytimes3xbfgragh.onion/hc/en-us/articles/115014792127-Copyright-notice}{©~2020~The
  New York Times Company}
\end{itemize}

\begin{itemize}
\tightlist
\item
  \href{https://www.nytco.com/}{NYTCo}
\item
  \href{https://help.nytimes3xbfgragh.onion/hc/en-us/articles/115015385887-Contact-Us}{Contact
  Us}
\item
  \href{https://www.nytco.com/careers/}{Work with us}
\item
  \href{https://nytmediakit.com/}{Advertise}
\item
  \href{http://www.tbrandstudio.com/}{T Brand Studio}
\item
  \href{https://www.nytimes3xbfgragh.onion/privacy/cookie-policy\#how-do-i-manage-trackers}{Your
  Ad Choices}
\item
  \href{https://www.nytimes3xbfgragh.onion/privacy}{Privacy}
\item
  \href{https://help.nytimes3xbfgragh.onion/hc/en-us/articles/115014893428-Terms-of-service}{Terms
  of Service}
\item
  \href{https://help.nytimes3xbfgragh.onion/hc/en-us/articles/115014893968-Terms-of-sale}{Terms
  of Sale}
\item
  \href{https://spiderbites.nytimes3xbfgragh.onion}{Site Map}
\item
  \href{https://help.nytimes3xbfgragh.onion/hc/en-us}{Help}
\item
  \href{https://www.nytimes3xbfgragh.onion/subscription?campaignId=37WXW}{Subscriptions}
\end{itemize}
