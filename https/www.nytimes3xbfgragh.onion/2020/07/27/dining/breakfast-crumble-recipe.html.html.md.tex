Sections

SEARCH

\protect\hyperlink{site-content}{Skip to
content}\protect\hyperlink{site-index}{Skip to site index}

\href{https://www.nytimes3xbfgragh.onion/section/food}{Food}

\href{https://myaccount.nytimes3xbfgragh.onion/auth/login?response_type=cookie\&client_id=vi}{}

\href{https://www.nytimes3xbfgragh.onion/section/todayspaper}{Today's
Paper}

\href{/section/food}{Food}\textbar{}A Breakfast Crumble for Early Birds
With a Sweet Tooth

\url{https://nyti.ms/305SIRq}

\begin{itemize}
\item
\item
\item
\item
\item
\item
\end{itemize}

\href{https://www.nytimes3xbfgragh.onion/spotlight/at-home?action=click\&pgtype=Article\&state=default\&region=TOP_BANNER\&context=at_home_menu}{At
Home}

\begin{itemize}
\tightlist
\item
  \href{https://www.nytimes3xbfgragh.onion/2020/07/28/books/time-for-a-literary-road-trip.html?action=click\&pgtype=Article\&state=default\&region=TOP_BANNER\&context=at_home_menu}{Take:
  A Literary Road Trip}
\item
  \href{https://www.nytimes3xbfgragh.onion/2020/07/29/magazine/bored-with-your-home-cooking-some-smoky-eggplant-will-fix-that.html?action=click\&pgtype=Article\&state=default\&region=TOP_BANNER\&context=at_home_menu}{Cook:
  Smoky Eggplant}
\item
  \href{https://www.nytimes3xbfgragh.onion/2020/07/27/travel/moose-michigan-isle-royale.html?action=click\&pgtype=Article\&state=default\&region=TOP_BANNER\&context=at_home_menu}{Look
  Out: For Moose}
\item
  \href{https://www.nytimes3xbfgragh.onion/interactive/2020/at-home/even-more-reporters-editors-diaries-lists-recommendations.html?action=click\&pgtype=Article\&state=default\&region=TOP_BANNER\&context=at_home_menu}{Explore:
  Reporters' Obsessions}
\end{itemize}

Advertisement

\protect\hyperlink{after-top}{Continue reading the main story}

Supported by

\protect\hyperlink{after-sponsor}{Continue reading the main story}

\hypertarget{a-breakfast-crumble-for-early-birds-with-a-sweet-tooth}{%
\section{A Breakfast Crumble for Early Birds With a Sweet
Tooth}\label{a-breakfast-crumble-for-early-birds-with-a-sweet-tooth}}

What started out as French toast, a morning favorite at Yotam
Ottolenghi's house, evolved into this warm, fruity treat.

\includegraphics{https://static01.graylady3jvrrxbe.onion/images/2020/07/29/dining/27Otto1/27Otto1-articleLarge.jpg?quality=75\&auto=webp\&disable=upscale}

\href{https://www.nytimes3xbfgragh.onion/by/yotam-ottolenghi}{\includegraphics{https://static01.graylady3jvrrxbe.onion/images/2019/05/23/reader-center/author-yotam-ottolenghi/5e47d0c5641b4785a3407fabbb491073-thumbLarge.png}}

By \href{https://www.nytimes3xbfgragh.onion/by/yotam-ottolenghi}{Yotam
Ottolenghi}

\begin{itemize}
\item
  July 27, 2020
\item
  \begin{itemize}
  \item
  \item
  \item
  \item
  \item
  \item
  \end{itemize}
\end{itemize}

LONDON --- Years of close observation have taught me this: People are
least adventurous in the morning. Even those who would normally thrust
themselves into anything new are likely to stop at breakfast. ``Oh, that
looks great,'' they say, ``but not so early in the day.'' It takes quite
some time for our taste buds to wake up, it seems.

The transition from sleep to fully awake therefore needs to be eased
into, perhaps with a comforting morning meal. What that looks like can
be very different depending on where you are from. My French and Italian
friends, on the minimalist end of the spectrum, are horrified by
anything beyond a continental breakfast. A cup of coffee (and a flaky
pastry, if one absolutely must) is as much as they can stomach in the
early hours. Anything cooked, highly flavored or rich gets a negative
response.

Where I've traveled in Asia, on the other hand, the kind of comfort that
sets you up for the day can be a full-blown meal, not dissimilar to
what's eaten at any other time. In different parts of India, I had rich
dals and curries, served with rice or flatbreads that could be baked or
deep-fried. In Thailand and Malaysia, things often kick off with intense
savory flavors: rice and noodles, often as porridge or soup, seasoned
liberally with chile condiments and garlic, grilled meats and steamed
Chinese-style savory buns.

\includegraphics{https://static01.graylady3jvrrxbe.onion/images/2020/07/29/dining/27otto2/merlin_174750102_3361188b-d36a-4047-9dc2-a34c2c74a936-articleLarge.jpg?quality=75\&auto=webp\&disable=upscale}

The Middle Eastern way is different, yet again. Generally speaking,
there is not as much breakfast cooking going on. In Jerusalem, where I
grew up, a popular morning spread included fresh vegetables, herbs,
young cheese and yogurt, olives and flatbreads. Cooked egg, often an
omelet, is the exception. And eggs, in general, seem to be the one
player with a truly cross-cultural breakfast reach.

In my adopted culture, the full English breakfast consists of fried egg,
grilled bacon, mushrooms, tomatoes and beans. I always found the beans
puzzling in this context --- too filling and too sweet --- but their
sweetness points to another major cultural variance on the breakfast
table: sugar, either plain, or in some kind of syrupy form, or as fresh
or cooked fruit.

Image

Stale bread also finds a home in this warming dish.Credit...Andrew
Scrivani for The New York Times

It seems to me that for many, particularly in North America, it is that
sweetness that has become the most customary form of breakfast. A stack
of thick pancakes with maple syrup, a bowl of oatmeal or breakfast
cereal with milk and a dusting or two of sugar, fresh or cooked fruit
with yogurt, jam on buttered toast --- all are expressions of the kind
of comfort that sugar brings to the breakfast table.

During the recent lockdown, when I was often cooking three daily meals
for my children, sweetness became the defining mark of breakfast in our
house as well. French toast became the embodiment of both my children's
joy and my wish not to waste a thing. My eggy bread, as we call it in
Britain, was the perfect kitchen raid, using up old bread, any fruit
that had seen better days --- quickly cooked with some sugar and vanilla
and spooned over --- and a pot of yogurt or cream lurking at the back of
the fridge.

Over time, with natural evolution and deeper rummaging through
cupboards, my variations on French toast moved farther and farther away
from the austere concept of the continental breakfast. My skillet berry
and brown butter toast crumble sits happily somewhere between breakfast
and dessert, holding on to some aspects of the original eggy bread ---
stale bread, burned butter and cooked fruit --- while being its own
deliciously rich thing. Not necessarily to everyone's liking early in
the day but, then again, you can easily hold off for a few hours and
have it as brunch, once you are fully awake and feel brave enough to try
something new.

Recipe:
\textbf{\href{https://cooking.nytimes3xbfgragh.onion/recipes/1021208-skillet-berry-and-brown-butter-toast-crumble}{Skillet
Berry and Brown Butter Toast Crumble}}

\emph{Follow} \href{https://twitter.com/nytfood}{\emph{NYT Food on
Twitter}} \emph{and}
\href{https://www.instagram.com/nytcooking/}{\emph{NYT Cooking on
Instagram}}\emph{,}
\href{https://www.facebookcorewwwi.onion/nytcooking/}{\emph{Facebook}}\emph{,}
\href{https://www.youtube.com/nytcooking}{\emph{YouTube}} \emph{and}
\href{https://www.pinterest.com/nytcooking/}{\emph{Pinterest}}\emph{.}
\href{https://www.nytimes3xbfgragh.onion/newsletters/cooking}{\emph{Get
regular updates from NYT Cooking, with recipe suggestions, cooking tips
and shopping advice}}\emph{.}

Advertisement

\protect\hyperlink{after-bottom}{Continue reading the main story}

\hypertarget{site-index}{%
\subsection{Site Index}\label{site-index}}

\hypertarget{site-information-navigation}{%
\subsection{Site Information
Navigation}\label{site-information-navigation}}

\begin{itemize}
\tightlist
\item
  \href{https://help.nytimes3xbfgragh.onion/hc/en-us/articles/115014792127-Copyright-notice}{©~2020~The
  New York Times Company}
\end{itemize}

\begin{itemize}
\tightlist
\item
  \href{https://www.nytco.com/}{NYTCo}
\item
  \href{https://help.nytimes3xbfgragh.onion/hc/en-us/articles/115015385887-Contact-Us}{Contact
  Us}
\item
  \href{https://www.nytco.com/careers/}{Work with us}
\item
  \href{https://nytmediakit.com/}{Advertise}
\item
  \href{http://www.tbrandstudio.com/}{T Brand Studio}
\item
  \href{https://www.nytimes3xbfgragh.onion/privacy/cookie-policy\#how-do-i-manage-trackers}{Your
  Ad Choices}
\item
  \href{https://www.nytimes3xbfgragh.onion/privacy}{Privacy}
\item
  \href{https://help.nytimes3xbfgragh.onion/hc/en-us/articles/115014893428-Terms-of-service}{Terms
  of Service}
\item
  \href{https://help.nytimes3xbfgragh.onion/hc/en-us/articles/115014893968-Terms-of-sale}{Terms
  of Sale}
\item
  \href{https://spiderbites.nytimes3xbfgragh.onion}{Site Map}
\item
  \href{https://help.nytimes3xbfgragh.onion/hc/en-us}{Help}
\item
  \href{https://www.nytimes3xbfgragh.onion/subscription?campaignId=37WXW}{Subscriptions}
\end{itemize}
