Sections

SEARCH

\protect\hyperlink{site-content}{Skip to
content}\protect\hyperlink{site-index}{Skip to site index}

\href{https://myaccount.nytimes3xbfgragh.onion/auth/login?response_type=cookie\&client_id=vi}{}

\href{https://www.nytimes3xbfgragh.onion/section/todayspaper}{Today's
Paper}

\href{/section/opinion}{Opinion}\textbar{}Italy Is Slowly Waking From
the Nightmare

\url{https://nyti.ms/3g2f4Jq}

\begin{itemize}
\item
\item
\item
\item
\item
\item
\end{itemize}

Advertisement

\protect\hyperlink{after-top}{Continue reading the main story}

\href{/section/opinion}{Opinion}

Supported by

\protect\hyperlink{after-sponsor}{Continue reading the main story}

\hypertarget{italy-is-slowly-waking-from-the-nightmare}{%
\section{Italy Is Slowly Waking From the
Nightmare}\label{italy-is-slowly-waking-from-the-nightmare}}

This is a very strange, subdued summer for a country with an economy
that relies heavily on tourism and merrymaking. But E.U. aid is on the
way.

\href{https://topics.nytimes3xbfgragh.onion/top/reference/timestopics/people/s/beppe_severgnini/index.html}{\includegraphics{https://static01.graylady3jvrrxbe.onion/images/2013/10/11/opinion/contributors_severgnini/contributors_severgnini-thumbLarge-v5.jpg}}

By
\href{https://topics.nytimes3xbfgragh.onion/top/reference/timestopics/people/s/beppe_severgnini/index.html}{Beppe
Severgnini}

Mr. Severgnini is a contributing opinion writer.

\begin{itemize}
\item
  July 27, 2020
\item
  \begin{itemize}
  \item
  \item
  \item
  \item
  \item
  \item
  \end{itemize}
\end{itemize}

\includegraphics{https://static01.graylady3jvrrxbe.onion/images/2020/07/27/opinion/27severgnini1/merlin_174869175_1251a1cf-017e-48e7-bcdd-9a0eea70d148-articleLarge.jpg?quality=75\&auto=webp\&disable=upscale}

ROME --- Streets are quiet, squares are empty, you can hear the gurgling
of the city's ubiquitous fountains in the daytime. The platoons of
Chinese tourists are nowhere to be seen; the American travelers ---
their hats, their sandals, their holiday shorts --- have disappeared.

Only a few German families with young children brave the afternoon heat.
Of Rome's 1,200 hotels, fewer than 200 reopened after the lockdown, some
with just five rooms occupied. On the central Via Veneto, only three
establishments are open for business. The others are shut --- officially
for maintenance. The truth is different. Those hotels cater to foreign
tourists --- and this summer, foreigners are at home, unable or
unwilling to travel.

Only a few Italian tourists roam the streets. The Romans who are not on
the nearby beaches (Ostia, Fregene) go for a stroll in the evening; some
drive to the top of the Janiculum Hill, where the fresh Ponentino west
wind brings relief from the summer heat. Rome glitters down below. The
Janiculum is home to the opulent Villa Pamphilj, where the Italian
government convened the Stati Generali dell'Economia in June: eight days
and more than 120 meetings with companies, unions, associations,
academics, writers and artists --- 82 of the meetings held personally by
Prime Minister Giuseppe Conte --- in order to find ideas to restart the
economy.

It will be tough. Italy's G.D.P. is likely to shrink by 11 percent in
2020, the worst prediction among the 27 members of the European Union.
The lack of visitors is a nightmare. The Italian tourism industry, which
includes hospitality, bars and restaurants, transport, museums
------accounts for 13 percent of G.D.P. Half of the travelers in 2019
were foreigners, and they're gone.

But national tourism is also in the doldrums. Predictions have it that
half of the Italians won't take a vacation this summer. Nine out of ten
of those who do will choose one in Italy, but they're staying away from
large cities (Milan, Turin, Naples, Palermo) and art cities (Rome,
Florence, Venice), and heading for less crowded places like Alto Adige,
Friuli, Abruzzo, and also tiny Molise, which is doubling last year's
tourist arrivals.

Last week I visited Corinaldo, in the Marche region, part of central
Italy. Nestled in the hills, 12 miles from the sandy Adriatic beaches,
surrounded by 14th-century walls, it's considered one of Italy's
prettiest villages. At 9 in the morning, tourists --- mostly Italians
--- were already roaming the streets, drinking cappuccino and seeking
the replica of the infamous fig cannon --- a cannon made of fragile fig
wood that exploded in the Middle Ages and killed all of the soldiers
attending to it.

This is the strangest summer. I drove around Italy for a week, with
stops in Romagna, Marche, Abruzzo, Puglia, Naples, Rome, Versilia in
Tuscany, then back to Lombardy --- and noticed how wary fellow Italians
are. Normality is frightening, after a while. Most people wear face
masks, even outdoors, where it's not compulsory; some masks droop sadly
from the face, instead of covering it.

We respected the rules during the spring lockdown because we were
afraid? So what? In a pandemic, fear is a form of wisdom, boldness a
show of carelessness.

Italy was the first country outside Asia to bear the brunt of the
coronavirus. Thirty-five thousand lost their lives, half of them in
Lombardy. There would have been many more without the painful spring
lockdown, which prevented the hospitals from being overwhelmed.

Mistakes were made: the area around Bergamo was not declared a no-go
area; family doctors were left on their own; patients were brought into
hospitals where doctors and nurses were infected; tests for the general
population were unavailable for too long.

But Italy coped.

Northern Italians showed resilience; Central and Southern Italians
stayed at home, even though the epidemic was less visible there. Here
and there, young crowds gathered dangerously, with the cover of night
life, beach life, politics, football, even an air show in Turin. But on
the whole, Italy stuck to the rules.

From early March to early May, the country found itself with its back to
the wall; and that's a position where we Italians give our best. We can
be disciplined, but somehow we don't like to admit it, as if it might
damage our reputation.

Of course some things didn't work. We were the first in Europe to shut
down the schools, and we'll be the last to reopen them (on Sept. 14,
hopefully). For millions of Italians with young children and small
apartments, working from home turned out to be a nightmare. And
political squabbles, after a lull, restarted. The political parties
sniff an early election, and are jockeying for position.

This slows down all decisions. Despite endless consultations, Mr. Conte
has not made up his mind about the European Stability Mechanism, whose
funds are earmarked for health expenditure. And, more important, he
hasn't decided how to allocate Italy's share of the E.U. Recovery Fund.
This has complicated negotiations at the recent European Council in
Brussels, and has given suspicious northern countries --- led by the
Netherlands --- an excuse to stall. But in the end, predictably, an
agreement was found.

Last Tuesday, after 90 hours of negotiation led by the European
Commission, the 27 leaders of the European Union agreed to look forward.
The 2021-2027 budget will be 1.8 trillion euros: of these, 750 billion
will go to the post-Covid recovery fund, called Next Generation E.U.
(390 billion will be in aid, 360 billion in loans). Italy --- one of two
countries in Europe hardest hit by the pandemic, alongside Spain ---
will be the main beneficiary. Each Italian citizen, on average, will
receive 500 euros; each German and each Dutchman will shell out 840 and
930 euros respectively.

Europe may be hyper-regulated; but in an emergency, rules and
regulations help to keep the situation under control. Slowly and
painfully, the European Union is getting out of it. Some countries
suffered more than others; but none was refused help, nor did any refuse
it. As of July 20, 135,000 deaths had been reported among the 445,000,00
people living in the Union. The day before, for the first time since
February, Lombardy --- where I live, and where it all started for Italy
--- registered no coronavirus deaths. We are still worried, but we can
finally breathe.

Beppe Severgnini, an editorial writer and editor at Corriere della Sera,
writes regularly about Italian and European politics, society and
culture.

\emph{The Times is committed to publishing}
\href{https://www.nytimes3xbfgragh.onion/2019/01/31/opinion/letters/letters-to-editor-new-york-times-women.html}{\emph{a
diversity of letters}} \emph{to the editor. We'd like to hear what you
think about this or any of our articles. Here are some}
\href{https://help.nytimes3xbfgragh.onion/hc/en-us/articles/115014925288-How-to-submit-a-letter-to-the-editor}{\emph{tips}}\emph{.
And here's our email:}
\href{mailto:letters@NYTimes.com}{\emph{letters@NYTimes.com}}\emph{.}

\emph{Follow The New York Times Opinion section on}
\href{https://www.facebookcorewwwi.onion/nytopinion}{\emph{Facebook}}\emph{,}
\href{http://twitter.com/NYTOpinion}{\emph{Twitter (@NYTopinion)}}
\emph{and}
\href{https://www.instagram.com/nytopinion/}{\emph{Instagram}}\emph{.}

Advertisement

\protect\hyperlink{after-bottom}{Continue reading the main story}

\hypertarget{site-index}{%
\subsection{Site Index}\label{site-index}}

\hypertarget{site-information-navigation}{%
\subsection{Site Information
Navigation}\label{site-information-navigation}}

\begin{itemize}
\tightlist
\item
  \href{https://help.nytimes3xbfgragh.onion/hc/en-us/articles/115014792127-Copyright-notice}{©~2020~The
  New York Times Company}
\end{itemize}

\begin{itemize}
\tightlist
\item
  \href{https://www.nytco.com/}{NYTCo}
\item
  \href{https://help.nytimes3xbfgragh.onion/hc/en-us/articles/115015385887-Contact-Us}{Contact
  Us}
\item
  \href{https://www.nytco.com/careers/}{Work with us}
\item
  \href{https://nytmediakit.com/}{Advertise}
\item
  \href{http://www.tbrandstudio.com/}{T Brand Studio}
\item
  \href{https://www.nytimes3xbfgragh.onion/privacy/cookie-policy\#how-do-i-manage-trackers}{Your
  Ad Choices}
\item
  \href{https://www.nytimes3xbfgragh.onion/privacy}{Privacy}
\item
  \href{https://help.nytimes3xbfgragh.onion/hc/en-us/articles/115014893428-Terms-of-service}{Terms
  of Service}
\item
  \href{https://help.nytimes3xbfgragh.onion/hc/en-us/articles/115014893968-Terms-of-sale}{Terms
  of Sale}
\item
  \href{https://spiderbites.nytimes3xbfgragh.onion}{Site Map}
\item
  \href{https://help.nytimes3xbfgragh.onion/hc/en-us}{Help}
\item
  \href{https://www.nytimes3xbfgragh.onion/subscription?campaignId=37WXW}{Subscriptions}
\end{itemize}
