Sections

SEARCH

\protect\hyperlink{site-content}{Skip to
content}\protect\hyperlink{site-index}{Skip to site index}

\href{https://myaccount.nytimes3xbfgragh.onion/auth/login?response_type=cookie\&client_id=vi}{}

\href{https://www.nytimes3xbfgragh.onion/section/todayspaper}{Today's
Paper}

\href{/section/opinion}{Opinion}\textbar{}As Federal Forces Confront
Protesters

\url{https://nyti.ms/3hDBNM8}

\begin{itemize}
\item
\item
\item
\item
\item
\end{itemize}

Advertisement

\protect\hyperlink{after-top}{Continue reading the main story}

\href{/section/opinion}{Opinion}

Supported by

\protect\hyperlink{after-sponsor}{Continue reading the main story}

Letters

\hypertarget{as-federal-forces-confront-protesters}{%
\section{As Federal Forces Confront
Protesters}\label{as-federal-forces-confront-protesters}}

Readers condemn President Trump's use of armed forces in Portland, Ore.
One reader describes ``a real war on American cities in a feeble attempt
to save his presidency.''

July 27, 2020

\begin{itemize}
\item
\item
\item
\item
\item
\end{itemize}

\includegraphics{https://static01.graylady3jvrrxbe.onion/images/2020/07/26/opinion/26kristof2/merlin_174823974_38af76d5-f3db-46e3-b9a6-01e3ec05bb9b-articleLarge.jpg?quality=75\&auto=webp\&disable=upscale}

\textbf{To the Editor:}

Re
``\href{https://www.nytimes3xbfgragh.onion/2020/07/25/opinion/sunday/portland-protest-federal-troops.html}{In
Portland's `War Zone,' Troops Menace}'' (column, July 26):

Nicholas Kristof's column is simultaneously illuminating and terrifying.
President Trump's ``Wag the Dog'' scenario isn't about a president
declaring a mythical election-year war on Albania, as portrayed in the
iconic film. No, Mr. Trump's scenario is to declare a real war on
American cities in a feeble attempt to save his presidency.

Real American forces and real American munitions are being used in
Portland, Ore. --- and undoubtedly very soon in other Democratic-led
cities --- as unpaid and unwitting people are being used as props and
extras in Trump campaign commercials. We've already seen a few such
commercials in which somber and serious sounding voice-overs tell us
that this is what we can expect in a Biden presidency.

Does no one perceive the irony and the deceitful hypocrisy? These scenes
are being filmed right here, right now, in the midst of a Trump
presidency.

America, wake up --- finally!

Mike Schneider\\
Truckee, Calif.

\textbf{To the Editor:}

Re
``\href{https://www.nytimes3xbfgragh.onion/2020/07/26/us/protests-portland-seattle-trump.html}{Cities
Are in Bind as Unrest Flares Over U.S. Actions}'' (front page, July 27):

If I could reach the protesters in Portland, I would say:

While I sympathize with your demands and complaints, including the
removal of federal forces from your city streets, I worry very much that
your actions are helpful to President Trump's appeal to parts of the
electorate. Your protests might be hurting your cause. I recommend that
you go home and work more effectively by:

Supporting the John Lewis voting rights bill in Congress.

Supporting voter registration and absentee ballot registration.

Supporting the campaigns of Democrats at all levels of government in the
upcoming November elections.

Katie Wittenberg\\
Arlington, Va.

\textbf{To the Editor:}

Re
``\href{https://www.nytimes3xbfgragh.onion/2020/07/25/opinion/sunday/portland-protests-white.html}{Who
Gets to Be a `Naked Athena?,'}'' by Mitchell S. Jackson (Sunday Review,
July 26):

I live in Portland, Ore. When I saw the video of ``Naked Athena,'' a
naked woman striding majestically out of the darkness and into the
glaring spotlights of the federal goon squad, I felt that I was
witnessing a magical reawakening of hope. There she stood, proudly and
unflinchingly, staring back at the anonymous camouflage-clad men. She
was unclad and they were heavily armed, yet she represented the stronger
force.

To me she was Aphrodite rather than Athena. She represented purity and
love, which must triumph over unreasoning brute force.

Carol Benson Knutson\\
Portland, Ore.

\textbf{To the Editor:}

Coming from a family with a history of military service, we display both
a Black Lives Matter sign and an American flag in honor of those who
defended our liberty. Both of these symbols represent to us the
continued pride of America's past struggles and the will to evolve and
fix what needs changing.

That is why I'm confused about why I don't see the American flag used in
the protests, or if they are being used, maybe the images are not being
published. Federal forces would be less likely to engage violently with
a protester holding an American flag or a Black Lives Matter sign with
an American flag incorporated into the design.

The image of federal forces attacking an American flag would be
powerful. Adding the flag to the protests may help keep them peaceful.

Dan King\\
Rockport, Mass.

\textbf{To the Editor:}

Re
``\href{https://www.nytimes3xbfgragh.onion/2020/07/26/us/selma-john-lewis-memorial.html}{Lewis's
Funeral Procession Follows His Footsteps Across Bridge}'' (news article,
July 27):

Is there any difference between what armed, federally employed violent
thugs are doing in Portland, Ore., and what armed, state-employed
violent thugs did in Selma, Ala., on the Edmund Pettus Bridge 55 years
ago?

Thank you, John Lewis, for your courage and bravery, and for your moral
integrity on that day. You led us by your example then, and by a
lifetime of examples of what it means to be an American, keeping alive
decency and mercy, and the dream of a country with liberty and justice
for all.

Stephen Tessler\\
Davis, Calif.

Advertisement

\protect\hyperlink{after-bottom}{Continue reading the main story}

\hypertarget{site-index}{%
\subsection{Site Index}\label{site-index}}

\hypertarget{site-information-navigation}{%
\subsection{Site Information
Navigation}\label{site-information-navigation}}

\begin{itemize}
\tightlist
\item
  \href{https://help.nytimes3xbfgragh.onion/hc/en-us/articles/115014792127-Copyright-notice}{©~2020~The
  New York Times Company}
\end{itemize}

\begin{itemize}
\tightlist
\item
  \href{https://www.nytco.com/}{NYTCo}
\item
  \href{https://help.nytimes3xbfgragh.onion/hc/en-us/articles/115015385887-Contact-Us}{Contact
  Us}
\item
  \href{https://www.nytco.com/careers/}{Work with us}
\item
  \href{https://nytmediakit.com/}{Advertise}
\item
  \href{http://www.tbrandstudio.com/}{T Brand Studio}
\item
  \href{https://www.nytimes3xbfgragh.onion/privacy/cookie-policy\#how-do-i-manage-trackers}{Your
  Ad Choices}
\item
  \href{https://www.nytimes3xbfgragh.onion/privacy}{Privacy}
\item
  \href{https://help.nytimes3xbfgragh.onion/hc/en-us/articles/115014893428-Terms-of-service}{Terms
  of Service}
\item
  \href{https://help.nytimes3xbfgragh.onion/hc/en-us/articles/115014893968-Terms-of-sale}{Terms
  of Sale}
\item
  \href{https://spiderbites.nytimes3xbfgragh.onion}{Site Map}
\item
  \href{https://help.nytimes3xbfgragh.onion/hc/en-us}{Help}
\item
  \href{https://www.nytimes3xbfgragh.onion/subscription?campaignId=37WXW}{Subscriptions}
\end{itemize}
