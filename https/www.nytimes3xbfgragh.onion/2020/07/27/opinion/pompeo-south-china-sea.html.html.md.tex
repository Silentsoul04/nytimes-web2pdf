Sections

SEARCH

\protect\hyperlink{site-content}{Skip to
content}\protect\hyperlink{site-index}{Skip to site index}

\href{https://myaccount.nytimes3xbfgragh.onion/auth/login?response_type=cookie\&client_id=vi}{}

\href{https://www.nytimes3xbfgragh.onion/section/todayspaper}{Today's
Paper}

\href{/section/opinion}{Opinion}\textbar{}China's Claims to the South
China Sea Are Unlawful. Now What?

\url{https://nyti.ms/2EmPmBf}

\begin{itemize}
\item
\item
\item
\item
\item
\end{itemize}

Advertisement

\protect\hyperlink{after-top}{Continue reading the main story}

\href{/section/opinion}{Opinion}

Supported by

\protect\hyperlink{after-sponsor}{Continue reading the main story}

\hypertarget{chinas-claims-to-the-south-china-sea-are-unlawful-now-what}{%
\section{China's Claims to the South China Sea Are Unlawful. Now
What?}\label{chinas-claims-to-the-south-china-sea-are-unlawful-now-what}}

Republican and Democratic administrations have failed to thwart
aggressive expansion in one of the world's busiest sea lanes. The
solution isn't flashy, but it could work.

By
\href{https://www.nytimes3xbfgragh.onion/interactive/opinion/editorialboard.html}{The
Editorial Board}

The editorial board is a group of opinion journalists whose views are
informed by expertise, research, debate and certain longstanding ****
\href{https://www.nytimes3xbfgragh.onion/interactive/2018/opinion/editorialboard.html}{values}.
It is separate from the newsroom.

\begin{itemize}
\item
  July 27, 2020
\item
  \begin{itemize}
  \item
  \item
  \item
  \item
  \item
  \end{itemize}
\end{itemize}

\includegraphics{https://static01.graylady3jvrrxbe.onion/images/2020/07/27/opinion/27china-sea-editorial/27china-sea-editorial-articleLarge-v6.jpg?quality=75\&auto=webp\&disable=upscale}

With China throwing its weight around in the South China Sea, Secretary
of State Mike Pompeo rightly declared this month that the country's
aggressive claims to offshore resources in the area were
``\href{https://www.theguardian.com/world/2020/jul/14/south-china-sea-us-says-beijings-claims-to-disputed-area-completely-unlawful}{completely
unlawful}.'' The message aligned the United States with international
law in one of the world's most critical waterways and showed support for
the smaller coastal states threatened by Chinese bullying.

The strategic importance of the South China Sea cannot be overstated. A
third of the world's shipping passes through it, its fisheries are
critical sources of food for millions of people in Southeast Asia and
its seabed covers vast reserves of oil and gas. China's exorbitant
claims and might-makes-right behavior are self-evidently destabilizing
to the international rule of law.

The problem is that Mr. Pompeo's statement is meaningful only if it is
accompanied by a firm commitment by the Trump administration to a robust
and coordinated policy. However incensed China's neighbors are by its
bullying, they are in no position to push back unless they can be
certain of American support and leadership.

Apart from the possibility that the Trump administration is in its final
months, the president and his revolving-door lieutenants have abdicated
multilateral leadership in the Pacific and elsewhere in the world in the
name of ``America first.'' Mr. Trump has vacillated between waging a
\href{https://www.nytimes3xbfgragh.onion/2019/08/06/opinion/china-currency.html}{tariff
war} against China's ``predatory practices'' in trade, singing paeans to
President Xi Jinping and, according to John Bolton's
\href{https://www.washingtonpost.com/politics/trump-asked-chinas-xi-to-help-him-win-reelection-according-to-bolton-book/2020/06/17/d4ea601c-ad7a-11ea-868b-93d63cd833b2_story.html}{account
of his time} as Mr. Trump's national security adviser, seeking favors
from Mr. Xi to help his re-election.

In the absence of any coherent China policy, the administration's
proclivity for tearing up treaties and its disdain for alliances, Mr.
Pompeo's belated declaration that China is violating international law
--- and especially
\href{https://www.britannica.com/topic/Law-of-the-Sea}{the Law of the
Sea treaty}, which the United States has never ratified --- sounds a bit
hollow.

It is, nonetheless, a message that is valid and long overdue. Over the
past decade, China has steadily hardened its claims to most of the South
China Sea, a zone circumscribed by a vague ``nine-dash line'' that one
American naval commander called the
``\href{https://www.bbc.com/news/world-asia-32126840}{Great Wall of
Sand}.'' The claims have included a campaign of building up shoals and
militarizing islands or proclaiming municipal districts and settling
people on contested islands. The reclamation of several reefs and atolls
in the Spratly Islands has included construction of runways, hangars,
barracks, missile silos and radar sites.

In recent months, with much of the world preoccupied with the Covid-19
pandemic, China has sharply escalated its coercive activities. In early
April, a Chinese
\href{https://www.defense.gov/Newsroom/Releases/Release/Article/2143925/china-coast-guard-sinking-of-a-vietnam-fishing-vessel/}{Coast
Guard vessel} sank a Vietnamese fishing boat close to islands claimed by
both China and Vietnam. A Chinese marine survey vessel harassed a
\href{https://www.reuters.com/article/us-china-security-malaysia/malaysian-oil-exploration-vessel-leaves-south-china-sea-waters-after-standoff-idUSKBN22O1M9\#:~:text=KUALA\%20LUMPUR\%20(Reuters)\%20\%2D\%20An,and\%20the\%20vessel\%20operator\%20said.}{Malaysian
oil exploration} vessel off Borneo. This month, the Department of
Defense
\href{https://www.defense.gov/Newsroom/Releases/Release/Article/2246604/peoples-republic-of-china-military-exercises-in-the-south-china-sea/}{voiced
concern} about the Chinese Navy's decision to seal off an area around
the Paracel Islands to conduct naval exercises. In response, the United
States
\href{https://www.nytimes3xbfgragh.onion/2020/07/04/us/politics/south-china-sea-aircraft-carrier.html?searchResultPosition=1}{increased
its own naval activities}, including joint exercises by two aircraft
carrier groups.

These confrontations have contributed to a sharp deterioration in
U.S.-China relations on other fronts. Over a few weeks, while Mr. Trump
has continued to publicly blame China for the coronavirus outbreak, the
United States
has\href{https://www.nytimes3xbfgragh.onion/2020/07/15/world/asia/china-trump-hong-kong.html}{punished
Chinese officials} over Beijing's crackdown in Hong Kong and the western
region of Xinjiang and accused the Chinese of stealing intellectual
property. Last week, the State Department
\href{https://www.bbc.com/news/world-us-canada-53497193}{ordered China
to shut down} its consulate in Houston, provoking the
\href{https://www.nytimes3xbfgragh.onion/2020/07/24/world/asia/china-us-consulate-chengdu.html}{closure
of the American consulate} in the southwestern city of Chengdu.

The strains are not likely to subside as China continues to grow in
wealth, power and technological prowess. Though the tensions between
Washington and Beijing have often been likened to the Cold War with the
Soviet Union, they are far different. China is a major trading partner
with the United States and much of the rest of the world. It does not
command an empire, and its economy is not likely to crumble under the
weight of Western challenges the way Moscow's command economy did.

Dealing with the new China will require a balance of diplomacy,
firmness, credible deterrents and a code of conduct, especially in the
South China Sea. It will require a broad consensus among China's
Southeast Asian neighbors and America's allies, all of which are opposed
to any restrictions on navigation through the South China Sea but are
also cognizant of the importance of trade with China. The Obama
administration's Asia strategy marked a recognition of these realities
and the need for a coherent approach.

In fact, Mr. Pompeo's statement did not break new ground, since the
United States already effectively recognized the 2016 decision of an
\href{https://www.nytimes3xbfgragh.onion/2016/07/13/world/asia/south-china-sea-hague-ruling-philippines.html}{international
arbitration court} that rebuffed China's ``nine-dash line'' claims in a
landmark case brought by the Philippines. (China rejects the ruling.)
What Mr. Pompeo did was to overtly declare the validity of the decision.

Though not necessarily a change of policy, the statement would give
greater authority to any American-led punitive actions, whether through
sanctions, United Nations resolutions or joint action by organizations
such as the Association of Southeast Asian Nations or the Group of 7.

Yet the administration has largely spurned international organizations,
and its approach to China has vacillated widely and unpredictably. Mr.
Trump tore up the
\href{https://www.cfr.org/backgrounder/what-trans-pacific-partnership-tpp}{Trans-Pacific
Partnership} trade agreement with 11 other countries and launched a
tariff war against China while at the same time heaping praise on Mr.
Xi, avoiding criticism of China's human-rights violations and claiming,
as he did last January at Davos, that ``our relationship with China has
now probably never, ever been better.''

Mr. Pompeo has said the right things about China's unlawful behavior.
But unless words are accompanied by a credible American re-engagement in
the region, including a clear commitment to diplomacy, investment and
security, the words are just more election-year bluster.

\emph{The Times is committed to publishing}
\href{https://www.nytimes3xbfgragh.onion/2019/01/31/opinion/letters/letters-to-editor-new-york-times-women.html}{\emph{a
diversity of letters}} \emph{to the editor. We'd like to hear what you
think about this or any of our articles. Here are some}
\href{https://help.nytimes3xbfgragh.onion/hc/en-us/articles/115014925288-How-to-submit-a-letter-to-the-editor}{\emph{tips}}\emph{.
And here's our email:}
\href{mailto:letters@NYTimes.com}{\emph{letters@NYTimes.com}}\emph{.}

\emph{Follow The New York Times Opinion section on}
\href{https://www.facebookcorewwwi.onion/nytopinion}{\emph{Facebook}}\emph{,}
\href{http://twitter.com/NYTOpinion}{\emph{Twitter (@NYTopinion)}}
\emph{and}
\href{https://www.instagram.com/nytopinion/}{\emph{Instagram}}\emph{.}

Advertisement

\protect\hyperlink{after-bottom}{Continue reading the main story}

\hypertarget{site-index}{%
\subsection{Site Index}\label{site-index}}

\hypertarget{site-information-navigation}{%
\subsection{Site Information
Navigation}\label{site-information-navigation}}

\begin{itemize}
\tightlist
\item
  \href{https://help.nytimes3xbfgragh.onion/hc/en-us/articles/115014792127-Copyright-notice}{©~2020~The
  New York Times Company}
\end{itemize}

\begin{itemize}
\tightlist
\item
  \href{https://www.nytco.com/}{NYTCo}
\item
  \href{https://help.nytimes3xbfgragh.onion/hc/en-us/articles/115015385887-Contact-Us}{Contact
  Us}
\item
  \href{https://www.nytco.com/careers/}{Work with us}
\item
  \href{https://nytmediakit.com/}{Advertise}
\item
  \href{http://www.tbrandstudio.com/}{T Brand Studio}
\item
  \href{https://www.nytimes3xbfgragh.onion/privacy/cookie-policy\#how-do-i-manage-trackers}{Your
  Ad Choices}
\item
  \href{https://www.nytimes3xbfgragh.onion/privacy}{Privacy}
\item
  \href{https://help.nytimes3xbfgragh.onion/hc/en-us/articles/115014893428-Terms-of-service}{Terms
  of Service}
\item
  \href{https://help.nytimes3xbfgragh.onion/hc/en-us/articles/115014893968-Terms-of-sale}{Terms
  of Sale}
\item
  \href{https://spiderbites.nytimes3xbfgragh.onion}{Site Map}
\item
  \href{https://help.nytimes3xbfgragh.onion/hc/en-us}{Help}
\item
  \href{https://www.nytimes3xbfgragh.onion/subscription?campaignId=37WXW}{Subscriptions}
\end{itemize}
