Sections

SEARCH

\protect\hyperlink{site-content}{Skip to
content}\protect\hyperlink{site-index}{Skip to site index}

\href{https://www.nytimes3xbfgragh.onion/section/opinion/sunday}{Sunday
Review}

\href{https://myaccount.nytimes3xbfgragh.onion/auth/login?response_type=cookie\&client_id=vi}{}

\href{https://www.nytimes3xbfgragh.onion/section/todayspaper}{Today's
Paper}

\href{/section/opinion/sunday}{Sunday Review}\textbar{}Medical Schools
Have Historically Been Wrong on Race

\url{https://nyti.ms/39KQ72P}

\begin{itemize}
\item
\item
\item
\item
\item
\end{itemize}

Advertisement

\protect\hyperlink{after-top}{Continue reading the main story}

\href{/section/opinion}{Opinion}

Supported by

\protect\hyperlink{after-sponsor}{Continue reading the main story}

\hypertarget{medical-schools-have-historically-been-wrong-on-race}{%
\section{Medical Schools Have Historically Been Wrong on
Race}\label{medical-schools-have-historically-been-wrong-on-race}}

While the blatant horrors of the past are gone, the ideas that fueled
race-based medicine stubbornly linger. We can change.

By Damon Tweedy

Dr. Tweedy is a psychiatrist at Duke University School of Medicine.

\begin{itemize}
\item
  July 27, 2020
\item
  \begin{itemize}
  \item
  \item
  \item
  \item
  \item
  \end{itemize}
\end{itemize}

\includegraphics{https://static01.graylady3jvrrxbe.onion/images/2020/07/27/opinion/27tweedy/27tweedy-articleLarge.jpg?quality=75\&auto=webp\&disable=upscale}

A senior white physician stands by silently as a white patient uses a
slur to describe a Black nurse.

A middle-aged doctor asks a Black student why the lower-income Black
patients in the clinic aren't able to speak and act the way she does.

Several young doctors make fun of ``Black-sounding'' names in a newborn
unit and speculate when each infant will later enter the penal system.

Instances of racism like these are happening in medical schools across
America today, just as they happened when I was a medical student 20
years ago.

I still vividly recall the afternoon many years ago when a patient
angrily suggested that I go back to Africa and stay there, and the shrug
that my white supervisor offered when I told him what happened. I'll
also never forget when a Muslim student's name was openly mocked by a
senior doctor who questioned whether he was a terrorist. Nor will I
forget the moments I saw Black patients' intelligence, motivation and
truthfulness derisively questioned.

Facing the power of those above us, I remember many of my peers and me
putting our heads down, just enduring, on our way to becoming
full-fledged doctors. The stings never faded.

Now, however, many future physicians of color and their white allies are
doing what we were too afraid to do: They are speaking up and demanding
change. And medical school leaders are being forced to listen to them.
Medicine, like other institutions in society, is now being called to
task by its own for the role it has played in perpetuating the
longstanding inequities that have led us to this moment.

The conversation took on new energy at the start of the Covid-19
pandemic, which exposed racial health disparities dating back to the
origins of our country. And it became inflamed in the aftermath of
George Floyd's video-recorded murder.

I see myself in all of the dimensions of this overdue conversation: I am
the medical student facing discrimination, a Black man with pre-existing
health conditions and a medical school professor forced to face the cold
reality that I have not done nearly enough to help my patients or my
students of color.

On so many levels, Black patients and Black doctors are perpetually
fighting upstream. Covid-19 has killed Black people at a rate
\href{https://www.npr.org/sections/health-shots/2020/06/22/881886733/black-medicare-patients-with-covid-19-nearly-4-times-as-likely-to-end-up-in-hosp}{roughly
two times greater} than would be expected based on their share of the
population. Still, diabetes, heart disease and many cancers have
disproportionately ravaged Black families for much longer. Leading
medical journals across all clinical specialties have chronicled these
stark realities for years. But wide-scale interventions are scarce.

\includegraphics{https://static01.graylady3jvrrxbe.onion/images/2020/07/27/opinion/27tweedy1/merlin_172498455_97141459-151b-48f8-951a-905b7791b1e5-articleLarge.jpg?quality=75\&auto=webp\&disable=upscale}

Racial health disparities can't change until the health system changes
itself. Starting that transformation means shifting the way that medical
students are taught the interplay between race and health. For far too
long, medical schools have neglected to tackle the full complexity of
race in their curriculums. And two problematic, longstanding prejudices
have filled that vacuum and impeded progress.

One is the focus on race as a category signifying distinct biological
difference, a belief that dates back to slavery, where it was used as
justification to maintain the practice. Yes, Black people are much more
likely to have sickle cell anemia, just as white people are more likely
to have cystic fibrosis. But these and other diseases that closely, but
not precisely, track with race (or more accurately, ethnic origin)
represent a very small fraction of what is encountered in medicine.

Still, distinctly race-based biology remained mainstream medicine
throughout much of the 20th century. And this allowed for some of
medicine's most egregious sins: As recently
\href{https://www.cdc.gov/tuskegee/timeline.htm}{as the 1970s}, Black
people were experimented upon under the guise of scientific study and
sterilized without their consent.

While those blatant horrors of the past are gone, the ideology that
fueled those actions stubbornly lingers.

A
\href{https://www.pnas.org/content/early/2016/03/30/1516047113.full}{2016
study} at one institution found that half of the medical students and
residents surveyed agreed with one or more false statements about
biological differences based on race, such as the idea that Black people
had thicker skin and less sensitive nerve endings than their white
counterparts.

The study only buttressed
\href{https://www.sciencedirect.com/science/article/abs/pii/S027795369900338X\#:~:text=Socioeconomic\%20status\%20moderates\%20the\%20relationship,pleasant\%20and\%20rational\%20than\%20Whites.}{earlier
surveys} in which white physicians consistently categorized Black
patients as less intelligent. Such false beliefs, left unchallenged, or
worse, tacitly reinforced by professors, can lead a new generation of
doctors to perpetuate discriminatory practices. To name just one
example: Black patients are
\href{https://journalofethics.ama-assn.org/article/pain-and-ethnicity/2013-05}{prescribed
less pain medication} for injuries comparable with those of their
non-Black counterparts.

The other major flawed way in which medical education has historically
been wrong on race is in its eagerness to ascribe health differences
primarily to Black people's supposed pathological misbehavior.

Too often, physicians assume certain groups of people bring an array of
maladies or misfortunes upon themselves with intentional bad choices.
During my residency training in psychiatry, a doctor I worked with
suggested that Black men were more likely to have poor judgment
(excluding me of course, he quickly added) and that this explained why
they faced higher rates of involuntary hospitalization and the
potentially negative consequences that come from it.

The overlapping prejudices embedded in the medical establishment are
ultimately harmful not because they hurt feelings but because they
alienate patients who need help and lead to bad medicine. They are
biases that prevent the profession from taking a more accurate and
enlightened view that emphasizes the pervasive environmental and
economic roots of patients' health problems.

Covid-19 has highlighted these issues. Pathologizing Black behavior
leads to blaming Black patients, like the theory of an Ohio physician
and politician
\href{https://www.washingtonpost.com/nation/2020/06/11/black-coronavirus-ohio-gop/}{who
publicly speculated} last month whether Black people are more
susceptible to Covid-19 because they don't wash their hands enough.

A more nuanced approach, informed by public health, leads to exploring
the real, underlying reasons the coronavirus has caused more destruction
in Black communities: crowded multigenerational housing arrangements,
more frequent use of public transportation and employment in newly
hazardous front-line service jobs.

In recent years, many medical schools have begun broadening curriculums
to include implicit bias and the social factors that influence the
health of diverse patient groups. But even the most dynamic lecture can
be easily drowned out by the hundreds of hours students spend
experiencing the broader informal curriculum in clinics and hospital,
where myths about biological difference and behavioral pathology still
linger.

So it's essential that we set up ways to ensure physician-educators are
also trained and periodically evaluated in a tangible and accountable
way. That way they can pass along a more empathic and open-minded
approach to treating patients. And it's a task too important to be
relegated to a certain lecture or delegated to Black faculty. If it
doesn't involve every component of the medical school, we will continue
to perpetuate the problem.

Thanks to the work of brave young medical students who have pushed us to
have this introspective discourse and re-examine our practices, medicine
is being presented with an opportunity to reckon with its troubled past
and redefine its societal role.

A profession sworn to heal can no longer passively accept the inequities
it has witnessed for decades --- or the hand that it has played in them.

Damon Tweedy is an associate professor of psychiatry at Duke University
School of Medicine and the author of ``Black Man in a White Coat: A
Doctor's Reflections on Race and Medicine.''

\emph{The Times is committed to publishing}
\href{https://www.nytimes3xbfgragh.onion/2019/01/31/opinion/letters/letters-to-editor-new-york-times-women.html}{\emph{a
diversity of letters}} \emph{to the editor. We'd like to hear what you
think about this or any of our articles. Here are some}
\href{https://help.nytimes3xbfgragh.onion/hc/en-us/articles/115014925288-How-to-submit-a-letter-to-the-editor}{\emph{tips}}\emph{.
And here's our email:}
\href{mailto:letters@NYTimes.com}{\emph{letters@NYTimes.com}}\emph{.}

\emph{Follow The New York Times Opinion section on}
\href{https://www.facebookcorewwwi.onion/nytopinion}{\emph{Facebook}}\emph{,}
\href{http://twitter.com/NYTOpinion}{\emph{Twitter (@NYTopinion)}}
\emph{and}
\href{https://www.instagram.com/nytopinion/}{\emph{Instagram}}\emph{.}

Advertisement

\protect\hyperlink{after-bottom}{Continue reading the main story}

\hypertarget{site-index}{%
\subsection{Site Index}\label{site-index}}

\hypertarget{site-information-navigation}{%
\subsection{Site Information
Navigation}\label{site-information-navigation}}

\begin{itemize}
\tightlist
\item
  \href{https://help.nytimes3xbfgragh.onion/hc/en-us/articles/115014792127-Copyright-notice}{©~2020~The
  New York Times Company}
\end{itemize}

\begin{itemize}
\tightlist
\item
  \href{https://www.nytco.com/}{NYTCo}
\item
  \href{https://help.nytimes3xbfgragh.onion/hc/en-us/articles/115015385887-Contact-Us}{Contact
  Us}
\item
  \href{https://www.nytco.com/careers/}{Work with us}
\item
  \href{https://nytmediakit.com/}{Advertise}
\item
  \href{http://www.tbrandstudio.com/}{T Brand Studio}
\item
  \href{https://www.nytimes3xbfgragh.onion/privacy/cookie-policy\#how-do-i-manage-trackers}{Your
  Ad Choices}
\item
  \href{https://www.nytimes3xbfgragh.onion/privacy}{Privacy}
\item
  \href{https://help.nytimes3xbfgragh.onion/hc/en-us/articles/115014893428-Terms-of-service}{Terms
  of Service}
\item
  \href{https://help.nytimes3xbfgragh.onion/hc/en-us/articles/115014893968-Terms-of-sale}{Terms
  of Sale}
\item
  \href{https://spiderbites.nytimes3xbfgragh.onion}{Site Map}
\item
  \href{https://help.nytimes3xbfgragh.onion/hc/en-us}{Help}
\item
  \href{https://www.nytimes3xbfgragh.onion/subscription?campaignId=37WXW}{Subscriptions}
\end{itemize}
