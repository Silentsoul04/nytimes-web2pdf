Sections

SEARCH

\protect\hyperlink{site-content}{Skip to
content}\protect\hyperlink{site-index}{Skip to site index}

\href{https://www.nytimes3xbfgragh.onion/section/opinion/sunday}{Sunday
Review}

\href{https://myaccount.nytimes3xbfgragh.onion/auth/login?response_type=cookie\&client_id=vi}{}

\href{https://www.nytimes3xbfgragh.onion/section/todayspaper}{Today's
Paper}

\href{/section/opinion/sunday}{Sunday Review}\textbar{}Colonialism Made
the Modern World. Let's Remake It.

\url{https://nyti.ms/3gkzBZD}

\begin{itemize}
\item
\item
\item
\item
\item
\end{itemize}

Advertisement

\protect\hyperlink{after-top}{Continue reading the main story}

\href{/section/opinion}{Opinion}

Supported by

\protect\hyperlink{after-sponsor}{Continue reading the main story}

\hypertarget{colonialism-made-the-modern-world-lets-remake-it}{%
\section{Colonialism Made the Modern World. Let's Remake
It.}\label{colonialism-made-the-modern-world-lets-remake-it}}

This is what real ``decolonization'' should look like.

By Adom Getachew

Dr. Getachew is the author of ``Worldmaking After Empire: The Rise and
Fall of Self-Determination.''

\begin{itemize}
\item
  July 27, 2020
\item
  \begin{itemize}
  \item
  \item
  \item
  \item
  \item
  \end{itemize}
\end{itemize}

\includegraphics{https://static01.graylady3jvrrxbe.onion/images/2020/07/27/opinion/27getachew1/merlin_173394009_27aeade5-01fb-41fa-a1df-4a3f132d39a8-articleLarge.jpg?quality=75\&auto=webp\&disable=upscale}

``Decolonize this place!'' ``Decolonize the university!'' ``Decolonize
the museum!''

In the past few years, decolonization has gained new political currency
--- inside the borders of the old colonial powers. Indigenous movements
have reclaimed the mantle of ``decolonization'' in protests like those
at Standing Rock against the Dakota Access pipeline. Students from South
Africa to Britain have marched under its banner to challenge Eurocentric
curriculums. Museums such as the Natural History Museum in New York and
the Royal Museum for Central Africa in Brussels have been compelled to
confront their representation of colonized African and Indigenous
peoples.

But what is ``decolonization?'' What the word means and what it requires
have been contested for a century.

After World War I, European colonial administrators viewed
decolonization as the process in which they would allow their imperial
charges to graduate to independence by modeling themselves on European
states. But in the mid-20th century, anticolonial activists and
intellectuals demanded immediate independence and refused to model their
societies on the terms set by imperialists. Between 1945 and 1975, as
struggles for independence were won in Africa and Asia, United Nations
membership grew from 51 to 144 countries. In that period, decolonization
was primarily political and economic.

As more colonies gained independence, however, cultural decolonization
became more significant. European political and economic domination
coincided with a Eurocentrism that valorized European civilization as
the apex of human achievement. Indigenous cultural traditions and
systems of knowledge were denigrated as backward and uncivilized. The
colonized were treated as people without history. The struggle against
this has been especially central in settler colonies in which the
displacement of Indigenous institutions was most violent.

South Africa, where a reckoning with the persistence of the settler
regime has gripped national politics, reignited the latest calls for
decolonization in 2015 with the
\href{https://www.sabcnews.com/sabcnews/rhodesmustfall-wants-statues-representing-oppression-racism-removed/}{\#RhodesMustFall
movement}. Students at the University of Cape Town targeted the statue
of the British imperialist Cecil Rhodes, but saw its removal as only the
opening act in a wider struggle to bring white supremacy to an end.
Under the banners of ``more than a statue'' and ``decolonize the
university,'' students called for social and economic transformation to
undo the racial hierarchies that persist in post-apartheid South Africa,
free university tuition and an Africa-centered curriculum.

Now, partly riding the global surge of Black Lives Matter mobilizations,
calls for decolonization have swept Europe's former imperial metropoles.
In Bristol, England, last month, protesters
\href{https://www.nytimes3xbfgragh.onion/2020/06/12/opinion/edward-colston-statue-racism.html}{tore
down the statue of Edward Colston}, the director of the Royal African
Company, which dominated the African slave trade in the 17th and 18th
centuries. Across Belgium, protesters have focused on statues of King
Leopold II, who ruled the Congo Free State (now the Democratic Republic
of Congo) as his personal property from 1885 to 1908. King Phillipe II
of Belgium
\href{https://www.nytimes3xbfgragh.onion/2020/06/30/world/europe/belgium-king-congo.html}{recently
expressed ``regret''} for his ancestor's brutal regime, which caused the
death of 10 million people.

Colonialism, the protesters insist, did not just shape the global south.
It made Europe and the modern world. Profits from the slave trade fueled
the rise of port cities like Bristol, Liverpool and London while the
Atlantic economy that slavery created helped to fuel the Industrial
Revolution. King Leopold amassed a fortune of well over \$1.1 billion in
today's dollars from Congo. His vision of the Royal Museum for Central
Africa, which opened in 1910 soon after his death, reproduced a
narrative of African backwardness while obscuring the violent
exploitation of the Congolese.

By tearing down or defacing these statues, protesters burst open the
national narrative and force a confrontation with the history of empire.
This is a decolonization of the sensory world, the illusion that empire
was somewhere else.

Laying a flag of the Democratic Republic of Congo on the statue of King
Leopold or hauling the Colston statue into the sea, where thousands of
enslaved women and men lost their lives, tears apart the blinders and
boundaries between past and present, metropole and colony. Insisting on
the presence of the past, the protests reveal Europe's romance with
itself, unmasking its political and economic achievements as the product
of enslavement and colonial exploitation.

This historical reckoning is only the first step. Acknowledging that
colonial history shapes the current inequalities and hierarchies that
structure the world sets the stage for the next one: reparations and
restitution.

Reparations is not a single act. The Caribbean Community has already
demanded reparations for slavery and Indigenous genocide from Britain,
France, Spain and the Netherlands. Although there is little movement at
the level of states, the University of Glasgow
\href{https://www.theguardian.com/uk-news/2019/aug/23/glasgow-university-slave-trade-reparations}{agreed
last year to pay 20 million pounds} (about \$25 million) for development
research with the University of the West Indies in recognition of how
the university benefited from the profits of the trans-Atlantic slave
trade.

The Herero of Namibia, who suffered
\href{https://www.ushmm.org/collections/bibliography/herero-and-nama-genocide}{the
20th century's first genocide} at the hands of Germany, have also called
for redress. Their efforts follow the
\href{https://www.bbc.co.uk/news/uk-22790037}{successful bid for
reparations} by the Mau Mau of Kenya, many of whom were tortured during
Britain's brutal suppression of their independence movement in the
mid-20th century. In other contexts, activists have focused on the
return of the looted artifacts that fill Europe's great museums. France,
for instance, has committed to returning 26 stolen artworks to Benin.

But reparations should not focus only on the former colonies and their
relations with European states. Colonialism lives on inside Europe's
borders, and Europe itself must be decolonized. Black Europeans
experience discrimination in employment and education, are racially
profiled and are subject to racist violence at the hands of the police
and fellow citizens.

The European Union recently avowed that ``Black lives matter,'' but its
policies deprive Black people of equal rights,
\href{https://www.nytimes3xbfgragh.onion/2018/10/03/opinion/greece-europe-refugees.html}{imprison
them in camps} and
\href{https://www.nytimes3xbfgragh.onion/interactive/2018/12/26/opinion/europe-migrant-crisis-mediterranean-libya.html}{drown
them in the Mediterranean}. Overseas imperialism was once believed to be
a political necessity for European states; today, anti-immigrant
politics plays the same role. In either case, European policymakers
disavow responsibility for the misery they bring about.

Repair and redress is owed as much to Black Europeans as it is to former
colonial states. It would mean treating Black Europeans, and all
migrants from the colonized world, as equal participants in European
society. And this form of reparation cannot be perceived as one-off
transactions. Instead, it must be the basis of building an inclusive and
egalitarian Europe.

This is no easy task and will not happen overnight. But we should
remember that just 80 years ago, colonial rule appeared to be a stable
and almost permanent feature of international politics. In just three
decades, anticolonial nationalists had transformed the world's map.

The struggle for racial equality in Europe is a fight for a truly
postcolonial condition, and its creation is implied by each dethroned
statue. If colonialism made the modern world, decolonization cannot be
complete until the world --- including Europe --- is remade.

Adom Getachew (\href{https://twitter.com/adomgetachew}{@adomgetachew})
is a professor of political science at the University of Chicago and the
author of ``Worldmaking After Empire: The Rise and Fall of
Self-Determination.''

\emph{The Times is committed to publishing}
\href{https://www.nytimes3xbfgragh.onion/2019/01/31/opinion/letters/letters-to-editor-new-york-times-women.html}{\emph{a
diversity of letters}} \emph{to the editor. We'd like to hear what you
think about this or any of our articles. Here are some}
\href{https://help.nytimes3xbfgragh.onion/hc/en-us/articles/115014925288-How-to-submit-a-letter-to-the-editor}{\emph{tips}}\emph{.
And here's our email:}
\href{mailto:letters@NYTimes.com}{\emph{letters@NYTimes.com}}\emph{.}

\emph{Follow The New York Times Opinion section on}
\href{https://www.facebookcorewwwi.onion/nytopinion}{\emph{Facebook}}\emph{,}
\href{http://twitter.com/NYTOpinion}{\emph{Twitter (@NYTopinion)}}
\emph{and}
\href{https://www.instagram.com/nytopinion/}{\emph{Instagram}}\emph{.}

Advertisement

\protect\hyperlink{after-bottom}{Continue reading the main story}

\hypertarget{site-index}{%
\subsection{Site Index}\label{site-index}}

\hypertarget{site-information-navigation}{%
\subsection{Site Information
Navigation}\label{site-information-navigation}}

\begin{itemize}
\tightlist
\item
  \href{https://help.nytimes3xbfgragh.onion/hc/en-us/articles/115014792127-Copyright-notice}{©~2020~The
  New York Times Company}
\end{itemize}

\begin{itemize}
\tightlist
\item
  \href{https://www.nytco.com/}{NYTCo}
\item
  \href{https://help.nytimes3xbfgragh.onion/hc/en-us/articles/115015385887-Contact-Us}{Contact
  Us}
\item
  \href{https://www.nytco.com/careers/}{Work with us}
\item
  \href{https://nytmediakit.com/}{Advertise}
\item
  \href{http://www.tbrandstudio.com/}{T Brand Studio}
\item
  \href{https://www.nytimes3xbfgragh.onion/privacy/cookie-policy\#how-do-i-manage-trackers}{Your
  Ad Choices}
\item
  \href{https://www.nytimes3xbfgragh.onion/privacy}{Privacy}
\item
  \href{https://help.nytimes3xbfgragh.onion/hc/en-us/articles/115014893428-Terms-of-service}{Terms
  of Service}
\item
  \href{https://help.nytimes3xbfgragh.onion/hc/en-us/articles/115014893968-Terms-of-sale}{Terms
  of Sale}
\item
  \href{https://spiderbites.nytimes3xbfgragh.onion}{Site Map}
\item
  \href{https://help.nytimes3xbfgragh.onion/hc/en-us}{Help}
\item
  \href{https://www.nytimes3xbfgragh.onion/subscription?campaignId=37WXW}{Subscriptions}
\end{itemize}
