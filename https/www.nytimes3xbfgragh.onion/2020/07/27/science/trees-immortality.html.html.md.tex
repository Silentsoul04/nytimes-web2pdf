Sections

SEARCH

\protect\hyperlink{site-content}{Skip to
content}\protect\hyperlink{site-index}{Skip to site index}

\href{https://www.nytimes3xbfgragh.onion/section/science}{Science}

\href{https://myaccount.nytimes3xbfgragh.onion/auth/login?response_type=cookie\&client_id=vi}{}

\href{https://www.nytimes3xbfgragh.onion/section/todayspaper}{Today's
Paper}

\href{/section/science}{Science}\textbar{}Can Trees Live Forever? New
Kindling for an Immortal Debate

\url{https://nyti.ms/2Eotoho}

\begin{itemize}
\item
\item
\item
\item
\item
\item
\end{itemize}

Advertisement

\protect\hyperlink{after-top}{Continue reading the main story}

Supported by

\protect\hyperlink{after-sponsor}{Continue reading the main story}

Trilobites

\hypertarget{can-trees-live-forever-new-kindling-for-an-immortal-debate}{%
\section{Can Trees Live Forever? New Kindling for an Immortal
Debate}\label{can-trees-live-forever-new-kindling-for-an-immortal-debate}}

Some trees can live for thousands of years, but we may not be around
long enough to really know whether they can die of old age.

\includegraphics{https://static01.graylady3jvrrxbe.onion/images/2020/07/27/science/27TB-IMMORTALTREES1/merlin_174998949_eac0e99d-5dc3-4fe0-be67-64b8419c6538-articleLarge.jpg?quality=75\&auto=webp\&disable=upscale}

By Cara Giaimo

\begin{itemize}
\item
  July 27, 2020
\item
  \begin{itemize}
  \item
  \item
  \item
  \item
  \item
  \item
  \end{itemize}
\end{itemize}

Trees
\href{https://dghaskell.com/2016/02/03/the-tree-that-owns-itself/}{do
not pay taxes}. Some seem to avoid death as well. Many of the world's
most ancient organisms \href{http://www.rmtrr.org/oldlist.htm}{are
trees}, including a 3,600-year-old cypress in Chile and a sacred fig in
Sri Lanka that was planted in the third century B.C. One bristlecone
pine known as Methuselah has been alive
\href{https://www.nytimes3xbfgragh.onion/2003/06/17/science/at-age-4600-plus-methuselah-pine-tree-begets-new-offspring.html}{for
nearly five millenniums}, standing in a forest in what is now called
California.

But according to a paper published Monday in the journal Trends in Plant
Science, time ravages us all in the end. The paper,
``\href{https://doi.org/10.1016/j.tplants.2020.06.006}{Long-Lived Trees
Are Not Immortal},'' argues that even the most venerable trees have
physiological limits --- though we, with our puny life spans, may never
be able to tell.

Sergi Munné-Bosch, a plant biologist at the University of Barcelona,
wrote the article in response to
\href{https://www.nytimes3xbfgragh.onion/2020/01/13/science/oldest-trees-ginkgos.html}{a
January study on ginkgo trees}, which can live for over a thousand
years. \href{https://www.pnas.org/content/117/4/2201}{The study} found
that 600-year-old ginkgos are as reproductively and photosynthetically
vigorous as their 20-year-old peers. Genetic analysis of the trees'
vascular cambium --- a thin layer of cells that lies just underneath the
bark, and creates new living tissue --- showed ``no evidence of
senescence,'' or cell death, the authors wrote.

Dr. Munné-Bosch said he found the paper ``very interesting,'' but
disagreed with how some readers of the study in popular media and beyond
had interpreted it.

``In my opinion at least, there is no immortality,'' he said.

Those tree species that can live for centuries or millenniums have a lot
of tricks for staying youthful. They have simple body plans, and develop
modularly, so they can replace parts they lose. They also build on their
own dead tissue, which provides support and volume at a low metabolic
cost. The trunk of a very old tree might be 95 percent dead, Dr.
Munné-Bosch said, a strategy used also by other plants.

For these reasons, it's much more likely that such a tree will die of
external causes than age-related ones. In some populations, this can
result in ``negative senescence'' --- a phenomenon where the durability
of older trees means they actually have a greater chance of survival
than younger ones, Dr. Munné-Bosch said.

Still, ``everything seems to indicate'' that individual trees are
mortal, he said.

But others have a different take.

``A modular organism such as a tree could hypothetically live forever,''
said Peter Brown, a forest scientist who runs an ancient tree database
called \href{http://www.rmtrr.org/oldlist.htm}{the OldList}. ``I don't
think there is any real physiological or anatomical limitation for them
not to just keep going.''

\includegraphics{https://static01.graylady3jvrrxbe.onion/images/2020/07/27/science/27TB-IMMORTALTREES2/27TB-IMMORTALTREES2-articleLarge.jpg?quality=75\&auto=webp\&disable=upscale}

In practice, though, ``something always comes along'' and interrupts,
whether that's a windstorm, a logging harvester or a swarm of bark
beetles, he said. Many trees on the OldList won the placement lottery,
Dr. Brown said --- they're rooted deep into rocks, hard to get to with
an ax, and far enough from other trees that pests can't spread.

Dr. Munné-Bosch points to some potential limits. For instance, the
vascular tissue that ginkgos produce gets thinner and thinner each year.
At some point, it could become too thin to function, killing the tree,
he said.

Ginkgos also suffer more physiological stress as time goes by, along
with a depleted supply of growth hormone. Despite their miraculous
vascular cambiums, ``it's probable that even ginkgo trees may die from
`natural causes,''' said Richard Dixon, one of the authors of January's
ginkgo paper.

Dr. Brown and Dr. Munné-Bosch agree that the question is almost
impossible to answer experimentally. Very old trees are rare, and the
same tricks that allow for their long-term survival make them hard to
find. (The oldest age group in the ginkgo study contained just three
trees, all younger than 700.) So it's difficult to design a
comprehensive study on them.

Plus, our own life spans are simply too short. Even if a scientist
dedicated her whole career to very old trees, she would be able to
follow her research subjects for only a small percentage of their lives.
And a long enough multigenerational study might see its own methods go
obsolete.

For these reasons, Dr. Munné-Bosch thinks ``we will never prove''
whether long-lived trees experience senescence, he said. So in his own
experimental work, he now focuses on shrubs with more manageable life
spans, of around 30 years.

``I think at the end,'' he said, ``we have to accept that we will all
die.''

Advertisement

\protect\hyperlink{after-bottom}{Continue reading the main story}

\hypertarget{site-index}{%
\subsection{Site Index}\label{site-index}}

\hypertarget{site-information-navigation}{%
\subsection{Site Information
Navigation}\label{site-information-navigation}}

\begin{itemize}
\tightlist
\item
  \href{https://help.nytimes3xbfgragh.onion/hc/en-us/articles/115014792127-Copyright-notice}{©~2020~The
  New York Times Company}
\end{itemize}

\begin{itemize}
\tightlist
\item
  \href{https://www.nytco.com/}{NYTCo}
\item
  \href{https://help.nytimes3xbfgragh.onion/hc/en-us/articles/115015385887-Contact-Us}{Contact
  Us}
\item
  \href{https://www.nytco.com/careers/}{Work with us}
\item
  \href{https://nytmediakit.com/}{Advertise}
\item
  \href{http://www.tbrandstudio.com/}{T Brand Studio}
\item
  \href{https://www.nytimes3xbfgragh.onion/privacy/cookie-policy\#how-do-i-manage-trackers}{Your
  Ad Choices}
\item
  \href{https://www.nytimes3xbfgragh.onion/privacy}{Privacy}
\item
  \href{https://help.nytimes3xbfgragh.onion/hc/en-us/articles/115014893428-Terms-of-service}{Terms
  of Service}
\item
  \href{https://help.nytimes3xbfgragh.onion/hc/en-us/articles/115014893968-Terms-of-sale}{Terms
  of Sale}
\item
  \href{https://spiderbites.nytimes3xbfgragh.onion}{Site Map}
\item
  \href{https://help.nytimes3xbfgragh.onion/hc/en-us}{Help}
\item
  \href{https://www.nytimes3xbfgragh.onion/subscription?campaignId=37WXW}{Subscriptions}
\end{itemize}
