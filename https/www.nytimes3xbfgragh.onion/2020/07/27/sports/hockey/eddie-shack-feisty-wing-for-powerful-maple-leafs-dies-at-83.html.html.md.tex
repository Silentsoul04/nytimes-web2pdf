Sections

SEARCH

\protect\hyperlink{site-content}{Skip to
content}\protect\hyperlink{site-index}{Skip to site index}

\href{https://www.nytimes3xbfgragh.onion/section/sports/hockey}{Hockey}

\href{https://myaccount.nytimes3xbfgragh.onion/auth/login?response_type=cookie\&client_id=vi}{}

\href{https://www.nytimes3xbfgragh.onion/section/todayspaper}{Today's
Paper}

\href{/section/sports/hockey}{Hockey}\textbar{}Eddie Shack, Feisty Wing
for Powerful Maple Leafs, Dies at 83

\url{https://nyti.ms/32ZLz7h}

\begin{itemize}
\item
\item
\item
\item
\item
\end{itemize}

Advertisement

\protect\hyperlink{after-top}{Continue reading the main story}

Supported by

\protect\hyperlink{after-sponsor}{Continue reading the main story}

\hypertarget{eddie-shack-feisty-wing-for-powerful-maple-leafs-dies-at-83}{%
\section{Eddie Shack, Feisty Wing for Powerful Maple Leafs, Dies at
83}\label{eddie-shack-feisty-wing-for-powerful-maple-leafs-dies-at-83}}

He helped Toronto win four Stanley Cups in the '60s, his pugnacity
earning love in Toronto (and, no surprise, hatred in rival Montreal).

\includegraphics{https://static01.graylady3jvrrxbe.onion/images/2020/07/28/obituaries/27Shack-sub/27Shack-sub-articleLarge-v3.jpg?quality=75\&auto=webp\&disable=upscale}

By
\href{https://www.nytimes3xbfgragh.onion/by/richard-goldstein}{Richard
Goldstein}

\begin{itemize}
\item
  July 27, 2020
\item
  \begin{itemize}
  \item
  \item
  \item
  \item
  \item
  \end{itemize}
\end{itemize}

Eddie Shack, a colorful, pugnacious wing who became a fan favorite at
Maple Leaf Gardens playing for Toronto's four Stanley Cup championship
teams of the 1960s, and who remained in the public eye as a pitchman on
Canadian television sporting a cowboy hat and flowing mustache, died on
Saturday in Toronto. He was 83.

Shack's death, at a hospital, was announced by the Maple Leafs. He had
been treated for throat cancer.

Shack played for 17 N.H.L. seasons, with six teams, and appeared in
three All-Star Games. He scored the winning goal for Toronto in Game 5
of the 1963 Stanley Cup finals against Detroit, though he took no
credit, saying the puck had caromed into the Red Wings' net off his
backside.

``He was a powerful skater,'' Darryl Sittler, the captain of the Maple
Leafs in Shack's last season with them, told
\href{https://www.thepeterboroughexaminer.com/ts/sports/leafs/2020/07/26/the-entertainer-eddie-shack-was-everything-leafs-fans-wanted-in-a-hockey-player.html}{The
Peterborough Examiner of Ontario}. ``His body was thick. His forearms.
And he could score goals.''

Shack whipped up the home crowds, at times with help from family.

``If I wasn't playing, my dad would stand up across from the Leafs'
bench and start yelling, `We want Shack!''' he told
\href{https://torontosun.com/sports/hockey/leafs-eddie-shack-the-entertainer-dies-at-83}{The
Toronto Sun}. ``Then I'd stand up on the end of the bench on my side and
start the cheer too.''

Shack was 6 feet 1 inches tall and 200 pounds or so, good size for a
forward of his time.

His roughhousing inspired the 1966 novelty song
``\href{https://www.youtube.com/watch?v=kb8_Ybbrn5A}{Clear the Track:
Here Comes Shack},'' written by the hockey broadcaster Brian McFarlane
and sung by Douglas Rankine with The Secrets.

As the tune put it: ``He knocks 'em down and he gives 'em a whack.''

Shack was known as the Entertainer for his exuberance and also as the
Nose, for his prominent one.

In his memoir ``Eddie Shack: Hockey's Most Entertaining Stories''
(2019), written with Ken Reid, Shack told of head-butting the Montreal
Canadiens' Henri Richard
\href{https://www.nytimes3xbfgragh.onion/2020/03/06/sports/hockey/henri-richard-dead.html}{(who
died in March)}in a game in Toronto, then hearing from his big brother,
\href{https://www.nytimes3xbfgragh.onion/2000/05/28/sports/rocket-richard-montreal-s-goal-scoring-hero-dies-at-78.html}{Maurice
the Rocket}, when the Leafs were in Montreal. ``Maurice said, `Thank God
you never hit him with your nose or you would have split my brother in
two.'''

After his playing days, Shack hawked, among other things, Schick razor
blades (he once shaved off his mustache in a commercial) and The Pop
Shoppe soft drinks, saying he had ``a nose for value.'' He was co-owner
of a golf club and established the Eddie Shack Donuts chain.

\includegraphics{https://static01.graylady3jvrrxbe.onion/images/2020/07/28/obituaries/27Shack2/merlin_174973449_eb49600f-babd-431e-8fb8-0c4a33fd64e6-articleLarge.jpg?quality=75\&auto=webp\&disable=upscale}

Edward Steven Philip Shack was born in Sudbury, Ontario, on Feb. 11,
1937, to Bill and Lena Shack, Ukrainian immigrants. His father was a
crane operator. Eddie was ill for several years as a boy and missed
school but was promoted from one grade to the next nonetheless, leaving
him unable to read and write. He quit school at 15 and was signed by the
Guelph Biltmores, a New York Rangers junior team.

Shack made his N.H.L. debut in the 1958-59 season with the Rangers, who
traded him to the Maple Leafs in November 1960. He played for their
Stanley Cup championship teams in 1962, '63, '64 and '67.

He was later with the Boston Bruins, the Los Angeles Kings, the Buffalo
Sabres, the Pittsburgh Penguins and the Leafs again. He retired after
the 1974-75 season with career totals of 239 goals, 226 assists and
1,431 penalty minutes in 1,047 regular-season games. He was in the
N.H.L.'s top 10 in penalty minutes four times.

Though he was illiterate, Shack proved astute in drawing on his
popularity for commercial opportunities. ``He once told me, `I can't
read or write but I can count,''' his former Maple Leafs teammate Dick
Duff told The Globe and Mail of Toronto. In his later years Shack was an
advocate for literacy programs in schools.

He and his wife, Norma (Givens) Shack, who survives him, had two
children. A complete list of survivors was not immediately available.

For fans of the Canadiens, the Maple Leafs' chief rival, Shack was a
villain, as Quebec's premier, François Legault, acknowledged wryly in
paying tribute to him in a Twitter posting on his death.

He saluted Shack ``with the sentiments of many fans and teams for which
he didn't play,'' Mr. Legault said, adding, ``We loved to hate him.''

Advertisement

\protect\hyperlink{after-bottom}{Continue reading the main story}

\hypertarget{site-index}{%
\subsection{Site Index}\label{site-index}}

\hypertarget{site-information-navigation}{%
\subsection{Site Information
Navigation}\label{site-information-navigation}}

\begin{itemize}
\tightlist
\item
  \href{https://help.nytimes3xbfgragh.onion/hc/en-us/articles/115014792127-Copyright-notice}{©~2020~The
  New York Times Company}
\end{itemize}

\begin{itemize}
\tightlist
\item
  \href{https://www.nytco.com/}{NYTCo}
\item
  \href{https://help.nytimes3xbfgragh.onion/hc/en-us/articles/115015385887-Contact-Us}{Contact
  Us}
\item
  \href{https://www.nytco.com/careers/}{Work with us}
\item
  \href{https://nytmediakit.com/}{Advertise}
\item
  \href{http://www.tbrandstudio.com/}{T Brand Studio}
\item
  \href{https://www.nytimes3xbfgragh.onion/privacy/cookie-policy\#how-do-i-manage-trackers}{Your
  Ad Choices}
\item
  \href{https://www.nytimes3xbfgragh.onion/privacy}{Privacy}
\item
  \href{https://help.nytimes3xbfgragh.onion/hc/en-us/articles/115014893428-Terms-of-service}{Terms
  of Service}
\item
  \href{https://help.nytimes3xbfgragh.onion/hc/en-us/articles/115014893968-Terms-of-sale}{Terms
  of Sale}
\item
  \href{https://spiderbites.nytimes3xbfgragh.onion}{Site Map}
\item
  \href{https://help.nytimes3xbfgragh.onion/hc/en-us}{Help}
\item
  \href{https://www.nytimes3xbfgragh.onion/subscription?campaignId=37WXW}{Subscriptions}
\end{itemize}
