\href{/section/food}{Food}\textbar{}How to Open a Top-Tier Restaurant in
a Pandemic? Rethink Everything

\url{https://nyti.ms/3ff8X3j}

\begin{itemize}
\item
\item
\item
\item
\item
\item
\end{itemize}

\href{https://www.nytimes3xbfgragh.onion/spotlight/at-home?action=click\&pgtype=Article\&state=default\&region=TOP_BANNER\&context=at_home_menu}{At
Home}

\begin{itemize}
\tightlist
\item
  \href{https://www.nytimes3xbfgragh.onion/2020/07/28/books/time-for-a-literary-road-trip.html?action=click\&pgtype=Article\&state=default\&region=TOP_BANNER\&context=at_home_menu}{Take:
  A Literary Road Trip}
\item
  \href{https://www.nytimes3xbfgragh.onion/2020/07/29/magazine/bored-with-your-home-cooking-some-smoky-eggplant-will-fix-that.html?action=click\&pgtype=Article\&state=default\&region=TOP_BANNER\&context=at_home_menu}{Cook:
  Smoky Eggplant}
\item
  \href{https://www.nytimes3xbfgragh.onion/2020/07/27/travel/moose-michigan-isle-royale.html?action=click\&pgtype=Article\&state=default\&region=TOP_BANNER\&context=at_home_menu}{Look
  Out: For Moose}
\item
  \href{https://www.nytimes3xbfgragh.onion/interactive/2020/at-home/even-more-reporters-editors-diaries-lists-recommendations.html?action=click\&pgtype=Article\&state=default\&region=TOP_BANNER\&context=at_home_menu}{Explore:
  Reporters' Obsessions}
\end{itemize}

\includegraphics{https://static01.graylady3jvrrxbe.onion/images/2020/07/29/dining/29virus-ever1/29virus-ever1-articleLarge.jpg?quality=75\&auto=webp\&disable=upscale}

Sections

\protect\hyperlink{site-content}{Skip to
content}\protect\hyperlink{site-index}{Skip to site index}

The Great Read

\hypertarget{how-to-open-a-top-tier-restaurant-in-a-pandemic-rethink-everything}{%
\section{How to Open a Top-Tier Restaurant in a Pandemic? Rethink
Everything}\label{how-to-open-a-top-tier-restaurant-in-a-pandemic-rethink-everything}}

This one, Ever, debuts Tuesday in Chicago with a pricey tasting menu and
a world-class chef. But on the road to its opening, lofty visions have
met hard reality.

Curtis Duffy, the chef and a co-owner of Ever, with a display of
ingredients that will greet guests when the restaurant opens Tuesday
night in Chicago.Credit...Joshua Lott for The New York Times

Supported by

\protect\hyperlink{after-sponsor}{Continue reading the main story}

By Mark Caro

\begin{itemize}
\item
  July 28, 2020
\item
  \begin{itemize}
  \item
  \item
  \item
  \item
  \item
  \item
  \end{itemize}
\end{itemize}

CHICAGO --- The chef de cuisine was looking up mask prices online as the
team at \href{https://www.ever-restaurant.com/}{Ever} discussed how to
approach guests entering the new restaurant.

Under normal circumstances this would not be an issue. These customers
would be paying at least \$285 a head to be transported by peerless
hospitality and the creativity of the chef Curtis Duffy, who is making
\href{https://www.nytimes3xbfgragh.onion/2019/06/24/dining/ever-restaurant-chicago-curtis-duffy.html}{his
long-awaited return} to Chicago's high-end dining scene after he and his
longtime business partner, Michael Muser,
\href{https://www.nytimes3xbfgragh.onion/2017/12/20/dining/grace-chicago-chefs-quit.html}{abruptly
departed} Grace in 2017.

Yet following a state recommendation, the restaurant would require
diners to wear masks except when seated at their tables. So the team was
contemplating providing each arriving diner with a tote bag containing
Ever-branded personal protective equipment such as masks, gloves and
hand sanitizer.

Days later, Mr. Muser reconsidered, and not just because the apparent
\$10-per-guest cost felt significant and most people carry their own
masks now. If diners are supposed to ``get lost in a world of food and
wine,'' he said, then maybe the restaurant shouldn't greet them with:
```Here's your first-aid survival kit. Don't die. Enjoy your dinner!''

In the best of times, building any restaurant from scratch is a fraught
venture. But Ever is a hugely ambitious, high-concept dining room with
an elaborate tasting menu, set to open Tuesday night in the thick of a
public-health crisis that for many people has made the very notion of
fine dining seem alien and scary.

As a result, preparations for the opening have become a series of lofty
visions and constant, real-world revisions.

``If the world has taught us anything in the past three months, `` Mr.
Muser said, ``it's that anything can happen.''

\includegraphics{https://static01.graylady3jvrrxbe.onion/images/2020/07/29/dining/29virus-ever3/merlin_174519840_4e7c96d6-875d-4465-8cb4-15aff58b9bce-articleLarge.jpg?quality=75\&auto=webp\&disable=upscale}

Image

In an appetizer of caviar and king crab, the restaurant's logo is made
from a roasted coconut pudding and then frozen so it will keep its shape
when set into a cucumber gel.Credit...Joshua Lott for The New York Times

Grace received three Michelin stars for four consecutive years before
Mr. Duffy and Mr. Muser left in a clash with the owner. In June 2019,
after waiting for a noncompete clause to expire, the two announced plans
to open Ever in a new office building in the surging West Loop district.
Mr. Muser estimated that backers have spent close to \$5 million ``to
deliver to the city of Chicago the greatest dining room it's ever
seen.''

Its layered plaster walls give the impression of an eroding canyon that
leads into a modern room punctuated by vertical wood slats, sliding
panels and widely spaced tables. In mid-March, though, with about 60
percent of construction done, Gov. J.B. Pritzker issued a
shelter-in-place order that would prohibit restaurant dining for months
in Illinois. When, weeks later, the two partners set a July 28 opening
date, it was a big roll of the dice.

On June 11, Ever emailed its mailing list to announce that two months'
worth of reservations were now on sale
\href{https://www.exploretock.com/ever/}{via the restaurant-reservation
platform Tock}. Chicago restaurants had been serving outdoor meals for
just eight days, and Mayor Lori Lightfoot and Governor Pritzker had yet
to announce a date when indoor dining could start. There was also no
guarantee that a Covid-19 resurgence wouldn't prompt another shutdown.

When the city did begin to allow indoor service, on June 26, it capped
capacity at 25 percent. Ever cut its seating plan to 40 guests a night,
not using all of the dining room's 14 tables at once, but still: Would
enough people feel comfortable spending \$285 or more for a 10-course
menu that would keep them indoors for about two hours? Is this kind of
luxury dining still appealing and viable in the pandemic?

Ever might be the test case.
\href{https://www.alinearestaurant.com/}{Alinea}, currently the city's
only restaurant with three Michelin stars, has been serving takeout
comfort food since March, and has made no plans to reopen its Lincoln
Park dining room.
(\href{https://www.exploretock.com/alinea/experience/156965/air-alinea-in-residence?date=2020-07-31\&size=4\&time=20\%3A00}{AIR:
Alinea in Residence}, an outdoor pop-up on a West Loop rooftop, opened
July 1, though it
\href{https://blockclubchicago.org/2020/07/11/alinea-pop-up-closed-after-staffer-tests-positive-for-coronavirus-days-after-covid-19-canape-backlash/}{closed
for three days} after an employee, who had not worked in several days,
reported testing positive for the coronavirus. AIR reopened after
co-workers tested negative.)

Nick Kokonas, a co-owner of Alinea and the chief executive of Tock, said
the Ever partners ``probably don't have much of a choice as to whether
or not to open.''

``They started raising money and building this out before the crisis
hit,'' he said. ``At some point if they don't attempt to open, the
financial obligations will be weighty enough that they cannot open.''
(Mr. Muser agreed with that assessment.)

John Shields, executive chef at the acclaimed tasting-menu restaurant
\href{https://www.smythandtheloyalist.com/smyth/}{Smyth}, noted that for
now, Ever also won't be able to count on Grace's out-of-town clientele.
``I don't envy them for trying to get started under this,'' he said.

\begin{center}\rule{0.5\linewidth}{\linethickness}\end{center}

Image

Michael Muser, Ever's co-owner and director of operations, has come to
expect constant change. ``I feel like we're all just waiting for the
next shoe to drop,'' he said.Credit...Joshua Lott for The New York Times

No wonder Mr. Muser was nervous when, at 6:30 a.m. on June 11, he
announced the sale of tickets. By 9:15 a.m., all of the available
two-tops were sold out, and tables for four weren't far behind.

``I don't think any one of us could have predicted that people were
going to want to buy reservations that fast,'' Amy Cordell, the general
manager, said as she and Mr. Muser, wearing masks in Ever's windowless
office, watched the tickets sell on her computer screen.

``We opened a restaurant today,'' Mr. Muser crowed.

But any feeling of triumph soon gave way to awareness that the clock was
ticking. The team was behind in so many areas, like hiring; they had
just lost two captains for travel and family reasons related to
Covid-19. Even obtaining staff uniforms had become a headache; the store
that supplied Grace had closed, so Ms. Cordell had to track down a
uniform manufacturer online and collect the workers' measurements
herself.

Those were the minor problems.

``We would be idiots, foolish, silly, ungrateful not to look at what
just happened with our reservations and be anything but humbled by it,''
Mr. Muser said. ``But inside my brain and my heart, I'm like every other
American on the planet. I feel like we're all just waiting for the next
shoe to drop.''

Mr. Muser, 46, is the talker of the Ever partners. Mr. Duffy, 45,
prefers to express himself through his cooking, though his intense glare
also gets his message across.

Image

From left, Mr. Duffy, the sous-chef Richie Farina, the chef de cuisine
Justin Selk and the chef de partie Chris Sullivan prepared an amuse
bouche last week for the first friends-and-family dinner.Credit...Joshua
Lott for The New York Times

The chef has three Michelin stars tattooed on his right hand, and he
aims to recapture them at Ever. But a week into July, he was still
conceptualizing dishes on paper because he couldn't get into his kitchen
amid the cacophony of behind-schedule work.

``I just want the construction to get the hell out,'' he said,
glowering.

Uncertainty over the availability of ingredients also delayed Mr.
Duffy's menu development, though the chef is prone to improvising.

``Is there less out there?'' he said. ``Yes. Doesn't mean we can't work
around it. If we can't get fennel right now, OK, fine, we'll find
something else.''

As it turned out, fennel would wind up on a dish with lamb loin and lamb
tongue, braised sunflower seeds, grapefruit sections and mâche in a
puddle of coffee gastrique.

In a conference room on the building's ninth floor, Mr. Duffy and his
chef de cuisine, Justin Selk, brainstormed an array of canapés to serve
guests before they reached their seats. The idea was that the diners
would pause in a corner to enjoy bites of food that hung from the
ceiling on clips and sat on little ledges built into the wall.

But complications loomed. ``They're wearing their mask, they get to the
corner --- that's going to involve them removing their mask to enjoy
whatever these bites are,'' Mr. Selk said. He suggested that towels be
handy.

The two chefs imagined what kind of vessels might contain these small
bites: spoons, shot glasses, hand bowls, corks, bark, cocktail pins,
leaves, branches, a raspberry bush, dry ice or even a bed of sodium
hexafluoride gas. Then they considered the food possibilities: Ibérico
ham, fruit leather, cured fish, cold soup, bubble tea, tapioca chips,
various crackers, savory meringues, frozen beets.

Image

Mr. Duffy had hoped that arriving guests could grab bites from clips
hanging from a hallway-corner ceiling, but safety concerns led him to
shift to this look-but-don't-eat display of ingredients.Credit...Joshua
Lott for The New York Times

``They were going to be hung for a half-hour window, with multiple
tables coming in,'' Mr. Duffy said. ``But with this whole Covid thing, I
don't know how many people are going to get weirded out by food that's
hanging there, knowing that someone else had walked through that. They
might be weirded out just at the idea of hanging food anyway.''

By the time the full staff of 33 assembled for orientation on July 13,
that last instinct had prevailed, and the bites had given way to ``more
of an art installation,'' as Mr. Duffy put it. Now the ceiling and wall
would showcase 75 ingredients to be featured in the coming meal.

``As opposed to eating it,'' he said, ``now you get to look at it.''

There may be
\href{https://www.nytimes3xbfgragh.onion/2020/07/28/dining/melbourne-restaurants-coronavirus.html}{some
advantages} to opening this sort of restaurant during a pandemic. The
tasting-menu format, coupled with the advance sale of tickets, removes
the uncertainties involved in food ordering, and the check average is
guaranteed to be high. The attention to detail and highly ritualized
style of service can help keep safety measures on track.

Ever is also able to start with a scaled-back work force rather than
trying to support a staff hired in more flush times. By offering just
one tasting menu instead of the two originally planned, Mr. Duffy can
employ fewer cooks than at Grace, though the sudden departure of two
last week left the kitchen short-handed, with four chefs and seven line
cooks just days before opening.

Kevin Boehm, a co-chief executive of the local
\href{https://www.bokagrp.com/}{Boka Restaurant Group} (which includes
Stephanie Izard's Girl \& the Goat and Lee Wolen's Boka), said Ever's
``silver lining'' is that the shutdown didn't kill its momentum.

``It's fortunate for them that they didn't open and then have to close
down,'' Mr. Boehm said. ``I love Curtis and Michael, so I root for them
as human beings. They're super-talented, and I think it's good for our
city to have another restaurant at that level.''

Mr. Muser's orientation-day speech to the staff hit safety concerns hard
and repeatedly. He urged everyone to stay away from other people when
not at the restaurant, saying he didn't want them to get sick or Ever to
be shut down. Each employee arriving at the restaurant must submit to a
temperature check, sign a health questionnaire and wear a mask at all
times.

Image

Members of Ever's front-of-the-house staff were briefed on the evening's
service and guests before the first friends-and-family night, last
Thursday.Credit...Joshua Lott for The New York Times

Image

Ever's scallop dish is plated with mango in three forms, a lardo veil
and the herb hoja santa.Credit...Joshua Lott for The New York Times

Image

Jessica and Doug Streicher were among Ever's first diners, at a
friends-and-family night last week.Credit...Joshua Lott for The New York
Times

``We get to come here and be a part of something awesome, and that does
not come free in 2020,'' Mr. Muser told the group. ``It comes at a
cost.''

Illinois has experienced nearly a week with more than 1,200 new
coronavirus cases a day, and
\href{https://slack-redir.net/link?url=https\%3A\%2F\%2Fwww.nytimes3xbfgragh.onion\%2Finteractive\%2F2020\%2Fus\%2Fcoronavirus-us-cases.html\%23states}{case
numbers are surging} in most of the United States. Hours after Mr.
Muser's speech, Gov. Gavin Newsom shut down indoor dining rooms and bars
across California. New York City had already postponed its return to
indoor dining, and last week Mayor Lightfoot announced that Chicago bars
that didn't serve food could not sell alcohol indoors.

``I don't read the news anymore,'' Mr. Duffy said. ``I deleted it from
my phone.''

Mr. Muser can't wrap his head around the possibility of a future
shutdown of indoor dining. ``It just removes all the joy from my
profession,'' he said. ``There's no such thing as three-Michelin-star
to-go anything.''

He checks the virus numbers every morning, then plows ahead, because
there can be no hesitation when you're in the final stretch of opening a
restaurant. Most of Ever's reservations through the end of September
have been sold.

Last Tuesday, the kitchen plated the menu for him and the sommeliers
Jessica Dennis and Ryan Rickelman so they could decide which wines to
pair with, say, caviar and king crab nestled into a cucumber gel in
which the Ever logo has been embedded with roasted coconut pudding.

Late Thursday afternoon, Mr. Selk and Chris Sullivan, a line cook, were
hanging dehydrated foods --- among them slices of dragonfruit, a maitake
mushroom, a Fresno pepper and herbs in rice paper --- in the hallway
corner as the team prepared for the first of three nights of
friends-and-family diners.

Mr. Muser was putting the finishing touches on his elegant solution to
the P.P.E. question. He had hired the same craftsman who designed the
walls to create a matching sculpturelike table that would sit inside the
entryway to hold masks and hand sanitizer.

``It's a \$6,000 problem solver,'' Mr. Muser said with a rueful smile.
He called it ``the Covid table.''

The automatic sanitizer dispenser wound up spewing so much goop onto the
first guests' hands that it dripped onto the floor. Mr. Muser made a
note. One more problem to solve.

\begin{center}\rule{0.5\linewidth}{\linethickness}\end{center}

Ever, 1340 West Fulton Street, Chicago;
\href{https://www.ever-restaurant.com/}{ever-restaurant.com};
reservations,
\href{https://www.exploretock.com/ever}{exploretock.com/ever}.

\emph{Follow} \href{https://twitter.com/nytfood}{\emph{NYT Food on
Twitter}} \emph{and}
\href{https://www.instagram.com/nytcooking/}{\emph{NYT Cooking on
Instagram}}\emph{,}
\href{https://www.facebookcorewwwi.onion/nytcooking/}{\emph{Facebook}}\emph{,}
\href{https://www.youtube.com/nytcooking}{\emph{YouTube}} \emph{and}
\href{https://www.pinterest.com/nytcooking/}{\emph{Pinterest}}\emph{.}
\href{https://www.nytimes3xbfgragh.onion/newsletters/cooking}{\emph{Get
regular updates from NYT Cooking, with recipe suggestions, cooking tips
and shopping advice}}\emph{.}

Advertisement

\protect\hyperlink{after-bottom}{Continue reading the main story}

\hypertarget{site-index}{%
\subsection{Site Index}\label{site-index}}

\hypertarget{site-information-navigation}{%
\subsection{Site Information
Navigation}\label{site-information-navigation}}

\begin{itemize}
\tightlist
\item
  \href{https://help.nytimes3xbfgragh.onion/hc/en-us/articles/115014792127-Copyright-notice}{©~2020~The
  New York Times Company}
\end{itemize}

\begin{itemize}
\tightlist
\item
  \href{https://www.nytco.com/}{NYTCo}
\item
  \href{https://help.nytimes3xbfgragh.onion/hc/en-us/articles/115015385887-Contact-Us}{Contact
  Us}
\item
  \href{https://www.nytco.com/careers/}{Work with us}
\item
  \href{https://nytmediakit.com/}{Advertise}
\item
  \href{http://www.tbrandstudio.com/}{T Brand Studio}
\item
  \href{https://www.nytimes3xbfgragh.onion/privacy/cookie-policy\#how-do-i-manage-trackers}{Your
  Ad Choices}
\item
  \href{https://www.nytimes3xbfgragh.onion/privacy}{Privacy}
\item
  \href{https://help.nytimes3xbfgragh.onion/hc/en-us/articles/115014893428-Terms-of-service}{Terms
  of Service}
\item
  \href{https://help.nytimes3xbfgragh.onion/hc/en-us/articles/115014893968-Terms-of-sale}{Terms
  of Sale}
\item
  \href{https://spiderbites.nytimes3xbfgragh.onion}{Site Map}
\item
  \href{https://help.nytimes3xbfgragh.onion/hc/en-us}{Help}
\item
  \href{https://www.nytimes3xbfgragh.onion/subscription?campaignId=37WXW}{Subscriptions}
\end{itemize}
