Sections

SEARCH

\protect\hyperlink{site-content}{Skip to
content}\protect\hyperlink{site-index}{Skip to site index}

\href{https://myaccount.nytimes3xbfgragh.onion/auth/login?response_type=cookie\&client_id=vi}{}

\href{https://www.nytimes3xbfgragh.onion/section/todayspaper}{Today's
Paper}

\href{/section/opinion}{Opinion}\textbar{}Being a Grandma During a
Pandemic

\url{https://nyti.ms/3jNLA42}

\begin{itemize}
\item
\item
\item
\item
\item
\end{itemize}

Advertisement

\protect\hyperlink{after-top}{Continue reading the main story}

\href{/section/opinion}{Opinion}

Supported by

\protect\hyperlink{after-sponsor}{Continue reading the main story}

letterS

\hypertarget{being-a-grandma-during-a-pandemic}{%
\section{Being a Grandma During a
Pandemic}\label{being-a-grandma-during-a-pandemic}}

Grandmothers agonize over whether it's safe to attend a grandson's
wedding and relate the joys of sharing space and a journal with
grandchildren.

July 28, 2020

\begin{itemize}
\item
\item
\item
\item
\item
\end{itemize}

\includegraphics{https://static01.graylady3jvrrxbe.onion/images/2020/07/20/opinion/20egan/17egan-articleLarge.jpg?quality=75\&auto=webp\&disable=upscale}

\textbf{To the Editor:}

I was having a heated discussion recently with my daughter about
exposing myself to the ``outside.'' She and her siblings criticize me
for, say, going to the supermarket or, worse, the hairdresser. She, in
her 50s, feels that I, in my 80s, should be more restrictive.

My feeling is that I live in a comparatively safe area, am in good
health --- I even teach yoga, on the lawn --- and take the necessary
precautions, so I can allow myself more leeway. The isolation since I've
become a widow, and with my family living very far away, is hard.

So now I have my grandson's wedding next month to consider, a small,
downsized wedding. Of course I don't want to get sick, go to a hospital
or die. But I've had four largely solitary months to think about this.
Concluding our talk, I told my daughter, without realizing the irony,
``Darling, I'm going, because there's more to life than just living!''

Sondra Markim\\
Piermont, N.Y.

\textbf{To the Editor:}

Re
``\href{https://www.nytimes3xbfgragh.onion/2020/07/17/opinion/coronavirus-family.html}{Our
Life Was Languid. Then My Daughter's Family Moved In}'' (column, July
18):

How I so agree with Timothy Egan's essay about three generations living
under one roof, as I'm living it. Two weeks after New York went into
quarantine my son, daughter-in-law and granddaughter from Manhattan
moved in and are still here.

Our living room became a playroom and classroom as my granddaughter
Zoomed her preschool program and ballet school. Cooking for five became
the norm, and we celebrated my granddaughter's third birthday and my
son's birthday by teaching her how to bake. Everyone pitches in on
cleaning up and doing laundry, and throughout this pandemic we have
learned to navigate and give each of us enough space to keep our sanity.

I will miss them when they go back to their apartment at the end of the
summer, but I'll miss more a little girl every morning calling out my
name to take her out of bed. It has been the worst of times, but it has
been the best of times as well.

Viviane Kovacs\\
Searingtown, N.Y.

\textbf{To the Editor:}

Re
``\href{https://www.nytimes3xbfgragh.onion/2020/07/14/well/family/children-coronavirus-stories.html}{Telling
Children the Story of the Pandemic}'' (Science Times, July 21):

Dr. Perri Klass's article rang true for me. As we began to face the
Covid-19 lockdown, I realized how little my grandparents had told us
about their experiences in the 1918 flu pandemic. I determined that my
five grandchildren would have a tangible record of their experience that
they could share with their children and grandchildren.

I set up a WhatsApp group with my children who live in Antalya, Turkey,
and Larchmont, N.Y., so we could all share pictures and thoughts. I was
impressed with the creativity and humor the parents were imparting to
their children, ages 6 to 15. I began to create a photo journal of their
activities, mixed with scenes of my own walks with my husband in Central
Park, where the field hospital was a sobering addition. I dedicated the
book to my children and their amazing spouses.

I am just finishing Volume I now, and I hope that Volume II will be a
short one with a happy ending. The act of assembling the book has
provided me with hours of joyful distraction to all the heartbreaking
news.

Susan Todd Johnson\\
New York

Advertisement

\protect\hyperlink{after-bottom}{Continue reading the main story}

\hypertarget{site-index}{%
\subsection{Site Index}\label{site-index}}

\hypertarget{site-information-navigation}{%
\subsection{Site Information
Navigation}\label{site-information-navigation}}

\begin{itemize}
\tightlist
\item
  \href{https://help.nytimes3xbfgragh.onion/hc/en-us/articles/115014792127-Copyright-notice}{©~2020~The
  New York Times Company}
\end{itemize}

\begin{itemize}
\tightlist
\item
  \href{https://www.nytco.com/}{NYTCo}
\item
  \href{https://help.nytimes3xbfgragh.onion/hc/en-us/articles/115015385887-Contact-Us}{Contact
  Us}
\item
  \href{https://www.nytco.com/careers/}{Work with us}
\item
  \href{https://nytmediakit.com/}{Advertise}
\item
  \href{http://www.tbrandstudio.com/}{T Brand Studio}
\item
  \href{https://www.nytimes3xbfgragh.onion/privacy/cookie-policy\#how-do-i-manage-trackers}{Your
  Ad Choices}
\item
  \href{https://www.nytimes3xbfgragh.onion/privacy}{Privacy}
\item
  \href{https://help.nytimes3xbfgragh.onion/hc/en-us/articles/115014893428-Terms-of-service}{Terms
  of Service}
\item
  \href{https://help.nytimes3xbfgragh.onion/hc/en-us/articles/115014893968-Terms-of-sale}{Terms
  of Sale}
\item
  \href{https://spiderbites.nytimes3xbfgragh.onion}{Site Map}
\item
  \href{https://help.nytimes3xbfgragh.onion/hc/en-us}{Help}
\item
  \href{https://www.nytimes3xbfgragh.onion/subscription?campaignId=37WXW}{Subscriptions}
\end{itemize}
