Sections

SEARCH

\protect\hyperlink{site-content}{Skip to
content}\protect\hyperlink{site-index}{Skip to site index}

\href{https://www.nytimes3xbfgragh.onion/podcasts/the-daily}{The Daily}

\href{https://myaccount.nytimes3xbfgragh.onion/auth/login?response_type=cookie\&client_id=vi}{}

\href{https://www.nytimes3xbfgragh.onion/section/todayspaper}{Today's
Paper}

\href{/podcasts/the-daily}{The Daily}\textbar{}Why \$600 Checks Are
Tearing Republicans Apart

\url{https://nyti.ms/3hK64Jm}

\begin{itemize}
\item
\item
\item
\item
\item
\item
\end{itemize}

Advertisement

\protect\hyperlink{after-top}{Continue reading the main story}

transcript

Back to The Daily

bars

0:00/26:13

-26:13

transcript

\hypertarget{why-600-checks-are-tearing-republicans-apart}{%
\subsection{Why \$600 Checks Are Tearing Republicans
Apart}\label{why-600-checks-are-tearing-republicans-apart}}

\hypertarget{hosted-by-michael-barbaro-produced-by-rachel-quester-and-daniel-guillemette-with-help-from-robert-jimison-and-stella-tan-and-edited-by-mj-davis-lin}{%
\subsubsection{Hosted by Michael Barbaro; produced by Rachel Quester and
Daniel Guillemette; with help from Robert Jimison and Stella Tan; and
edited by M.J. Davis
Lin}\label{hosted-by-michael-barbaro-produced-by-rachel-quester-and-daniel-guillemette-with-help-from-robert-jimison-and-stella-tan-and-edited-by-mj-davis-lin}}

\hypertarget{as-republicans-consider-whether-to-extend-weekly-payments-for-those-without-work-during-the-pandemic-the-election-looms-large}{%
\paragraph{As Republicans consider whether to extend weekly payments for
those without work during the pandemic, the election looms
large.}\label{as-republicans-consider-whether-to-extend-weekly-payments-for-those-without-work-during-the-pandemic-the-election-looms-large}}

Tuesday, July 28th, 2020

\begin{itemize}
\item
  {[}music{]}
\item
  michael barbaro\\
  From The New York Times, I'm Michael Barbaro. This is ``The Daily.''

  Today: A fight has erupted among congressional Republicans over how
  long and how generously government should help the unemployed during
  the pandemic. Nick Fandos on what that battle is really about.

  It's Tuesday, July 28.

  Nick, tell me about this deadline coming up on Friday.
\item
  nick fandos\\
  So on Friday, at the end of July, one of the key programs in the \$2
  trillion economic relief package, called the CARES Act, that Congress
  passed this spring to deal with the coronavirus pandemic, is set to
  expire. This is the federal unemployment benefit, this extra \$600
  that the federal government has been putting into unemployment checks,
  on top of whatever states give the tens of millions of Americans that
  are out of work.
\item
  michael barbaro\\
  Right. And the thinking was that state unemployment benefits, which is
  how most people get by when they are laid off, are kind of stingy. And
  because these layoffs were so widespread, the federal government
  needed to step in an unusual way.
\item
  nick fandos\\
  That's right. And you know, \$600 was arrived at by congressional
  Democrats and the Treasury Secretary, Steve Mnuchin, as something like
  a kind of average wage that they thought might be lost across the
  board. And though some Republicans were uneasy ---
\item
  archived recording\\
  Mr. President, the majority leader of the Senate.
\end{itemize}

nick fandos

--- they ultimately set aside their concerns and ended up voting
unanimously to put this program and others in place.

\begin{itemize}
\item
  archived recording (mitch mcconnell)\\
  Our nation needed us to go big and go fast. And they did.

  So today, Mr. President, the Senate will act to help the people of
  this country weather this storm.
\end{itemize}

michael barbaro

Right. And I think for many Americans the sense was that this program
--- \$600 a week from the federal government --- would probably last as
long as widespread unemployment lasted, stemming from the pandemic.

nick fandos

I think that that's right, that that was the assumption of many
Americans. But Republicans never quite viewed it that way.

\begin{itemize}
\tightlist
\item
  archived recording (john cornyn)\\
  We have spent a lot of money in the last couple of months. But we've
  done so in the face of an emergency, kind of like the civilian
  equivalent of World War II.
\end{itemize}

nick fandos

They saw the whole stimulus bill, including this benefit, as a kind of
extraordinary measure for extraordinary circumstances. And that this was
kind of a bridge to float the economy and float the American people
through this period where the government was asking them to stay home,
so that we could get the virus under control.

\begin{itemize}
\tightlist
\item
  archived recording (ted cruz)\\
  Look, I supported every one of these bills that has come through. I
  agree that we need emergency relief to help people, to help people
  through the crisis as a short-term bridge loan.
\end{itemize}

nick fandos

But you know, if that was a gamble --- and it was, that this is going to
be a temporary thing --- Republicans do not come out where they want to.
The virus has resurged in many states now across the South and West, you
know, in states that are traditionally red states and are represented by
Republicans.

\begin{itemize}
\tightlist
\item
  archived recording (mitch mcconnell)\\
  So the question today is where are we? And where do we go from here?
\end{itemize}

nick fandos

And the party now has to kind of come to terms with the fact that what
they hoped would be a bridge is going to be a lot longer than they
initially thought.

\begin{itemize}
\tightlist
\item
  archived recording (mitch mcconnell)\\
  We had hoped we'd be on the way to saying goodbye to this health care
  pandemic. Clearly, it is not over.
\end{itemize}

michael barbaro

Right. Which brings us back to this Friday expiration date. So do
Republicans have intrinsic objections to just renewing the \$600 a week?

nick fandos

So for most Republicans, the answer is yes.

michael barbaro

Hm.

nick fandos

That \$600 figure, as we said, was arrived at honestly, but somewhat
hastily back in March. And Republicans started voicing concerns at the
time.

\begin{itemize}
\tightlist
\item
  archived recording (ted cruz)\\
  For 68 percent of people receiving it right now, they are being paid
  more on unemployment than they made in their job.
\end{itemize}

nick fandos

And they've grown a lot louder since. That \$600 from the federal
government, on top of whatever states were giving people that were out
of work, was simply too generous.

\begin{itemize}
\tightlist
\item
  archived recording (ted cruz)\\
  And I'll tell you, I've spoken to small business owners all over the
  state of Texas who are trying to reopen.
\end{itemize}

nick fandos

And actually was disincentivizing and has disincentivized many Americans
from going back to work.

\begin{itemize}
\tightlist
\item
  archived recording (ted cruz)\\
  --- and they're calling their waiters and waitresses, they're calling
  their busboys. And they won't come back. And of course they won't come
  back. Because the federal government is paying, in some instances,
  twice as much money to stay home.
\end{itemize}

nick fandos

So ideologically, many Republicans in Congress were never comfortable
with this \$600 benefit at that level in the first place. And then,
they're certainly not comfortable with extending it into perpetuity.

michael barbaro

So Nick, with this program running out of time, how is this playing out
among the Republicans?

nick fandos

So as Republicans are approaching these deadlines at the end of July,
they're looking around and seeing a bunch of different inputs that are
really difficult for them. On the one hand, Democrats are, you know,
unabashedly and enthusiastically pushing to extend this \$600 benefit
through the end of the year and as long as it's needed.

michael barbaro

Mhm.

nick fandos

And at the same time, Republicans are having to reconcile themselves to
the fact that the virus is spreading around the country. There are signs
in the last few weeks that the economy, which was recovering, is
starting to potentially soften again. And they recognize for a variety
of reasons --- economically, for the livelihood of the country, and
politically, as they're looking ahead to November's elections --- that
it's simply not going to be an option not to have a plan.

michael barbaro

Mhm.

nick fandos

And so Republicans start trying to put together their own proposal for
how to fix unemployment benefits going forward and a range of other
programs to keep the economy afloat. And it turns out it's a lot harder
than they think it's going to be.

michael barbaro

What do you mean?

nick fandos

Well, it turns out, as they try to unpack this and get into the details
of what might we do next, that there's a pretty big split between two
different camps of Republicans.

\begin{itemize}
\tightlist
\item
  archived recording (ted cruz)\\
  I asked my Republican colleagues, what in the hell are we doing?
\end{itemize}

nick fandos

So one of them are the kind of arch conservatives that are really
worried about federal spending. People like Ted Cruz.

\begin{itemize}
\tightlist
\item
  archived recording (ted cruz)\\
  A number of senators at lunch get up and say, well gosh, we need \$20
  billion for this. We need \$100 billion for this. And they're just
  really eager to spend money. I'm, like, what are you guys doing?
\end{itemize}

nick fandos

Or Rand Paul, who compared his colleagues to a bunch of Bernie bros with
the way they were talking.

\begin{itemize}
\tightlist
\item
  archived recording (rand paul)\\
  I find it extraordinary that I just came from a Republican caucus
  meeting that could be sort of the Bernie bros progressive caucus.
\end{itemize}

nick fandos

And that is a sharp pejorative in the Senate Republican conference.

michael barbaro

I would think.

\begin{itemize}
\tightlist
\item
  archived recording (rand paul)\\
  This is insane. It's got to stop. We're ruining the country. And there
  has to be some voice left for fiscal conservatism in this country.
\end{itemize}

nick fandos

This group is just, frankly, uneasy about the \$2 trillion that they
spent back in the spring and is not interested in seeing the federal
government add to the deficit, add to the debt and further involve
itself in the U.S. economy.

\begin{itemize}
\tightlist
\item
  archived recording (rand paul)\\
  I, for one, am alarmed at where the country is heading. I'm also
  alarmed that my party has forgotten what they actually stand for.
  There is no difference now between the two parties on spending.
\end{itemize}

nick fandos

Now, at the other end of the spectrum are a group of more moderate or
middle-of-the-road Republicans, who are up for re-election this fall and
are actually having to face the voters, in many cases, in swing states
or blue states where President Trump and the Republican response to the
pandemic have been deeply unpopular. People like Cory Gardner or Thom
Tillis ---

\begin{itemize}
\tightlist
\item
  archived recording (thom tillis)\\
  Well, I think we have to build on what we did with the CARES Act,
  almost \$3 trillion dollars to help individuals, to provide a
  supplement for unemployment.
\end{itemize}

nick fandos

--- who have really staked their re-election on the government's
response to this crisis, and on showing that they are effectively
leading the country through one of its most challenging periods in
anybody's memory. And joining with them on that side ---

\begin{itemize}
\tightlist
\item
  archived recording (mitch mcconnell)\\
  This crisis is far from over.
\end{itemize}

nick fandos

--- are some of the best known leaders of the Republican Party on
Capitol Hill.

michael barbaro

Hm.

\begin{itemize}
\tightlist
\item
  archived recording (mitch mcconnell)\\
  For weeks now, I have made it clear that further legislation out of
  the Senate will be a serious response to the crisis.
\end{itemize}

nick fandos

So Mitch McConnell, the majority leader from Kentucky, and John Cornyn,
a Republican from Texas who's one of his longtime deputies ---

\begin{itemize}
\tightlist
\item
  archived recording (john cornyn)\\
  But as the impact of Covid-19 has grown, so has the need for
  assistance.
\end{itemize}

nick fandos

--- seem to recognize that not only are the fates of individual senators
up in the air, but the Republican Party's prospects up and down the
ticket this fall may well be tied up into how they are judged to have
handled this crisis. And doing what the conservatives want and basically
stopping now and saying, ``we've done what we need to do'' is not an
option for that group.

michael barbaro

Nick, how much of that debate you just described is being informed by
the political realities surrounding the single most important person in
the party at this moment, which is President Trump?

nick fandos

I think it's inescapable for elected Republicans. And it's not just the
way that the public seems to be viewing President Trump and giving him
very poor grades on handling the pandemic, which could hurt the whole
Republican Party in November. It's also the kind of erratic nature of
his leadership and engagement on this issue itself. And so they're
working with his Treasury secretary to iron out the details. But this is
not a negotiation that President Trump is leading or even all that
active in. They're trying to do whatever they can to bail out the party,
not to please President Trump in this case.

michael barbaro

Hm.

nick fandos

And that has added another kind of layer of interest and
unpredictability to this whole thing which, you know, we have not seen a
lot of in the last three and a half years.

michael barbaro

And what does that tell you, that they're choosing this moment to do
that?

nick fandos

Well, I think whether they want to acknowledge it or not, Republicans
are starting to sense that their party is really in trouble. That if
things aren't turned around quickly, they may not only lose the White
House, but really get wiped out in November. And are thinking in
different ways about why that is and what the party may need to look
like in a world that's just starting to dawn on them as a possibility of
being kind of post-Trump.

michael barbaro

So in other words, this battle over \$600 a week and what this entire
new version of a relief package looks like, it's not really just about
what's in a piece of legislation like this. It's about the identity of
the Republican Party at a time where it may need a new identity. Because
theoretically, Donald Trump could lose. And the Republican Party would
no longer be just the party of Donald Trump.

nick fandos

That's right. So while they're very much focused on how is the party
going to be viewed in November, they're really kind of foreshadowing or
staking out positioning for this potentially larger battle to come, over
what Republicanism really looks like after Donald Trump has defined it
for four or five years.

{[}music{]}

And you know, some of these folks are not new to their positions. But
they recognize that there may soon be more of a need to kind of assert
their views, and the primacy of those views, against others in the
Republican Party.

michael barbaro

We'll be right back.

{[}music{]}

So Nick, where does this very high stakes ideological battle within the
Republican Party, where does it leave this economic relief package?

nick fandos

So it's up to Mitch McConnell, basically, to try and pull together these
different factions and arrive at a bill that deals with the expiring
unemployment benefits and a host of other kind of programs and
priorities. Basically, to try and reconcile those differences and put
together a bill that can be Republicans' starting point when they go to
the negotiating table with Democrats.

michael barbaro

Mhm.

nick fandos

And so that's where we were by the middle of last week. And as he tries
to work out those details with the White House and run it by his
Republican colleagues, there's a bunch of snafus along the way. They
push past some small deadlines. But in the end, they're unable to
introduce their bill, because those differences turn out to have been
more significant than Republicans even wanted to let on.

michael barbaro

So the Republicans cannot come up with any kind of consensus bill to
salvage this program that we've been talking about?

nick fandos

So as of Thursday morning, no. And as lawmakers head for the exits for
the weekend, without a proposal for how to fix a whole host of programs,
they have not arrived at a solution on a range of issues, including what
to do about this expiring \$600 unemployment benefit. But their staff
and Treasury Secretary Mnuchin, Meadows, the White House chief of staff,
work through the weekend to try and iron out some of these details.

\begin{itemize}
\tightlist
\item
  archived recording (mitch mcconnell)\\
  Well, good afternoon, everyone. The Senate Republicans and the
  administration have been consulting over the last few weeks.
\end{itemize}

nick fandos

By Monday afternoon, what they finally introduce ---

\begin{itemize}
\tightlist
\item
  archived recording (mitch mcconnell)\\
  --- with what we think is an appropriate amount of additional debt to
  be added. We think it is about a trillion dollars.
\end{itemize}

nick fandos

--- is a plan that is roughly a trillion dollars.

\begin{itemize}
\tightlist
\item
  archived recording (mitch mcconnell)\\
  And we've allocated that in a way that we think makes the most sense.
\end{itemize}

nick fandos

Some of that goes to schools to help them reopen and for more testing
and contact tracing.

\begin{itemize}
\tightlist
\item
  archived recording (mitch mcconnell)\\
  So with that, I'm going to call on my colleagues who have developed
  the various ---
\end{itemize}

nick fandos

And on this key question of unemployment benefits, Republicans propose a
real overhaul to the way that they would work conceptually.

\begin{itemize}
\item
  archived recording (mitch mcconnell)\\
  Do we know who's next?
\item
  archived recording\\
  Chairman Grassley.
\item
  archived recording (mitch mcconnell)\\
  Senator Grassley.
\item
  archived recording (chuck grassley)\\
  Number one, we're going to continue ---
\end{itemize}

nick fandos

So they say that for the short term, we're going to cut that \$600 down
to \$200 a week.

michael barbaro

Big cut.

nick fandos

A pretty dramatic cut.

\begin{itemize}
\tightlist
\item
  archived recording (chuck grassley)\\
  So we want to continue to help the unemployed. But we want to
  encourage work. And we've learned a very tough lesson, that when you
  pay people not to work, what do you expect?
\end{itemize}

nick fandos

And they say, that's just going to buy us time over the next few months
for us to basically help states set up a new system, where what we're
going to try and do is make sure that every individual that's
unemployed, between the state government and the federal government ends
up getting about 70 percent of what their old wages would have been.

\begin{itemize}
\tightlist
\item
  archived recording (chuck grassley)\\
  We're going to have further tax relief for businesses to encourage
  hiring and rehiring. And we want to do that to encourage people to get
  back to work and help the employer, in the process, support people in
  the meantime.
\end{itemize}

nick fandos

And so what Republicans are trying to do here is keep a safety net in
place, but remove what they think is hindering people from going back to
work.

\begin{itemize}
\tightlist
\item
  archived recording (chuck grassley)\\
  Lastly, I hope that Democrats will come to the table and we can work
  out a bipartisan agreement. Thank you very much.
\end{itemize}

nick fandos

So in other words, if they can get this program up and operating, it
will always make sense from a financial point of view for somebody to go
and take their old job back or take a new job back, but not be so
draconian that they're making the economic situation drastically worse,
or can be accused of forcing people towards soup kitchens or the
streets.

michael barbaro

So this is a classic compromise. In other words, we're going to keep the
benefits but not at \$600 a week, because they see that as not
conservative and not incentivizing an economic recovery.

nick fandos

That's right. But remember, this is just kind of the first step. This
should have been the easy part for Republicans. Because what they have
coming is negotiations with Democrats, who are in favor of keeping the
benefit totally as it is, and are already lining up to say basically
that Republicans are giving a massive economic financial hit to
individuals and the economy right when they need it most, and at this
moment where the country's recovery seems to be teetering. Is it going
to keep going up? Or is it about to collapse again? And Democrats are
not going to settle for \$200 for any period of time.

michael barbaro

So given all that, what is likely to happen to this Republican bill in
the Senate?

nick fandos

So the interesting thing about where Republicans find themselves is,
this bill that they're introducing probably couldn't even pass the
Senate just on Republican votes. And that leaves them in a pretty weak
position as they head into negotiations with the Democrats. Because
remember, to pass anything into law, even if there's a Republican
president or a Republican Senate, you need the Democrats to get it
through Congress. And they have a very long and expensive wish list of
things that they'd like to see in legislation. And they're not going to
be easy on the Republicans.

michael barbaro

Nick, this may sound like a strange question. But do you think
Republicans now regret ever agreeing to these enhanced unemployment
benefits? I'm mindful of the fact that it was not a Republican idea. It
was Democrats who pushed for it. As you have said, it cuts against a lot
of Republican principles. But they agreed to it as a short-term fix. And
it turns out it's not going to be a short-term term fix, because there's
nothing short-term about this pandemic. And it is inevitably hard to
take something like this away from people once you give it to them. So
is it possible Republicans look back and think we should have never
agreed to do this?

nick fandos

I think there may be a small subset of fiscally conservative Republicans
that feel that way. But my guess is that the vast majority felt like,
hey, we did what we had to do back then in the springtime. I mean, the
economy was in freefall, remember. And the course of the virus was
highly uncertain. And the fundamental problem for them is that they
envisioned the federal government having a relatively short-term role to
play in getting the country back on its feet and ready to fight against
this virus. And it's just turned out to be, for a lot of different
reasons, a much more complicated, prolonged, expensive fight than they
wanted.

And honestly, Michael, at this point, it's hard to see how this
situation resolves itself. Usually, when you cover Congress for a while,
you can kind of see the pattern of how these negotiations will work. But
Republicans really find themselves pretty far up the stream without a
paddle right now. And there seem to be risks for them and consequences
in every direction.

And it's going to be a pretty fascinating next couple of weeks to see
how and if they can reach an agreement with Democrats --- and one that
some members of the party feel like doesn't completely undermine what
they stand for.

michael barbaro

Of course, weeks is not what people who are on this program have. They
have days. Because this thing really does expire on Friday.

nick fandos

That's right. Many of the people receiving these benefits are living
paycheck to paycheck or don't have a lot of savings to fall back on.
There can and will be very real consequences to this delay. And that's
not to mention the whole host of other programs that are being debated
by Congress right now that are touching different aspects of people's
lives.

{[}music{]}

The longer this goes on, the effects just get magnified. Bigger and
bigger and bigger. And it frankly makes the problem even harder to
solve.

michael barbaro

Thank you, Nick.

nick fandos

Thank you, Michael.

michael barbaro

On Monday night, Democratic leaders, including House Speaker Nancy
Pelosi, met with White House officials to begin negotiations over a new
economic relief package, including federal unemployment benefits.

\begin{itemize}
\tightlist
\item
  archived recording (nancy pelosi)\\
  Suffice to say that we hoped that we would be able to reach an
  agreement. We clearly do not have shared values.
\end{itemize}

michael barbaro

Little progress was made during the two-hour session. But afterward, the
Democratic leaders made one thing clear. Congressional Republicans lack
the votes to pass their own bill.

We'll be right back.

Here's what else you need to know today.

On Monday, the pandemic touched the worlds of politics, business and
sports. The Trump administration said that its national security
adviser, Robert O'Brien, had contracted the virus, becoming the most
senior White House official yet to test positive.

Meanwhile, the parent company of Google --- Alphabet --- told employees
that they would not be expected to return to the office until next
summer, suggesting that work-from-home policies will extend well past
the end of the year.

Finally, the Miami Marlins canceled two upcoming baseball games after 12
players and two coaches tested positive for the coronavirus. The
outbreak was disclosed just four days after the beginning of the
baseball season.

\begin{itemize}
\tightlist
\item
  archived recording (dave martinez)\\
  My level of concern went from about an eight to a 12. You know, it
  hits home now that you see half a team get infected and it go from one
  city to another. So ---
\end{itemize}

michael barbaro

During a news conference, the manager of the Washington Nationals
expressed alarm over the news.

\begin{itemize}
\tightlist
\item
  archived recording (dave martinez)\\
  Yeah, I got friends on that Miami team. And it really stinks. Now I'm
  not going to lie. I'm not going to sugarcoat it. Seeing those guys go
  down like that, it's not good for them. It's not good for anybody.
\end{itemize}

michael barbaro

That's it for ``The Daily.'' I'm Michael Barbaro. See you tomorrow.

\href{https://www.nytimes3xbfgragh.onion/column/the-daily}{\includegraphics{https://static01.graylady3jvrrxbe.onion/images/2017/01/29/podcasts/the-daily-album-art/the-daily-album-art-square320-v4.png}The
Daily}Subscribe:

\begin{itemize}
\tightlist
\item
  \href{https://itunes.apple.com/us/podcast/id1200361736}{Apple
  Podcasts}
\item
  \href{https://www.google.com/podcasts?feed=aHR0cHM6Ly9yc3MuYXJ0MTkuY29tL3RoZS1kYWlseQ\%3D\%3D}{Google
  Podcasts}
\end{itemize}

\hypertarget{why-600-checks-are-tearing-republicans-apart-1}{%
\section{Why \$600 Checks Are Tearing Republicans
Apart}\label{why-600-checks-are-tearing-republicans-apart-1}}

\hypertarget{as-republicans-consider-whether-to-extend-weekly-payments-for-those-without-work-during-the-pandemic-the-election-looms-large-1}{%
\subsection{As Republicans consider whether to extend weekly payments
for those without work during the pandemic, the election looms
large.}\label{as-republicans-consider-whether-to-extend-weekly-payments-for-those-without-work-during-the-pandemic-the-election-looms-large-1}}

Hosted by Michael Barbaro; produced by Rachel Quester and Daniel
Guillemette; with help from Robert Jimison and Stella Tan; and edited by
M.J. Davis Lin

Transcript

transcript

Back to The Daily

bars

0:00/26:13

-0:00

transcript

\hypertarget{why-600-checks-are-tearing-republicans-apart-2}{%
\subsection{Why \$600 Checks Are Tearing Republicans
Apart}\label{why-600-checks-are-tearing-republicans-apart-2}}

\hypertarget{hosted-by-michael-barbaro-produced-by-rachel-quester-and-daniel-guillemette-with-help-from-robert-jimison-and-stella-tan-and-edited-by-mj-davis-lin-1}{%
\subsubsection{Hosted by Michael Barbaro; produced by Rachel Quester and
Daniel Guillemette; with help from Robert Jimison and Stella Tan; and
edited by M.J. Davis
Lin}\label{hosted-by-michael-barbaro-produced-by-rachel-quester-and-daniel-guillemette-with-help-from-robert-jimison-and-stella-tan-and-edited-by-mj-davis-lin-1}}

\hypertarget{as-republicans-consider-whether-to-extend-weekly-payments-for-those-without-work-during-the-pandemic-the-election-looms-large-2}{%
\paragraph{As Republicans consider whether to extend weekly payments for
those without work during the pandemic, the election looms
large.}\label{as-republicans-consider-whether-to-extend-weekly-payments-for-those-without-work-during-the-pandemic-the-election-looms-large-2}}

Tuesday, July 28th, 2020

\begin{itemize}
\item
  {[}music{]}
\item
  michael barbaro\\
  From The New York Times, I'm Michael Barbaro. This is ``The Daily.''

  Today: A fight has erupted among congressional Republicans over how
  long and how generously government should help the unemployed during
  the pandemic. Nick Fandos on what that battle is really about.

  It's Tuesday, July 28.

  Nick, tell me about this deadline coming up on Friday.
\item
  nick fandos\\
  So on Friday, at the end of July, one of the key programs in the \$2
  trillion economic relief package, called the CARES Act, that Congress
  passed this spring to deal with the coronavirus pandemic, is set to
  expire. This is the federal unemployment benefit, this extra \$600
  that the federal government has been putting into unemployment checks,
  on top of whatever states give the tens of millions of Americans that
  are out of work.
\item
  michael barbaro\\
  Right. And the thinking was that state unemployment benefits, which is
  how most people get by when they are laid off, are kind of stingy. And
  because these layoffs were so widespread, the federal government
  needed to step in an unusual way.
\item
  nick fandos\\
  That's right. And you know, \$600 was arrived at by congressional
  Democrats and the Treasury Secretary, Steve Mnuchin, as something like
  a kind of average wage that they thought might be lost across the
  board. And though some Republicans were uneasy ---
\item
  archived recording\\
  Mr. President, the majority leader of the Senate.
\end{itemize}

nick fandos

--- they ultimately set aside their concerns and ended up voting
unanimously to put this program and others in place.

\begin{itemize}
\item
  archived recording (mitch mcconnell)\\
  Our nation needed us to go big and go fast. And they did.

  So today, Mr. President, the Senate will act to help the people of
  this country weather this storm.
\end{itemize}

michael barbaro

Right. And I think for many Americans the sense was that this program
--- \$600 a week from the federal government --- would probably last as
long as widespread unemployment lasted, stemming from the pandemic.

nick fandos

I think that that's right, that that was the assumption of many
Americans. But Republicans never quite viewed it that way.

\begin{itemize}
\tightlist
\item
  archived recording (john cornyn)\\
  We have spent a lot of money in the last couple of months. But we've
  done so in the face of an emergency, kind of like the civilian
  equivalent of World War II.
\end{itemize}

nick fandos

They saw the whole stimulus bill, including this benefit, as a kind of
extraordinary measure for extraordinary circumstances. And that this was
kind of a bridge to float the economy and float the American people
through this period where the government was asking them to stay home,
so that we could get the virus under control.

\begin{itemize}
\tightlist
\item
  archived recording (ted cruz)\\
  Look, I supported every one of these bills that has come through. I
  agree that we need emergency relief to help people, to help people
  through the crisis as a short-term bridge loan.
\end{itemize}

nick fandos

But you know, if that was a gamble --- and it was, that this is going to
be a temporary thing --- Republicans do not come out where they want to.
The virus has resurged in many states now across the South and West, you
know, in states that are traditionally red states and are represented by
Republicans.

\begin{itemize}
\tightlist
\item
  archived recording (mitch mcconnell)\\
  So the question today is where are we? And where do we go from here?
\end{itemize}

nick fandos

And the party now has to kind of come to terms with the fact that what
they hoped would be a bridge is going to be a lot longer than they
initially thought.

\begin{itemize}
\tightlist
\item
  archived recording (mitch mcconnell)\\
  We had hoped we'd be on the way to saying goodbye to this health care
  pandemic. Clearly, it is not over.
\end{itemize}

michael barbaro

Right. Which brings us back to this Friday expiration date. So do
Republicans have intrinsic objections to just renewing the \$600 a week?

nick fandos

So for most Republicans, the answer is yes.

michael barbaro

Hm.

nick fandos

That \$600 figure, as we said, was arrived at honestly, but somewhat
hastily back in March. And Republicans started voicing concerns at the
time.

\begin{itemize}
\tightlist
\item
  archived recording (ted cruz)\\
  For 68 percent of people receiving it right now, they are being paid
  more on unemployment than they made in their job.
\end{itemize}

nick fandos

And they've grown a lot louder since. That \$600 from the federal
government, on top of whatever states were giving people that were out
of work, was simply too generous.

\begin{itemize}
\tightlist
\item
  archived recording (ted cruz)\\
  And I'll tell you, I've spoken to small business owners all over the
  state of Texas who are trying to reopen.
\end{itemize}

nick fandos

And actually was disincentivizing and has disincentivized many Americans
from going back to work.

\begin{itemize}
\tightlist
\item
  archived recording (ted cruz)\\
  --- and they're calling their waiters and waitresses, they're calling
  their busboys. And they won't come back. And of course they won't come
  back. Because the federal government is paying, in some instances,
  twice as much money to stay home.
\end{itemize}

nick fandos

So ideologically, many Republicans in Congress were never comfortable
with this \$600 benefit at that level in the first place. And then,
they're certainly not comfortable with extending it into perpetuity.

michael barbaro

So Nick, with this program running out of time, how is this playing out
among the Republicans?

nick fandos

So as Republicans are approaching these deadlines at the end of July,
they're looking around and seeing a bunch of different inputs that are
really difficult for them. On the one hand, Democrats are, you know,
unabashedly and enthusiastically pushing to extend this \$600 benefit
through the end of the year and as long as it's needed.

michael barbaro

Mhm.

nick fandos

And at the same time, Republicans are having to reconcile themselves to
the fact that the virus is spreading around the country. There are signs
in the last few weeks that the economy, which was recovering, is
starting to potentially soften again. And they recognize for a variety
of reasons --- economically, for the livelihood of the country, and
politically, as they're looking ahead to November's elections --- that
it's simply not going to be an option not to have a plan.

michael barbaro

Mhm.

nick fandos

And so Republicans start trying to put together their own proposal for
how to fix unemployment benefits going forward and a range of other
programs to keep the economy afloat. And it turns out it's a lot harder
than they think it's going to be.

michael barbaro

What do you mean?

nick fandos

Well, it turns out, as they try to unpack this and get into the details
of what might we do next, that there's a pretty big split between two
different camps of Republicans.

\begin{itemize}
\tightlist
\item
  archived recording (ted cruz)\\
  I asked my Republican colleagues, what in the hell are we doing?
\end{itemize}

nick fandos

So one of them are the kind of arch conservatives that are really
worried about federal spending. People like Ted Cruz.

\begin{itemize}
\tightlist
\item
  archived recording (ted cruz)\\
  A number of senators at lunch get up and say, well gosh, we need \$20
  billion for this. We need \$100 billion for this. And they're just
  really eager to spend money. I'm, like, what are you guys doing?
\end{itemize}

nick fandos

Or Rand Paul, who compared his colleagues to a bunch of Bernie bros with
the way they were talking.

\begin{itemize}
\tightlist
\item
  archived recording (rand paul)\\
  I find it extraordinary that I just came from a Republican caucus
  meeting that could be sort of the Bernie bros progressive caucus.
\end{itemize}

nick fandos

And that is a sharp pejorative in the Senate Republican conference.

michael barbaro

I would think.

\begin{itemize}
\tightlist
\item
  archived recording (rand paul)\\
  This is insane. It's got to stop. We're ruining the country. And there
  has to be some voice left for fiscal conservatism in this country.
\end{itemize}

nick fandos

This group is just, frankly, uneasy about the \$2 trillion that they
spent back in the spring and is not interested in seeing the federal
government add to the deficit, add to the debt and further involve
itself in the U.S. economy.

\begin{itemize}
\tightlist
\item
  archived recording (rand paul)\\
  I, for one, am alarmed at where the country is heading. I'm also
  alarmed that my party has forgotten what they actually stand for.
  There is no difference now between the two parties on spending.
\end{itemize}

nick fandos

Now, at the other end of the spectrum are a group of more moderate or
middle-of-the-road Republicans, who are up for re-election this fall and
are actually having to face the voters, in many cases, in swing states
or blue states where President Trump and the Republican response to the
pandemic have been deeply unpopular. People like Cory Gardner or Thom
Tillis ---

\begin{itemize}
\tightlist
\item
  archived recording (thom tillis)\\
  Well, I think we have to build on what we did with the CARES Act,
  almost \$3 trillion dollars to help individuals, to provide a
  supplement for unemployment.
\end{itemize}

nick fandos

--- who have really staked their re-election on the government's
response to this crisis, and on showing that they are effectively
leading the country through one of its most challenging periods in
anybody's memory. And joining with them on that side ---

\begin{itemize}
\tightlist
\item
  archived recording (mitch mcconnell)\\
  This crisis is far from over.
\end{itemize}

nick fandos

--- are some of the best known leaders of the Republican Party on
Capitol Hill.

michael barbaro

Hm.

\begin{itemize}
\tightlist
\item
  archived recording (mitch mcconnell)\\
  For weeks now, I have made it clear that further legislation out of
  the Senate will be a serious response to the crisis.
\end{itemize}

nick fandos

So Mitch McConnell, the majority leader from Kentucky, and John Cornyn,
a Republican from Texas who's one of his longtime deputies ---

\begin{itemize}
\tightlist
\item
  archived recording (john cornyn)\\
  But as the impact of Covid-19 has grown, so has the need for
  assistance.
\end{itemize}

nick fandos

--- seem to recognize that not only are the fates of individual senators
up in the air, but the Republican Party's prospects up and down the
ticket this fall may well be tied up into how they are judged to have
handled this crisis. And doing what the conservatives want and basically
stopping now and saying, ``we've done what we need to do'' is not an
option for that group.

michael barbaro

Nick, how much of that debate you just described is being informed by
the political realities surrounding the single most important person in
the party at this moment, which is President Trump?

nick fandos

I think it's inescapable for elected Republicans. And it's not just the
way that the public seems to be viewing President Trump and giving him
very poor grades on handling the pandemic, which could hurt the whole
Republican Party in November. It's also the kind of erratic nature of
his leadership and engagement on this issue itself. And so they're
working with his Treasury secretary to iron out the details. But this is
not a negotiation that President Trump is leading or even all that
active in. They're trying to do whatever they can to bail out the party,
not to please President Trump in this case.

michael barbaro

Hm.

nick fandos

And that has added another kind of layer of interest and
unpredictability to this whole thing which, you know, we have not seen a
lot of in the last three and a half years.

michael barbaro

And what does that tell you, that they're choosing this moment to do
that?

nick fandos

Well, I think whether they want to acknowledge it or not, Republicans
are starting to sense that their party is really in trouble. That if
things aren't turned around quickly, they may not only lose the White
House, but really get wiped out in November. And are thinking in
different ways about why that is and what the party may need to look
like in a world that's just starting to dawn on them as a possibility of
being kind of post-Trump.

michael barbaro

So in other words, this battle over \$600 a week and what this entire
new version of a relief package looks like, it's not really just about
what's in a piece of legislation like this. It's about the identity of
the Republican Party at a time where it may need a new identity. Because
theoretically, Donald Trump could lose. And the Republican Party would
no longer be just the party of Donald Trump.

nick fandos

That's right. So while they're very much focused on how is the party
going to be viewed in November, they're really kind of foreshadowing or
staking out positioning for this potentially larger battle to come, over
what Republicanism really looks like after Donald Trump has defined it
for four or five years.

{[}music{]}

And you know, some of these folks are not new to their positions. But
they recognize that there may soon be more of a need to kind of assert
their views, and the primacy of those views, against others in the
Republican Party.

michael barbaro

We'll be right back.

{[}music{]}

So Nick, where does this very high stakes ideological battle within the
Republican Party, where does it leave this economic relief package?

nick fandos

So it's up to Mitch McConnell, basically, to try and pull together these
different factions and arrive at a bill that deals with the expiring
unemployment benefits and a host of other kind of programs and
priorities. Basically, to try and reconcile those differences and put
together a bill that can be Republicans' starting point when they go to
the negotiating table with Democrats.

michael barbaro

Mhm.

nick fandos

And so that's where we were by the middle of last week. And as he tries
to work out those details with the White House and run it by his
Republican colleagues, there's a bunch of snafus along the way. They
push past some small deadlines. But in the end, they're unable to
introduce their bill, because those differences turn out to have been
more significant than Republicans even wanted to let on.

michael barbaro

So the Republicans cannot come up with any kind of consensus bill to
salvage this program that we've been talking about?

nick fandos

So as of Thursday morning, no. And as lawmakers head for the exits for
the weekend, without a proposal for how to fix a whole host of programs,
they have not arrived at a solution on a range of issues, including what
to do about this expiring \$600 unemployment benefit. But their staff
and Treasury Secretary Mnuchin, Meadows, the White House chief of staff,
work through the weekend to try and iron out some of these details.

\begin{itemize}
\tightlist
\item
  archived recording (mitch mcconnell)\\
  Well, good afternoon, everyone. The Senate Republicans and the
  administration have been consulting over the last few weeks.
\end{itemize}

nick fandos

By Monday afternoon, what they finally introduce ---

\begin{itemize}
\tightlist
\item
  archived recording (mitch mcconnell)\\
  --- with what we think is an appropriate amount of additional debt to
  be added. We think it is about a trillion dollars.
\end{itemize}

nick fandos

--- is a plan that is roughly a trillion dollars.

\begin{itemize}
\tightlist
\item
  archived recording (mitch mcconnell)\\
  And we've allocated that in a way that we think makes the most sense.
\end{itemize}

nick fandos

Some of that goes to schools to help them reopen and for more testing
and contact tracing.

\begin{itemize}
\tightlist
\item
  archived recording (mitch mcconnell)\\
  So with that, I'm going to call on my colleagues who have developed
  the various ---
\end{itemize}

nick fandos

And on this key question of unemployment benefits, Republicans propose a
real overhaul to the way that they would work conceptually.

\begin{itemize}
\item
  archived recording (mitch mcconnell)\\
  Do we know who's next?
\item
  archived recording\\
  Chairman Grassley.
\item
  archived recording (mitch mcconnell)\\
  Senator Grassley.
\item
  archived recording (chuck grassley)\\
  Number one, we're going to continue ---
\end{itemize}

nick fandos

So they say that for the short term, we're going to cut that \$600 down
to \$200 a week.

michael barbaro

Big cut.

nick fandos

A pretty dramatic cut.

\begin{itemize}
\tightlist
\item
  archived recording (chuck grassley)\\
  So we want to continue to help the unemployed. But we want to
  encourage work. And we've learned a very tough lesson, that when you
  pay people not to work, what do you expect?
\end{itemize}

nick fandos

And they say, that's just going to buy us time over the next few months
for us to basically help states set up a new system, where what we're
going to try and do is make sure that every individual that's
unemployed, between the state government and the federal government ends
up getting about 70 percent of what their old wages would have been.

\begin{itemize}
\tightlist
\item
  archived recording (chuck grassley)\\
  We're going to have further tax relief for businesses to encourage
  hiring and rehiring. And we want to do that to encourage people to get
  back to work and help the employer, in the process, support people in
  the meantime.
\end{itemize}

nick fandos

And so what Republicans are trying to do here is keep a safety net in
place, but remove what they think is hindering people from going back to
work.

\begin{itemize}
\tightlist
\item
  archived recording (chuck grassley)\\
  Lastly, I hope that Democrats will come to the table and we can work
  out a bipartisan agreement. Thank you very much.
\end{itemize}

nick fandos

So in other words, if they can get this program up and operating, it
will always make sense from a financial point of view for somebody to go
and take their old job back or take a new job back, but not be so
draconian that they're making the economic situation drastically worse,
or can be accused of forcing people towards soup kitchens or the
streets.

michael barbaro

So this is a classic compromise. In other words, we're going to keep the
benefits but not at \$600 a week, because they see that as not
conservative and not incentivizing an economic recovery.

nick fandos

That's right. But remember, this is just kind of the first step. This
should have been the easy part for Republicans. Because what they have
coming is negotiations with Democrats, who are in favor of keeping the
benefit totally as it is, and are already lining up to say basically
that Republicans are giving a massive economic financial hit to
individuals and the economy right when they need it most, and at this
moment where the country's recovery seems to be teetering. Is it going
to keep going up? Or is it about to collapse again? And Democrats are
not going to settle for \$200 for any period of time.

michael barbaro

So given all that, what is likely to happen to this Republican bill in
the Senate?

nick fandos

So the interesting thing about where Republicans find themselves is,
this bill that they're introducing probably couldn't even pass the
Senate just on Republican votes. And that leaves them in a pretty weak
position as they head into negotiations with the Democrats. Because
remember, to pass anything into law, even if there's a Republican
president or a Republican Senate, you need the Democrats to get it
through Congress. And they have a very long and expensive wish list of
things that they'd like to see in legislation. And they're not going to
be easy on the Republicans.

michael barbaro

Nick, this may sound like a strange question. But do you think
Republicans now regret ever agreeing to these enhanced unemployment
benefits? I'm mindful of the fact that it was not a Republican idea. It
was Democrats who pushed for it. As you have said, it cuts against a lot
of Republican principles. But they agreed to it as a short-term fix. And
it turns out it's not going to be a short-term term fix, because there's
nothing short-term about this pandemic. And it is inevitably hard to
take something like this away from people once you give it to them. So
is it possible Republicans look back and think we should have never
agreed to do this?

nick fandos

I think there may be a small subset of fiscally conservative Republicans
that feel that way. But my guess is that the vast majority felt like,
hey, we did what we had to do back then in the springtime. I mean, the
economy was in freefall, remember. And the course of the virus was
highly uncertain. And the fundamental problem for them is that they
envisioned the federal government having a relatively short-term role to
play in getting the country back on its feet and ready to fight against
this virus. And it's just turned out to be, for a lot of different
reasons, a much more complicated, prolonged, expensive fight than they
wanted.

And honestly, Michael, at this point, it's hard to see how this
situation resolves itself. Usually, when you cover Congress for a while,
you can kind of see the pattern of how these negotiations will work. But
Republicans really find themselves pretty far up the stream without a
paddle right now. And there seem to be risks for them and consequences
in every direction.

And it's going to be a pretty fascinating next couple of weeks to see
how and if they can reach an agreement with Democrats --- and one that
some members of the party feel like doesn't completely undermine what
they stand for.

michael barbaro

Of course, weeks is not what people who are on this program have. They
have days. Because this thing really does expire on Friday.

nick fandos

That's right. Many of the people receiving these benefits are living
paycheck to paycheck or don't have a lot of savings to fall back on.
There can and will be very real consequences to this delay. And that's
not to mention the whole host of other programs that are being debated
by Congress right now that are touching different aspects of people's
lives.

{[}music{]}

The longer this goes on, the effects just get magnified. Bigger and
bigger and bigger. And it frankly makes the problem even harder to
solve.

michael barbaro

Thank you, Nick.

nick fandos

Thank you, Michael.

michael barbaro

On Monday night, Democratic leaders, including House Speaker Nancy
Pelosi, met with White House officials to begin negotiations over a new
economic relief package, including federal unemployment benefits.

\begin{itemize}
\tightlist
\item
  archived recording (nancy pelosi)\\
  Suffice to say that we hoped that we would be able to reach an
  agreement. We clearly do not have shared values.
\end{itemize}

michael barbaro

Little progress was made during the two-hour session. But afterward, the
Democratic leaders made one thing clear. Congressional Republicans lack
the votes to pass their own bill.

We'll be right back.

Here's what else you need to know today.

On Monday, the pandemic touched the worlds of politics, business and
sports. The Trump administration said that its national security
adviser, Robert O'Brien, had contracted the virus, becoming the most
senior White House official yet to test positive.

Meanwhile, the parent company of Google --- Alphabet --- told employees
that they would not be expected to return to the office until next
summer, suggesting that work-from-home policies will extend well past
the end of the year.

Finally, the Miami Marlins canceled two upcoming baseball games after 12
players and two coaches tested positive for the coronavirus. The
outbreak was disclosed just four days after the beginning of the
baseball season.

\begin{itemize}
\tightlist
\item
  archived recording (dave martinez)\\
  My level of concern went from about an eight to a 12. You know, it
  hits home now that you see half a team get infected and it go from one
  city to another. So ---
\end{itemize}

michael barbaro

During a news conference, the manager of the Washington Nationals
expressed alarm over the news.

\begin{itemize}
\tightlist
\item
  archived recording (dave martinez)\\
  Yeah, I got friends on that Miami team. And it really stinks. Now I'm
  not going to lie. I'm not going to sugarcoat it. Seeing those guys go
  down like that, it's not good for them. It's not good for anybody.
\end{itemize}

michael barbaro

That's it for ``The Daily.'' I'm Michael Barbaro. See you tomorrow.

Previous

More episodes ofThe Daily

\href{https://www.nytimes3xbfgragh.onion/2020/08/04/podcasts/the-daily/mail-in-voting-president-trump.html?action=click\&module=audio-series-bar\&region=header\&pgtype=Article}{\includegraphics{https://static01.graylady3jvrrxbe.onion/images/2020/07/30/us/politics/04daily/30trump-election1-thumbLarge.jpg}}

August 4, 2020Is the U.S. Ready to Vote by Mail?

\href{https://www.nytimes3xbfgragh.onion/2020/08/03/podcasts/the-daily/algorithmic-justice-racism.html?action=click\&module=audio-series-bar\&region=header\&pgtype=Article}{\includegraphics{https://static01.graylady3jvrrxbe.onion/images/2020/06/24/business/03daily/24michigan-arrest1-thumbLarge.jpg}}

August 3, 2020~~•~ 28:13Wrongfully Accused by an Algorithm

\href{https://www.nytimes3xbfgragh.onion/2020/08/02/podcasts/the-daily/on-female-rage.html?action=click\&module=audio-series-bar\&region=header\&pgtype=Article}{\includegraphics{https://static01.graylady3jvrrxbe.onion/images/2018/01/21/magazine/21mag-femaleanger1-copy/21mag-femaleanger1-thumbLarge.jpg}}

August 2, 2020The Sunday Read: `On Female Rage'

\href{https://www.nytimes3xbfgragh.onion/2020/07/31/podcasts/the-daily/vanessa-guillen-military-metoo.html?action=click\&module=audio-series-bar\&region=header\&pgtype=Article}{\includegraphics{https://static01.graylady3jvrrxbe.onion/images/2020/07/12/us/politics/31daily/00dc-army-metoo-thumbLarge.jpg}}

July 31, 2020A \#MeToo Moment in the Military

\href{https://www.nytimes3xbfgragh.onion/2020/07/30/podcasts/the-daily/congress-facebook-amazon-google-apple.html?action=click\&module=audio-series-bar\&region=header\&pgtype=Article}{\includegraphics{https://static01.graylady3jvrrxbe.onion/images/2020/07/30/reader-center/30daily/merlin_175077825_5ebc931b-baa1-489a-960c-34e4d845e997-thumbLarge.jpg}}

July 30, 2020~~•~ 35:19The Big Tech Hearing

\href{https://www.nytimes3xbfgragh.onion/2020/07/29/podcasts/the-daily/china-trump-foreign-policy.html?action=click\&module=audio-series-bar\&region=header\&pgtype=Article}{\includegraphics{https://static01.graylady3jvrrxbe.onion/images/2020/07/26/world/29daily/00china-us-clash1-thumbLarge.jpg}}

July 29, 2020~~•~ 28:40Confronting China

\href{https://www.nytimes3xbfgragh.onion/2020/07/28/podcasts/the-daily/unemployment-benefits-coronavirus.html?action=click\&module=audio-series-bar\&region=header\&pgtype=Article}{\includegraphics{https://static01.graylady3jvrrxbe.onion/images/2020/07/23/business/28daily/23virus-uiexplain1-thumbLarge.jpg}}

July 28, 2020~~•~ 26:13Why \$600 Checks Are Tearing Republicans Apart

\href{https://www.nytimes3xbfgragh.onion/2020/07/27/podcasts/the-daily/new-york-hospitals-covid.html?action=click\&module=audio-series-bar\&region=header\&pgtype=Article}{\includegraphics{https://static01.graylady3jvrrxbe.onion/images/2020/07/27/world/27daily-hospitals/27daily-hospitals-thumbLarge.jpg}}

July 27, 2020~~•~ 33:28The Mistakes New York Made

\href{https://www.nytimes3xbfgragh.onion/2020/07/26/podcasts/the-daily/the-accusation-the-sunday-read.html?action=click\&module=audio-series-bar\&region=header\&pgtype=Article}{\includegraphics{https://static01.graylady3jvrrxbe.onion/images/2020/03/22/magazine/26audm-2/22mag-titleix-thumbLarge.jpg}}

July 26, 2020The Sunday Read: `The Accusation'

\href{https://www.nytimes3xbfgragh.onion/2020/07/24/podcasts/the-daily/mlb-baseball-season-coronavirus.html?action=click\&module=audio-series-bar\&region=header\&pgtype=Article}{\includegraphics{https://static01.graylady3jvrrxbe.onion/images/2020/07/22/sports/24daily/22mlb-previewlede1-thumbLarge.jpg}}

July 24, 2020~~•~ 45:34The Battle for a Baseball Season

\href{https://www.nytimes3xbfgragh.onion/2020/07/23/podcasts/the-daily/portland-protests.html?action=click\&module=audio-series-bar\&region=header\&pgtype=Article}{\includegraphics{https://static01.graylady3jvrrxbe.onion/images/2020/07/22/us/23daily-image/22portland-tactics02-thumbLarge.jpg}}

July 23, 2020~~•~ 30:04The Showdown in Portland

\href{https://www.nytimes3xbfgragh.onion/2020/07/22/podcasts/the-daily/school-reopenings-coronavirus.html?action=click\&module=audio-series-bar\&region=header\&pgtype=Article}{\includegraphics{https://static01.graylady3jvrrxbe.onion/images/2020/07/12/science/22daily/00virus-schools-reopen01-thumbLarge.jpg}}

July 22, 2020~~•~ 27:24The Science of School Reopenings

\href{https://www.nytimes3xbfgragh.onion/column/the-daily}{See All
Episodes ofThe Daily}

Next

July 28, 2020

\begin{itemize}
\item
\item
\item
\item
\item
\item
\end{itemize}

\emph{\textbf{Listen and subscribe to our podcast from your mobile
device:}}\\
\textbf{\href{https://itunes.apple.com/us/podcast/the-daily/id1200361736?mt=2}{\emph{Via
Apple Podcasts}}} \emph{\textbf{\textbar{}}}
\textbf{\href{https://open.spotify.com/show/3IM0lmZxpFAY7CwMuv9H4g?si=SfuMSC55R1qprFsRZU3_zw}{\emph{Via
Spotify}}} \emph{\textbf{\textbar{}}}
\textbf{\href{http://www.stitcher.com/podcast/the-new-york-times/the-daily-10}{\emph{Via
Stitcher}}}

A fight has erupted among congressional Republicans over how long and
how generously the government should help those who are unemployed
during the pandemic. But what is that battle really about?

\textbf{On today's episode:}

\begin{itemize}
\tightlist
\item
  \href{https://www.nytimes3xbfgragh.onion/by/nicholas-fandos}{Nicholas
  Fandos}, who covers Congress for The New York Times.
\end{itemize}

\includegraphics{https://static01.graylady3jvrrxbe.onion/images/2020/07/23/business/28daily/merlin_174871614_89a856cc-1ddf-4333-937d-221e23031599-articleLarge.jpg?quality=75\&auto=webp\&disable=upscale}

\textbf{Background reading:}

\begin{itemize}
\item
  Supplemental checks for laid-off workers are set to stop at the end of
  July. Republicans and Democrats
  \href{https://www.nytimes3xbfgragh.onion/2020/07/23/business/economy/unemployment-benefits.html}{disagree
  on what to do next}.
\item
  \href{https://www.nytimes3xbfgragh.onion/2020/07/26/us/politics/coronavirus-stimulus-bill-unemployment.html}{Why
  the two parties are unlikely to reach a deal} before the end of the
  month.
\end{itemize}

\emph{Tune in, and tell us what you think. Email us at}
\href{mailto:thedaily@NYTimes.com}{\emph{thedaily@NYTimes.com}}\emph{.
Follow Michael Barbaro on Twitter:}
\href{https://twitter.com/mikiebarb}{\emph{@mikiebarb}}\emph{. And if
you're interested in advertising with ``The Daily,'' write to us at}
\href{mailto:thedaily-ads@NYTimes.com}{\emph{thedaily-ads@NYTimes.com}}\emph{.}

Nicholas Fandos contributed reporting.

``The Daily'' is made by Theo Balcomb, Andy Mills, Lisa Tobin, Rachel
Quester, Lynsea Garrison, Annie Brown, Clare Toeniskoetter, Paige
Cowett, Michael Simon Johnson, Brad Fisher, Larissa Anderson, Wendy
Dorr, Chris Wood, Jessica Cheung, Stella Tan, Alexandra Leigh Young,
Jonathan Wolfe, Lisa Chow, Eric Krupke, Marc Georges, Luke Vander Ploeg,
Adizah Eghan, Kelly Prime, Julia Longoria, Sindhu Gnanasambandan, M.J.
Davis Lin, Austin Mitchell, Sayre Quevedo, Neena Pathak, Dan Powell,
Dave Shaw, Sydney Harper, Daniel Guillemette, Hans Buetow, Robert
Jimison, Mike Benoist, Bianca Giaever and Asthaa Chaturvedi. Our theme
music is by Jim Brunberg and Ben Landsverk of Wonderly. Special thanks
to Sam Dolnick, Mikayla Bouchard, Lauren Jackson, Julia Simon, Mahima
Chablani and Nora Keller.

Advertisement

\protect\hyperlink{after-bottom}{Continue reading the main story}

\hypertarget{site-index}{%
\subsection{Site Index}\label{site-index}}

\hypertarget{site-information-navigation}{%
\subsection{Site Information
Navigation}\label{site-information-navigation}}

\begin{itemize}
\tightlist
\item
  \href{https://help.nytimes3xbfgragh.onion/hc/en-us/articles/115014792127-Copyright-notice}{©~2020~The
  New York Times Company}
\end{itemize}

\begin{itemize}
\tightlist
\item
  \href{https://www.nytco.com/}{NYTCo}
\item
  \href{https://help.nytimes3xbfgragh.onion/hc/en-us/articles/115015385887-Contact-Us}{Contact
  Us}
\item
  \href{https://www.nytco.com/careers/}{Work with us}
\item
  \href{https://nytmediakit.com/}{Advertise}
\item
  \href{http://www.tbrandstudio.com/}{T Brand Studio}
\item
  \href{https://www.nytimes3xbfgragh.onion/privacy/cookie-policy\#how-do-i-manage-trackers}{Your
  Ad Choices}
\item
  \href{https://www.nytimes3xbfgragh.onion/privacy}{Privacy}
\item
  \href{https://help.nytimes3xbfgragh.onion/hc/en-us/articles/115014893428-Terms-of-service}{Terms
  of Service}
\item
  \href{https://help.nytimes3xbfgragh.onion/hc/en-us/articles/115014893968-Terms-of-sale}{Terms
  of Sale}
\item
  \href{https://spiderbites.nytimes3xbfgragh.onion}{Site Map}
\item
  \href{https://help.nytimes3xbfgragh.onion/hc/en-us}{Help}
\item
  \href{https://www.nytimes3xbfgragh.onion/subscription?campaignId=37WXW}{Subscriptions}
\end{itemize}
