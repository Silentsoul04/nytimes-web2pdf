Sections

SEARCH

\protect\hyperlink{site-content}{Skip to
content}\protect\hyperlink{site-index}{Skip to site index}

\href{https://www.nytimes3xbfgragh.onion/section/science}{Science}

\href{https://myaccount.nytimes3xbfgragh.onion/auth/login?response_type=cookie\&client_id=vi}{}

\href{https://www.nytimes3xbfgragh.onion/section/todayspaper}{Today's
Paper}

\href{/section/science}{Science}\textbar{}Bringing Mars Rocks to Earth:
Our Greatest Interplanetary Circus Act

\url{https://nyti.ms/39zCmUp}

\begin{itemize}
\item
\item
\item
\item
\item
\item
\end{itemize}

\href{https://www.nytimes3xbfgragh.onion/news-event/summer-of-mars?action=click\&pgtype=Article\&state=default\&region=TOP_BANNER\&context=storylines_menu}{Missions
to Mars}

\begin{itemize}
\tightlist
\item
  \href{https://www.nytimes3xbfgragh.onion/2020/07/30/science/nasa-mars-launch.html?action=click\&pgtype=Article\&state=default\&region=TOP_BANNER\&context=storylines_menu}{NASA
  Launch Highlights}
\item
  \href{https://www.nytimes3xbfgragh.onion/interactive/2020/science/mars-perseverance-tianwen-hope.html?action=click\&pgtype=Article\&state=default\&region=TOP_BANNER\&context=storylines_menu}{Meet
  the Spacecraft}
\item
  \href{https://www.nytimes3xbfgragh.onion/2020/07/28/science/nasa-jezero-perseverance.html?action=click\&pgtype=Article\&state=default\&region=TOP_BANNER\&context=storylines_menu}{NASA's
  Destination}
\item
  \href{https://www.nytimes3xbfgragh.onion/2020/07/28/science/mars-nasa-science.html?action=click\&pgtype=Article\&state=default\&region=TOP_BANNER\&context=storylines_menu}{Too
  Much Mars?}
\end{itemize}

Advertisement

\protect\hyperlink{after-top}{Continue reading the main story}

Supported by

\protect\hyperlink{after-sponsor}{Continue reading the main story}

\hypertarget{bringing-mars-rocks-to-earth-our-greatest-interplanetary-circus-act}{%
\section{Bringing Mars Rocks to Earth: Our Greatest Interplanetary
Circus
Act}\label{bringing-mars-rocks-to-earth-our-greatest-interplanetary-circus-act}}

NASA and the European Space Agency plan to toss rocks from one
spacecraft to another before the samples finally land on Earth in 2031.

\includegraphics{https://static01.graylady3jvrrxbe.onion/images/2020/07/28/science/28MARSSAMPLE/28MARSSAMPLE-articleLarge.jpg?quality=75\&auto=webp\&disable=upscale}

\href{https://www.nytimes3xbfgragh.onion/by/kenneth-chang}{\includegraphics{https://static01.graylady3jvrrxbe.onion/images/2018/02/16/multimedia/author-kenneth-chang/author-kenneth-chang-thumbLarge.jpg}}

By \href{https://www.nytimes3xbfgragh.onion/by/kenneth-chang}{Kenneth
Chang}

\begin{itemize}
\item
  Published July 28, 2020Updated July 30, 2020
\item
  \begin{itemize}
  \item
  \item
  \item
  \item
  \item
  \item
  \end{itemize}
\end{itemize}

\href{https://www.nytimes3xbfgragh.onion/es/2020/07/29/espanol/ciencia-y-tecnologia/mision-marte-nasa.html}{Leer
en español}

Send a robotic spacecraft to Mars, grab some rocks and dirt and bring
those back to Earth.

How hard could that be?

It's more like an interplanetary circus act than you might imagine, but
NASA and the European Space Agency think that now is the time they can
finally pull off this complex choreography, tossing the rocks from one
spacecraft to another before the samples finally land on Earth in 2031.

``The science community, of course, has lusted after doing this for
quite some time,'' said James Watzin, the director of the Mars
exploration program at NASA.

Over the last couple of decades, robotic explorers have revealed an
increasingly complex picture of Mars, but planetary scientists are
limited by the amount of science that can be packed in a spacecraft.

``You can only carry so much instrumentation into the field,
robotically,'' Mr. Watzin said. ``To really get into some of the really
intriguing questions at a detail level means we need to parse the
evidence down on the molecular level and try to tease the information
out of very, very old material. And that requires a whole suite of
instrumentation that was clearly too large to shrink and send to another
planet.''

With fresh Mars rocks on Earth, more scientists will be able to examine
them, employing a wide array of the most sophisticated equipment in
laboratories around the world.

The first step of this epic undertaking, known as Mars sample return,
starts soon with Perseverance, the next NASA rover. It is scheduled to
liftoff on July 30, headed for Jezero, a crater that was once a lake
about 3.5 billion years ago, and is a promising place where signs of
past life on Mars could be preserved.

One of the key tasks for Perseverance is to drill up to 39 rock cores,
each a half-inch wide and 2.4 inches long, that look interesting enough
to merit additional scrutiny on Earth. Each sample of rock and dirt,
weighing about half an ounce, will be sealed in an ultraclean cigar-size
metal tube.

But initially, NASA had no plans to bring those tubes back to Earth.
Perseverance has no way of flinging the rocks off Mars.

\hypertarget{perseverance}{%
\subsection{Perseverance}\label{perseverance}}

The NASA mission includes Perseverance, a 2,200-pound rover, and
Ingenuity, an experimental Mars helicopter.

Ingenuity Helicopter

The four-pound aircraft will communicate wirelessly with the
Perseverance rover.

Solar Panel

Blades

Four carbon-fiber blades will spin at about 2,400 r.p.m.

Power

The plutonium-based power supply will charge the rover's batteries.

MAST

Instruments will take videos, panoramas and photographs. A laser will
study the chemistry of Martian rocks.

PiXl

Will identify chemical elements to seek signs of past life on Mars.

Antenna

Will transmit data directly to Earth.

Robotic arm

A turret with many instruments is attached to a 7-foot robotic arm. A
drill will extract samples from Martian rocks. The Sherloc device will
identify molecules and minerals to detect potential biosignatures, with
help from the Watson camera.

Perseverance Rover

The 2,200 pound rover will explore Jezero Crater. It has aluminum wheels
and a suspension system to drive over obstacles.

Ingenuity Helicopter

The aircraft will communicate wirelessly with the rover.

Solar Panel

Blades

Power

The plutonium-based power supply will charge the rover's batteries.

MAST

Instruments will take videos, panoramas and photographs. A laser will
study the chemistry of Martian rocks.

PiXl

Will identify chemical elements to seek signs of past life on Mars.

Antenna

Robotic arm

A turret with many instruments is attached to a 7-foot robotic arm. A
drill will extract samples from Martian rocks. The Sherloc device will
identify molecules and minerals to detect potential biosignatures, with
help from the Watson camera.

Perseverance Rover

The 2,200 pound rover will explore Jezero Crater. It has aluminum wheels
and a suspension system to drive over obstacles.

Solar panel

Ingenuity Helicopter

Blades

Power

Mast

PIXL

Antenna

Suspension

Perseverance rover

Robotic arm

A turret with many instruments is attached to a 7-foot robotic arm. A
drill will extract samples from Martian rocks. The Sherloc device will
identify molecules and minerals to detect potential biosignatures, with
help from the Watson camera. PiXl will identify chemical elements to
seek signs of past life on Mars.

By Eleanor Lutz \textbar{} Source: NASA

Three years ago, a team of engineers at NASA's Jet Propulsion Laboratory
in California, began taking a closer look at when the return part of
Mars sample return could be undertaken. They considered the possibility
of launching the retrieval spacecraft in 2026 with the samples returning
three years later.

That timeline, they found, was too ambitious.

But if the landing on Earth was pushed back to 2031, the schedule
appeared to be feasible. ``We actually feel like we could do this,'' Mr.
Watzin said.

The Trump administration's budget request for NASA for fiscal year 2021
included \$233 million to continue development, two years after the
agency received \$50 million for the initial studies. Last month, the 22
member nations of the European Space Agency gave the go-ahead on the
collaboration with NASA.

The Perseverance science team has already begun preliminary geological
analysis about what should be brought back to Earth.

``We became ever more focused on how to do that element of it right,''
said Kenneth Farley, the project scientist for Perseverance. ``We've
kind of transitioned from a `yeah, someday these samples will get picked
up' to `yeah, they might get picked up pretty soon.' It's been an
important evolution.''

Space agency officials have not yet announced a total price tag, but the
cost is expected to run several billion dollars.

``We're trying to keep this under a certain cost target,'' said Brian K.
Muirhead, who is leading the sample return design at the Jet Propulsion
Laboratory. ``We're really coming up with the estimates --- `This is
what we think it's going to take' --- and so far, NASA has said, `OK,
keep going.'''

If everything goes to plan, two spacecraft will blast off to Mars in
2026. One will be a NASA-built lander that will be the heaviest vehicle
ever put on the surface of Mars. It will be carrying a rover, built by
the Europeans, to fetch the rock samples, and a small rocket that will
launch the rocks to orbit around Mars.

The lander will take a roundabout trajectory to Mars, arriving in August
2028, the beginning of the Martian spring. The solar-powered fetch rover
will then roll off the lander, make a dash to collect at least some of
the rock samples and bring them back and transfer them to the lander.
The samples, in turn, will be robotically moved to the top of the Mars
ascent vehicle, the rocket that will launch the rocks off Mars.

The second spacecraft, the Earth Return Orbiter, will be built by the
European Space Agency. It will take a quicker path to Mars, pulling into
orbit before the lander's arrival. That will allow the orbiter to serve
as the relay for communications from the lander as it zooms to the
surface.

The launch of the ascent vehicle will deposit a container, about the
size of a soccer ball, with the rock samples circling around Mars about
200 miles above the surface. The orbiter then has to find this
container, like a baseball outfielder chasing down a fly ball. The
orbiter will be tracking the launch of the rocket, but for simplicity,
the container itself does not possess any thrusters or a radio beacon.
It is, however, white, which should make it easier to spot against the
darkness of space.

``This is obviously one of the key issues: How do you find it?'' Mr.
Muirhead said. ``Once you know where its orbit is, it's very easy to
match orbit.''

A door on the orbiter will open to capture the container. A 1,000-pound
contraption within the orbiter then rotates and slides the container to
the proper configuration within the spacecraft, taking care to seal off
the possibility that anything from Mars could contaminate anything
outside of the sample container.

The orbiter would then depart Mars. As it approached Earth, it would
eject the samples, now mounted within what is called the Earth entry
vehicle, on a collision course with the Utah desert.

Parachutes were another complication that engineers decided was
unnecessary, so the entry vehicle, which resembles a large sombrero, is
to hit the ground at a speed comparable to a highway car crash: 90 miles
per hour.

The scientific cargo --- rocks and dirt, which are not fragile --- will
easily survive that impact.

Many of the details like where the lander will set down, remain
undecided. If Perseverance is still in good working condition, it might
head to a second site outside of Jezero where there might have been
geothermal hot springs, another environment where life could have
thrived.

But these decisions do not have to be made for years, and the best
answers may not become apparent until Perseverance gets a good look at
Jezero.

If one piece breaks, the sample return mission does not necessarily
fail. Perseverance will likely drop some of the sample tubes on the
ground in case it suffers a malfunction later in the mission. If the
fetch rover breaks, then Perseverance could bring samples to the lander
instead.

Even if the orbiter fails, its soccer ball-size container holding
samples could remain circling Mars for years until another spacecraft
could be sent to catch it.

``That's been my job as the architect,'' Mr. Muirhead said. ``To think
through the process from the concept of operations, develop the concepts
that can achieve the objectives of the different phases and make sure
there's good margins built in everywhere. So that the design isn't
fragile.''

\href{https://www.nytimes3xbfgragh.onion/interactive/2020/science/exploring-the-solar-system.html}{}

\includegraphics{https://static01.graylady3jvrrxbe.onion/images/2020/07/24/us/exploring-the-solar-system-promo-1595620746754/exploring-the-solar-system-promo-1595620746754-articleLarge.png}

\hypertarget{exploring-the-solar-system}{%
\subsection{Exploring the Solar
System}\label{exploring-the-solar-system}}

A guide to the spacecraft beyond Earth's orbit.

\href{https://www.nytimes3xbfgragh.onion/interactive/2020/science/2020-astronomy-space-calendar.html}{}

\includegraphics{https://static01.graylady3jvrrxbe.onion/images/2019/12/04/science/04SUN1/04SUN1-articleLarge.png}

\hypertarget{sync-your-calendar-with-the-solar-system}{%
\subsection{Sync your calendar with the solar
system}\label{sync-your-calendar-with-the-solar-system}}

Never miss an eclipse, a meteor shower, a rocket launch or any other
astronomical and space event that's out of this world.

Advertisement

\protect\hyperlink{after-bottom}{Continue reading the main story}

\hypertarget{site-index}{%
\subsection{Site Index}\label{site-index}}

\hypertarget{site-information-navigation}{%
\subsection{Site Information
Navigation}\label{site-information-navigation}}

\begin{itemize}
\tightlist
\item
  \href{https://help.nytimes3xbfgragh.onion/hc/en-us/articles/115014792127-Copyright-notice}{©~2020~The
  New York Times Company}
\end{itemize}

\begin{itemize}
\tightlist
\item
  \href{https://www.nytco.com/}{NYTCo}
\item
  \href{https://help.nytimes3xbfgragh.onion/hc/en-us/articles/115015385887-Contact-Us}{Contact
  Us}
\item
  \href{https://www.nytco.com/careers/}{Work with us}
\item
  \href{https://nytmediakit.com/}{Advertise}
\item
  \href{http://www.tbrandstudio.com/}{T Brand Studio}
\item
  \href{https://www.nytimes3xbfgragh.onion/privacy/cookie-policy\#how-do-i-manage-trackers}{Your
  Ad Choices}
\item
  \href{https://www.nytimes3xbfgragh.onion/privacy}{Privacy}
\item
  \href{https://help.nytimes3xbfgragh.onion/hc/en-us/articles/115014893428-Terms-of-service}{Terms
  of Service}
\item
  \href{https://help.nytimes3xbfgragh.onion/hc/en-us/articles/115014893968-Terms-of-sale}{Terms
  of Sale}
\item
  \href{https://spiderbites.nytimes3xbfgragh.onion}{Site Map}
\item
  \href{https://help.nytimes3xbfgragh.onion/hc/en-us}{Help}
\item
  \href{https://www.nytimes3xbfgragh.onion/subscription?campaignId=37WXW}{Subscriptions}
\end{itemize}
