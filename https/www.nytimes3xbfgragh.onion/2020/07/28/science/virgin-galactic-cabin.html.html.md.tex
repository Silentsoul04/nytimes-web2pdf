Sections

SEARCH

\protect\hyperlink{site-content}{Skip to
content}\protect\hyperlink{site-index}{Skip to site index}

\href{https://www.nytimes3xbfgragh.onion/section/science}{Science}

\href{https://myaccount.nytimes3xbfgragh.onion/auth/login?response_type=cookie\&client_id=vi}{}

\href{https://www.nytimes3xbfgragh.onion/section/todayspaper}{Today's
Paper}

\href{/section/science}{Science}\textbar{}Virgin Galactic Unveils Comfy
Cabin for Jet-Setting to the Edge of Space

\url{https://nyti.ms/3gl5DVH}

\begin{itemize}
\item
\item
\item
\item
\item
\end{itemize}

Advertisement

\protect\hyperlink{after-top}{Continue reading the main story}

Supported by

\protect\hyperlink{after-sponsor}{Continue reading the main story}

\hypertarget{virgin-galactic-unveils-comfy-cabin-for-jet-setting-to-the-edge-of-space}{%
\section{Virgin Galactic Unveils Comfy Cabin for Jet-Setting to the Edge
of
Space}\label{virgin-galactic-unveils-comfy-cabin-for-jet-setting-to-the-edge-of-space}}

Passengers able to pay hundreds of thousands of dollars for a seat can
escape gravity for a few minutes.

\includegraphics{https://static01.graylady3jvrrxbe.onion/images/2020/07/28/science/28VIRGINGALACTIC2/28VIRGINGALACTIC2-videoSixteenByNineJumbo1600.jpg}

\href{https://www.nytimes3xbfgragh.onion/by/kenneth-chang}{\includegraphics{https://static01.graylady3jvrrxbe.onion/images/2018/02/16/multimedia/author-kenneth-chang/author-kenneth-chang-thumbLarge.jpg}}

By \href{https://www.nytimes3xbfgragh.onion/by/kenneth-chang}{Kenneth
Chang}

\begin{itemize}
\item
  July 28, 2020
\item
  \begin{itemize}
  \item
  \item
  \item
  \item
  \item
  \end{itemize}
\end{itemize}

The inside of Virgin Galactic's space plane is like a space-age
executive jet.

The seats recline to absorb the forces of acceleration toward space.
Mood lighting shifts during each phase of the flight. Twelve windows ---
two for each of the six passengers, who have paid hundreds of thousands
of dollars each for a seat --- provide an impressive view of Earth and
the darkness of space. Sixteen cameras will capture you floating. And
the back of the cabin includes a big circular mirror so that you can
watch yourself enjoying a few minutes escaping the effects of gravity.

Virgin Galactic will be offering short up-and-down trips to the edge of
space, essentially like giant roller coaster rides with better views, in
its space plane, SpaceShipTwo.

But how can the company unveil the fancy new interior of its space plane
in the middle of a global pandemic when journalists are not able to
gather for a fancy media event?

Modern technology provided an imaginative solution. Virgin Galactic sent
Oculus virtual reality headsets as loaners to journalists so that they
could chat with the designers of the cabin while walking through a
computer-generated version of it --- an experience of almost being there
while being nowhere near there.

For me, that was standing in my cluttered home office on Monday, trying
to avoid walking into the door, walls and stacks of boxes, while gazing
at Earth above.

``Obviously, you're on your journey to become an astronaut yourself,''
Jeremy Brown, design director at Virgin Galactic, said during our voyage
into virtual reality. In this Oculus world, the avatars of other people
are mannequin heads with a couple of disembodied gesticulating hands.

If you weren't sent a VR headset, the company presented a livestream
event on YouTube on Tuesday, the first public viewing of the inside of
SpaceShipTwo.

Aboard SpaceShipTwo, every seat is, of course, a window seat, with one
window to the side and a second above. Unlike commercial jets, the
cockpit area where the two pilots sit is not walled off, so passengers
can also look out the front windows.

The virtual-reality experience offers a unique perspective not possible
in real life. I was able to walk through the wall of SpaceShipTwo to
look at the outside of the spacecraft and the full panorama of Earth.

The mirror ``is the largest mirror on board a spaceship,'' Mr. Brown
said, adding, ``That's a really nice memory for you as a customer to
have that kind of analog memory of seeing you and your fellow crew
members floating around.''

\includegraphics{https://static01.graylady3jvrrxbe.onion/images/2020/07/28/science/28VIRGINGALACTIC1/28VIRGINGALACTIC1-articleLarge.jpg?quality=75\&auto=webp\&disable=upscale}

The company,
\href{https://www.nytimes3xbfgragh.onion/2019/07/09/business/richard-branson-virgin-galactic-space.html}{which
went public in October}, is not yet promising when virtual reality will
turn into actual reality. This year, it moved its operations from
Mojave, Calif., where SpaceShipTwo was developed and initially tested,
to Spaceport America in New Mexico, where the commercial flights will
take place.

Also making the move was the White Knight Two airplane that will carry
the space plane to an altitude of about 50,000 feet before dropping it.
SpaceShipTwo's engine then ignites, taking it up to more than 50 miles.
At the top of the arc, passengers will float for about four minutes
before the space plane re-enters the atmosphere and glides to a runway
landing.

The actual interior is not quite finished, said George Whitesides, who
just turned over the chief executive role at Virgin Galactic to Michael
Colglazier, who had been in charge of the theme parks at Disney. Mr.
Whitesides assumed a newly created position of chief space officer.

Virgin Galactic has more than 600 customers who paid
\href{https://www.nytimes3xbfgragh.onion/2019/10/16/science/virgin-galactic-spacesuit.html}{up
to \$250,000 each} for seats on its earliest flights. But it stopped
selling tickets in December 2018. The company has not yet said when it
will restart sales or how much a seat will cost now, except that the
price will be higher. However, it allows people to get in line by paying
a \$1,000 deposit.

In the future, the company will face competition from
\href{https://www.nytimes3xbfgragh.onion/2020/04/30/science/nasa-moon-lander.html}{Blue
Origin, the private rocket company founded by Jeffrey P. Bezos}, the
chief executive of Amazon. The company also offers suborbital trips to
the edge of space on its reusable New Shepard rocket and capsule, but
Amazon has not yet tested flights with people aboard or announced when
it will start selling tickets or how much they will cost.

And customers with tens of millions of dollars to spend may be flying to
Earth orbit or
\href{https://www.nytimes3xbfgragh.onion/2020/03/05/science/axiom-space-station.html}{perhaps
even the International Space Station} in the coming years aboard Crew
Dragon, the capsule built for NASA by SpaceX, the rocket company founded
by Elon Musk. SpaceX flew two NASA astronauts successfully to the
station in May, and plans to bring them home on Sunday if weather
permits.

After a couple of successful glide tests at the New Mexico spaceport,
the final
\href{https://www.nytimes3xbfgragh.onion/2018/12/13/science/virgin-galactic-spaceship.html}{rounds
of testing} will consist of powered flights, including ones with
employees who will play the role of paying passengers. The company has
not set a target date for the first commercial flight --- the one that
will have its founder, Richard Branson, aboard.

``We're still working hard to fly Richard as soon as we can,'' Mr.
Whitesides said.

Advertisement

\protect\hyperlink{after-bottom}{Continue reading the main story}

\hypertarget{site-index}{%
\subsection{Site Index}\label{site-index}}

\hypertarget{site-information-navigation}{%
\subsection{Site Information
Navigation}\label{site-information-navigation}}

\begin{itemize}
\tightlist
\item
  \href{https://help.nytimes3xbfgragh.onion/hc/en-us/articles/115014792127-Copyright-notice}{©~2020~The
  New York Times Company}
\end{itemize}

\begin{itemize}
\tightlist
\item
  \href{https://www.nytco.com/}{NYTCo}
\item
  \href{https://help.nytimes3xbfgragh.onion/hc/en-us/articles/115015385887-Contact-Us}{Contact
  Us}
\item
  \href{https://www.nytco.com/careers/}{Work with us}
\item
  \href{https://nytmediakit.com/}{Advertise}
\item
  \href{http://www.tbrandstudio.com/}{T Brand Studio}
\item
  \href{https://www.nytimes3xbfgragh.onion/privacy/cookie-policy\#how-do-i-manage-trackers}{Your
  Ad Choices}
\item
  \href{https://www.nytimes3xbfgragh.onion/privacy}{Privacy}
\item
  \href{https://help.nytimes3xbfgragh.onion/hc/en-us/articles/115014893428-Terms-of-service}{Terms
  of Service}
\item
  \href{https://help.nytimes3xbfgragh.onion/hc/en-us/articles/115014893968-Terms-of-sale}{Terms
  of Sale}
\item
  \href{https://spiderbites.nytimes3xbfgragh.onion}{Site Map}
\item
  \href{https://help.nytimes3xbfgragh.onion/hc/en-us}{Help}
\item
  \href{https://www.nytimes3xbfgragh.onion/subscription?campaignId=37WXW}{Subscriptions}
\end{itemize}
