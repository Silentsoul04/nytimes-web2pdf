Sections

SEARCH

\protect\hyperlink{site-content}{Skip to
content}\protect\hyperlink{site-index}{Skip to site index}

\href{https://www.nytimes3xbfgragh.onion/section/technology}{Technology}

\href{https://myaccount.nytimes3xbfgragh.onion/auth/login?response_type=cookie\&client_id=vi}{}

\href{https://www.nytimes3xbfgragh.onion/section/todayspaper}{Today's
Paper}

\href{/section/technology}{Technology}\textbar{}Amazon Is Jeff Bezos

\url{https://nyti.ms/3gl5nGd}

\begin{itemize}
\item
\item
\item
\item
\item
\end{itemize}

Advertisement

\protect\hyperlink{after-top}{Continue reading the main story}

Supported by

\protect\hyperlink{after-sponsor}{Continue reading the main story}

on Tech

\hypertarget{amazon-is-jeff-bezos}{%
\section{Amazon Is Jeff Bezos}\label{amazon-is-jeff-bezos}}

Bezos built Amazon into a central force in the world. This week he'll
have to answer for it.

\includegraphics{https://static01.graylady3jvrrxbe.onion/images/2020/07/28/business/28ontech-videostill/28ontech-videostill-videoSixteenByNineJumbo1600.png}

\href{https://www.nytimes3xbfgragh.onion/by/shira-ovide}{\includegraphics{https://static01.graylady3jvrrxbe.onion/images/2020/03/18/reader-center/author-shira-ovide/author-shira-ovide-thumbLarge-v2.png}}

By \href{https://www.nytimes3xbfgragh.onion/by/shira-ovide}{Shira Ovide}

\begin{itemize}
\item
  July 28, 2020
\item
  \begin{itemize}
  \item
  \item
  \item
  \item
  \item
  \end{itemize}
\end{itemize}

\emph{This article is part of the On Tech newsletter. You can}
\href{https://www.nytimes3xbfgragh.onion/newsletters/signup/OT}{\emph{sign
up here}} \emph{to receive it weekdays.}

A quarter-century ago, Jeff Bezos was a finance nerd with a tiny
bookselling website. You know what happened next.

Bezos's career arc tracks the shift of technology from a relatively
fringe industry into a central force in the world. And that's exactly
why Bezos and the chief executives of three other American tech stars
will be
\href{https://www.nytimes3xbfgragh.onion/2020/07/28/technology/amazon-apple-facebook-google-antitrust-hearing.html}{testifying
this week} at a congressional panel investigating possible abuses of
their power. The congressional hot seat shows how far the industry has
come.

I talked to
\href{https://www.nytimes3xbfgragh.onion/by/karen-weise}{Karen Weise},
my colleague who covers Amazon, about how Bezos thinks and the meaning
behind the scrutiny of Amazon.

\textbf{Shira: How much of Amazon is Jeff Bezos?}

\textbf{Karen:} He's far less hands-on than most people realize,
\href{https://www.nytimes3xbfgragh.onion/2020/04/22/technology/bezos-amazon-coronavirus.html}{at
least he was until recently}. But Amazon is a reflection of Bezos. It's
built on his ideals and ideas, and Bezos has a clarity of thought about
what the company should be.

Amazon is also structured around a set of principles and mechanisms that
Bezos created. These ``Jeff-isms'' can sound like meaningless
corporate-speak to an outsider, but many employees completely buy into
them, and the principles are infused into everything.

\textbf{Oh fun! What are some notable Jeff-isms?}

One is this idea of
``\href{https://www.inc.com/jeff-haden/amazon-founder-jeff-bezos-this-is-how-successful-people-make-such-smart-decisions.html}{one
way'' versus ``two way'' doors}. The first are irreversible decisions
that should be made with care, versus changeable choices that can be
made fast. People who worked at Amazon said they used that framework to
make life decisions, too.

Another Bezos principle is orienting every decision around what the
customer wants. It's an obsession that makes Amazon what it is. The
downside is acting in the best interests of shoppers can sometimes
justify actions that put pressure on Amazon's workers or marketplace
sellers.

\textbf{What's the significance of Bezos and the other tech C.E.O.s
testifying at the}
\textbf{\href{https://www.nytimes3xbfgragh.onion/2020/07/28/technology/amazon-apple-facebook-google-antitrust-hearing.html}{congressional
hearing}} \textbf{on potential abuses of power?}

For a long time these companies thought they didn't need to concern
themselves with policy, politics and regulation. Bezos certainly didn't.
\href{https://www.nytimes3xbfgragh.onion/2020/07/27/business/jeff-bezos-amazon-congress.html}{That's
changing now} because of the growing influence of technology everywhere.

Several of the most valuable companies in the world are tech companies.
Bezos is the world's richest person. Amazon is the second largest
corporate employer in the United States. My coverage of Amazon touches
on retail, transportation, labor, economics, consumer electronics and
the functioning of cities.

\textbf{Americans generally}
\textbf{\href{https://knightfoundation.org/reports/techlash-americas-growing-concern-with-major-technology-companies/}{don't
trust technology companies}, but Amazon}
\textbf{\href{https://www.usatoday.com/story/money/2019/04/02/companies-with-the-best-and-worst-reputations/39271085/}{has
a good reputation}.}

Yes, and the company has had a sense that customers' love and trust
would carry it through everything. But people can love shopping on
Amazon and not love its record on politics, labor or the environment.

We saw that in Amazon's hometown, Seattle, where the company
\href{https://www.nytimes3xbfgragh.onion/2019/10/18/technology/amazon-seattle-council-election.html}{put
a lot of money into City Council races} last year, and it completely
\href{https://www.seattletimes.com/seattle-news/politics/amazon-lost-the-seattle-city-council-elections-after-a-1-million-power-play-will-it-see-a-new-head-tax/}{backfired}.
People felt that the company was trying to buy the vote. In New York
City, there were people who believed that Amazon was trying to bully its
way into
\href{https://www.nytimes3xbfgragh.onion/2019/02/14/nyregion/amazon-hq2-queens.html}{building
a big corporate campus.}

\textbf{Does Amazon understand that people may love the product but
mistrust the company?}

It understands it
\href{https://www.seattletimes.com/business/amazon/amazon-draws-praise-protests-at-annual-shareholders-meeting}{intellectually}.
I don't think it does emotionally.

\emph{If you don't already get this newsletter in your inbox,}
\href{https://www.nytimes3xbfgragh.onion/newsletters/signup/OT}{\emph{please
sign up here}}\emph{.}

\begin{center}\rule{0.5\linewidth}{\linethickness}\end{center}

\hypertarget{coronavirus-misinformation-goes-wild-again}{%
\subsection{Coronavirus misinformation goes wild
again}\label{coronavirus-misinformation-goes-wild-again}}

In just a few hours yesterday, another video with false information
about the coronavirus spread like wildfire on Facebook before the
company started to stamp it out.

The video --- which I won't link to here, but you can find on Breitbart
News --- showed a group of purported doctors touting unproven
treatments.

One of the videos racked up 14 million views in six hours, my colleague
\href{https://twitter.com/kevinroose/status/1287906751069581318}{Kevin
Roose tweeted}. A few months ago, another video filled with coronavirus
conspiracies, called ``Plandemic,'' was
\href{https://www.nytimes3xbfgragh.onion/2020/05/20/technology/plandemic-movie-youtube-facebook-coronavirus.html}{watched
more than eight million times} on YouTube, Facebook and other spots over
multiple days.

Some of you may be wondering why it's so bad for people to watch a
couple of videos that go against the consensus of health experts. After
all, there's a lot about the virus we don't understand.

The problem is that it's not so easy to correct the record once someone
sees bogus ideas. We've seen that good information doesn't necessarily
undo bad information. Doses of falsehoods can make people doubt the
recommendations of proven health experts --- or even, the validity of
elections.

That's why Facebook, YouTube and other internet companies, which have
highlighted coronavirus information from authoritative sources such as
the Centers for Disease Control and Prevention, have said they also
would be aggressive about deleting false information related to the
virus. (On Tuesday, Twitter
\href{https://www.washingtonpost.com/nation/2020/07/28/trump-coronavirus-misinformation-twitter/}{temporarily
limited some functions} of the account of Donald Trump Jr., one of the
president's sons, as punishment for posting the video with misleading
information.)

And yet, this latest bogus video went wild, again making me wonder
whether Facebook and other popular internet sites are so sprawling that
the companies can't control even the most high-profile kinds of false
information.

\begin{center}\rule{0.5\linewidth}{\linethickness}\end{center}

\hypertarget{before-we-go-}{%
\subsection{Before we go \ldots{}}\label{before-we-go-}}

\begin{itemize}
\item
  \textbf{What to expect, and what's the big deal:} My
  \href{https://www.nytimes3xbfgragh.onion/2020/07/28/technology/amazon-apple-facebook-google-antitrust-hearing.html}{colleagues
  explain why} Congress is digging into how Amazon, Facebook, Google and
  Apple use their influence, and the possible trouble areas for each
  company.

  A related article: One of the questions about Apple is whether it
  drives up what we pay for online services because it charges up to a
  30 percent commission on many app transactions, and most app makers
  have little choice but to pay Apple. The company is now
  \href{https://www.nytimes3xbfgragh.onion/2020/07/28/technology/apple-app-store-airbnb-classpass.html}{starting
  to apply those fees to apps that never had to pay the fee before.}
\item
  \textbf{Is posting a glam selfie fun or activism?} My colleague Taylor
  Lorenz has a thoughtful article about an Instagram trend of women
  \href{https://www.nytimes3xbfgragh.onion/2020/07/27/style/challenge-accepted-instagram.html}{challenging
  one another to post black-and-white self portraits} as a celebration
  of female empowerment. To some participants this is lovely, but to
  others it's a shallow form of activism or an excuse to post a fun
  photo that might otherwise seem tacky in tough times.
\item
  \textbf{Netflix made a hit (and some haters) on multiple continents:}
  Netflix wants to be the first global television network, and one
  element of its strategy is to make series that appeal to people in
  many different countries. It seems to have done that with its reality
  show about Indian matchmakers set in India and the United States,
  Bloomberg News
  \href{https://www.bloomberg.com/news/articles/2020-07-24/netflix-s-reality-series-on-roving-indian-matchmaker-goes-viral}{writes}.
  Some South Asians love the series, ``Indian Matchmaking,'' while
  others believe it
  \href{https://www.thecut.com/2020/07/indian-matchmaking-review.html}{enforces
  outdated stereotypes}. **** Either way, the attention is good for
  Netflix.
\end{itemize}

\hypertarget{hugs-to-this}{%
\subsubsection{Hugs to this}\label{hugs-to-this}}

Get prepped for Wednesday's hearing by eyeballing a
\href{https://www.consumer.ftc.gov/sites/default/files/games/off-site/youarehere/pages/pdf/FTC-Competition_Antitrust-Laws.pdf}{bag
of sugar with a malicious glint in its eye} and other delightfully weird
drawings representing industrial monopolies that the U.S. government
once broke apart. (Thanks to my colleague Cecilia Kang for sharing this
document from the U.S. Federal Trade Commission.)

\begin{center}\rule{0.5\linewidth}{\linethickness}\end{center}

\emph{We want to hear from you. Tell us what you think of this
newsletter and what else you'd like us to explore. You can reach us at}
\href{mailto:ontech@NYTimes.com?subject=On\%20Tech\%20Feedback}{\emph{ontech@NYTimes.com.}}
**

\emph{If you don't already get this newsletter in your inbox,}
\href{https://www.nytimes3xbfgragh.onion/newsletters/signup/OT}{\emph{please
sign up here}}\emph{.}

Advertisement

\protect\hyperlink{after-bottom}{Continue reading the main story}

\hypertarget{site-index}{%
\subsection{Site Index}\label{site-index}}

\hypertarget{site-information-navigation}{%
\subsection{Site Information
Navigation}\label{site-information-navigation}}

\begin{itemize}
\tightlist
\item
  \href{https://help.nytimes3xbfgragh.onion/hc/en-us/articles/115014792127-Copyright-notice}{©~2020~The
  New York Times Company}
\end{itemize}

\begin{itemize}
\tightlist
\item
  \href{https://www.nytco.com/}{NYTCo}
\item
  \href{https://help.nytimes3xbfgragh.onion/hc/en-us/articles/115015385887-Contact-Us}{Contact
  Us}
\item
  \href{https://www.nytco.com/careers/}{Work with us}
\item
  \href{https://nytmediakit.com/}{Advertise}
\item
  \href{http://www.tbrandstudio.com/}{T Brand Studio}
\item
  \href{https://www.nytimes3xbfgragh.onion/privacy/cookie-policy\#how-do-i-manage-trackers}{Your
  Ad Choices}
\item
  \href{https://www.nytimes3xbfgragh.onion/privacy}{Privacy}
\item
  \href{https://help.nytimes3xbfgragh.onion/hc/en-us/articles/115014893428-Terms-of-service}{Terms
  of Service}
\item
  \href{https://help.nytimes3xbfgragh.onion/hc/en-us/articles/115014893968-Terms-of-sale}{Terms
  of Sale}
\item
  \href{https://spiderbites.nytimes3xbfgragh.onion}{Site Map}
\item
  \href{https://help.nytimes3xbfgragh.onion/hc/en-us}{Help}
\item
  \href{https://www.nytimes3xbfgragh.onion/subscription?campaignId=37WXW}{Subscriptions}
\end{itemize}
