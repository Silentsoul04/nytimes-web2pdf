Sections

SEARCH

\protect\hyperlink{site-content}{Skip to
content}\protect\hyperlink{site-index}{Skip to site index}

\href{https://www.nytimes3xbfgragh.onion/section/arts/design}{Art \&
Design}

\href{https://myaccount.nytimes3xbfgragh.onion/auth/login?response_type=cookie\&client_id=vi}{}

\href{https://www.nytimes3xbfgragh.onion/section/todayspaper}{Today's
Paper}

\href{/section/arts/design}{Art \& Design}\textbar{}Kerry James
Marshall's Black Birds Take Flight in a New Series

\url{https://nyti.ms/2P6Dx4o}

\begin{itemize}
\item
\item
\item
\item
\item
\item
\end{itemize}

Advertisement

\protect\hyperlink{after-top}{Continue reading the main story}

Supported by

\protect\hyperlink{after-sponsor}{Continue reading the main story}

The Great Read

\hypertarget{kerry-james-marshalls-black-birds-take-flight-in-a-new-series}{%
\section{Kerry James Marshall's Black Birds Take Flight in a New
Series}\label{kerry-james-marshalls-black-birds-take-flight-in-a-new-series}}

Inspired by John James Audubon, the painter explores the societal
``pecking order'' in two works that dovetail with ``this mystery about
whether or not Audubon himself was Black.''

\includegraphics{https://static01.graylady3jvrrxbe.onion/images/2020/07/31/arts/30kerryjames-audobon7/30kerryjames-audobon7-articleLarge.jpg?quality=75\&auto=webp\&disable=upscale}

\href{https://www.nytimes3xbfgragh.onion/by/ted-loos}{\includegraphics{https://static01.graylady3jvrrxbe.onion/images/2018/12/10/multimedia/author-ted-loos/author-ted-loos-thumbLarge.png}}

By \href{https://www.nytimes3xbfgragh.onion/by/ted-loos}{Ted Loos}

\begin{itemize}
\item
  July 29, 2020Updated 9:22 a.m. ET
\item
  \begin{itemize}
  \item
  \item
  \item
  \item
  \item
  \item
  \end{itemize}
\end{itemize}

About 10 years ago, the artist Kerry James Marshall caught a crow with
his bare hands.

The bird was cornered awkwardly near Mr. Marshall's home on the South
Side of Chicago, and curiosity got the better of him. ``I've always been
impressed by that kind of bird,'' he recalled the other day.

Mr. Marshall, widely acknowledged as one of the best painters working
today, wanted to photograph and take video of the crow, since he often
used such documentation as the basis for his work (he prefers props
now). So he grabbed it and took it home.

``At first he screamed like he was being murdered,'' Mr. Marshall said.
``The minute I put him by my side, he got quiet.''

On his second-floor deck, Mr. Marshall tied a cord to the crow's leg,
and provided a meal of mulberries ``so he wouldn't starve.'' He showed
the crow to his wife and documented the bird as planned. The next day,
he let the bird go.

Some days later, he saw the crow being menaced by a cat. Mr. Marshall
recalled: ``So I picked up a rock and threw it at the cat. And I swear
to God, that same bird, he stood there just looking at me. And I said,
`You better keep your butt off the ground because I'm not going to be
around to save you the next time.'''

The crow meeting, which started out as research, somehow edged into a
metaphysical encounter with deeper meanings, and it now informs Mr.
Marshall's newest series of paintings. His first two canvases officially
debut Thursday in an online show,
``\href{https://www.davidzwirner.com/viewing-room/studio-kerry-james-marshall}{Studio:
Kerry James Marshall},'' at David Zwirner Gallery.

As he has for decades, Mr. Marshall, 64, has harnessed history,
especially the history of painting, in these new canvases: They are his
reimagining of John James Audubon's landmark series,
\href{https://www.audubon.org/birds-of-america}{``Birds of America,''}
the painstakingly rendered 435 watercolors made in the first half of the
19th century, significant achievements in the fields of both ornithology
and art.

In one image, ``Black and part Black Birds in America: (Crow,
Goldfinch),'' a large crow dominates the canvas, clearly too large for
the birdhouses depicted behind it. There are glorious leaves, flowers
and a small goldfinch in the bottom left corner. In the other picture,
finished just last week, ``Black and part Black Birds in America:
(Grackle, Cardinal \& Rose-breasted Grosbeak)\emph{,''} a grackle is the
protagonist with a dainty birdhouse and brightly colored flowers. The
cardinal and grosbeak are both flying in different directions, giving
them a sense of being at cross purposes with the grackle.

\includegraphics{https://static01.graylady3jvrrxbe.onion/images/2020/07/30/arts/30kerryjames-audubon6/merlin_175012575_53e0b19b-45a5-4dc7-87b3-ef3ad800cc14-articleLarge.jpg?quality=75\&auto=webp\&disable=upscale}

``There's a disconnect between the house that's built and the birds,''
Mr. Marshall said of the crow and grackle. ``It's not designed for them,
you know?'' The scene considers, he said, ``the pecking order.''

The series itself has been brewing in Mr. Marshall's mind for eight or
nine years, he said, and he began painting the works just before
transmissions of the coronavirus accelerated in the United States in
March.

A casual bird enthusiast who has been fascinated by Audubon's
draughtsmanship since he was a child, Mr. Marshall has long put Black
protagonists at the center of his complex, richly layered compositions.
\href{https://www.artic.edu/artworks/137125/many-mansions}{``Many
Mansions''}(1994), one of his large-scale depictions of housing
projects, features three Black men gardening --- and, not incidentally,
there are two bluebirds holding up a banner, too. The pointed inclusion
of Black figures is part of what he has called a ``counter-archive'' to
the familiar, white-centered story of Western art.

For the new series, the images hinge on Audubon's own racial heritage:
Many people believe he was, as Mr. Marshall's title suggests, ``part
Black'' --- born in what is now Haiti, as Jean Rabin, to a white,
plantation-owning father and a Creole chambermaid who may have been of
racially-mixed descent. But, the theory goes, he was able to pass as
white.

Not everyone agrees on this narrative. The biographer Richard Rhodes,
author of
``\href{https://www.nytimes3xbfgragh.onion/2004/10/31/books/review/john-james-audubon-the-birdman-of-america.html}{John
James Audubon: The Making of an American,}'' said that Audubon's
biological mother was a white French chambermaid who died months after
childbirth. ``I know Audubon has been an inspiration to many people of
color,'' Mr. Rhodes said, adding that he felt ``terrible'' about not
being able to support the theory.

But for Mr. Marshall, what he called the ``mystery'' of Audubon's
parentage has fueled his imagination since 1976, when he saw the Los
Angeles County Museum of Art exhibition
\href{https://www.lacma.org/art/exhibition/two-centuries-black-american-art}{``Two
Centuries of Black American Art: 1750-1950.''} Organized by the curator
and scholar
\href{https://driskellcenterarchives.wordpress.com/tag/two-centuries-of-black-american-art-1750-1950/}{David
C. Driskell}, the show included Audubon's work, a surprise to many at
the time.

``I didn't know what to make of it, honestly,'' said Mr. Marshall, who
was a student at what is now the Otis College of Art and Design. ``If
somebody did the research and put it in a book, then maybe it must be
true. And I never forgot that assertion was made.''

Image

Kerry James Marshall, ``Black and part Black Birds in America: (Grackle,
Cardinal \& Rose-breasted Grosbeak),'' 2020.~Credit...Kerry James
Marshall and David Zwirner

He referenced the notorious ``one drop rule'' --- that someone with one
drop of Black blood made the person Black.

``That's the key to the whole thing,'' Mr. Marshall said of his new
series, noting that in ``Black and part Black'' he included a goldfinch,
a bird that also has black markings but is named for its yellow color.
``And it dovetails with this mystery about whether or not Audubon
himself was Black.''

\href{https://www.moca.org/exhibition/kerry-james-marshall-mastry}{Helen
Molesworth}, who was a co-organizer of a 2016-17 retrospective of Mr.
Marshall's work,
``\href{https://www.nytimes3xbfgragh.onion/2016/10/21/arts/design/kerry-james-marshalls-paintings-show-what-it-means-to-be-black-in-america.html}{Kerry
James Marshall: Mastry},'' when she was chief curator of the Museum of
Contemporary Art in Los Angeles, said that his foregrounding of birds
was significant.

``He's known as a figurative painter, but in these he has left the human
figure out,'' said Ms. Molesworth, who has seen photographs of the new
paintings.

``His paintings have been filled with birds all along,'' she added. ``If
you wanted to go birding in a Kerry James Marshall show, you could.
People were paying so much attention to the human figure in his work,
the birds may have gone unexamined.''

Examples include
\href{https://www.metmuseum.org/art/collection/search/668312}{``They
Know That I Know''} (1992),
\href{https://www.metmuseum.org/art/collection/search/668319}{``Bang''}
(1994) and
``\href{https://mcachicago.org/Collection/Items/2003/Kerry-James-Marshall-7am-Sunday-Morning-2003}{7
am Sunday Morning''} (2003), all depicting birds as supporting players.

Ms. Molesworth, a birder herself, said the new works were evidence that
Mr. Marshall is a ``polymath, deeply interested in a lot of things. He
thinks the world is filled with knowledge, and all of it is available to
him.''

His deep dives started early. Born in Birmingham, Ala., in 1955, Mr.
Marshall moved to the South Central area of Los Angeles when he was a
child, and the public library on Central Avenue was a primary
destination as of age 8 or 9.

``I'd pick out books by the stack,'' he said. ``You had a limit of 10,
so I would get 10 every time I went.''

Books depicting reptiles, birds and insects were first, and soon after
came Audubon's images. ``They appealed to me for two reasons,'' he said.
``One, the way he set up the images and tableaus to create some drama,
they were beautifully done --- and ** they were hand-drawn.''

The Belgian painter
\href{https://www.davidzwirner.com/artists/luc-tuymans}{Luc Tuymans}, a
friend of Mr. Marshall's, noted that he is an intensely ``deliberate''
painter, and that Audubon's obsessive meticulousness would naturally
have appealed.

James Rondeau, the director of the
\href{https://www.artic.edu/artists/47905/kerry-james-marshall}{Art
Institute of Chicago}, hadn't seen the new series yet, but as someone
who knows Mr. Marshall well, he said it was typical of the artist to
attempt to highlight ``not only the Black experience, but Black
expertise,'' referring to painting as well as ornithology.

Image

``Black and part Black Birds in America: (Grackle, Cardinal \&
Rose-breasted Grosbeak),'' in progress at Kerry James Marshall's studio
in Chicago.~~Credit...Kerry James Marshall and David Zwirner

Mr. Marshall was well underway with his series when, in May, Christian
Cooper, a director of New York City Audubon, who is
Black,\href{https://www.nytimes3xbfgragh.onion/2020/06/14/nyregion/central-park-amy-cooper-christian-racism.html}{was
birding in Central Park}, and he asked a white woman to leash her dog.
She threatened to call the police and tell them ``an African-American
man is threatening my life.'' The collision exposed a deep vein of
racial bias and was a blatant example of the routine humiliations in the
daily life of African-Americans.

Mr. Marshall's reaction to news of the incident did not dwell on the
conflict. Rather, he said he felt some kind of kinship to Mr. Cooper ---
who was in the park pursuing a field he knows well and had memorized
``The Birds of North America'' when he was 11 --- and related to
expertise that transcends race.

``There are assumptions about the kinds of things that Black people do
and are interested in,'' he said, adding that he wanted to push back on
the idea that ``all Black people's lives are consumed by trauma. I'm not
thinking about trauma all day.''

What consumes this artist is paint itself.

Mr. Marshall can talk about color theory for hours. The crow and the
grackle in the ``Black and part Black'' pictures are particularly
nuanced.

``I have to be able to show that it's not just a silhouette; it has
volume, it breathes,'' he said. ``And so I had to figure out how to make
that happen but not diminish the fundamental blackness of the thing.''

To do that, Mr. Marshall painstakingly adjusts both the chroma (the
warmth or coolness) and the value (the amount of light or dark) by
mixing colors like raw sienna, chrome green, cobalt blue, and violet
with black pigments.

It's among the things that Mr. Tuymans noticed first in the 1990s, when
he got to know Mr. Marshall and his work. He called Mr. Marshall's
attention to blackness, at a time when it was a more radical move,
``decisive and unapologetic.''

True, but in painting, bravery only makes a difference if the artist has
the tools, and the focus, to get the message across.

``The picture plane is the site of every action,'' Mr. Marshall said. He
seemed to be speaking not only about the painting process but also how
he conducts his whole life --- after all, this is a man who captured a
live crow to get to know it better. ``How things occupy that space,'' he
added, ``matters more than anything.''

Advertisement

\protect\hyperlink{after-bottom}{Continue reading the main story}

\hypertarget{site-index}{%
\subsection{Site Index}\label{site-index}}

\hypertarget{site-information-navigation}{%
\subsection{Site Information
Navigation}\label{site-information-navigation}}

\begin{itemize}
\tightlist
\item
  \href{https://help.nytimes3xbfgragh.onion/hc/en-us/articles/115014792127-Copyright-notice}{©~2020~The
  New York Times Company}
\end{itemize}

\begin{itemize}
\tightlist
\item
  \href{https://www.nytco.com/}{NYTCo}
\item
  \href{https://help.nytimes3xbfgragh.onion/hc/en-us/articles/115015385887-Contact-Us}{Contact
  Us}
\item
  \href{https://www.nytco.com/careers/}{Work with us}
\item
  \href{https://nytmediakit.com/}{Advertise}
\item
  \href{http://www.tbrandstudio.com/}{T Brand Studio}
\item
  \href{https://www.nytimes3xbfgragh.onion/privacy/cookie-policy\#how-do-i-manage-trackers}{Your
  Ad Choices}
\item
  \href{https://www.nytimes3xbfgragh.onion/privacy}{Privacy}
\item
  \href{https://help.nytimes3xbfgragh.onion/hc/en-us/articles/115014893428-Terms-of-service}{Terms
  of Service}
\item
  \href{https://help.nytimes3xbfgragh.onion/hc/en-us/articles/115014893968-Terms-of-sale}{Terms
  of Sale}
\item
  \href{https://spiderbites.nytimes3xbfgragh.onion}{Site Map}
\item
  \href{https://help.nytimes3xbfgragh.onion/hc/en-us}{Help}
\item
  \href{https://www.nytimes3xbfgragh.onion/subscription?campaignId=37WXW}{Subscriptions}
\end{itemize}
