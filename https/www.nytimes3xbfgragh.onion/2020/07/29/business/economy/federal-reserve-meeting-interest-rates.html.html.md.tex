Sections

SEARCH

\protect\hyperlink{site-content}{Skip to
content}\protect\hyperlink{site-index}{Skip to site index}

\href{https://www.nytimes3xbfgragh.onion/section/business/economy}{Economy}

\href{https://myaccount.nytimes3xbfgragh.onion/auth/login?response_type=cookie\&client_id=vi}{}

\href{https://www.nytimes3xbfgragh.onion/section/todayspaper}{Today's
Paper}

\href{/section/business/economy}{Economy}\textbar{}Federal Reserve
Leaves Rates Near Zero as Economic Recovery Sputters

\url{https://nyti.ms/3hLhjkG}

\begin{itemize}
\item
\item
\item
\item
\item
\end{itemize}

\hypertarget{the-coronavirus-outbreak}{%
\subsubsection{\texorpdfstring{\href{https://www.nytimes3xbfgragh.onion/news-event/coronavirus?name=styln-coronavirus-markets\&region=TOP_BANNER\&variant=undefined\&block=storyline_menu_recirc\&action=click\&pgtype=Article\&impression_id=623e47b0-e3a1-11ea-912a-61b8543016b1}{The
Coronavirus
Outbreak}}{The Coronavirus Outbreak}}\label{the-coronavirus-outbreak}}

\begin{itemize}
\tightlist
\item
  live\href{https://www.nytimes3xbfgragh.onion/2020/08/21/world/covid-19-coronavirus.html?name=styln-coronavirus-markets\&region=TOP_BANNER\&variant=undefined\&block=storyline_menu_recirc\&action=click\&pgtype=Article\&impression_id=623e6ec0-e3a1-11ea-912a-61b8543016b1}{Latest
  Updates}
\item
  \href{https://www.nytimes3xbfgragh.onion/interactive/2020/us/coronavirus-us-cases.html?name=styln-coronavirus-markets\&region=TOP_BANNER\&variant=undefined\&block=storyline_menu_recirc\&action=click\&pgtype=Article\&impression_id=623e6ec1-e3a1-11ea-912a-61b8543016b1}{Maps
  and Cases}
\item
  \href{https://www.nytimes3xbfgragh.onion/interactive/2020/science/coronavirus-vaccine-tracker.html?name=styln-coronavirus-markets\&region=TOP_BANNER\&variant=undefined\&block=storyline_menu_recirc\&action=click\&pgtype=Article\&impression_id=623e6ec2-e3a1-11ea-912a-61b8543016b1}{Vaccine
  Tracker}
\item
  \href{https://www.nytimes3xbfgragh.onion/2020/08/19/us/colleges-closing-covid.html?name=styln-coronavirus-markets\&region=TOP_BANNER\&variant=undefined\&block=storyline_menu_recirc\&action=click\&pgtype=Article\&impression_id=623e6ec3-e3a1-11ea-912a-61b8543016b1}{Colleges
  Closing}
\item
  \href{https://www.nytimes3xbfgragh.onion/live/2020/08/20/business/stock-market-today-coronavirus?name=styln-coronavirus-markets\&region=TOP_BANNER\&variant=undefined\&block=storyline_menu_recirc\&action=click\&pgtype=Article\&impression_id=623e6ec4-e3a1-11ea-912a-61b8543016b1}{Economy}
\end{itemize}

Advertisement

\protect\hyperlink{after-top}{Continue reading the main story}

Supported by

\protect\hyperlink{after-sponsor}{Continue reading the main story}

\hypertarget{federal-reserve-leaves-rates-near-zero-as-economic-recovery-sputters}{%
\section{Federal Reserve Leaves Rates Near Zero as Economic Recovery
Sputters}\label{federal-reserve-leaves-rates-near-zero-as-economic-recovery-sputters}}

Federal Reserve officials made clear that they will work to bolster the
economy as the pandemic hurts business activity and the job market.

\includegraphics{https://static01.graylady3jvrrxbe.onion/images/2020/07/29/business/29DC-Fed-01/merlin_171588552_71b5518c-f3c6-4426-9279-319d6a501b3f-articleLarge.jpg?quality=75\&auto=webp\&disable=upscale}

\href{https://www.nytimes3xbfgragh.onion/by/jeanna-smialek}{\includegraphics{https://static01.graylady3jvrrxbe.onion/images/2020/07/03/reader-center/author-jeanna-smialek/author-jeanna-smialek-thumbLarge.png}}

By \href{https://www.nytimes3xbfgragh.onion/by/jeanna-smialek}{Jeanna
Smialek}

\begin{itemize}
\item
  July 29, 2020
\item
  \begin{itemize}
  \item
  \item
  \item
  \item
  \item
  \end{itemize}
\end{itemize}

WASHINGTON --- The Federal Reserve left interest rates near zero on
Wednesday and Jerome H. Powell, the Fed chair, predicted a long road
ahead as a recent spike in virus cases saps momentum from the nascent
economic recovery.

``The path forward for the economy is extraordinarily uncertain and will
depend in large part on our success in keeping the virus in check,'' Mr.
Powell said at a news conference following the Fed's two-day meeting,
noting that infections have surged since late June and the ``pace of
recovery looks like it has slowed.''

Mr. Powell said policymakers needed more data before drawing firm
conclusions about the scope of the pullback, but he noted that debit and
credit card spending were slowing and labor market indicators suggested
that recent job gains might be weakening. More than
\href{https://fred.stlouisfed.org/series/PAYEMS\#:~:text=Notes\%3A,and\%20the\%20unincorporated\%20self\%2Demployed.}{14
million} people who held jobs in February are no longer employed, Mr.
Powell said, warning that it will take a while for workers in certain
industries, like restaurants, hotels and travel, to find new jobs.

``There's probably going to be a long tail where a large number of
people are struggling to get back to work,'' he said, adding that the
Fed was ``not even thinking about thinking about thinking about''
raising rates.

The labor market rebound ``is going to take a while,'' he said, and
``we're going to be there for all of that.''

While the Fed took no major actions on Wednesday, Mr. Powell's comments
underlined both the peril ahead for American workers and the reality
that interest rates are likely to be very low --- making money cheap to
borrow --- for an extended period of time. Stock prices climbed
following his remarks as investors took heart in the Fed's patient
stance.

Ahead of Mr. Powell's comments, the central bank reiterated in its
\href{https://www.federalreserve.gov/newsevents/pressreleases/monetary20200729a.htm}{post-meeting
statement} that the Fed would keep low rates in place ``until it is
confident that the economy has weathered recent events.''

\hypertarget{latest-updates-the-coronavirus-outbreak-and-the-economy}{%
\section{\texorpdfstring{\href{https://www.nytimes3xbfgragh.onion/live/2020/08/20/business/stock-market-today-coronavirus?action=click\&pgtype=Article\&state=default\&region=MAIN_CONTENT_1\&context=storylines_live_updates}{Latest
Updates: The Coronavirus Outbreak and the
Economy}}{Latest Updates: The Coronavirus Outbreak and the Economy}}\label{latest-updates-the-coronavirus-outbreak-and-the-economy}}

\href{https://www.nytimes3xbfgragh.onion/live/2020/08/20/business/stock-market-today-coronavirus?action=click\&pgtype=Article\&state=default\&region=MAIN_CONTENT_1\&context=storylines_live_updates\#american-airlines-to-stop-flights-to-15-cities-after-government-aid-ends}{21h
ago}

\href{https://www.nytimes3xbfgragh.onion/live/2020/08/20/business/stock-market-today-coronavirus?action=click\&pgtype=Article\&state=default\&region=MAIN_CONTENT_1\&context=storylines_live_updates\#american-airlines-to-stop-flights-to-15-cities-after-government-aid-ends}{American
Airlines to stop flights to 15 cities after government aid ends.}

\href{https://www.nytimes3xbfgragh.onion/live/2020/08/20/business/stock-market-today-coronavirus?action=click\&pgtype=Article\&state=default\&region=MAIN_CONTENT_1\&context=storylines_live_updates\#without-school-plays-and-assemblies-a-technicians-livelihood-withers}{22h
ago}

\href{https://www.nytimes3xbfgragh.onion/live/2020/08/20/business/stock-market-today-coronavirus?action=click\&pgtype=Article\&state=default\&region=MAIN_CONTENT_1\&context=storylines_live_updates\#without-school-plays-and-assemblies-a-technicians-livelihood-withers}{Without
school plays and assemblies, a technician's livelihood withers.}

\href{https://www.nytimes3xbfgragh.onion/live/2020/08/20/business/stock-market-today-coronavirus?action=click\&pgtype=Article\&state=default\&region=MAIN_CONTENT_1\&context=storylines_live_updates\#finding-a-job-after-a-long-search-but-settling-for-less-pay}{22h
ago}

\href{https://www.nytimes3xbfgragh.onion/live/2020/08/20/business/stock-market-today-coronavirus?action=click\&pgtype=Article\&state=default\&region=MAIN_CONTENT_1\&context=storylines_live_updates\#finding-a-job-after-a-long-search-but-settling-for-less-pay}{Finding
a job after a long search, but settling for less pay.}

\href{https://www.nytimes3xbfgragh.onion/live/2020/08/20/business/stock-market-today-coronavirus?action=click\&pgtype=Article\&state=default\&region=MAIN_CONTENT_1\&context=storylines_live_updates}{See
more updates}

More live coverage:
\href{https://www.nytimes3xbfgragh.onion/2020/08/21/world/covid-19-coronavirus.html?action=click\&pgtype=Article\&state=default\&region=MAIN_CONTENT_1\&context=storylines_live_updates}{Global}

The Fed's announcement
\href{https://www.nytimes3xbfgragh.onion/2020/07/28/business/us-lawmakers-deliberate-over-another-aid-package.html}{came
as another round of tense negotiations continued in Congress} over
providing more support to workers and businesses still struggling amid
the pandemic, including whether to extend an extra \$600 per week in
unemployment benefits that is set to expire this week.

Mr. Powell said the support lawmakers had already provided had been
critical for workers and businesses and, in turn, the economy. While he
did not weigh in on how high unemployment insurance benefits should be
set, he said it would be important to help the large number of workers
who were likely to be displaced even if the economy reopened
successfully.

``There won't be enough jobs for them --- those people will need
support,'' he said, noting that government policy so far has ``kept
people in their homes, it's kept businesses in business.''

Mr. Powell said both Congress and the central bank would need to do more
in the months ahead.

Since March, the Fed has put in place a series of measures to help
cushion the economic fallout as businesses close or reduce capacity and
as shoppers stay home from malls and movie theaters to control the
spread of the coronavirus. The central bank has rolled out nine
emergency lending programs, which are meant to keep credit flowing to
businesses and state and local governments, and is purchasing
government-backed bonds to keep markets functioning normally. It has
slashed interest rates to rock bottom to entice borrowing and spending.

On Tuesday,
\href{https://www.nytimes3xbfgragh.onion/2020/07/28/business/economy/coronavirus-federal-reserve-policy.html}{officials
announced} that they would extend their emergency lending programs
through the end of the year. Seven of the programs were initially set to
expire around the end of September, but could still be needed past that
as coronavirus cases have continued to rise.

The Fed said on Wednesday that it would also extend its programs meant
to keep dollar funding readily available to foreign central banks
through March.

Mr. Powell said it was important that the facilities stay in place
``until we're very confident that the turmoil from the pandemic and the
economic fallout are behind us.''

That could take time. The unemployment rate, while falling, remains
historically high at 11.1 percent. Initial jobless claims
\href{https://www.nytimes3xbfgragh.onion/2020/07/23/business/economy/unemployment-economy-coronavirus.html}{ticked
up last week} after months of gradual improvement, stoking concerns that
the economy might be backsliding. Data
\href{https://www.washingtonpost.com/business/2020/07/23/permanent-business-closures-yelp/}{suggest}
that many
\href{https://www.nytimes3xbfgragh.onion/2020/07/13/business/small-businesses-coronavirus.html}{businesses}
are beginning to close permanently.

The job losses are hitting disadvantaged communities particularly hard.
The Fed's own surveys have shown that poorer people were
\href{https://www.nytimes3xbfgragh.onion/2020/05/14/business/economy/coronavirus-jobless-unemployment.html}{more
likely to lose jobs}, and those with less education often did not have
the option to work from home. The jobless rate for Black workers has
skyrocketed to more than
\href{https://www.bls.gov/news.release/empsit.t02.htm}{15 percent}, and
the unemployment rate for Black men continued to tick up in June even as
the rate for other racial and gender groups began to fall.

Mr. Powell acknowledged the unequal brunt of the pandemic on Wednesday,
and said that what the Fed can do is focus on fostering a strong labor
market.

``What we're trying to do is create an environment, in the financial
markets and in the economy, where those people have the best chance they
can have to go back to work to their old job or to a new job,'' Mr.
Powell said.

While Fed officials' June economic projections suggested that they
expected
\href{https://www.federalreserve.gov/monetarypolicy/files/fomcprojtabl20200610.pdf}{unemployment}
to fall below 10 percent by the end of the year, based on the central
forecast, policymakers made it clear that conditions were extremely
uncertain. The recent surge in infections could temper the more
optimistic takes.

The central bank's policies do seem to be offering support, at least
around the edges. House buying has ticked up, fueled by cheap mortgage
rates, and the
\href{https://www.bloomberg.com/news/articles/2020-07-28/homeownership-rate-in-the-u-s-soars-to-highest-level-since-2008?sref=oZtxD6sa}{U.S.
homeownership rate} is now at levels last seen before the 2008 financial
crisis.

Key credit markets have calmed down after a disorderly March and April,
as has the market for U.S. government debt.

While investors expect the Fed to eventually make a more concrete
commitment to maintaining low rates for months or years --- by pegging
them to the unemployment or inflation rate, or by pledging to keep rates
low until a calendar date --- Mr. Powell said on Wednesday that
conversations about such approaches would continue at future meetings.

He also said the Federal Open Market Committee's longer-run framework
review, which could guide the central bank's strategies, would be
completed in the near future. Some economists took that news to mean
that more action is coming at the Fed's
\href{https://www.federalreserve.gov/monetarypolicy/fomccalendars.htm}{Sept.
15-16} meeting.

``The July F.O.M.C. meeting was expected to be a placeholder event until
more important decisions are made at the next meeting in September,''
Michael Feroli, the chief U.S. economist at J.P. Morgan, said in a note.
``The committee met those expectations.''

Advertisement

\protect\hyperlink{after-bottom}{Continue reading the main story}

\hypertarget{site-index}{%
\subsection{Site Index}\label{site-index}}

\hypertarget{site-information-navigation}{%
\subsection{Site Information
Navigation}\label{site-information-navigation}}

\begin{itemize}
\tightlist
\item
  \href{https://help.nytimes3xbfgragh.onion/hc/en-us/articles/115014792127-Copyright-notice}{©~2020~The
  New York Times Company}
\end{itemize}

\begin{itemize}
\tightlist
\item
  \href{https://www.nytco.com/}{NYTCo}
\item
  \href{https://help.nytimes3xbfgragh.onion/hc/en-us/articles/115015385887-Contact-Us}{Contact
  Us}
\item
  \href{https://www.nytco.com/careers/}{Work with us}
\item
  \href{https://nytmediakit.com/}{Advertise}
\item
  \href{http://www.tbrandstudio.com/}{T Brand Studio}
\item
  \href{https://www.nytimes3xbfgragh.onion/privacy/cookie-policy\#how-do-i-manage-trackers}{Your
  Ad Choices}
\item
  \href{https://www.nytimes3xbfgragh.onion/privacy}{Privacy}
\item
  \href{https://help.nytimes3xbfgragh.onion/hc/en-us/articles/115014893428-Terms-of-service}{Terms
  of Service}
\item
  \href{https://help.nytimes3xbfgragh.onion/hc/en-us/articles/115014893968-Terms-of-sale}{Terms
  of Sale}
\item
  \href{https://spiderbites.nytimes3xbfgragh.onion}{Site Map}
\item
  \href{https://help.nytimes3xbfgragh.onion/hc/en-us}{Help}
\item
  \href{https://www.nytimes3xbfgragh.onion/subscription?campaignId=37WXW}{Subscriptions}
\end{itemize}
