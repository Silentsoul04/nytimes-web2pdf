Sections

SEARCH

\protect\hyperlink{site-content}{Skip to
content}\protect\hyperlink{site-index}{Skip to site index}

\href{https://www.nytimes3xbfgragh.onion/section/climate}{Climate}

\href{https://myaccount.nytimes3xbfgragh.onion/auth/login?response_type=cookie\&client_id=vi}{}

\href{https://www.nytimes3xbfgragh.onion/section/todayspaper}{Today's
Paper}

\href{/section/climate}{Climate}\textbar{}It's Summer. Let's Talk About
Hockey.

\url{https://nyti.ms/30af2JZ}

\begin{itemize}
\item
\item
\item
\item
\item
\end{itemize}

\href{https://www.nytimes3xbfgragh.onion/section/climate?action=click\&pgtype=Article\&state=default\&region=TOP_BANNER\&context=storylines_menu}{Climate
and Environment}

\begin{itemize}
\tightlist
\item
  \href{https://www.nytimes3xbfgragh.onion/2020/07/30/climate/sea-level-inland-floods.html?action=click\&pgtype=Article\&state=default\&region=TOP_BANNER\&context=storylines_menu}{Rising
  Seas}
\item
  \href{https://www.nytimes3xbfgragh.onion/interactive/2020/climate/trump-environment-rollbacks.html?action=click\&pgtype=Article\&state=default\&region=TOP_BANNER\&context=storylines_menu}{Trump's
  Changes}
\item
  \href{https://www.nytimes3xbfgragh.onion/interactive/2020/04/19/climate/climate-crash-course-1.html?action=click\&pgtype=Article\&state=default\&region=TOP_BANNER\&context=storylines_menu}{Climate
  101}
\item
  \href{https://www.nytimes3xbfgragh.onion/interactive/2018/08/30/climate/how-much-hotter-is-your-hometown.html?action=click\&pgtype=Article\&state=default\&region=TOP_BANNER\&context=storylines_menu}{Is
  Your Hometown Hotter?}
\item
  \href{https://www.nytimes3xbfgragh.onion/newsletters/climate-change?action=click\&pgtype=Article\&state=default\&region=TOP_BANNER\&context=storylines_menu}{Newsletter}
\end{itemize}

Advertisement

\protect\hyperlink{after-top}{Continue reading the main story}

Supported by

\protect\hyperlink{after-sponsor}{Continue reading the main story}

Climate Fwd:

\hypertarget{its-summer-lets-talk-about-hockey}{%
\section{It's Summer. Let's Talk About
Hockey.}\label{its-summer-lets-talk-about-hockey}}

Also this week, health benefits from cutting emissions

By \href{https://www.nytimes3xbfgragh.onion/by/david-waldstein}{David
Waldstein}, Veronica Penney and
\href{https://www.nytimes3xbfgragh.onion/by/lisa-friedman}{Lisa
Friedman}

\begin{itemize}
\item
  July 29, 2020
\item
  \begin{itemize}
  \item
  \item
  \item
  \item
  \item
  \end{itemize}
\end{itemize}

\emph{Welcome to the} \emph{\textbf{Climate Fwd:}} \emph{newsletter. The
New York Times climate team emails readers once a week with stories and
insights about climate change.}
\href{https://www.nytimes3xbfgragh.onion/newsletters/climate-change}{\emph{Sign
up here}} \emph{to get it in your inbox.}

\includegraphics{https://static01.graylady3jvrrxbe.onion/images/2020/07/28/climate/28cli-newsletter-hockey/merlin_134340137_ebca4404-4563-49ad-a299-2fea43001f6f-articleLarge.jpg?quality=75\&auto=webp\&disable=upscale}

\href{https://www.nytimes3xbfgragh.onion/by/david-waldstein}{\includegraphics{https://static01.graylady3jvrrxbe.onion/images/2018/02/20/multimedia/author-david-waldstein/author-david-waldstein-thumbLarge.jpg}}

By \href{https://www.nytimes3xbfgragh.onion/by/david-waldstein}{David
Waldstein}

Summer might seem like an odd time to talk about hockey, but we have two
good reasons: First, the National Hockey League is scheduled to start
playoffs on Saturday in Edmonton and Toronto. Second, a team of
researchers in Canada published
\href{https://onlinelibrary.wiley.com/doi/10.1111/cag.12640}{a report
this month looking at climate change and outdoor rinks}.

After a four-month hiatus because of the coronavirus pandemic, the
outcome of the N.H.L. season is anyone's guess. The takeaway from the
study is more clear: Rising global temperatures are puncturing the
viability of homemade rinks because there are fewer days each winter
when it's cold enough to maintain them.

The researchers looked at backyard rinks in the Original Six N.H.L.
cities --- Boston, Chicago, Detroit, Montreal, New York and Toronto ---
and found a steady decline in the number of days when the weather was
cold enough to make ice and skate on the rinks.

Toronto saw the greatest reduction in ice time. In the winter of
1942-43, the first year of the Original Six era, there were close to 60
days when Torontonians could expect high-quality skating conditions in
backyard rinks. Last year, there were about 20.

``Starting in the 1980s you see this downward slope on virtually every
indicator for all the cities,'' said Robert McLeman, an environmental
scientist at Wilfrid Laurier University in Waterloo, Ontario, and one of
the authors of the study. ``Skating seasons are shifting toward shorter
seasons and more of a yo-yo effect with a mild winter followed by a cold
one. You did not see that much prior to 1980.''

For the past seven years the scientists have been collecting data from
outdoor rink owners in a project known as
\href{https://twitter.com/rinkwatchorg?lang=en}{RinkWatch} to determine
the number of high-quality skating days available to them. To discover
the conditions in the previous seven decades, they gathered data from
weather stations in those towns to determine how many skateable days
were available.

They used outdoor rinks as the vehicle to highlight the data because
they capture the imagination, especially in Canada. Dr. McLeman said
that connecting climate data to outdoor rinks in the six original N.H.L.
cities made the study, published in The Canadian Geographer, an exercise
in science communication as much as a research paper.

By making it about hockey, it becomes more relatable.

``It's hard for individuals to sense change in average temperature,''
Dr. McLeman said. ``It's easier for us as humans to sense changes in our
own behaviors or activities.''

\href{https://www.nytimes3xbfgragh.onion/section/climate?action=click\&pgtype=Article\&state=default\&region=MAIN_CONTENT_1\&context=storylines_keepup}{}

\hypertarget{climate-and-environment-}{%
\subsubsection{Climate and Environment
›}\label{climate-and-environment-}}

\hypertarget{keep-up-on-the-latest-climate-news}{%
\paragraph{Keep Up on the Latest Climate
News}\label{keep-up-on-the-latest-climate-news}}

Updated July 30, 2020

Here's what you need to know about the latest climate change news this
week:

\begin{itemize}
\item
  \begin{itemize}
  \tightlist
  \item
    \href{https://www.nytimes3xbfgragh.onion/2020/07/30/climate/bangladesh-floods.html?action=click\&pgtype=Article\&state=default\&region=MAIN_CONTENT_1\&context=storylines_keepup}{Floods
    in}\href{https://www.nytimes3xbfgragh.onion/2020/07/30/climate/bangladesh-floods.html?action=click\&pgtype=Article\&state=default\&region=MAIN_CONTENT_1\&context=storylines_keepup}{Bangladesh}
    are punishing the people least responsible for climate change.
  \item
    As climate change raises sea levels,
    \href{https://www.nytimes3xbfgragh.onion/2020/07/30/climate/sea-level-inland-floods.html?action=click\&pgtype=Article\&state=default\&region=MAIN_CONTENT_1\&context=storylines_keepup}{storm
    surges and high tides} are likely to push farther inland.
  \item
    The E.P.A. inspector general plans to investigate whether a rollback
    of fuel efficiency standards
    \href{https://www.nytimes3xbfgragh.onion/2020/07/27/climate/trump-fuel-efficiency-rule.html?action=click\&pgtype=Article\&state=default\&region=MAIN_CONTENT_1\&context=storylines_keepup}{violated
    government rules}.
  \end{itemize}
\end{itemize}

The scientists also measured the time period when good skating was
possible. For example, in Montreal, where the N.H.L.'s Canadiens are an
institution, the outdoor skating season used to begin in late November.
It has mostly shifted to early December and it did not start until
January a few times in recent years.

Since 1995, there have been six late starts to the outdoor rink season
in Montreal, whereas in the previous 45 years there had only been only
one.

Unless governments take action to sharply reduce greenhouse gas
emissions, scientists say, that trend will continue. And, with fewer
good skating days each winter, according to Dr. McLeman, people will be
less likely to make the considerable effort to build backyard rinks.

That means hockey could become a sport for the privileged few.

``You take away the outdoor rink,'' Dr. McLeman said, ``and only the
rich kids get to play.''

\begin{center}\rule{0.5\linewidth}{\linethickness}\end{center}

Image

Small people got some fresh air in the Boston Public Garden in
May.Credit...Cassandra Klos for The New York Times

\hypertarget{air-pollution-and-childrens-health}{%
\subsection{Air pollution and children's
health}\label{air-pollution-and-childrens-health}}

By Veronica Penney

A program in the Eastern United States to reduce greenhouse gas
emissions from power plants also has significant health benefits for
children in the region, according to a new study led by researchers at
Columbia University.

The emissions program, known as the
\href{https://www.rggi.org/program-overview-and-design/elements}{Regional
Greenhouse Gas Initiative}, was established in 2009 and sets a cap on
planet-warming carbon dioxide pollution, giving plants emissions
allowances via permit auctions and
\href{https://www.nytimes3xbfgragh.onion/interactive/2019/04/02/climate/pricing-carbon-emissions.html}{allowing
them to trade those allowances}.

According to the study, which was
\href{https://ehp.niehs.nih.gov/doi/full/10.1289/EHP6706}{published
Wednesday in the journal Environmental Health Perspectives} and included
researchers from Boston University and private groups, the program may
also have reduced the number of underweight births in the region and
lowered the incidence of asthma and autism by hundreds of cases.

Scientists say that's because the tiny pollution particles known as fine
particulate matter, a byproduct of burning fossil fuels, when inhaled by
a pregnant mother, can cross the placenta barrier and interfere with
fetal development.

Frederica Perera, the lead researcher on the study, said that fine
particulate matter can wreak havoc on the rapid and ``highly
choreographed'' development of unborn babies.

The study built on
\href{https://www.abtassociates.com/insights/publications/report/analysis-of-the-public-health-impacts-of-the-regional-greenhouse-gas}{earlier
estimates} of how much more polluted air in Eastern states would have
been without the agreement. Researchers paired that data with population
measurements and estimated that cleaner air between 2009 and 2014 could
have prevented nearly 100 children in the region from developing autism
spectrum disorder, and over 500 from developing childhood asthma. The
study also found that more than 100 preterm births and more than 50
cases of low birth weight may have been avoided.

The regional initiative includes Connecticut, Delaware, Maine, Maryland,
Massachusetts, New Hampshire, New Jersey, New York, Rhode Island and
Vermont.

The researchers looked at county-level data in those states and some
neighboring downwind states. Neighborhood-level analysis would most
likely have shown disparities in exposure to pollution based on race and
income, Dr. Perera said, but air quality measurements at that level are
generally unavailable outside big cities and would have been too
expensive for researchers to compile.

Jonathan Buonocore, a research scientist at the Harvard T.H. Chan School
of Public Health, said that the study's methodology was sound, but if
anything, it undercounted the value of health benefits. ``If a child has
a disability and needs special care and doesn't have the same life as an
adult as they would have had they been born at normal birth weight,
that's not included,'' Dr. Buonocore said.

The emissions program, often referred to as ``Reggie'' because of its
acronym, has already surpassed its 2020 goal for carbon reductions.
Proponents say it has allowed states to invest in clean energy and
impose limits on big polluters. Critics counter that it could
\href{https://energynews.us/2019/10/25/northeast/how-much-credit-does-rggi-deserve-for-the-northeasts-progress-on-emissions/}{distract
states} from pursuing more aggressive carbon reduction policies.

The study estimated that air quality improvements in the region and
subsequent child health improvements have resulted in \$200 to \$350
million in economic benefits. That's in addition to the \$5.7 billion
estimated by a 2017 study of the carbon dioxide initiative's health
benefits for adults.

``Dollars cannot possibly capture the cost in terms of the psychological
cost and the disruption of the family'' when it comes to children with
serious health problems, Dr. Perera said. But she and her colleagues
decided that it was important to include economic projections in the
study to counter what she called false arguments against pollution rules
and regulations.

She cited the example of catalytic converters in cars, which automakers
used to help meet requirements set by the Clean Air Act. The benefits of
converters and other technology are estimated to have
\href{https://www.epa.gov/clean-air-act-overview/benefits-and-costs-clean-air-act-1990-2020-second-prospective-study}{outweighed
the costs 30 to 1}, according to the Environmental Protection Agency.

``We've gone ahead and done it, simply because we think it does reach
some people when you say, `Not only do we have this health benefit, but
we are saving money,''' Dr. Perera said.

``There are many of these exercises that really blow up that theory of
false dichotomy, between health and expenditure.''

\begin{center}\rule{0.5\linewidth}{\linethickness}\end{center}

\hypertarget{the-ohio-corruption-case-an-update}{%
\subsection{The Ohio corruption case: An
update}\label{the-ohio-corruption-case-an-update}}

\href{https://www.nytimes3xbfgragh.onion/by/lisa-friedman}{\includegraphics{https://static01.graylady3jvrrxbe.onion/images/2018/07/18/multimedia/author-lisa-friedman/author-lisa-friedman-thumbLarge.png}}

By \href{https://www.nytimes3xbfgragh.onion/by/lisa-friedman}{Lisa
Friedman}

Last week, we reported on
\href{https://www.nytimes3xbfgragh.onion/2020/07/22/climate/nyt-climate-newsletter-premature-babies.html}{the
arrest of the Ohio House speaker}, who is accused of talking bribes in
connection with a giant bailout measure to support foundering coal and
nuclear energy plants.

The Ohio governor, Mike DeWine, has called on lawmakers to repeal and
replace the bailout package, which provided about \$150 million in
annual subsidies to energy plants while also gutting Ohio's renewable
energy standards.

Governor DeWine initially supported the measure, but said the arrest of
Larry Householder, the speaker, and others on bribery charges calls for
a rethinking of the state's energy plans. ``While the policy in my
opinion is good, the process by which it was created stinks,'' Governor
DeWine said.

Mr. Householder has rejected calls from leaders of both parties in Ohio
to resign.

\emph{We'd love your feedback on this newsletter. We read every message,
and reply to many! Please email thoughts and suggestions to}
\href{mailto:climateteam@NYTimes.com?subject=Newsletter\%20Feedback}{\emph{climateteam@NYTimes.com}}\emph{.}

\emph{If you like what we're doing, please spread the word and send this
to your friends. You can}
\href{https://www.nytimes3xbfgragh.onion/newsletters/climate-change}{\emph{sign
up here}} \emph{to get our newsletter delivered to your inbox each
week.}

\emph{And be sure to check out}
\href{https://www.nytimes3xbfgragh.onion/newsletters}{\emph{our full
assortment of free newsletters}} \emph{from The Times.}

Advertisement

\protect\hyperlink{after-bottom}{Continue reading the main story}

\hypertarget{site-index}{%
\subsection{Site Index}\label{site-index}}

\hypertarget{site-information-navigation}{%
\subsection{Site Information
Navigation}\label{site-information-navigation}}

\begin{itemize}
\tightlist
\item
  \href{https://help.nytimes3xbfgragh.onion/hc/en-us/articles/115014792127-Copyright-notice}{©~2020~The
  New York Times Company}
\end{itemize}

\begin{itemize}
\tightlist
\item
  \href{https://www.nytco.com/}{NYTCo}
\item
  \href{https://help.nytimes3xbfgragh.onion/hc/en-us/articles/115015385887-Contact-Us}{Contact
  Us}
\item
  \href{https://www.nytco.com/careers/}{Work with us}
\item
  \href{https://nytmediakit.com/}{Advertise}
\item
  \href{http://www.tbrandstudio.com/}{T Brand Studio}
\item
  \href{https://www.nytimes3xbfgragh.onion/privacy/cookie-policy\#how-do-i-manage-trackers}{Your
  Ad Choices}
\item
  \href{https://www.nytimes3xbfgragh.onion/privacy}{Privacy}
\item
  \href{https://help.nytimes3xbfgragh.onion/hc/en-us/articles/115014893428-Terms-of-service}{Terms
  of Service}
\item
  \href{https://help.nytimes3xbfgragh.onion/hc/en-us/articles/115014893968-Terms-of-sale}{Terms
  of Sale}
\item
  \href{https://spiderbites.nytimes3xbfgragh.onion}{Site Map}
\item
  \href{https://help.nytimes3xbfgragh.onion/hc/en-us}{Help}
\item
  \href{https://www.nytimes3xbfgragh.onion/subscription?campaignId=37WXW}{Subscriptions}
\end{itemize}
