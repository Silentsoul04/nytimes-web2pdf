Sections

SEARCH

\protect\hyperlink{site-content}{Skip to
content}\protect\hyperlink{site-index}{Skip to site index}

\href{https://www.nytimes3xbfgragh.onion/section/health}{Health}

\href{https://myaccount.nytimes3xbfgragh.onion/auth/login?response_type=cookie\&client_id=vi}{}

\href{https://www.nytimes3xbfgragh.onion/section/todayspaper}{Today's
Paper}

\href{/section/health}{Health}\textbar{}School Closures in the Spring
Saved Lives, Study Asserts

\url{https://nyti.ms/3fhm0Bi}

\begin{itemize}
\item
\item
\item
\item
\item
\end{itemize}

\href{https://www.nytimes3xbfgragh.onion/news-event/coronavirus?action=click\&pgtype=Article\&state=default\&region=TOP_BANNER\&context=storylines_menu}{The
Coronavirus Outbreak}

\begin{itemize}
\tightlist
\item
  live\href{https://www.nytimes3xbfgragh.onion/2020/08/04/world/coronavirus-covid-19.html?action=click\&pgtype=Article\&state=default\&region=TOP_BANNER\&context=storylines_menu}{Latest
  Updates}
\item
  \href{https://www.nytimes3xbfgragh.onion/interactive/2020/us/coronavirus-us-cases.html?action=click\&pgtype=Article\&state=default\&region=TOP_BANNER\&context=storylines_menu}{Maps
  and Cases}
\item
  \href{https://www.nytimes3xbfgragh.onion/interactive/2020/science/coronavirus-vaccine-tracker.html?action=click\&pgtype=Article\&state=default\&region=TOP_BANNER\&context=storylines_menu}{Vaccine
  Tracker}
\item
  \href{https://www.nytimes3xbfgragh.onion/2020/08/02/us/covid-college-reopening.html?action=click\&pgtype=Article\&state=default\&region=TOP_BANNER\&context=storylines_menu}{College
  Reopening}
\item
  \href{https://www.nytimes3xbfgragh.onion/live/2020/08/03/business/stock-market-today-coronavirus?action=click\&pgtype=Article\&state=default\&region=TOP_BANNER\&context=storylines_menu}{Economy}
\end{itemize}

Advertisement

\protect\hyperlink{after-top}{Continue reading the main story}

Supported by

\protect\hyperlink{after-sponsor}{Continue reading the main story}

\hypertarget{school-closures-in-the-spring-saved-lives-study-asserts}{%
\section{School Closures in the Spring Saved Lives, Study
Asserts}\label{school-closures-in-the-spring-saved-lives-study-asserts}}

But, experts caution, the findings highlight a period when few
precautions were in place, and do not apply to current discussions about
reopening schools.

\includegraphics{https://static01.graylady3jvrrxbe.onion/images/2020/07/29/science/29VIRUS-SCHOOLS/merlin_170293230_f9a4e1c1-c884-4fa3-aeb4-3a7c1633328b-articleLarge.jpg?quality=75\&auto=webp\&disable=upscale}

\href{https://www.nytimes3xbfgragh.onion/by/benedict-carey}{\includegraphics{https://static01.graylady3jvrrxbe.onion/images/2018/02/16/multimedia/author-benedict-carey/author-benedict-carey-thumbLarge.jpg}}\href{https://www.nytimes3xbfgragh.onion/by/pam-belluck}{\includegraphics{https://static01.graylady3jvrrxbe.onion/images/2018/02/16/multimedia/author-pam-belluck/author-pam-belluck-thumbLarge-v2.png}}

By \href{https://www.nytimes3xbfgragh.onion/by/benedict-carey}{Benedict
Carey} and \href{https://www.nytimes3xbfgragh.onion/by/pam-belluck}{Pam
Belluck}

\begin{itemize}
\item
  Published July 29, 2020Updated July 31, 2020
\item
  \begin{itemize}
  \item
  \item
  \item
  \item
  \item
  \end{itemize}
\end{itemize}

In a new analysis, pediatric researchers have estimated that the states'
decisions to close schools last spring likely saved tens of thousands of
lives from Covid-19 and prevented many more coronavirus infections.

The findings come amid a worldwide debate on whether, when and
\href{https://www.nytimes3xbfgragh.onion/interactive/2020/07/31/us/coronavirus-school-reopening-risk.html}{how
to reopen schools}, including for some 56 million American students,
kindergarten through high school.

Outside experts cautioned that the effect of school closings is
extremely difficult to predict because of unknowns like how infectious
children are and because of the difficulty in separating out the effect
of school closures from other measures that states took to control the
virus. In addition, early in the pandemic, testing was especially
limited and spotty, raising questions about how well the number of
confirmed cases reflected actual infections.

The paper,
\href{https://jamanetwork.com/journals/jama/fullarticle/10.1001/jama.2020.14348}{published
Wednesday in JAMA}, is likely to remind school districts and
policymakers how consequential their decisions could be for the shape of
the pandemic this fall. The study focuses on a six-week period in the
spring and concluded that school closure ``may have been associated with
approximately 1.37 million fewer cases of Covid-19 over a 26-day period
and 40,600 fewer deaths over a 16-day period.''

However, they continued, ``these figures do not account for uncertainty
in the model assumptions and the resulting estimates.'' The authors
acknowledged that their findings are not broadly applicable today
because before closings in the spring, there were no precautions, such
as smaller classes and the wearing of masks, that are being recommended
for reopening.

\hypertarget{latest-updates-global-coronavirus-outbreak}{%
\section{\texorpdfstring{\href{https://www.nytimes3xbfgragh.onion/2020/08/04/world/coronavirus-covid-19.html?action=click\&pgtype=Article\&state=default\&region=MAIN_CONTENT_1\&context=storylines_live_updates}{Latest
Updates: Global Coronavirus
Outbreak}}{Latest Updates: Global Coronavirus Outbreak}}\label{latest-updates-global-coronavirus-outbreak}}

Updated 2020-08-04T09:15:14.275Z

\begin{itemize}
\tightlist
\item
  \href{https://www.nytimes3xbfgragh.onion/2020/08/04/world/coronavirus-covid-19.html?action=click\&pgtype=Article\&state=default\&region=MAIN_CONTENT_1\&context=storylines_live_updates\#link-6b644638}{`Long
  days, long nights': Washington prepares for a prolonged fight over
  virus relief.}
\item
  \href{https://www.nytimes3xbfgragh.onion/2020/08/04/world/coronavirus-covid-19.html?action=click\&pgtype=Article\&state=default\&region=MAIN_CONTENT_1\&context=storylines_live_updates\#link-7af9fca0}{Israel's
  rocky reopening of its schools may be a lesson for the U.S.}
\item
  \href{https://www.nytimes3xbfgragh.onion/2020/08/04/world/coronavirus-covid-19.html?action=click\&pgtype=Article\&state=default\&region=MAIN_CONTENT_1\&context=storylines_live_updates\#link-33bf9168}{Hurricane
  Isaias arrives in North Carolina as officials along the East Coast
  scramble.}
\end{itemize}

\href{https://www.nytimes3xbfgragh.onion/2020/08/04/world/coronavirus-covid-19.html?action=click\&pgtype=Article\&state=default\&region=MAIN_CONTENT_1\&context=storylines_live_updates}{See
more updates}

More live coverage:
\href{https://www.nytimes3xbfgragh.onion/live/2020/08/03/business/stock-market-today-coronavirus?action=click\&pgtype=Article\&state=default\&region=MAIN_CONTENT_1\&context=storylines_live_updates}{Markets}

``At the time, there wasn't any masking in schools, there wasn't
physical distancing, there wasn't an increase in hygiene and that sort
of thing,'' said Dr. Katherine Auger, an associate professor of
pediatrics at Cincinnati Children's Hospital and the lead author of the
study. ``The findings of our studies took place before any of those
measures were in.''

Still, Dr. Auger said, she believed that the conclusions of the study
supported the idea that decisions on reopening schools should strongly
consider the prevalence of Covid-19 cases in the community. ``Are there
a lot of cases right now, is it a hot spot, is it an outbreak, is it a
surge?'' Dr. Auger said. She said that the
\href{https://mailcastr.com/links/t/MTExMDY2Ojo6MjRkZTk2ZWM3NjNiNDVlNjg1NGI2ZWNhM2MwMmJlNGU6OnYzOjoxNTk1OTg4MDMyOjox/brMxW}{American
Academy of Pediatrics guidelines}``emphasize making sure that the
community numbers are reasonable before trying to open schools, and I
think this manuscript sort of bolsters that a bit.''

Experts not involved in the study said that its findings could reflect
not only reduced infections at schools but broader disruptions due to
students being at home.

``This is a nice study that provides clear interpretations and
caveats,'' said Lauren Ancel Meyers, a professor of biology and
statistics at the University of Texas, Austin. Dr. Meyers said that the
school closures in March ``not only disrupted daily interactions among
students, but also raised awareness of the emerging threat of Covid-19
and forced parents to stay home to care for their children.''

But other experts noted that the impact of preventive measures taken by
states --- including stay-at-home orders, closures of restaurants and
nonessential businesses and limits on large social gatherings ---
simultaneously or shortly after school closings, made it hard to
determine the specific role of schools in such analyses.

``I think we have to be incredibly cautious when interpreting estimates
from a study like this,'' said Julie Donohue, a professor of public
health at the University of Pittsburgh who
co-wrote\href{https://jamanetwork.com/journals/jama/fullarticle/10.1001/jama.2020.13092}{an
editorial about the study}. ``In particular, I think it's important to
emphasize that we really can't isolate the impact of school closures
from other interventions.''

Dr. Donohue added that ``even if these numbers were accurate or valid,
we don't know how much of the effect would be derived from reducing
contacts among kids at school, versus reducing contacts among parents
who have to stay home from work because their children are out of
school.''

Given that the study is being published just as districts around the
country are struggling with whether and how to reopen schools, some
experts expressed concern that its estimates about the impact of closing
schools early in the pandemic would be seized upon as an argument that
schools should remain closed. Experts on public health and education
have recommended that communities and schools should work toward
reopening with strong health precautions in place, because in-person
schooling has such tremendous value for children's academic, social and
emotional development.

\href{https://www.nytimes3xbfgragh.onion/news-event/coronavirus?action=click\&pgtype=Article\&state=default\&region=MAIN_CONTENT_3\&context=storylines_faq}{}

\hypertarget{the-coronavirus-outbreak-}{%
\subsubsection{The Coronavirus Outbreak
›}\label{the-coronavirus-outbreak-}}

\hypertarget{frequently-asked-questions}{%
\paragraph{Frequently Asked
Questions}\label{frequently-asked-questions}}

Updated August 3, 2020

\begin{itemize}
\item ~
  \hypertarget{im-a-small-business-owner-can-i-get-relief}{%
  \paragraph{I'm a small-business owner. Can I get
  relief?}\label{im-a-small-business-owner-can-i-get-relief}}

  \begin{itemize}
  \tightlist
  \item
    The
    \href{https://www.nytimes3xbfgragh.onion/article/small-business-loans-stimulus-grants-freelancers-coronavirus.html?action=click\&pgtype=Article\&state=default\&region=MAIN_CONTENT_3\&context=storylines_faq}{stimulus
    bills enacted in March} offer help for the millions of American
    small businesses. Those eligible for aid are businesses and
    nonprofit organizations with fewer than 500 workers, including sole
    proprietorships, independent contractors and freelancers. Some
    larger companies in some industries are also eligible. The help
    being offered, which is being managed by the Small Business
    Administration, includes the Paycheck Protection Program and the
    Economic Injury Disaster Loan program. But lots of folks have
    \href{https://www.nytimes3xbfgragh.onion/interactive/2020/05/07/business/small-business-loans-coronavirus.html?action=click\&pgtype=Article\&state=default\&region=MAIN_CONTENT_3\&context=storylines_faq}{not
    yet seen payouts.} Even those who have received help are confused:
    The rules are draconian, and some are stuck sitting on
    \href{https://www.nytimes3xbfgragh.onion/2020/05/02/business/economy/loans-coronavirus-small-business.html?action=click\&pgtype=Article\&state=default\&region=MAIN_CONTENT_3\&context=storylines_faq}{money
    they don't know how to use.} Many small-business owners are getting
    less than they expected or
    \href{https://www.nytimes3xbfgragh.onion/2020/06/10/business/Small-business-loans-ppp.html?action=click\&pgtype=Article\&state=default\&region=MAIN_CONTENT_3\&context=storylines_faq}{not
    hearing anything at all.}
  \end{itemize}
\item ~
  \hypertarget{what-are-my-rights-if-i-am-worried-about-going-back-to-work}{%
  \paragraph{What are my rights if I am worried about going back to
  work?}\label{what-are-my-rights-if-i-am-worried-about-going-back-to-work}}

  \begin{itemize}
  \tightlist
  \item
    Employers have to provide
    \href{https://www.osha.gov/SLTC/covid-19/standards.html}{a safe
    workplace} with policies that protect everyone equally.
    \href{https://www.nytimes3xbfgragh.onion/article/coronavirus-money-unemployment.html?action=click\&pgtype=Article\&state=default\&region=MAIN_CONTENT_3\&context=storylines_faq}{And
    if one of your co-workers tests positive for the coronavirus, the
    C.D.C.} has said that
    \href{https://www.cdc.gov/coronavirus/2019-ncov/community/guidance-business-response.html}{employers
    should tell their employees} -\/- without giving you the sick
    employee's name -\/- that they may have been exposed to the virus.
  \end{itemize}
\item ~
  \hypertarget{should-i-refinance-my-mortgage}{%
  \paragraph{Should I refinance my
  mortgage?}\label{should-i-refinance-my-mortgage}}

  \begin{itemize}
  \tightlist
  \item
    \href{https://www.nytimes3xbfgragh.onion/article/coronavirus-money-unemployment.html?action=click\&pgtype=Article\&state=default\&region=MAIN_CONTENT_3\&context=storylines_faq}{It
    could be a good idea,} because mortgage rates have
    \href{https://www.nytimes3xbfgragh.onion/2020/07/16/business/mortgage-rates-below-3-percent.html?action=click\&pgtype=Article\&state=default\&region=MAIN_CONTENT_3\&context=storylines_faq}{never
    been lower.} Refinancing requests have pushed mortgage applications
    to some of the highest levels since 2008, so be prepared to get in
    line. But defaults are also up, so if you're thinking about buying a
    home, be aware that some lenders have tightened their standards.
  \end{itemize}
\item ~
  \hypertarget{what-is-school-going-to-look-like-in-september}{%
  \paragraph{What is school going to look like in
  September?}\label{what-is-school-going-to-look-like-in-september}}

  \begin{itemize}
  \tightlist
  \item
    It is unlikely that many schools will return to a normal schedule
    this fall, requiring the grind of
    \href{https://www.nytimes3xbfgragh.onion/2020/06/05/us/coronavirus-education-lost-learning.html?action=click\&pgtype=Article\&state=default\&region=MAIN_CONTENT_3\&context=storylines_faq}{online
    learning},
    \href{https://www.nytimes3xbfgragh.onion/2020/05/29/us/coronavirus-child-care-centers.html?action=click\&pgtype=Article\&state=default\&region=MAIN_CONTENT_3\&context=storylines_faq}{makeshift
    child care} and
    \href{https://www.nytimes3xbfgragh.onion/2020/06/03/business/economy/coronavirus-working-women.html?action=click\&pgtype=Article\&state=default\&region=MAIN_CONTENT_3\&context=storylines_faq}{stunted
    workdays} to continue. California's two largest public school
    districts --- Los Angeles and San Diego --- said on July 13, that
    \href{https://www.nytimes3xbfgragh.onion/2020/07/13/us/lausd-san-diego-school-reopening.html?action=click\&pgtype=Article\&state=default\&region=MAIN_CONTENT_3\&context=storylines_faq}{instruction
    will be remote-only in the fall}, citing concerns that surging
    coronavirus infections in their areas pose too dire a risk for
    students and teachers. Together, the two districts enroll some
    825,000 students. They are the largest in the country so far to
    abandon plans for even a partial physical return to classrooms when
    they reopen in August. For other districts, the solution won't be an
    all-or-nothing approach.
    \href{https://bioethics.jhu.edu/research-and-outreach/projects/eschool-initiative/school-policy-tracker/}{Many
    systems}, including the nation's largest, New York City, are
    devising
    \href{https://www.nytimes3xbfgragh.onion/2020/06/26/us/coronavirus-schools-reopen-fall.html?action=click\&pgtype=Article\&state=default\&region=MAIN_CONTENT_3\&context=storylines_faq}{hybrid
    plans} that involve spending some days in classrooms and other days
    online. There's no national policy on this yet, so check with your
    municipal school system regularly to see what is happening in your
    community.
  \end{itemize}
\item ~
  \hypertarget{is-the-coronavirus-airborne}{%
  \paragraph{Is the coronavirus
  airborne?}\label{is-the-coronavirus-airborne}}

  \begin{itemize}
  \tightlist
  \item
    The coronavirus
    \href{https://www.nytimes3xbfgragh.onion/2020/07/04/health/239-experts-with-one-big-claim-the-coronavirus-is-airborne.html?action=click\&pgtype=Article\&state=default\&region=MAIN_CONTENT_3\&context=storylines_faq}{can
    stay aloft for hours in tiny droplets in stagnant air}, infecting
    people as they inhale, mounting scientific evidence suggests. This
    risk is highest in crowded indoor spaces with poor ventilation, and
    may help explain super-spreading events reported in meatpacking
    plants, churches and restaurants.
    \href{https://www.nytimes3xbfgragh.onion/2020/07/06/health/coronavirus-airborne-aerosols.html?action=click\&pgtype=Article\&state=default\&region=MAIN_CONTENT_3\&context=storylines_faq}{It's
    unclear how often the virus is spread} via these tiny droplets, or
    aerosols, compared with larger droplets that are expelled when a
    sick person coughs or sneezes, or transmitted through contact with
    contaminated surfaces, said Linsey Marr, an aerosol expert at
    Virginia Tech. Aerosols are released even when a person without
    symptoms exhales, talks or sings, according to Dr. Marr and more
    than 200 other experts, who
    \href{https://academic.oup.com/cid/article/doi/10.1093/cid/ciaa939/5867798}{have
    outlined the evidence in an open letter to the World Health
    Organization}.
  \end{itemize}
\end{itemize}

``I do worry that these large estimates of the effect of school closures
will lead people to give up because it is going to be challenging to
open schools,'' Dr. Donohue said. ``I do worry that some districts will
look at these numbers and say, well, it's just too hard and it's not
safe to reopen.''

Analyses of
\href{https://www.nytimes3xbfgragh.onion/interactive/2020/04/02/us/coronavirus-social-distancing.html}{cellphone
mobility data}in March, by The New York Times and others, revealed that
people in some states began spending more time at home and less time
congregating in public spaces days to weeks before official
shelter-in-place orders were enacted.

In the new report, a research team led by Dr. Auger analyzed case counts
and deaths in all 50 states beginning on March 7, before the first
closings, and ending on May 7, six weeks after the last closing orders.
Some states closed their schools a week or more before implementing a
more comprehensive lockdown; others did so simultaneously, or close to
it. Covid-19 prevalence rates also varied widely, with states like New
York and California at the high end and others, like Nebraska and Texas,
on the lower end.

The researchers used these differences to help isolate the probable
effects of closings from those that were the result of other measures,
like closing businesses and stay-at-home policies. The study also
accounted for state differences in population vulnerability to the
virus, according to an index from the Centers for Disease Control and
Prevention that accounts for household income, demographics, underlying
levels of chronic diseases and other factors. The earlier a state closed
its schools, the fewer cases and deaths accumulated in the weeks that
followed, the study found.

A co-author, Dr. Samir Shah, a professor of pediatrics and director of
the division of hospital medicine at Cincinnati Children's Hospital,
emphasized the complicated balancing act in deciding whether to reopen
schools. ``It's not simply a decision of, what is the risk of Covid,''
he said. ``It also has to factor in what are the benefits versus harms
of not having children in school in person.''

Advertisement

\protect\hyperlink{after-bottom}{Continue reading the main story}

\hypertarget{site-index}{%
\subsection{Site Index}\label{site-index}}

\hypertarget{site-information-navigation}{%
\subsection{Site Information
Navigation}\label{site-information-navigation}}

\begin{itemize}
\tightlist
\item
  \href{https://help.nytimes3xbfgragh.onion/hc/en-us/articles/115014792127-Copyright-notice}{©~2020~The
  New York Times Company}
\end{itemize}

\begin{itemize}
\tightlist
\item
  \href{https://www.nytco.com/}{NYTCo}
\item
  \href{https://help.nytimes3xbfgragh.onion/hc/en-us/articles/115015385887-Contact-Us}{Contact
  Us}
\item
  \href{https://www.nytco.com/careers/}{Work with us}
\item
  \href{https://nytmediakit.com/}{Advertise}
\item
  \href{http://www.tbrandstudio.com/}{T Brand Studio}
\item
  \href{https://www.nytimes3xbfgragh.onion/privacy/cookie-policy\#how-do-i-manage-trackers}{Your
  Ad Choices}
\item
  \href{https://www.nytimes3xbfgragh.onion/privacy}{Privacy}
\item
  \href{https://help.nytimes3xbfgragh.onion/hc/en-us/articles/115014893428-Terms-of-service}{Terms
  of Service}
\item
  \href{https://help.nytimes3xbfgragh.onion/hc/en-us/articles/115014893968-Terms-of-sale}{Terms
  of Sale}
\item
  \href{https://spiderbites.nytimes3xbfgragh.onion}{Site Map}
\item
  \href{https://help.nytimes3xbfgragh.onion/hc/en-us}{Help}
\item
  \href{https://www.nytimes3xbfgragh.onion/subscription?campaignId=37WXW}{Subscriptions}
\end{itemize}
