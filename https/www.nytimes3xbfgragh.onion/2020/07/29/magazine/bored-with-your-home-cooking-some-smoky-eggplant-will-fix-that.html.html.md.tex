Sections

SEARCH

\protect\hyperlink{site-content}{Skip to
content}\protect\hyperlink{site-index}{Skip to site index}

\href{https://myaccount.nytimes3xbfgragh.onion/auth/login?response_type=cookie\&client_id=vi}{}

\href{https://www.nytimes3xbfgragh.onion/section/todayspaper}{Today's
Paper}

Bored With Your Home Cooking? Some Smoky Eggplant Will Fix That

\url{https://nyti.ms/3gbNcCr}

\begin{itemize}
\item
\item
\item
\item
\item
\end{itemize}

\href{https://www.nytimes3xbfgragh.onion/spotlight/at-home?action=click\&pgtype=Article\&state=default\&region=TOP_BANNER\&context=at_home_menu}{At
Home}

\begin{itemize}
\tightlist
\item
  \href{https://www.nytimes3xbfgragh.onion/2020/07/28/books/time-for-a-literary-road-trip.html?action=click\&pgtype=Article\&state=default\&region=TOP_BANNER\&context=at_home_menu}{Take:
  A Literary Road Trip}
\item
  \href{https://www.nytimes3xbfgragh.onion/2020/07/29/magazine/bored-with-your-home-cooking-some-smoky-eggplant-will-fix-that.html?action=click\&pgtype=Article\&state=default\&region=TOP_BANNER\&context=at_home_menu}{Cook:
  Smoky Eggplant}
\item
  \href{https://www.nytimes3xbfgragh.onion/2020/07/27/travel/moose-michigan-isle-royale.html?action=click\&pgtype=Article\&state=default\&region=TOP_BANNER\&context=at_home_menu}{Look
  Out: For Moose}
\item
  \href{https://www.nytimes3xbfgragh.onion/interactive/2020/at-home/even-more-reporters-editors-diaries-lists-recommendations.html?action=click\&pgtype=Article\&state=default\&region=TOP_BANNER\&context=at_home_menu}{Explore:
  Reporters' Obsessions}
\end{itemize}

Advertisement

\protect\hyperlink{after-top}{Continue reading the main story}

Supported by

\protect\hyperlink{after-sponsor}{Continue reading the main story}

\href{/column/magazine-eat}{Eat}

\hypertarget{bored-with-your-home-cooking-some-smoky-eggplant-will-fix-that}{%
\section{Bored With Your Home Cooking? Some Smoky Eggplant Will Fix
That}\label{bored-with-your-home-cooking-some-smoky-eggplant-will-fix-that}}

\includegraphics{https://static01.graylady3jvrrxbe.onion/images/2020/08/02/magazine/02mag-eat/02mag-eat-articleLarge.jpg?quality=75\&auto=webp\&disable=upscale}

By Gabrielle Hamilton

\begin{itemize}
\item
  July 29, 2020
\item
  \begin{itemize}
  \item
  \item
  \item
  \item
  \item
  \end{itemize}
\end{itemize}

Lately I've been worn out by my own home cooking. It's not the toil
that's getting to me; it's more that I feel fatigued by the familiarity
of it: the predictability of my own kitchen habits, the recognizability
of my own cooking voice, the monotony of what I always throw in my cart
as I zigzag the aisles at the grocery store on autopilot. Everyone I
know has had a similar malaise recently --- all of us feeling rather
short on enthusiasm for this monogamous intimacy with our
three-ingredient tendencies and our everything-shoved-on-a-sheet-pan
ways, no matter how beloved and delicious the results have reliably
been. And so I've begun reaching for new ideas, new paths, new routes to
new pleasures.

I now shop in my grocery store in opposite zigzags than I did the week
before, starting at the exit aisles and making my way back to the
entrance ones, walking corridors I normally avoid to see if there's
something on those shelves that speaks to me in a new way. I've trekked
many extra crosstown blocks to visit grocery stores in neighborhoods
farther afield to see what they stock that I might be attracted to. I've
also been re-employing restaurant techniques that aren't entirely
convenient in the home kitchen, but somehow it hasn't been wearying;
it's been animating.

For the decades that I've cooked eggplant, I've cooked it pretty much
only this one way, and it's undeniably fun and still delivers: You take
the whole fat globe eggplant, and you set it directly on the burner
grate on the stovetop. Set the gas flame to high and scorch it. The skin
forms a carbonized black bark --- the kind that would usually signal
ruin --- while the flesh inside steams and softens until it collapses to
silken and rather smoky perfection. While the eggplant cools after its
scorching, a viscous liquid as dark as brewed coffee collects in the
bowl, which steeps the interior fruit in its smokiness. Once strained,
this liquid can be kept in your fridge for other uses --- it's a
brilliant byproduct. The cooked eggplant is removed from its charred
casing with a spoon, just as you might scoop ripe avocado out of its
leathery shell.

Usually I finish that bowlful of smoky eggplant pulp by adding a long
pour of fruity olive oil and seasoning it one of three ways: salt,
microplaned raw garlic and lemon juice; \emph{tom yum} paste and
scallions and sesame seeds stirred in; or a heavy shower of freshly
chopped parsley, some crumbled feta cheese and a little red onion.

That smoky pulp responds in the same way that a satin-finish white wall
does --- uncannily well --- to a boldly painted trim. It has been a
staple of my cooking for decades, to spread on warm toast, to dip into
with bitter Belgian endive spears, to spoon next to a grilled lamb chop.

But my reliable three finishing sets for smoked eggplant have become, to
me, almost claustrophobically reliable. So I have begun doing something
new: forming, freezing, breading, refreezing and finally frying the
mixture into croquettes. These knockouts reflect that elusive next-level
quality of cooking that comes from the extra steps and inconveniences
restaurant chefs embrace, which I think we have all so sorely been
missing these past many months.

You could stop after Phase 1 and enjoy well-seasoned smoky eggplant the
way I have for decades. It's truly delicious, and for you probably not
deadly familiar. You could stop as well with these crispy,
silken-centered croquettes, which will turn heads at cocktail hour.

But if you want to keep going, there are even further steps to take.
After you have your golden fried croquettes, you could also top them
with a little tomato sauce and slivers of mozzarella and a grating of
Parmesan cheese, run them under the broiler and enjoy a family meal of
eggplant parmigiana that dissolves on the tongue and stuns your children
into loving awe.

However you choose to do it, I hope the product will bring you the same
spark in your home kitchen as it did me in mine --- delicious, exciting,
complex and entirely new bites from the same old girl who just yesterday
was starting to bore herself to tears.

Recipe:
\href{https://cooking.nytimes3xbfgragh.onion/recipes/1021285-smoky-eggplant-croquettes?action=click\&module=Global\%20Search\%20Recipe\%20Card\&pgType=search\&rank=2}{Smoky
Eggplant Croquettes}

Advertisement

\protect\hyperlink{after-bottom}{Continue reading the main story}

\hypertarget{site-index}{%
\subsection{Site Index}\label{site-index}}

\hypertarget{site-information-navigation}{%
\subsection{Site Information
Navigation}\label{site-information-navigation}}

\begin{itemize}
\tightlist
\item
  \href{https://help.nytimes3xbfgragh.onion/hc/en-us/articles/115014792127-Copyright-notice}{©~2020~The
  New York Times Company}
\end{itemize}

\begin{itemize}
\tightlist
\item
  \href{https://www.nytco.com/}{NYTCo}
\item
  \href{https://help.nytimes3xbfgragh.onion/hc/en-us/articles/115015385887-Contact-Us}{Contact
  Us}
\item
  \href{https://www.nytco.com/careers/}{Work with us}
\item
  \href{https://nytmediakit.com/}{Advertise}
\item
  \href{http://www.tbrandstudio.com/}{T Brand Studio}
\item
  \href{https://www.nytimes3xbfgragh.onion/privacy/cookie-policy\#how-do-i-manage-trackers}{Your
  Ad Choices}
\item
  \href{https://www.nytimes3xbfgragh.onion/privacy}{Privacy}
\item
  \href{https://help.nytimes3xbfgragh.onion/hc/en-us/articles/115014893428-Terms-of-service}{Terms
  of Service}
\item
  \href{https://help.nytimes3xbfgragh.onion/hc/en-us/articles/115014893968-Terms-of-sale}{Terms
  of Sale}
\item
  \href{https://spiderbites.nytimes3xbfgragh.onion}{Site Map}
\item
  \href{https://help.nytimes3xbfgragh.onion/hc/en-us}{Help}
\item
  \href{https://www.nytimes3xbfgragh.onion/subscription?campaignId=37WXW}{Subscriptions}
\end{itemize}
