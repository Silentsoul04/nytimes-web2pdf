Sections

SEARCH

\protect\hyperlink{site-content}{Skip to
content}\protect\hyperlink{site-index}{Skip to site index}

\href{https://myaccount.nytimes3xbfgragh.onion/auth/login?response_type=cookie\&client_id=vi}{}

\href{https://www.nytimes3xbfgragh.onion/section/todayspaper}{Today's
Paper}

Inside the Iraqi Kleptocracy

\url{https://nyti.ms/2PbfgtL}

\begin{itemize}
\item
\item
\item
\item
\item
\item
\end{itemize}

Advertisement

\protect\hyperlink{after-top}{Continue reading the main story}

Supported by

\protect\hyperlink{after-sponsor}{Continue reading the main story}

Feature

\hypertarget{inside-the-iraqi-kleptocracy}{%
\section{Inside the Iraqi
Kleptocracy}\label{inside-the-iraqi-kleptocracy}}

\includegraphics{https://static01.graylady3jvrrxbe.onion/images/2020/08/02/magazine/02mag-corruption/02mag-corruption-articleLarge.jpg?quality=75\&auto=webp\&disable=upscale}

By \href{https://www.nytimes3xbfgragh.onion/by/robert-f-worth}{Robert F.
Worth}

\begin{itemize}
\item
  July 29, 2020
\item
  \begin{itemize}
  \item
  \item
  \item
  \item
  \item
  \item
  \end{itemize}
\end{itemize}

\hypertarget{listen-to-this-article}{%
\subsubsection{Listen to This Article}\label{listen-to-this-article}}

Audio Recording by Audm

\emph{To hear more audio stories from publishers like The New York
Times, download}
\emph{\href{https://www.audm.com/?utm_source=nyt\&utm_medium=embed\&utm_campaign=wages_of_corruption}{Audm
for iPhone or Android}.}

Early last October, while working in his office in Baghdad, a
businessman named Hussein Laqees got a phone call from a number he'd
never seen before. ``We need to talk,'' the caller said. The man's voice
was gruff and self-assured, a little menacing. He demanded that Laqees
come meet him but refused to give his name.

Laqees demurred, and the call ended. He might have forgotten the whole
exchange had a colleague not been in touch a few minutes later with
worrisome news. The mystery caller, he said, was from Kataib Hezbollah,
a powerful Iraqi militia with strong ties to Iran's Revolutionary
Guards. They had a business proposal to discuss.

When the militiaman called again, Laqees reluctantly agreed to a
meeting. He gathered a few colleagues, and they all drove to a house off
Sadoun Street in downtown Baghdad, arriving near dusk. Inside, he was
led into a dim office and introduced to a small, bald man who got right
to the point. ``You need to work with us, there is no other choice,''
the bald man said. ``You can keep your staff, but you must do as we
say.'' He explained that Kataib Hezbollah would take 20 percent of
Laqees's gross revenue --- about 50 percent of his profits.

Laqees refused. His company, Palm Jet, had a five-year government
contract to run a V.I.P. terminal at Baghdad's international airport,
along with a nearby hotel; it also works routinely with Western
aeronautics firms like Lockheed Martin. He could not have any dealings
with a group like Kataib Hezbollah, which is listed by the U.S.
government as a foreign terrorist organization (as is the unrelated
Lebanese group also called Hezbollah). The bald man replied that if
Laqees refused, he would seize everything he owned in Baghdad. Laqees
looked at him in disbelief. ``I'm an investor,'' he said. ``There is
law.'' The bald man shot back: ``We are the law.'' He told Laqees to
give him an answer by noon the next day.

The following afternoon, five Chevrolet S.U.V.s rolled up outside the
V.I.P. terminal. Twelve men got out, dressed in black paramilitary gear
and carrying guns. They found Laqees in the cafe of the airport hotel,
smoking and sipping coffee. He had been calling all his government
contacts since the night before, along with the airport's department
heads. No one had called back. It was as if they'd been warned --- or
perhaps paid off. The militiamen took Laqees's phone and told him to
sign a document relinquishing his contract. He stalled for time. One of
his employees slipped outside to take a cellphone picture of the
militiamen's vehicles, but they caught him, smashed his phone and beat
him up. Laqees, who is Lebanese, had been working in Iraq since 2011. He
knew the country was troubled by crime and corruption, but he believed
that the airport, with its hundreds of uniformed immigration and
security officials, was different. ``I wait 20 minutes, maybe someone
will come,'' Laqees told me later. ``Police, something.'' Finally, he
walked to the departures hall and caught a flight to Dubai. Days later,
Kataib Hezbollah installed its preferred contractor in his place. Laqees
has not returned to Iraq since.

\emph{\href{https://www.nytimes3xbfgragh.onion/2018/03/14/magazine/how-a-ransom-for-royal-falconers-reshaped-the-middle-east.html}{{[}Kidnapped
Royalty Become Pawns in Iran's Deadly Plot{]}}}

The airport raid took place just four days after the start of
\href{https://www.nytimes3xbfgragh.onion/2019/10/02/world/middleeast/iraq-corruption-protests.html}{Iraq's
anti-government protests}, as thousands of young demonstrators were
flooding the streets of Baghdad and other cities, chanting their
poignant trademark slogan: ``\emph{Nureed watan},'' or ``We want a
country.'' The protesters quickly took over Tahrir Square in the heart
of Baghdad, setting up tents and fighting pitched battles with the
police. Although the chaos brought much of the city's business and
government to a standstill, it also won the sympathy of Arabs across the
region, igniting an equally powerful protest movement in Lebanon. To
those who took part in the rallies, groups like Kataib Hezbollah are not
just Iranian proxies; they are the newest faces of a kleptocracy that
has enriched itself at the expense of Iraq's youth, who have been left
jobless and destitute in ever-increasing numbers. Some militia leaders,
meanwhile, have joined the ranks of Iraq's richest men, becoming famous
for buying upscale restaurants, nightclubs and opulent farms on the
Tigris.

The militias have been aided and abetted by a new Iraqi political class
whose sole ethic is self-enrichment. Over the years, this
cross-sectarian cabal has mastered scams at every level: routine
checkpoint shakedowns, bank fraud, embezzling from the government
payroll. Adel Abdul Mahdi, who was hailed as a potential reformer when
he became Iraq's prime minister in 2018, hoped to subordinate the
militias to the state. Instead, they outmaneuvered and overpowered him.
His cabinet included people with ties to some of the worst graft schemes
afflicting the country.

The United States is deeply implicated in all this, and not just because
its serial invasions wrecked the country and helped ravage the economy.
America provides the money that sustains it, even as U.S. officials wink
at the self-dealing of Iraqi allies. The Federal Reserve of New York
still supplies Iraq with at least \$10 billion a year in hard currency
from the country's oil sales. Much of that is passed on to commercial
banks, ostensibly for imports, in a process that was hijacked long ago
by Iraq's money-laundering cartels. At the same time, the United States
inflicts punishing sanctions on two countries --- Iran and Syria ---
with which Iraq shares notoriously permeable borders. It is the ideal
breeding ground for corruption.

The Trump administration may have shocked Iraq's militias with the
\href{https://www.nytimes3xbfgragh.onion/2020/01/11/us/politics/iran-trump.html}{unexpected
assassination in January of Qassim Suleimani,} the powerful Iranian spy
chief, at the Baghdad airport. But Iranian proxies like Kataib Hezbollah
do not seem overly worried. They know President Trump has little stomach
for a war, especially in the Covid-19 era of soaring deficits. Their
greatest priority is maintaining an Iraqi system in which literally
everything is for sale.

The coronavirus pandemic has now pushed Iraq to the brink of an
existential crisis. The global collapse of demand for oil has brought
prices to historic lows, delivering a terrible shock to a country whose
economy depends almost entirely on oil revenue. But it could also offer
the new Iraqi prime minister, Mustafa al-Kadhimi, an extraordinary
opportunity to face his country's most intractable problem. Corruption
can now be framed as a life-or-death issue: Iraq must choose between
feeding its people and enriching its kleptocrats. Kadhimi has promised
to take up this challenge. He is not likely to succeed unless the United
States seizes this chance to undo some of the damage it has done in
Iraq, and to make common cause with the protesters who are hoping to
re-establish their country on a new footing.

\includegraphics{https://static01.graylady3jvrrxbe.onion/images/2020/08/02/magazine/02mag-corruption-02/02mag-corruption-02-articleLarge.jpg?quality=75\&auto=webp\&disable=upscale}

\textbf{In the annals of American diplomacy,} corruption has long had an
equivocal status: deplored in public but in practice often regarded as a
tolerable, even useful vice. The United States has a long history of
supporting kleptocrats who were on the ``right side'' of one
geopolitical rivalry or another. The price of these bargains, often paid
in blood, has fueled a reappraisal. ``Corruption is not just a
fundamental political problem but the most significant driver of most of
the security problems we are supposed to be trying to address,'' Sarah
Chayes told me in May. Chayes's 2015 book, ``Thieves of State,''
documents the destructive effects of corruption across a range of
countries in Africa and Asia. The book grew out of her experiences in
Afghanistan, where she lived for years before becoming a Pentagon
adviser and saw how the rampant extortion and graft of the U.S.-backed
government helped push the local population into the arms of the
Taliban.

Iraq may be an even more vivid object lesson. As recently as the 1980s,
corruption was rare, and ministries in Saddam Hussein's autocratic
government were mostly clean and well run. The change came during the
1990s, when the United Nations imposed crippling sanctions following
Hussein's invasion of Kuwait. Over a period of just seven years, Iraq's
per-capita income dropped to \$450 from about \$3,500. As the value of
their salaries collapsed, government officials couldn't survive without
taking bribes, which became the currency of everyday life. The rot grew
worse after the invasion of 2003, when American officers began handing
out bricks of \$100 bills in an attempt to make friends and jump-start
the economy. They may have meant well, but their clumsy haste was
catastrophic. A new group of opportunists, including returning Iraqi
exiles, lined up for big government contracts. Billions went missing.
The theft expanded in scale after the oil boom of 2008, thanks to a
network of oligarchs empowered by Prime Minister Nuri Kamal al-Maliki.

When ISIS surged into northwestern Iraq in mid-2014, the Iraqi forces
that rose to defend it were officially 350,000 strong, much larger than
the attacking jihadi brigades. In reality, the army had been eviscerated
by ``ghost soldier'' kickback schemes, with commanders pocketing
hundreds, even thousands, of salaries. These practices destroyed morale
inside the army and fed popular anger among civilians in Mosul, who
became more receptive to ISIS than they otherwise might have been. A
recent survey of people in the Mosul region, led by the Harvard
Humanitarian Initiative,
\href{https://hhi.harvard.edu/publications/english-version-never-forget-views-peace-and-justice-within-conflict-affected}{found
that they saw corruption as a chief cause of the emergence of ISIS.}

Weighing the full cost of what has been stolen from Iraq is not easy.
Deals are done in cash, documents are hard to come by and the
government's statistics are often unreliable. Still, the available
information suggests that Iraq may have had more of its national wealth
illicitly drained abroad than any other nation. One Iraqi elder
statesman with long experience in finance recently assembled a
confidential assessment for the Atlantic Council, an American think
tank, based on conversations with bankers, investigators and contacts in
a variety of foreign countries. He concluded that \$125 billion to \$150
billion is held by Iraqis overseas, most of it ``illegitimately
acquired.'' He noted that other estimates run as high as \$300 billion.
Some \$10 billion in stolen money, he estimated, is invested in London
real estate alone. A full reckoning would extend well beyond the
financial bill to the damage inflicted on Iraq's culture and society ---
a point I often heard older Iraqis make with great sadness during the
years I lived there.

Image

Baghdad's Tahrir SquareCredit...Moises Saman/Magnum, for The New York
Times

Iraqi political life may look like gang warfare to outsiders, but on
most days its turbulent surface conceals a calm and cheerful business of
looting. At every government ministry, the biggest spoils are allocated
by unwritten agreement to one faction or another. The Sadrists have the
Health Ministry, the Badr Organization has long had the Interior
Ministry and the Oil Ministry belongs to Al-Hikma. Newcomers sometimes
have trouble adjusting to this state of affairs. One former cabinet
minister --- a technocrat who spent decades abroad --- discovered, on
arriving in his post, that his ministry was procuring vaccines with a
\$92 million contract. He found another way to buy the same vaccines for
less than \$15 million. ``Once I did this, I faced a great deal of
resistance, a fierce campaign against me,'' he told me. His priority was
addressing the gap between Iraq's oil wealth and its devastated health
system, which lacks access to many basic medicines. To his opponents,
the only imperative was their own and their party's interests. The
minister eventually decided that these two philosophies were
irreconcilable, and he resigned. (Like most of the people I spoke to for
this article, he spoke on condition that I not use his name. Corruption
is the third rail of Iraqi politics: Touching it can easily get you or
your relatives killed.)

The political bosses who preside over this graft are well known; some
are staunch American allies. The Barzani and Talabani families of
Kurdistan have
\href{https://newrepublic.com/article/158609/iraq-barzani-pentagon-oil-beverly-hills-mansion}{used
their control over that region's contracts and its central bank} to
become immensely rich. Maliki and his ring of powerful cronies still
loom over the Iraqi political scene. Moktada al-Sadr, the mercurial
Shiite cleric, is another godfather figure whose followers are notorious
for demanding hefty kickbacks. This system should have received a jolt
in 2014, when its depredations led directly to the country's
near-takeover by ISIS. Instead, the main consequence was the rise of a
new breed of parasite: the militias who helped defeat ISIS, known
collectively as the Hashd al-Shaabi, or the Popular Mobilization Forces.
The Hashd is a loose confederation of armed groups, some of which have
been around for decades. In 2016, Prime Minister Haider al-Abadi
recognized them as part of the country's security sector, and they now
receive regular salaries just as soldiers and police officers do.

Among the most powerful is Kataib Hezbollah. It was accused of mounting
an attack on an Iraqi air base in December that killed an American
contractor and led to the assassination of Suleimani --- its ultimate
patron --- a week later. Despite its high profile, it is surrounded by
mystery. ``We know almost nothing about the leadership,'' says Michael
Knights, an analyst at the Washington Institute for Near East Policy who
has tracked the group since its founding. ``It's like the Masons. You
can be in it and be in another movement at the same time.'' It has built
an economic empire, partly by forcing its way into legitimate businesses
and government contracts.

Among the militia's least-known and most troubling ventures has been its
gradual assertion of control over the Baghdad airport. It started
several years ago, when Kataib Hezbollah and another Iranian-backed
militia called Asaib Ahl al-Haq began stealthily placing workers loyal
to them throughout the airport, according to a senior airport official I
spoke with. They were also able to get G4S, a British company that has a
long-term contract for security at the airport, to hire their people, he
said. (G4S did not respond to requests for comment.) As a result, the
two militias now have access to all the airport's CCTV cameras and to a
limited-access road called Kilometer One that connects the runways to
the airport perimeter, bypassing the security barriers, the official
told me. (When Qassim Suleimani and his entourage were struck by an
American drone in January, they had just come off this road.) The
militias' efforts became more aggressive about a year ago, the official
told me, when its members held the Baghdad airport's director of civil
aviation at gunpoint and forced him to hire a man loyal to them as his
nominal deputy. In late October, a Kataib Hezbollah front company
received a 12-year contract at the Baghdad and Basra airports, worth
tens of millions of dollars a year, even though the firm --- the blandly
named Gulf corporation --- was only two months old and did not have the
necessary accreditation or licensing and its founder had been barred
from the airport. The contract has since been terminated, but the
company that took over the V.I.P. terminal and hotel from Hussein Laqees
remains in place.

The Baghdad airport is just one of the economic gateways that the
militias now control. They have used the ISIS threat to install
themselves at most of the country's land borders. And the militias have
dominated much of the trade through Iraq's southern seaports for more
than a decade. In effect, the militias operate a shadow state, charging
importers higher fees in exchange for expedited processing and delivery.
They have economic committees with offices in Baghdad, where private
companies can make deals that brazenly circumvent the country's legal
channels. ``For example, if I'm bringing 100 cars in from Dubai, if I do
the legal process it might take two months to clear,'' the airport
official told me. ``If I pay Kataib Hezbollah, say, \$10,000 to
\$15,000, it might take only two days.''

\textbf{The cash that has fueled Iraq's descent} into kleptocracy
originates, for the most part, from a heavily guarded Federal Reserve
compound in East Rutherford, N.J. There, every month or so, a truck is
loaded with more than 10 tons of plastic-wrapped U.S. currency, a haul
worth \$1 billion to \$2 billion. The money is then driven to an Air
Force base and flown to Baghdad. It belongs to the Iraqi government,
which routes the proceeds of its oil sales through an account at the New
York Federal Reserve. This unusual arrangement is a legacy of the U.S.
occupation, when America directly controlled the Iraqi government and
its finances. It has remained in place because it suits both sides: The
Iraqis get quick, preferential access to dollars, and the United States
retains tremendous leverage over Iraq's economy. Ostensibly, the
periodic dollar shipments (a small part of the country's overall oil
revenue) are to meet the needs of Iraqi exchange houses and importers,
who require hard cash. In practice, many of the dollars have found their
way into the hands of money launderers, terrorist groups and Iran's
Revolutionary Guards, thanks to a little-known ritual run by the central
bank of Iraq: the ``dollar auction.''

The dollar auction has been called the ``sewage system of Iraqi
corruption,'' but its inner workings have rarely been written about. The
fraud schemes that revolve around it have fueled every side in the
Syrian civil war, including ISIS. The U.S. Treasury Department has made
serious efforts to keep auction dollars out of the hands of ISIS and
Iran, but it has often turned a blind eye to other kinds of money
laundering. And terrorists have repeatedly found new companies and
methods to disguise their participation in the auction, often with the
complicity of central-bank officials.

The auction's name is misleading; it is a daily process in which Iraq's
central bank provides dollars to a limited number of the country's
commercial banks in exchange for Iraqi dinars. The American occupation
authorities established it in 2003 to serve two purposes: collecting
enough dinars to pay salaries in cash to Iraq's vast armada of
government employees and helping the country pay for badly needed
imports in dollars. In principle, the auction is similar to the process
used by some other countries to facilitate foreign trade. It was meant
to work like this: A company intending to import shoes from India, for
instance, would go its local Iraqi bank with an invoice from the Indian
shoe company. The local bank would authenticate the transaction and
deposit the required amount in Iraqi dinars with the central bank, which
would wire dollars to a correspondent account belonging, ostensibly, to
the exporter.

The trouble started with a swelling tide of dirty money: Iraqis who had
stolen large sums through fraudulent contracts or kickback schemes were
hungry to trade their dinars for dollars, so that they could use them
abroad. To meet the need, a new class of opportunists began registering
fake companies and fabricating the invoices required to simulate an
import deal, which would then be funded via the dollar auction. In a
matter of days, someone who had defrauded his country of millions could
become the owner of a London townhouse. The phony imports left little
trace, because they were documented with ID cards and photographs of
real people, who would agree to play company officials in exchange for a
bribe.

Each time the authorities at the Iraqi central bank or the New York Fed
got suspicious, the fraudsters would up their game correspondingly.
``There were small offices of young people to produce
professional-looking forgeries,'' a former Iraqi banker, one of several
financiers and former government officials who described the scheme,
told me. ``Then they cook the whole file around it.'' To avoid paying
taxes on the phony imports, the launderers would register dozens of
companies, abandoning them and creating new ones whenever their taxes
were due. They got the border authorities involved, paying officials to
provide fake manifests with realistic-looking stamps. The launderers
eventually commandeered much of the central bank's daily dollar sales,
which have totaled, according to the central bank's own figures, well
over \$500 billion since 2003. (That figure is much higher than the
number of physical dollars flown to Iraq from the Fed, because most of
the dollars sold by the central bank are electronic transfers from
Iraq's oil revenues.)

The fraud was sometimes laughably obvious. In 2017, Iraq officially
imported \$1.66 billion worth of tomatoes from Iran --- more than a
thousand times the amount it imported in 2016. It also listed imports of
\$2.86 billion in watermelons from Iran, up from \$16 million the year
before. These amounts would be ludicrous even if Iraq didn't grow large
amounts of its own tomatoes and watermelons. Economists told me these
official import numbers --- still visible on the Iraqi planning
ministry's website --- appear to be a poorly disguised cover for money
laundering via the dollar auction.

The auction has also enabled a large-scale embezzlement scheme that has
funneled billions of dollars to Iraq's power brokers. This fraud was
based on the difference between the fixed exchange rate offered by the
central bank --- which is pegged to the dollar --- and the fluctuating
market rate, which is often much higher. Soon after the auction started
in 2003, the money launderers realized that if they could fake an import
deal, they could then resell the dollars they'd acquired from the
central bank, realizing an instant profit on the rate spread. As soon as
Iraq's political bosses realized how much money was to be made, they
seized control of access to the auction. Ordinary companies and banks
wanting to do legitimate imports or lending were squeezed out by those
with backing from the main political parties and militias. To disguise
this takeover, the newly minted plutocrats bought up almost all the
remaining commercial banks, turning them into mere vehicles for the
auction scheme.

It is impossible to say exactly how many billions have been stolen
through exchange-rate arbitrage, but several former bankers and Iraqi
officials told me that this kind of fraud accounts for most of the
ostensible imports financed by the dollar auction since around 2008. My
own estimate, based on figures from the central bank's website and
information from Iraqi bankers and finance officials, is about \$20
billion, all of it stolen from the Iraqi people. The businessmen running
the scheme are virtually printing their own money, because their costs
--- paying for fake invoices and bribing bank and government officials
--- are low. Some of the banks posting enormous profits from the auction
are little more than fronts, with dilapidated branch offices and
scarcely any employees. One bank bought \$4 billion in dollars on the
auction, I was told by a member of Parliament who has investigated
corruption cases, a total that would correspond to a profit of \$200
million. ``We checked on this bank,'' the lawmaker said. ``It has one
room, one computer and some guards.''

The damage caused by the auction fraud was not just about illicit
profits. As Iraq's commercial banks transformed into instruments for
arbitrage, ordinary businesses were left without access to the loans
they needed to grow. Some legitimate importers, unable to get dollars
from the auction, were forced to use foreign banks instead. It is hard
to know how much harm this did to the economy, but all the analysts I
spoke to said it has been devastating, starving the country's private
sector and making Iraq even more dependent on its oil proceeds, which
have been cut in half in recent months.

Only one Iraqi leader has made serious efforts to expose the crimes
surrounding the dollar auction, and he was an unlikely hero. Ahmad
Chalabi, the banker and politician who helped the Bush administration
justify its invasion of Iraq, led a parliamentary investigation of the
dollar auction starting in 2014. He uncovered documents that implicated
some of the country's biggest banks and their owners in large-scale
fraud. Just as he was expected to reveal more about the scandal in
November 2015, Chalabi died of a heart attack. (Despite the suspicious
timing, autopsies found no evidence of foul play.) The bankers he
identified in his investigations suffered no consequences and are still
in business.

The auction continues to this day, and so do the money laundering and
theft that surround it. On some days in mid-March, the central bank's
website registered dollar sales of well over \$200 million --- more than
\$1 billion in a single working week --- all of it supposedly to pay for
imports. At the time, the coronavirus pandemic was shuttering Iraq's
economy. Some of those imports may be legitimate, but the bankers I
spoke to said the numbers suggest the persistence of large-scale money
laundering. Another glaring sign of fraud is the daily total of dollars
sold by the central bank to Iraq's exchange houses, which are supposed
to be used only by Iraqis traveling abroad. In mid-July, they were still
averaging \$10 million to \$11 million per day, even though Baghdad's
airport was closed from March until July 23 and travel restrictions
remain in place. There is also evidence that the auction continues to
provide money for terrorist groups. In October, the New York Federal
Reserve issued a letter to the Iraqi central bank demanding that it bar
two banks and an exchange house from using the dollar auction, stating
that it has reason to believe that the three entities are ``affiliated
with or engaged in material dealings with'' ISIS or a group with ties to
it. The three entities are owned by a financier named Hassan Nasser
Jaafar al-Lami, also known in Iraqi financial circles as ``the king of
the fake invoices.'' In January, an employee at Iraq's central bank gave
an interview to a Lebanese TV station claiming that al-Lami was still
using the auction, through other banks than those named by the Fed.

In some cases, the central bank appears to have deliberately
circumvented efforts by the Fed or the U.S. Treasury. In 2018, the
Treasury Department placed sanctions on Aras Habib Kareem, a political
figure charged with funneling money to the Revolutionary Guards and
Lebanon's Hezbollah movement. It also placed sanctions on the bank he
ran, known as Al-Bilad Islamic Bank. But instead of freezing Kareem's
assets, the Iraqi central bank in October directed that 40 million
shares in Al-Bilad bank owned by Kareem and his family be returned to
them, according to a central-bank document I obtained. When I asked
Treasury officials about the Iraqi central bank's action, they provided
a canned statement: ``Treasury continues to work closely with the
government of Iraq on compliance with U.S. sanctions.''

\textbf{Iraq is a cautionary tale} for the rest of the world,
illustrating how quickly the rawest forms of corruption can catch on and
how hard it is to reverse that process. It doesn't take long for the
dust of complicity to cover almost everyone, much as it has in
Afghanistan, Somalia or Venezuela. ``It becomes a self-sustaining
system,'' says Richard Messick, who has spent decades studying the
subject and is the senior contributor to
\href{https://globalanticorruptionblog.com/}{an influential blog that
monitors global anticorruption efforts.}``You can't just intervene in
one area, because they're all linked together, so you have to change
multiple institutions at the same time.'' It is hard to do that without
a powerful outside force. The U.S. government was essential in
eradicating the rampant graft in Chicago that peaked during the 1920s,
when the gangster Al Capone had the city's mayor on his payroll. There
are few precedents for cleaning up an entire country in modern times,
apart from autocratic city-states like Singapore, where former Prime
Minister Lee Kuan Yew cracked down hard in the 1960s.

The largest single obstacle to reform in Iraq is the country's
overwhelming dependence on cash, which is hard to trace and thus more
vulnerable to money laundering. Moving more Iraqis into the banking
system, where payments leave a record that can be verified, has been a
goal of the country's lonely anticorruption advocates for years. But the
transition away from cash is itself fraught with risk: New technologies
are vulnerable to capture by the oligarchs, who can turn them into even
more effective tools for laundering.

Image

Tens of millions of dollars meant to refurbish Baghdad's Sadr al Qanaat
area have been lost to corruption.Credit...Moises Saman/Magnum, for The
New York Times

One of Iraq's most brazen embezzlement schemes offers a near-perfect
illustration of this danger. It involves the use of a device called Qi
Card, which was intended to move the country toward electronic payments.
Developed by a company named International Smart Card, it allows
government employees and pensioners to retrieve their monthly payments
in cash at any one of thousands of terminals across the country. It is a
popular innovation: Before Qi Card came along in 2007, workers often had
to wait in line for hours outside a government bank to get their money.
The company now competes with other, smaller card companies and
advertises on huge billboards with the slogan ``Join the largest
family.'' It presents itself as a homegrown tech firm helping to bring
Iraq into the information age, with images on its website of biometric
registration processes and happy customers making cash-free payments.
But its link to the state payroll has given it immense power. In 2019,
according to a report issued by the central bank, the government paid
nearly \$47.5 billion to its employees and pensioners --- a vast sum for
a country of Iraq's size --- and much of that went through Qi Card.

This makes it all the more remarkable that the company appears to
operate with almost no oversight, according to officials I spoke with
and documents I obtained from Iraq's Finance Ministry and the central
bank. It has sidestepped a legal requirement to integrate its payment
system with the national card-payment network. This would allow the
central bank to monitor its transactions. The documents describe
frustrated efforts to make Qi Card accountable for its transactions,
along with complaints from Iraqi retirees who say Qi Card has been used
to skim from their salaries. (Qi Card's chief executive, reached by
email, said that the company complies with all relevant regulations and
that its transactions are monitored directly by the central bank, in
addition to being audited periodically by independent firms.)

Beneath this data blackout, Qi Card is being used by Iranian-backed
militia figures who operate a large-scale ``ghost employees'' scheme to
steal hundreds of millions of dollars from the state payroll, I was told
by several government officials, including one who is close to the
financial bureau of the Hashd. This official told me that the Hashd has
registered about 70,000 fictional soldiers for electronic payments via
Qi Card. (It was not clear whether this was done with or without the
knowledge of Qi Card's managers.) Ghost soldiers have been a standard
self-enrichment scheme for high-ranking officers in Iraq's Army and
police forces for years, but Qi Card appears to have allowed this ruse
to be taken to a higher level. The average salary of a Hashd member is
almost \$1,000 per month, which would put the scheme's revenues at more
than \$800 million a year. This operation, the official told me, has
been run in strict secrecy by powerful figures with deep ties to Iran,
including Abu Mahdi al-Muhandis, the militia leader who was assassinated
in January with Qassim Suleimani. Qi Card also earns enormous profits
from the fees it charges for electronic transactions. Some of that
profit, I was told by another senior Iraqi official, is shared with
other leading Iran-backed figures.

Qi Card's founder, a businessman named Bahaa Abdul Hadi, appears, the
senior official told me, to have insulated himself from scrutiny and
criticism for years by forming business relationships with Iraq's most
powerful people, including militia leaders with close ties to Iran. One
of them is Ammar al-Hakim, a prominent and wealthy Shiite cleric and
political figure. Another is Shibl al-Zaydi, the general secretary of a
militia called Kataib Imam Ali, whom the U.S. Treasury Department issued
sanctions against in 2018 for his financial dealings with the
Revolutionary Guards and Hezbollah. A third connection is Nasser
al-Shammari, a leader of another Iran-backed group called Hezbollah
al-Nujaba. (A Qi Card spokeswoman told me that Abdul Hadi has no
relationship with al-Hakim, al-Zaydi or al-Shammari.)

At the same time, Qi Card has made efforts to endear itself to American
officials, some of which appear to have paid off. In early 2018, an
incoming Trump political appointee, Max Primorac, suggested to a United
Nations agency that it use Qi Card for transactions, according to a
report
\href{https://www.propublica.org/article/pences-special-envoy-in-foreign-aid-office-sparked-an-ethics-complaint-just-weeks-after-he-started-his-job}{published
in May by ProPublica.} Primorac was doing consulting work at the time
for Markez, an American-Iraqi firm hired by Qi Card. The United Nations
did not hire Qi Card, but Primorac's pitch triggered an ethics complaint
by a State Department official, ProPublica reported. (When asked about
the complaint, Primorac responded by forwarding a memo indicating that
no investigation was opened.) He went on to become an aide to Vice
President Mike Pence.

Qi Card's forays into the American influence industry are a reminder
that corruption can encompass far more than crude payoffs and tropical
tax havens. The 2008 global financial crisis, which exposed unsavory
links between politicians and speculators, helped fuel the populist
movements still roiling Europe and the election of Donald Trump, who has
made corruption an increasingly apt description of our own political
life even as he flings the word indiscriminately at his opponents.

\textbf{On a warm afternoon in February,} I drove out to a construction
site in eastern Baghdad called Sadr al Qanaat. It is a narrow strip of
vacant land --- almost a median --- that runs for 15 miles between two
sides of a major highway on the western edge of the Sadr City slum, with
a canal at the center. The Baghdad city authorities had talked for years
of an ambitious project to turn the corridor into a vast outdoor
pleasure area, encompassing sports fields, parks, restaurants and
playgrounds. Decorative bridges would be built over the canal, where
visitors would ride back and forth on boats. In 2011, the city
government signed a contract with three construction companies for about
\$148 million.

Today the site is a dismal dumping ground with little sign that anything
was ever spent on it. Stepping off the highway onto the grass, I found
my feet covered in a thick slurry of plastic trash. I walked up and down
for 20 minutes or so and found only a few signs of construction: a cheap
prefab children's playground gathering dust, a couple of unfinished
concrete bunkers. In the concrete-lined canal, the water looked fetid.

No one seems to know exactly what happened to the money thrown at Sadr
al Qanaat, but a report by Iraq's integrity commission rings sadly
familiar notes: delays, disagreements and a former mayor who, along with
one of his deputies, fled the country after ``causing a deliberate
damage of more than \$12 million,'' most of which, presumably, ended up
in his pocket. There are projects like this all over Iraq. Abandoned
cranes rust by half-built mosques and housing projects. Many of them are
tied up in legal and political disputes. Billions of dollars have been
spent on electricity, yet Iraq still has power outages of up to 20 hours
a day.

Iraqis have a word for the shady businessmen and power brokers who grow
immensely rich at their country's expense: \emph{hitaan}, or whales.
They are widely said to be above the law. I was repeatedly warned, while
reporting this article, that my life would be in great danger if I
confronted any one of them about his illicit activities. But I did
manage eventually to speak to a whale.

He was an Iraqi construction magnate who told me he had spent years
paying off politicians to secure contracts worth many millions of
dollars. He described a world of cynical back-room deals in which deadly
rivalries are common, political alliances shift easily and the ultimate
currency is ``cash, always in dollars, always in advance.'' It was clear
that he accepted graft as his everyday reality; I didn't sense any
unease or guilt about it. He had offices and homes in multiple
countries, but he spoke in the guttural Iraqi dialect of a man without
much formal education. I was introduced via a government official who
met him through a friend. It was impossible for me to verify the details
of the stories he told. But they are consistent with everything I heard
from government insiders and bankers about the way high-level corruption
works. We spoke by phone for about two hours. He told me about one deal
he had managed, a major construction project in which the government
allocated about 40 billion dinars (about \$33.6 million).

``In reality I spent only about 10 billion dinars on construction,'' he
said. Of the rest, most went to paying off government and party
officials, along with other expenses. The remainder, about five billion
dinars (\$4.2 million), was pure profit.

He told me that for the past six or seven years, Iraq's provincial
governors --- who have great power over contracts --- have been elected
almost exclusively through deals with businessmen who pay off the
provincial council (which elects the governor) in exchange for a share
of the province's contracts. ``Anyone who has money can manipulate these
things,'' he said. The deals are elaborately constructed, with deputy
mayors loyal to different political parties dividing up the expected
proceeds from inflated contracts. A single big contract can supply
enough kickbacks to cover the bribery costs of getting a governor
elected, he said.

Government officials are not just passive recipients of bribes. The
members of the provincial council, he told me, ``knock on the doors of
businessmen and say, How can we help? Do you have someone you want to
smear? Is there a conspiracy you want to promote, someone you want to
refer to the integrity commission?'' These dark arts transcend party
loyalty; money is all that counts. ``If you want to conspire against the
Dawa Party, provincial council members from that party will cooperate
with you'' if you pay them, he said.

Behind all these deals, he told me, lurk the militias, providing muscle
and taking their cut of the cash. ``Any businessman, any bank owner
without a militant group backing him will not be able to operate,'' he
said.

Image

The self-styled anticorruption crusader Mishaan
al-Jabouri.Credit...Moises Saman/Magnum, for The New York Times

None of this is a surprise to Iraqis. They have become so cynical that
they now see even the country's various anticorruption bodies as
vehicles for extortion and bribes. Sadly, this is not an entirely
groundless charge. When I was in Baghdad, I went to see Mishaan
al-Jabouri, a businessman and politician who is famous for his jeremiads
against graft. Jabouri is a big man of 63 with a lumpy, bald head and
protruding eyes who has become a kind of mascot for his country's
struggle with corruption; he has been on every side of it. He was a
businessman in the 1980s and fled the country at the end of the decade
to join the opposition. In 2006, Jabouri had to flee Iraq again after
being accused of an elaborate extortion scheme involving attacks on oil
pipelines. He made his way back, was elected to Parliament and joined
its anticorruption committee.

``Everyone is involved, the religious, the secular, in villages, in
cities, from the top commanders to the porters,'' Jabouri told me as we
sat in a cavernous, half-furnished house he owns in the Harthiya
neighborhood. ``It became a culture. It's something people are proud
of.''

In 2016, Jabouri made headlines around the world by telling a reporter
from The Guardian that he too was corrupt;
\href{https://www.theguardian.com/world/2016/feb/19/post-war-iraq-corruption-oil-prices-revenues}{he
had taken a \$5 million bribe} from a man who wanted him to drop a fraud
investigation. ``At least I am honest about it,'' he told the newspaper.

When I saw him in February, Jabouri recanted his confession, claiming to
have invented the \$5 million bribe. I stared at him in disbelief. He
stared back. ``I needed to shake the society,'' he said. Now, he added,
such falsehoods were no longer necessary. ``The current protests are
doing this.''

\textbf{To those watching from another continent,} the street
demonstrations that captured Iraq's cities last October looked like a
sudden eruption of rage. In fact, this anger had been simmering for
years in cities and towns all over the country. One protest leader I met
was a short, squarely built 28-year-old named Mousa, who grew up in a
poor farming family in Samawa, a southern Iraqi city. (He asked that I
not use his last name because he remains in hiding and fears
repercussions.) Like many others I spoke to, Mousa repeatedly bumped up
against the cruelty of Iraq's gangster economy, where real
qualifications are often irrelevant and most job offers come with a
hefty upfront price, the equivalent of several months' salary. After
spending five years earning an advanced degree in veterinary science, he
could find only one veterinary job --- a one-year contract paying \$200
a month --- which he was fired from after he refused his boss's offer to
join a militia. He had no choice but to take a job at the regional
electricity ministry, which paid \$375 a month.

Image

Mousa, a leader of Iraq's protest movement, in hiding after receiving
threats from militias.Credit...Moises Saman/Magnum, for The New York
Times

His rebellion began more than two years ago, when, he told me, he found
documents suggesting that a regional director at the electricity
ministry was getting rich by taking kickbacks on government contracts.
Mousa helped organize protests calling for his boss to be fired. (The
boss was later dismissed, Mousa told me.) Over the following year, he
began making connections with other young people across Iraq who had had
similar experiences and shared his feelings. Many of them believed their
country was becoming a vassal of Iran and its local gun-toting thugs. By
the summer of 2019, a loose archipelago of local protest networks was
coalescing into something larger. Mousa was among the organizers who
called for a national insurrection starting on Oct. 1.

Barely a week later, he found himself sitting on a couch across from
Prime Minister Adel Abdul Mahdi. Outside the ministry's high wooden
doors, the country was on fire. More than 100 people had been killed in
chaotic clashes with the police, and the economy was at a standstill.
Abdul Mahdi was desperate to restore order, and he invited Mousa and
eight other protest leaders to hear them out. Mousa handed him a piece
of paper outlining the protesters' demands, which he read quickly, in
silence. Dealing with corruption was one of them. After a brief
exchange, one of Abdul Mahdi's advisers said: ``Give us a list of the
most corrupt people.''

Mousa, who has a patient manner and big, earnest eyes, was baffled and
irritated by the request, he told me. He already knew he was wanted for
arrest by the security services; not long afterward he would be forced
into hiding, like many other protest leaders. He also knew that some of
the country's most corrupt figures had very likely been welcomed on that
same couch. ``That's not our job, that's your job,'' he answered. The
mood soured, and the meeting broke up after only 10 minutes. The prime
minister declared soon after that the protest movement was leaderless.
The same thing might have been said about Abdul Mahdi's own floundering
government. Less than two months later, facing even wider insurrections
and a rising death toll, he announced his resignation.

The depth and fury of the protest movement took everyone by surprise.
The militias were on the defensive for the first time in years, with
some demonstrators deriding them as Iranian stooges. Even some members
of the Hashd took part. One of them described to me a tense phone call
in which he told his former boss: ``This is a revolution against you.''
In December, the Iraqi Parliament passed a landmark law that allows the
country's integrity commission to check a public servant's income
against his assets and to impose large fines or even jail time if he or
she cannot show a legitimate source for the money. A new demand for
accountability was seeping into all kinds of unexpected places. In
Baghdad, I met a young lawyer named Marwa Abd al-Rida, who took out her
cellphone and showed me documents about a curious little scandal at the
Iraqi Bar Association, which had submitted vastly inflated expenses for
the construction of a swimming pool. The fraud had been uncovered just a
day earlier, before any of the lawyers had a chance to don a bathing
suit. ``In the past, there was lots of spending and no complaints,'' she
said. ``Now, lawyers are speaking out.''

Image

Marwa Abd al-Rida, one of many lawyers now speaking out about corruption
in Iraq.Credit...Moises Saman/Magnum, for The New York Times

The protest movement's uncompromising spirit helped keep it alive --- at
least until the pandemic struck --- but also limited its impact. As the
months dragged on, the protesters adamantly refused to nominate anyone
for office. They seemed caught in an Iraqi Catch-22: They wanted to
change the system, but anyone who touched that system, even on their
behalf, became instantly suspect. Their only heroes were their martyred
comrades, whose faces appear in graffiti and posters all over the
protest squares.

\textbf{At the heart} of Iraq's protest movement is a struggle to break
free of the country's tortured history. Many in the younger generation
understand that Iraq --- like many other former colonies across Africa
and Asia --- has too often elevated its military men and clerics into
gods, only to see them transform into monsters. That is one reason the
protesters have refused to delegate any leader to represent them. They
know that what matters now is the slow, unglamorous work of building
institutions, not anointing saviors. But they are also starved for
admirable public figures. Like anyone else, they want to be inspired and
led.

Image

Caring for protesters in a makeshift first-aid station near Al-Khilani
Square this February.Credit...Moises Saman/Magnum, for The New York
Times

One leader did emerge early in the protests, and it seems fitting that
he is something of an antihero. Abdul-Wahab al-Saadi is one of Iraq's
most senior counterterrorism officials. He is beloved throughout the
country not only for his military record --- he led a series of decisive
battles against ISIS --- but also because, almost alone among Iraqi
officers, he is resolutely nonpartisan and is said never to have taken a
bribe. Last September, Iraq's prime minister abruptly sidelined him. The
protesters quickly seized on him as a political martyr who lost out
because he didn't play the game (this was partly true, though factional
rivalries played a role as well). They began carrying posters with his
face on them and chanting his name. Some demanded that Saadi be
nominated to replace Abdul Mahdi as prime minister. Saadi responded to
these overtures with a characteristic diffidence. He said he was a
military man who wasn't qualified for political office. Some protesters
were disappointed, but others were delighted, seeing his renunciation as
a badge of honor.

Saadi is a tall, razor-lean 57-year-old with an air of lonely sobriety
and a quiff of steel gray hair. Although he is Shiite, the people of
Mosul --- which is overwhelmingly Sunni --- revere him as a liberator
from the plague of ISIS, and last year a statue of him was erected
there. (The government, apparently threatened by this gesture, removed
the statue before it could be unveiled.) When I met him in February,
Saadi still seemed mildly amused by the attention he was receiving. He
told me about a series of phone calls he had received from political
heavyweights, all of them hoping to recruit him or gain his endorsement.
``The prime minister wants to hire me to get benefits in public
opinion,'' he said dismissively, dragging on a cigarette.

Saadi seemed uncomfortable talking about himself. He has a kind of
austere modesty, his hands often jammed into his pockets, gaze fixed in
the distance as if he were quietly assessing a field maneuver. For
anyone accustomed to the self-important manners of most Iraqi political
figures, Saadi forms an almost ludicrous contrast. Where they are often
plump and garrulous, he is gaunt and self-effacing. They often own
townhouses in London and Amman; he lives in a Baghdad apartment. I don't
have proof that Saadi has never taken a bribe. But there are plenty of
people in Iraq who would love to embarrass him, and no compromising
evidence has emerged. He is so uncorrupt that when his son joined the
army, he refused to use his own position to help the boy in any way ---
a level of personal integrity that some of his colleagues saw as
unnatural. When I asked him about this, he told me his own father died
young, and his older brother was executed by Saddam Hussein. He had been
forced to make his own way, and he felt it had saved him from the
slackness that infected the military. He wanted the same for his son.
``I told him, `You have to rely on yourself, I have nothing to do with
it,''' Saadi told me. ``I never helped him with rank, vacations,
privileges.''

Image

Abdul-Wahab al-Saadi, center, a senior counterterrorism official,
beloved throughout Iraq for his military record and who is said never to
have taken a bribe.Credit...Moises Saman/Magnum, for The New York Times

One Friday evening, I met Saadi at a coffee shop called Ridha Alwan, in
a lively middle-class neighborhood. We sat at an outdoor table,
surrounded by warm drafts of coffee and cardamom and flavored tobacco.
He has a stiff, taciturn demeanor, but he seemed to relax a little as we
chatted about politics and history, with frequent interruptions from
customers who wanted a handshake or a selfie with the hero of Mosul.
Saadi obliged them all with a shy grin, and when they asked if he would
play a role in the new government, he would wave them off with a
noncommittal ``\emph{inshallah}'' --- God willing. (After our meeting,
Iraq's new prime minister, Mustafa al-Kadhimi, restored and promoted
Saadi.)

Then we tried to leave the cafe. Almost as soon as he stood up, people
on the street recognized him, and he was surrounded by a thick crowd of
admirers. He patiently posed for selfies and shook hands. Cars slowed to
get a look. ``Hey, look, it's Saadi!'' I heard someone shout. A woman
began ululating. His bodyguards were looking nervous, but there was
nothing they could do. Everyone wanted a moment with him.

After 15 minutes, he was still only a few feet from the cafe, and the
street was impassable. A middle-aged man began to improvise a rhyming
ballad about Saadi and his role in saving Iraq from ISIS. The onlookers
clapped along, delighted, and snapped photos. A young taxi driver in a
black \emph{djellaba} ran up, pushed his way through the crowd and began
telling Saadi that his brother had been killed in the Baghdad protest
square. He thanked Saadi for everything he'd done and then stepped back
to let other fans in. A soldier in a helmet and flak jacket walked up
and began imploring Saadi to become the next minister of defense. Then a
police officer pushed in, saying, ``We want him as interior minister.''

Standing in the darkness, I was moved by the sight of those eager,
hopeful faces. All they ask is what many of us are lucky enough to take
for granted, at least for the moment: relatively honest bureaucrats,
clean streets, police officers who don't demand bribes. They want a
country.

Advertisement

\protect\hyperlink{after-bottom}{Continue reading the main story}

\hypertarget{site-index}{%
\subsection{Site Index}\label{site-index}}

\hypertarget{site-information-navigation}{%
\subsection{Site Information
Navigation}\label{site-information-navigation}}

\begin{itemize}
\tightlist
\item
  \href{https://help.nytimes3xbfgragh.onion/hc/en-us/articles/115014792127-Copyright-notice}{©~2020~The
  New York Times Company}
\end{itemize}

\begin{itemize}
\tightlist
\item
  \href{https://www.nytco.com/}{NYTCo}
\item
  \href{https://help.nytimes3xbfgragh.onion/hc/en-us/articles/115015385887-Contact-Us}{Contact
  Us}
\item
  \href{https://www.nytco.com/careers/}{Work with us}
\item
  \href{https://nytmediakit.com/}{Advertise}
\item
  \href{http://www.tbrandstudio.com/}{T Brand Studio}
\item
  \href{https://www.nytimes3xbfgragh.onion/privacy/cookie-policy\#how-do-i-manage-trackers}{Your
  Ad Choices}
\item
  \href{https://www.nytimes3xbfgragh.onion/privacy}{Privacy}
\item
  \href{https://help.nytimes3xbfgragh.onion/hc/en-us/articles/115014893428-Terms-of-service}{Terms
  of Service}
\item
  \href{https://help.nytimes3xbfgragh.onion/hc/en-us/articles/115014893968-Terms-of-sale}{Terms
  of Sale}
\item
  \href{https://spiderbites.nytimes3xbfgragh.onion}{Site Map}
\item
  \href{https://help.nytimes3xbfgragh.onion/hc/en-us}{Help}
\item
  \href{https://www.nytimes3xbfgragh.onion/subscription?campaignId=37WXW}{Subscriptions}
\end{itemize}
