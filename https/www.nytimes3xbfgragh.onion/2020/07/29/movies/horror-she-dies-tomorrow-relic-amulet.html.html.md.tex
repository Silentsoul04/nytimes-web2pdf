Sections

SEARCH

\protect\hyperlink{site-content}{Skip to
content}\protect\hyperlink{site-index}{Skip to site index}

\href{https://www.nytimes3xbfgragh.onion/section/movies}{Movies}

\href{https://myaccount.nytimes3xbfgragh.onion/auth/login?response_type=cookie\&client_id=vi}{}

\href{https://www.nytimes3xbfgragh.onion/section/todayspaper}{Today's
Paper}

\href{/section/movies}{Movies}\textbar{}The Horror of Isolation

\url{https://nyti.ms/39BnswZ}

\begin{itemize}
\item
\item
\item
\item
\item
\end{itemize}

\href{https://www.nytimes3xbfgragh.onion/spotlight/at-home?action=click\&pgtype=Article\&state=default\&region=TOP_BANNER\&context=at_home_menu}{At
Home}

\begin{itemize}
\tightlist
\item
  \href{https://www.nytimes3xbfgragh.onion/2020/07/28/books/time-for-a-literary-road-trip.html?action=click\&pgtype=Article\&state=default\&region=TOP_BANNER\&context=at_home_menu}{Take:
  A Literary Road Trip}
\item
  \href{https://www.nytimes3xbfgragh.onion/2020/07/29/magazine/bored-with-your-home-cooking-some-smoky-eggplant-will-fix-that.html?action=click\&pgtype=Article\&state=default\&region=TOP_BANNER\&context=at_home_menu}{Cook:
  Smoky Eggplant}
\item
  \href{https://www.nytimes3xbfgragh.onion/2020/07/27/travel/moose-michigan-isle-royale.html?action=click\&pgtype=Article\&state=default\&region=TOP_BANNER\&context=at_home_menu}{Look
  Out: For Moose}
\item
  \href{https://www.nytimes3xbfgragh.onion/interactive/2020/at-home/even-more-reporters-editors-diaries-lists-recommendations.html?action=click\&pgtype=Article\&state=default\&region=TOP_BANNER\&context=at_home_menu}{Explore:
  Reporters' Obsessions}
\end{itemize}

Advertisement

\protect\hyperlink{after-top}{Continue reading the main story}

Supported by

\protect\hyperlink{after-sponsor}{Continue reading the main story}

Critic's Notebook

\hypertarget{the-horror-of-isolation}{%
\section{The Horror of Isolation}\label{the-horror-of-isolation}}

Three new movies show why this cinematic genre is best suited to the
Covid-19 era, when isolation has become not just a way of life, but
necessary to avoid deaths.

\includegraphics{https://static01.graylady3jvrrxbe.onion/images/2020/08/02/arts/23horror-isolation1/23horror-isolation1-articleLarge.jpg?quality=75\&auto=webp\&disable=upscale}

\href{https://www.nytimes3xbfgragh.onion/by/jason-zinoman}{\includegraphics{https://static01.graylady3jvrrxbe.onion/images/2018/12/10/multimedia/author-jason-zinoman/author-jason-zinoman-thumbLarge.png}}

By \href{https://www.nytimes3xbfgragh.onion/by/jason-zinoman}{Jason
Zinoman}

\begin{itemize}
\item
  July 29, 2020
\item
  \begin{itemize}
  \item
  \item
  \item
  \item
  \item
  \end{itemize}
\end{itemize}

The first essential of a good horror film is an isolated place. Any
desolate locale will do: cabin in the woods, empty motel, middle of the
ocean,
\href{https://www.freep.com/story/entertainment/movies/julie-hinds/2016/08/25/dont-breathe-horror-film-detroit/89304296/}{Detroit},
just somewhere no one can hear you scream. Populate it with potential
victims, add a monster and you've got everything you need to make things
go bump in the night.

This is why horror is the cinematic genre best suited to the Covid-19
era, when isolation has become not just a way of life, but necessary to
avoid deaths. Social distancing has quarantined us in our homes,
increasingly alienated and lonely, eyeing strangers a little more
warily. The frustration that Jack Torrance feels toward his family in
``The Shining'' doesn't seem quite as foreign after a few months of
remote learning. And the masks of so many serial killers in slasher
movies suddenly seem fashion forward.

Even though they were made before the pandemic, three new bold and
chilling horror movies, all directed by women, have a new kind of
topical resonance.

``She Dies Tomorrow,'' which premieres on Friday, comes off as the most
prescient since it's actually about a contagion, a peculiar one where a
woman's sudden premonition that she will die the next day spreads, from
one person to another sharing the same space. The first great shock
in\href{https://www.nytimes3xbfgragh.onion/2020/07/22/movies/amulet-review.html}{``Amulet''}
occurs when a scaly bat emerges from a toilet, a terrifying image that
cannot help but remind one of wet markets in Wuhan. And even an intimate
portrait of an older, declining woman whose daughter wants to put her in
a home, like
\href{https://www.nytimes3xbfgragh.onion/2020/07/09/movies/relic-review-a-haunted-house-and-a-clouded-mind.html}{``Relic,''}takes
on additional charge considering that more than 50,000 Americans have
died of Covid-19 in such facilities.

\includegraphics{https://static01.graylady3jvrrxbe.onion/images/2020/08/02/arts/02horror-isolation5/02horror-isolation5-articleLarge.jpg?quality=75\&auto=webp\&disable=upscale}

And yet, the sturdiest connective tissue among these dread-filled movies
is a sensitivity to the punishing nature of loneliness and the sinister
aspects of solitude.

``She Dies Tomorrow'' is very different from the virus movies like
``Contagion'' and ``Outbreak'' that have suddenly become popular again
with scientists racing against the clock to save the world. Amy Seimetz,
who starred in the recent remake of ``Pet Sematary'' and
\href{https://www.nytimes3xbfgragh.onion/2016/04/03/arts/television/starz-the-girlfriend-experience.html}{helped
create the TV series ``The Girlfriend Experience,''} has made a more
eccentric, startlingly assured mood piece with the whispering vibe of a
moody indie record. Its first 15 minutes portrays a woman named Amy
(Kate Lyn Sheil) in her house alone, starting with a shot of her eye and
then often lingering in close-ups. She is convinced of her impending
doom, but seems oddly resigned to it.

In dreamy visuals Seimetz films her listening to music, dancing, online
shopping, gazing into the distance, trying to cry but failing. When she
tells a friend Jane (played with exquisite fragility by Jane Adams), she
is met by disbelief. No one connects in this movie. Everyone appears in
their own world, staring past the person they are talking to, if they
are looking at them at all.

Jane retreats to her empty home, putters about the basement, and is
suddenly struck by a terrible realization: She knows she will also die
tomorrow. Then this sense of doom keeps spreading.

There's something unsettling (and creepily familiar) about the lack of
panic. What if the apocalypse came and everyone sadly shrugged? Or maybe
more to the point: What if no one tried to stave it off?

One woman regrets she stuck in a relationship too long. Another starts
kissing a guy and while neither seems particularly passionate, what
little interest they have peters out. But everyone seems depressed and
haunted by the sense that they will ultimately die alone, an old theme
with new urgency in a time when the pandemic limits loved ones from
mourning together at a funeral. The real monster of this movie is not a
virus, but loneliness itself. This pandemic hasn't created what the
former surgeon general Vivek Murthy calls a
\href{https://www.wbur.org/onpoint/2020/03/23/vivek-murthy-loneliness}{``loneliness
epidemic,''} so much as laid it bare. More Americans live alone than
ever before and those who tell pollsters they are
\href{https://www.rollingstone.com/culture/culture-features/coronavirus-isolation-negative-effects-health-loneliness-1016478/}{lonely
have doubled} since the 1980s. Research has shown that lack of social
support does not just increase depression and mental health problems,
but also has a physical impact, particularly for older adults.

Two of these horror movies, which portray the tense relationship between
an older woman and her caregiver, speak to this situation.

``Relic'' is a nuanced character study, a portrait of a disordered mind
that hints, at supernatural terror. In her debut film, the director
Natalie Erika James shows us a family whose ties have frayed. The
grandmother, Edna (Robyn Nevin), has vanished and her daughter, Kay
(Emily Mortimer), and granddaughter (Bella Heathcote) search for her.
Kay has grown distant from her mother, and argues with her daughter
about putting her in a home. In between these mundane family scenes are
flashbacks to a cabin in the woods housing a solitary figure. The house
begins to work like a metaphor both for the shaky foundations of their
relationships as well as the mind of Edna.

James uses the tools of scary movies (ominous strings, titled
camerawork, buzzing flies) but roots them in realism. The result is one
of the most emotionally draining movies in memory, the rare scary movie
that evokes Kenneth Lonergan's sensitive play ``The Waverly Gallery,''
another portrait of a family dealing with the declining mind of a
matriarch suffering from Alzheimer's.

Image

``Relic,'' starring Robyn Nevin, shows how the gradual deterioration of
one mind can scar an entire family.Credit...Jackson Finter/IFC Midnight

With wild white hair, dirty bare feet and glassy eyes, Nevin looks like
an aging Ophelia. Like so many great horror characters, Edna is both
frightening and frightened, lashing out at her relatives, before wailing
in tears: ``Where's everyone?'' When her granddaughter asks her if she
ever gets lonely, she doesn't even answer. This is a movie about an
isolation worse than solitude: that of being separated from your mind.

While there are enough grotesque images to satisfy most horror fans, the
most terrifying shots of this movie are Post-it notes Edna places
throughout the house, reminders that say ``take pills'' or ``flush.'' As
the tension escalates, these notes become more heartbreaking, signposts
that signal growing tension. The viscerally gross and emotionally
complex climax is kicked off when her daughter finds the final one that
hits her with a devastating impact: ``I am loved.''

If ``She Dies Tomorrow'' imagines the uncanny despair of knowing you're
going to die quickly, ``Relic'' shows the pain of dying slowly, how the
gradual deterioration of one mind can scar an entire family.

Despite a large female audience, the horror genre has historically and
shamefully ignored female directors. Only two years ago, Jason Blum, the
most powerful producer in the genre, laid the blame on the lack of women
wanting to direct horror films. (After blowback, he apologized.)

These three movies demonstrate how much is lost by showing only male
perspectives. Whereas all these movies focus on complex relationships
between women, ``Amulet'' digs the deepest into gender dynamics.

Like the other films, it portrays several figures in solitude, starting
with long scenes of a lonely soldier, Tomaz (Alec Secareanu), perhaps
suffering from post-traumatic stress disorder, standing guard in the
woods. Then the action fast-forwards to after the war when he signs up
to help a reclusive young woman, Magda (Carla Juri), take care of her
invalid mother, who lives in the attic.

Image

In ``Amulet,'' a former soldier (Alec Secareanu)~helps a reclusive young
woman (Carla Juri) take care of her invalid mother, who lives in the
attic.Credit...Nick Wall/Magnet Releasing

No one is exactly who he or she appear to be. And neither is the movie.
There are many twists, but what begins as a story about a man's tortured
past pivots ferociously into a supernatural revenge tale. It explores
the question of forgiving men who did bad things. It's not didactic, but
if you want to find \#metoo themes, they are there.

The actor Romola Garai makes an audacious directing debut, staging
scenes of ugly horror with subtlety and misdirection, before setting you
up for the full-on assault, generating memorable set pieces, including
one that makes overt what the creators of the chest-burster scene in
``Alien'' only tried to imply. She also teases out some stellar
performances, from an inscrutably tender Juri to a raucously
entertaining one from Imelda Staunton, who plays a nun who introduces
Tomaz to Magda and her mother. She's having a ball playing in gothic
melodramatic style. Not since Darth Vader has anyone said ``It is your
destiny'' with as much gravitas.

I've always been skeptical of the idea that bad times make for good
horror. The best scary movies work on fears more primal than those you
find in the headlines. But clearly, horror articulates buried cultural
anxieties, and right now, while the escalating case numbers and death
tolls are the most important measures of the current crisis, there are
other, less obvious, disasters going on, ones that will linger. Human
beings are social animals, and pushing against those instincts will have
consequences, some of which are the stuff of horror.

These movies dredge up those hidden monsters. And it's fitting that they
all had their premieres at drive-in theaters, since there's something
about watching images of isolation separated by glass and metal that
only adds to their chill.

Advertisement

\protect\hyperlink{after-bottom}{Continue reading the main story}

\hypertarget{site-index}{%
\subsection{Site Index}\label{site-index}}

\hypertarget{site-information-navigation}{%
\subsection{Site Information
Navigation}\label{site-information-navigation}}

\begin{itemize}
\tightlist
\item
  \href{https://help.nytimes3xbfgragh.onion/hc/en-us/articles/115014792127-Copyright-notice}{©~2020~The
  New York Times Company}
\end{itemize}

\begin{itemize}
\tightlist
\item
  \href{https://www.nytco.com/}{NYTCo}
\item
  \href{https://help.nytimes3xbfgragh.onion/hc/en-us/articles/115015385887-Contact-Us}{Contact
  Us}
\item
  \href{https://www.nytco.com/careers/}{Work with us}
\item
  \href{https://nytmediakit.com/}{Advertise}
\item
  \href{http://www.tbrandstudio.com/}{T Brand Studio}
\item
  \href{https://www.nytimes3xbfgragh.onion/privacy/cookie-policy\#how-do-i-manage-trackers}{Your
  Ad Choices}
\item
  \href{https://www.nytimes3xbfgragh.onion/privacy}{Privacy}
\item
  \href{https://help.nytimes3xbfgragh.onion/hc/en-us/articles/115014893428-Terms-of-service}{Terms
  of Service}
\item
  \href{https://help.nytimes3xbfgragh.onion/hc/en-us/articles/115014893968-Terms-of-sale}{Terms
  of Sale}
\item
  \href{https://spiderbites.nytimes3xbfgragh.onion}{Site Map}
\item
  \href{https://help.nytimes3xbfgragh.onion/hc/en-us}{Help}
\item
  \href{https://www.nytimes3xbfgragh.onion/subscription?campaignId=37WXW}{Subscriptions}
\end{itemize}
