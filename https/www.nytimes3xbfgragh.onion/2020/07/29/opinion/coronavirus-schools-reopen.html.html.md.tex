Sections

SEARCH

\protect\hyperlink{site-content}{Skip to
content}\protect\hyperlink{site-index}{Skip to site index}

\href{https://myaccount.nytimes3xbfgragh.onion/auth/login?response_type=cookie\&client_id=vi}{}

\href{https://www.nytimes3xbfgragh.onion/section/todayspaper}{Today's
Paper}

\href{/section/opinion}{Opinion}\textbar{}Opening Schools Won't Be Easy,
but Here's How to Do It Safely

\url{https://nyti.ms/2X6SAj1}

\begin{itemize}
\item
\item
\item
\item
\item
\end{itemize}

Advertisement

\protect\hyperlink{after-top}{Continue reading the main story}

\href{/section/opinion}{Opinion}

Supported by

\protect\hyperlink{after-sponsor}{Continue reading the main story}

\hypertarget{opening-schools-wont-be-easy-but-heres-how-to-do-it-safely}{%
\section{Opening Schools Won't Be Easy, but Here's How to Do It
Safely}\label{opening-schools-wont-be-easy-but-heres-how-to-do-it-safely}}

Not all schools will be able to restart. For those that can, the focus
should be on more than just the classroom.

By \href{https://hcmg.wharton.upenn.edu/profile/zemanuel/}{Ezekiel J.
Emanuel}, Saskia Popescu and
\href{https://smhs.gwu.edu/emed/education-training/fellowships/ems-disaster}{James
Phillips}

Drs. Emanuel and Phillips are medical doctors. Dr. Popescu is an
infectious disease epidemiologist.

\begin{itemize}
\item
  July 29, 2020
\item
  \begin{itemize}
  \item
  \item
  \item
  \item
  \item
  \end{itemize}
\end{itemize}

\includegraphics{https://static01.graylady3jvrrxbe.onion/images/2020/07/29/opinion/29Emanuel/merlin_174675372_0a5d9563-0119-4db0-9d1a-a22802434163-articleLarge.jpg?quality=75\&auto=webp\&disable=upscale}

As we wrestle in the next few weeks over which of the nation's schools
to reopen --- and let's be clear, not all of them will be safe to
restart --- we need to understand this: Kindergarten through the 12th
grade involves more than just the classroom.

To create safe schools is much more complex than just having students
wear face masks and sit physically distanced from one another in class.
We must ensure that all five of the core school-based activities ---
transportation, time in the classroom, mealtimes, gym and
extracurricular activities --- are safe.

To do this, we created a school risk index to assess the dangers and
offer recommendations to reduce the chance of spreading the virus, not
only among students but also among teachers and other employees.

\hypertarget{safety-first}{%
\subsection{Safety First}\label{safety-first}}

Different school activities have different risk levels, offering options
to reopen schools as safely as possible. In all reopenings, basic
precautions are essential: low community spread, physical distancing,
mask-wearing, hand hygiene, adequate ventilation and cleaning of
facilities at night.

LOW RISK

MEDIUM RISK

HIGH RISK

Transportation

to and from school

Walk or ride a bicycle

Carpool with non-household members

School bus

Public transportation

(subway, bus)

Riding in a car

Household members only

Routine classwork

Desk-based instruction

Unmonitored study hall

Lockers/changing rooms between classes

Shop and

vocational/

technical class

Going to the restroom

Lunchtime

Assuming 6 feet of distancing at all times

Picking up prepackaged meals

Cafeteria lunch line

Indoor seating in the

cafeteria

Indoor eating

in the classroom

Outdoor eating

Arts \& Humanities

Art indoor

Band and orchestra

Supervised clubs and organizations

Choir

Drama performances

Recess \& Athletics

Outdoor

playgrounds

Indoor non- contact sports

All contact sports, indoor or outdoor

Outdoor

non-contact sports

Locker rooms

Transportation to and from school

LOW RISK

MEDIUM RISK

HIGH RISK

Walk or ride a bicycle

Carpool with non-household members

School bus

Public

transportation

Automobile

Household members only

(subway, bus)

Routine classwork

LOW RISK

MEDIUM RISK

HIGH RISK

Unmonitored study hall

Desk-based instruction

Lockers/

changing rooms between classes

Shop and

vocational/

technical class

Going to the restroom

Lunchtime

Assuming 6 feet of distancing at all times

LOW RISK

MEDIUM RISK

HIGH RISK

Cafeteria lunch line

Picking up prepackaged meals

Indoor seating in the cafeteria

Indoor eating

in the classroom

Outdoor eating

Arts \& Humanities

LOW RISK

MEDIUM RISK

HIGH RISK

Band and orchestra

Art indoor

Supervised clubs and organizations

Choir

Drama performances

Recess \& Athletics

LOW RISK

MEDIUM RISK

HIGH RISK

Indoor non- contact sports

Outdoor

playgrounds

All contact sports, indoor or outdoor

Outdoor

non-contact sports

Locker rooms

Source: Based on a chart by Covid19 Reopen using information from the
C.D.C. and the National Academies of Sciences, Engineering and Medicine
\textbar{} Adapted by The New York Times

Four principles stand out.

First, schools cannot reopen safely when community transmission is high
and climbing. In our view, schools should open only in places that have
fewer than 75 confirmed cases per 100,000 people cumulatively over the
previous seven days, and that have a test positivity rate
\href{https://coronavirus.jhu.edu/testing/tracker/overview}{below 5
percent}. By our count,
\href{https://www.nytimes3xbfgragh.onion/interactive/2020/us/coronavirus-us-cases.html\#states}{12
states and the District of Columbia} meet both metrics. In many larger
states, some counties or cities meet those criteria. Even with those
numbers, about one in 1,300 people might return to school with a case of
the coronavirus, meaning a school of 350 students, faculty and staff
will have roughly a one-in-four chance of someone coming in with
Covid-19. (Many countries, such as Japan, Austria and Italy, have
suppressed the virus to the extent that they have fewer than one in
10,000 people with confirmed cases.)

Second, schools should avoid high-risk activities. This means no contact
sports either in the gym or in competitive athletics for high school
students. It also means no band, choir or drama performances. We know
that this will be both disappointing and difficult. But close contact
for prolonged periods of time with forced exhalations is what increases
the risk of transmission. Playing football and basketball and wrestling
simply cannot be done safely. We understand that missing a season could
lead to missed scholarships for student athletes. But these activities
will have to wait a year. That said, we should allow outside physical
activity on playgrounds, ideally with masks, and noncontact sports like
track and field.

For meals it means no cafeteria time, where crowding would be likely to
encourage spread. The best alternative is to have meals packaged and
delivered to classrooms, which would also reduce contact between
cafeteria workers and students. Schools should also bar the use of
lockers because they cause crowding and congestion and therefore
increase the risk of transmission.

Third, focus on the basics where risks are tolerable --- that is at the
medium level or lower on our chart. Yes, the classroom may pose the most
risk, as students will spend the most time there and are in proximity to
others who may cough or sneeze. This will certainly be the case during
cold and flu season. But we think that with proper funding, classrooms
can be made relatively safe.

Which leads us to our fourth point. Schools must adhere to public health
measures and reduce density in classrooms and elsewhere on campus.

Students, faculty and other employees should wear masks properly at all
times. Face shields, while not as good as masks --- droplets can enter
or exit from the bottom or sides --- are better than nothing and an
option for those who for medical reasons are unable to wear a mask.
Everyone should wash hands or sanitize them every hour or so. Adequate
ventilation and rigorous nightly cleaning of facilities with
disinfectants will also need to be aspects of these return-to-school
strategies.

Schools need to reduce class size to allow students to be 6 feet apart.
\href{https://www.nationalacademies.org/news/2020/07/schools-should-prioritize-reopening-in-fall-2020-especially-for-grades-k-5-while-weighing-risks-and-benefits}{Recommendations}
from the National Academies of Sciences, Engineering and Medicine and
\href{https://www.sciencemag.org/news/2020/07/school-openings-across-globe-suggest-ways-keep-coronavirus-bay-despite-outbreaks}{data
from other countries} suggest some 10 to 15 students in a regular
classroom is about right. Students should also be assigned to ``pods''
or ``cohorts'' --- small groups to limit the number of people they
interact with throughout the day. Students should not change classrooms
--- teachers should --- and student arrivals and departures should be
staggered.

How can we do this? One step is to use cafeterias, gyms, band rooms and
other spaces as classrooms. Other options for instruction might be
turning to playing fields, tents and mobile classrooms. Music and gym
teachers and athletic coaches could assist in teaching outside their
normal assignments.

Being safe is not free. Unless school safety becomes something
exclusively for the rich, the federal government will have to provide
the funds for school districts to carry out these measures. Schools will
need to hire more staff, put up new classrooms and have available
personal protective equipment --- hand sanitizer, air filters and other
safety gear.

Funding will also be needed to pay for off-campus or virtual options for
students whose immune systems are compromised or are otherwise at risk.
Teachers too, who are at risk, should not be forced to teach in person
and should be assigned to virtual classrooms or after-school tutoring
without being penalized for it.

We all want schools to open, even as we recognize the risks attached.
Will the measures we outline permit a normal school experience? No. But
these recommendations will permit relatively safe schools and allow
in-person education and as much socializing as possible.

During this pandemic, that is the best we can hope for.

Ezekiel J. Emanuel
(\href{https://twitter.com/zekeemanuel?lang=en}{@ZekeEmanuel}) is vice
provost of global initiatives and professor of medical ethics and health
policy at the University of Pennsylvania. Saskia Popescu
(\href{https://twitter.com/SaskiaPopescu?ref_src=twsrc\%5Egoogle\%7Ctwcamp\%5Eserp\%7Ctwgr\%5Eauthor}{@SaskiaPopescu})
is an infectious disease epidemiologist. James Phillips
(@\href{https://twitter.com/DrPhillipsMD}{DrPhillipsMD}) is chief of
disaster medicine at George Washington University's School of Medicine
and Health Sciences. The three are partners in a consulting firm that
advises companies on how to respond to Covid-19.

\emph{The Times is committed to publishing}
\href{https://www.nytimes3xbfgragh.onion/2019/01/31/opinion/letters/letters-to-editor-new-york-times-women.html}{\emph{a
diversity of letters}} \emph{to the editor. We'd like to hear what you
think about this or any of our articles. Here are some}
\href{https://help.nytimes3xbfgragh.onion/hc/en-us/articles/115014925288-How-to-submit-a-letter-to-the-editor}{\emph{tips}}\emph{.
And here's our email:}
\href{mailto:letters@NYTimes.com}{\emph{letters@NYTimes.com}}\emph{.}

\emph{Follow The New York Times Opinion section on}
\href{https://www.facebookcorewwwi.onion/nytopinion}{\emph{Facebook}}\emph{,}
\href{http://twitter.com/NYTOpinion}{\emph{Twitter (@NYTopinion)}}
\emph{and}
\href{https://www.instagram.com/nytopinion/}{\emph{Instagram}}\emph{.}

Advertisement

\protect\hyperlink{after-bottom}{Continue reading the main story}

\hypertarget{site-index}{%
\subsection{Site Index}\label{site-index}}

\hypertarget{site-information-navigation}{%
\subsection{Site Information
Navigation}\label{site-information-navigation}}

\begin{itemize}
\tightlist
\item
  \href{https://help.nytimes3xbfgragh.onion/hc/en-us/articles/115014792127-Copyright-notice}{©~2020~The
  New York Times Company}
\end{itemize}

\begin{itemize}
\tightlist
\item
  \href{https://www.nytco.com/}{NYTCo}
\item
  \href{https://help.nytimes3xbfgragh.onion/hc/en-us/articles/115015385887-Contact-Us}{Contact
  Us}
\item
  \href{https://www.nytco.com/careers/}{Work with us}
\item
  \href{https://nytmediakit.com/}{Advertise}
\item
  \href{http://www.tbrandstudio.com/}{T Brand Studio}
\item
  \href{https://www.nytimes3xbfgragh.onion/privacy/cookie-policy\#how-do-i-manage-trackers}{Your
  Ad Choices}
\item
  \href{https://www.nytimes3xbfgragh.onion/privacy}{Privacy}
\item
  \href{https://help.nytimes3xbfgragh.onion/hc/en-us/articles/115014893428-Terms-of-service}{Terms
  of Service}
\item
  \href{https://help.nytimes3xbfgragh.onion/hc/en-us/articles/115014893968-Terms-of-sale}{Terms
  of Sale}
\item
  \href{https://spiderbites.nytimes3xbfgragh.onion}{Site Map}
\item
  \href{https://help.nytimes3xbfgragh.onion/hc/en-us}{Help}
\item
  \href{https://www.nytimes3xbfgragh.onion/subscription?campaignId=37WXW}{Subscriptions}
\end{itemize}
