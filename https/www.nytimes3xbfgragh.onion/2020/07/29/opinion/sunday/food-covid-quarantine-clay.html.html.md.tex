\href{/section/opinion/sunday}{Sunday Review}\textbar{}I Cured My
Pandemic Anxiety by Making Tiny Food Out of Clay

\url{https://nyti.ms/3gmRIy3}

\begin{itemize}
\item
\item
\item
\item
\item
\end{itemize}

\href{https://www.nytimes3xbfgragh.onion/spotlight/at-home?action=click\&pgtype=Article\&state=default\&region=TOP_BANNER\&context=at_home_menu}{At
Home}

\begin{itemize}
\tightlist
\item
  \href{https://www.nytimes3xbfgragh.onion/2020/07/28/books/time-for-a-literary-road-trip.html?action=click\&pgtype=Article\&state=default\&region=TOP_BANNER\&context=at_home_menu}{Take:
  A Literary Road Trip}
\item
  \href{https://www.nytimes3xbfgragh.onion/2020/07/29/magazine/bored-with-your-home-cooking-some-smoky-eggplant-will-fix-that.html?action=click\&pgtype=Article\&state=default\&region=TOP_BANNER\&context=at_home_menu}{Cook:
  Smoky Eggplant}
\item
  \href{https://www.nytimes3xbfgragh.onion/2020/07/27/travel/moose-michigan-isle-royale.html?action=click\&pgtype=Article\&state=default\&region=TOP_BANNER\&context=at_home_menu}{Look
  Out: For Moose}
\item
  \href{https://www.nytimes3xbfgragh.onion/interactive/2020/at-home/even-more-reporters-editors-diaries-lists-recommendations.html?action=click\&pgtype=Article\&state=default\&region=TOP_BANNER\&context=at_home_menu}{Explore:
  Reporters' Obsessions}
\end{itemize}

\includegraphics{https://static01.graylady3jvrrxbe.onion/images/2020/08/02/opinion/sunday/29Ackermann-06/29Ackermann-06-articleLarge.jpg?quality=75\&auto=webp\&disable=upscale}

Sections

\protect\hyperlink{site-content}{Skip to
content}\protect\hyperlink{site-index}{Skip to site index}

\href{/section/opinion}{Opinion}

\hypertarget{i-cured-my-pandemic-anxiety-by-making-tiny-food-out-of-clay}{%
\section{I Cured My Pandemic Anxiety by Making Tiny Food Out of
Clay}\label{i-cured-my-pandemic-anxiety-by-making-tiny-food-out-of-clay}}

The subtle pleasures of trying to make one small piece of the world just
right.

Credit...

Supported by

\protect\hyperlink{after-sponsor}{Continue reading the main story}

Text and Illustrations by Rebecca Ackermann

\begin{itemize}
\item
  July 29, 2020
\item
  \begin{itemize}
  \item
  \item
  \item
  \item
  \item
  \end{itemize}
\end{itemize}

I've always hated cooking. My mother thought it was drudgery; her mother
felt the same. As for me, a tech worker, I hid in my profession's
disdain for inefficiency. The calculus of exchanging time or money for
feeding my family never balanced in cooking's favor. For a perfectionist
like me (another matrilineal inheritance) long recipes looked like
baroque forms of self-torture; there wasn't even time to shop at the
good grocery store.

\includegraphics{https://static01.graylady3jvrrxbe.onion/images/2020/08/02/opinion/sunday/29Ackermann-03/29Ackermann-03-articleLarge.jpg?quality=75\&auto=webp\&disable=upscale}

But in quarantine, I watched friends extract comfort from making food.
In the muck of working and parenting and worrying and scrolling, I
wanted that for myself. Always the diligent student, I bookmarked
recipes, sourced flour, bought a stand mixer. I felt pride in my
no-knead bread, my fridge-cleaning frittata, my gut-destroying
double-chocolate cookies.

Still, sometimes my family didn't want to eat what I made, or it looked
funny, or worst of all, it tasted bad and I didn't know why. I couldn't
find the patience to learn from imperfection. Instead, I boiled over
with annoyance and exhaustion. Cooking was no different from scrubbing
the dishes or supervising Zoom preschool. It was just another thing I
was failing at.

Image

Image

Image

Image

After trying tie dye, YouTube yoga, face paint, Instagram read-alongs
and homemade coloring books, one day I bought some polymer clay to pass
the hours with my daughter. She wanted to make fruit, so we rolled baby
apples and oranges in our palms. She demanded cookies, so we carefully
placed minuscule chocolate chips onto tiny balls of dough. She enjoyed
it; I was addicted. The soft clay in my hands slowed my anxious
breathing. I fell into a trance mixing the perfect shade of icing.
``Clay today?'' I'd beg my child. After spaghetti, she lost interest,
but I found my source of comfort.

I started to stay up too long after everyone's bedtime making beautiful
little meals I could never master at life size. It was the first time in
a long time I was really alone, the first time my mind was still. The
only thing at stake was my own satisfaction.

Image

Image

My husband does most of the cooking now, and my heart belongs to clay. I
keep lists of new clay meals to make. I follow genius chefs on social
media and save screenshots for inspiration. I bore my friends over text
with my plans. My daughter and I still do clay together when she's in
the mood, but she gets angry if her results don't look like mine. So I'm
working to teach her the word ``experiment'' and the notion that each
time she tries, the trying makes her better. It's a lesson I'm still
learning at the end of every strange, horrible or hopeful day in
quarantine, when I sit down with my clay and my little tools and I try
again to make one small piece of the world just right.

Rebecca Ackermann is a designer and writer.

\emph{The Times is committed to publishing}
\href{https://www.nytimes3xbfgragh.onion/2019/01/31/opinion/letters/letters-to-editor-new-york-times-women.html}{\emph{a
diversity of letters}} \emph{to the editor. We'd like to hear what you
think about this or any of our articles. Here are some}
\href{https://help.nytimes3xbfgragh.onion/hc/en-us/articles/115014925288-How-to-submit-a-letter-to-the-editor}{\emph{tips}}\emph{.
And here's our email:}
\href{mailto:letters@NYTimes.com}{\emph{letters@NYTimes.com}}\emph{.}

\emph{Follow The New York Times Opinion section on}
\href{https://www.facebookcorewwwi.onion/nytopinion}{\emph{Facebook}}\emph{,}
\href{http://twitter.com/NYTOpinion}{\emph{Twitter (@NYTopinion)}}
\emph{and}
\href{https://www.instagram.com/nytopinion/}{\emph{Instagram}}\emph{.}

Advertisement

\protect\hyperlink{after-bottom}{Continue reading the main story}

\hypertarget{site-index}{%
\subsection{Site Index}\label{site-index}}

\hypertarget{site-information-navigation}{%
\subsection{Site Information
Navigation}\label{site-information-navigation}}

\begin{itemize}
\tightlist
\item
  \href{https://help.nytimes3xbfgragh.onion/hc/en-us/articles/115014792127-Copyright-notice}{©~2020~The
  New York Times Company}
\end{itemize}

\begin{itemize}
\tightlist
\item
  \href{https://www.nytco.com/}{NYTCo}
\item
  \href{https://help.nytimes3xbfgragh.onion/hc/en-us/articles/115015385887-Contact-Us}{Contact
  Us}
\item
  \href{https://www.nytco.com/careers/}{Work with us}
\item
  \href{https://nytmediakit.com/}{Advertise}
\item
  \href{http://www.tbrandstudio.com/}{T Brand Studio}
\item
  \href{https://www.nytimes3xbfgragh.onion/privacy/cookie-policy\#how-do-i-manage-trackers}{Your
  Ad Choices}
\item
  \href{https://www.nytimes3xbfgragh.onion/privacy}{Privacy}
\item
  \href{https://help.nytimes3xbfgragh.onion/hc/en-us/articles/115014893428-Terms-of-service}{Terms
  of Service}
\item
  \href{https://help.nytimes3xbfgragh.onion/hc/en-us/articles/115014893968-Terms-of-sale}{Terms
  of Sale}
\item
  \href{https://spiderbites.nytimes3xbfgragh.onion}{Site Map}
\item
  \href{https://help.nytimes3xbfgragh.onion/hc/en-us}{Help}
\item
  \href{https://www.nytimes3xbfgragh.onion/subscription?campaignId=37WXW}{Subscriptions}
\end{itemize}
