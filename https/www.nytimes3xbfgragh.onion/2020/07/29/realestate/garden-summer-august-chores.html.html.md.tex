Sections

SEARCH

\protect\hyperlink{site-content}{Skip to
content}\protect\hyperlink{site-index}{Skip to site index}

\href{https://www.nytimes3xbfgragh.onion/section/realestate}{Real
Estate}

\href{https://myaccount.nytimes3xbfgragh.onion/auth/login?response_type=cookie\&client_id=vi}{}

\href{https://www.nytimes3xbfgragh.onion/section/todayspaper}{Today's
Paper}

\href{/section/realestate}{Real Estate}\textbar{}It's Summertime, and
the Gardening Should Be Easy, Right?

\url{https://nyti.ms/2X2g0pF}

\begin{itemize}
\item
\item
\item
\item
\item
\item
\end{itemize}

\href{https://www.nytimes3xbfgragh.onion/spotlight/at-home?action=click\&pgtype=Article\&state=default\&region=TOP_BANNER\&context=at_home_menu}{At
Home}

\begin{itemize}
\tightlist
\item
  \href{https://www.nytimes3xbfgragh.onion/2020/07/28/books/time-for-a-literary-road-trip.html?action=click\&pgtype=Article\&state=default\&region=TOP_BANNER\&context=at_home_menu}{Take:
  A Literary Road Trip}
\item
  \href{https://www.nytimes3xbfgragh.onion/2020/07/29/magazine/bored-with-your-home-cooking-some-smoky-eggplant-will-fix-that.html?action=click\&pgtype=Article\&state=default\&region=TOP_BANNER\&context=at_home_menu}{Cook:
  Smoky Eggplant}
\item
  \href{https://www.nytimes3xbfgragh.onion/2020/07/27/travel/moose-michigan-isle-royale.html?action=click\&pgtype=Article\&state=default\&region=TOP_BANNER\&context=at_home_menu}{Look
  Out: For Moose}
\item
  \href{https://www.nytimes3xbfgragh.onion/interactive/2020/at-home/even-more-reporters-editors-diaries-lists-recommendations.html?action=click\&pgtype=Article\&state=default\&region=TOP_BANNER\&context=at_home_menu}{Explore:
  Reporters' Obsessions}
\end{itemize}

Advertisement

\protect\hyperlink{after-top}{Continue reading the main story}

Supported by

\protect\hyperlink{after-sponsor}{Continue reading the main story}

IN THE GARDEN

\hypertarget{its-summertime-and-the-gardening-should-be-easy-right}{%
\section{It's Summertime, and the Gardening Should Be Easy,
Right?}\label{its-summertime-and-the-gardening-should-be-easy-right}}

Actually, there's a lot you still need to do. Here's a list to get you
started.

\includegraphics{https://static01.graylady3jvrrxbe.onion/images/2020/08/02/realestate/29garden1/29garden1-articleLarge.jpg?quality=75\&auto=webp\&disable=upscale}

By Margaret Roach

\begin{itemize}
\item
  July 29, 2020
\item
  \begin{itemize}
  \item
  \item
  \item
  \item
  \item
  \item
  \end{itemize}
\end{itemize}

For any gardener, high summer is quiet compared to spring or fall, with
the requisite cleanup. But while it may not be a time for ambitious
planting projects, it is a time for ongoing maintenance --- especially
chores that pay off long-term.

At public gardens like Untermyer Park and Gardens in Yonkers, N.Y., a
43-acre former estate on the Hudson, it was an eerily quiet spring and
early summer without visitors, even before the ``summer pause,'' as
Timothy Tilghman, the head gardener calls it. But there is still work to
be done.

``Patrons elevate the standards,'' Mr. Tilghman said, but even without
them, ``a public garden should always be display-worthy.'' He keeps his
eye on his list of tasks, reshuffling entries as priorities and staffing
levels shift.

\begin{center}\rule{0.5\linewidth}{\linethickness}\end{center}

\begin{center}\rule{0.5\linewidth}{\linethickness}\end{center}

Almost no one saw the pink-and-purple palette carefully planned to
delight early season visitors, although the seeds had been tracked down
and propagated. To spare those sweet peas, purple-leaf mustards and
violets a fate as compost before their time, the bare-bones crew took
some home for private enjoyment, which eliminated one chore --- watering
them --- from their to-do list.

The lockdown did allow a little cheating: The crew left sprinklers and
hoses out between waterings of the 2,000 newly planted perennials,
rather than dragging them back into storage daily, as they would in a
normal year.

\includegraphics{https://static01.graylady3jvrrxbe.onion/images/2020/08/02/realestate/29garden2/29garden2-articleLarge.jpg?quality=75\&auto=webp\&disable=upscale}

But in other ways, the public's absence got back at the gardeners by
creating additional tasks: ``We never have to weed our paths, because
usually visitors walk on them,'' Mr. Tilghman said. ``But this year, no
feet.''

Perhaps he can cross that job off his list now, as
\href{https://www.untermyergardens.org/visit.html}{Untermyer reopened}
on Fridays, Saturdays and Sundays beginning July 10, with timed tickets.
Guests are back, and as August beckons, Mr. Tilghman shared his
high-summer to-do list.

\hypertarget{water-and-weed-consistently}{%
\subsection{Water and Weed
Consistently}\label{water-and-weed-consistently}}

``If you can't enjoy weeding, you won't be a happy gardener,'' said Mr.
Tilghman, citing its importance to a garden's health and visuals.
``Everyone enjoys the neatness of a fresh planting, but unless you're
willing and eager to get in there and weed \ldots{}''

Deep, diligent watering, like weeding, is also crucial. (Although after
each session at Untermyer, the gear must once again be stowed.)

And while you're tending to both: Observe and make note of what needs
fixing.

``We look for scale, vigor, composition --- and aesthetic worthiness,''
Mr. Tilghman said. ``Does a plant look good in the border, and is it
worth growing?''

Sometimes what was irresistible in a catalog isn't as appealing in your
garden. ``Once established, maybe it doesn't add a strong visual
element, or it's too compact --- or colonizes when you didn't expect it
to,'' he said.

At Untermyer, those insights become essential fodder for action plans in
the fall and the following spring.

Image

Although some blooms may be sacrificed, the gardeners at Untermyer may
cut out badly placed branches of long-neglected cherry trees in the
relative quiet of August, rather than let them get more overgrown and
compromise the aging trees' health.Credit...Jessica Norman, Untermyer
Gardens Conservancy

\hypertarget{deadhead-and-groom}{%
\subsection{Deadhead and Groom}\label{deadhead-and-groom}}

Untermyer's lavish annual and tropical schemes in beds and pots are
deadheaded regularly and pinched back for scale. But not just the
annuals.

Some shrubs, including messy-looking spent roses and even certain
\href{https://www.nytimes3xbfgragh.onion/2020/07/01/realestate/how-to-grow-hydrangea-uncommon-varieties.html/}{hydrangeas},
also need grooming.

``While most hydrangeas look great through winter, some don't,'' Mr.
Tilghman said. ``The arborescens flowers that won't look good dry and
tawny, and pull the plant down into a flattened mess --- they get
deadheaded, too,'' along with any floppy blue mopheads or macrophyllas.

``We'd rather have nice green shrubs,'' he said. ``Thankfully, the
oakleaf types, you usually don't have to touch.''

\hypertarget{or-plan-to-save-seed}{%
\subsection{Or Plan to Save Seed}\label{or-plan-to-save-seed}}

Some
\href{https://www.nytimes3xbfgragh.onion/2020/04/28/realestate/shopping-your-garden-plants-flowers.html?/}{annuals
self-sow} if they are allowed to set seed --- including Nicotiana,
Verbena bonariensis and annual poppies. So don't deadhead every last
fading flower as late summer approaches.

``When I go into a nursery and there's a flat of 12 seedlings at \$5 a
plant, I just can't buy it,'' Mr. Tilghman said. ``With the really
prolific self-sowers, it's much more economical to gather seed this
summer and fall --- or just plan to leave plants in place to sow
themselves.''

Image

In August 2018, the Untermyer gardeners began turning the estate's
former gatehouse into the Ruin Garden, using shade plants with diverse
colors and textures of foliage. Because the property had been in decline
for decades, they had to dig down three feet to remove years of debris
and bring in new soil. But they left the graffiti on the walls.
~Credit...Jessica Norman, Untermyer Gardens Conservancy

\hypertarget{dont-forget-to-edge}{%
\subsection{Don't Forget to Edge}\label{dont-forget-to-edge}}

Attention to detail was instilled in Mr. Tilghman when he worked for
Marco Polo Stufano, the founding director of horticulture at Wave Hill,
a public garden not far down the Hudson, in the Bronx.

``A secret: You can make your garden look pretty good, no matter what's
really going on,'' he said, ``as long as path and bed edges are crisp
and weed-free. Marco used to remind us that `God is in the edges.'''

Although the traditional method involves the use of an edging tool or
spade, or even sheep shears to clip errant grass, Mr. Tilghman automates
the process (always wearing eye and ear protection): He turns a weed
whip, or string trimmer, 180 degrees to cut a vertical slice rather than
a horizontal one. The trigger will be on top of the pole, meaning you
use your thumb to operate it.

``Rather than working in a sweeping motion,'' he said, ``I put the motor
against my hip, so I'm not moving the whip, but slowly walking with it
in position.'' It takes practice, but is a timesaver.

``I got yelled at the first time I did it at Wave Hill, but then Marco
followed me around for a minute and acknowledged it wasn't as good, but
good enough --- which I took for an approval.''

Image

A ring of mulch applied two or three inches deep right out to the drip
line, rather than ground cover-plantings or even grass, is best beneath
trees, reducing competition for moisture and nutrients. But mulch should
never be piled against the trunk, where it can invite pests and disease,
and the tree's decline.Credit...Jessica Norman, Untermyer Gardens
Conservancy

\hypertarget{theres-a-right-and-a-wrong-way-to-mulch-around-trees}{%
\subsection{There's a Right (and a Wrong) Way to Mulch Around
Trees}\label{theres-a-right-and-a-wrong-way-to-mulch-around-trees}}

A ring of mulch right out to the drip line is better for a tree than
ground cover or even grass, which compete for moisture and nutrients.

Untermyer's two most important trees, old weeping beeches, had been
underplanted with pachysandra, which Mr. Tilghman removed in favor of
mulch. But just two or three inches, no deeper --- never
\href{https://bygl.osu.edu/node/1006}{the dreaded ``volcano mulch''} ---
and never mulch against the trunk, where it can harm the bark and invite
decline.

``If you do have ground cover growing around trees,'' he said, ``this is
a good time to edit and get it six to 12 inches away from the trunk,
like the mulch.'' Rodents love to tuck in and gnaw on bark, especially
in winter.

\hypertarget{summer-prune-that-wisteria}{%
\subsection{Summer-Prune That
Wisteria}\label{summer-prune-that-wisteria}}

Rampant vines like wisteria --- which got their precise, hard pruning at
Untermyer back to three to five nodes for each strong shoot in late
winter --- need touch-ups once or twice in season.

``It's so vigorous it outgrows its space, and just looks bad,'' Mr.
Tilghman said. ``We don't cut back as far as the detailed March pruning,
but we keep it from going wild.''

Lusty climbers like Dutchman's pipe (Aristolochia) and trumpet vine
(Campsis) also benefit from pruning. And with the trumpet vine, he said,
``look for any growth coming up from underground runners now, too, and
cut it out.''

Other summer-pruning targets include fruit trees. At Untermyer, Mr.
Tilghman identifies and removes any bad branching in the ornamental
cherries and other trees, including suckers at the base, vertical shoots
jutting up off branches and inward-facing or crossing ones.

``Sometimes it's easier to prune for shape and scale when you see the
plant in leaf,'' he said. ``Some structural cuts may sacrifice a little
spring show, but it's healthier for the tree than getting more
overgrown.''

He added: ``Any textbook that gives you the exact right time to prune
each plant is leaving out something: Realistically, sometimes you just
have to do it when you have time.''

Image

When the view's the thing, keep plantings simple. The Vista at Untermyer
is embellished with ancient Roman columns installed by its long-ago
owner, Samuel Untermyer, and beyond that, by the river and Palisades. To
frame the landscape, but avoid competing with it, the gardeners created
an allée of 99 evergreen Cryptomeria and twin swaths of golden Japanese
forest grass (Hakonechloa macra Aureola).Credit...Jessica Norman,
Untermyer Gardens Conservancy

\hypertarget{sometimes-simpler-is-better}{%
\subsection{Sometimes Simpler Is
Better}\label{sometimes-simpler-is-better}}

For all Untermyer's artful plant combinations, sometimes simplicity is
preferable. Are there places in your garden that need quieting?

The Vista, one the garden's most important Hudson views, is embellished
by ancient Roman columns installed by the long-ago property owner,
Samuel Untermyer, and beyond that, by the river and rugged Palisades.

The team created an allée of 99 Cryptomeria and, beneath them, twin
stretches of golden Japanese forest grass (Hakonechloa macra Aureola).
Both plants have year-round appeal and neither distracts from the real
show. Instead, they frame it.

``On any main axis or transition moment within a smaller garden, or
anywhere you have a bigger view,'' Mr. Tilghman said, ``an overly fussy
planting can actually compete and add visual static. Let the big moment
nearby show.''

\hypertarget{if-theres-a-break-in-the-heat-prep-future-beds}{%
\subsection{If There's a Break in the Heat, Prep Future
Beds}\label{if-theres-a-break-in-the-heat-prep-future-beds}}

That weedy spot out back you've been meaning to start over on, or the
new bed you've been imagining? If the heat relents even briefly, push
through and prep them.

The Untermyer crew did that and more in August 2018, in the former
estate gatehouse that is now the Ruin Garden. Because the estate and its
gardens had been abandoned and in decline for decades before being
rescued in 2011, a major cleanup was required.

``Hopefully, your task will not be quite as daunting,'' Mr. Tilghman
said. ``We had soil full of trash, broken bottles --- so we had to
excavate three-plus feet and build it back up.''

What they didn't remove: the graffiti on the walls. It remains as a
testament to Untermyer's incredible history.

\begin{center}\rule{0.5\linewidth}{\linethickness}\end{center}

For weekly email updates on residential real estate news,
\href{http://www.nytimes3xbfgragh.onion/newsletters/realestate/}{sign up
here}. Follow us on Twitter:
\href{https://twitter.com/nytrealestate}{@nytrealestate}.

Advertisement

\protect\hyperlink{after-bottom}{Continue reading the main story}

\hypertarget{site-index}{%
\subsection{Site Index}\label{site-index}}

\hypertarget{site-information-navigation}{%
\subsection{Site Information
Navigation}\label{site-information-navigation}}

\begin{itemize}
\tightlist
\item
  \href{https://help.nytimes3xbfgragh.onion/hc/en-us/articles/115014792127-Copyright-notice}{©~2020~The
  New York Times Company}
\end{itemize}

\begin{itemize}
\tightlist
\item
  \href{https://www.nytco.com/}{NYTCo}
\item
  \href{https://help.nytimes3xbfgragh.onion/hc/en-us/articles/115015385887-Contact-Us}{Contact
  Us}
\item
  \href{https://www.nytco.com/careers/}{Work with us}
\item
  \href{https://nytmediakit.com/}{Advertise}
\item
  \href{http://www.tbrandstudio.com/}{T Brand Studio}
\item
  \href{https://www.nytimes3xbfgragh.onion/privacy/cookie-policy\#how-do-i-manage-trackers}{Your
  Ad Choices}
\item
  \href{https://www.nytimes3xbfgragh.onion/privacy}{Privacy}
\item
  \href{https://help.nytimes3xbfgragh.onion/hc/en-us/articles/115014893428-Terms-of-service}{Terms
  of Service}
\item
  \href{https://help.nytimes3xbfgragh.onion/hc/en-us/articles/115014893968-Terms-of-sale}{Terms
  of Sale}
\item
  \href{https://spiderbites.nytimes3xbfgragh.onion}{Site Map}
\item
  \href{https://help.nytimes3xbfgragh.onion/hc/en-us}{Help}
\item
  \href{https://www.nytimes3xbfgragh.onion/subscription?campaignId=37WXW}{Subscriptions}
\end{itemize}
