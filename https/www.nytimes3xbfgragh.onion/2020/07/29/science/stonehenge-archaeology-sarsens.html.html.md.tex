Sections

SEARCH

\protect\hyperlink{site-content}{Skip to
content}\protect\hyperlink{site-index}{Skip to site index}

\href{https://www.nytimes3xbfgragh.onion/section/science}{Science}

\href{https://myaccount.nytimes3xbfgragh.onion/auth/login?response_type=cookie\&client_id=vi}{}

\href{https://www.nytimes3xbfgragh.onion/section/todayspaper}{Today's
Paper}

\href{/section/science}{Science}\textbar{}Whence Came Stonehenge's
Stones? Now We Know

\url{https://nyti.ms/3jUEtXw}

\begin{itemize}
\item
\item
\item
\item
\item
\item
\end{itemize}

Advertisement

\protect\hyperlink{after-top}{Continue reading the main story}

Supported by

\protect\hyperlink{after-sponsor}{Continue reading the main story}

\hypertarget{whence-came-stonehenges-stones-now-we-know}{%
\section{Whence Came Stonehenge's Stones? Now We
Know}\label{whence-came-stonehenges-stones-now-we-know}}

Last year archaeologists pinpointed the origin of many of the ancient
monument's massive stones. A new study identifies the source of the
rest.

\includegraphics{https://static01.graylady3jvrrxbe.onion/images/2020/07/29/science/29SCI-STONEHENGE1/29SCI-STONEHENGE1-articleLarge.jpg?quality=75\&auto=webp\&disable=upscale}

By Franz Lidz

\begin{itemize}
\item
  July 29, 2020
\item
  \begin{itemize}
  \item
  \item
  \item
  \item
  \item
  \item
  \end{itemize}
\end{itemize}

Back in the 30s --- the 1130s --- the Welsh cleric Geoffrey of Monmouth
created the impression that Stonehenge was built as a memorial to a
bunch of British nobles slain by the Saxons. In his
``\href{https://en.wikipedia.org/wiki/Historia_Regum_Britanniae}{Historia
Regum Britanniae},'' Geoffrey tells us that Merlin, the wizard of
Arthurian legend, was enlisted to move a ring of giant mystical stones
from Mount Killaraus in Ireland to what is commonly believed to be
Salisbury Plain, a chalk plateau in southern England, where Stonehenge
is located.

Back in the 50s --- the 1950s --- a chunk of rock went missing from the
magical tumble of megaliths that now compose Stonehenge. The chunk, a
three-and-a-half foot cylindrical core, had been drilled out of one of
the site's massive sarsen stones during repairs and taken home by an
employee of the diamond-cutting firm that carried out the work.

The core,
\href{https://www.nytimes3xbfgragh.onion/2019/05/08/world/europe/uk-stonehenge-missing-piece.html}{recently
repatriated} after 60 years, turned out to be pivotal to an
\href{https://advances.sciencemag.org/content/6/31/eabc0133}{academic
paper} published on Wednesday in the journal Science Advances. The study
pinpointed the source of the sarsens, a mystery that has long bedeviled
geologists and archaeologists.

Although the project did not identify the specific spot where the stones
came from, Mike Pitts, editor of the magazine British Archaeology,
believes that the discovery makes the search for sarsen quarries a
realistic option. ``If we can find them, we could learn about how they
were dressed and moved, and importantly we might be able to date that
activity,'' he said. ``Dating matters, because then we can say what else
was present in the landscape at the same time, what was old or gone and
what was still to come --- other sites are better dated --- and of
course who actually built the thing.''

\includegraphics{https://static01.graylady3jvrrxbe.onion/images/2020/07/29/science/29SCI-STONEHENGE2/merlin_154554093_fb698f44-04d0-45d1-a05a-284be387cfc0-articleLarge.jpg?quality=75\&auto=webp\&disable=upscale}

Two kinds of stones make up the roughly 5,000-year-old monument known as
Stonehenge. A small inner horseshoe consists of 2- to 4-ton blocks of
varied geology, called bluestone after the bluish-gray hue they have
when wet or freshly broken. The sarsens, sandstone slabs that weigh 20
tons on average, form Stonehenge's enormous central horseshoe, the
uprights and lintels of the ragged outer circle, as well as the outlying
Heel Stone, Slaughter Stone and Station Stones.

Geologists determined nearly a century ago that the bluestones were
dragged, carried or rolled to Stonehenge from somewhere in the Preseli
Hills in western Wales, some 180 miles away. Last year a team of
archaeologists led by Michael Parker Pearson of University College
London revealed evidence of the exact location of two of the quarries.

As for the sarsens, conventional wisdom holds that they derived from
deposits on the highest points of the Marlborough Downs, 18 miles north
of Stonehenge.

David Nash, a geomorphologist at the University of Brighton and lead
author on the new sarsen study, said the idea that the slabs hailed from
the Downs dates to the writings of William Lambarde, a 16th-century
antiquarian.

``Lambarde came to that conclusion based on little more than the
appearance of the stones on the Downs and their similarity to those at
Stonehenge,'' Dr. Nash said. ``This idea has stuck around for more than
400 years but has never been tested.''

Dr. Nash has traced the source of almost all the sarsens to West Woods,
on the southern edge of the Downs and several miles closer to
Stonehenge. His team analyzed the geochemical fingerprint of the 52
sarsens that remain in situ at the ancient site.

The breakthrough came last summer when the long-lost core from Stone 58
was returned to English Heritage, the charity that manages Stonehenge.
The sarsen cylinder offered Dr. Nash the unique opportunity to analyze a
sample unaffected by surface weathering, which can slightly alter the
chemical composition. Drilling through the ancient stones is now
discouraged.

``There are literally thousands of pieces of sarsen sitting in museums
across Britain,'' he said. ``However, to my knowledge, the core from 58
is the only piece where we can identify precisely which stone it came
from.''

Image

David Nash, a geomorphologist at University of Brighton, examining a
core from Stone 58 that was stolen in the 1950s and was recently
repatriated.Credit...Sam Frost/English Heritage

To determine its chemical makeup, researchers used a variety of
noninvasive spectrometry techniques. Once the geochemical signature was
established, they sampled sarsens from 20 locations **** across southern
England, including six on the Downs. A data set comparison resulted in a
single match, West Woods.

Only two of the sarsens, Stones 26 and 160, appear to have come from
elsewhere in the region. ``The biggest surprise for me was finding out
that the chemistry of the remaining sarsens was so consistent,'' said
Dr. Nash. ``I expected a little more variability.''

The only other authority thought to have linked West Woods and
Stonehenge is John Aubrey, biographer and philosopher who surveyed the
monument in the 17th century and was the first to record a ring of 56
chalk pits, now called Aubrey Holes. ``Aubrey reckoned that he'd found
the source of Stonehenge's sarsens, a large quarry pit just 14 miles
north of Stonehenge,'' said Dr. Parker Pearson. ``Given that West Woods
is around 15 miles away, it's very possible that he got it right the
first time! And it's taken us 340 years to find out again.''

For his part, Dr. Nash writes it off to archaeologists ``not seeing the
wood for the trees.''

Or perhaps leaving stones unturned.

\textbf{\emph{{[}}\href{http://on.fb.me/1paTQ1h}{\emph{Like the Science
Times page on Facebook.}}} ****** \emph{\textbar{} Sign up for the}
\textbf{\href{http://nyti.ms/1MbHaRU}{\emph{Science Times
newsletter.}}\emph{{]}}}

Advertisement

\protect\hyperlink{after-bottom}{Continue reading the main story}

\hypertarget{site-index}{%
\subsection{Site Index}\label{site-index}}

\hypertarget{site-information-navigation}{%
\subsection{Site Information
Navigation}\label{site-information-navigation}}

\begin{itemize}
\tightlist
\item
  \href{https://help.nytimes3xbfgragh.onion/hc/en-us/articles/115014792127-Copyright-notice}{©~2020~The
  New York Times Company}
\end{itemize}

\begin{itemize}
\tightlist
\item
  \href{https://www.nytco.com/}{NYTCo}
\item
  \href{https://help.nytimes3xbfgragh.onion/hc/en-us/articles/115015385887-Contact-Us}{Contact
  Us}
\item
  \href{https://www.nytco.com/careers/}{Work with us}
\item
  \href{https://nytmediakit.com/}{Advertise}
\item
  \href{http://www.tbrandstudio.com/}{T Brand Studio}
\item
  \href{https://www.nytimes3xbfgragh.onion/privacy/cookie-policy\#how-do-i-manage-trackers}{Your
  Ad Choices}
\item
  \href{https://www.nytimes3xbfgragh.onion/privacy}{Privacy}
\item
  \href{https://help.nytimes3xbfgragh.onion/hc/en-us/articles/115014893428-Terms-of-service}{Terms
  of Service}
\item
  \href{https://help.nytimes3xbfgragh.onion/hc/en-us/articles/115014893968-Terms-of-sale}{Terms
  of Sale}
\item
  \href{https://spiderbites.nytimes3xbfgragh.onion}{Site Map}
\item
  \href{https://help.nytimes3xbfgragh.onion/hc/en-us}{Help}
\item
  \href{https://www.nytimes3xbfgragh.onion/subscription?campaignId=37WXW}{Subscriptions}
\end{itemize}
