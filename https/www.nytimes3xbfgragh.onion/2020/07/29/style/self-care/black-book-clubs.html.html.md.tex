\href{/section/style/self-care/}{Self-Care}\textbar{}The Black Book Club
Takes It to the Next Level

\url{https://nyti.ms/2DjST2B}

\begin{itemize}
\item
\item
\item
\item
\item
\item
\end{itemize}

\href{https://www.nytimes3xbfgragh.onion/spotlight/at-home?action=click\&pgtype=Article\&state=default\&region=TOP_BANNER\&context=at_home_menu}{At
Home}

\begin{itemize}
\tightlist
\item
  \href{https://www.nytimes3xbfgragh.onion/2020/08/03/well/family/the-benefits-of-talking-to-strangers.html?action=click\&pgtype=Article\&state=default\&region=TOP_BANNER\&context=at_home_menu}{Talk:
  To Strangers}
\item
  \href{https://www.nytimes3xbfgragh.onion/2020/08/01/at-home/coronavirus-make-pizza-on-a-grill.html?action=click\&pgtype=Article\&state=default\&region=TOP_BANNER\&context=at_home_menu}{Make:
  Grilled Pizza}
\item
  \href{https://www.nytimes3xbfgragh.onion/2020/07/31/arts/television/goldbergs-abc-stream.html?action=click\&pgtype=Article\&state=default\&region=TOP_BANNER\&context=at_home_menu}{Watch:
  'The Goldbergs'}
\item
  \href{https://www.nytimes3xbfgragh.onion/interactive/2020/at-home/even-more-reporters-editors-diaries-lists-recommendations.html?action=click\&pgtype=Article\&state=default\&region=TOP_BANNER\&context=at_home_menu}{Explore:
  Reporters' Google Docs}
\end{itemize}

\includegraphics{https://static01.graylady3jvrrxbe.onion/images/2020/08/02/fashion/29BLACK-BOOKCLUBS-promo/29BLACK-BOOKCLUBS-promo-articleLarge-v2.jpg?quality=75\&auto=webp\&disable=upscale}

Sections

\protect\hyperlink{site-content}{Skip to
content}\protect\hyperlink{site-index}{Skip to site index}

\hypertarget{the-black-book-club-takes-it-to-the-next-level}{%
\section{The Black Book Club Takes It to the Next
Level}\label{the-black-book-club-takes-it-to-the-next-level}}

Noname and other Black thought leaders have taken what Oprah built and
made something new.

Noname, the rapper, started a book club in 2019.Credit...Erik Carter for
The New York Times

Supported by

\protect\hyperlink{after-sponsor}{Continue reading the main story}

By Iman Stevenson

\begin{itemize}
\item
  July 29, 2020
\item
  \begin{itemize}
  \item
  \item
  \item
  \item
  \item
  \item
  \end{itemize}
\end{itemize}

``I want people to think radically,'' said Noname, the 28-year-old
rapper, in a phone interview this month from her home in Los Angeles.
She is outspoken, especially on Twitter, about dismantling patriarchy,
white supremacy and capitalism, but over the last year she has also been
opening people's minds through a more analog medium.

It started in July 2019, when she posted a photo of ``Jackson Rising:
The Struggle for Economic Democracy and Black Self‑Determination in
Jackson, Mississippi,'' a collection of essays about the movement to
develop cooperative economic practices in the capital of America's
poorest state. Later, another Twitter user replied with a photo of the
book and suggested that they become ``pen pals and swap notes.''

Thus began the
\href{https://www.instagram.com/nonamereads/?hl=en}{Noname Book Club}, a
reading group focused on texts by authors of color (tagline: ``reading
material for the homies''). Hers is one of many Black- and women-led
book clubs people are turning to in the midst of a virus that has
alienated people from their communities and a continuing global
conversation about anti-Black racism.

What is essential to each of these groups --- and why members find them
appealing --- has a lot to do with leaders creating a space free of the
white gaze.

\hypertarget{black-book-clubs-then-and-now}{%
\subsection{Black Book Clubs Then and
Now}\label{black-book-clubs-then-and-now}}

The Black book club has, over time, served as a space of critical study,
leisure and fellowship. In the 19th century, free Black Americans in the
North saw literary societies and the organized literary activities that
they sponsored ``as one way to arrest the attention of the public,
assert their racial and American identities, and give voice to their
belief in the promises of democracy,'' Elizabeth McHenry wrote in
``Forgotten Readers: Recovering the Lost History of African American
Literary Societies.''

Dr. McHenry also notes in her book that ``not every member of African
American literary societies wanted to be a writer or enjoyed an
unmediated relationship with texts.'' Some members weren't even
literate, so they relied on others to share information.

At the end of the 20th century, the Black-led book club became a
national phenomenon and a commercial success. Oprah's Book Club, founded
by Oprah Winfrey in 1996, introduced readers to Black authors including
Toni Morrison and Pearl Cleage in the 1990s and early aughts. Ms.
Winfrey helped bring Black literature to non-Black consumers and created
a blueprint for celebrities of all stripes to become literary
tastemakers.

Of course, many of Oprah's readers --- and
\href{https://www.oprahmag.com/entertainment/books/g23067476/oprah-book-club-list/}{the
authors she has recommended} to them --- have been white.

The protests following the killings of George Floyd, Tony McDade,
Breonna Taylor and countless others have led many people, most of them
white, to immerse themselves in books about race in America, like the
best-selling titles
``\href{https://bookshop.org/books/how-to-be-an-antiracist-9780593396803/9780593396803}{How
to Be an Antiracist},'' by Ibram X. Kendi, and
``\href{https://www.nytimes3xbfgragh.onion/2020/07/15/magazine/white-fragility-robin-diangelo.html}{White
Fragility,}'' by Robin DiAngelo.

At\href{https://www.goodbooksatl.com/}{Good Books Atlanta}, a pop-up and
online bookstore owned by Katie Mitchell and her mother, Katherine,
recent purchasing trends reflect non-Black customers' desire to better
understand race and whiteness.

``I've definitely seen a surge in demand for anti-racist reading,
nonfiction that really exposes systemic racism,'' Ms. Mitchell said.
Those books are usually purchased by newer customers.

\includegraphics{https://static01.graylady3jvrrxbe.onion/images/2020/08/02/fashion/29BLACK-BOOKCLUBS-2/29BLACK-BOOKCLUBS-2-articleLarge.jpg?quality=75\&auto=webp\&disable=upscale}

\hypertarget{the-radicalization-of-noname}{%
\subsection{The Radicalization of
Noname}\label{the-radicalization-of-noname}}

Since its founding in August 2019, Noname's book club has grown to
nearly 10,000 Patreon subscribers, who pay at least \$1 a month for
membership. Others follow her book recommendations on Twitter and
support the club by buying merchandise.

Though her mother,
\href{https://twitter.com/nonamebooks/status/1153723696990715906?lang=en}{Desiree
Sanders}, owned a bookstore in Chicago, Noname did not necessarily
inherit her bookish tendencies.

``I wasn't really interested in reading,'' she said of her grade-school
years. And even now, ``I'm pretty insecure about it. I'm just doing it
now because I think it's important, and I do love language and
literature.''

Her more recent interest in reading was born out of the formation of her
politics. ``I'm still doing the reading, learning,'' she said. She
supports abolition --- ``of the U.S. empire,'' she said, but especially
\href{https://slack-redir.net/link?url=https\%3A\%2F\%2Fwww.nytimes3xbfgragh.onion\%2F2020\%2F06\%2F12\%2Fopinion\%2Fsunday\%2Ffloyd-abolish-defund-police.html}{the
police}, an idea that has gained broader support in recent months.
Activists have pushed for local policing budgets to be redirected toward
social and mental health services.

She posts about many of these ideas on Twitter, where she is the
recipient of much adoration and vitriol.

``Anything that's going to be pro-liberation, I'll always tweet that
without hesitation,'' she said. ``I think when you start questioning
systems, it helps you to open up other parts of your humanity.''

Her work expands to other forms of social activism, including sending
literature to incarcerated people. The future of her book club includes
plans to start a grocery drive and provide cooked meals for the homeless
in Los Angeles, where she lives.

``The Free Reading Program,'' which she plans to start once she reaches
10,000 subscribers, will focus on one essay a month and will be
facilitated by organizers and educators, according to the book club's
Patreon page. The program is specifically focused on political
education, and will focus on themes like Marxism and feminism.

``We read books, but under the umbrella that I am continuously trying to
expand different initiatives through book club,'' she said.

\hypertarget{black-book-clubs-everywhere}{%
\subsection{Black Book Clubs
Everywhere}\label{black-book-clubs-everywhere}}

The vlogger known
as\href{https://www.youtube.com/user/jouelzy}{Jouelzy}, who founded the
book club \href{https://www.smartbrowngirl.com/}{Smart Brown Girl},
believes there is a privilege associated with selecting and interpreting
texts, so the space she has created is an attempt to remedy that.

``We have a cohort of Black women graduate-level researchers who produce
what we call syllabi that walk you through the readings we're doing to
make the books more accessible,'' she said. Those materials include
background on each author, a book overview, themes and motifs, reading
tips, discussion questions and suggestions of similar books to read
next.

Expanding the kinds of Black stories that are centered is what prompted
K Bailey Obazee, 30, to
start\href{https://www.instagram.com/prim.black/}{OKHA}, a book club in
London that she describes as ``hella Black and hella gay.'' (``Okha''
means ``tale'' in Edo, the language spoken in the state in Nigeria where
she is from.)

Ms. Obazee grew tired of seeing the same Black authors promoted over and
over again. ``We kind of try and make sure that you are reading books,
not just by the same people and not just by people who are well known,''
she said.

Beyond the texts that are discussed, physical space is also key,
particularly for Ms. Obazee's largely queer membership (though the
pandemic has put in-person meetings on hold).

``It's nice to create a safe space, somewhere that can primarily be a
sober space, particularly for those who don't drink, for those who don't
want to party but still want to engage and connect with their
community,'' Ms. Obazee said.

The club serves as an exhibition as well, where Black and queer artists'
work is on display.

``That's an opportunity for people who come to the book club to also
become accustomed to the other forms of storytelling,'' she said,
``because it's not just written. We tell stories in varying mediums.''

There are chapters of the Well-Read Black Girl book club, founded by
Glory Edim, 36, in independent bookstores across all 50 states, thanks
to a partnership with the American Booksellers Association. The clubs
are self-directed but follow the same book recommendations, which
include one monthly title for adults and one for Y.A. readers.

The book club also holds an annual
\href{https://www.wellreadblackgirl.org/}{festival}, now in its fourth
year. Past participants have included the novelist Jacqueline Woodson
and the poets Aja Monet and Staceyann Chin.

``Last year our theme was reading as resistance, and it being a kind of
radical act,'' Ms. Edim said. This year's event, which does not yet have
a date, will be virtual, but its ambition is just as huge: ``Trying to
find ways to just have a really positive impact and move towards the
Black liberation,'' she said.

Advertisement

\protect\hyperlink{after-bottom}{Continue reading the main story}

\hypertarget{site-index}{%
\subsection{Site Index}\label{site-index}}

\hypertarget{site-information-navigation}{%
\subsection{Site Information
Navigation}\label{site-information-navigation}}

\begin{itemize}
\tightlist
\item
  \href{https://help.nytimes3xbfgragh.onion/hc/en-us/articles/115014792127-Copyright-notice}{©~2020~The
  New York Times Company}
\end{itemize}

\begin{itemize}
\tightlist
\item
  \href{https://www.nytco.com/}{NYTCo}
\item
  \href{https://help.nytimes3xbfgragh.onion/hc/en-us/articles/115015385887-Contact-Us}{Contact
  Us}
\item
  \href{https://www.nytco.com/careers/}{Work with us}
\item
  \href{https://nytmediakit.com/}{Advertise}
\item
  \href{http://www.tbrandstudio.com/}{T Brand Studio}
\item
  \href{https://www.nytimes3xbfgragh.onion/privacy/cookie-policy\#how-do-i-manage-trackers}{Your
  Ad Choices}
\item
  \href{https://www.nytimes3xbfgragh.onion/privacy}{Privacy}
\item
  \href{https://help.nytimes3xbfgragh.onion/hc/en-us/articles/115014893428-Terms-of-service}{Terms
  of Service}
\item
  \href{https://help.nytimes3xbfgragh.onion/hc/en-us/articles/115014893968-Terms-of-sale}{Terms
  of Sale}
\item
  \href{https://spiderbites.nytimes3xbfgragh.onion}{Site Map}
\item
  \href{https://help.nytimes3xbfgragh.onion/hc/en-us}{Help}
\item
  \href{https://www.nytimes3xbfgragh.onion/subscription?campaignId=37WXW}{Subscriptions}
\end{itemize}
