Sections

SEARCH

\protect\hyperlink{site-content}{Skip to
content}\protect\hyperlink{site-index}{Skip to site index}

\href{https://www.nytimes3xbfgragh.onion/section/travel}{Travel}

\href{https://myaccount.nytimes3xbfgragh.onion/auth/login?response_type=cookie\&client_id=vi}{}

\href{https://www.nytimes3xbfgragh.onion/section/todayspaper}{Today's
Paper}

\href{/section/travel}{Travel}\textbar{}The New College Drop-Off

\url{https://nyti.ms/3f4jxKg}

\begin{itemize}
\item
\item
\item
\item
\item
\item
\end{itemize}

\href{https://www.nytimes3xbfgragh.onion/news-event/coronavirus?action=click\&pgtype=Article\&state=default\&region=TOP_BANNER\&context=storylines_menu}{The
Coronavirus Outbreak}

\begin{itemize}
\tightlist
\item
  live\href{https://www.nytimes3xbfgragh.onion/2020/08/04/world/coronavirus-cases.html?action=click\&pgtype=Article\&state=default\&region=TOP_BANNER\&context=storylines_menu}{Latest
  Updates}
\item
  \href{https://www.nytimes3xbfgragh.onion/interactive/2020/us/coronavirus-us-cases.html?action=click\&pgtype=Article\&state=default\&region=TOP_BANNER\&context=storylines_menu}{Maps
  and Cases}
\item
  \href{https://www.nytimes3xbfgragh.onion/interactive/2020/science/coronavirus-vaccine-tracker.html?action=click\&pgtype=Article\&state=default\&region=TOP_BANNER\&context=storylines_menu}{Vaccine
  Tracker}
\item
  \href{https://www.nytimes3xbfgragh.onion/2020/08/02/us/covid-college-reopening.html?action=click\&pgtype=Article\&state=default\&region=TOP_BANNER\&context=storylines_menu}{College
  Reopening}
\item
  \href{https://www.nytimes3xbfgragh.onion/live/2020/08/04/business/stock-market-today-coronavirus?action=click\&pgtype=Article\&state=default\&region=TOP_BANNER\&context=storylines_menu}{Economy}
\end{itemize}

Advertisement

\protect\hyperlink{after-top}{Continue reading the main story}

Supported by

\protect\hyperlink{after-sponsor}{Continue reading the main story}

\hypertarget{the-new-college-drop-off}{%
\section{The New College Drop-Off}\label{the-new-college-drop-off}}

A bittersweet family tradition has become an exercise in risk
assessment, logistics and trying to understand ever-changing rules.

\includegraphics{https://static01.graylady3jvrrxbe.onion/images/2020/07/28/travel/28college/merlin_175044948_c9c97e00-b9bd-4332-8ad5-f1d0520b78ab-articleLarge.jpg?quality=75\&auto=webp\&disable=upscale}

By Julie Weed

\begin{itemize}
\item
  Published July 29, 2020Updated July 31, 2020
\item
  \begin{itemize}
  \item
  \item
  \item
  \item
  \item
  \item
  \end{itemize}
\end{itemize}

Maureen Rayhill of Seattle sounds like a public health official as she
describes the current process for coronavirus testing, rattling off
research she's done on in-person testing centers versus mail-order
companies and how their turnaround times for results compare. But she's
not. She's a mother, just trying to get her oldest child to college.

The poignant annual tradition of college drop-off --- parents driving
the new, nervous college student to school, bringing along brothers and
sisters to see their sibling's new home, setting up the tiny dorm room
together, sharing one last meal with the entire family, then waving
goodbye as the almost-adult runs off with a big pack of possible new
best friends --- has become the latest family milestone rendered almost
unrecognizable by the coronavirus pandemic.

Ms. Rayhill, 49, has already canceled the family vacation in Maine that
she had dreamed of taking before bringing Corrigan to Colby College in
Waterville next month. Instead, the retired nurse and homemaker is
frantically caught up with how to get a virus test done within 72 hours
of departure to meet the Maine state requirements, when current test
results are taking up to five days to be returned.

``It's nothing like what we thought it would be,'' she says.

The drop-off has always been a momentous trip, fraught with strong
emotions felt by parents and children alike. Now pile on the additional
stress of Covid-19. Families need to navigate how to best get to campus
while minimizing their exposure to the virus, all while trying to adhere
to changing and often confusing school and state health, safety and
travel rules.

``When we dropped off my son three years ago, the whole family went and
it was this bittersweet fun event for us all,'' said Mark Lorence of
Needham, Mass., as he recalled the first trip to leave his oldest son,
Matthew, at New York University. This fall will likely see Mr. Lorence,
58, driving down to New York, with masks and food from home, and back
again in one day. ``Now we have Plan A, B and C, depending on what's
going on.''

Randy Dunbar, a father trying to coordinate the cross-country trip for
his daughter, Alex, from Colorado to North Carolina, echoed the
sentiment.

``It's supposed to be a time to contemplate this great goodbye,'' he
said. ``But I'm thinking, `Where am I allowed to park?'''

Image

Corrigan Rayhill in Seattle, packing for Colby College.Credit...Reeny
Rayhill

\hypertarget{logistics-at-the-state-college-and-personal-level}{%
\subsection{Logistics at the state, college and personal
level}\label{logistics-at-the-state-college-and-personal-level}}

Complications and confusion come way before those campus gates. Nearly
half of the country's states currently have
\href{https://www.nytimes3xbfgragh.onion/2020/07/10/travel/state-travel-restrictions.html}{statewide
travel restrictions}, with various degrees of self-quarantines orders
--- encouraged, strongly encouraged, mandatory --- not to mention
suggested or required testing. Some counties, metro areas or
municipalities have issued their own rules for travelers.

New York is one of many states with extremely strict travel policies,
requiring anyone traveling from Puerto Rico, Washington, D.C., and
\href{https://coronavirus.health.ny.gov/covid-19-travel-advisory}{34
other states} --- those with high infection rates --- to self-quarantine
for 14 days.

\hypertarget{latest-updates-global-coronavirus-outbreak}{%
\section{\texorpdfstring{\href{https://www.nytimes3xbfgragh.onion/2020/08/04/world/coronavirus-cases.html?action=click\&pgtype=Article\&state=default\&region=MAIN_CONTENT_1\&context=storylines_live_updates}{Latest
Updates: Global Coronavirus
Outbreak}}{Latest Updates: Global Coronavirus Outbreak}}\label{latest-updates-global-coronavirus-outbreak}}

Updated 2020-08-04T20:50:09.557Z

\begin{itemize}
\tightlist
\item
  \href{https://www.nytimes3xbfgragh.onion/2020/08/04/world/coronavirus-cases.html?action=click\&pgtype=Article\&state=default\&region=MAIN_CONTENT_1\&context=storylines_live_updates\#link-1228a480}{Novavax
  sees encouraging results from two studies of its experimental
  vaccine.}
\item
  \href{https://www.nytimes3xbfgragh.onion/2020/08/04/world/coronavirus-cases.html?action=click\&pgtype=Article\&state=default\&region=MAIN_CONTENT_1\&context=storylines_live_updates\#link-4825b93}{Public
  and private schools in Maryland and elsewhere are divided over
  in-person instruction.}
\item
  \href{https://www.nytimes3xbfgragh.onion/2020/08/04/world/coronavirus-cases.html?action=click\&pgtype=Article\&state=default\&region=MAIN_CONTENT_1\&context=storylines_live_updates\#link-50f7386d}{The
  United Nations calls on policymakers to `plan thoroughly for school
  reopenings.'}
\end{itemize}

\href{https://www.nytimes3xbfgragh.onion/2020/08/04/world/coronavirus-cases.html?action=click\&pgtype=Article\&state=default\&region=MAIN_CONTENT_1\&context=storylines_live_updates}{See
more updates}

More live coverage:
\href{https://www.nytimes3xbfgragh.onion/live/2020/08/04/business/stock-market-today-coronavirus?action=click\&pgtype=Article\&state=default\&region=MAIN_CONTENT_1\&context=storylines_live_updates}{Markets}

Jennifer Overholt of Menlo Park, Calif., 56, said she paged through
screen after screen of quarantine-related comments and questions on a
Facebook page for parents of students attending Rensselaer Polytechnic
Institute in Troy, N.Y., where her son, Cole Mediratta, will go for his
sophomore year.

``There's worry and concern,'' said Ms. Overholt, a nonprofit executive.
First-year parents, who already had questions, like whether rice cookers
are allowed in dorm rooms, are now wondering where their child will be
quarantined if they test positive for the virus.

Two questions about travel bubbled up again and again. Does a 14-day
quarantine mean that if you arrive in New York, you have to stay for 14
days? (No,
\href{https://coronavirus.health.ny.gov/system/files/documents/2020/06/interimguidance_traveladvisory.pdf}{you
can leave anytime, but must quarantine the whole time you are in the
state}.) Do you have to quarantine if your home state is not on the
restricted list, but you drive through a hot zone state on the way to
New York?
(\href{https://coronavirus.health.ny.gov/system/files/documents/2020/06/interimguidance_traveladvisory.pdf}{Not}
\href{https://coronavirus.health.ny.gov/system/files/documents/2020/06/interimguidance_traveladvisory.pdf}{as
long as you spent fewer than 24 hours in hot zone states}.)

``It was kind of overwhelming, so I stopped looking,'' she said.

Parents are discovering that, regardless of what guidelines are posted,
policies are changing with new data and little notice almost daily.
Washington, D.C., home to around 20 colleges and universities, announced
last Friday that beginning this Monday, travelers coming into the city
from a high-risk area need to self-quarantine for 14 days. This group
includes students. The only silver lining:
\href{https://coronavirus.dc.gov/page/mayor\%E2\%80\%99s-order-2020-081-requirement-self-quarantine-after-non-essential-travel-during-covid-19}{Travel
that brings people into the area for less than 24 hours is allowed}, so
parents can drop-off students. No lingering.

The area's schools are now determining how to comply. The order is so
new and has such broad implications that colleges did not immediately
specify how they would respond. A spokeswoman at Georgetown University,
Ruth McBain, wrote in an email that officials were reviewing the new
order and would ensure that the schools reopening plan would comply with
the district's guidance.

But families across the country are waiting for details to be finalized.

Mr. Dunbar, a management consultant who will be taking his daughter from
their home in Boulder, Colo., to the University of North Carolina in
Chapel Hill, said he's already had to change their flights three times
as Alex's move-in dates shifted. His wife, Shawna, he said, spends about
an hour a day scanning the school's website for new information and
comparing plans with friends who are also sending offspring to college.

They are laser-focused on the latest updates from other schools, Mr.
Dunbar, 57, said, ``because it seems like when one rolls out a new
policy, others follow it.''

Other parents may have their offspring around the house a bit longer.

Mr. Lorence, the management consultant from Needham, Mass., expected to
drop off Matthew at New York University in August. But just last week
Matthew decided to stay home until October. His classes are all online,
and his musical performance was canceled. He thinks a later move-in
would perhaps help him snag an apartment with reduced rent.

Meanwhile, the Lorences' oldest daughter, Audrey, will be a first-year
student at University of Pittsburgh. For her drop-off in mid-August, the
entire family (Mark, his wife, Jean, and younger siblings Meredith and
Luke) wanted to go along for the ride. Now it will only be the parents,
and they struggled on whether to stay with family in town ---
grandparents or an uncle's family usually host them. This year, they
decided to stay in a hotel.

``Uncertainty is the word to describe it,'' Mr. Lorence said.

\hypertarget{to-fly-or-drive}{%
\subsection{To fly or drive}\label{to-fly-or-drive}}

Other families are deciding to fly or drive. Sure, air circulation on
planes is excellent, and the journey should be shorter, but it's hard to
know
\href{https://www.nytimes3xbfgragh.onion/2020/07/21/travel/crowded-flights-coronavirus.html}{how
full the flight will be} or if flight attendants will be enforcing
mask-wearing. There are also
\href{https://www.nytimes3xbfgragh.onion/2020/04/09/travel/coronavirus-regional-airports-cancellations.html}{fewer
nonstop flights between smaller cities}, fewer flights period, and
airlines have been frequently changing times and dates of flights to
optimize revenues. All these headaches from the air need to be weighed
against a drive that might require hours in the car, food stops and a
hotel stay.

The Feder-Johnson family of Madison, Wis., typically flies to New
Orleans each school year, to drop off their daughter, Nora, at Tulane
University. This year, mother, father and daughter are driving the 14
hours each way. At restroom and fuel stops, Nora's mother, Elizabeth
Feder, a public health researcher, will be looking to see if the people
around her are taking the necessary precautions.

``If we pull into a gas station and the people there aren't wearing
masks, we'll go on by to the next one,'' Dr. Feder, 62, said.

Louisiana has had more than
\href{https://eur01.safelinks.protection.outlook.com/?url=https\%3A\%2F\%2Fwww.nytimes3xbfgragh.onion\%2Finteractive\%2F2020\%2Fus\%2Flouisiana-coronavirus-cases.html\&data=02\%7C01\%7C\%7Cf98faf91db4c457cd27e08d82cdc651d\%7C84df9e7fe9f640afb435aaaaaaaaaaaa\%7C1\%7C0\%7C637308672381698693\&sdata=VRikgzNv7F\%2B0pyPzti\%2FBaZGxDJK0RPTBOb7Ya7c7ssM\%3D\&reserved=0}{100,000
Covid cases} (including a significant increase in July), so when they
arrive in New Orleans, instead of exploring the sights, hearing live
music and eating at favorite restaurants as they did in years past, the
family plans to eat takeout and make ``essential trips only'' to help
Nora set up her home. Household items will be ordered online and picked
up curbside.

Finding a hotel for their short stay has been a worry. ``The websites of
the national chains make the cleaning sound so thorough you could do
surgery in their rooms,'' Ms. Feder said, but it's hard to know what a
particular property is doing.

Even packing for the college student is different this year. With
concerns that the virus may flare on campus and cause the school to shut
down, Gina Anstey, 48, is sending her daughter Elise from Seattle to her
first year at Fordham University in New York City with just two large
suitcases, eschewing everything but the essentials.

\href{https://www.nytimes3xbfgragh.onion/news-event/coronavirus?action=click\&pgtype=Article\&state=default\&region=MAIN_CONTENT_3\&context=storylines_faq}{}

\hypertarget{the-coronavirus-outbreak-}{%
\subsubsection{The Coronavirus Outbreak
›}\label{the-coronavirus-outbreak-}}

\hypertarget{frequently-asked-questions}{%
\paragraph{Frequently Asked
Questions}\label{frequently-asked-questions}}

Updated August 4, 2020

\begin{itemize}
\item ~
  \hypertarget{i-have-antibodies-am-i-now-immune}{%
  \paragraph{I have antibodies. Am I now
  immune?}\label{i-have-antibodies-am-i-now-immune}}

  \begin{itemize}
  \tightlist
  \item
    As of right
    now,\href{https://www.nytimes3xbfgragh.onion/2020/07/22/health/covid-antibodies-herd-immunity.html?action=click\&pgtype=Article\&state=default\&region=MAIN_CONTENT_3\&context=storylines_faq}{that
    seems likely, for at least several months.} There have been
    frightening accounts of people suffering what seems to be a second
    bout of Covid-19. But experts say these patients may have a
    drawn-out course of infection, with the virus taking a slow toll
    weeks to months after initial exposure. People infected with the
    coronavirus typically
    \href{https://www.nature.com/articles/s41586-020-2456-9}{produce}
    immune molecules called antibodies, which are
    \href{https://www.nytimes3xbfgragh.onion/2020/05/07/health/coronavirus-antibody-prevalence.html?action=click\&pgtype=Article\&state=default\&region=MAIN_CONTENT_3\&context=storylines_faq}{protective
    proteins made in response to an
    infection}\href{https://www.nytimes3xbfgragh.onion/2020/05/07/health/coronavirus-antibody-prevalence.html?action=click\&pgtype=Article\&state=default\&region=MAIN_CONTENT_3\&context=storylines_faq}{.
    These antibodies may} last in the body
    \href{https://www.nature.com/articles/s41591-020-0965-6}{only two to
    three months}, which may seem worrisome, but that's perfectly normal
    after an acute infection subsides, said Dr. Michael Mina, an
    immunologist at Harvard University. It may be possible to get the
    coronavirus again, but it's highly unlikely that it would be
    possible in a short window of time from initial infection or make
    people sicker the second time.
  \end{itemize}
\item ~
  \hypertarget{im-a-small-business-owner-can-i-get-relief}{%
  \paragraph{I'm a small-business owner. Can I get
  relief?}\label{im-a-small-business-owner-can-i-get-relief}}

  \begin{itemize}
  \tightlist
  \item
    The
    \href{https://www.nytimes3xbfgragh.onion/article/small-business-loans-stimulus-grants-freelancers-coronavirus.html?action=click\&pgtype=Article\&state=default\&region=MAIN_CONTENT_3\&context=storylines_faq}{stimulus
    bills enacted in March} offer help for the millions of American
    small businesses. Those eligible for aid are businesses and
    nonprofit organizations with fewer than 500 workers, including sole
    proprietorships, independent contractors and freelancers. Some
    larger companies in some industries are also eligible. The help
    being offered, which is being managed by the Small Business
    Administration, includes the Paycheck Protection Program and the
    Economic Injury Disaster Loan program. But lots of folks have
    \href{https://www.nytimes3xbfgragh.onion/interactive/2020/05/07/business/small-business-loans-coronavirus.html?action=click\&pgtype=Article\&state=default\&region=MAIN_CONTENT_3\&context=storylines_faq}{not
    yet seen payouts.} Even those who have received help are confused:
    The rules are draconian, and some are stuck sitting on
    \href{https://www.nytimes3xbfgragh.onion/2020/05/02/business/economy/loans-coronavirus-small-business.html?action=click\&pgtype=Article\&state=default\&region=MAIN_CONTENT_3\&context=storylines_faq}{money
    they don't know how to use.} Many small-business owners are getting
    less than they expected or
    \href{https://www.nytimes3xbfgragh.onion/2020/06/10/business/Small-business-loans-ppp.html?action=click\&pgtype=Article\&state=default\&region=MAIN_CONTENT_3\&context=storylines_faq}{not
    hearing anything at all.}
  \end{itemize}
\item ~
  \hypertarget{what-are-my-rights-if-i-am-worried-about-going-back-to-work}{%
  \paragraph{What are my rights if I am worried about going back to
  work?}\label{what-are-my-rights-if-i-am-worried-about-going-back-to-work}}

  \begin{itemize}
  \tightlist
  \item
    Employers have to provide
    \href{https://www.osha.gov/SLTC/covid-19/standards.html}{a safe
    workplace} with policies that protect everyone equally.
    \href{https://www.nytimes3xbfgragh.onion/article/coronavirus-money-unemployment.html?action=click\&pgtype=Article\&state=default\&region=MAIN_CONTENT_3\&context=storylines_faq}{And
    if one of your co-workers tests positive for the coronavirus, the
    C.D.C.} has said that
    \href{https://www.cdc.gov/coronavirus/2019-ncov/community/guidance-business-response.html}{employers
    should tell their employees} -\/- without giving you the sick
    employee's name -\/- that they may have been exposed to the virus.
  \end{itemize}
\item ~
  \hypertarget{should-i-refinance-my-mortgage}{%
  \paragraph{Should I refinance my
  mortgage?}\label{should-i-refinance-my-mortgage}}

  \begin{itemize}
  \tightlist
  \item
    \href{https://www.nytimes3xbfgragh.onion/article/coronavirus-money-unemployment.html?action=click\&pgtype=Article\&state=default\&region=MAIN_CONTENT_3\&context=storylines_faq}{It
    could be a good idea,} because mortgage rates have
    \href{https://www.nytimes3xbfgragh.onion/2020/07/16/business/mortgage-rates-below-3-percent.html?action=click\&pgtype=Article\&state=default\&region=MAIN_CONTENT_3\&context=storylines_faq}{never
    been lower.} Refinancing requests have pushed mortgage applications
    to some of the highest levels since 2008, so be prepared to get in
    line. But defaults are also up, so if you're thinking about buying a
    home, be aware that some lenders have tightened their standards.
  \end{itemize}
\item ~
  \hypertarget{what-is-school-going-to-look-like-in-september}{%
  \paragraph{What is school going to look like in
  September?}\label{what-is-school-going-to-look-like-in-september}}

  \begin{itemize}
  \tightlist
  \item
    It is unlikely that many schools will return to a normal schedule
    this fall, requiring the grind of
    \href{https://www.nytimes3xbfgragh.onion/2020/06/05/us/coronavirus-education-lost-learning.html?action=click\&pgtype=Article\&state=default\&region=MAIN_CONTENT_3\&context=storylines_faq}{online
    learning},
    \href{https://www.nytimes3xbfgragh.onion/2020/05/29/us/coronavirus-child-care-centers.html?action=click\&pgtype=Article\&state=default\&region=MAIN_CONTENT_3\&context=storylines_faq}{makeshift
    child care} and
    \href{https://www.nytimes3xbfgragh.onion/2020/06/03/business/economy/coronavirus-working-women.html?action=click\&pgtype=Article\&state=default\&region=MAIN_CONTENT_3\&context=storylines_faq}{stunted
    workdays} to continue. California's two largest public school
    districts --- Los Angeles and San Diego --- said on July 13, that
    \href{https://www.nytimes3xbfgragh.onion/2020/07/13/us/lausd-san-diego-school-reopening.html?action=click\&pgtype=Article\&state=default\&region=MAIN_CONTENT_3\&context=storylines_faq}{instruction
    will be remote-only in the fall}, citing concerns that surging
    coronavirus infections in their areas pose too dire a risk for
    students and teachers. Together, the two districts enroll some
    825,000 students. They are the largest in the country so far to
    abandon plans for even a partial physical return to classrooms when
    they reopen in August. For other districts, the solution won't be an
    all-or-nothing approach.
    \href{https://bioethics.jhu.edu/research-and-outreach/projects/eschool-initiative/school-policy-tracker/}{Many
    systems}, including the nation's largest, New York City, are
    devising
    \href{https://www.nytimes3xbfgragh.onion/2020/06/26/us/coronavirus-schools-reopen-fall.html?action=click\&pgtype=Article\&state=default\&region=MAIN_CONTENT_3\&context=storylines_faq}{hybrid
    plans} that involve spending some days in classrooms and other days
    online. There's no national policy on this yet, so check with your
    municipal school system regularly to see what is happening in your
    community.
  \end{itemize}
\end{itemize}

``They might decide on a dime, `you gotta go' and she'll have to take it
all home again,'' said Ms. Anstey, a philanthropic consultant.

For some students, that heartbreaking scenario became real before they
even arrived. On July 20, Spelman College in Atlanta made the decision
to \href{https://www.spelman.edu/academics/path-forward-2020-21}{move
instruction online}. Just three weeks earlier the school had announced a
plan to welcome students back to campus, but in that short period, the
health crisis worsened. Other schools, from Occidental College in
California to Lafayette College in Pennsylvania, are following suit to
help ensure the safety of their students and faculty.

\hypertarget{get-in-get-settled-get-out}{%
\subsection{Get in, get settled, get
out}\label{get-in-get-settled-get-out}}

Once on campus, forget the once-common niceties like parent orientations
and let's-get-to-know-you coffees. No more chitchat with your child's
new roommate and their family, or meanderings around campus to check out
the new science lab.

The 19-page move-in guide issued by the University of North Carolina,
Chapel Hill this year asks families to ``leave as soon as is possible
once all student belongings have been brought to their rooms.'' Like
many other schools, it also has limits on how many helpers can enter the
dorm and asks families to
\href{https://housing.unc.edu/sites/housing.unc.edu/files/Move-In\%20Guide\%2020-21.pdf}{share
the elevator with their own move-in group} only.

Stores and restaurants, used to welcoming the influx of back-to-school
families, are under new rules as well. North Carolina restaurants are
under
\href{https://files.nc.gov/covid/documents/guidance/NCDHHS-Interim-Guidance-for-Restaurants-Phase-2.pdf}{orders
to separate tables of diners by six feet and to operate at no more than
50 percent capacity}. Mr. Dunbar and Alex have decided not to dine
anywhere indoors, and were relieved to learn that their favorite fried
chicken on biscuits from Time Out in Chapel Hill was available for
takeout.

The two are avoiding going inside stores as well by ordering dorm
essentials to be shipped to a friend's house in town where they will
pick them up. ``There's an overwhelming focus on logistics,'' Mr. Dunbar
said.

\hypertarget{get-tested}{%
\subsection{Get tested}\label{get-tested}}

Students who arrive on campus should expect a heightened focus on
health. The University of Idaho will
\href{https://www.uidaho.edu/-/media/UIdaho-Responsive/Files/health-clinic/covid-19/memos/memo-200716-green.pdf}{test
all students returning to the Moscow campus} in August. Colby College
will test students a number of times during the first few weeks of
school, and they
\href{https://covid19.colby.edu/fall-2020-planning/}{will not be allowed
to leave} the state until the end of the academic term.

Cornell University, in upstate New York, is asking all students to
quarantine at home for two weeks before departure and all will be
\href{https://covid.cornell.edu/students/move-in/}{tested when they
arrive in Ithaca}. In addition, students arriving from the many
\href{https://coronavirus.health.ny.gov/covid-19-travel-advisory}{states
under Governor Cuomo's executive travel order} will need to quarantine
in New York State, or another state that is not on the New York list
before stepping onto campus. Students scheduled to live in dorms had
been told they would be asked to quarantine in a Cornell-provided
location, but now they are expected to quarantine themselves before
arriving on campus, or take classes online until their state comes off
the list. \href{https://covid.cornell.edu/students/move-in/}{Parents are
not allowed into the dorms} during move-in or at any time during the
fall semester.

Some parents ultimately are deciding to send their offspring to school
by themselves, particularly if they are returning students. Ms.
Overholt's son, Cole, will travel alone from California to begin his
second year at Rensselaer. Ms. Overholt was planning to accompany him,
but then the virus hit.

``I don't see any reason I should get on a plane right now,'' she said,
assured that Cole is capable of moving himself in. ``I don't need to add
to the problem.''

Indeed, college students are much less likely than their parents to get
very sick from the coronavirus. You could say that part of growing up is
learning to be safe. At least that hasn't changed.

\emph{\textbf{Follow New York Times Travel}}
\emph{on}\href{https://www.instagram.com/nytimestravel/}{\emph{Instagram}}\emph{,}\href{https://twitter.com/nytimestravel}{\emph{Twitter}}
\emph{and}\href{https://www.facebookcorewwwi.onion/nytimestravel/}{\emph{Facebook}}\emph{.
And}\href{https://www.nytimes3xbfgragh.onion/newsletters/traveldispatch}{\emph{sign
up for our weekly Travel Dispatch newsletter}} \emph{to receive expert
tips on traveling smarter and inspiration for your next vacation.}

Advertisement

\protect\hyperlink{after-bottom}{Continue reading the main story}

\hypertarget{site-index}{%
\subsection{Site Index}\label{site-index}}

\hypertarget{site-information-navigation}{%
\subsection{Site Information
Navigation}\label{site-information-navigation}}

\begin{itemize}
\tightlist
\item
  \href{https://help.nytimes3xbfgragh.onion/hc/en-us/articles/115014792127-Copyright-notice}{©~2020~The
  New York Times Company}
\end{itemize}

\begin{itemize}
\tightlist
\item
  \href{https://www.nytco.com/}{NYTCo}
\item
  \href{https://help.nytimes3xbfgragh.onion/hc/en-us/articles/115015385887-Contact-Us}{Contact
  Us}
\item
  \href{https://www.nytco.com/careers/}{Work with us}
\item
  \href{https://nytmediakit.com/}{Advertise}
\item
  \href{http://www.tbrandstudio.com/}{T Brand Studio}
\item
  \href{https://www.nytimes3xbfgragh.onion/privacy/cookie-policy\#how-do-i-manage-trackers}{Your
  Ad Choices}
\item
  \href{https://www.nytimes3xbfgragh.onion/privacy}{Privacy}
\item
  \href{https://help.nytimes3xbfgragh.onion/hc/en-us/articles/115014893428-Terms-of-service}{Terms
  of Service}
\item
  \href{https://help.nytimes3xbfgragh.onion/hc/en-us/articles/115014893968-Terms-of-sale}{Terms
  of Sale}
\item
  \href{https://spiderbites.nytimes3xbfgragh.onion}{Site Map}
\item
  \href{https://help.nytimes3xbfgragh.onion/hc/en-us}{Help}
\item
  \href{https://www.nytimes3xbfgragh.onion/subscription?campaignId=37WXW}{Subscriptions}
\end{itemize}
