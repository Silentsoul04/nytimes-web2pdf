Sections

SEARCH

\protect\hyperlink{site-content}{Skip to
content}\protect\hyperlink{site-index}{Skip to site index}

\href{https://www.nytimes3xbfgragh.onion/section/world/americas}{Americas}

\href{https://myaccount.nytimes3xbfgragh.onion/auth/login?response_type=cookie\&client_id=vi}{}

\href{https://www.nytimes3xbfgragh.onion/section/todayspaper}{Today's
Paper}

\href{/section/world/americas}{Americas}\textbar{}U.N. Panel Takes Aim
at Heavy-Handed Police Tactics at Protests

\url{https://nyti.ms/3077ZBI}

\begin{itemize}
\item
\item
\item
\item
\item
\end{itemize}

\href{https://www.nytimes3xbfgragh.onion/news-event/george-floyd-protests-minneapolis-new-york-los-angeles?action=click\&pgtype=Article\&state=default\&region=TOP_BANNER\&context=storylines_menu}{Race
and America}

\begin{itemize}
\tightlist
\item
  \href{https://www.nytimes3xbfgragh.onion/2020/07/26/us/protests-portland-seattle-trump.html?action=click\&pgtype=Article\&state=default\&region=TOP_BANNER\&context=storylines_menu}{Protesters
  Return to Other Cities}
\item
  \href{https://www.nytimes3xbfgragh.onion/2020/07/24/us/portland-oregon-protests-white-race.html?action=click\&pgtype=Article\&state=default\&region=TOP_BANNER\&context=storylines_menu}{Portland
  at the Center}
\item
  \href{https://www.nytimes3xbfgragh.onion/2020/07/23/podcasts/the-daily/portland-protests.html?action=click\&pgtype=Article\&state=default\&region=TOP_BANNER\&context=storylines_menu}{Podcast:
  Showdown in Portland}
\item
  \href{https://www.nytimes3xbfgragh.onion/interactive/2020/07/16/us/black-lives-matter-protests-louisville-breonna-taylor.html?action=click\&pgtype=Article\&state=default\&region=TOP_BANNER\&context=storylines_menu}{45
  Days in Louisville}
\end{itemize}

Advertisement

\protect\hyperlink{after-top}{Continue reading the main story}

Supported by

\protect\hyperlink{after-sponsor}{Continue reading the main story}

\hypertarget{un-panel-takes-aim-at-heavy-handed-police-tactics-at-protests}{%
\section{U.N. Panel Takes Aim at Heavy-Handed Police Tactics at
Protests}\label{un-panel-takes-aim-at-heavy-handed-police-tactics-at-protests}}

The comments by the United Nations Human Rights Committee come as the
Trump administration faces growing criticism for deploying federal
agents to confront Black Lives Matter protesters.

\includegraphics{https://static01.graylady3jvrrxbe.onion/images/2020/07/29/world/29un-protest/merlin_174823956_d754649b-e487-4dcd-968a-ea82b04851fa-articleLarge.jpg?quality=75\&auto=webp\&disable=upscale}

By \href{https://www.nytimes3xbfgragh.onion/by/nick-cumming-bruce}{Nick
Cumming-Bruce}

\begin{itemize}
\item
  July 29, 2020
\item
  \begin{itemize}
  \item
  \item
  \item
  \item
  \item
  \end{itemize}
\end{itemize}

GENEVA --- Law-enforcement authorities are obligated to protect and
facilitate peaceful demonstrations, an influential United Nations human
rights panel said on Wednesday, challenging tactics the police have used
against anti-racism protests in American cities and around the world.

The international treaty governing civil and political rights requires
states to allow peaceful demonstrations, not to block or disrupt them
without a compelling reason, the United Nations Human Rights Committee
said. Authorities should also seek to de-escalate situations that might
lead to violence and to use only the minimum force necessary to disperse
crowds.

``A failure to respect and ensure the right of peaceful assembly is
typically a marker of repression,'' the panel said in its finding.

The committee of 18 international law experts monitors compliance with
the International Convention on Civil and Political Rights, a
cornerstone of human rights law signed by 173 countries, including the
United States.

The panel's comment, a product of more than two years of deliberations,
sets out international guidelines that speak to issues central to
America's deepening discord over the Trump administration's response to
the Black Lives Matter protests in Portland and elsewhere. The protests
in Portland and elsewhere swelled after the administration
\href{https://www.nytimes3xbfgragh.onion/2020/07/28/us/portland-protests-fact-check.html}{dispatched
federal law enforcement officials} to confront demonstrators over the
objections of local officials.

\includegraphics{https://static01.graylady3jvrrxbe.onion/images/2020/07/29/world/29un-protest2/merlin_173244918_0e97b7ce-d36a-45fe-9bc9-9ae0b75110d5-articleLarge.jpg?quality=75\&auto=webp\&disable=upscale}

The U.S. Attorney General, William P. Barr, defended the
administration's intervention during heated exchanges on Tuesday with
the House Judiciary Committee, asserting that ``violent rioters and
anarchists have hijacked legitimate protests to wreak senseless havoc
and destruction.''

On Wednesday, Gov. Kate Brown of Oregon
\href{https://www.nytimes3xbfgragh.onion/2020/07/29/us/protests-portland-federal-withdrawal.html}{said
the U.S. government had agreed to remove tactical teams} that had been
deployed to Portland, and that the state would provide extra security
for the federal courthouse in the city.

The panel's findings are not legally binding. But the timing of the
comments thrusts the U.N. into a divisive debate roiling the United
States during the heat of a presidential campaign.

The U.N. panel said that law enforcement authorities should facilitate
demonstrations and restrict the use of force to control or disperse
protesters. It said crowds should be dispersed ``only in exceptional
cases'' and after the authorities have exhausted nonviolent means to
de-escalate tensions.

``It is a fundamental human right for individuals to join a peaceful
assembly to express themselves, to celebrate, or to air grievances,''
said Christof Heyns, a law professor at South Africa's University of
Pretoria and the panel member who drafted the document. ``Together with
other rights related to political freedom, it constitutes the very
foundation of a democratic society.''

The United States, while endorsing some of the committee's guidance,
said in a commentary that the panel's observations were ``beyond its
competence and mandate and contrary to international law.''

The panel said military or paramilitary forces should not be used to
police gatherings except in extraordinary circumstances and that
concealing the identity of the police, a practice followed by federal
agents in unmarked vehicles in Portland, undercut accountability.

It also took aim at the use of tear gas, which has been heavily used by
law enforcement agencies in Portland and elsewhere. It said tear gas
should only be used as a last resort, and that rubber bullets should
only be fired in response to an imminent threat of death or serious
injury.

Image

The authorities in Kentucky used tear gas to disperse crowds in
Louisville during a Black Lives Matter Protest in June.Credit...Luke
Sharrett for The New York Times

The committee also asserted the ``particular importance'' of journalists
and human rights defenders in monitoring demonstrations, emphasizing
they should not be harassed or their equipment confiscated or damaged.
Even if violence erupted to a point that justified dispersing a crowd,
it did not justify dispersing journalists.

``All of these are lessons that the government of the United States
should be taking very much to heart,'' said Gerald Neuman, a professor
of international law at Harvard University who served on the committee.

Human rights experts said the panel's findings put a critical spotlight
on the actions of federal agents in Portland and other cities and
reinforce concerns that the heavy-handed policing there falls far short
of international standards.

``There are serious questions as to whether the administration's
response in Portland is sufficiently de-escalatory, respectful of
rights, discerning about differences between peaceful and violent
protesters, restrained in the use of force and weapons,'' said Harold
Hongju Koh, a Yale Law School professor who served as the State
Department's legal adviser in the Obama administration.

``Of special concern is whether the federal agents being deployed have
appropriate training,'' he added.

Advertisement

\protect\hyperlink{after-bottom}{Continue reading the main story}

\hypertarget{site-index}{%
\subsection{Site Index}\label{site-index}}

\hypertarget{site-information-navigation}{%
\subsection{Site Information
Navigation}\label{site-information-navigation}}

\begin{itemize}
\tightlist
\item
  \href{https://help.nytimes3xbfgragh.onion/hc/en-us/articles/115014792127-Copyright-notice}{©~2020~The
  New York Times Company}
\end{itemize}

\begin{itemize}
\tightlist
\item
  \href{https://www.nytco.com/}{NYTCo}
\item
  \href{https://help.nytimes3xbfgragh.onion/hc/en-us/articles/115015385887-Contact-Us}{Contact
  Us}
\item
  \href{https://www.nytco.com/careers/}{Work with us}
\item
  \href{https://nytmediakit.com/}{Advertise}
\item
  \href{http://www.tbrandstudio.com/}{T Brand Studio}
\item
  \href{https://www.nytimes3xbfgragh.onion/privacy/cookie-policy\#how-do-i-manage-trackers}{Your
  Ad Choices}
\item
  \href{https://www.nytimes3xbfgragh.onion/privacy}{Privacy}
\item
  \href{https://help.nytimes3xbfgragh.onion/hc/en-us/articles/115014893428-Terms-of-service}{Terms
  of Service}
\item
  \href{https://help.nytimes3xbfgragh.onion/hc/en-us/articles/115014893968-Terms-of-sale}{Terms
  of Sale}
\item
  \href{https://spiderbites.nytimes3xbfgragh.onion}{Site Map}
\item
  \href{https://help.nytimes3xbfgragh.onion/hc/en-us}{Help}
\item
  \href{https://www.nytimes3xbfgragh.onion/subscription?campaignId=37WXW}{Subscriptions}
\end{itemize}
