Sections

SEARCH

\protect\hyperlink{site-content}{Skip to
content}\protect\hyperlink{site-index}{Skip to site index}

\href{https://www.nytimes3xbfgragh.onion/section/world/asia}{Asia
Pacific}

\href{https://myaccount.nytimes3xbfgragh.onion/auth/login?response_type=cookie\&client_id=vi}{}

\href{https://www.nytimes3xbfgragh.onion/section/todayspaper}{Today's
Paper}

\href{/section/world/asia}{Asia Pacific}\textbar{}Hong Kong Is Keeping
Pro-Democracy Candidates Out of Its Election

\url{https://nyti.ms/30cPrjv}

\begin{itemize}
\item
\item
\item
\item
\item
\item
\end{itemize}

Advertisement

\protect\hyperlink{after-top}{Continue reading the main story}

Supported by

\protect\hyperlink{after-sponsor}{Continue reading the main story}

\hypertarget{hong-kong-is-keeping-pro-democracy-candidates-out-of-its-election}{%
\section{Hong Kong Is Keeping Pro-Democracy Candidates Out of Its
Election}\label{hong-kong-is-keeping-pro-democracy-candidates-out-of-its-election}}

Twelve candidates, including several prominent democracy advocates, were
barred from an upcoming legislative election, and four activists were
arrested over online posts.

\includegraphics{https://static01.graylady3jvrrxbe.onion/images/2020/07/30/world/30hongkong-arrests-1sub/merlin_175101765_3a087a89-e1a2-4ffc-ac68-03df3a0283b1-articleLarge.jpg?quality=75\&auto=webp\&disable=upscale}

By \href{https://www.nytimes3xbfgragh.onion/by/austin-ramzy}{Austin
Ramzy}, Elaine Yu and
\href{https://www.nytimes3xbfgragh.onion/by/tiffany-may}{Tiffany May}

\begin{itemize}
\item
  July 29, 2020
\item
  \begin{itemize}
  \item
  \item
  \item
  \item
  \item
  \item
  \end{itemize}
\end{itemize}

\href{https://cn.nytimes3xbfgragh.onion/china/20200730/hong-kong-arrests-security-law/}{阅读简体中文版}\href{https://cn.nytimes3xbfgragh.onion/china/20200730/hong-kong-arrests-security-law/zh-hant/}{閱讀繁體中文版}

HONG KONG --- Weeks after the Chinese government imposed a new national
security law on Hong Kong, raising fears of a broader crackdown on the
semiautonomous territory, the city's authorities have taken aggressive
steps against the pro-democracy opposition.

Officials on Thursday barred 12 candidates, including well-known
pro-democracy figures, from the September legislative election. The
disqualifications came a day after the police made what appeared to be
the first targeted arrests of four activists accused of posting
pro-independence messages online.

Local news outlets also reported that the government was considering
postponing the election by as much as a year because of the coronavirus
pandemic, though pro-democracy lawmakers argued it would be a naked
attempt to avoid a loss at the polls.

Opposition candidates said they had hoped to ride a wave of protests and
public discontent to electoral success on Sept. 6. But they had also
acknowledged the fear that the government would disqualify candidates on
the nebulous grounds that they would not uphold the Basic Law, the Hong
Kong Constitution.

The candidates who said they were barred included Joshua Wong, a
prominent activist, and Gwyneth Ho, a former journalist, both of them
front-runners in an unofficial democratic primary this month. The list
also includes four sitting lawmakers, including members of the moderate,
pro-democracy Civic Party.

Beijing's liaison office in Hong Kong said it supported the
disqualifications. The Hong Kong government said more could follow.

It also said in a
\href{https://www.info.gov.hk/gia/general/202007/30/P2020073000481.htm}{statement}
that grounds for disqualification included advocating Hong Kong's
independence or self-determination, soliciting intervention from foreign
governments, expressing an objection in principle to the national
security law Beijing imposed last month, or vowing to indiscriminately
vote against government proposals.

``The excuse they use is that I describe national security law as a
draconian law, which shows that I do not support this sweeping law,''
Mr. Wong, 23,
\href{https://www.facebookcorewwwi.onion/joshuawongchifung/photos/a.313299448762570/3193915534034266/}{wrote
on Facebook}. ``Clearly, Beijing shows a total disregard for the will of
the Hongkongers, tramples upon the city's last pillar of vanishing
autonomy and attempts to keep Hong Kong's legislature under its firm
grip.''

\includegraphics{https://static01.graylady3jvrrxbe.onion/images/2020/07/30/world/30hongkong-arrests-2/merlin_175101201_593b46d1-007b-47f1-9611-c029d40e3e3b-articleLarge.jpg?quality=75\&auto=webp\&disable=upscale}

Disqualification letters sent to the pro-democracy lawmakers Alvin
Yeung, Dennis Kwok and Kenneth Leung said their calls for the U.S. to
impose sanctions on those responsible for rights abuses in Hong Kong
would violate the national security law.

Mr. Yeung and Mr. Kwok said in their reply that their visit to New York
last August and a joint letter they sent in September to U.S. senators
took place months before the national security law took effect.

Though the security law could not be applied retroactively, election
officers said candidates' past actions and remarks reflected their true
intentions.

Mr. Yeung's disqualification letter also accused him of planning, along
with other members of the Civic Party, to ``indiscriminately vote down''
government proposals. Under the Basic Law, the chief executive must call
for a new legislative election if the government cannot pass a budget.
If it happens again under a new legislature, the chief executive must
step down.

Eric Cheung, a law lecturer at the University of Hong Kong, said he
believed lawmakers had the constitutional right to block government
proposals and compel the leader to step down. He said the mass
disqualifications showed that Hong Kong was growing increasingly similar
to mainland China.

``According to the legal system in the mainland, one cannot openly
oppose the regime, which is considered to be beyond the limits of free
speech,'' he said. ``Only allowing certain people they find acceptable
to run isn't a free election --- it's what totalitarian governments
do.''

The disqualifications came on the heels of the arrests of four people
accused of publishing social media posts that called for the city to
become independent from China. The sweep on Wednesday was an early
indication that the authorities would strictly enforce the new law and
crack down on speech that was now considered illegal.

Image

The activists, who ranged in age from 16 to 21, were arrested on
Wednesday and accused of promoting secession.Credit...Tyrone Siu/Reuters

The police said the activists, three men and one woman whose ages range
from 16 to 21, were arrested in the New Territories area of Hong Kong
for the ``publishing of content about secession, and inciting or
abetting others for the commission of secession.'' Officers seized
mobile phones, computers and documents during the roundup.

Li Kwai-wah, a senior superintendent of the Hong Kong police's new
national security department, said the arrests had been made after an
organization posted on social media about creating a new party to
promote Hong Kong independence. Its ``declarations'' referred to
establishing a ``Hong Kong country'' and using ``all means'' to achieve
its goals, Mr. Li said.

Mr. Li gave no further information about the organization or about the
four people arrested. He said the comments had been posted after the
security law took effect, but he would not say whether they had been
taken down or elaborate further on their content.

A political organization called Studentlocalism said its former
convener, Tony Chung, was among those arrested. When the national
security law took effect, the group said it had ended its operations in
Hong Kong but that some members would continue to work overseas.

The legislation came into force a month ago and gives the Chinese
government
\href{https://www.nytimes3xbfgragh.onion/2020/06/29/world/asia/china-hong-kong-security-law-rules.html?searchResultPosition=6}{broad
new powers over the semiautonomous territory}. The new law targets
subversion, secession, terrorism and collusion with foreign powers, and
many of its clauses indicate they were written to curb the protests. For
a city that had generally had strong protections for free speech, the
legislation represented a drastic shift.

The law had already been cited in the arrests of about a dozen people
\href{https://www.nytimes3xbfgragh.onion/2020/07/01/world/asia/hong-kong-security-law-china.html?searchResultPosition=8}{during
several demonstrations, including on July 1,} the anniversary of Hong
Kong's return to Chinese control. Human rights groups denounced those
street arrests, saying they showed that the authorities intended to use
the new powers to clamp down on peaceful activities. The Wednesday
arrests, rights groups said, sent another chilling message and raised
concerns about a crackdown on activism and political speech in Hong
Kong.

``The gross misuse of this draconian law makes clear that the aim is to
silence dissent, not protect national security,'' said Sophie
Richardson, the China director for Human Rights Watch.

Under the new law, Chinese security
agencies\href{https://www.nytimes3xbfgragh.onion/2020/07/02/world/asia/hong-kong-security-china.html?searchResultPosition=5}{can
now operate openly in Hong Kong}. But the Hong Kong police said the
arrests had been carried out by its own national security department.

Beijing imposed the security law after more than a year of large
protests in Hong Kong, many of which involved violent clashes with the
police. The protests were set off by a proposal to allow extraditions to
mainland China from Hong Kong, which is guaranteed its own legal system
under the terms of the former British colony's return to Chinese rule in
1997.

The demonstrations evolved to encompass a range of issues, including the
police's use of force and calls for expanding direct elections.

This month, more than 600,000 people participated in a primary for the
pro-democracy camp despite warnings from the government that it was
possibly illegal. That show of support followed
\href{https://www.nytimes3xbfgragh.onion/2019/11/24/world/asia/hong-kong-election-results.html}{a
landslide win for opposition candidates in district council elections in
November}.

Advertisement

\protect\hyperlink{after-bottom}{Continue reading the main story}

\hypertarget{site-index}{%
\subsection{Site Index}\label{site-index}}

\hypertarget{site-information-navigation}{%
\subsection{Site Information
Navigation}\label{site-information-navigation}}

\begin{itemize}
\tightlist
\item
  \href{https://help.nytimes3xbfgragh.onion/hc/en-us/articles/115014792127-Copyright-notice}{©~2020~The
  New York Times Company}
\end{itemize}

\begin{itemize}
\tightlist
\item
  \href{https://www.nytco.com/}{NYTCo}
\item
  \href{https://help.nytimes3xbfgragh.onion/hc/en-us/articles/115015385887-Contact-Us}{Contact
  Us}
\item
  \href{https://www.nytco.com/careers/}{Work with us}
\item
  \href{https://nytmediakit.com/}{Advertise}
\item
  \href{http://www.tbrandstudio.com/}{T Brand Studio}
\item
  \href{https://www.nytimes3xbfgragh.onion/privacy/cookie-policy\#how-do-i-manage-trackers}{Your
  Ad Choices}
\item
  \href{https://www.nytimes3xbfgragh.onion/privacy}{Privacy}
\item
  \href{https://help.nytimes3xbfgragh.onion/hc/en-us/articles/115014893428-Terms-of-service}{Terms
  of Service}
\item
  \href{https://help.nytimes3xbfgragh.onion/hc/en-us/articles/115014893968-Terms-of-sale}{Terms
  of Sale}
\item
  \href{https://spiderbites.nytimes3xbfgragh.onion}{Site Map}
\item
  \href{https://help.nytimes3xbfgragh.onion/hc/en-us}{Help}
\item
  \href{https://www.nytimes3xbfgragh.onion/subscription?campaignId=37WXW}{Subscriptions}
\end{itemize}
