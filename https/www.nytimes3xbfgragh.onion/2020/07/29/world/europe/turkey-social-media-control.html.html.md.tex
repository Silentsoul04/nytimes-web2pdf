Sections

SEARCH

\protect\hyperlink{site-content}{Skip to
content}\protect\hyperlink{site-index}{Skip to site index}

\href{https://www.nytimes3xbfgragh.onion/section/world/europe}{Europe}

\href{https://myaccount.nytimes3xbfgragh.onion/auth/login?response_type=cookie\&client_id=vi}{}

\href{https://www.nytimes3xbfgragh.onion/section/todayspaper}{Today's
Paper}

\href{/section/world/europe}{Europe}\textbar{}Turkey Passes Law
Extending Sweeping Powers Over Social Media

\url{https://nyti.ms/2BClU9k}

\begin{itemize}
\item
\item
\item
\item
\item
\item
\end{itemize}

Advertisement

\protect\hyperlink{after-top}{Continue reading the main story}

Supported by

\protect\hyperlink{after-sponsor}{Continue reading the main story}

\hypertarget{turkey-passes-law-extending-sweeping-powers-over-social-media}{%
\section{Turkey Passes Law Extending Sweeping Powers Over Social
Media}\label{turkey-passes-law-extending-sweeping-powers-over-social-media}}

The legislation extends control over platforms like Facebook, Twitter
and YouTube. Critics worry it will be used to stifle dissent and
criticism of the government.

\includegraphics{https://static01.graylady3jvrrxbe.onion/images/2020/07/29/world/29turkey-social-media01/29turkey-social-media01-articleLarge.jpg?quality=75\&auto=webp\&disable=upscale}

\href{https://www.nytimes3xbfgragh.onion/by/marc-santora}{\includegraphics{https://static01.graylady3jvrrxbe.onion/images/2018/02/20/multimedia/author-marc-santora/author-marc-santora-thumbLarge-v2.png}}

By \href{https://www.nytimes3xbfgragh.onion/by/marc-santora}{Marc
Santora}

\begin{itemize}
\item
  July 29, 2020
\item
  \begin{itemize}
  \item
  \item
  \item
  \item
  \item
  \item
  \end{itemize}
\end{itemize}

Turkish lawmakers passed legislation on Wednesday that would give the
government sweeping new powers to regulate social media content, raising
concerns that one of the few remaining spaces for free public debate in
the country could fall under greater government control.

The bill orders social media platforms with over one million daily users
--- such as Facebook,
\href{https://www.nytimes3xbfgragh.onion/2020/08/03/technology/ftc-twitter-privacy-violations.html}{Twitter}
and YouTube --- to open offices in Turkey and imposes stiff penalties if
the international companies refuse, including slowing the bandwidth of
the sites and making them largely inaccessible.

These offices would be responsible for responding to the demands of the
government and individuals to block or remove content hosted on their
platforms that is deemed offensive. They would have 48 hours to comply
and could be fined more than \$700,000 if they fail to respond.

The new law, which is expected to go into effect Oct. 1, also requires
the social media companies to store user data inside Turkey, raising
privacy concerns.

President Recep Tayyip Erdogan and his governing A.K.P. party, having
already taken control over most of the nation's traditional media
outlets, were behind the legislation, arguing that it was needed to
protect citizens from cybercrime and slander. Critics, however, say it
is part of a broader effort to control the flow of information in the
country and stifle dissent.

``The new law will enable the government to control social media, to get
content removed at will and to arbitrarily target individual users,''
Tom Porteous, deputy program director at Human Rights Watch, said in a
statement released hours before the overnight vote. ``Social media is a
lifeline for many people who use it to access news, so this law signals
a new dark era of online censorship.''

The attempts to gain control over social media in Turkey highlight the
paradox the platforms present in the digital age.

They have been used to spread disinformation in Western democracies,
including by hostile foreign powers bent on sowing chaos and influencing
elections. Efforts by social media platforms to police the content they
host have repeatedly fallen short, and governments have yet to devise
successful strategies to regulate content without unwittingly infringing
on free speech.

But they have also proved to be an increasingly essential tool for
debate and dialogue in repressive and autocratic nations, one of the
last arenas where opposition figures can connect with the public, and
citizens can attempt to hold politicians to account.

More than 90 percent of Turkey's conventional media is now controlled by
conglomerates close to the government. Hundreds of reporters have been
jailed or fled the country out of fear and Mr. Erdogan has made himself
so omnipresent on TV and radio that his voice can drown out all others.
The internet is now, for many, the last open public forum.

In passing the bill, supporters acknowledged that it would allow the
government to exert more authority in the digital realm.

``Today, while all the conventional media is acting within a certain
discipline and order, we will be regulating social media who is acting
entirely on its own,'' Cahit Ozkan, deputy head of the A.K.P., said on
Tuesday in televised remarks.

Mr. Erdogan has
\href{https://www.nytimes3xbfgragh.onion/interactive/2019/11/13/magazine/internet-turkey.html}{made
no secret of his disdain for social media and of his desire to exert
control over digital spaces}, much in the same way his government has
gained control over traditional media.

\includegraphics{https://static01.graylady3jvrrxbe.onion/images/2020/07/29/world/29turkey-social-media02sub/merlin_174798120_95ea4653-86ae-4284-a719-4b173761cfa5-articleLarge.jpg?quality=75\&auto=webp\&disable=upscale}

Six years ago, when he was embroiled in a corruption scandal widely
reported on social media, he vowed to restrict access to the sites.

``We will not allow this nation to be devoured by YouTube, Facebook or
whatever,'' Mr. Erdogan said at the time. ``We will take necessary steps
in the firmest way.''

In 2016, months before an attempted coup, Turkey moved even more
aggressively to censor content on the internet, as the number of people
prosecuted for insulting Mr. Erdogan in posts on social media
skyrocketed. At the same time, internet trolls loyal to the government
used social media platforms to attack critics and journalists.

In 2017, the country shocked many international observers when it banned
Wikipedia, a restriction that was lifted only this past January.

Last summer, Turkey gave its media regulatory authority sweeping control
over streaming services, including entertainment sites like Netflix.

Netflix, which is the most prominent streaming service in Turkey, has
\href{https://www.nytimes3xbfgragh.onion/2019/10/31/arts/television/netflix-censorship-turkey-india.html}{already
run into difficulties with Turkish censors} and has had to edit content
of its Turkish-produced programs, removing cigarettes and blurring
suggestive gestures.

Last week, Netflix canceled its fourth Turkish production a day before
filming was scheduled to begin, after the Turkish media regulatory
authority refused permission for filming. Ece Yorenc, the screenplay
writer, said the authority had refused to grant permission for the
series ``If Only'' because of a gay character in the series. Netflix
chose to cancel the entire series rather than change the content, she
said.

The issue of social media content took a personal turn more recently,
when Mr. Erdogan's newborn grandson --- the fourth child of his
daughter, Esra Albayrak, and Berat Albayrak, who is also the finance
minister --- was insulted on various social media platforms.

``Those spaces where lies, defamation, attack to personal rights,
character assassinations are running wild, should be put in order,'' Mr.
Erdogan told party officials in a televised speech on July 1.

``We want such social media spaces to be entirely removed, to be
controlled,'' he added.

But criticism of the law was swift, filling the very sites the
government was seeking to control.

Yaman Akdeniz, an expert in online rights, wrote on Twitter that ``a new
and dark period in Turkey is starting.''

``The aim is to silence,'' he wrote. ``It is the intention of the
government to clean its past from critical content including news
coverage of corruption allegations as well as all sorts of
irregularities.''

Even without the new bill, Turkey blocked access to more than 400,000
websites by the end of 2019, according to Mr. Akdeniz, whose
organization, the Freedom of Expression Association, compiles an annual
report on internet access in the country.

According to his analysis, last year more than 130,000 web addresses
were blocked; 40,000 posts on Twitter taken down; 10,000 YouTube videos
removed; and 6,200 Facebook posts scrubbed from the site.

The new bill will give the government even greater freedom to remove
content of independent news outlets which are the almost the only source
of unbiased local news coverage in Turkish. For many in Turkey, social
media is also the last forum where they can publicly express grievances
on topics raging from violence against women to political corruption.

Before the vote, the United Nations High Commissioner for Human Rights
warned that the law ``would give the state powerful tools for asserting
even more control over the media landscape.''

Quinn McKew, executive director of Article 19, an advocacy group for an
open internet and free expression, said that the law placed companies in
a difficult position.

``Social media companies face either becoming the long arm of the state
censorship or having access to their platforms slowed so much that they
are in effect blocked in Turkey,'' she said. ``Given the erosion of the
rule of law in Turkey under the current government, tech companies
cannot rely on the courts to challenge blocking decisions or requests
for user data.''

Twitter and Facebook declined to comment on the bill. According to
Twitter's transparency report for the first half of 2019, Turkey had the
highest number of content removal request, with more than 6,000.

Raymond Zhong, Adam Satariano and Carlotta Gall contributed reporting.

Advertisement

\protect\hyperlink{after-bottom}{Continue reading the main story}

\hypertarget{site-index}{%
\subsection{Site Index}\label{site-index}}

\hypertarget{site-information-navigation}{%
\subsection{Site Information
Navigation}\label{site-information-navigation}}

\begin{itemize}
\tightlist
\item
  \href{https://help.nytimes3xbfgragh.onion/hc/en-us/articles/115014792127-Copyright-notice}{©~2020~The
  New York Times Company}
\end{itemize}

\begin{itemize}
\tightlist
\item
  \href{https://www.nytco.com/}{NYTCo}
\item
  \href{https://help.nytimes3xbfgragh.onion/hc/en-us/articles/115015385887-Contact-Us}{Contact
  Us}
\item
  \href{https://www.nytco.com/careers/}{Work with us}
\item
  \href{https://nytmediakit.com/}{Advertise}
\item
  \href{http://www.tbrandstudio.com/}{T Brand Studio}
\item
  \href{https://www.nytimes3xbfgragh.onion/privacy/cookie-policy\#how-do-i-manage-trackers}{Your
  Ad Choices}
\item
  \href{https://www.nytimes3xbfgragh.onion/privacy}{Privacy}
\item
  \href{https://help.nytimes3xbfgragh.onion/hc/en-us/articles/115014893428-Terms-of-service}{Terms
  of Service}
\item
  \href{https://help.nytimes3xbfgragh.onion/hc/en-us/articles/115014893968-Terms-of-sale}{Terms
  of Sale}
\item
  \href{https://spiderbites.nytimes3xbfgragh.onion}{Site Map}
\item
  \href{https://help.nytimes3xbfgragh.onion/hc/en-us}{Help}
\item
  \href{https://www.nytimes3xbfgragh.onion/subscription?campaignId=37WXW}{Subscriptions}
\end{itemize}
