Sections

SEARCH

\protect\hyperlink{site-content}{Skip to
content}\protect\hyperlink{site-index}{Skip to site index}

\href{https://www.nytimes3xbfgragh.onion/section/world/europe}{Europe}

\href{https://myaccount.nytimes3xbfgragh.onion/auth/login?response_type=cookie\&client_id=vi}{}

\href{https://www.nytimes3xbfgragh.onion/section/todayspaper}{Today's
Paper}

\href{/section/world/europe}{Europe}\textbar{}U.S. to Cut 12,000 Forces
in Germany

\url{https://nyti.ms/3hPY0H8}

\begin{itemize}
\item
\item
\item
\item
\item
\item
\end{itemize}

Advertisement

\protect\hyperlink{after-top}{Continue reading the main story}

Supported by

\protect\hyperlink{after-sponsor}{Continue reading the main story}

\hypertarget{us-to-cut-12000-forces-in-germany}{%
\section{U.S. to Cut 12,000 Forces in
Germany}\label{us-to-cut-12000-forces-in-germany}}

Defense Department officials say the redeployments will enhance American
security and its ability to respond to threats. Allies and some in
Congress see it as punishment to Germany.

\includegraphics{https://static01.graylady3jvrrxbe.onion/images/2020/07/29/us/politics/29dc-military/merlin_170021838_f29a04b0-d7ed-41f6-ac3a-d3dc02bf35b0-articleLarge.jpg?quality=75\&auto=webp\&disable=upscale}

\href{https://www.nytimes3xbfgragh.onion/by/thomas-gibbons-neff}{\includegraphics{https://static01.graylady3jvrrxbe.onion/images/2018/07/12/multimedia/author-thomas-gibbons-neff/author-thomas-gibbons-neff-thumbLarge.png}}

By
\href{https://www.nytimes3xbfgragh.onion/by/thomas-gibbons-neff}{Thomas
Gibbons-Neff}

\begin{itemize}
\item
  July 29, 2020Updated 11:30 a.m. ET
\item
  \begin{itemize}
  \item
  \item
  \item
  \item
  \item
  \item
  \end{itemize}
\end{itemize}

The United States is cutting back its deployments in Germany by nearly
12,000 troops and shifting some of those forces around the continent,
including relocating some units to Belgium and Italy, Defense Secretary
Mark T. Esper announced Wednesday.

About 6,400 troops are set to return to the United States.

The move is certain to rankle European leaders and anger lawmakers on
both sides of the aisle who see the American troop presence on the
continent, especially in Germany, as a cornerstone of post-World War II
order.

``I am confident that the alliance will be all the better and stronger
for it,'' Mr. Esper told reporters. ``We can see some moves begin within
weeks.''

The Pentagon's decision to cut American troops in Germany from roughly
36,000 to about 24,000 is in keeping with President Trump's idea of
``America First'' and his deep-seated drive to bring home U.S. forces
from wars launched after the attacks of Sept. 11, 2001.

He also has vigorously demanded that European allies carry more of the
burden for their own protection. In particular, he has long criticized
Germany for being ``delinquent'' in meeting its commitment to spend 2
percent of its gross domestic product on defense. The plan announced by
Mr. Esper on Wednesday will affect
\href{https://www.nytimes3xbfgragh.onion/2020/06/06/world/europe/germany-troop-withdrawal-america.html}{American
deployments to Germany} more than any other NATO ally.

``This is so clearly a punitive move against Germany that its hard to
see any benefit from this,'' said Rachel Rizzo, the director of programs
at the Truman Center for National Policy and who focuses on European
security issues. ``It really puts future administrations in a bind, it
gives them no room to maneuver and will stick in Europeans' mind well
into the future.''

Mr. Trump addressed the issue when he spoke to reporters on Wednesday,
stressing that Germany had not paid its full share for defense under
NATO commitments.

``Germany is delinquent,'' Mr. Trump said.

\href{https://www.nytimes3xbfgragh.onion/2020/06/05/world/europe/trump-troops-europe-nato-germany.html}{The
outlines of the move}, reported earlier in June by The Wall Street
Journal, blindsided German officials and some American military
officials, who have long seen the U.S. presence in Germany as the
bedrock of the American commitment to NATO.

Mr. Esper said the change was a part of an ongoing review of U.S. troop
presence around the world that was ``accelerated'' by Mr. Trump's
announcement to cut forces in Germany. Mr. Esper sought to explain to
reporters that the move was not a punitive action prompted by the
president.

``I'm telling you that this is going to accomplish what the president
said with regard to getting us down to a lower number in Europe, and it
meets those other objectives I outlined with the regard to the strategic
piece,'' Mr. Esper said.

Repositioning the troops will cost ``several billion dollars,'' Mr.
Esper added. The withdrawal and shifting of troops will most likely
takes months, if not years.

Roughly 5,600 troops leaving Germany will move elsewhere in Europe,
including an F-16 fighter squadron to Italy and an armored unit that
will return to the United States and start a rotational deployment in
the Black Sea region. The military's European Command headquarters in
Stuttgart, Germany, will move to Mons, Belgium. The Africa Command
headquarters, also in Germany, will probably move, although Defense
Department officials did not provide a location.

The shift of forces will reduce the Americans' presence at some U.S.
bases in Germany but none are expected to close. The more remote town of
Spangdahlem, where the F-16 jet squadron is based, depends on the
American presence there and will probably suffer some economic fallout
from the withdrawal.

Earlier this month, the governors from the four German states that host
U.S. troops sent a letter to more than a dozen U.S. lawmakers pushing
them to urge Mr. Trump not to scale back the American troop presence in
Germany.

``For decades, Americans and Germans have worked together to build and
develop these unique and highly capable structures,'' the letter said.
``They provide the necessary foundation for a partnership-based
contribution to peace in Europe and the world, to which we all share a
common commitment.''

In 2012, the Obama administration withdrew two combat brigades from
Germany to a mostly muted response, though it was opposed by many
officials in Europe.

But that was two years before Russia invaded Crimea and fueled an
insurgency in eastern Ukraine with weapons and troops, prompting Europe
and American officials to call for a reinvigorated NATO and a harsher
stance against Russia.

American bases in Germany span the gamut of missions, from training
areas such as the one in Grafenwoehr, to the Landstuhl Regional Medical
Center, often one of the first stops for U.S. troops wounded in combat
in places such as Iraq or Afghanistan as they return stateside for
medical care.

Senator Mitt Romney, Republican of Utah and his party's former
presidential candidate, has called Mr. Trump's plan to cut troops in
Germany ``a gift to Russia.'' The Kremlin has long seen American troops
in Europe as a threat to Russian borders.

Mr. Esper did say that some American forces could increase rotational
deployments to the Black Sea region near Russia, as well as to Poland
and the Baltic States.

At the height of the Cold War, when military officials planned for
armored blitzes through the Fulda Gap, a patch of land in Germany long
seen as a flash point for a shooting war with the Soviet Union, the U.S.
military had nearly 300,000 troops stationed in Europe.

Melissa Eddy contributed reporting from Berlin.

Advertisement

\protect\hyperlink{after-bottom}{Continue reading the main story}

\hypertarget{site-index}{%
\subsection{Site Index}\label{site-index}}

\hypertarget{site-information-navigation}{%
\subsection{Site Information
Navigation}\label{site-information-navigation}}

\begin{itemize}
\tightlist
\item
  \href{https://help.nytimes3xbfgragh.onion/hc/en-us/articles/115014792127-Copyright-notice}{©~2020~The
  New York Times Company}
\end{itemize}

\begin{itemize}
\tightlist
\item
  \href{https://www.nytco.com/}{NYTCo}
\item
  \href{https://help.nytimes3xbfgragh.onion/hc/en-us/articles/115015385887-Contact-Us}{Contact
  Us}
\item
  \href{https://www.nytco.com/careers/}{Work with us}
\item
  \href{https://nytmediakit.com/}{Advertise}
\item
  \href{http://www.tbrandstudio.com/}{T Brand Studio}
\item
  \href{https://www.nytimes3xbfgragh.onion/privacy/cookie-policy\#how-do-i-manage-trackers}{Your
  Ad Choices}
\item
  \href{https://www.nytimes3xbfgragh.onion/privacy}{Privacy}
\item
  \href{https://help.nytimes3xbfgragh.onion/hc/en-us/articles/115014893428-Terms-of-service}{Terms
  of Service}
\item
  \href{https://help.nytimes3xbfgragh.onion/hc/en-us/articles/115014893968-Terms-of-sale}{Terms
  of Sale}
\item
  \href{https://spiderbites.nytimes3xbfgragh.onion}{Site Map}
\item
  \href{https://help.nytimes3xbfgragh.onion/hc/en-us}{Help}
\item
  \href{https://www.nytimes3xbfgragh.onion/subscription?campaignId=37WXW}{Subscriptions}
\end{itemize}
