Sections

SEARCH

\protect\hyperlink{site-content}{Skip to
content}\protect\hyperlink{site-index}{Skip to site index}

\href{https://www.nytimes3xbfgragh.onion/section/arts/music}{Music}

\href{https://myaccount.nytimes3xbfgragh.onion/auth/login?response_type=cookie\&client_id=vi}{}

\href{https://www.nytimes3xbfgragh.onion/section/todayspaper}{Today's
Paper}

\href{/section/arts/music}{Music}\textbar{}Beyoncé's `Black Is King' Is
No Secret, but Still Comes With Mystery

\url{https://nyti.ms/2DmaZB6}

\begin{itemize}
\item
\item
\item
\item
\item
\end{itemize}

Advertisement

\protect\hyperlink{after-top}{Continue reading the main story}

Supported by

\protect\hyperlink{after-sponsor}{Continue reading the main story}

\hypertarget{beyoncuxe9s-black-is-king-is-no-secret-but-still-comes-with-mystery}{%
\section{Beyoncé's `Black Is King' Is No Secret, but Still Comes With
Mystery}\label{beyoncuxe9s-black-is-king-is-no-secret-but-still-comes-with-mystery}}

Her latest project is her first with Disney+: a visual album connected
to the music she oversaw for the ``Lion King'' remake. And as usual,
she's captured fans' attention by saying little.

\includegraphics{https://static01.graylady3jvrrxbe.onion/images/2020/07/30/arts/30beyonce-walkup/merlin_173319969_107c0b60-2dae-4ba0-ac83-7f3b31ab262c-articleLarge.jpg?quality=75\&auto=webp\&disable=upscale}

\href{https://www.nytimes3xbfgragh.onion/by/ben-sisario}{\includegraphics{https://static01.graylady3jvrrxbe.onion/images/2018/02/20/multimedia/author-ben-sisario/author-ben-sisario-thumbLarge.jpg}}

By \href{https://www.nytimes3xbfgragh.onion/by/ben-sisario}{Ben Sisario}

\begin{itemize}
\item
  Published July 30, 2020Updated Aug. 1, 2020, 9:35 a.m. ET
\item
  \begin{itemize}
  \item
  \item
  \item
  \item
  \item
  \end{itemize}
\end{itemize}

The playbook is familiar, though the particulars are new: Beyoncé
unveils a new project. Details, though scant, are pored over for clues.
Social media immediately bubbles with anticipation and debate.

On Friday,
\href{https://www.nytimes3xbfgragh.onion/2020/07/31/arts/music/beyonce-black-is-king.html}{Beyoncé
will release ``Black Is King},'' a visual album connected to Disney's
remake last year of ``The Lion King,'' on the Disney+ streaming
platform. Announced a month ago, ``Black Is King'' is a typically
ambitious latter-day project for Beyoncé --- she wrote and directed it,
and is executive producer --- that adapts the ``Lion King'' story to a
wider narrative of African history and heritage. It also represents
Beyoncé's latest move as a self-directed business figure, aligning
herself with a major media partner, as she has done before with Tidal,
HBO, Apple and Netflix.

``Black Is King,'' which is based on songs that Beyoncé created for
``\href{https://www.nytimes3xbfgragh.onion/2019/07/24/arts/music/beyonce-the-lion-king-the-gift-review.html}{The
Lion King: The Gift},'' a companion album to last year's remake, carries
added weight since Beyoncé herself has made a case for its topical
significance.

``The events of 2020 have made the film's vision and message even more
relevant,'' she wrote in a rare
\href{https://www.instagram.com/tv/CCAMxfrHjAL/}{explanatory post} on
Instagram. ``I believe that when Black people tell our own stories, we
can shift the axis of the world and tell our REAL history of
generational wealth and richness of soul that are not told in our
history books.''

Beyoncé and Disney have offered few details about the project itself. It
was made with an international creative team, including many Africans,
and its cast has boldface names like Lupita Nyong'o, Pharrell Williams,
Naomi Campbell, Jay-Z and
\href{https://www.nytimes3xbfgragh.onion/2017/01/21/fashion/tina-knowles-lawson-beyonce-solange-matriarch.html}{Tina
Knowles-Lawson}, Beyoncé's mother. The list of directors who worked with
Beyoncé on the project includes Emmanuel Adjei, Blitz Bazawule, Pierre
Debusschere, Jenn Nkiru, Ibra Ake, Dikayl Rimmasch, Jake Nava and Kwasi
Fordjour.

Even basic points remain mysterious. Officially called a visual album,
it appears to be a series of music videos linked through a narrative
sequence, though it is not clear even how many songs or films are
included. Representatives for Beyoncé and Disney declined to comment.

But a lack of information has only stirred the pot, as online
commentators --- having seen just two brief trailers --- have debated
topics like whether Beyoncé is
\href{https://www.washingtonpost.com/world/africa/beyonce-black-is-king-africa-backlash/2020/07/08/cfaa2dd2-c079-11ea-864a-0dd31b9d6917_story.html}{exploiting
African stereotypes}, and whether the apparent presence of a
\href{https://twitter.com/lulu_4_lu_puffs/status/1287890438976929792/photo/1}{white
butler} at a Black women's tea party is a sign of racism.

In some ways, that reflects one of Beyoncé's great talents --- stoking
public conversation with her art, while explaining very little about it.

``She is allowing her art to speak for itself,'' said Treva Lindsey, an
associate professor of women's, gender and sexuality studies at Ohio
State University, who has
\href{https://wfpl.org/strange-fruit-unpacking-symbolism-beyonces-lemonade/}{commented
frequently} on Beyoncé's work. ``I always see Beyoncé as opening up
space for robust conversations. It often says more about us as consumers
and critics than it does about her.''

What is more clear, however, is Beyoncé's media strategy, which she has
been developing in plain sight over the last decade. After beginning her
career as a teenager in Destiny's Child --- and doing what is expected
of all rising stars, like giving interviews --- by the early 2010s she
had largely abandoned the standard pop-star script, and remade herself
as a self-contained cultural brand. She now
\href{https://www.nytimes3xbfgragh.onion/2015/08/20/fashion/beyonce-is-seen-but-not-heard.html}{almost
never speaks} to the news media.

Part of her approach has involved leapfrogging from one platform to
another to suit the needs of each project. In early 2013, HBO showed her
autobiographical film ``Life Is but a Dream''; later that year, she
melted the internet --- and
\href{https://www.nytimes3xbfgragh.onion/2016/04/24/arts/music/beyonce-hbo-lemonade.html}{upended
the music business} --- by releasing her album ``Beyoncé'' on Apple's
iTunes with no notice.

``Lemonade,'' her 2016 album, was first released on Tidal, the streaming
service taken over by Jay-Z, her husband, in which she is a partner, and
had a companion film shown on HBO, with segments directed by Mark
Romanek, Jonas Akerlund, Melina Matsoukas and others. Last year, Netflix
carried
\href{https://www.nytimes3xbfgragh.onion/2019/04/17/arts/music/beyonce-netflix-homecoming.html}{``Homecoming,''}
the film of her performance at Coachella from 2018.

In this trajectory, Disney+ is simply the next hot media platform with
something to offer Beyoncé, said Dan Runcie, who writes about the
business of streaming and hip-hop on his site
\href{https://trapital.co/2019/04/18/beyonces-streaming-strategy-explained/}{Trapital}.

``This is well within the wheelhouse of the Beyoncé empire,'' Runcie
said, ``given how much she's not locked herself into one particular
partner, but thought of herself as a broader enterprise and kept her
options open.''

With greater control, Beyoncé has changed her musical priorities. No
longer chasing pop hits, she has used her albums and multimedia projects
to explore challenging material, and made issues like gender and race
central topics of her art, with the Black experience --- and Black
womanhood, in particular --- becoming her overarching theme in recent
years.

This has, perhaps paradoxically, made Beyoncé even more famous and
influential, with her every appearance, utterance or Instagram post
scrutinized for hidden meanings. That fame can bring more attention to
her themes of Black lives and Black struggles --- like her Black
Panther-inspired dancers at her Super Bowl appearance in 2016, or images
invoking the toll of Hurricane Katrina from the video of her song
``Formation'' --- said Robin M. Boylorn, an associate professor of
communications at the University of Alabama.

Boylorn also pointed to Beyoncé's Coachella appearance, where the star
performed an ode to the dances and marching bands of historically Black
colleges and universities --- with signifiers that may have gone over
the heads of many white people in the audience, though their use by
Beyoncé drew attention and led to
\href{https://time.com/5574035/beyonce-homecoming-hbcu-history/}{wide
media coverage}.

``Her taking a space like Coachella, that is inherently white, and
making it a celebration of Blackness,'' Boylorn said, ``speaks to her
being able to shift the narrative and also literally shift the face of
the conversation. That is just a remarkable use of her platform.''

What statement Beyoncé makes with ``Black Is King'' remains to be seen
(at least for one more day). But that statement is likely to come
primarily through the film and not any comment.

``She says less,'' Lindsey said, ``as she has more power.''

Advertisement

\protect\hyperlink{after-bottom}{Continue reading the main story}

\hypertarget{site-index}{%
\subsection{Site Index}\label{site-index}}

\hypertarget{site-information-navigation}{%
\subsection{Site Information
Navigation}\label{site-information-navigation}}

\begin{itemize}
\tightlist
\item
  \href{https://help.nytimes3xbfgragh.onion/hc/en-us/articles/115014792127-Copyright-notice}{©~2020~The
  New York Times Company}
\end{itemize}

\begin{itemize}
\tightlist
\item
  \href{https://www.nytco.com/}{NYTCo}
\item
  \href{https://help.nytimes3xbfgragh.onion/hc/en-us/articles/115015385887-Contact-Us}{Contact
  Us}
\item
  \href{https://www.nytco.com/careers/}{Work with us}
\item
  \href{https://nytmediakit.com/}{Advertise}
\item
  \href{http://www.tbrandstudio.com/}{T Brand Studio}
\item
  \href{https://www.nytimes3xbfgragh.onion/privacy/cookie-policy\#how-do-i-manage-trackers}{Your
  Ad Choices}
\item
  \href{https://www.nytimes3xbfgragh.onion/privacy}{Privacy}
\item
  \href{https://help.nytimes3xbfgragh.onion/hc/en-us/articles/115014893428-Terms-of-service}{Terms
  of Service}
\item
  \href{https://help.nytimes3xbfgragh.onion/hc/en-us/articles/115014893968-Terms-of-sale}{Terms
  of Sale}
\item
  \href{https://spiderbites.nytimes3xbfgragh.onion}{Site Map}
\item
  \href{https://help.nytimes3xbfgragh.onion/hc/en-us}{Help}
\item
  \href{https://www.nytimes3xbfgragh.onion/subscription?campaignId=37WXW}{Subscriptions}
\end{itemize}
