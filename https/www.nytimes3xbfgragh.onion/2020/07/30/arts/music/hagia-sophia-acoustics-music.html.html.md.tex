Sections

SEARCH

\protect\hyperlink{site-content}{Skip to
content}\protect\hyperlink{site-index}{Skip to site index}

\href{https://www.nytimes3xbfgragh.onion/section/arts/music}{Music}

\href{https://myaccount.nytimes3xbfgragh.onion/auth/login?response_type=cookie\&client_id=vi}{}

\href{https://www.nytimes3xbfgragh.onion/section/todayspaper}{Today's
Paper}

\href{/section/arts/music}{Music}\textbar{}How a Historian Stuffed Hagia
Sophia's Sound Into a Studio

\url{https://nyti.ms/3ffGc6e}

\begin{itemize}
\item
\item
\item
\item
\item
\end{itemize}

Advertisement

\protect\hyperlink{after-top}{Continue reading the main story}

Supported by

\protect\hyperlink{after-sponsor}{Continue reading the main story}

\hypertarget{how-a-historian-stuffed-hagia-sophias-sound-into-a-studio}{%
\section{How a Historian Stuffed Hagia Sophia's Sound Into a
Studio}\label{how-a-historian-stuffed-hagia-sophias-sound-into-a-studio}}

Bissera Pentcheva used virtual acoustics to bring Istanbul to California
and reconstruct the sonic world of Byzantine cathedral music.

\includegraphics{https://static01.graylady3jvrrxbe.onion/images/2020/08/01/arts/31hagia-1/merlin_57568990_558ffcfa-4624-4b0b-a20b-d127a0ae1bce-articleLarge.jpg?quality=75\&auto=webp\&disable=upscale}

By Corinna da Fonseca-Wollheim

\begin{itemize}
\item
  July 30, 2020
\item
  \begin{itemize}
  \item
  \item
  \item
  \item
  \item
  \end{itemize}
\end{itemize}

Turquoise carpets covered the marble floor, with its geometric designs.
White drapes concealed the mosaic of the Virgin and Christ. Scaffolding
obscured crosses and other Christian symbols.

Footage broadcast around the world last week captured some of these
striking changes to Hagia Sophia, the Byzantine cathedral in Istanbul,
which served as a mosque under Ottoman rule before becoming a museum in
1934. On the orders of Turkey's president, Recep Tayyip Erdogan,
\href{https://www.nytimes3xbfgragh.onion/2020/07/24/world/europe/turkey-hagia-sophia-mosque-prayers.html}{it
is now once again used as a mosque}.

But for a group of scholars, scientists and musicians, Hagia Sophia's
rededication as a Muslim place of worship threatens to cloak a less
tangible treasure: its sound. Bissera Pentcheva,
\href{https://art.stanford.edu/people/bissera-pentcheva}{an art
historian at Stanford University} and an expert in the burgeoning field
of acoustic archaeology, has spent the past decade studying the
building's extravagantly reverberant acoustics to reconstruct
\href{https://ccrma.stanford.edu/groups/iconsofsound/film/}{the sonic
world of Byzantine cathedral music}. Ms. Pentcheva argues that Hagia
Sophia's mystical brilliance reveals itself fully only if it is viewed
as a vessel for animated light --- and sound.

\includegraphics{https://static01.graylady3jvrrxbe.onion/images/2020/08/01/arts/31hagia-2/31hagia-2-articleLarge.jpg?quality=75\&auto=webp\&disable=upscale}

``The void is a stage,'' she said in a recent interview over Zoom.

Conducting research inside this contested monument has required a
mixture of diplomacy, ingenuity and technology. Turkish authorities
forbade singing inside Hagia Sophia, even when it was operated as a
museum. Now that the building falls under the jurisdiction of religious
authorities, that ban will harden, and further research may be even more
difficult.

But Ms. Pentcheva's existing work culminated last fall in the release of
``The Lost Voices of Hagia Sophia,'' an album that brings to life the
stately mystery of Byzantine cathedral liturgy, bathed in the glittering
acoustics of the space for which it was written --- even though it was
recorded in a studio in California.

For about 20 years, it has been possible to superimpose the acoustics of
a particular space onto recorded music during postproduction. A pioneer
was Altiverb, a plug-in software that draws on
\href{https://www.audioease.com/altiverb/browse.php}{a large library} of
virtual spaces so that a recorded track can be retrofitted to seem like
it was done in, for example, the Berlin Philharmonie or the King's
Chamber inside the Great Pyramid of Giza.

But in what has become known as live virtual acoustics, processors and
speakers provide the acoustic feedback of a particular space in real
time, so that musicians can adjust their performance as if they were
really in another building.

\href{https://ccrma.stanford.edu/people/jonathan-abel}{Jonathan Abel}, a
consulting professor at the Center for Computer Research in Music and
Acoustics at Stanford, devised a plan with Ms. Bissera that allowed her
to capture vital information about the acoustic properties of Hagia
Sophia with the help of a balloon, discreet recording equipment and a
cooperative security guard.

In the winter of 2010, Ms. Pentcheva obtained permission to enter what
was then a museum at dawn, when Istanbul was quiet. She persuaded a
guard to stand in a spot that would have been occupied by singers during
the Byzantine era and to pop a balloon. In the meantime, she stationed
herself where a privileged member of the public might have experienced
mass. Microphones captured the explosion of sound and the ensuing wash
of reverberations.

Ms. Pentcheva was allowed to capture only four such pops over two
visits. But those bursts of sound yielded a wealth of data.

Image

One of the balloon pops in 2010 that helped Bissera Pentcheva capture
the acoustics of Hagia Sophia.Credit...Bissera V. Pentcheva

``That little balloon pop brings back all the information about the
material and the size of the space,'' Mr. Abel said. ``You can think of
a human voice as being made up of a whole bunch of balloon pops. Each
voice drags behind it a bunch of impulse responses, like streamers
behind a wedding car.''

The balloon noises, along with maps of the interior, allowed Mr. Abel to
identify what he called the acoustic fingerprint of the building,
including the multidirectional refraction of sound as it bounces off the
dome and marble colonnades. His computer simulation was then integrated
into a set of microphones and speakers.

Thus the members of \href{https://cappellaromana.org/}{Cappella Romana},
a vocal ensemble based in Portland, Ore., specializing in Byzantine
chant, recorded ``The Lost Voices'' in a space that persuasively
mimicked the acoustics of Hagia Sophia --- with its luscious
reverberation, cross echoes and amplification of particular frequencies.

Alexander Lingas, a musicologist and the music director of Cappella
Romana, said that the live virtual acoustics were transformative to his
understanding of the group's repertory. The long reverberation time
dictated slower tempos. Basses singing drones made subtle pitch
adjustments to match frequencies of maximum resonance.

Mr. Lingas said that some pieces only ``made sense'' inside the
simulated acoustics. One example featured on the album is a cherubic
hymn that likens the singers to angels.

``The music is designed to convey that,'' Mr. Lingas said. ``But I
remember editing this piece and thinking, `My, this is really
strange.''' Yet, he added, as the group rehearsed it with the virtual
acoustics, a pattern of repeated undulating motifs built up rippling
momentum until, as he described it, ``the sound essentially achieved
liftoff.''

\href{https://vimeo.com/357467398}{The Voice of Hagia Sophia - Full
Movie} from \href{https://vimeo.com/duyguerucman}{Duygu Eruçman} on
\href{https://vimeo.com}{Vimeo}.

Ms. Pentcheva believed that in Byzantine cathedral chant, reverberation
was key to invoking the divine presence. She pointed to the exuberant
amount of melisma in the repertory, where a single syllable is stretched
over multiple notes. In the liquid acoustics of Hagia Sophia, words sung
in this way blur, the way a line drawn in ink bleeds on wet paper.

``Rather than containing this smearing of semantics, the music itself
actually intensifies it,'' Ms. Pentcheva said. ``So there is this
process of alienation and estrangement from the register of human
language that happens in Hagia Sophia, and is a desired goal.''

In Greek Orthodox rites, Ms. Pentcheva argued, acoustics and chant
interact in a way that ``is not about sound carrying information, but
sound precipitating experience. It is a fully corporeal investment.''

The recording provides a glimpse of that experience. Phrases chanted in
unison leave a ghostly imprint. Rhythmic shudders and grace notes set
off blurry squiggles of overlapping echoes. Chords unfurl in reverberant
bloom.

The acoustic drama of Hagia Sophia would have unfolded alongside the
changing light and curling smoke of burning incense, enveloping the
senses. The effect is described in a 6th-century description of the
building by Paul the Silentiary, an aristocrat and poet at the court of
Justinian.

``He speaks about a human action that brings into presence the divine
reaction, the divine voice,'' Ms. Pentcheva said. ``In a sense that is
the reverberation of the space: After the human voice stops singing, the
building continues.''

Advertisement

\protect\hyperlink{after-bottom}{Continue reading the main story}

\hypertarget{site-index}{%
\subsection{Site Index}\label{site-index}}

\hypertarget{site-information-navigation}{%
\subsection{Site Information
Navigation}\label{site-information-navigation}}

\begin{itemize}
\tightlist
\item
  \href{https://help.nytimes3xbfgragh.onion/hc/en-us/articles/115014792127-Copyright-notice}{©~2020~The
  New York Times Company}
\end{itemize}

\begin{itemize}
\tightlist
\item
  \href{https://www.nytco.com/}{NYTCo}
\item
  \href{https://help.nytimes3xbfgragh.onion/hc/en-us/articles/115015385887-Contact-Us}{Contact
  Us}
\item
  \href{https://www.nytco.com/careers/}{Work with us}
\item
  \href{https://nytmediakit.com/}{Advertise}
\item
  \href{http://www.tbrandstudio.com/}{T Brand Studio}
\item
  \href{https://www.nytimes3xbfgragh.onion/privacy/cookie-policy\#how-do-i-manage-trackers}{Your
  Ad Choices}
\item
  \href{https://www.nytimes3xbfgragh.onion/privacy}{Privacy}
\item
  \href{https://help.nytimes3xbfgragh.onion/hc/en-us/articles/115014893428-Terms-of-service}{Terms
  of Service}
\item
  \href{https://help.nytimes3xbfgragh.onion/hc/en-us/articles/115014893968-Terms-of-sale}{Terms
  of Sale}
\item
  \href{https://spiderbites.nytimes3xbfgragh.onion}{Site Map}
\item
  \href{https://help.nytimes3xbfgragh.onion/hc/en-us}{Help}
\item
  \href{https://www.nytimes3xbfgragh.onion/subscription?campaignId=37WXW}{Subscriptions}
\end{itemize}
