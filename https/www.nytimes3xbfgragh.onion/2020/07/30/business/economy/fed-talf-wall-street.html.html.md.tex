Sections

SEARCH

\protect\hyperlink{site-content}{Skip to
content}\protect\hyperlink{site-index}{Skip to site index}

\href{https://www.nytimes3xbfgragh.onion/section/business/economy}{Economy}

\href{https://myaccount.nytimes3xbfgragh.onion/auth/login?response_type=cookie\&client_id=vi}{}

\href{https://www.nytimes3xbfgragh.onion/section/todayspaper}{Today's
Paper}

\href{/section/business/economy}{Economy}\textbar{}How Pimco's
Cayman-Based Hedge Fund Can Profit From the Fed's Rescue

\url{https://nyti.ms/2BHdy0e}

\begin{itemize}
\item
\item
\item
\item
\item
\item
\end{itemize}

\href{https://www.nytimes3xbfgragh.onion/news-event/coronavirus?action=click\&pgtype=Article\&state=default\&region=TOP_BANNER\&context=storylines_menu}{The
Coronavirus Outbreak}

\begin{itemize}
\tightlist
\item
  live\href{https://www.nytimes3xbfgragh.onion/2020/08/01/world/coronavirus-covid-19.html?action=click\&pgtype=Article\&state=default\&region=TOP_BANNER\&context=storylines_menu}{Latest
  Updates}
\item
  \href{https://www.nytimes3xbfgragh.onion/interactive/2020/us/coronavirus-us-cases.html?action=click\&pgtype=Article\&state=default\&region=TOP_BANNER\&context=storylines_menu}{Maps
  and Cases}
\item
  \href{https://www.nytimes3xbfgragh.onion/interactive/2020/science/coronavirus-vaccine-tracker.html?action=click\&pgtype=Article\&state=default\&region=TOP_BANNER\&context=storylines_menu}{Vaccine
  Tracker}
\item
  \href{https://www.nytimes3xbfgragh.onion/interactive/2020/07/29/us/schools-reopening-coronavirus.html?action=click\&pgtype=Article\&state=default\&region=TOP_BANNER\&context=storylines_menu}{What
  School May Look Like}
\item
  \href{https://www.nytimes3xbfgragh.onion/live/2020/07/31/business/stock-market-today-coronavirus?action=click\&pgtype=Article\&state=default\&region=TOP_BANNER\&context=storylines_menu}{Economy}
\end{itemize}

Advertisement

\protect\hyperlink{after-top}{Continue reading the main story}

Supported by

\protect\hyperlink{after-sponsor}{Continue reading the main story}

\hypertarget{how-pimcos-cayman-based-hedge-fund-can-profit-from-the-feds-rescue}{%
\section{How Pimco's Cayman-Based Hedge Fund Can Profit From the Fed's
Rescue}\label{how-pimcos-cayman-based-hedge-fund-can-profit-from-the-feds-rescue}}

Congress said borrowers in taxpayer-backed rescue programs had to be
from the United States. Wall Street has a workaround.

\includegraphics{https://static01.graylady3jvrrxbe.onion/images/2020/07/31/business/29JPDC-FedProfit1-print/29DC-FedProfit-sub-articleLarge.jpg?quality=75\&auto=webp\&disable=upscale}

\href{https://www.nytimes3xbfgragh.onion/by/jeanna-smialek}{\includegraphics{https://static01.graylady3jvrrxbe.onion/images/2020/07/03/reader-center/author-jeanna-smialek/author-jeanna-smialek-thumbLarge.png}}

By \href{https://www.nytimes3xbfgragh.onion/by/jeanna-smialek}{Jeanna
Smialek}

\begin{itemize}
\item
  July 30, 2020
\item
  \begin{itemize}
  \item
  \item
  \item
  \item
  \item
  \item
  \end{itemize}
\end{itemize}

WASHINGTON --- Pacific Investment Management Company runs a
\href{https://sec.report/Document/0001569540-16-000003/}{hedge fund}
registered in the Cayman Islands, a common structure for avoiding
certain U.S. taxes. But when a profit opportunity arose from the ashes
of America's
\href{https://www.nytimes3xbfgragh.onion/news-event/coronavirus}{coronavirus
crisis}, that international location did not stop it from seizing the
moment.

The Federal Reserve opened a highly anticipated emergency lending
program in June, a revamped version of one it used during the 2008
financial crisis. This time around,
\href{https://assets.documentcloud.org/documents/20059055/final-final-cares-act.pdf}{Congress
stipulated} that only American companies can participate as borrowers in
such programs. Despite being offshore, Pimco's fund had an easy way to
benefit.

The offshore fund is invested in an entity
\href{https://icis.corp.delaware.gov/ecorp/entitysearch/NameSearch.aspx}{registered
in Delawar}e. That entity can be used by investment managers to buy and
sell bonds. The Delaware operation borrowed \$13.1 million from the Fed
program by pledging a bundle of debt as collateral. Investors in the
Cayman-based hedge fund ultimately stand to profit from the transaction.

The Pimco example is not unique --- other foreign investors have put
money into U.S.-based funds that are tapping the Fed program. That they
found a way to participate in a program restricted to American borrowers
**** highlights the potential for financial firms to make money from the
Fed's market rescue programs, even if doing so means maneuvering around
congressional limitations on eligibility. Investors earned double-digit
returns on the program during the 2008 financial crisis and they stand
to profit this time around as well, as they collect interest on the debt
bundles and, thanks to the Fed's cheap funding, pay very little to hold
them.

The Fed's program is intended to keep credit flowing through the
economy, but its design has provided an opportunity for global financial
players to profit from **** an initiative backed by taxpayer funding.
That side effect could **** draw further scrutiny **** to **** the Fed's
rescue efforts, which are already prompting questions from lawmakers
about who benefits, and on what terms.

The lending programs ``drag the Fed into political crossfire,'' said
Mark Spindel, chief investment officer at Potomac River Capital and an
author of a book on the politics of the Fed. ``The Fed is seen as the
honest broker in town --- but just because you're the honest broker
today, doesn't mean you're not going to face questions down the road.''

The goal of the Fed program in question, known as the Term Asset-Backed
Securities Loan Facility or TALF, is to bolster a critical corner of
U.S. debt markets, one where loans are bundled and sold off to investors
who are willing to take on risk in exchange for interest payments. That
helps to keep the market for commercial mortgages functioning, and
allows student loans and credit card debt to continue flowing to
end-users.

\hypertarget{latest-updates-economy}{%
\section{\texorpdfstring{\href{https://www.nytimes3xbfgragh.onion/live/2020/07/31/business/stock-market-today-coronavirus?action=click\&pgtype=Article\&state=default\&region=MAIN_CONTENT_1\&context=storylines_live_updates}{Latest
Updates:
Economy}}{Latest Updates: Economy}}\label{latest-updates-economy}}

\href{https://www.nytimes3xbfgragh.onion/live/2020/07/31/business/stock-market-today-coronavirus?action=click\&pgtype=Article\&state=default\&region=MAIN_CONTENT_1\&context=storylines_live_updates\#kodaks-chief-executive-was-given-stock-options-then-the-share-price-spiked-1000-percent}{20h
ago}

\href{https://www.nytimes3xbfgragh.onion/live/2020/07/31/business/stock-market-today-coronavirus?action=click\&pgtype=Article\&state=default\&region=MAIN_CONTENT_1\&context=storylines_live_updates\#kodaks-chief-executive-was-given-stock-options-then-the-share-price-spiked-1000-percent}{Kodak's
chief executive was given stock options. Then the share price spiked
1,000 percent.}

\href{https://www.nytimes3xbfgragh.onion/live/2020/07/31/business/stock-market-today-coronavirus?action=click\&pgtype=Article\&state=default\&region=MAIN_CONTENT_1\&context=storylines_live_updates\#fitch-ratings-downgrades-its-outlook-on-us-debt}{23h
ago}

\href{https://www.nytimes3xbfgragh.onion/live/2020/07/31/business/stock-market-today-coronavirus?action=click\&pgtype=Article\&state=default\&region=MAIN_CONTENT_1\&context=storylines_live_updates\#fitch-ratings-downgrades-its-outlook-on-us-debt}{Fitch
Ratings downgrades its outlook on U.S. debt.}

\href{https://www.nytimes3xbfgragh.onion/live/2020/07/31/business/stock-market-today-coronavirus?action=click\&pgtype=Article\&state=default\&region=MAIN_CONTENT_1\&context=storylines_live_updates\#us-sanctions-more-chinese-officials-over-human-rights-violations-as-tensions-flare}{30h
ago}

\href{https://www.nytimes3xbfgragh.onion/live/2020/07/31/business/stock-market-today-coronavirus?action=click\&pgtype=Article\&state=default\&region=MAIN_CONTENT_1\&context=storylines_live_updates\#us-sanctions-more-chinese-officials-over-human-rights-violations-as-tensions-flare}{U.S.
sanctions more Chinese officials over human rights violations as
tensions flare}

\href{https://www.nytimes3xbfgragh.onion/live/2020/07/31/business/stock-market-today-coronavirus?action=click\&pgtype=Article\&state=default\&region=MAIN_CONTENT_1\&context=storylines_live_updates}{See
more updates}

More live coverage:
\href{https://www.nytimes3xbfgragh.onion/2020/08/01/world/coronavirus-covid-19.html?action=click\&pgtype=Article\&state=default\&region=MAIN_CONTENT_1\&context=storylines_live_updates}{Global}

The program was not created to make money for investment vehicles or the
investors they represent. But because of the way TALF works, financial
firms like Pimco's hedge fund can make a profit from it.

It operates by encouraging investors to purchase a certain type of debt
called asset-backed securities.

\includegraphics{https://static01.graylady3jvrrxbe.onion/images/2020/07/31/business/29JPdc-virus-fedprofit2-print/29dc-virus-fedprofit-02-articleLarge.jpg?quality=75\&auto=webp\&disable=upscale}

A fund can buy those securities using some combination of cash and
short-term loans and then take them to the Fed in exchange for a TALF
loan.

The TALF loan can be used to pay back whatever money the fund borrowed
to make the purchase in the first place, so that its holdings are
financed mostly by the cheap Fed loan, and a sliver of its own money
(what is known as a ``haircut'' in financial parlance). It essentially
earns the difference between what it makes in interest from the
securities and what it is paying on the Fed loan.

Because investors have just a small amount of money at stake, returns on
each invested dollar can be quite high. Investors said they anticipated
high single-digit returns in 2020, far lower than the double-digit
returns in 2008 but still generous.

The Fed has so far released detailed data only on TALF's first round of
loans, though the program has since finalized
\href{https://www.newyorkfed.org/markets/term-asset-backed-securities-loan-facility/term-asset-backed-securities-loan-facility-rates}{two
more round}s. The Fed will most likely release additional data in
mid-August.

Pimco's Cayman Islands-based fund, which has borrowed via a U.S.-based
entity called TOCU IX, is one of several foreign investors using an
American investment vehicle to gain access to TALF. The pension plan of
the Oxford University Press Group will tap the program through a fund
set up by the New York-based investment manager MacKay Shields. A
Singapore-based fund is a material investor in an offering by the giant
financial firm BlackRock, according to the Fed's first round of
\href{https://www.federalreserve.gov/reports-to-congress-covid-19.htm}{detailed
disclosures}.

The fact that some investors based overseas can make money from TALF
does not break Congress's rules, but it may fall shy of what some
lawmakers intended. They
\href{https://www.govinfo.gov/content/pkg/BILLS-116hr748enr/pdf/BILLS-116hr748enr.pdf}{specified
that} loans, advances and asset purchases made under the Fed's programs
should be restricted to ``businesses that are created or organized in
the United States or under the laws of the United States.'' But they
said nothing about who could ultimately benefit.

``There are going to be people who focus on this like a laser,'' Peter
Conti-Brown, a Fed historian at the University of Pennsylvania's Wharton
School, said of the fact that foreign investors in some cases benefit
from Fed programs. But the reality, he pointed out, is that financial
markets are global.

``Others are going to say that there's no way to provide liquidity
without benefiting international counterparties.''

And while Pimco's fund and other foreign investors may profit by
participating in the program, their investment is also helping to keep
more money flowing into the Fed's program, smoothing over U.S.
securitization markets.

That reality has presented a challenge for the Fed, which has had to
walk a fine line between creating emergency programs that are effective
while also making them politically palatable. Lawmakers want the Fed to
help the economy, but have also warned the central bank against allowing
companies to take advantage of taxpayer-backed funding.

When Republicans and Democrats were hammering out the details of their
coronavirus rescue package in March, congressional leaders agreed to
give the Treasury Department \$454 billion to back up Fed emergency
programs.

The Fed requires a Treasury backstop for many of those efforts, to
insure against losses in case borrowers default. But because the Fed did
not expect to lose every dollar it lent out, it could use the \$454
billion to field a huge rescue: potentially more than \$4 trillion in
loans to businesses, states and cities.

The ability to supersize the coronavirus response package was an
attractive proposition. But many lawmakers in both parties were wary
about handing the Fed and the Treasury so much money. Many remembered
the 2008 bank bailouts and the bad taste they had left behind. They did
not want a repeat.

Image

Treasury Secretary Steve Mnuchin in the Capitol on Tuesday.
Congressional leaders earlier agreed to give the Treasury Department
\$454 billion to back up Fed emergency programs.Credit...Anna Moneymaker
for The New York Times

So Steven Mnuchin, the Treasury secretary, and key lawmakers agreed to
terms that attached strings to the funding. Companies borrowing direct
loans might be asked to try to maintain their payroll. Those who
borrowed directly would also be banned from making dividend payments,
and executives would face compensation limits. Only U.S. companies could
borrow.

Those requirements are generally guidelines rather than binding rules,
given the way the programs work. The Fed has found itself being hammered
on both sides --- some lawmakers have questioned whether the central
bank is precluding companies from using its programs by being too
strict, while others have warned it against letting big corporations and
Wall Street firms benefit.

Foreign investor participation in the TALF program could raise similar
questions from lawmakers and the oversight groups set up to police where
the funds are going. Mr. Conti-Brown and others say that while Congress
gave the Fed the room to make design choices, that will not insulate the
central bank from critique.

The fact that the Fed has discretion ``is a byproduct of political
compromise,'' Mr. Conti-Brown said. But ``the Fed is always open to
criticism down the road.''

Advertisement

\protect\hyperlink{after-bottom}{Continue reading the main story}

\hypertarget{site-index}{%
\subsection{Site Index}\label{site-index}}

\hypertarget{site-information-navigation}{%
\subsection{Site Information
Navigation}\label{site-information-navigation}}

\begin{itemize}
\tightlist
\item
  \href{https://help.nytimes3xbfgragh.onion/hc/en-us/articles/115014792127-Copyright-notice}{©~2020~The
  New York Times Company}
\end{itemize}

\begin{itemize}
\tightlist
\item
  \href{https://www.nytco.com/}{NYTCo}
\item
  \href{https://help.nytimes3xbfgragh.onion/hc/en-us/articles/115015385887-Contact-Us}{Contact
  Us}
\item
  \href{https://www.nytco.com/careers/}{Work with us}
\item
  \href{https://nytmediakit.com/}{Advertise}
\item
  \href{http://www.tbrandstudio.com/}{T Brand Studio}
\item
  \href{https://www.nytimes3xbfgragh.onion/privacy/cookie-policy\#how-do-i-manage-trackers}{Your
  Ad Choices}
\item
  \href{https://www.nytimes3xbfgragh.onion/privacy}{Privacy}
\item
  \href{https://help.nytimes3xbfgragh.onion/hc/en-us/articles/115014893428-Terms-of-service}{Terms
  of Service}
\item
  \href{https://help.nytimes3xbfgragh.onion/hc/en-us/articles/115014893968-Terms-of-sale}{Terms
  of Sale}
\item
  \href{https://spiderbites.nytimes3xbfgragh.onion}{Site Map}
\item
  \href{https://help.nytimes3xbfgragh.onion/hc/en-us}{Help}
\item
  \href{https://www.nytimes3xbfgragh.onion/subscription?campaignId=37WXW}{Subscriptions}
\end{itemize}
