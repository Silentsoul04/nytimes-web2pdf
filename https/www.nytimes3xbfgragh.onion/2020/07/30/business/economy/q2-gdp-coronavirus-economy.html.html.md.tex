Sections

SEARCH

\protect\hyperlink{site-content}{Skip to
content}\protect\hyperlink{site-index}{Skip to site index}

\href{https://www.nytimes3xbfgragh.onion/section/business/economy}{Economy}

\href{https://myaccount.nytimes3xbfgragh.onion/auth/login?response_type=cookie\&client_id=vi}{}

\href{https://www.nytimes3xbfgragh.onion/section/todayspaper}{Today's
Paper}

\href{/section/business/economy}{Economy}\textbar{}A Collapse That Wiped
Out 5 Years of Growth, With No Bounce in Sight

\url{https://nyti.ms/3jWdfjp}

\begin{itemize}
\item
\item
\item
\item
\item
\end{itemize}

\href{https://www.nytimes3xbfgragh.onion/news-event/coronavirus?action=click\&pgtype=Article\&state=default\&region=TOP_BANNER\&context=storylines_menu}{The
Coronavirus Outbreak}

\begin{itemize}
\tightlist
\item
  live\href{https://www.nytimes3xbfgragh.onion/2020/08/01/world/coronavirus-covid-19.html?action=click\&pgtype=Article\&state=default\&region=TOP_BANNER\&context=storylines_menu}{Latest
  Updates}
\item
  \href{https://www.nytimes3xbfgragh.onion/interactive/2020/us/coronavirus-us-cases.html?action=click\&pgtype=Article\&state=default\&region=TOP_BANNER\&context=storylines_menu}{Maps
  and Cases}
\item
  \href{https://www.nytimes3xbfgragh.onion/interactive/2020/science/coronavirus-vaccine-tracker.html?action=click\&pgtype=Article\&state=default\&region=TOP_BANNER\&context=storylines_menu}{Vaccine
  Tracker}
\item
  \href{https://www.nytimes3xbfgragh.onion/interactive/2020/07/29/us/schools-reopening-coronavirus.html?action=click\&pgtype=Article\&state=default\&region=TOP_BANNER\&context=storylines_menu}{What
  School May Look Like}
\item
  \href{https://www.nytimes3xbfgragh.onion/live/2020/07/31/business/stock-market-today-coronavirus?action=click\&pgtype=Article\&state=default\&region=TOP_BANNER\&context=storylines_menu}{Economy}
\end{itemize}

Advertisement

\protect\hyperlink{after-top}{Continue reading the main story}

Supported by

\protect\hyperlink{after-sponsor}{Continue reading the main story}

\hypertarget{a-collapse-that-wiped-out-5-years-of-growth-with-no-bounce-in-sight}{%
\section{A Collapse That Wiped Out 5 Years of Growth, With No Bounce in
Sight}\label{a-collapse-that-wiped-out-5-years-of-growth-with-no-bounce-in-sight}}

The second-quarter contraction set a grim record, and it would have been
worse without government aid that is expiring.

The percentage decrease in G.D.P.

is by far the biggest on record.

+

4

\%

+

2

0

--

2

--

4

--

6

--9.5\%

Percentage change from previous quarter

--

8

--

10

1950

1960

1970

1980

1990

2000

2010

2020

The percentage decrease in G.D.P. is by far the biggest on record.

+

4

\%

+

2

0

--

2

--

4

--

6

--9.5\%

Percentage change from previous quarter

--

8

--

10

1950

1960

1970

1980

1990

2000

2010

2020

The percentage decrease in G.D.P. is by far the biggest on record.

+

4

\%

+

2

0

--

2

--

4

--

6

Percentage change from previous quarter

--9.5\%

--

8

--

10

1950

1960

1970

1980

1990

2000

2010

2020

Source: Bureau of Economic Analysis

By Karl Russell

\href{https://www.nytimes3xbfgragh.onion/by/ben-casselman}{\includegraphics{https://static01.graylady3jvrrxbe.onion/images/2018/11/09/multimedia/author-ben-casselman/author-ben-casselman-thumbLarge.png}}

By \href{https://www.nytimes3xbfgragh.onion/by/ben-casselman}{Ben
Casselman}

\begin{itemize}
\item
  July 30, 2020
\item
  \begin{itemize}
  \item
  \item
  \item
  \item
  \item
  \end{itemize}
\end{itemize}

The coronavirus pandemic's toll on the nation's economy became
emphatically clearer Thursday as the government detailed the most
devastating three-month collapse on record, which wiped away nearly five
years of growth.

Gross domestic product, the broadest measure of goods and services
produced, fell 9.5 percent in the second quarter of the year as
consumers cut back spending, businesses pared investments and global
trade dried up,
\href{https://www.bea.gov/sites/default/files/2020-07/gdp2q20_adv.pdf}{the
Commerce Department said}.

The drop --- the equivalent of a 32.9 percent annual rate of decline ---
would have been even more severe without trillions of dollars in
government aid to households and businesses.

But there is mounting evidence that the attempt to freeze the economy
and defeat the virus has not produced the rapid rebound that many
envisioned. A surge in coronavirus cases and deaths across the country
has led to a renewed
\href{https://www.nytimes3xbfgragh.onion/2020/07/15/business/economy/economic-recovery-coronavirus-resurgence.html?action=click\&module=RelatedLinks\&pgtype=Article}{pullback
in economic activity}, reflecting consumer unease and renewed shutdowns.
And much of the government support is on the verge of running out, with
Washington at an impasse over next steps.

``In another world, a sharp drop in activity would have been just a
good, necessary blip while we addressed the virus,'' said Heather
Boushey, president of the Washington Center for Equitable Growth, a
progressive think tank. ``From where we sit in July, we know that this
wasn't just a short-term blip. We did not get the virus under control.''

Data from Europe shows what might have been. Germany on Thursday
reported a
\href{https://www.nytimes3xbfgragh.onion/live/2020/07/30/business/stock-market-today-coronavirus/the-german-economy-had-its-biggest-slump-in-50-years}{drop
in second-quarter G.D.P.} that was even steeper than the U.S. decline.
But in Germany, coronavirus cases fell sharply and remain low, which has
allowed a much stronger economic rebound in recent weeks.

In the United States, the rebound appears to have stalled. More than 1.4
million Americans filed new claims for state unemployment benefits last
week, the \href{https://oui.doleta.gov/press/2020/073020.pdf}{Labor
Department said Thursday}. It was the 19th straight week that the tally
exceeded one million, an unheard-of figure before the pandemic. A
further 830,000 people filed for benefits under the federal Pandemic
Unemployment Assistance program, which supports freelancers, the
self-employed and other workers not covered by traditional unemployment
benefits.

\includegraphics{https://static01.graylady3jvrrxbe.onion/images/2020/07/30/business/30virus-econ/merlin_174916941_09986855-52f6-4220-8c68-107e59a7cc27-articleLarge.jpg?quality=75\&auto=webp\&disable=upscale}

In total, some 30 million people are receiving unemployment benefits, a
number that has come down only slowly as new layoffs --- many of them
permanent job losses, as opposed to the spring's temporary furloughs ---
offset gradual rehiring. Some economists now fear that the monthly jobs
report coming next week will show that total employment fell in July
after two months of strong gains. The slow recovery, and signs of
backsliding, are taking a toll on consumer confidence, which fell in
July after rising in June.

``Not only have we plateaued, but we may be losing ground,'' said Diane
Swonk, chief economist at the accounting firm Grant Thornton in Chicago.
``To have these kinds of numbers in July when many in Congress hoped
this would be over by summer underscores how unique and persistent the
Covid crisis is.''

\hypertarget{latest-updates-economy}{%
\section{\texorpdfstring{\href{https://www.nytimes3xbfgragh.onion/live/2020/07/31/business/stock-market-today-coronavirus?action=click\&pgtype=Article\&state=default\&region=MAIN_CONTENT_1\&context=storylines_live_updates}{Latest
Updates:
Economy}}{Latest Updates: Economy}}\label{latest-updates-economy}}

\href{https://www.nytimes3xbfgragh.onion/live/2020/07/31/business/stock-market-today-coronavirus?action=click\&pgtype=Article\&state=default\&region=MAIN_CONTENT_1\&context=storylines_live_updates\#kodaks-chief-executive-was-given-stock-options-then-the-share-price-spiked-1000-percent}{20h
ago}

\href{https://www.nytimes3xbfgragh.onion/live/2020/07/31/business/stock-market-today-coronavirus?action=click\&pgtype=Article\&state=default\&region=MAIN_CONTENT_1\&context=storylines_live_updates\#kodaks-chief-executive-was-given-stock-options-then-the-share-price-spiked-1000-percent}{Kodak's
chief executive was given stock options. Then the share price spiked
1,000 percent.}

\href{https://www.nytimes3xbfgragh.onion/live/2020/07/31/business/stock-market-today-coronavirus?action=click\&pgtype=Article\&state=default\&region=MAIN_CONTENT_1\&context=storylines_live_updates\#fitch-ratings-downgrades-its-outlook-on-us-debt}{23h
ago}

\href{https://www.nytimes3xbfgragh.onion/live/2020/07/31/business/stock-market-today-coronavirus?action=click\&pgtype=Article\&state=default\&region=MAIN_CONTENT_1\&context=storylines_live_updates\#fitch-ratings-downgrades-its-outlook-on-us-debt}{Fitch
Ratings downgrades its outlook on U.S. debt.}

\href{https://www.nytimes3xbfgragh.onion/live/2020/07/31/business/stock-market-today-coronavirus?action=click\&pgtype=Article\&state=default\&region=MAIN_CONTENT_1\&context=storylines_live_updates\#us-sanctions-more-chinese-officials-over-human-rights-violations-as-tensions-flare}{30h
ago}

\href{https://www.nytimes3xbfgragh.onion/live/2020/07/31/business/stock-market-today-coronavirus?action=click\&pgtype=Article\&state=default\&region=MAIN_CONTENT_1\&context=storylines_live_updates\#us-sanctions-more-chinese-officials-over-human-rights-violations-as-tensions-flare}{U.S.
sanctions more Chinese officials over human rights violations as
tensions flare}

\href{https://www.nytimes3xbfgragh.onion/live/2020/07/31/business/stock-market-today-coronavirus?action=click\&pgtype=Article\&state=default\&region=MAIN_CONTENT_1\&context=storylines_live_updates}{See
more updates}

More live coverage:
\href{https://www.nytimes3xbfgragh.onion/2020/08/01/world/coronavirus-covid-19.html?action=click\&pgtype=Article\&state=default\&region=MAIN_CONTENT_1\&context=storylines_live_updates}{Global}

The economic collapse in the second quarter was unrivaled in its speed
and breathtaking in its severity. The decline was more than twice as
large as in the Great Recession a decade ago, but occurred in a fraction
of the time. The only possible comparisons in modern American history
came during the Great Depression and the demobilization after World War
II, both of which predated modern economic statistics.

Economists and epidemiologists alike describe
t\href{https://www.nytimes3xbfgragh.onion/2020/07/18/us/politics/trump-coronavirus-response-failure-leadership.html}{he
U.S. failure to control the virus} during the initial shutdown as a
missed opportunity. The government's efforts at financial support were
largely successful: After plummeting in March and April, retail sales
rose in May and June as stimulus payments and a \$600 weekly federal
supplement to unemployment benefits began flowing into consumers' bank
accounts. Loans made under the Paycheck Protection Program allowed small
businesses to begin bringing back furloughed workers.

The wave of evictions and foreclosures that many economists predicted
early in the recession largely failed to materialize. But those programs
have expired or are about to do so. And efforts to extend them have been
delayed in Congress by disagreements --- between the parties, and among
Republicans --- about how and how much to spend.

``The lesson from the early policy experiments is that it is possible to
support the income of people and to offset those financial
constraints,'' said Tara Sinclair, a George Washington University
economist and a senior fellow at the Indeed Hiring Lab. ``But now as of
today, we're not going to have that anymore.''

The up-and-down nature of the recovery is shown by Russian River Brewing
Company in Sonoma County, Calif.

Image

State shutdowns were a disruption at Russian River Brewing Company's two
brew pubs.Credit...Felix Uribe for The New York Times

Image

The company's revenues are back more or less to normal levels, but
profits are still down.Credit...Felix Uribe for The New York Times

Before the pandemic, half the company's revenue came from retail sales:
food and drink at its two brew pubs, tours and tastings at the brewery
itself, in-person purchases of bottled beer. When California ordered
restaurants to shut down in mid-March, all of that revenue disappeared.

``We were panicking for 48 hours,'' said Natalie Cilurzo, who owns
Russian River with her husband, Vinnie.

Once the panic passed, the Cilurzos began finding ways to plug the hole.
With restaurants shut down, grocery-store sales surged, and online
ordering proved to be a hit. A loan through the Paycheck Protection
Program helped cover employee salaries and other expenses. And in early
June, Russian River was allowed to reopen its brew pubs.

But it has been a partial and uneven rebound. Revenues are back more or
less to normal levels, but profits are still down because margins are
lower for grocery-store sales. The company has brought back most of its
furloughed workers, but it has permanently laid off 20 percent of its
staff. And one of its locations had to close again when California
reimposed restrictions on indoor dining.

Image

Natalie Cilurzo, a co-owner, said Russian River had to change plans more
than once because of the pandemic. ``We're kind of in this yo-yo here in
California,'' she said.Credit...Felix Uribe for The New York Times

``We're open, we're closed, we're open, we're closed --- we're kind of
in this yo-yo here in California,'' Ms. Cilurzo said.

The G.D.P. report shows the severity of the temporary slowdown and hints
at evidence of more lasting damage. Consumer spending fell 10.1 percent,
led by a near-total collapse in spending on restaurant meals,
recreational activities and other services. Even health care spending
fell, as patients canceled elective procedures and delayed routine care.

Businesses, too, pulled back sharply on their investments, which Ms.
Sinclair said was a worrying sign because it suggested they do not
expect a rapid recovery in demand. And trade --- both imports and
exports --- plunged, reflecting the global nature of the pandemic.

There were glimmers of optimism. Spending on goods fell a modest 3
percent, and some quarantine-friendly categories had increases. And
while residential construction slumped in the second quarter, more
recent data suggests the housing market has experienced a strong
rebound, buoyed by low interest rates.

Many economists, however, caution that spending could fall further if
Congress reduces or eliminates aid to households and businesses. And it
would add to the stress on the unemployed, who are facing the expiration
of extra jobless benefits at a time when the virus remains prevalent and
jobs remain scarce.

Louise Francis had worked as a banquet cook at the Sheraton Hotel in New
Orleans for nearly two decades when she was furloughed last spring. It
took three months of effort to get her first unemployment check, and she
relied on her adult daughters for help in the meantime. But when she
began receiving the money, the \$600 weekly federal supplement to
regular state benefits allowed her to find some stability.

Image

With the \$600 federal supplement to unemployment payments ending,
Louise Francis will have to get by on \$247 a week in state
benefits.Credit...L. Kasimu Harris for The New York Times

``With the \$600, you could see your way a little bit,'' Ms. Francis
said. ``You could feel a little more comfortable. You could pay three or
four bills and not feel so far behind.''

With the supplement at an end and
\href{https://www.nytimes3xbfgragh.onion/2020/07/29/business/economy/unemployment-benefits-coronavirus.html}{no
congressional consensus on replacing it}, Ms. Francis, 59, isn't sure
what she will do. Her age, combined with her diabetes and high blood
pressure, puts her at high risk of severe illness if she contracts the
coronavirus, which makes her reluctant to take any job that puts her
into face-to-face contact with the public, especially with
\href{https://www.nytimes3xbfgragh.onion/interactive/2020/us/louisiana-coronavirus-cases.html}{cases
surging in Louisiana}.

Ms. Francis's husband is retired, leaving her as the family's
breadwinner, and she will have to get by on \$247 a week in state
benefits.

``If they take that \$600 from us, how am I supposed to be able to
continue paying my bills?'' she said. ``You still have to eat, to pay
insurance. If they take it away, they're going to push us back into
poverty.''

Nelson D. Schwartz contributed reporting.

Advertisement

\protect\hyperlink{after-bottom}{Continue reading the main story}

\hypertarget{site-index}{%
\subsection{Site Index}\label{site-index}}

\hypertarget{site-information-navigation}{%
\subsection{Site Information
Navigation}\label{site-information-navigation}}

\begin{itemize}
\tightlist
\item
  \href{https://help.nytimes3xbfgragh.onion/hc/en-us/articles/115014792127-Copyright-notice}{©~2020~The
  New York Times Company}
\end{itemize}

\begin{itemize}
\tightlist
\item
  \href{https://www.nytco.com/}{NYTCo}
\item
  \href{https://help.nytimes3xbfgragh.onion/hc/en-us/articles/115015385887-Contact-Us}{Contact
  Us}
\item
  \href{https://www.nytco.com/careers/}{Work with us}
\item
  \href{https://nytmediakit.com/}{Advertise}
\item
  \href{http://www.tbrandstudio.com/}{T Brand Studio}
\item
  \href{https://www.nytimes3xbfgragh.onion/privacy/cookie-policy\#how-do-i-manage-trackers}{Your
  Ad Choices}
\item
  \href{https://www.nytimes3xbfgragh.onion/privacy}{Privacy}
\item
  \href{https://help.nytimes3xbfgragh.onion/hc/en-us/articles/115014893428-Terms-of-service}{Terms
  of Service}
\item
  \href{https://help.nytimes3xbfgragh.onion/hc/en-us/articles/115014893968-Terms-of-sale}{Terms
  of Sale}
\item
  \href{https://spiderbites.nytimes3xbfgragh.onion}{Site Map}
\item
  \href{https://help.nytimes3xbfgragh.onion/hc/en-us}{Help}
\item
  \href{https://www.nytimes3xbfgragh.onion/subscription?campaignId=37WXW}{Subscriptions}
\end{itemize}
