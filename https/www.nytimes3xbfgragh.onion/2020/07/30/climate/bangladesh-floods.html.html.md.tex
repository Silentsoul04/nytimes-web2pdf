Sections

SEARCH

\protect\hyperlink{site-content}{Skip to
content}\protect\hyperlink{site-index}{Skip to site index}

\href{https://www.nytimes3xbfgragh.onion/section/climate}{Climate}

\href{https://myaccount.nytimes3xbfgragh.onion/auth/login?response_type=cookie\&client_id=vi}{}

\href{https://www.nytimes3xbfgragh.onion/section/todayspaper}{Today's
Paper}

\href{/section/climate}{Climate}\textbar{}A Quarter of Bangladesh Is
Flooded. Millions Have Lost Everything.

\url{https://nyti.ms/334VB7c}

\begin{itemize}
\item
\item
\item
\item
\item
\item
\end{itemize}

\href{https://www.nytimes3xbfgragh.onion/section/climate?action=click\&pgtype=Article\&state=default\&region=TOP_BANNER\&context=storylines_menu}{Climate
and Environment}

\begin{itemize}
\tightlist
\item
  \href{https://www.nytimes3xbfgragh.onion/2020/07/30/climate/sea-level-inland-floods.html?action=click\&pgtype=Article\&state=default\&region=TOP_BANNER\&context=storylines_menu}{Rising
  Seas}
\item
  \href{https://www.nytimes3xbfgragh.onion/interactive/2020/climate/trump-environment-rollbacks.html?action=click\&pgtype=Article\&state=default\&region=TOP_BANNER\&context=storylines_menu}{Trump's
  Changes}
\item
  \href{https://www.nytimes3xbfgragh.onion/interactive/2020/04/19/climate/climate-crash-course-1.html?action=click\&pgtype=Article\&state=default\&region=TOP_BANNER\&context=storylines_menu}{Climate
  101}
\item
  \href{https://www.nytimes3xbfgragh.onion/interactive/2018/08/30/climate/how-much-hotter-is-your-hometown.html?action=click\&pgtype=Article\&state=default\&region=TOP_BANNER\&context=storylines_menu}{Is
  Your Hometown Hotter?}
\item
  \href{https://www.nytimes3xbfgragh.onion/newsletters/climate-change?action=click\&pgtype=Article\&state=default\&region=TOP_BANNER\&context=storylines_menu}{Newsletter}
\end{itemize}

Advertisement

\protect\hyperlink{after-top}{Continue reading the main story}

Supported by

\protect\hyperlink{after-sponsor}{Continue reading the main story}

Inequity at the boiling point

\hypertarget{a-quarter-of-bangladesh-is-flooded-millions-have-lost-everything}{%
\section{A Quarter of Bangladesh Is Flooded. Millions Have Lost
Everything.}\label{a-quarter-of-bangladesh-is-flooded-millions-have-lost-everything}}

The country's latest calamity illustrates a striking inequity of our
time: The people least responsible for climate change are among those
most hurt by its consequences.

\includegraphics{https://static01.graylady3jvrrxbe.onion/images/2020/07/30/climate/30CLI-BANGLADESH1/merlin_174670302_d722e08d-7dcb-4371-965a-6453e3508c4f-articleLarge.jpg?quality=75\&auto=webp\&disable=upscale}

By \href{https://www.nytimes3xbfgragh.onion/by/somini-sengupta}{Somini
Sengupta} and Julfikar Ali Manik

\begin{itemize}
\item
  July 30, 2020
\item
  \begin{itemize}
  \item
  \item
  \item
  \item
  \item
  \item
  \end{itemize}
\end{itemize}

Torrential rains have submerged at least a quarter of Bangladesh,
washing away the few things that count as assets for some of the world's
poorest people --- their goats and chickens, houses of mud and tin,
sacks of rice stored for the lean season.

It is the latest calamity to strike the delta nation of 165 million
people. Only two months ago,
\href{https://www.nytimes3xbfgragh.onion/2020/05/20/world/asia/cyclone-amphan-india-bangladesh.html}{a
cyclone pummeled the country's southwest}. Along the coast, a rising sea
has swallowed entire villages. And while it's too soon to ascertain what
role climate change has played in these latest floods, Bangladesh is
already witnessing a pattern of more severe and more frequent river
flooding than in the past along the mighty Brahmaputra River, scientists
say, and that is projected to worsen in the years ahead as climate
change intensifies the rains.

BHUTAN

NEPAL

INDIA

Kurigram

Rangpur

Flooded areas July 19-24

Usual bodies of water

Brahmaputra River

Jamalpur

Sylhet

Rajshahi

Tangail

Pabna

INDIA

Dhaka

Comilla

Jessore

BANGLADESH

Khulna

Kolkata

Chittagong

MYANMAR

100 KM

Bay of Bengal

60 MILES

BHUTAN

NEPAL

INDIA

Kurigram

Rangpur

Flooded areas July 19-24

Usual bodies of water

Brahmaputra River

Jamalpur

Sylhet

Rajshahi

Tangail

Pabna

INDIA

Dhaka

Comilla

Jessore

BANGLADESH

Khulna

Kolkata

Chittagong

MYANMAR

100 KM

Bay of Bengal

60 MILES

BHUTAN

INDIA

Kurigram

Rangpur

Flooded areas July 19-24

Usual bodies of water

Brahmaputra River

Jamalpur

Sylhet

Rajshahi

Tangail

Pabna

Dhaka

Comilla

INDIA

Jessore

BANGLADESH

Khulna

Chittagong

Kolkata

100 KM

Bay of Bengal

60 MILES

BHUTAN

NEPAL

INDIA

Kurigram

Rangpur

Flooded areas July 19-24

Usual bodies of water

Brahmaputra River

Jamalpur

Sylhet

Rajshahi

INDIA

Tangail

Pabna

Dhaka

Comilla

Jessore

BANGLADESH

Khulna

Kolkata

MYANMAR

Chittagong

100 KM

Bay of Bengal

60 MILES

By Blacki Migliozzi·Source: Institute of Water and Flood Management,
Bangladesh University of Engineering and Technology

``The suffering will go up,'' said Sajedul Hasan, the humanitarian
director of BRAC, an international development organization based in
Bangladesh that is distributing food, cash and liquid soap to displaced
people.

This is one of the most striking inequities of the modern era. Those who
are least responsible for polluting Earth's atmosphere are among those
most hurt by its consequences. The average American is responsible for
33 times more planet-warming carbon dioxide than the average
Bangladeshi.

This chasm has bedeviled diplomacy for a generation, and it is once
again in stark relief as the coronavirus pandemic upends the global
economy and threatens to push the world's most vulnerable people deeper
into ruin.

An estimated 24 to 37 percent of the country's landmass is submerged,
according to government estimates and satellite data By Tuesday,
\href{https://reliefweb.int/disaster/fl-2020-000161-bgd}{according to
the most recent figures available,} nearly a million homes were
inundated and 4.7 million people were affected. At least 54 have died,
most of them children.

The current floods, which are a result of intense rains upstream on the
Brahmaputra, could last through the middle of August. Until then, Taijul
Islam, a 30-year-old sharecropper whose house has washed away, will have
to camp out in a makeshift bamboo shelter on slightly higher ground. At
least he was able to salvage the tin sheet that was once the roof of his
house. Without it, he said, his extended family of nine would be exposed
to the elements.

Mr. Islam's predicament is multiplied by the millions among those on the
front lines of climate change. Vanuatu is literally sinking into the
Pacific. Pastoralists in the Horn of Africa are being pushed
\href{https://www.nytimes3xbfgragh.onion/2018/03/12/climate/kenya-drought.html}{to
the edge of survival} by back-to-back droughts. In the
\href{https://www.nytimes3xbfgragh.onion/interactive/2019/11/25/climate/india-monsoon-drought.html}{megacity
of Mumbai}, the rains come in terrifying cloudbursts.

\includegraphics{https://static01.graylady3jvrrxbe.onion/images/2020/07/30/climate/30CLI-BANGLADESH3/merlin_172705776_4f923d33-0b0d-4a53-8f5e-315af616a89a-articleLarge.jpg?quality=75\&auto=webp\&disable=upscale}

The inequity is striking, no matter which way you slice it. One recent
analysis found that the
\href{https://www.nature.com/articles/s41467-020-16941-y}{world's
richest 10 percent} are responsible for up to 40 percent of global
environmental damage, including climate change, while the poorest 10
percent account for less than 5 percent. Another estimated that warming
had
\href{https://www.nytimes3xbfgragh.onion/2019/04/22/climate/climate-change-global-wealth-gap.html}{reduced
incomes in the world's poorest countries} by between 17 percent and 30
percent.

\href{https://www.nytimes3xbfgragh.onion/section/climate?action=click\&pgtype=Article\&state=default\&region=MAIN_CONTENT_1\&context=storylines_keepup}{}

\hypertarget{climate-and-environment-}{%
\subsubsection{Climate and Environment
›}\label{climate-and-environment-}}

\hypertarget{keep-up-on-the-latest-climate-news}{%
\paragraph{Keep Up on the Latest Climate
News}\label{keep-up-on-the-latest-climate-news}}

Updated July 30, 2020

Here's what you need to know about the latest climate change news this
week:

\begin{itemize}
\item
  \begin{itemize}
  \tightlist
  \item
    \href{https://www.nytimes3xbfgragh.onion/2020/07/30/climate/bangladesh-floods.html?action=click\&pgtype=Article\&state=default\&region=MAIN_CONTENT_1\&context=storylines_keepup}{Floods
    in}\href{https://www.nytimes3xbfgragh.onion/2020/07/30/climate/bangladesh-floods.html?action=click\&pgtype=Article\&state=default\&region=MAIN_CONTENT_1\&context=storylines_keepup}{Bangladesh}
    are punishing the people least responsible for climate change.
  \item
    As climate change raises sea levels,
    \href{https://www.nytimes3xbfgragh.onion/2020/07/30/climate/sea-level-inland-floods.html?action=click\&pgtype=Article\&state=default\&region=MAIN_CONTENT_1\&context=storylines_keepup}{storm
    surges and high tides} are likely to push farther inland.
  \item
    The E.P.A. inspector general plans to investigate whether a rollback
    of fuel efficiency standards
    \href{https://www.nytimes3xbfgragh.onion/2020/07/27/climate/trump-fuel-efficiency-rule.html?action=click\&pgtype=Article\&state=default\&region=MAIN_CONTENT_1\&context=storylines_keepup}{violated
    government rules}.
  \end{itemize}
\end{itemize}

Poor countries have long sought a kind of reparations for what they call
loss and damage from climate change. Rich countries, led by the United
States and European Union, have resisted, mainly out of concern that
they could be saddled with liability claims for climate damage.

It doesn't help that the rich world has failed to deliver on a \$100
billion aid package to help poor countries cope, promised as part of the
2015 Paris accord.

Coronavirus recovery plans have lately begun to open the door to new
discussions about debt relief linked to climate resilience.

In June, the
\href{https://www.aosis.org/wp-content/uploads/2020/07/AOSIS-Statement-on-Debt_verJune-29.pdf}{Alliance
of Small Island Developing States}, led by Belize, pressed for what it
called a new compact with private and bilateral creditors ``to deliver
debt relief and increase resilience financing.''

Caribbean countries, with their economies ravaged by hurricanes in
recent years, now find themselves falling deeper into debt as the
pandemic dries up tourism revenues. A
\href{https://unfccc.int/news/climate-change-is-driving-debt-for-developing-countries}{study
commissioned by the United Nations} found that the 20 most
climate-vulnerable countries have paid more than \$40 billion in
additional interest payments because of losses stemming from extreme
weather events.

In Bangladesh, the flooding of the Brahmaputra reflects the unequal pain
of extreme weather.

The floods began in June. In most cases, heavy rains upstream in
neighboring India swelled the river basins that flow through Bangladesh
before draining into the Bay of Bengal. Those who live along the
Brahmaputra are no strangers to flooding. When the river swells, work
stops, land erodes, people move to higher ground and wait for the waters
to recede. They rely on their savings or aid to feed themselves.

This year was different, though. By the time the river flooded, in June,
people were already running out of food, said Mr. Hasan of BRAC.

Because of the lockdown, working people had all but stopped working.
\href{https://www.nytimes3xbfgragh.onion/2020/07/27/business/global-remittances-coronavirus.html}{Remittances
from relatives abroad}, many of them newly unemployed, had dried up. In
the countryside, people had begun to sell their goats and cattle at
bargain prices. They had no food to eat.

When the river first swelled, Taijul Islam, the sharecropper from the
Kurigram district in the country's north, rushed to save his livestock
--- cattle, goats, chickens, ducks. A few, he rescued. Many, he lost.
The river took away the small vegetable garden next to his house, then
his house, where he had stashed roughly 1,300 pounds of rice. Then it
washed away a small shop that he ran when he wasn't working on other
people's land. Also the school that his 6-year-old son attended in the
village.

Image

Dhaka, the Bangladeshi capital, in late July. The damage from this
year's flooding has been compounded by the global coronavirus
pandemic.Credit...Munir Uz Zaman/Agence France-Presse --- Getty Images

All he can think of now is where he can go to earn a living. He is the
sole breadwinner of his extended family. All nine of them had been
living on rice, boiled potato and lentils. Vegetables are unaffordable,
let alone fish or meat, which, he said, ``are now unimaginable.''

Akkas Ali, 48, had already been through a bad flood. He moved to Mr.
Islam's village six years ago, when his old village washed into the
Brahmaputra. Two weeks ago, as the river rose, breaking through its
embankments, his four acres of farmland went underwater. The village
mosque and market washed away. So, too, a secondary school where more
than 250 children were enrolled. Mr. Ali worried where they would go to
school now, if at all.

His house still stood this week, but the river, which had been a quarter
mile away, had rushed dangerously close. He was sure it, too, would wash
away soon.

The Brahmaputra is a fearsome, shape-shifting 2,400-mile river that
erupts from the Tibetan Himalayas and spills into northeastern India
before merging with the Ganges in Bangladesh and emptying into the Bay
of Bengal. It irrigates vast areas of farmland but it's also
unpredictable, often swallowing the islands that form within it, like
the one where Mr. Ali's village once stood.

Climate change, too, is altering its fate --- and that of the people who
live along its banks. The rains are more unpredictable and the river is
rising above dangerous levels far more frequently than it did before,
according to 35 years of flooding data analyzed by A.K.M. Saiful Islam,
a water management expert at the Bangladesh University of Engineering
and Technology in Dhaka.

The last five years alone have brought four major floods, eroding
people's capacity to adapt, Dr. Islam said.

More and worse floods loom.

Even if average global temperature increase modestly --- by 2 degrees
Celsius over the average for preindustrial times --- flooding along the
\href{https://ascelibrary.org/doi/abs/10.1061/\%28ASCE\%29HE.1943-5584.0001705}{Brahmaputra
in Bangladesh is projected to increase} by 24 percent. With an increase
of 4 degrees Celsius, flooding is projected to increase by over 60
percent.

No matter what, Dr. Islam said, the country will have to adapt. That
requires money to dredge rivers, maintain embankments, improve drainage
and offer aid to those who are repeatedly displaced and impoverished.

Advocates for the poor say Bangladesh's predicament with disasters
illustrates exactly why climate negotiations, postponed until 2021, need
to deliver compensation for people who have not caused the problem.

``People are losing whatever little they have,'' said Farah Kabir, the
Bangladesh country director for ActionAid International. ``When and how
are they going to be supported? When is the global community going to
take responsibility?''

Somini Sengupta reported from New York, and Julfikar Ali Manik from
Dhaka, Bangladesh.

Advertisement

\protect\hyperlink{after-bottom}{Continue reading the main story}

\hypertarget{site-index}{%
\subsection{Site Index}\label{site-index}}

\hypertarget{site-information-navigation}{%
\subsection{Site Information
Navigation}\label{site-information-navigation}}

\begin{itemize}
\tightlist
\item
  \href{https://help.nytimes3xbfgragh.onion/hc/en-us/articles/115014792127-Copyright-notice}{©~2020~The
  New York Times Company}
\end{itemize}

\begin{itemize}
\tightlist
\item
  \href{https://www.nytco.com/}{NYTCo}
\item
  \href{https://help.nytimes3xbfgragh.onion/hc/en-us/articles/115015385887-Contact-Us}{Contact
  Us}
\item
  \href{https://www.nytco.com/careers/}{Work with us}
\item
  \href{https://nytmediakit.com/}{Advertise}
\item
  \href{http://www.tbrandstudio.com/}{T Brand Studio}
\item
  \href{https://www.nytimes3xbfgragh.onion/privacy/cookie-policy\#how-do-i-manage-trackers}{Your
  Ad Choices}
\item
  \href{https://www.nytimes3xbfgragh.onion/privacy}{Privacy}
\item
  \href{https://help.nytimes3xbfgragh.onion/hc/en-us/articles/115014893428-Terms-of-service}{Terms
  of Service}
\item
  \href{https://help.nytimes3xbfgragh.onion/hc/en-us/articles/115014893968-Terms-of-sale}{Terms
  of Sale}
\item
  \href{https://spiderbites.nytimes3xbfgragh.onion}{Site Map}
\item
  \href{https://help.nytimes3xbfgragh.onion/hc/en-us}{Help}
\item
  \href{https://www.nytimes3xbfgragh.onion/subscription?campaignId=37WXW}{Subscriptions}
\end{itemize}
