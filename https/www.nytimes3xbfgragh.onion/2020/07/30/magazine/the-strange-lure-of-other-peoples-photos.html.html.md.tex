The Strange Lure of Other People's Photos

\url{https://nyti.ms/3jQ9FY6}

\begin{itemize}
\item
\item
\item
\item
\item
\end{itemize}

\href{https://www.nytimes3xbfgragh.onion/spotlight/at-home?action=click\&pgtype=Article\&state=default\&region=TOP_BANNER\&context=at_home_menu}{At
Home}

\begin{itemize}
\tightlist
\item
  \href{https://www.nytimes3xbfgragh.onion/2020/08/03/well/family/the-benefits-of-talking-to-strangers.html?action=click\&pgtype=Article\&state=default\&region=TOP_BANNER\&context=at_home_menu}{Talk:
  To Strangers}
\item
  \href{https://www.nytimes3xbfgragh.onion/2020/08/01/at-home/coronavirus-make-pizza-on-a-grill.html?action=click\&pgtype=Article\&state=default\&region=TOP_BANNER\&context=at_home_menu}{Make:
  Grilled Pizza}
\item
  \href{https://www.nytimes3xbfgragh.onion/2020/07/31/arts/television/goldbergs-abc-stream.html?action=click\&pgtype=Article\&state=default\&region=TOP_BANNER\&context=at_home_menu}{Watch:
  'The Goldbergs'}
\item
  \href{https://www.nytimes3xbfgragh.onion/interactive/2020/at-home/even-more-reporters-editors-diaries-lists-recommendations.html?action=click\&pgtype=Article\&state=default\&region=TOP_BANNER\&context=at_home_menu}{Explore:
  Reporters' Google Docs}
\end{itemize}

\includegraphics{https://static01.graylady3jvrrxbe.onion/images/2020/02/08/magazine/02mag-vernacular-photos-image/02mag-vernacular-photos-image-mediumSquareAt3X.jpg}

Sections

\protect\hyperlink{site-content}{Skip to
content}\protect\hyperlink{site-index}{Skip to site index}

Letter of Recommendation

\hypertarget{the-strange-lure-of-other-peoples-photos}{%
\section{The Strange Lure of Other People's
Photos}\label{the-strange-lure-of-other-peoples-photos}}

When I consider all that these people lived through --- world wars, the
Depression, epidemics with no medicine --- I'm given a far longer view.

Credit...

Supported by

\protect\hyperlink{after-sponsor}{Continue reading the main story}

By Bill Shapiro

\begin{itemize}
\item
  July 30, 2020
\item
  \begin{itemize}
  \item
  \item
  \item
  \item
  \item
  \end{itemize}
\end{itemize}

My phone is so stuffed with photos that it takes me, no lie, 143
vigorous thumb flicks --- past 41,407 pictures --- to scroll back
through time to its very first snap, a ho-hum shot of music-festival
fans under a wide Texas sky. I've got plenty more elsewhere. Everyone
does: on hard drives, backed up in the cloud, hung on walls and perched
on shelves, scrambled in shoe boxes, sorted in old-school albums.

We're all drowning in our own pictures --- last year, we humans took an
estimated 1.3 \emph{trillion} of them. I keep pictures that I never look
at (an acrobatic squirrel), others I look at immediately after I take
them but rarely if ever again (rooftop sunset), some I flip to often
(children, girlfriend).

And then there are the photos I reach for, with intention, a couple of
times a year, when I find myself needing to look at life with different
eyes.

\includegraphics{https://static01.graylady3jvrrxbe.onion/images/2020/08/02/magazine/02mag-lor-1/02mag-lor-1-articleLarge.jpg?quality=75\&auto=webp\&disable=upscale}

Photos like this one: an older couple smack in the center of the driest,
dustiest, emptiest parcel of land you've ever seen. It's probably a
hundred years old, one corner bent, its surface slightly faded by time.
He sits, she stands, and both look ticked off. I do not know these
people.

Image

Credit...Photograph by Horacio Salinas

And this: A large family gathers at a long table in the kind of moment
people build beer ads around. It's golden hour, and the sun filters
through the trees, the windows, the half-full pitchers. This might be
the 1930s, and yet you can practically hear the clinking, the laughing.

These pictures, taken by average people with average cameras, are among
the thousand or so that I've picked up at flea markets, junk shops,
garage sales and, once in a while, on eBay. I started noticing these
``found'' photos (the fancy name is ``vernacular photography'') maybe a
decade ago. \emph{Noticing} turned to \emph{looking} turned to
\emph{hunting}. Somehow, I've accumulated enough photos of long-dead
people that I have no connection with to stuff a dozen slate gray
11-by-17-by-3.5 archival boxes.

I'm particularly drawn to quietly composed pictures that hold the sense
of an unfinished story. Exhibit A: that older couple staring out from
that dusty landscape. Why is there a chair in a spot where it appears
nothing else exists for miles? I looked at the picture a few times
before I noticed that thin sticks sprout from the ground. Were they
homesteaders? Would a town grow here? A city? Someone posed them here,
so the moment or place held some importance. But what?

Not every shot is so mysterious. I have photos from the 1920s of people
doing pretty much the same things we do today: drinking booze, kissing,
cross-dressing, picnicking by a pond, holding their children in the air
with a love so fierce you can feel it a hundred years later.

Of course, those children are gone now. As is everyone sitting around
that long table as the sun sets and the glasses clink. They lived,
worked, made their share of bad decisions, loved a bunch and surely
suffered some. But these ``everyday'' photos haunt me for the simple
reason that I have pictures just like them, where I am the full-eyed
father stretching my own children toward a brilliant blue sky. Images
like this hang all over my house, reminding me of moments when my heart
felt full to bursting. I love these pictures.

I also hate them. They remind me of time going by. They remind me of
what I had and what's gone. These pictures warn me how fast and fragile
those moments are. There's my son learning to ride a tricycle; as I
write this, he's driving across the country with his girlfriend.
Probably speeding. Get out of the way, these pictures say; something new
is coming. They leave me wobbly, unsure whether to look forward or back.

Which is why at moments of uncertainty and confusion, I turn to my gray
boxes of found photos. When it looked as if Covid-19 would swallow New
York, I pulled a box off the shelf. ``I need to categorize the new
finds,'' I told my girlfriend. She arched her eyebrows. Even I didn't
buy that line. Those hundred-year-old photos center me. They give me
something that my own photos don't. When I look at the found photos and
consider all that these people lived through --- world wars, the
Depression, epidemics with no medicine, loss and hardship I can hardly
grasp --- I'm given a far longer view. They take me out of myself, make
my pangs of the heart feel less about me and more about all of us.

I get emotional when I look at them, but not in the same way as I do the
photos of my children. With my own photos, I hear the fast ticking of
the secondhand. The old pictures keep a more steady time: humanity's
slow and sweeping waltz.

It's not lost on me that the only reason I'm able to pluck these
beautiful images from some forlorn flea-market bin and meditate on the
lives that came before mine is that they were discarded. Did the younger
generation not recognize that child on the porch as their great-grandma?
Did they know but not care? And then this question arises: Will I be the
last person on Earth to ever see her face?

The neuroscientist and author David Eagleman has written that we all die
three deaths: ``The first is when the body ceases to function. The
second is when the body is consigned to the grave. The third is that
moment, sometime in the future, when your name is spoken for the last
time.'' I would say there's a fourth: the moment the last remaining
picture of you is seen for the final time. These found photographs not
only remind me of this delicate thing we run both toward and away from
--- time --- but they also hold something else. The humbling, steadying
truth that, one day, that's all we'll be: a photo.

Image

Credit...Photograph by Horacio Salinas

\begin{center}\rule{0.5\linewidth}{\linethickness}\end{center}

Bill Shapiro is a former editor in chief of Life magazine and a
co-author of ``What We Keep.''

Photographs from the collection of Bill Shapiro

Advertisement

\protect\hyperlink{after-bottom}{Continue reading the main story}

\hypertarget{site-index}{%
\subsection{Site Index}\label{site-index}}

\hypertarget{site-information-navigation}{%
\subsection{Site Information
Navigation}\label{site-information-navigation}}

\begin{itemize}
\tightlist
\item
  \href{https://help.nytimes3xbfgragh.onion/hc/en-us/articles/115014792127-Copyright-notice}{©~2020~The
  New York Times Company}
\end{itemize}

\begin{itemize}
\tightlist
\item
  \href{https://www.nytco.com/}{NYTCo}
\item
  \href{https://help.nytimes3xbfgragh.onion/hc/en-us/articles/115015385887-Contact-Us}{Contact
  Us}
\item
  \href{https://www.nytco.com/careers/}{Work with us}
\item
  \href{https://nytmediakit.com/}{Advertise}
\item
  \href{http://www.tbrandstudio.com/}{T Brand Studio}
\item
  \href{https://www.nytimes3xbfgragh.onion/privacy/cookie-policy\#how-do-i-manage-trackers}{Your
  Ad Choices}
\item
  \href{https://www.nytimes3xbfgragh.onion/privacy}{Privacy}
\item
  \href{https://help.nytimes3xbfgragh.onion/hc/en-us/articles/115014893428-Terms-of-service}{Terms
  of Service}
\item
  \href{https://help.nytimes3xbfgragh.onion/hc/en-us/articles/115014893968-Terms-of-sale}{Terms
  of Sale}
\item
  \href{https://spiderbites.nytimes3xbfgragh.onion}{Site Map}
\item
  \href{https://help.nytimes3xbfgragh.onion/hc/en-us}{Help}
\item
  \href{https://www.nytimes3xbfgragh.onion/subscription?campaignId=37WXW}{Subscriptions}
\end{itemize}
