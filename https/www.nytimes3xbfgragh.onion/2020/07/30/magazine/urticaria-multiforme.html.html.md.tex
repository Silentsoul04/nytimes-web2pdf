Sections

SEARCH

\protect\hyperlink{site-content}{Skip to
content}\protect\hyperlink{site-index}{Skip to site index}

\href{https://myaccount.nytimes3xbfgragh.onion/auth/login?response_type=cookie\&client_id=vi}{}

\href{https://www.nytimes3xbfgragh.onion/section/todayspaper}{Today's
Paper}

His Face, Hands and Feet Swelled Up. What Was Going On?

\url{https://nyti.ms/2XqZD6n}

\begin{itemize}
\item
\item
\item
\item
\item
\item
\end{itemize}

Advertisement

\protect\hyperlink{after-top}{Continue reading the main story}

Supported by

\protect\hyperlink{after-sponsor}{Continue reading the main story}

\href{/column/diagnosis}{Diagnosis}

\hypertarget{his-face-hands-and-feet-swelled-up-what-was-going-on}{%
\section{His Face, Hands and Feet Swelled Up. What Was Going
On?}\label{his-face-hands-and-feet-swelled-up-what-was-going-on}}

\includegraphics{https://static01.graylady3jvrrxbe.onion/images/2020/08/02/magazine/02mag-diagnosis-1/02mag-diagnosis-1-articleLarge.jpg?quality=75\&auto=webp\&disable=upscale}

By \href{https://www.nytimes3xbfgragh.onion/by/lisa-sanders-md}{Lisa
Sanders, M.D.}

\begin{itemize}
\item
  July 30, 2020
\item
  \begin{itemize}
  \item
  \item
  \item
  \item
  \item
  \item
  \end{itemize}
\end{itemize}

The 5-year-old boy lay in his grandmother's arms. ``My feet, my feet,''
he wailed. She smoothed the tangled bangs away from his forehead then
reached for her phone. She wasn't sure what was going on with her
grandson, but she had cared for enough children and grandchildren to
know that something was really wrong. His feet were swollen and painful
and so were his hands. On his legs there was a rash that exploded over
the several hours he was at her house into a series of raised red rings.
She called her son and daughter-in-law. The little boy needed to go to
the hospital.

He'd always been delicate, his mother explained to the triage nurse at
the E.R. of Lowell General Hospital in Lowell, Mass. If one of his
brothers had a runny nose, he'd get a runny nose and a fever. He was
sick so often that she'd once taken him to the pediatrician to see if
there was anything wrong with his immune system. The pediatrician
assured her that it was a typical response to starting kindergarten,
when first-time students would catch all the viruses going around. Just
the previous week he had an awful cold. But he'd never had anything like
this.

\hypertarget{perhaps-an-infection}{%
\subsection{\texorpdfstring{\textbf{Perhaps an
Infection}}{Perhaps an Infection}}\label{perhaps-an-infection}}

It started with a stomachache the night before. The boy was up for
hours. Then in the early morning, when his father took him to the
bathroom, he noticed that the tip of the boy's penis was red and
inflamed. As soon as his wife awakened, he told her about the situation
and took him to the hospital.

The child was seen right away. He didn't have a fever and was drinking
well, though reluctant to eat. He said he didn't have a stomachache
anymore, and his belly wasn't tender. His penis, though inflamed, didn't
seem to hurt, either. But his white-blood-cell count was elevated, and
that suggested the child had an infection. The E.R. doctor figured he
probably had balanitis, an infection of the head of the penis, and gave
him a week of an oral antibiotic.

When the boy was discharged from the hospital late that morning, the
father called the boy's grandmother. Could she look after him? He and
his wife had to work, and the child and his grandmother were very close.

When they arrived at the home of his Lela --- that's how the boy
pronounced abuela, the Spanish word for grandmother --- he didn't want
to walk. He held up his arms in that familiar wordless plea: Carry me.
When his father picked him up and took him to his grandmother, she
received him happily, and he snuggled into her arms.

\hypertarget{scary-swelling}{%
\subsection{\texorpdfstring{\textbf{Scary
Swelling}}{Scary Swelling}}\label{scary-swelling}}

He slept for much of the afternoon but ate and played when he was awake.
In the evening, the boy seemed feverish --- hot and uncomfortable. His
grandmother gave him a bath. As she washed him, she saw what looked like
a couple of bites on his legs. When it was time to get out of the tub,
he refused to stand up. He said his feet hurt. And when his abuela
looked, his feet were puffy. His hands, too, seemed larger than normal,
as if they had been blown up like balloons. Even his face looked swollen
--- his eyelids were distended, and his lips seemed enormous. And what
she thought were bites were now strange red lines that snaked around his
legs onto his thighs and belly. She called her daughter-in-law.
Something's not right, she told her. He needs to go back to the
hospital.

The mother left work and hurried to her mother-in-law's house. To her
maternal eye, her little boy looked like a monster --- he was so
strangely swollen.

\includegraphics{https://static01.graylady3jvrrxbe.onion/images/2020/08/02/magazine/02mag-diagnosis-2/02mag-diagnosis-2-articleLarge.jpg?quality=75\&auto=webp\&disable=upscale}

She snapped him into his car seat and sped off to the hospital. She
carried the boy to the front desk; she was worried that the swelling of
his lips could extend and cut off his airway, she told the triage nurse.
The nurse was concerned, too, and they quickly took him into the busy
emergency room.

\hypertarget{a-dangerous-possibility}{%
\subsection{\texorpdfstring{\textbf{A Dangerous
Possibility}}{A Dangerous Possibility}}\label{a-dangerous-possibility}}

The child was cranky and didn't want to be examined. Both of his feet
were obviously swollen. So were his hands and face. He cried when anyone
tried to move his hands and feet. And he had this itchy, spreading rash.
The E.R. doctor worried about a rare but potentially dangerous disease
of the smallest blood vessels called Henoch-Schönlein Purpura (H.S.P.).
(It has recently been renamed IgA vasculitis.) An autoimmune disorder,
it primarily affects children and is characterized by a triad of
symptoms: abdominal pain, arthritis and a rash.

The rash is caused by the affected vessels leaking blood into the
surrounding tissues. When this happens close to the surface of the skin,
it creates dark red or bruise-colored spots. When these lesions are
pressed, they don't change color. But the boy's rash was a light, not
dark color, and when touched those spots lost the pink coloring and
became pale. H.S.P. also causes a rise in inflammatory markers in the
blood, so the doctor ordered a test to look for that as well. Those
markers were slightly elevated, and so to be safe, the E.R. doctor
decided to admit the child to the hospital to be monitored overnight.
She ordered ibuprofen for the pain and an antihistamine for the itch.

\hypertarget{finding-a-new-diagnosis}{%
\subsection{\texorpdfstring{\textbf{Finding a New
Diagnosis}}{Finding a New Diagnosis}}\label{finding-a-new-diagnosis}}

The next morning, the boy was seen by Dr. Krista Birnie, a pediatric
hospitalist. By then he looked more comfortable, though he still didn't
want anyone to touch his swollen hands or feet. It probably wasn't
H.S.P., Birnie decided, now that she saw the child. Although the boy
didn't want to move his hands or feet, it looked to her as if that was
because of the swelling rather than some form of arthritis, a symptom of
H.S.P. And the pink circles of rash had not turned into the angry red
blotches characteristic of H.S.P.

The child's parents reported that the rash started out looking like
hives, which then grew into expanding circles and faded away --- only to
be replaced by others. The E.R. doctor had also ordered a test for Lyme
disease. There was a field behind the boy's house where he often played,
and it could harbor deer ticks. Birnie didn't think this looked like the
typical Lyme rash, though Lyme disease was a common infection in
northern Massachusetts.

\hypertarget{an-allergic-reaction}{%
\subsection{\texorpdfstring{\textbf{An Allergic
Reaction}}{An Allergic Reaction}}\label{an-allergic-reaction}}

Birnie was thinking along different lines. This child had first come
into the hospital with a lesion on his penis. Just 12 hours later that
lesion was gone. And the boy's parents noted that his lesions were
transient like hives. There was a disorder Birnie had heard of but never
seen called Urticaria multiforme (U.M.). It's a type of allergic
reaction, she remembered, seen in kids usually after a viral infection,
but Birnie couldn't recall more than that. After her rounds, she found a
computer and looked up the disorder. The description fit this child
perfectly. It's an allergic response triggered by either a medication or
more commonly a viral infection. This child had a cold the week before.

The allergic reaction causes a release of histamine, a body chemical
that causes (among other things) plasma to leak out of blood vessels,
causing hives and the characteristic swelling of the feet, hands and
face. It is usually treated with antihistamines --- to block the
histamine. The boy had an antihistamine on admission because he felt
itchy. There's no test for Urticaria multiforme, but given his symptoms,
Birnie felt confident he had it. Still, she wanted to watch him one more
day.

\hypertarget{slowly-back-to-normal}{%
\subsection{\texorpdfstring{\textbf{Slowly Back to
Normal}}{Slowly Back to Normal}}\label{slowly-back-to-normal}}

The next morning the child was a little less swollen, though he
continued to get new hives. The Lyme test hadn't come back, but Birnie
was pretty sure that when it did, it would be negative. The child had a
follow-up appointment with his primary doctor a couple of days later.
Birnie called a few days after that to make sure the boy was feeling
better. He was. And she was right --- the Lyme test was in fact
negative.

To the boy's mother, this was just one more illness experienced by her
delicate child. But it made an impression on him. Three years later,
when he sometimes tells his mother that his feet hurt, she knows that
means he's feeling sick.

Advertisement

\protect\hyperlink{after-bottom}{Continue reading the main story}

\hypertarget{site-index}{%
\subsection{Site Index}\label{site-index}}

\hypertarget{site-information-navigation}{%
\subsection{Site Information
Navigation}\label{site-information-navigation}}

\begin{itemize}
\tightlist
\item
  \href{https://help.nytimes3xbfgragh.onion/hc/en-us/articles/115014792127-Copyright-notice}{©~2020~The
  New York Times Company}
\end{itemize}

\begin{itemize}
\tightlist
\item
  \href{https://www.nytco.com/}{NYTCo}
\item
  \href{https://help.nytimes3xbfgragh.onion/hc/en-us/articles/115015385887-Contact-Us}{Contact
  Us}
\item
  \href{https://www.nytco.com/careers/}{Work with us}
\item
  \href{https://nytmediakit.com/}{Advertise}
\item
  \href{http://www.tbrandstudio.com/}{T Brand Studio}
\item
  \href{https://www.nytimes3xbfgragh.onion/privacy/cookie-policy\#how-do-i-manage-trackers}{Your
  Ad Choices}
\item
  \href{https://www.nytimes3xbfgragh.onion/privacy}{Privacy}
\item
  \href{https://help.nytimes3xbfgragh.onion/hc/en-us/articles/115014893428-Terms-of-service}{Terms
  of Service}
\item
  \href{https://help.nytimes3xbfgragh.onion/hc/en-us/articles/115014893968-Terms-of-sale}{Terms
  of Sale}
\item
  \href{https://spiderbites.nytimes3xbfgragh.onion}{Site Map}
\item
  \href{https://help.nytimes3xbfgragh.onion/hc/en-us}{Help}
\item
  \href{https://www.nytimes3xbfgragh.onion/subscription?campaignId=37WXW}{Subscriptions}
\end{itemize}
