Sections

SEARCH

\protect\hyperlink{site-content}{Skip to
content}\protect\hyperlink{site-index}{Skip to site index}

\href{https://www.nytimes3xbfgragh.onion/section/opinion/sunday}{Sunday
Review}

\href{https://myaccount.nytimes3xbfgragh.onion/auth/login?response_type=cookie\&client_id=vi}{}

\href{https://www.nytimes3xbfgragh.onion/section/todayspaper}{Today's
Paper}

\href{/section/opinion/sunday}{Sunday Review}\textbar{}The Last Days of
the Tech Emperors?

\url{https://nyti.ms/39NhJ7c}

\begin{itemize}
\item
\item
\item
\item
\item
\end{itemize}

Advertisement

\protect\hyperlink{after-top}{Continue reading the main story}

\href{/section/opinion}{Opinion}

Supported by

\protect\hyperlink{after-sponsor}{Continue reading the main story}

\hypertarget{the-last-days-of-the-tech-emperors}{%
\section{The Last Days of the Tech
Emperors?}\label{the-last-days-of-the-tech-emperors}}

Congress was once filled with ``Atari Democrats.'' This week's hearings
showed their transformation into trust busters.

\includegraphics{https://static01.graylady3jvrrxbe.onion/images/2019/08/09/opinion/margaret-omara/margaret-omara-thumbLarge.jpg}

By Margaret O'Mara

Contributing Opinion Writer

\begin{itemize}
\item
  July 30, 2020
\item
  \begin{itemize}
  \item
  \item
  \item
  \item
  \item
  \end{itemize}
\end{itemize}

\includegraphics{https://static01.graylady3jvrrxbe.onion/images/2020/07/30/opinion/30OMara1-Sub/30OMara1-Sub-articleLarge-v2.jpg?quality=75\&auto=webp\&disable=upscale}

On Wednesday, Representative David Cicilline, Democrat of Rhode Island
and chairman of the House Judiciary Committee's antitrust subcommittee,
opened a half-virtual hearing on ``Online Platforms and Market Power''
with a combative opening statement: ``Our founders would not bow before
a king. Nor should we bow before the emperors of the online economy.''

That set the tone for the hours of sharp questioning of four of the
wealthiest people on the planet: Jeff Bezos of Amazon, Tim Cook of
Apple, Sundar Pichai of Google and Mark Zuckerberg of Facebook, whose
companies have a combined market value roughly equivalent to the
\href{https://data.worldbank.org/indicator/NY.GDP.MKTP.CD?locations=JP}{G.D.P.
of Japan}.

Given the history of Silicon Valley's relationship with Washington, the
intensity and precision of some subcommittee members' questions were
remarkable. It is a sign that significant tech regulation may be closer
than we think.

Despite its techno-libertarian image, the tech industry has had close
political ties for decades and remarkable success in getting what it
wants.

In the late 1970s, venture capitalists and semiconductor chief
executives\href{https://books.google.com/books/about/H_R_9549_the_Capital_Investment_and_Busi.html?id=tG3SAAAAMAAJ}{got}
Capitol Hill and the Carter White House to agree to tax cuts and looser
financial regulations. In the 1980s, a group of young legislators became
such boosters of the industry that they were known as
``\href{https://www.nytimes3xbfgragh.onion/1982/09/26/business/designing-a-new-economics-for-the-atari-democrats.html}{Atari
Democrats}.'' Ronald Reagan extolled Silicon Valley entrepreneurship and
helped tech companies fend off Japanese competition.

The bipartisan love affair intensified in the 1990s as Bill Clinton and
Al Gore invited tech executives to shape early internet-era
policymaking. Newt Gingrich, then the Republican speaker of the House,
\href{https://www.nytimes3xbfgragh.onion/1995/01/17/opinion/mr-gingrichs-cyberrevolution.html?searchResultPosition=1}{talked
up cyberspace} and formed close alliances with libertarian-minded tech
thinkers. His party's leaders convened
``\href{https://www.nytimes3xbfgragh.onion/1999/06/16/business/gates-testifies-and-the-republicans-seek-to-forge-closer-links.html?searchResultPosition=24}{high-tech
summits}'' on Capitol Hill.

The lightly regulated online economy we have today is a product of that
decade, when Silicon Valley leaders persuaded starry-eyed lawmakers that
young, scrappy internet companies could regulate themselves.

Washington's embrace of tech continued even as questions emerged about
the industry's wealth and power. A 2013
\href{https://www.nytimes3xbfgragh.onion/2013/05/22/technology/ceo-denies-that-apple-is-avoiding-taxes.html}{Senate
hearing} to interrogate Mr. Cook about Apple's tax avoidance quickly was
sidetracked by lawmakers gushing to the chief executive about his
company's innovative products. Mr. Pichai faced tough questions at a
2018 House Judiciary hearing, but also was showered with praise.

``Google is still the story of the American dream,''
\href{https://www.nytimes3xbfgragh.onion/2018/12/11/business/sundar-pichai-google-house-hearing.html}{declared}
Representative Robert W. Goodlatte of Virginia, the committee's chairman
at the time.

Those days seemed a dim memory Wednesday. Instead, the mood recalled the
traffic safety debates of the mid-1960s that helped catalyze
significantly more regulation for the auto industry. After a steady
drumbeat of studies and some short-lived congressional inquiries,
traffic safety exploded into the public consciousness starting with
Senate hearings in
\href{https://www.google.com/books/edition/Federal_Role_in_Traffic_Safety/Kk1KAQAAIAAJ?hl=en\&gbpv=0}{the
summer of 1965}, where top auto executives faced sharp questions about
their lax approach to safety.

The evening network news programs showed Robert F. Kennedy, a newly
elected senator from New York, grilling the leaders of General Motors
about the tiny amount the company spent on safety research. Later that
year a young lawyer advising the Senate committee, Ralph Nader,
published a blockbuster exposé of the industry, ``Unsafe at Any Speed.''

This combination of political and media scrutiny led to passage of the
National Traffic and Motor Vehicle Safety Act of 1966, which mandated
seatbelts and additional car safety features, as well as road
improvements like guardrails and traffic barriers.

Wednesday felt like Big Tech's Ralph Nader moment: the pointed
questioning by committee members, notably its Democratic women like
Representatives Val Demings of Florida, Pramila Jayapal of Washington,
Lucy McBath of Georgia and Mary Gay Scanlon of Pennsylvania; the
crescendo of investigative journalism that, in part, led to this week's
hearing by shining a critical light on Big Tech's practices. And now,
this House subcommittee is merely one of several legislative or
regulatory bodies considering limits on Big Tech's power.

Image

Val Demings of Florida.Credit...Pool photo by Mandel Ngan

Image

Pramila Jayapal of Washington.Credit...Pool photo by Graeme Jennings

Image

Lucy McBath of Georgia.Credit...Poll Photo by Greg Nash

Image

Mary Gay Scanlon of Pennsylvania.Credit...Pool photo by Mandel Ngan

There are of course many reasons tech regulation may not come to pass.
The issues at stake are wickedly complex, and quite different for each
of these companies, something chief executives sought to underscore in
the hearing.

``It appears to me,'' Mr. Bezos observed, ``that social media is a
nuance-destruction machine, and I don't think that's helpful for a
democracy.'' (Mr. Zuckerberg's reaction to that statement sadly was not
visible to the audience.)

Large tech companies also have prepared for the regulatory onslaught by
starting some of the most well-funded lobbying operations in Washington.
They learned a lesson from Microsoft, whose presence in the capital
before its antitrust case in 1998 consisted of one employee who worked
out of the back of his car because he lacked proper office space.

Although the trial didn't end with Microsoft being ordered to break
itself apart, it taught the company that government regulators needed to
be taken seriously. And as a result Microsoft tamped down its most
aggressive market practices, and escaped much of the yearslong policy
scrutiny now facing its peers.

Then there is the sticky problem of public opinion. During other seminal
moments --- carmakers in the 1960s, tobacco in the 1990s --- the
problems posed by
\href{https://www.nytimes3xbfgragh.onion/2018/11/10/opinion/sunday/fascism-economy-monopoly.html}{unregulated
bigness} were clear-cut. Cigarettes killed people. Cars were unsafe.

Tech's consumer dangers are harder to see and acutely feel on an average
day: misinformation, an incomplete search result, an unfairly promoted
link, privacy erosion, a skewed algorithm. We may wish we used our
smartphones less, or worry about what overuse of social media is doing
to our communities and brains.

But we still routinely check our Facebook pages, buy apps via Apple, and
click ``buy'' on Amazon Prime. Even if, as some representatives noted,
we do so because we have little alternative.

What happens next will depend on many things, including the November
election. But this week marks the end of Washington's great love affair
with tech, one that helped make these companies' bigness possible in the
first place.

\emph{The Times is committed to publishing}
\href{https://www.nytimes3xbfgragh.onion/2019/01/31/opinion/letters/letters-to-editor-new-york-times-women.html}{\emph{a
diversity of letters}} \emph{to the editor. We'd like to hear what you
think about this or any of our articles. Here are some}
\href{https://help.nytimes3xbfgragh.onion/hc/en-us/articles/115014925288-How-to-submit-a-letter-to-the-editor}{\emph{tips}}\emph{.
And here's our email:}
\href{mailto:letters@NYTimes.com}{\emph{letters@NYTimes.com}}\emph{.}

\emph{Follow The New York Times Opinion section on}
\href{https://www.facebookcorewwwi.onion/nytopinion}{\emph{Facebook}}\emph{,}
\href{http://twitter.com/NYTOpinion}{\emph{Twitter (@NYTopinion)}}
\emph{and}
\href{https://www.instagram.com/nytopinion/}{\emph{Instagram}}\emph{.}

Advertisement

\protect\hyperlink{after-bottom}{Continue reading the main story}

\hypertarget{site-index}{%
\subsection{Site Index}\label{site-index}}

\hypertarget{site-information-navigation}{%
\subsection{Site Information
Navigation}\label{site-information-navigation}}

\begin{itemize}
\tightlist
\item
  \href{https://help.nytimes3xbfgragh.onion/hc/en-us/articles/115014792127-Copyright-notice}{©~2020~The
  New York Times Company}
\end{itemize}

\begin{itemize}
\tightlist
\item
  \href{https://www.nytco.com/}{NYTCo}
\item
  \href{https://help.nytimes3xbfgragh.onion/hc/en-us/articles/115015385887-Contact-Us}{Contact
  Us}
\item
  \href{https://www.nytco.com/careers/}{Work with us}
\item
  \href{https://nytmediakit.com/}{Advertise}
\item
  \href{http://www.tbrandstudio.com/}{T Brand Studio}
\item
  \href{https://www.nytimes3xbfgragh.onion/privacy/cookie-policy\#how-do-i-manage-trackers}{Your
  Ad Choices}
\item
  \href{https://www.nytimes3xbfgragh.onion/privacy}{Privacy}
\item
  \href{https://help.nytimes3xbfgragh.onion/hc/en-us/articles/115014893428-Terms-of-service}{Terms
  of Service}
\item
  \href{https://help.nytimes3xbfgragh.onion/hc/en-us/articles/115014893968-Terms-of-sale}{Terms
  of Sale}
\item
  \href{https://spiderbites.nytimes3xbfgragh.onion}{Site Map}
\item
  \href{https://help.nytimes3xbfgragh.onion/hc/en-us}{Help}
\item
  \href{https://www.nytimes3xbfgragh.onion/subscription?campaignId=37WXW}{Subscriptions}
\end{itemize}
