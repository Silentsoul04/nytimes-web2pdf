Sections

SEARCH

\protect\hyperlink{site-content}{Skip to
content}\protect\hyperlink{site-index}{Skip to site index}

\href{https://myaccount.nytimes3xbfgragh.onion/auth/login?response_type=cookie\&client_id=vi}{}

\href{https://www.nytimes3xbfgragh.onion/section/todayspaper}{Today's
Paper}

\href{/section/opinion}{Opinion}\textbar{}The Supreme Court's Religious
Crusaders Take On the Pandemic Response

\url{https://nyti.ms/338wu3l}

\begin{itemize}
\item
\item
\item
\item
\item
\item
\end{itemize}

Advertisement

\protect\hyperlink{after-top}{Continue reading the main story}

\href{/section/opinion}{Opinion}

Supported by

\protect\hyperlink{after-sponsor}{Continue reading the main story}

\hypertarget{the-supreme-courts-religious-crusaders-take-on-the-pandemic-response}{%
\section{The Supreme Court's Religious Crusaders Take On the Pandemic
Response}\label{the-supreme-courts-religious-crusaders-take-on-the-pandemic-response}}

The fight over limits on church attendance divides the justices.

\href{https://www.nytimes3xbfgragh.onion/by/linda-greenhouse}{\includegraphics{https://static01.graylady3jvrrxbe.onion/images/2018/04/02/opinion/linda-greenhouse/linda-greenhouse-thumbLarge.png}}

By \href{https://www.nytimes3xbfgragh.onion/by/linda-greenhouse}{Linda
Greenhouse}

Contributing Opinion Writer

\begin{itemize}
\item
  July 30, 2020
\item
  \begin{itemize}
  \item
  \item
  \item
  \item
  \item
  \item
  \end{itemize}
\end{itemize}

\includegraphics{https://static01.graylady3jvrrxbe.onion/images/2020/07/30/opinion/30Greehouse1/30Greehouse1-articleLarge.jpg?quality=75\&auto=webp\&disable=upscale}

I know I should be jaded by now by the persistence of the Supreme
Court's conservative justices in seeking to elevate religious interests
over those of secular society. After all, in the closing days of the
court's term,
\href{https://www.supremecourt.gov/opinions/19pdf/19-431_5i36.pdf}{religious
employers won} the right to withhold from female employees the
contraception coverage to which federal law entitled them.
\href{https://www.supremecourt.gov/opinions/19pdf/19-267_1an2.pdf}{Religious
schools gained} a broad exemption from the anti-discrimination laws that
would otherwise protect classroom teachers and soon, no doubt, other
employees as well.

But I was still startled last week to see Justices Samuel Alito, Brett
Kavanaugh, Neil Gorsuch and Clarence Thomas vote to turn a public health
issue into a religious crusade. Fortunately for the people of rural Lyon
County, Nev., where a church went to federal court for the right to have
90 people at a worship service instead of the permitted 50, the four
justices failed to find a fifth vote and the 50-person cap remains.

What surprised me was not that a church would run to federal court with
such a case, rather than add a second service or meet outside under a
tent. Representing the church, Calvary Chapel Dayton Valley, was the
Alliance Defending Freedom, which used to focus primarily on
representing people seeking a religious exemption from having to do
business with couples in same-sex marriages. Lately, the alliance has
been bringing
\href{https://www.adflegal.org/case/calvary-chapel-dayton-valley-v-sisolak}{cases}
around the country to challenge Covid-19-related limits on in-person
church services.

I was mildly surprised that this case got to the Supreme Court;
\href{https://www.supremecourt.gov/DocketPDF/19/19A1044/144133/20200523140701636_Emergency\%20Application\%20for\%20Writ\%20of\%20Injunction.pdf}{another
church case} reached the court in May, with a similar outcome. In that
case, as in
\href{https://www.supremecourt.gov/opinions/19pdf/19a1070_08l1.pdf}{the
Nevada case}, Chief Justice John Roberts refused to go along with the
four dissenters.

What did astonish me was the ferocity of the main dissenting opinion,
written by Justice Alito and joined by Justices Thomas and Kavanaugh.
They appear oblivious to the facts on the ground, particularly the
well-documented role of religious services in
\href{https://www.nytimes3xbfgragh.onion/2020/07/08/us/coronavirus-churches-outbreaks.html?searchResultPosition=3}{spreading
the virus}. (Reflecting the nationwide pattern, a small church in New
Haven, Conn., where I live, was
\href{https://www.newhavenindependent.org/index.php/archives/entry/covid-struck_church_/}{identified}
this month as the likely source of an unexpected uptick in Covid-19
cases.) People who are sitting --- and breathing --- together for a
prolonged period in an enclosed space might as well put out a welcome
mat for the coronavirus. We knew that back in May. It is even more
evident now. Thus the growing prevalence of official orders limiting
people who can come together in that fashion to a certain number or a
certain percentage of the venue's capacity.

\includegraphics{https://static01.graylady3jvrrxbe.onion/images/2020/07/30/opinion/30Greenhouse2/merlin_174807525_e87a5f58-5e75-4bc4-a026-458b851d842c-articleLarge.jpg?quality=75\&auto=webp\&disable=upscale}

Nevada's governor, Steve Sisolak, a Democrat, took several approaches in
an order he issued as part of the state's Phase 2 reopening in May.
Spectators remain barred at live entertainment events. The order raised
to 50 from 10 the number of people who can gather indoors at most
places, including movie theaters, lecture spaces, museums, trade schools
and yes, churches. (Churches have no limit on the number of people who
can gather outside, assuming proper social distancing, an exception not
offered to other venues.) Restaurants and theme parks are capped at 50
percent of capacity, as are the state's casinos. Since casinos can be
very big, the number of people inside can easily run into the hundreds.

Calvary Chapel had been holding its services online, but said it could
not continue to do so because of its belief that the Bible requires
in-person worship. ``If a body of believers fails to hold in-person
gatherings, Calvary Chapel views it as ceasing to be a church in the
biblical sense,'' according to the
\href{https://www.supremecourt.gov/DocketPDF/19/19A1070/147153/20200707185749882_Calvary\%20Chapel\%20Emergency\%20Application\%20for\%20Injunction\%20with\%20Appendix\%20FINAL.pdf}{emergency
appeal} that Alliance Defending Freedom filed in the Supreme Court after
two lower courts rejected the church's arguments.

One mystery is why the church, which said it was willing to cut its
90-minute service in half, couldn't add a second service to make room
for the 90 people it needed to accommodate. The deeper mystery is why
the church went to the Supreme Court at all, given the court's recent
rejection of a similar claim. One reason might be that crusades don't
take no for an answer. But the reason here was casinos.

The first page of the appeal contains a color photograph of an
unidentified Las Vegas casino taken on June 4, the frame filled with an
unmasked, cheek-by-jowl crowd of reveling gamblers (with predictable
results: Covid-19 cases
\href{https://thehill.com/changing-america/well-being/medical-advances/509247-over-a-hundred-coronavirus-cases-in-nevada}{surged}
and the governor soon imposed a mask requirement). The fact that casinos
are treated more generously than movie theaters, concerts, museums and
churches handed Alliance Defending Freedom its opening. ``The Free
Exercise clause protects the exercise of religion,'' the church's
Supreme Court appeal began, adding that ``no constitutional provision
protects the right to gamble at casinos, eat at restaurants, or frolic
at indoor amusement parks.''

This statement of the case obviously appealed to Justice Alito, who
began his 11-page dissent with this observation: ``The Constitution
guarantees the free exercise of religion. It says nothing about the
freedom to play craps or blackjack, to feed tokens into a slot machine,
or to engage in any other game of chance.'' He added, ``But the governor
of Nevada apparently has different priorities.'' The state, he said,
``blatantly discriminates against houses of worship.''

Let's assume that opening the casinos was a political and economic
necessity for the state. But finding a constitutional violation requires
more than clever rhetoric. It requires a very precise chain of analysis,
a hard climb that Justice Alito failed to make. His burden was to show
not simply that churches are limited to 50 people and casinos aren't,
but that the state singled churches out for onerous treatment without
reason. Federal District Judge Richard Boulware II, in rejecting the
church's request for an injunction, offered the analysis that Justice
Alito's opinion skipped over:

\begin{quote}
Other secular entities and activities similar in nature to church
services have been subject to similar or more restrictive limitations on
their operations. The court notes that church services consist of
activities, such as sermons and corporate worship, that are comparable
in terms of large numbers of people gathering for an extended period of
time to lectures, museums, movie theaters, specified trade/technical
schools, nightclubs and concerts. All of these latter activities are
also subject to the 50-person cap or remain banned altogether.
\end{quote}

In other words, it's casinos that Nevada singles out for special
treatment, not religion.

Invoking the First Amendment's free speech clause, Justice Alito also
claimed that the state was discriminating against churches on the basis
of their ``viewpoint.'' ``Religion counts as a viewpoint,'' he wrote, a
sentence I found baffling. Isn't belief in public health a viewpoint? In
any event, he went on: ``Compare the directive's treatment of casino
entertainment and church services. Both involve expression, but the
directive favors the secular expression in casino shows over the
religious expression in houses of worship.''

The contorted effort to shoehorn this dispute into standard free-speech
doctrine was only a prelude to a further charge: that Governor Sisolak
``has favored certain speakers over others.'' It seems that the governor
supported and even took part in a protest after the killing of George
Floyd by the Minneapolis police, a protest that drew more than 50 people
and yet was allowed to take place. ``Respecting some First Amendment
rights is not a shield for violating others,'' Justice Alito wrote. In
other words, the governor of Nevada is not only anti-religious but a
hypocrite as well.

Why does any of this matter? After all, Justice Alito's opinion
attracted only two other votes. (Justice Gorsuch filed a separate
one-paragraph dissent, pithily observing that ``there is no world in
which the Constitution permits Nevada to favor Caesars Palace over
Calvary Chapel.'')

I think it matters because rhetoric like Justice Alito's, by design or
not, inevitably places the Supreme Court where it least belongs and
where it is least qualified to be, at the center of a national debate
over how best to protect the American public from a deadly pandemic.
It's beyond dispute that public health has become politicized to a
degree that would have been unimaginable just six months ago. Does the
court really want to become a part of that politicization by opening its
doors to a battle not its own? That even a minority of justices would
seize this moment to advance their religious agenda, especially given
that agenda's nearly unqualified success in recent years, is deeply
unsettling.

(I've been fascinated that
\href{https://www.vox.com/2020/7/25/21338216/supreme-court-churches-pandemic-covid-samuel-alito-brett-kavanaugh-calvary-chapel}{some
liberal commentators} found the dissenting opinions persuasive and the
case a close one. I understand the impulse not to appear unduly
antagonistic toward religion, but I think that generosity toward the
religious claim here loses sight of the broader context in which the
dissenting justices were writing.)

Chief Justice Roberts understands why the court's intervention in such
disputes would be unwise. He said as much in the first church case back
in May, when he explained why he was not supporting the church's effort
to enlist the Supreme Court in a battle against California's Covid-19
regulations. The ``unelected judiciary,'' he wrote then, ``lacks the
background, competence, and expertise to assess public health and is not
accountable to the people.''

This time, the chief justice remained silent; the only voices were those
of the dissenters. Chief Justice Roberts is second to no one on the
court in pursuing
\href{https://www.nytimes3xbfgragh.onion/2020/07/16/opinion/supreme-court-roberts-religion.html?searchResultPosition=1}{his
goal} to afford religion an ever greater place of privilege in American
life. It's just that he's not willing to sacrifice the court to achieve
it.

\emph{The Times is committed to publishing}
\href{https://www.nytimes3xbfgragh.onion/2019/01/31/opinion/letters/letters-to-editor-new-york-times-women.html}{\emph{a
diversity of letters}} \emph{to the editor. We'd like to hear what you
think about this or any of our articles. Here are some}
\href{https://help.nytimes3xbfgragh.onion/hc/en-us/articles/115014925288-How-to-submit-a-letter-to-the-editor}{\emph{tips}}\emph{.
And here's our email:}
\href{mailto:letters@NYTimes.com}{\emph{letters@NYTimes.com}}\emph{.}

\emph{Follow The New York Times Opinion section on}
\href{https://www.facebookcorewwwi.onion/nytopinion}{\emph{Facebook}}\emph{,}
\href{http://twitter.com/NYTOpinion}{\emph{Twitter (@NYTopinion)}}
\emph{and}
\href{https://www.instagram.com/nytopinion/}{\emph{Instagram}}\emph{.}

Advertisement

\protect\hyperlink{after-bottom}{Continue reading the main story}

\hypertarget{site-index}{%
\subsection{Site Index}\label{site-index}}

\hypertarget{site-information-navigation}{%
\subsection{Site Information
Navigation}\label{site-information-navigation}}

\begin{itemize}
\tightlist
\item
  \href{https://help.nytimes3xbfgragh.onion/hc/en-us/articles/115014792127-Copyright-notice}{©~2020~The
  New York Times Company}
\end{itemize}

\begin{itemize}
\tightlist
\item
  \href{https://www.nytco.com/}{NYTCo}
\item
  \href{https://help.nytimes3xbfgragh.onion/hc/en-us/articles/115015385887-Contact-Us}{Contact
  Us}
\item
  \href{https://www.nytco.com/careers/}{Work with us}
\item
  \href{https://nytmediakit.com/}{Advertise}
\item
  \href{http://www.tbrandstudio.com/}{T Brand Studio}
\item
  \href{https://www.nytimes3xbfgragh.onion/privacy/cookie-policy\#how-do-i-manage-trackers}{Your
  Ad Choices}
\item
  \href{https://www.nytimes3xbfgragh.onion/privacy}{Privacy}
\item
  \href{https://help.nytimes3xbfgragh.onion/hc/en-us/articles/115014893428-Terms-of-service}{Terms
  of Service}
\item
  \href{https://help.nytimes3xbfgragh.onion/hc/en-us/articles/115014893968-Terms-of-sale}{Terms
  of Sale}
\item
  \href{https://spiderbites.nytimes3xbfgragh.onion}{Site Map}
\item
  \href{https://help.nytimes3xbfgragh.onion/hc/en-us}{Help}
\item
  \href{https://www.nytimes3xbfgragh.onion/subscription?campaignId=37WXW}{Subscriptions}
\end{itemize}
