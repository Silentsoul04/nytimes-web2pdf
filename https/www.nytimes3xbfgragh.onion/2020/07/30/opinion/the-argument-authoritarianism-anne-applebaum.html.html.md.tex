Sections

SEARCH

\protect\hyperlink{site-content}{Skip to
content}\protect\hyperlink{site-index}{Skip to site index}

\href{https://myaccount.nytimes3xbfgragh.onion/auth/login?response_type=cookie\&client_id=vi}{}

\href{https://www.nytimes3xbfgragh.onion/section/todayspaper}{Today's
Paper}

\href{/section/opinion}{Opinion}\textbar{}When Conservatives Fall for
Demagogues

\url{https://nyti.ms/30ZDf4W}

\begin{itemize}
\item
\item
\item
\item
\item
\end{itemize}

Advertisement

\protect\hyperlink{after-top}{Continue reading the main story}

transcript

Back to The Argument

bars

0:00/0:00

-0:00

transcript

\hypertarget{when-conservatives-fall-for-demagogues}{%
\subsection{When Conservatives Fall for
Demagogues}\label{when-conservatives-fall-for-demagogues}}

\hypertarget{with-frank-bruni-and-michelle-goldberg}{%
\subsubsection{With Frank Bruni and Michelle
Goldberg}\label{with-frank-bruni-and-michelle-goldberg}}

\hypertarget{a-debate-with-twilight-of-democracy-author-anne-applebaum}{%
\paragraph{A debate with ``Twilight of Democracy'' author Anne
Applebaum.}\label{a-debate-with-twilight-of-democracy-author-anne-applebaum}}

Thursday, July 30th, 2020

\begin{itemize}
\item
  frank bruni\\
  I'm Frank Bruni.
\item
  michelle goldberg\\
  I'm Michelle Goldberg. And this is ``The Argument.'' {[}MUSIC
  PLAYING{]}

  Today, our guest is writer and historian Anne Applebaum. We'll talk
  about how authoritarianism became so appealing to conservative
  thinkers in the 21st century. {[}MUSIC PLAYING{]}

  Anne Applebaum's latest book is ``Twilight of Democracy: The Seductive
  Lure of Authoritarianism.'' It's part memoir, part history, the story
  of people she once called friends, anti-communist intellectuals and
  conservative defenders of liberal ideals who, during the Cold War,
  fought for things like an independent press and a free market. In
  recent years, these same friends have become foot soldiers in the rise
  of nationalism and authoritarianism in the West. I read the book and
  found it utterly fascinating, even though I disagreed with a big part
  of Applebaum's premise. As I wrote in a recent column, ``there's no
  mystery in the right's surrender to authoritarianism because for many
  of the people Applebaum describes, it wasn't a surrender at all, it
  was a liberation.'' Anne, not surprisingly, disagrees, writing on
  Twitter, ``It is just as incorrect to write, quote, `conservatism
  always contained the seeds of authoritarianism,' as it is to write
  progressivism always contained the seeds of authoritarianism.'' It is
  an honor to get to debate it with her on the show today. Anne, welcome
  to ``The Argument.''
\item
  anne applebaum\\
  Thanks so much for talking to me.
\item
  michelle goldberg\\
  Before you explain why you think my argument about your book is wrong,
  maybe you could just give a sense of the case that the book is making
  or the story that the book is telling.
\item
  anne applebaum\\
  So first of all, to be clear, the book doesn't really make a case.
  This isn't a political science book.
\item
  michelle goldberg\\
  Mmhmm.
\item
  anne applebaum\\
  It doesn't have a thesis. It's not an op-ed column. It rather is a
  reflection on a series of events that I took part in or I was aware of
  or people that I knew over the last 20 and 30 years. My previous books
  were works of history in which I did lots of research, and then I
  tried to show a problem from lots of points of view. And this is
  really the opposite of that. It's a very narrow and subjective view,
  including some events that I have a role in or I'm somehow implicated
  in or biased about. It's about four countries, mostly about the United
  States, the United Kingdom, and Poland, with a big chunk about Hungary
  and some reflections on Spain. And it looks at the evolution of the
  right in all of those countries or the thing that used to be called
  the conservative movement or the center right or the Tories or the
  Polish Right or the Republican Party, and it observes that some people
  who used to think they were on the same side 20 or 30 years ago now
  are not. I start with a party at my house that took place in 1999. It
  was not a fancy party. And no, it's not a book about parties or
  catering. That just was a metaphor that I used for this alliance.
\item
  michelle goldberg\\
  No, it's a cinematic intro.
\item
  anne applebaum\\
  {[}LAUGHING{]} And I try to explain that some of the people who were
  at that party are now people who I no longer speak to and many of the
  other people at the party no longer speak to. And the reasons are not
  personal, they're political. We're now all on opposite sides of this
  very profound political divide in Poland, which is a very profoundly
  polarized country, some ways weirdly like the United States in which
  people live in alternate informational universes and don't speak to
  each other and have a completely different view of the world. And so
  the book was a kind of reflection about how those deep divides happen,
  what causes polarization. There are a few historical references. I
  talk about the Dreyfus trial in France, which was a similarly
  polarizing moment in French history. I talk about division over
  Brexit, which was different in many ways from the Polish division, but
  also caused everybody to get mad at each other. And I tried to show
  what were some of the deeper reasons for these changes. Again, the
  book has no thesis. It doesn't argue that there is one explanation,
  and it looks rather for historical echoes across countries rather than
  tying everything together in a neat knot. I spent a lot of time, like
  you both do, writing op-ed columns that have to end with a single
  paragraph saying, x is y, {[}BRUNI LAUGHS{]} and therefore we should
  do z. And it was really a relief in this book to write something in
  which, no, sorry, I'm not going to tie up all the loose ends for you.
  You're just going to have to figure it out yourself.
\item
  michelle goldberg\\
  It was really striking to me, your use of Fritz Stern's ``The Politics
  of Cultural Despair'' to sort of diagnose some of what you were seeing
  on the contemporary right. Stern was a history professor and a refugee
  from Nazi Germany who wrote about conservative intellectuals. So I
  also used ``The Politics of Cultural Despair'' to try to diagnose the
  contemporary right, but I did it in a book that came out in 2006. So
  I'm just going to put this quote out there that I used to try to
  explain what was going on in George W. Bush's re-election campaign in
  2004. ``The ideologists of the conservative revolution superimposed
  division of national redemption upon the dissatisfaction with liberal
  culture and with the loss of authoritative faith. They posed as the
  true champions of nationalism and berated the socialists for their
  internationalism and the liberals for their pacifism and their
  indifference to national greatness.'' And so I guess this is maybe the
  heart of our disagreement, right? I mean, I think I agree with you
  obviously about the horrors of Trumpism, but I do see it as more of a
  sort of evolutionary maybe inevitability of trends that were certainly
  present in the American right through most of my adult life as opposed
  to this real break with what had previously been known as
  conservatism. One of the things I saw specifically when I was covering
  George W. Bush's re-election in 2004 was a lot of talk about the,
  quote unquote, ``homosexual agenda'' as a kind of totalizing ideology.
  So when I read what you've written about, the use of anti-gay
  prejudice almost as a kind of conspiracy theory in Polish politics,
  that seemed really familiar to me.
\item
  anne applebaum\\
  So I'm not really disagreeing with you about that. I don't think it's
  incorrect to say that some of the seeds of what we're seeing now were
  present 15 or 20 years ago. I think my disagreement is that I believe
  that the anti-communist movement, which is, remember, the book begins
  at a slightly different moment, not in 2007, but in 1999, and it's
  reflecting on an earlier period, which is the late 1980s and early
  1990s. I'm arguing that the anti-communist movement in that period had
  different components, and people were anti-communist for different
  reasons. Some of them were anti-communist because they believed in
  Realpolitik and they were worried about Soviet nuclear weapons, OK?
  And some people were anti-communist because they cared about democracy
  and human rights and the rule of law. And there were some people who
  were anti-communist because they were Christians and the Soviet Union
  was Marxist and atheist, and therefore they were against it for that
  reason. And one of the things that I trace in the book is this idea
  that that coalition over time break up. I think I even say at one
  point that it was probably artificially held together by 9/11, which
  kind of kept some people on the same team for longer than they might
  have been anyway. I mean, you could even argue that it was breaking up
  by the end of the `90s in the United States. And the same is true in
  Poland, where the anti-communist movement in Poland had people in it
  whom you would describe as left-wing. There were a lot of
  anti-communist children of communists and so on. As these different
  bits of it broke up, some people went in one direction and some people
  went in another. So I'm not actually disagreeing about that. I mean, I
  think what I'm disagreeing about is your implication that this was
  somehow inevitable, that, of course, the right would end up like that
  because the right was always authoritarian, and that's what it was
  about. That I disagree with. If you'd been looking at the left in the
  1980s, you might have been able to make that same argument. I was
  actually looking for a copy of my own book, which I can't seem to
  find, which is my history of the Gulag that was published in 2004.
\item
  michelle goldberg\\
  Right.
\item
  anne applebaum\\
  And in the introduction to that book, I talk a little bit about the
  communist tolerant part of the Western left, which chose to ignore and
  dismiss and downplay the existence of Soviet concentration camps
  because it was politically inconvenient. There was a piece of the left
  that was always also charmed by authoritarianism and tempted by
  extremism. Well, that's also in my new book a little bit too. But if
  I'd said at that time, well, of course, it was inevitable that the
  left will therefore from this moment develop into an authoritarian
  movement, I would have been wrong. I mean, it could have been that
  those elements took over the left, and actually in one or two
  countries they did.
\item
  michelle goldberg\\
  Right.
\item
  anne applebaum\\
  The point is that these movements were, particularly in the United
  States where we're cursed with our two party system, these movements
  are coalitions, and I was simply dissecting what happened to the
  right-wing coalition or the center right coalition that I felt myself
  to be a part of and clearly no longer do. I simply don't think that
  Trumpism was some kind of inevitability, that it had to be that way
  because that was the way the party was going. That there are these
  elements of cultural despair of deep pessimism about American society
  of fear and anxiety caused by modernity and technological change and
  social change, that that is all present on the right right now, of
  course, it's absolutely true. I write about it all the time, and it's
  part of my book. What I don't agree with is that retrospectively
  colors all anti-communists as proto-authoritarians, which would be
  ridiculous.
\item
  michelle goldberg\\
  Oh, I don't think--- no, I mean, I don't think it retrospectively
  colors all anti-communists as proto-authoritarians. I just think ---
  and I think that this comes out in your book --- that the segment of
  that coalition that was genuinely concerned about human rights and
  democracy is smaller than, I think, people would have admitted at the
  time.
\item
  anne applebaum\\
  I mean, depends on which country, and it depends which time we're
  talking about. A lot of this is also to do with who won power and how
  and when. The anti-liberal or anti-democratic part of the Polish
  right, for example, was a fringe minority for 25 years, which had very
  little influence on Polish politics and was not even really visible.
  And even the Law and Justice party, which is now the ruling party, it
  was very briefly in power once before for a year and a half, and it
  exhibited a much more limited and narrow and less ambitious version of
  what it's doing now. It was still then a coalition that contained a
  wide range of people. Even the Law and Justice president, who was the
  one who died in the tragic plane crash, was somebody who cared a lot
  about the rule of law, for example, and polish judges and so on, and
  it's hard to imagine him presiding over this destruction of the
  judicial system that's taking place now. So that's around 2005. Even
  then, they were the truly anti-democratic anti-pluralist piece of the
  party was a fringe piece of the extreme. And that it has now moved to
  the center is the drama of the current story.
\item
  frank bruni\\
  Anne, I kept on wondering as I was reading you, when you look at the
  group of you on the right, conservatives, however you want to call
  yourselves, when you look at your kind of wide group of friends and
  acquaintances, not just in Poland, but also in Britain and the United
  States, and I assume you had some in Hungary as well, and you kind of
  observe who drifted readily, or at least eventually, toward the lure
  of authoritarianism or some of these more nativist administrations or
  governments, and who, like you, resisted or like the Never Trumpers
  here, what are the kind of character or logical or socioeconomic
  differences, what distinguishes the people who decided to go along
  versus the people who said, no way?
\item
  anne applebaum\\
  So first of all, to be clear, there are no sociological differences. I
  mean, we are talking about elite people. We're talking about educated
  sophisticated people who were not damaged by Poland's post-communist
  resurgence. Quite the contrary. I hesitate to generalize because one
  of the things I liked about this book is that, unlike as in writing
  columns, I didn't really have to generalize. I could just focus on a
  few people. But very often in Poland, a lot of the people that I'm
  talking about are people who, for one reason or another, became
  dissatisfied. I mean, so for example, in Poland with the
  post-communist state. And sometimes they were dissatisfied for
  political reasons, and often they were dissatisfied for personal
  reasons. So they personally had not prospered to the degree to which
  they felt they should have done.
\item
  frank bruni\\
  Right.
\item
  anne applebaum\\
  And so there is a quality of resentment, a feeling that the wrong
  people have risen to power, the wrong people are achieving prominence.
  Our business class is not the right kind of people. So there's a
  feeling that the wrong people have somehow won or are somehow in
  charge. And you can look back through history, and the sort of classic
  profile of the revolutionary is somebody who's part of the elite, but
  not on top. So look at Lenin. Lenin came from the Russian aristocracy,
  but the lowest rung of the aristocracy, right? And then his family
  even dropped out of the aristocracy after his brother was sentenced as
  a revolutionary, and he has fantastic resentment, even though he's not
  from the working class. He's not the poorest part of the country, but
  he has this feeling of resentment against Russia's rulers. OK, the
  Russian revolution is a more complicated story than that. It wasn't
  all about Lenin's personal psychosis. I don't want to keep going with
  that analogy. I'm just saying that if you do look back, you do find
  that a lot of radicals often fit that profile. And I think even in the
  U.S., Laura Ingraham, who I write about at some length, isn't somebody
  that I do know that well. She doesn't really count as a friend. She
  was just somebody I met a few times. And we have some mutual friends
  and so on, but I'm not close to her. But she is somebody who I know
  was always very resentful of the fact that her achievements weren't
  more recognized. So she also has that quality of resenting the elite,
  which she is also somehow part of. I mean, to say that she's not an
  elite is ridiculous. I mean, she's ---
\item
  michelle goldberg\\
  Did you ever read her book about what an elite is?
\item
  anne applebaum\\
  I read several of her books. I can't remember.
\item
  michelle goldberg\\
  She wrote this book in 2003 where she basically defines elitism as a
  state of mind that sort of has nothing to do---
\item
  anne applebaum\\
  With your actual---
\item
  michelle goldberg\\
  ---with class.
\item
  anne applebaum\\
  ---class. Yeah, right. Right.
\item
  michelle goldberg\\
  So it's kind of this bizarre fantasmagorical version of elitism in
  which a person with the right politics can never, by definition, be
  part of the elite.
\item
  anne applebaum\\
  Right. So she has defined herself as somebody --- even though she's
  extremely wealthy, television presenter with millions of social media
  followers and a huge fan base, and a large house and all that, she
  nevertheless defines herself as being somehow anti-elite, which is, of
  course, ridiculous.
\item
  michelle goldberg\\
  Right.
\item
  anne applebaum\\
  If you're asking what people like that have in common, this is often
  it. There's some element of resentment or a feeling of outsiderism.
\item
  michelle goldberg\\
  But do you think she's changed? Because in your book, you sort of
  associate her with this optimistic post-Cold War milieu of Reaganism
  and people who felt like we won, and now we're going to go on to keep
  winning. Whereas when I look at her history, I mean, starting with
  Dartmouth where she is the editor of this school newspaper with Dinesh
  D'Souza, where they become famous for trying to humiliate gay students
  by sending an undercover reporter to the gay student group's meeting
  and publishing excerpts of things that people talked about. Her book
  in 2003 is obviously extremely resentful and nationalist. I guess I
  don't see a Laura Ingraham that was at one point different or better
  than the one that we see today.
\item
  anne applebaum\\
  No, I'm not sure that Laura Ingraham was ever different or better. Her
  views about the world have certainly changed. I mean, she has a much
  different view of American foreign policy. She has a different
  analysis of American history than she once had. I'm not intimate with
  her. I don't know her well enough to tell you exactly all the
  components of it, although I tease it apart a little bit in the book.
  But that she was an anti-communist at a time when other people on the
  left weren't is true. And so I probably gave her more credit for that
  than she deserved in the `80s and `90s. But to me, that was a really
  important quality, and it remained an important quality for me for a
  long time.
\item
  frank bruni\\
  Anne, I want to say something very cynical and ask something very
  cynical, but these last years have made me pretty cynical, which is,
  you used the verb ``prospered'' before. You were just talking about
  Laura Ingraham's views have changed. I find myself wondering as we
  talk about resentment as a motivating factor, the size of the chip on
  someone's shoulder, to what extent do they actually have convictions
  that change in ideologies, or are they just grabbing onto the winning
  formula that puts them on the top of the team or in the winner's
  circle or make sure that they prosper? I mean, are they just kind of
  choosing horses rather than, in fact, evolving ideologically?
\item
  anne applebaum\\
  Some of them. I mean, some of them are deeply cynical. So one other
  person who I talk about in the book who won't be familiar to American
  listeners is a Hungarian woman called Maria Schmidt, whom I suspect of
  deep and profound cynicism. She's somebody who has a number of both
  business and political interests in Hungary, and she's made great
  efforts to stay on the right side of the current ruling party and is
  now one of their chief propagandists, even though I know that in
  earlier eras she was more doubtful about them. And I suspect her of
  being profoundly cynical. So one of the things that I tried to argue
  in the book is that, again, the human personality is very complex, and
  people can be motivated by more than one thing. And some people are
  motivated by an idealism or an ideology or politics, and some people
  are motivated by personal ambition. Some people are motivated by
  resentment. Some people are motivated by the pleasure in seeing their
  enemies squirm. And some people are, as you say, profoundly cynical,
  and sometimes those qualities dominate at different moments. I mean, I
  actually end the book with this discussion of cynicism and nihilism,
  and I say that this is the really dangerous sentiment for liberal
  democracy is that. Because once people become cynical and once they
  think it's all a game and it doesn't matter who wins, then you do
  begin to lose the virtue of the state, and then you really are open
  for all kinds of extreme politics. But yeah, I mean, are they cynical?
  Some of them really are, yes.
\item
  michelle goldberg\\
  But there's also a relationship, right, between kind of doubt and
  fanaticism. If you're going to insist on something that on some level
  you know it not to be true, it's likely to make you more histrionic
  and more sort of uncompromising.
\item
  anne applebaum\\
  I think there's a quote in the book that I'm not going to be able to
  recite to you off the top of my head, but it's a quote actually that
  comes from a guy who was a young communist in Poland in the 1940s,
  Jacek Trzynadel. And he described in a famous memoir that he wrote
  years later, he described what it felt like to be talking to a crowd
  of people. People are shouting and cheering, and he was lecturing the
  crowd whatever, support our Marxist state. I mean, I'm just making up
  what he said. I don't remember the exact thing. And he writes, the
  more I was shouting, the more I was doubting what I was saying. Did I
  really believe what I was shouting? And I felt I had to shout louder
  not just to convince the crowd, but to convince myself.
\item
  frank bruni\\
  I think I was talking a little bit about something else, which is just
  rank opportunism. We were talking moments ago about Laura Ingraham.
  One of her Fox News colleagues Tucker Carlson, if you go back and you
  listen to Tucker Carlson 15, 20 years ago, he doesn't sound exactly or
  really all that much like he sounds now. And his voice has evolved in
  relationship to where he's gotten the largest audience, the most
  traction, the most power. And Donald Trump, I mean, does any one of us
  really believe that Donald Trump has an intrinsic and abiding ideology
  versus a set of hunches and instincts about which direction to turn in
  to maximize the adulation and the purchase on power?
\item
  michelle goldberg\\
  But actually, Frank, I mean, I would say I actually think that racism
  is pretty deep in Donald Trump and is as close to a sincere belief as
  he's possible of holding.
\item
  anne applebaum\\
  So actually, Tucker Carlson, who I didn't write about, but is probably
  a better example, as you say, of somebody who has changed very
  profoundly. David Frum has a thesis about Carlson, which is that it
  was because what he really, really, really, really always wanted to do
  was be on television and be good at television. He tried various
  television projects, and they kept failing. And then finally, he hit
  on the formula that would win him millions of fans, and that was this
  kind of strange racist, nationalist, nativist language that is
  designed to appeal to mobs of like thinkers in America and that that
  was what finally brought him fame and glory, and that's why he's
  sticking with it. And so, yes, that would be an example of really sort
  of pure opportunism and a kind of nihilism. And that, as I say, is one
  of the enemies of good politics.
\item
  frank bruni\\
  But where would you put Trump, Anne? Is Trump more Tucker or more
  Lenin? {[}GOLDBERG AND APPLEBAUM LAUGH{]}
\item
  anne applebaum\\
  Actually, a bit of --- {[}LAUGHS{]} golly, that's not a question I
  ever thought I was going to have to answer.
\item
  frank bruni\\
  It's not a question I ever thought I'd ask. {[}BOTH LAUGH{]}
\item
  anne applebaum\\
  So, I mean, actually, you can see elements of both. I mean, think
  about it. Trump is somebody who he's very rich. He's on the front page
  of tabloids. And yet, is he ever really accepted by the New York elite
  who he resents? Is he accepted by the real movers and shakers, the
  real bankers and businessmen in New York? Do they see him as a real
  player? I mean, no, they don't. I mean, for all I know, he may be
  deeply resentful of the people who look down on him because he was
  vulgar and so on. So he may well fit into that sort of, in that sense,
  the Leninist category, and at the same time be someone who, as
  Michelle said, is somebody who just has this instinct for kind of
  negative publicity. He understands how to move crowds and how to
  appeal to really base instincts. So he's somebody you can see both of
  those aspects in.
\item
  michelle goldberg\\
  Now, one of the things that I think is really valuable about your book
  is its kind of international scope. So it's not just about what
  happened to the Republican Party. But to me, one of the questions that
  keeps --- this may be a provincial question, but compared to maybe if
  Hungary is the worst case scenario out of all of the countries that
  you're writing about in this book, possibly followed by Poland, these
  are both countries where, as you write about, there's sort of a
  conspiracy at the very center of the government. Where do you see the
  United States on that trajectory? How far are we from Poland, or how
  far are we from Victor Orbán's Hungary?
\item
  anne applebaum\\
  So this may sound very bizarre, and I accept that it's my own weird
  provincialism, and it's my strange perspective as an American who
  lives at least part of the time in Poland. I think the United States
  and Poland are really similar in the sense that I do think that Trump
  came to power also on the back of a conspiracy that all of us ---
  including me, but also many other people --- underrated, which was the
  conspiracy of birtherism. And it's now sort of forgotten because so
  many things have happened, that this was the moment when he really
  broke into national politics was by being the kind of the loudest
  spokesman for birtherism. And what was birtherism? It was the argument
  that the president of the United States is illegitimate. He's not even
  American. And this argument had a tremendous amount of power and was
  believed by an enormous number of people. I saw a statistic once, it's
  between 20 percent and 30 percent of Americans believed this to be
  true. Now, think about what that means. If the president is
  illegitimate, that means that everybody --- the media, the courts, the
  Congress, the civil service --- everybody is lying to you. So the
  entire state is covering up this fact that the president is an illegal
  outsider and should not be president at all. If you believe that, OK,
  then you are ready for all kinds of radical changes and all kinds of
  radical politics because you have come to doubt all of the
  institutions of your democracy.
\item
  michelle goldberg\\
  Oh, that's really interesting. So that creates much more of a kind of
  --- I don't think that I've thought before that there is such a
  through line between birtherism and then the deep state conspiracy.
\item
  anne applebaum\\
  Oh, absolutely. No, no, I mean, the use of conspiracy theories and
  conspiracy thinking by Trump is very similar to the way this was used
  in Poland and where it's been used in other places. And by the way,
  throughout history, OK? The Bolsheviks were great conspiracy
  theorists. And the way in which he used it and the way in which he has
  consistently sought to undermine public trust in a range of
  institutions --- the deep state, the fake news media, the judges who
  are really Mexicans, they're not Americans --- all that language used
  over and over. Some of it I think he does in a calculating way and
  some of it is instinctive. I mean, this is genuinely, I think,
  probably how he sees the world. All of this has built up and amplified
  the existing lack of trust in American society and helped to undermine
  people's faith in institutions. I mean, if you think American
  democracy is so rotten that we had a president for eight years who
  wasn't even American and it was all lied to us, then you're willing to
  see all kinds of things overthrown. Then what do you care about the
  State Department being decimated or the Inspector General of the
  C.I.A. being fired? Why should you care? All those people, it's so
  obvious that all of it is corrupt. And this is something that Trumpism
  has in common with the way politics are done not just in Poland, but
  in Brazil and in many other countries.
\item
  frank bruni\\
  Anne, I ask you this because you are a student, a scholar of
  authoritarians and authoritarianism, and because I suspect it's a
  question that's on a lot of our listeners' minds. Donald Trump, let's
  say he loses in early November. Do you think there's a real
  possibility that he rejects the results? And what does that look like?
  And if he wins, what do you suspect the next four years look like?
\item
  anne applebaum\\
  So those are two different questions. I mean, first of all, I think
  it's almost certain that he is going to question the results. He's
  said so already. He's been asked point blank, will you accept the
  result? And he's refused to say yes. So I think that is a clear
  possibility. It's something everybody should be prepared for. It may
  be that there is a very easy solution to that if it happens, and the
  solution is I think Biden himself has said, well, on January the 20th,
  he'll walk into the White House, and the Secret Service will escort
  Mr. Trump out because Trump will have trouble fighting the legality of
  it. But that doesn't mean that he can't cause an enormous amount of
  trouble or do an enormous amount of damage by campaigning around the
  country between November and January and saying that the election was
  rigged and seek once again to use conspiracy theory to move people and
  prepare them for some other political project still to come. And so,
  yes, it's possible. But I should also say that I also think it's
  possible he'll try to cheat in other ways. I mean, I think the attempt
  to undermine absentee ballots, the hints that he might undermine the
  work of the post office so that it couldn't deliver absentee ballots,
  I mean, all of that is preparing both to try and cheat and to prevent
  people from voting, and also to begin to make the case that the
  election was rigged. So it is highly possible he'll do that. Second
  question --- well, no, sorry, your second question was about what if
  he wins.
\item
  frank bruni\\
  If he wins, is that it? Are we on a fast track, an express train to
  authoritarianism?
\item
  anne applebaum\\
  So if he wins, I mean, the main difference between Trump's first term
  and Trump's second term will be the kinds of people working for him.
  Because if you remember, in the early part of his presidency, there
  were still people working for him who believed they were there to do
  something good and they were there to protect the institutions of the
  United States. And although I've criticized James Mattis, I do accept
  that he thought that what he was doing there was making sure that the
  army was OK and that elements of U.S. foreign policy, like our
  relationship with allies, were maintained. And a number of people who
  worked for Trump in the beginning had that commitment to the
  institutions and to democracy. In Trump's second term, there will be
  no such people. And so the government, the institutions will be run by
  people who are intent on destroying what they can and who will do so
  rather aggressively. And yes, I do think that would cause an enormous
  crisis in American democracy.
\item
  frank bruni\\
  The other scenario is Biden wins, Trump leaves voluntarily or is
  escorted out. What then happens to the authoritarian fervor that he
  has cultivated during his four years when he's out of office?
\item
  anne applebaum\\
  So this is a really interesting question, and one that I have been
  talking to people about. Most think that --- and these are mostly
  ex-Republicans, but people who know the party quite well --- most
  people think that everything depends on how he loses. So if he loses
  by a lot, if it's a wipeout, if the Republican Party loses the Senate
  and if it loses even more seats in the House, then it is possible that
  a part of the party leadership will say we're right, that was a big
  disaster, and that was a really bad idea. And Trumpism was a road to
  nowhere, and it's a road to an ever-shrinking base. And therefore,
  there will be an election or an argument inside the party, and a new
  kind of leader will be chosen. And then as somebody said to me
  recently, and then there's at least a chance that we get rid of this
  authoritarian or we push to the fringes this authoritarian streak in
  the Republican Party. It doesn't mean that we will, but it means that
  then there's a chance. If the election is very close, on the other
  hand, if Trump loses very slightly, if the Republicans retain the
  Senate, and if the party still feels that this kind of rhetoric and
  this kind of language is a winning ticket, then the next presidential
  candidate may well be Tucker Carlson or Don Jr. {[}BRUNI LAUGHS{]} or
  Ivanka or someone who ---
\item
  frank bruni\\
  So if it's Tom Cotton, we're lucky, right?
\item
  anne applebaum\\
  If it's Tom Cotton, you're lucky.
\item
  frank bruni\\
  That's actually --- {[}APPLEBAUM LAUGHS{]}
\item
  anne applebaum\\
  Or Mike Pompeo. I mean ---
\item
  frank bruni\\
  Ugh.
\item
  anne applebaum\\
  --- there's a range of people who will be competing for the role of
  the next and possibly more efficient and more intelligent Trumpist
  leader. And so then we are in a real national conundrum because then
  we have one political party which is not dedicated to playing by the
  rules of the game, and then the kind of polarization that we've had
  will continue and get worse. And then we will find ourselves in the
  position of having every four years a kind of electoral moment of
  crisis.
\item
  michelle goldberg\\
  So obviously, a lot of these trends that we're talking about in other
  countries predate the election of Donald Trump. But I'm curious how
  you think the --- I mean, collapse isn't the right word, but the kind
  of damage incurred to the ideal of liberal democracy in America has
  empowered authoritarianism in other countries and how the sort of fate
  of this election will then--- the echoes that will have in other
  countries that have basically argued that liberal democracy has
  failed, that it was a joke, that its time is over.
\item
  anne applebaum\\
  So the election of Donald Trump, the re-election of Donald Trump,
  would have an enormous negative impact around the world, and it would
  absolutely empower other authoritarians, both sitting ones, whether in
  Russia or in China, for that matter, or in the Middle East, as well as
  aspiring ones in other countries that are current democracies. And
  this is both by the power of example. Trump and his language are
  quoted and imitated by people all over the world. I mean, for example,
  his use of the expression ``fake news'' to undermine the media is one
  that has been picked up and repeated by dozens of other authoritarian
  leaders or would-be authoritarian leaders. I mean, the other part of
  the story is that the authoritarian right, the new radical right
  around the world has deep links. There are deep financial links. There
  are deep links, kind of organic links on social media. These groups of
  people speak to one another. They share one another's memes. They
  communicate in different ways across borders. And the influence of the
  American alt-right on the European far right or the international
  alt-right is quite profound and vice versa. I think they influence one
  another. And there is no question that the re-election of Donald Trump
  would embolden and motivate those different groups who, as I say, are
  speaking to one another all the time anyway and help them propagate
  their ideas further.
\item
  frank bruni\\
  Anne, can I ask you a meta question, so to speak? In the very title of
  your book Twilight of Democracy and I think sort of the subtext of a
  lot of the discussion that we're all having here today is this notion
  that the trend line is away from democracy. The trend line is toward,
  very scarily, authoritarianism. I just want to kind of play devil's
  advocate maybe in a very hopeful vein. Poland just had an election. I
  believe--- you can correct me if I'm wrong --- was the closest
  election since the end of communism. And in fact, the authoritarian
  government barely won. You go back to 2016. Trump got almost 3 million
  fewer votes than Hillary Clinton did in the popular vote. He won only
  after Russian interference, only after a late in the game assist from
  Jim Comey, and possibly also because so many people were so convinced
  the outcome was predetermined in Hillary Clinton's favor that they
  didn't rouse themselves to vote. And since then, almost every metric,
  almost every measure, almost every election has suggested the limits
  of Trump's appeal. Is it possible that what we're seeing in Poland and
  what we're seeing in America is, at the end of the day, a blip, he
  asked hopefully.
\item
  anne applebaum\\
  I mean, maybe. I, too, hope that that's the correct interpretation.
  And our country and other countries have overcome even worse crises
  before. I mean, the Civil War was a pretty big blip on the history of
  American democracy. It was a moment of total collapse, and somehow we
  nevertheless recovered from that with some caveats. I think one of the
  reasons I wrote the book and one of the reasons I concluded it the way
  I did is that what I don't want is for Americans to become complacent,
  for us to say, well, our democracy is so great and so strong and our
  Constitution is so fantastic, and we're a superpower. I don't have to
  try that hard to make sure that everything goes well in our country,
  and it's all going to go back to how it was eventually. And anyway,
  Poland is far away, and they are central Europeans, and they have
  nothing to do with us. I mean, I really believe that it is that
  conviction of inevitability which was the big mistake, I should say,
  of my entourage or my milieu, my friends in the `90s. There was a kind
  of complacency that now that we're on this road to democracy, nothing
  can take us back. Because that is wrong. Countries do reverse
  themselves, and they do cease to be democracies. And we are going
  through a really extraordinary moment of technological and
  informational and demographic and social change, and changes make
  society very volatile and all kinds of outcomes are possible. And I
  don't want people to be complacent or certain that we will somehow go
  back to everything being the way it was before, and you don't have to
  really do anything about it or worry about it. because it's that
  feeling of inevitability that turned out to be wrong.
\item
  michelle goldberg\\
  So I think we will end it there. And even though we have some
  disagreements about the nature of pre-Trump conservatism, it was a
  pleasure to read ``The Twilight of Democracy.'' It's a fascinating
  book, and I'm thrilled that you came on the show. Thank you so much.
\item
  frank bruni\\
  Anne, thank you so much for your time and your thoughtfulness.
\item
  anne applebaum\\
  Oh, thanks for talking to me.

  {[}MUSIC PLAYING{]}
\item
  frank bruni\\
  And we're back. Michelle, what did you think?
\item
  michelle goldberg\\
  So I think that she's extraordinarily brilliant obviously. And it's
  often funny when you disagree with someone on Twitter in a column, and
  then you end up talking and you find out you agree more than you might
  have thought. And I should say that Anne Applebaum is someone who her
  willingness to kind of call out authoritarianism on the right
  certainly didn't begin with Trump. Something I probably should have
  brought up is that she was at one time a big fan of John McCain, who
  refused to vote for him, I think very honorably, in 2008 because he
  put Sarah Palin on the ticket, right? And Sarah Palin's rise to
  prominence was in a lot of ways a premonition of what we're living
  through now. And so one of the things that is so interesting about
  this moment is how you end up being in political coalitions with
  people that you never could have imagined, right? It's such a reminder
  of how long life is. That it would have been hard enough for me to
  imagine maybe being on the same side as Anne Applebaum in 2000 or
  2003, never mind David Frum and Bill Kristol. But one of the things I
  wrote in my piece is that I think a lot of people on the left are
  very, very suspicious of Never Trump conservatives, and I really
  admire them because they've really sacrificed something to stand up to
  him. And they've sacrificed more than I've had to or more than people
  on the left have had to just in terms of the fissuring of personal
  relationships, professional networks, right? I mean, I think it's
  really not easy to turn your back on the social and political world
  that you've lived your whole life in.
\item
  frank bruni\\
  What you just described or what you just talked about is actually
  where I get my hope during these dark times. And what I mean by that
  is, you're right, you and Anne Applebaum agree more than you disagree,
  and that becomes clear when you talk with each other, and you have
  made temporary whatever kind of partnership with people whom you
  didn't expect to. I think that if I'm going to be an optimist, one
  thing that could come out of all of this --- and our conversation with
  Anne reacquainted me with this optimism --- is that we can focus on
  what unites us more than what divides us. Understanding what a threat
  Trump is, being so deeply offended by some of the divisions he's
  exploited and the ways he's governed has reminded so many people from
  various points of the political spectrum of what they have in common
  and of how overlapping their fundamental values are in so many ways.
  Don't you think?
\item
  michelle goldberg\\
  I guess, but I really wonder if that's more of an elite phenomenon,
  right? That sort of people who make their living by kind of arguing
  and writing in the search for truth and good faith argumentation are
  sort of necessarily in this moment on the same side. Instead of kind
  of politics--- I've written this before. Instead of politics being a
  war of ideas, the people who takes ideas seriously are aside in
  politics. I think if you look at the country as a whole, I don't see
  how you look out at America right now and say that this is a country
  where people are more aware of what unites them than divides them. I
  mean, to me, it looks like a country that's coming apart at the seams.
\item
  frank bruni\\
  I mean, it looks and feels that way. And yet, I don't think it's just
  an elite phenomenon. I mean, if I flash back over the last couple of
  years, once Trump had been in office for a while and the offenses had
  mounted and mounted and mounted, the number of times somebody in a
  store, in a fast food restaurant, wherever, has kind of set an aside,
  an aside of disgust that lets me know that they've developed the same
  reservations or deep concerns that someone writing op-ed pieces and
  simply using a different vocabulary is expressing. I mean, that
  happens all the time. And I look more recently --- and I think this
  has been so fascinating and hopeful --- I look at the polls regarding
  how Americans feel about Black Lives Matter and where that puts them
  in opposition to Trump, and I see an overwhelming majority of
  Americans coming together with the same idea of justice and rejecting
  Donald Trump. And ---
\item
  michelle goldberg\\
  No, I think you're right that Donald Trump has created --- and I think
  Jemele wrote about this, right, that Donald Trump has created, I don't
  know if it's a silent majority, but he's definitely created a fairly
  solid anti-Trump majority.
\item
  frank bruni\\
  Again, when I want to wear my optimist hat, I think Donald Trump in
  the long run, if things turn out better and not worse, I think Donald
  Trump has shown us who we don't want to be.

  At the end of the conversation with Anne, we talked in what I found to
  be a very fascinating manner about various scenarios. Trump barely
  loses. Trump loses big. Trump wins. Beyond all of those, there's a
  scenario where he loses, we move on. And again, this is my hope. This
  is me at my most hopeful and optimistic. Call me Pollyanna. We
  remember what Donald Trump did to us. We remember what he showed us in
  terms of the kind of America we don't want to be. And maybe we're the
  better for it.
\item
  michelle goldberg\\
  So I think that's certainly a possibility, right? I mean, if you look
  at all the reforms that you saw after Richard Nixon, I mean, most of
  them stood up fairly well until Donald Trump dismantled them. But
  there was not just this huge wave after, this huge Democratic wave
  that followed Nixon's resignation, but a whole flurry of lawmaking to
  try to solidify accountability, solidify transparency, solidify
  democratic institutions. I think it's going to be really, really
  important that people keep pushing on that. When Barack Obama first
  became president, there were people who wanted some sort of
  accountability for the people who had both misled the country into war
  under George W. Bush and also the people responsible for the financial
  crisis. And I understand why Obama, facing all these interlocking
  crises, thought that it was the time to sort of look forward rather
  than backward. But I think that was a mistake in retrospect, and it
  would be a fatal mistake if, inshallah, Biden becomes president. I
  think it will be really important to have something--- I don't know if
  it's a Truth and Reconciliation Commission, if it's a commission in
  the Justice Department --- to sort of expose both the kind of
  instances of corruption and lawlessness that we maybe half know about.
  And I'm sure there are a lot that we don't know about, right? I'm sure
  there are a lot of utterly corrupt phone calls with foreign leaders as
  bad as the one with the president of Ukraine that we just haven't
  heard about. And so I think it'll be really, really important to
  figure out the places where norms need to be replaced by laws.
\item
  frank bruni\\
  I suspect we will be excavating the muck of the Trump administration
  for decades to come. I just hope that excavation begins sooner rather
  than later. {[}GOLDBERG LAUGHS{]} But we should probably wrap it up
  there, Michelle. It's just you and me this week, which means 50/50
  chance that you're going to be the one doing the recommendation. And I
  just tossed a coin. Heads it's you. What's your recommendation this
  week?
\item
  michelle goldberg\\
  So I am going to recommend --- it's not a new book. I'm going to
  recommend an old book that I've been thinking about a lot recently and
  that I recently re-read, Milan Kundera's ``The Joke.'' So I don't know
  about you, but I have a lot of these books that I feel like have been
  really formative, and yet at the same time I can barely remember. But
  I decided to reread ``The Joke'' recently because we've been, or I've
  been at any rate, embroiled in this discussion over cancel culture,
  which even though I don't like the phrase cancel culture, it seems
  somewhat inescapable, and sort of what it means to be publicly shamed,
  what it means for a ideological movement to enforce its boundaries.
  And when I see people kind of cast out, I often think of this book.
\item
  frank bruni\\
  What is the storyline of ``The Joke``?
\item
  michelle goldberg\\
  OK, so the storyline of ``The Joke'' is there is a young communist in
  Czechoslovakia in the `40s, a very ardent communist, but somebody with
  a sort of slightly ironic temperament, who has a crush on this girl
  who is kind of earnest to a fault. And he teases her by sending her
  this jokey postcard --- hold on, I'll just get the actual line because
  I have it underlined here. {[}PAGES FLIPPING{]} So he teases her by
  sending her this jokey postcard that says, ``Optimism is the opium of
  the people. A healthy atmosphere stinks of stupidity. Long live
  Trotsky.'' And this postcard, this joke, ends up completely derailing
  his life. He's drummed out of the party, drummed out of the
  university. He's sent to go work in the mines. So in some ways, this
  is not at all comparable to what we talk about when we talk about
  cancellation, right? Because the sort of horror of it is that it has
  the power of the state behind it. And sometimes I think that people my
  age and older, those of us who grew up on Milan Kundera and books like
  it, there are certain rhetorical modes that signal to us
  authoritarianism. And we see them on Twitter and maybe overreact
  because what was so frightening about them was that they had the power
  of the state behind them. And yet at the same time, I think he sort of
  really captures the emotional experience, or what I imagine to be the
  emotional experience, of being sort of drummed out of a movement that
  you identify with. Because the significant thing about Ludvik is that
  he really is a communist, and so he really does identify with the
  people who are casting him out. He doesn't really want to be a
  dissident. I'll just quote this line: ``I came to realize that there
  was no power capable of changing the image of my person lodged
  somewhere in the supreme court of human destinies. But this image,
  even though it bore no resemblance to me, was much more real than my
  actual self, that I was its shadow and not it mine.'' I think that
  line captures some of what happens when people become reduced to
  either the worst thing that they've ever said or someone's impression
  of the worst thing that they ever said.
\item
  frank bruni\\
  It's so interesting to hear you describe it, Michelle, because one of
  the more celebrated novels to come out in the last six months and one
  of the novels I most recently read is called ``A Burning.'' I'm going
  to probably mispronounce the author's name, for which I apologize, but
  her name is Megha Majumdar, and it's set in India. And I'm realizing
  how timeless, what stamina the story you're describing from ``The
  Joke'' has. It is, again, about someone who, with a single
  communication, miscommunication, call it what you will, is branded
  forevermore and becomes a convenient object of analysis, demonization,
  caricature, whatever, for the society around her. So this is clearly a
  story that, regardless of epoch of decade, speaks to certain eternal
  verities of human nature.
\item
  michelle goldberg\\
  I had heard of that book, but I hadn't realized that that was what it
  was about. But now I'm gonna --- we can make that a recommendation too
  because I'm going to order it.
\item
  frank bruni\\
  But this is your week. Your recommendation again is?
\item
  michelle goldberg\\
  My recommendation is Milan Kundera's ``The Joke.'' That's our show
  this week. Thanks for listening. ``The Argument'' is a production of
  The New York Times opinion section. The team includes Phoebe Lett,
  Paula Schumann, and Pedro Rafael Rosado. Special thanks to Brad Fisher
  and Kristin Lin. See you next week. {[}MUSIC PLAYING{]}
\end{itemize}

\href{https://www.nytimes3xbfgragh.onion/column/the-argument}{\includegraphics{https://static01.graylady3jvrrxbe.onion/images/2018/10/03/opinion/the-argument-album-art/the-argument-album-art-square320-v3.png}The
Argument}Subscribe:

\begin{itemize}
\tightlist
\item
  \href{https://itunes.apple.com/us/podcast/id1438024613}{Apple
  Podcasts}
\item
  \href{https://www.google.com/podcasts?feed=aHR0cHM6Ly9yc3MuYXJ0MTkuY29tL3RoZS1hcmd1bWVudA\%3D\%3D}{Google
  Podcasts}
\end{itemize}

\hypertarget{when-conservatives-fall-for-demagogues-1}{%
\section{When Conservatives Fall for
Demagogues}\label{when-conservatives-fall-for-demagogues-1}}

\hypertarget{a-debate-with-twilight-of-democracy-author-anne-applebaum-1}{%
\subsection{A debate with ``Twilight of Democracy'' author Anne
Applebaum.}\label{a-debate-with-twilight-of-democracy-author-anne-applebaum-1}}

With Frank Bruni and Michelle Goldberg

Transcript

transcript

Back to The Argument

bars

0:00/0:00

-0:00

transcript

\hypertarget{when-conservatives-fall-for-demagogues-2}{%
\subsection{When Conservatives Fall for
Demagogues}\label{when-conservatives-fall-for-demagogues-2}}

\hypertarget{with-frank-bruni-and-michelle-goldberg-1}{%
\subsubsection{With Frank Bruni and Michelle
Goldberg}\label{with-frank-bruni-and-michelle-goldberg-1}}

\hypertarget{a-debate-with-twilight-of-democracy-author-anne-applebaum-2}{%
\paragraph{A debate with ``Twilight of Democracy'' author Anne
Applebaum.}\label{a-debate-with-twilight-of-democracy-author-anne-applebaum-2}}

Thursday, July 30th, 2020

\begin{itemize}
\item
  frank bruni\\
  I'm Frank Bruni.
\item
  michelle goldberg\\
  I'm Michelle Goldberg. And this is ``The Argument.'' {[}MUSIC
  PLAYING{]}

  Today, our guest is writer and historian Anne Applebaum. We'll talk
  about how authoritarianism became so appealing to conservative
  thinkers in the 21st century. {[}MUSIC PLAYING{]}

  Anne Applebaum's latest book is ``Twilight of Democracy: The Seductive
  Lure of Authoritarianism.'' It's part memoir, part history, the story
  of people she once called friends, anti-communist intellectuals and
  conservative defenders of liberal ideals who, during the Cold War,
  fought for things like an independent press and a free market. In
  recent years, these same friends have become foot soldiers in the rise
  of nationalism and authoritarianism in the West. I read the book and
  found it utterly fascinating, even though I disagreed with a big part
  of Applebaum's premise. As I wrote in a recent column, ``there's no
  mystery in the right's surrender to authoritarianism because for many
  of the people Applebaum describes, it wasn't a surrender at all, it
  was a liberation.'' Anne, not surprisingly, disagrees, writing on
  Twitter, ``It is just as incorrect to write, quote, `conservatism
  always contained the seeds of authoritarianism,' as it is to write
  progressivism always contained the seeds of authoritarianism.'' It is
  an honor to get to debate it with her on the show today. Anne, welcome
  to ``The Argument.''
\item
  anne applebaum\\
  Thanks so much for talking to me.
\item
  michelle goldberg\\
  Before you explain why you think my argument about your book is wrong,
  maybe you could just give a sense of the case that the book is making
  or the story that the book is telling.
\item
  anne applebaum\\
  So first of all, to be clear, the book doesn't really make a case.
  This isn't a political science book.
\item
  michelle goldberg\\
  Mmhmm.
\item
  anne applebaum\\
  It doesn't have a thesis. It's not an op-ed column. It rather is a
  reflection on a series of events that I took part in or I was aware of
  or people that I knew over the last 20 and 30 years. My previous books
  were works of history in which I did lots of research, and then I
  tried to show a problem from lots of points of view. And this is
  really the opposite of that. It's a very narrow and subjective view,
  including some events that I have a role in or I'm somehow implicated
  in or biased about. It's about four countries, mostly about the United
  States, the United Kingdom, and Poland, with a big chunk about Hungary
  and some reflections on Spain. And it looks at the evolution of the
  right in all of those countries or the thing that used to be called
  the conservative movement or the center right or the Tories or the
  Polish Right or the Republican Party, and it observes that some people
  who used to think they were on the same side 20 or 30 years ago now
  are not. I start with a party at my house that took place in 1999. It
  was not a fancy party. And no, it's not a book about parties or
  catering. That just was a metaphor that I used for this alliance.
\item
  michelle goldberg\\
  No, it's a cinematic intro.
\item
  anne applebaum\\
  {[}LAUGHING{]} And I try to explain that some of the people who were
  at that party are now people who I no longer speak to and many of the
  other people at the party no longer speak to. And the reasons are not
  personal, they're political. We're now all on opposite sides of this
  very profound political divide in Poland, which is a very profoundly
  polarized country, some ways weirdly like the United States in which
  people live in alternate informational universes and don't speak to
  each other and have a completely different view of the world. And so
  the book was a kind of reflection about how those deep divides happen,
  what causes polarization. There are a few historical references. I
  talk about the Dreyfus trial in France, which was a similarly
  polarizing moment in French history. I talk about division over
  Brexit, which was different in many ways from the Polish division, but
  also caused everybody to get mad at each other. And I tried to show
  what were some of the deeper reasons for these changes. Again, the
  book has no thesis. It doesn't argue that there is one explanation,
  and it looks rather for historical echoes across countries rather than
  tying everything together in a neat knot. I spent a lot of time, like
  you both do, writing op-ed columns that have to end with a single
  paragraph saying, x is y, {[}BRUNI LAUGHS{]} and therefore we should
  do z. And it was really a relief in this book to write something in
  which, no, sorry, I'm not going to tie up all the loose ends for you.
  You're just going to have to figure it out yourself.
\item
  michelle goldberg\\
  It was really striking to me, your use of Fritz Stern's ``The Politics
  of Cultural Despair'' to sort of diagnose some of what you were seeing
  on the contemporary right. Stern was a history professor and a refugee
  from Nazi Germany who wrote about conservative intellectuals. So I
  also used ``The Politics of Cultural Despair'' to try to diagnose the
  contemporary right, but I did it in a book that came out in 2006. So
  I'm just going to put this quote out there that I used to try to
  explain what was going on in George W. Bush's re-election campaign in
  2004. ``The ideologists of the conservative revolution superimposed
  division of national redemption upon the dissatisfaction with liberal
  culture and with the loss of authoritative faith. They posed as the
  true champions of nationalism and berated the socialists for their
  internationalism and the liberals for their pacifism and their
  indifference to national greatness.'' And so I guess this is maybe the
  heart of our disagreement, right? I mean, I think I agree with you
  obviously about the horrors of Trumpism, but I do see it as more of a
  sort of evolutionary maybe inevitability of trends that were certainly
  present in the American right through most of my adult life as opposed
  to this real break with what had previously been known as
  conservatism. One of the things I saw specifically when I was covering
  George W. Bush's re-election in 2004 was a lot of talk about the,
  quote unquote, ``homosexual agenda'' as a kind of totalizing ideology.
  So when I read what you've written about, the use of anti-gay
  prejudice almost as a kind of conspiracy theory in Polish politics,
  that seemed really familiar to me.
\item
  anne applebaum\\
  So I'm not really disagreeing with you about that. I don't think it's
  incorrect to say that some of the seeds of what we're seeing now were
  present 15 or 20 years ago. I think my disagreement is that I believe
  that the anti-communist movement, which is, remember, the book begins
  at a slightly different moment, not in 2007, but in 1999, and it's
  reflecting on an earlier period, which is the late 1980s and early
  1990s. I'm arguing that the anti-communist movement in that period had
  different components, and people were anti-communist for different
  reasons. Some of them were anti-communist because they believed in
  Realpolitik and they were worried about Soviet nuclear weapons, OK?
  And some people were anti-communist because they cared about democracy
  and human rights and the rule of law. And there were some people who
  were anti-communist because they were Christians and the Soviet Union
  was Marxist and atheist, and therefore they were against it for that
  reason. And one of the things that I trace in the book is this idea
  that that coalition over time break up. I think I even say at one
  point that it was probably artificially held together by 9/11, which
  kind of kept some people on the same team for longer than they might
  have been anyway. I mean, you could even argue that it was breaking up
  by the end of the `90s in the United States. And the same is true in
  Poland, where the anti-communist movement in Poland had people in it
  whom you would describe as left-wing. There were a lot of
  anti-communist children of communists and so on. As these different
  bits of it broke up, some people went in one direction and some people
  went in another. So I'm not actually disagreeing about that. I mean, I
  think what I'm disagreeing about is your implication that this was
  somehow inevitable, that, of course, the right would end up like that
  because the right was always authoritarian, and that's what it was
  about. That I disagree with. If you'd been looking at the left in the
  1980s, you might have been able to make that same argument. I was
  actually looking for a copy of my own book, which I can't seem to
  find, which is my history of the Gulag that was published in 2004.
\item
  michelle goldberg\\
  Right.
\item
  anne applebaum\\
  And in the introduction to that book, I talk a little bit about the
  communist tolerant part of the Western left, which chose to ignore and
  dismiss and downplay the existence of Soviet concentration camps
  because it was politically inconvenient. There was a piece of the left
  that was always also charmed by authoritarianism and tempted by
  extremism. Well, that's also in my new book a little bit too. But if
  I'd said at that time, well, of course, it was inevitable that the
  left will therefore from this moment develop into an authoritarian
  movement, I would have been wrong. I mean, it could have been that
  those elements took over the left, and actually in one or two
  countries they did.
\item
  michelle goldberg\\
  Right.
\item
  anne applebaum\\
  The point is that these movements were, particularly in the United
  States where we're cursed with our two party system, these movements
  are coalitions, and I was simply dissecting what happened to the
  right-wing coalition or the center right coalition that I felt myself
  to be a part of and clearly no longer do. I simply don't think that
  Trumpism was some kind of inevitability, that it had to be that way
  because that was the way the party was going. That there are these
  elements of cultural despair of deep pessimism about American society
  of fear and anxiety caused by modernity and technological change and
  social change, that that is all present on the right right now, of
  course, it's absolutely true. I write about it all the time, and it's
  part of my book. What I don't agree with is that retrospectively
  colors all anti-communists as proto-authoritarians, which would be
  ridiculous.
\item
  michelle goldberg\\
  Oh, I don't think--- no, I mean, I don't think it retrospectively
  colors all anti-communists as proto-authoritarians. I just think ---
  and I think that this comes out in your book --- that the segment of
  that coalition that was genuinely concerned about human rights and
  democracy is smaller than, I think, people would have admitted at the
  time.
\item
  anne applebaum\\
  I mean, depends on which country, and it depends which time we're
  talking about. A lot of this is also to do with who won power and how
  and when. The anti-liberal or anti-democratic part of the Polish
  right, for example, was a fringe minority for 25 years, which had very
  little influence on Polish politics and was not even really visible.
  And even the Law and Justice party, which is now the ruling party, it
  was very briefly in power once before for a year and a half, and it
  exhibited a much more limited and narrow and less ambitious version of
  what it's doing now. It was still then a coalition that contained a
  wide range of people. Even the Law and Justice president, who was the
  one who died in the tragic plane crash, was somebody who cared a lot
  about the rule of law, for example, and polish judges and so on, and
  it's hard to imagine him presiding over this destruction of the
  judicial system that's taking place now. So that's around 2005. Even
  then, they were the truly anti-democratic anti-pluralist piece of the
  party was a fringe piece of the extreme. And that it has now moved to
  the center is the drama of the current story.
\item
  frank bruni\\
  Anne, I kept on wondering as I was reading you, when you look at the
  group of you on the right, conservatives, however you want to call
  yourselves, when you look at your kind of wide group of friends and
  acquaintances, not just in Poland, but also in Britain and the United
  States, and I assume you had some in Hungary as well, and you kind of
  observe who drifted readily, or at least eventually, toward the lure
  of authoritarianism or some of these more nativist administrations or
  governments, and who, like you, resisted or like the Never Trumpers
  here, what are the kind of character or logical or socioeconomic
  differences, what distinguishes the people who decided to go along
  versus the people who said, no way?
\item
  anne applebaum\\
  So first of all, to be clear, there are no sociological differences. I
  mean, we are talking about elite people. We're talking about educated
  sophisticated people who were not damaged by Poland's post-communist
  resurgence. Quite the contrary. I hesitate to generalize because one
  of the things I liked about this book is that, unlike as in writing
  columns, I didn't really have to generalize. I could just focus on a
  few people. But very often in Poland, a lot of the people that I'm
  talking about are people who, for one reason or another, became
  dissatisfied. I mean, so for example, in Poland with the
  post-communist state. And sometimes they were dissatisfied for
  political reasons, and often they were dissatisfied for personal
  reasons. So they personally had not prospered to the degree to which
  they felt they should have done.
\item
  frank bruni\\
  Right.
\item
  anne applebaum\\
  And so there is a quality of resentment, a feeling that the wrong
  people have risen to power, the wrong people are achieving prominence.
  Our business class is not the right kind of people. So there's a
  feeling that the wrong people have somehow won or are somehow in
  charge. And you can look back through history, and the sort of classic
  profile of the revolutionary is somebody who's part of the elite, but
  not on top. So look at Lenin. Lenin came from the Russian aristocracy,
  but the lowest rung of the aristocracy, right? And then his family
  even dropped out of the aristocracy after his brother was sentenced as
  a revolutionary, and he has fantastic resentment, even though he's not
  from the working class. He's not the poorest part of the country, but
  he has this feeling of resentment against Russia's rulers. OK, the
  Russian revolution is a more complicated story than that. It wasn't
  all about Lenin's personal psychosis. I don't want to keep going with
  that analogy. I'm just saying that if you do look back, you do find
  that a lot of radicals often fit that profile. And I think even in the
  U.S., Laura Ingraham, who I write about at some length, isn't somebody
  that I do know that well. She doesn't really count as a friend. She
  was just somebody I met a few times. And we have some mutual friends
  and so on, but I'm not close to her. But she is somebody who I know
  was always very resentful of the fact that her achievements weren't
  more recognized. So she also has that quality of resenting the elite,
  which she is also somehow part of. I mean, to say that she's not an
  elite is ridiculous. I mean, she's ---
\item
  michelle goldberg\\
  Did you ever read her book about what an elite is?
\item
  anne applebaum\\
  I read several of her books. I can't remember.
\item
  michelle goldberg\\
  She wrote this book in 2003 where she basically defines elitism as a
  state of mind that sort of has nothing to do---
\item
  anne applebaum\\
  With your actual---
\item
  michelle goldberg\\
  ---with class.
\item
  anne applebaum\\
  ---class. Yeah, right. Right.
\item
  michelle goldberg\\
  So it's kind of this bizarre fantasmagorical version of elitism in
  which a person with the right politics can never, by definition, be
  part of the elite.
\item
  anne applebaum\\
  Right. So she has defined herself as somebody --- even though she's
  extremely wealthy, television presenter with millions of social media
  followers and a huge fan base, and a large house and all that, she
  nevertheless defines herself as being somehow anti-elite, which is, of
  course, ridiculous.
\item
  michelle goldberg\\
  Right.
\item
  anne applebaum\\
  If you're asking what people like that have in common, this is often
  it. There's some element of resentment or a feeling of outsiderism.
\item
  michelle goldberg\\
  But do you think she's changed? Because in your book, you sort of
  associate her with this optimistic post-Cold War milieu of Reaganism
  and people who felt like we won, and now we're going to go on to keep
  winning. Whereas when I look at her history, I mean, starting with
  Dartmouth where she is the editor of this school newspaper with Dinesh
  D'Souza, where they become famous for trying to humiliate gay students
  by sending an undercover reporter to the gay student group's meeting
  and publishing excerpts of things that people talked about. Her book
  in 2003 is obviously extremely resentful and nationalist. I guess I
  don't see a Laura Ingraham that was at one point different or better
  than the one that we see today.
\item
  anne applebaum\\
  No, I'm not sure that Laura Ingraham was ever different or better. Her
  views about the world have certainly changed. I mean, she has a much
  different view of American foreign policy. She has a different
  analysis of American history than she once had. I'm not intimate with
  her. I don't know her well enough to tell you exactly all the
  components of it, although I tease it apart a little bit in the book.
  But that she was an anti-communist at a time when other people on the
  left weren't is true. And so I probably gave her more credit for that
  than she deserved in the `80s and `90s. But to me, that was a really
  important quality, and it remained an important quality for me for a
  long time.
\item
  frank bruni\\
  Anne, I want to say something very cynical and ask something very
  cynical, but these last years have made me pretty cynical, which is,
  you used the verb ``prospered'' before. You were just talking about
  Laura Ingraham's views have changed. I find myself wondering as we
  talk about resentment as a motivating factor, the size of the chip on
  someone's shoulder, to what extent do they actually have convictions
  that change in ideologies, or are they just grabbing onto the winning
  formula that puts them on the top of the team or in the winner's
  circle or make sure that they prosper? I mean, are they just kind of
  choosing horses rather than, in fact, evolving ideologically?
\item
  anne applebaum\\
  Some of them. I mean, some of them are deeply cynical. So one other
  person who I talk about in the book who won't be familiar to American
  listeners is a Hungarian woman called Maria Schmidt, whom I suspect of
  deep and profound cynicism. She's somebody who has a number of both
  business and political interests in Hungary, and she's made great
  efforts to stay on the right side of the current ruling party and is
  now one of their chief propagandists, even though I know that in
  earlier eras she was more doubtful about them. And I suspect her of
  being profoundly cynical. So one of the things that I tried to argue
  in the book is that, again, the human personality is very complex, and
  people can be motivated by more than one thing. And some people are
  motivated by an idealism or an ideology or politics, and some people
  are motivated by personal ambition. Some people are motivated by
  resentment. Some people are motivated by the pleasure in seeing their
  enemies squirm. And some people are, as you say, profoundly cynical,
  and sometimes those qualities dominate at different moments. I mean, I
  actually end the book with this discussion of cynicism and nihilism,
  and I say that this is the really dangerous sentiment for liberal
  democracy is that. Because once people become cynical and once they
  think it's all a game and it doesn't matter who wins, then you do
  begin to lose the virtue of the state, and then you really are open
  for all kinds of extreme politics. But yeah, I mean, are they cynical?
  Some of them really are, yes.
\item
  michelle goldberg\\
  But there's also a relationship, right, between kind of doubt and
  fanaticism. If you're going to insist on something that on some level
  you know it not to be true, it's likely to make you more histrionic
  and more sort of uncompromising.
\item
  anne applebaum\\
  I think there's a quote in the book that I'm not going to be able to
  recite to you off the top of my head, but it's a quote actually that
  comes from a guy who was a young communist in Poland in the 1940s,
  Jacek Trzynadel. And he described in a famous memoir that he wrote
  years later, he described what it felt like to be talking to a crowd
  of people. People are shouting and cheering, and he was lecturing the
  crowd whatever, support our Marxist state. I mean, I'm just making up
  what he said. I don't remember the exact thing. And he writes, the
  more I was shouting, the more I was doubting what I was saying. Did I
  really believe what I was shouting? And I felt I had to shout louder
  not just to convince the crowd, but to convince myself.
\item
  frank bruni\\
  I think I was talking a little bit about something else, which is just
  rank opportunism. We were talking moments ago about Laura Ingraham.
  One of her Fox News colleagues Tucker Carlson, if you go back and you
  listen to Tucker Carlson 15, 20 years ago, he doesn't sound exactly or
  really all that much like he sounds now. And his voice has evolved in
  relationship to where he's gotten the largest audience, the most
  traction, the most power. And Donald Trump, I mean, does any one of us
  really believe that Donald Trump has an intrinsic and abiding ideology
  versus a set of hunches and instincts about which direction to turn in
  to maximize the adulation and the purchase on power?
\item
  michelle goldberg\\
  But actually, Frank, I mean, I would say I actually think that racism
  is pretty deep in Donald Trump and is as close to a sincere belief as
  he's possible of holding.
\item
  anne applebaum\\
  So actually, Tucker Carlson, who I didn't write about, but is probably
  a better example, as you say, of somebody who has changed very
  profoundly. David Frum has a thesis about Carlson, which is that it
  was because what he really, really, really, really always wanted to do
  was be on television and be good at television. He tried various
  television projects, and they kept failing. And then finally, he hit
  on the formula that would win him millions of fans, and that was this
  kind of strange racist, nationalist, nativist language that is
  designed to appeal to mobs of like thinkers in America and that that
  was what finally brought him fame and glory, and that's why he's
  sticking with it. And so, yes, that would be an example of really sort
  of pure opportunism and a kind of nihilism. And that, as I say, is one
  of the enemies of good politics.
\item
  frank bruni\\
  But where would you put Trump, Anne? Is Trump more Tucker or more
  Lenin? {[}GOLDBERG AND APPLEBAUM LAUGH{]}
\item
  anne applebaum\\
  Actually, a bit of --- {[}LAUGHS{]} golly, that's not a question I
  ever thought I was going to have to answer.
\item
  frank bruni\\
  It's not a question I ever thought I'd ask. {[}BOTH LAUGH{]}
\item
  anne applebaum\\
  So, I mean, actually, you can see elements of both. I mean, think
  about it. Trump is somebody who he's very rich. He's on the front page
  of tabloids. And yet, is he ever really accepted by the New York elite
  who he resents? Is he accepted by the real movers and shakers, the
  real bankers and businessmen in New York? Do they see him as a real
  player? I mean, no, they don't. I mean, for all I know, he may be
  deeply resentful of the people who look down on him because he was
  vulgar and so on. So he may well fit into that sort of, in that sense,
  the Leninist category, and at the same time be someone who, as
  Michelle said, is somebody who just has this instinct for kind of
  negative publicity. He understands how to move crowds and how to
  appeal to really base instincts. So he's somebody you can see both of
  those aspects in.
\item
  michelle goldberg\\
  Now, one of the things that I think is really valuable about your book
  is its kind of international scope. So it's not just about what
  happened to the Republican Party. But to me, one of the questions that
  keeps --- this may be a provincial question, but compared to maybe if
  Hungary is the worst case scenario out of all of the countries that
  you're writing about in this book, possibly followed by Poland, these
  are both countries where, as you write about, there's sort of a
  conspiracy at the very center of the government. Where do you see the
  United States on that trajectory? How far are we from Poland, or how
  far are we from Victor Orbán's Hungary?
\item
  anne applebaum\\
  So this may sound very bizarre, and I accept that it's my own weird
  provincialism, and it's my strange perspective as an American who
  lives at least part of the time in Poland. I think the United States
  and Poland are really similar in the sense that I do think that Trump
  came to power also on the back of a conspiracy that all of us ---
  including me, but also many other people --- underrated, which was the
  conspiracy of birtherism. And it's now sort of forgotten because so
  many things have happened, that this was the moment when he really
  broke into national politics was by being the kind of the loudest
  spokesman for birtherism. And what was birtherism? It was the argument
  that the president of the United States is illegitimate. He's not even
  American. And this argument had a tremendous amount of power and was
  believed by an enormous number of people. I saw a statistic once, it's
  between 20 percent and 30 percent of Americans believed this to be
  true. Now, think about what that means. If the president is
  illegitimate, that means that everybody --- the media, the courts, the
  Congress, the civil service --- everybody is lying to you. So the
  entire state is covering up this fact that the president is an illegal
  outsider and should not be president at all. If you believe that, OK,
  then you are ready for all kinds of radical changes and all kinds of
  radical politics because you have come to doubt all of the
  institutions of your democracy.
\item
  michelle goldberg\\
  Oh, that's really interesting. So that creates much more of a kind of
  --- I don't think that I've thought before that there is such a
  through line between birtherism and then the deep state conspiracy.
\item
  anne applebaum\\
  Oh, absolutely. No, no, I mean, the use of conspiracy theories and
  conspiracy thinking by Trump is very similar to the way this was used
  in Poland and where it's been used in other places. And by the way,
  throughout history, OK? The Bolsheviks were great conspiracy
  theorists. And the way in which he used it and the way in which he has
  consistently sought to undermine public trust in a range of
  institutions --- the deep state, the fake news media, the judges who
  are really Mexicans, they're not Americans --- all that language used
  over and over. Some of it I think he does in a calculating way and
  some of it is instinctive. I mean, this is genuinely, I think,
  probably how he sees the world. All of this has built up and amplified
  the existing lack of trust in American society and helped to undermine
  people's faith in institutions. I mean, if you think American
  democracy is so rotten that we had a president for eight years who
  wasn't even American and it was all lied to us, then you're willing to
  see all kinds of things overthrown. Then what do you care about the
  State Department being decimated or the Inspector General of the
  C.I.A. being fired? Why should you care? All those people, it's so
  obvious that all of it is corrupt. And this is something that Trumpism
  has in common with the way politics are done not just in Poland, but
  in Brazil and in many other countries.
\item
  frank bruni\\
  Anne, I ask you this because you are a student, a scholar of
  authoritarians and authoritarianism, and because I suspect it's a
  question that's on a lot of our listeners' minds. Donald Trump, let's
  say he loses in early November. Do you think there's a real
  possibility that he rejects the results? And what does that look like?
  And if he wins, what do you suspect the next four years look like?
\item
  anne applebaum\\
  So those are two different questions. I mean, first of all, I think
  it's almost certain that he is going to question the results. He's
  said so already. He's been asked point blank, will you accept the
  result? And he's refused to say yes. So I think that is a clear
  possibility. It's something everybody should be prepared for. It may
  be that there is a very easy solution to that if it happens, and the
  solution is I think Biden himself has said, well, on January the 20th,
  he'll walk into the White House, and the Secret Service will escort
  Mr. Trump out because Trump will have trouble fighting the legality of
  it. But that doesn't mean that he can't cause an enormous amount of
  trouble or do an enormous amount of damage by campaigning around the
  country between November and January and saying that the election was
  rigged and seek once again to use conspiracy theory to move people and
  prepare them for some other political project still to come. And so,
  yes, it's possible. But I should also say that I also think it's
  possible he'll try to cheat in other ways. I mean, I think the attempt
  to undermine absentee ballots, the hints that he might undermine the
  work of the post office so that it couldn't deliver absentee ballots,
  I mean, all of that is preparing both to try and cheat and to prevent
  people from voting, and also to begin to make the case that the
  election was rigged. So it is highly possible he'll do that. Second
  question --- well, no, sorry, your second question was about what if
  he wins.
\item
  frank bruni\\
  If he wins, is that it? Are we on a fast track, an express train to
  authoritarianism?
\item
  anne applebaum\\
  So if he wins, I mean, the main difference between Trump's first term
  and Trump's second term will be the kinds of people working for him.
  Because if you remember, in the early part of his presidency, there
  were still people working for him who believed they were there to do
  something good and they were there to protect the institutions of the
  United States. And although I've criticized James Mattis, I do accept
  that he thought that what he was doing there was making sure that the
  army was OK and that elements of U.S. foreign policy, like our
  relationship with allies, were maintained. And a number of people who
  worked for Trump in the beginning had that commitment to the
  institutions and to democracy. In Trump's second term, there will be
  no such people. And so the government, the institutions will be run by
  people who are intent on destroying what they can and who will do so
  rather aggressively. And yes, I do think that would cause an enormous
  crisis in American democracy.
\item
  frank bruni\\
  The other scenario is Biden wins, Trump leaves voluntarily or is
  escorted out. What then happens to the authoritarian fervor that he
  has cultivated during his four years when he's out of office?
\item
  anne applebaum\\
  So this is a really interesting question, and one that I have been
  talking to people about. Most think that --- and these are mostly
  ex-Republicans, but people who know the party quite well --- most
  people think that everything depends on how he loses. So if he loses
  by a lot, if it's a wipeout, if the Republican Party loses the Senate
  and if it loses even more seats in the House, then it is possible that
  a part of the party leadership will say we're right, that was a big
  disaster, and that was a really bad idea. And Trumpism was a road to
  nowhere, and it's a road to an ever-shrinking base. And therefore,
  there will be an election or an argument inside the party, and a new
  kind of leader will be chosen. And then as somebody said to me
  recently, and then there's at least a chance that we get rid of this
  authoritarian or we push to the fringes this authoritarian streak in
  the Republican Party. It doesn't mean that we will, but it means that
  then there's a chance. If the election is very close, on the other
  hand, if Trump loses very slightly, if the Republicans retain the
  Senate, and if the party still feels that this kind of rhetoric and
  this kind of language is a winning ticket, then the next presidential
  candidate may well be Tucker Carlson or Don Jr. {[}BRUNI LAUGHS{]} or
  Ivanka or someone who ---
\item
  frank bruni\\
  So if it's Tom Cotton, we're lucky, right?
\item
  anne applebaum\\
  If it's Tom Cotton, you're lucky.
\item
  frank bruni\\
  That's actually --- {[}APPLEBAUM LAUGHS{]}
\item
  anne applebaum\\
  Or Mike Pompeo. I mean ---
\item
  frank bruni\\
  Ugh.
\item
  anne applebaum\\
  --- there's a range of people who will be competing for the role of
  the next and possibly more efficient and more intelligent Trumpist
  leader. And so then we are in a real national conundrum because then
  we have one political party which is not dedicated to playing by the
  rules of the game, and then the kind of polarization that we've had
  will continue and get worse. And then we will find ourselves in the
  position of having every four years a kind of electoral moment of
  crisis.
\item
  michelle goldberg\\
  So obviously, a lot of these trends that we're talking about in other
  countries predate the election of Donald Trump. But I'm curious how
  you think the --- I mean, collapse isn't the right word, but the kind
  of damage incurred to the ideal of liberal democracy in America has
  empowered authoritarianism in other countries and how the sort of fate
  of this election will then--- the echoes that will have in other
  countries that have basically argued that liberal democracy has
  failed, that it was a joke, that its time is over.
\item
  anne applebaum\\
  So the election of Donald Trump, the re-election of Donald Trump,
  would have an enormous negative impact around the world, and it would
  absolutely empower other authoritarians, both sitting ones, whether in
  Russia or in China, for that matter, or in the Middle East, as well as
  aspiring ones in other countries that are current democracies. And
  this is both by the power of example. Trump and his language are
  quoted and imitated by people all over the world. I mean, for example,
  his use of the expression ``fake news'' to undermine the media is one
  that has been picked up and repeated by dozens of other authoritarian
  leaders or would-be authoritarian leaders. I mean, the other part of
  the story is that the authoritarian right, the new radical right
  around the world has deep links. There are deep financial links. There
  are deep links, kind of organic links on social media. These groups of
  people speak to one another. They share one another's memes. They
  communicate in different ways across borders. And the influence of the
  American alt-right on the European far right or the international
  alt-right is quite profound and vice versa. I think they influence one
  another. And there is no question that the re-election of Donald Trump
  would embolden and motivate those different groups who, as I say, are
  speaking to one another all the time anyway and help them propagate
  their ideas further.
\item
  frank bruni\\
  Anne, can I ask you a meta question, so to speak? In the very title of
  your book Twilight of Democracy and I think sort of the subtext of a
  lot of the discussion that we're all having here today is this notion
  that the trend line is away from democracy. The trend line is toward,
  very scarily, authoritarianism. I just want to kind of play devil's
  advocate maybe in a very hopeful vein. Poland just had an election. I
  believe--- you can correct me if I'm wrong --- was the closest
  election since the end of communism. And in fact, the authoritarian
  government barely won. You go back to 2016. Trump got almost 3 million
  fewer votes than Hillary Clinton did in the popular vote. He won only
  after Russian interference, only after a late in the game assist from
  Jim Comey, and possibly also because so many people were so convinced
  the outcome was predetermined in Hillary Clinton's favor that they
  didn't rouse themselves to vote. And since then, almost every metric,
  almost every measure, almost every election has suggested the limits
  of Trump's appeal. Is it possible that what we're seeing in Poland and
  what we're seeing in America is, at the end of the day, a blip, he
  asked hopefully.
\item
  anne applebaum\\
  I mean, maybe. I, too, hope that that's the correct interpretation.
  And our country and other countries have overcome even worse crises
  before. I mean, the Civil War was a pretty big blip on the history of
  American democracy. It was a moment of total collapse, and somehow we
  nevertheless recovered from that with some caveats. I think one of the
  reasons I wrote the book and one of the reasons I concluded it the way
  I did is that what I don't want is for Americans to become complacent,
  for us to say, well, our democracy is so great and so strong and our
  Constitution is so fantastic, and we're a superpower. I don't have to
  try that hard to make sure that everything goes well in our country,
  and it's all going to go back to how it was eventually. And anyway,
  Poland is far away, and they are central Europeans, and they have
  nothing to do with us. I mean, I really believe that it is that
  conviction of inevitability which was the big mistake, I should say,
  of my entourage or my milieu, my friends in the `90s. There was a kind
  of complacency that now that we're on this road to democracy, nothing
  can take us back. Because that is wrong. Countries do reverse
  themselves, and they do cease to be democracies. And we are going
  through a really extraordinary moment of technological and
  informational and demographic and social change, and changes make
  society very volatile and all kinds of outcomes are possible. And I
  don't want people to be complacent or certain that we will somehow go
  back to everything being the way it was before, and you don't have to
  really do anything about it or worry about it. because it's that
  feeling of inevitability that turned out to be wrong.
\item
  michelle goldberg\\
  So I think we will end it there. And even though we have some
  disagreements about the nature of pre-Trump conservatism, it was a
  pleasure to read ``The Twilight of Democracy.'' It's a fascinating
  book, and I'm thrilled that you came on the show. Thank you so much.
\item
  frank bruni\\
  Anne, thank you so much for your time and your thoughtfulness.
\item
  anne applebaum\\
  Oh, thanks for talking to me.

  {[}MUSIC PLAYING{]}
\item
  frank bruni\\
  And we're back. Michelle, what did you think?
\item
  michelle goldberg\\
  So I think that she's extraordinarily brilliant obviously. And it's
  often funny when you disagree with someone on Twitter in a column, and
  then you end up talking and you find out you agree more than you might
  have thought. And I should say that Anne Applebaum is someone who her
  willingness to kind of call out authoritarianism on the right
  certainly didn't begin with Trump. Something I probably should have
  brought up is that she was at one time a big fan of John McCain, who
  refused to vote for him, I think very honorably, in 2008 because he
  put Sarah Palin on the ticket, right? And Sarah Palin's rise to
  prominence was in a lot of ways a premonition of what we're living
  through now. And so one of the things that is so interesting about
  this moment is how you end up being in political coalitions with
  people that you never could have imagined, right? It's such a reminder
  of how long life is. That it would have been hard enough for me to
  imagine maybe being on the same side as Anne Applebaum in 2000 or
  2003, never mind David Frum and Bill Kristol. But one of the things I
  wrote in my piece is that I think a lot of people on the left are
  very, very suspicious of Never Trump conservatives, and I really
  admire them because they've really sacrificed something to stand up to
  him. And they've sacrificed more than I've had to or more than people
  on the left have had to just in terms of the fissuring of personal
  relationships, professional networks, right? I mean, I think it's
  really not easy to turn your back on the social and political world
  that you've lived your whole life in.
\item
  frank bruni\\
  What you just described or what you just talked about is actually
  where I get my hope during these dark times. And what I mean by that
  is, you're right, you and Anne Applebaum agree more than you disagree,
  and that becomes clear when you talk with each other, and you have
  made temporary whatever kind of partnership with people whom you
  didn't expect to. I think that if I'm going to be an optimist, one
  thing that could come out of all of this --- and our conversation with
  Anne reacquainted me with this optimism --- is that we can focus on
  what unites us more than what divides us. Understanding what a threat
  Trump is, being so deeply offended by some of the divisions he's
  exploited and the ways he's governed has reminded so many people from
  various points of the political spectrum of what they have in common
  and of how overlapping their fundamental values are in so many ways.
  Don't you think?
\item
  michelle goldberg\\
  I guess, but I really wonder if that's more of an elite phenomenon,
  right? That sort of people who make their living by kind of arguing
  and writing in the search for truth and good faith argumentation are
  sort of necessarily in this moment on the same side. Instead of kind
  of politics--- I've written this before. Instead of politics being a
  war of ideas, the people who takes ideas seriously are aside in
  politics. I think if you look at the country as a whole, I don't see
  how you look out at America right now and say that this is a country
  where people are more aware of what unites them than divides them. I
  mean, to me, it looks like a country that's coming apart at the seams.
\item
  frank bruni\\
  I mean, it looks and feels that way. And yet, I don't think it's just
  an elite phenomenon. I mean, if I flash back over the last couple of
  years, once Trump had been in office for a while and the offenses had
  mounted and mounted and mounted, the number of times somebody in a
  store, in a fast food restaurant, wherever, has kind of set an aside,
  an aside of disgust that lets me know that they've developed the same
  reservations or deep concerns that someone writing op-ed pieces and
  simply using a different vocabulary is expressing. I mean, that
  happens all the time. And I look more recently --- and I think this
  has been so fascinating and hopeful --- I look at the polls regarding
  how Americans feel about Black Lives Matter and where that puts them
  in opposition to Trump, and I see an overwhelming majority of
  Americans coming together with the same idea of justice and rejecting
  Donald Trump. And ---
\item
  michelle goldberg\\
  No, I think you're right that Donald Trump has created --- and I think
  Jemele wrote about this, right, that Donald Trump has created, I don't
  know if it's a silent majority, but he's definitely created a fairly
  solid anti-Trump majority.
\item
  frank bruni\\
  Again, when I want to wear my optimist hat, I think Donald Trump in
  the long run, if things turn out better and not worse, I think Donald
  Trump has shown us who we don't want to be.

  At the end of the conversation with Anne, we talked in what I found to
  be a very fascinating manner about various scenarios. Trump barely
  loses. Trump loses big. Trump wins. Beyond all of those, there's a
  scenario where he loses, we move on. And again, this is my hope. This
  is me at my most hopeful and optimistic. Call me Pollyanna. We
  remember what Donald Trump did to us. We remember what he showed us in
  terms of the kind of America we don't want to be. And maybe we're the
  better for it.
\item
  michelle goldberg\\
  So I think that's certainly a possibility, right? I mean, if you look
  at all the reforms that you saw after Richard Nixon, I mean, most of
  them stood up fairly well until Donald Trump dismantled them. But
  there was not just this huge wave after, this huge Democratic wave
  that followed Nixon's resignation, but a whole flurry of lawmaking to
  try to solidify accountability, solidify transparency, solidify
  democratic institutions. I think it's going to be really, really
  important that people keep pushing on that. When Barack Obama first
  became president, there were people who wanted some sort of
  accountability for the people who had both misled the country into war
  under George W. Bush and also the people responsible for the financial
  crisis. And I understand why Obama, facing all these interlocking
  crises, thought that it was the time to sort of look forward rather
  than backward. But I think that was a mistake in retrospect, and it
  would be a fatal mistake if, inshallah, Biden becomes president. I
  think it will be really important to have something--- I don't know if
  it's a Truth and Reconciliation Commission, if it's a commission in
  the Justice Department --- to sort of expose both the kind of
  instances of corruption and lawlessness that we maybe half know about.
  And I'm sure there are a lot that we don't know about, right? I'm sure
  there are a lot of utterly corrupt phone calls with foreign leaders as
  bad as the one with the president of Ukraine that we just haven't
  heard about. And so I think it'll be really, really important to
  figure out the places where norms need to be replaced by laws.
\item
  frank bruni\\
  I suspect we will be excavating the muck of the Trump administration
  for decades to come. I just hope that excavation begins sooner rather
  than later. {[}GOLDBERG LAUGHS{]} But we should probably wrap it up
  there, Michelle. It's just you and me this week, which means 50/50
  chance that you're going to be the one doing the recommendation. And I
  just tossed a coin. Heads it's you. What's your recommendation this
  week?
\item
  michelle goldberg\\
  So I am going to recommend --- it's not a new book. I'm going to
  recommend an old book that I've been thinking about a lot recently and
  that I recently re-read, Milan Kundera's ``The Joke.'' So I don't know
  about you, but I have a lot of these books that I feel like have been
  really formative, and yet at the same time I can barely remember. But
  I decided to reread ``The Joke'' recently because we've been, or I've
  been at any rate, embroiled in this discussion over cancel culture,
  which even though I don't like the phrase cancel culture, it seems
  somewhat inescapable, and sort of what it means to be publicly shamed,
  what it means for a ideological movement to enforce its boundaries.
  And when I see people kind of cast out, I often think of this book.
\item
  frank bruni\\
  What is the storyline of ``The Joke``?
\item
  michelle goldberg\\
  OK, so the storyline of ``The Joke'' is there is a young communist in
  Czechoslovakia in the `40s, a very ardent communist, but somebody with
  a sort of slightly ironic temperament, who has a crush on this girl
  who is kind of earnest to a fault. And he teases her by sending her
  this jokey postcard --- hold on, I'll just get the actual line because
  I have it underlined here. {[}PAGES FLIPPING{]} So he teases her by
  sending her this jokey postcard that says, ``Optimism is the opium of
  the people. A healthy atmosphere stinks of stupidity. Long live
  Trotsky.'' And this postcard, this joke, ends up completely derailing
  his life. He's drummed out of the party, drummed out of the
  university. He's sent to go work in the mines. So in some ways, this
  is not at all comparable to what we talk about when we talk about
  cancellation, right? Because the sort of horror of it is that it has
  the power of the state behind it. And sometimes I think that people my
  age and older, those of us who grew up on Milan Kundera and books like
  it, there are certain rhetorical modes that signal to us
  authoritarianism. And we see them on Twitter and maybe overreact
  because what was so frightening about them was that they had the power
  of the state behind them. And yet at the same time, I think he sort of
  really captures the emotional experience, or what I imagine to be the
  emotional experience, of being sort of drummed out of a movement that
  you identify with. Because the significant thing about Ludvik is that
  he really is a communist, and so he really does identify with the
  people who are casting him out. He doesn't really want to be a
  dissident. I'll just quote this line: ``I came to realize that there
  was no power capable of changing the image of my person lodged
  somewhere in the supreme court of human destinies. But this image,
  even though it bore no resemblance to me, was much more real than my
  actual self, that I was its shadow and not it mine.'' I think that
  line captures some of what happens when people become reduced to
  either the worst thing that they've ever said or someone's impression
  of the worst thing that they ever said.
\item
  frank bruni\\
  It's so interesting to hear you describe it, Michelle, because one of
  the more celebrated novels to come out in the last six months and one
  of the novels I most recently read is called ``A Burning.'' I'm going
  to probably mispronounce the author's name, for which I apologize, but
  her name is Megha Majumdar, and it's set in India. And I'm realizing
  how timeless, what stamina the story you're describing from ``The
  Joke'' has. It is, again, about someone who, with a single
  communication, miscommunication, call it what you will, is branded
  forevermore and becomes a convenient object of analysis, demonization,
  caricature, whatever, for the society around her. So this is clearly a
  story that, regardless of epoch of decade, speaks to certain eternal
  verities of human nature.
\item
  michelle goldberg\\
  I had heard of that book, but I hadn't realized that that was what it
  was about. But now I'm gonna --- we can make that a recommendation too
  because I'm going to order it.
\item
  frank bruni\\
  But this is your week. Your recommendation again is?
\item
  michelle goldberg\\
  My recommendation is Milan Kundera's ``The Joke.'' That's our show
  this week. Thanks for listening. ``The Argument'' is a production of
  The New York Times opinion section. The team includes Phoebe Lett,
  Paula Schumann, and Pedro Rafael Rosado. Special thanks to Brad Fisher
  and Kristin Lin. See you next week. {[}MUSIC PLAYING{]}
\end{itemize}

Previous

More episodes ofThe Argument

\href{https://www.nytimes3xbfgragh.onion/2020/07/30/opinion/the-argument-authoritarianism-anne-applebaum.html?action=click\&module=audio-series-bar\&region=header\&pgtype=Article}{\includegraphics{https://static01.graylady3jvrrxbe.onion/images/2020/07/31/opinion/30argumentWeb-print/30argumentWeb-thumbLarge.jpg}}

July 30, 2020When Conservatives Fall for Demagogues

\href{https://www.nytimes3xbfgragh.onion/2020/07/23/opinion/the-argument-israel-palestinian.html?action=click\&module=audio-series-bar\&region=header\&pgtype=Article}{\includegraphics{https://static01.graylady3jvrrxbe.onion/images/2020/07/25/opinion/25audio/21argumentWeb-thumbLarge.jpg}}

July 23, 2020The Case for a One-State Solution

\href{https://www.nytimes3xbfgragh.onion/2020/07/16/opinion/the-argument-tammy-duckworth.html?action=click\&module=audio-series-bar\&region=header\&pgtype=Article}{\includegraphics{https://static01.graylady3jvrrxbe.onion/images/2020/07/17/opinion/16argumentWeb-print/16argumentWeb-thumbLarge.jpg}}

July 16, 2020A Conversation With Tammy Duckworth

\href{https://www.nytimes3xbfgragh.onion/2020/07/09/opinion/is-trumps-fate-sealed.html?action=click\&module=audio-series-bar\&region=header\&pgtype=Article}{\includegraphics{https://static01.graylady3jvrrxbe.onion/images/2020/07/10/opinion/10a2_audio/09argument1-thumbLarge.jpg}}

July 9, 2020Is Trump's Fate Sealed?

\href{https://www.nytimes3xbfgragh.onion/2020/07/02/opinion/the-argument-protest-statue-revolution.html?action=click\&module=audio-series-bar\&region=header\&pgtype=Article}{\includegraphics{https://static01.graylady3jvrrxbe.onion/images/2020/07/05/opinion/02argument-eightyfive1/02argument-eightyfive1-thumbLarge.jpg}}

July 2, 2020Whose Statue Must Fall?

\href{https://www.nytimes3xbfgragh.onion/2020/06/25/opinion/the-argument-biden-vice-president-supreme-court.html?action=click\&module=audio-series-bar\&region=header\&pgtype=Article}{\includegraphics{https://static01.graylady3jvrrxbe.onion/images/2020/06/28/opinion/25argument-eightyfour1/25argument-eightyfour1-thumbLarge.jpg}}

June 25, 2020Place Your Bets on Biden's V.P.

\href{https://www.nytimes3xbfgragh.onion/2020/06/18/opinion/the-argument-tom-cotton-resignation.html?action=click\&module=audio-series-bar\&region=header\&pgtype=Article}{\includegraphics{https://static01.graylady3jvrrxbe.onion/images/2020/06/20/opinion/18argument-eightythree1/18argument-eightythree1-thumbLarge.jpg}}

June 18, 2020Which Opinions Are Out of Bounds?

\href{https://www.nytimes3xbfgragh.onion/2020/06/04/opinion/the-argument-protest-riot-violence.html?action=click\&module=audio-series-bar\&region=header\&pgtype=Article}{\includegraphics{https://static01.graylady3jvrrxbe.onion/images/2020/06/04/opinion/04argument1/04argument1-thumbLarge.jpg}}

June 4, 2020Can Riots Force Change?

\href{https://www.nytimes3xbfgragh.onion/2020/05/28/opinion/the-argument-tara-reade-norma-mccorvey.html?action=click\&module=audio-series-bar\&region=header\&pgtype=Article}{\includegraphics{https://static01.graylady3jvrrxbe.onion/images/2020/05/28/opinion/28argument1/28argument1-thumbLarge.jpg}}

May 28, 2020Credibility and Converts: Revisiting Tara Reade and Jane Roe

\href{https://www.nytimes3xbfgragh.onion/2020/05/21/opinion/the-argument-de-blasio-cuomo-coronavirus.html?action=click\&module=audio-series-bar\&region=header\&pgtype=Article}{\includegraphics{https://static01.graylady3jvrrxbe.onion/images/2020/05/21/opinion/21argument-eighty1/21argument-eighty1-thumbLarge.jpg}}

May 21, 2020Did de Blasio Bungle the Crisis?

\href{https://www.nytimes3xbfgragh.onion/2020/05/14/opinion/the-argument-flynn-barr-coronavirus.html?action=click\&module=audio-series-bar\&region=header\&pgtype=Article}{\includegraphics{https://static01.graylady3jvrrxbe.onion/images/2020/05/14/opinion/14arguement-seventynine1/14arguement-seventynine1-thumbLarge.jpg}}

May 14, 2020Bill Barr's Junk Justice

\href{https://www.nytimes3xbfgragh.onion/2020/05/07/opinion/justin-amash-trump-biden.html?action=click\&module=audio-series-bar\&region=header\&pgtype=Article}{\includegraphics{https://static01.graylady3jvrrxbe.onion/images/2020/05/07/opinion/00argueSub/00argueSub-thumbLarge.jpg}}

May 7, 2020Who's Afraid of Justin Amash?

\href{https://www.nytimes3xbfgragh.onion/column/the-argument}{See All
Episodes ofThe Argument}

Next

July 30, 2020

\begin{itemize}
\item
\item
\item
\item
\item
\end{itemize}

\emph{\textbf{Listen and subscribe to our podcast from your mobile
device:}}

\textbf{\href{https://itunes.apple.com/us/podcast/the-argument/id1438024613?mt=2}{\emph{Apple
Podcasts}}} \emph{\textbf{\textbar{}}}
\textbf{\href{https://open.spotify.com/show/6bmhSFLKtApYClEuSH8q42}{\emph{Spotify}}}
\emph{\textbf{\textbar{}}}
\textbf{\href{https://play.google.com/music/m/Idxib4hsg3yviao4gtym76knjjy?t=The_Argument}{\emph{Google
Play}}} \emph{\textbf{\textbar{}}}
\textbf{\href{https://radiopublic.com/the-argument-Wdbepr}{\emph{RadioPublic}}}
\emph{\textbf{\textbar{}}}
\textbf{\href{https://www.stitcher.com/podcast/the-new-york-times/the-argument}{\emph{Stitcher}}}
\emph{\textbf{\textbar{}}}
\textbf{\href{https://rss.art19.com/the-argument}{\emph{RSS Feed}}}

How did the conservative defenders of classical liberal ideals like free
speech and the rule of law wind up abetting authoritarians across Europe
and the U.S.? With Ross out for the week, Frank and Michelle are joined
by Anne Applebaum, author of
``\href{https://www.penguinrandomhouse.com/books/621076/twilight-of-democracy-by-anne-applebaum/}{Twilight
of Democracy: The Seductive Lure of Authoritarianism},'' to debate the
origins of the center right's schism over nationalism. Then, if you've
got consternation over cancel culture, Michelle has ``The Joke'' for
you.

\includegraphics{https://static01.graylady3jvrrxbe.onion/images/2020/07/31/opinion/30argumentWeb-print/merlin_174247911_b0900092-2902-43d1-bf97-80a223ff9261-articleLarge.jpg?quality=75\&auto=webp\&disable=upscale}

\begin{center}\rule{0.5\linewidth}{\linethickness}\end{center}

\textbf{Background Reading:}

\begin{itemize}
\item
  Michelle's column
  ``\href{https://www.nytimes3xbfgragh.onion/2020/07/27/opinion/anne-applebaum-twilight-of-democracy.html}{Twilight
  of the Liberal Right,}'' responding to Anne's book.
\item
  Frank on the
  \href{https://www.nytimes3xbfgragh.onion/2020/07/11/opinion/sunday/republican-party-trump-2020.html}{Republican
  Party fissure}.
\item
  Anne Applebaum for The Atlantic:
  ``\href{https://www.theatlantic.com/ideas/archive/2020/07/polands-rulers-manufactured-a-rainbow-plague/614113/}{Poland's
  Rulers Made Up a `Rainbow Plague'}.''
\item
  Jamelle Bouie on the
  ``\href{https://www.nytimes3xbfgragh.onion/2020/07/24/opinion/trump-silent-majority.html}{Great
  Silent Majority}'' against Trump.
\end{itemize}

\begin{center}\rule{0.5\linewidth}{\linethickness}\end{center}

\textbf{How to listen to ``The Argument'':}

\emph{Press play or read the transcript at the top of this page, or tune
in on}
\href{https://itunes.apple.com/us/podcast/the-argument/id1438024613?mt=2}{\emph{iTunes}}\emph{,}
\href{https://play.google.com/music/listen?u=0\#/ps/Idxib4hsg3yviao4gtym76knjjy}{\emph{Google
Play}}\emph{,}
\href{https://open.spotify.com/episode/5fIsHqqunLBwoxPSUUSGre?si=Rz5D9VnlRFKdGMu8ixzBOw}{\emph{Spotify}}\emph{,}
\href{https://www.stitcher.com/podcast/the-new-york-times/the-argument}{\emph{Stitcher}}
\emph{or your preferred podcast listening app. After you listen, share
your opinion with us at}
\href{mailto:argument@NYTimes.com}{\emph{argument@NYTimes.com.}}

\begin{center}\rule{0.5\linewidth}{\linethickness}\end{center}

\hypertarget{meet-the-hosts}{%
\section{Meet the Hosts}\label{meet-the-hosts}}

\hypertarget{frank-bruni}{%
\subsection{Frank Bruni}\label{frank-bruni}}

Image

I've been an Op-Ed columnist for The Times since 2011, but my career
with the newspaper stretches back to 1995 and includes many twists and
turns that reflect my embarrassingly scattered interests. I covered
Congress, the White House and several political campaigns; I also spent
five years in the role of chief restaurant critic. As the Rome bureau
chief, I reported on the Vatican; as a staff writer for The Times's
Sunday magazine, I wrote many celebrity profiles. That jumble has
informed my various books, which focus on the Roman Catholic Church,
George W. Bush, my strange eating life, the college admissions process
and meatloaf. Politically, I'm grief-stricken over the way President
Trump has governed and I'm left of center, but I don't think that the
center is a bad place or ``compromise'' a dirty word. I'm
Italian-American, I'm gay and I write a
\href{https://www.nytimes3xbfgragh.onion/newsletters/frank-bruni}{weekly
Times newsletter} in which you'll occasionally encounter my dog, Regan,
who has the run of our Manhattan apartment.

\hypertarget{michelle-goldberg}{%
\subsection{Michelle Goldberg}\label{michelle-goldberg}}

Image

I've been an Op-Ed columnist at The New York Times since 2017, writing
mainly about politics, ideology and gender. These days people on the
right and the left both use ``liberal'' as an epithet, but that's
basically what I am, though the nightmare of Donald Trump's presidency
has radicalized me and pushed me leftward. I've written three books,
including one, in 2006, about the danger of right-wing populism in its
religious fundamentalist guise. (My other two were about the global
battle over reproductive rights and, in a brief detour from politics,
about an adventurous Russian émigré who helped bring yoga to the West.)
I love to travel; a long time ago, after my husband and I eloped, we
spent a year backpacking through Asia. Now we live in Brooklyn with our
son and daughter.

\begin{center}\rule{0.5\linewidth}{\linethickness}\end{center}

``The Argument'' is a production of The New York Times Opinion section.
The team includes Phoebe Lett, Paula Szuchman and Pedro Rafael Rosado.
Special thanks to Brad Fisher and Kristin Lin. Theme by Allison
Leyton-Brown.

Advertisement

\protect\hyperlink{after-bottom}{Continue reading the main story}

\hypertarget{site-index}{%
\subsection{Site Index}\label{site-index}}

\hypertarget{site-information-navigation}{%
\subsection{Site Information
Navigation}\label{site-information-navigation}}

\begin{itemize}
\tightlist
\item
  \href{https://help.nytimes3xbfgragh.onion/hc/en-us/articles/115014792127-Copyright-notice}{©~2020~The
  New York Times Company}
\end{itemize}

\begin{itemize}
\tightlist
\item
  \href{https://www.nytco.com/}{NYTCo}
\item
  \href{https://help.nytimes3xbfgragh.onion/hc/en-us/articles/115015385887-Contact-Us}{Contact
  Us}
\item
  \href{https://www.nytco.com/careers/}{Work with us}
\item
  \href{https://nytmediakit.com/}{Advertise}
\item
  \href{http://www.tbrandstudio.com/}{T Brand Studio}
\item
  \href{https://www.nytimes3xbfgragh.onion/privacy/cookie-policy\#how-do-i-manage-trackers}{Your
  Ad Choices}
\item
  \href{https://www.nytimes3xbfgragh.onion/privacy}{Privacy}
\item
  \href{https://help.nytimes3xbfgragh.onion/hc/en-us/articles/115014893428-Terms-of-service}{Terms
  of Service}
\item
  \href{https://help.nytimes3xbfgragh.onion/hc/en-us/articles/115014893968-Terms-of-sale}{Terms
  of Sale}
\item
  \href{https://spiderbites.nytimes3xbfgragh.onion}{Site Map}
\item
  \href{https://help.nytimes3xbfgragh.onion/hc/en-us}{Help}
\item
  \href{https://www.nytimes3xbfgragh.onion/subscription?campaignId=37WXW}{Subscriptions}
\end{itemize}
