Sections

SEARCH

\protect\hyperlink{site-content}{Skip to
content}\protect\hyperlink{site-index}{Skip to site index}

\href{https://myaccount.nytimes3xbfgragh.onion/auth/login?response_type=cookie\&client_id=vi}{}

\href{https://www.nytimes3xbfgragh.onion/section/todayspaper}{Today's
Paper}

\href{/section/opinion}{Opinion}\textbar{}Trump Might Try to Postpone
the Election. That's Unconstitutional.

\url{https://nyti.ms/33azE6H}

\begin{itemize}
\item
\item
\item
\item
\item
\item
\end{itemize}

Advertisement

\protect\hyperlink{after-top}{Continue reading the main story}

\href{/section/opinion}{Opinion}

Supported by

\protect\hyperlink{after-sponsor}{Continue reading the main story}

\hypertarget{trump-might-try-to-postpone-the-election-thats-unconstitutional}{%
\section{Trump Might Try to Postpone the Election. That's
Unconstitutional.}\label{trump-might-try-to-postpone-the-election-thats-unconstitutional}}

He should be removed unless he relents.

By Steven G. Calabresi

Mr. Calabresi is a co-founder of the Federalist Society and a professor
at Northwestern University's Pritzker School of Law.

\begin{itemize}
\item
  July 30, 2020
\item
  \begin{itemize}
  \item
  \item
  \item
  \item
  \item
  \item
  \end{itemize}
\end{itemize}

\includegraphics{https://static01.graylady3jvrrxbe.onion/images/2020/07/30/opinion/30Calabresi2/30Calabresi2-articleLarge.jpg?quality=75\&auto=webp\&disable=upscale}

I have voted Republican in every presidential election since 1980,
including voting for Donald Trump in 2016. I wrote op-eds and a law
review article protesting what I believe was an unconstitutional
investigation by Robert Mueller. I also wrote an op-ed opposing
President Trump's impeachment.

But I am frankly appalled by the president's recent tweet seeking to
postpone the November election. Until recently, I had taken as political
hyperbole the Democrats' assertion that President Trump is a fascist.
But this latest tweet is fascistic and is itself grounds for the
president's immediate impeachment again by the House of Representatives
and his removal from office by the Senate.

Here is what President Trump tweeted:

\begin{quote}
With Universal Mail-In Voting (not Absentee Voting, which is good), 2020
will be the most INACCURATE \& FRAUDULENT Election in history. It will
be a great embarrassment to the USA. Delay the Election until people can
properly, securely and safely vote???

--- Donald J. Trump (@realDonaldTrump)
\href{https://twitter.com/realDonaldTrump/status/1288818160389558273?ref_src=twsrc\%5Etfw}{July
30, 2020}
\end{quote}

The nation has faced grave challenges before, just as it does today with
the spread of the coronavirus. But it has never canceled or delayed a
presidential election. Not in 1864, when President Abraham Lincoln was
expected to lose and the South looked as if it might defeat the North.
Not in 1932 in the depths of the Great Depression. Not in 1944 during
World War II.

So we certainly should not even consider canceling this fall's election
because of the president's concern about mail-in voting, which is likely
to increase because of fears about Covid-19. It is up to each of the 50
states whether to allow universal mail-in voting for presidential
elections, and
\href{https://constitutioncenter.org/interactive-constitution/article/article-ii}{Article
II of the Constitution} explicitly gives the states total power over the
selection of presidential electors.

\includegraphics{https://static01.graylady3jvrrxbe.onion/images/2020/07/30/opinion/30Calabresi1/merlin_174987366_1e15384e-a2a6-4a17-962e-7968fc3552a7-articleLarge.jpg?quality=75\&auto=webp\&disable=upscale}

Election Day was fixed by a
\href{https://www.loc.gov/law/help/statutes-at-large/28th-congress/session-2/c28s2ch1.pdf}{federal
law} passed in 1845, and the Constitution itself in the
\href{https://constitutioncenter.org/interactive-constitution/amendment/amendment-xx}{20th
Amendment} specifies that the newly elected Congress meet at noon on
Jan. 3, 2021, and that the terms of the president and vice president end
at noon on Jan. 20, 2021. Even if President Trump disputed an election
he lost, his term would still be over on that day. And if no newly
elected president is available, the speaker of the House of
Representatives becomes acting president.

President Trump needs to be told by every Republican in Congress that he
cannot postpone the federal election. Doing so would be illegal,
unconstitutional and without precedent in American history. Anyone who
says otherwise should never be elected to Congress again.

\href{https://www.law.northwestern.edu/faculty/profiles/StevenCalabresi/}{Steven
G. Calabresi} is a co-founder of the Federalist Society and a professor
at Northwestern University's Pritzker School of Law.

\begin{center}\rule{0.5\linewidth}{\linethickness}\end{center}

\emph{The Times is committed to publishing}
\href{https://www.nytimes3xbfgragh.onion/2019/01/31/opinion/letters/letters-to-editor-new-york-times-women.html}{\emph{a
diversity of letters}} \emph{to the editor. We'd like to hear what you
think about this or any of our articles. Here are some}
\href{https://help.nytimes3xbfgragh.onion/hc/en-us/articles/115014925288-How-to-submit-a-letter-to-the-editor}{\emph{tips}}\emph{.
And here's our email:}
\href{mailto:letters@NYTimes.com}{\emph{letters@NYTimes.com}}\emph{.}

\emph{Follow The New York Times Opinion section on}
\href{https://www.facebookcorewwwi.onion/nytopinion}{\emph{Facebook}}\emph{,}
\href{http://twitter.com/NYTOpinion}{\emph{Twitter (@NYTopinion)}}
\emph{and}
\href{https://www.instagram.com/nytopinion/}{\emph{Instagram}}\emph{.}

Advertisement

\protect\hyperlink{after-bottom}{Continue reading the main story}

\hypertarget{site-index}{%
\subsection{Site Index}\label{site-index}}

\hypertarget{site-information-navigation}{%
\subsection{Site Information
Navigation}\label{site-information-navigation}}

\begin{itemize}
\tightlist
\item
  \href{https://help.nytimes3xbfgragh.onion/hc/en-us/articles/115014792127-Copyright-notice}{©~2020~The
  New York Times Company}
\end{itemize}

\begin{itemize}
\tightlist
\item
  \href{https://www.nytco.com/}{NYTCo}
\item
  \href{https://help.nytimes3xbfgragh.onion/hc/en-us/articles/115015385887-Contact-Us}{Contact
  Us}
\item
  \href{https://www.nytco.com/careers/}{Work with us}
\item
  \href{https://nytmediakit.com/}{Advertise}
\item
  \href{http://www.tbrandstudio.com/}{T Brand Studio}
\item
  \href{https://www.nytimes3xbfgragh.onion/privacy/cookie-policy\#how-do-i-manage-trackers}{Your
  Ad Choices}
\item
  \href{https://www.nytimes3xbfgragh.onion/privacy}{Privacy}
\item
  \href{https://help.nytimes3xbfgragh.onion/hc/en-us/articles/115014893428-Terms-of-service}{Terms
  of Service}
\item
  \href{https://help.nytimes3xbfgragh.onion/hc/en-us/articles/115014893968-Terms-of-sale}{Terms
  of Sale}
\item
  \href{https://spiderbites.nytimes3xbfgragh.onion}{Site Map}
\item
  \href{https://help.nytimes3xbfgragh.onion/hc/en-us}{Help}
\item
  \href{https://www.nytimes3xbfgragh.onion/subscription?campaignId=37WXW}{Subscriptions}
\end{itemize}
