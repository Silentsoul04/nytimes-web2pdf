Sections

SEARCH

\protect\hyperlink{site-content}{Skip to
content}\protect\hyperlink{site-index}{Skip to site index}

\href{https://www.nytimes3xbfgragh.onion/section/science}{Science}

\href{https://myaccount.nytimes3xbfgragh.onion/auth/login?response_type=cookie\&client_id=vi}{}

\href{https://www.nytimes3xbfgragh.onion/section/todayspaper}{Today's
Paper}

\href{/section/science}{Science}\textbar{}How the Ultimate Live-in
Boyfriend Evolved His Way Around Rejection

\url{https://nyti.ms/2PbWFy0}

\begin{itemize}
\item
\item
\item
\item
\item
\end{itemize}

Advertisement

\protect\hyperlink{after-top}{Continue reading the main story}

Supported by

\protect\hyperlink{after-sponsor}{Continue reading the main story}

\hypertarget{how-the-ultimate-live-in-boyfriend-evolved-his-way-around-rejection}{%
\section{How the Ultimate Live-in Boyfriend Evolved His Way Around
Rejection}\label{how-the-ultimate-live-in-boyfriend-evolved-his-way-around-rejection}}

Commingling tissues and blood would normally prompt a massive immune
response. These deep sea lovers found a workaround.

\includegraphics{https://static01.graylady3jvrrxbe.onion/images/2020/07/30/science/30TB-ANGLERFISH1/30TB-ANGLERFISH1-articleLarge.jpg?quality=75\&auto=webp\&disable=upscale}

By
\href{https://www.nytimes3xbfgragh.onion/by/katherine-j--wu}{Katherine
J. Wu}

\begin{itemize}
\item
  July 30, 2020
\item
  \begin{itemize}
  \item
  \item
  \item
  \item
  \item
  \end{itemize}
\end{itemize}

In a biopic about the mating rituals of
\href{https://www.nytimes3xbfgragh.onion/2019/07/29/science/anglerfish-bioluminescence-deep-sea.html}{anglerfish},
it's unclear what would earn the film its R-rating: the sex or the
violence.

Born into an inky deep sea world, the males of certain anglerfish
species exist solely to sniff out their mates. Upon locating his lady
(who might be up to 60 times his size), the male will nip at her belly.
His mouth then dissolves in a sludge of chemicals that physically fuse
him to his bioluminescent bride, forever commingling his blood and
tissues with hers.

This grotesque ritual, called sexual parasitism, looks just as bizarre
as it sounds. To an immunologist, though, the aesthetics of this macabre
form of unholy matrimony aren't actually the weirdest part.

Two genetically distinct animals, no matter how much in love, shouldn't
be able to merge their flesh without serious consequences, said Dr.
Thomas Boehm of the Max Planck Institute of Immunobiology and
Epigenetics. It's the same reason that transplanted organs often get
rejected by a recipient's body: Vertebrate immune systems are built to
wage war on any foreign matter.

And yet, some male anglerfish are full-time grafts --- the ultimate
live-in boyfriends. ``It really is a mysterious phenomenon,'' Dr. Boehm
said.

With the help of modern genetic sequencing, Dr. Boehm and his colleagues
may have solved this deep sea dilemma. Anglerfish have largely
jettisoned a branch of the immune system that's been a fixture of
vertebrate bodies for the last
\href{https://www.ncbi.nlm.nih.gov/pmc/articles/PMC3805090/}{500 million
years}, they report in
\href{https://science.sciencemag.org/lookup/doi/10.1126/science.aaz9445}{a
study published Thursday in Science}. The adaptation may help the
clingiest of couples stick together.

It's a substantial sacrifice to make, even for sex: Similar changes
would be lethal for humans --- and no other animals have yet been
documented doing the same.

``This is some of the cooler science I've read in a while,'' said Jesyka
Meléndez Rosa, an evolutionary biologist and expert in the genetics of
the immune system at the University of Puerto Rico who wasn't involved
in the study. ``It just goes to show, nothing is sacred.''

Anglerfish have good reason to
\href{https://www.nytimes3xbfgragh.onion/2020/07/16/science/ultra-black-fish.html}{resort
to extreme evolution}. Thousands of feet below the surface of the sea,
where the sun's rays don't shine, both food and friends are scarce. Many
males never actually manage to find a female. ``So if they do meet up,
what better thing to do than to bite and hold and stay that way?'' said
Theodore Pietsch, an evolutionary biologist at the University of
Washington and an author on the study.

That's probably why sexual parasitism has supposedly
\href{https://pubmed.ncbi.nlm.nih.gov/20178642/}{evolved}
\href{https://bioone.org/journals/copeia/volume-2007/issue-1/0045-8511(2007)7\%5b1\%3aPRODAO\%5d2.0.CO\%3b2/Phylogenetic-Relationships-of-Deep-sea-Anglerfishes-of-the-Suborder-Ceratioidei/10.1643/0045-8511(2007)7\%5B1:PRODAO\%5D2.0.CO;2.short}{multiple
times} across the anglerfish family tree. In some cases, the attachments
are temporary --- the boys hop on and off as they please. In others,
though, the males become permanent fixtures on the females.

These longer-lasting hookups come with a price. After males glom onto
their girls, their innards rapidly atrophy until little more is left
than a bulbous pair of testes, fringed with gills, protruding from the
female's flank like a sperm-filled saddlebag. ``There's basically no
integrity at this point,'' Dr. Pietsch said.

\includegraphics{https://static01.graylady3jvrrxbe.onion/images/2020/07/30/science/30TB-ANGLERFISH2/merlin_175072869_4710cf62-0fb5-4381-a810-f180c12e4842-articleLarge.jpg?quality=75\&auto=webp\&disable=upscale}

In the most extreme version of this trait, females of some species will
host up to eight male consorts at a time.

To figure out how the fish tissues tolerate each other, the researchers
sequenced the genes of 10 types of anglerfishes. They found that two of
the most decorated species, where females sported entourages of males,
had lost their ability to produce functional antibodies and T cells ---
two types of immune system sentinels that greatly underpin the body's
ability to distinguish its own cells from unfamiliar ones, and
annihilate inbound threats.

That strategy comes in handy when animals must contend with germs or
cancerous cells, said Zuri Sullivan, an immunologist at Yale University
who wasn't involved in the study. The so-called adaptive immune system,
to which antibodies and T cells belong, also
\href{https://www.nytimes3xbfgragh.onion/2019/10/31/health/measles-vaccine-immune-system.html}{helps
the body remember past encounters with pathogens} so they can be
vanquished again. ``It provides this huge benefit,'' Ms. Sullivan said.
``It's a big thing to lose.''

Similar genetic changes were present in anglerfish that melded
monogamously, though to a lesser degree. These more faithful fish still
had genes that allowed them to manufacture a limited selection of
disease-fighting antibodies, for instance.

Least altered of all were the ephemeral attachers, who seem to have
retained the genes for T cells and a blunted antibody response. In
general, the less durable the bond, the more intact the immune system,
Dr. Boehm said. That pattern makes sense: Short-lived flings between
partners can withstand some tissue rejection, but the stakes are far
higher when a relationship is for keeps.

Dr. Boehm said the data so far point to the possibility that
deterioration of anglerfish immunity preceded the rise of sexual
parasitism. But figuring out the order of these events is really ``a
chicken or egg situation,'' Dr. Meléndez Rosa said.

The researchers also don't yet know how anglerfish manage to ward off
infections. But there's more to the immune system than antibodies and T
cells; perhaps other members of this complex cavalry have risen in the
ranks to compensate, Ms. Sullivan said. ``Clearly, these animals are
doing fine,'' she said.

Finding these answers will likely require finding more rare deep sea
anglerfish. It took several years to amass 31 specimens with enough DNA
to analyze, Dr. Boehm said. But the researchers are up for the
challenge.

``We are not quite sure what lessons the anglerfish will teach us,'' Dr.
Boehm said. ``But we know they have done something really incredible.''

Advertisement

\protect\hyperlink{after-bottom}{Continue reading the main story}

\hypertarget{site-index}{%
\subsection{Site Index}\label{site-index}}

\hypertarget{site-information-navigation}{%
\subsection{Site Information
Navigation}\label{site-information-navigation}}

\begin{itemize}
\tightlist
\item
  \href{https://help.nytimes3xbfgragh.onion/hc/en-us/articles/115014792127-Copyright-notice}{©~2020~The
  New York Times Company}
\end{itemize}

\begin{itemize}
\tightlist
\item
  \href{https://www.nytco.com/}{NYTCo}
\item
  \href{https://help.nytimes3xbfgragh.onion/hc/en-us/articles/115015385887-Contact-Us}{Contact
  Us}
\item
  \href{https://www.nytco.com/careers/}{Work with us}
\item
  \href{https://nytmediakit.com/}{Advertise}
\item
  \href{http://www.tbrandstudio.com/}{T Brand Studio}
\item
  \href{https://www.nytimes3xbfgragh.onion/privacy/cookie-policy\#how-do-i-manage-trackers}{Your
  Ad Choices}
\item
  \href{https://www.nytimes3xbfgragh.onion/privacy}{Privacy}
\item
  \href{https://help.nytimes3xbfgragh.onion/hc/en-us/articles/115014893428-Terms-of-service}{Terms
  of Service}
\item
  \href{https://help.nytimes3xbfgragh.onion/hc/en-us/articles/115014893968-Terms-of-sale}{Terms
  of Sale}
\item
  \href{https://spiderbites.nytimes3xbfgragh.onion}{Site Map}
\item
  \href{https://help.nytimes3xbfgragh.onion/hc/en-us}{Help}
\item
  \href{https://www.nytimes3xbfgragh.onion/subscription?campaignId=37WXW}{Subscriptions}
\end{itemize}
