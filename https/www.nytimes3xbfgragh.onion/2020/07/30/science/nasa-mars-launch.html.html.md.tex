Sections

SEARCH

\protect\hyperlink{site-content}{Skip to
content}\protect\hyperlink{site-index}{Skip to site index}

\href{https://www.nytimes3xbfgragh.onion/section/science}{Science}

\href{https://myaccount.nytimes3xbfgragh.onion/auth/login?response_type=cookie\&client_id=vi}{}

\href{https://www.nytimes3xbfgragh.onion/section/todayspaper}{Today's
Paper}

\href{/section/science}{Science}\textbar{}NASA Launches Perseverance
Rover, Capping Summer of Missions to Mars

\url{https://nyti.ms/3fcM4xp}

\begin{itemize}
\item
\item
\item
\item
\item
\item
\end{itemize}

\href{https://www.nytimes3xbfgragh.onion/news-event/summer-of-mars?action=click\&pgtype=Article\&state=default\&region=TOP_BANNER\&context=storylines_menu}{Missions
to Mars}

\begin{itemize}
\tightlist
\item
  \href{https://www.nytimes3xbfgragh.onion/2020/07/30/science/nasa-mars-launch.html?action=click\&pgtype=Article\&state=default\&region=TOP_BANNER\&context=storylines_menu}{NASA
  Launch Highlights}
\item
  \href{https://www.nytimes3xbfgragh.onion/interactive/2020/science/mars-perseverance-tianwen-hope.html?action=click\&pgtype=Article\&state=default\&region=TOP_BANNER\&context=storylines_menu}{Meet
  the Spacecraft}
\item
  \href{https://www.nytimes3xbfgragh.onion/2020/07/28/science/nasa-jezero-perseverance.html?action=click\&pgtype=Article\&state=default\&region=TOP_BANNER\&context=storylines_menu}{NASA's
  Destination}
\item
  \href{https://www.nytimes3xbfgragh.onion/2020/07/28/science/mars-nasa-science.html?action=click\&pgtype=Article\&state=default\&region=TOP_BANNER\&context=storylines_menu}{Too
  Much Mars?}
\end{itemize}

Advertisement

\protect\hyperlink{after-top}{Continue reading the main story}

Supported by

\protect\hyperlink{after-sponsor}{Continue reading the main story}

\hypertarget{nasa-launches-perseverance-rover-capping-summer-of-missions-to-mars}{%
\section{NASA Launches Perseverance Rover, Capping Summer of Missions to
Mars}\label{nasa-launches-perseverance-rover-capping-summer-of-missions-to-mars}}

The third and final mission to the red planet of the month lifted off on
Thursday.

\includegraphics{https://static01.graylady3jvrrxbe.onion/images/2020/07/30/us/30hpmars/30hpmars-videoSixteenByNine3000-v2.jpg}

\href{https://www.nytimes3xbfgragh.onion/by/kenneth-chang}{\includegraphics{https://static01.graylady3jvrrxbe.onion/images/2018/02/16/multimedia/author-kenneth-chang/author-kenneth-chang-thumbLarge.jpg}}

By \href{https://www.nytimes3xbfgragh.onion/by/kenneth-chang}{Kenneth
Chang}

\begin{itemize}
\item
  July 30, 2020
\item
  \begin{itemize}
  \item
  \item
  \item
  \item
  \item
  \item
  \end{itemize}
\end{itemize}

NASA's Perseverance rover is headed to Mars, the third spacecraft to
head that way this month.

Perseverance, a robotic wheeled vehicle designed to look for signs of
past life on Mars, lifted off from Cape Canaveral in Florida on Thursday
at 7:50 a.m. Eastern time. The launch was pushed back a couple of weeks
by a series of technical delays and overcame challenges imposed by the
coronavirus pandemic, which required many of its engineers to work from
home.

The rover's destination is a crater, Jezero, which was once a lake in
the northern hemisphere of Mars. Scientists believe it is a promising
location where signs of ancient Martian life could be preserved if life
ever existed on Mars.

The Atlas 5 rocket lofted the spacecraft away from Earth and on a
trajectory to arrive at Mars in six-and-a-half months. It follows July's
earlier launches by the
\href{https://www.nytimes3xbfgragh.onion/2020/07/19/science/emirates-mars-mission.html}{United
Arab Emirates} and
\href{https://www.nytimes3xbfgragh.onion/2020/07/22/science/mars-china-launch.html}{China}.
While Perseverance is last to leave, all three missions should arrive at
the red planet at about the same time, in February.

For people at NASA's Jet Propulsion Laboratory in Southern California,
which will be responsible for operating the mission during its journey
to Mars, an earthquake with a magnitude of 4.2 provided a bit of an
extra jolt to the countdown. It did not affect the launch, but employees
working on the mission expressed their surprise on Twitter.

The launch was largely flawless, but a couple of hiccups emerged once it
began its movements toward Mars.

First, a few hours after launch, NASA was having some trouble
communicating with the spacecraft. ``It's something we've seen before
with other Mars missions,'' Jim Bridenstine, the NASA administrator,
said during a post-launch news conference.

The
\href{https://www.nytimes3xbfgragh.onion/2020/03/04/science/voyager-2-nasa-deep-space-network.html}{large
radio dishes of the Deep Space Network} that communicate with distant
spacecraft were receiving Perseverance's radio signals loud and clear
--- in effect, too loud.

As Mr. Bridenstine was speaking, Matt Wallace, the deputy project
manager, received a text message that engineers at the Jet Propulsion
Laboratory had made the adjustments that allowed the dishes to lock onto
the telemetry data.

Second, as the mission's controllers on the ground looked through the
telemetry, they learned that Perseverance had entered ``safe mode'' ---
a precautionary state when a spacecraft detects something not quite
right and waits for instructions from Earth. In a statement, the agency
said that part of the spacecraft was colder than expected while it
traveled in Earth's shadow. Temperatures have since warmed, and
engineers are working to bring Perseverance back into normal working
condition.

\hypertarget{what-is-the-perseverance-rover}{%
\subsection{What is the Perseverance
rover?}\label{what-is-the-perseverance-rover}}

Perseverance is a car-size wheeled robot nearly identical in design to
NASA's previous Mars rover, Curiosity, which landed in 2012. However,
Perseverance is headed to a different place --- a crater named Jezero
that was once a lake --- carrying a different set of instruments.
Curiosity was designed to look for habitable environments, and it found
signs of a freshwater lake. Perseverance is to go a step farther and
search for evidence of past life that might have lived in the lake at
Jezero.

\hypertarget{perseverance}{%
\subsection{Perseverance}\label{perseverance}}

The NASA mission includes Perseverance, a 2,200-pound rover, and
Ingenuity, an experimental Mars helicopter.

Ingenuity Helicopter

The four-pound aircraft will communicate wirelessly with the
Perseverance rover.

Solar Panel

Blades

Four carbon-fiber blades will spin at about 2,400 r.p.m.

Power

The plutonium-based power supply will charge the rover's batteries.

MAST

Instruments will take videos, panoramas and photographs. A laser will
study the chemistry of Martian rocks.

PiXl

Will identify chemical elements to seek signs of past life on Mars.

Antenna

Will transmit data directly to Earth.

Robotic arm

A turret with many instruments is attached to a 7-foot robotic arm. A
drill will extract samples from Martian rocks. The Sherloc device will
identify molecules and minerals to detect potential biosignatures, with
help from the Watson camera.

Perseverance Rover

The 2,200 pound rover will explore Jezero Crater. It has aluminum wheels
and a suspension system to drive over obstacles.

Ingenuity Helicopter

The aircraft will communicate wirelessly with the rover.

Solar Panel

Blades

Power

The plutonium-based power supply will charge the rover's batteries.

MAST

Instruments will take videos, panoramas and photographs. A laser will
study the chemistry of Martian rocks.

PiXl

Will identify chemical elements to seek signs of past life on Mars.

Antenna

Robotic arm

A turret with many instruments is attached to a 7-foot robotic arm. A
drill will extract samples from Martian rocks. The Sherloc device will
identify molecules and minerals to detect potential biosignatures, with
help from the Watson camera.

Perseverance Rover

The 2,200 pound rover will explore Jezero Crater. It has aluminum wheels
and a suspension system to drive over obstacles.

Solar panel

Ingenuity Helicopter

Blades

Power

Mast

PIXL

Antenna

Suspension

Perseverance rover

Robotic arm

A turret with many instruments is attached to a 7-foot robotic arm. A
drill will extract samples from Martian rocks. The Sherloc device will
identify molecules and minerals to detect potential biosignatures, with
help from the Watson camera. PiXl will identify chemical elements to
seek signs of past life on Mars.

By Eleanor Lutz \textbar{} Source: NASA

Perseverance is also carrying a couple of devices that are more fun than
scientific: several cameras, which will record various views as the
spacecraft zooms through the atmosphere en route to landing; and two
microphones, which will be the first to record sounds on another planet.

It is carrying an experimental helicopter, too.

\hypertarget{wait-a-helicopter}{%
\subsection{Wait, a helicopter?}\label{wait-a-helicopter}}

\includegraphics{https://static01.graylady3jvrrxbe.onion/images/2020/06/22/science/23sci-marscopter-animation-image/23sci-marscopter-animation-image-videoSixteenByNineJumbo1600.jpg}

Yup,
\href{https://www.nytimes3xbfgragh.onion/2020/06/23/science/mars-helicopter-nasa.html}{it's
called Ingenuity}. The four-pound Marscopter is a technology experiment,
and if it works, it will be the first powered flight on another planet.
The rotors have to spin at 2,400 revolutions a minute to generate lift
in the thin atmosphere of Mars, just one percent as dense at Earth's at
the surface.

\hypertarget{how-will-the-mission-help-future-astronauts}{%
\subsection{How will the mission help future
astronauts?}\label{how-will-the-mission-help-future-astronauts}}

A couple of experiments on Perseverance have nothing to do with
searching for past life, but they could help future life on Mars ---
astronauts from Earth.

One of the crucial supplies that astronauts will need is oxygen, for
breathing and as a rocket propellant.

The Mars Oxygen In-Situ Resource Utilization Experiment, or MOXIE, will
take carbon dioxide molecules from the Martian atmosphere and split them
into oxygen atoms and carbon monoxide.

MOXIE will try to demonstrate that is possible on the surface of the red
planet. But the amount of oxygen it could produce --- less than ounce
per hour --- is tiny.

``We're only making about enough oxygen to keep a small dog alive,''
said Michael Hecht, the principal investigator for MOXIE.

But if the idea works, the technique could be employed in the future on
a much larger scale to fill up a rocket. ``So astronauts in a future
Mars mission could take off from Mars to come home,'' he said.

Perseverance is also carrying samples of materials used in spacesuits,
mounted on a target used to calibrate one of the rover's instruments.

``When I send somebody to Mars in my spacesuit, I want to make sure that
they stay alive that whole time,'' Amy Ross, one of NASA's spacesuit
designers said during a news conference on Tuesday.

With Perseverance taking repeated measurements over a couple of years on
Mars, ``we can understand how our materials hold up or don't in that
environment,'' she said.

\hypertarget{when-will-the-mission-arrive-on-mars}{%
\subsection{When will the mission arrive on
Mars?}\label{when-will-the-mission-arrive-on-mars}}

Perseverance will land on Mars on Feb. 18 next year at 3:40 p.m. Eastern
time.

Every 26 months, Earth and Mars come close to each other, which allows
the quickest, most efficient trip from Earth to Mars. If the launch does
not occur by the middle of August, NASA would have to wait until the
next opportunity, in 2022.

\hypertarget{where-will-perseverance-land-and-why-is-it-going-there}{%
\subsection{Where will Perseverance land, and why is it going
there?}\label{where-will-perseverance-land-and-why-is-it-going-there}}

Jezero crater was filled with water about 3.5 billion years ago when
Mars was warmer and wetter. From orbit, earlier NASA spacecraft spotted
a dried-up river on one side of Jezero and an outflow channel can be
seen on the other side. The sediments of a fan-shape delta can be seen
where the river spilled into the crater. No one knows if anything ever
lived on Mars, but if it did, Jezero would be a prime place to look,
scientists decided.

\hypertarget{inside-jezero-crater}{%
\subsection{Inside Jezero Crater}\label{inside-jezero-crater}}

NASA's Perseverance rover will attempt to land in Jezero Crater, an
ancient Martian lake roughly the size of Lake Tahoe. If successful, the
rover will spend years exporing the river delta and making its way to
the crater rim.

Crater

rim

MARS

Jezero

Crater

Shoreline

JEZERO CRATER

River

delta

Possible

path of

rover

Possible

landing site

TARGET

LANDING AREA

1/2 mile

Canyon

carved by

a river

JEZERO CRATER

Shoreline

River

delta

Crater

rim

Possible

path of

rover

Possible

landing site

MARS

Jezero

Crater

TARGET

LANDING AREA

1/2 mile

Canyon

carved by

a river

JEZERO CRATER

Shoreline

River

delta

Crater

rim

Possible

path of

rover

Possible

landing site

MARS

Jezero

Crater

TARGET

LANDING AREA

1/2 mile

By Jonathan Corum \textbar{} Image by NASA, Jet Propulsion Laboratory,
European Space Agency, German Aerospace Center, Freie Universität Berlin
and Justin Cowart. Inset image by NASA and J.P.L.

\href{https://www.nytimes3xbfgragh.onion/2020/07/29/science/nasa-mars-perseverance-rover.html}{Landing
on Mars is difficult}. The planet's thin atmosphere isn't thick enough
to provide enough drag to slow down a spacecraft like Perseverance,
which will be arriving at more than 12,000 miles per hour. But the
atmosphere is still thick enough to generate thousands of degrees of
heat, complicating the task of slowing down Perseverance before it slams
into the ground. Quite a few landing attempts by NASA and other space
agencies have ended with creating new craters on the red planet's
surface.

But NASA has pulled off five consecutive successful landings. To
increase the likelihood that Perseverance rover will be the sixth, NASA
has made adjustments to the parachute that slows the spacecraft when it
reaches the Martian atmosphere. It has also improved the rover's ability
to identify a smooth landing site.

\hypertarget{what-else-is-going-to-mars-this-summer}{%
\subsection{What else is going to Mars this
summer?}\label{what-else-is-going-to-mars-this-summer}}

The
\href{https://www.nytimes3xbfgragh.onion/2020/07/14/science/mars-united-arab-emirates.html}{Emirates
Mars Mission} successfully lifted off on a Japanese rocket
\href{https://www.nytimes3xbfgragh.onion/2020/07/19/science/emirates-mars-mission.html}{on
July 20}.

The space program of the United Arab Emirates is modest, and its bid to
join the ranks of countries that have reached Mars is part of an
ambitious effort to inspire Emirati youth to take up careers in science
and technology.

Its Hope spacecraft will orbit Mars for a number of years, helping
scientists study the planet's weather cycles.

Image

The United Arab Emirates Hope mission launched from Tanegashima Space
Center in Kagoshima, Japan, on July 20.Credit...Hiroki Yamauchi/Kyodo
News, via Associated Press

Image

China's Tianwen-1 Mars mission lifted off from Wenchang Spacecraft
Launch Site in Hainan Province on July 23.Credit...Yang Guanyu/Xinhua,
via Associated Press

China launched
\href{https://www.nytimes3xbfgragh.onion/2020/07/22/science/mars-china-launch.html}{the
second mission, Tianwen-1,}on July 23.

The country's space program has seen a number of successes in recent
years, including
\href{https://www.nytimes3xbfgragh.onion/2020/02/26/science/china-moon-far-side.html}{two
rovers that landed on Earth's moon} as well as a pair of space stations
deployed in orbit. But its previous attempt to get to Mars in 2011 was
lost when the Russian rocket it was riding on failed and burned up in
Earth's atmosphere.

The new Chinese mission includes an orbiter, a lander and a rover. While
other countries have taken a staggered approach to visiting Mars --- an
orbiter first, then a lander, then finally a rover --- China emphasizes
that
\href{https://www.nytimes3xbfgragh.onion/2020/07/22/science/china-mars-mission.html}{it
will attempt to operate all of these components for the first time at
once}.

The orbiter, according to
\href{https://www.nature.com/articles/s41550-020-1148-6}{four scientists
involved in the mission}, will study Mars and its atmosphere for about
one Martian year, or 687 days on Earth. In addition to two cameras, the
spacecraft carries subsurface radar, a detector to study the Martian
magnetic field and three other scientific instruments.

The rover will try to land in the Utopia Planitia region in the
mid-northern Martian latitudes.
\href{https://mars.nasa.gov/mars-exploration/missions/viking-1-2/}{NASA's
Viking 2 mission} touched down there in 1976. Earlier studies using data
from NASA's Mars Reconnaissance Orbiter showed that Utopia Planitia has
\href{https://www.jpl.nasa.gov/news/news.php?feature=6680}{a layer of
water ice equivalent to what is found in Lake Superior} on Earth.

If it manages the perilous Martian landing, the rover will use a mix of
cameras, ground-penetrating radar and other instruments to better
understand the distribution of underground ice, which future human
colonists on Mars could use to sustain themselves. China's mission is to
last about 90 Martian days.

A fourth mission,
\href{https://www.nytimes3xbfgragh.onion/2020/03/12/science/mars-rover-coronavirus.html}{the
joint Russian-European Rosalind Franklin rover}, was to launch this
summer, too. But technical hurdles, aggravated by the coronavirus
pandemic, could not be overcome in time. It is now scheduled to launch
in 2022.

\hypertarget{what-other-spacecraft-are-currently-studying-mars}{%
\subsection{What other spacecraft are currently studying
Mars?}\label{what-other-spacecraft-are-currently-studying-mars}}

It's getting a bit crowded around the red planet.

Six orbiters are currently studying the planet from space. Three were
sent there by NASA: Mars Odyssey, launched in 2001; Mars Reconnaissance
Orbiter, in 2005; and MAVEN, which left Earth in 2013.

Europe has two spacecraft in orbit. Its Mars Express orbiter was
launched in 2003, and
\href{https://www.nytimes3xbfgragh.onion/2019/04/10/science/mars-methane-life.html}{the
ExoMars Trace Gas Orbiter}, which is shared with Russia's space program,
lifted off in 2016.

India operates the sixth spacecraft, the Mars Orbiter Mission, also
known as Mangalyaan, which launched in 2013.

Two American missions are currently operating on the ground.
C\href{https://www.nytimes3xbfgragh.onion/2019/06/22/science/nasa-mars-rover-life.html}{uriosity
has been roving since 2012}. It is joined by
\href{https://www.nytimes3xbfgragh.onion/interactive/2018/05/01/science/mars-nasa-insight-ar-3d-ul.html}{InSight,
a stationary lander that has been studying Marsquakes} and other inner
properties of the red planet since 2018. A third American mission, the
\href{https://www.nytimes3xbfgragh.onion/2019/02/13/science/mars-opportunity-rover-dead.html}{Opportunity
rover}, expired in 2019 when a dust storm caused it to lose power.

Advertisement

\protect\hyperlink{after-bottom}{Continue reading the main story}

\hypertarget{site-index}{%
\subsection{Site Index}\label{site-index}}

\hypertarget{site-information-navigation}{%
\subsection{Site Information
Navigation}\label{site-information-navigation}}

\begin{itemize}
\tightlist
\item
  \href{https://help.nytimes3xbfgragh.onion/hc/en-us/articles/115014792127-Copyright-notice}{©~2020~The
  New York Times Company}
\end{itemize}

\begin{itemize}
\tightlist
\item
  \href{https://www.nytco.com/}{NYTCo}
\item
  \href{https://help.nytimes3xbfgragh.onion/hc/en-us/articles/115015385887-Contact-Us}{Contact
  Us}
\item
  \href{https://www.nytco.com/careers/}{Work with us}
\item
  \href{https://nytmediakit.com/}{Advertise}
\item
  \href{http://www.tbrandstudio.com/}{T Brand Studio}
\item
  \href{https://www.nytimes3xbfgragh.onion/privacy/cookie-policy\#how-do-i-manage-trackers}{Your
  Ad Choices}
\item
  \href{https://www.nytimes3xbfgragh.onion/privacy}{Privacy}
\item
  \href{https://help.nytimes3xbfgragh.onion/hc/en-us/articles/115014893428-Terms-of-service}{Terms
  of Service}
\item
  \href{https://help.nytimes3xbfgragh.onion/hc/en-us/articles/115014893968-Terms-of-sale}{Terms
  of Sale}
\item
  \href{https://spiderbites.nytimes3xbfgragh.onion}{Site Map}
\item
  \href{https://help.nytimes3xbfgragh.onion/hc/en-us}{Help}
\item
  \href{https://www.nytimes3xbfgragh.onion/subscription?campaignId=37WXW}{Subscriptions}
\end{itemize}
