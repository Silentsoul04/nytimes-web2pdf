Sections

SEARCH

\protect\hyperlink{site-content}{Skip to
content}\protect\hyperlink{site-index}{Skip to site index}

\href{https://www.nytimes3xbfgragh.onion/section/sports}{Sports}

\href{https://myaccount.nytimes3xbfgragh.onion/auth/login?response_type=cookie\&client_id=vi}{}

\href{https://www.nytimes3xbfgragh.onion/section/todayspaper}{Today's
Paper}

\href{/section/sports}{Sports}\textbar{}SEC Becomes Latest College
Football League to Shrink Schedule

\url{https://nyti.ms/2PcOu4i}

\begin{itemize}
\item
\item
\item
\item
\item
\end{itemize}

Advertisement

\protect\hyperlink{after-top}{Continue reading the main story}

Supported by

\protect\hyperlink{after-sponsor}{Continue reading the main story}

\hypertarget{sec-becomes-latest-college-football-league-to-shrink-schedule}{%
\section{SEC Becomes Latest College Football League to Shrink
Schedule}\label{sec-becomes-latest-college-football-league-to-shrink-schedule}}

The Southeastern Conference, the most influential league in college
football, said it would delay the start of its season and have its teams
play only within the conference.

\includegraphics{https://static01.graylady3jvrrxbe.onion/images/2020/07/30/sports/30collegefootball/merlin_164191302_f199e0e3-0deb-4c75-beac-cc367da17c0b-articleLarge.jpg?quality=75\&auto=webp\&disable=upscale}

\href{https://www.nytimes3xbfgragh.onion/by/billy-witz}{\includegraphics{https://static01.graylady3jvrrxbe.onion/images/2018/02/16/multimedia/author-billy-witz/author-billy-witz-thumbLarge.jpg}}

By \href{https://www.nytimes3xbfgragh.onion/by/billy-witz}{Billy Witz}

\begin{itemize}
\item
  July 30, 2020
\item
  \begin{itemize}
  \item
  \item
  \item
  \item
  \item
  \end{itemize}
\end{itemize}

As universities desperately try to rescue the college football season
while
\href{https://www.nytimes3xbfgragh.onion/news-event/coronavirus}{the
coronavirus pandemic} continues to grip the country, the Southeastern
Conference became the latest --- and most influential --- league to push
back the start of its season and trim its schedule to conference-only
games.

The SEC's decision, made by the conference's 14 university presidents,
comes one day after the Atlantic Coast Conference came to a similar
conclusion --- also including Notre Dame, which is a member of the
conference in other sports but usually an independent in football.

The SEC announced it would move the start of the season back by at least
three weeks, to Sept. 26, and play 10 games, which would leave each
school with an open date --- along with another open weekend preceding
the conference championship game, which is now scheduled for Dec. 19.

The Big Ten and the Pac-12 conferences had previously announced they
would play only conference games this season. That leaves only the Big
12 among the so-called Power 5 conferences that has not amended its
schedule, though the conference on Thursday did cancel its media day,
which had already been pushed back.

The shrinking and fluid schedules are the extraordinary measures being
taken to ensure that universities can salvage the lucrative television
revenues that, along with an unpaid labor force --- the players --- fuel
the college athletics industrial complex. The SEC brought in \$720.6
million in revenue for the fiscal year ending last Aug. 31,
\href{https://www.usatoday.com/story/sports/ncaaf/sec/2020/01/30/sec-generated-721-million-revenue-still-trails-big-ten/2856234001/}{according
to USA Today}.

Even without fans, the schools still stand to receive considerable
payouts from their broadcast deals. The Big Ten has the richest TV
contract, paying its schools as much as \$54 million each last season.

The conference-only schedules have meant the cancellation of certain
rivalry games, like South Carolina and Clemson, and marquee interleague
games,
\href{https://www.nytimes3xbfgragh.onion/2020/05/17/sports/ncaafootball/college-football-no-fans.html}{like
Alabama and Southern California}, which had been scheduled in Arlington,
Tex., and Texas at Louisiana State.

But by playing only conference games, schools can mostly avoid traveling
outside their regions and also can more firmly standardize virus testing
and safety protocols.

``We believe these schedule adjustments offer the best opportunity to
complete a full season by giving us the ability to adapt to the fluid
nature of the virus and the flexibility to adjust schedules as necessary
if disruptions occur,'' SEC Commissioner Greg Sankey said in a statement
announcing the changes.

The SEC plans to announce its new schedule next month.

The college football season, though, is increasingly in jeopardy at a
time when the virus, which had caused
\href{https://www.nytimes3xbfgragh.onion/interactive/2020/us/coronavirus-us-cases.html}{nearly
152,000 deaths in the United States}as of Thursday, has shown few signs
of slowing. The difficulty of playing even a truncated season has been
on display in the last week in Major League Baseball, which saw the
Miami Marlins temporarily shut down --- and their opponents rescheduled
--- after half the roster contracted the virus.

More than a dozen college football teams --- with Michigan State,
Colorado State and Rutgers among the latest --- have been forced to
abandon football workouts over the summer after outbreaks. Meanwhile,
Penn State reported eight positive coronavirus tests among its athletes
on Wednesday.

Conditions have recently been dire in many of the 11 states that are
part of the SEC footprint --- including Florida and Mississippi, which
is experiencing the
\href{https://www.nytimes3xbfgragh.onion/interactive/2020/us/coronavirus-us-cases.html}{greatest
increase in cases per capita.}

Advertisement

\protect\hyperlink{after-bottom}{Continue reading the main story}

\hypertarget{site-index}{%
\subsection{Site Index}\label{site-index}}

\hypertarget{site-information-navigation}{%
\subsection{Site Information
Navigation}\label{site-information-navigation}}

\begin{itemize}
\tightlist
\item
  \href{https://help.nytimes3xbfgragh.onion/hc/en-us/articles/115014792127-Copyright-notice}{©~2020~The
  New York Times Company}
\end{itemize}

\begin{itemize}
\tightlist
\item
  \href{https://www.nytco.com/}{NYTCo}
\item
  \href{https://help.nytimes3xbfgragh.onion/hc/en-us/articles/115015385887-Contact-Us}{Contact
  Us}
\item
  \href{https://www.nytco.com/careers/}{Work with us}
\item
  \href{https://nytmediakit.com/}{Advertise}
\item
  \href{http://www.tbrandstudio.com/}{T Brand Studio}
\item
  \href{https://www.nytimes3xbfgragh.onion/privacy/cookie-policy\#how-do-i-manage-trackers}{Your
  Ad Choices}
\item
  \href{https://www.nytimes3xbfgragh.onion/privacy}{Privacy}
\item
  \href{https://help.nytimes3xbfgragh.onion/hc/en-us/articles/115014893428-Terms-of-service}{Terms
  of Service}
\item
  \href{https://help.nytimes3xbfgragh.onion/hc/en-us/articles/115014893968-Terms-of-sale}{Terms
  of Sale}
\item
  \href{https://spiderbites.nytimes3xbfgragh.onion}{Site Map}
\item
  \href{https://help.nytimes3xbfgragh.onion/hc/en-us}{Help}
\item
  \href{https://www.nytimes3xbfgragh.onion/subscription?campaignId=37WXW}{Subscriptions}
\end{itemize}
