Sections

SEARCH

\protect\hyperlink{site-content}{Skip to
content}\protect\hyperlink{site-index}{Skip to site index}

\href{/section/style}{Style}\textbar{}Renée Fleming, Songbird in
Seclusion

\href{https://nyti.ms/3hTq9wX}{https://nyti.ms/3hTq9wX}

\begin{itemize}
\item
\item
\item
\item
\item
\item
\end{itemize}

\href{https://www.nytimes3xbfgragh.onion/spotlight/at-home?action=click\&pgtype=Article\&state=default\&region=TOP_BANNER\&context=at_home_menu}{At
Home}

\begin{itemize}
\tightlist
\item
  \href{https://www.nytimes3xbfgragh.onion/2020/07/28/books/time-for-a-literary-road-trip.html?action=click\&pgtype=Article\&state=default\&region=TOP_BANNER\&context=at_home_menu}{Take:
  A Literary Road Trip}
\item
  \href{https://www.nytimes3xbfgragh.onion/2020/07/29/magazine/bored-with-your-home-cooking-some-smoky-eggplant-will-fix-that.html?action=click\&pgtype=Article\&state=default\&region=TOP_BANNER\&context=at_home_menu}{Cook:
  Smoky Eggplant}
\item
  \href{https://www.nytimes3xbfgragh.onion/2020/07/27/travel/moose-michigan-isle-royale.html?action=click\&pgtype=Article\&state=default\&region=TOP_BANNER\&context=at_home_menu}{Look
  Out: For Moose}
\item
  \href{https://www.nytimes3xbfgragh.onion/interactive/2020/at-home/even-more-reporters-editors-diaries-lists-recommendations.html?action=click\&pgtype=Article\&state=default\&region=TOP_BANNER\&context=at_home_menu}{Explore:
  Reporters' Obsessions}
\end{itemize}

\includegraphics{https://static01.graylady3jvrrxbe.onion/images/2020/07/30/fashion/30WITH-RENEEFLEMING-1/merlin_175065252_1e20ad0d-1cb7-4453-a077-4efa109dd49e-articleLarge.jpg?quality=75\&auto=webp\&disable=upscale}

with \ldots{}

\hypertarget{renuxe9e-fleming-songbird-in-seclusion}{%
\section{Renée Fleming, Songbird in
Seclusion}\label{renuxe9e-fleming-songbird-in-seclusion}}

America's top soprano will return to the Met stage in 2022, starring in
a new opera based on ``The Hours,'' by Michael Cunningham.

Shelf care: Ms. Fleming at home, among books, piano and
flowers.Credit...Justin T. Gellerson for The New York Times

Supported by

\protect\hyperlink{after-sponsor}{Continue reading the main story}

\href{https://www.nytimes3xbfgragh.onion/by/maureen-dowd}{\includegraphics{https://static01.graylady3jvrrxbe.onion/images/2018/04/02/opinion/maureen-dowd/maureen-dowd-thumbLarge.png}}

By \href{https://www.nytimes3xbfgragh.onion/by/maureen-dowd}{Maureen
Dowd}

\begin{itemize}
\item
  Published July 30, 2020Updated July 31, 2020
\item
  \begin{itemize}
  \item
  \item
  \item
  \item
  \item
  \item
  \end{itemize}
\end{itemize}

McLEAN, Va. --- Tim Jessell, a lanky divorced Washington corporate
attorney and father of three, had given up on love in 2008 when he was
set up on a blind date with a beautiful blond New Yorker who had also
given up on love.

They clicked, and many years later began that old sitcom debate about
splitting drawers and closets. If she sold her Upper West Side apartment
and moved to D.C., would he be able to make enough space in his
townhouse?

``She told me she was bringing her piano,'' Mr. Jessell said, with a
smile. ``That was serious.'' And, of course, the designer gowns were
going to overrun the closets.

Mr. Jessell, a sports fan and Bruce Springsteen fanatic who knew nothing
about opera before that first date, ended up in a new place, an airy
glass and stone contemporary house beside a creek in McLean, Va., with a
hammock and the most famous American soprano since Beverly Sills.

Curled on a white couch in their living room, wearing a silky cream
blouse, black pinstriped pants and turquoise jewelry, Renée Fleming said
that she was glad Mr. Jessell had never heard of her.

``If they're not a fan, it gives you a chance to kind of develop
something based on who you are,'' she said, noting wryly that their love
bloomed even though the first performance of hers Mr. Jessell saw was
``Lucrezia Borgia.''

``That's an opera in which I fall in love with my own son and then kill
all his friends and him by mistake,'' she said. ``He embraced the whole
thing.''

Moreover, Mr. Jessell was consistently willing to get on a plane to
wherever she was. Ms. Fleming had seen plenty of glimmers with men
evaporate over her grueling travel schedule. ``Someone would introduce
me to someone and we'd go out and I'd say `Oh, I had so much fun
tonight, I'll be back in three weeks,''' she said. ``I could sort of see
their eyes glaze over.''

After her divorce from the actor Rick Ross, when she was raising their
two daughters, ``I was single for a long time,'' Ms. Fleming said. ``And
there was a period in which I just felt really angry about the fact that
it's hard for accomplished, gifted women to be with men of similar
talents.''

\includegraphics{https://static01.graylady3jvrrxbe.onion/images/2020/07/30/fashion/30WITH-RENEEFLEMING-jessell/30WITH-RENEEFLEMING-jessell-articleLarge.jpg?quality=75\&auto=webp\&disable=upscale}

\hypertarget{the-all-american-diva}{%
\subsection{`The All-American Diva'}\label{the-all-american-diva}}

Now Ms. Fleming, her Steinway and her gowns are happily ensconced in
their new house, where she is rehearsing a solo program that includes
Handel, ``Over the Rainbow'' and much more for a Metropolitan Opera
concert at Dumbarton Oaks in Georgetown,
\href{https://www.metopera.org/discover/video/?videoName=renee-fleming-live-in-concert\&videoId=6174507070001}{livestreaming
Saturday} at 1 p.m. Her audience, the smallest she has ever performed
for, will be four cameras, two of them robotic.

The pay-per-view event is designed to help the Met survive during a
pandemic that is strangling her profession. The virus can be
\href{https://www.nytimes3xbfgragh.onion/2020/06/09/arts/music/choirs-singing-coronavirus-safe.html}{easily
spread by singing} and through crowds, which makes opera --- which was
already struggling --- exceedingly vulnerable.

The shimmery, creamy voice of the ``undiva,'' as she is known, is
ingrained in America's cultural memory, at both sad and happy moments.
She sang ``Amazing Grace'' at a memorial service at ground zero
\href{https://www.c-span.org/video/?c4619588/user-clip-world-trade-center-family-memorial-service}{after
9/11} and ``\href{https://www.youtube.com/watch?v=EPThqvkuky8}{Danny
Boy}'' at John McCain's funeral at the Washington National Cathedral.
She \href{https://www.youtube.com/watch?v=qzN7d94JXjI}{sang in}Sindarin,
the Elvish language for ``The Lord of the Rings'' soundtrack. She sang a
Top 10 list on David Letterman's
\href{https://www.youtube.com/watch?v=RZ59lj4Y_a0}{show}, Verdi
\href{https://www.youtube.com/watch?v=IPdP7XJdvlM\&feature=emb_title}{with
the Muppets}, and a goose-bumps-inducing rendition of the national
anthem at the \href{https://www.youtube.com/watch?v=N1h4zfO8Ais}{2014
Super Bowl}.

Image

Singing ``Amazing Grace'' with the Orchestra of St. Luke's during a
memorial service honoring those killed in the Sept. 11 terrorist attack
on New York City.Credit...Beth A. Keiser/Associated Press

Image

Perfoming the National Anthem before Super Bowl XLVIII in 2014, in East
Rutherford, N.J.Credit...Ben Liebenberg/Associated Press

``We even rehearsed with the Black Hawk helicopters so I wouldn't be
thrown off,'' she recalled. ``Imagine walking across the turf in
five-inch platform shoes with your adrenaline pumping. I wasn't going to
compromise on the shoes after Vera Wang did this extraordinary dress for
me which is now
\href{https://americanhistory.si.edu/blog/2014/06/ren\%C3\%A9e-flemings-super-bowl-gown-a-curatorial-jackpot.html}{in
the Smithsonian}.''

Peter Gelb, the general manager of the Met, calls Ms. Fleming ``the
all-American diva.'' The daughter of two music teachers from upstate New
York, she is known for both her wholesome style and her sensuality, her
natural stage presence and her preternatural voice, described by the
mezzo-soprano Susan Graham as ``pure gold.''

``The sensuality of her face goes with the sensuality of her sound,''
said Christine Baranski, her friend and a fellow Juilliard alum.

Anna Deavere Smith described watching from the audience as Ms. Fleming
stood on the side of the stage, preparing to perform at a dinner at the
Museum of Modern Art. As the singer paused to gather her focus, Ms.
Smith recalled, ``She looked to me like a lion going after a kind of
prey.''

Ann Patchett, the novelist, got to know the opera superstar when her
novel ``Bel Canto'' came out. Ms. Fleming, like the beautiful soprano in
the novel, was known for singing the aria ``Song to the Moon'' from
``Rusalka'' by Dvorak. ``People thought it was Renée,'' Ms. Patchett
said about the soprano she conjured. ``In retrospect, it probably was.''

The writer said that whenever she goes out with Ms. Fleming, ``people
come up to the table and say, `You were singing at my wedding' or `You
were singing when my daughter was born' or `You were singing when my
father died.' And she really takes it in and is so appreciative to make
that connection personally, every single time.''

When Ms. Fleming was at the height of her spectacular career, jetting
around the world to different opera houses, she sometimes told reporters
she fantasized about a quiet life in the suburbs, growing a vegetable
garden. Now that the pandemic has mandated that life, how does she like
it? It's as though an exotic bird has perched in the gray, bureaucratic
confines of the nation's capital.

``I had nice terrace gardens in New York, which I really loved, but here
I'm able to do it myself,'' she said, in a voice that has been described
as ``jewels floating like butterflies.'' ``Right now, I have cucumbers
and tomatoes and then I'll have squash.''

Image

Don't fence her in: The wide-ranging Ms. Fleming's garden includes
cucumbers and tomatoes. Squash is on the horizon.Credit...Justin T.
Gellerson for The New York Times

\hypertarget{twists-in-the-plot}{%
\subsection{Twists in the Plot}\label{twists-in-the-plot}}

Ms. Fleming, 61, who does not like to be ranked against other sopranos,
decided not to compete with her earlier performances and sang one of her
signature roles as the Marschallin in ``Der Rosenkavalier'' for the last
time at the Met in 2017. Headlines mourned, ``The Diva Departs.'' But
reports of the diva's departure are greatly exaggerated. She's busier
than ever.

``I'm still concertizing like I always was and I have been doing a lot
of new work,'' she said. ``That's where my heart is.''

She has dipped into other genres, starring in a Broadway revival of
``Carousel'' and the London premiere of ``The Light in the Piazza.'' She
acted opposite Ben Whishaw in ``Norma Jeane Baker of Troy'' at the Shed.

And, in a move bound to thrill her fans, she will be returning to the
Met in the fall of 2022 to star in a new opera based on Michael
Cunningham's novel ``The Hours'' --- and the movie of the same name ---
as Clarissa, the role played by Meryl Streep. Kelli O'Hara will play
Laura, and Joyce DiDonato will play Virginia Woolf.

Image

The house of Strauss: Elina Garanca, left, as Octavian and Ms. Fleming
as the Marschallin in the opera "Der Rosenkavalier" at the Metropolitan
Opera in 2017.Credit...Sara Krulwich/The New York Times

The idea was first suggested to Ms. Fleming by her man Friday, Paul
Batsel, and the work was composed by Kevin Puts, who won the 2012
Pulitzer for his debut opera, ``Silent Night.''

They brought the concept to Mr. Gelb. ``I thought for sure he would say
no and he immediately said, `I want this project,''' she recalled.
``It's a triple threat of actresses and fascinating characters.''

Ms. Fleming also hosts a
\href{https://www.kennedy-center.org/whats-on/festivals-series/sound-health/music-and-the-mind/}{webinar}
on music and health. ``It's the last memory to go, music memory, so
music affects more parts of the brain than anywhere else,'' she said.

Her friends say she has gotten more political since she came to
Washington, so I ask what she thinks of the president.

``I just think it's a disaster,'' she said. ``I mean, please, let's have
people who are entertaining and shocking in entertainment, not at the
head of government. I'm on the arts council for Biden. I love the
interview recently presenting Biden and Obama together as calm,
measured, sane.''

The opera world, which has always been very white and patriarchal and
known for backstage drama, was dealing with shocks of sexual harassment
even before it was rocked by the pandemic.

Ms. Fleming was a favorite of James Levine, the Met conductor who was
\href{https://www.nytimes3xbfgragh.onion/2018/03/12/arts/music/james-levine-metropolitan-opera.html}{fired
in 2018} over sexual misconduct charges involving young men. (Mr. Levine
has denied the accusations.) And she played Desdemona to Plácido
Domingo's ``Otello'' at the Met in
\href{https://www.nytimes3xbfgragh.onion/1995/10/04/arts/opera-review-otello-outshines-all-the-accessories.html}{1995}.
After he was accused of sexual misconduct by a slew of women, Mr.
Domingo dropped out of Verdi's ``Macbeth'' at the Met at
\href{https://www.nytimes3xbfgragh.onion/2019/09/24/arts/music/placido-domingo-met-opera-harassment.html}{the
last minute} and ended his career in America.

Ms. Fleming recalled that, as a young soprano, she almost fainted when
she had to go on as the understudy at the Met in 1994 in ``Otello'' with
the world-famous tenor. ``He was so frightening in the confrontation
scene and he's crazy with jealousy and he almost hits Desdemona, and it
was also a huge opportunity for me,'' she said, ``and the adrenaline was
just off the charts and I could hardly walk off stage.''

She continues, ``I'm really happy that the Me Too movement has occurred,
that people have been called out and hopefully people think twice before
they behave badly.''

Image

Rumpled reputation: Mr. Domingo as the titular character in Giuseppe
Verdi's ``Otello," and Ms. Fleming as Desdemona during the final dress
rehearsal for the Met's 2002 opening night gala at Lincoln Center in New
York City.Credit...Jack Vartoogian/Getty Images

She also echoed the calls for race reform in her profession, saying that
impresarios must identify and nurture Black talent. ``There are some
things about opera that are extremely democratic because it's about the
voice,'' she said. ``There are some extraordinary young artists,
Lawrence Brownlee, Julia Bullock, J'Nai Bridges and other gifted singers
who are stars or on their way to being stars. But that's not enough. We
want to come into the opera house and see the diversity reflected
everywhere.'' She enlisted Ms. Smith to write a libretto about Chicago
inspired by her ``Notes From the Field,'' an examination of systemic
racism.

I ask Ms. Fleming about Leontyne Price, the legendary Black soprano and
pioneer at the Met, and she described her as a ``phenomenal mentor.''

``I think the most important thing that she got me to think about was
that period of time where everyone wants a piece of you and you know you
have to set up filters, you have to protect yourself, because there
aren't enough hours in the day to respond to everything,'' she said.
``And she referred to all of that as `the noise.' And she said, `You're
hearing the noise and you need to tune out the noise and focus on
this''' --- here Ms. Fleming stroked her throat to indicate her voice
--- ``because if this goes wrong, the noise will disappear overnight.''

She continued: ``I try not to be too neurotic about my voice, but I
can't speak over loud parties. Even after a concert, I would go to a
reception, which is part of my job, and find speaking really tiring,
really taxing.''

Ms. Baranski testifies to how protective Ms. Fleming is of her voice.
``When an opera singer is your girlfriend, you find yourself at Orso or
some other restaurant,'' she said, ``and you've just ordered your
martinis and then you've got to get up and leave or go to a back room
because it's too chilly or you have to speak over too many people.''

Image

The kindness of stages: Teddy Tahu Rhodes, left, as Stanley Kowalski and
Ms. Fleming as Blanche DuBois during a 2013 performance of ``A Streetcar
Named Desire'' at Carnegie Hall.Credit...Ruby Washington/The New York
Times

\hypertarget{nothing-is-uncool}{%
\subsection{`Nothing Is Uncool'}\label{nothing-is-uncool}}

Ms. Fleming's favorite heroine is Tosca, but said the heroine closest to
her personality is Tatyana in ``Eugene Onegin.'' If she could sing any
tenor role, it would be Don Jose in ``Carmen.'' And if she could dine
with any composer? Richard Strauss, no contest. ``A secretive man who
created the most sublime music,'' she said. ``And there are secrets in
his music that I find compelling, phrases that haunted me for months.''

Ms. Fleming, who was shy and insecure growing up and became a bit bolder
only when she sang in jazz clubs in college, had crippling bouts of
stage fright. ``I had it early on as I was just trying to transition
from being a student to a professional and then I had it again in 1998.
I had a huge number of new roles that year, a divorce, too many things
at one time. It lasted for almost a year. I almost quit singing. It's
intense, it's very hard to walk on stage.''

She was heckled that year at La Scala in Milan, performing ``Lucrezia
Borgia'' --- just for being an American singing an Italian opera in
Italy --- and found it ``really traumatic.''

When she played Blanche DuBois in Andre Previn's opera of ``A Streetcar
Named Desire,'' also in that year, she was channeling her own neuroses.
``That was in a period when I was really troubled and having panic
attacks and the anxiety around performance was so great, I thought
either I'm going to crash in this role or it's going to be a really
extraordinary vehicle for me to begin to heal. And I think it was the
latter.''

She had therapy and read books about stage fright. ``Instead of seeing
the audience as judgmental, as a critical body, and thinking it's not
going to go well because you're a fraud or you're not good enough, you
see yourself as sharing something with them.''

Fashion and costume has helped armor her. ``I would say my performance
wardrobe for concerts and television is unique in that the number of
couture designs from top designers is --- I don't think there's anybody
else who's had this in history,'' she said. ``The extraordinary media
push around Maria Callas's CDs and her glamour, I think, fed a desire on
the part of designers to dress a diva. So I think I just happened to be
in the right place at the right time.''

In 2008, she became the first woman, à la Pavarotti and Domingo, to
headline an opening of the Met season in her very own gala, starring in
scenes from three different operas, Verdi's ``La Traviata,'' Massenet's
``Manon,'' and Strauss's ``Capriccio.'' Lacroix, Lagerfeld and Galliano
each designed a different costume for her.

Issey Miyake, Oscar de la Renta and Vivienne Westwood have also
festooned her. ``I started with Gianfranco Ferré, and he would send 17
gowns and many of them in the end came in boxes and without fittings and
they were perfect,'' Ms. Fleming said. ``And I had a form, a model in
his studio that was next to Elizabeth Taylor's, so I was in good
company.''

She is also renowned for her gems, some of which she borrows from her
good friend,
\href{https://www.nytimes3xbfgragh.onion/2018/10/17/arts/design/show-us-your-wall-ann-ziff.html}{Ann
Ziff}, the chairwoman of the Met who runs a jewelry label.

Image

Culture and couture: Ms. Fleming arrives at the 2018 Met Gala in New
York City wearing a Vivienne Westwood gown.Credit...Hector
Retamal/Agence France-Presse --- Getty Images

Is she worried about the crepuscular nature of her art, that opera will
wither away if its grand poobahs can't figure out a way to attract young
people?

``There's an anti-opera wave in the mainstream media now, I've seen it
in commercials where opera is the joke, `Oh, you don't like opera, do
you?''' she said. ``It's unfortunate because it makes people who do like
it afraid to say they like it. It's a manufactured prejudice. It should
be a matter of taste just like any other style of music. It shouldn't
have some negativity ascribed to it that's it's uncool. Nothing should
be uncool. It should be, do you like it or do you not like it?

``Look at `The Voice' or `America's Got Talent.' When a 14-year-old boy
gets up and sings `Nessun Dorma,' everybody's crying and cheering. Why
is the connection not made between that and the art world?''

Ms. Fleming walks me out. When she opens the door, she finds a package
from DSW, the discount shoe store.

``I've been found out,'' she laughed. ``I needed some clogs. I don't
think I'll ever wear high heels again.''

Somehow, I don't believe her.

{[}\emph{A bonus coda of Confirm or Deny}{]}

\textbf{Maureen Dowd: Tenors are dumber and more neurotic than
sopranos.}

Renée Fleming: I wouldn't say dumber. I would say more neurotic probably
because they have a greater risk. If a tenor fails, it's much more
obvious than when a soprano fails because they are singing with their
chest voice, full throated. And so when they fail, there's a huge split
or a crack. And when a soprano fails, it's just unattractive.

\textbf{Tenors are often shorter than their soprano partners.}

Tenors have not, by and large, been a tall singing block. And there have
been a lot of Latin American tenors. There's something in the language
and maybe the Mariachi tradition.

\textbf{You hate your fach.}

Yeah. I wish my fach were broader but I'm happy being a lyric soprano.

\textbf{You are the only opera singer to have an asteroid named for
you.}

Truth, yes. I said, ``That's thrilling. I really am a star!''

\textbf{You sometimes talk to your husband in Sindarin in romantic
situations.}

No, but he would probably love that. We could try to improv.

\textbf{You baked cookies with Martha Stewart and Snoop Dogg.}

I did and that was surreal. And of that trio, I am the one who is out
because Martha Stewart and Snoop Dogg have collaborated in other ways.
But I'm not a good cook, so I'm sure that was it.

\textbf{You can get glamorous in 10 minutes.}

I would say that's true. I don't like to spend a lot of time on it with
one exception, my hair. There's a reason why my escape plan is just to
shave my head. It's a pain.

\textbf{You've learned 50 operas in six languages.}

Confirm. The first time I learned Russian for ``Eugene Onegin,'' my
first child was a month old and I kept thinking, ``What is wrong with
me? This is not sinking in.'' And then, of course, a couple years later,
they started talking about what happens to women hormonally when they've
had a baby, which is they're stupid for a while.

\textbf{Both your daughters can sing.}

Definitely, they're wonderful singers, but they know too much. They've
opted for other careers.

Advertisement

\protect\hyperlink{after-bottom}{Continue reading the main story}

\hypertarget{site-index}{%
\subsection{Site Index}\label{site-index}}

\hypertarget{site-information-navigation}{%
\subsection{Site Information
Navigation}\label{site-information-navigation}}

\begin{itemize}
\tightlist
\item
  \href{https://help.nytimes3xbfgragh.onion/hc/en-us/articles/115014792127-Copyright-notice}{©~2020~The
  New York Times Company}
\end{itemize}

\begin{itemize}
\tightlist
\item
  \href{https://www.nytco.com/}{NYTCo}
\item
  \href{https://help.nytimes3xbfgragh.onion/hc/en-us/articles/115015385887-Contact-Us}{Contact
  Us}
\item
  \href{https://www.nytco.com/careers/}{Work with us}
\item
  \href{https://nytmediakit.com/}{Advertise}
\item
  \href{http://www.tbrandstudio.com/}{T Brand Studio}
\item
  \href{https://www.nytimes3xbfgragh.onion/privacy/cookie-policy\#how-do-i-manage-trackers}{Your
  Ad Choices}
\item
  \href{https://www.nytimes3xbfgragh.onion/privacy}{Privacy}
\item
  \href{https://help.nytimes3xbfgragh.onion/hc/en-us/articles/115014893428-Terms-of-service}{Terms
  of Service}
\item
  \href{https://help.nytimes3xbfgragh.onion/hc/en-us/articles/115014893968-Terms-of-sale}{Terms
  of Sale}
\item
  \href{https://spiderbites.nytimes3xbfgragh.onion}{Site Map}
\item
  \href{https://help.nytimes3xbfgragh.onion/hc/en-us}{Help}
\item
  \href{https://www.nytimes3xbfgragh.onion/subscription?campaignId=37WXW}{Subscriptions}
\end{itemize}
