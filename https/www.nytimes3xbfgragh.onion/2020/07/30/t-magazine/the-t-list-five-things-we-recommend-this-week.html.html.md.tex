Sections

SEARCH

\protect\hyperlink{site-content}{Skip to
content}\protect\hyperlink{site-index}{Skip to site index}

\href{https://myaccount.nytimes3xbfgragh.onion/auth/login?response_type=cookie\&client_id=vi}{}

\href{https://www.nytimes3xbfgragh.onion/section/todayspaper}{Today's
Paper}

The T List: Five Things We Recommend This Week

\url{https://nyti.ms/312SGt6}

\begin{itemize}
\item
\item
\item
\item
\item
\end{itemize}

Advertisement

\protect\hyperlink{after-top}{Continue reading the main story}

Supported by

\protect\hyperlink{after-sponsor}{Continue reading the main story}

\hypertarget{the-t-list-five-things-we-recommend-this-week}{%
\section{The T List: Five Things We Recommend This
Week}\label{the-t-list-five-things-we-recommend-this-week}}

Farm dining, a Parisian floral gallery, Ruth Asawa stamps --- and more.

Published July 30, 2020Updated July 31, 2020

\begin{itemize}
\item
\item
\item
\item
\item
\end{itemize}

\emph{Welcome to the T List, a newsletter from the editors of T
Magazine. Each week, we're sharing things we're eating, wearing,
listening to or coveting now.}
\textbf{\href{https://www.nytimes3xbfgragh.onion/newsletters/t-list?module=inline}{\emph{Sign
up here}}} \emph{\textbf{to find us in your inbox every Wednesday.}}
\emph{You can always reach us at}
\href{mailto:tlist@NYTimes.com}{\emph{tlist@NYTimes.com}}\emph{.}

\begin{center}\rule{0.5\linewidth}{\linethickness}\end{center}

\hypertarget{shop-this}{%
\subsubsection{Shop This}\label{shop-this}}

\hypertarget{in-paris-an-art-deco-inspired-honduran-florist}{%
\subsection{In Paris, an Art Deco-Inspired Honduran
Florist}\label{in-paris-an-art-deco-inspired-honduran-florist}}

\includegraphics{https://static01.graylady3jvrrxbe.onion/images/2020/07/29/t-magazine/29tmag-newsletter-slide-TQW9/29tmag-newsletter-slide-TQW9-articleLarge.jpg?quality=75\&auto=webp\&disable=upscale}

By Ahnna Lee

A few blocks from Paris's stately Palais-Royal in the First
Arrondissement exists a dramatic enclave of flowers. This eponymous new
floral gallery, created by the Honduran (and formerly New York
City-based) florist \href{https://www.instagram.com/arturoarita/}{Arturo
Arita}, is an all-white grotto that enshrines fresh blooms --- not
mythological deities --- in its rocky alcoves. Arita's shop was designed
by the Parisian architect \href{https://www.sophiedries.com/}{Sophie
Dries}, whose assignment was to come up with something inspired by the
pair's shared appreciation for the Mannerist-style caves found in
Renaissance-era Italian gardens, including those of Villa d'Este in
Tivoli and Villa Medici in Rome. There's also a
\href{https://www.barovier.com/en/collections/chandeliers}{Barovier \&
Toso} chandelier that looks like a piece of fruit hanging from the vine,
a leaf-shaped lamp by André Arbus and a dramatic center table made of a
raw-edged slab of Brazilian marble upon which Arita creates his designs.
He tends to feature vibrantly hued, sculptural flowers such as Heliconia
She Kongs and Protea Venuses along with geometric fronds, placed in
contrasting-colored, Art Deco-style vases. The space's mirrored walls
heighten its lushness so much so that you might feel as though you're in
a botanical garden, not the center of Paris.
\href{https://www.arturoarita.com/}{\emph{arturoarita.com}}\emph{.}

\begin{center}\rule{0.5\linewidth}{\linethickness}\end{center}

\hypertarget{see-this}{%
\subsubsection{See This}\label{see-this}}

\hypertarget{new-yorkers-phone-calls-turned-into-poetry}{%
\subsection{New Yorkers' Phone Calls Turned Into
Poetry}\label{new-yorkers-phone-calls-turned-into-poetry}}

Image

The poet Asiya Wadud stands by one of her pieces from ``Echo Exhibit''
(2020) in New York City's Seaport district, co-presented by the Lower
Manhattan Cultural Council (LMCC), the Seaport district and the Howard
Hughes Corporation as part of River To River 2020: Four
Voices.Credit...Ian Douglas

By Samuel Rutter

Founded in the aftermath of 9/11, the
\href{https://www.nytimes3xbfgragh.onion/2019/06/26/arts/dance/river-to-river-festival.html}{River
to River Festival} is an annual celebration of the arts in Lower
Manhattan that in the past has included work by Yoko Ono and Pam
Tanowitz. The challenges of the present moment led the organizers to
rethink how the public can interact with art --- and the result is River
to River 2020: Four Voices, where projects take on a physical presence
that can be viewed in open spaces over time. In ``Blessing of the
Boats*,\emph{'' the artist Muna Malik asks us to consider what concrete
actions can lead us to a better future, write them on an origami boat
and place them in a 20-foot maquette of a ship in the Battery's
Belvedere Plaza. The artist Jean Shin has two pieces --- ``Floating
MAiZE'' ** and ``The Last Straw'' --- ** that repurpose plastic waste to
raise questions of belonging and identity. The multidisciplinary artist
Mona Chalabi, known for transforming statistical truths into visual
narratives, has used census data to condense the city's sprawling
population down to a graphic project called ``100 New Yorkers,}''* which
will be displayed on posters and screens around the Oculus building.
Finally, the poet Asiya Wadud presents ``Echo Exhibit,'' in which she
and other writers have transformed 15-minute phone calls with residents
of Lower Manhattan into poems that are displayed on vinyl posters in
storefronts throughout the neighborhood. From over 200 hours of
conversations in half a dozen languages --- including American Sign
Language --- Wadud has curated a series of 20 poems, which she calls
echoes. ``Poems in public spaces create the potential to encounter
something new and unexpected,'' said Wadud, who teaches poetry at Saint
Ann's School in Brooklyn. \emph{All events are free and run through
August 30 in various locations in New York City,}
\href{https://lmcc.net/river-to-river-festival/}{\emph{lmcc.net}}\emph{.}

\begin{center}\rule{0.5\linewidth}{\linethickness}\end{center}

\hypertarget{eat-this}{%
\subsubsection{Eat This}\label{eat-this}}

\hypertarget{outside-san-francisco-quince-goes-to-the-source}{%
\subsection{Outside San Francisco, Quince Goes to the
Source}\label{outside-san-francisco-quince-goes-to-the-source}}

Image

Left: the open-air greenhouse built with reclaimed wood at Fresh Run
Farm in Bolinas, Calif., for Quince at the Farm, the temporary home of
the three-Michelin-starred restaurant Quince. Right: a dish of Watson
Farm spring lamb with fava beans, English peas and spring
onions.Credit...Joe Weaver

By Kurt Soller

As many restaurateurs around the globe wonder when, if ever, they'll be
able to reopen their dining rooms, the husband-and-wife duo Michael and
Lindsay Tusk --- who run San Francisco's renowned Quince, among other
spots --- have come up with a creative solution: bringing diners to the
farm. Two farms, in fact: Fresh Run in Bolinas, Calif., and McEvoy Ranch
in nearby Marin County, where, each weekend from now through the end of
October, they'll host a rotating series of socially distanced alfresco
lunches and dinners across the rolling acreage from which many of their
ingredients are sourced. The meals, which are meant to be more casual
than those served at their flagship --- flat shoes and hats are
encouraged --- begin with California-inspired aperitifs and passed
canapés (such as local halibut tartare and Romano bean salad) before
moving onto a hyperlocal, multicourse offering that changes by the week
but might include, say, fresh corn with edible farm flowers and Stemple
roast beef ``grilled from up the road in its own smoked fat,'' as the
menu explains. The idea isn't simply to relocate their business but to
fully reimagine it, bringing Quince closer to its roots. Fittingly, at
the end of each feast, guests will be invited to pick vegetables or
berries to take home. \emph{For details, pricing and availability,
visit}
\href{http://quincerestaurant.com/}{\emph{quincerestaurant.com}}\emph{.}

\begin{center}\rule{0.5\linewidth}{\linethickness}\end{center}

\hypertarget{buy-this}{%
\subsubsection{Buy This}\label{buy-this}}

\hypertarget{ruth-asawa-postage-stamps}{%
\subsection{Ruth Asawa Postage Stamps}\label{ruth-asawa-postage-stamps}}

Image

Ten different wire sculptures by the artist Ruth Asawa will be featured
on a USPS Forever stamps this August.~Credit...Courtesy of the United
States Postal Service

By Thessaly La Force

Earlier this year, it was announced that 10 works of art by Ruth Asawa
--- about whom I
\href{https://www.nytimes3xbfgragh.onion/2020/07/20/t-magazine/ruth-asawa.html}{wrote}
for T's online art issue, ``True Believers,'' which launched last week
--- would appear on United States postage stamps. Asawa, who was born in
Norwalk, Calif., in 1926 and was incarcerated during World War II along
with over 100,000 other Japanese-Americans and people of Japanese
descent, eventually found her way to Black Mountain College, where she
studied with Buckminster Fuller and Josef Albers. A gifted artist, she
began creating masterful abstract sculpture work made from wire in 1947.
Though she was represented by a well-respected midtown gallery called
Peridot in the 1950s and '60s, Asawa eventually withdrew from the New
York City art scene, becoming more engaged with public sculpture,
community-based arts education and activism in San Francisco. More
recently, Asawa has become the subject of what some might call a
rediscovery, so it comes as no surprise that she's been given the honor
of being featured on a postage stamp, alongside other great Americans,
such as the artist Ellsworth Kelly, the writer Nella Larsen and the poet
Walt Whitman. Designed by Ethel Kessler, the stamps can be preordered
now and will be made available Aug. 13. \emph{\$11 for a sheet of 20
stamps,}
\href{https://store.usps.com/store/product/buy-stamps/ruth-asawa-S_476304}{\emph{usps.com}}\emph{.}

\begin{center}\rule{0.5\linewidth}{\linethickness}\end{center}

\hypertarget{try-this}{%
\subsubsection{Try This}\label{try-this}}

\hypertarget{write-a-digital-love-note-to-new-york-city}{%
\subsection{Write a Digital Love Note to New York
City}\label{write-a-digital-love-note-to-new-york-city}}

Image

A screenshot of the web page Dear New York (2020) by Pentagram's~Giorgia
Lupi, Talia Cotton and Phil Cox.Credit...Courtesy of Pentagram

By Flo Wales Bonner

As New York cautiously reopens its public spaces, a project by the
design group Pentagram is offering the city's residents a digital one in
which to both reflect and look forward. Visitors to the web page
\href{https://dear-new-york.com/}{Dear New York} are invited to explore
and add to a collection of messages written on virtual sticky notes,
each one expressing what its author has missed about the city in recent
months or desires for its future. Overseen by Pentagram partner Giorgia
Lupi, an authority in data visualization, the project has been an
exercise in creating **** a humanized digital interface that encourages
genuine connection. The ``crowdsourced love letter,'' as Lupi describes
it, presents a rich, dynamic **** world --- notes can be precisely
positioned on one of a number of randomized backgrounds that depict
intimate surface details of everyday New York, from dust-streaked
storefront windows to mottled patches of sidewalk. As contributions
multiply, Lupi hopes the project will become an engaging emotional
record of a city in the midst of social and cultural upheaval --- a
``monument of a moment in time.''
\href{https://dear-new-york.com/}{\emph{dear-new-york.com}}\emph{.}

\begin{center}\rule{0.5\linewidth}{\linethickness}\end{center}

\hypertarget{from-ts-instagram}{%
\subsubsection{From T's Instagram}\label{from-ts-instagram}}

\hypertarget{tartissue-la-monte-young}{%
\subsection{\#TArtIssue: La Monte
Young}\label{tartissue-la-monte-young}}

Advertisement

\protect\hyperlink{after-bottom}{Continue reading the main story}

\hypertarget{site-index}{%
\subsection{Site Index}\label{site-index}}

\hypertarget{site-information-navigation}{%
\subsection{Site Information
Navigation}\label{site-information-navigation}}

\begin{itemize}
\tightlist
\item
  \href{https://help.nytimes3xbfgragh.onion/hc/en-us/articles/115014792127-Copyright-notice}{©~2020~The
  New York Times Company}
\end{itemize}

\begin{itemize}
\tightlist
\item
  \href{https://www.nytco.com/}{NYTCo}
\item
  \href{https://help.nytimes3xbfgragh.onion/hc/en-us/articles/115015385887-Contact-Us}{Contact
  Us}
\item
  \href{https://www.nytco.com/careers/}{Work with us}
\item
  \href{https://nytmediakit.com/}{Advertise}
\item
  \href{http://www.tbrandstudio.com/}{T Brand Studio}
\item
  \href{https://www.nytimes3xbfgragh.onion/privacy/cookie-policy\#how-do-i-manage-trackers}{Your
  Ad Choices}
\item
  \href{https://www.nytimes3xbfgragh.onion/privacy}{Privacy}
\item
  \href{https://help.nytimes3xbfgragh.onion/hc/en-us/articles/115014893428-Terms-of-service}{Terms
  of Service}
\item
  \href{https://help.nytimes3xbfgragh.onion/hc/en-us/articles/115014893968-Terms-of-sale}{Terms
  of Sale}
\item
  \href{https://spiderbites.nytimes3xbfgragh.onion}{Site Map}
\item
  \href{https://help.nytimes3xbfgragh.onion/hc/en-us}{Help}
\item
  \href{https://www.nytimes3xbfgragh.onion/subscription?campaignId=37WXW}{Subscriptions}
\end{itemize}
