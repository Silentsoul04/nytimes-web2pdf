Sections

SEARCH

\protect\hyperlink{site-content}{Skip to
content}\protect\hyperlink{site-index}{Skip to site index}

\href{https://www.nytimes3xbfgragh.onion/section/technology}{Technology}

\href{https://myaccount.nytimes3xbfgragh.onion/auth/login?response_type=cookie\&client_id=vi}{}

\href{https://www.nytimes3xbfgragh.onion/section/todayspaper}{Today's
Paper}

\href{/section/technology}{Technology}\textbar{}Grilled by Lawmakers,
Big Tech Turns Up the Gaslight

\href{https://nyti.ms/39IyOPC}{https://nyti.ms/39IyOPC}

\begin{itemize}
\item
\item
\item
\item
\item
\item
\end{itemize}

Advertisement

\protect\hyperlink{after-top}{Continue reading the main story}

Supported by

\protect\hyperlink{after-sponsor}{Continue reading the main story}

The Shift

\hypertarget{grilled-by-lawmakers-big-tech-turns-up-the-gaslight}{%
\section{Grilled by Lawmakers, Big Tech Turns Up the
Gaslight}\label{grilled-by-lawmakers-big-tech-turns-up-the-gaslight}}

It is less clear that tech executives' strategy of evasive answers will
continue to work now that lawmakers have begun doing their homework.

\includegraphics{https://static01.graylady3jvrrxbe.onion/images/2020/07/31/business/31roose/31roose-articleLarge.jpg?quality=75\&auto=webp\&disable=upscale}

\href{https://www.nytimes3xbfgragh.onion/by/kevin-roose}{\includegraphics{https://static01.graylady3jvrrxbe.onion/images/2018/02/20/multimedia/author-kevin-roose/author-kevin-roose-thumbLarge.jpg}}

By \href{https://www.nytimes3xbfgragh.onion/by/kevin-roose}{Kevin Roose}

\begin{itemize}
\item
  July 30, 2020
\item
  \begin{itemize}
  \item
  \item
  \item
  \item
  \item
  \item
  \end{itemize}
\end{itemize}

When Mark Zuckerberg appeared in front of Congress two years ago, the
Facebook chief executive's memorable retort to a clueless questioner was
``Senator, we run ads.'' After Wednesday's marathon appearance by Mr.
Zuckerberg and three other tech titans at a
\href{https://www.nytimes3xbfgragh.onion/2020/07/29/technology/big-tech-hearing-apple-amazon-facebook-google.html}{House
hearing on competition} in the tech industry, a more fitting quote might
be ``Congresswoman, I'm not sure what you would mean by `threaten.'''

That was Mr. Zuckerberg's evasive answer to a question asked by
Representative Pramila Jayapal, a Democrat from Washington, about
whether Facebook had ever threatened to squash smaller competitors by
copying their products if they wouldn't let Facebook acquire them.

It was a good question with a clear-cut answer. Facebook's
copy-and-crush approach has been
\href{https://www.wired.com/story/copycat-how-facebook-tried-to-squash-snapchat/}{well
documented} for years, and Ms. Jayapal brought even more receipts ---
previously undisclosed messages in which Mr. Zuckerberg issued thinly
veiled threats to Kevin Systrom, the co-founder of Instagram, about what
would happen to his company if he refused to sell.

An honest Mr. Zuckerberg might have replied, ``Yes, Congresswoman, like
most successful tech companies, we acquire potential competitors all the
time, and copy the ones we can't buy. That's how we've avoided going
extinct like MySpace or Friendster, and we're about to do it again with
\href{https://techcrunch.com/2020/07/16/instagram-confirms-its-tiktok-rival-reels-will-launch-in-the-us-in-early-august/}{Instagram
Reels,} our new TikTok clone.'' That would have been an illuminating
answer, and one that could have let lawmakers in on the
kill-or-be-killed ethos of Silicon Valley. Instead, he dodged and
weaved, trying to explain away the emails without admitting the obvious.

He did the same thing when Representative Hank Johnson, Democrat of
Georgia, pressed him for answers about Facebook Research --- an app that
was used to
\href{https://www.wired.com/story/facebook-research-app-root-certificate/}{snoop
on users' smartphone} usage and give Facebook detailed data about its
competitors. Mr. Zuckerberg initially said he wasn't familiar with the
app, even though Apple's decision to bar it from its App Store
\href{https://www.nytimes3xbfgragh.onion/2019/01/31/technology/apple-blocks-facebook.html}{nearly
caused} a meltdown at his company last year. (He later said he misspoke,
and that he remembered it.)

I don't mean to pick on Mr. Zuckerberg. Every other witness at
Wednesday's hearing --- Jeff Bezos of Amazon, Sundar Pichai of Google
and Tim Cook of Apple --- also dodged lawmakers' most pointed questions,
or professed their ignorance.

The result was a hearing that, at times, felt less like a reckoning than
an attempted gaslighting --- a group of savvy executives trying to
convince lawmakers that the evidence that their yearslong antitrust
investigation had dug up wasn't really evidence of anything.

The performance wasn't particularly convincing. You don't become a tech
mogul by being sloppy or forgetful, and it strains credulity to imagine
that these four hypercompetitive, detail-obsessed men --- all of whom
had many weeks to prepare for Wednesday's hearing --- simply didn't
remember major decisions they'd made.

At one point, Mr. Bezos was asked about a recent
\href{https://www.wsj.com/articles/amazon-tech-startup-echo-bezos-alexa-investment-fund-11595520249}{Wall
Street Journal report} that Amazon had set up a venture capital fund to
invest in start-ups, only to then introduce its own versions of those
start-ups' products.

``I don't know the specifics of that situation,'' Mr. Bezos replied.

At another point, Mr. Pichai, asked to explain whether it was
anticompetitive for Google to threaten to delist Yelp if it didn't allow
the company's search engine to use its listings in its featured
snippets, said he was ``happy to engage and understand the specifics''
at a later date. Yelp has complained publicly about Google's search
engine for years.

And confronted with emails acquired by House investigators in which
Apple employees promised to fast-track a company through its App Store
approval process --- a seeming contradiction of Mr. Cook's assertion
that it treated all developers equally --- Mr. Cook responded, ``I don't
know about that, sir.''

Give the executives this: It's hard to have nuanced conversations about
complex and often technical topics in front of an audience of hostile
politicians, some of whom seemed more interested in generating fiery
clips for their Facebook pages than investigating antitrust concerns.
And the panel's format --- rapid-fire rounds of questioning conducted
over video conference, with each member given only five minutes at a
time to question the witnesses --- nearly guaranteed that the
conversation would remain surface-level.

In addition, many Republican members of the subcommittee seemed to have
no interest in antitrust issues at all, preferring instead to ride
partisan hobby horses like claims of anti-conservative bias on social
media.

But many Democratic members came armed with real, substantive questions
that deserved a fuller airing. The executives' choice to sidestep these
questions --- or their inability to answer before being cut off --- may
signal that they still believe they can run circles around Congress
without engaging in hard, detailed conversations about how they exercise
their power.

That may have been a reasonable conclusion to draw after the past
several years, a period in which Silicon Valley giants added hundreds of
billions of dollars in market value while Washington barely gave it a
passing glance. Despite plenty of clamoring, lawmakers have failed to
pass any meaningful privacy or data protection laws during President
Trump's time in office, and Mr. Trump has shown little interest in any
tech regulation that does not involve his own Twitter account.

But it is less clear that a say-nothing strategy will continue to work,
now that lawmakers have begun doing their homework. Sure, some members
of Congress may still need their iPhones explained to them, but there is
real expertise on Capitol Hill that wasn't there even a year ago, and
new allies who are willing to give Congress the ammunition it needs.

When Mr. Zuckerberg appeared before Congress in 2018, tech expertise in
Washington was a rare commodity. Now, antitrust experts schooled in the
intricacies of tech platforms are helping Congress investigate.

One such specialist, Lina Khan, whose
\href{https://www.nytimes3xbfgragh.onion/2018/09/07/technology/monopoly-antitrust-lina-khan-amazon.html}{analysis
of Amazon's anti-competitive business practices} made her a star of the
``hipster antitrust'' movement, appeared at Wednesday's hearing sitting
behind Representative David Cicilline of Rhode Island, whom she now
advises. Tech industry leaders, who once refrained from criticizing
their fellow titans out of courtesy or cowardice, are now
\href{https://gamerant.com/epic-games-ceo-tim-sweeney-criticizes-apples-app-store-fees/}{speaking
up} about issues like the 30 percent cut Apple takes from purchases made
through its App Store. And if Mr. Trump loses his re-election bid in
November, the calls from Democrats to break up or regulate big tech
companies will only grow louder.

None of this necessarily means that Congress is on the verge of reining
in Silicon Valley's excesses. There are still plenty of lawmakers who
would rather focus on promoting dubious claims of partisan censorship on
Facebook, or complaining about fund-raising emails getting stuck in
constituents' spam filters. And tech companies' generally good
reputations with Americans --- not to mention their deep lobbying
pockets --- may protect them in the end.

But at certain moments on Wednesday, each of the four tech executives
appeared to be taken off guard by the rigor and depth of the questions
they faced. If they were expecting to teach Tech 101 to a group of
clueless lawmakers, they instead found themselves in the principal's
office, being confronted with evidence of the spitballs they'd thrown.
And they must have realized, in those moments, that they were seeing the
beginnings of accountability.

Advertisement

\protect\hyperlink{after-bottom}{Continue reading the main story}

\hypertarget{site-index}{%
\subsection{Site Index}\label{site-index}}

\hypertarget{site-information-navigation}{%
\subsection{Site Information
Navigation}\label{site-information-navigation}}

\begin{itemize}
\tightlist
\item
  \href{https://help.nytimes3xbfgragh.onion/hc/en-us/articles/115014792127-Copyright-notice}{©~2020~The
  New York Times Company}
\end{itemize}

\begin{itemize}
\tightlist
\item
  \href{https://www.nytco.com/}{NYTCo}
\item
  \href{https://help.nytimes3xbfgragh.onion/hc/en-us/articles/115015385887-Contact-Us}{Contact
  Us}
\item
  \href{https://www.nytco.com/careers/}{Work with us}
\item
  \href{https://nytmediakit.com/}{Advertise}
\item
  \href{http://www.tbrandstudio.com/}{T Brand Studio}
\item
  \href{https://www.nytimes3xbfgragh.onion/privacy/cookie-policy\#how-do-i-manage-trackers}{Your
  Ad Choices}
\item
  \href{https://www.nytimes3xbfgragh.onion/privacy}{Privacy}
\item
  \href{https://help.nytimes3xbfgragh.onion/hc/en-us/articles/115014893428-Terms-of-service}{Terms
  of Service}
\item
  \href{https://help.nytimes3xbfgragh.onion/hc/en-us/articles/115014893968-Terms-of-sale}{Terms
  of Sale}
\item
  \href{https://spiderbites.nytimes3xbfgragh.onion}{Site Map}
\item
  \href{https://help.nytimes3xbfgragh.onion/hc/en-us}{Help}
\item
  \href{https://www.nytimes3xbfgragh.onion/subscription?campaignId=37WXW}{Subscriptions}
\end{itemize}
