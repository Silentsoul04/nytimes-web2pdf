\href{/section/theater}{Theater}\textbar{}How I Spent My Summer
Vacation: Singing, Dancing, Knife Fighting

\url{https://nyti.ms/3jU29vc}

\begin{itemize}
\item
\item
\item
\item
\item
\end{itemize}

\href{https://www.nytimes3xbfgragh.onion/spotlight/at-home?action=click\&pgtype=Article\&state=default\&region=TOP_BANNER\&context=at_home_menu}{At
Home}

\begin{itemize}
\tightlist
\item
  \href{https://www.nytimes3xbfgragh.onion/2020/07/28/books/time-for-a-literary-road-trip.html?action=click\&pgtype=Article\&state=default\&region=TOP_BANNER\&context=at_home_menu}{Take:
  A Literary Road Trip}
\item
  \href{https://www.nytimes3xbfgragh.onion/2020/07/29/magazine/bored-with-your-home-cooking-some-smoky-eggplant-will-fix-that.html?action=click\&pgtype=Article\&state=default\&region=TOP_BANNER\&context=at_home_menu}{Cook:
  Smoky Eggplant}
\item
  \href{https://www.nytimes3xbfgragh.onion/2020/07/27/travel/moose-michigan-isle-royale.html?action=click\&pgtype=Article\&state=default\&region=TOP_BANNER\&context=at_home_menu}{Look
  Out: For Moose}
\item
  \href{https://www.nytimes3xbfgragh.onion/interactive/2020/at-home/even-more-reporters-editors-diaries-lists-recommendations.html?action=click\&pgtype=Article\&state=default\&region=TOP_BANNER\&context=at_home_menu}{Explore:
  Reporters' Obsessions}
\end{itemize}

\includegraphics{https://static01.graylady3jvrrxbe.onion/images/2020/07/31/arts/31Theater-Skils-Illo/31Theater-Skils-Illo-articleLarge.jpg?quality=75\&auto=webp\&disable=upscale}

Sections

\protect\hyperlink{site-content}{Skip to
content}\protect\hyperlink{site-index}{Skip to site index}

\hypertarget{how-i-spent-my-summer-vacation-singing-dancing-knife-fighting}{%
\section{How I Spent My Summer Vacation: Singing, Dancing, Knife
Fighting}\label{how-i-spent-my-summer-vacation-singing-dancing-knife-fighting}}

When actor training migrated online, our reporter gave herself two weeks
to learn as many theater skills --- and knife skills --- as she could.

Credit...Marly Gallardo

Supported by

\protect\hyperlink{after-sponsor}{Continue reading the main story}

By \href{https://www.nytimes3xbfgragh.onion/by/alexis-soloski}{Alexis
Soloski}

\begin{itemize}
\item
  July 30, 2020
\item
  \begin{itemize}
  \item
  \item
  \item
  \item
  \item
  \end{itemize}
\end{itemize}

A few Fridays ago, just before what I had come to think of as
``showtime,'' I lined my eyes, stepped into my costume, readied a prop
and adjusted the lighting, which mostly meant fiddling with a bedside
lamp. Then I logged into a Zoom meeting.

That meeting, an online recital for friends and colleagues, capped a
brief and frantic curriculum of voice, movement, scene study, stage
combat and some dubious dialect work. After the spring closure of
theaters and studios, nearly every training institution adopted a remote
learning model. Artists, suddenly unemployed, could advertise their
services on new online agorae, like
\href{https://www.arenastage.org/artistsmarketplace}{Arena Stage's
Theater Artists Marketplace} and \href{https://hireartists.org/}{Hire
Artists}. Which means that theater training, at a variety of price
points, has never been more available or accessible.

Screen-to-screen classes don't exactly parallel in-person ones. They are
shorter and often smaller --- three-hour classes have shrunk to 80
minutes, and breakout groups are the new norm --- with altered
methodology. ``Zoom has its own vocabulary,'' said
\href{https://tisch.nyu.edu/about/directory/grad-acting/102885218}{Laurence
Maslon}, associate chair of New York University's graduate acting
program. ``It isn't live. It isn't in the room. That doesn't mean you
can't achieve something.''

I wasn't sure. As an undergraduate 20 years ago, I had majored in
theater and back then, our training was exclusively and incontrovertibly
face to face. Good acting happened in the moment, in the room, in the
space between bodies and breath, action and intention. You couldn't
teach that online! (Admittedly, ``online'' back then meant ``dial-up
internet.'')

Or could you?

\hypertarget{we-have-work-to-do}{%
\subsection{`We have work to do'}\label{we-have-work-to-do}}

For two humbling and sometimes humiliating weeks, I tried. With the help
of friends, social media, frantic Googling and enough Disney+ shows to
keep the children occupied, I designed a mostly live, all-remote
conservatory training program. I wanted to see if someone like me ---
busy, amateur, with an instrument almost fully oxidized --- could learn
theater skills.

I started with vocal work, arranging a voice lesson via
\href{https://broadwayplus.com/}{Broadway Plus}, a concierge service
that used to arrange V.I.P. access to Broadway performances and has
since pivoted to online meet-and-greets and private lessons. As part of
a publicity push for the ``Hamilton'' movie, Denée Benton, a Tony
Award-nominated actress and a replacement Eliza, had volunteered to do
some coaching. I am not a singer, which is less false modesty than true
and harrowing fact, and Benton,
\href{https://www.nytimes3xbfgragh.onion/2017/05/11/theater/denee-benton-tony-nominee-embraces-her-inner-natasha.html}{whom
I had interviewed} during her run in the Broadway production of
``\href{https://www.nytimes3xbfgragh.onion/2016/11/15/theater/natasha-pierre-and-the-great-comet-of-1812-review.html}{Natasha,
Pierre \& the Great Comet of 1812,}'' seemed extremely sympathetic.

\includegraphics{https://static01.graylady3jvrrxbe.onion/images/2020/07/30/arts/30theater-skills-videostill/30theater-skills-videostill-videoSixteenByNine3000.png}

After polling friends about a good song for a nice lady with a
Playbill-slim range and a shaky grasp of pitch, I picked
``\href{https://www.youtube.com/watch?v=_HHmHv26tao}{Sonya Alone},''
from ``Natasha, Pierre.'' Because I love it, because you can kind of
talk your way through the opening, because it hadn't been one of
Benton's numbers. I rehearsed when I could --- in the shower, cooking
dinner, under my breath at various playgrounds. By the time the lesson
came around, I had it down.

Then, sitting at my desk, with Benton smiling back at me from somewhere
in the Midwest, I didn't. My shoulders tensed, my throat closed, a
chipmunk hijacked my voice box. The piano intro started, and I sang as
only a squeaky toy can. But worse somehow.

``We have work to do,'' Benton said when the song let me go. ``That's
the point.''

Gently, she helped me break apart the song --- unlocking character and
emotion --- then put it back together. ``When you focus on the
storytelling, it can make anyone a singer,'' she said. Not quite. But by
the end of the hour I could approach the chorus with emotions beside
dread.

\hypertarget{cant-fake-a-rond-de-jambe}{%
\subsection{Can't fake a rond de
jambe}\label{cant-fake-a-rond-de-jambe}}

Though I once won a limbo contest at a classmate's bar mitzvah, dance
has also never been my thing. Still, I figured that Beginner Theater
Dance, which I signed up for through
\href{https://www.aileyextension.com/}{Ailey Extension}, couldn't be so
hard. I figured wrong.

We warmed up to selections from ``The Lion King'' and ``The Prince of
Egypt.'' I even learned a Fosse hip roll. But as we danced to ``No Day
but Today,'' the ``Rent'' finale, the ballet terms --- passé, coupé,
rond de jambe --- proliferated and the eight counts came worryingly
fast. Though I had positioned my laptop camera so that it showed only me
from the rib cage up. I couldn't even fake the arms.

Maybe that's because, as I soon learned, a level exists even below
Beginner. That level is Basic. So am I. The following week, I tried
\href{https://www.stepsnyc.com/}{Steps on Broadway}'s Basic Theater
Dance. The instructor, Tera-Lee Pollin, a Broadway veteran with inhuman
exuberance, guided a handful of students through ``Waterloo,'' the
curtain number for ``Mamma Mia!'' and a song about defeat. Together,
delightedly, we ponied, we swam, we
\href{https://www.youtube.com/watch?v=1BESmZUXIJs}{grapevined}. No
jambes were ronded.

\includegraphics{https://static01.graylady3jvrrxbe.onion/images/2020/07/30/arts/30theater-skills-1/30theater-skills-1-articleLarge.jpg?quality=75\&auto=webp\&disable=upscale}

I could also just about manage the footwork required for knife
techniques, which I learned through
\href{http://www.swordplaystagecombat.com/}{Swordplay}. In advance of
the course, the instructor, Joseph Travers, had sent me a bubble-wrapped
training knife. (Was I disappointed to discover it was merely a hunk of
ridged plastic? I was.) Through YouTube videos and private tutorials, I
learned various grips, stances, cuts and blocks. This may just be
pent-up pandemic anxiety talking, but I love stage combat now. My new
party trick, assuming we ever have parties again: a fan grip switch, a
flip from the overhand forward grip to the reverse ``Psycho'' grip.

I asked Travers how much combat, a skill that seems to demand physical
intimacy, could be taught online. ``There's plenty of groundwork to be
laid for the individual actor,'' he said. ``But ultimately, we have to
face each other and fight.''

Dialect work, however, has been learned remotely from the days of the
phonograph. At the urging of an editor who may not have had my best
interests at heart, I chose Scottish, working through a few MP3 files
each day --- learning to position resonance lower in my mouth, lilt
internal vowels, trill Rs and drop most Gs. The first days were
unspeakable, with an accent that vacillated between demented Valley Girl
and Southern Belle with cognitive difficulties. But a week in something
shifted and I began to sound reliably, if hammily, Scottish. I wrote to
a Scottish friend and asked if I could test it out on him. He asked
after the region: Border? Highlands? Glasgow? ``Brigadoon,'' I told him.
He never wrote back.

\hypertarget{tell-a-story-and-be-real}{%
\subsection{`Tell a story and be real'}\label{tell-a-story-and-be-real}}

The acting component felt trickier, mostly because I used to act and I
like to believe I wasn't terrible at it --- and I prefer that belief
uncrushed. With the help of a contact at the Juilliard School, I wrote
to two alumni who do online training one on one: Jimonn Cole, who would
coach me on a classical speech, and Jo Mei, who would work with me on a
contemporary one.

I met Cole first. He suggested a monologue from ``As You Like It'' and
after a series of vocal warm-ups --- tongue twisters, meowing --- and a
guided meditation that helped to establish the look and feel and precise
pH of the Forest of Arden, we went into it. With calm and rigor, he had
me note rhythm, punctuation, language, intention, plus vocal register.
``Shakespeare is still just talking,'' he said when he saw me start to
tense up.

During our second meeting, with the piece now memorized, we worked on
character, and he told me to make my Rosalind meaner, more vicious. ``If
that was venom at level 5, scold at level 9,'' he said. I am sorry to
shatter anyone's preconceived notions about critics, but this was very
hard for me! I made it to about a 6.

For my sessions with Mei, I had chosen a quieter piece, the opening of
Lucas Hnath's ``The Thin Place,'' which begins casually and gets
creepier. Mei asked questions about the character and she pointed out
punctuation, too, like the marked differences among a dash, a period and
an ellipsis. But her method was less prescriptive, mostly jokes and
friendly suggestions, like picking just one place to smile and not
over-relying on a particular hand gesture. ``The challenge of this one
is how to relax into it and just tell a story and be real,'' she said.
Each time I went through it I felt as if I was acting a little less and
being a little more.

Mei thought that we should work toward a goal, so she emailed a few
friends. I did the same, and the day after our second session, we all
met up on Zoom. In the moments before I went on --- ``on'' meaning that
I dropped into a chair shoved between the desk and the bed --- I felt a
paler version of what I had felt backstage 20 years ago, the
butterflies, the flop sweat, the jolting adrenaline.

I thought of that scene, from ``42nd Street,'' in which the director
tells the ingénue, ``You're going out a youngster, but you've got to
come back a star.'' I was pretty sure I was going out a theater critic
and coming back a theater critic. And I was. And I did. But even through
a screen, it made me remember --- viscerally, a little regretfully ---
that strange magic of speaking someone else's words and feeling someone
else's feelings and making them, for a moment, your own.

What did I learn? I mean, beyond saber grip and a ``Mamma Mia!'' move
called ``the coffee grinder,'' which terrifies and delights the
children? I learned --- or I was reminded --- that acting and its
associated skills are hard, that they require real vulnerability, that
it takes weeks and months and years of thankless exertion, solitary
muttering and practice, practice, practice to make an effortful thing
seem effortless. I learned that when I thought I couldn't miss live
theater any more acutely, I was wrong. I learned that as soon as it is
safe to do so, I will absolutely knife fight someone.

So, yes, any amateur with enough time and resilience and discretionary
income --- a class can run anywhere from \$12 to \$100 --- can probably
learn theater basics remotely. Then again, as Travers said, ultimately
we have to face one another, with or without knives. Because the alchemy
of live acting before a live audience almost comes through onscreen. But
not quite. Until it can, I will think of the thousands and thousands of
people in their thousands and thousands of homes, practicing their
pentameter, arabesques and key changes, waiting for curtains to rise.

Advertisement

\protect\hyperlink{after-bottom}{Continue reading the main story}

\hypertarget{site-index}{%
\subsection{Site Index}\label{site-index}}

\hypertarget{site-information-navigation}{%
\subsection{Site Information
Navigation}\label{site-information-navigation}}

\begin{itemize}
\tightlist
\item
  \href{https://help.nytimes3xbfgragh.onion/hc/en-us/articles/115014792127-Copyright-notice}{©~2020~The
  New York Times Company}
\end{itemize}

\begin{itemize}
\tightlist
\item
  \href{https://www.nytco.com/}{NYTCo}
\item
  \href{https://help.nytimes3xbfgragh.onion/hc/en-us/articles/115015385887-Contact-Us}{Contact
  Us}
\item
  \href{https://www.nytco.com/careers/}{Work with us}
\item
  \href{https://nytmediakit.com/}{Advertise}
\item
  \href{http://www.tbrandstudio.com/}{T Brand Studio}
\item
  \href{https://www.nytimes3xbfgragh.onion/privacy/cookie-policy\#how-do-i-manage-trackers}{Your
  Ad Choices}
\item
  \href{https://www.nytimes3xbfgragh.onion/privacy}{Privacy}
\item
  \href{https://help.nytimes3xbfgragh.onion/hc/en-us/articles/115014893428-Terms-of-service}{Terms
  of Service}
\item
  \href{https://help.nytimes3xbfgragh.onion/hc/en-us/articles/115014893968-Terms-of-sale}{Terms
  of Sale}
\item
  \href{https://spiderbites.nytimes3xbfgragh.onion}{Site Map}
\item
  \href{https://help.nytimes3xbfgragh.onion/hc/en-us}{Help}
\item
  \href{https://www.nytimes3xbfgragh.onion/subscription?campaignId=37WXW}{Subscriptions}
\end{itemize}
