Sections

SEARCH

\protect\hyperlink{site-content}{Skip to
content}\protect\hyperlink{site-index}{Skip to site index}

\href{https://www.nytimes3xbfgragh.onion/section/us}{U.S.}

\href{https://myaccount.nytimes3xbfgragh.onion/auth/login?response_type=cookie\&client_id=vi}{}

\href{https://www.nytimes3xbfgragh.onion/section/todayspaper}{Today's
Paper}

\href{/section/us}{U.S.}\textbar{}John Lewis, a Man of `Unbreakable
Perseverance,' Is Laid to Rest

\url{https://nyti.ms/2DjGQm2}

\begin{itemize}
\item
\item
\item
\item
\item
\item
\end{itemize}

Advertisement

\protect\hyperlink{after-top}{Continue reading the main story}

Supported by

\protect\hyperlink{after-sponsor}{Continue reading the main story}

\hypertarget{john-lewis-a-man-of-unbreakable-perseverance-is-laid-to-rest}{%
\section{John Lewis, a Man of `Unbreakable Perseverance,' Is Laid to
Rest}\label{john-lewis-a-man-of-unbreakable-perseverance-is-laid-to-rest}}

The civil rights leader was eulogized by former presidents and family in
Atlanta, the city he represented in Congress for more than three
decades.

\includegraphics{https://static01.graylady3jvrrxbe.onion/images/2020/07/30/us/30lewis-funeral5/merlin_175122897_9cfada3c-07bb-4531-8889-888d54df0318-articleLarge.jpg?quality=75\&auto=webp\&disable=upscale}

By \href{https://www.nytimes3xbfgragh.onion/by/richard-fausset}{Richard
Fausset} and
\href{https://www.nytimes3xbfgragh.onion/by/rick-rojas}{Rick Rojas}

\begin{itemize}
\item
  July 30, 2020
\item
  \begin{itemize}
  \item
  \item
  \item
  \item
  \item
  \item
  \end{itemize}
\end{itemize}

ATLANTA --- Three former presidents and dozens of other dignitaries were
drawn to Ebenezer Baptist Church on Thursday to bid farewell to John
Lewis, a giant of Congress and the civil rights era whose courageous
protests guaranteed him a place in American history. But even as the
funeral
\href{https://www.nytimes3xbfgragh.onion/2020/07/17/us/john-lewis-dead.html}{looked
back over Mr. Lewis's long life}, it also focused very much on the
tumultuous state of affairs in the country today.

The most pointed
\href{https://www.nytimes3xbfgragh.onion/2020/07/30/us/obama-eulogy-john-lewis-full-transcript.html}{eulogy
came from former President Barack Obama}, who issued a blistering
critique of the Trump administration, the brutality of police officers
toward Black people and efforts to limit the right to vote that Mr.
Lewis had shed his blood to secure.

The political tone of the ceremony came as little surprise. Mr. Lewis,
who died July 17 **** at the age of 80 after a battle with pancreatic
cancer, had spent more than three decades in Congress as a thorn in the
side of Republican administrations. And he and President Trump had
traded public slights since before Mr. Trump took office.

\includegraphics{https://static01.graylady3jvrrxbe.onion/images/2020/07/30/us/30lewis-briefing2/merlin_175114953_3815007c-1217-437b-ae11-ed3939c05d88-articleLarge.jpg?quality=75\&auto=webp\&disable=upscale}

Mr. Obama compared Mr. Lewis to an Old Testament prophet and credited
him with directly paving the way for the nation's first Black president.
He also took aim at the forces that he said were working against the
equality for Black Americans and other oppressed people that Mr. Lewis
had spent a lifetime championing.

``Bull Connor may be gone,'' Mr. Obama said, referring to the 1960s-era
public safety commissioner of Birmingham, Ala., who turned fire hoses
and dogs on civil rights protesters. ``But today, we witness, with our
own eyes, police officers kneeling on the necks of Black Americans.''

George Wallace, the Alabama governor who endorsed segregation and used
racist language, may also be gone, Mr. Obama continued. ``But we can
witness our federal government sending agents to use tear gas and batons
against peaceful demonstrators.''

And while insuperable poll tests for Black people may be a thing of the
past, Mr. Obama said, ``Even as we sit here, there are those in power
who are doing their darnedest to discourage people from voting by
closing polling locations, and targeting minorities and students with
restrictive ID laws, and attacking our voting rights with surgical
precision.''

The critique elicited a torrent of applause from the invitation-only
audience at Ebenezer Baptist Church, the famed institution that Mr.
Lewis attended and where the Rev. Dr. Martin Luther King Jr., Mr.
Lewis's mentor and ally, once preached.

The mourners, masked to prevent the spread of the coronavirus, were
strategically limited in number to ensure social distancing. Some took
their seats as an organist played ``We Shall Overcome,'' a protest
anthem sung by Mr. Lewis countless times during his nonviolent
confrontations with segregationist forces in the South who beat and
injured him on several occasions.

Mr. Obama praised Mr. Lewis in his eulogy as an ``American whose faith
was tested again and again to produce a man of pure joy and unbreakable
perseverance.''

Image

Mr. Obama delivered the eulogy,~comparing Mr. Lewis to an Old Testament
prophet.Credit...Alyssa Pointer/Via Reuters

In death, Mr. Lewis drew a bipartisan crowd, including former presidents
George W. Bush and Bill Clinton, although Mr. Trump did not attend.
Speaker Nancy Pelosi and dozens of members of Congress were also at the
three-hour service, presided over by Ebenezer's pastor, the Rev. Dr.
Raphael G. Warnock, who is running as a Democrat for a Senate seat.

Mr. Bush gave a short, warm speech in which he praised Mr. Lewis's
Christian faith and recalled working with him to establish the National
Museum of African American History and Culture in Washington.

``Listen, John and I had our disagreements, of course,'' said Mr. Bush,
a Republican. ``But in the America John Lewis fought for, and the
America I believe in, differences of opinion are inevitable elements and
evidence of democracy in action.''

The line was as well received as Mr. Bush himself: Dr. Warnock noted
that Mr. Bush was president ``the last time we reauthorized the Voting
Rights Act.'' It was a markedly different tone from the 2006 funeral of
Coretta Scott King, Dr. King's widow, in which numerous speakers
criticized the Bush administration while Mr. Bush, then in his second
term, looked on.

Mr. Clinton called Mr. Lewis ``a man I loved for a long time'' and
someone who was ``on a mission that was bigger than personal ambition.''

He also said that Mr. Lewis had learned a lesson after he was asked by
other civil rights leaders to tone down a fiery speech that he had
written for the March on Washington in August 1963. ``He listened to
people that he knew had the same goals say, `Well, we have to be careful
how we say **** this because we're trying to get converts, not more
adversaries.'''

It came as little surprise that Mr. Lewis's funeral would dwell as much
on the present as on his younger days, including his beating in 1965 by
Alabama state troopers at the Edmund Pettus Bridge in Selma, Ala., an
atrocity that helped spur Congress to pass the Voting Rights Act. On
Thursday, The New York Times, at Mr. Lewis's request, published an essay
he wrote that
\href{https://www.nytimes3xbfgragh.onion/2020/07/30/opinion/john-lewis-civil-rights-america.html}{praised
the Black Lives Matter movement} and urged continued participation in
the democratic process.

Mr. Lewis's funeral also came amid a fraught season in Atlanta, which he
represented as part of his Fifth Congressional District. In recent
weeks, the city has been rocked by protests, both peaceful and violent,
over systemic racism and police brutality.

His coffin was carried into the sanctuary not long after Mr. Trump made
unsubstantiated assertions of potential vote-by-mail fraud and floated
the idea of postponing the election this November --- a suggestion that
shocked both critics and allies and heightened concerns that he might
not accept the results if he were to lose.

At the same time, however, the funeral underscored how Mr. Lewis
believed that his decades-long civil rights fight could be waged in a
spirit of comity --- and with a belief that the American project was not
fatally flawed, but perfectible in the hands of a citizenry willing to
go to the polls and engage in nonviolent protest.

The need to form coalitions with converts to the civil rights cause,
including white people, was a cornerstone of Mr. Lewis's belief system,
and it clashed in the mid-1960s with a more radical branch of the
movement that was skeptical of nonviolence as an effective strategy and
prioritized Black political and economic power over integration.

Stokely Carmichael, a proponent of this worldview who would later change
his name to Kwame Ture, was chosen to replace Mr. Lewis in 1966 as
chairman of the Student Nonviolent Coordinating Committee.

``There were two or three years there where the movement went a little
bit too far toward Stokely,'' Mr. Clinton said. ``But in the end John
Lewis prevailed.''

That philosophical tension, however, continues to play out among
African-Americans and the activists powering the current Black Lives
Matter movement.

In his many years in public life, Mr. Lewis played crucial roles as a
critic who stood --- often literally --- against government power, but
also as one who worked within the system as a 33-year member of
Congress.

The encomiums were not all of a political nature. The ceremony, which
began with a tolling of a bell for each of his 80 years, also featured
warm remembrances from family and staff members. A number of speakers
revived the story of how a young Mr. Lewis, who grew up on an
impoverished farm near Troy, Ala., used to preach to his family's
chickens.

Called to something bigger, he eventually met Dr. King, who famously
nicknamed him ``the boy from Troy.''

Image

Xernona Clayton, a civil rights leader and the godmother of Mr. Lewis's
son, addressed the service.Credit...Alyssa Pointer/via Reuters

Xernona Clayton, a longtime civil rights advocate, humorously recalled
her strong-armed and ultimately successful efforts to effect a love
match between a young Mr. Lewis and his future wife, Lillian Miles
Lewis, who died in 2013. Mr. Lewis, she said, seemed like a man who was
going places, unlike ``the bums'' who had approached Lillian in the
past.

In his eulogy, Mr. Obama, among other things, called on Congress to pass
a new Voting Rights Act named for Mr. Lewis, for the end of
gerrymandering and for the establishment of a national holiday on
Election Day to make it easier for working people to get to the polls.

Echoing a favored theme, Mr. Obama also praised Mr. Lewis for
understanding that it takes not only faith but hard work to improve the
country and keep a healthy democracy on course.

Mr. Lewis exhibited, he said, ``that most American of ideas --- the idea
that any of us ordinary people, without rank or wealth or title or fame,
can somehow point out the imperfections of this nation and come together
and challenge the status quo and decide that it is in our power to
remake this country that we love until it more closely aligns with our
highest ideals.''

For that, he said, Mr. Lewis would come to be viewed as ``a founding
father of that fuller, fairer, better America.''

The sentiment resonated with the crowd that had gathered outside the
church. Many dashed over to get a look as Mr. Lewis's coffin was carried
out.

But Latasha Cosby-Woods stood off to the side with her hand raised in
prayer. She prayed, she said, for unity and justice, and for a new
younger generation of activists: ``the John Lewises of the world,'' she
said, who might ``go forth and make a difference.''

Lucy Tompkins contributed reporting from New York.

Advertisement

\protect\hyperlink{after-bottom}{Continue reading the main story}

\hypertarget{site-index}{%
\subsection{Site Index}\label{site-index}}

\hypertarget{site-information-navigation}{%
\subsection{Site Information
Navigation}\label{site-information-navigation}}

\begin{itemize}
\tightlist
\item
  \href{https://help.nytimes3xbfgragh.onion/hc/en-us/articles/115014792127-Copyright-notice}{©~2020~The
  New York Times Company}
\end{itemize}

\begin{itemize}
\tightlist
\item
  \href{https://www.nytco.com/}{NYTCo}
\item
  \href{https://help.nytimes3xbfgragh.onion/hc/en-us/articles/115015385887-Contact-Us}{Contact
  Us}
\item
  \href{https://www.nytco.com/careers/}{Work with us}
\item
  \href{https://nytmediakit.com/}{Advertise}
\item
  \href{http://www.tbrandstudio.com/}{T Brand Studio}
\item
  \href{https://www.nytimes3xbfgragh.onion/privacy/cookie-policy\#how-do-i-manage-trackers}{Your
  Ad Choices}
\item
  \href{https://www.nytimes3xbfgragh.onion/privacy}{Privacy}
\item
  \href{https://help.nytimes3xbfgragh.onion/hc/en-us/articles/115014893428-Terms-of-service}{Terms
  of Service}
\item
  \href{https://help.nytimes3xbfgragh.onion/hc/en-us/articles/115014893968-Terms-of-sale}{Terms
  of Sale}
\item
  \href{https://spiderbites.nytimes3xbfgragh.onion}{Site Map}
\item
  \href{https://help.nytimes3xbfgragh.onion/hc/en-us}{Help}
\item
  \href{https://www.nytimes3xbfgragh.onion/subscription?campaignId=37WXW}{Subscriptions}
\end{itemize}
