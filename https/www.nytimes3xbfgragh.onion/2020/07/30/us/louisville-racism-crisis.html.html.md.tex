Sections

SEARCH

\protect\hyperlink{site-content}{Skip to
content}\protect\hyperlink{site-index}{Skip to site index}

\href{https://www.nytimes3xbfgragh.onion/section/us}{U.S.}

\href{https://myaccount.nytimes3xbfgragh.onion/auth/login?response_type=cookie\&client_id=vi}{}

\href{https://www.nytimes3xbfgragh.onion/section/todayspaper}{Today's
Paper}

\href{/section/us}{U.S.}\textbar{}Louisville May Become Latest City to
Declare Racism a Public Health Crisis

\url{https://nyti.ms/3fexGEP}

\begin{itemize}
\item
\item
\item
\item
\item
\end{itemize}

\href{https://www.nytimes3xbfgragh.onion/news-event/george-floyd-protests-minneapolis-new-york-los-angeles?action=click\&pgtype=Article\&state=default\&region=TOP_BANNER\&context=storylines_menu}{Race
and America}

\begin{itemize}
\tightlist
\item
  \href{https://www.nytimes3xbfgragh.onion/2020/07/26/us/protests-portland-seattle-trump.html?action=click\&pgtype=Article\&state=default\&region=TOP_BANNER\&context=storylines_menu}{Protesters
  Return to Other Cities}
\item
  \href{https://www.nytimes3xbfgragh.onion/2020/07/24/us/portland-oregon-protests-white-race.html?action=click\&pgtype=Article\&state=default\&region=TOP_BANNER\&context=storylines_menu}{Portland
  at the Center}
\item
  \href{https://www.nytimes3xbfgragh.onion/2020/07/23/podcasts/the-daily/portland-protests.html?action=click\&pgtype=Article\&state=default\&region=TOP_BANNER\&context=storylines_menu}{Podcast:
  Showdown in Portland}
\item
  \href{https://www.nytimes3xbfgragh.onion/interactive/2020/07/16/us/black-lives-matter-protests-louisville-breonna-taylor.html?action=click\&pgtype=Article\&state=default\&region=TOP_BANNER\&context=storylines_menu}{45
  Days in Louisville}
\end{itemize}

Advertisement

\protect\hyperlink{after-top}{Continue reading the main story}

Supported by

\protect\hyperlink{after-sponsor}{Continue reading the main story}

\hypertarget{louisville-may-become-latest-city-to-declare-racism-a-public-health-crisis}{%
\section{Louisville May Become Latest City to Declare Racism a Public
Health
Crisis}\label{louisville-may-become-latest-city-to-declare-racism-a-public-health-crisis}}

``We must have a new sense of urgency to make this declaration,'' Mayor
Greg Fischer said. The city has been the focus of protests after the
police killing of Breonna Taylor.

\includegraphics{https://static01.graylady3jvrrxbe.onion/images/2020/07/29/multimedia/29xp-racism-louisville/29xp-racism-louisville-articleLarge.jpg?quality=75\&auto=webp\&disable=upscale}

\href{https://www.nytimes3xbfgragh.onion/by/azi-paybarah}{\includegraphics{https://static01.graylady3jvrrxbe.onion/images/2019/02/14/multimedia/author-azi-paybarah/author-azi-paybarah-thumbLarge.png}}

By \href{https://www.nytimes3xbfgragh.onion/by/azi-paybarah}{Azi
Paybarah}

\begin{itemize}
\item
  July 30, 2020
\item
  \begin{itemize}
  \item
  \item
  \item
  \item
  \item
  \end{itemize}
\end{itemize}

Officials in Louisville, Ky., where there have been continuing protests
over the police killing of Breonna Taylor and where coronavirus cases
have begun to rise, said on Wednesday that the city might join a growing
list of local governments that have declared racism a public health
crisis.

Local officials in 19 states, including California, Connecticut,
Georgia, Michigan, Texas and Wisconsin, have passed or are considering
similar resolutions,
\href{https://www.apha.org/topics-and-issues/health-equity/racism-and-health/racism-declarations}{according
to the American Public Health Association}.

``We must have a new sense of urgency to make this declaration and do
the hard work of dismantling racism and creating real transformation,''
Mayor Greg Fischer of Louisville
\href{https://twitter.com/louisvillemayor/status/1288602379408084994}{wrote
on Twitter on Wednesday}. ``I look forward to partnering with Council on
this work of declaring racism as a public health emergency.''

How exactly Louisville makes its declaration is still being negotiated,
one health official said. It could be made through a resolution or
ordinance passed by the Louisville Metro Council, or through an
executive order from the mayor.

Once declared, it could help establish funding priorities and possibly
lead to regular reports on how local government officials are addressing
racial inequality, according to T. Benicio Gonzales, director of the
Center for Health Equity at the Louisville health department.

If racism is identified as a public health emergency, then local
lawmakers could say to the heads of city agencies, ``What are you doing
in order to address this important health issue?'' Mx. Gonzales said in
an interview.

Mx. Gonzales said the declaration could provide ``an opportunity to be
explicit about what are the causes and barriers to good health.''

Earlier on Wednesday, officials at a Louisville Metro Council meeting
said health and income disparities fell disproportionately along racial
lines in the city.

Kendall Boyd, chief equity officer for the mayor's office, presented a
report at the hearing that found Black residents of the city were 2.6
times more likely to be arrested by the local police than white
residents.

The report also found that in 2016, the unemployment rate among Black
residents was 11 percent, compared with 5 percent for white residents,
and the mortality rate for Black infants was 2.5 times higher than for
white babies, from 2011 through 2015.

``When someone doesn't look at racism and equity every day like I do, it
can be hard to connect those dots,'' Mr. Boyd said. ``So that's why if
we declare racism and racist attitudes and racist structures as part of
this overall public health crisis, we can see how racism literally can
kill people.''

Mr. Boyd said that the coronavirus pandemic and the recent protests for
racial justice sparked by police killings of Black people --- including
Ms. Taylor in Louisville and George Floyd in Minneapolis --- had given
these centuries-old problems new urgency.

Ms. Taylor, a 26-year-old emergency room technician,
\href{https://www.nytimes3xbfgragh.onion/2020/05/14/us/breonna-taylor-louisville-shooting.html}{was
shot and killed} on March 13 by Louisville police officers who entered
her home using a battering ram while executing a search warrant. The
police have said that they returned fire when Ms. Taylor's boyfriend
shot an officer in the leg. Several investigations into Ms. Taylor's
killing are
\href{https://twitter.com/louisvillemayor/status/1288464475272679426}{underway}.

One Louisville officer was
\href{https://www.nytimes3xbfgragh.onion/article/breonna-taylor-police.html}{terminated
in June}; two other officers have been placed
\href{https://www.nytimes3xbfgragh.onion/2020/06/23/us/breonna-taylor-brett-hankison-fired.html}{on
administrative reassignment}.

On Wednesday morning, the mayor wrote on Twitter that he shared ``the
frustration at the pace of justice as we all await'' the results of the
investigations into Ms. Taylor's killing.

Some residents on social media criticized the proposed declaration,
responding to the mayor's tweet and Facebook post about the news with
calls to bring justice for Ms. Taylor's family. Other residents saw the
move as a distraction.

Tim Wohlford, a Louisville resident, said the declaration was a symbolic
move by the mayor, rather than a substantive one. ``Everybody was
yelling at him for six years to do something about the police,'' Mr.
Wohlford, 58, said.

``Racism is an all-encompassing term,'' he said, and it was unfair to
expect health officials to address such a wide-ranging problem. ``No way
in hell. They don't have the funding or the expertise.''

Since the beginning of July, the coronavirus has continued to spread
throughout Kentucky,
\href{https://www.nytimes3xbfgragh.onion/interactive/2020/us/kentucky-coronavirus-cases.html\#county}{according
to a New York Times database}. The seven-day average of new cases at the
beginning of the month was 227. By July 28, that figure was 605.

In Jefferson County, which includes Louisville, there are 854 cases per
100,000 people, above the statewide total of 660 cases per 100,000.

According to Mr. Boyd, Black residents are 23 percent of the city's
population but account for about 27 percent of coronavirus-related
deaths.

Moving forward, Mr. Boyd said funding was needed to address racism as a
health crisis.

``Just putting something on paper and declaring it a public health
crisis is one thing,'' Mr. Boyd said. ``There has to be a specific
intentionality and commitment of resources addressing this as a public
health crisis.''

Advertisement

\protect\hyperlink{after-bottom}{Continue reading the main story}

\hypertarget{site-index}{%
\subsection{Site Index}\label{site-index}}

\hypertarget{site-information-navigation}{%
\subsection{Site Information
Navigation}\label{site-information-navigation}}

\begin{itemize}
\tightlist
\item
  \href{https://help.nytimes3xbfgragh.onion/hc/en-us/articles/115014792127-Copyright-notice}{©~2020~The
  New York Times Company}
\end{itemize}

\begin{itemize}
\tightlist
\item
  \href{https://www.nytco.com/}{NYTCo}
\item
  \href{https://help.nytimes3xbfgragh.onion/hc/en-us/articles/115015385887-Contact-Us}{Contact
  Us}
\item
  \href{https://www.nytco.com/careers/}{Work with us}
\item
  \href{https://nytmediakit.com/}{Advertise}
\item
  \href{http://www.tbrandstudio.com/}{T Brand Studio}
\item
  \href{https://www.nytimes3xbfgragh.onion/privacy/cookie-policy\#how-do-i-manage-trackers}{Your
  Ad Choices}
\item
  \href{https://www.nytimes3xbfgragh.onion/privacy}{Privacy}
\item
  \href{https://help.nytimes3xbfgragh.onion/hc/en-us/articles/115014893428-Terms-of-service}{Terms
  of Service}
\item
  \href{https://help.nytimes3xbfgragh.onion/hc/en-us/articles/115014893968-Terms-of-sale}{Terms
  of Sale}
\item
  \href{https://spiderbites.nytimes3xbfgragh.onion}{Site Map}
\item
  \href{https://help.nytimes3xbfgragh.onion/hc/en-us}{Help}
\item
  \href{https://www.nytimes3xbfgragh.onion/subscription?campaignId=37WXW}{Subscriptions}
\end{itemize}
