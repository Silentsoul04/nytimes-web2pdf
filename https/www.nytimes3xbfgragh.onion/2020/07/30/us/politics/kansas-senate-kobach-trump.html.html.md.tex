Sections

SEARCH

\protect\hyperlink{site-content}{Skip to
content}\protect\hyperlink{site-index}{Skip to site index}

\href{https://www.nytimes3xbfgragh.onion/section/politics}{Politics}

\href{https://myaccount.nytimes3xbfgragh.onion/auth/login?response_type=cookie\&client_id=vi}{}

\href{https://www.nytimes3xbfgragh.onion/section/todayspaper}{Today's
Paper}

\href{/section/politics}{Politics}\textbar{}Republicans and White House
at Odds Over Kansas Senate Race

\url{https://nyti.ms/3hTvcgU}

\begin{itemize}
\item
\item
\item
\item
\item
\item
\end{itemize}

Advertisement

\protect\hyperlink{after-top}{Continue reading the main story}

Supported by

\protect\hyperlink{after-sponsor}{Continue reading the main story}

\hypertarget{republicans-and-white-house-at-odds-over-kansas-senate-race}{%
\section{Republicans and White House at Odds Over Kansas Senate
Race}\label{republicans-and-white-house-at-odds-over-kansas-senate-race}}

Some in the G.O.P. want President Trump to endorse the opponent of Kris
Kobach, who they worry could cost them a traditionally safe Senate seat.
So far, the White House has declined to do so.

\includegraphics{https://static01.graylady3jvrrxbe.onion/images/2020/07/30/us/politics/30kansas1/merlin_171905724_8041c89b-51ec-4084-8e6d-21bd6273a72f-articleLarge.jpg?quality=75\&auto=webp\&disable=upscale}

\href{https://www.nytimes3xbfgragh.onion/by/jonathan-martin}{\includegraphics{https://static01.graylady3jvrrxbe.onion/images/2018/11/06/multimedia/author-jonathan-martin/author-jonathan-martin-thumbLarge.png}}\href{https://www.nytimes3xbfgragh.onion/by/katie-glueck}{\includegraphics{https://static01.graylady3jvrrxbe.onion/images/2020/01/29/reader-center/author-katie-glueck/author-katie-glueck-thumbLarge.png}}

By \href{https://www.nytimes3xbfgragh.onion/by/jonathan-martin}{Jonathan
Martin} and
\href{https://www.nytimes3xbfgragh.onion/by/katie-glueck}{Katie Glueck}

\begin{itemize}
\item
  July 30, 2020
\item
  \begin{itemize}
  \item
  \item
  \item
  \item
  \item
  \item
  \end{itemize}
\end{itemize}

WASHINGTON --- As the
\href{https://www.nytimes3xbfgragh.onion/2020/08/04/us/elections/primary-election-michigan-arizona-kansas.html}{Kansas
Senate primary} barrels to a close, tensions are rising between Senate
Republicans and the White House over the potential nomination of Kris
Kobach, who party officials fear would jeopardize the seat and further
imperil their Senate majority.

Senator Mitch McConnell is worried that Mr. Kobach, the controversial
former Kansas secretary of state who
\href{https://www.nytimes3xbfgragh.onion/2018/11/06/us/laura-kelly-wins-kansas-governors-race.html}{lost
the 2018 governor's race}, may win the nomination in Tuesday's primary,
only to lose the seat in November --- and he is frustrated that
President Trump is not intervening in the race, according to multiple
G.O.P. officials.

Mr. McConnell and other Senate Republican leaders have made urgent pleas
to the president to block Mr. Kobach by endorsing one of his opponents,
Representative Roger Marshall. But Mr. Trump has so far declined to do
so, and his aides said they had no plans to change course. Compounding
the frustration of Capitol Hill Republicans, White House aides have
refused to tell Mr. Kobach, a longtime booster of Mr. Trump, to stop
using the president's imagery in his campaign materials.

With a number of incumbent Senate Republicans trailing in polls, and
being out-raised by their Democratic rivals, they have little margin for
error as they seek to protect their 53-47 majority. And because of Mr.
Trump's broad unpopularity, and a health crisis
\href{https://www.nytimes3xbfgragh.onion/2020/07/30/business/economy/q2-gdp-coronavirus-economy.html?action=click\&module=Top\%20Stories\&pgtype=Homepage}{that
has devastated the economy}, even a deeply conservative state like
Kansas, which has not sent a Democrat to the Senate since the 1930s, is
no sure thing for Senate Republicans this year.

``We have eight months of data that says the majority is gone if Kris
Kobach is the nominee,'' said Josh Holmes, a top lieutenant to Mr.
McConnell. ``It's that simple.''

Mr. Trump's reluctance to wade into the race illustrates his growing
anxiety about his conservative base, the core of which is supporting Mr.
Kobach in Kansas. The president has recently sought to shore up his
standing on the right by taking a series of positions, particularly on
race and protests, aimed at solidifying Republican voters who have
drifted from him over his ineffective response to the coronavirus
outbreak.

On a related note, Mr. Trump, even as he sinks in general election
polling, is proud of his win-loss record in those Republican primaries
where he has endorsed candidates, and he is skittish about being seen as
having diminished clout within the G.O.P.

Mr. Kobach has long been an incendiary figure in Kansas politics,
associated with hard-line views on immigration, voting rights and a host
of other issues. He is especially unpopular in the Kansas City suburbs,
home to traditionally moderate Republicans who have moved away from the
party in the Trump era. Republicans in Kansas and in Washington remain
frustrated that he captured the nomination for governor two years ago,
which they feel cost them the seat.

Senate Republicans have long been concerned about Mr. Kobach's
candidacy, and for months they sought to woo Secretary of State Mike
Pompeo, a former Kansas congressman, into the race. They have grown even
more uneasy in recent days, though, after reviewing the results of
Senate Republican polling: The surveys showed Mr. Trump leading only
narrowly in the state and found that nearly 30 percent of Republican
primary voters indicated they would support the Democrat in the Senate
race, state Senator Barbara Bollier, if Mr. Kobach were the nominee,
according to two Republicans familiar with the data.

Mr. Trump has expressed frustration that he endorsed Mr. Kobach's bid
for governor two years ago only to watch him lose, and many
congressional Republicans believed the president would try to halt Mr.
Kobach's candidacy this year.

But party officials became gravely alarmed Thursday after getting word
that the president was not inclined to support Mr. Marshall.

According to two people familiar with the conversation, Senator Ted Cruz
of Texas used an Air Force One flight with the president on Wednesday to
steer Mr. Trump away from supporting Mr. Marshall. Mr. Cruz told Mr.
Trump that Mr. Marshall had supported former Ohio Gov. John Kasich, now
a vocal Trump critic, in the 2016 primary. Mr. Cruz, who himself ran
against Mr. Trump, has sought to lift some anti-establishment candidates
and his top political adviser is working for another candidate in the
race.

Representatives for Mr. Cruz declined to comment.

The push and pull between Mr. McConnell and Mr. Cruz reflects the degree
to which Mr. Trump is increasingly being used by competing G.O.P.
factions, who recognize that he's a useful ally in intraparty fights and
that the way to appeal to him is to play to his insecurities.

Establishment-aligned Republicans, however, are especially nervous about
how the race is shaping up, and about Mr. Trump's role in it.

``I don't understand it,'' Robert Blizzard, a Republican strategist and
a pollster for Mr. Marshall, said, asked about Mr. Trump's perch on the
sidelines so far. ``Given how tight things look in terms of holding the
majority, why we wouldn't be all in on trying to make sure we have a
chance of winning, holding this Kansas Senate seat. It's concerning for
sure.''

David Kensinger, a veteran Kansas Republican strategist, noted that Mr.
Trump's 2018 endorsement of Mr. Kobach had not been forgotten.

``Kris Kobach's career would have been over two years ago but for the
Trump endorsement --- this is the opportunity for the president to put
it right,'' he said, still hoping for a Marshall endorsement.

But asked if he had seen any indication that Mr. Trump would do so, he
replied, ``I have not,'' a view shared by another Republican strategist
involved in the race.

\includegraphics{https://static01.graylady3jvrrxbe.onion/images/2020/07/30/us/politics/30kansas2/merlin_168229068_5b189dbd-6a99-4ab6-a169-8b621e67323b-articleLarge.jpg?quality=75\&auto=webp\&disable=upscale}

Mr. Kobach and Mr. Marshall are brawling in a brutal, crowded contest
that has been defined by negative ads and mailers. Particularly
worrisome to Senate Republicans, an outside group, which
\href{https://www.kansascity.com/news/politics-government/article244514962.html}{appears
to be linked} to Democrats, has also been advertising heavily and has
attacked Mr. Marshall in an effort to elevate Mr. Kobach.

In a statement, Mr. Kobach said that data showed that he would also be
competitive against Ms. Bollier.

``They know that their claimed reason for supporting Marshall is a false
one,'' he said of what he called the Republican establishment. ``The
real reason is that they want a yes man in the Senate --- not a
principled conservative.''

Certainly, a statewide race in Kansas remains a challenge for any
Democrat, regardless of opponent, in a state where many religious voters
prioritize issues like abortion and the courts, and some moderate voters
are leery of the prospect of a Democratic-controlled Senate, whatever
their feelings about Mr. Trump, who is still expected to win the state.

And there are plenty of party leaders who are hoping that in the end,
Mr. Marshall will prevail. The crowded, negative nature of the
Republican primary has made the race fluid and unpredictable, and Mr.
Marshall has support from a stable of prominent, deep-pocketed groups
and influential figures including former Senator
\href{https://www.kansascity.com/news/politics-government/article239180118.html}{Bob
Dole}, the Chamber of Commerce and a host of other conservative
organizations.

``Next week is a huge week for the Republicans' chance to hold onto the
Senate,'' said Scott W. Reed, the senior political strategist at the
U.S. Chamber of Commerce, pointing to the race in Kansas as well as one
in Tennessee. ``Nominating a stone-cold loser in Kansas will guarantee
we lose in the fall.''

The winner of the primary is expected to face Ms. Bollier, a retired
anesthesiologist who was
\href{https://www.kansas.com/news/politics-government/article222990385.html}{until
recently a Republican} and did not have a significant primary battle of
her own.

The chairman of the Kansas Republican Party, Mike Kuckelman, said he did
not fault Mr. Trump for not intervening, but he said that the president
``probably would have affected the race had he endorsed someone.''

``Maybe it would have made the race a little easier,'' he said.

Advertisement

\protect\hyperlink{after-bottom}{Continue reading the main story}

\hypertarget{site-index}{%
\subsection{Site Index}\label{site-index}}

\hypertarget{site-information-navigation}{%
\subsection{Site Information
Navigation}\label{site-information-navigation}}

\begin{itemize}
\tightlist
\item
  \href{https://help.nytimes3xbfgragh.onion/hc/en-us/articles/115014792127-Copyright-notice}{©~2020~The
  New York Times Company}
\end{itemize}

\begin{itemize}
\tightlist
\item
  \href{https://www.nytco.com/}{NYTCo}
\item
  \href{https://help.nytimes3xbfgragh.onion/hc/en-us/articles/115015385887-Contact-Us}{Contact
  Us}
\item
  \href{https://www.nytco.com/careers/}{Work with us}
\item
  \href{https://nytmediakit.com/}{Advertise}
\item
  \href{http://www.tbrandstudio.com/}{T Brand Studio}
\item
  \href{https://www.nytimes3xbfgragh.onion/privacy/cookie-policy\#how-do-i-manage-trackers}{Your
  Ad Choices}
\item
  \href{https://www.nytimes3xbfgragh.onion/privacy}{Privacy}
\item
  \href{https://help.nytimes3xbfgragh.onion/hc/en-us/articles/115014893428-Terms-of-service}{Terms
  of Service}
\item
  \href{https://help.nytimes3xbfgragh.onion/hc/en-us/articles/115014893968-Terms-of-sale}{Terms
  of Sale}
\item
  \href{https://spiderbites.nytimes3xbfgragh.onion}{Site Map}
\item
  \href{https://help.nytimes3xbfgragh.onion/hc/en-us}{Help}
\item
  \href{https://www.nytimes3xbfgragh.onion/subscription?campaignId=37WXW}{Subscriptions}
\end{itemize}
