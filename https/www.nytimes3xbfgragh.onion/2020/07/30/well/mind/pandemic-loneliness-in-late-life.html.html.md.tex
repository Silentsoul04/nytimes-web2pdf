Sections

SEARCH

\protect\hyperlink{site-content}{Skip to
content}\protect\hyperlink{site-index}{Skip to site index}

\href{https://www.nytimes3xbfgragh.onion/section/well/mind}{Mind}

\href{https://myaccount.nytimes3xbfgragh.onion/auth/login?response_type=cookie\&client_id=vi}{}

\href{https://www.nytimes3xbfgragh.onion/section/todayspaper}{Today's
Paper}

\href{/section/well/mind}{Mind}\textbar{}Pandemic Loneliness in Late
Life

\url{https://nyti.ms/2X7URdu}

\begin{itemize}
\item
\item
\item
\item
\item
\item
\end{itemize}

\href{https://www.nytimes3xbfgragh.onion/spotlight/at-home?action=click\&pgtype=Article\&state=default\&region=TOP_BANNER\&context=at_home_menu}{At
Home}

\begin{itemize}
\tightlist
\item
  \href{https://www.nytimes3xbfgragh.onion/2020/07/28/books/time-for-a-literary-road-trip.html?action=click\&pgtype=Article\&state=default\&region=TOP_BANNER\&context=at_home_menu}{Take:
  A Literary Road Trip}
\item
  \href{https://www.nytimes3xbfgragh.onion/2020/07/29/magazine/bored-with-your-home-cooking-some-smoky-eggplant-will-fix-that.html?action=click\&pgtype=Article\&state=default\&region=TOP_BANNER\&context=at_home_menu}{Cook:
  Smoky Eggplant}
\item
  \href{https://www.nytimes3xbfgragh.onion/2020/07/27/travel/moose-michigan-isle-royale.html?action=click\&pgtype=Article\&state=default\&region=TOP_BANNER\&context=at_home_menu}{Look
  Out: For Moose}
\item
  \href{https://www.nytimes3xbfgragh.onion/interactive/2020/at-home/even-more-reporters-editors-diaries-lists-recommendations.html?action=click\&pgtype=Article\&state=default\&region=TOP_BANNER\&context=at_home_menu}{Explore:
  Reporters' Obsessions}
\end{itemize}

Advertisement

\protect\hyperlink{after-top}{Continue reading the main story}

Supported by

\protect\hyperlink{after-sponsor}{Continue reading the main story}

Living with Cancer

\hypertarget{pandemic-loneliness-in-late-life}{%
\section{Pandemic Loneliness in Late
Life}\label{pandemic-loneliness-in-late-life}}

Many seniors comply with the physical distancing and stay-at-home
orders, even as we understand that social isolation generates the lethal
byproducts of loneliness.

\includegraphics{https://static01.graylady3jvrrxbe.onion/images/2020/07/22/multimedia/00well-loneliness/00well-loneliness-articleLarge.jpg?quality=75\&auto=webp\&disable=upscale}

By Susan Gubar

\begin{itemize}
\item
  July 30, 2020
\item
  \begin{itemize}
  \item
  \item
  \item
  \item
  \item
  \item
  \end{itemize}
\end{itemize}

By the end of June, when my husband landed in the hospital, its
administrators had begun allowing one visitor per patient, much to my
relief. At 92, Don had fallen and fractured a hip. He would need an
operation and then rehab in a facility that responded to the ongoing
coronavirus by extending its ban on all visitors. This last prospect
filled me with dread.

A month earlier, Don's sister, Mary, had fallen in Chicago and her four
daughters --- scattered across the States --- promptly got into their
cars to help her through the subsequent ordeal. But they spent their
visits outside a window, displaying the baked goods and flowers they
would deliver to a receptionist, miming acts of affection.

The pandemic has made isolates of the elderly. We have all read ghastly
stories about
\href{https://www.cnn.com/2020/03/29/world/funerals-dying-alone-coronavirus/index.html}{coronavirus
patients dying alone}. (I can barely wrap my mind around such a forlorn
fate.) Aware that the Covid-19
\href{https://www.statnews.com/2020/03/30/what-explains-coronavirus-lethality-for-elderly/}{mortality
rate among the elderly} is much higher than that of the young, many
seniors comply with the physical distancing and stay-at-home orders,
even as we understand that social isolation generates the lethal
byproducts of loneliness --- depression, food and sleep disorders,
anxiety, substance abuse, self-harm --- particularly for those who do
not have family, friends or neighbors to help with grocery and pharmacy
runs as well as daily communication.

Several years ago, when Don had gone through a stint in rehab for a torn
knee tendon, I understood the importance of daily contact. The second
bed in the room was empty and I could stay from 8 a.m. till 8 p.m. If he
was in pain, I could run to the nurses' station; if he needed a warm
sweater, I could fetch one. Most of the time, I simply commiserated,
knitted and \emph{kibitzed} to bolster his spirits.

After that accident, we had moved to a ground-floor apartment so neither
of us would have to climb stairs. Yet even so, in his current
immobilized state, Don would not be safe at home. I therefore spent the
three days we had together in the hospital scheming to alleviate the
misery of our pending two- or three-week separation.

Virtual visits were one obvious solution; however, they pose a problem
for older people who may be technologically challenged. Don did own an
iPad; however, he had no idea how to FaceTime. We practiced in his
hospital room. There would be a landline next to his bed in rehab. Would
these two ways of chatting see us through?

It turned out that they provided anemic substitutes for the intimacy I
longed for. As is often the case in times of misfortune, troubles piled
up. When Don entered rehab, I suffered a bowel obstruction (resulting
from umpteen abdominal operations for ovarian cancer). As I fasted and
hydrated and focused on staying out of the hospital, I was reassured by
seeing his face or hearing his voice, but both phone and FaceTime
sessions precluded the tactile comfort of touch while accentuating the
distance between us. I could see only his head --- if he managed to hold
the screen up properly --- not his surroundings; I could hear his voice
--- not see his body. Tethered to our separate devices, we could express
but not assuage the helplessness we felt at not being able to solace
each other.

In
``\href{https://www.nytimes3xbfgragh.onion/2016/03/20/books/review/the-lonely-city-by-olivia-laing.html}{The
Lonely City,}'' ** a study of visual artists from Edward Hopper to Andy
Warhol, Olivia Laing characterizes loneliness as ``a state of lack,'' in
which one urgently \emph{needs} to be loved, touched, held. This feeling
of deficiency can be accompanied by ``hypervigilance'' about social
threats --- especially in those who feel shamed and unsafe because they
are stigmatized for their looks or disabilities, their race or religion
or sexual orientation. If lonely people arm themselves with expectations
of rudeness and rejection from what they perceive to be an increasingly
negative world, they grow more isolated, suspicious and withdrawn.
Accretive, ``loneliness grows \ldots{} like a mould or fur, a
prophylactic that inhibits contact, no matter how badly contact is
desired.''

Before the epidemic, of course, many seniors found that physical or
cognitive problems, hearing or vision losses could make them feel shamed
or unsafe and thus contributed to their isolation. In late life, Don and
I certainly realized that apprehension about leaving home --- because
getting out into the world is difficult and risks accidents ---
inhibited our socializing.

For many elderly people, it seems that
\href{https://ajp.psychiatryonline.org/doi/10.1176/appi.ajp.2013.12091235}{late-onset
agoraphobia} has been aggravated by the coronavirus, which makes the
social world an even less hospitable place requiring hypervigilance
about face-to-face interactions. Quarantined for some five months, many
oldsters feel more isolated, suspicious and withdrawn or, worse, in
danger of vanishing. Fear of contagion has turned us into prisoners
under house arrest.

About one week into my separation from Don, Lester Holt (one of my
personal heroes) had a story on the nightly news about a woman who
managed to hug her husband in an Alzheimer's facility by
\href{https://www.facebookcorewwwi.onion/watch/?v=669777473751880}{signing
up to become a dishwasher} in it. ``Love does always find a way,'' Mr.
Holt said. I wept at my inadequacy, for I could not find a way. In the
midst of this meltdown, I doubted that Don and I would ever see our
grandchildren in person again.

At less miserable moments, I wonder if widespread isolation will help
more of us cultivate empathy for those, old and young, whose loneliness
derives from irrational and alienating stigmas. But rattling around the
empty apartment, I discover that I cannot do what Don and I used to do
together: listen to music, eat or sleep. My condition is the obverse of
the solitude I have cultivated and prized throughout my adult life.

Only my study serves as a refuge as I sit at my laptop --- the landline
and cellphone close at hand --- with a bitter foretaste of bereavement
in my mouth. We do what we can to stay united apart, alone together.

\emph{Susan Gubar, who has been dealing with ovarian cancer since 2008,
is distinguished emerita professor of English at Indiana University. Her
latest book is
``}\href{https://www.nytimes3xbfgragh.onion/2018/12/28/books/review/susan-gubar-late-life-love.html?module=inline}{\emph{Late-Life
Love}}\emph{.''}

Advertisement

\protect\hyperlink{after-bottom}{Continue reading the main story}

\hypertarget{site-index}{%
\subsection{Site Index}\label{site-index}}

\hypertarget{site-information-navigation}{%
\subsection{Site Information
Navigation}\label{site-information-navigation}}

\begin{itemize}
\tightlist
\item
  \href{https://help.nytimes3xbfgragh.onion/hc/en-us/articles/115014792127-Copyright-notice}{©~2020~The
  New York Times Company}
\end{itemize}

\begin{itemize}
\tightlist
\item
  \href{https://www.nytco.com/}{NYTCo}
\item
  \href{https://help.nytimes3xbfgragh.onion/hc/en-us/articles/115015385887-Contact-Us}{Contact
  Us}
\item
  \href{https://www.nytco.com/careers/}{Work with us}
\item
  \href{https://nytmediakit.com/}{Advertise}
\item
  \href{http://www.tbrandstudio.com/}{T Brand Studio}
\item
  \href{https://www.nytimes3xbfgragh.onion/privacy/cookie-policy\#how-do-i-manage-trackers}{Your
  Ad Choices}
\item
  \href{https://www.nytimes3xbfgragh.onion/privacy}{Privacy}
\item
  \href{https://help.nytimes3xbfgragh.onion/hc/en-us/articles/115014893428-Terms-of-service}{Terms
  of Service}
\item
  \href{https://help.nytimes3xbfgragh.onion/hc/en-us/articles/115014893968-Terms-of-sale}{Terms
  of Sale}
\item
  \href{https://spiderbites.nytimes3xbfgragh.onion}{Site Map}
\item
  \href{https://help.nytimes3xbfgragh.onion/hc/en-us}{Help}
\item
  \href{https://www.nytimes3xbfgragh.onion/subscription?campaignId=37WXW}{Subscriptions}
\end{itemize}
