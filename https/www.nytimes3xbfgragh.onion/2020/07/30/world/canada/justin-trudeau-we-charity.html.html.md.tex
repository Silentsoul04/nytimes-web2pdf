Sections

SEARCH

\protect\hyperlink{site-content}{Skip to
content}\protect\hyperlink{site-index}{Skip to site index}

\href{https://www.nytimes3xbfgragh.onion/section/world/canada}{Canada}

\href{https://myaccount.nytimes3xbfgragh.onion/auth/login?response_type=cookie\&client_id=vi}{}

\href{https://www.nytimes3xbfgragh.onion/section/todayspaper}{Today's
Paper}

\href{/section/world/canada}{Canada}\textbar{}A Cool Trudeau Rebuffs
Conflict Charges. Will Canadians Buy His Story?

\url{https://nyti.ms/30ZQTox}

\begin{itemize}
\item
\item
\item
\item
\item
\end{itemize}

Advertisement

\protect\hyperlink{after-top}{Continue reading the main story}

Supported by

\protect\hyperlink{after-sponsor}{Continue reading the main story}

\hypertarget{a-cool-trudeau-rebuffs-conflict-charges-will-canadians-buy-his-story}{%
\section{A Cool Trudeau Rebuffs Conflict Charges. Will Canadians Buy His
Story?}\label{a-cool-trudeau-rebuffs-conflict-charges-will-canadians-buy-his-story}}

There was no corruption, just a government working to save lives during
a pandemic. That's the message the prime minister delivered to skeptical
lawmakers, and to Canadians.

\includegraphics{https://static01.graylady3jvrrxbe.onion/images/2020/07/30/world/30trudeau4/merlin_175124229_55283d52-744e-4720-8333-ab6448f70f57-articleLarge.jpg?quality=75\&auto=webp\&disable=upscale}

By
\href{https://www.nytimes3xbfgragh.onion/by/catherine-porter}{Catherine
Porter}

\begin{itemize}
\item
  July 30, 2020
\item
  \begin{itemize}
  \item
  \item
  \item
  \item
  \item
  \end{itemize}
\end{itemize}

TORONTO --- There was no corruption, no conflict of interest, just a
government and prime minister working around the clock to save lives and
livelihoods during a pandemic.

That was Justin Trudeau's message on Thursday to Canadians, who for the
past month have lifted their heads from the exhausting struggle with the
coronavirus to watch a growing political scandal over his government's
decision to award a hefty no-bid contract to a charity that has ties
with his family.

``Nothing of this program was in any way going to benefit any members of
my family,'' the prime minister said during a rare virtual appearance
before the standing finance committee of the Canadian Parliament. ``I
was not in a conflict of interest.''

The highly anticipated session, which stretched for 90 minutes, had
elements of drama --- heated exchanges, baiting questions and even a
power outage during a storm over the chairman's home.

But Mr. Trudeau predictably kept his cool, displaying his well-honed
political talents.

The contract, made to the WE Charity, was to oversee hundreds of
millions of government dollars for an emergency summer youth volunteer
program. Mr. Trudeau defended the decision to give the charity the
contract by presenting himself as a longtime champion of youths.

And he said he was following the advice of public servants to start a
program that, like others his government has set up, would help
thousands of people across the country.

``We moved quickly to try to get help out to people as fast as we could
as flexibly as we could,'' Mr. Trudeau said.

The question is: Was his performance enough to persuade Canadians that
he did nothing wrong and put to rest thorny ethical questions that have
set the media ablaze for the past month and dragged down his party's
polling numbers?

The WE Charity is tied not only to Mr. Trudeau's family, but also to his
finance minister, Bill Morneau.

Mr. Trudeau's mother and brother earned more than \$200,000 over the
past five years for speaking engagements with the charity. Mr. Morneau's
daughter works there, and his family has traveled overseas with the
charity twice in recent years.

Both Mr. Trudeau and Mr. Morneau have
\href{https://www.nytimes3xbfgragh.onion/2020/07/13/world/canada/trudeau-we-charity-apology.html}{apologized}
for not recusing themselves from the cabinet decision. Both are under
investigation by the country's ethics commissioner.

But during Thursday's testimony, Mr. Trudeau reinforced what others have
said before him --- that officials of Canada's apolitical public service
chose the charity and that his cabinet was given a ``binary choice'' of
either agreeing to the plan, or having to abandon it altogether.

The perception of a conflict with his family, he added, caused him to
put the decision on hold for two weeks and push the public service to
``make sure that everything was done exactly right, because I knew there
would be questions asked.''

\includegraphics{https://static01.graylady3jvrrxbe.onion/images/2020/07/30/world/30trudeau/merlin_174844728_caf57cf5-f0be-4ede-8b3e-a92e553395dc-articleLarge.jpg?quality=75\&auto=webp\&disable=upscale}

Lori Turnbull, director of the school of public administration at
Dalhousie University in Halifax was among the rapt viewers who blocked
off hours to see the unfolding testimonies.

``He got his message out --- you want the Liberals to be there to take
care of you when a crisis like this hits,'' she said, referring to Mr.
Trudeau's party. ``To me, the prime minister left this meeting as a
youth champion who made a little error.''

Still, the prime minister has already been found in breach of the
country's conflict-of-interest rules twice since coming to office five
years ago. That he acknowledged a perceived conflict but did not recuse
himself raises troubling questions about his political judgment, said
\href{https://twitter.com/ShachiKurl?ref_src=twsrc\%5Egoogle\%7Ctwcamp\%5Eserp\%7Ctwgr\%5Eauthor}{Shachi
Kurl,} the executive director of the Angus Reid Institute, a Canadian
nonprofit polling firm based in Vancouver.

``In some ways, I think he did very well,'' Ms. Kurl said. ``But it
doesn't fix other problems. Now, we have to assess to what extent these
ongoing examples of the prime minister not having good judgment will
become a ballot question.''

The story has unfolded since late June, when the Trudeau government
announced that it had awarded the job of administering the summer
program, and its budget of 912 million Canadian dollars, to the WE
Charity.

Two brothers, Craig and Marc Kielburger, founded the charity as
teenagers. Since then, it has grown into a network of organizations that
have built schools and
\href{https://www.metowe.com/2016/09/09/power-clean-water/}{wells} in
countries like Kenya and Nicaragua.

But it is best known for inspiring young Canadians to get involved in
social justice issues through school programs and huge concert-like
events featuring motivational speakers, including Prince Harry, Malala
Yousafzai, Mr. Trudeau and his wife, Sophie Grégoire Trudeau.

Image

Fans lining the red carpet at WE Day Toronto in
2018.Credit...Christopher Katsarov/CP, via Associated Press

The charity says the Trudeaus volunteered their time and were never
paid, with one exception in 2012, when Ms. Gregoire Trudeau received
about \$1,000 for a speaking engagement. Mr. Trudeau was not prime
minister at the time.

Since then, Ms. Gregoire Trudeau has become an ``ambassador'' of the
charity, speaking regularly at events and hosting a podcast called WE
Well-being. Mr. Trudeau said Thursday that the country's ethics
commissioner had vetted these roles.

But the WE Charity
\href{https://www.nytimes3xbfgragh.onion/2020/07/09/world/canada/trudeau-ethics-charity.html}{did
pay Mr. Trudeau's brother Alexandre, a filmmaker, and mother, Margaret,}
the country's former first lady, for speaking over the past four years
on behalf of the charity at various events. The Kielburger brothers said
attracting sponsors for these events was their main way of raising
money.

On Thursday, Mr. Trudeau said he did not have a personal relationship
with the Kielburger brothers and had no contact with them about this
contract.

In recent days, witnesses before the committee bolstered what Mr.
Trudeau said Thursday --- that he was not involved in selecting the
charity to administer the program, which involved overseeing up to
100,000 students volunteering for public service jobs.

WE was to have received as much as 43.5 million Canadian dollars to run
the program, according to documents released this week. But after the
controversy erupted, the government announced it was taking the program
back, and the Kielburgers said they would return all the money.

The program is still on hold.

Hours before Mr. Morneau, the finance minister, was set to testify last
week before the committee, he announced he had written a check to the
charity for 41,366 Canadian dollars, stating that he hadn't realized he
and his family had not paid the full fare for the two trips they took
with the organization.

He also disclosed that his family had donated 100,000 Canadian dollars
to the charity in recent years.

``You don't donate \$100,000, take trips with a charity and have a
daughter who works there and not know there's a conflict of interest,''
said Duff Conacher, the co-founder of a nonprofit watchdog organization,
Democracy Watch, which is calling for a criminal investigation.

``This is a story about how every government has friends, those friends
help them get elected and promote them, and then the government wants to
help those friends because they will promote and boost them more,'' he
said.

The committee has not heard any evidence that Mr. Trudeau stood to
financially benefit from the contract, several political experts pointed
out. If there has been any damage, it has been in optics.

``Technically speaking, he didn't need to recuse himself, in my
opinion,'' said Ian Stedman, a professor of Canadian public law and
governance at York University in Toronto, and an expert on Canadian
parliamentary ethics law.

But, he added: ``But pragmatically speaking, what he did was bad
politics. He allowed himself to be the story, instead of his policies.''

Canadians have been generally happy with the Trudeau government's
handling of the coronavirus epidemic, which has leveled off across the
country, allowing hairdressers and restaurants in most places to reopen
and plans for school openings to be drafted.

But in recent weeks, since news about the charity has continually made
the front pages, Mr. Trudeau's approval rating has fallen. It is unclear
whether that is a blip or the start of a trend.

``People have concerns about this, but they have so many other issues
right now,'' said Jean-Marc Léger, the chief executive of the Léger
polling firm based in Montreal. ``They are worried about the pandemic,
the economic crisis, relations with the United States.''

The Kielburger brothers maintain they would have made no profit from the
contract.

Image

Craig, left, and Marc Kielburger at WE Day Toronto last
year.Credit...Jeremy Chan/Getty Images

Because of the intense scrutiny over the past few weeks,
\href{https://www.theglobeandmail.com/world/article-royal-bank-cuts-ties-with-embattled-we-charity/}{many
big sponsors have cut ties with WE} --- the organization has called the
decisions mutual --- and former staff members have poured out stories on
social media about the charity's ``culture of fear.''

Craig Kielburger said the unfolding controversy might ``destroy'' the
organization.

``Frankly, there are days that we wish that we never answered the
phone,'' when the government called asking them ``to help,'' he told the
committee near the grueling end of the brothers' four-hour appearance
earlier this week.

Dan Bilefsky contributed reporting from Montreal.

Advertisement

\protect\hyperlink{after-bottom}{Continue reading the main story}

\hypertarget{site-index}{%
\subsection{Site Index}\label{site-index}}

\hypertarget{site-information-navigation}{%
\subsection{Site Information
Navigation}\label{site-information-navigation}}

\begin{itemize}
\tightlist
\item
  \href{https://help.nytimes3xbfgragh.onion/hc/en-us/articles/115014792127-Copyright-notice}{©~2020~The
  New York Times Company}
\end{itemize}

\begin{itemize}
\tightlist
\item
  \href{https://www.nytco.com/}{NYTCo}
\item
  \href{https://help.nytimes3xbfgragh.onion/hc/en-us/articles/115015385887-Contact-Us}{Contact
  Us}
\item
  \href{https://www.nytco.com/careers/}{Work with us}
\item
  \href{https://nytmediakit.com/}{Advertise}
\item
  \href{http://www.tbrandstudio.com/}{T Brand Studio}
\item
  \href{https://www.nytimes3xbfgragh.onion/privacy/cookie-policy\#how-do-i-manage-trackers}{Your
  Ad Choices}
\item
  \href{https://www.nytimes3xbfgragh.onion/privacy}{Privacy}
\item
  \href{https://help.nytimes3xbfgragh.onion/hc/en-us/articles/115014893428-Terms-of-service}{Terms
  of Service}
\item
  \href{https://help.nytimes3xbfgragh.onion/hc/en-us/articles/115014893968-Terms-of-sale}{Terms
  of Sale}
\item
  \href{https://spiderbites.nytimes3xbfgragh.onion}{Site Map}
\item
  \href{https://help.nytimes3xbfgragh.onion/hc/en-us}{Help}
\item
  \href{https://www.nytimes3xbfgragh.onion/subscription?campaignId=37WXW}{Subscriptions}
\end{itemize}
