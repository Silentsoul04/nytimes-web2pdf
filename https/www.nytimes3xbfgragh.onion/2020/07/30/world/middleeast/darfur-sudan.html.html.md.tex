Sections

SEARCH

\protect\hyperlink{site-content}{Skip to
content}\protect\hyperlink{site-index}{Skip to site index}

\href{https://www.nytimes3xbfgragh.onion/section/world/middleeast}{Middle
East}

\href{https://myaccount.nytimes3xbfgragh.onion/auth/login?response_type=cookie\&client_id=vi}{}

\href{https://www.nytimes3xbfgragh.onion/section/todayspaper}{Today's
Paper}

\href{/section/world/middleeast}{Middle East}\textbar{}The Dictator Who
Waged War on Darfur Is Gone, but the Killing Goes On

\url{https://nyti.ms/3hQnzHO}

\begin{itemize}
\item
\item
\item
\item
\item
\item
\end{itemize}

Advertisement

\protect\hyperlink{after-top}{Continue reading the main story}

Supported by

\protect\hyperlink{after-sponsor}{Continue reading the main story}

\hypertarget{the-dictator-who-waged-war-on-darfur-is-gone-but-the-killing-goes-on}{%
\section{The Dictator Who Waged War on Darfur Is Gone, but the Killing
Goes
On}\label{the-dictator-who-waged-war-on-darfur-is-gone-but-the-killing-goes-on}}

The ouster of Sudan's longtime ruler last year stoked hopes for peace in
brutalized Darfur. But violence has surged amid fresh accusations
against Sudan's security forces.

\includegraphics{https://static01.graylady3jvrrxbe.onion/images/2020/07/28/world/28darfur/merlin_174229890_2075153e-3d25-422a-b09e-9fd8a5165e02-articleLarge.jpg?quality=75\&auto=webp\&disable=upscale}

\href{https://www.nytimes3xbfgragh.onion/by/declan-walsh}{\includegraphics{https://static01.graylady3jvrrxbe.onion/images/2018/10/15/multimedia/author-declan-walsh/author-declan-walsh-thumbLarge-v3.png}}

By \href{https://www.nytimes3xbfgragh.onion/by/declan-walsh}{Declan
Walsh}

\begin{itemize}
\item
  July 30, 2020
\item
  \begin{itemize}
  \item
  \item
  \item
  \item
  \item
  \item
  \end{itemize}
\end{itemize}

CAIRO --- On camels, horses and motorbikes, dozens of Arab militiamen
stormed into the remote village in Darfur, in western Sudan, firing
wildly, witnesses said. Houses were pillaged, animals stolen and water
tanks smashed. Villagers ran for their lives.

United Nations peacekeepers scrambled to the scene but said they found
the road blocked by obstacles placed in their way, and continued on
foot. When they arrived after two and a half hours, it was too late.

At least nine people lay dead, including a 15-year-old boy, and another
20 were seriously wounded, according to the United Nations.

The attack in Fata Bornu, a remote hamlet of 4,000 people, echoes the
\href{https://www.nytimes3xbfgragh.onion/2005/02/23/opinion/the-secret-genocide-archive.html}{grimmest
days} of the Darfur conflict in the 2000s. But it happened just this
month --- over a year since
\href{https://www.nytimes3xbfgragh.onion/2019/04/23/world/africa/sudan-protesters-al-bashir-abdelkhalig-.html}{euphoric
protests} toppled Omar Hassan al-Bashir, the detested dictator whose
alleged atrocities in Darfur earned him an indictment on genocide
charges in an international court.

Now Mr. al-Bashir languishes in jail and Sudan is run by a joint
civilian-military government that has promised to usher in a new era of
democracy, and finally --- after 17 years of suffering --- bring peace
to Darfur.

\includegraphics{https://static01.graylady3jvrrxbe.onion/images/2020/07/28/world/28darfur7/merlin_174795249_60496b4f-9d6f-42bb-b1e1-506cb1d70a0f-articleLarge.jpg?quality=75\&auto=webp\&disable=upscale}

But while the revolution brought
\href{https://www.nytimes3xbfgragh.onion/2020/07/13/world/africa/sudan-fgm-alcohol-flogging.html}{some
change} to Sudan's cities, that is not the case in Darfur, where the
notorious janjaweed --- nomadic Arab militias --- still ride free.
Heavily armed gangs continue to massacre, plunder and rape in
scorched-earth tactics that recall the worst days of Mr. al-Bashir's
rule.

And the Sudanese military has faced accusations of negligence, as in
Fata Bornu, or even being complicit.

``They stand silent in front of the janjaweed attacks,'' said Adam
Mohamed, a senior leader of Darfur's displaced community, in a telephone
interview. ``They do nothing.''

The tempo of violence quickened in recent days with another 60 civilians
massacred on Saturday in an assault by about 500 Arab militiamen,
according to the United Nations --- the deadliest attack in months. Two
days earlier, 15 people were killed in a different part of Darfur, local
news media
\href{https://www.dabangasudan.org/en/all-news/article/attack-on-south-darfur-village-leaves-15-dead}{reported}.

After Saturday's massacre, in Masteri village in West Darfur state,
Sudan's civilian prime minister, Abdalla Hamdok, promised to send the
police and army reinforcements ``to protect citizens and the farming
season.''

300 MILES

EGYPT

Red

Sea

Nile

CHAD

DARFUR

SUDAN

ERITREA

Fata

Bornu

Khartoum

Masteri

ETHIOPIA

CENTRAL

AFRICAN

REP.

SOUTH

SUDAN

By The New York Times

But distrust of the security forces runs high in Darfur, where the
military has an ignominious history of manipulating, and frequently
directing, ethnic violence, according to human rights groups.

``When you see attacks like this one, and it's just another Tuesday in
Darfur, you realize that not much has changed since the revolution,''
said Cameron Hudson, a former State Department official and Sudan expert
at the Atlantic Council's Africa Center, a research group, speaking of
the attack on Fata Bornu.

Image

A funeral for those killed earlier this month in the hamlet of Fata
Bornu, in Darfur. Witnesses said Arab militiamen on horseback and
motorcycles stormed in with guns.Credit...via Sudanese Marginalization
Force Group

While Sudan's dictator has been vanquished, his legacy has proved harder
to dismantle, and the transition to democracy has stalled perilously in
key areas.

A transitional legislature, announced last year, has yet to be formed.
The youthful democracy campaigners who helped to oust Mr. al-Bashir have
split into factions. A campaign to get the United States to lift its
designation of Sudan as a state sponsor of terrorism, which perpetuates
Sudan's pariah status and blocks foreign investment, has
\href{https://www.nytimes3xbfgragh.onion/2020/07/25/us/politics/sudan-compensation-embassy-bombings.html}{run
into sand}.

And the country's transitional government, which is supposed to pave the
way for a general election in 2022, is troubled by tensions between
military and civilian leaders, Western diplomats say, as well as
\href{https://www.nytimes3xbfgragh.onion/2020/04/17/world/africa/Sudan-coup-coronavirus.html}{periodic
rumors} of a military coup.

Young Darfuris
\href{https://www.nytimes3xbfgragh.onion/2019/05/11/world/africa/sudan-darfur-revolution-protest.html}{were
at the forefront} of last year's uprising, when some were arrested and
tortured by Mr. al-Bashir's security forces. Their high profile fueled
their expectations that the revolution might bring radical change to
Darfur. But the dynamic of the conflict remains the same.

The war in Darfur erupted in 2003 when Darfuri rebels, angered by
longstanding discrimination against the region's non-Arab population,
rose up against the government. Mr. al-Bashir hit back by arming the
janjaweed, who killed thousands of civilians monthly at the height of
the war, in attacks that the International Criminal Court has classified
as genocide.

The United Nations estimates that at least 300,000 people have died in a
conflict that decreased in recent years, but never stopped.

Image

Abdalla Hamdok, Sudan's civilian prime minister, recently promised to
send police and army reinforcements to Darfur ``to protect citizens and
the farming season.''Credit...Ashraf Shazly/Agence France-Presse ---
Getty Images

Since the ouster of Mr. al-Bashir in April of 2019, one of the most
powerful figures in the new government, to the dismay of many Darfuris,
has been
\href{https://www.nytimes3xbfgragh.onion/2019/06/15/world/africa/sudan-leader-hemeti.html}{Lt.
Gen. Mohamed Hamdan}, a commander whose Rapid Support Forces
paramilitaries have been linked to atrocities in Darfur.

Mr. Hamdan is now part of a government-led effort to negotiate peace
with two Darfuri rebel groups. But many Darfuris see the talks as little
more than window-dressing because the largest rebel group, led by a
former lawyer, Abdul Wahid al Nur, has spurned them.

The escalating crisis comes just as the international community is
shifting attention away from Sudan.

The United Nations peacekeeping mission to Darfur, formed in 2007, is
expected to shut down by the end of the year. And Western governments
have slashed their aid budgets for Sudan, even as the pandemic plunges
the country deeper into poverty. Some~9.6 million people urgently need
food aid, the most ever for Sudan,
\href{http://www.ipcinfo.org/ipcinfo-website/ipc-alerts/issue-23/en/}{according
to the United Nations}.

Foreign officials, eager to prop up Sudan's shaky transition to
democracy, appear to be reluctant to point blame at its fledgling
government when trouble erupts in Darfur.

After the attack on Fata Bornu on July 13, the United Nations
peacekeeping mission, which is run jointly with the African Union,
framed the violence as a seasonal dispute between herders and
pastoralists. ``The farming seasons in Darfur have witnessed such
occurrences in the past,'' it said
\href{https://unamid.unmissions.org/unamid-deeply-concerned-about-violent-incidents-kutum-town-and-fata-borno-idps-camp-north-darfur}{in
a statement}.

The top U.S. State Department official for Africa, Tibor Nagy,
reinforced that narrative a day later with
\href{https://twitter.com/AsstSecStateAF/status/1283877883094749189}{praise
on Twitter} for cooperation between Sudanese security forces and United
Nations peacekeepers in Darfur.

But internal United Nations reports reviewed by The New York Times, as
well as interviews with Darfuri tribal leaders and United Nations
officials, paint a different picture.

Image

A camp for internally displaced people in Darfur last October. The U.N.
peacekeeping mission is winding down.Credit...Ashraf Shazly/Agence
France-Presse --- Getty Images

The villagers, mostly people displaced by earlier fighting, had been
protesting for two weeks before the attack. They demanded the
resignation of the local governor, Maj. Gen. Malik Al-Tayeb Khojaly,
accusing him of siding with the local Arab community.

On July 12, General Khojaly responded to the villagers' demands by
withdrawing security from the area, villagers told United Nations
officials. The Arab militiamen, who were at least 100 strong, attacked
the next morning.

The United Nations peacekeepers, mostly from Senegal, heard gunfire and
saw plumes of smoke as they approached the camp. When a contingent of at
least 130 peacekeepers finally arrived, they found terrified villagers
huddled outside a school or hiding in the bushes.

It was unclear if Sudanese troops had stood aside to allow the attack to
take place, but an internal United Nations report pointedly noted their
absence.

``None of the GoS security forces were on the ground to defend the
IDPs,'' it said, using shorthand for the Sudanese government and
internally displaced people.

A spokesman for the Khartoum government did not respond to requests for
comment. An official delegation visited Fata Bornu and promised to
investigate the attack.

Image

Members of a Sudanese government delegation visiting victims of a recent
attack in Darfur.

Ashraf Eissa, a spokesman for the United Nations peacekeepers, said the
mission was ``doing its best'' to protect civilians in the parts of
Darfur where it still deployed. Since 2018, the mission has closed bases
across Darfur and reduced its strength to 6,500 peacekeepers, down from
19,500.

``This is a mission in exit mode,'' he said.

Beyond Darfur, last year's revolution has brought some
\href{https://www.nytimes3xbfgragh.onion/2020/07/13/world/africa/sudan-fgm-alcohol-flogging.html}{significant
change}s to Sudan.

Mr. al-Bashir's once-powerful Islamist party has been dissolved and Mr.
Hamdok, a mild-mannered technocrat, has introduced a raft of modernizing
legal reforms. Female genital cutting has been outlawed. Women can no
longer be arrested for wearing clothing deemed insufficiently modest,
and flogging has been abolished for all lawbreakers.

The apostasy laws have been scrapped, Christians are allowed to consume
alcohol, and any citizen can leave Sudan without an exit visa.

Gay sex is no longer punishable by the death penalty, though it is still
subject to a seven-year jail sentence.

And Mr. al-Bashir, 76, is behind bars. Sentenced to two years
imprisonment for corruption in December, the deposed autocrat reappeared
in court last week to face separate charges over the 1989 military coup
that catapulted him to power. If convicted, he faces the death penalty.

Image

Protesters in Khartoum last year, listening to activists from Darfur
give speeches against the regime of~Omar Hassan al-Bashir.Credit...Bryan
Denton for The New York Times

On June 30, frustration with the slow pace of change prompted giant
street protests across Sudan, the first since last year. ``Freedom,
peace and justice,'' chanted the crowds, echoing a slogan of the
anti-Bashir movement. One of the protests took place in Fata Bornu.

But many other Darfuris, abandoning their dreams of change, have fled to
Libya to try the perilous sea crossing to Europe, or have ended up in
refugee camps in neighboring Chad.

``The Khartoum government is telling them to be patient,'' said Jérôme
Tubiana, co-author of a
\href{http://www.smallarmssurveysudan.org/fileadmin/docs/reports/HSBA-Report-Darfur-mobility.pdf}{recent
report} on the exodus for the Small Arms Survey, a research institute in
Switzerland. ``But they don't trust them. For many, the enemy is no
longer Bashir ---~it's the center of Sudan.''

Civil society leaders in Darfur say it is crucial that some form of
peacekeeping mission step in when the United Nations mission leaves
Darfur.

``We need to protect the lives of the people,'' said El Sadig Hassan,
secretary-general of the Darfur Bar Association. ``Otherwise, the crisis
will continue.''

Advertisement

\protect\hyperlink{after-bottom}{Continue reading the main story}

\hypertarget{site-index}{%
\subsection{Site Index}\label{site-index}}

\hypertarget{site-information-navigation}{%
\subsection{Site Information
Navigation}\label{site-information-navigation}}

\begin{itemize}
\tightlist
\item
  \href{https://help.nytimes3xbfgragh.onion/hc/en-us/articles/115014792127-Copyright-notice}{©~2020~The
  New York Times Company}
\end{itemize}

\begin{itemize}
\tightlist
\item
  \href{https://www.nytco.com/}{NYTCo}
\item
  \href{https://help.nytimes3xbfgragh.onion/hc/en-us/articles/115015385887-Contact-Us}{Contact
  Us}
\item
  \href{https://www.nytco.com/careers/}{Work with us}
\item
  \href{https://nytmediakit.com/}{Advertise}
\item
  \href{http://www.tbrandstudio.com/}{T Brand Studio}
\item
  \href{https://www.nytimes3xbfgragh.onion/privacy/cookie-policy\#how-do-i-manage-trackers}{Your
  Ad Choices}
\item
  \href{https://www.nytimes3xbfgragh.onion/privacy}{Privacy}
\item
  \href{https://help.nytimes3xbfgragh.onion/hc/en-us/articles/115014893428-Terms-of-service}{Terms
  of Service}
\item
  \href{https://help.nytimes3xbfgragh.onion/hc/en-us/articles/115014893968-Terms-of-sale}{Terms
  of Sale}
\item
  \href{https://spiderbites.nytimes3xbfgragh.onion}{Site Map}
\item
  \href{https://help.nytimes3xbfgragh.onion/hc/en-us}{Help}
\item
  \href{https://www.nytimes3xbfgragh.onion/subscription?campaignId=37WXW}{Subscriptions}
\end{itemize}
