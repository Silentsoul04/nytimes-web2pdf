Sections

SEARCH

\protect\hyperlink{site-content}{Skip to
content}\protect\hyperlink{site-index}{Skip to site index}

\href{https://www.nytimes3xbfgragh.onion/section/arts/design}{Art \&
Design}

\href{https://myaccount.nytimes3xbfgragh.onion/auth/login?response_type=cookie\&client_id=vi}{}

\href{https://www.nytimes3xbfgragh.onion/section/todayspaper}{Today's
Paper}

\href{/section/arts/design}{Art \& Design}\textbar{}An Artist Having Fun
While Waiting for Catastrophe

\url{https://nyti.ms/30ZojDO}

\begin{itemize}
\item
\item
\item
\item
\item
\end{itemize}

Advertisement

\protect\hyperlink{after-top}{Continue reading the main story}

Supported by

\protect\hyperlink{after-sponsor}{Continue reading the main story}

\hypertarget{an-artist-having-fun-while-waiting-for-catastrophe}{%
\section{An Artist Having Fun While Waiting for
Catastrophe}\label{an-artist-having-fun-while-waiting-for-catastrophe}}

Heather Phillipson's works make viewers smile, but underneath their
bright, over-the-top exteriors are dark, urgent messages.

\includegraphics{https://static01.graylady3jvrrxbe.onion/images/2020/07/31/arts/31phillipson1/merlin_175105821_1ec4a135-77cb-4e72-9f53-1949e9ef1b7f-articleLarge.jpg?quality=75\&auto=webp\&disable=upscale}

\href{https://www.nytimes3xbfgragh.onion/by/alex-marshall}{\includegraphics{https://static01.graylady3jvrrxbe.onion/images/2018/09/10/multimedia/author-alex-marshall/author-alex-marshall-thumbLarge.png}}

By \href{https://www.nytimes3xbfgragh.onion/by/alex-marshall}{Alex
Marshall}

\begin{itemize}
\item
  July 31, 2020
\item
  \begin{itemize}
  \item
  \item
  \item
  \item
  \item
  \end{itemize}
\end{itemize}

LONDON --- The artist Heather Phillipson's latest work is a 31-foot
statue of a dollop of whipped cream, with a fly on it.

This one hasn't been easy. In March, the work was meant to be installed
on an empty plinth in Trafalgar Square, the latest in
\href{https://www.nytimes3xbfgragh.onion/2017/01/19/arts/design/london-fourth-plinth-art.html}{a
series of commissions} that brings contemporary art to the central
London plaza. But on the day the installation was scheduled to begin,
Britain went into lockdown.

Soon after, she was having conversations with the London city officials
about whether the work could be installed during the pandemic at all.
The work's title, ``The End,'' didn't have the best connotations at a
moment when thousands were dying.

``It started to feel like there'd never be a good time, or a right time,
for it to go up,'' Phillipson said in a recent interview at her East
London studio.

\includegraphics{https://static01.graylady3jvrrxbe.onion/images/2020/07/31/arts/31phillipson8/merlin_175105725_b88998e8-88df-4a54-8d24-c9a8d2ef4866-articleLarge.jpg?quality=75\&auto=webp\&disable=upscale}

On Thursday, ``The End'' was finally unveiled. Phillipson said the work
had been conceived in 2016, not long after Britain voted to leave the
European Union, and she had wanted the creamy sculpture, which looks as
if it could ooze off its platform, to look precarious, because that's
how the world felt back then. Recently, she added, things have gotten
worse.

But people could read the statue however they wanted, Ms. Phillipson
said: She would even be happy if they just saw it as a bit of fun.

``Personally, I'm drawn to stuff that baffles me,'' she said. ``If I
don't get it, that's when I'm hooked.''

Image

The statue includes a working drone, which films the square below. A
feed of the video is available online.~Credit...Tom Jamieson for The New
York Times

Enjoying being confused is central to the charm of Ms. Phillipson's
works, whose bright, over-the-top exteriors often belie their dark,
urgent messages about environmental destruction or humanity's treatment
of animals. She is a vegan
(\href{https://www.theguardian.com/artanddesign/2018/aug/27/heather-phillipson-interview-vegan-artist-we-torture-eggs}{since
``before it was fashionable''}) and her interviews are littered with
talk of impending planetary doom.

``The End'' is a more ambiguous piece, but a huge planned installation
at Tate Britain is perhaps more typical: Ms. Phillipson will turn the
museum's central gallery into ``a suite of deranged landscapes,
addressing the earth as a thinking eruption, on the verge of
termination,'' she said. That work was supposed to be unveiled this
summer, but has been postponed because of the coronavirus and
\href{https://www.tate.org.uk/whats-on/tate-britain/exhibition/heather-phillipson}{is
now scheduled for 2021}.

Image

A still from ``The Age of Love,'' a 2018 work by Ms.
Phillipson.Credit...Heather Phillipson

In 2018, she staged
``\href{https://baltic.art/whats-on/heatherphillipson}{The Age of
Love}'' at the Baltic Center for Contemporary Art in northern England,
in which she filled a floor of the museum with agricultural machinery
and psychedelic videos of snails mating and swivel-eyed cats, all set to
booming dance music. A critic
\href{http://www.thecourieronline.co.uk/review-the-age-of-love-by-heather-phillipson-baltic/}{from
a local newspaper} wrote that her work ``speaks to our current
environmental state, scaring us into working harder to change the
world.''

That same year,
\href{https://art.tfl.gov.uk/projects/my-name-is-lettie-eggsyrub/}{Phillipson
made a 260-foot-long installation} on a disused subway platform in
London. The work featured TV screens that seemed to be walking on giant
chicken legs, and cartoonish egg sculptures, some of which appeared to
be releasing bad smells. ``It is all enough to turn you vegan,''
\href{https://www.theguardian.com/artanddesign/2018/jun/07/heather-phillipson-review-eggs-sculpture-underground-gloucester-road-tube-london}{the
critic Adrian Searle wrote in a review for The Guardian}.

Image

A still from ``My Name Is Lettie Eggsyrub,'' which was installed in a
London subway station in 2018.Credit...Heather Phillipson

Ms. Phillipson insisted her work was not simply about her political
views or lifestyle choices. ``Yes, I'm a vegan, but I'm also a woman, a
feminist,'' she said. ``All kinds of things feed into my art, because
whatever ideologies I have will be in there at some level. But I'm not
presenting an argument.''

Ekow Eshun, the chairman of the group that commissions works for the
Fourth Plinth, as the pedestal in Trafalgar Square is known, said in a
telephone interview that Ms. Phillipson was very good at ``summoning the
strangeness and discomfort and absurdity of the contemporary moment and
assembling that into forms that are unexpected.'' Her work also happened
to be ``extremely enjoyable,'' he added.

Iwona Blazwick, the director of London's Whitechapel Gallery, which
\href{https://www.whitechapelgallery.org/exhibitions/heather-phillipson/}{has
commissioned work by Ms. Phillipson}, said in a Zoom interview that her
art managed to be both ``hilarious and terrifying at the same time.''

``She reminds me of the Surrealists, actually,'' Ms. Blazwick said. Like
them, Phillipson juxtaposes unrelated items to give them new meaning.
``That is what sets her apart, and makes her a great sculptor,'' Ms.
Blazwick added.

In her studio, Ms. Phillipson --- who has no gallery representation and
worked as an office administrator until about five years ago --- seemed
surprised by her recent success. She never expected to get the Fourth
Plinth commission, she said. When she received an email in 2016 inviting
her to submit an idea, her response, she said, was, ``This is hilarious.
There's no way I'm going to get it.''

Image

``The more one thinks about the state of global politics, the harder it
is not to feel like there's a catastrophe coming,'' Ms. Phillipson said.
Credit...Tom Jamieson for The New York Times

Born in London, Ms. Phillipson spent much of her teenage years in rural
Wales. Her mother was a social worker and her father a musician who also
made art and wrote poetry.
(\href{https://www.nytimes3xbfgragh.onion/2016/02/11/t-magazine/art/heather-phillipson-british-artist.html}{Phillipson
is a prize winning poet} herself, has DJed at illegal raves and
\href{https://www.bbc.co.uk/programmes/m000gdzs}{makes sound collages
that have played on BBC radio}.)

She said she couldn't remember any specific moment that turned her onto
art --- it was always there, she said. Likewise, she added, she couldn't
remember a time when she didn't fear for the planet's future.

``My parents were vegetarian, so I was always politically tuned into our
relationship to other species and how that can be a problem,'' she said.

Her parents also talked to her about feminism, anti-racism and other
political issues from ``a really young age,'' she said, and those
conversations influenced her way of looking at the world.

``The more one thinks about the state of global politics, the harder it
is not to feel like there's a catastrophe coming,'' she said.

But she insisted her worldview wasn't actually just about doom and
gloom. ``The world is a disturbing place isn't it? But there's a lot of
joy in there,'' she said. Her works are ``holding a position of
conflict'' between those points, she added.

Image

``I love it!'' said Cheryl Lawrence, a scuba diving instructor, not
pictured. ``It's colorful, it's festive.''Credit...Tom Jamieson for The
New York Times

On Thursday morning, Ms. Phillipson, wearing three Black Lives Matter
badges, looked nervous as she waited in Trafalgar Square for ``The End''
to be unveiled. Her hands shook as she put on a face mask.

If she was still worried about whether it was a good time to unveil the
sculpture, she needn't have been. As soon as ``The End'' emerged from
underneath a huge black sheet, the few passers-by in Trafalgar Square
stopped to gawk at it, then take photos with bemused smiles.

In interviews, three commuters and one tourist from Belgium all said
they liked the work. ``I love it!'' said Cheryl Lawrence, a scuba diving
instructor. ``It's colorful, it's festive.''

When told about Phillipson's political motivations in making the work,
Ms. Lawrence waved the comment away. ``The average person isn't going to
think about that,'' she **** said: ``It'll probably just make them want
an ice cream.''

Advertisement

\protect\hyperlink{after-bottom}{Continue reading the main story}

\hypertarget{site-index}{%
\subsection{Site Index}\label{site-index}}

\hypertarget{site-information-navigation}{%
\subsection{Site Information
Navigation}\label{site-information-navigation}}

\begin{itemize}
\tightlist
\item
  \href{https://help.nytimes3xbfgragh.onion/hc/en-us/articles/115014792127-Copyright-notice}{©~2020~The
  New York Times Company}
\end{itemize}

\begin{itemize}
\tightlist
\item
  \href{https://www.nytco.com/}{NYTCo}
\item
  \href{https://help.nytimes3xbfgragh.onion/hc/en-us/articles/115015385887-Contact-Us}{Contact
  Us}
\item
  \href{https://www.nytco.com/careers/}{Work with us}
\item
  \href{https://nytmediakit.com/}{Advertise}
\item
  \href{http://www.tbrandstudio.com/}{T Brand Studio}
\item
  \href{https://www.nytimes3xbfgragh.onion/privacy/cookie-policy\#how-do-i-manage-trackers}{Your
  Ad Choices}
\item
  \href{https://www.nytimes3xbfgragh.onion/privacy}{Privacy}
\item
  \href{https://help.nytimes3xbfgragh.onion/hc/en-us/articles/115014893428-Terms-of-service}{Terms
  of Service}
\item
  \href{https://help.nytimes3xbfgragh.onion/hc/en-us/articles/115014893968-Terms-of-sale}{Terms
  of Sale}
\item
  \href{https://spiderbites.nytimes3xbfgragh.onion}{Site Map}
\item
  \href{https://help.nytimes3xbfgragh.onion/hc/en-us}{Help}
\item
  \href{https://www.nytimes3xbfgragh.onion/subscription?campaignId=37WXW}{Subscriptions}
\end{itemize}
