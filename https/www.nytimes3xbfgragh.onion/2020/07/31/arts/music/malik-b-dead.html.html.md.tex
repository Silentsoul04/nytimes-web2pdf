Sections

SEARCH

\protect\hyperlink{site-content}{Skip to
content}\protect\hyperlink{site-index}{Skip to site index}

\href{https://www.nytimes3xbfgragh.onion/section/arts/music}{Music}

\href{https://myaccount.nytimes3xbfgragh.onion/auth/login?response_type=cookie\&client_id=vi}{}

\href{https://www.nytimes3xbfgragh.onion/section/todayspaper}{Today's
Paper}

\href{/section/arts/music}{Music}\textbar{}Malik B., Longtime Member of
the Roots, Is Dead at 47

\url{https://nyti.ms/39VbU89}

\begin{itemize}
\item
\item
\item
\item
\item
\end{itemize}

Advertisement

\protect\hyperlink{after-top}{Continue reading the main story}

Supported by

\protect\hyperlink{after-sponsor}{Continue reading the main story}

\hypertarget{malik-b-longtime-member-of-the-roots-is-dead-at-47}{%
\section{Malik B., Longtime Member of the Roots, Is Dead at
47}\label{malik-b-longtime-member-of-the-roots-is-dead-at-47}}

He brought his understatedly gritty lyrics and cadence to the band
during its formative years. After he went solo, he dropped in on a few
Roots albums.

\includegraphics{https://static01.graylady3jvrrxbe.onion/images/2020/08/02/obituaries/02malik-obit/merlin_175140180_769f61d6-b332-437e-9c36-349c4f4c20a5-articleLarge.jpg?quality=75\&auto=webp\&disable=upscale}

By \href{https://www.nytimes3xbfgragh.onion/by/julia-carmel}{Julia
Carmel}

\begin{itemize}
\item
  July 31, 2020
\item
  \begin{itemize}
  \item
  \item
  \item
  \item
  \item
  \end{itemize}
\end{itemize}

Malik B., the elusive M.C. best known for his work with the Roots, died
on Wednesday. He was 47.

His death was
\href{https://twitter.com/DonChampionTV/status/1288520216574488577}{confirmed
by his cousin} Don Champion. Members of the Roots also posted public
statements on their
\href{https://www.instagram.com/p/CDPLAtJAFr7/}{Instagram} and
\href{https://twitter.com/theroots/status/1288571089019265025}{Twitter}
accounts, though the statements did not say where he died or specify the
cause.

Malik joined the hip-hop group then known as the Square Roots after he
met the founders, Questlove (Ahmir Thompson) and Black Thought (Tariq
Trotter), in 1991 at Millersville University in rural Millersville, Pa.
By 1993, the Roots had dropped ``Square'' from their name and
self-released their debut album, ``Organix.'' Touring relentlessly, they
soon developed a cult following in Europe.

New members filtered in and out each year. Malik appeared on three more
albums --- ``Do You Want More?!!!??!'' (1995), ``Illadelph Halflife''
(1996) and ``Things Fall Apart'' (1999) --- and then left the band.

Before his departure, the Roots were churning out critically acclaimed
albums and inching further into the hip-hop mainstream. Collaborating
with artists like D'Angelo, Common and Erykah Badu, the Roots made their
name merging the seemingly disparate worlds of live jazz and gritty East
Coast rap.

Though none of the members ever explicitly said why Malik had left,
``Water,'' a song off the group's album ``Phrenology'' (2002), openly
referred to his departure. Black Thought recalled meeting Malik, whom he
called ``Slacks,'' and hinted at the ways they had grown apart:

``But inside people down with me started to change/It was a couple
things, lil' syrup, lil' pills,/Instead of riding out on the road you'd
rather chill.''

``Things Fall Apart'' --- the title was taken from a 1958 novel by the
Nigeran writer
\href{https://www.nytimes3xbfgragh.onion/2013/03/23/world/africa/chinua-achebe-nigerian-writer-dies-at-82.html\#:~:text=Chinua\%20Achebe\%2C\%20the\%20Nigerian\%20author,died\%20after\%20a\%20brief\%20illness.}{Chinua
Achebe} --- became the group's breakthrough album, peaking at No. 4 on
the Billboard chart. ``You Got Me,'' a song from that album featuring
Erykah Badu and Eve, earned the Roots their first Grammy, in 2000.

Though Malik left the ground around that time, the other members have
continued to acknowledge his influence as the grounded emotional core of
the group.

``I always felt as if I possessed only a mere fraction of your true gift
and potential,'' Black Thought
\href{https://www.instagram.com/p/CDPLu1nFpft/}{wrote on Instagram}
after Malik B.'s death. ``Your steel sharpened my steel as I watched you
create cadences from the ether and set them free into the universe to
become poetic law, making the English language your bitch.''

Malik released solo music after leaving the Roots, including the EP
``Psychological'' in 2006, and two studio albums: ``Street Assault''
(2005) and ``Unpredictable'' (2015), which was a collaboration with the
producer known as Mr. Green.

He also returned as a featured artist on the Roots' albums ``Game
Theory'' (2006) and ``Rising Down'' (2008).

Malik Smart Abdul-Basit was born on Nov. 14, 1972, in Philadelphia.
Information on his survivors was not available.

Though the Roots toured tirelessly in the mid-1990s, Malik was often
absent. He notably didn't join the band's 1996 tour, on which the Roots
opened for the Beastie Boys. He joked about his absence on the 1999
track ``\href{https://genius.com/20831}{Adrenaline}'': ``Yeah, Malik B.
from the Roots, he ain't gone/I took the wrong exit, the sign said
Langhorne.''

Later in his career, when Malik performed and recorded as a solo artist
and became a member of the Philadelphia rap collective Beard Gang, he
maintained his quiet yet distinctive spirit.

``I'm a survivor, by any means,'' he said in
\href{https://arena.com/article/malik-b-and-mr-green-rooted-in-reality}{a
2015 interview with Arena} magazine. ``I'll work with whatever I have.
I'm that type of person. Regardless of the situation, you have your ups
and your downs, I'm definitely gonna make it.''

Advertisement

\protect\hyperlink{after-bottom}{Continue reading the main story}

\hypertarget{site-index}{%
\subsection{Site Index}\label{site-index}}

\hypertarget{site-information-navigation}{%
\subsection{Site Information
Navigation}\label{site-information-navigation}}

\begin{itemize}
\tightlist
\item
  \href{https://help.nytimes3xbfgragh.onion/hc/en-us/articles/115014792127-Copyright-notice}{©~2020~The
  New York Times Company}
\end{itemize}

\begin{itemize}
\tightlist
\item
  \href{https://www.nytco.com/}{NYTCo}
\item
  \href{https://help.nytimes3xbfgragh.onion/hc/en-us/articles/115015385887-Contact-Us}{Contact
  Us}
\item
  \href{https://www.nytco.com/careers/}{Work with us}
\item
  \href{https://nytmediakit.com/}{Advertise}
\item
  \href{http://www.tbrandstudio.com/}{T Brand Studio}
\item
  \href{https://www.nytimes3xbfgragh.onion/privacy/cookie-policy\#how-do-i-manage-trackers}{Your
  Ad Choices}
\item
  \href{https://www.nytimes3xbfgragh.onion/privacy}{Privacy}
\item
  \href{https://help.nytimes3xbfgragh.onion/hc/en-us/articles/115014893428-Terms-of-service}{Terms
  of Service}
\item
  \href{https://help.nytimes3xbfgragh.onion/hc/en-us/articles/115014893968-Terms-of-sale}{Terms
  of Sale}
\item
  \href{https://spiderbites.nytimes3xbfgragh.onion}{Site Map}
\item
  \href{https://help.nytimes3xbfgragh.onion/hc/en-us}{Help}
\item
  \href{https://www.nytimes3xbfgragh.onion/subscription?campaignId=37WXW}{Subscriptions}
\end{itemize}
