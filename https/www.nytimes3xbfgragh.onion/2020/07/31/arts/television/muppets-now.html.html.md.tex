Sections

SEARCH

\protect\hyperlink{site-content}{Skip to
content}\protect\hyperlink{site-index}{Skip to site index}

\href{https://www.nytimes3xbfgragh.onion/section/arts/television}{Television}

\href{https://myaccount.nytimes3xbfgragh.onion/auth/login?response_type=cookie\&client_id=vi}{}

\href{https://www.nytimes3xbfgragh.onion/section/todayspaper}{Today's
Paper}

\href{/section/arts/television}{Television}\textbar{}Muppet Meta Mania,
Revived for the Streaming Era

\url{https://nyti.ms/2D5M9pu}

\begin{itemize}
\item
\item
\item
\item
\item
\end{itemize}

Advertisement

\protect\hyperlink{after-top}{Continue reading the main story}

Supported by

\protect\hyperlink{after-sponsor}{Continue reading the main story}

Critic's Notebook

\hypertarget{muppet-meta-mania-revived-for-the-streaming-era}{%
\section{Muppet Meta Mania, Revived for the Streaming
Era}\label{muppet-meta-mania-revived-for-the-streaming-era}}

The Muppets were made of, by and for TV. Two new shows, ``Muppets Now''
on Disney+ and ``The Not-Too-Late Show With Elmo'' on HBO Max, reimagine
the media-savvy furry friends for a new age.

\includegraphics{https://static01.graylady3jvrrxbe.onion/images/2020/08/03/arts/03muppets/03muppets-articleLarge.jpg?quality=75\&auto=webp\&disable=upscale}

\href{https://www.nytimes3xbfgragh.onion/by/james-poniewozik}{\includegraphics{https://static01.graylady3jvrrxbe.onion/images/2018/02/16/multimedia/author-james-poniewozik/author-james-poniewozik-thumbLarge.jpg}}

By \href{https://www.nytimes3xbfgragh.onion/by/james-poniewozik}{James
Poniewozik}

\begin{itemize}
\item
  July 31, 2020
\item
  \begin{itemize}
  \item
  \item
  \item
  \item
  \item
  \end{itemize}
\end{itemize}

What is a Muppet made of? One of the first corrections I ever had to
make in this newspaper, and still the best, involved my review of the
2015 ABC sitcom ``The Muppets.'' I referred to the covering that makes
up the outside of Jim Henson's creations as felt; a reader informed me
that it was, in fact, fleece.
\href{https://www.nytimes3xbfgragh.onion/2015/09/22/arts/television/review-muppets-abc-kermit-ms-piggy.html}{\emph{Noted}}.

That truth, however, is only skin-deep. What Muppets are really made out
of is television.

This goes back to the earliest days of ``Sesame Street,'' in the 1960s,
when the creators conceived a kids' show with the metabolism and spirit
of ``Laugh-In,'' full of TV parodies and faux sponsorships. It continued
through that ill-fated ABC comedy, an unsettlingly edgy
behind-the-scenes look at a talk show starring Miss Piggy.

And that maniacal meta spirit powered ``The Muppet Show,'' a comedy
about a faux variety show that was also, itself, one of TV's best
variety shows (and the inspiration for a series of movies). Stressed-out
Kermit, melodramatic Piggy, hyperactive Animal and the rest lovingly
embodied the craziness of showbiz, for a mass-media era when TV
delivered dance, romance and seltzered pants for audiences of all ages
under one big tent.

As the show's
\href{https://www.youtube.com/watch?v=9KorhvVQRUM}{original pitch reel}
to TV executives promised, accurately: ``Small children will love the
cute, cuddly characters! Young people will love the fresh and innovative
comedy! College kids and intellectual eggheads will love the underlying
symbolism of everything!''

Cut to 2020, when TV is splintered and siloed, and so are the Muppets as
a property. The kids' end of the franchise, ``Sesame Street,''
\href{https://www.nytimes3xbfgragh.onion/2019/10/03/business/media/sesame-street-hbo-max.html}{belongs
to HBO Max}, after a move to the
\href{https://www.nytimes3xbfgragh.onion/2016/01/15/arts/television/tv-review-sesame-street-hbo.html}{gentrified
neighborhood of HBO} in 2016. The kids-of-all-ages end, populated by
``Muppet Show'' alumni, wear the sigil of House Disney.

So what, in the streaming era, is a Muppet now? That's the question of,
appropriately,
\href{https://www.nytimes3xbfgragh.onion/2020/07/23/arts/television/muppets-now-disney.html}{``Muppets
Now,''} on Disney+, which recaptures some of the bomb-throwing brio of
the 1970s ``Muppet Show,'' but in a more compartmentalized format.

Like its forebear, this is a show about the making of the show that
you're watching. This time, the puppety pals are not putting on a giant
theater-scaled production but uploading a package of mini-episodes, on
an unforgiving deadline, to a streaming service. Goodbye, Rainbow
Connection; hello, broadband connection.

Kermit and his lieutenant Scooter still sweat deadlines and suffer
fools, but virtually, through a teleconferencing screen. There are so
many chat windows in the new show, you might think it was developed
under coronavirus quarantine. It wasn't, but it all seems awfully
familiar right now.

Each half-hour episode collects a handful of recurring, Quibi-sized
segments. Miss Piggy hosts a lifestyle (rather, ``lifesty'') mini-show,
with sporting appearances from Taye Diggs and Linda Cardellini (the
latter joined by a talking hunk of brie). The Swedish Chef is ruining
dishes and endangering lives on a celebrity cooking-competition show.
Gonzo is shooting a wilderness survival show that we may never see
because bringing along a camera ``would be cheating.''

``Muppets Now'' improves on the ABC sitcom because it understands what
the Muppets are and why we love them. They're not mopey stand-ins for us
but wild, demonic imaginings of ourselves, unburdened by impulse control
and the laws of physics. Like Dr. Bunsen Honeydew, the bespectacled
scientist of Muppet Labs, this show knows there's no point in getting
access to a budget and a camera if you're not going to blow things up.

But with the segmented format of ``Muppets Now,'' you lose the big-scale
interaction among characters that animated the 1970s variety show. The
connective tissue here mostly consists of Kermit and Scooter
teleconferencing. There are some nice throwaway jokes there. (Scooter's
shared computer desktop includes the random folder ``UFOs?'') But just
like all the Zoom webinars you're attending these days, it's not quite
the same.

The best segments don't lean too hard into the ``Now'' part of ``Muppets
Now, but use the premise of quickie reality TV to resurrect the
old-fashioned appeal of entertainment made by maniacs. Pepe the King
Prawn steals the new episodes as the host of a game show whose
complicated rules and questions he invents on the fly. (``What was
Christopher Columbus's maiden name?'')

The Muppet Labs update, ``Field Test,'' finds an apt reality-video
corollary: the alleged science show whose real purpose is creative
destruction. Features include ``Will It Melt or Will It Burn?,'' a
question to which the only legitimate Muppet answer is ``Fetch me a
blowtorch.'' You may retain the odd scientific fact from it, but
Honeydew captures the show's, and the Muppets', true spirit: ``Let's
stop learning and let's start burning!''

\includegraphics{https://static01.graylady3jvrrxbe.onion/images/2020/08/03/arts/03muppets-2/merlin_172483533_d45ec44c-638b-4b2b-8ff3-179176109cea-articleLarge.jpg?quality=75\&auto=webp\&disable=upscale}

HBO Max, meanwhile, is putting its intellectual property to use in a
show for younger viewers that calls back to an older form of TV. ``The
Not-Too-Late Show With Elmo'' imagines that, for around 15 minutes
before bedtime, the ticklish young star of ``Sesame Street'' hosts a
full-on talk show from his home. (Far-fetched?
\href{https://www.nytimes3xbfgragh.onion/2020/04/29/arts/television/john-krasinski-some-good-news.html}{Tell
John Krasinski}.)

You may remember Elmo as the adorable/exasperating toddler-Muppet who
gradually hijacked ``Sesame Street'' starting in the 1980s. If you're
not a fan to begin with {[}raises hand{]}, ``Not-Too-Late'' will not
convert you.

But it's charmingly true to the character, who in retrospect has the
kind of insistent energy, nosiness and thirst for attention that makes
him perfect for late-night. ``Not-Too-Late'' is actually closer than
``Muppets Now'' to the format of the old ``Muppet Show,'' with chaos
backstage and Bert and Ernie squabbling in the control room. But the
spirit is all Elmo.

Each episode has a featured guest, a well-chosen group that includes
Andy Cohen (in disguise as Grover) and John Mulaney, fresh off his own
brilliant
\href{https://www.nytimes3xbfgragh.onion/2019/12/24/arts/television/john-mulaney-sack-lunch-bunch-netflix.html}{``Sack
Lunch Bunch''} kids' show sendup. There are also musical guests,
delivering sweetly oddball covers of lullabies and ``Sesame Street''
standards, like Lil Nas X taking ``Elmo's Song'' down the Old Town Road.

Elmo, however, remains the star. He high-fives his M.C., Cookie Monster;
he croons a good-night song; he tells knock-knock jokes. (``Who's
there?'' ``Tank.'' ``Tank who?'' ``You're welcome!'') Like Jimmy Fallon
(who visits the first episode), he challenges his guests to games and
goofy races.

Part of me, I will admit, fantasizes a more snarky, more adult --- more
``Muppet Show'' --- version of this series that looked at Elmo as a
Larry Sanders-esque needy diva, exploring his hunger for attention, his
thirst for validation, his insistence on seeing all the world as Elmo's
World.

That version will have to wait for another reboot. TV will keep morphing
and evolving. But the Muppets, it seems, will always be there for it ---
sometimes with a lullaby, sometimes with a blowtorch.

Advertisement

\protect\hyperlink{after-bottom}{Continue reading the main story}

\hypertarget{site-index}{%
\subsection{Site Index}\label{site-index}}

\hypertarget{site-information-navigation}{%
\subsection{Site Information
Navigation}\label{site-information-navigation}}

\begin{itemize}
\tightlist
\item
  \href{https://help.nytimes3xbfgragh.onion/hc/en-us/articles/115014792127-Copyright-notice}{©~2020~The
  New York Times Company}
\end{itemize}

\begin{itemize}
\tightlist
\item
  \href{https://www.nytco.com/}{NYTCo}
\item
  \href{https://help.nytimes3xbfgragh.onion/hc/en-us/articles/115015385887-Contact-Us}{Contact
  Us}
\item
  \href{https://www.nytco.com/careers/}{Work with us}
\item
  \href{https://nytmediakit.com/}{Advertise}
\item
  \href{http://www.tbrandstudio.com/}{T Brand Studio}
\item
  \href{https://www.nytimes3xbfgragh.onion/privacy/cookie-policy\#how-do-i-manage-trackers}{Your
  Ad Choices}
\item
  \href{https://www.nytimes3xbfgragh.onion/privacy}{Privacy}
\item
  \href{https://help.nytimes3xbfgragh.onion/hc/en-us/articles/115014893428-Terms-of-service}{Terms
  of Service}
\item
  \href{https://help.nytimes3xbfgragh.onion/hc/en-us/articles/115014893968-Terms-of-sale}{Terms
  of Sale}
\item
  \href{https://spiderbites.nytimes3xbfgragh.onion}{Site Map}
\item
  \href{https://help.nytimes3xbfgragh.onion/hc/en-us}{Help}
\item
  \href{https://www.nytimes3xbfgragh.onion/subscription?campaignId=37WXW}{Subscriptions}
\end{itemize}
