Sections

SEARCH

\protect\hyperlink{site-content}{Skip to
content}\protect\hyperlink{site-index}{Skip to site index}

\href{https://www.nytimes3xbfgragh.onion/section/books/review}{Book
Review}

\href{https://myaccount.nytimes3xbfgragh.onion/auth/login?response_type=cookie\&client_id=vi}{}

\href{https://www.nytimes3xbfgragh.onion/section/todayspaper}{Today's
Paper}

\href{/section/books/review}{Book Review}\textbar{}Stabbed in the
Convent, Murdered in the Yard

\url{https://nyti.ms/33e4Wtu}

\begin{itemize}
\item
\item
\item
\item
\item
\end{itemize}

Advertisement

\protect\hyperlink{after-top}{Continue reading the main story}

Supported by

\protect\hyperlink{after-sponsor}{Continue reading the main story}

\href{/column/crime}{Crime}

\hypertarget{stabbed-in-the-convent-murdered-in-the-yard}{%
\section{Stabbed in the Convent, Murdered in the
Yard}\label{stabbed-in-the-convent-murdered-in-the-yard}}

\includegraphics{https://static01.graylady3jvrrxbe.onion/images/2020/08/02/books/review/02Stasio-Crime/02Stasio-Crime-articleLarge.jpg?quality=75\&auto=webp\&disable=upscale}

By Marilyn Stasio

\begin{itemize}
\item
  July 31, 2020
\item
  \begin{itemize}
  \item
  \item
  \item
  \item
  \item
  \end{itemize}
\end{itemize}

Political protest is a sacred rite of passage for the young. In
\textbf{LADY CHEVY (Pegasus Crime, 296 pp., \$25.95),} John Woods
delivers a righteous cause to a high school senior named Amy Wirkner,
whose family lives in Barnesville, a bucolic town in the Ohio Valley
that's being chewed up by fracking and strip mining.

``The fracking rig is an industrial spire at the back of our property
line, about a quarter mile from our trailer,'' Amy tells us. ``At night,
its 20-foot flame enchants the orange horizon, a fire's dance, a hellish
light that is no light.'' The land still belongs to the family, but
sometimes they can set their water on fire and, she dispassionately
adds, ``Soon after the towers came, my baby brother was born deformed.''

Amy does what she must to survive. Her strength comes from the same
place where she finds the grit to shrug off the teasing she gets
``because I have a wide backside.'' But even as she flips off her
tormentors, ``on the inside, deep down, a dark red ball of anger keeps
me warm.''

Then, along with a misfit friend, Paul McCormick, Amy commits an act of
industrial terrorism that leaves a man dead, jeopardizing her plans for
college and maybe the rest of her life. Despite the crime, Woods makes
Amy so sympathetic, her life so bleak and her options so limited that
she becomes a paradigm for the entire valley, robbed of its youth and
denied its future. She's quite a remarkable character.

Amy comes from tough stock, a clan of miners whose faces are bared in
the old photos people hang in their homes, black-and-white reminders of
``the unhappy proud, strong like whipped horses.'' Woods writes in the
same style as those photographs, wrapping beauty in shocking misery.

♦

What's a border noir mystery without vicious drug cartels and desperate
illegal migrants? Alex Gilly obligingly provides those joys in
\textbf{DEATH RATTLE (Forge, 300 pp., \$27.99),} which is set in a town
on the California-Mexico border laughably named Paradise. The story
opens with the heart-thumping sea rescue of 22 men, women and children
who set out from Tijuana in a leaky boat and hit a storm. The lucky
survivors make it to the Paradise Detention Center, where the really
lucky ones find a humane lawyer like Mona Jimenez.

Carmen Vega, one of the really lucky ones, entrusts Mona with her
horrific history (battery acid is involved) as the property of Salvador
Soto, an enforcer for the notorious Caballeros drug cartel. If Carmen is
sent back to Mexico, Soto, a connoisseur and collector of snakes, will
put her in a box crawling with them.

Rather than dwell on what happens to Carmen, or start counting all the
snakes (including a \$10,000 black mamba) in this story, let's talk
about Mona. She's smart in the courtroom, she's tough in the field and
she has heart. She's also married to Nick Finn, a marine interdiction
agent for Customs and Border Protection, who makes the other half of a
super new team of crime-busters.

♦

\textbf{THE GIRL FROM WIDOW HILLS (Simon \& Schuster, 323 pp., \$26.99)}
has an offbeat plot and startling ending, both features we've come to
expect from Megan Miranda. As narrator of her own history, Olivia Meyer
has a good story to tell. When she was 6 years old and living with her
mother in the nothing-doing-here town of Widow Hills, Ky., Olivia became
famous as a ``miracle girl'' who survived being swept away in a storm
while she was sleepwalking. Twenty years later, Olivia has returned to
her hometown, where she works as an administrator in the local hospital.

Although she has issues with ``lingering night terrors,'' Olivia doesn't
sleepwalk anymore; at least, not until someone sends her a box of her
late mother's belongings. Unnerved by the memories, she starts
sleepwalking again, stopping only when she trips over a bloody corpse
one night in her yard. As Miranda reminds us, ``You become the stories
you tell.''

If only she didn't talk so much.

♦

In his novels, John Guzlowski --- the son of Polish slave laborers in
Nazi Germany --- reimagines the 1950s Chicago neighborhood he was raised
in, a place shaped by immigrants and strivers. \textbf{LITTLE ALTAR BOY
(Kasva Press, 323 pp., paper, \$14.95)} once again features Hank Purcell
and Marvin Bondarowicz, the two veteran cops whom we met in
``\href{https://www.nytimes3xbfgragh.onion/2018/11/30/books/review/louise-penny-kingdom-of-the-blind.html}{Suitcase
Charlie}'' and are happy to see again.

This time out Guzlowski is taking on pedophilia among the clergy, and
it's not pretty. Sister Mary Philomena, a nun at St. Fidelis Parish,
shows up at Purcell's home one snowy winter night. ``I need your help,''
she tells him. ``There's something terrible happening. I saw it today
\ldots{} and it stopped me like a death.'' What she witnessed was a
priest molesting an altar boy, a terrified sixth grader.

A few days later, the nun is found stabbed to death in the cellar of the
convent. In the classic procedural that follows, the cops choke down
their own cynicism (``People don't take that kind of accusation against
priests seriously. Never have, never will'') to investigate a crime that
officially doesn't exist.

Advertisement

\protect\hyperlink{after-bottom}{Continue reading the main story}

\hypertarget{site-index}{%
\subsection{Site Index}\label{site-index}}

\hypertarget{site-information-navigation}{%
\subsection{Site Information
Navigation}\label{site-information-navigation}}

\begin{itemize}
\tightlist
\item
  \href{https://help.nytimes3xbfgragh.onion/hc/en-us/articles/115014792127-Copyright-notice}{©~2020~The
  New York Times Company}
\end{itemize}

\begin{itemize}
\tightlist
\item
  \href{https://www.nytco.com/}{NYTCo}
\item
  \href{https://help.nytimes3xbfgragh.onion/hc/en-us/articles/115015385887-Contact-Us}{Contact
  Us}
\item
  \href{https://www.nytco.com/careers/}{Work with us}
\item
  \href{https://nytmediakit.com/}{Advertise}
\item
  \href{http://www.tbrandstudio.com/}{T Brand Studio}
\item
  \href{https://www.nytimes3xbfgragh.onion/privacy/cookie-policy\#how-do-i-manage-trackers}{Your
  Ad Choices}
\item
  \href{https://www.nytimes3xbfgragh.onion/privacy}{Privacy}
\item
  \href{https://help.nytimes3xbfgragh.onion/hc/en-us/articles/115014893428-Terms-of-service}{Terms
  of Service}
\item
  \href{https://help.nytimes3xbfgragh.onion/hc/en-us/articles/115014893968-Terms-of-sale}{Terms
  of Sale}
\item
  \href{https://spiderbites.nytimes3xbfgragh.onion}{Site Map}
\item
  \href{https://help.nytimes3xbfgragh.onion/hc/en-us}{Help}
\item
  \href{https://www.nytimes3xbfgragh.onion/subscription?campaignId=37WXW}{Subscriptions}
\end{itemize}
