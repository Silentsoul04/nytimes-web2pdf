Sections

SEARCH

\protect\hyperlink{site-content}{Skip to
content}\protect\hyperlink{site-index}{Skip to site index}

\href{https://www.nytimes3xbfgragh.onion/section/books/review}{Book
Review}

\href{https://myaccount.nytimes3xbfgragh.onion/auth/login?response_type=cookie\&client_id=vi}{}

\href{https://www.nytimes3xbfgragh.onion/section/todayspaper}{Today's
Paper}

\href{/section/books/review}{Book Review}\textbar{}A Native American
Coming-of-Age and the Uses of Enchantment

\url{https://nyti.ms/2Xesri1}

\begin{itemize}
\item
\item
\item
\item
\item
\end{itemize}

Advertisement

\protect\hyperlink{after-top}{Continue reading the main story}

Supported by

\protect\hyperlink{after-sponsor}{Continue reading the main story}

\href{/column/childrens-books}{Children's Books}

\hypertarget{a-native-american-coming-of-age-and-the-uses-of-enchantment}{%
\section{A Native American Coming-of-Age and the Uses of
Enchantment}\label{a-native-american-coming-of-age-and-the-uses-of-enchantment}}

\includegraphics{https://static01.graylady3jvrrxbe.onion/images/2020/08/02/books/02BKS-TREUER-KIDS/02BKS-TREUER-KIDS-articleLarge.jpg?quality=75\&auto=webp\&disable=upscale}

Buy Book ▾

\begin{itemize}
\tightlist
\item
  \href{https://www.amazon.com/gp/search?index=books\&tag=NYTBSREV-20\&field-keywords=The+Brave+James+Bird}{Amazon}
\item
  \href{https://du-gae-books-dot-nyt-du-prd.appspot.com/buy?title=The+Brave\&author=James+Bird}{Apple
  Books}
\item
  \href{https://www.anrdoezrs.net/click-7990613-11819508?url=https\%3A\%2F\%2Fwww.barnesandnoble.com\%2Fw\%2F\%3Fean\%3D9781250247759}{Barnes
  and Noble}
\item
  \href{https://www.anrdoezrs.net/click-7990613-35140?url=https\%3A\%2F\%2Fwww.booksamillion.com\%2Fp\%2FThe\%2BBrave\%2FJames\%2BBird\%2F9781250247759}{Books-A-Million}
\item
  \href{https://bookshop.org/a/3546/9781250247759}{Bookshop}
\item
  \href{https://www.indiebound.org/book/9781250247759?aff=NYT}{Indiebound}
\end{itemize}

When you purchase an independently reviewed book through our site, we
earn an affiliate commission.

By David Treuer

\begin{itemize}
\item
  July 31, 2020
\item
  \begin{itemize}
  \item
  \item
  \item
  \item
  \item
  \end{itemize}
\end{itemize}

James Bird's \textbf{THE BRAVE (Feiwel and Friends, 320 pp., \$16.99;
ages 9 to 12)} is a bighearted book about Collin. Collin has a few
problems. The most pressing is his compulsion to blurt out the number of
letters in the words spoken to him. This leads to bullying and
misunderstandings, and turns a sweet, smart adolescent into a problem
child. When the novel opens, Collin has caused one problem too many for
his beleaguered father, who sends him to live with his Ojibwe mother
(whom he's never met) on a reservation in Minnesota.

Off he goes on a dive into reservation life and an even deeper dive into
himself. His mother is beautiful and wise. His grandmother (who turns
out to be a ghost/spirit) is also beautiful and wise. His dead half
brother was beautiful and wise (and gay). But most important, he meets
and falls in love with his beautiful and wise neighbor Orenda --- really
dying of A.L.S. but convinced she is (spiritually and possibly
literally) turning into a butterfly. With the help of his new family and
the spirits, Collin learns to overcome his problems. The same cannot be
said for the book.

The place where all this happens, Fond du Lac Indian Reservation in
Minnesota, is real, and the larger world in which ``The Brave'' is set
is our world: airplanes, cars, bullies, math, pets, parents, gravity.
Spirituality and what we could call magic appear as well, apportioned to
the reservation and the Native American characters only. But realistic
or fantastical, fiction must create a ground-floor reality. And the
floor on which this novel is built is shaky.

Every Native American in it --- in addition to being beautiful and wise
--- is nice, brave and witty. Everything is a lesson. And none of it, to
my ear, is derived from Ojibwe culture or Ojibwe life as it's actually
lived at Fond du Lac.

Reservations have long been magic meaning machines for outsiders: dirty
prisons and proof of white perfidy if you've got a historical bent;
diminished gardens tended by sage earthkeepers if you're into folklore;
troubled places where policies fail if you're into politics. But in
every case reservations are imagined as places apart, in but not of
America, the land that time either rejected or forgot. ``The Brave'' is
no exception.

At one point, Collin attends a ceremony of sorts. Ushered into a teepee
with a fire in the middle, he is soon joined by four people wearing
robes of different colors and holding matching stones, which they place
in the fire. ``The stones sizzle to life, sending gray clouds of smoke
into the teepee. The heat immediately engulfs my body. I've never been
to a sauna before, but I imagine this is what it feels like.'' This
isn't how Ojibwe ceremony works. This isn't even how physics works.

Writers don't get to make Native American life mean whatever they want
it to mean. They don't get to do this because Native people have been
erased, silenced and willfully misunderstood for too many years.

It's especially important that they not do it in fiction for young
people, which may be the only stage of life when most Americans think
about us at all, as our history and present tense is inaccurately and
glancingly taught to them in school.

The world depicted in ``The Brave'' is not Native American life as I
know it. It's summer camp, complete with exotic names and faux rituals;
chock-full of crafts, bravery tests and self-discovery.

I want better books for my Ojibwe/Seneca children to read: books that
add to the stock of available reality, that incorporate our Native lives
in a way that informs those lives and makes them larger. ``The Brave''
does none of those things.

Advertisement

\protect\hyperlink{after-bottom}{Continue reading the main story}

\hypertarget{site-index}{%
\subsection{Site Index}\label{site-index}}

\hypertarget{site-information-navigation}{%
\subsection{Site Information
Navigation}\label{site-information-navigation}}

\begin{itemize}
\tightlist
\item
  \href{https://help.nytimes3xbfgragh.onion/hc/en-us/articles/115014792127-Copyright-notice}{©~2020~The
  New York Times Company}
\end{itemize}

\begin{itemize}
\tightlist
\item
  \href{https://www.nytco.com/}{NYTCo}
\item
  \href{https://help.nytimes3xbfgragh.onion/hc/en-us/articles/115015385887-Contact-Us}{Contact
  Us}
\item
  \href{https://www.nytco.com/careers/}{Work with us}
\item
  \href{https://nytmediakit.com/}{Advertise}
\item
  \href{http://www.tbrandstudio.com/}{T Brand Studio}
\item
  \href{https://www.nytimes3xbfgragh.onion/privacy/cookie-policy\#how-do-i-manage-trackers}{Your
  Ad Choices}
\item
  \href{https://www.nytimes3xbfgragh.onion/privacy}{Privacy}
\item
  \href{https://help.nytimes3xbfgragh.onion/hc/en-us/articles/115014893428-Terms-of-service}{Terms
  of Service}
\item
  \href{https://help.nytimes3xbfgragh.onion/hc/en-us/articles/115014893968-Terms-of-sale}{Terms
  of Sale}
\item
  \href{https://spiderbites.nytimes3xbfgragh.onion}{Site Map}
\item
  \href{https://help.nytimes3xbfgragh.onion/hc/en-us}{Help}
\item
  \href{https://www.nytimes3xbfgragh.onion/subscription?campaignId=37WXW}{Subscriptions}
\end{itemize}
