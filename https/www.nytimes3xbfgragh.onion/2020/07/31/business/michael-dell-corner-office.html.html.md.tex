\href{/section/business}{Business}\textbar{}`America, What a Country.'
Michael Dell on His Life and Business

\href{https://nyti.ms/3hVsBmA}{https://nyti.ms/3hVsBmA}

\begin{itemize}
\item
\item
\item
\item
\item
\end{itemize}

\includegraphics{https://static01.graylady3jvrrxbe.onion/images/2020/08/02/business/31CORNEROFFICE-DELL/31CORNEROFFICE-DELL-articleLarge.jpg?quality=75\&auto=webp\&disable=upscale}

Sections

\protect\hyperlink{site-content}{Skip to
content}\protect\hyperlink{site-index}{Skip to site index}

corner office

\hypertarget{america-what-a-country-michael-dell-on-his-life-and-business}{%
\section{`America, What a Country.' Michael Dell on His Life and
Business}\label{america-what-a-country-michael-dell-on-his-life-and-business}}

The billionaire tech executive discusses capitalism, immigration,
innovation and why 2020 has given us ``a glimpse of the future.''

Credit...Guerin Blask for The New York Times

Supported by

\protect\hyperlink{after-sponsor}{Continue reading the main story}

\href{https://www.nytimes3xbfgragh.onion/by/david-gelles}{\includegraphics{https://static01.graylady3jvrrxbe.onion/images/2018/07/24/multimedia/author-david-gelles/author-david-gelles-thumbLarge.png}}

By \href{https://www.nytimes3xbfgragh.onion/by/david-gelles}{David
Gelles}

\begin{itemize}
\item
  Published July 31, 2020Updated Aug. 1, 2020
\item
  \begin{itemize}
  \item
  \item
  \item
  \item
  \item
  \end{itemize}
\end{itemize}

The first time I sat down to talk with Michael Dell,
\href{https://www.nytimes3xbfgragh.onion/news-event/coronavirus}{the
coronavirus} had not disrupted life and business around the globe.

We discussed his journey from a precocious Texas schoolboy with an
interest in computers to the billionaire chief executive of one of the
world's biggest technology companies, along with his views on taxes,
immigration and philanthropy.

Much has changed since then. In a follow-up conversation this week, Mr.
Dell described how his company, Dell Technologies, weathered early
supply chain disruptions as the virus emerged in China, and went remote.
Most of Dell's more than 150,000 employees are still working from home.
And while the company has not had to carry out large-scale layoffs, many
of its small business customers are struggling to make payments amid the
economic disruption.

Over all, however, Mr. Dell struck an optimistic tone when asked how the
pandemic would shape the world in the years ahead.

``While 2020 will be seen as a kind of a tragic year with economic
disruption and loss of life, there's a couple of other stories that are
going on here,'' Mr. Dell said. ``One is, it's kind of amazing how much
business and commerce and education and health care and everything else
continued while all that was going on. That would not have been the case
15 or 20 years ago.''

And this is just the beginning, he said.

``I also think that 2020 will be a year of kind of great
accelerations,'' he said. ``We've kind of got a glimpse of the future
here.''

The following conversation, which was condensed and edited for clarity,
was conducted in New York before the pandemic.

\begin{center}\rule{0.5\linewidth}{\linethickness}\end{center}

\textbf{What were your parents like?}

My parents weren't into sports. They talked about the economy, the oil
crisis, interest rates. They read Scientific American, Barron's,
Fortune, Forbes and The Wall Street Journal. My mother was a real estate
agent, and then she became a financial analyst. She was the financial
brains of the family. My dad was accumulating some money as an
orthodontist, and she was investing it and doing quite well. They were
always talking about that around the dinner table. I liked the math and
science, and numbers.

\textbf{How did you become interested in technology?}

We lived in Houston, and NASA and the Johnson Space Center wasn't very
far away. My parents would take us there and we'd see the rockets
launch, and that was super exciting. Then, when I was about 8 years old,
I got an electronic calculator, which in 1973 was a big deal. It was a
National Semiconductor calculator. I was amazed that this thing could do
math --- multiplication and division. When I went to junior high school,
I wasn't on the track team or the football team or the basketball team.
I was in the Number Sense Club. You multiply three numbers by three
numbers in your head, and you compete at the district level, the state
level.

After school, I would go to RadioShack and hang out. I'd stay there
until they'd kick me out because I wasn't buying anything. Then, Apple
comes out with the Apple II, and I heard about Stephen Wozniak and Steve
Jobs and I said, ``I've got to have one of these things.'' I had saved
up some money from early entrepreneurial things --- trading baseball
cards and stamps, trading gold and silver, investing in stocks when I
was really pretty young. I bought an Apple II, and immediately took it
apart, which totally flabbergasted my parents.

\textbf{How did you start Dell?}

Fast forward to 1981, I'm 16 years old, and IBM comes out with the IBM
PC. ``OK, this IBM PC's going to be a really big deal.'' I got one of
those, took it apart, and I started upgrading those and training other
kids. I go off to college, and I go off and I am sort of continuing the
upgrading computer thing. And it becomes a bigger business while I'm in
my freshman year at college. My parents learn about it, get really,
really upset with me and said, ``You've got to stop doing it. You're
supposed to go to college.''

My parents were the first in their generation to go to college, and the
idea that I would give up an education to mess around with computers ---
they couldn't understand that at all. They pleaded with me. It was a
highly emotionally charged situation. So I stopped for like 10 days, and
it was in those 10 days that I really decided this wasn't a hobby. It
was actually what I wanted to do. So I did what any young 18-year-old
would do: I just did it, and didn't tell my parents. I moved out of my
dorm room and into a little office, which we outgrew in 30 days. And
here we are.

\textbf{You didn't graduate college?}

America, what a country.

\textbf{How did you go from essentially upgrading machines to figuring
out, ``Oh, there's a market to actually design and build new
machines?''}

When I took apart the IBM PC, one of the striking things was none of the
parts was from IBM. They were selling it for \$3,000 but it had, as far
as I could tell, maybe \$500 worth of parts. It seemed a bit like a
criminal enterprise. I mean, in terms of the math.

I started by upgrading the computers, and I would also buy the
stripped-down IBM computers and upgrade those and sell them. The
business quickly became making these hard-disk drive kits to upgrade IBM
computers that didn't have hard drives, because the early versions
didn't. We were making hundreds and thousands of these kits and selling
them all over the place.

The first eight years, we grew compounded 80 percent per year. The six
years after that we grew about 60 percent per year. Any number you start
with, if you put that into your calculator, you get like tens of
billions of dollars. That's what happened. America, what a country.

\textbf{What about tech gets you excited today in the way you were
excited about the personal computer 40 years ago?}

The incredible explosion in the amount of data. If you think about all
the buzzwords in the industry, the 5G, the A.I., the V.R., the IoT, blah
blah blah, insert your buzzword here. Behind all that stuff is
incredible quantities of data. Turning that data into useful insights,
outcomes, results, and the infrastructure, hardware, software required
to make all that happen, to secure it, to integrate it and to protect it
--- that requires an enormous set of new capabilities. In many ways,
it's just the beginning. And while the last 35 years have been amazing,
I think it's all going to pale in comparison to what's coming. In many,
many ways, it's just the beginning of tech.

\textbf{Last year at Davos you said you didn't support a steep increase
to the individual tax rate on the wealthiest Americans. Can you say a
bit more about that? Why isn't a higher individual tax rate a good thing
at a moment when the federal government clearly needs real resources to
do things like educate our kids?}

It may very well be. My wife and I have a foundation. We focus a lot on
education. We've contributed \$2.5 billion into our foundation, and it
does enormous work in the education space in the United States and
around the world. You've got myriad proposals out there for how to
improve the system. We'll let the marketplace of ideas do its thing. I
won't be shy in saying that I believe in entrepreneurship. I think
having a system where you can take risk and innovate is incredibly
important. Now, all that has to be balanced with the public interests.
There you go.

\textbf{Many of your contemporaries are not shy about saying, ``The
system's broken.'' Marc Benioff is out there saying, ``Capitalism's
broken.'' Ray Dalio is out there saying it. Do you have those sort of
same existential concerns as some of them?}

Probably not as inflammatory. Is it a perfect system? No. Can it be
improved? Yes. But let's go back to the entrepreneurship and
risk-taking, the innovation. We have, in this country, an engine that is
creating a lot of new businesses, and a lot of new innovation that is
globally relevant. I think any of those other countries would love to
have that, right?

\textbf{Is your contention that high taxes stifle that entrepreneurial
spirit, or that innovation?}

No, my contention is, I'm not a tax policy expert, and I'm not going to
be setting tax policy. It's just not what I do.

\textbf{When you talk about a system that supports entrepreneurship and
innovation, what does that look like?}

We have something pretty precious in our system that's a combination of
culture and capital. As we tweak it and improve it, we want to make sure
we preserve that, so that new, small businesses and entrepreneurs are
able to be created in the process.

\textbf{Does that extend to issues like immigration?}

I think it's pretty clear that we need comprehensive immigration reform,
and the only way you're going to do that is a bipartisan-type thing. I
don't see a lot of that happening. I don't see any George Washingtons
running around right now.

Advertisement

\protect\hyperlink{after-bottom}{Continue reading the main story}

\hypertarget{site-index}{%
\subsection{Site Index}\label{site-index}}

\hypertarget{site-information-navigation}{%
\subsection{Site Information
Navigation}\label{site-information-navigation}}

\begin{itemize}
\tightlist
\item
  \href{https://help.nytimes3xbfgragh.onion/hc/en-us/articles/115014792127-Copyright-notice}{©~2020~The
  New York Times Company}
\end{itemize}

\begin{itemize}
\tightlist
\item
  \href{https://www.nytco.com/}{NYTCo}
\item
  \href{https://help.nytimes3xbfgragh.onion/hc/en-us/articles/115015385887-Contact-Us}{Contact
  Us}
\item
  \href{https://www.nytco.com/careers/}{Work with us}
\item
  \href{https://nytmediakit.com/}{Advertise}
\item
  \href{http://www.tbrandstudio.com/}{T Brand Studio}
\item
  \href{https://www.nytimes3xbfgragh.onion/privacy/cookie-policy\#how-do-i-manage-trackers}{Your
  Ad Choices}
\item
  \href{https://www.nytimes3xbfgragh.onion/privacy}{Privacy}
\item
  \href{https://help.nytimes3xbfgragh.onion/hc/en-us/articles/115014893428-Terms-of-service}{Terms
  of Service}
\item
  \href{https://help.nytimes3xbfgragh.onion/hc/en-us/articles/115014893968-Terms-of-sale}{Terms
  of Sale}
\item
  \href{https://spiderbites.nytimes3xbfgragh.onion}{Site Map}
\item
  \href{https://help.nytimes3xbfgragh.onion/hc/en-us}{Help}
\item
  \href{https://www.nytimes3xbfgragh.onion/subscription?campaignId=37WXW}{Subscriptions}
\end{itemize}
