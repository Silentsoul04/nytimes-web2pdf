Sections

SEARCH

\protect\hyperlink{site-content}{Skip to
content}\protect\hyperlink{site-index}{Skip to site index}

\href{https://www.nytimes3xbfgragh.onion/section/health}{Health}

\href{https://myaccount.nytimes3xbfgragh.onion/auth/login?response_type=cookie\&client_id=vi}{}

\href{https://www.nytimes3xbfgragh.onion/section/todayspaper}{Today's
Paper}

\href{/section/health}{Health}\textbar{}Should Youth Come First in
Coronavirus Care?

\url{https://nyti.ms/30itXBP}

\begin{itemize}
\item
\item
\item
\item
\item
\item
\end{itemize}

\href{https://www.nytimes3xbfgragh.onion/news-event/coronavirus?action=click\&pgtype=Article\&state=default\&region=TOP_BANNER\&context=storylines_menu}{The
Coronavirus Outbreak}

\begin{itemize}
\tightlist
\item
  live\href{https://www.nytimes3xbfgragh.onion/2020/08/03/world/coronavirus-covid-19.html?action=click\&pgtype=Article\&state=default\&region=TOP_BANNER\&context=storylines_menu}{Latest
  Updates}
\item
  \href{https://www.nytimes3xbfgragh.onion/interactive/2020/us/coronavirus-us-cases.html?action=click\&pgtype=Article\&state=default\&region=TOP_BANNER\&context=storylines_menu}{Maps
  and Cases}
\item
  \href{https://www.nytimes3xbfgragh.onion/interactive/2020/science/coronavirus-vaccine-tracker.html?action=click\&pgtype=Article\&state=default\&region=TOP_BANNER\&context=storylines_menu}{Vaccine
  Tracker}
\item
  \href{https://www.nytimes3xbfgragh.onion/2020/08/02/us/covid-college-reopening.html?action=click\&pgtype=Article\&state=default\&region=TOP_BANNER\&context=storylines_menu}{College
  Reopening}
\item
  \href{https://www.nytimes3xbfgragh.onion/live/2020/08/03/business/stock-market-today-coronavirus?action=click\&pgtype=Article\&state=default\&region=TOP_BANNER\&context=storylines_menu}{Economy}
\end{itemize}

Advertisement

\protect\hyperlink{after-top}{Continue reading the main story}

Supported by

\protect\hyperlink{after-sponsor}{Continue reading the main story}

The New Old Age

\hypertarget{should-youth-come-first-in-coronavirus-care}{%
\section{Should Youth Come First in Coronavirus
Care?}\label{should-youth-come-first-in-coronavirus-care}}

If medical rationing becomes necessary, some older adults are prepared
to step aside. But many have the opposite concern: that they will be
arbitrarily sent to the rear of the line.

\includegraphics{https://static01.graylady3jvrrxbe.onion/images/2020/08/04/science/31SCI-SPAN/merlin_174983262_bf283f8c-a469-488f-b94c-b27e2b9d92cb-articleLarge.jpg?quality=75\&auto=webp\&disable=upscale}

By \href{https://www.nytimes3xbfgragh.onion/by/paula-span}{Paula Span}

\begin{itemize}
\item
  July 31, 2020
\item
  \begin{itemize}
  \item
  \item
  \item
  \item
  \item
  \item
  \end{itemize}
\end{itemize}

In April, as the coronavirus was rampaging through the Northeast, Larry
Churchill considered what he would do if the pandemic caused medical
shortages. Should he, a 75-year-old, direct care to younger people
before him if he got sick?

He was in a good position to raise the question. A bioethicist retired
from Vanderbilt University, he published an
\href{https://www.thehastingscenter.org/on-being-an-elder-in-a-pandemic/}{essay}
on the Hastings Center's bioethics forum saying that he intended to
avoid hospitals if they became overwhelmed and forgo a ventilator if
equipment grew scarce. When a vaccine became available, he would move to
the end of the line.

Fortunately, Dr. Churchill has not had to face such decisions. He
remains healthy, writing and teaching, and hiking in the Blue Ridge
Mountains. And enough ventilators were produced to meet demand.

But as new cases surge across the country and the flu season looms, the
prospect of rationing has resurfaced. Some policies adopted by states or
health care systems to allocate medical resources --- equipment, drugs,
critical care and intensive-care beds --- specifically make age part of
the equation. Other guidelines appear more neutral, but incorporate
factors that nevertheless disfavor older people, like other health
conditions or life expectancy.

After much discussion of what younger people owe their elders, who are
at greater risk from Covid-19, some older adults are contemplating their
obligations to the young.

``Part of the moral meaning of aging lies in a sense of reciprocity
across generations,'' Dr. Churchill wrote in his essay.

He does not believe that older people are less deserving of care, nor
would he want his personal philosophy to become public policy, he said;
other older adults will reach different conclusions.

\hypertarget{latest-updates-global-coronavirus-outbreak}{%
\section{\texorpdfstring{\href{https://www.nytimes3xbfgragh.onion/2020/08/03/world/coronavirus-covid-19.html?action=click\&pgtype=Article\&state=default\&region=MAIN_CONTENT_1\&context=storylines_live_updates}{Latest
Updates: Global Coronavirus
Outbreak}}{Latest Updates: Global Coronavirus Outbreak}}\label{latest-updates-global-coronavirus-outbreak}}

Updated 2020-08-04T07:33:06.428Z

\begin{itemize}
\tightlist
\item
  \href{https://www.nytimes3xbfgragh.onion/2020/08/03/world/coronavirus-covid-19.html?action=click\&pgtype=Article\&state=default\&region=MAIN_CONTENT_1\&context=storylines_live_updates\#link-4547638f}{Fauci
  defends Birx after she is criticized by Trump.}
\item
  \href{https://www.nytimes3xbfgragh.onion/2020/08/03/world/coronavirus-covid-19.html?action=click\&pgtype=Article\&state=default\&region=MAIN_CONTENT_1\&context=storylines_live_updates\#link-15e7f995}{Trump
  derides Democrats as lawmakers and administration officials try to
  break stimulus impasse.}
\item
  \href{https://www.nytimes3xbfgragh.onion/2020/08/03/world/coronavirus-covid-19.html?action=click\&pgtype=Article\&state=default\&region=MAIN_CONTENT_1\&context=storylines_live_updates\#link-e5a2cda}{The
  deadline for 2020 census counting has been moved up by a month.}
\end{itemize}

\href{https://www.nytimes3xbfgragh.onion/2020/08/03/world/coronavirus-covid-19.html?action=click\&pgtype=Article\&state=default\&region=MAIN_CONTENT_1\&context=storylines_live_updates}{See
more updates}

More live coverage:
\href{https://www.nytimes3xbfgragh.onion/live/2020/08/03/business/stock-market-today-coronavirus?action=click\&pgtype=Article\&state=default\&region=MAIN_CONTENT_1\&context=storylines_live_updates}{Markets}

But he subscribes to a ``life span approach'' to ethics, sometimes
called the ``fair innings'' approach: He has had his turns at bat.
Younger people have had less time to experience life's opportunities and
pleasures.

``That's why the death of a 40-year-old is tragic --- we see so much
unfulfilled potential,'' he said in an interview. ``My death at 75 would
be sad for the people who love me, but not tragic.''

His is a minority opinion among those considering medical ethics and
legal protections in a pandemic. Professional associations like the
American Geriatrics Society and advocacy groups like Justice in Aging
have focused more on the opposite possibility: They want to prevent
older and disabled people from being arbitrarily sent to the rear of the
line.

``You can have a fairly young individual who is frail and behaves like
an 85-year-old, and 85-year-olds who are running marathons,'' said Dr.
Timothy Farrell, the vice-chair of the American Geriatrics Society's
ethics committee and a geriatrician at University of Utah Health. ``So
to say that 85-year-olds don't have a claim on resources seems unjust.''

In June, the society published a
\href{https://onlinelibrary.wiley.com/doi/epdf/10.1111/jgs.16537}{position
statement} aimed at government policymakers and hospital administrators.
It warned that age categories, which some states apply to determine
access to care in disease outbreaks, natural disasters and other crises,
should never be used to exclude patients from treatment.

The statement also opposes criteria like ``life-years saved'' and
long-term life expectancy, which similarly disadvantage the older
population. Instead, it recommends treating patients based on the
likelihood of being discharged from the hospital and surviving for six
months.

``Our ability as physicians to prognosticate is generally poor,
particularly in the heat of the moment,'' Dr. Farrell said. Short-term
predictions are more accurate.

The society took no position on whether age should serve as a
``tiebreaker,'' as some ethics guidelines permit in cases where two
patients are deemed equally likely to survive. But ``that would be bias,
in our view,'' said Regan Bailey, the director of litigation for Justice
in Aging.

Her organization, along with others representing disabled people and
older adults, has filed several complaints with the Office of Civil
Rights at the federal Department of Health and Human Services, charging
that state policies for allocating health care during a crisis violate
federal civil rights laws.

In March, the Office of Civil
Rights\href{https://www.hhs.gov/about/news/2020/03/28/ocr-issues-bulletin-on-civil-rights-laws-and-hipaa-flexibilities-that-apply-during-the-covid-19-emergency.html}{reaffirmed}
that the Affordable Care Act and other federal statutes prohibit
discrimination, in health facilities receiving federal funds, on the
basis of age, disability and other characteristics.

Nevertheless, the coalition has argued that guidelines in Oregon,
\href{https://www.justiceinaging.org/older-adults-and-people-with-disabilities-challenge-discriminatory-surge-care-guidelines-in-covid-19-hotspots-arizona-and-texas/}{Arizona
and northern Texas} remain discriminatory. For example, Arizona, which
activated its crisis guidelines in June, includes long-term mortality as
a consideration, along with the ability to experience ``life stages.''
While those cases are pending, other civil rights complaints have
brought changes in
\href{https://www.hhs.gov/about/news/2020/06/26/ocr-resolves-complaint-tennessee-after-it-revises-its-triage-plans-protect-against-disability.html}{Tennessee},
\href{https://www.hhs.gov/about/news/2020/04/16/ocr-resolves-civil-rights-complaint-against-pennsylvania-after-it-revises-its-pandemic-health-care.html}{Pennsylvania}
and
\href{https://www.hhs.gov/about/news/2020/04/08/ocr-reaches-early-case-resolution-alabama-after-it-removes-discriminatory-ventilator-triaging.html}{Alabama}.

\href{https://www.nytimes3xbfgragh.onion/news-event/coronavirus?action=click\&pgtype=Article\&state=default\&region=MAIN_CONTENT_3\&context=storylines_faq}{}

\hypertarget{the-coronavirus-outbreak-}{%
\subsubsection{The Coronavirus Outbreak
›}\label{the-coronavirus-outbreak-}}

\hypertarget{frequently-asked-questions}{%
\paragraph{Frequently Asked
Questions}\label{frequently-asked-questions}}

Updated August 3, 2020

\begin{itemize}
\item ~
  \hypertarget{im-a-small-business-owner-can-i-get-relief}{%
  \paragraph{I'm a small-business owner. Can I get
  relief?}\label{im-a-small-business-owner-can-i-get-relief}}

  \begin{itemize}
  \tightlist
  \item
    The
    \href{https://www.nytimes3xbfgragh.onion/article/small-business-loans-stimulus-grants-freelancers-coronavirus.html?action=click\&pgtype=Article\&state=default\&region=MAIN_CONTENT_3\&context=storylines_faq}{stimulus
    bills enacted in March} offer help for the millions of American
    small businesses. Those eligible for aid are businesses and
    nonprofit organizations with fewer than 500 workers, including sole
    proprietorships, independent contractors and freelancers. Some
    larger companies in some industries are also eligible. The help
    being offered, which is being managed by the Small Business
    Administration, includes the Paycheck Protection Program and the
    Economic Injury Disaster Loan program. But lots of folks have
    \href{https://www.nytimes3xbfgragh.onion/interactive/2020/05/07/business/small-business-loans-coronavirus.html?action=click\&pgtype=Article\&state=default\&region=MAIN_CONTENT_3\&context=storylines_faq}{not
    yet seen payouts.} Even those who have received help are confused:
    The rules are draconian, and some are stuck sitting on
    \href{https://www.nytimes3xbfgragh.onion/2020/05/02/business/economy/loans-coronavirus-small-business.html?action=click\&pgtype=Article\&state=default\&region=MAIN_CONTENT_3\&context=storylines_faq}{money
    they don't know how to use.} Many small-business owners are getting
    less than they expected or
    \href{https://www.nytimes3xbfgragh.onion/2020/06/10/business/Small-business-loans-ppp.html?action=click\&pgtype=Article\&state=default\&region=MAIN_CONTENT_3\&context=storylines_faq}{not
    hearing anything at all.}
  \end{itemize}
\item ~
  \hypertarget{what-are-my-rights-if-i-am-worried-about-going-back-to-work}{%
  \paragraph{What are my rights if I am worried about going back to
  work?}\label{what-are-my-rights-if-i-am-worried-about-going-back-to-work}}

  \begin{itemize}
  \tightlist
  \item
    Employers have to provide
    \href{https://www.osha.gov/SLTC/covid-19/standards.html}{a safe
    workplace} with policies that protect everyone equally.
    \href{https://www.nytimes3xbfgragh.onion/article/coronavirus-money-unemployment.html?action=click\&pgtype=Article\&state=default\&region=MAIN_CONTENT_3\&context=storylines_faq}{And
    if one of your co-workers tests positive for the coronavirus, the
    C.D.C.} has said that
    \href{https://www.cdc.gov/coronavirus/2019-ncov/community/guidance-business-response.html}{employers
    should tell their employees} -\/- without giving you the sick
    employee's name -\/- that they may have been exposed to the virus.
  \end{itemize}
\item ~
  \hypertarget{should-i-refinance-my-mortgage}{%
  \paragraph{Should I refinance my
  mortgage?}\label{should-i-refinance-my-mortgage}}

  \begin{itemize}
  \tightlist
  \item
    \href{https://www.nytimes3xbfgragh.onion/article/coronavirus-money-unemployment.html?action=click\&pgtype=Article\&state=default\&region=MAIN_CONTENT_3\&context=storylines_faq}{It
    could be a good idea,} because mortgage rates have
    \href{https://www.nytimes3xbfgragh.onion/2020/07/16/business/mortgage-rates-below-3-percent.html?action=click\&pgtype=Article\&state=default\&region=MAIN_CONTENT_3\&context=storylines_faq}{never
    been lower.} Refinancing requests have pushed mortgage applications
    to some of the highest levels since 2008, so be prepared to get in
    line. But defaults are also up, so if you're thinking about buying a
    home, be aware that some lenders have tightened their standards.
  \end{itemize}
\item ~
  \hypertarget{what-is-school-going-to-look-like-in-september}{%
  \paragraph{What is school going to look like in
  September?}\label{what-is-school-going-to-look-like-in-september}}

  \begin{itemize}
  \tightlist
  \item
    It is unlikely that many schools will return to a normal schedule
    this fall, requiring the grind of
    \href{https://www.nytimes3xbfgragh.onion/2020/06/05/us/coronavirus-education-lost-learning.html?action=click\&pgtype=Article\&state=default\&region=MAIN_CONTENT_3\&context=storylines_faq}{online
    learning},
    \href{https://www.nytimes3xbfgragh.onion/2020/05/29/us/coronavirus-child-care-centers.html?action=click\&pgtype=Article\&state=default\&region=MAIN_CONTENT_3\&context=storylines_faq}{makeshift
    child care} and
    \href{https://www.nytimes3xbfgragh.onion/2020/06/03/business/economy/coronavirus-working-women.html?action=click\&pgtype=Article\&state=default\&region=MAIN_CONTENT_3\&context=storylines_faq}{stunted
    workdays} to continue. California's two largest public school
    districts --- Los Angeles and San Diego --- said on July 13, that
    \href{https://www.nytimes3xbfgragh.onion/2020/07/13/us/lausd-san-diego-school-reopening.html?action=click\&pgtype=Article\&state=default\&region=MAIN_CONTENT_3\&context=storylines_faq}{instruction
    will be remote-only in the fall}, citing concerns that surging
    coronavirus infections in their areas pose too dire a risk for
    students and teachers. Together, the two districts enroll some
    825,000 students. They are the largest in the country so far to
    abandon plans for even a partial physical return to classrooms when
    they reopen in August. For other districts, the solution won't be an
    all-or-nothing approach.
    \href{https://bioethics.jhu.edu/research-and-outreach/projects/eschool-initiative/school-policy-tracker/}{Many
    systems}, including the nation's largest, New York City, are
    devising
    \href{https://www.nytimes3xbfgragh.onion/2020/06/26/us/coronavirus-schools-reopen-fall.html?action=click\&pgtype=Article\&state=default\&region=MAIN_CONTENT_3\&context=storylines_faq}{hybrid
    plans} that involve spending some days in classrooms and other days
    online. There's no national policy on this yet, so check with your
    municipal school system regularly to see what is happening in your
    community.
  \end{itemize}
\item ~
  \hypertarget{is-the-coronavirus-airborne}{%
  \paragraph{Is the coronavirus
  airborne?}\label{is-the-coronavirus-airborne}}

  \begin{itemize}
  \tightlist
  \item
    The coronavirus
    \href{https://www.nytimes3xbfgragh.onion/2020/07/04/health/239-experts-with-one-big-claim-the-coronavirus-is-airborne.html?action=click\&pgtype=Article\&state=default\&region=MAIN_CONTENT_3\&context=storylines_faq}{can
    stay aloft for hours in tiny droplets in stagnant air}, infecting
    people as they inhale, mounting scientific evidence suggests. This
    risk is highest in crowded indoor spaces with poor ventilation, and
    may help explain super-spreading events reported in meatpacking
    plants, churches and restaurants.
    \href{https://www.nytimes3xbfgragh.onion/2020/07/06/health/coronavirus-airborne-aerosols.html?action=click\&pgtype=Article\&state=default\&region=MAIN_CONTENT_3\&context=storylines_faq}{It's
    unclear how often the virus is spread} via these tiny droplets, or
    aerosols, compared with larger droplets that are expelled when a
    sick person coughs or sneezes, or transmitted through contact with
    contaminated surfaces, said Linsey Marr, an aerosol expert at
    Virginia Tech. Aerosols are released even when a person without
    symptoms exhales, talks or sings, according to Dr. Marr and more
    than 200 other experts, who
    \href{https://academic.oup.com/cid/article/doi/10.1093/cid/ciaa939/5867798}{have
    outlined the evidence in an open letter to the World Health
    Organization}.
  \end{itemize}
\end{itemize}

None of these efforts, however, prevent individuals like Dr. Churchill
from voluntarily declining treatment to benefit the young, or for any
other reason. Specifying one's choices, using an advance directive, is
of greater importance than ever during a pandemic.

For those who want to yield their place in the health care line, an
organization called \href{https://www.saveothersouls.org/}{Save Other
Souls} has developed a document that takes effect during a declared
state of emergency for Covid-19. Vetted by lawyers, it allows people of
any age to cede medical equipment, drugs or hospital care to others. The
directive lapses when the emergency ends, or after 18 months.

``It's akin to the person who runs into a burning building or gives up
the last seat on the lifeboat,'' said Dr. Andrea Kittrell, an
otolaryngologist in Lynchburg, Va., who created the organization in
March. ``There are those people who are selfless and generous and value
other people's lives as much or more than their own.''

Winnona Merritt, for example, works daily in her vegetable garden in
High Point, N.C., sharing cucumbers and squash with her neighbors.
Vigorous at 82, Ms. Merritt said she would welcome more good years. But
in a pandemic, ``I'm afraid I could go to the head of the line, ahead of
someone younger, with a family,'' she said. ``I don't need that. I've
had a wonderful life.'' With her family's support, she signed an S.O.S.
directive.

Research suggests that altruism and
\href{https://academic.oup.com/psychsocgerontology/article/74/1/52/3105859}{generosity
increase at older ages}. For example, in recruiting volunteers for
Experience Corps, which trains seniors to assist in public schools, the
\href{https://ajph.aphapublications.org/doi/full/10.2105/AJPH.2009.169151}{most
effective recruiting messages} appealed to a person's desire to help the
next generation.

Successful programs like Experience Corps also point to a common
criticism of the fair innings philosophy: The last innings of the game
can be among the most significant. ``We may be missing out on the
contributions people make later in life,'' Dr. Farrell said. Moreover,
some individuals, for reasons including race, gender or poverty, never
got their fair innings.

Of course, forgoing health care isn't the only altruistic response to
Covid-19. Jeffrey Balkind, 73, a retired World Bank administrator in
Washington, has
\href{https://www.coronaviruspreventionnetwork.org/}{volunteered for
clinical vaccine trials}. Mr. Balkind has had two close encounters with
death, as a hostage during a
\href{https://www.washingtonpost.com/archive/lifestyle/1994/09/23/amateur-chronicles-crisis/0a961fb6-af41-4857-8884-5bfc0a3511fd/}{1981
airliner}hijacking, then after a 2017 accident with his Vespa scooter.
Injured and hospitalized, he resolved that ``if I ever had a chance, I
would try to do something useful in the medical field.''

It wasn't simple. For the initial trial of
\href{https://www.nytimes3xbfgragh.onion/2020/07/27/health/moderna-vaccine-covid.html}{a
vaccine candidate developed by the biotech company Moderna} and the
National Institutes of Health, Mr. Balkind underwent a health screening
by phone and a two-hour physical exam, then signed a 23-page consent
form. He was deeply disappointed to be placed on a standby list.

But he was accepted into the much larger Phase 3 trial now underway, and
expected to receive his first dose on Monday. ``I am thrilled,'' he
said. ``I feel I'm on a fascinating adventure.''

Advertisement

\protect\hyperlink{after-bottom}{Continue reading the main story}

\hypertarget{site-index}{%
\subsection{Site Index}\label{site-index}}

\hypertarget{site-information-navigation}{%
\subsection{Site Information
Navigation}\label{site-information-navigation}}

\begin{itemize}
\tightlist
\item
  \href{https://help.nytimes3xbfgragh.onion/hc/en-us/articles/115014792127-Copyright-notice}{©~2020~The
  New York Times Company}
\end{itemize}

\begin{itemize}
\tightlist
\item
  \href{https://www.nytco.com/}{NYTCo}
\item
  \href{https://help.nytimes3xbfgragh.onion/hc/en-us/articles/115015385887-Contact-Us}{Contact
  Us}
\item
  \href{https://www.nytco.com/careers/}{Work with us}
\item
  \href{https://nytmediakit.com/}{Advertise}
\item
  \href{http://www.tbrandstudio.com/}{T Brand Studio}
\item
  \href{https://www.nytimes3xbfgragh.onion/privacy/cookie-policy\#how-do-i-manage-trackers}{Your
  Ad Choices}
\item
  \href{https://www.nytimes3xbfgragh.onion/privacy}{Privacy}
\item
  \href{https://help.nytimes3xbfgragh.onion/hc/en-us/articles/115014893428-Terms-of-service}{Terms
  of Service}
\item
  \href{https://help.nytimes3xbfgragh.onion/hc/en-us/articles/115014893968-Terms-of-sale}{Terms
  of Sale}
\item
  \href{https://spiderbites.nytimes3xbfgragh.onion}{Site Map}
\item
  \href{https://help.nytimes3xbfgragh.onion/hc/en-us}{Help}
\item
  \href{https://www.nytimes3xbfgragh.onion/subscription?campaignId=37WXW}{Subscriptions}
\end{itemize}
