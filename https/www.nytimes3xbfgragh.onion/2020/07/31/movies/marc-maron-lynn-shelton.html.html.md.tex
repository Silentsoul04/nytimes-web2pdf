Sections

SEARCH

\protect\hyperlink{site-content}{Skip to
content}\protect\hyperlink{site-index}{Skip to site index}

\href{https://www.nytimes3xbfgragh.onion/section/movies}{Movies}

\href{https://myaccount.nytimes3xbfgragh.onion/auth/login?response_type=cookie\&client_id=vi}{}

\href{https://www.nytimes3xbfgragh.onion/section/todayspaper}{Today's
Paper}

\href{/section/movies}{Movies}\textbar{}The Lynn Shelton That Marc Maron
Knew

\url{https://nyti.ms/3hUNM8d}

\begin{itemize}
\item
\item
\item
\item
\item
\item
\end{itemize}

Advertisement

\protect\hyperlink{after-top}{Continue reading the main story}

Supported by

\protect\hyperlink{after-sponsor}{Continue reading the main story}

\hypertarget{the-lynn-shelton-that-marc-maron-knew}{%
\section{The Lynn Shelton That Marc Maron
Knew}\label{the-lynn-shelton-that-marc-maron-knew}}

The comedian and podcaster reflects on his relationship with the
filmmaker and TV director who died unexpectedly in May.

\includegraphics{https://static01.graylady3jvrrxbe.onion/images/2020/08/01/arts/31shelton-maron1/merlin_175081011_7bbc3ef6-09e8-4ee7-9e6b-56386ac67617-articleLarge.jpg?quality=75\&auto=webp\&disable=upscale}

\href{https://www.nytimes3xbfgragh.onion/by/dave-itzkoff}{\includegraphics{https://static01.graylady3jvrrxbe.onion/images/2018/02/16/multimedia/author-dave-itzkoff/author-dave-itzkoff-thumbLarge.jpg}}

By \href{https://www.nytimes3xbfgragh.onion/by/dave-itzkoff}{Dave
Itzkoff}

\begin{itemize}
\item
  July 31, 2020
\item
  \begin{itemize}
  \item
  \item
  \item
  \item
  \item
  \item
  \end{itemize}
\end{itemize}

When the writer-director Lynn Shelton
\href{https://www.nytimes3xbfgragh.onion/2020/05/16/obituaries/lynn-shelton-dead.html?searchResultPosition=4}{died
in May}, it was a devastating loss to film, to moviegoers who knew her
for intimate low-budget features like
\href{https://www.youtube.com/watch?v=RKy63pehrvw}{``Humpday''} and
\href{https://www.youtube.com/watch?v=wglzH4dUoLQ}{``Your Sister's
Sister,''} and to TV audiences starting to discover her work as a
director on series like ``GLOW'' and ``The Morning Show.'' On Tuesday,
she received a posthumous Emmy directing nomination for the Hulu limited
series ``Little Fires Everywhere.''

Shelton, 54, left behind family, friends and frequent collaborators who
included Marc Maron, the comic, actor and host of
\href{http://www.wtfpod.com/}{the long-running ``WTF'' podcast}. Shelton
had directed Maron in her 2019 film,
\href{https://www.youtube.com/watch?v=vWmQyBjrbfU\&t=5s}{``Sword of
Trust,''} his stand-up specials ``Too Real'' (2017) and ``End Times
Fun'' (2020), and his IFC series, ``Maron,'' among other shows. They had
also been romantically involved for about a year, after a lengthy and
sometimes awkward courtship that began when Shelton appeared
\href{http://www.wtfpod.com/podcast/episodes/episode_627_-_lynn_shelton}{as
a guest on ``WTF'' in 2015.}

Maron, 56, has not had much opportunity to mourn Shelton privately. As a
comedian and podcaster, he is known for his blunt and relentlessly
confessional approach, and he has embraced the notion of candor even in
his suffering. In the midst of the coronavirus pandemic, Maron said, it
was important to be open about his sorrow for listeners who may be going
through their own difficulties as well as for himself. Reflecting on his
grief, he said, has not diminished it but has helped him understand it
as unavoidable and universal.

``It's a terrible experience but it is a fundamental human experience,''
he said. ``It's as common as love. It's devastating, but we are built to
carry it, for ourselves and for others.''

\includegraphics{https://static01.graylady3jvrrxbe.onion/images/2020/08/01/arts/31shelton-maron6/merlin_175087587_20e88c2d-53be-4bc6-b877-e271545393d2-articleLarge.jpg?quality=75\&auto=webp\&disable=upscale}

In a phone interview from Los Angeles, during which Maron also stopped
to get a drive-through test for Covid-19 (``I'm not sick,'' he
explained, ``I'm just crazy''), he talked about his memories of Shelton,
his appreciation for her work and how he has tried to process her death,
publicly and in real time. These are edited excerpts from that
conversation.

\textbf{You first met Lynn Shelton when she came on your podcast. How
did that come about?}

I had seen in one of the trades that she was working on some project
with my ex-wife {[}the writer Mishna Wolff{]}. I had never really heard
of her, but I watched a couple of her movies and she seemed like the
real deal. I had her on the podcast and she was very charming and
eloquent about her own process and who she was. I got her and she
clearly got me. That was the beginning of something that became rather
intense.

\textbf{Listening to that interview now, does it sound like you were
already starting to connect with each other?}

That was definitely there and we both knew it. But we weren't in a
situation to really do anything about it. {[}Shelton was married and
Maron was in another relationship at the time.{]} It was difficult. From
that connection, we started our creative relationship, because I wanted
to see her again and get to know her. I knew she was a great director
and why wouldn't she want to direct my television show?

\textbf{How did you find it working with her on ``Maron''?}

She was always very disarming to me. I'm a stubborn, difficult man. And
I was difficult with her, too. She had a vision and she was also
stubborn. But she wasn't difficult. I would go through my own thing ---
``I don't want to do it that way. Why can't I do it this way?'' It was
my show. But she would eventually do it the way she wanted and convince
you to do it that way too. And it was usually the right way.

\textbf{She knew how to get the performance she wanted from you?}

It's not that she was unassuming. You definitely knew she was in charge.
But she embraced the collaborative process. She would watch what you did
in a take and focus on finding the naturalism that's possible. Get you
away from locking into a line reading, enable you to open up your
emotional space and get present in what needs to be done in the scene.
She just had a way of doing it that didn't feel oppressive or wasn't
something I wanted to fight against, come the third time. {[}Laughs{]}

Image

The two at the Vanity Fair Oscar party earlier this year. They kept
their relationship low-profile for the first few months.Credit...Frazer
Harrison/Getty Images

\textbf{You started writing together and she directed you in episodes of
``GLOW.'' Did she also appreciate you as a stand-up?}

She was my best audience. {[}Chokes up{]} There was some sort of
connection that we had --- I lost all my self-consciousness, which is no
easy trick. And I don't even know why. We had similar interests around
food, around music, around humor, around film. Intellectually and
emotionally --- spiritually, I don't know. She meditated twice a day. I
never did that. But she loved to laugh and she was kind of an easy
laugh, but that's OK. I'll take it. She would go to the Comedy Store
with me and, over the years, became acutely sensitive to my process. It
got to the point where she would suggest things and I would take the
note, even with my stand-up, which is saying a lot.

\textbf{So it made sense to have her direct your stand-up specials,
starting with ``Too Real''?}

I told Netflix I wanted her to direct it. They were concerned that she
had never directed a comedy special. I'm like: ``She's done seven
movies! You think this is rocket science?'' She'd already made
\href{https://www.youtube.com/watch?v=s0SdJfplC80}{``Outside In,''} and
that's a little masterpiece. But she was pretty humble. She got what she
wanted as an artist from realizing her vision. Her focus was on the
work. But half of this game, for most of us, is getting it out there and
that's the hardest part.

\textbf{How was Netflix finally persuaded to let her do it?}

I told my manager I won't do the special if she can't direct it. And
they came back with, they're also going to put another director who's
directed specials with her, to guide her through the process. And I said
to my manager, that's not a great look. Especially if they pull out some
dude. You're going to have some guy teach her? That's ridiculous. Just
let her direct it herself. And that's all.

Image

Shelton directed Maron in two comedy specials, including ``Too Real,''
above.Credit...Adam Bettcher/Netflix

\textbf{Was it a different dynamic on ``Sword of Trust,'' which, though
you starred in it, was ultimately her film?}

By that time, we were wrestling with feelings that were deeper than
creative and friendship. She was separated {[}from her husband, the host
and actor Kevin Seal{]} and the tension had become more intense. We'd
been working on a script together for years and we just weren't getting
it done. She's like, ``I'm going to write you a movie.'' I'm like,
``Yeah, OK, sure.'' And she did. She and Mike O'Brien came up with an
idea --- she decided, because she drove past a pawnshop, that I would be
a pawnshop owner --- and she willed it into existence.

\textbf{What was the day-to-day experience like on the film?}

I was cranky and difficult. It was hot down there {[}in Birmingham,
Ala.{]}. I felt fat. It was very daunting for me, because I was on set
with a lot of monster improvisers. And after a day or two, I said to
Lynn, if you don't reel them in, I'm going to be this dumb straight man
to a bunch of over-the-top characters. I'm going to be filled with
resentment and I'm not going to be able to ease into this. And Lynn made
an interesting choice to act in that movie {[}as the troubled
ex-girlfriend of Maron's character{]}. That scene when she comes to pawn
that ring {[}chokes up{]}, the emotions were real. It's really Lynn
holding me open, which is something she did in real life, too. Once it
got to a point where she had resolved some stuff in her relationship and
we were able to acknowledge a real love for each other, I actually said
to her --- there was drama here --- I said, ``Look, if we don't try
this, whatever's going on here, I'm going to regret it for the rest of
my life.'' And so, we did.

\textbf{Were you hesitant to acknowledge your relationship on the
podcast?}

It took a long time and I've learned a lot of lessons around that. My
audience has been through several relationships with me. When you talk
about somebody, they don't have a voice in it. So choose what you're
doing there. Once Lynn and I were able to start seeing each other, we
kept a low profile for like three to six months and then she started to
ease into the conversation. I think she was getting upset with it. She
was like: ``When can we be who we are? Enough already.'' And I'm like,
``Yeah, you're right.'' This quarantine was not the greatest of
situations but it accelerated things. It enabled us to double up our
time together. She had moved all her stuff down here and we were
together all the time. We were cooking and talking and eventually we
started working on the script again.

Image

In ``Sword of Trust,'' Maron was surrounded by Jillian Bell, left, Jon
Bass and Michaela Watkins.Credit...IFC Films

Image

Shelton not only directed ``Sword of Trust'' but also acted in
it.Credit...IFC Films

\textbf{In the time you were together with her, did you see any warnings
that her health was at risk?}

She was not a big doctor person. She was used to a fairly elaborate
daily regimen of supplements and talking to her naturopath. She had
swollen glands in her throat on a Friday. She went and got the Covid
test immediately. It was negative. We got an online appointment with a
physician. She started antibiotics on Saturday. Come Monday, she had a
104-point-something fever. On Wednesday she said her throat didn't hurt
anymore but the fever was relentless. On Thursday, I'm like, we've got
to go to the doctor. She was going to go in that Friday morning to get a
blood panel and that's when I woke up to her collapsed. I called the
ambulance and she was dead within 18 hours. Acute myeloid leukemia is
what's on the death certificate. Organ failure is the primary cause and
then acute myeloid leukemia is what they signed off on.

I went {[}to the hospital{]} that night and spent a few minutes with her
body. It was the heaviest thing I've ever done. It was just devastating.
I was blown out, totally traumatized. Totally heartbroken.

\textbf{You never took a break from your podcast. Did you consider
taking any time off from the show?}

I didn't think I owed it to anybody but myself. This is the type of
thing I do. My producer was like: ``Dude, we don't have to do anything.
You just tell me what you want to do.'' And I said, I have no control
over these feelings. They're monstrous. But they're real. So we posted
\href{http://www.wtfpod.com/podcast/in-memoriam-remembering-lynn-shelton}{the
posthumous episode {[}of Shelton's 2015 interview{]}}. I got on the mic
and stayed in what I was feeling to {[}introduce{]} that episode. And I
thought, this is going to be good for me. The people that have been in
my audience for a decade can handle it. Ultimately, I felt like it was
the right thing.

\textbf{Do you find it at all difficult to ask for people's sympathies
or for acknowledgment of your grief during a global pandemic?}

That's one of the reasons I thought it was good to do it. There's
nothing but grief around. It's a tough emotion for people to sit in and
accept. The one thing the pandemic has given me is time to process and
sit with the feelings. I cry every day. The shock and the trauma have
dissipated a little bit, so now I deal with the loss. I have her jacket
that she always wore, and her hat and boots. I have the shirt that I met
her in. I touch these things when I can and try to keep her with me. But
to answer your question, it's been challenging to be in this much
sadness in a fairly hopeless world. In terms of really experiencing the
feelings that one has with grief and loss, I've had the presence to be
in those. Because I have nothing else to do, man.

\textbf{Having been able to mourn her with other friends, family members
and colleagues, has that affected your understanding of her?}

I didn't know her as well as many people knew her, which is something I
found out at memorial events. My history with her was pretty short and
sweet in a way. Romantically it was just a year and change. There are
people who have known her for 25 years. Hearing everybody's experiences
with her, working on all these films, I'm like, what stories do I have?
But I realized we had a unique frequency to our connection. We saw
ourselves through each other's eyes. I was really the best version of
me, the way she saw me.

Image

For Maron, ``the challenge now is to not get bitter or sad or
angry.''Credit...Erik Carter for The New York Times

\textbf{Will you do anything with the screenplay that the two of you
were working on?}

It's a complicated story. We wrote the draft over a period of years and
there's a lot of funny stuff in it. We just needed to tighten up this
final scene, and it revolves around how this main character dies of
cancer. {[}Pauses{]} Yeah. So. That was the unfinished business. I've
gotten some people who are asking me, like, why don't you finish that
script and make it happen? Maybe find another woman to direct it. I
don't know that it would honor her. Because this was always the thing
that she was going to direct me in. That was the whole idea of it. It
would be upsetting.

\textbf{You said in a recent podcast that your grief for Lynn was
``expanding your sense of what love can be.'' What did you mean by
that?}

I knew that we were just beginning something and I was very excited to
have felt that. I was still difficult --- it wasn't hostile, it was just
kind of childish. But she seemed to understand that. That opened up my
heart aperture a little bit, to experience things differently. Once I'd
gotten to that place, I could take that openness in the world. I didn't
have to be afraid of it anymore. The challenge now is to not get bitter
or sad or angry. How do I not do that? How do you stay loving in
something as relentless as what's going on now? I don't know. I sit on
my porch and it's nice out here. It was a lot to lose. But I just try to
stay in her light as much as I can.

\textbf{You sounded a little more exuberant at the start of}
\textbf{\href{http://www.wtfpod.com/podcast/episode-1143-seth-rogen}{Monday's
show}. Is that an accurate reflection of how you were feeling at the
time?}

Yesterday, when I recorded that, was a fairly awful day. There's a whole
arc of feelings that happen to me every day, and every day in
quarantine's like a week. I've got to wake up and battle the darkness.
I'm going to get up, I'll make my bed. I'm going to look at a picture of
Lynn. I'm going to maybe pray to nothing, because it feels like a
meditation of some kind. A little trick I learned in sobriety. And then
I'm going to wonder if it's worth being alive for a while and wonder if
there's a way to hang myself from the exercise belt that you hook to the
door of the closet. {[}Laughs{]} And then that goes away, and I cook
some breakfast and I go hike up the mountain or listen to some music.
Feelings aren't facts. You should have them but don't act on anything
too rashly. Move through them. So I guess by the time I got to the
garage {[}where he records the podcast{]}, I was having a good 10
minutes and I was able to capture that. {[}Laughs{]}

Advertisement

\protect\hyperlink{after-bottom}{Continue reading the main story}

\hypertarget{site-index}{%
\subsection{Site Index}\label{site-index}}

\hypertarget{site-information-navigation}{%
\subsection{Site Information
Navigation}\label{site-information-navigation}}

\begin{itemize}
\tightlist
\item
  \href{https://help.nytimes3xbfgragh.onion/hc/en-us/articles/115014792127-Copyright-notice}{©~2020~The
  New York Times Company}
\end{itemize}

\begin{itemize}
\tightlist
\item
  \href{https://www.nytco.com/}{NYTCo}
\item
  \href{https://help.nytimes3xbfgragh.onion/hc/en-us/articles/115015385887-Contact-Us}{Contact
  Us}
\item
  \href{https://www.nytco.com/careers/}{Work with us}
\item
  \href{https://nytmediakit.com/}{Advertise}
\item
  \href{http://www.tbrandstudio.com/}{T Brand Studio}
\item
  \href{https://www.nytimes3xbfgragh.onion/privacy/cookie-policy\#how-do-i-manage-trackers}{Your
  Ad Choices}
\item
  \href{https://www.nytimes3xbfgragh.onion/privacy}{Privacy}
\item
  \href{https://help.nytimes3xbfgragh.onion/hc/en-us/articles/115014893428-Terms-of-service}{Terms
  of Service}
\item
  \href{https://help.nytimes3xbfgragh.onion/hc/en-us/articles/115014893968-Terms-of-sale}{Terms
  of Sale}
\item
  \href{https://spiderbites.nytimes3xbfgragh.onion}{Site Map}
\item
  \href{https://help.nytimes3xbfgragh.onion/hc/en-us}{Help}
\item
  \href{https://www.nytimes3xbfgragh.onion/subscription?campaignId=37WXW}{Subscriptions}
\end{itemize}
