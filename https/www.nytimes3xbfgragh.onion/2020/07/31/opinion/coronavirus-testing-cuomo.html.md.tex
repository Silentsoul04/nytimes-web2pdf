Sections

SEARCH

\protect\hyperlink{site-content}{Skip to
content}\protect\hyperlink{site-index}{Skip to site index}

\href{https://myaccount.nytimes3xbfgragh.onion/auth/login?response_type=cookie\&client_id=vi}{}

\href{https://www.nytimes3xbfgragh.onion/section/todayspaper}{Today's
Paper}

\href{/section/opinion}{Opinion}\textbar{}Let's End the Wait for
Coronavirus Test Results. Here's How.

\url{https://nyti.ms/30gEXzB}

\begin{itemize}
\item
\item
\item
\item
\item
\item
\end{itemize}

Advertisement

\protect\hyperlink{after-top}{Continue reading the main story}

\href{/section/opinion}{Opinion}

Supported by

\protect\hyperlink{after-sponsor}{Continue reading the main story}

\hypertarget{lets-end-the-wait-for-coronavirus-test-results-heres-how}{%
\section{Let's End the Wait for Coronavirus Test Results. Here's
How.}\label{lets-end-the-wait-for-coronavirus-test-results-heres-how}}

States should look to New York's strategies.

By Andrew M. Cuomo

Mr. Cuomo is the governor of New York.

\begin{itemize}
\item
  July 31, 2020
\item
  \begin{itemize}
  \item
  \item
  \item
  \item
  \item
  \item
  \end{itemize}
\end{itemize}

\includegraphics{https://static01.graylady3jvrrxbe.onion/images/2020/07/30/opinion/30Cuomo1/merlin_171976521_6137293f-4c6e-4588-8ae0-fdb17e813038-articleLarge.jpg?quality=75\&auto=webp\&disable=upscale}

It's been six months since the United States reported its first
coronavirus case, and getting a test can still take days. National labs
are overwhelmed, leaving people to wait as much as two weeks for
results. Every day that testing falls short is another day the virus can
spread undetected, costing lives and delaying the reopening of our
economy, schools and society.

As states try to control the virus and as Congress considers the fourth
Covid-19 relief bill, New York offers important lessons on how to fix
the testing mess.

Over the last 10 weeks, New York has used testing to not only flatten
the curve, but actually reduce the rate of infection since our phased
reopening started. We have kept our testing rates high through
partnerships with federal and local governments. In February and early
March, New York worked with the Food and Drug Administration to gain the
necessary approvals to begin using our own coronavirus test and mobilize
a network of hundreds of labs. In April, when our labs were struggling
because of shortages of a necessary chemical ingredient, reagents,
President Trump and I reached an agreement that helped double New York's
capacity.

Here's what states should do to build a sustainable testing operation,
and how Congress can help.

\textbf{Mobilize smaller local labs.} Almost all states are now using a
handful of national testing companies, and they are overwhelmed. New
York has managed to avoid the delays because more than 80 percent of our
testing does not depend on the national laboratories experiencing long
turnaround times for results.

In the early days of the pandemic, New York organized hundreds of local
labs to conduct as many tests as possible. We moved equipment sitting
idle to labs that could run them around the clock. Today, more than 250
labs in the state report results each day --- some conducting 10 tests
daily, some thousands. All together, New York can now conduct on average
65,000 tests a day.

And while any lag time is not ideal, over the past week, more than 85
percent of New York's tests took a median of just two days (and an
average of three days) from collection to result, and lags will continue
to shorten as we move tests from labs with backlogs to labs without.

Each state should mobilize its own network of laboratories, which will
take pressure off the major national labs, reduce reporting times and
arm states with data that can help slow the spread of the virus.
Congress should dedicate money to help develop the capacity of local
laboratories and ensure federal agencies can provide speedy approvals
and technical assistance to states.

\textbf{Streamline the supply chain.} In New York and other states,
there are high-capacity labs running at partial capacity because they
don't have enough supplies.

How can it be, six months after America's first case was reported, that
the United States still doesn't have an adequate supply chain? What labs
need --- reagents and plastic pipette tips --- are not complicated to
manufacture. They can, and should, be made in mass quantity,
immediately, and here at home.

New York invested \$750,000 in Rheonix, an Ithaca-based manufacturer, to
build lab instruments and make reagent kits, which are now being used
for thousands of tests daily. States should tap their local
manufacturing companies to compensate for international shortages, and
Congress should allocate funding for businesses that fill these needs.

\textbf{Invest in innovative solutions.} The Food and Drug
Administration recently
\href{https://www.fda.gov/news-events/press-announcements/coronavirus-covid-19-update-fda-issues-first-emergency-authorization-sample-pooling-diagnostic}{approved}
pooled testing, where multiple samples are run at once, increasing
capacity and saving lab supplies. But for one national lab, the approved
pool size is just four samples. In Wuhan, China, up to 10 specimens were
pooled, allowing the city to increase its capacity to 1.5 million tests
daily, up from 46,000 tests daily.

The federal government should direct research money so that labs can
increase their pool size, while ensuring accuracy. With flu season on
the way, Congress and federal agencies should also invest in developing
widely available single tests that can detect multiple respiratory
viruses, including the coronavirus and different types of influenza.

Congress should also invest in developing more tests that can give
results in minutes and that can be administered at workplaces, not just
labs. The F.D.A. has approved only a handful of these devices, and they
are not widely available.

\textbf{Fund all necessary testing.} Currently, under federal rules,
``medically necessary'' testing is free for those with Covid-19
symptoms, as well as asymptomatic people who have been exposed to the
virus.

But states should be able to conduct broad community screening --- 40
percent of infected people are asymptomatic --- to detect the virus and
control its spread. For example, Congress should ensure testing is free
for individuals who attend mass gatherings, regularly ride public
transportation or interact with members of the public at work.

New York is proof that a real testing strategy can control Covid-19. But
our future success depends on other states to do the same --- a virus
anywhere is a virus everywhere.

There can be no economic recovery without each state having a
sustainable testing strategy. New York has already advised other cities,
and we stand ready to help any state or local government replicate our
success.

Andrew M. Cuomo is the governor of New York.

\emph{The Times is committed to publishing}
\href{https://www.nytimes3xbfgragh.onion/2019/01/31/opinion/letters/letters-to-editor-new-york-times-women.html}{\emph{a
diversity of letters}} \emph{to the editor. We'd like to hear what you
think about this or any of our articles. Here are some}
\href{https://help.nytimes3xbfgragh.onion/hc/en-us/articles/115014925288-How-to-submit-a-letter-to-the-editor}{\emph{tips}}\emph{.
And here's our email:}
\href{mailto:letters@NYTimes.com}{\emph{letters@NYTimes.com}}\emph{.}

\emph{Follow The New York Times Opinion section on}
\href{https://www.facebookcorewwwi.onion/nytopinion}{\emph{Facebook}}\emph{,}
\href{http://twitter.com/NYTOpinion}{\emph{Twitter (@NYTopinion)}}
\emph{and}
\href{https://www.instagram.com/nytopinion/}{\emph{Instagram}}\emph{.}

Advertisement

\protect\hyperlink{after-bottom}{Continue reading the main story}

\hypertarget{site-index}{%
\subsection{Site Index}\label{site-index}}

\hypertarget{site-information-navigation}{%
\subsection{Site Information
Navigation}\label{site-information-navigation}}

\begin{itemize}
\tightlist
\item
  \href{https://help.nytimes3xbfgragh.onion/hc/en-us/articles/115014792127-Copyright-notice}{©~2020~The
  New York Times Company}
\end{itemize}

\begin{itemize}
\tightlist
\item
  \href{https://www.nytco.com/}{NYTCo}
\item
  \href{https://help.nytimes3xbfgragh.onion/hc/en-us/articles/115015385887-Contact-Us}{Contact
  Us}
\item
  \href{https://www.nytco.com/careers/}{Work with us}
\item
  \href{https://nytmediakit.com/}{Advertise}
\item
  \href{http://www.tbrandstudio.com/}{T Brand Studio}
\item
  \href{https://www.nytimes3xbfgragh.onion/privacy/cookie-policy\#how-do-i-manage-trackers}{Your
  Ad Choices}
\item
  \href{https://www.nytimes3xbfgragh.onion/privacy}{Privacy}
\item
  \href{https://help.nytimes3xbfgragh.onion/hc/en-us/articles/115014893428-Terms-of-service}{Terms
  of Service}
\item
  \href{https://help.nytimes3xbfgragh.onion/hc/en-us/articles/115014893968-Terms-of-sale}{Terms
  of Sale}
\item
  \href{https://spiderbites.nytimes3xbfgragh.onion}{Site Map}
\item
  \href{https://help.nytimes3xbfgragh.onion/hc/en-us}{Help}
\item
  \href{https://www.nytimes3xbfgragh.onion/subscription?campaignId=37WXW}{Subscriptions}
\end{itemize}
