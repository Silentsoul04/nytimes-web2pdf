Sections

SEARCH

\protect\hyperlink{site-content}{Skip to
content}\protect\hyperlink{site-index}{Skip to site index}

\href{https://myaccount.nytimes3xbfgragh.onion/auth/login?response_type=cookie\&client_id=vi}{}

\href{https://www.nytimes3xbfgragh.onion/section/todayspaper}{Today's
Paper}

\href{/section/opinion}{Opinion}\textbar{}John Lewis Was the Anti-Trump

\url{https://nyti.ms/39Hc9Dz}

\begin{itemize}
\item
\item
\item
\item
\item
\item
\end{itemize}

Advertisement

\protect\hyperlink{after-top}{Continue reading the main story}

\href{/section/opinion}{Opinion}

Supported by

\protect\hyperlink{after-sponsor}{Continue reading the main story}

\hypertarget{john-lewis-was-the-anti-trump}{%
\section{John Lewis Was the
Anti-Trump}\label{john-lewis-was-the-anti-trump}}

The president doesn't seem to know what democracy is; the congressman
embodied it.

\href{https://www.nytimes3xbfgragh.onion/column/jamelle-bouie}{\includegraphics{https://static01.graylady3jvrrxbe.onion/images/2019/01/24/opinion/jamelle-bouie/jamelle-bouie-thumbLarge-v3.png}}

By
\href{https://www.nytimes3xbfgragh.onion/column/jamelle-bouie}{Jamelle
Bouie}

Opinion Columnist

\begin{itemize}
\item
  July 31, 2020
\item
  \begin{itemize}
  \item
  \item
  \item
  \item
  \item
  \item
  \end{itemize}
\end{itemize}

\includegraphics{https://static01.graylady3jvrrxbe.onion/images/2020/07/31/opinion/31bouieNew/merlin_175050603_216037fa-ae00-46ca-901d-574c43de1725-articleLarge.jpg?quality=75\&auto=webp\&disable=upscale}

\hypertarget{listen-to-this-op-ed}{%
\subsubsection{Listen to This Op-Ed}\label{listen-to-this-op-ed}}

Audio Recording by Audm

\emph{To hear more audio stories from publishers like The New York
Times,
download}\href{https://www.audm.com/?utm_source=nytopinion\&utm_medium=embed\&utm_campaign=trump_democracy_lewis}{\emph{Audm
for iPhone or Android}}\emph{.}

In his
\href{https://www.nytimes3xbfgragh.onion/2020/07/30/opinion/john-lewis-civil-rights-america.html}{final
essay}, published on Thursday in The New York Times, Representative John
Lewis of Georgia offered welcome words of encouragement and wisdom for
everyone protesting discrimination and injustice. He also made a crucial
point about our political system, one that bears repeating as we face
powerful threats to self-government and the rule of law.

``Democracy is not a state,'' Lewis wrote. ``It is an act, and each
generation must do its part to help build what we called the Beloved
Community, a nation and world society at peace with itself.''

Americans have lived with democratic institutions for so long that it's
become easy to think of democracy as something that is defined and
embodied by those institutions. But the Constitution and Congress and
elections and courts aren't democracy themselves as much as they're
instruments for its realization. Democracy itself is something larger
and more expansive; it is an ethic, a way of living and, as Lewis wrote,
an \emph{act}, something that you must do in order to summon it into
existence.

I am reminded, by all of this, of John Dewey, the American philosopher
and psychologist who devoted his long career to the explication of life
in a modern industrial democracy and its implications for a wide range
of social and political activity.

In ``The Ethics of Democracy,'' an 1888 essay written while he was
teaching at the University of Michigan, Dewey described his expansive
vision of democracy. Against contemporary skeptics who saw democracy as
little more than simple majority rule by ignorant, isolated individuals,
he argued that we should understand democracy as ``a form of moral and
spiritual association'' that takes ``personality'' --- meaning
individual potential --- as its ``first and final reality.'' Democracy
recognizes the ``infinite and universal possibility'' within each person
and seeks to foster its expression, not for ``mere self-assertion'' or
``unregulated'' desire but for ``an individualism of freedom, of
responsibility, of initiative to and for the ethical ideal.''

For Dewey, democracy was an ethical project for individual and
collective flourishing. And a democratic society was one in which people
could develop their
``\href{https://www.gutenberg.org/files/852/852-h/852-h.htm}{distinctive
capacities}'' to the fullest and then use them for the sake of their
communities.

Of course, Dewey knew that American democracy was far from this ideal.
And to the extent that the United States could be an example to the
world, it was only if it demonstrated progress toward ``securing and
maintaining an ever-increasing release of the powers of human nature, in
service of a freedom which is cooperative and a cooperation which is
voluntary.''

The only way to make this happen, Dewey argued, was to live this
democratic belief in the ``potentialities of every human being'' and
work to ``provide the conditions that will enable these potentialities
to come to realization.'' Decades later, in 1941, as the world battled
fascism, Dewey wrote that democracy ``is a faith which becomes
sentimental when it is not put systematically into practice every day in
all the relationships of living.''

The reason to connect Lewis to Dewey is to highlight and emphasize this
idea of democracy as a social and ethical commitment, something that
cannot be limited to the ballot box, something that must be lived and
practiced in all spheres of life. Marching, speaking, deliberating,
educating, persuading --- these are just some of the actions that help
make democracy real. They're also the tools we'll need to defend
democracy against the looming threat of autocracy.

Just a few hours before Lewis's funeral in Atlanta, President Trump
denounced mail-in voting, in one of his now regular attempts to
delegitimize the upcoming election. He also raised the idea of pushing
the election back, to another date. ``With Universal Mail-In Voting (not
Absentee Voting, which is good), 2020 will be the most INACCURATE \&
FRAUDULENT Election in history,''
\href{https://twitter.com/realdonaldtrump/status/1288818160389558273?s=21}{he
wrote on Twitter}. ``It will be a great embarrassment to the USA. Delay
the Election until people can properly, securely and safely vote???''

There's no legal way the president can delay or postpone the election.
Its date is set by state and federal law and moving it would require a
herculean political effort. Trump lacks the patience or capacity to
coordinate. But that doesn't mean his language isn't dangerous. Trump is
sowing chaos. He's undermining public faith in the election process and
building a constituency of supporters who will treat any result short of
his re-election as evidence of fraud and misconduct. And he's been
backed thus far by an attorney general
\href{https://www.cnn.com/2020/06/25/politics/barr-mail-in-voting-election-fraud-npr/index.html}{who
repeats his false claims} and gives ominously conditional answers to
questions about honoring the democratic process.
\href{https://abcnews.go.com/Politics/barr-testimony-live-updates-ag-faces-questions-2020/story?id=72029139}{Asked}
during a hearing of the House Judiciary Committee on Tuesday what he
would do if Trump loses the election but refuses to concede, Bill Barr
answered, ``If the results are clear I would leave office,'' a response
that leaves open the possibility of unclear results and a contested
outcome.

It's fair to say that over the last three and a half years our
democratic ``norms'' have done little to restrain Trump's most corrupt
and authoritarian instincts. Our ``checks and balances'' have proved
inadequate in the face of a president who sees the Constitution as
merely a few pieces of paper. As we've seen with child separation on the
border or secretive federal police in Portland, Ore., Trump has tried to
extend and expand his authority as much as he can, daring the political
system to stop him each time.

But while many of our institutions have not been up to the task of
confronting Trump, our democracy, meaning individuals and communities
and civil society, has. Protest put Trump on the defensive in the days
after he took office; protest drew attention to his abuses at the
border; and protest over the last three months has helped galvanized
many millions more against him. If Trump is defeated, and if he does
leave office, it will be because Americans understood, and took
seriously, the idea that democracy is a way of living as much as it is a
form of government --- that it is, as Lewis told us, an act and not a
state.

\includegraphics{https://static01.graylady3jvrrxbe.onion/images/2020/08/02/opinion/31bouieOption/31bouieOption-articleLarge.jpg?quality=75\&auto=webp\&disable=upscale}

\emph{The Times is committed to publishing}
\href{https://www.nytimes3xbfgragh.onion/2019/01/31/opinion/letters/letters-to-editor-new-york-times-women.html}{\emph{a
diversity of letters}} \emph{to the editor. We'd like to hear what you
think about this or any of our articles. Here are some}
\href{https://help.nytimes3xbfgragh.onion/hc/en-us/articles/115014925288-How-to-submit-a-letter-to-the-editor}{\emph{tips}}\emph{.
And here's our email:}
\href{mailto:letters@NYTimes.com}{\emph{letters@NYTimes.com}}\emph{.}

\emph{Follow The New York Times Opinion section on}
\href{https://www.facebookcorewwwi.onion/nytopinion}{\emph{Facebook}}\emph{,}
\href{http://twitter.com/NYTOpinion}{\emph{Twitter (@NYTopinion)}}
\emph{and}
\href{https://www.instagram.com/nytopinion/}{\emph{Instagram}}\emph{.}

Advertisement

\protect\hyperlink{after-bottom}{Continue reading the main story}

\hypertarget{site-index}{%
\subsection{Site Index}\label{site-index}}

\hypertarget{site-information-navigation}{%
\subsection{Site Information
Navigation}\label{site-information-navigation}}

\begin{itemize}
\tightlist
\item
  \href{https://help.nytimes3xbfgragh.onion/hc/en-us/articles/115014792127-Copyright-notice}{©~2020~The
  New York Times Company}
\end{itemize}

\begin{itemize}
\tightlist
\item
  \href{https://www.nytco.com/}{NYTCo}
\item
  \href{https://help.nytimes3xbfgragh.onion/hc/en-us/articles/115015385887-Contact-Us}{Contact
  Us}
\item
  \href{https://www.nytco.com/careers/}{Work with us}
\item
  \href{https://nytmediakit.com/}{Advertise}
\item
  \href{http://www.tbrandstudio.com/}{T Brand Studio}
\item
  \href{https://www.nytimes3xbfgragh.onion/privacy/cookie-policy\#how-do-i-manage-trackers}{Your
  Ad Choices}
\item
  \href{https://www.nytimes3xbfgragh.onion/privacy}{Privacy}
\item
  \href{https://help.nytimes3xbfgragh.onion/hc/en-us/articles/115014893428-Terms-of-service}{Terms
  of Service}
\item
  \href{https://help.nytimes3xbfgragh.onion/hc/en-us/articles/115014893968-Terms-of-sale}{Terms
  of Sale}
\item
  \href{https://spiderbites.nytimes3xbfgragh.onion}{Site Map}
\item
  \href{https://help.nytimes3xbfgragh.onion/hc/en-us}{Help}
\item
  \href{https://www.nytimes3xbfgragh.onion/subscription?campaignId=37WXW}{Subscriptions}
\end{itemize}
