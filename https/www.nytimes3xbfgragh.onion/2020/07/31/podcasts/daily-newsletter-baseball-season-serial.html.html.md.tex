Sections

SEARCH

\protect\hyperlink{site-content}{Skip to
content}\protect\hyperlink{site-index}{Skip to site index}

\href{https://www.nytimes3xbfgragh.onion/spotlight/podcasts}{Podcasts}

\href{https://myaccount.nytimes3xbfgragh.onion/auth/login?response_type=cookie\&client_id=vi}{}

\href{https://www.nytimes3xbfgragh.onion/section/todayspaper}{Today's
Paper}

\href{/spotlight/podcasts}{Podcasts}\textbar{}Life Without Sports?

\url{https://nyti.ms/30ivgAJ}

\begin{itemize}
\item
\item
\item
\item
\item
\item
\end{itemize}

\href{https://www.nytimes3xbfgragh.onion/spotlight/at-home?action=click\&pgtype=Article\&state=default\&region=TOP_BANNER\&context=at_home_menu}{At
Home}

\begin{itemize}
\tightlist
\item
  \href{https://www.nytimes3xbfgragh.onion/2020/08/03/well/family/the-benefits-of-talking-to-strangers.html?action=click\&pgtype=Article\&state=default\&region=TOP_BANNER\&context=at_home_menu}{Talk:
  To Strangers}
\item
  \href{https://www.nytimes3xbfgragh.onion/2020/08/01/at-home/coronavirus-make-pizza-on-a-grill.html?action=click\&pgtype=Article\&state=default\&region=TOP_BANNER\&context=at_home_menu}{Make:
  Grilled Pizza}
\item
  \href{https://www.nytimes3xbfgragh.onion/2020/07/31/arts/television/goldbergs-abc-stream.html?action=click\&pgtype=Article\&state=default\&region=TOP_BANNER\&context=at_home_menu}{Watch:
  'The Goldbergs'}
\item
  \href{https://www.nytimes3xbfgragh.onion/interactive/2020/at-home/even-more-reporters-editors-diaries-lists-recommendations.html?action=click\&pgtype=Article\&state=default\&region=TOP_BANNER\&context=at_home_menu}{Explore:
  Reporters' Google Docs}
\end{itemize}

Advertisement

\protect\hyperlink{after-top}{Continue reading the main story}

Supported by

\protect\hyperlink{after-sponsor}{Continue reading the main story}

the daily newsletter

\hypertarget{life-without-sports}{%
\section{Life Without Sports?}\label{life-without-sports}}

It's ``deeply boring,'' our producer reports.

\includegraphics{https://static01.graylady3jvrrxbe.onion/images/2020/07/30/reader-center/author-daniel-guillemette/author-daniel-guillemette-thumbLarge.png}

By Daniel Guillemette

\begin{itemize}
\item
  Published July 31, 2020Updated Aug. 4, 2020, 11:21 a.m. ET
\item
  \begin{itemize}
  \item
  \item
  \item
  \item
  \item
  \item
  \end{itemize}
\end{itemize}

\includegraphics{https://static01.graylady3jvrrxbe.onion/images/2020/07/22/sports/00Baseball-Preview-copy/00Baseball-Preview-articleLarge.jpg?quality=75\&auto=webp\&disable=upscale}

\emph{Our producer and sports enthusiast Daniel Guillemette on}
\href{https://www.nytimes3xbfgragh.onion/2020/07/24/podcasts/the-daily/mlb-baseball-season-coronavirus.html}{\emph{last
Friday's episode}}\emph{:}

When the world moved into quarantine in March, I began to wonder: What
would I do with all the time I normally spent watching, reading, talking
and thinking about my favorite sports? It's been an interesting
experiment to go without them --- and by ``interesting'' I mean deeply
boring.

So it was only natural I'd want to fill the void by telling a sports
story on The Daily. Thankfully,
\href{https://www.nytimes3xbfgragh.onion/by/michael-s-schmidt}{Mike
Schmidt}, a Washington correspondent, was thinking the same thing.

Mike's a regular on The Daily; he has been our guide through all things
politics, including the
\href{https://www.nytimes3xbfgragh.onion/2019/05/30/podcasts/the-daily/mueller-special-counsel.html}{special
counsel investigation} and the
\href{https://www.nytimes3xbfgragh.onion/2019/11/13/podcasts/the-daily/impeachment-hearings.html}{impeachment
hearings}. But Mike's first beat at The Times was baseball. He has a
long history with Rob Manfred, Major League Baseball's commissioner ---
a relationship that began back when Mike was, in his words, an
``overaggressive young reporter'' and Manfred was a labor lawyer for the
M.L.B. It started with them mostly yelling at each other on speaker
phone, Mike said (it was the steroids era --- lots to yell about), but
later developed into a more civil reporter-source relationship.

In March, Mike thought it would be interesting to periodically check in
with Manfred, who was home in Florida, trying to figure out how to make
the baseball season happen. I got to listen in on these calls, along
with a producer, Clare Toensikoetter, and two of our editors, Dave Shaw
and Lisa Tobin.

In our first conversations, Manfred was focused on the health protocols
for coronavirus-era baseball (no high-fives). But as the weeks went on,
Manfred became consumed by salary negotiations with the players' union
--- the primary obstacle, beyond the pandemic, to restarting the season.
A sign of how tense those salary negotiations got? Mike had seen Manfred
in countless stressful situations over the years. But Manfred claimed
this was the first time he had actually \emph{felt} stress in his
professional career.

To learn more about the players' perspectives on the negotiations, we
interviewed Travis Shaw of my beloved Toronto Blue Jays. He started by
telling us that his life without baseball was fairly dull (that day, he
was planning to make dinner at 2 p.m.). Still, he explained why he
didn't want the players' union to back down on the salary negotiations:
The average career for an M.L.B. player is a little less than 6 years,
so for Shaw it was only fair that the players maximized each year they
had.

After months of negotiations between Manfred and the union, the players
got the salary deal they wanted, and Manfred got his season. A few hours
before the first game, Mike got through to Manfred at Nationals Park in
Washington, D.C. In all of the calls I listened in on, I hadn't heard
Manfred in a mood like this before: The guy was downright cautiously
optimistic.

I doubt that mood lasted long, though. Since
\href{https://www.nytimes3xbfgragh.onion/2020/07/24/podcasts/the-daily/mlb-baseball-season-coronavirus.html?action=click\&module=audio-series-bar\&region=header\&pgtype=Article}{our
episode} ran last Friday, at least 18 (18!) players on the Miami Marlins
have tested positive for Covid-19, and the team has suspended play. All
of which, understandably, has
\href{https://www.nytimes3xbfgragh.onion/2020/07/28/sports/baseball/marlins-outbreak-mlb-coronavirus.html}{led
to a lot of unease throughout the M.L.B.} and has put the season in some
doubt. So who knows what'll happen with baseball this year? At least
I've still got the N.B.A. Go Raptors!

\begin{center}\rule{0.5\linewidth}{\linethickness}\end{center}

\hypertarget{introducing-nice-white-parents}{%
\subsection{Introducing Nice White
Parents}\label{introducing-nice-white-parents}}

Image

Credit...The New York Times

We spend most of our time covering the news on The Daily, but this week
our team has some news to share:
\href{https://www.nytimes3xbfgragh.onion/2020/07/22/business/media/new-york-times-serial.html}{Serial
Productions is now a New York Times company}.

We're excited to introduce you to our first podcast from Serial: Nice
White Parents, a five-part series exploring the complicated relationship
between white parents and the public education system.

For the past five years, Chana Joffe-Walt, a reporter and producer, has
been examining inequality in education. In the process, she saw that
most reforms focused on who schools were failing: Black and brown kids.
But what about who the schools are serving? In this show, she turns her
attention to what is arguably the most powerful force in our schools:
White parents.

You can
\href{https://www.nytimes3xbfgragh.onion/2020/07/30/podcasts/nice-white-parents-serial.html}{listen
to the first two episodes} of Nice White Parents now, and if you're
interested in learning more about the history and research behind this
series, here are some of the books Chana recommends.

\hypertarget{on-the-daily-this-week}{%
\subsection{On The Daily this week}\label{on-the-daily-this-week}}

\textbf{Monday:} Surviving the coronavirus in New York
\href{https://www.nytimes3xbfgragh.onion/2020/07/27/podcasts/the-daily/new-york-hospitals-covid.html}{had
a lot to do with which hospital} a patient went to, Brian M. Rosenthal
reports.

\textbf{Tuesday:} Nicholas Fandos takes us inside the
\href{https://www.nytimes3xbfgragh.onion/2020/07/28/podcasts/the-daily/unemployment-benefits-coronavirus.html}{battle
over unemployment benefits} in Congress --- and explains the identity
crisis the fight has created for Republicans.

\textbf{Wednesday:} China and the United States have tried to play nice
for over half a century.
\href{https://www.nytimes3xbfgragh.onion/2020/07/29/podcasts/the-daily/china-trump-foreign-policy.html}{Edward
Wong examines why that status quo is changing now}.

\textbf{Thursday:}
\href{https://www.nytimes3xbfgragh.onion/2020/07/30/podcasts/the-daily/congress-facebook-amazon-google-apple.html}{Are
tech giants too powerful}? Four C.E.O.s were grilled on this question in
congress this week, and Cecilia Kang was in the room.

\textbf{Friday:} Jennifer Steinhauer tells the
\href{https://www.nytimes3xbfgragh.onion/2020/07/31/podcasts/the-daily/vanessa-guillen-military-metoo.html}{story
of 20-year-old Army Specialist Vanessa Guillen} --- and why her death
has incited a \#MeToo reckoning inside the military.

\begin{center}\rule{0.5\linewidth}{\linethickness}\end{center}

\hypertarget{thats-it-for-the-daily-newsletter-see-you-next-week}{%
\subsection{That's it for The Daily newsletter. See you next
week.}\label{thats-it-for-the-daily-newsletter-see-you-next-week}}

\emph{Have thoughts about the show? Tell us what you think at}
\href{mailto:thedaily@NYTimes.com?subject=The\%20Daily\%20newsletter}{\emph{thedaily@NYTimes.com}}\emph{.}

\emph{Were you forwarded this newsletter?}
\href{https://www.nytimes3xbfgragh.onion/newsletters/the-daily?te=1\&nl=the-daily\&emc=edit_dy_20190627}{\emph{Subscribe
here}} \emph{to get it delivered to your inbox.}

\emph{Love podcasts? Join The New York Times Podcast Club on}
\href{https://www.facebookcorewwwi.onion/groups/nytpodcastclub}{\emph{Facebook}}\emph{.}

Advertisement

\protect\hyperlink{after-bottom}{Continue reading the main story}

\hypertarget{site-index}{%
\subsection{Site Index}\label{site-index}}

\hypertarget{site-information-navigation}{%
\subsection{Site Information
Navigation}\label{site-information-navigation}}

\begin{itemize}
\tightlist
\item
  \href{https://help.nytimes3xbfgragh.onion/hc/en-us/articles/115014792127-Copyright-notice}{©~2020~The
  New York Times Company}
\end{itemize}

\begin{itemize}
\tightlist
\item
  \href{https://www.nytco.com/}{NYTCo}
\item
  \href{https://help.nytimes3xbfgragh.onion/hc/en-us/articles/115015385887-Contact-Us}{Contact
  Us}
\item
  \href{https://www.nytco.com/careers/}{Work with us}
\item
  \href{https://nytmediakit.com/}{Advertise}
\item
  \href{http://www.tbrandstudio.com/}{T Brand Studio}
\item
  \href{https://www.nytimes3xbfgragh.onion/privacy/cookie-policy\#how-do-i-manage-trackers}{Your
  Ad Choices}
\item
  \href{https://www.nytimes3xbfgragh.onion/privacy}{Privacy}
\item
  \href{https://help.nytimes3xbfgragh.onion/hc/en-us/articles/115014893428-Terms-of-service}{Terms
  of Service}
\item
  \href{https://help.nytimes3xbfgragh.onion/hc/en-us/articles/115014893968-Terms-of-sale}{Terms
  of Sale}
\item
  \href{https://spiderbites.nytimes3xbfgragh.onion}{Site Map}
\item
  \href{https://help.nytimes3xbfgragh.onion/hc/en-us}{Help}
\item
  \href{https://www.nytimes3xbfgragh.onion/subscription?campaignId=37WXW}{Subscriptions}
\end{itemize}
