Sections

SEARCH

\protect\hyperlink{site-content}{Skip to
content}\protect\hyperlink{site-index}{Skip to site index}

\href{/section/sports/basketball}{Pro Basketball}\textbar{}`The Better I
Got in Sports, the Worse the Racism Got'

\url{https://nyti.ms/33f1NcE}

\begin{itemize}
\item
\item
\item
\item
\item
\end{itemize}

\href{https://www.nytimes3xbfgragh.onion/news-event/george-floyd-protests-minneapolis-new-york-los-angeles?action=click\&pgtype=Article\&state=default\&region=TOP_BANNER\&context=storylines_menu}{Race
and America}

\begin{itemize}
\tightlist
\item
  \href{https://www.nytimes3xbfgragh.onion/interactive/2020/07/03/us/george-floyd-protests-crowd-size.html?action=click\&pgtype=Article\&state=default\&region=TOP_BANNER\&context=storylines_menu}{Black
  Lives Matter Movement}
\item
  \href{https://www.nytimes3xbfgragh.onion/interactive/2020/06/28/us/i-cant-breathe-police-arrest.html?action=click\&pgtype=Article\&state=default\&region=TOP_BANNER\&context=storylines_menu}{History
  of `I Can't Breathe'}
\item
  \href{https://www.nytimes3xbfgragh.onion/interactive/2020/06/10/upshot/black-lives-matter-attitudes.html?action=click\&pgtype=Article\&state=default\&region=TOP_BANNER\&context=storylines_menu}{How
  Public Opinion Shifted}
\item
  \href{https://www.nytimes3xbfgragh.onion/interactive/2020/07/16/us/black-lives-matter-protests-louisville-breonna-taylor.html?action=click\&pgtype=Article\&state=default\&region=TOP_BANNER\&context=storylines_menu}{45
  Days in Louisville}
\end{itemize}

\includegraphics{https://static01.graylady3jvrrxbe.onion/images/2020/08/02/sports/00nba-pattymills-1/merlin_174856419_0f49b851-2eee-49b0-a7c0-ea3c416a8179-articleLarge.jpg?quality=75\&auto=webp\&disable=upscale}

\hypertarget{the-better-i-got-in-sports-the-worse-the-racism-got}{%
\section{`The Better I Got in Sports, the Worse the Racism
Got'}\label{the-better-i-got-in-sports-the-worse-the-racism-got}}

Patty Mills, a key reserve for the San Antonio Spurs, said he developed
an emotional shield from a lifetime of racial abuse, but he is using the
platform of the N.B.A.'s restart to speak out against racism.

Credit...Reginald Thomas II/San Antonio Spurs

Supported by

\protect\hyperlink{after-sponsor}{Continue reading the main story}

\href{https://www.nytimes3xbfgragh.onion/by/marc-stein}{\includegraphics{https://static01.graylady3jvrrxbe.onion/images/2018/06/14/multimedia/author-marc-stein/author-marc-stein-thumbLarge.png}}

By \href{https://www.nytimes3xbfgragh.onion/by/marc-stein}{Marc Stein}

\begin{itemize}
\item
  July 31, 2020
\item
  \begin{itemize}
  \item
  \item
  \item
  \item
  \item
  \end{itemize}
\end{itemize}

LAKE BUENA VISTA, Fla. --- There were a few must-pack accessories for
Patty Mills of the San Antonio Spurs for his stay at Walt Disney World.
Mills brought a stack of books on Black history in the United States and
his native Australia. He made sure to carefully transport his favorite
flat-brimmed hat, which bears two flags representing Australia's
Indigenous populations.

Mills also arrived for the restart of the N.B.A. season with what he
referred to as ``my shield'' --- his internal defense mechanism to ward
off hurtful words and actions. The shield, he said, is a byproduct of a
lifetime of racial abuse that began on his first day of kindergarten,
soon after Mills's parents had moved to the Australian capital of
Canberra to take jobs with the federal government.

``I'm the only Black kid in the room,'' Mills said of that first day.
``It didn't take long before the biggest kid in the room walked up to me
and threw a straight uppercut to the guts, completely knocking the wind
out of me and leaving me in all sorts of tears.''

``I saw this boy coming from the left, and he came from a fair
distance,'' said his mother, Yvonne Mills, who was against a wall
nearby, observing alongside other parents. ``I can still feel the punch
in my stomach, too.''

More than 25 years later, with a slew of similar stories to tell from
throughout his life, Mills trusts his shield as much as his jump shot.
That is no small thing given Mills's ability to produce instant offense
off the bench, which has enabled him, as a 6-foot, 180-pound guard, to
last for nine seasons as a trusty change-of-pace option for the Spurs
and become one of the most feared scorers in international basketball.

\includegraphics{https://static01.graylady3jvrrxbe.onion/images/2020/07/30/sports/00nba-pattymills-7-a/merlin_175123713_3f8087fd-a0fb-4b97-b496-ef394ec4513d-articleLarge.jpg?quality=75\&auto=webp\&disable=upscale}

``A lot of things that are said just bounce off me because of the shield
I've created,'' Mills said. ``I just need to work out the appropriate
times to lower it, or when to take it off completely.''

This moment is one of those times. Mills, 31, has joined the global push
to focus on social matters as much as his basketball job in Florida will
allow --- even if that means revealing painful tales from the past. He
has been finding his voice as an activist in recent years and pledged to
join the many N.B.A. players who are determined to use the platform of
the league's rebooted season to fight against racism and police
brutality.

``It's the same battle on two continents,'' Mills said, referring to his
home nation several time zones away.

The Spurs may play as few as eight games at Disney World because they
are a long shot to make the playoffs, which they have not missed since
1997. Yet Mills has ensured that his time here will resonate no matter
how short. He is donating his remaining salary of about \$1 million to
Black Lives Matter Australia, Black Deaths in Custody and a new campaign
--- We Got You --- he helped launch to show support for athletes as they
fight racism in Australian sport.

After participating in the first game of the N.B.A. restart on Thursday
night, Utah's Donovan Mitchell mentioned Mills and Jrue Holiday of the
New Orleans Pelicans as emerging leaders of the N.B.A.'s social justice
movement. He lauded both for agreeing to donate the remainder of their
salaries to Black causes and said players who are speaking out are ``not
really asking for permission.''

``Given the unfortunate events that have happened in this country, we
have the ears of people,'' Mills said, referring to the deaths of George
Floyd and Breonna Taylor, who were killed by the police. ``For the first
time in my career, I've had teammates, old teammates, coaches, old
coaches, even old friends --- the almost universal question they ask is:
`What can I do to help?' Just by so many people saying that, I feel like
there is a tide change.''

He formed We Got You with both Black and white athletes in part because
campaigning against racism, Mills said, is much tougher for Black
athletes in Australia. Adam Goodes, a former Australian Rules football
star, was an anti-racism advocate who in 2015 began to be relentlessly
booed by opposing fans. The backlash ultimately prompted Goodes, at 35,
to retire suddenly, one year after he had won the country's Australian
of the Year award for national good citizenship.

Image

Mills said he wants to give aspiring Indigenous basketball players in
Australia someone to emulate.Credit...Matthew Adekponya

Of course, as Mills has also learned, speaking out invariably leads to
the sort of storytelling that requires ``putting yourself in a
vulnerable position because you have to relive traumatic experiences.''
Shield down.

Mills's mother is Aboriginal, and his father is from the Torres Strait
Islands. The two flags on the hat Patty Mills brought with him to
Florida represent his two identities.

In a one-hour conversation after a recent practice, Mills shared some of
the names he was called during his childhood, including ``darkie,''
``blackie,'' ``petrol sniffer,'' ``monkey,'' ``chimp,'' ``abo'' (a
derogatory term short for Aboriginal) and other disparaging terms that
he was called ``regularly at school or on the sporting fields.''

``The better I got in sports,'' Mills said, ``the worse the racism
got.''

The Mills family moved to Canberra because his parents got jobs working
in Aboriginal affairs for the government. ``It was a bit like going to
Washington,'' said Benny Mills, Patty's father.

Yet leaving their home on Thursday Island in Torres Strait --- where,
Patty said, everyone ``looked like me and spoke like me'' --- landed him
in that kindergarten classroom where he was first punched.

``It was the very beginning of how I was going to be treated for the
rest of my time at school, not only by students but, more appallingly,
by teachers and principals,'' Mills said.

Within a few years, when Mills was 9, his parents began explaining the
traumatic past of his mother, Yvonne Mills. One of five siblings born to
a white man and an Aboriginal woman, Yvonne and the other four children
were taken from their mother, Gladys Haynes, in 1949 after their parents
had separated. Yvonne, the youngest, was 2 years old. The children were
moved to group homes as wards of the state and sent to separate foster
families in a government-sponsored social engineering program designed,
in effect, to assimilate Aboriginal children into white society.

Throughout their childhoods, Yvonne and her siblings were told that
their mother did not want them. The falsehoods were exposed by a
government inquiry in the mid-1990s, which confirmed decades of human
rights violations that made Yvonne part of what became known as
Australia's ``Stolen Generations'' --- although she said she did not
receive a written acknowledgment of such status from the South
Australian government until 2018. Yvonne had virtually no contact with
her mother between the ages of 2 and 17; Haynes died in 1979.

Image

In his youth, Mills's instinct was to ``let my game do the talking.'' In
adulthood, he is trying, like his parents did, to more forcefully
influence change.Credit...Thomas Peter/Reuters

Patty Mills said he can still picture the car ride during which this was
first discussed. Patty was sitting behind his mother in the left-side
passenger seat of his parents' white Toyota Corolla and got out of the
car when they arrived at their home, before Yvonne could even open the
door.

``I remember looking down at my arms and the back of my calves as she's
getting out of the car,'' Mills said. ``I look up to her and I say, `So
does this mean they will come and take me away, too?'

``I remember her answer very vividly,'' he said, recalling his mother's
words: `` `Your dad is going to stand on that big rock in the driveway
with a big stick. And no one is going to come anywhere near you.' ''

Learning about his mother's torment, Mills said, was ``a turning
point.'' His athletic talent was already blossoming in the basketball
and social club his parents founded in Canberra for Indigenous
Australians called Shadows, but Mills said that was when he began to
realize ``why I''m being treated differently at school.''

In his youth, Mills's instinct was to ``let my game do the talking.'' In
adulthood, he is trying, like his parents did, to more forcefully
influence change. Just staying visible, Mills said, is a big part of it
--- to give aspiring Indigenous basketball players in Australia someone
to emulate. Mills was only the third Black Australian to represent the
country in basketball at the Olympics, joining Michael Ah Matt (1964)
and Mills's uncle Danny Morseu (1980 and 1984).

``My Uncle Danny played 30 years before I did,'' Mills said. ``I don't
want another 30 years to go by before another Indigenous Australian
plays for Australia.''

He has thrust himself into anti-racism causes across sports in Australia
with the support of Spurs Coach Gregg Popovich, whose bond with Mills
was cemented on the eve of the 2014 N.B.A. finals. Popovich stunned
Mills when he began a crucial practice by
\href{https://www.si.com/nba/2015/03/24/patty-mills-australia-san-antonio-spurs-bala-gregg-popovich}{introducing
the story of Eddie Mabo} to the team. Mabo, who was Mills's great-uncle,
is revered by Indigenous Australians to such a degree that his landmark
campaign for their land ownership rights has long prompted calls for a
national holiday in his name.

Image

San Antonio Spurs Coach Gregg Popovich described Mills as ``a great
citizen of the world.''Credit...David Zalubowski/Associated Press

``It's been very satisfying to watch Patty grow into someone who is much
more interested in our world than basketball,'' Popovich said. ``He has
grown into a great citizen of the world. And Patty's story is pretty
important and very timely, because normally all we think about is the
race problem here. It's in many places.''

Mills said he feels fortunate to play for a franchise and a coach ---
and in a league --- that encourages him to ``speak out on these
things.'' The Spurs, he said, urge him ``to continue to show who I am as
a Black Australian.''

Advertisement

\protect\hyperlink{after-bottom}{Continue reading the main story}

\hypertarget{site-index}{%
\subsection{Site Index}\label{site-index}}

\hypertarget{site-information-navigation}{%
\subsection{Site Information
Navigation}\label{site-information-navigation}}

\begin{itemize}
\tightlist
\item
  \href{https://help.nytimes3xbfgragh.onion/hc/en-us/articles/115014792127-Copyright-notice}{©~2020~The
  New York Times Company}
\end{itemize}

\begin{itemize}
\tightlist
\item
  \href{https://www.nytco.com/}{NYTCo}
\item
  \href{https://help.nytimes3xbfgragh.onion/hc/en-us/articles/115015385887-Contact-Us}{Contact
  Us}
\item
  \href{https://www.nytco.com/careers/}{Work with us}
\item
  \href{https://nytmediakit.com/}{Advertise}
\item
  \href{http://www.tbrandstudio.com/}{T Brand Studio}
\item
  \href{https://www.nytimes3xbfgragh.onion/privacy/cookie-policy\#how-do-i-manage-trackers}{Your
  Ad Choices}
\item
  \href{https://www.nytimes3xbfgragh.onion/privacy}{Privacy}
\item
  \href{https://help.nytimes3xbfgragh.onion/hc/en-us/articles/115014893428-Terms-of-service}{Terms
  of Service}
\item
  \href{https://help.nytimes3xbfgragh.onion/hc/en-us/articles/115014893968-Terms-of-sale}{Terms
  of Sale}
\item
  \href{https://spiderbites.nytimes3xbfgragh.onion}{Site Map}
\item
  \href{https://help.nytimes3xbfgragh.onion/hc/en-us}{Help}
\item
  \href{https://www.nytimes3xbfgragh.onion/subscription?campaignId=37WXW}{Subscriptions}
\end{itemize}
