Sections

SEARCH

\protect\hyperlink{site-content}{Skip to
content}\protect\hyperlink{site-index}{Skip to site index}

\href{https://www.nytimes3xbfgragh.onion/section/technology}{Technology}

\href{https://myaccount.nytimes3xbfgragh.onion/auth/login?response_type=cookie\&client_id=vi}{}

\href{https://www.nytimes3xbfgragh.onion/section/todayspaper}{Today's
Paper}

\href{/section/technology}{Technology}\textbar{}Amazon Wins Without Even
Trying

\href{https://nyti.ms/39H2xbY}{https://nyti.ms/39H2xbY}

\begin{itemize}
\item
\item
\item
\item
\item
\end{itemize}

Advertisement

\protect\hyperlink{after-top}{Continue reading the main story}

Supported by

\protect\hyperlink{after-sponsor}{Continue reading the main story}

on tech

\hypertarget{amazon-wins-without-even-trying}{%
\section{Amazon Wins Without Even
Trying}\label{amazon-wins-without-even-trying}}

As America's economy suffers, Big Tech does shockingly well.

\includegraphics{https://static01.graylady3jvrrxbe.onion/images/2020/08/03/business/01ontech1-print/31ontech-BOXESvideostill-threeByTwoMediumAt2X.png}

\href{https://www.nytimes3xbfgragh.onion/by/shira-ovide}{\includegraphics{https://static01.graylady3jvrrxbe.onion/images/2020/03/18/reader-center/author-shira-ovide/author-shira-ovide-thumbLarge-v2.png}}

By \href{https://www.nytimes3xbfgragh.onion/by/shira-ovide}{Shira Ovide}

\begin{itemize}
\item
  July 31, 2020
\item
  \begin{itemize}
  \item
  \item
  \item
  \item
  \item
  \end{itemize}
\end{itemize}

\emph{This article is part of the On Tech newsletter. You can}
\href{https://www.nytimes3xbfgragh.onion/newsletters/signup/OT}{\emph{sign
up here}} \emph{to receive it weekdays.}

Three months ago, the Amazon chief executive Jeff Bezos effectively
declared that his company would try to lose money. Instead, Amazon
declared on Thursday the
\href{https://www.nytimes3xbfgragh.onion/live/2020/07/30/business/stock-market-today-coronavirus/amazons-earnings-double-as-sales-surge}{largest
profit in its history}.

It was a bit awkward.

Companies are supposed to make money, for sure. But this comes at a
moment when politicians and the public are wondering if America's
digital superstars are so powerful --- and perhaps, tilt the game to
their advantage --- that they simply can't be beaten.

A company like Amazon planning to lose money and instead making billions
of dollars in profit is a pretty compelling sign of dominance.

This week in technology made me think of
\href{https://www.nytimes3xbfgragh.onion/1992/11/06/business/is-gm-fate-still-crucial-to-us.html}{that
old line} about a once dominant car company: What's good for the United
States was good for General Motors, and what was good for GM was good
for the country. (There's a debate about what the GM executive meant by
this, but it's still a good line. Stay with me.)

The bosses of four of America's tech giants,
\href{https://www.nytimes3xbfgragh.onion/2020/07/29/technology/big-tech-hearing-apple-amazon-facebook-google.html}{dragged
(virtually) in front of Congress this week}, said some version of that
old saw. They said that their successes are uniquely American, and that
their companies enrich the country and the lives of people who live in
it.

That's true. It is, however, hard to ignore that the fortunes of the
country and its leading corporate citizens are currently going in
opposite directions.

We learned on Thursday that the United States
\href{https://www.nytimes3xbfgragh.onion/2020/07/30/business/economy/q2-gdp-coronavirus-economy.html}{wiped
out five years of economic growth in a matter of months}, as my
colleague Ben Casselman put it. During that period, Amazon, Apple,
Google and Facebook
\href{https://www.nytimes3xbfgragh.onion/2020/07/30/technology/tech-company-earnings-amazon-apple-facebook-google.html}{mostly
raked in money hand over fist}.

Mostly, this makes sense. During a pandemic, we have needed the products
and services these companies provide. That does not, however, guarantee
them financial success.

\emph{(Read more: Last year, my colleague Kashmir Hill wrote about}
\href{https://gizmodo.com/c/goodbye-big-five}{\emph{trying and mostly
failing}} \emph{to cut the five big U.S. technology companies out of her
life. Now, Kash is reflecting on}
\href{https://www.nytimes3xbfgragh.onion/2020/07/31/technology/blocking-the-tech-giants.html}{\emph{what
she learned from that experiment}}\emph{.)}

Facebook's Mark Zuckerberg said a few months ago that the way his
company makes money --- selling ads to a local bakery or an online
luggage maker ---
\href{https://www.nytimes3xbfgragh.onion/2020/05/01/technology/coronavirus-big-tech-earnings.html}{tends
to naturally rise and fall} in tune with the economy. That's generally
true, but not right now. The economy is tanking at its worst rate in
many decades. Facebook's advertising sales are fine.

What has been bad for the United States hasn't yet been bad for Big
Tech. Is, then, what's good for Big Tech good for the country? I'm not
sure.

There's an
\href{https://www.oreilly.com/radar/gradually-then-suddenly/\#:~:text=There's\%20a\%20passage\%20in\%20Ernest,happens\%20much\%20the\%20same\%20way.}{axiom}
in technology that change happens gradually, then suddenly. Tech
companies can seem unbeatable until they aren't --- often because of
some rapid evolutionary change. It happened to Nokia and Sun
Microsystems --- whose old headquarters was taken over by Facebook in a
symbol of one empire replacing a crumbled one.

So could there be a Fall of Rome moment for today's tech superpowers?
Yes, in theory, and we might never see it coming. Right now, though,
despite broader economic pains and a growing backlash to their power,
these four American tech superpowers appear to be as close to
invulnerable as you can get.

\hypertarget{your-take}{%
\subsubsection{Your Take}\label{your-take}}

We've spent a lot of time this week talking about the congressional
antitrust hearing and potential abuses of power by Big Tech. We want to
change things up a bit and hear from our readers.

Tell us about one tech invention of the past decade that makes your life
fabulous, or at least easier, and why.

A reader in Allentown, Pa., Arthur Weinrach, inspired us, writing in to
mention the many technological changes that he's grateful for, including
the E-Z Pass.

Tell us yours at ontech@nytimes. Please include your name and location.
We will publish a selection of them.

\begin{center}\rule{0.5\linewidth}{\linethickness}\end{center}

\hypertarget{what-didnt-get-attention-at-the-big-tech-hearing}{%
\subsection{What didn't get attention at the Big Tech
hearing}\label{what-didnt-get-attention-at-the-big-tech-hearing}}

The antitrust code was written to tackle railroads and steel companies
that grew strong enough to raise their prices at will.

A hot conversation in legal scholarship is whether those laws apply to
Google, Facebook and other companies that offer many products for no
(monetary) cost to us. (My colleague Cecilia Kang
\href{https://www.nytimes3xbfgragh.onion/2020/07/30/podcasts/the-daily/congress-facebook-amazon-google-apple.html}{talked
about this on The Daily.})

There are, however, at least a couple of examples in which tech
companies are being accused of behavior that has led to higher sticker
prices for us. In other words, there are conventional,
railroad-baron-type antitrust claims against the tech giants, too.

These instances didn't get much of an airing during the congressional
hearing this week into tech company power, but they're worth paying
attention to.

One issue involves Apple's App Store. A lawsuit that is
\href{https://www.nytimes3xbfgragh.onion/2019/05/13/us/politics/supreme-court-antitrust-apple.html}{winding
its way through U.S. courts} claims that Apple's commission of as much
as 30 percent on digital app transactions makes all iPhone apps more
expensive than they would be without Apple's monopoly over iPhone app
distribution.

Another involves Amazon's marketplace. Some merchants have said that
Amazon
\href{https://www.nytimes3xbfgragh.onion/2019/12/19/technology/amazon-sellers.html}{punishes
them} if they list what they sell on Amazon for lower prices on
Walmart.com or other spots. Those sellers claim that Amazon is in
essence
\href{https://www.bloomberg.com/news/articles/2019-08-05/amazon-is-squeezing-sellers-that-offer-better-prices-on-walmart}{pushing
up the prices} on products on competitor's shopping sites.

Members of Congress didn't ask Apple and Amazon about these allegations,
and the companies have previously denied them.

Tim Wu, a professor at Columbia Law School and a
\href{https://www.nytimes3xbfgragh.onion/by/tim-wu}{contributing Opinion
writer} for The New York Times, told me that he believed those price
claims were the strongest potential antitrust case against Amazon on
legal grounds.

He said, though, that there's a distinction between ``technical
antitrust and public opinion antitrust.'' Intricate discussions about
price setting are boring in congressional hearings.

\emph{If you don't already get this newsletter in your inbox,}
\href{https://www.nytimes3xbfgragh.onion/newsletters/signup/OT}{\emph{please
sign up here}}\emph{.}

\begin{center}\rule{0.5\linewidth}{\linethickness}\end{center}

\hypertarget{before-we-go-}{%
\subsection{Before we go \ldots{}}\label{before-we-go-}}

\begin{itemize}
\item
  \textbf{Europe vs. Big Tech:} The European Union and some of its
  member countries have been relatively aggressive in suing America's
  tech giants and restricting them through new laws. But, as my
  colleague Adam Satariano writes, there's a belief that those tactics
  haven't been effective, and now officials in Europe are drafting
  several new laws and regulations that
  \href{https://www.nytimes3xbfgragh.onion/2020/07/30/technology/europe-new-phase-tech-amazon-apple-facebook-google.html}{aim
  at the heart of how the U.S. digital stars operate}.
\item
  \textbf{Bond with your co-workers by robbing a (virtual) bank:} Bored
  by Zoom calls for work? My colleague David Segal has a
  \href{https://www.nytimes3xbfgragh.onion/2020/07/31/business/hey-you-free-on-friday-for-a-meeting-and-a-bank-heist.html}{fun
  look} at people holding business meetings and work bonding sessions in
  Minecraft, Grand Theft Auto and other video games. Just don't get
  killed by zombies on your lunch break.
\item
  \textbf{Seven. Billion. Video. Views:} If you have kids, they have
  probably watched the slightly unnerving YouTube videos released by
  CoComelon and Blippi, two giants of children's entertainment. Both are
  \href{https://www.bloomberg.com/news/articles/2020-07-30/blippi-cocomelon-purchased-by-kids-media-giant-moonbug}{now
  part of a single empire} whose YouTube videos generate more than seven
  billion views each month, Bloomberg News writes. Children's
  programming is among the most popular destinations on YouTube, which
  has made some parents and children's advocates uncomfortable.
\end{itemize}

\hypertarget{hugs-to-this}{%
\subsubsection{Hugs to this}\label{hugs-to-this}}

Here is a
\href{https://twitter.com/kwatt/status/1285728300128665601}{cat eating
corn on the cob} --- rather elegantly, I think.

\begin{center}\rule{0.5\linewidth}{\linethickness}\end{center}

\emph{We want to hear from you. Tell us what you think of this
newsletter and what else you'd like us to explore. You can reach us at}
\href{mailto:ontech@NYTimes.com?subject=On\%20Tech\%20Feedback}{\emph{ontech@NYTimes.com.}}
**

\emph{If you don't already get this newsletter in your inbox,}
\href{https://www.nytimes3xbfgragh.onion/newsletters/signup/OT}{\emph{please
sign up here}}\emph{.}

Advertisement

\protect\hyperlink{after-bottom}{Continue reading the main story}

\hypertarget{site-index}{%
\subsection{Site Index}\label{site-index}}

\hypertarget{site-information-navigation}{%
\subsection{Site Information
Navigation}\label{site-information-navigation}}

\begin{itemize}
\tightlist
\item
  \href{https://help.nytimes3xbfgragh.onion/hc/en-us/articles/115014792127-Copyright-notice}{©~2020~The
  New York Times Company}
\end{itemize}

\begin{itemize}
\tightlist
\item
  \href{https://www.nytco.com/}{NYTCo}
\item
  \href{https://help.nytimes3xbfgragh.onion/hc/en-us/articles/115015385887-Contact-Us}{Contact
  Us}
\item
  \href{https://www.nytco.com/careers/}{Work with us}
\item
  \href{https://nytmediakit.com/}{Advertise}
\item
  \href{http://www.tbrandstudio.com/}{T Brand Studio}
\item
  \href{https://www.nytimes3xbfgragh.onion/privacy/cookie-policy\#how-do-i-manage-trackers}{Your
  Ad Choices}
\item
  \href{https://www.nytimes3xbfgragh.onion/privacy}{Privacy}
\item
  \href{https://help.nytimes3xbfgragh.onion/hc/en-us/articles/115014893428-Terms-of-service}{Terms
  of Service}
\item
  \href{https://help.nytimes3xbfgragh.onion/hc/en-us/articles/115014893968-Terms-of-sale}{Terms
  of Sale}
\item
  \href{https://spiderbites.nytimes3xbfgragh.onion}{Site Map}
\item
  \href{https://help.nytimes3xbfgragh.onion/hc/en-us}{Help}
\item
  \href{https://www.nytimes3xbfgragh.onion/subscription?campaignId=37WXW}{Subscriptions}
\end{itemize}
