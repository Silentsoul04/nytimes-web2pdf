Sections

SEARCH

\protect\hyperlink{site-content}{Skip to
content}\protect\hyperlink{site-index}{Skip to site index}

\href{https://www.nytimes3xbfgragh.onion/section/technology}{Technology}

\href{https://myaccount.nytimes3xbfgragh.onion/auth/login?response_type=cookie\&client_id=vi}{}

\href{https://www.nytimes3xbfgragh.onion/section/todayspaper}{Today's
Paper}

\href{/section/technology}{Technology}\textbar{}William English, Who
Helped Build the Computer Mouse, Dies at 91

\url{https://nyti.ms/3hVKD8d}

\begin{itemize}
\item
\item
\item
\item
\item
\end{itemize}

Advertisement

\protect\hyperlink{after-top}{Continue reading the main story}

Supported by

\protect\hyperlink{after-sponsor}{Continue reading the main story}

\hypertarget{william-english-who-helped-build-the-computer-mouse-dies-at-91}{%
\section{William English, Who Helped Build the Computer Mouse, Dies at
91}\label{william-english-who-helped-build-the-computer-mouse-dies-at-91}}

He was one of the computing pioneers who ``showed what a computer
interface could --- and should --- look like,'' a colleague said.

\includegraphics{https://static01.graylady3jvrrxbe.onion/images/2020/08/02/obituaries/02English-obit1/merlin_175064223_d65d496f-ad90-40fe-bdb3-fb8df215ffc1-articleLarge.jpg?quality=75\&auto=webp\&disable=upscale}

\href{https://www.nytimes3xbfgragh.onion/by/cade-metz}{\includegraphics{https://static01.graylady3jvrrxbe.onion/images/2018/11/26/multimedia/author-cade-metz/author-cade-metz-thumbLarge.png}}

By \href{https://www.nytimes3xbfgragh.onion/by/cade-metz}{Cade Metz}

\begin{itemize}
\item
  Published July 31, 2020Updated Aug. 4, 2020, 2:14 p.m. ET
\item
  \begin{itemize}
  \item
  \item
  \item
  \item
  \item
  \end{itemize}
\end{itemize}

William English, the engineer and researcher who helped build the first
computer mouse and, in 1968, orchestrated an elaborate demonstration of
the technology that foretold the computers, tablets and smartphones of
today, died on July 26 in San Rafael, Calif. He was 91.

His death, at a medical facility, was confirmed by his wife, Roberta
English, who said the cause was respiratory failure.

In the late 1950s, after leaving a career in the Navy, Mr. English
joined a Northern California research lab called the Stanford Research
Institute, or S.R.I. (now known as \href{https://www.sri.com/}{SRI
International}). There he met Douglas Engelbart, a fellow engineer who
hoped to build a new kind of computer.

At a time when only specialists used computers, entering and retrieving
information through punched cards, typewriters and printouts, Mr.
Engelbart envisioned a machine that anyone could use simply by
manipulating images on a screen. It was a concept that would come to
define the information age, but by his own admission Mr. Engelbart had
struggled to explain his vision to others.

\includegraphics{https://static01.graylady3jvrrxbe.onion/images/2020/08/02/obituaries/02English-obit2/31English-02-articleLarge.jpg?quality=75\&auto=webp\&disable=upscale}

Mr. English, known to everyone as Bill, was one of the few who
understood these ideas and who had the engineering talent, patience and
social skills needed to realize them. ``He was the guy who made
everything happen,'' said Bill Duvall, who worked alongside Mr. English
during those years. ``If you told him something needed to be done, he
figured out how to do it.''

Among other things, Mr. Engelbart,
\href{https://www.nytimes3xbfgragh.onion/2013/07/04/technology/douglas-c-engelbart-inventor-of-the-computer-mouse-dies-at-88.html\#:~:text=Engelbart\%20died\%20on\%20Tuesday\%20at,the\%20cause\%20was\%20kidney\%20failure.}{who
died in 2013 at 88}, envisioned a mechanical device that could move a
cursor across a screen and perform discrete tasks by selecting
particular symbols or images. Mr. English made this a reality, building
the first computer mouse and, through a series of tests, showing that it
could navigate a screen faster than any other device developed at S.R.I.

Their multifaceted experimental computer was called oNLine System, or
NLS, and on Dec. 9, 1968, they unveiled it at an event in San Francisco
that became known as ``The Mother of All Demos.''

As Mr. Engelbart demonstrated the machine onstage at the Civic
Auditorium, a live video appeared on the wall behind him showing the
seamless interaction between his mouse and the computer screen. Mr.
English directed this elaborate production from the back of the
auditorium, relying on cameras and microphones both there and at the lab
that housed the computer in Menlo Park, Calif., more than 30 miles away.

This Mother of All Demos --- showing early forms of online text editing,
video conferencing and ``hypertext,'' the links now used to navigate web
pages on the internet --- presaged not only the desktop and laptop
computers that rose to the fore in the 1980s and '90s, but also the
smartphones and tablets that would come to suffuse everyday life.

``It showed what a computer interface could --- and should --- look
like,'' said Doug Fairbairn, a director of the Computer History Museum
in Mountain View, Calif. He worked alongside Mr. English and many other
computing pioneers in the 1970s.

Image

Mr. English in an undated photo. In 1968, he led a demonstration of a
multifaceted experimental computer in a famous San Francisco event that
came to be called ``The Mother of All Demos.'' Credit...via English
family

William Kirk English was born on Jan. 27, 1929, in Lexington, Ky., the
only son of Harry and Caroline (Gray) English. His father was an
electrical engineer who managed coal mines in eastern Kentucky and West
Virginia; his mother was a homemaker. His father had two other sons,
John and Robert, from a previous marriage.

In the early 1940s, after the outbreak of the Second World War, Mr.
English left for Arizona to attend a boarding school --- what was called
a ``ranch school,'' where he learned to ride horses as well --- near
Tucson. After returning home, he studied electrical engineering at the
University of Kentucky.

His time in the Navy included postings in Northern California and Japan.
He then took his research position at the Stanford Research Institute,
at first working on a new kind of computer memory --- a rotating metal
drum the size of a desk that could store as many as three pages of text
--- before embracing the project that became NLS.

After Mr. Engelbart had envisaged the computer mouse and drawn a rough
sketch of it on a notepad, Mr. English built it in the mid-1960s. Housed
inside a small pinewood case, the device consisted of two electrical
mechanisms, called potentiometers, that tracked the movement of two
small wheels as they moved across a desktop. They called it a mouse
because of the way the computer's on-screen cursor, called a CAT, seemed
to chase the device's path.

As they were developing the system, both Mr. English and Mr. Engelbart
were part of the government-funded L.S.D. tests conducted by a nearby
lab called the International Foundation of Advanced Study. Both took the
psychedelic as part of a sweeping effort to determine whether it could
``open the mind'' and foster creativity.

Though Mr. Engelbart oversaw the NLS project, the 1968 demonstration in
San Francisco was led by Mr. English, who brought both engineering and
theater skills to the task. In the mid-1950s he had volunteered as a
stage manager for a Bay Area theater troupe called The Actor's Workshop.

For the San Francisco event, he used a video projector the size of a
Volkswagen Beetle (borrowed it from a nearby NASA lab) to arrange and
project the live images behind Mr. Engelbart as he demonstrated NLS from
the stage. He had been able to set up the wireless link that sent video
between the Menlo Park computer lab and the auditorium after befriending
a telephone company technician.

Image

Mr. English helped orchestrate an elaborate demonstration of the
technology that foretold the computers, tablets and smartphones of
today.Credit...via English family

Three years after the demonstration, Mr. English left S.R.I. and joined
a new Xerox lab called the \href{https://www.parc.com/}{Palo Alto
Research Center, or PARC}. There he helped adapt many of the NLS ideas
for a new machine called the Alto, which became a template for the Apple
Macintosh, the first Microsoft Windows personal computers and other
internet-connected devices.

``The whole aim was to implement the ideas behind NLS'' across a network
of personal computers, said Alan Kay, a key researcher behind the Alto.

Mr. English's first marriage, to Patricia Dickson, ended in divorce. He
had met his future second wife, Roberta Mercer, while they were both
working at S.R.I. In addition to her, he is survived by two sons from
his first marriage, Aaron and John; a stepdaughter, Patricia; and a
granddaughter.

This week, when asked if he remembered that Mr. English had arranged the
wireless video feed for the Mother of All Demos through a chance meeting
with a telephone technician in Silicon Valley, Mr. Kay said he did not.

``But it doesn't surprise me,'' he added. ``That sounds like Bill.''

Advertisement

\protect\hyperlink{after-bottom}{Continue reading the main story}

\hypertarget{site-index}{%
\subsection{Site Index}\label{site-index}}

\hypertarget{site-information-navigation}{%
\subsection{Site Information
Navigation}\label{site-information-navigation}}

\begin{itemize}
\tightlist
\item
  \href{https://help.nytimes3xbfgragh.onion/hc/en-us/articles/115014792127-Copyright-notice}{©~2020~The
  New York Times Company}
\end{itemize}

\begin{itemize}
\tightlist
\item
  \href{https://www.nytco.com/}{NYTCo}
\item
  \href{https://help.nytimes3xbfgragh.onion/hc/en-us/articles/115015385887-Contact-Us}{Contact
  Us}
\item
  \href{https://www.nytco.com/careers/}{Work with us}
\item
  \href{https://nytmediakit.com/}{Advertise}
\item
  \href{http://www.tbrandstudio.com/}{T Brand Studio}
\item
  \href{https://www.nytimes3xbfgragh.onion/privacy/cookie-policy\#how-do-i-manage-trackers}{Your
  Ad Choices}
\item
  \href{https://www.nytimes3xbfgragh.onion/privacy}{Privacy}
\item
  \href{https://help.nytimes3xbfgragh.onion/hc/en-us/articles/115014893428-Terms-of-service}{Terms
  of Service}
\item
  \href{https://help.nytimes3xbfgragh.onion/hc/en-us/articles/115014893968-Terms-of-sale}{Terms
  of Sale}
\item
  \href{https://spiderbites.nytimes3xbfgragh.onion}{Site Map}
\item
  \href{https://help.nytimes3xbfgragh.onion/hc/en-us}{Help}
\item
  \href{https://www.nytimes3xbfgragh.onion/subscription?campaignId=37WXW}{Subscriptions}
\end{itemize}
