Sections

SEARCH

\protect\hyperlink{site-content}{Skip to
content}\protect\hyperlink{site-index}{Skip to site index}

\href{https://www.nytimes3xbfgragh.onion/section/theater}{Theater}

\href{https://myaccount.nytimes3xbfgragh.onion/auth/login?response_type=cookie\&client_id=vi}{}

\href{https://www.nytimes3xbfgragh.onion/section/todayspaper}{Today's
Paper}

\href{/section/theater}{Theater}\textbar{}Provincetown: Go for the Mask
Compliance, Stay for a Show

\url{https://nyti.ms/2Xeymnu}

\begin{itemize}
\item
\item
\item
\item
\item
\end{itemize}

Advertisement

\protect\hyperlink{after-top}{Continue reading the main story}

Supported by

\protect\hyperlink{after-sponsor}{Continue reading the main story}

\hypertarget{provincetown-go-for-the-mask-compliance-stay-for-a-show}{%
\section{Provincetown: Go for the Mask Compliance, Stay for a
Show}\label{provincetown-go-for-the-mask-compliance-stay-for-a-show}}

In this gay haven known for its nightlife, the crowds are smaller this
summer. And the nightclubs are closed. But by the pool, the show goes
on.

\includegraphics{https://static01.graylady3jvrrxbe.onion/images/2020/08/01/arts/31ptown-7/merlin_174899316_5a42b24a-ff98-4fa7-96c4-3d686bd254b1-articleLarge.jpg?quality=75\&auto=webp\&disable=upscale}

\href{https://www.nytimes3xbfgragh.onion/by/laura-collins-hughes}{\includegraphics{https://static01.graylady3jvrrxbe.onion/images/2019/04/03/multimedia/author-laura-collins-hughes/author-laura-collins-hughes-thumbLarge.png}}

By
\href{https://www.nytimes3xbfgragh.onion/by/laura-collins-hughes}{Laura
Collins-Hughes}

\begin{itemize}
\item
  July 31, 2020
\item
  \begin{itemize}
  \item
  \item
  \item
  \item
  \item
  \end{itemize}
\end{itemize}

PROVINCETOWN, Mass. --- Varla Jean Merman has a good arm, and when she
threw her hairpiece into the swimming pool the other evening at the end
of an increasingly frenzied number in her cabaret show, it landed on the
surface just right. Then it floated there, inert and disheveled.

``Everyone loves a wig in a pool,'' Varla said, like a breathy
midcentury hostess reassuring her guests. ``It looks like an Irish
setter's in there, taking a nap.''

The pool deck of the
\href{https://onlyatthecrown.com/entertainment}{Crown \& Anchor}, a
hotel and nightlife complex known for its drag shows, is not where Varla
--- or Jeffery Roberson, the performer who plays her --- had planned to
spend the season, in front of an audience on folding chairs. Under a
lighting truss framed by tall trees in full leaf, the stage there is a
new addition: an improvised attempt to salvage this coronavirus summer
by moving at least some entertainment outdoors.

This artsy, eccentric beach town on the tip of Cape Cod --- long-ago
stomping ground of Tennessee Williams and Eugene O'Neill; longtime safe
harbor for queer folk --- ordinarily pulses with activity in July and
August. This year, though, it's hitting a low-tide mark in what should
be its high season.

\includegraphics{https://static01.graylady3jvrrxbe.onion/images/2020/08/01/arts/31ptown-2/merlin_174899625_81d4a0c1-cf62-4d12-99a8-a82701552b88-articleLarge.jpg?quality=75\&auto=webp\&disable=upscale}

Art galleries are open, and so is the pirate museum. Whale-watching
boats are running, and restaurants seat diners inside and al fresco. But
\href{https://theboatslip.com/tea-dance}{tea dance} at the Boatslip is
on indefinite hold, the cabaret is empty at the darkened
\href{https://www.postofficecabaret.com}{Post Office Cafe}, and no
revelers spill out of bars at 1 a.m. to throng
\href{http://spirituspizza.com}{Spiritus Pizza} until 2. The nightclubs
are closed;
\href{https://provincetownindependent.org/featured/2020/07/09/last-call-at-11-p-m-in-provincetown/}{so
are bars}, unless they've morphed into restaurants. Either way, 11 p.m.
is last call.

And with indoor entertainment spaces shuttered, only two establishments
--- the Crown and
\href{https://www.pilgrimhouseptown.com/provincetown-entertainment/}{Pilgrim
House} --- have shifted to open-air stages since that became an option,
just after Independence Day. Neither offers shows the customary seven
nights a week.

Visitors are here, but in diminished numbers. That's to be expected,
given that millions nationwide are unemployed because of the pandemic,
and that travelers to Massachusetts from 42 states
\href{https://www.mass.gov/info-details/covid-19-travel-order\#lower-risk-states-}{must
quarantine} for two weeks upon arrival.

So there is a curious quiet along Commercial Street, Provincetown's
narrow main thoroughfare, which would normally be clogged with festive,
free-spirited masses.

Image

A sign in a shop window on Commercial Street reminds visitors of face
mask requirements.Credit...M. Scott Brauer for The New York Times

Image

Credit...M. Scott Brauer for The New York Times

Image

Outside the Mews Restaurant and Cafe on Commercial Street.Credit...M.
Scott Brauer for The New York Times

In this town of 3,000, which
\href{https://www.mass.gov/doc/weekly-covid-19-public-health-report-july-29-2020/download}{as
of Wednesday} had reported just one new coronavirus case in the previous
14 days, tourism is the main industry, bringing in more than \$250
million in 2019. Yet concerns about economic survival coexist with
vigilance about the virus --- not least because the population includes
a significant number of older residents and the state's highest rate of
people living with H.I.V.

To the comedian Judy Gold, who has owned a second home here since 1994
and has been performing on local stages even longer, there is a clear
link between the community's memory of the 1980s and '90s and its
mindfulness now. When people ask her what it's like in Provincetown
these days, she has a simple response.

``We went through the AIDS crisis here,'' she tells them. ``Everyone's
wearing a mask.''

\hypertarget{uniting-for-survival}{%
\subsection{Uniting for survival}\label{uniting-for-survival}}

Along Route 6 on Cape Cod this summer, electronic signs in town after
town flash variations on the same public-health mantra. Cover your face.
Practice social distancing.

At the Provincetown border, a sign on the median repeats those
entreaties --- and adds a third that might tug at your heart, if this is
a place that you love.

``KEEP PTOWN SAFE,'' it says.

Doing that has required confronting some difficult realities. Mark
Cortale, a producer and artist manager who programs
\href{https://www.ptownarthouse.com}{the Art House} on Commercial
Street, said he hoped until mid-May that he could open its two intimate
stages for a 10th season. The audience, he thought, could be capped at
half capacity, with jauntily masked blowup dolls filling empty seats.

Then Kristin Chenoweth, whom he had booked for two August performances
in the 700-seat auditorium at Provincetown Town Hall, called to postpone
until the same weekend next year. And Cortale's principal client, Seth
Rudetsky, who hosts the starry Broadway @ the Art House series, told him
bluntly that those concerts had
\href{https://thesethconcertseries.com}{to move online}.

``He was like, `Wake up,''' Cortale said. ```Are you watching the
news?'''

Determined not to be foiled completely, Cortale hunted around for an
outdoor space for performers who were eager to play Provincetown this
year. Maybe an old amphitheater in the Cape Cod National Seashore would
do, if he could rig up a generator?

In late May, on Facebook, he spied the solution in a post by Rick
Murray, the owner of the Crown: a photo of a poolside outdoor stage,
with socially distanced seating.

Entrenched rivals, the Art House and the Crown both draw acts from the
worlds of drag, Broadway and cabaret. Even in a good year, the window
for making money is tight in Provincetown, and competition can be
brutal. But when Cortale proposed putting some Art House performers,
Roberson and Gold among them, on that stage, Murray agreed.

Their willingness to work together, Murray acknowledged in an interview,
``turned a few heads in town.''

Or, as Roberson jovially said, it ``probably wouldn't have happened
unless it was the end of the world.''

Image

Provincetown Town Hall at 11 p.m., which has become last call.
Nightclubs are closed, and so are bars unless they have morphed into
restaurants.Credit...M. Scott Brauer for The New York Times

\hypertarget{a-strangely-different-crowd}{%
\subsection{A strangely different
crowd}\label{a-strangely-different-crowd}}

It may not be the end of the world, but for now at least, the pandemic
has altered Provincetown --- changed the mix of people in its streets,
dimmed its spectacle, dulled its sparkle.

``You know what it is?'' Gold mused the other afternoon, from a safe
social distance in an airy room at the Crown. ``It's the magic. The
magic is gone this year.''

No show tunes waft through the windows of piano bars; no dance music
throbs from the clubs. Performers in drag don't weave through
sweat-slicked crowds on bikes and motor scooters, calling, ``Come to my
show!'' And the artists who were always out sketching --- they've
disappeared, too.

Gold misses all of that, and with it the cherished sense of a place
where straight people understand that they are the exception, not the
rule. The strange, skewed thing about Provincetown this summer, she and
others said, is how disproportionately heterosexual the day-tripping
visitors are.

Town Hall, where Jennifer Holliday, Alan Cumming and Margaret Cho would
have played this season, sits silent in the evenings. But its Commercial
Street facade stops passers-by in their tracks.

Bathed in blue and red light, it has a caduceus --- a symbol for
medicine, with winged staff and twined serpents --- projected high on
either side. The display's designers, Chris Racine and Shelley Jennings,
mean it as a tribute to front-line workers.

It is a striking complement to the plentiful street signs labeled
``MANDATORY MASK ZONE,'' and to the friendly ``community ambassadors''
in red pageant-style sashes, whose paid job it is to remind people to
mask up properly. Compliance is startlingly close to universal.

The town's director of health, Morgan Clark, said she was trying to walk
the fine line of keeping everyone safe while protecting both their
physical and mental well-being. In Provincetown, artistic expression is
part of that.

``My favorite kind of movie,'' she said, ``is where people sing or dance
against all odds.''

\hypertarget{bittersweet-gratitude}{%
\subsection{Bittersweet gratitude}\label{bittersweet-gratitude}}

That's pretty much what's happening at Pilgrim House --- singing and
joking, anyway. The drag artist Russ King, a.k.a. Miss Richfield 1981,
ordinarily would be selling out the hotel's 180 indoor seats. Instead
he's onstage in its pebble-paved parking lot, where the capacity is 56,
with social distancing.

Given a cast and crew of four, that means just 52 audience members in a
space that David Nelson Burbank, Pilgrim House's entertainment manager,
aptly described as ``homey.''

In the course of a normal year, King does more than 100 shows, 60 of
them in Provincetown, from Memorial Day to mid-September. This summer,
he said, barring any cancellations because of weather or closures
because of the pandemic, he will do only 36.

Hard as it is to build audience cohesion when people are seated at a
distance from one another and from him, he is grateful to be there.

``I'm really blessed to be employed,'' he said.

Over at what Varla drolly calls **** ``the Crown \& Anchor Poolside
Emergency Theater,'' about 80 spectators are permitted at each
performance. Most take their masks off once they're in their seats, to
have a drink or a snack, though in my experience on two consecutive
nights, there was much less than the
\href{https://www.mass.gov/info-details/safety-standards-and-checklist-theaters-and-performance-venues}{state-mandated}
six feet between audience members in different parties.

Gold and Roberson do solo nights at the Crown, but ``The Judy \& Varla
Show'' is their joint enterprise. For that, their microphone stands are
placed to keep them two yards apart --- and because there is singing,
they must be at least 25 feet from the front row. It's not an ideal way
to work: too far from the audience, in too much darkness, to see many
faces properly, and without walls for the laughter to bounce off.

So, for them it's bittersweet --- joy and relief at being back onstage,
tinged with some frustration. There is also the pang of being forbidden
by state regulations from doing meet-and-greets with fans. For Gold, who
has a new book,
\href{https://www.harpercollins.com/products/yes-i-can-say-that-judy-gold}{``Yes,
I Can Say That,''} to promote, there goes the marketing synergy.

Image

Judy Gold (left) and Varla Jean Merman (Jeffery Roberson) performing in
the ``Judy and Varla Show'' at what Varla calls the Crown \& Anchor
Poolside Emergency Theater.Credit...M. Scott Brauer for The New York
Times

But what a ripe time to be among the few performers with a live outlet
for social commentary.

In Roberson's solo show, ``Superspreader,'' there is a pointed moment
when Varla holds up a little white mask and ridicules those who say
having to wear one is ``ripping away their right to breathe fresh air.''

``Well,'' she says, looking out at the crowd, ``I remember a time not
too long ago in our country where almost everybody in this room was
federally prohibited from the right to get married.''

It's a risky line, because it isn't a joke. It's an indignant assertion
of what real oppression is, and what's just selfishness masquerading as
righteousness.

The other night, as a soft breeze floated in off the harbor, she let
that idea land. Then she turned to her keyboard player, said, ``Hit it,
honey,'' and got on with the show.

Advertisement

\protect\hyperlink{after-bottom}{Continue reading the main story}

\hypertarget{site-index}{%
\subsection{Site Index}\label{site-index}}

\hypertarget{site-information-navigation}{%
\subsection{Site Information
Navigation}\label{site-information-navigation}}

\begin{itemize}
\tightlist
\item
  \href{https://help.nytimes3xbfgragh.onion/hc/en-us/articles/115014792127-Copyright-notice}{©~2020~The
  New York Times Company}
\end{itemize}

\begin{itemize}
\tightlist
\item
  \href{https://www.nytco.com/}{NYTCo}
\item
  \href{https://help.nytimes3xbfgragh.onion/hc/en-us/articles/115015385887-Contact-Us}{Contact
  Us}
\item
  \href{https://www.nytco.com/careers/}{Work with us}
\item
  \href{https://nytmediakit.com/}{Advertise}
\item
  \href{http://www.tbrandstudio.com/}{T Brand Studio}
\item
  \href{https://www.nytimes3xbfgragh.onion/privacy/cookie-policy\#how-do-i-manage-trackers}{Your
  Ad Choices}
\item
  \href{https://www.nytimes3xbfgragh.onion/privacy}{Privacy}
\item
  \href{https://help.nytimes3xbfgragh.onion/hc/en-us/articles/115014893428-Terms-of-service}{Terms
  of Service}
\item
  \href{https://help.nytimes3xbfgragh.onion/hc/en-us/articles/115014893968-Terms-of-sale}{Terms
  of Sale}
\item
  \href{https://spiderbites.nytimes3xbfgragh.onion}{Site Map}
\item
  \href{https://help.nytimes3xbfgragh.onion/hc/en-us}{Help}
\item
  \href{https://www.nytimes3xbfgragh.onion/subscription?campaignId=37WXW}{Subscriptions}
\end{itemize}
