Sections

SEARCH

\protect\hyperlink{site-content}{Skip to
content}\protect\hyperlink{site-index}{Skip to site index}

\href{https://www.nytimes3xbfgragh.onion/section/world/middleeast}{Middle
East}

\href{https://myaccount.nytimes3xbfgragh.onion/auth/login?response_type=cookie\&client_id=vi}{}

\href{https://www.nytimes3xbfgragh.onion/section/todayspaper}{Today's
Paper}

\href{/section/world/middleeast}{Middle East}\textbar{}Scorching
Temperatures Bake Middle East Amid Eid al-Adha Celebrations

\url{https://nyti.ms/2Da4Wjk}

\begin{itemize}
\item
\item
\item
\item
\item
\end{itemize}

\href{https://www.nytimes3xbfgragh.onion/news-event/coronavirus?action=click\&pgtype=Article\&state=default\&region=TOP_BANNER\&context=storylines_menu}{The
Coronavirus Outbreak}

\begin{itemize}
\tightlist
\item
  live\href{https://www.nytimes3xbfgragh.onion/2020/08/04/world/coronavirus-cases.html?action=click\&pgtype=Article\&state=default\&region=TOP_BANNER\&context=storylines_menu}{Latest
  Updates}
\item
  \href{https://www.nytimes3xbfgragh.onion/interactive/2020/us/coronavirus-us-cases.html?action=click\&pgtype=Article\&state=default\&region=TOP_BANNER\&context=storylines_menu}{Maps
  and Cases}
\item
  \href{https://www.nytimes3xbfgragh.onion/interactive/2020/science/coronavirus-vaccine-tracker.html?action=click\&pgtype=Article\&state=default\&region=TOP_BANNER\&context=storylines_menu}{Vaccine
  Tracker}
\item
  \href{https://www.nytimes3xbfgragh.onion/2020/08/02/us/covid-college-reopening.html?action=click\&pgtype=Article\&state=default\&region=TOP_BANNER\&context=storylines_menu}{College
  Reopening}
\item
  \href{https://www.nytimes3xbfgragh.onion/live/2020/08/04/business/stock-market-today-coronavirus?action=click\&pgtype=Article\&state=default\&region=TOP_BANNER\&context=storylines_menu}{Economy}
\end{itemize}

Advertisement

\protect\hyperlink{after-top}{Continue reading the main story}

Supported by

\protect\hyperlink{after-sponsor}{Continue reading the main story}

\hypertarget{scorching-temperatures-bake-middle-east-amid-eid-al-adha-celebrations}{%
\section{Scorching Temperatures Bake Middle East Amid Eid al-Adha
Celebrations}\label{scorching-temperatures-bake-middle-east-amid-eid-al-adha-celebrations}}

Record high temperatures were recorded in Baghdad and Damascus, and
experts warned of the effects of prolonged heat waves as the planet
warms.

\includegraphics{https://static01.graylady3jvrrxbe.onion/images/2020/07/31/world/31heatwave/merlin_175001865_05582ad2-bf6c-4d4f-acee-82bf1880bac9-articleLarge.jpg?quality=75\&auto=webp\&disable=upscale}

By Falih Hassan and
\href{https://www.nytimes3xbfgragh.onion/by/elian-peltier}{Elian
Peltier}

\begin{itemize}
\item
  July 31, 2020
\item
  \begin{itemize}
  \item
  \item
  \item
  \item
  \item
  \end{itemize}
\end{itemize}

BAGHDAD --- A sweltering 125 degrees Fahrenheit in Baghdad on Tuesday; a
record 115 degrees in Damascus on Wednesday. And extreme levels of heat
in Israel and Lebanon.

Several countries in the Middle East experienced record high
temperatures this week as many marked the Muslim celebration of Eid
al-Adha amid the coronavirus pandemic. The heat wave left cities
sweltering in scorching temperatures of 120 degrees (48 degrees Celsius)
or more for days, raising concerns it was a sign of future misery under
the warming effects of climate change.

Iraq has been hit especially hard, with Baghdad recording its all-time
highest temperature on Tuesday, followed by its second hottest day on
record on Wednesday.

The southern city of Basra also recorded temperatures of 120 degrees and
higher for days, with the mercury hitting 122 degrees on Thursday, a
temperature also recorded in Amara, in the southeast.

``The heat is unbearable,'' said Ahmed Hashim, a 30-year-old Baghdad
resident. ``There's a psychological pressure, people can easily get into
a fight.''

Mr. Hashim said he had seen people faint from the heat in the streets of
the Iraqi capital. Some have tried to find respite from the scorching
temperatures in public fountains.

\includegraphics{https://static01.graylady3jvrrxbe.onion/images/2020/07/31/world/31heatwave05/merlin_175029189_80afc09d-7cd8-4724-9f27-fb1741d9711f-articleLarge.jpg?quality=75\&auto=webp\&disable=upscale}

The heat wave is hitting Iraq as the country struggles with a worsening
shortage of electricity, which has pushed people to rely even more on
private generators to power refrigerators, air-conditioners and fans.
Mr. Hashim said generators were being switched off every few hours
because of power cuts, worsening the misery.

``The cooler in the house cannot cool the rooms --- electricity is a
disaster,'' he added.

Two protesters were killed by security forces in Baghdad on Monday in
demonstrations over the worsening lack of electricity. The killings were
the first in months near
\href{https://www.nytimes3xbfgragh.onion/2019/11/20/world/middleeast/iraq-protests-sadr-city.html}{Tahrir
Square}, which became a symbol of
\href{https://www.nytimes3xbfgragh.onion/2019/12/21/world/middleeast/Iraq-protests-Iran.html}{protests
against endemic corruption and foreign interference} last year during a
monthslong period of unrest.

On Thursday, Iraqi authorities ordered a nationwide eight-day holiday
for Eid al-Adha to bring some relief across the country.

\hypertarget{latest-updates-global-coronavirus-outbreak}{%
\section{\texorpdfstring{\href{https://www.nytimes3xbfgragh.onion/2020/08/04/world/coronavirus-cases.html?action=click\&pgtype=Article\&state=default\&region=MAIN_CONTENT_1\&context=storylines_live_updates}{Latest
Updates: Global Coronavirus
Outbreak}}{Latest Updates: Global Coronavirus Outbreak}}\label{latest-updates-global-coronavirus-outbreak}}

Updated 2020-08-04T21:34:02.738Z

\begin{itemize}
\tightlist
\item
  \href{https://www.nytimes3xbfgragh.onion/2020/08/04/world/coronavirus-cases.html?action=click\&pgtype=Article\&state=default\&region=MAIN_CONTENT_1\&context=storylines_live_updates\#link-2daa96b5}{As
  talks drag on, McConnell signals openness to jobless aid extension
  that Republicans have opposed.}
\item
  \href{https://www.nytimes3xbfgragh.onion/2020/08/04/world/coronavirus-cases.html?action=click\&pgtype=Article\&state=default\&region=MAIN_CONTENT_1\&context=storylines_live_updates\#link-1228a480}{Novavax
  sees encouraging results from two studies of its experimental
  vaccine.}
\item
  \href{https://www.nytimes3xbfgragh.onion/2020/08/04/world/coronavirus-cases.html?action=click\&pgtype=Article\&state=default\&region=MAIN_CONTENT_1\&context=storylines_live_updates\#link-4825b93}{Public
  and private schools in Maryland and elsewhere are divided over
  in-person instruction.}
\end{itemize}

\href{https://www.nytimes3xbfgragh.onion/2020/08/04/world/coronavirus-cases.html?action=click\&pgtype=Article\&state=default\&region=MAIN_CONTENT_1\&context=storylines_live_updates}{See
more updates}

More live coverage:
\href{https://www.nytimes3xbfgragh.onion/live/2020/08/04/business/stock-market-today-coronavirus?action=click\&pgtype=Article\&state=default\&region=MAIN_CONTENT_1\&context=storylines_live_updates}{Markets}

Meteorologists define a heat wave as a prolonged period of unusually
high temperatures that span across several days, usually three. Combined
with high humidity or the lack of cool temperatures at night, extreme
temperatures pose risks to the elderly and children.

With average worldwide temperatures rising as a result of carbon dioxide
emissions and other heat-trapping gases, periods of extreme heat are
becoming more frequent and more intense, with the situation particularly
dire near the Equator.

But cooler regions have not been spared. Intense heat afflicted Europe
this week, a year after
\href{https://www.nytimes3xbfgragh.onion/2019/06/26/world/europe/europe-heat-wave.html}{extreme
temperatures soared across the continent}, and several cities like Paris
and Glasgow
\href{https://www.nytimes3xbfgragh.onion/2019/07/25/world/europe/heatwave-record-temperatures.html}{recorded
all-time high temperatures}.

And a study of a
\href{https://www.nytimes3xbfgragh.onion/2020/07/15/climate/siberia-heat-wave-climate-change.html}{prolonged
heat wave in Siberia} earlier this year found that global warming made
the extreme temperatures 600 times more likely there.

Temperatures regularly go above 115 degrees in the summer in Riyadh,
Saudi Arabia; Amman, Jordan; and Baghdad; 120-degree days are no longer
exceptional. But meteorologists have warned that the current heat wave
may be longer and
\href{https://public.wmo.int/en/media/news/above-normal-temperatures-likely-over-arab-region}{more
widespread across the region}.

Image

An Eid al-Adha prayer service on Friday in Amman, Jordan, where
temperatures regularly go above 115 degrees.Credit...Andre Pain/EPA, via
Shutterstock

A report about
\href{https://www.unescwa.org/sites/www.unescwa.org/files/events/files/riccar_main_report_2017.pdf}{climate
change in the Middle East} published by the United Nations in 2017
estimated that average temperatures could increase nearly 10 degrees in
certain areas of the region by the end of the century, and that the
number of days over 104 degrees would increase significantly.

The number of nights per year where temperatures remain above 68 degrees
Fahrenheit may also jump from 40 to 80 or 90 by the end of the century
in the region, said Paolo Ruti, a meteorologist and the director of the
World Weather Research Program at the World Meteorological Organization,
a United Nations agency.

\href{https://www.nytimes3xbfgragh.onion/news-event/coronavirus?action=click\&pgtype=Article\&state=default\&region=MAIN_CONTENT_3\&context=storylines_faq}{}

\hypertarget{the-coronavirus-outbreak-}{%
\subsubsection{The Coronavirus Outbreak
›}\label{the-coronavirus-outbreak-}}

\hypertarget{frequently-asked-questions}{%
\paragraph{Frequently Asked
Questions}\label{frequently-asked-questions}}

Updated August 4, 2020

\begin{itemize}
\item ~
  \hypertarget{i-have-antibodies-am-i-now-immune}{%
  \paragraph{I have antibodies. Am I now
  immune?}\label{i-have-antibodies-am-i-now-immune}}

  \begin{itemize}
  \tightlist
  \item
    As of right
    now,\href{https://www.nytimes3xbfgragh.onion/2020/07/22/health/covid-antibodies-herd-immunity.html?action=click\&pgtype=Article\&state=default\&region=MAIN_CONTENT_3\&context=storylines_faq}{that
    seems likely, for at least several months.} There have been
    frightening accounts of people suffering what seems to be a second
    bout of Covid-19. But experts say these patients may have a
    drawn-out course of infection, with the virus taking a slow toll
    weeks to months after initial exposure. People infected with the
    coronavirus typically
    \href{https://www.nature.com/articles/s41586-020-2456-9}{produce}
    immune molecules called antibodies, which are
    \href{https://www.nytimes3xbfgragh.onion/2020/05/07/health/coronavirus-antibody-prevalence.html?action=click\&pgtype=Article\&state=default\&region=MAIN_CONTENT_3\&context=storylines_faq}{protective
    proteins made in response to an
    infection}\href{https://www.nytimes3xbfgragh.onion/2020/05/07/health/coronavirus-antibody-prevalence.html?action=click\&pgtype=Article\&state=default\&region=MAIN_CONTENT_3\&context=storylines_faq}{.
    These antibodies may} last in the body
    \href{https://www.nature.com/articles/s41591-020-0965-6}{only two to
    three months}, which may seem worrisome, but that's perfectly normal
    after an acute infection subsides, said Dr. Michael Mina, an
    immunologist at Harvard University. It may be possible to get the
    coronavirus again, but it's highly unlikely that it would be
    possible in a short window of time from initial infection or make
    people sicker the second time.
  \end{itemize}
\item ~
  \hypertarget{im-a-small-business-owner-can-i-get-relief}{%
  \paragraph{I'm a small-business owner. Can I get
  relief?}\label{im-a-small-business-owner-can-i-get-relief}}

  \begin{itemize}
  \tightlist
  \item
    The
    \href{https://www.nytimes3xbfgragh.onion/article/small-business-loans-stimulus-grants-freelancers-coronavirus.html?action=click\&pgtype=Article\&state=default\&region=MAIN_CONTENT_3\&context=storylines_faq}{stimulus
    bills enacted in March} offer help for the millions of American
    small businesses. Those eligible for aid are businesses and
    nonprofit organizations with fewer than 500 workers, including sole
    proprietorships, independent contractors and freelancers. Some
    larger companies in some industries are also eligible. The help
    being offered, which is being managed by the Small Business
    Administration, includes the Paycheck Protection Program and the
    Economic Injury Disaster Loan program. But lots of folks have
    \href{https://www.nytimes3xbfgragh.onion/interactive/2020/05/07/business/small-business-loans-coronavirus.html?action=click\&pgtype=Article\&state=default\&region=MAIN_CONTENT_3\&context=storylines_faq}{not
    yet seen payouts.} Even those who have received help are confused:
    The rules are draconian, and some are stuck sitting on
    \href{https://www.nytimes3xbfgragh.onion/2020/05/02/business/economy/loans-coronavirus-small-business.html?action=click\&pgtype=Article\&state=default\&region=MAIN_CONTENT_3\&context=storylines_faq}{money
    they don't know how to use.} Many small-business owners are getting
    less than they expected or
    \href{https://www.nytimes3xbfgragh.onion/2020/06/10/business/Small-business-loans-ppp.html?action=click\&pgtype=Article\&state=default\&region=MAIN_CONTENT_3\&context=storylines_faq}{not
    hearing anything at all.}
  \end{itemize}
\item ~
  \hypertarget{what-are-my-rights-if-i-am-worried-about-going-back-to-work}{%
  \paragraph{What are my rights if I am worried about going back to
  work?}\label{what-are-my-rights-if-i-am-worried-about-going-back-to-work}}

  \begin{itemize}
  \tightlist
  \item
    Employers have to provide
    \href{https://www.osha.gov/SLTC/covid-19/standards.html}{a safe
    workplace} with policies that protect everyone equally.
    \href{https://www.nytimes3xbfgragh.onion/article/coronavirus-money-unemployment.html?action=click\&pgtype=Article\&state=default\&region=MAIN_CONTENT_3\&context=storylines_faq}{And
    if one of your co-workers tests positive for the coronavirus, the
    C.D.C.} has said that
    \href{https://www.cdc.gov/coronavirus/2019-ncov/community/guidance-business-response.html}{employers
    should tell their employees} -\/- without giving you the sick
    employee's name -\/- that they may have been exposed to the virus.
  \end{itemize}
\item ~
  \hypertarget{should-i-refinance-my-mortgage}{%
  \paragraph{Should I refinance my
  mortgage?}\label{should-i-refinance-my-mortgage}}

  \begin{itemize}
  \tightlist
  \item
    \href{https://www.nytimes3xbfgragh.onion/article/coronavirus-money-unemployment.html?action=click\&pgtype=Article\&state=default\&region=MAIN_CONTENT_3\&context=storylines_faq}{It
    could be a good idea,} because mortgage rates have
    \href{https://www.nytimes3xbfgragh.onion/2020/07/16/business/mortgage-rates-below-3-percent.html?action=click\&pgtype=Article\&state=default\&region=MAIN_CONTENT_3\&context=storylines_faq}{never
    been lower.} Refinancing requests have pushed mortgage applications
    to some of the highest levels since 2008, so be prepared to get in
    line. But defaults are also up, so if you're thinking about buying a
    home, be aware that some lenders have tightened their standards.
  \end{itemize}
\item ~
  \hypertarget{what-is-school-going-to-look-like-in-september}{%
  \paragraph{What is school going to look like in
  September?}\label{what-is-school-going-to-look-like-in-september}}

  \begin{itemize}
  \tightlist
  \item
    It is unlikely that many schools will return to a normal schedule
    this fall, requiring the grind of
    \href{https://www.nytimes3xbfgragh.onion/2020/06/05/us/coronavirus-education-lost-learning.html?action=click\&pgtype=Article\&state=default\&region=MAIN_CONTENT_3\&context=storylines_faq}{online
    learning},
    \href{https://www.nytimes3xbfgragh.onion/2020/05/29/us/coronavirus-child-care-centers.html?action=click\&pgtype=Article\&state=default\&region=MAIN_CONTENT_3\&context=storylines_faq}{makeshift
    child care} and
    \href{https://www.nytimes3xbfgragh.onion/2020/06/03/business/economy/coronavirus-working-women.html?action=click\&pgtype=Article\&state=default\&region=MAIN_CONTENT_3\&context=storylines_faq}{stunted
    workdays} to continue. California's two largest public school
    districts --- Los Angeles and San Diego --- said on July 13, that
    \href{https://www.nytimes3xbfgragh.onion/2020/07/13/us/lausd-san-diego-school-reopening.html?action=click\&pgtype=Article\&state=default\&region=MAIN_CONTENT_3\&context=storylines_faq}{instruction
    will be remote-only in the fall}, citing concerns that surging
    coronavirus infections in their areas pose too dire a risk for
    students and teachers. Together, the two districts enroll some
    825,000 students. They are the largest in the country so far to
    abandon plans for even a partial physical return to classrooms when
    they reopen in August. For other districts, the solution won't be an
    all-or-nothing approach.
    \href{https://bioethics.jhu.edu/research-and-outreach/projects/eschool-initiative/school-policy-tracker/}{Many
    systems}, including the nation's largest, New York City, are
    devising
    \href{https://www.nytimes3xbfgragh.onion/2020/06/26/us/coronavirus-schools-reopen-fall.html?action=click\&pgtype=Article\&state=default\&region=MAIN_CONTENT_3\&context=storylines_faq}{hybrid
    plans} that involve spending some days in classrooms and other days
    online. There's no national policy on this yet, so check with your
    municipal school system regularly to see what is happening in your
    community.
  \end{itemize}
\end{itemize}

At Houche al Oumara in Lebanon, temperatures rose to nearly 114 degrees
--- one of the hottest temperatures ever recorded in the country.
Temperatures also soared across much of Israel this week, reaching 111
degrees at the Red Sea resort town of Eilat, and 103 in the northern
city of Tiberias.

Climate experts said the heat wave was part of a trend of warmer summer
temperatures across Israel.

``This heat wave didn't break any records,'' said Hadas Saaroni, a
professor of climatology at Tel Aviv University. ``But over the past
three decades, we have witnessed higher temperatures as well as longer
summers and heat waves.''

Even as countries in the Middle East deal with the crippling heat, they
are struggling to contain the coronavirus pandemic, which has infected
hundreds of thousands of people in the region.

In May, the
\href{https://public.wmo.int/en/media/news/global-partnership-urges-stronger-preparation-hot-weather-during-covid-19}{World
Meteorological Organization} warned that the pandemic would increase
health risks caused by high temperatures, forcing people to congregate
indoors in air-conditioned public spaces, while leaving vulnerable
people more exposed to heat stroke.

In Saudi Arabia, where the temperature hit 115 degrees in its capital,
Riyadh, the pandemic has pushed authorities to sharply reduce the number
of pilgrims undertaking the hajj,
\href{https://www.nytimes3xbfgragh.onion/2020/07/30/world/middleeast/pilgrims-hajj-mecca-coronavirus-pandemic.html}{the
annual pilgrimage to the holy city of Mecca}.

Image

In this image released by the Saudi ministry of media, pilgrims are seen
at the Grand Mosque in Mecca on Friday as the country recorded extremely
high temperatures.Credit...Saudi Ministry of Media, via Associated Press

In Iraq, Jowdat Abdul Rahman, a spokesman of the country's civil defense
forces, said the heat wave's impact had been aggravated by the pandemic.

``Iraqis used to go to swimming pools when temperatures would rise,
while now, they can't,'' he said.

And for the first time, he said, birds are dying because of the heat.
``I had never seen such a thing before.''

Falih Hassan reported from Baghdad, and Elian Peltier from London. Adam
Rasgon contributed reporting from Tel Aviv, and Henry Fountain from
Albuquerque, N.M.

Advertisement

\protect\hyperlink{after-bottom}{Continue reading the main story}

\hypertarget{site-index}{%
\subsection{Site Index}\label{site-index}}

\hypertarget{site-information-navigation}{%
\subsection{Site Information
Navigation}\label{site-information-navigation}}

\begin{itemize}
\tightlist
\item
  \href{https://help.nytimes3xbfgragh.onion/hc/en-us/articles/115014792127-Copyright-notice}{©~2020~The
  New York Times Company}
\end{itemize}

\begin{itemize}
\tightlist
\item
  \href{https://www.nytco.com/}{NYTCo}
\item
  \href{https://help.nytimes3xbfgragh.onion/hc/en-us/articles/115015385887-Contact-Us}{Contact
  Us}
\item
  \href{https://www.nytco.com/careers/}{Work with us}
\item
  \href{https://nytmediakit.com/}{Advertise}
\item
  \href{http://www.tbrandstudio.com/}{T Brand Studio}
\item
  \href{https://www.nytimes3xbfgragh.onion/privacy/cookie-policy\#how-do-i-manage-trackers}{Your
  Ad Choices}
\item
  \href{https://www.nytimes3xbfgragh.onion/privacy}{Privacy}
\item
  \href{https://help.nytimes3xbfgragh.onion/hc/en-us/articles/115014893428-Terms-of-service}{Terms
  of Service}
\item
  \href{https://help.nytimes3xbfgragh.onion/hc/en-us/articles/115014893968-Terms-of-sale}{Terms
  of Sale}
\item
  \href{https://spiderbites.nytimes3xbfgragh.onion}{Site Map}
\item
  \href{https://help.nytimes3xbfgragh.onion/hc/en-us}{Help}
\item
  \href{https://www.nytimes3xbfgragh.onion/subscription?campaignId=37WXW}{Subscriptions}
\end{itemize}
