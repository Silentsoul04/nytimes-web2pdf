Sections

SEARCH

\protect\hyperlink{site-content}{Skip to
content}\protect\hyperlink{site-index}{Skip to site index}

\href{https://www.nytimes3xbfgragh.onion/section/arts}{Arts}

\href{https://myaccount.nytimes3xbfgragh.onion/auth/login?response_type=cookie\&client_id=vi}{}

\href{https://www.nytimes3xbfgragh.onion/section/todayspaper}{Today's
Paper}

\href{/section/arts}{Arts}\textbar{}Lady Red Couture, `Mother Hen' of a
Drag Scene, Dies at 43

\url{https://nyti.ms/3fhGhXb}

\begin{itemize}
\item
\item
\item
\item
\item
\end{itemize}

Advertisement

\protect\hyperlink{after-top}{Continue reading the main story}

Supported by

\protect\hyperlink{after-sponsor}{Continue reading the main story}

\hypertarget{lady-red-couture-mother-hen-of-a-drag-scene-dies-at-43}{%
\section{Lady Red Couture, `Mother Hen' of a Drag Scene, Dies at
43}\label{lady-red-couture-mother-hen-of-a-drag-scene-dies-at-43}}

A fixture in Los Angeles clubs who stood 7-foot-2 in heels, she also
co-hosted the L.G.B.T.Q. talk show ``Hey Qween!''

\includegraphics{https://static01.graylady3jvrrxbe.onion/images/2020/08/03/obituaries/00ladyred/merlin_175015461_8055e454-898c-4072-8dc9-1547fb7d253c-articleLarge.jpg?quality=75\&auto=webp\&disable=upscale}

\href{https://www.nytimes3xbfgragh.onion/by/john-leland}{\includegraphics{https://static01.graylady3jvrrxbe.onion/images/2018/02/20/multimedia/author-john-leland/author-john-leland-thumbLarge.jpg}}

By \href{https://www.nytimes3xbfgragh.onion/by/john-leland}{John Leland}

\begin{itemize}
\item
  Aug. 1, 2020
\item
  \begin{itemize}
  \item
  \item
  \item
  \item
  \item
  \end{itemize}
\end{itemize}

Lady Red Couture, a comedian, singer and talk-show host who called
herself ``the largest live-singing drag queen in captivity,'' and who
became a mentor to younger drag performers, died on July 25 in Los
Angeles. She was 43.

The cause was complications of
\href{https://ghr.nlm.nih.gov/condition/cyclic-vomiting-syndrome\#inheritance}{cyclic
vomiting syndrome}, a condition thought to be related to migraines, said
Jonny McGovern, her co-host on
``\href{https://www.youtube.com/channel/UCStjBZ-91yduQVNtsBkjgJQ}{Hey
Qween}!,'' a popular L.G.B.T.Q. talk show that began on YouTube and is
now also available on Amazon Prime.

In her \href{https://www.youtube.com/watch?v=9R825RVrB78}{audition reel}
for Season 6 of
``\href{https://www.nytimes3xbfgragh.onion/2011/08/07/fashion/at-lunch-with-rupaul-main-course.html?searchResultPosition=11}{RuPaul's
Drag Race},'' Lady Red offered a proud mantra of herself. ``I live, I
breathe, I do everything in drag --- this is me 24/7,'' she said. ``I am
nothing to play with. Especially with my mouth.''

Like some of the influential early drag queens, she also identified as a
transgender woman. In an Instagram
\href{https://www.instagram.com/p/CBO43e4go6R/}{video} posted in June in
response to the Black Lives Matter protests, she described herself as
``a Black American trans woman who is making a difference in this
world.''

``At a time when the word authentic is overused, honey, she was
unique,'' said \href{http://ladybunny.net/}{Lady Bunny}, the New York
City drag queen
\href{https://www.nytimes3xbfgragh.onion/2018/09/29/style/lady-bunny-drag-queen.html?searchResultPosition=1}{impresaria},
who toured with Lady Red. ``She was 6-foot-7, wore Size 16 Converse
sneakers with an evening gown'' --- she also stood 7-foot-2 in heels ---
``and wore the showbiz eyelashes during the day.''

``And,'' Lady Bunny added, ``she would drop her voice real low to freak
you out.''

Though she was never chosen for ``RuPaul's Drag Race,'' Lady Red cut a
radiant figure within the Los Angeles drag scene, performing in clubs
like \href{https://www.hamburgermarys.com/weho/}{Hamburger Mary's} and
the District and on ``Hey Qween!,'' where she served as a vivacious Ed
McMahon to Mr. McGovern's bearded Johnny Carson.

``To meet her was to fall in love with her,'' Frank DeCaro, the author
of
\href{https://www.hollywoodreporter.com/news/frank-decaro-unravels-drags-origins-new-book-1208975}{``Drag:
Combing Through the Big Wigs of Show Business''} (2019), said in an
email. ``She was one of the few people about whom I never heard anyone
say anything shady. ''

Younger drag performers were drawn to her.
\href{https://gingerminj.com/}{Ginger Minj}, who competed on ``RuPaul's
Drag Race,'' described Lady Red as ``the mother hen'' of the circuit,
adding: ``Everybody knew her and looked up to her. You could always go
to her with your problems.''

Image

Lady Red at a Los Angeles Film Festival party in 2011. ``Everybody knew
her and looked up to her,'' a fellow drag performer said.Credit...Craig
Barritt/WireImage

Lady Red Couture was born in Los Angeles on May 30, 1977. Her mother,
Kathleen Barnes, was a nondenominational Christian minister, and the two
did not see eye to eye.

``She's a wonderful person --- to other people,'' Lady Red said on her
``Drag Race'' audition reel in 2017.

Her father left the family when she was young and was not a part of her
life, said Krystle Butler, Lady Red's younger sister. Her sister and her
mother survive her.

At George Washington Preparatory High School in Los Angeles, Lady Red
acted in plays and played trombone and tuba in the marching band,
performing in the Tournament of Roses Parade. But she was also picked on
by her peers and fought with her mother.

The arguments escalated until, Lady Red said, her mother told her, ``If
you want to be a girl, go be a girl,'' and threw her out of the house.
``It hurt me a lot,'' she added.

It was a difficult relationship, Ms. Butler said, but it was based in
love. ``My mother is very religious, so at first she had to digest the
whole concept. But over time she adjusted, because God loves everyone.''

``She said, `Whatever you decide to be, be the best at it,''' Ms. Butler
added. ```If you're going to be a queen, be the best queen.'''

Lady Red received an associate degree from the Salt Lake Community
College Culinary Institute in Utah and worked briefly as a chef with
Amtrak. After returning to Los Angeles, she held a variety of jobs,
including ``security diva'' at Gym Sportsbar and ``budtender'' at a
MedMen cannabis dispensary, both in West Hollywood.

She started performing in drag in 1995, according to her Internet Movie
Database
\href{https://www.imdb.com/name/nm4366207/bio?ref_=nm_ov_bio_sm}{profile}.
Mr. McGovern, a comedian and actor, saw her in clubs and tapped her to
be his co-host on ``Hey Qween!'' The show, which began in 2014, rose in
parallel with ``RuPaul's Drag Race,'' becoming a go-to spot for cast
members to ``spill some tea,'' or tell some truth.

During the first season, Mr. McGovern learned that Lady Red was living
in a transient hotel and invited her to move in with him. He was among
the few who ever saw her without full makeup, he said.

``Lady Red's look was 1,000 percent all the time,'' Mr. McGovern said.
``She was calling herself a full-time drag queen before accepting that
she was a woman, not just a drag queen. She lived a life with glamour,
but as a Black trans woman, there was a lot of hardship as well.''

She never brought her hardships to the stage. The drag queen
\href{http://www.missjackiebeat.com/}{Jackie Beat} recalled one tour
when she and Sherry Vine, another drag queen, were in a bad mood before
a show. ``That's a bad place to be when it's time to sit down and paint
your face and then entertain a theater filled with people who, frankly,
don't care what you're going through,'' she said by email.

But Lady Red was not having it. Whenever she was not feeling it, she
told them, she would look in the mirror and tell herself, ``Get in the
mood, bitch!''

``We roared with laughter and our mood instantly improved,'' Jackie Beat
said. ``We say that all the time now.''

Advertisement

\protect\hyperlink{after-bottom}{Continue reading the main story}

\hypertarget{site-index}{%
\subsection{Site Index}\label{site-index}}

\hypertarget{site-information-navigation}{%
\subsection{Site Information
Navigation}\label{site-information-navigation}}

\begin{itemize}
\tightlist
\item
  \href{https://help.nytimes3xbfgragh.onion/hc/en-us/articles/115014792127-Copyright-notice}{©~2020~The
  New York Times Company}
\end{itemize}

\begin{itemize}
\tightlist
\item
  \href{https://www.nytco.com/}{NYTCo}
\item
  \href{https://help.nytimes3xbfgragh.onion/hc/en-us/articles/115015385887-Contact-Us}{Contact
  Us}
\item
  \href{https://www.nytco.com/careers/}{Work with us}
\item
  \href{https://nytmediakit.com/}{Advertise}
\item
  \href{http://www.tbrandstudio.com/}{T Brand Studio}
\item
  \href{https://www.nytimes3xbfgragh.onion/privacy/cookie-policy\#how-do-i-manage-trackers}{Your
  Ad Choices}
\item
  \href{https://www.nytimes3xbfgragh.onion/privacy}{Privacy}
\item
  \href{https://help.nytimes3xbfgragh.onion/hc/en-us/articles/115014893428-Terms-of-service}{Terms
  of Service}
\item
  \href{https://help.nytimes3xbfgragh.onion/hc/en-us/articles/115014893968-Terms-of-sale}{Terms
  of Sale}
\item
  \href{https://spiderbites.nytimes3xbfgragh.onion}{Site Map}
\item
  \href{https://help.nytimes3xbfgragh.onion/hc/en-us}{Help}
\item
  \href{https://www.nytimes3xbfgragh.onion/subscription?campaignId=37WXW}{Subscriptions}
\end{itemize}
