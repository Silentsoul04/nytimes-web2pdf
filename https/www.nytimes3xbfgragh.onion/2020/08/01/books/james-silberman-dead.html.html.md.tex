Sections

SEARCH

\protect\hyperlink{site-content}{Skip to
content}\protect\hyperlink{site-index}{Skip to site index}

\href{https://www.nytimes3xbfgragh.onion/section/books}{Books}

\href{https://myaccount.nytimes3xbfgragh.onion/auth/login?response_type=cookie\&client_id=vi}{}

\href{https://www.nytimes3xbfgragh.onion/section/todayspaper}{Today's
Paper}

\href{/section/books}{Books}\textbar{}James Silberman, Editor Who
Nurtured Literary Careers, Dies at 93

\url{https://nyti.ms/2BSK4wA}

\begin{itemize}
\item
\item
\item
\item
\item
\end{itemize}

Advertisement

\protect\hyperlink{after-top}{Continue reading the main story}

Supported by

\protect\hyperlink{after-sponsor}{Continue reading the main story}

\hypertarget{james-silberman-editor-who-nurtured-literary-careers-dies-at-93}{%
\section{James Silberman, Editor Who Nurtured Literary Careers, Dies at
93}\label{james-silberman-editor-who-nurtured-literary-careers-dies-at-93}}

At Random House and elsewhere, including his own Summit imprint, he
worked with James Baldwin, Marilyn French, Hunter S. Thompson and many
others.

\includegraphics{https://static01.graylady3jvrrxbe.onion/images/2020/08/03/obituaries/Silberman1/Silberman1-articleLarge.jpg?quality=75\&auto=webp\&disable=upscale}

\href{https://www.nytimes3xbfgragh.onion/by/sam-roberts}{\includegraphics{https://static01.graylady3jvrrxbe.onion/images/2018/02/20/multimedia/author-sam-roberts/author-sam-roberts-thumbLarge.jpg}}

By \href{https://www.nytimes3xbfgragh.onion/by/sam-roberts}{Sam Roberts}

\begin{itemize}
\item
  Aug. 1, 2020
\item
  \begin{itemize}
  \item
  \item
  \item
  \item
  \item
  \end{itemize}
\end{itemize}

James Silberman, a revered book editor whose meticulousness, intuition
and patience helped propel the publishing careers of a distinguished
roster of authors, including James Baldwin, Marilyn French, Hunter S.
Thompson and Alvin Toffler, died on July 26 at his home in Manhattan. He
was 93.

His son, Michael, said the cause was complications of a stroke.

Mr. Silberman was ``a man who knows how to edit a manuscript, to read a
manuscript and to publish a manuscript,'' another of his authors, Elie
Wiesel, told
\href{https://www.nytimes3xbfgragh.onion/1991/10/30/business/the-media-business-head-of-summit-books-is-said-to-be-leaving.html}{The
New York Times} in 1991.

Mr. Silberman's career path was serendipitous. A government major at
Harvard, he enrolled in the Radcliffe Publishing Course (now the
Columbia Publishing Course) after graduating in 1950, then got hired in
the shipping department of The Writer, which, he recalled in an oral
history, was in the business of ``selling a magazine to aspiring
writers, telling them how to become rich and famous.''

He found an advertising job at Little, Brown \& Company, then became a
publicist for the Dial Press in New York in 1953. When the company's
sole editor left to have her second child, he was promoted to replace
her and assumed the title that would define his vocation.

After Alfred A. Knopf, James Baldwin's first publisher, rejected
``Giovanni's Room'' because they felt its gay white characters might
alienate Mr. Baldwin's Black audience, Mr. Silberman scooped it up for
Dial. He went on to edit Mr. Baldwin's ``Another Country'' and ``The
Fire Next Time.''

In 1963, Mr. Silberman was lured to Random House as senior editor by
Bennett Cerf, the company's co-founder, who later named him editor in
chief and publisher of adult trade books.

Joining an impressive editorial team that included Robert Loomis, Jason
Epstein and Joe Fox, Mr. Silberman published Hunter S. Thompson's
``Hell's Angels'' (1967), Alvin Toffler's ``Future Shock'' (1970),
Stewart Brand's ``The Last Whole Earth Catalog'' (1971, in association
with the Portola Institute), David Halberstam's ``The Best and the
Brightest'' (1972) and E.L. Doctorow's ``The Book of Daniel'' (1971) and
``Ragtime'' (1975).

Mr. Silberman left Random House in 1975 after refusing to fire Selma
Shapiro, the company's vice president for publicity, with whom he was
having an affair and whom he later married; he blamed the company's
``moral rigidity.'' He was immediately hired by Richard E. Snyder, Simon
\& Schuster's competitive chairman, to launch his own imprint, Summit
Books.

\includegraphics{https://static01.graylady3jvrrxbe.onion/images/2020/08/03/obituaries/30Silberman2/30Silberman2-articleLarge.jpg?quality=75\&auto=webp\&disable=upscale}

At Summit he published
\href{https://www.nytimes3xbfgragh.onion/2009/05/04/arts/04french.html}{Marilyn
French}'s debut novel, ``The Women's Room'' (1977), which sold some 20
million copies; Seymour Hersh's ``The Price of Power: Kissinger in the
Nixon White House'' (1983); and Oliver Sacks's ``The Man Who Mistook His
Wife for a Hat'' (1985).

``Jim could see things in what I was doing as a reporter that I did not
see,'' Mr. Hersh said by email, citing his books on Mr. Kissinger and
John F. Kennedy. ``Amidst constant negative pressure from the subjects,
he never flinched and had my back all the way.''

Mr. Silberman lost his job at Summit in 1991 when the imprint was
eliminated to cut costs. He was a vice president and senior editor at
Little Brown until 1998 and then established James H. Silberman Books.

Over the course of his career, his authors also included Muhammad Ali,
Betty Friedan, George Goodman (who wrote about economics under the name
Adam Smith), John Irving and Chris Matthews, whom he encouraged to write
``Hardball: How Politics Is Played Told by One Who Knows the Game''
(1988).

``He spotted a piece I'd done for The New Republic as Tip O'Neill's guy
going to daily war with the Reagan White House,'' Mr. Matthews said by
email. ``He asked me to write a book about the inside political world to
match `The Money Game,''' Mr. Goodman's influential 1968 book. ``It
became `Hardball.'''

Invoking the editor who fostered Hemingway and Fitzgerald, Mr. Matthews
said, ``Jim was my Max Perkins.''

James Henry Silberman was born on March 21, 1927, in Boston to Henry R.
Silberman, who ran a news clipping service and was the executive
director of the Massachusetts Progressive Party, and Dorothy (Conrad)
Silberman.

After graduating from Cambridge Latin School, he served in the Army
after World War II and then attended Harvard.

He married
\href{https://www.nytimes3xbfgragh.onion/2005/12/15/arts/leona-nevler-editor-dies-at-79-shepherded-peyton-place.html}{Leona
Nevler}, an editor, in 1960; they divorced in 1976. In 1986 he married
\href{https://www.nytimes3xbfgragh.onion/1986/08/27/style/selma-shapiro-wed-to-james-silberman.html}{Ms.
Shapiro}, who survives him, along with two children from his first
marriage, Michael and Ellen Silberman; his sister, Dorothy Altman; and
four grandchildren.

Mr. Silberman was a natty dresser, a dashing wheelman (he became an
amateur pilot at 50 and drove a Mazda RX-7 convertible sports car on
weekends) and a scrupulous wordsmith who at 86, even after suffering a
stroke, finished editing two books.

Mr. Cerf, who took pride in all his top editors, said in the mid-1960s
that ``the best one of all for the purposes of great corporate handling
of manuscripts is Jim Silberman, who is now being made editor in chief,
because he's the one willing to do all of the dirty work of seeing what
happens to all of these manuscripts.''

Among the authors with whom Mr. Silberman had especially tortured
relationships was Mr. Thompson, the gonzo journalist who wrote books
about ``Fear and Loathing'' and whose struggle to write a book
tentatively called ``The Death of the American Dream'' is recorded in
his letters to Mr. Silberman in books edited by Douglas Brinkley.

Mr. Silberman once said of Mr. Thompson, ``Your method of research is to
tie yourself to a railroad track when you know a train is coming to it,
and see what happens.'' And, when
\href{https://www.nytimes3xbfgragh.onion/2005/02/21/books/hunter-s-thompson-67-author-commits-suicide.html}{Mr.
Thompson} killed himself at 67 in 2005, Mr. Silberman remarked, ``He
spent his life in search of an honest man, and he seldom found any.''

Coaxing a book out of Mr. Thompson, or for that matter a more
conventional writer, meant ``helping the author write the best book he
or she can write at that moment in time,'' which requires that ``every
time you turn that page, you are open and hopeful,'' Mr. Silberman once
said.

``It's very difficult to think your way into a story,'' he added. ``You
have to feel your way into it, which requires you to approach the
manuscript with a certain kind of naïveté. You have to return to the
kind of reader all of us once were.''

Advertisement

\protect\hyperlink{after-bottom}{Continue reading the main story}

\hypertarget{site-index}{%
\subsection{Site Index}\label{site-index}}

\hypertarget{site-information-navigation}{%
\subsection{Site Information
Navigation}\label{site-information-navigation}}

\begin{itemize}
\tightlist
\item
  \href{https://help.nytimes3xbfgragh.onion/hc/en-us/articles/115014792127-Copyright-notice}{©~2020~The
  New York Times Company}
\end{itemize}

\begin{itemize}
\tightlist
\item
  \href{https://www.nytco.com/}{NYTCo}
\item
  \href{https://help.nytimes3xbfgragh.onion/hc/en-us/articles/115015385887-Contact-Us}{Contact
  Us}
\item
  \href{https://www.nytco.com/careers/}{Work with us}
\item
  \href{https://nytmediakit.com/}{Advertise}
\item
  \href{http://www.tbrandstudio.com/}{T Brand Studio}
\item
  \href{https://www.nytimes3xbfgragh.onion/privacy/cookie-policy\#how-do-i-manage-trackers}{Your
  Ad Choices}
\item
  \href{https://www.nytimes3xbfgragh.onion/privacy}{Privacy}
\item
  \href{https://help.nytimes3xbfgragh.onion/hc/en-us/articles/115014893428-Terms-of-service}{Terms
  of Service}
\item
  \href{https://help.nytimes3xbfgragh.onion/hc/en-us/articles/115014893968-Terms-of-sale}{Terms
  of Sale}
\item
  \href{https://spiderbites.nytimes3xbfgragh.onion}{Site Map}
\item
  \href{https://help.nytimes3xbfgragh.onion/hc/en-us}{Help}
\item
  \href{https://www.nytimes3xbfgragh.onion/subscription?campaignId=37WXW}{Subscriptions}
\end{itemize}
