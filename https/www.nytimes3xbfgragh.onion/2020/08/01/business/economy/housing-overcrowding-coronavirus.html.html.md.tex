\href{/section/business/economy}{Economy}\textbar{}12 People in a
3-Bedroom House, Then the Virus Entered the Equation

\begin{itemize}
\item
\item
\item
\item
\item
\end{itemize}

\href{https://www.nytimes3xbfgragh.onion/news-event/coronavirus?action=click\&pgtype=Article\&state=default\&region=TOP_BANNER\&context=storylines_menu}{The
Coronavirus Outbreak}

\begin{itemize}
\tightlist
\item
  live\href{https://www.nytimes3xbfgragh.onion/2020/08/01/world/coronavirus-covid-19.html?action=click\&pgtype=Article\&state=default\&region=TOP_BANNER\&context=storylines_menu}{Latest
  Updates}
\item
  \href{https://www.nytimes3xbfgragh.onion/interactive/2020/us/coronavirus-us-cases.html?action=click\&pgtype=Article\&state=default\&region=TOP_BANNER\&context=storylines_menu}{Maps
  and Cases}
\item
  \href{https://www.nytimes3xbfgragh.onion/interactive/2020/science/coronavirus-vaccine-tracker.html?action=click\&pgtype=Article\&state=default\&region=TOP_BANNER\&context=storylines_menu}{Vaccine
  Tracker}
\item
  \href{https://www.nytimes3xbfgragh.onion/interactive/2020/07/29/us/schools-reopening-coronavirus.html?action=click\&pgtype=Article\&state=default\&region=TOP_BANNER\&context=storylines_menu}{What
  School May Look Like}
\item
  \href{https://www.nytimes3xbfgragh.onion/live/2020/07/31/business/stock-market-today-coronavirus?action=click\&pgtype=Article\&state=default\&region=TOP_BANNER\&context=storylines_menu}{Economy}
\end{itemize}

\includegraphics{https://static01.graylady3jvrrxbe.onion/images/2020/08/02/business/00virus-crowding1/00virus-crowding1-articleLarge.jpg?quality=75\&auto=webp\&disable=upscale}

Sections

\protect\hyperlink{site-content}{Skip to
content}\protect\hyperlink{site-index}{Skip to site index}

\hypertarget{12-people-in-a-3-bedroom-house-then-the-virus-entered-the-equation}{%
\section{12 People in a 3-Bedroom House, Then the Virus Entered the
Equation}\label{12-people-in-a-3-bedroom-house-then-the-virus-entered-the-equation}}

Overcrowding, not density, has defined many coronavirus hot spots.
Service workers' quarters skirting Silicon Valley are no exception.

Karla Lorenzo made about \$16 an hour mopping floors and vacuuming
carpets before the pandemic. She shared a three-bedroom house with her
partner, her three children and seven others.Credit...Brian L. Frank for
The New York Times

Supported by

\protect\hyperlink{after-sponsor}{Continue reading the main story}

\href{https://www.nytimes3xbfgragh.onion/by/conor-dougherty}{\includegraphics{https://static01.graylady3jvrrxbe.onion/images/2018/07/27/multimedia/author-conor-dougherty/author-conor-dougherty-thumbLarge.png}}

By \href{https://www.nytimes3xbfgragh.onion/by/conor-dougherty}{Conor
Dougherty}

\begin{itemize}
\item
  Aug. 1, 2020
\item
  \begin{itemize}
  \item
  \item
  \item
  \item
  \item
  \end{itemize}
\end{itemize}

It was not surprising when three-quarters of the house tested positive.
There were 12 people in three bedrooms, with a bathroom whose door
frequently required a knock and a kitchen where dinnertime shifts
extended from 5 p.m. well into the evening.

Karla Lorenzo, a Guatemalan immigrant who cleaned houses in San
Francisco and Silicon Valley, lived in the big room along the driveway.
Big is a relative term when a room has five people in it. She and her
partner, Abel, slept in a queen-size bed along the wall. There was a
crib for the baby at the foot, with the older children's bunk bed next
to that. The other housemates had similar layouts.

Living among many people, as Ms. Lorenzo put it in Spanish, you cannot
really avoid your housemates. The sounds, the smells, the moods ---
everyone is pressed against all of it, and they understood that if one
of them got
\href{https://www.nytimes3xbfgragh.onion/news-event/coronavirus}{the
coronavirus}, the rest probably would.

That happened in April, and now the house is returning to health. Abel,
referred to by his first name because his immigration status is
uncertain, is home after three weeks in the hospital, where Ms. Lorenzo
feared he would die alone gasping for air. And she is no longer
squirreled in the closet where she spent days to avoid giving the virus
to the children.

Now comes a second struggle: figuring out how to pay rent. Abel is back
at work at a home supply store, but Ms. Lorenzo's housecleaning jobs
dried up and one of the other families moved out --- increasing the
monthly bill by \$850. ``We don't know how we are going to do it,'' she
said.

From the early outbreaks to the economic destruction that has come
after, the coronavirus pandemic has mapped itself onto America's
longstanding affordable housing problem and the gaping inequality that
underlies it. To offset rising rents in a nation where
\href{https://www.jchs.harvard.edu/sites/default/files/Harvard_JCHS_Americas_Rental_Housing_2020.pdf}{one
in four tenant households} spend more than half of their pretax income
on shelter, a multitude of low-wage service workers have piled into ever
more crowded homes.

Image

Shoes sitting outside Ms. Lorenzo's home to disinfect. She confined
herself to the closet for days to avoid spreading the virus to her
children.Credit...Brian L. Frank for The New York Times

Image

Space for parking is at a premium in her neighborhood.Credit...Jim
McAuley for The New York Times

Living in overstuffed units subdivided by hinged partitions and
tacked-up sheets, these households --- many of them retail and service
workers who are unable to do their jobs from home --- were acutely
susceptible to the virus's spread. With double-digit unemployment
projected to persist
\href{https://www.nytimes3xbfgragh.onion/2020/07/15/business/economy/economic-recovery-coronavirus-resurgence.html}{through
next year}, the same families face
\href{https://www.nytimes3xbfgragh.onion/2020/05/27/us/coronavirus-evictions-renters.html}{losing
the crowded homes} that make it so easy to get sick in the first place.

To combat the virus, Americans of every income are being encouraged to
wear masks and keep their distance. But for low-income families who
crowd together to stretch their budgets, home has its own risks.

For these families, a good amount of the response has included triaging
a decades-old
\href{https://reports.nlihc.org/gap\#:~:text=The\%20U.S.\%20has\%20a\%20shortage,extremely\%20low\%2Dincome\%20renter\%20households.}{shortage
of affordable housing}. Cities and states are
\href{https://www.kqed.org/news/11825653/california-found-hotels-for-10000-homeless-residents-what-next}{renting
hotel rooms} for people who normally sleep on the streets. There are
trailers to quarantine those whose apartments are too crowded for
isolation. Fearing a wave of homelessness, governments have followed up
with rental aid and
\href{https://evictionlab.org/covid-policy-scorecard/}{moratoriums on
evictions}.

Combined with federal stimulus funds, and
\href{https://www.nytimes3xbfgragh.onion/2020/07/29/business/economy/unemployment-benefits-coronavirus.html}{\$600
a week in supplemental unemployment benefits} that have just lapsed,
these measures have
\href{https://www.nytimes3xbfgragh.onion/2020/05/31/business/economy/coronavirus-rent-landlords-tenants.html}{prevented
the dire predictions} of mass displacement. Congress is working on
another emergency package, and property owners and affordable-housing
advocates have pressed for direct rental assistance.

But evictions are
\href{https://www.nytimes3xbfgragh.onion/2020/07/23/business/evictions-moratorium-cares-act.html}{already
ramping back up}, and the longer the economic malaise continues, the
more housing insecurity there will be. Some of the evicted will become
homeless, but if the past is a guide, most are likely to find somewhere
else to go, and that somewhere is likely to be overcrowded ---
compounding the conditions that make it so easy to spread the virus.

``We have clients struggling to choose between living in an overcrowded
home or facing eviction for not being able to make rent,'' said Nazanin
Salehi, a lawyer with the nonprofit group Community Legal Services in
East Palo Alto. ``No matter what they decide, the risk is more exposure
to this virus.''

\hypertarget{latest-updates-economy}{%
\section{\texorpdfstring{\href{https://www.nytimes3xbfgragh.onion/live/2020/07/31/business/stock-market-today-coronavirus?action=click\&pgtype=Article\&state=default\&region=MAIN_CONTENT_1\&context=storylines_live_updates}{Latest
Updates:
Economy}}{Latest Updates: Economy}}\label{latest-updates-economy}}

\href{https://www.nytimes3xbfgragh.onion/live/2020/07/31/business/stock-market-today-coronavirus?action=click\&pgtype=Article\&state=default\&region=MAIN_CONTENT_1\&context=storylines_live_updates\#kodaks-chief-executive-was-given-stock-options-then-the-share-price-spiked-1000-percent}{32h
ago}

\href{https://www.nytimes3xbfgragh.onion/live/2020/07/31/business/stock-market-today-coronavirus?action=click\&pgtype=Article\&state=default\&region=MAIN_CONTENT_1\&context=storylines_live_updates\#kodaks-chief-executive-was-given-stock-options-then-the-share-price-spiked-1000-percent}{Kodak's
chief executive was given stock options. Then the share price spiked
1,000 percent.}

\href{https://www.nytimes3xbfgragh.onion/live/2020/07/31/business/stock-market-today-coronavirus?action=click\&pgtype=Article\&state=default\&region=MAIN_CONTENT_1\&context=storylines_live_updates\#fitch-ratings-downgrades-its-outlook-on-us-debt}{35h
ago}

\href{https://www.nytimes3xbfgragh.onion/live/2020/07/31/business/stock-market-today-coronavirus?action=click\&pgtype=Article\&state=default\&region=MAIN_CONTENT_1\&context=storylines_live_updates\#fitch-ratings-downgrades-its-outlook-on-us-debt}{Fitch
Ratings downgrades its outlook on U.S. debt.}

\href{https://www.nytimes3xbfgragh.onion/live/2020/07/31/business/stock-market-today-coronavirus?action=click\&pgtype=Article\&state=default\&region=MAIN_CONTENT_1\&context=storylines_live_updates\#us-sanctions-more-chinese-officials-over-human-rights-violations-as-tensions-flare}{41h
ago}

\href{https://www.nytimes3xbfgragh.onion/live/2020/07/31/business/stock-market-today-coronavirus?action=click\&pgtype=Article\&state=default\&region=MAIN_CONTENT_1\&context=storylines_live_updates\#us-sanctions-more-chinese-officials-over-human-rights-violations-as-tensions-flare}{U.S.
sanctions more Chinese officials over human rights violations as
tensions flare}

\href{https://www.nytimes3xbfgragh.onion/live/2020/07/31/business/stock-market-today-coronavirus?action=click\&pgtype=Article\&state=default\&region=MAIN_CONTENT_1\&context=storylines_live_updates}{See
more updates}

More live coverage:
\href{https://www.nytimes3xbfgragh.onion/2020/08/01/world/coronavirus-covid-19.html?action=click\&pgtype=Article\&state=default\&region=MAIN_CONTENT_1\&context=storylines_live_updates}{Global}

\hypertarget{two-sides-of-silicon-valley}{%
\subsection{Two Sides of Silicon
Valley}\label{two-sides-of-silicon-valley}}

Residential overcrowding

Share of housing units within each census tract with more than one
person per room

0

3

8

13

22

30

45\%

Danville

Oakland

San

Francisco

San Leandro

Pleasanton

San Bruno

Hayward

Hillsborough

San Mateo

Redwood

City

East Palo

Alto

Half Moon

Bay

Fremont

Palo Alto

Milpitas

Menlo

Park

Mountain

View

Santa

Clara

San

Jose

Cupertino

Highways

Los Gatos

Residential

overcrowding

Danville

Oakland

San

Francisco

Share of housing units within each census tract with more than one
person per room

San Leandro

Daly City

Pleasanton

45

\%

Hayward

San Bruno

Millbrae

Hillsborough

San Mateo

Redwood

City

30

Half

Moon

Bay

East Palo

Alto

Fremont

Palo Alto

Milpitas

Menlo

Park

22

Mountain

View

Sunnyvale

Santa

Clara

San

Jose

13

Cupertino

8

Highways

3

0

Los Gatos

Source: Census Bureau

By Karl Russell

Visitors to Silicon Valley may take a wrong turn or freeway exit on the
way to this or that office park and find themselves in an area like the
North Central neighborhood of San Mateo, Calif. That is where Ms.
Lorenzo lives on a block of faded homes on small lots, with packed
driveways and cars parked liberally on the sidewalk. The scene is one
side of the tech economy.

For much of the peninsula stretching south from San Francisco, there is
a rough economic split. Cities and neighborhoods to the east, places
like East Palo Alto, North Fair Oaks and the Belle Haven section of
Menlo Park, are more overcrowded and have a larger share of low-income
and Black and Latino residents, many of whom have been
\href{https://www.nytimes3xbfgragh.onion/interactive/2020/07/05/us/coronavirus-latinos-african-americans-cdc-data.html}{disproportionately
affected by the virus}. Towns and neighborhoods to the west, places like
Hillsborough and Palo Alto, are whiter and rich.

This geography is as fundamental to how the place operates as the
invention of the microchip. Every day, throngs of clerks, landscapers
and elder-care workers wake up on the eastern parts and travel to homes
on the western parts or to the corporate campuses of tech companies to
do subcontracting work. And every night, they return to overcrowded
homes.

Ms. Lorenzo was one of them. She immigrated to the United States six
years ago from Guatemala with her two children, fleeing a broken
relationship and looking for a new start. Now she is a green-card holder
with a new partner and a 2-year-old. Until the pandemic hit, she made
about \$16 an hour mopping floors and vacuuming carpets in homes on the
other side of the peninsula.

For a while, her wages and Abel's were enough for their own small place
--- a \$1,600-a-month studio that had a bed for them and a shared
mattress for the children. Then the rent jumped to \$2,100. And then to
\$2,650.

The couple went looking for cheaper housing and roommates, a quest that
has become a Bay Area ritual. Since the Great Recession, a growing share
of Bay Area movers, from all but the most well-off households, have gone
to homes with four or more adults from ones with one or two adults,
\href{https://www.frbsf.org/community-development/blog/overcrowding-in-the-bay-area-where-the-housing-crisis-meets-covid-19/}{according
to a study} by researchers at Stanford University and the Federal
Reserve Bank of San Francisco.

The high-end version is dressed up with a description like ``co-living''
or explained as a culturally in-tune couple sacrificing an extra bedroom
in the suburbs for a life of less driving closer to the city. The
low-end version is poverty. Whatever it is called, the economic calculus
is the same.

Wages are higher in coastal California than in inland areas, where
housing is cheaper, so all but the very rich have to make a trade-off
between a commute and space. It is just that the choices for poorer
workers are more extreme, like
\href{https://www.nytimes3xbfgragh.onion/2017/08/17/business/economy/san-francisco-commute.html}{a
three-hour commute} from cities like Stockton or huddling together in
homes where nearly every space is the site of someone's bed.

Image

Hillsborough is one of the wealthy towns near Silicon Valley with demand
for service work.Credit...Jim McAuley for The New York Times

Image

San Mateo's North Central neighborhood is part of the San Francisco
Peninsula's denser east side.Credit...Jim McAuley for The New York Times

Researchers define extreme overcrowding as any home that is occupied by
more than one person for every room without a toilet. By this
measurement, overcrowding has increased nationwide since the mid-2000s,
and the problem is particularly acute in California. About 13.4 percent
of rental units --- more than double the national average --- were
considered overcrowded in 2018, according to the Census Bureau. San
Mateo and Santa Clara Counties, which roughly outline Silicon Valley,
have one of the world's densest concentrations of billionaires as well
as some of the country's most overcrowded homes.

After the studio, Ms. Lorenzo found a \$1,250-a-month room in her
current home, a blue stucco house at the back of a two-unit lot, with
chalk drawings on the driveway and a dirt yard in the back. There were
11 occupants after Ms. Lorenzo moved in, 12 after her younger child was
born.

Dividing the rent had benefits, like allowing Ms. Lorenzo to save money
and buy her first television. The children's shared mattress from the
studio was replaced with a new bunk bed. ``More clothes, more shoes for
the children,'' she said, ``because we were limited in many things.''

The catch was living with personalities, rules and understandings.
Cooking privileges were on a first-come basis, which meant that the last
family to use the kitchen might not eat until 9:30 p.m. There was no
official time limit on the bathroom, but people knew to be fast. If
anyone got a cold, everyone was exposed.

Crowded homes have been a concern practically as long as public health
has been a field. Living with a pile of roommates has long been
associated with faster-spreading infections, inescapable stress,
irregular sleep and the effects that follow, including higher blood
pressure and weakened immune systems.

But those take years to develop. The coronavirus spreads in days. By
moving so fast and furiously, the virus has exposed in weeks something
doctors have been worried about for generations, said Dr. Margot Kushel,
an internist and director of the Benioff Homelessness and Housing
Initiative at the University of California, San Francisco. ``Covid has
really become a story of essential workers living in crowded housing,''
she said.

\hypertarget{games-in-the-closet}{%
\subsection{Games in the Closet}\label{games-in-the-closet}}

\includegraphics{https://static01.graylady3jvrrxbe.onion/images/2020/08/01/business/01virus-crowding3/merlin_174571317_2a13c932-f841-4aff-b790-d8bd18b33a9f-articleLarge.jpg?quality=75\&auto=webp\&disable=upscale}

The sickness began, as it does, with worry.

In mid-April, after schools shut down and the children were sent home
with worksheets, Abel returned from his job with a report that two of
his co-workers had been out sick. He showered with the garden hose and
slept in the car that night. But it was too late.

His symptoms were initially mild, before escalating to a 104-degree
fever and a shortness of breath that prompted Ms. Lorenzo to take him to
the hospital. The county health department, worried that a crowded home
would accelerate the spread of what was confirmed to be the coronavirus,
dispatched a case worker to test everyone in the house, Ms. Lorenzo
said. Eight --- all except her children --- were also positive.

Ms. Lorenzo never got more than a headache and a sore throat, which in
normal times would not have even prevented her from going to work.
Suddenly she had to isolate herself in a house where everything was
shared.

She settled on the closet, running a phone charger under the door and
sitting there for six to eight hours a day, playing word games on her
phone, calling relatives in Guatemala, sometimes just napping. Her
10-year-old son took over cooking meals and changing diapers. All the
while, Abel was in the hospital. Improving or worsening, alive or dead,
Ms. Lorenzo had no idea.

``There was no communication with him, so my head was spinning,'' she
said.

Image

Ms. Lorenzo's children were the only ones living at the house who
avoided getting the coronavirus.Credit...Brian L. Frank for The New York
Times

Ms. Lorenzo sprayed down the bathroom whenever she or the children used
it. She avoided the kitchen and had her sister, who lives more than a
half-hour away in Oakland, deliver food through the bedroom window. One
time, the sister brought a thermos of hot coffee that Ms. Lorenzo said
might as well have been hot water; the virus had so ruined her sense of
taste that she could not tell the difference.

Still, the house got tense. One of the housemates accused Abel of
infecting them. She told Ms. Lorenzo that if anyone in her family died,
she would figure out a way to sue her. After that came the silent
treatment --- ``no hablaba'' --- and as house relations plummeted, Ms.
Lorenzo feared she would be evicted with nowhere to go.

After two weeks, a county health worker returned to test the house
again. Ms. Lorenzo's children were still negative, which seemed so
unlikely, given the crowding, that the county retested them several
times. All negative, she said. Worried that this luck would soon run
out, the county moved her and the children to an emergency trailer.

They lived there for nine days, leaving only to collect stale salad and
sandwiches left on an outdoor table. When they finally went home, Abel
was back from the hospital.

Days of deep cleaning ensued. Ms. Lorenzo, back to health, is wondering
when the world will return to some semblance of normality. Yet she feels
lucky that things are not worse, because she thought her partner was
going to die. ``We are trying to cope with it,'' she said. ``Trying to
leave everything in the past.''

\hypertarget{crowding-vs-density}{%
\subsection{Crowding vs. Density}\label{crowding-vs-density}}

Image

Chelsea, Mass., had one of the nation's worst coronavirus outbreaks.
``It's not how many people you run into on the street but how many
people you see when you come home,'' an affordable-housing provider
there said.Credit...Brian Snyder/Reuters

Early in the outbreak,
\href{https://www.nytimes3xbfgragh.onion/2020/03/24/upshot/coronavirus-urban-density-risks.html}{Gov.
Andrew M. Cuomo} of New York and
\href{https://www.latimes.com/opinion/story/2020-04-26/coronavirus-cities-density-los-angeles-transit}{some
commentators} blamed dense housing and public transit for the spread of
the virus. The proof seemed as intuitive as New York's status as an
\href{https://www.nytimes3xbfgragh.onion/2020/03/22/nyregion/Coronavirus-new-York-epicenter.html}{early
epicenter.} The recent surge of cases in the more sprawling metropolitan
areas of
\href{https://www.nytimes3xbfgragh.onion/2020/06/14/us/coronavirus-united-states.html}{the
South and the West} has undercut that thesis, and
\href{https://jamanetwork.com/journals/jama/fullarticle/2767631?widget=personalizedcontent\&previousarticle=2768532}{a
number} of
\href{https://www.tandfonline.com/doi/full/10.1080/01944363.2020.1777891?scroll=top\&needAccess=true\&}{new
studies} suggest that density, the number of housing units per acre, is
less important than crowding, the number of people per bedroom.

One
\href{https://furmancenter.org/thestoop/entry/covid-19-cases-in-new-york-city-a-neighborhood-level-analysis}{widely
cited report} was from New York University's Furman Center, which found
that infections were much more intense in Queens neighborhoods with high
rates of overcrowding than in Manhattan neighborhoods with higher
density but fewer people per unit. The link between crowding and
transmission has since shown up in suburbs,
\href{https://www.wsj.com/articles/covid-19-households-spread-coronavirus-families-navajo-california-second-wave-11591553896}{rural
America} and
\href{https://www.azcentral.com/story/news/local/arizona-health/2020/05/12/close-knit-navajo-nation-communities-arizona-virus-takes-hold/3115961001/}{Native
American reservations}. There is even some evidence that dense
metropolitan counties, while suffering higher raw numbers of infections,
have a
\href{https://www.ncbi.nlm.nih.gov/pmc/articles/PMC7315990/}{lower death
rate} because it is easier to get to a hospital.

San Mateo County has been a bright spot, with a rate of about 700
coronavirus cases per 100,000,
\href{https://www.nytimes3xbfgragh.onion/interactive/2020/us/california-coronavirus-cases.html}{about
half the rate of the state}. Still, the county's cases have been
concentrated in low-income households, with most coming lately from
front-line workers who ``live in crowded multigenerational conditions,''
according to the
\href{https://www.smchealth.org/coronavirus-health-officer-updates}{county
health officer}.

In Chelsea, Mass., which had one of the nation's
\href{https://www.nytimes3xbfgragh.onion/2020/04/25/us/coronavirus-chelsea-massachusetts.html}{worst
outbreaks}, there is a compelling suggestion that less-crowded quarters
can help control the spread. Sleeved into the same blocks where
buildings were overrun with infection are 375 subsidized apartments
owned by The Neighborhood Developers, a housing nonprofit. The 968
tenants are mostly nonwhite, have the same mix of low-paid service jobs
as their neighbors, and live in multistory buildings. But their units
are subsidized and less crowded --- and so far, healthier.

The Neighborhood Developers has had eight reported cases of the
coronavirus in Chelsea, or 826 per 100,000 people, about
\href{https://www.mass.gov/doc/weekly-covid-19-public-health-report-july-22-2020/download}{a
tenth the rate} of the surrounding community. ``It's not how many people
you run into on the street but how many people you see when you come
home,'' said Rafael Mares, executive director of The Neighborhood
Developers.

The story is tempered by its rarity. The United States has a deficit of
seven million apartments available to the lowest-income households, or
an average of 36 available affordable units for every 100 extremely
low-income families in search of one, according to the
\href{https://reports.nlihc.org/sites/default/files/gap/Gap-Report_2019.pdf}{National
Low Income Housing Coalition}.

In April of last year, The Neighborhood Developers opened a five-story
building with 34 apartments for homeless and low-income families. It
received 3,598 applications.

\hypertarget{rent-is-due}{%
\subsection{Rent Is Due}\label{rent-is-due}}

Image

Before the pandemic, Ms. Lorenzo's sister used three red-and-white
coolers to sell ice cream on the street. Now they are stacked against a
wall, a reminder of lost income.Credit...Brian L. Frank for The New York
Times

Stacked against a wall in Ms. Lorenzo's living room are three
red-and-white coolers that her sister used to fill with ice cream to
sell on the street. They are furloughed because of the lack of demand
and have become just another obstacle that her cooped-up children have
to dodge while zipping around the house.

Abel still gets headaches and a tremor in his left arm, but the virus is
gone and he is well enough to work. Ms. Lorenzo has not cleaned a house
since March but recently got a new job cleaning offices. The family has
also been relying on nonprofit organizations and Christian charities for
staples.

Once a week Ms. Lorenzo joins the procession of cars that roll through a
parking-lot food bank set up by Samaritan House, a San Mateo-based
organization that has seen demand for food double and is spending
\$200,000 a week on rental assistance. Since April 1, 4,000 families
have applied for some \$8 million in assistance on rent and utilities
``and it hasn't even really hit yet,'' said Bart Charlow, Samaritan
House's chief executive.

Ms. Lorenzo's name could soon be on the list. In June, the departure of
the angry housemates opened up an extra bedroom, and her family spread
out, with the older children moving across the hall --- the sort of
arrangement that the San Mateo County Health Department has been
recommending \href{http://www.gethealthysmc.org/healthy-housing}{for
years}, except that it is financially unsustainable.

After taking the extra bedroom, Ms. Lorenzo's family's share of the rent
jumped to \$2,100 from \$1,250. Their savings got them through July. Now
that money is gone, and August is here.

Liliana Michelena and Ben Casselman contributed reporting.

Advertisement

\protect\hyperlink{after-bottom}{Continue reading the main story}

\hypertarget{site-index}{%
\subsection{Site Index}\label{site-index}}

\hypertarget{site-information-navigation}{%
\subsection{Site Information
Navigation}\label{site-information-navigation}}

\begin{itemize}
\tightlist
\item
  \href{https://help.nytimes3xbfgragh.onion/hc/en-us/articles/115014792127-Copyright-notice}{©~2020~The
  New York Times Company}
\end{itemize}

\begin{itemize}
\tightlist
\item
  \href{https://www.nytco.com/}{NYTCo}
\item
  \href{https://help.nytimes3xbfgragh.onion/hc/en-us/articles/115015385887-Contact-Us}{Contact
  Us}
\item
  \href{https://www.nytco.com/careers/}{Work with us}
\item
  \href{https://nytmediakit.com/}{Advertise}
\item
  \href{http://www.tbrandstudio.com/}{T Brand Studio}
\item
  \href{https://www.nytimes3xbfgragh.onion/privacy/cookie-policy\#how-do-i-manage-trackers}{Your
  Ad Choices}
\item
  \href{https://www.nytimes3xbfgragh.onion/privacy}{Privacy}
\item
  \href{https://help.nytimes3xbfgragh.onion/hc/en-us/articles/115014893428-Terms-of-service}{Terms
  of Service}
\item
  \href{https://help.nytimes3xbfgragh.onion/hc/en-us/articles/115014893968-Terms-of-sale}{Terms
  of Sale}
\item
  \href{https://spiderbites.nytimes3xbfgragh.onion}{Site Map}
\item
  \href{https://help.nytimes3xbfgragh.onion/hc/en-us}{Help}
\item
  \href{https://www.nytimes3xbfgragh.onion/subscription?campaignId=37WXW}{Subscriptions}
\end{itemize}
