Sections

SEARCH

\protect\hyperlink{site-content}{Skip to
content}\protect\hyperlink{site-index}{Skip to site index}

\href{https://www.nytimes3xbfgragh.onion/section/business}{Business}

\href{https://myaccount.nytimes3xbfgragh.onion/auth/login?response_type=cookie\&client_id=vi}{}

\href{https://www.nytimes3xbfgragh.onion/section/todayspaper}{Today's
Paper}

\href{/section/business}{Business}\textbar{}Like Old Hollywood Movies,
Video Games Get a Polish for New Audiences

\url{https://nyti.ms/2BRz42w}

\begin{itemize}
\item
\item
\item
\item
\item
\end{itemize}

Advertisement

\protect\hyperlink{after-top}{Continue reading the main story}

Supported by

\protect\hyperlink{after-sponsor}{Continue reading the main story}

\hypertarget{like-old-hollywood-movies-video-games-get-a-polish-for-new-audiences}{%
\section{Like Old Hollywood Movies, Video Games Get a Polish for New
Audiences}\label{like-old-hollywood-movies-video-games-get-a-polish-for-new-audiences}}

Game publishers are digging through their vaults to remake or remaster
popular titles in a bid to kindle players' nostalgia.

\includegraphics{https://static01.graylady3jvrrxbe.onion/images/2020/07/22/business/00Remake-RE2-lede/00Remake-RE2-lede-articleLarge-v2.jpg?quality=75\&auto=webp\&disable=upscale}

By Imad Khan

\begin{itemize}
\item
  Aug. 1, 2020, 5:00 a.m. ET
\item
  \begin{itemize}
  \item
  \item
  \item
  \item
  \item
  \end{itemize}
\end{itemize}

Nostalgia has always been a powerful source of revenue for Hollywood.
Turns out, it's equally lucrative for video games.

From its beginnings with the likes of Pong, a two-dimensional table
tennis game, the video game industry has grown into a
\href{https://www.businessinsider.com/video-game-industry-120-billion-future-innovation-2019-9}{\$120
billion} business. Over the years, memorable games have garnered strong
followings. Like Hollywood remakes or remasters old movies, video game
publishers are overhauling and rereleasing games to tap into ready-made
fan bases for popular franchises like The Legend of Zelda, Crash
Bandicoot, Spyro the Dragon and World of Warcraft.

``I think nostalgia is the major driving force for the success of a
remake,'' said Doug Clinton, managing partner for the venture capitalist
firm Loup Ventures, which focuses on emerging technology and gaming.
``Any game that doesn't have meaningful nostalgic value isn't likely to
be successful.''

In May, Activision Blizzard, the developer behind World of Warcraft,
announced that two games from the Tony Hawk's Pro Skater series,
originally released in 1999, would be
\href{https://www.barrons.com/articles/activision-is-bringing-back-tony-hawks-skateboarding-games-51589364900}{brought
back later this year}.

The remake trend isn't extending only to the most highly rated games
either. Children (and adults) who received SpongeBob SquarePants: Battle
for Bikini Bottom in Christmas of 2003 can now buy a ``rehydrated''
remake, which hit stores in June. Though the game received decent
reviews when it was first released, it was by no means a classic. But
the remastering shows how nostalgia is driving publishers' decision
making.

``Because you can actually revisit those virtual spaces, it's a more
powerful type of nostalgia,'' said Alyse Knorr, assistant professor of
English at Regis University and author of the book
``\href{https://bossfightbooks.com/products/super-mario-bros-3-by-alyse-knorr}{Super
Mario Bros. 3}.'' ``It's the same when you go back to it; it's the same
as it was when you were 7.''

That sentimentality does not necessarily lead to instant sales. Some
titles that have been rereleased or remastered in hopes of cashing in on
cult status fall back into obscurity, like 2017's Constructor HD or
White Day: A Labyrinth Named School. Generally, games that have high
review scores and strong followings tend to be safe financial bets for a
second look.

``When you're taking a game that you know has a Metacritic of 90-plus,
the only thing you can do at that point is screw it up,'' said Marco
Thrush, president of Bluepoint Games, a studio known for developing high
quality remasters and remakes.

Initially, publishers capitalized on the nostalgia trend by curating
games from the 1990s on plug-and-play devices. Nintendo's
\href{https://www.nytimes3xbfgragh.onion/2016/07/16/technology/nintendo-nes.html}{NES
Classic}, which offered 30 games like Super Mario Bros. and Donkey Kong
packaged in a replica of the original Nintendo Entertainment System
console,
\href{https://www.nytimes3xbfgragh.onion/2016/11/27/business/nintendos-new-console-may-feed-your-nostalgia-if-you-can-get-one.html}{was
a huge hit} when it came out in 2016, selling out almost immediately.
Other developers like Sega and Sony
\href{https://www.nytimes3xbfgragh.onion/2018/12/02/business/retro-video-games.html}{quickly
followed suit}.

But developers saw an opportunity to make even more money by investing
in substantial upgrades. One of the biggest this year was the release of
Final Fantasy VII Remake.

\includegraphics{https://static01.graylady3jvrrxbe.onion/images/2020/07/22/business/00REMAKE/00REMAKE-articleLarge.jpg?quality=75\&auto=webp\&disable=upscale}

Image

The original Final Fantasy VII~had no voice acting, blocky-looking
characters and no 3-D backgrounds.Credit... .

In 1997, Square Enix released the original Final Fantasy VII, a
futuristic cyberpunk epic with multiple characters and twisting
plotlines that became one of the most beloved titles in the Final
Fantasy series.

Visually, however, the creators had to make do with the technology at
the time. For example, the game had blocky-looking characters, no voice
acting and no 3-D backgrounds.

After years of teasing, Square Enix remade the game to match a modern
experience. Final Fantasy VII Remake used entire teams of voice actors,
artists, animators, engineers and producers to create a game that could
stand up to any contemporary release.

The strategy paid off: It became the best-selling game of April,
according to data from the NPD Group, a research firm that covers the
video game industry.

Fans have largely been receptive to the reimagined game, and its modern
systems have made it accessible to new players, who found the original
mechanics difficult.

``I tried the Final Fantasy VII remaster on Xbox; it was a little too
far gone for me,'' said Preston Bakies, 27, of Findlay, Ohio. ``But when
the remake came out --- I've put a lot of time into it. It's been a lot
of fun.''

The original Final Fantasy VII
cost\href{https://www.polygon.com/a/final-fantasy-7}{\$40 million to
make}, which was considered a high sum for a video game in the `90s.
Given the technological demands of modern games, costs have grown
considerably more expensive, experts say.

``I haven't come across a single game which took more than \$100 million
in Japan'' to get made, said Atul Goyal, a managing director at
investment bank Jefferies \& Company, who pegged the budget for Final
Fantasy VII Remake at up to \$140 million.

Others felt it was even higher. ``If we assume the number of sales for
Final Fantasy VII Remake is six million units, \$144 million is the
budget,'' said Yuhsuke Koyama, a professor at Shibaura Institute of
Technology in Tokyo and author of
``\href{https://www.amazon.co.jp/dp/4409241079/ref=cm_sw_r_tw_dp_U_x_NWONEbES7SRZ9}{A
History of the Japanese Video Game Industry}.''

In a twist, Square Enix has broken Final Fantasy VII Remake into
multiple parts, although it would not say how many. There are risks
associated with this strategy, including irking fans who have to shell
out more money for the other parts of the game.

``We are writing in our reports that it will be a two-part series. Not
three, not four, not 10,'' Mr. Goyal said. ``And the subsequent chapter
will be coming out soon in the next fiscal year.''

Image

Capcom revealed in a Japanese documentary that at least 800 developers
worked on the remake of Resident Evil 2.Credit...Capcom

Image

The original Resident Evil 2 was released in 1998 for the PlayStation
console.Credit...Capcom

Square Enix is not the only publisher capitalizing on this trend.
Capcom, the publisher behind Street Fighter and Mega Man, has also been
rummaging through its back catalog. Last year, it released Resident Evil
2, a remake of the 1998 PlayStation original. Not only was the remake
loved by critics, it has sold
\href{http://www.capcom.co.jp/ir/english/news/html/e200413.html}{6.5
million units} as of April.

The success prompted Capcom to greenlight a Resident Evil 3 remake,
which was released in April.

Capcom declined to comment for this article, but it did
\href{https://comicbook.com/gaming/news/resident-evil-2-remake-dev-team-size-over-800/}{reveal}
in a Japanese documentary that at least 800 developers worked on
Resident Evil 2.

Mr. Goyal estimates that Resident Evil 2 likely did not cost more than
\$100 million to remake. And given that Resident Evil 3 reused some
assets from its predecessor, and clocks in at a shorter run time, it
likely cost Capcom even less.

The savings for Resident Evil 3 would have been significant, said
Michael Pachter, an analyst at Wedbush Securities who follows the video
game industry.

``The level design is the complicated part, art is relatively
inexpensive, probably 30 percent of the cost of the game,'' he said.
``If they reused 20 percent of the art, it's a 6 percent savings.''

Resident Evil 3 was not the same breakout success that
\href{https://venturebeat.com/2019/02/20/january-2019-npd-sales-down-as-kingdom-hearts-tops-charts/}{Resident
Evil 2 was} in 2019, but it still landed at a respectable sixth place in
April, according to NPD. Capcom did confirm that it shipped two million
units, and it has already announced that Resident Evil 4 will be getting
a remake.

Remastering and remaking have become so common that some studios are
dedicated to bringing old games to modern hardware. Bluepoint Games in
Austin, Texas, has a reputation for creating some of the highest quality
updates in the industry. In 2018, it released a high-definition remake
of Shadow of the Colossus, which originally came out in 2005 by Sony.

Image

Bluepoint Games has reputation for creating some of the highest quality
updates in the industry, including its remake of Shadow of the
Colossus.Credit... Bluepoint Games

Image

The original Shadow of the Colossus, which came out in 2005, but was
revamped in 2011 and again in 2018.Credit... Bluepoint Games

Bluepoint revamped the game in 2011, bringing the original up to 1080p
standards, then substantially reworked it again in 2018 for 4K
televisions. Mr. Thrush, Bluepoint's president, declined to reveal the
costs of remaking the game.

``We revitalize an older game, somebody's baby,'' said Mr. Thrush. ``New
gamers get to play games they otherwise wouldn't.''

Advertisement

\protect\hyperlink{after-bottom}{Continue reading the main story}

\hypertarget{site-index}{%
\subsection{Site Index}\label{site-index}}

\hypertarget{site-information-navigation}{%
\subsection{Site Information
Navigation}\label{site-information-navigation}}

\begin{itemize}
\tightlist
\item
  \href{https://help.nytimes3xbfgragh.onion/hc/en-us/articles/115014792127-Copyright-notice}{©~2020~The
  New York Times Company}
\end{itemize}

\begin{itemize}
\tightlist
\item
  \href{https://www.nytco.com/}{NYTCo}
\item
  \href{https://help.nytimes3xbfgragh.onion/hc/en-us/articles/115015385887-Contact-Us}{Contact
  Us}
\item
  \href{https://www.nytco.com/careers/}{Work with us}
\item
  \href{https://nytmediakit.com/}{Advertise}
\item
  \href{http://www.tbrandstudio.com/}{T Brand Studio}
\item
  \href{https://www.nytimes3xbfgragh.onion/privacy/cookie-policy\#how-do-i-manage-trackers}{Your
  Ad Choices}
\item
  \href{https://www.nytimes3xbfgragh.onion/privacy}{Privacy}
\item
  \href{https://help.nytimes3xbfgragh.onion/hc/en-us/articles/115014893428-Terms-of-service}{Terms
  of Service}
\item
  \href{https://help.nytimes3xbfgragh.onion/hc/en-us/articles/115014893968-Terms-of-sale}{Terms
  of Sale}
\item
  \href{https://spiderbites.nytimes3xbfgragh.onion}{Site Map}
\item
  \href{https://help.nytimes3xbfgragh.onion/hc/en-us}{Help}
\item
  \href{https://www.nytimes3xbfgragh.onion/subscription?campaignId=37WXW}{Subscriptions}
\end{itemize}
