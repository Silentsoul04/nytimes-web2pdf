\href{/section/opinion}{Opinion}\textbar{}Talking to the Dead in the
Sunshine State

\url{https://nyti.ms/316Q517}

\begin{itemize}
\item
\item
\item
\item
\item
\end{itemize}

\includegraphics{https://static01.graylady3jvrrxbe.onion/images/2020/08/01/opinion/sunday/01gerard-top/01gerard-03-articleLarge.jpg?quality=75\&auto=webp\&disable=upscale}

Sections

\protect\hyperlink{site-content}{Skip to
content}\protect\hyperlink{site-index}{Skip to site index}

\href{/section/opinion}{Opinion}

\hypertarget{talking-to-the-dead-in-the-sunshine-state}{%
\section{Talking to the Dead in the Sunshine
State}\label{talking-to-the-dead-in-the-sunshine-state}}

I was searching for facts, for a lost friend, and for what couldn't be
explained.

Credit...Illustrations by Xia Gordon

Supported by

\protect\hyperlink{after-sponsor}{Continue reading the main story}

By Sarah Gerard

Ms. Gerard is a novelist.

\begin{itemize}
\item
  Aug. 1, 2020
\item
  \begin{itemize}
  \item
  \item
  \item
  \item
  \item
  \end{itemize}
\end{itemize}

\begin{center}\rule{0.5\linewidth}{\linethickness}\end{center}

Image

\textbf{I. THE ROAD TO CASSADAGA}

It was August of last year, and I was driving north through the Florida
swamp to the Cassadaga Spiritualist Camp, also known as the psychic
capital of the world. I had flown from New York City to Orlando and
rented a car at the airport. It was midafternoon and so, as usual,
raining, though the sun was still spilling through live oaks. The air
was thick, the roadside draped with sagging Spanish moss. I was going to
commune with the dead.

It was sunflower season back home, and I had taken the proliferation of
this symbol of Spiritualism in my New Jersey neighborhood as a sign that
I should make this long-anticipated trip.

Spiritualism, a movement that flourished in the late 19th and early 20th
centuries in America as a sort of alt-religion, today is mainly
associated with the paranormal --- auras, crystals, ghosts, Ouija boards
--- but to me it's more familiar. I grew up in Tampa Bay, two hours
southwest of Cassadaga, and was raised by my parents in the New Thought
Movement --- which, along with Christian Science, grew in part out of
early practices in mesmerism and mind-cure, as did Spiritualism. I
practiced metaphysics as a child, and had been instilled with a belief
in the power of my own mind to shape material reality. I had always been
curious about the Spiritualists at Cassadaga. Friends would drive across
the state to spend weekends there, especially around Halloween. Maybe
because of this upbringing, I approach Spiritualism and its phenomena
with an open mind: I am a skeptical believer.

Cassadaga is an unincorporated community of 13 historically registered
city blocks with about 100 full-time residents, and is the oldest
continuously operating Spiritualist camp in the South. It's located in
Trump Country: Volusia County, home of the Daytona 500, which is heavily
evangelical. As the legend goes, the town was founded by George P.
Colby, an itinerant trance medium from Pike, N.Y., who in 1875 was
visited by a Native American spirit guide calling himself Seneca. The
spirit instructed Colby to travel to Florida, and search for a place
with seven hills, fed by lakes and springs. When he found it, he named
it after the Cassadaga Lake Free Association, a Spiritualist community
in New York. The name means ``water beneath the rocks'' in the Seneca
language. There are no town lines marking Cassadaga; it is more an idea
than a precise location.

When I moved to New York City to attend college I was eager to escape
the gun-toting, overt racism and religious conservatism of that I
experienced in the South. But over time, I find myself increasingly
called back to my primordial homeland.

Now I was going in search of facts, but also what couldn't be explained.
The presidential race was beginning to heat up, and Americans couldn't
seem to agree on what was real. I had also lost a friend the year
before, a death that felt unresolved. I wanted to know what Americans
believed --- what I believed --- what it even means to believe. And I
knew that belief is rarely based in fact.

Image

\textbf{II. THE REVEREND FROM WALL STREET}

Once inside Cassadaga, I walked the length of Stevens Street, which
divided the two sides of the town: the Cassadaga Hotel on one side, the
Spiritualist camp on the other. On the way, I passed wooden bungalows
with lush gardens, structures erected at the end of the 19th century as
boardinghouses for Spiritualists wintering from northern camps like Lily
Dale, in New York, and Camp Etna, in Maine.

My first meeting was with the Rev. Claire Van Cott, who had recently
bought and was refurbishing one of the camp's historic homes. Only
Spiritualists approved by the board of trustees are allowed to buy them,
and they seldom go up for sale. The Spiritualists have to undergo a
certification process that can take four to six years; it includes an
academic curriculum on the history of Spiritualism, as well as a lot of
soul-searching and character development.

The reverend had been a Wall Street stockbroker for 25 years, and moved
to Florida after the Sept. 11 attacks. Soon after, she retired from
finance and devoted herself full time to mediumship and lecturing at
churches throughout the state.

Her house was in the process of having its gray clapboards painted
amethyst. She invited me to join her on the sun porch, where I took note
of a Buddha, a stained glass butterfly, a wind chime, a crystal window
ornament, a basket of rose quartz, and a wooden angel sculpture
scattered among stacks of New Age CDs.

She set a timer between us and instructed me to respond to each of her
statements with, `Yes,'' ``No,'' or, ``I don't know,'' and not to
elaborate further. This reminded me of a common sales tactic: get
someone to say yes the first time, and the second time is much easier.
Few people will feel comfortable saying no; they are much more likely to
say ``I don't know'' --- another form of yes. This forces the listener
into an active role in the crafting of the narrative.

She asked if there was anyone I wanted to visit with, and I asked her
about Daniel. Daniel was a friend from college; we were close but had
drifted apart over the years, as he became consumed by alcoholism and
trouble with the law. He died in September 2018, and the anniversary of
his death was approaching. I didn't know if his death was accidental or
if he meant to take his own life. I hadn't communicated with him for
years, and did not return his last phone call, made the previous March.
I had told him why in a text message. The last message he sent me, just
weeks before he died, read, \emph{I still think of you and hope you're
well.}

She asked me how old he was when he died, and I said 34.

``He can be a pretty serious person. Would that be correct?'' she said.

I agreed.

She asked more questions. Whether Daniel had dark hair (yes), sometimes
wore glasses (I wasn't sure).

``Would you understand that he's not a big fan of entertaining people at
his house?'' she asked.

``He's not?''

``That's a question.''

I considered it.

``In other words, did you go to his home?''

Yes. Daniel had often hosted parties at his apartment in college. And
the last time I saw him was at his parents' house in North Carolina,
nine years before he died. A friend and I had stayed there while on a
road trip to a concert. Daniel had been living with his parents for
several years. After a late night partying, I awoke to find him in my
bed, which he refused to leave. He groped me, held my wrists and said,
``I'm not done with you, yet,'' insisting he wanted to have sex --- at
which point, my friend, pretending to sleep in the next bed, moved over
to make room for me. ``You stay here, Daniel,'' I said, indicating the
bed we were in. I pried his hands from my wrists, climbed out of that
bed, and slid into hers.

Not yet thinking of this, though, I said, ``He was homeless for a while
before he died, so I couldn't go to his house. '' Which was true.

``I feel like he's a private kind of person,'' she said. ``Does that
make sense?''

``Yes.''

``He's the kind of person who does not want people to know everything
about him. If he's not a private drinker, then he's a private something
else. So, did he have addictive issues?''

``Yes.''

``I want to give you the opportunity to ask questions if you have any.''

I asked if she could tell me how he died.

She hesitated. ``From Daniel's perspective, he would have to take
responsibility for his own passing,'' she said. ``I don't mean he put a
gun to his head. What I mean is that he neglected himself to such a
medical state. Would you understand that with him?''

``Yes.''

``Sometimes I find --- this is a generalization --- that people who
participate in their own passing do not necessarily always want to go
there.''

Image

\textbf{III. DO YOU BELIEVE IN MAGIC?}

After the reading I walked to the Cassadaga Hotel. I had read that it
was haunted, but also that it had been renovated and was not the
original structure. I was planning to stay there that night and was
hoping to see a ``psychic imprint, '' a common apparition in which the
emotion of what happened in a particular space imprints so deeply on the
energy of the area that the scene plays out over and over, eternally.
But I wasn't sure whether, as with the Argo, Jason's ship of myth, a
haunted structure remained haunted even if all of its original parts
were replaced.

I checked in at the gift shop, which doubles as the front desk. The
crone at the register recorded my credit card number in pencil in a
ledger book. She looked at me. ``Your room has two doors,'' she said.

``One of them goes out to the porch, and one of them opens up on the
hallway. We have a very spirited hotel.'' She smirked. ``You just never
know who's going to walk away with your keys.''

In my room, I found a sign on the door warning that there was limited
hot water. I considered keeping my luggage in the bathroom rather than
on the carpet, to avoid bedbugs, but there wasn't enough floor space in
the bathroom to do that. Plus, the linoleum was peeling.

I went looking for dinner at Sinatra's L'aldila Ristoranté, which
occupies a third of the hotel's first floor. A lounge singer accompanied
himself dramatically on keyboard, sounding as though Antony and the
Johnsons were covering ``My Way.'' About 30 people, mostly over 50,
mostly white, filled some of the tables. I spaced out on the
color-changing lights. The singer segued into ``Hey Jude'' and the
patrons sang along to the ``na-nas.''

I sat at a table and began taking notes, and noticed an older couple on
the other side of the keyboardist watching me. The man appeared to be in
his early 80s, the woman a good deal younger. He wore a Hawaiian shirt
tucked into wide-leg khaki shorts. They approached me and asked if I was
a food writer here to review Sinatra's. I asked if we could talk outside
on the wraparound porch, where the night was an inky black but for
yellow light from the hotel and the white glow of the Spiritualist camp
across the street. Two of the village's stray cats knelt at a pool of
water reflecting the moon.

The man introduced himself as John Platania, and his companion as Patti
Young. They both lived nearby in DeLand and were frequent visitors to
Cassadaga, but I was surprised to learn they weren't married. ``We're a
dancing couple,'' Patti said.

Patti had identified as a Southern Baptist until her divorce six years
before, and was now studying chakra healing under Katharina ``Kat''
Moonchild, who rented a room upstairs in the hotel. I had a reading
scheduled with Moonchild the next day.

John had started coming here six years ago after seeing an ad for the
piano bar in a coupon book. He was looking for a new place to kill time
since his partner died. He couldn't go to the places they went to
anymore. I was learning that most people came to Cassadaga with their
grief.

``But John doesn't believe in this stuff,'' said Patti.

``No, I don't,'' he said. He removed a worn deck of playing cards from
his pocket. ``Now,'' He turned to me. ``I'm going to show you a card,''
he said.

He proceeded to perform several tricks for me. Before my eyes, the deck
became entirely made of Sevens of Diamonds, then changed back again, to
52 different cards. He moved my Queen of Clubs from his hand to my hand,
to the table, back into the deck, now facing upward.

``No way,'' I said, disbelieving.

``What does it mean that you `don't believe this stuff?''' I asked,
pointing out that he was performing a magic trick.

``There's no such thing as magic,'' he said. ``It's done through hands.
It's mathematics.''

He paused.

``I mean, some of it might be true.'' He burst into laughter.

I asked if he was religious. He told me he was Catholic --- then he
began a sentence with, ``Young kids today,'' and griped for a long time
about how they don't go to church. This mutated into a story about
attending service at an African-American church in Washington, D.C.,
just after Donald Trump's election.

``I'm a white guy in a Black church, and I'm a Trump man, and I thought
I'd get crucified.'' He laughed. ``Instead, I was loved.''

I asked John where he sees the election going.

``I see a bunch of people that are not Americans running for
president,'' he said. ``I see three women in particular that hate
America, running for president. That one from New York, man, she's an
idiot.''

I asked if he meant Kirsten Gillibrand.

``No, what's her name there, the one from the Bronx.''

I told him Alexandria Ocasio-Cortez wasn't running for president.

``Not anymore.''

``She was never running for president,'' I said, turning to go inside.

He followed me, swiping through cellphone pictures of attractive female
companions. I attempted to say goodbye at the door of the restaurant,
but he pulled me onto the dance floor and held me there in his grip
until the end of ``Don't Stop Believin'.''

Image

\textbf{IV. GOOD VIBRATIONS}

I awoke on the second day having heard no bumps in the night. I was on
the porch awaiting my appointment with Kat Moonchild. The rain had let
up but hung in the air, threatening to come down again.

Since browsing the gift shop the day before, I was beginning to wonder
whether rose quartz might shift me to a higher vibration --- a
faster-moving energy field expressing love and peace. When I was a girl
I collected rocks and gems. I had amassed a huge number by the third
grade, when I made a science project displaying my specimens. My teacher
gave me a C on my project: I'd failed to understand the assignment,
which was to conduct an experiment: a series of discrete actions
designed to create a repeatable outcome. An experiment has to prove
something, she said. But I didn't know how to prove what simply made
sense to me.

A man joined me, smoking a cigar. He sat in a rocking chair and stroked
a black cat that came up to greet him. He had a mustache and gray hair,
and a monogram on his shirt that read CST-100 STARLINER. He introduced
himself as Gary Wedekind, a rocket engineer building new spacecraft at
Kennedy Space Center, down the coast. Like me, he was awaiting a psychic
reading.

``You're a man of science,'' I said, surprised he'd see a psychic.

``It's a different art, a different skill.'' He and his wife had both
grown up in DeLand and first came to Cassadaga as teenagers. They had
been taking classes here since they'd started dating, he said. ``It's
just there,'' he said, describing what he experiences when he has a
vision. ``Like you sitting here in front of me. You see those things
manifesting.''

``With your eyes or in your mind?''

``In your mind. You close your eyes.'' He closed his own eyes to
demonstrate. ``I see you sitting there, and I would be able to see other
things or other events going on around you. It's not a visual thing.''

A woman came to fetch him for his reading. He left me on the porch
hypothesizing that what I called a spontaneous act of imagination, and
do dozens, if not hundreds of times a day as a professional writer, was
really a form of prophecy. That perhaps intuition, in life as in my
writing, like scientific data, was simply information --- or simply
language. If artist and psychic were not so different, then it followed
that a psychic reading was a form of relational aesthetics, or
co-creation based on our social context. It followed that our life
itself was the material.

Image

\textbf{V. THE WORK OF GOD}

I found Kat Moonchild upstairs in a small room furnished with what
appeared to be several altars of draping fabrics, dried flowers, cigar
boxes, candles, wind chimes, seashells, dragon figurines, dream catchers
and a variety of oracle cards spread about in stacks on the floor.
During the Great Depression the Spiritualist camp sold the Cassadaga
Hotel to a private owner, which eventually led to the building being
sold again and renovated. Though many psychics still operated out of the
hotel, they didn't undergo any certification process. It was clear that
the Spiritualists from the camp across the street saw them as hucksters.

Moonchild and I sat cross-legged facing each other. She was in her early
30s, and wore a long, Hare Krishna-orange skirt, and a lace vest, with a
loose braid over her shoulder. She told me that she reads ``based on
`law of attraction,''' borrowing the titular phrase of a book by New
Thought authors Esther and Jerry Hicks, by which I gathered she meant we
would be discussing manifestations in my life, rather than communicating
with the dead.

Moonchild laid seven cards on the cloth between us. She saw my divorce
in the Ten of Swords. She then identified, in the Two of Cups, my
current partner, and my excitement about our upcoming wedding. She
proceeded to describe the dynamic of attachment in our relationship to a
rather uncanny level of accuracy. So far, I saw little difference in
skill between the readers of the hotel and the camp.

She then took out her phone and navigated to a general area on Google
Maps, around the Arizona-California border. My partner and I would move
there, she said, which puzzled me.

I asked if she'd ever faced discrimination for her beliefs.

``My spiritual belief system has nothing to do with the fact that I do
readings,'' she corrected me. ``There are Catholic readers, there are
Wiccan readers, there are people in Santeria who are readers. There are
Christian readers, there are nondenominational readers. There's Agnostic
readers, there are Spiritualist readers. This is a craft or a trade.''

This struck me as different from the mediums and healers of the camp,
whose approach to reading followed from their spiritual beliefs.
Moonchild's practice was her career. She was able to compartmentalize it
from her personal philosophy.

``But yes, we've had people come and protest,'' she said. ``It's
basically, `This is not the work of God.'''

Image

\textbf{VI. GHOSTED}

I returned to the camp for the ghost photography tour. Dawn Medley, the
camp's activity director, set up a row of screw-on flashlights at the
front of the fellowship hall, one of them fitted inside a teddy bear.
She twisted each of their caps just enough to sever the connection,
explaining that ``spirits can come in and use their energy to complete
that circuit.'' She instructed us to cheer when they flickered on. ``In
order for a spirit to manifest in this physical world, they have to
utilize energy,'' and we would be providing it. It was simple, she said,
according to the Law of Conservation of Energy: energy the spirits use
must, after all, come from somewhere.

She led us through a slide show of ghostly photographs, coaching us in
how to see them. She was interrupted now and then with scattered
applause in reaction to the flashlights --- first enthusiastic, then
less so. I questioned our collective belief in the flashlights if our
enthusiasm could wane so quickly, but it occurred to me that, like
saying the ``Pledge of Allegiance,'' their purpose might be less to
prove the existence of spirits than to foster group cohesion, and
thereby, collectively, bring spirits forth.

We gathered outside. The night was thick and buggy; Dawn passed around
mosquito repellent. Most of what we'd see in the pictures would be orbs,
she said, manifestations of spirit, which skeptics explain as
photographic backscatter. People took their cells from their pockets and
turned on the flash.

We proceeded toward a ``hot spot,'' by Spirit Pond. Dawn warned that it
would be even darker, and denser with biting insects. She marched
through high grass to the ``Portal Trees'': two palms at the edge of the
pond, pulsing with frogs and cicadas. One-by-one, she instructed us to
assume our positions between them. ``We have energy centers in the palms
of our hands, so I ask that everybody, in a relaxed state, turn them
upward,'' she said. ``I'm going to cue you to invite someone specific
in.''

When it came my turn, I stepped to the water. I turned my palms upward
and closed my eyes. I invited Daniel into my energy. I heard, in his
voice, the text message he sent me when I failed to return his last
phone call --- when I'd explained that my reason was his assault. He was
a slave to his addiction then, he'd explained. \emph{I could regale you
with the devastation and pain I endured during the several years I spent
after you saw me, but it has little to do with how I currently feel
about what happened.} Through my eyelids, the darkness flared.

Image

\textbf{VII. TAKE ME TO CHURCH}

I was in the fellowship hall for the Sunday morning lyceum. The room
looked like the modest sanctuary of a rural Southern church: wood
panels; long tables draped in tablecloths patterned with sunflowers; and
an old man, the Rev. Louis Gates, at the podium. He wore a gray suit and
a heavy gold necklace. ``Mediumship and healing is about getting
yourself to a higher vibration,'' he said.

I proceeded to the back of the room, where a handful of Spiritualists
stood over straight-backed wooden chairs, and I sat in one presided over
by a smiling woman who resembled my childhood pastor, with fiery curls.
She thanked me for being there and placed her hands on my shoulders. It
was nice to be comforted, held calmly. I closed my eyes. We stayed that
way for several minutes. There was warmth and tingling, her hands
vibrating against my shirt. I rose and returned to the first pew.

The message service was the portion of each Sunday's meeting in which
mediums from the camp addressed members of the audience with insights
and communiqués from loved ones who had died. Also a practicum for
mediums-in-training, this was a chance for them to refine their skills,
delivering messages to as many people as possible, then receiving
written feedback.

A silvery-blonde woman in a floral dress stood from the front row. She
identified herself as a student. ``May I come to you?'' she asked me.

``There's a few folks that are stepping into your vibration. That was
hard for you, watching all of them. It's important, for you and for
them, that you feel them.''

I felt a shock go through my body. I was suddenly aware of being
surrounded, at all times, by the people I'd lost. I felt both of my
grandmothers there. My grandfather. My friend Carolyn. My friend Brook
and my uncles Brian, Dennis and Mike. My friend Dylan from high school.
Daniel.

``You're not alone here,'' the student continued, as I welled with
emotion. ``My message from all of these people is simply: Continue to
reach out.''

Image

\textbf{VII. DO YOU SEE WHAT I SEE?}

An hour later, I was still in the fellowship hall, but now sitting
across the table from another fellow student, named Dennis. We held
hands. Our eyes were closed. The Rev. Joy Sagar was leading the
clairvoyance class. He'd instructed us to raise our vibrations, and
envision the homes of each other's deceased loved ones. Dennis and I had
not specified to each other who these loved ones would be. I hadn't told
him that I'd decided to picture the childhood home of his dead brother,
whom I'd intuited correctly, moments before, had died in a farming
accident.

We opened our eyes and wrote down in our notebooks what we'd seen. Two
stories, I wrote. Woven rug in the entryway. ** Glass panes in the door.
Wood floors. Dining room to the left of the entrance. White candlesticks
on the table*.*

I read this list aloud to Dennis. \emph{That was the house he grew up
in, he said, the farm}.

I couldn't decide how much of what I'd envisioned of his childhood home
was coming from my presuppositions about him, the few details I'd
already gathered, or my own psychic abilities. Regardless, there was
something beautiful in this activity. Together, Dennis and I had cleared
a space in which it was safe to speak of our dead loved ones, and to
share the intimate details of our lives, though we were virtual
strangers.

The reverend directed us to repeat a version of the activity, this time
looking into each other's eyes. Dennis told me that he'd intuited that
my grandmother was a neat housekeeper. She was a great cook and loved
gardening. She always had fresh produce. She enjoyed sewing. She lived
on some nice acreage. All of this was true except the sewing, as far as
I knew, though she did crochet --- I asked Dennis if he might have meant
to say that she liked crochet. His tableau was a set of clichés about
grandmothers, but I was coming to understand that it gave me a set of
points on which to palpate my grief with his assistance. There was no
harm in us feeling our grief together, and sharing each other's burden.
``When it makes sense, accept and embrace that,'' said the reverend,
circulating the room.

He explained that, over the years, he'd had to learn to distinguish, in
these visions, between realistic and symbolic imagery. They appear
side-by-side, he said --- but for instance, he often pictures horses,
and is aware that horses hold great metaphorical significance for him in
his personal mythology. He must not take them too literally when they
arise in a reading. He knows that when he sees a horse, he has to
translate it, as though interpreting a dream.

Image

\textbf{IX. HAUNTED}

I stood on the porch of the Cassadaga Bookstore with some stragglers
from the class. The sun was still high, sucking the sweat from our
faces, and Nick Christensen was telling us all about guns. He was a
firearms instructor, and an 11-year Army veteran, and had been raised
with guns since he was \emph{this tall}. He wore a yellow tee with a
screen-printed sunset on it, and the words TRANSCEND BOUNDARIES, a
Hawaiian shirt, and an N.R.A. baseball cap embroidered with a bald eagle
waving an American flag. ``The more education you have, OK, then the
more stupid people you run into --- and it's not their fault --- ''

``Well, the thing is, they just, they're not educated,'' said his wife,
Pat.

``They have every chance that everybody else does, OK, and --- ''

``Who does? Who is this?'' I asked.

``People who oppose guns, or people who oppose --- they'll have an
opinion about something, but they don't have the facts about it. I'm a
history fanatic, OK? People who come to our house say it looks like a
museum. There's Native American everywhere.''

I asked --- because it seemed like I was expected to --- whether someone
in their family was Native American.

``I'm part Native,'' said Nick, ``and so is she.''

I asked what tribes they were affiliated with. Pat explained that she
didn't have a family member who was federally recognized, so therefore
she isn't considered Native by the U.S. government, but her ``history is
from the Northeast, with the Iroquois people.'' This dovetailed with
what she'd said during the group introduction about growing up with the
Spiritualists in Lily Dale, which is in an Iroquois Nation.

It occurred to me that the Spiritualist tradition was in a way haunted
by the imperialist history of white Americans destroying and
appropriating Native cultures. I remembered thinking this earlier, too,
on the photography tour, walking to Spirit Pond. Dawn Medley's
introduction to the body of water had opened with a general promise of
finding enhanced magic there. Then she proceeded to locate the origin of
that magic in the early touch of Indigenous people, who by the simple
fact of their existence exuded it. The structure of her story followed
an almost allegorical arc: The hero's awakening to the sacredness of a
thing, and the shame he feels afterward for the desecration civilization
has already wrought upon it.

I asked Nick what tribe he was affiliated with.

``Drunk Norwegians,'' he said.

``Lakota and Apache,'' he said more seriously. ``And I actually lived on
a couple different reservations as a kid. And the first thing you learn
being white-bred is that you don't want to play cowboys and Indians.''

```Cuz you'd end up being the cowboy tied up,'' said Pat.

Image

\textbf{X. START BELIEVING THE STORIES}

The Rev. Lori Carter was the camp's volunteer P.R. director and had been
my invisible guide over these last three days in Cassadaga. She lived
steps away from the bookstore, on the first floor of Harmony Hall. I had
scheduled a reading with her, and was planning to ask her to bless a
deck of tarot cards I had bought recently. I asked her if she had any
experience with them, and she told me that she kept a deck in her car
for personal use, but that the certified psychics of Cassadaga were
trained not to need tools.

It was raining again but still sunny as she sat across from me, and drew
the sheer curtains over the window behind her, shielding our activity
from the street. She was already operating at a high vibration when I
came in.

``I remember you asking me the other day if it's imagination or it's
real, and of course I'm going to say: it's real,'' she began.

She encouraged me to develop my intuitive abilities with a daily
meditation practice. She told me that I'm on a journey of healing and
invited me to ask questions. I asked if there was a man with us about my
age, who had passed on. She asked if he had brown hair and I said yes.
``And did he have bright eyes?'' she said. I confirmed.

She described a sensation of gulping for air. The intensity of her
experience was apparent

``I felt that gulping for air, and then I just felt very serene,'' she
said. ``Did he drown?''

``In a way, yes,'' I said. ``He drank himself to death.''

She described what she saw happening to Daniel in the hotel room where
he was found. She told me that he is part of my soul group, and that we
were here on this earth-plane together to teach each other. ``Sometimes
the lessons are painful. Just know that he's OK now, and he's working
through things. It's interesting, he shows me a heart with an arrow
through it.''

This is a joke he would make. Sweet on the surface, yet dark and
disturbing underneath. In the myth of Cupid, he seduces and kidnaps
Psyche and, night after night, rapes her after she falls asleep. She is
never permitted to see who he is, so one night, she hides a lamp in her
room and casts the light on his face. She's surprised to find that he's
beautiful, and pierces herself on one of his poisoned arrows, causing
her to fall in love --- though I was never in love with Daniel, in my
grief I felt acutely the love I held for my friend, despite what he'd
done.

Then Cupid flees Psyche, and though she tries to pursue him, he outruns
her. She wanders the earth searching for him. As I was now searching for
Daniel.

Carter asked if Daniel had ever had a broken wrist. A few months before
he attacked me in my sleep, he had messaged me on Facebook. He told me
he was in the hospital after trying to kill himself by taking a drill
bit to his wrist. I called him immediately but he didn't answer, and
since he'd logged off Facebook, I called every hospital in his county
looking for him. Finding him in none, I called the local police. The
officer who answered laughed when he heard my story. They were used to
getting calls from people about Daniel, he said. Sure, they would do a
wellness check. ``I feel like he is a little remorseful,'' said Carter.
``He wishes he could have had his act together more in the end.''

She tells me not to give up on my gifts, to start believing the stories
as they come to me --- more will come as long as I believe they will.
``When you open up about it, other people feel free to talk about it,
too. Some people are fearful, and who knows why? Their belief system,
their upbringing, the society they lived in growing up. It could be many
things.''

I asked if she meant that it's our altruistic responsibility to educate
one another. She said yes. I thought back to my conversation with Nick
the day before.

``How do you educate someone who pisses you off?'' I said.

``God bless it,'' she said. ``You have to be centered in your own
mind.''

Sarah Gerard is the author the novels
``\href{https://www.twodollarradio.com/products/binary-star}{Binary
Star}'' and
``\href{https://www.harpercollins.com/products/true-love-sarah-gerard}{True
Love}'' and the essay collection
``\href{https://www.harpercollins.com/products/sunshine-state-sarah-gerard}{Sunshine
State}.''

\emph{The Times is committed to publishing}
\href{https://www.nytimes3xbfgragh.onion/2019/01/31/opinion/letters/letters-to-editor-new-york-times-women.html}{\emph{a
diversity of letters}} \emph{to the editor. We'd like to hear what you
think about this or any of our articles. Here are some}
\href{https://help.nytimes3xbfgragh.onion/hc/en-us/articles/115014925288-How-to-submit-a-letter-to-the-editor}{\emph{tips}}\emph{.
And here's our email:}
\href{mailto:letters@NYTimes.com}{\emph{letters@NYTimes.com}}\emph{.}

\emph{Follow The New York Times Opinion section on}
\href{https://www.facebookcorewwwi.onion/nytopinion}{\emph{Facebook}}\emph{,}
\href{http://twitter.com/NYTOpinion}{\emph{Twitter (@NYTopinion)}}
\emph{and}
\href{https://www.instagram.com/nytopinion/}{\emph{Instagram}}\emph{.}

Advertisement

\protect\hyperlink{after-bottom}{Continue reading the main story}

\hypertarget{site-index}{%
\subsection{Site Index}\label{site-index}}

\hypertarget{site-information-navigation}{%
\subsection{Site Information
Navigation}\label{site-information-navigation}}

\begin{itemize}
\tightlist
\item
  \href{https://help.nytimes3xbfgragh.onion/hc/en-us/articles/115014792127-Copyright-notice}{©~2020~The
  New York Times Company}
\end{itemize}

\begin{itemize}
\tightlist
\item
  \href{https://www.nytco.com/}{NYTCo}
\item
  \href{https://help.nytimes3xbfgragh.onion/hc/en-us/articles/115015385887-Contact-Us}{Contact
  Us}
\item
  \href{https://www.nytco.com/careers/}{Work with us}
\item
  \href{https://nytmediakit.com/}{Advertise}
\item
  \href{http://www.tbrandstudio.com/}{T Brand Studio}
\item
  \href{https://www.nytimes3xbfgragh.onion/privacy/cookie-policy\#how-do-i-manage-trackers}{Your
  Ad Choices}
\item
  \href{https://www.nytimes3xbfgragh.onion/privacy}{Privacy}
\item
  \href{https://help.nytimes3xbfgragh.onion/hc/en-us/articles/115014893428-Terms-of-service}{Terms
  of Service}
\item
  \href{https://help.nytimes3xbfgragh.onion/hc/en-us/articles/115014893968-Terms-of-sale}{Terms
  of Sale}
\item
  \href{https://spiderbites.nytimes3xbfgragh.onion}{Site Map}
\item
  \href{https://help.nytimes3xbfgragh.onion/hc/en-us}{Help}
\item
  \href{https://www.nytimes3xbfgragh.onion/subscription?campaignId=37WXW}{Subscriptions}
\end{itemize}
