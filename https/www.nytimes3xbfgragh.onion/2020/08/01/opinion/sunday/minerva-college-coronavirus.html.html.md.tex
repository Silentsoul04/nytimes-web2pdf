Sections

SEARCH

\protect\hyperlink{site-content}{Skip to
content}\protect\hyperlink{site-index}{Skip to site index}

\href{https://www.nytimes3xbfgragh.onion/section/opinion/sunday}{Sunday
Review}

\href{https://myaccount.nytimes3xbfgragh.onion/auth/login?response_type=cookie\&client_id=vi}{}

\href{https://www.nytimes3xbfgragh.onion/section/todayspaper}{Today's
Paper}

\href{/section/opinion/sunday}{Sunday Review}\textbar{}How to Go to
College During a Pandemic

\url{https://nyti.ms/2PiJZFz}

\begin{itemize}
\item
\item
\item
\item
\item
\item
\end{itemize}

\href{https://www.nytimes3xbfgragh.onion/spotlight/at-home?action=click\&pgtype=Article\&state=default\&region=TOP_BANNER\&context=at_home_menu}{At
Home}

\begin{itemize}
\tightlist
\item
  \href{https://www.nytimes3xbfgragh.onion/2020/07/28/books/time-for-a-literary-road-trip.html?action=click\&pgtype=Article\&state=default\&region=TOP_BANNER\&context=at_home_menu}{Take:
  A Literary Road Trip}
\item
  \href{https://www.nytimes3xbfgragh.onion/2020/07/29/magazine/bored-with-your-home-cooking-some-smoky-eggplant-will-fix-that.html?action=click\&pgtype=Article\&state=default\&region=TOP_BANNER\&context=at_home_menu}{Cook:
  Smoky Eggplant}
\item
  \href{https://www.nytimes3xbfgragh.onion/2020/07/27/travel/moose-michigan-isle-royale.html?action=click\&pgtype=Article\&state=default\&region=TOP_BANNER\&context=at_home_menu}{Look
  Out: For Moose}
\item
  \href{https://www.nytimes3xbfgragh.onion/interactive/2020/at-home/even-more-reporters-editors-diaries-lists-recommendations.html?action=click\&pgtype=Article\&state=default\&region=TOP_BANNER\&context=at_home_menu}{Explore:
  Reporters' Obsessions}
\end{itemize}

Advertisement

\protect\hyperlink{after-top}{Continue reading the main story}

\href{/section/opinion}{Opinion}

Supported by

\protect\hyperlink{after-sponsor}{Continue reading the main story}

\hypertarget{how-to-go-to-college-during-a-pandemic}{%
\section{How to Go to College During a
Pandemic}\label{how-to-go-to-college-during-a-pandemic}}

Students at Minerva spend four years entirely online. Is this the future
of higher education?

\href{https://www.nytimes3xbfgragh.onion/by/frank-bruni}{\includegraphics{https://static01.graylady3jvrrxbe.onion/images/2018/04/03/opinion/frank-bruni/frank-bruni-thumbLarge.png}}

By \href{https://www.nytimes3xbfgragh.onion/by/frank-bruni}{Frank Bruni}

Opinion Columnist

\begin{itemize}
\item
  Aug. 1, 2020
\item
  \begin{itemize}
  \item
  \item
  \item
  \item
  \item
  \item
  \end{itemize}
\end{itemize}

\includegraphics{https://static01.graylady3jvrrxbe.onion/images/2020/08/02/opinion/sunday/02bruni/02bruni-articleLarge.jpg?quality=75\&auto=webp\&disable=upscale}

Hundreds of thousands of undergraduates in America won't be allowed on
their campuses this fall, or the campuses welcoming them will be
hollowed-out, locked-down, revelry-leeched shadows of their former
selves. What kind of college experience is that?

The kind that Natalie Kanter had by design. She did college without the
campus --- four demanding and exhilarating years of it. And I don't mean
that she lived \emph{off} campus, commuting in as needed. There was no
campus to commute to. No lecture halls. No rec center. No football
stadium.

For her and her schoolmates, remote learning wasn't a crisis-prompted
compromise. It was the whole point.

Kanter, 23, belonged to the first graduating class of a sort of start-up
college,
\href{https://www.minervaproject.com/solutions/educational-solutions/}{Minerva},
which opened about five years ago. All of its instruction is online,
from professors scattered far and wide.

And while students in a given grade live together in a residential
building, so that they have peers at hand and a center of gravity, they
do so all around the globe, moving periodically to a new city that
becomes their new campus, but only temporarily.

Kanter and her roughly 105 classmates spent their first two semesters in
San Francisco, where Minerva's bare-bones administration is, before
migrating for one semester each to Berlin, Buenos Aires, Seoul,
Hyderabad (India), London and then San Francisco again. Minerva has a
footprint --- well, more a toe dimple --- in each of those places plus
Taipei, Taiwan.

It's defined not by physical structures but by a proprietary, highly
interactive digital platform that professors use for their seminars. The
seminars are capped at 20 students (but are usually smaller) and
emphasize participation to a point where the platform --- a far cry from
Zoom --- shows a professor how long he or she has been droning on.

``Having a campus is one of the least important parts of the university
experience,'' Kanter, who graduated in May 2019 and now works for the
social advocacy organization
\href{https://www.dosomething.org/us}{DoSomething.org}, told me.

Yes, she said, the ``additional pizazz'' of grand buildings, weathered
statues and ``rubbing the left foot of this or jumping into that
fountain when you graduate'' might have been nice. But necessary? Not
for learning. Not for extracurricular enrichment, to which a campus can
sometimes be a cloistering, coddling barrier.

A campus also inflates the cost of college. Tuition, fees, room and
board at Minerva are about \$32,000 a year --- easily half the sticker
price of many prestigious private colleges --- for students paying full
freight, which is only about 20 percent of them. That's made possible by
the absence of gleaming campus structures.

Bob Kerrey, the former senator and a Minerva board member, told me that
the importance of a snazzy or traditional setting is a falsehood ``set
up by people who really don't want to lose the investment in
classrooms.''

``I'm part of it,'' he added, referring to the 16-story, roughly \$350
million University Center that he helped to create when he was president
of the New School in Manhattan from 2001 to 2010. ``It's lovely. The
students love it. But it's built for the old way of doing things. Once
you've made the investment, it's not easy to let it go.''

Minerva
\href{https://www.theatlantic.com/magazine/archive/2014/09/the-future-of-college/375071/}{drew
attention} at the outset, because it was one answer to intensifying
questions about the affordability of college, the country-club excesses
of some campuses and the erratic devotion to actual instruction.

But it's an object of even greater fascination now, because it willingly
made changes that other schools are suddenly forced to consider, thanks
to the coronavirus.

Mitchell Stevens, an associate professor of education at Stanford, told
me that even before the pandemic higher education ``was in many ways
being held together by prayers, Band-Aids, international students and a
lot of debt.''

``What the pandemic creates,'' he said, ``is a kind of existential
challenge to so many colleges and universities and business-model
presumptions. That's an opportunity for fairly radical rethinking.''

Minerva's rethinking isn't \emph{the} solution. For many students,
Minerva would be a disastrous psychological or practical fit. Others
have obligations or limitations that forbid globe-trotting.

Its course catalog is a haiku next to a conventional university's epic
poem. It doesn't own and give students instant access to cutting-edge
scientific equipment or stacks of old books.

But it's a creative mix of disruptions and rebellions that could, in
some form, have application elsewhere.

For example, it completely bucks the trend at many schools toward a
dizzying array of clubs, activities and amenities. Minerva's founder,
Ben Nelson, argues that those are often a distraction from academic
pursuits and that students who want to perform music or play soccer with
one another can arrange that for themselves, foraging for resources in
the diverse, teeming city around them.

``The richness that students can get by being independent, by figuring
out how to navigate a place, is infinitely better for their personal
development than the glee club having a room in the student center,'' he
said. ``They're entrepreneurial. They can leverage what happens in the
real world.''

That they use kitchens in their residential buildings to cook for
themselves rather than plopping down in a student cafeteria isn't a
hardship, he added. It's, well, an education.

Nelson, of course, has an investment in believing that. But he's hardly
the first person to suggest that the country's richer schools are more
concerned with pleasing students than with challenging them and are
better at delaying adulthood than at ushering their charges into it.

And Minerva students and graduates I spoke to said that the school
indeed gave them something --- maturity, initiative, ingenuity and,
ultimately, confidence --- in return for what they sacrificed. Alexander
Sanchez, who left Washington University in St. Louis after one year to
do Minerva from start to finish, told me that he ``lost access to a
dance studio and to the critical mass of students I would need to form a
dance club.''

But Sanchez, 23, who graduated this year and works in the psychology
department at Harvard, told me that he learned the tango in Buenos
Aires, studied hip-hop in Berlin and developed ``friendships that
challenged my conceptions of what `normal' was every single day.''

``We were exposed to such a diversity of opinions and cultural
upbringings,'' he said, referring to himself and his Minerva classmates,
most of whom aren't from America. He's Colombian-American and went to
high school in Puerto Rico.

Minerva is an antidote to provincialism. Megan Perpich, 21, who will
head to Berlin this fall to begin her junior year, told me that the
school made her feel ``small.'' That's a compliment. She feels small,
she explained, because the world feels so very inexhaustibly big.

And Minerva has carefully considered and tried to address the pitfalls
of online education. One reason that someone taking a large lecture
course alone on a computer at home may not finish or glean much from it
is the absence of a nearby support network of peers. But Minerva
students have one another for counsel, company and community, and
several told me that a classmate taking the same seminar that you are
might be doing so on a laptop set up next to yours. There might be three
or four of you huddled together, although not this coming semester, when
Minerva will exhort students to observe social distancing in their
residences.

Those seminars all happen live, and Minerva's digital platform has
interactive facets that prevent students from dozing or drifting away.
Their distant professors hold office hours --- virtually. And there are
a few adults on hand in each Minerva city to assist and check in on the
students.

``But what,'' I asked Kanter, ``about school spirit?'' Does it survive a
reliance on wireless and airports?

``It's reimagined,'' she said. ``It's not sitting in bleachers and
chanting.'' It's about being in an unconventional group of undaunted
adventurers who are having an unfamiliar college experience, in part
because they're fashioning it themselves. ``That definitely gives you an
adrenaline rush,'' she said --- a rush that may even be immune to a
pandemic.

\emph{I invite you to sign up for my free}
\href{https://www.nytimes3xbfgragh.onion/newsletters/frank-bruni}{\emph{weekly
email newsletter}}\emph{. You can follow me on Twitter
(}\href{https://twitter.com/FrankBruni}{\emph{@FrankBruni}}\emph{).}

\emph{Listen to}
\href{https://www.nytimes3xbfgragh.onion/column/the-argument}{\emph{``The
Argument'' podcast}} \emph{every Thursday morning, with Ross Douthat,
Michelle Goldberg and me.}

Advertisement

\protect\hyperlink{after-bottom}{Continue reading the main story}

\hypertarget{site-index}{%
\subsection{Site Index}\label{site-index}}

\hypertarget{site-information-navigation}{%
\subsection{Site Information
Navigation}\label{site-information-navigation}}

\begin{itemize}
\tightlist
\item
  \href{https://help.nytimes3xbfgragh.onion/hc/en-us/articles/115014792127-Copyright-notice}{©~2020~The
  New York Times Company}
\end{itemize}

\begin{itemize}
\tightlist
\item
  \href{https://www.nytco.com/}{NYTCo}
\item
  \href{https://help.nytimes3xbfgragh.onion/hc/en-us/articles/115015385887-Contact-Us}{Contact
  Us}
\item
  \href{https://www.nytco.com/careers/}{Work with us}
\item
  \href{https://nytmediakit.com/}{Advertise}
\item
  \href{http://www.tbrandstudio.com/}{T Brand Studio}
\item
  \href{https://www.nytimes3xbfgragh.onion/privacy/cookie-policy\#how-do-i-manage-trackers}{Your
  Ad Choices}
\item
  \href{https://www.nytimes3xbfgragh.onion/privacy}{Privacy}
\item
  \href{https://help.nytimes3xbfgragh.onion/hc/en-us/articles/115014893428-Terms-of-service}{Terms
  of Service}
\item
  \href{https://help.nytimes3xbfgragh.onion/hc/en-us/articles/115014893968-Terms-of-sale}{Terms
  of Sale}
\item
  \href{https://spiderbites.nytimes3xbfgragh.onion}{Site Map}
\item
  \href{https://help.nytimes3xbfgragh.onion/hc/en-us}{Help}
\item
  \href{https://www.nytimes3xbfgragh.onion/subscription?campaignId=37WXW}{Subscriptions}
\end{itemize}
