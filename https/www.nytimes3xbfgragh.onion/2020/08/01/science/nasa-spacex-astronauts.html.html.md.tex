Sections

SEARCH

\protect\hyperlink{site-content}{Skip to
content}\protect\hyperlink{site-index}{Skip to site index}

\href{https://www.nytimes3xbfgragh.onion/section/science}{Science}

\href{https://myaccount.nytimes3xbfgragh.onion/auth/login?response_type=cookie\&client_id=vi}{}

\href{https://www.nytimes3xbfgragh.onion/section/todayspaper}{Today's
Paper}

\href{/section/science}{Science}\textbar{}SpaceX Crew Dragon to Bring 2
NASA Astronauts Home

\url{https://nyti.ms/2D7F5Zm}

\begin{itemize}
\item
\item
\item
\item
\item
\end{itemize}

\href{https://www.nytimes3xbfgragh.onion/2020/08/01/science/nasa-spacex-astronauts.html?action=click\&pgtype=Article\&state=default\&region=TOP_BANNER\&context=storylines_menu}{SpaceX's
Astronaut Trip}

\begin{itemize}
\tightlist
\item
  \href{https://www.nytimes3xbfgragh.onion/2020/08/01/science/nasa-spacex-astronauts.html?action=click\&pgtype=Article\&state=default\&region=TOP_BANNER\&context=storylines_menu}{The
  Voyage Home}
\item
  \href{https://www.nytimes3xbfgragh.onion/2020/05/26/science/spacex-launch-nasa.html?action=click\&pgtype=Article\&state=default\&region=TOP_BANNER\&context=storylines_menu}{Why
  NASA Picked SpaceX}
\item
  \href{https://www.nytimes3xbfgragh.onion/interactive/2020/05/26/science/spacex-nasa.html?action=click\&pgtype=Article\&state=default\&region=TOP_BANNER\&context=storylines_menu}{Inside
  SpaceX's Capsule}
\item
  \href{https://www.nytimes3xbfgragh.onion/2020/05/27/science/bob-behnken-doug-hurley.html?action=click\&pgtype=Article\&state=default\&region=TOP_BANNER\&context=storylines_menu}{Meet
  the Astronauts}
\end{itemize}

Advertisement

\protect\hyperlink{after-top}{Continue reading the main story}

Supported by

\protect\hyperlink{after-sponsor}{Continue reading the main story}

\hypertarget{spacex-crew-dragon-to-bring-2-nasa-astronauts-home}{%
\section{SpaceX Crew Dragon to Bring 2 NASA Astronauts
Home}\label{spacex-crew-dragon-to-bring-2-nasa-astronauts-home}}

Bob Behnken and Doug Hurley are getting ready to splash down after two
months in orbit.

\includegraphics{https://static01.graylady3jvrrxbe.onion/images/2020/08/01/science/01sci-astronauts01/merlin_174995049_2c12a19d-5a92-4615-b582-a236222d7def-articleLarge.jpg?quality=75\&auto=webp\&disable=upscale}

\href{https://www.nytimes3xbfgragh.onion/by/kenneth-chang}{\includegraphics{https://static01.graylady3jvrrxbe.onion/images/2018/02/16/multimedia/author-kenneth-chang/author-kenneth-chang-thumbLarge.jpg}}

By \href{https://www.nytimes3xbfgragh.onion/by/kenneth-chang}{Kenneth
Chang}

\begin{itemize}
\item
  Aug. 1, 2020Updated 3:30 p.m. ET
\item
  \begin{itemize}
  \item
  \item
  \item
  \item
  \item
  \end{itemize}
\end{itemize}

If a hurricane does not get in the way, two astronauts who took the
first commercial trip to orbit will return home on Sunday.

Two astronauts, Robert L. Behnken and Douglas G. Hurley, traveled to the
International Space Station in May aboard a Crew Dragon capsule built
and run by SpaceX, the private rocket company started by Elon Musk.

The Crew Dragon is scheduled to undock from the space station at about
7:34 p.m. Eastern time on Saturday and splash down in the Gulf of Mexico
off Pensacola, Fla., at 2:41 p.m. on Sunday, Jim Bridenstine, the NASA
administrator, announced on Twitter.

\begin{quote}
BREAKING: \href{https://twitter.com/NASA?ref_src=twsrc\%5Etfw}{@NASA}
and \href{https://twitter.com/SpaceX?ref_src=twsrc\%5Etfw}{@SpaceX} are
targeting Pensacola as the primary return location for Crew Dragon w/
\href{https://twitter.com/Astro_Doug?ref_src=twsrc\%5Etfw}{@Astro\_Doug}
and
\href{https://twitter.com/AstroBehnken?ref_src=twsrc\%5Etfw}{@AstroBehnken}
from the
\href{https://twitter.com/Space_Station?ref_src=twsrc\%5Etfw}{@Space\_Station}.
We are targeting undocking at 7:34 p.m. EDT today. Read more:
\url{https://t.co/Xdh8iIpLxz}
\href{https://t.co/9p5TRMevAH}{pic.twitter.com/9p5TRMevAH}

--- Jim Bridenstine (@JimBridenstine)
\href{https://twitter.com/JimBridenstine/status/1289617675572969472?ref_src=twsrc\%5Etfw}{August
1, 2020}
\end{quote}

A safe return would open up more trips to and from orbit for future
astronaut crews, and possibly space tourists, aboard the spacecraft.

Hurricane Isaias is forecast to sweep up along the Atlantic coast of
Florida over the weekend. NASA and SpaceX have seven splashdown sites in
the Gulf of Mexico and the Atlantic, but the track of Isaias ruled out
the three in the Atlantic.

The departure could still be pushed back to Sunday night with a
splashdown on Monday afternoon. NASA and SpaceX will make a final
decision around 5 p.m. whether to proceed with the undocking.

``We have confidence that the teams on the ground are, of course,
watching that much more closely than we are,'' Mr. Behnken said during a
news conference on Friday, ``and we won't leave the space station
without some good landing opportunities in front of us, good splashdown
weather in front of us.''

\hypertarget{how-can-i-watch-the-return-of-the-astronauts}{%
\subsection{How can I watch the return of the
astronauts?}\label{how-can-i-watch-the-return-of-the-astronauts}}

\href{https://www.nasa.gov/multimedia/nasatv/\#public}{NASA Television}
broadcast \href{https://twitter.com/NASA/status/1289548827012333570}{a
farewell ceremony on Saturday morning}. Coverage of the undocking will
begin at 5:15 p.m. and is to continue through splashdown. The coverage
would be pushed back if weather delays the departure.

\hypertarget{returning-to-earth}{%
\subsection{Returning to Earth}\label{returning-to-earth}}

The SpaceX Crew Dragon is scheduled to splash down near Florida on
Sunday, though
\href{https://www.nytimes3xbfgragh.onion/interactive/2020/07/31/us/hurricane-isaias-tracker-map.html}{Hurricane
Isaias} could change those plans.

Category

2

1

Tropical storm

Forecasted path of Hurricane Isaias

Atlantic Ocean

Mon. 2 a.m.

Seven possible

splashdown sites

(approximate)

Gulf of Mexico

Sun. 2 a.m.

Sat. 2 a.m.

Last updated Sat. 9 a.m.

Category

2

1

Tropical storm

Forecasted path of Hurricane Isaias

Mon. 2 a.m.

Seven possible

splashdown sites

(approximate)

Gulf of Mexico

Sun. 2 a.m.

Sat. 2 a.m.

Last updated Sat. 9 a.m.

Category

2

1

Tropical storm

Forecasted path of Hurricane Isaias

Mon.

2 a.m.

Seven possible

splashdown sites

(approximate)

Gulf of Mexico

Sun.

2 a.m.

Last updated Sat. 9 a.m.

By The New York Times \textbar{} Sources: NASA, National Hurricane
Center, Mapbox, OpenStreetMap

\hypertarget{why-does-hurricane-isaias-affect-the-departure}{%
\subsection{Why does Hurricane Isaias affect the
departure?}\label{why-does-hurricane-isaias-affect-the-departure}}

The storm complicates when and where splashdown can take place. At the
splashdown site, winds must be less than 10 miles per hour for the
capsule to land safely. There are additional constraints on waves, rain
and lightning. In addition, helicopters that take part in the recovery
of the capsule must be able to fly and land safely.

The first landing opportunity will aim for only the primary site,
Pensacola. If weather there is inconsistent with the rules, the capsule
and the astronauts will remain in orbit for another day or two, and
managers will consider the backup site, which is Panama City.

\hypertarget{what-will-happen-after-they-leave-the-station}{%
\subsection{What will happen after they leave the
station?}\label{what-will-happen-after-they-leave-the-station}}

After undocking on Saturday night, the spacecraft will perform a series
of maneuvers, first firing the engines several times to move away from
the space station and, a few hour hours later, to line up the spacecraft
with the splashdown zone.

For most of the trip, Mr. Behnken and Mr. Hurley will be sleeping. Their
\href{https://twitter.com/NASA/status/1289245570565992449}{schedule sets
aside 10 hours of shut eye}.

Any return journey that exceeds six hours has to be long enough for the
crew to get some sleep between undocking and splashdown, Daniel Huot, a
NASA spokesman, said in an email.

Otherwise, because of the extended process that leads up to undocking,
the crew would end up working more than 20 hours straight, ``which is
not safe for dynamic operations like water splashdown and recovery,''
Mr. Huot said.

Just before a final burn that will drop the Crew Dragon out of orbit on
Sunday afternoon, it will jettison the bottom part of the spacecraft,
known as the trunk, which will then burn up in the atmosphere.

At re-entry, the Crew Dragon will be traveling at about 17,500 miles per
hour. Two small parachutes will deploy at an altitude of 18,000 feet
when the spacecraft has already been slowed by Earth's atmosphere to
about 350 miles per hour. The four main parachutes deploy at an altitude
of about 6,000 feet.

Once the capsule splashes in the water, it is expected to take 45 to 60
minutes to pluck them out.

\hypertarget{why-is-the-return-trip-an-important-part-of-the-crew-dragons-first-flight}{%
\subsection{Why is the return trip an important part of the Crew
Dragon's first
flight?}\label{why-is-the-return-trip-an-important-part-of-the-crew-dragons-first-flight}}

After launch, re-entry through Earth's atmosphere is the second most
dangerous phase of spaceflight. Friction of air rushing past will heat
the bottom of the capsule to about 3,500 degrees Fahrenheit. A test
flight of the Crew Dragon last year successfully splashed down, so
engineers know the system works.

A successful conclusion to the trip opens the door to more people flying
to space. Some companies have already announced plans to use Crew
Dragons to lift wealthy tourists to orbit.

In the past, NASA astronauts launched on spacecraft like the Saturn 5
moon rocket and the space shuttles that NASA itself operated. After the
retirement of the space shuttles in 2011, NASA had to rely on Russia,
buying seats on the Soyuz capsules for trips to and from orbit.

Under the Obama administration, NASA hired two companies, SpaceX and
Boeing, to build spacecraft to take astronauts to the space station.
NASA financed much of the work to develop the spacecraft but will now
buy rides at fixed prices. For SpaceX, the trip by Mr. Behnken and Mr.
Hurley --- the first launch of astronauts from American soil since the
last space shuttle flight --- was the last major demonstration needed
before NASA officially certifies that the Crew Dragon is ready to begin
regular flights.

\hypertarget{who-are-the-astronauts}{%
\subsection{Who are the astronauts?}\label{who-are-the-astronauts}}

The astronauts are
\href{https://www.nytimes3xbfgragh.onion/2020/05/27/science/bob-behnken-doug-hurley.html}{Robert
L. Behnken and Douglas G. Hurley}, who have been friends and colleagues
since both were selected by NASA to be astronauts in 2000.

Both men have backgrounds as military test pilots and each has flown
twice before on space shuttle missions, although this is the first time
they have worked together on a mission. Mr. Hurley flew on the space
shuttle's final mission in 2011.

In 2015, they were among the astronauts chosen to work with Boeing and
SpaceX on the commercial space vehicles that the companies were
developing. In 2018, they were assigned to the first SpaceX flight.

Advertisement

\protect\hyperlink{after-bottom}{Continue reading the main story}

\hypertarget{site-index}{%
\subsection{Site Index}\label{site-index}}

\hypertarget{site-information-navigation}{%
\subsection{Site Information
Navigation}\label{site-information-navigation}}

\begin{itemize}
\tightlist
\item
  \href{https://help.nytimes3xbfgragh.onion/hc/en-us/articles/115014792127-Copyright-notice}{©~2020~The
  New York Times Company}
\end{itemize}

\begin{itemize}
\tightlist
\item
  \href{https://www.nytco.com/}{NYTCo}
\item
  \href{https://help.nytimes3xbfgragh.onion/hc/en-us/articles/115015385887-Contact-Us}{Contact
  Us}
\item
  \href{https://www.nytco.com/careers/}{Work with us}
\item
  \href{https://nytmediakit.com/}{Advertise}
\item
  \href{http://www.tbrandstudio.com/}{T Brand Studio}
\item
  \href{https://www.nytimes3xbfgragh.onion/privacy/cookie-policy\#how-do-i-manage-trackers}{Your
  Ad Choices}
\item
  \href{https://www.nytimes3xbfgragh.onion/privacy}{Privacy}
\item
  \href{https://help.nytimes3xbfgragh.onion/hc/en-us/articles/115014893428-Terms-of-service}{Terms
  of Service}
\item
  \href{https://help.nytimes3xbfgragh.onion/hc/en-us/articles/115014893968-Terms-of-sale}{Terms
  of Sale}
\item
  \href{https://spiderbites.nytimes3xbfgragh.onion}{Site Map}
\item
  \href{https://help.nytimes3xbfgragh.onion/hc/en-us}{Help}
\item
  \href{https://www.nytimes3xbfgragh.onion/subscription?campaignId=37WXW}{Subscriptions}
\end{itemize}
