Sections

SEARCH

\protect\hyperlink{site-content}{Skip to
content}\protect\hyperlink{site-index}{Skip to site index}

\href{https://www.nytimes3xbfgragh.onion/section/sports/hockey}{Hockey}

\href{https://myaccount.nytimes3xbfgragh.onion/auth/login?response_type=cookie\&client_id=vi}{}

\href{https://www.nytimes3xbfgragh.onion/section/todayspaper}{Today's
Paper}

\href{/section/sports/hockey}{Hockey}\textbar{}Edmonton, `City of
Champions,' Readies for a Hockey Marathon

\url{https://nyti.ms/2DoP10q}

\begin{itemize}
\item
\item
\item
\item
\item
\end{itemize}

Advertisement

\protect\hyperlink{after-top}{Continue reading the main story}

Supported by

\protect\hyperlink{after-sponsor}{Continue reading the main story}

\hypertarget{edmonton-city-of-champions-readies-for-a-hockey-marathon}{%
\section{Edmonton, `City of Champions,' Readies for a Hockey
Marathon}\label{edmonton-city-of-champions-readies-for-a-hockey-marathon}}

The N.H.L. chose the small-market, hockey-mad city as one of two ``hub
cities'' --- along with Toronto --- to host its playoffs.

\includegraphics{https://static01.graylady3jvrrxbe.onion/images/2020/08/01/sports/01nhl-edmonton1-print/merlin_175138452_9486317d-6b51-411a-a58b-933de68b92ca-articleLarge.jpg?quality=75\&auto=webp\&disable=upscale}

By Carol Schram

\begin{itemize}
\item
  Aug. 1, 2020
\item
  \begin{itemize}
  \item
  \item
  \item
  \item
  \item
  \end{itemize}
\end{itemize}

EDMONTON, Alberta --- It is said that hockey is the heartbeat of
Alberta's capital city. If that's so then the
\href{https://www.nytimes3xbfgragh.onion/2020/07/06/sports/hockey/nhl-playoffs.html}{2020
N.H.L. playoffs} are like a defibrillator that has shocked the city's
rhythm back to life.

For months it looked like Las Vegas --- with its massive resorts and
status as host to the league's off-season awards --- would be chosen as
the primary hub city for the N.H.L.'s summer restart after the regular
season was paused in March because of the coronavirus pandemic. But
Edmonton, a city of just less than 1 million people, persisted and the
dogged effort paid off in early July when the league instead selected it
and Toronto as the sites for postseason play.

``We've had a great staff doing a ton of work,'' said Bob Nicholson, the
Edmonton Oilers' chairman. Nicholson singled out the team's owner, Daryl
Katz, for pestering N.H.L. Commissioner Gary Bettman during
deliberations. ``But really it was Daryl, starting with the vision. He
called Gary a ton.''

On Saturday, 12 Western Conference teams will begin the qualifying round
and round-robin seeding tournament at Rogers Place, the four-year-old
arena that sits at the center of the city's Ice District, a \$2.5
billion (CAN) mixed-use sports and entertainment zone. It will be the
site of both conference finals and the Stanley Cup final.

The Ice District may not have the same global profile as the Las Vegas
Strip, but in Edmonton, which once billed itself the ``City of
Champions,'' hockey keeps the community pumping.

``We are oil country and we are a hockey town,'' said Janet Riopel, the
president of the city's chamber of commerce. ``Our kids start early.
They play through most of their lives, male and female. We are a hockey
community and we've been very proud of our team. Oil country fans are
die-hard fans.''

\includegraphics{https://static01.graylady3jvrrxbe.onion/images/2020/08/01/sports/01nhl-edmonton2-print/merlin_175166319_4b851fc2-8f3d-4f66-bde8-a8446e949faa-articleLarge.jpg?quality=75\&auto=webp\&disable=upscale}

Kevin Lowe, the six-time Stanley Cup winner, Hall of Famer and former
Oilers general manager, arrived in the city in 1979, the year the former
World Hockey Association franchise joined the N.H.L. Championship hockey
quickly became a way of life for the city. Building around the league's
career leading scorer, Wayne Gretzky, the Oilers won five Stanley Cups
between 1984 and 1990 --- and made sure they shared their success with
the community.

``In all likelihood, if you grew up in Edmonton during the `80s, you
probably either were in a bar with a couple of us, or you might even
have had a sip from the Cup,'' Lowe said.

Sandy Langley, 53, is one of those people. She started working for the
Oilers as a 15-year-old usher at the old Northlands Coliseum. Since
1993, she has worked in the team's front office in various
administrative capacities.

``My husband was a bouncer at one of the main bars here,'' Langley said.
``Back then, all of us became really, really good friends. They were
just very approachable. They went out quite a bit, so you saw them, you
know, at the grocery store. People felt that they could talk to them.''

Through another schoolmate, Langley said she got to know the former
Oiler Esa Tikkanen and his first wife, Lotta.

``I think as soon as a player feels comfortable with you, they kind of
welcome you into their whole group. So when we became friends with Lotta
and Esa, we would go to their house. Grant Fuhr and his wife would be
there, and Jari Kurri and his wife. We were almost like a family for
them, because they didn't have family here.''

\hypertarget{the-games-resume}{%
\subsubsection{The Games Resume}\label{the-games-resume}}

\hypertarget{sports-and-the-virus}{%
\paragraph{Sports and the Virus}\label{sports-and-the-virus}}

Updated Aug. 3, 2020

Here's what's happening as the world of sports slowly comes back to
life:

\begin{itemize}
\item
  \begin{itemize}
  \tightlist
  \item
    On all but the two biggest courts, automated line calls
    \href{https://www.nytimes3xbfgragh.onion/2020/08/03/sports/tennis/us-open-hawkeye-line-judges.html?action=click\&pgtype=Article\&state=default\&region=MAIN_CONTENT_2\&context=storylines_keepup}{will
    replace human judges} at the U.S. Open to reduce the number of
    people on site during the pandemic.
  \item
    Mets star Yoenis Cespedes is healthy, but
    \href{https://www.nytimes3xbfgragh.onion/2020/08/02/sports/baseball/Yoenis-cespedes-opt-out-rule.html?action=click\&pgtype=Article\&state=default\&region=MAIN_CONTENT_2\&context=storylines_keepup}{has
    decided to opt out} of the 2020 baseball season for Covid-related
    reasons.
  \item
    Britain tried to bring fans back to indoor sports.
    \href{https://www.nytimes3xbfgragh.onion/2020/08/02/sports/snooker-world-championship.html?action=click\&pgtype=Article\&state=default\&region=MAIN_CONTENT_2\&context=storylines_keepup}{It
    lasted a day}.
  \end{itemize}
\end{itemize}

Langley and her husband got married in June 1988, two months before the
blockbuster trade that sent Gretzky to the Los Angeles Kings. The
Tikkanens were wedding guests.

``It wasn't anything, to ask them if they would come to our wedding,''
Langley said. ``Then, for my husband's stag, Esa brought the Stanley
Cup. That was unbelievable.''

Image

Hockey's most hard-core supporters will not be allowed to cheer from
inside Rogers Place or stake out the player entrance to ask for
autographs, as the N.H.L. will keep players in a fan-free
``bubble.''Credit...Jeff Vinnick/Getty Images

When the regular season was paused in March, the Oilers were on track to
return to the playoffs for the first time in three years. Forward Leon
Draisaitl led the league's scoring race by 13 points, and is the
\href{https://www.nytimes3xbfgragh.onion/2020/07/27/sports/hockey/nhl-awards-restart-postseason.html}{favorite
to win the Hart Trophy,} awarded to the league's most valuable player.
The team also plays behind the 2017 Hart winner, Connor McDavid, and got
a spark in December when winger Kailer Yamamoto, 21, was called up from
the A.H.L. and scored at a point-per-game pace.

The Oilers start the postseason facing the
\href{https://www.nytimes3xbfgragh.onion/2020/07/07/sports/hockey/chicago-blackhawks-stand-by-logo.html}{Chicago
Blackhawks} in a best-of-five series but fans won't be able to pierce
the league's ``bubble.'' That means some of hockey's most hard-core
supporters not allowed to cheer from inside Rogers Place or stake out
the player entrance to ask for autographs.

They'll be on the outside looking in as the local arena hosts up to
three games a day in the early rounds, a feat that required packing what
should have been months of planning into the span of two weeks.

``As soon as we started to get inklings that we were going to be in ---
because we kind of felt that we might not be --- we really had to time
it right, because a lot of our staff were not working,'' said Stu
Ballantyne, the Oilers' senior vice president of operations.

Their preparation included bouncing back when a storm ripped away part
of the building's roof in mid-July, causing flooding that damaged a
small portion of the entrance and mezzanine. Ballantyne said the damage
did not set the organization's plans back in a significant way.

Among the other considerations were sanitizing and facilitating social
distancing as teams come and go from the building's six dressing rooms.
Arena staff will also have to maintain the ice for more than 12 hours of
daily hockey, cooling down the building temperature even more than
usual, since there will be no fans in the stands to keep comfortable.

In essence, Rogers Place has become a massive soundstage for a
made-for-television event.

``At times you think, `Holy smokes, you won't get there,''' Nicholson
said. ``Hopefully, we add things and we're going to get better every day
from here on out, too. You know, we have to do that for the players.''

Outside the building, the plaza near the main entrance to Rogers Place
has been turned into an outdoor recreation area where players can get a
bite to eat or play basketball, enjoying the pleasant Edmonton summer,
where daily temperatures top out in the 70s and there are 16 hours of
daylight.

Though fans aren't allowed inside to watch games, Lowe believes they'll
find new ways to enjoy summer hockey.

``I think the biggest difference is that people will be sitting on their
patios, next to their pools, by a lake, by a river,'' Lowe said from his
off-season home in British Columbia's Shuswap region.

``It's summertime, right? So they'll be, in all likelihood, watching in
the strangest of places.''

Advertisement

\protect\hyperlink{after-bottom}{Continue reading the main story}

\hypertarget{site-index}{%
\subsection{Site Index}\label{site-index}}

\hypertarget{site-information-navigation}{%
\subsection{Site Information
Navigation}\label{site-information-navigation}}

\begin{itemize}
\tightlist
\item
  \href{https://help.nytimes3xbfgragh.onion/hc/en-us/articles/115014792127-Copyright-notice}{©~2020~The
  New York Times Company}
\end{itemize}

\begin{itemize}
\tightlist
\item
  \href{https://www.nytco.com/}{NYTCo}
\item
  \href{https://help.nytimes3xbfgragh.onion/hc/en-us/articles/115015385887-Contact-Us}{Contact
  Us}
\item
  \href{https://www.nytco.com/careers/}{Work with us}
\item
  \href{https://nytmediakit.com/}{Advertise}
\item
  \href{http://www.tbrandstudio.com/}{T Brand Studio}
\item
  \href{https://www.nytimes3xbfgragh.onion/privacy/cookie-policy\#how-do-i-manage-trackers}{Your
  Ad Choices}
\item
  \href{https://www.nytimes3xbfgragh.onion/privacy}{Privacy}
\item
  \href{https://help.nytimes3xbfgragh.onion/hc/en-us/articles/115014893428-Terms-of-service}{Terms
  of Service}
\item
  \href{https://help.nytimes3xbfgragh.onion/hc/en-us/articles/115014893968-Terms-of-sale}{Terms
  of Sale}
\item
  \href{https://spiderbites.nytimes3xbfgragh.onion}{Site Map}
\item
  \href{https://help.nytimes3xbfgragh.onion/hc/en-us}{Help}
\item
  \href{https://www.nytimes3xbfgragh.onion/subscription?campaignId=37WXW}{Subscriptions}
\end{itemize}
