Sections

SEARCH

\protect\hyperlink{site-content}{Skip to
content}\protect\hyperlink{site-index}{Skip to site index}

\href{https://www.nytimes3xbfgragh.onion/section/us}{U.S.}

\href{https://myaccount.nytimes3xbfgragh.onion/auth/login?response_type=cookie\&client_id=vi}{}

\href{https://www.nytimes3xbfgragh.onion/section/todayspaper}{Today's
Paper}

\href{/section/us}{U.S.}\textbar{}Connie Culp, First Face Transplant
Recipient in U.S., Dies at 57

\url{https://nyti.ms/3jVN6kK}

\begin{itemize}
\item
\item
\item
\item
\item
\end{itemize}

Advertisement

\protect\hyperlink{after-top}{Continue reading the main story}

Supported by

\protect\hyperlink{after-sponsor}{Continue reading the main story}

\hypertarget{connie-culp-first-face-transplant-recipient-in-us-dies-at-57}{%
\section{Connie Culp, First Face Transplant Recipient in U.S., Dies at
57}\label{connie-culp-first-face-transplant-recipient-in-us-dies-at-57}}

Her near-total transplant in 2008 was at the time the most complex one
ever performed. She was the fourth patient in the world to undergo such
a procedure.

\includegraphics{https://static01.graylady3jvrrxbe.onion/images/2020/08/04/multimedia/01xp-connieculp-pix1/merlin_175200339_febffdf5-1c8c-47cb-96b2-27c133295c8d-articleLarge.jpg?quality=75\&auto=webp\&disable=upscale}

By \href{https://www.nytimes3xbfgragh.onion/by/bryan-pietsch}{Bryan
Pietsch}

\begin{itemize}
\item
  Published Aug. 1, 2020Updated Aug. 3, 2020
\item
  \begin{itemize}
  \item
  \item
  \item
  \item
  \item
  \end{itemize}
\end{itemize}

Connie Culp, the first patient in the United States to receive a face
transplant, died on Wednesday at the Cleveland Clinic, where her
procedure was performed in 2008. She was 57.

The cause was an infection unrelated to her transplant, a spokeswoman
for the hospital said. She said Ms. Culp had been the longest-living
face transplant patient in the world.

Dr. Frank Papay, chairman of the Cleveland Clinic's Dermatology and
Plastic Surgery Institute, said of Ms. Culp, ``Her decision to undergo a
sometimes-daunting procedure is an enduring gift for all of humanity.''

Dr. Papay was part of the surgical team that performed the operation,
replacing Ms. Culp's damaged face with that of a recently deceased
woman. Lasting 23 hours, it was the most extensive and complicated face
transplant
\href{https://www.nytimes3xbfgragh.onion/2008/12/17/health/17face.html}{at
the time}. Three face transplants had been done before hers: two in
France and one in China.

A Cleveland Clinic ethics committee had approved such a procedure only
in 2004, the first such committee to do so. Dr. Eric Kodish, then the
chairman of the clinic's bioethics department,
\href{https://www.nytimes3xbfgragh.onion/2008/12/18/health/s18face.html}{told
The New York Times} that before the transplant Ms. Culp had undergone
psychological testing, in which she was asked whether it was she or a
family member who wanted the transplant. She was also asked how she felt
about the prospect of living with the face of someone who had died.

Most of her face had been damaged when she was shot by her husband,
Thomas Culp, in 2004; her wounds had left her unable to breathe or eat
on her own.

Her husband, with whom she had a common-law marriage, according to
\href{https://www.cleveland.com/healthfit/2010/11/woman_who_underwent_first_near.html}{The
Plain Dealer} of Cleveland, was sentenced to seven years in prison for
aggravated attempted murder and released in 2011.

After shooting his wife, Mr. Culp turned the shotgun on himself but lost
only a few teeth and some vision in his left eye.

Ms. Culp said she had forgiven her husband. ``I still love my husband,''
she told the ABC program
``\href{https://abcnews.go.com/Health/MindMoodNews/story?id=7535591}{Good
Morning America}'' in 2009. ``I forgave him the day he did it. I have
to.''

About 40 such surgeries have been performed worldwide since Ms. Culp's,
said Dr. Thomas Romo III, director of facial plastic and reconstructive
surgery at Lenox Hill Hospital and Manhattan Eye, Ear and Throat
Hospital.

\href{https://www.nytimes3xbfgragh.onion/2008/12/18/health/s18face.html}{Her
procedure was successful}, and for the rest of her life she needed to
take drugs to prevent her immune system from rejecting the transplant.
The drugs, however, make the recipient more susceptible to infections.

Charla Nash, who received a full face transplant in 2011 after being
mauled by
\href{https://abcnews.go.com/US/sandra-herold-owner-chimp-travis-mauled-charla-nash-dies/story?id=10740652}{a
friend's pet chimpanzee} in Stamford, Conn.,
\href{https://www.nytimes3xbfgragh.onion/2016/05/06/nyregion/chimpanzee-attack-victim-who-got-face-transplant-is-hospitalized.html}{was
hospitalized} in 2016 after she participated in a trial to determine if
transplant patients could be weaned off the anti-rejection drugs.

Most face transplant procedures are necessitated by gunshot wounds or
accidents involving animals, Dr. Romo said.

After successful transplants, most patients are able to speak, eat and
otherwise live a more normal life. If not for the surgery, Ms. Culp
would not have been able to smile or talk, Dr. Romo said, adding that
face transplants can have positive psychological effects for patients.

She was chosen for the experimental surgery because of her optimism and
willingness to follow medical orders, according to a
\href{https://www.cleveland.com/healthfit/2010/11/woman_who_underwent_first_near.html}{2010
profile} of her in The Plain Dealer, which described her as having been
``a hard-working, fun-loving, Harley-riding, thumb-wrestling, small-town
Ohio woman.''

Ms. Culp and her husband ran a drywall, painting and wallpapering
business before they bought a restaurant and bar in 2004. She often
worked there from the early morning until late at night, The Plain
Dealer reported.

After spending six years in hospitals and nursing homes, she returned to
her home in coal-mining country in southeastern Ohio, where a daughter,
Alicia, helped care for her, the newspaper said. She also had a son,
Steven. Complete information on her survivors was not available.

Connie Wagoner was born on March 26, 1963, and grew up in eastern Ohio,
near the Pennsylvania border.

At a news conference
\href{https://www.nytimes3xbfgragh.onion/2009/05/06/science/06face.html}{unveiling
her new face} in 2009, Ms. Culp asked that others be kind to people with
facial disfigurements.

``Don't judge people who don't look the same as you do,'' she said.
``Because you never know. One day it might be all taken away.''

Advertisement

\protect\hyperlink{after-bottom}{Continue reading the main story}

\hypertarget{site-index}{%
\subsection{Site Index}\label{site-index}}

\hypertarget{site-information-navigation}{%
\subsection{Site Information
Navigation}\label{site-information-navigation}}

\begin{itemize}
\tightlist
\item
  \href{https://help.nytimes3xbfgragh.onion/hc/en-us/articles/115014792127-Copyright-notice}{©~2020~The
  New York Times Company}
\end{itemize}

\begin{itemize}
\tightlist
\item
  \href{https://www.nytco.com/}{NYTCo}
\item
  \href{https://help.nytimes3xbfgragh.onion/hc/en-us/articles/115015385887-Contact-Us}{Contact
  Us}
\item
  \href{https://www.nytco.com/careers/}{Work with us}
\item
  \href{https://nytmediakit.com/}{Advertise}
\item
  \href{http://www.tbrandstudio.com/}{T Brand Studio}
\item
  \href{https://www.nytimes3xbfgragh.onion/privacy/cookie-policy\#how-do-i-manage-trackers}{Your
  Ad Choices}
\item
  \href{https://www.nytimes3xbfgragh.onion/privacy}{Privacy}
\item
  \href{https://help.nytimes3xbfgragh.onion/hc/en-us/articles/115014893428-Terms-of-service}{Terms
  of Service}
\item
  \href{https://help.nytimes3xbfgragh.onion/hc/en-us/articles/115014893968-Terms-of-sale}{Terms
  of Sale}
\item
  \href{https://spiderbites.nytimes3xbfgragh.onion}{Site Map}
\item
  \href{https://help.nytimes3xbfgragh.onion/hc/en-us}{Help}
\item
  \href{https://www.nytimes3xbfgragh.onion/subscription?campaignId=37WXW}{Subscriptions}
\end{itemize}
