Sections

SEARCH

\protect\hyperlink{site-content}{Skip to
content}\protect\hyperlink{site-index}{Skip to site index}

\href{https://www.nytimes3xbfgragh.onion/section/world/africa}{Africa}

\href{https://myaccount.nytimes3xbfgragh.onion/auth/login?response_type=cookie\&client_id=vi}{}

\href{https://www.nytimes3xbfgragh.onion/section/todayspaper}{Today's
Paper}

\href{/section/world/africa}{Africa}\textbar{}Ethiopian Workers Are
Forced to Return Home, Some With Coronavirus

\url{https://nyti.ms/3k0lTxf}

\begin{itemize}
\item
\item
\item
\item
\item
\end{itemize}

\href{https://www.nytimes3xbfgragh.onion/news-event/coronavirus?action=click\&pgtype=Article\&state=default\&region=TOP_BANNER\&context=storylines_menu}{The
Coronavirus Outbreak}

\begin{itemize}
\tightlist
\item
  live\href{https://www.nytimes3xbfgragh.onion/2020/08/02/world/coronavirus-updates.html?action=click\&pgtype=Article\&state=default\&region=TOP_BANNER\&context=storylines_menu}{Latest
  Updates}
\item
  \href{https://www.nytimes3xbfgragh.onion/interactive/2020/us/coronavirus-us-cases.html?action=click\&pgtype=Article\&state=default\&region=TOP_BANNER\&context=storylines_menu}{Maps
  and Cases}
\item
  \href{https://www.nytimes3xbfgragh.onion/interactive/2020/science/coronavirus-vaccine-tracker.html?action=click\&pgtype=Article\&state=default\&region=TOP_BANNER\&context=storylines_menu}{Vaccine
  Tracker}
\item
  \href{https://www.nytimes3xbfgragh.onion/interactive/2020/07/29/us/schools-reopening-coronavirus.html?action=click\&pgtype=Article\&state=default\&region=TOP_BANNER\&context=storylines_menu}{What
  School May Look Like}
\item
  \href{https://www.nytimes3xbfgragh.onion/live/2020/07/31/business/stock-market-today-coronavirus?action=click\&pgtype=Article\&state=default\&region=TOP_BANNER\&context=storylines_menu}{Economy}
\end{itemize}

Advertisement

\protect\hyperlink{after-top}{Continue reading the main story}

Supported by

\protect\hyperlink{after-sponsor}{Continue reading the main story}

\hypertarget{ethiopian-workers-are-forced-to-return-home-some-with-coronavirus}{%
\section{Ethiopian Workers Are Forced to Return Home, Some With
Coronavirus}\label{ethiopian-workers-are-forced-to-return-home-some-with-coronavirus}}

Stigmatized, out of work and facing dangers, migrant laborers are
returning by the thousands --- and may be fueling a growing outbreak in
Ethiopia.

\includegraphics{https://static01.graylady3jvrrxbe.onion/images/2020/08/02/world/02ethiopia-migrants/merlin_173729919_f42842bb-de6f-4b2d-a84e-7034e168495a-articleLarge.jpg?quality=75\&auto=webp\&disable=upscale}

By Simon Marks

\begin{itemize}
\item
  Aug. 1, 2020
\item
  \begin{itemize}
  \item
  \item
  \item
  \item
  \item
  \end{itemize}
\end{itemize}

ADDIS ABABA, Ethiopia --- Unemployed and shunned as possible coronavirus
carriers, Ethiopian migrant laborers are returning home by the
thousands, placing a huge strain on Ethiopia's poorly equipped medical
system.

More than 30,000 workers have re-entered Ethiopia since mid-March,
according to the government, some of them after suffering abuse and
detention in unhealthy conditions in the countries they left, often on
the Persian Gulf or in other parts of Africa.

At least 927 migrant laborers were infected with the virus when they
returned, Ethiopian officials say, but the true number is probably much
higher. The government has not updated that figure for more than a
month, and it does not include those who have slipped back into the
country unnoticed.

Ethiopia has had more than 16,000 confirmed infections and 250 Covid-19
deaths, according to
\href{https://www.nytimes3xbfgragh.onion/interactive/2020/world/coronavirus-maps.html\#countries}{figures
compiled by The New York Times}. Those are very low counts for a nation
of 115 million people, but the numbers are rising and many cases go
undetected by the country's sparse testing.

Doctors fear the outbreak may be primed to explode, fueled in part by
returning migrants whose journeys often include crowded, unsanitary
conditions --- jails in the countries where they worked, informal
migrant camps in countries like Yemen and Djibouti and quarantine
centers once they arrive back in Ethiopia.

\includegraphics{https://static01.graylady3jvrrxbe.onion/images/2020/08/02/world/02ethiopia-migrants3/merlin_173930694_6e769e07-4519-4ecc-af53-93b23b4070de-articleLarge.jpg?quality=75\&auto=webp\&disable=upscale}

Dr. Yohanes Tesfaye, who runs a government Covid-19 treatment center
near the eastern city of Dire Dawa, said that within a month of opening,
the center had treated 248 infected migrants. And, he warned, ``we have
a long border, so we can't be sure'' whether many more people with the
virus are entering the country undetected.

All this is occurring in a country that has just one respiratory
therapist, ill-equipped public hospitals and few medical resources in
rural areas, and is also suffering the economic blow of the pandemic.
Major hotels in the capital city, Addis Ababa, are almost empty, jobs in
tourism and construction have disappeared and the flow of money sent
home by workers overseas has dried up.

Adding to Ethiopia's struggles have been deadly conflicts between ethnic
groups that prompted the government to shut down the internet for more
than three weeks before recently restoring it. Hundreds of people died
in clashes and anti-government protests following the killing in June of
\href{https://www.nytimes3xbfgragh.onion/2020/07/11/world/africa/ethiopia-hachalu-hundessa.html}{the
singer Hachaluu Hundessa}, who was particularly revered by the Oromo
ethnic group.

Many of the migrants have returned voluntarily, suddenly unable to work
abroad after the pandemic shut down entire economies. Whether out of
economic hardship or fear of contagion, employers have
\href{https://www.nytimes3xbfgragh.onion/2020/07/06/world/middleeast/coronavirus-saudi-domestic-workers-maids-arab.html}{abruptly
laid off migrant domestic workers}, often leaving them at their
countries' embassies.

\hypertarget{latest-updates-global-coronavirus-outbreak}{%
\section{\texorpdfstring{\href{https://www.nytimes3xbfgragh.onion/2020/08/01/world/coronavirus-covid-19.html?action=click\&pgtype=Article\&state=default\&region=MAIN_CONTENT_1\&context=storylines_live_updates}{Latest
Updates: Global Coronavirus
Outbreak}}{Latest Updates: Global Coronavirus Outbreak}}\label{latest-updates-global-coronavirus-outbreak}}

Updated 2020-08-02T17:52:35.962Z

\begin{itemize}
\tightlist
\item
  \href{https://www.nytimes3xbfgragh.onion/2020/08/01/world/coronavirus-covid-19.html?action=click\&pgtype=Article\&state=default\&region=MAIN_CONTENT_1\&context=storylines_live_updates\#link-34047410}{The
  U.S. reels as July cases more than double the total of any other
  month.}
\item
  \href{https://www.nytimes3xbfgragh.onion/2020/08/01/world/coronavirus-covid-19.html?action=click\&pgtype=Article\&state=default\&region=MAIN_CONTENT_1\&context=storylines_live_updates\#link-780ec966}{Top
  U.S. officials work to break an impasse over the federal jobless
  benefit.}
\item
  \href{https://www.nytimes3xbfgragh.onion/2020/08/01/world/coronavirus-covid-19.html?action=click\&pgtype=Article\&state=default\&region=MAIN_CONTENT_1\&context=storylines_live_updates\#link-2bc8948}{Its
  outbreak untamed, Melbourne goes into even greater lockdown.}
\end{itemize}

\href{https://www.nytimes3xbfgragh.onion/2020/08/01/world/coronavirus-covid-19.html?action=click\&pgtype=Article\&state=default\&region=MAIN_CONTENT_1\&context=storylines_live_updates}{See
more updates}

More live coverage:
\href{https://www.nytimes3xbfgragh.onion/live/2020/07/31/business/stock-market-today-coronavirus?action=click\&pgtype=Article\&state=default\&region=MAIN_CONTENT_1\&context=storylines_live_updates}{Markets}

But many others have been rounded up, confined and deported by host
governments that had previously tolerated them. In interviews, senior
government officials, doctors, health workers and more than a dozen
returnees from Kuwait, Saudi Arabia, Yemen, Djibouti and Lebanon
describe scenes of workers being mistreated in detention before being
sent back to Ethiopia.

Image

An Ethiopian domestic worker at the Ethiopian Consulate outside Beirut,
where she and others were abandoned by their Lebanese employers last
month.Credit...Hassan Ammar/Associated Press

``The police were throwing racial slurs at us. They called me an
animal,'' said Selam Bizuneh, 26, who worked as a maid in Kuwait until
her employer stopped paying her. She said she spent 40 days in a
detention center in Kuwait City's Al Farwaniyah district in May and
June, adding, ``we were roughed up and forced to stand.''

Shortly after arriving back in Ethiopia in late June, she said, she
tested positive for the coronavirus.

Birhan Tesfay, 27, left Ethiopia hoping to find work in Saudi Arabia,
but turned back as the pandemic spread. He said he paid smugglers \$300
to cross the Red Sea from Yemen to Djibouti in the middle of the night
on June 5.

``We were shot at by Djibouti's navy on our way back,'' he said in a
telephone interview from a quarantine center. ``One migrant died while
the smugglers attempted to escape.''

His story was verified by a United Nations staff member, who spoke on
the condition of anonymity because she was not authorized to discuss it.
Mr. Birhan was arrested by Djiboutian security forces and returned to
Ethiopia.

People who work with them say that out-of-work migrants, fearing
detention and abuse, stay on the fringes of society, where infections
are unlikely to be detected --- much less treated --- and band together
in ways that prevent social distancing.

``It is extremely important that at this time we restrict the movement
of people as there is a direct link between movement and the number of
Covid cases,'' said Tsion Teklu, Ethiopia's State Minister for Economic
Diplomacy and Diaspora Affairs.

``This is particularly important in Ethiopia, where the number of cases
is increasing, the economy is strained right down to the level of local
vendors and the health sector is under pressure.''

Image

Doctors caring for a coronavirus patient in the ICU Saint Petros
Hospital in Addis Ababa, Ethiopia, this month.~Credit...Amanuel
Sileshi/Agence France-Presse --- Getty Images

Before the pandemic, about 100,000 Ethiopians made the perilous trip
each year to other parts of the world to find work --- often illegally
--- as maids, construction workers, drivers, hairdressers, guards and
more. The largest number make their way to the Arabian Peninsula, though
workers have also been sent back this year by Lebanon, India, Pakistan,
the United States, Kenya and other countries.

\href{https://www.nytimes3xbfgragh.onion/news-event/coronavirus?action=click\&pgtype=Article\&state=default\&region=MAIN_CONTENT_3\&context=storylines_faq}{}

\hypertarget{the-coronavirus-outbreak-}{%
\subsubsection{The Coronavirus Outbreak
›}\label{the-coronavirus-outbreak-}}

\hypertarget{frequently-asked-questions}{%
\paragraph{Frequently Asked
Questions}\label{frequently-asked-questions}}

Updated July 27, 2020

\begin{itemize}
\item ~
  \hypertarget{should-i-refinance-my-mortgage}{%
  \paragraph{Should I refinance my
  mortgage?}\label{should-i-refinance-my-mortgage}}

  \begin{itemize}
  \tightlist
  \item
    \href{https://www.nytimes3xbfgragh.onion/article/coronavirus-money-unemployment.html?action=click\&pgtype=Article\&state=default\&region=MAIN_CONTENT_3\&context=storylines_faq}{It
    could be a good idea,} because mortgage rates have
    \href{https://www.nytimes3xbfgragh.onion/2020/07/16/business/mortgage-rates-below-3-percent.html?action=click\&pgtype=Article\&state=default\&region=MAIN_CONTENT_3\&context=storylines_faq}{never
    been lower.} Refinancing requests have pushed mortgage applications
    to some of the highest levels since 2008, so be prepared to get in
    line. But defaults are also up, so if you're thinking about buying a
    home, be aware that some lenders have tightened their standards.
  \end{itemize}
\item ~
  \hypertarget{what-is-school-going-to-look-like-in-september}{%
  \paragraph{What is school going to look like in
  September?}\label{what-is-school-going-to-look-like-in-september}}

  \begin{itemize}
  \tightlist
  \item
    It is unlikely that many schools will return to a normal schedule
    this fall, requiring the grind of
    \href{https://www.nytimes3xbfgragh.onion/2020/06/05/us/coronavirus-education-lost-learning.html?action=click\&pgtype=Article\&state=default\&region=MAIN_CONTENT_3\&context=storylines_faq}{online
    learning},
    \href{https://www.nytimes3xbfgragh.onion/2020/05/29/us/coronavirus-child-care-centers.html?action=click\&pgtype=Article\&state=default\&region=MAIN_CONTENT_3\&context=storylines_faq}{makeshift
    child care} and
    \href{https://www.nytimes3xbfgragh.onion/2020/06/03/business/economy/coronavirus-working-women.html?action=click\&pgtype=Article\&state=default\&region=MAIN_CONTENT_3\&context=storylines_faq}{stunted
    workdays} to continue. California's two largest public school
    districts --- Los Angeles and San Diego --- said on July 13, that
    \href{https://www.nytimes3xbfgragh.onion/2020/07/13/us/lausd-san-diego-school-reopening.html?action=click\&pgtype=Article\&state=default\&region=MAIN_CONTENT_3\&context=storylines_faq}{instruction
    will be remote-only in the fall}, citing concerns that surging
    coronavirus infections in their areas pose too dire a risk for
    students and teachers. Together, the two districts enroll some
    825,000 students. They are the largest in the country so far to
    abandon plans for even a partial physical return to classrooms when
    they reopen in August. For other districts, the solution won't be an
    all-or-nothing approach.
    \href{https://bioethics.jhu.edu/research-and-outreach/projects/eschool-initiative/school-policy-tracker/}{Many
    systems}, including the nation's largest, New York City, are
    devising
    \href{https://www.nytimes3xbfgragh.onion/2020/06/26/us/coronavirus-schools-reopen-fall.html?action=click\&pgtype=Article\&state=default\&region=MAIN_CONTENT_3\&context=storylines_faq}{hybrid
    plans} that involve spending some days in classrooms and other days
    online. There's no national policy on this yet, so check with your
    municipal school system regularly to see what is happening in your
    community.
  \end{itemize}
\item ~
  \hypertarget{is-the-coronavirus-airborne}{%
  \paragraph{Is the coronavirus
  airborne?}\label{is-the-coronavirus-airborne}}

  \begin{itemize}
  \tightlist
  \item
    The coronavirus
    \href{https://www.nytimes3xbfgragh.onion/2020/07/04/health/239-experts-with-one-big-claim-the-coronavirus-is-airborne.html?action=click\&pgtype=Article\&state=default\&region=MAIN_CONTENT_3\&context=storylines_faq}{can
    stay aloft for hours in tiny droplets in stagnant air}, infecting
    people as they inhale, mounting scientific evidence suggests. This
    risk is highest in crowded indoor spaces with poor ventilation, and
    may help explain super-spreading events reported in meatpacking
    plants, churches and restaurants.
    \href{https://www.nytimes3xbfgragh.onion/2020/07/06/health/coronavirus-airborne-aerosols.html?action=click\&pgtype=Article\&state=default\&region=MAIN_CONTENT_3\&context=storylines_faq}{It's
    unclear how often the virus is spread} via these tiny droplets, or
    aerosols, compared with larger droplets that are expelled when a
    sick person coughs or sneezes, or transmitted through contact with
    contaminated surfaces, said Linsey Marr, an aerosol expert at
    Virginia Tech. Aerosols are released even when a person without
    symptoms exhales, talks or sings, according to Dr. Marr and more
    than 200 other experts, who
    \href{https://academic.oup.com/cid/article/doi/10.1093/cid/ciaa939/5867798}{have
    outlined the evidence in an open letter to the World Health
    Organization}.
  \end{itemize}
\item ~
  \hypertarget{what-are-the-symptoms-of-coronavirus}{%
  \paragraph{What are the symptoms of
  coronavirus?}\label{what-are-the-symptoms-of-coronavirus}}

  \begin{itemize}
  \tightlist
  \item
    Common symptoms
    \href{https://www.nytimes3xbfgragh.onion/article/symptoms-coronavirus.html?action=click\&pgtype=Article\&state=default\&region=MAIN_CONTENT_3\&context=storylines_faq}{include
    fever, a dry cough, fatigue and difficulty breathing or shortness of
    breath.} Some of these symptoms overlap with those of the flu,
    making detection difficult, but runny noses and stuffy sinuses are
    less common.
    \href{https://www.nytimes3xbfgragh.onion/2020/04/27/health/coronavirus-symptoms-cdc.html?action=click\&pgtype=Article\&state=default\&region=MAIN_CONTENT_3\&context=storylines_faq}{The
    C.D.C. has also} added chills, muscle pain, sore throat, headache
    and a new loss of the sense of taste or smell as symptoms to look
    out for. Most people fall ill five to seven days after exposure, but
    symptoms may appear in as few as two days or as many as 14 days.
  \end{itemize}
\item ~
  \hypertarget{does-asymptomatic-transmission-of-covid-19-happen}{%
  \paragraph{Does asymptomatic transmission of Covid-19
  happen?}\label{does-asymptomatic-transmission-of-covid-19-happen}}

  \begin{itemize}
  \tightlist
  \item
    So far, the evidence seems to show it does. A widely cited
    \href{https://www.nature.com/articles/s41591-020-0869-5}{paper}
    published in April suggests that people are most infectious about
    two days before the onset of coronavirus symptoms and estimated that
    44 percent of new infections were a result of transmission from
    people who were not yet showing symptoms. Recently, a top expert at
    the World Health Organization stated that transmission of the
    coronavirus by people who did not have symptoms was ``very rare,''
    \href{https://www.nytimes3xbfgragh.onion/2020/06/09/world/coronavirus-updates.html?action=click\&pgtype=Article\&state=default\&region=MAIN_CONTENT_3\&context=storylines_faq\#link-1f302e21}{but
    she later walked back that statement.}
  \end{itemize}
\end{itemize}

Coronavirus travel restrictions adopted by Saudi Arabia have left many
migrants stranded in Yemen and Djibouti. In Yemen, in particular,
African migrants have been scapegoated as virus spreaders, and some have
been
\href{https://www.nytimes3xbfgragh.onion/2020/06/28/world/middleeast/coronavirus-yemen-african-migrants.html}{shot
by the Houthi militia}, as it tries to chase them out of the parts of
the country it controls.

Saudi Arabia alone said in May that 12,000 Ethiopians were in the
country illegally and were to be repatriated, though not all of them
have been.

The Ethiopian government said on July 20 that it had helped 30,087 of
its citizens return home since the pandemic began. The government and
the United Nations have been in talks with other nations about managing
repatriation in a way that does not overwhelm Ethiopia.

``Mass movements at a time like this will only exacerbate the spread of
Covid-19,'' said Maureen Achieng, chief of mission in Ethiopia for the
International Organization for Migration, an arm of the United Nations.
``In this regard, government-to-government discussions will be critical
to reaching agreement on a common approach to containing Covid-19.''

Image

Ethiopian migrants resting in a building under construction in Aden,
Yemen, in June.Credit...Reuters

Once the migrants make it back to Ethiopia, there remain large gaps in
the services they need.

Those who return through official channels are tested for the
coronavirus. If they test positive, they are supposed to go to treatment
centers; those who test negative are supposed to go to quarantine
centers, where they are monitored until they are released.

But the facilities have serious shortages of equipment and staff, and
local officials sometimes confuse the two kinds of centers, said Zia
Hassan, program director for the Norwegian Refugee Council in Ethiopia,
which has provided beds and other goods to several quarantine centers.

Ethiopia's health ministry is ``really struggling to understand the
situation to provide the necessary action that has to be taken,'' he
said.

Even if they are healthy and free, returning migrant workers face
uncertain prospects in a poor country facing a multitude of challenges.

Zeytuna Kemal, 33, said she left her job as a maid in Kuwait after her
employer failed to pay her for three months of work. She decided to flee
the country, she said, but the police arrested her and jailed her for
four days without food or water.

Then she was transferred with dozens of other Ethiopians to a detention
center near the international airport and eventually flown back to her
home country. ``I am now lost and confused,'' she said. And she is
worried about providing for her children and mother.

``I will not find a job here.''

Tiksa Negeri contributed reporting from Addis Ababa.

Advertisement

\protect\hyperlink{after-bottom}{Continue reading the main story}

\hypertarget{site-index}{%
\subsection{Site Index}\label{site-index}}

\hypertarget{site-information-navigation}{%
\subsection{Site Information
Navigation}\label{site-information-navigation}}

\begin{itemize}
\tightlist
\item
  \href{https://help.nytimes3xbfgragh.onion/hc/en-us/articles/115014792127-Copyright-notice}{©~2020~The
  New York Times Company}
\end{itemize}

\begin{itemize}
\tightlist
\item
  \href{https://www.nytco.com/}{NYTCo}
\item
  \href{https://help.nytimes3xbfgragh.onion/hc/en-us/articles/115015385887-Contact-Us}{Contact
  Us}
\item
  \href{https://www.nytco.com/careers/}{Work with us}
\item
  \href{https://nytmediakit.com/}{Advertise}
\item
  \href{http://www.tbrandstudio.com/}{T Brand Studio}
\item
  \href{https://www.nytimes3xbfgragh.onion/privacy/cookie-policy\#how-do-i-manage-trackers}{Your
  Ad Choices}
\item
  \href{https://www.nytimes3xbfgragh.onion/privacy}{Privacy}
\item
  \href{https://help.nytimes3xbfgragh.onion/hc/en-us/articles/115014893428-Terms-of-service}{Terms
  of Service}
\item
  \href{https://help.nytimes3xbfgragh.onion/hc/en-us/articles/115014893968-Terms-of-sale}{Terms
  of Sale}
\item
  \href{https://spiderbites.nytimes3xbfgragh.onion}{Site Map}
\item
  \href{https://help.nytimes3xbfgragh.onion/hc/en-us}{Help}
\item
  \href{https://www.nytimes3xbfgragh.onion/subscription?campaignId=37WXW}{Subscriptions}
\end{itemize}
