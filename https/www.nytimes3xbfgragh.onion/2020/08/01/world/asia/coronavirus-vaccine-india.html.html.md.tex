Sections

SEARCH

\protect\hyperlink{site-content}{Skip to
content}\protect\hyperlink{site-index}{Skip to site index}

\href{https://www.nytimes3xbfgragh.onion/section/world/asia}{Asia
Pacific}

\href{https://myaccount.nytimes3xbfgragh.onion/auth/login?response_type=cookie\&client_id=vi}{}

\href{https://www.nytimes3xbfgragh.onion/section/todayspaper}{Today's
Paper}

\href{/section/world/asia}{Asia Pacific}\textbar{}Indian Billionaires
Bet Big on Head Start in Coronavirus Vaccine Race

\url{https://nyti.ms/2XicLKP}

\begin{itemize}
\item
\item
\item
\item
\item
\item
\end{itemize}

\href{https://www.nytimes3xbfgragh.onion/news-event/coronavirus?action=click\&pgtype=Article\&state=default\&region=TOP_BANNER\&context=storylines_menu}{The
Coronavirus Outbreak}

\begin{itemize}
\tightlist
\item
  live\href{https://www.nytimes3xbfgragh.onion/2020/08/01/world/coronavirus-covid-19.html?action=click\&pgtype=Article\&state=default\&region=TOP_BANNER\&context=storylines_menu}{Latest
  Updates}
\item
  \href{https://www.nytimes3xbfgragh.onion/interactive/2020/us/coronavirus-us-cases.html?action=click\&pgtype=Article\&state=default\&region=TOP_BANNER\&context=storylines_menu}{Maps
  and Cases}
\item
  \href{https://www.nytimes3xbfgragh.onion/interactive/2020/science/coronavirus-vaccine-tracker.html?action=click\&pgtype=Article\&state=default\&region=TOP_BANNER\&context=storylines_menu}{Vaccine
  Tracker}
\item
  \href{https://www.nytimes3xbfgragh.onion/interactive/2020/07/29/us/schools-reopening-coronavirus.html?action=click\&pgtype=Article\&state=default\&region=TOP_BANNER\&context=storylines_menu}{What
  School May Look Like}
\item
  \href{https://www.nytimes3xbfgragh.onion/live/2020/07/31/business/stock-market-today-coronavirus?action=click\&pgtype=Article\&state=default\&region=TOP_BANNER\&context=storylines_menu}{Economy}
\end{itemize}

Advertisement

\protect\hyperlink{after-top}{Continue reading the main story}

Supported by

\protect\hyperlink{after-sponsor}{Continue reading the main story}

\hypertarget{indian-billionaires-bet-big-on-head-start-in-coronavirus-vaccine-race}{%
\section{Indian Billionaires Bet Big on Head Start in Coronavirus
Vaccine
Race}\label{indian-billionaires-bet-big-on-head-start-in-coronavirus-vaccine-race}}

The world's largest vaccine producer, the Serum Institute, announced a
plan to make hundreds of millions of doses of an unproven inoculation.
It's a gamble with a huge upside. And huge risks.

\includegraphics{https://static01.graylady3jvrrxbe.onion/images/2020/08/02/world/02virus-india-vaccine/merlin_175095756_ab6859c7-a529-4f1f-809d-0b4ccf992203-articleLarge.jpg?quality=75\&auto=webp\&disable=upscale}

\href{https://www.nytimes3xbfgragh.onion/by/jeffrey-gettleman}{\includegraphics{https://static01.graylady3jvrrxbe.onion/images/2018/10/10/multimedia/author-jeffrey-gettleman/author-jeffrey-gettleman-thumbLarge.png}}

By
\href{https://www.nytimes3xbfgragh.onion/by/jeffrey-gettleman}{Jeffrey
Gettleman}

\begin{itemize}
\item
  Aug. 1, 2020Updated 9:19 a.m. ET
\item
  \begin{itemize}
  \item
  \item
  \item
  \item
  \item
  \item
  \end{itemize}
\end{itemize}

PUNE, India --- In early May, an extremely well-sealed steel box arrived
at the cold room of the Serum Institute of India, the world's largest
vaccine maker.

Inside, packed in dry ice, sat a tiny 1-milliliter vial from Oxford,
England, containing the cellular material for one of the
\href{https://www.nytimes3xbfgragh.onion/2020/04/27/world/europe/coronavirus-vaccine-update-oxford.html}{world's
most promising coronavirus vaccines}.

Scientists in white lab coats brought the vial to Building 14, carefully
poured the contents into a flask, added a medium of vitamins and sugar
and began growing billions of cells. Thus began one of the biggest
gambles yet in the quest to find the vaccine that will bring the world's
Covid-19 nightmare to an end.

\href{https://www.seruminstitute.com/index.php}{The Serum Institute},
which is exclusively controlled by a small and fabulously rich Indian
family and started out years ago as a horse farm, is doing what a few
other companies in the race for a vaccine are doing: mass-producing
hundreds of millions of doses of a vaccine candidate that is still in
trials and might not even work.

But if it does, Adar Poonawalla, Serum's chief executive and the only
child of the company's founder, will become one of the most tugged-at
men in the world. He will have on hand what everyone wants, possibly in
greater quantities before anyone else.

His company, which has teamed up with the Oxford scientists developing
the vaccine,
\href{https://www.news18.com/news/business/indias-serum-institute-to-make-millions-of-potential-coronavirus-vaccine-doses-2597041.html}{was
one of the first to boldly announce}, in April, that it was going to
mass-produce a vaccine before clinical trials even ended. Now, Mr.
Poonawalla's fastest vaccine assembly lines are being readied to crank
out 500 doses each minute, and his phone rings endlessly.

National health ministers, prime ministers and other heads of state (he
wouldn't say who) and friends he hasn't heard from in years have been
calling him, he said, begging for the first batches.

``I've had to explain to them that, `Look I can't just give it to you
like this,''' he said.

\includegraphics{https://static01.graylady3jvrrxbe.onion/images/2020/07/29/world/00virus-india-vaccine-2/merlin_174611103_eea92714-64a6-48e3-837d-a8774e80e04c-articleLarge.jpg?quality=75\&auto=webp\&disable=upscale}

With the coronavirus pandemic turning the world upside down and all
hopes pinned on a vaccine, the Serum Institute finds itself in the
middle of an extremely competitive and murky endeavor. To get the
vaccine out as soon as possible, vaccine developers say they need
Serum's mammoth assembly lines --- each year, it churns out 1.5 billion
doses of other vaccines, mostly for poor countries, more than any other
company.

Half of the world's children have been vaccinated with Serum's products.
Scale is its specialty. Just the other day, Mr. Poonawalla received a
shipment of 600 million glass vials.

But right now it's not entirely clear how much of the coronavirus
vaccine that Serum will mass-produce will be kept by India or who will
fund its production, leaving the Poonawallas to navigate a torrent of
cross-pressures, political, financial, external and domestic.

\href{https://www.nytimes3xbfgragh.onion/2020/07/16/world/asia/coronavirus-india-million-cases.html}{India
has been walloped by the coronavirus}, and with 1.3 billion people, it
needs vaccine doses as much as anywhere. It's also led by a highly
nationalistic prime minister, Narendra Modi, whose government has
already
\href{https://www.nytimes3xbfgragh.onion/2020/03/03/business/coronavirus-india-drugs.html}{blocked
exports of drugs} that were believed to help treat Covid-19, the disease
caused by the coronavirus.

Adar Poonawalla, 39, says that he will split the hundreds of millions of
vaccine doses he produces 50-50 between India and the rest of the world,
with a focus on poorer countries, and that Mr. Modi's government has not
objected to this.

\hypertarget{latest-updates-global-coronavirus-outbreak}{%
\section{\texorpdfstring{\href{https://www.nytimes3xbfgragh.onion/2020/08/01/world/coronavirus-covid-19.html?action=click\&pgtype=Article\&state=default\&region=MAIN_CONTENT_1\&context=storylines_live_updates}{Latest
Updates: Global Coronavirus
Outbreak}}{Latest Updates: Global Coronavirus Outbreak}}\label{latest-updates-global-coronavirus-outbreak}}

Updated 2020-08-01T19:08:55.687Z

\begin{itemize}
\tightlist
\item
  \href{https://www.nytimes3xbfgragh.onion/2020/08/01/world/coronavirus-covid-19.html?action=click\&pgtype=Article\&state=default\&region=MAIN_CONTENT_1\&context=storylines_live_updates\#link-3ac56579}{Top
  officials work to break impasse over jobless benefit.}
\item
  \href{https://www.nytimes3xbfgragh.onion/2020/08/01/world/coronavirus-covid-19.html?action=click\&pgtype=Article\&state=default\&region=MAIN_CONTENT_1\&context=storylines_live_updates\#link-8796723}{The
  virus picks up dangerous speed in the Midwest, and in areas that had
  seen success.}
\item
  \href{https://www.nytimes3xbfgragh.onion/2020/08/01/world/coronavirus-covid-19.html?action=click\&pgtype=Article\&state=default\&region=MAIN_CONTENT_1\&context=storylines_live_updates\#link-25930521}{Thousands
  in Berlin protest Germany's coronavirus measures.}
\end{itemize}

\href{https://www.nytimes3xbfgragh.onion/2020/08/01/world/coronavirus-covid-19.html?action=click\&pgtype=Article\&state=default\&region=MAIN_CONTENT_1\&context=storylines_live_updates}{See
more updates}

More live coverage:
\href{https://www.nytimes3xbfgragh.onion/live/2020/07/31/business/stock-market-today-coronavirus?action=click\&pgtype=Article\&state=default\&region=MAIN_CONTENT_1\&context=storylines_live_updates}{Markets}

But he added, ``Look, they may still invoke some kind of emergency if
they deem fit or if they want to.''

Image

A multi-eyed, computer-driven camera carefully inspects each vial on the
assembly line, looking for cracks or other flaws.Credit...Atul Loke for
The New York Times

The
\href{https://www.nytimes3xbfgragh.onion/2020/04/27/world/europe/coronavirus-vaccine-update-oxford.html}{Oxford-designed
vaccine} is just one of several promising contenders that will soon be
mass-produced, in different factories around the world, before they are
proven to work. Vaccines take time not just to perfect but to
manufacture. Live cultures need weeks to grow inside bioreactors, for
instance, and each vial needs to be carefully cleaned, filled,
stoppered, sealed and packaged.

The idea is to conduct these two processes simultaneously and start
production now, while the vaccines are still in trials, so that as soon
as the trials are finished --- at best within the next six months,
though no one really knows --- vaccine doses will be on hand, ready for
a world desperate to protect itself.

American and European governments have committed billions of dollars to
this effort, cutting deals with pharmaceutical giants such as Johnson \&
Johnson, Pfizer, Sanofi and AstraZeneca to speed up the development and
production of select vaccine candidates in exchange for hundreds of
millions of doses.

Image

Adar Poonawalla turned this vintage plane that no longer flies into an
office suite on Serum's campus in Pune.Credit...Atul Loke for The New
York Times

AstraZeneca is the lead partner with the Oxford scientists, and it has
signed government contracts worth more than \$1 billion to manufacture
the vaccine for Europe, the United States and other markets. But it has
allowed the Serum Institute to produce it as well. The difference, Mr.
Poonawalla said, is that his company is shouldering the cost of
production on its own.

But Serum is distinct from all other major vaccine producers in an
important way. Like many highly successful Indian businesses, it is
family-run. It can make decisions quickly and take big risks, like the
one it's about to, which could cost the family hundreds of millions of
dollars.

Mr. Poonawalla said he was ``70 to 80 percent'' sure the Oxford vaccine
would work.

But, he added, ``I hope we don't go in too deep.''

Unbeholden to shareholders, the Serum Institute is steered by only two
men: Mr. Poonawalla and his father, Cyrus, a horse breeder turned
billionaire.

Image

Cyrus Poonawalla, center right, at a Poonawala-sponsored horse race in
Mumbai, in 2006. The elder Poonawalla began as a horse breeder and has
become a billionaire.Credit...Prodip Guha/Getty Images

More than 50 years ago, the Serum Institute began as a shed on the
family's thoroughbred horse farm. The elder Poonawalla realized that
instead of donating horses to a vaccine laboratory that needed horse
serum --- one way of producing vaccines is to inject horses with small
amounts of toxins and then extract their antibody-rich blood serum ---
he could process the serum and make the vaccines himself.

He started with tetanus in 1967. Then snake bite antidotes. Then shots
for tuberculosis, hepatitis, polio and the flu. From his stud farm in
the fertile and pleasantly humid town of Pune, Mr. Poonawalla built a
vaccine empire, and a staggering fortune.

Capitalizing on India's combination of cheap labor and advanced
technology, the Serum Institute won contracts from Unicef, the Pan
American Health Organization and scores of countries, many of them poor,
to supply low-cost vaccines. The Poonawallas have now entered the
\href{https://www.forbes.com/sites/naazneenkarmali/2020/04/07/indias-10-richest-billionaires-in-2020/\#316831d47c23}{pantheon
of India's richest families}, worth more than \$5 billion.

Horses are still everywhere. Live ones trot around emerald paddocks,
topiary ones guard the front gates, and fancy glass ornaments frozen in
mid-strut stand on the tabletop of Serum's baronial boardroom
overlooking its industrial park, where 5,000 people work.

Inside the facility producing the coronavirus vaccine candidate,
white-hooded scientists monitor the vital signs of the bioreactors, huge
stainless steel vats where the vaccine's cellular material is
reproduced. Visitors are not allowed inside but can peer through
double-paned glass.

Image

Technicians running one of the many high-speed vaccine assembly lines at
the Serum Institute in early July.Credit...Atul Loke for The New York
Times

``These cells are very delicate,'' said Santosh Narwade, a Serum
scientist. ``We have to take care with oxygen levels and mixing speed or
the cells get ruptured.''

His voice was jumpy with excitement.

``We all feel like we're giving the solution to our nation and our
world,'' he said.

Initial trial results of the Oxford-designed vaccine showed that it
activated antibody levels similar to those seen in recovering Covid-19
patients, which was considered very good news.

Serum has already produced millions of doses of this vaccine for
research and development, including large batches for the ongoing
trials. By the time the trials finish, expected around November, Serum
plans to have stockpiled 300 million doses for commercial use.

\href{https://www.nytimes3xbfgragh.onion/news-event/coronavirus?action=click\&pgtype=Article\&state=default\&region=MAIN_CONTENT_3\&context=storylines_faq}{}

\hypertarget{the-coronavirus-outbreak-}{%
\subsubsection{The Coronavirus Outbreak
›}\label{the-coronavirus-outbreak-}}

\hypertarget{frequently-asked-questions}{%
\paragraph{Frequently Asked
Questions}\label{frequently-asked-questions}}

Updated July 27, 2020

\begin{itemize}
\item ~
  \hypertarget{should-i-refinance-my-mortgage}{%
  \paragraph{Should I refinance my
  mortgage?}\label{should-i-refinance-my-mortgage}}

  \begin{itemize}
  \tightlist
  \item
    \href{https://www.nytimes3xbfgragh.onion/article/coronavirus-money-unemployment.html?action=click\&pgtype=Article\&state=default\&region=MAIN_CONTENT_3\&context=storylines_faq}{It
    could be a good idea,} because mortgage rates have
    \href{https://www.nytimes3xbfgragh.onion/2020/07/16/business/mortgage-rates-below-3-percent.html?action=click\&pgtype=Article\&state=default\&region=MAIN_CONTENT_3\&context=storylines_faq}{never
    been lower.} Refinancing requests have pushed mortgage applications
    to some of the highest levels since 2008, so be prepared to get in
    line. But defaults are also up, so if you're thinking about buying a
    home, be aware that some lenders have tightened their standards.
  \end{itemize}
\item ~
  \hypertarget{what-is-school-going-to-look-like-in-september}{%
  \paragraph{What is school going to look like in
  September?}\label{what-is-school-going-to-look-like-in-september}}

  \begin{itemize}
  \tightlist
  \item
    It is unlikely that many schools will return to a normal schedule
    this fall, requiring the grind of
    \href{https://www.nytimes3xbfgragh.onion/2020/06/05/us/coronavirus-education-lost-learning.html?action=click\&pgtype=Article\&state=default\&region=MAIN_CONTENT_3\&context=storylines_faq}{online
    learning},
    \href{https://www.nytimes3xbfgragh.onion/2020/05/29/us/coronavirus-child-care-centers.html?action=click\&pgtype=Article\&state=default\&region=MAIN_CONTENT_3\&context=storylines_faq}{makeshift
    child care} and
    \href{https://www.nytimes3xbfgragh.onion/2020/06/03/business/economy/coronavirus-working-women.html?action=click\&pgtype=Article\&state=default\&region=MAIN_CONTENT_3\&context=storylines_faq}{stunted
    workdays} to continue. California's two largest public school
    districts --- Los Angeles and San Diego --- said on July 13, that
    \href{https://www.nytimes3xbfgragh.onion/2020/07/13/us/lausd-san-diego-school-reopening.html?action=click\&pgtype=Article\&state=default\&region=MAIN_CONTENT_3\&context=storylines_faq}{instruction
    will be remote-only in the fall}, citing concerns that surging
    coronavirus infections in their areas pose too dire a risk for
    students and teachers. Together, the two districts enroll some
    825,000 students. They are the largest in the country so far to
    abandon plans for even a partial physical return to classrooms when
    they reopen in August. For other districts, the solution won't be an
    all-or-nothing approach.
    \href{https://bioethics.jhu.edu/research-and-outreach/projects/eschool-initiative/school-policy-tracker/}{Many
    systems}, including the nation's largest, New York City, are
    devising
    \href{https://www.nytimes3xbfgragh.onion/2020/06/26/us/coronavirus-schools-reopen-fall.html?action=click\&pgtype=Article\&state=default\&region=MAIN_CONTENT_3\&context=storylines_faq}{hybrid
    plans} that involve spending some days in classrooms and other days
    online. There's no national policy on this yet, so check with your
    municipal school system regularly to see what is happening in your
    community.
  \end{itemize}
\item ~
  \hypertarget{is-the-coronavirus-airborne}{%
  \paragraph{Is the coronavirus
  airborne?}\label{is-the-coronavirus-airborne}}

  \begin{itemize}
  \tightlist
  \item
    The coronavirus
    \href{https://www.nytimes3xbfgragh.onion/2020/07/04/health/239-experts-with-one-big-claim-the-coronavirus-is-airborne.html?action=click\&pgtype=Article\&state=default\&region=MAIN_CONTENT_3\&context=storylines_faq}{can
    stay aloft for hours in tiny droplets in stagnant air}, infecting
    people as they inhale, mounting scientific evidence suggests. This
    risk is highest in crowded indoor spaces with poor ventilation, and
    may help explain super-spreading events reported in meatpacking
    plants, churches and restaurants.
    \href{https://www.nytimes3xbfgragh.onion/2020/07/06/health/coronavirus-airborne-aerosols.html?action=click\&pgtype=Article\&state=default\&region=MAIN_CONTENT_3\&context=storylines_faq}{It's
    unclear how often the virus is spread} via these tiny droplets, or
    aerosols, compared with larger droplets that are expelled when a
    sick person coughs or sneezes, or transmitted through contact with
    contaminated surfaces, said Linsey Marr, an aerosol expert at
    Virginia Tech. Aerosols are released even when a person without
    symptoms exhales, talks or sings, according to Dr. Marr and more
    than 200 other experts, who
    \href{https://academic.oup.com/cid/article/doi/10.1093/cid/ciaa939/5867798}{have
    outlined the evidence in an open letter to the World Health
    Organization}.
  \end{itemize}
\item ~
  \hypertarget{what-are-the-symptoms-of-coronavirus}{%
  \paragraph{What are the symptoms of
  coronavirus?}\label{what-are-the-symptoms-of-coronavirus}}

  \begin{itemize}
  \tightlist
  \item
    Common symptoms
    \href{https://www.nytimes3xbfgragh.onion/article/symptoms-coronavirus.html?action=click\&pgtype=Article\&state=default\&region=MAIN_CONTENT_3\&context=storylines_faq}{include
    fever, a dry cough, fatigue and difficulty breathing or shortness of
    breath.} Some of these symptoms overlap with those of the flu,
    making detection difficult, but runny noses and stuffy sinuses are
    less common.
    \href{https://www.nytimes3xbfgragh.onion/2020/04/27/health/coronavirus-symptoms-cdc.html?action=click\&pgtype=Article\&state=default\&region=MAIN_CONTENT_3\&context=storylines_faq}{The
    C.D.C. has also} added chills, muscle pain, sore throat, headache
    and a new loss of the sense of taste or smell as symptoms to look
    out for. Most people fall ill five to seven days after exposure, but
    symptoms may appear in as few as two days or as many as 14 days.
  \end{itemize}
\item ~
  \hypertarget{does-asymptomatic-transmission-of-covid-19-happen}{%
  \paragraph{Does asymptomatic transmission of Covid-19
  happen?}\label{does-asymptomatic-transmission-of-covid-19-happen}}

  \begin{itemize}
  \tightlist
  \item
    So far, the evidence seems to show it does. A widely cited
    \href{https://www.nature.com/articles/s41591-020-0869-5}{paper}
    published in April suggests that people are most infectious about
    two days before the onset of coronavirus symptoms and estimated that
    44 percent of new infections were a result of transmission from
    people who were not yet showing symptoms. Recently, a top expert at
    the World Health Organization stated that transmission of the
    coronavirus by people who did not have symptoms was ``very rare,''
    \href{https://www.nytimes3xbfgragh.onion/2020/06/09/world/coronavirus-updates.html?action=click\&pgtype=Article\&state=default\&region=MAIN_CONTENT_3\&context=storylines_faq\#link-1f302e21}{but
    she later walked back that statement.}
  \end{itemize}
\end{itemize}

But even if this vaccine fails to win the race, the Serum Institute will
still be instrumental. It has teamed up with other vaccine designers, at
earlier stages of development, to manufacture four other vaccines,
though those are not being mass produced yet.

And if all of those fail, Mr. Poonawalla says he can quickly adapt his
assembly lines to manufacture whatever vaccine candidate does work,
wherever it comes from.

``Very few people can produce it at this cost, this scale and this
speed,'' he said.

Under the AstraZeneca deal, Serum can make 1 billion doses of the Oxford
vaccine for India and lower- and middle-income countries during the
pandemic and charge an amount that is no more than its production costs.

After the pandemic passes, Mr. Poonawalla expects that he will be able
to sell the vaccine at a profit --- if it works --- but his biggest
concern is the near term and covering his cash flow. He estimates that
he is spending around \$450 million to mass-produce the Oxford vaccine.

Image

The final stage of the production process: packaging sealed vaccines in
cardboard boxes.Credit...Atul Loke for The New York Times

Many of his expenses might never be recouped, like the costs for the
vials holding the vaccine and the chemicals used in the process. For the
first time, the Poonawallas say they are considering turning to
sovereign wealth or private equity funds for help.

Contrast that with the deals made under
\href{https://www.nytimes3xbfgragh.onion/2020/06/03/us/politics/coronavirus-vaccine-trump-moderna.html}{President
Trump's Warp Speed project}, and the similar ones in Europe. In the
scramble to secure hundreds of millions of doses for their people,
richer countries have already paid or committed to pay drug companies
handsomely to offset the risks of mass-producing a vaccine candidate
that might not work and end up being thrown out.

What this spells is ``vaccine nationalism,'' said
\href{http://www.lse.ac.uk/health-policy/people/dr-olivier-wouters}{Dr.
Olivier Wouters}, a health policy professor at the London School of
Economics.

``Rich countries are getting to the front of the queue and poorer
countries are at risk of getting left behind.''

Analysts said it was likely that Serum would eventually get some
financial help from the Bill \& Melinda Gates Foundation, which supports
global immunization programs, or maybe the Indian government. Both
declined to comment.

But any deal will probably be far smaller than what the big
pharmaceutical companies have landed. Another difference is that those
companies are vaccine developers and producers. Serum's role, at least
for the Oxford vaccine, is purely production.

Image

Technicians from a German company installing a new high-speed vaccine
assembly line on Serum's campus.Credit...Atul Loke for The New York
Times

Either way, Mr. Poonawalla said he felt an obligation to take this risk.

``We just felt that this was our sort of moment,'' he said.

Since Adar Poonawalla took over as Serum's chief executive from his
father in 2011, the company has expanded into new markets, pushing
revenues to more than \$800 million.

A few years ago,
\href{https://www.cnbc.com/2015/09/14/cyrus-poonawalla-buys-mumbais-lincoln-house-setting-india-record.html}{the
Poonawallas decided to buy the former American consulate building in
Mumbai}, which used to be a maharajah palace, for \$113 million --- for
a weekend retreat. They have
\href{https://www.youtube.com/watch?v=Da0-uMocTMc}{more Rolls-Royces and
Ferraris} than you can shake a stick at, and a Batmobile.

Adar Poonawalla acknowledged that his family was better known for
``being seen in some fancy car or a jet or whatever,'' than making
lifesaving vaccines.

``A lot of people didn't even know in India what the hell I did,'' he
said. ``They thought, `Oh, you do something with horses or something,
you must be making money.'''

Mr. Poonawalla senses this is about to change.

He is confident that the Oxford vaccine his gleaming stainless-steel
machines are churning out has the best shot of working. If it does, he
plans to roll up his sleeve and brace for an injection.

``It would be ridiculous,'' he said, ``if I spent all this money,
committed to everything, and I didn't take it myself.''

Image

Construction for a new facility at the Serum Institute. The company is
investing in a major expansion, to produce even more
vaccines.Credit...Atul Loke for The New York Times

Kai Schultz contributed reporting from New Delhi.

Advertisement

\protect\hyperlink{after-bottom}{Continue reading the main story}

\hypertarget{site-index}{%
\subsection{Site Index}\label{site-index}}

\hypertarget{site-information-navigation}{%
\subsection{Site Information
Navigation}\label{site-information-navigation}}

\begin{itemize}
\tightlist
\item
  \href{https://help.nytimes3xbfgragh.onion/hc/en-us/articles/115014792127-Copyright-notice}{©~2020~The
  New York Times Company}
\end{itemize}

\begin{itemize}
\tightlist
\item
  \href{https://www.nytco.com/}{NYTCo}
\item
  \href{https://help.nytimes3xbfgragh.onion/hc/en-us/articles/115015385887-Contact-Us}{Contact
  Us}
\item
  \href{https://www.nytco.com/careers/}{Work with us}
\item
  \href{https://nytmediakit.com/}{Advertise}
\item
  \href{http://www.tbrandstudio.com/}{T Brand Studio}
\item
  \href{https://www.nytimes3xbfgragh.onion/privacy/cookie-policy\#how-do-i-manage-trackers}{Your
  Ad Choices}
\item
  \href{https://www.nytimes3xbfgragh.onion/privacy}{Privacy}
\item
  \href{https://help.nytimes3xbfgragh.onion/hc/en-us/articles/115014893428-Terms-of-service}{Terms
  of Service}
\item
  \href{https://help.nytimes3xbfgragh.onion/hc/en-us/articles/115014893968-Terms-of-sale}{Terms
  of Sale}
\item
  \href{https://spiderbites.nytimes3xbfgragh.onion}{Site Map}
\item
  \href{https://help.nytimes3xbfgragh.onion/hc/en-us}{Help}
\item
  \href{https://www.nytimes3xbfgragh.onion/subscription?campaignId=37WXW}{Subscriptions}
\end{itemize}
