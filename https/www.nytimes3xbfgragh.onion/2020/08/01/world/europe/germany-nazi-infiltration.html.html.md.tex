Sections

SEARCH

\protect\hyperlink{site-content}{Skip to
content}\protect\hyperlink{site-index}{Skip to site index}

\href{https://www.nytimes3xbfgragh.onion/section/world/europe}{Europe}

\href{https://myaccount.nytimes3xbfgragh.onion/auth/login?response_type=cookie\&client_id=vi}{}

\href{https://www.nytimes3xbfgragh.onion/section/todayspaper}{Today's
Paper}

\href{/section/world/europe}{Europe}\textbar{}Body Bags and Enemy Lists:
How Far-Right Police Officers and Ex-Soldiers Planned for `Day X'

\url{https://nyti.ms/39JJzkO}

\begin{itemize}
\item
\item
\item
\item
\item
\end{itemize}

Advertisement

\protect\hyperlink{after-top}{Continue reading the main story}

Supported by

\protect\hyperlink{after-sponsor}{Continue reading the main story}

\hypertarget{body-bags-and-enemy-lists-how-far-right-police-officers-and-ex-soldiers-planned-for-day-x}{%
\section{Body Bags and Enemy Lists: How Far-Right Police Officers and
Ex-Soldiers Planned for `Day
X'}\label{body-bags-and-enemy-lists-how-far-right-police-officers-and-ex-soldiers-planned-for-day-x}}

Germany has woken up to a problem of far-right extremism in its elite
special forces. But the threat of neo-Nazi infiltration of state
institutions is much broader.

\includegraphics{https://static01.graylady3jvrrxbe.onion/images/2020/07/15/world/00germany-nordkreuz1/merlin_169848747_56fc1d28-734b-4450-830e-9a7ce957c287-articleLarge.jpg?quality=75\&auto=webp\&disable=upscale}

\href{https://www.nytimes3xbfgragh.onion/by/katrin-bennhold}{\includegraphics{https://static01.graylady3jvrrxbe.onion/images/2018/07/13/multimedia/author-katrin-bennhold/author-katrin-bennhold-thumbLarge.png}}

By \href{https://www.nytimes3xbfgragh.onion/by/katrin-bennhold}{Katrin
Bennhold}

\begin{itemize}
\item
  Aug. 1, 2020Updated 9:44 a.m. ET
\item
  \begin{itemize}
  \item
  \item
  \item
  \item
  \item
  \end{itemize}
\end{itemize}

GÜSTROW, Germany --- The plan sounded frighteningly concrete. The group
would round up political enemies and those defending migrants and
refugees, put them on trucks and drive them to a secret location.

Then they would kill them.

One member had already bought 30 body bags. More body bags were on an
order list, investigators say, along with quicklime, used to decompose
organic material.

On the surface, those discussing the plan seemed reputable. One was a
lawyer and local politician, but with a special hatred of immigrants.
Two were active army reservists. Two others were police officers,
including Marko Gross, a police sniper and former parachutist who acted
as their unofficial leader.

The group grew out of
a\href{https://www.nytimes3xbfgragh.onion/2020/07/03/world/europe/germany-military-neo-nazis-ksk.html?searchResultPosition=2}{nationwide
chat network for soldiers} and others with far-right sympathies set up
by a member of Germany's elite special forces, the KSK. Over time, under
Mr. Gross's supervision, they formed a parallel group of their own.
Members included a doctor, an engineer, a decorator, a gym owner, even a
local fisherman.

They called themselves Nordkreuz, or Northern Cross.

``Between us, we were a whole village,'' recalled Mr. Gross, one of
several Nordkreuz members who described to me in various interviews this
year how the group came together and began making plans.

They denied they had plotted to kill anyone. But investigators and
prosecutors, as well an account one member gave to the police ---
transcripts of which were seen by The New York Times --- indicate their
planning took a more sinister turn.

Germany has belatedly begun dealing with far-right networks that
officials now say are far more extensive than they ever understood. The
reach of far-right extremists into its armed forces is particularly
alarming in a country that has worked to cleanse itself of its Nazi past
and the horrors of the Holocaust. In July the government
\href{https://www.nytimes3xbfgragh.onion/2020/07/01/world/europe/german-special-forces-far-right.html}{disbanded
an entire company} infiltrated by extremists in the nation's special
forces.

But the Nordkreuz case, which only recently came to trial after being
uncovered more than three years ago, shows that the problem of far-right
infiltration is neither new nor confined to to the KSK, or even the
military.

Far-right extremism penetrated multiple layers of German society in the
years when the authorities underestimated the threat or were reluctant
to countenance it fully, officials and lawmakers acknowledge. Now they
are struggling to uproot it.

One central motivation of the extremists has seemed so far-fetched and
fantastical that for a long time the authorities and investigators did
not take it seriously, even as it gained broader currency in far-right
circles.

Neo-Nazi groups and other extremists call it Day X --- a mythical moment
when Germany's social order collapses, requiring committed far-right
extremists, in their telling, to save themselves and rescue the nation.

Today Day X preppers are drawing serious people with serious skills and
ambition. Increasingly, the German authorities consider the scenario a
pretext for domestic terrorism by far-right plotters or even for a
takeover of the government.

``I fear we've only seen the tip of the iceberg,'' said Dirk
Friedriszik, a lawmaker in the~northeastern state of Mecklenburg-Western
Pomerania, where Nordkreuz was founded. ``It isn't just the KSK. The
real worry is: These cells are everywhere. In the army, in the police,
in reservist units.''

Nordkreuz was one of those groups elaborately preparing for Day X. The
domestic intelligence service got a tip in late 2016, and prosecutors
started investigating in the summer of 2017. But it took years before
the network, or a small sliver of it, came before a court.

Even now, only one member of the group, Mr. Gross, has faced charges ---
for illegal weapons possession, not for any larger conspiracy.

\includegraphics{https://static01.graylady3jvrrxbe.onion/images/2020/07/20/world/00germany-nordkreuz2/merlin_173112810_62fa023a-97e4-4804-881c-d802ab71cfb9-articleLarge.jpg?quality=75\&auto=webp\&disable=upscale}

Late last year, Mr. Gross was handed a 21-month suspended sentence. The
verdict was so mild that this year state prosecutors appealed it,
kicking the case into another protracted round of deliberations.

Of some 30 Nordkreuz members, only two others, a lawyer and another
police officer, are currently under investigation by the federal
prosecutor on suspicion of plotting terrorism.

The outcome is typical of the authorities' handling of far-right cases,
extremism experts say. The charges brought are often woefully narrow for
the elaborate plots they are meant to deter and punish. Almost always
they focus on individuals, not the networks themselves.

But the obstacles to prosecuting such cases more aggressively point to
another problem making the German authorities increasingly anxious:
Infiltration of the very institutions, like the police, that are
supposed to be doing the investigating.

In July the police chief of the western state of Hesse resigned after
police computers had been repeatedly accessed for confidential
information that was then used by neo-Nazis in death threats. It was in
Hesse that a well-known neo-Nazi assassinated a
\href{https://www.nytimes3xbfgragh.onion/2019/06/26/world/europe/germany-walter-lubcke-neo-nazi.html?searchResultPosition=8}{regional
politician} last summer in a case that woke many Germans to the threat
of far-right terrorism.

Some Nordkreuz members were serious enough that they had compiled a list
of political enemies. Heiko Böhringer, a local politician in the area
where the group was based, had received death threats.

``I used to think these preppers, they're harmless crazies who've
watched too many horror movies,'' Mr. Böhringer said. ``I changed my
mind.''

Mr. Friedriszik, the state lawmaker, tried for years to focus public
attention on the building danger of the far right, but found himself a
voice in the wilderness.

``This movement has its fingertips in lots of places,'' he said. ``All
this talk of Day X can seem like pure fantasy. But if you look closer,
you can see how quickly it turns into serious planning --- and
plotting.''

\hypertarget{northern-cross}{%
\subsection{Northern Cross}\label{northern-cross}}

The shooting range in~Güstrow, a rural town in a northeast corner of
Germany, sits at the end of a long dirt path secured by a heavy gate.
Barbed wire surrounds the area. A German flag flutters in the wind.

``This is where it all started,'' Axel Moll, a local decorator and
Nordkreuz member with a hunting license and gun cabinet at home, told me
when I was touring the area earlier this year.

Mr. Gross, the police officer, was a regular at the range. He had been a
parachutist and long-distance reconnaissance officer in the German army
before his battalion was absorbed by Germany's elite special forces, the
KSK. He never joined the KSK but knows several men who did.

Another regular was Frank Thiel, a champion in handgun competitions and
sought-after tactical shooting instructor for police and military units
across Germany.

In the fall of 2015, as hundreds of thousand of asylum seekers from wars
in Syria, Iraq and Afghanistan arrived in Germany, the men were
appalled. In their eyes, Germany faced a potential invasion from
terrorists, a possible breakdown of its welfare system, maybe even
unrest.

And their own government was welcoming the migrants.

Image

The shooting range in Güstrow, where the members of Nordkreuz met.
Credit...Gordon Welters for The New York Times

``We were worried,'' Mr. Gross, 49, recalled in one of several
conversations with me this year.

In late 2015, while conducting a shooting workshop for
\href{https://www.nytimes3xbfgragh.onion/2020/07/03/world/europe/germany-military-neo-nazis-ksk.html?searchResultPosition=2}{the
KSK} in southern Germany, Mr. Thiel learned about an encrypted,
countrywide chat network to share privileged information about the
security situation in Germany, and how to prepare for a crisis.

It was run by a soldier named André Schmitt. But everyone knew him as
\href{https://www.nytimes3xbfgragh.onion/2020/07/03/world/europe/germany-military-neo-nazis-ksk.html?searchResultPosition=1}{Hannibal}.

Who wanted in?

Soon some~30~people, many of them regulars at the shooting range in
Güstrow, joined the northern chapter of Mr. Schmitt's network, avidly
following his updates. It was not long before Mr. Gross decided to
create a~parallel group so they could communicate and meet up
locally.~Members lived in towns and villages in the region, shared
far-right sympathies and considered themselves concerned citizens.

By January 2016, this network had become Nordkreuz.

There were two criteria for joining, Mr. Moll recalled: ``The right
skills and the right attitude.''

Mr. Gross and another police officer in the group were members of what
was then an emerging far-right party, the Alternative for Germany, now
the third largest force in the national Parliament. At least two others
in the group had visited the Thule Seminar, an organization whose
leaders had a portrait of Hitler on their wall and preach white
supremacy.

Nordkreuz held meetings every few weeks, on the floor above a gym owned
by one member or in Mr. Moll's showroom, where the two of us also
talked. Sometimes they had a barbecue. Other times, they invited guest
speakers.

Once a retired military officer came and talked about crisis management,
Mr. Moll recalled. Another time they invited a ``Reichsbürger,'' or
citizen of the Reich, a movement that does not recognize the postwar
German state.

Over time, Nordkreuz members recalled, their group morphed into a
close-knit brotherhood with a shared ambition that would come to
dominate their lives: preparing for Day X.

They began hoarding enough supplies to survive for 100 days, including
food, gasoline, toiletries, walkie-talkies, medicine and ammunition. Mr.
Gross collected 600 euros from each member of the group to pay for it.
In all, he amassed more than 55,000 rounds of ammunition.

The group identified a ``safe house,'' where members would decamp with
their families on Day X: a former Communist vacation village deep in the
woods.

Image

Axel Moll, a local decorator and Nordkreuz member, in the showroom where
an order list for Day X was compiled that included body bags and quick
lime.Credit...Gordon Welters for The New York Times

The place was ``ideal,'' Mr. Moll said. There was a stream providing
fresh water, a small lake to wash themselves and clothes, a forest with
wood to build and deer to hunt, even an old septic tank.

Didn't all this seem a little far-fetched to them? I asked.

Mr. Moll smiled at my ``Western naïveté.''

The region where they live is nestled between the former Iron Curtain
and the Polish border. Members had grown up in the former East Germany.

``Under Communism, everything was scarce,'' Mr. Moll explained. ``You
had to get creative getting things through certain channels. You could
not rely on things being in the supermarket. You could say we're used to
prepping.''

And, he said, they had already seen one system collapse. ``You learn how
to read between the lines. It's an advantage.''

Through 2016, as hundreds of thousands more migrants arrived in Germany
and a number of Islamist terrorist attacks took place in Europe, the
planning got more serious.

Mr. Gross and other Nordkreuz members traveled in the fall to an arms
fair in Nuremberg **** and **** met Mr. Schmitt, the special forces
soldier running the nationwide chat network, in person.

Members of the group learned how to rappel down the tower of a disused
fire station. Two pickup points were designated as Day X meeting spots.
Two fully functioning operating theaters were built as makeshift field
hospitals, in a basement and a mobile home.

``The scenario was that something bad would happen,'' Mr. Gross told me.
``We asked ourselves, what did we want to prepare for? And we decided
that if we were going to do this, we would go all the way.''

\hypertarget{body-bags-and-quicklime}{%
\subsection{Body Bags and Quicklime}\label{body-bags-and-quicklime}}

The question investigators are now scrutinizing is what did it mean to
``go all the way.''

Mr. Gross insisted to me that the group was only prepping for what they
saw as the day that the social order would collapse, for Day X. He said
they never planned any murders, or intended to cause any harm.

But at least one member of the group portrays a more ominous story.

``People were to be gathered and murdered,'' Horst Schelski told
investigators in 2017, according to transcripts of his statement shared
with The New York Times.

Mr. Schelski is a former air force officer whose account is disputed by
the others. It pivots on a meeting he said took place at the end of 2016
at a highway truck stop in Sternberg, a small town about 40 minutes west
of the shooting range the men frequented.

There, at a coffee stand that today resembles little more than a shed
facing a bleak parking lot, Mr. Gross met with a handful of other men,
in what had become a concentrated cell within Nordkreuz.

Among the others present were two men now under investigation on
suspicion of plotting terrorism. Under German law, they cannot be fully
named. One was Haik J., who like Mr. Gross was a police officer. Another
was a lawyer and local politician, Jan Henrik H. Both declined to speak
with me.

Jan Henrik H. was described by other members as particularly fervent and
hateful. On his birthdays, he held a shooting contest on a field behind
his house in Rostock, a nearby city on Germany's northern coast,
Nordkreuz members recalled.

The winner got a trophy named for Mehmet Turgut, a Turkish street vendor
killed in Rostock in 2004 by the National Socialist Underground, a
far-right terrorist group.

Mr. Gross was the most recent winner.

Image

A memorial for Mehmet Turgut, a Turkish street vendor killed in Rostock
in 2004 by far-right terrorists. A Nordkreuz member was furious when the
memorial was unveiled.Credit...Gordon Welters for The New York Times

Mr. Schelski told the police that Jan Henrik H. kept a thick binder in
his garage with the names, addresses and photos of local politicians and
activists whom he considered to be political enemies. Some had sought to
help refugees by seeking real estate to turn into shelters.

Much in the file came from publicly available sources. But there were
also handwritten notes with information obtained from a police computer.

As they drank coffee at the truck stop, Jan Henrik H. turned the
conversation to ``the people in the file,'' who he said were ``harmful''
to the state and needed to be ``done away with,'' Mr. Schelski later
told the police.

Jan Henrik H. wanted advice on how best to transport their captives once
they had been rounded up. He asked Mr. Schelski, a major in the state
reservist unit, how they could get them past any checkpoints that might
be created in a time of unrest. Would uniforms help? Army trucks?

After that meeting, Mr. Schelski told the police, he distanced himself
from the group.

By then, the intelligence service was already watching. Some eight
months after the truck stop meeting, the authorities conducted the first
in a series of raids on the homes of several Nordkreuz members.

Over two years, the raids and intelligence work uncovered weapons,
ammunition, enemy lists, and a handwritten order list for Day X that
included the body bags and quick lime.

I asked Mr. Gross about the body bags. He told me they were
``multipurpose vessels,'' usable as cheap waterproof sleeping bag covers
or for transporting large items.

The disclosure that the group had identified political enemies has
rattled Mr. Böhringer, the local politician. In 2015, two police
officers came to sketch his house after he started receiving death
threats.

``We want to know where you can get in, where you sleep, so that we can
protect you,'' they told him.

Image

``I used to think these preppers, they're harmless crazies who've
watched too many horror movies,'' said Heiko Böhringer, a local
politician. ``I changed my mind.''Credit...Gordon Welters for The New
York Times

He said he wasn't too concerned. But in June 2018, Mr. Böhringer was
called to the police station. The homes of two Nordkreuz members had
recently been raided, one of them a police officer based in his
hometown: Haik J., who had been at the truck stop meeting.

``They showed me a handmade sketch of my home,'' Mr. Böhringer said.
```Do you recognize this?' they had asked.''

``It was the exact same sketch that those officers had made in my
home,'' he said.

``I had to swallow pretty hard,'' he recalled. ``The very people who
said they wanted to protect me then passed this on to people who wanted
to harm me.''

``They didn't just want to survive Day X, they wanted to kill their
enemies,'' he said. ``It was concrete, what they were planning.''

\hypertarget{meeting-with-marko}{%
\subsection{Meeting With Marko}\label{meeting-with-marko}}

The first time I knocked on Mr. Gross's door, in the village of Banzkow,
about an hour's drive from the shooting range, we ended up talking
outside for two hours.

The second time, it started raining and he invited me into his red brick
farmhouse on ``Liberation Street,'' named for Germany's liberation from
the Nazis at the end of World War II.

In the hallway his old military badge and uniform were on display. A
large map of Germany in 1937 dominated the wall. Images of guns were
ubiquitous. On refrigerator magnets. On mugs. On a calendar.

It was the same home that the police had raided years earlier, in August
2017, and found more than two dozen weapons and 23,800 rounds of
ammunition, some of it stolen from police and military stockpiles.

Image

Mr. Gross's house, second from left, in~Banzkow.Credit...Gordon Welters
for The New York Times

Another police raid in June 2019 uncovered another 31,500 rounds of
ammunition and an Uzi submachine gun. This time they arrested him.

In court, it took prosecutors almost 45 minutes to read the list of
cartridges, guns, explosives and knives they had found. He was only
charged with illegal weapons possession. In the ongoing terrorism
investigation he is a witness, not a suspect.

``It's pretty astounding,'' said Lorenz Caffier, the state's interior
minister, who used to shake Mr. Gross's hand at the annual special
forces workshop in Güstrow. ``Someone who hoards that much ammunition at
home, is close to far-right tendencies and also makes extremist comments
in chats is no harmless prepper.''

``Marko G. has a key role,'' he said.

Prosecutors have traced the illegal ammunition in Mr. Gross's home to a
dozen police and military depots across the country, indicating possible
collaborators. Several of the units shot in Güstrow.

``We don't know how it got from there to him,'' said Claudia Lange, a
prosecutor.

Three other police officers are being investigated on suspicion of
helping Mr. Gross. Asked during the trial, Mr. Gross said he did not
remember how he got the ammunition. When I met him, he stuck to that
line.

But otherwise he was not shy about sharing his views.

Chancellor Angela Merkel belongs ``in the dock,'' he said. The
multicultural cities in western Germany are ``the caliphate.'' The best
way to escape creeping migration was to move to the East German
countryside, ``where people are still called Schmidt, Schneider and
Müller.''

A copy of Compact, a prominent far-right magazine, with President
Trump's face on the cover, lay on a shelf. A selection of the
president's speeches had been translated into German in the issue. ``I
like Trump,'' Mr. Gross said.

As far back as 2009, some fellow police officers had voiced concerns
about Mr. Gross's far-right views, noting that he had brought books
about the Nazis to work. But no one intervened, and he was even groomed
for promotion.

``There is no danger from the far right,'' he insisted. ``I don't know a
single neo-Nazi.''

Soldiers and police officers are ``frustrated,'' he told me the third
time we met, ticking off complaints about migrants, crime and the
mainstream media. He likens the coverage of coronavirus to the censored
state broadcaster during Communism. Instead, he says, he has a YouTube
subscription to RT, the Russian state-controlled channel and other
alternative media.

In that parallel universe of disinformation, he learns that the
government is secretly flying in refugees after midnight. That
coronavirus is a ploy to deprive citizens of their rights. That Ms.
Merkel works for what he calls the ``deep state.''

``The deep state is global,'' Mr. Gross said. ``It's big capital, the
big banks, Bill Gates.''

He still expects Day X, sooner or later. Riots linked to an economic
meltdown. Or a blackout, because the German government is shuttering
coal plants.

Nordkreuz members never told me, nor the authorities, the location of
the disused vacation village that was their safe house for Day X.

The safe house is still active, said Mr. Gross, who at the height of
Nordkreuz's planning had boasted to a fellow member that his network
contained 2,000 like-minded people in Germany and beyond.

``The network is still there,'' he said.

Christopher F. Schuetze contributed reporting.

Advertisement

\protect\hyperlink{after-bottom}{Continue reading the main story}

\hypertarget{site-index}{%
\subsection{Site Index}\label{site-index}}

\hypertarget{site-information-navigation}{%
\subsection{Site Information
Navigation}\label{site-information-navigation}}

\begin{itemize}
\tightlist
\item
  \href{https://help.nytimes3xbfgragh.onion/hc/en-us/articles/115014792127-Copyright-notice}{©~2020~The
  New York Times Company}
\end{itemize}

\begin{itemize}
\tightlist
\item
  \href{https://www.nytco.com/}{NYTCo}
\item
  \href{https://help.nytimes3xbfgragh.onion/hc/en-us/articles/115015385887-Contact-Us}{Contact
  Us}
\item
  \href{https://www.nytco.com/careers/}{Work with us}
\item
  \href{https://nytmediakit.com/}{Advertise}
\item
  \href{http://www.tbrandstudio.com/}{T Brand Studio}
\item
  \href{https://www.nytimes3xbfgragh.onion/privacy/cookie-policy\#how-do-i-manage-trackers}{Your
  Ad Choices}
\item
  \href{https://www.nytimes3xbfgragh.onion/privacy}{Privacy}
\item
  \href{https://help.nytimes3xbfgragh.onion/hc/en-us/articles/115014893428-Terms-of-service}{Terms
  of Service}
\item
  \href{https://help.nytimes3xbfgragh.onion/hc/en-us/articles/115014893968-Terms-of-sale}{Terms
  of Sale}
\item
  \href{https://spiderbites.nytimes3xbfgragh.onion}{Site Map}
\item
  \href{https://help.nytimes3xbfgragh.onion/hc/en-us}{Help}
\item
  \href{https://www.nytimes3xbfgragh.onion/subscription?campaignId=37WXW}{Subscriptions}
\end{itemize}
