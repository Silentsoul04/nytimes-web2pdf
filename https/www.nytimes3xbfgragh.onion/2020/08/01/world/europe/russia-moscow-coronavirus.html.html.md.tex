Sections

SEARCH

\protect\hyperlink{site-content}{Skip to
content}\protect\hyperlink{site-index}{Skip to site index}

\href{https://www.nytimes3xbfgragh.onion/section/world/europe}{Europe}

\href{https://myaccount.nytimes3xbfgragh.onion/auth/login?response_type=cookie\&client_id=vi}{}

\href{https://www.nytimes3xbfgragh.onion/section/todayspaper}{Today's
Paper}

\href{/section/world/europe}{Europe}\textbar{}`Cocktails and Masks Don't
Really Go Together'

\url{https://nyti.ms/3hUtdZF}

\begin{itemize}
\item
\item
\item
\item
\item
\end{itemize}

\href{https://www.nytimes3xbfgragh.onion/news-event/coronavirus?action=click\&pgtype=Article\&state=default\&region=TOP_BANNER\&context=storylines_menu}{The
Coronavirus Outbreak}

\begin{itemize}
\tightlist
\item
  live\href{https://www.nytimes3xbfgragh.onion/2020/08/04/world/coronavirus-cases.html?action=click\&pgtype=Article\&state=default\&region=TOP_BANNER\&context=storylines_menu}{Latest
  Updates}
\item
  \href{https://www.nytimes3xbfgragh.onion/interactive/2020/us/coronavirus-us-cases.html?action=click\&pgtype=Article\&state=default\&region=TOP_BANNER\&context=storylines_menu}{Maps
  and Cases}
\item
  \href{https://www.nytimes3xbfgragh.onion/interactive/2020/science/coronavirus-vaccine-tracker.html?action=click\&pgtype=Article\&state=default\&region=TOP_BANNER\&context=storylines_menu}{Vaccine
  Tracker}
\item
  \href{https://www.nytimes3xbfgragh.onion/2020/08/02/us/covid-college-reopening.html?action=click\&pgtype=Article\&state=default\&region=TOP_BANNER\&context=storylines_menu}{College
  Reopening}
\item
  \href{https://www.nytimes3xbfgragh.onion/live/2020/08/04/business/stock-market-today-coronavirus?action=click\&pgtype=Article\&state=default\&region=TOP_BANNER\&context=storylines_menu}{Economy}
\end{itemize}

Advertisement

\protect\hyperlink{after-top}{Continue reading the main story}

Supported by

\protect\hyperlink{after-sponsor}{Continue reading the main story}

Moscow Dispatch

\hypertarget{cocktails-and-masks-dont-really-go-together}{%
\section{`Cocktails and Masks Don't Really Go
Together'}\label{cocktails-and-masks-dont-really-go-together}}

Young Russians are partying again, seeking a return to normal life and
willing to risk a coronavirus surge. ``We are people, not robots, and
want to have a life,'' said a bar patron (and doctor).

\includegraphics{https://static01.graylady3jvrrxbe.onion/images/2020/08/02/world/02moscowdispatch2/merlin_174764931_b934ad94-2385-453a-83a6-f8e42d4ad928-articleLarge.jpg?quality=75\&auto=webp\&disable=upscale}

\href{https://www.nytimes3xbfgragh.onion/by/andrew-higgins}{\includegraphics{https://static01.graylady3jvrrxbe.onion/images/2018/10/10/multimedia/author-andrew-higgins/author-andrew-higgins-thumbLarge.png}}

By \href{https://www.nytimes3xbfgragh.onion/by/andrew-higgins}{Andrew
Higgins}

\begin{itemize}
\item
  Aug. 1, 2020
\item
  \begin{itemize}
  \item
  \item
  \item
  \item
  \item
  \end{itemize}
\end{itemize}

MOSCOW --- When Nest, a cramped Moscow cocktail lounge, reopened for
business in late June after more than two months of lockdown, it offered
free masks and antiseptic lotion at the entrance to help calm any fears
drinkers might have about sitting just inches from each other around
tiny round tables.

It needn't have bothered.

``Nobody could care less so we quickly stopped offering,'' said Roman R.
Pometkov, the head bartender and a coronavirus survivor. He became
infected soon after the pandemic first hit the Russian capital with full
force in March and now, after recuperating in isolation at home for 28
days, is back at work.

Like everyone else in his packed bar on a recent evening, Mr. Pometkov
was not wearing a face mask. ``Everyone just wants to get back to a
normal life,'' he said. ``Cocktails and masks don't really go
together.''

Following a path taken by many people in Florida, Texas and other parts
of the United States in early summer, Moscow and most of Russia in
recent weeks have thrown caution to the wind.

Even restrictions that technically remain in force, like mandatory mask-
and glove-wearing in the Moscow subway and on city buses, are mostly
ignored. The authorities are making little effort to enforce them,
though a few random people have been fined.

``That's not my department,'' answered a gruff turnstile guard on Monday
when asked what the rules are.

\includegraphics{https://static01.graylady3jvrrxbe.onion/images/2020/08/02/world/02moscowdispatch1/merlin_174943362_b25d7ff8-0ca4-40d2-b2cf-d29c36bdfd81-articleLarge.jpg?quality=75\&auto=webp\&disable=upscale}

The Moscow city government led the way in sounding the alarm over the
pandemic in Russia and imposed draconian controls in late March, which
were largely observed, at least at the start.

Muscovites were ordered to stay at home except to buy food and medicine
or to walk their dogs within 100 yards of home. All restaurants and bars
were shut down, and masks and gloves were mandatory for anyone venturing
out, a rule the police enforced with vigor.

But on July 24, the city hosted a party in Gorky Park for recently
graduated high school students. More than 10,000 young people attended,
virtually all without masks, for a long night of dancing, hugging and
boisterous close-quarters celebration.

\hypertarget{latest-updates-global-coronavirus-outbreak}{%
\section{\texorpdfstring{\href{https://www.nytimes3xbfgragh.onion/2020/08/04/world/coronavirus-cases.html?action=click\&pgtype=Article\&state=default\&region=MAIN_CONTENT_1\&context=storylines_live_updates}{Latest
Updates: Global Coronavirus
Outbreak}}{Latest Updates: Global Coronavirus Outbreak}}\label{latest-updates-global-coronavirus-outbreak}}

Updated 2020-08-05T07:58:24.076Z

\begin{itemize}
\tightlist
\item
  \href{https://www.nytimes3xbfgragh.onion/2020/08/04/world/coronavirus-cases.html?action=click\&pgtype=Article\&state=default\&region=MAIN_CONTENT_1\&context=storylines_live_updates\#link-762df92}{As
  talks drag on, McConnell signals openness to jobless aid extension,
  and negotiators agree on a deadline.}
\item
  \href{https://www.nytimes3xbfgragh.onion/2020/08/04/world/coronavirus-cases.html?action=click\&pgtype=Article\&state=default\&region=MAIN_CONTENT_1\&context=storylines_live_updates\#link-1228a480}{Novavax
  sees encouraging results from two studies of its experimental
  vaccine.}
\item
  \href{https://www.nytimes3xbfgragh.onion/2020/08/04/world/coronavirus-cases.html?action=click\&pgtype=Article\&state=default\&region=MAIN_CONTENT_1\&context=storylines_live_updates\#link-794484ed}{Mississippians
  must now wear masks in public, governor says.}
\end{itemize}

\href{https://www.nytimes3xbfgragh.onion/2020/08/04/world/coronavirus-cases.html?action=click\&pgtype=Article\&state=default\&region=MAIN_CONTENT_1\&context=storylines_live_updates}{See
more updates}

More live coverage:
\href{https://www.nytimes3xbfgragh.onion/live/2020/08/04/business/stock-market-today-coronavirus?action=click\&pgtype=Article\&state=default\&region=MAIN_CONTENT_1\&context=storylines_live_updates}{Markets}

Schools across Russia have been told to reopen for the new school year
on Sept 1. Traffic on the Moscow subway, which plunged by 85 percent at
the peak of the crisis, has bounced back to near normal levels, with
more than 5.4 million passengers riding trains on Tuesday, a
post-lockdown record.

Unlike
\href{https://www.nytimes3xbfgragh.onion/2020/06/14/us/coronavirus-united-states.html}{the
Sun Belt in the United States}, however, Russia has so far seen no surge
in new cases, at least according to official statistics. The daily
infection rate nationwide has hovered between 5,000 and 6,000 cases ever
since President Vladimir V. Putin last month declared the battle against
the pandemic won.

Mr. Putin pressed ahead with
\href{https://www.nytimes3xbfgragh.onion/2020/06/24/world/europe/russia-coronavirus-parade.html}{a
huge, mask-free military parade} in Red Square on June 24, and seven
days of nationwide voting through July 1 on constitutional amendments
that
\href{https://www.nytimes3xbfgragh.onion/2020/07/01/world/europe/putin-referendum-vote-russia.html}{allow
him to stay in power until 2036}.

Kremlin critics say the figures are being massaged to avoid exposing Mr.
Putin's confidence as premature. And there are some small signs, even in
the official numbers, of possible trouble ahead.

The daily number of new infections in Moscow, after falling steadily to
530 in mid July, has started inching up, rising to 695 on Friday. This
is a big improvement on the more than 6,000 cases reported each day in
the Russian capital at the peak of the outbreak in May --- and a far cry
from the more than 50,000 new cases reported each day this week in the
United States --- but the upward trend in Moscow, if it continues and
accelerates, could quickly undo progress.

The Moscow city government on Friday warned of stiff fines for not
wearing masks on public transport and in shops.

The mayor of Norilsk, an industrial city in the Arctic, resigned
recently after accusing regional officials of underreporting coronavirus
figures. He said the real number of cases was more than twice the
official count.

Image

Customers gathering outside the Bambule bar.Credit...Sergey Ponomarev
for The New York Times

But others are more sanguine, and in a country long accustomed to
calamity, any worries Russians have of an American-style fiasco have
been far outweighed by delight at the end of restrictions.

Recent reports of a rash of infections among celebrities prompted more
jokes than cries of alarm. Eljay, a popular rap musician, described his
own infection with the coronavirus over the weekend with a gleeful
\href{https://www.instagram.com/sayonaraboy/}{message to his more than
four million followers on Instagram}: ``I have a corona so I am the
king. And if I am king we have a future.''

While not politicized as it has been in the United States, mask-wearing
is seen as seriously uncool among young people and by many older men,
whose often reckless disregard for health warnings, public health
experts agree, is reflected in their average life expectancy of 67
years. Some hip restaurants popular with youth have even started banning
masks.

When Kristina Orbakaite, a Russian pop star, posted a photograph of
herself wearing a designer mask on Instagram, she incited a storm of
protest from fans who accused her, variously, of spreading panic, empty
virtue signaling and pandering to the ``herd instinct of Russian show
business.''

The criticism became so vicious that Ms. Orbakaite, whose mother, Alla
Pugacheva, is an elderly but still hugely popular Russian singer,
\href{https://www.instagram.com/tv/CCHJPXkpV72/?utm_source=ig_embed}{posted
an audio message} to her ``dear subscribers and haters.'' In it, she
expressed dismay that her fashion statement in favor of good hygiene had
prompted ``such a violent reaction'' and explained that wearing a mask
might not save the wearer but does protect others from infection.

``Frankly, I am discouraged,'' she said.

Image

The Noor bar on a Saturday night.Credit...Sergey Ponomarev for The New
York Times

Russia's
\href{https://www.nytimes3xbfgragh.onion/2017/08/05/world/europe/vladimir-putin-russia-summer-vacation.html}{macho
leader}, Mr. Putin, while avoiding health hazards like guzzling vodka
that have traditionally been seen as marks of manliness in Russia, has
outdone even President Trump in shunning the face mask. The only time he
has appeared in public with his face covered was in March when he
visited a Moscow coronavirus clinic wearing a respirator and a hazmat
suit.

\href{https://www.nytimes3xbfgragh.onion/news-event/coronavirus?action=click\&pgtype=Article\&state=default\&region=MAIN_CONTENT_3\&context=storylines_faq}{}

\hypertarget{the-coronavirus-outbreak-}{%
\subsubsection{The Coronavirus Outbreak
›}\label{the-coronavirus-outbreak-}}

\hypertarget{frequently-asked-questions}{%
\paragraph{Frequently Asked
Questions}\label{frequently-asked-questions}}

Updated August 4, 2020

\begin{itemize}
\item ~
  \hypertarget{i-have-antibodies-am-i-now-immune}{%
  \paragraph{I have antibodies. Am I now
  immune?}\label{i-have-antibodies-am-i-now-immune}}

  \begin{itemize}
  \tightlist
  \item
    As of right
    now,\href{https://www.nytimes3xbfgragh.onion/2020/07/22/health/covid-antibodies-herd-immunity.html?action=click\&pgtype=Article\&state=default\&region=MAIN_CONTENT_3\&context=storylines_faq}{that
    seems likely, for at least several months.} There have been
    frightening accounts of people suffering what seems to be a second
    bout of Covid-19. But experts say these patients may have a
    drawn-out course of infection, with the virus taking a slow toll
    weeks to months after initial exposure. People infected with the
    coronavirus typically
    \href{https://www.nature.com/articles/s41586-020-2456-9}{produce}
    immune molecules called antibodies, which are
    \href{https://www.nytimes3xbfgragh.onion/2020/05/07/health/coronavirus-antibody-prevalence.html?action=click\&pgtype=Article\&state=default\&region=MAIN_CONTENT_3\&context=storylines_faq}{protective
    proteins made in response to an
    infection}\href{https://www.nytimes3xbfgragh.onion/2020/05/07/health/coronavirus-antibody-prevalence.html?action=click\&pgtype=Article\&state=default\&region=MAIN_CONTENT_3\&context=storylines_faq}{.
    These antibodies may} last in the body
    \href{https://www.nature.com/articles/s41591-020-0965-6}{only two to
    three months}, which may seem worrisome, but that's perfectly normal
    after an acute infection subsides, said Dr. Michael Mina, an
    immunologist at Harvard University. It may be possible to get the
    coronavirus again, but it's highly unlikely that it would be
    possible in a short window of time from initial infection or make
    people sicker the second time.
  \end{itemize}
\item ~
  \hypertarget{im-a-small-business-owner-can-i-get-relief}{%
  \paragraph{I'm a small-business owner. Can I get
  relief?}\label{im-a-small-business-owner-can-i-get-relief}}

  \begin{itemize}
  \tightlist
  \item
    The
    \href{https://www.nytimes3xbfgragh.onion/article/small-business-loans-stimulus-grants-freelancers-coronavirus.html?action=click\&pgtype=Article\&state=default\&region=MAIN_CONTENT_3\&context=storylines_faq}{stimulus
    bills enacted in March} offer help for the millions of American
    small businesses. Those eligible for aid are businesses and
    nonprofit organizations with fewer than 500 workers, including sole
    proprietorships, independent contractors and freelancers. Some
    larger companies in some industries are also eligible. The help
    being offered, which is being managed by the Small Business
    Administration, includes the Paycheck Protection Program and the
    Economic Injury Disaster Loan program. But lots of folks have
    \href{https://www.nytimes3xbfgragh.onion/interactive/2020/05/07/business/small-business-loans-coronavirus.html?action=click\&pgtype=Article\&state=default\&region=MAIN_CONTENT_3\&context=storylines_faq}{not
    yet seen payouts.} Even those who have received help are confused:
    The rules are draconian, and some are stuck sitting on
    \href{https://www.nytimes3xbfgragh.onion/2020/05/02/business/economy/loans-coronavirus-small-business.html?action=click\&pgtype=Article\&state=default\&region=MAIN_CONTENT_3\&context=storylines_faq}{money
    they don't know how to use.} Many small-business owners are getting
    less than they expected or
    \href{https://www.nytimes3xbfgragh.onion/2020/06/10/business/Small-business-loans-ppp.html?action=click\&pgtype=Article\&state=default\&region=MAIN_CONTENT_3\&context=storylines_faq}{not
    hearing anything at all.}
  \end{itemize}
\item ~
  \hypertarget{what-are-my-rights-if-i-am-worried-about-going-back-to-work}{%
  \paragraph{What are my rights if I am worried about going back to
  work?}\label{what-are-my-rights-if-i-am-worried-about-going-back-to-work}}

  \begin{itemize}
  \tightlist
  \item
    Employers have to provide
    \href{https://www.osha.gov/SLTC/covid-19/standards.html}{a safe
    workplace} with policies that protect everyone equally.
    \href{https://www.nytimes3xbfgragh.onion/article/coronavirus-money-unemployment.html?action=click\&pgtype=Article\&state=default\&region=MAIN_CONTENT_3\&context=storylines_faq}{And
    if one of your co-workers tests positive for the coronavirus, the
    C.D.C.} has said that
    \href{https://www.cdc.gov/coronavirus/2019-ncov/community/guidance-business-response.html}{employers
    should tell their employees} -\/- without giving you the sick
    employee's name -\/- that they may have been exposed to the virus.
  \end{itemize}
\item ~
  \hypertarget{should-i-refinance-my-mortgage}{%
  \paragraph{Should I refinance my
  mortgage?}\label{should-i-refinance-my-mortgage}}

  \begin{itemize}
  \tightlist
  \item
    \href{https://www.nytimes3xbfgragh.onion/article/coronavirus-money-unemployment.html?action=click\&pgtype=Article\&state=default\&region=MAIN_CONTENT_3\&context=storylines_faq}{It
    could be a good idea,} because mortgage rates have
    \href{https://www.nytimes3xbfgragh.onion/2020/07/16/business/mortgage-rates-below-3-percent.html?action=click\&pgtype=Article\&state=default\&region=MAIN_CONTENT_3\&context=storylines_faq}{never
    been lower.} Refinancing requests have pushed mortgage applications
    to some of the highest levels since 2008, so be prepared to get in
    line. But defaults are also up, so if you're thinking about buying a
    home, be aware that some lenders have tightened their standards.
  \end{itemize}
\item ~
  \hypertarget{what-is-school-going-to-look-like-in-september}{%
  \paragraph{What is school going to look like in
  September?}\label{what-is-school-going-to-look-like-in-september}}

  \begin{itemize}
  \tightlist
  \item
    It is unlikely that many schools will return to a normal schedule
    this fall, requiring the grind of
    \href{https://www.nytimes3xbfgragh.onion/2020/06/05/us/coronavirus-education-lost-learning.html?action=click\&pgtype=Article\&state=default\&region=MAIN_CONTENT_3\&context=storylines_faq}{online
    learning},
    \href{https://www.nytimes3xbfgragh.onion/2020/05/29/us/coronavirus-child-care-centers.html?action=click\&pgtype=Article\&state=default\&region=MAIN_CONTENT_3\&context=storylines_faq}{makeshift
    child care} and
    \href{https://www.nytimes3xbfgragh.onion/2020/06/03/business/economy/coronavirus-working-women.html?action=click\&pgtype=Article\&state=default\&region=MAIN_CONTENT_3\&context=storylines_faq}{stunted
    workdays} to continue. California's two largest public school
    districts --- Los Angeles and San Diego --- said on July 13, that
    \href{https://www.nytimes3xbfgragh.onion/2020/07/13/us/lausd-san-diego-school-reopening.html?action=click\&pgtype=Article\&state=default\&region=MAIN_CONTENT_3\&context=storylines_faq}{instruction
    will be remote-only in the fall}, citing concerns that surging
    coronavirus infections in their areas pose too dire a risk for
    students and teachers. Together, the two districts enroll some
    825,000 students. They are the largest in the country so far to
    abandon plans for even a partial physical return to classrooms when
    they reopen in August. For other districts, the solution won't be an
    all-or-nothing approach.
    \href{https://bioethics.jhu.edu/research-and-outreach/projects/eschool-initiative/school-policy-tracker/}{Many
    systems}, including the nation's largest, New York City, are
    devising
    \href{https://www.nytimes3xbfgragh.onion/2020/06/26/us/coronavirus-schools-reopen-fall.html?action=click\&pgtype=Article\&state=default\&region=MAIN_CONTENT_3\&context=storylines_faq}{hybrid
    plans} that involve spending some days in classrooms and other days
    online. There's no national policy on this yet, so check with your
    municipal school system regularly to see what is happening in your
    community.
  \end{itemize}
\end{itemize}

Polina Fedotova, a 27-year-old customer at the Nest cocktail bar, said
she has many friends in the United States, so is well aware of what she
called the ``hellish'' situation there. While not entirely confident
Russia won't end up in the same place, she has decided that the benefits
of having a normal life far outweigh any potential risks.

``It is better to get out and live normally and perhaps even get sick
than to stay at home forever doing nothing,'' she said.

Ms. Fedotova's companion for an evening of cocktails was a 28-year-old
doctor who works at a large Moscow hospital and who contracted the
virus, but barely had any symptoms and has now recovered.

``It was not so bad,'' said the doctor, who declined to give her last
name. ``We are people, not robots, and want to have a life.''

Image

At the Black Swan bar.Credit...Sergey Ponomarev for The New York Times

Younger Russians, often highly skeptical of all official claims, have
been the most eager to embrace the idea that the danger has passed or at
least that, whatever the risks, they are worth taking.

Yuri Kravchenko, the manager of Pod Mukhoi, a popular basement bar in
central Moscow, said ``people are frightened deep down but the desire
for a normal life is just too strong.''

He wears a mask himself and has had all his serving staff tested for the
virus, one of whom found out that he had had the virus without knowing
it.

None of his customers on a recent weekend, packed on stools at the
crowded wooden bar and huddled shoulder-to-shoulder around tables in the
dining area, had a mask.

Among them was a group of students from the Plekhanov Russian University
of Economics. They were celebrating the end of their course work, done
mostly online for the past three months, and the chance to meet
face-to-face for the first time since early spring.

Elizaveta Kolesnik, 21, said she had been in France when the pandemic
arrived in Europe and been so frightened that she rushed back to Moscow
just before Russia closed its borders. After more than two months cooped
up at home in Moscow, however, she has put aside her earlier fear,
deciding that ``fate will decide what happens now'' and ``if you are
afraid you only lose what life has to give.''

Francesco Spatola, an Italian-Russian classmate, said he has no faith at
all in official statistics but is happy to go along with the
authorities' rosy story line. He also sees no point in trying to resist
the forces of nature.

``It is sad to say,'' he said, ``but it is natural that people die.''

Image

Younger Russians have been the most eager to embrace the idea that the
danger has passed, or at least the idea that the risks are worth
it.Credit...Sergey Ponomarev for The New York Times

Ivan Nechepurenko contributed reporting.

Advertisement

\protect\hyperlink{after-bottom}{Continue reading the main story}

\hypertarget{site-index}{%
\subsection{Site Index}\label{site-index}}

\hypertarget{site-information-navigation}{%
\subsection{Site Information
Navigation}\label{site-information-navigation}}

\begin{itemize}
\tightlist
\item
  \href{https://help.nytimes3xbfgragh.onion/hc/en-us/articles/115014792127-Copyright-notice}{©~2020~The
  New York Times Company}
\end{itemize}

\begin{itemize}
\tightlist
\item
  \href{https://www.nytco.com/}{NYTCo}
\item
  \href{https://help.nytimes3xbfgragh.onion/hc/en-us/articles/115015385887-Contact-Us}{Contact
  Us}
\item
  \href{https://www.nytco.com/careers/}{Work with us}
\item
  \href{https://nytmediakit.com/}{Advertise}
\item
  \href{http://www.tbrandstudio.com/}{T Brand Studio}
\item
  \href{https://www.nytimes3xbfgragh.onion/privacy/cookie-policy\#how-do-i-manage-trackers}{Your
  Ad Choices}
\item
  \href{https://www.nytimes3xbfgragh.onion/privacy}{Privacy}
\item
  \href{https://help.nytimes3xbfgragh.onion/hc/en-us/articles/115014893428-Terms-of-service}{Terms
  of Service}
\item
  \href{https://help.nytimes3xbfgragh.onion/hc/en-us/articles/115014893968-Terms-of-sale}{Terms
  of Sale}
\item
  \href{https://spiderbites.nytimes3xbfgragh.onion}{Site Map}
\item
  \href{https://help.nytimes3xbfgragh.onion/hc/en-us}{Help}
\item
  \href{https://www.nytimes3xbfgragh.onion/subscription?campaignId=37WXW}{Subscriptions}
\end{itemize}
