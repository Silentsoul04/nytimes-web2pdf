Sections

SEARCH

\protect\hyperlink{site-content}{Skip to
content}\protect\hyperlink{site-index}{Skip to site index}

\href{https://www.nytimes3xbfgragh.onion/section/world/middleeast}{Middle
East}

\href{https://myaccount.nytimes3xbfgragh.onion/auth/login?response_type=cookie\&client_id=vi}{}

\href{https://www.nytimes3xbfgragh.onion/section/todayspaper}{Today's
Paper}

\href{/section/world/middleeast}{Middle East}\textbar{}U.A.E. Becomes
First Arab Nation to Open a Nuclear Power Plant

\url{https://nyti.ms/2BQdSK9}

\begin{itemize}
\item
\item
\item
\item
\item
\end{itemize}

Advertisement

\protect\hyperlink{after-top}{Continue reading the main story}

Supported by

\protect\hyperlink{after-sponsor}{Continue reading the main story}

\hypertarget{uae-becomes-first-arab-nation-to-open-a-nuclear-power-plant}{%
\section{U.A.E. Becomes First Arab Nation to Open a Nuclear Power
Plant}\label{uae-becomes-first-arab-nation-to-open-a-nuclear-power-plant}}

The launch is raising concerns about the growing number of nuclear
programs in the volatile Middle East.

\includegraphics{https://static01.graylady3jvrrxbe.onion/images/2020/08/01/world/01uae-nuclear/01uae-nuclear-articleLarge.jpg?quality=75\&auto=webp\&disable=upscale}

\href{https://www.nytimes3xbfgragh.onion/by/vivian-yee}{\includegraphics{https://static01.graylady3jvrrxbe.onion/images/2018/02/20/multimedia/author-vivian-yee/author-vivian-yee-thumbLarge-v2.png}}

By \href{https://www.nytimes3xbfgragh.onion/by/vivian-yee}{Vivian Yee}

\begin{itemize}
\item
  Aug. 1, 2020
\item
  \begin{itemize}
  \item
  \item
  \item
  \item
  \item
  \end{itemize}
\end{itemize}

BEIRUT, Lebanon --- The United Arab Emirates became the first Arab
country to open a nuclear power plant on Saturday, raising concerns
about the long-term consequences of introducing more nuclear programs to
the Middle East.

Two other countries in the region --- Israel and Iran --- already have
nuclear capabilities. Israel has an unacknowledged nuclear weapons
arsenal and Iran has a controversial uranium enrichment program that it
insists is solely for peaceful purposes.

The U.A.E., a tiny nation that has become a regional heavyweight and
international business center, said it built the plant to decrease its
reliance on the oil that has powered and enriched the country and its
Gulf neighbors for decades. It said that once its four units were all
running, the South Korean-designed plant would provide a quarter of the
country's electricity.

Seeking to quiet fears that it was trying to build muscle to use against
its regional rivals, it has insisted that it intends to use its nuclear
program only for energy purposes.

But with Iran in a standoff with Western powers over its nuclear
program, Israel in the neighborhood and tensions high among Gulf
countries, some analysts view the new plant --- and any that may follow
--- as a security and environmental headache. Other Arab countries,
including Saudi Arabia, are also starting or planning nuclear energy
programs.

The Middle East is already riven with enmities that pit Saudi Arabia and
the U.A.E. against Iran, Qatar and Iran's regional proxies. One of those
proxies, the Yemen-based Houthi rebel group, claimed an attack on the
Barakah plant when it was under construction in 2017.

And Iran is widely believed to be behind a series of attacks on Saudi
oil facilities and oil tankers passing through the Gulf over the last
year.

``The UAE's investment in these four
\href{https://www.nuclearconsult.com/wp/wp-content/uploads/2019/12/Gulf-Nuclear-Ambition-NCG-Dec-2019.pdf}{nuclear
reactors risks} further destabilizing the volatile Gulf region, damaging
the environment and raising the possibility of nuclear proliferation,''
Paul Dorfman, a researcher at University College London's Energy
Institute, wrote in an
\href{https://www.ucl.ac.uk/news/2020/mar/opinion-why-uae-about-open-four-nuclear-reactors}{op-ed}
in March.

Noting that the U.A.E. had other energy options, including ``some of the
best solar energy resources in the world,'' he added that ``the nature
of Emirate interest in nuclear may lie hidden in plain sight --- nuclear
weapon proliferation.''

But the U.A.E. has said it considered natural gas and renewable energy
sources before dismissing them in favor of nuclear energy because they
would not produce enough for its needs.

Offering evidence that its intentions are peaceful, it points to its
collaborations with the International Atomic Energy Agency, which has
reviewed the Barakah project, and the United States, with which it
signed a
\href{https://www.reuters.com/article/us-saudi-nuclear-usa/u-s-tells-saudi-arabia-nuclear-push-depends-on-snap-inspections-deal-idUSKBN1W2245}{nuclear
energy cooperation agreement} in 2009 that allows it to receive nuclear
materials and technical assistance from the United States while barring
it from uranium enrichment and other possible bomb-development
activities.

That has not persuaded Qatar, which last year
\href{https://www.reuters.com/article/us-qatar-emirates-nuclearpower-exclusive/exclusive-qatar-asks-iaea-to-intervene-over-threat-posed-by-uae-nuclear-plant-idUSKCN1R120L}{lodged
a complaint} with the international nuclear watchdog group over the
Barakah plant, calling it ``a serious threat to the stability of the
region and its environment.''

The U.A.E.'s oil exports account for about a quarter of its total gross
domestic product. Despite its gusher of oil, it has imported increasing
amounts of natural gas in recent years in part to power its
energy-intensive desalination plants.

``We proudly witness the start of Barakah nuclear power plant
operations, in alignment with the highest international safety
standards,'' Mohammed bin Zayed, the U.A.E.'s de facto ruler, tweeted on
Saturday.

The new nuclear facility, which is in the Gharbiya region on the coast,
close to Qatar and Saudi Arabia, is the first of several prospective
Middle East nuclear plants. Egypt plans to build a power plant with four
nuclear reactors.

Saudi Arabia is also building a civilian nuclear reactor while pursuing
a nuclear cooperation deal with the United States, though the Trump
administration has said it would sign such an agreement
\href{https://www.reuters.com/article/us-saudi-nuclear-usa/u-s-tells-saudi-arabia-nuclear-push-depends-on-snap-inspections-deal-idUSKBN1W2245}{only
if it includes safeguards} against weapons development.

Advertisement

\protect\hyperlink{after-bottom}{Continue reading the main story}

\hypertarget{site-index}{%
\subsection{Site Index}\label{site-index}}

\hypertarget{site-information-navigation}{%
\subsection{Site Information
Navigation}\label{site-information-navigation}}

\begin{itemize}
\tightlist
\item
  \href{https://help.nytimes3xbfgragh.onion/hc/en-us/articles/115014792127-Copyright-notice}{©~2020~The
  New York Times Company}
\end{itemize}

\begin{itemize}
\tightlist
\item
  \href{https://www.nytco.com/}{NYTCo}
\item
  \href{https://help.nytimes3xbfgragh.onion/hc/en-us/articles/115015385887-Contact-Us}{Contact
  Us}
\item
  \href{https://www.nytco.com/careers/}{Work with us}
\item
  \href{https://nytmediakit.com/}{Advertise}
\item
  \href{http://www.tbrandstudio.com/}{T Brand Studio}
\item
  \href{https://www.nytimes3xbfgragh.onion/privacy/cookie-policy\#how-do-i-manage-trackers}{Your
  Ad Choices}
\item
  \href{https://www.nytimes3xbfgragh.onion/privacy}{Privacy}
\item
  \href{https://help.nytimes3xbfgragh.onion/hc/en-us/articles/115014893428-Terms-of-service}{Terms
  of Service}
\item
  \href{https://help.nytimes3xbfgragh.onion/hc/en-us/articles/115014893968-Terms-of-sale}{Terms
  of Sale}
\item
  \href{https://spiderbites.nytimes3xbfgragh.onion}{Site Map}
\item
  \href{https://help.nytimes3xbfgragh.onion/hc/en-us}{Help}
\item
  \href{https://www.nytimes3xbfgragh.onion/subscription?campaignId=37WXW}{Subscriptions}
\end{itemize}
