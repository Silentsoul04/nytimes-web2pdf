Sections

SEARCH

\protect\hyperlink{site-content}{Skip to
content}\protect\hyperlink{site-index}{Skip to site index}

\href{https://www.nytimes3xbfgragh.onion/section/arts/music}{Music}

\href{https://myaccount.nytimes3xbfgragh.onion/auth/login?response_type=cookie\&client_id=vi}{}

\href{https://www.nytimes3xbfgragh.onion/section/todayspaper}{Today's
Paper}

\href{/section/arts/music}{Music}\textbar{}Leon Fleisher, 92, Dies;
Spellbinding Pianist With One Hand or Two

\url{https://nyti.ms/2BQplcD}

\begin{itemize}
\item
\item
\item
\item
\item
\end{itemize}

Advertisement

\protect\hyperlink{after-top}{Continue reading the main story}

Supported by

\protect\hyperlink{after-sponsor}{Continue reading the main story}

\hypertarget{leon-fleisher-92-dies-spellbinding-pianist-with-one-hand-or-two}{%
\section{Leon Fleisher, 92, Dies; Spellbinding Pianist With One Hand or
Two}\label{leon-fleisher-92-dies-spellbinding-pianist-with-one-hand-or-two}}

Unable to use his right hand, he performed pieces written for left hand
only, conducted and taught. After 30 years, he made a triumphant
two-handed comeback.

\includegraphics{https://static01.graylady3jvrrxbe.onion/images/2019/08/29/obituaries/00fleisher1/merlin_18504701_f19acbe9-202c-4815-bd7c-0e86390beb86-articleLarge.jpg?quality=75\&auto=webp\&disable=upscale}

By \href{https://www.nytimes3xbfgragh.onion/by/allan-kozinn}{Allan
Kozinn}

\begin{itemize}
\item
  Aug. 2, 2020
\item
  \begin{itemize}
  \item
  \item
  \item
  \item
  \item
  \end{itemize}
\end{itemize}

Leon Fleisher, a leading American pianist in the 1950s and early '60s
who was forced by an injury to his right hand to channel his career into
conducting, teaching and mastering the left-hand repertoire, died on
Sunday in a hospice in Baltimore. He was 92.

His death was confirmed by his son Julian, who said he was still
teaching and conducting master classes as recently as last week.

Mr. Fleisher came to believe that his career-altering malady, focal
dystonia, was caused by overpracticing --- ``seven or eight hours a day
of pumping ivory,'' as he told The New York Times in a 1996 interview
--- and for 30 years he tried virtually any cure that looked promising,
including shots of lidocaine, rehabilitation therapy, psychotherapy,
shock treatments, Rolfing and EST. At times, he later said, he was so
despondent that he considered suicide.

But he also realized that the musicality and incisiveness that had been
so widely admired in his early years could be mined in other ways. He
had joined the faculty of the Peabody Conservatory, in Baltimore, in
1959, and he devoted himself more fully to teaching, both at Peabody and
at the Tanglewood Music Center, where he was artistic director from 1986
to 1997.

He also made his way through the estimable catalog of works composed by
Ravel, Prokofiev and many others for the pianist Paul Wittgenstein (the
brother of the philosopher Ludwig Wittgenstein), who lost his right arm
during World War I, and commissioned new left-hand works from American
composers. He helped start the Theater Chamber Players in Washington.
And he began conducting.

\includegraphics{https://static01.graylady3jvrrxbe.onion/images/2020/07/30/obituaries/Fleisher-2/merlin_9068922_c7b55747-d4eb-465b-bf57-f1a76b484386-articleLarge.jpg?quality=75\&auto=webp\&disable=upscale}

Eventually, a combination of Rolfing --- a deep massage technique ---
and Botox injections provided sufficient relief that he was able to
resume his career as a two-handed pianist in 1995. He continued to play
recitals and concertos, and to make recordings, until last year.

Mr. Fleisher often pointed out after his comeback that he was not, and
never would be, fully cured. But he also acknowledged, late in life,
that the incapacitation of his right hand in 1964 ultimately gave him a
far more varied musical life than he might have had if he had been able
to pursue a conventional career as a virtuoso pianist.

That realization is implicit in the title of his autobiography, ``My
Nine Lives: A Memoir of Many Careers in Music'' (2010), which he wrote
with the music critic Anne Midgette.

Early in his career, though, Mr. Fleisher was a commanding pianist who
produced a warm, sharply etched and thoughtfully contoured sound that
was ideally suited to 19th-century Viennese classics --- Beethoven,
Brahms and Schubert, most notably --- but also yielded illuminating
readings of Rachmaninoff, Debussy and Liszt, and of contemporary
American composers like Roger Sessions (with whom he briefly studied
music theory) and Aaron Copland.

Mr. Fleisher's recordings of the Brahms and Beethoven piano concertos
with George Szell and the Cleveland Orchestra, made between 1958 and
1963, are still considered among the most vivid and moving accounts of
those works.

In the 1990s, he recorded spellbinding performances of the peaks of the
left-hand repertoire, including concertos by Ravel, Prokofiev and
Britten, chamber music by Korngold and Schmidt, and solo works by
Saint-Saëns, Godowsky and Bach (Brahms's left-hand arrangement of the
Chaconne from the Partita No. 2 for solo violin).

Even after he returned to recording two-hand works, on the albums ``Two
Hands'' (2004) and ``The Journey'' (2006), he continued to revisit the
left-hand works that had kept him going for three decades.

His album ``All the Things You Are'' (2014) included not only left-hand
arrangements of Gershwin's ``The Man I Love'' and the Jerome Kern song
that gave the collection its name, but also pieces composed for Mr.
Fleisher by George Perle and Leon Kirchner, and a deeply thoughtful,
spacious reconsideration of the Bach-Brahms Chaconne.

Leon Fleisher was born in San Francisco on July 23, 1928, to Isidore and
Bertha Fleisher. His parents, Jewish immigrants from Eastern Europe ---
he was from Odessa, then in Russia, now in Ukraine; she was from Poland
--- each managed one of the family's two hat shops.

An older brother, Raymond, was given piano lessons. He showed little
interest in them, but when Raymond went out to play after his lessons,
Leon, who was then 4 years old, would go to the piano and repeat, by
ear, everything he had heard.

His mother soon decided that Leon, rather than Raymond, should study the
instrument. She made her intentions for her younger son clear: He would
either be the first Jewish president of the United States or he would be
a concert pianist.

So devoted was his mother to his musical training that after two weeks
of kindergarten, during which he objected strenuously to nap time, she
withdrew him from public school and hired tutors so he could devote his
time to practicing at the piano. She also found ways of bringing him to
the attention of two important San Francisco conductors, Pierre Monteux
and Alfred Hertz, who in turn persuaded the pianist Artur Schnabel to
take Leon on as a student in 1938, when he was 9, despite his policy of
not teaching children.

By the time Leon began working with Schnabel, he had already played a
few concerts, but Schnabel's single condition for teaching the boy was
that there be no more concerts. Schnabel relaxed the rule in 1944 and
allowed Mr. Fleisher to play the Brahms Piano Concerto No. 1 in D minor
with Monteux and the San Francisco Symphony and then with the New York
Philharmonic at Carnegie Hall, also with Monteux conducting.

Noel Strauss, reviewing the performance for The New York Times, wrote
that Mr. Fleisher, making his New York debut, ``scored heavily in the
exacting work and at once established himself as one of the most
remarkably gifted of the younger generation of American keyboard
artists.''

In 1945, at Ravinia, Mr. Fleisher played the Brahms again --- it quickly
became one of his signature pieces --- as well as the Liszt Concerto No.
2 in A, with Leonard Bernstein conducting the Chicago Symphony
Orchestra. He also performed four concertos at Ravinia the next summer,
under the direction of William Steinberg and Szell, who soon engaged Mr.
Fleisher to perform with the Cleveland Orchestra, which he took over
later that year.

By 1949, although he had played with many of the major American
orchestras and had given recitals across the country, engagements began
to dry up. Mr. Fleisher moved to Paris in 1950 and remained in Europe
--- relocating first to the Netherlands, then to Italy --- until 1958.

In 1952, he became the first American to win the gold medal at the Queen
Elisabeth Competition in Brussels. That victory included a substantial
list of engagements in Europe; it also revived interest in Mr. Fleisher
among American orchestras, managers and concert promoters.

When Szell and the Cleveland Orchestra were signed to a new recording
contract with the Epic label in 1954, Szell invited Mr. Fleisher to be
his go-to soloist for recordings of the great piano concertos.

Shortly after his return to the United States in the late 1950s, Mr.
Fleisher accepted an offer to teach at the Peabody Conservatory, while
also pursuing a hefty performing and recording schedule.

Image

Mr. Fleisher in concert at Carnegie Hall in 2003.~Credit...Chris Lee for
The New York Times

``I was driven, if anything, even harder by all of my successes,'' he
wrote in his memoir. ``There was always more to attain, and more to
achieve, and more musical depths to plumb, and lurking behind it all,
the terrifying risk of failure.''

Failure was not far away. During the winter of 1963, he noticed what he
described as laziness in his right index finger, as well as ``a creeping
numbness'' in his right hand. By the summer, the fourth and fifth
fingers of his right hand had begun to curl inward toward his palm.

The timing was disastrous. Mr. Fleisher had planned to celebrate the
20th anniversary of his New York debut with a busy season that included
20 performances in New York alone and a spring 1964 tour of the Soviet
Union, in which he was to be the soloist in Mozart's Concerto No. 25 in
C (K. 503) with Szell and the Cleveland Orchestra.

Shortly before the tour, Mr. Fleisher performed the Mozart in Cleveland.
Szell noted the strain Mr. Fleisher was under and told him that he did
not feel he could undertake the tour. The pianist Grant Johannesen
traveled with the orchestra instead.

``The initial problem was a very stupid kind of overwork,'' Mr. Fleisher
said in 1996, cautioning young pianists against following his path. ``I
see kids still falling into this, and there are many reasons for it. The
perfection that they're bombarded with from recordings. The kind of
sound a Horowitz produced, which is wonderful, but people don't realize
that he had his technician work very hard on the piano, so the piano
itself helped. So when kids go to an acoustically dead hall, and get a
dead piano, and try to make these Horowitz kinds of sounds, they end up
brutalizing themselves.''

Mr. Fleisher resisted taking up the left-hand repertoire, partly because
he felt that to do so would be an admission that he would never regain
the use of his right hand. But after two years without playing concerts,
he reconsidered, agreeing to play both Ravel's Concerto for the Left
Hand and Benjamin Britten's left-hand work ``Diversions'' with Seiji
Ozawa and the Toronto Symphony in 1967.

The next year, with the pianist and composer Dina Koston, he started the
Theater Chamber Players, a flexible chamber group meant to present both
contemporary music and classics.

The ensemble --- initially based at the Washington Theater Club, later
at the Smithsonian National Museum of Natural History and ultimately at
the Kennedy Center --- provided an opportunity for Mr. Fleisher to both
play and conduct. And an invitation to become music director of the
Annapolis Symphony Orchestra in Maryland, a semiprofessional community
group, gave him a chance to work on the symphonic repertoire.

Soon, Mr. Fleisher was guest-conducting around the country --- his debut
at the head of a professional orchestra took place at Lincoln Center's
Mostly Mozart Festival in 1970 --- and in 1973 he became associate
conductor of the Baltimore Symphony Orchestra.

He held that post for only five years, but he maintained a close
relationship with the orchestra thereafter. When the ensemble was
preparing to inaugurate the new Joseph Meyerhoff Symphony Hall in 1982,
its music director, Sergiu Comissiona, invited Mr. Fleisher to be the
opening-night soloist.

Having recently had an operation to relieve carpal tunnel syndrome, Mr.
Fleisher began to regain the use of his right hand, if only partly and
inconsistently. But he felt he could make the jump back to two-handed
playing, with the televised opening of Meyerhoff Hall as the occasion
for his comeback.

In a bold moment, he told the orchestra that he would play Beethoven's
Fourth Piano Concerto. But as the occasion drew near, he decided to play
Franck's shorter and less pianistically exposed Symphonic Variations
instead.

Image

Mr. Fleisher performing in 2001 at the New York String Orchestera's
annual~ Christmas Eve concert at Carnegie Hall.Credit...Chris Lee/The
New York Times

Most listeners thought the performance went well. But Mr. Fleisher was
not satisfied. In his view, the amount of effort he expended working to
control his right hand precluded the kind of interpretive depth he hoped
for, and he dropped plans for a broader return to two-handed playing.

Shortly after the Baltimore performance, Mr. Fleisher married Katherine
Jacobson, a pianist who had been one of his students at Peabody.

She survives him as do his children from his first marriage, to Dorothy
Druzinsky Fleisher, Deborah Fleisher, Leah Fleisher and Richard; and his
children from his second marriage, to Rikki Rosenthal, Paula Fleisher
and Julian; and two grandchildren. Both of Mr. Fleisher's earlier
marriages ended in divorce.

In 1991, Mr. Fleisher found a doctor who was experimenting with Botox
injections for injuries like his. At first he found that the injections
loosened up his still-cramped fourth and fifth fingers, to the point
where he could play. But the injections wore off, and Mr. Fleisher was
still looking for a permanent cure.

Having tried Rolfing in the 1970s, he decided to try again in 1994. This
time he had better results, and he found that a regimen of Rolfing and
Botox injections was enough to keep him in playing trim.

As an experiment, he played Mozart's Piano Concerto No. 12 (K. 414) with
the Theater Chamber Players in April 1995, and with the Cleveland
Orchestra and at Tanglewood shortly thereafter.

``Nothing felt sweeter,'' he wrote in his memoir of those first
performances, ``than the feeling of those notes falling into place, the
right hand singing, the left hand balancing it on the lower part of the
keyboard, and the piece growing into something whole and complete, a
dream become reality.''

Mr. Fleisher gradually reclaimed the repertoire he had been unable to
play for more than three decades --- but cautiously, building his
recital programs with both two-hand and left-hand works, and playing
programs of piano four-hand works with his wife.

Image

Mr. Fleisher and his wife, the pianist Katherine Jacobson Fleisher,
arriving at the Kennedy Center Honors gala in Washington in 2007. He was
among that year's recipients.~Credit...Nicholas Kamm/Agence
France-Presse --- Getty Images

He was made a Commander of the Order of Arts and Letters by the French
government in 2006, and in 2007 he was a recipient of a Kennedy Center
Honor. A film about his struggle with focal dystonia, ``Two Hands,''
directed by Nathaniel Kahn, was nominated for an Academy Award for best
short documentary in 2006.

Toward the end of his life, Mr. Fleisher spoke about the level of
despair he felt when he was unable to use his right hand. But, having
regained that ability, he was also philosophical about the challenges
life presents.

``There are forces out there,'' he told The International Herald Tribune
in 2007, ``and if you keep yourself open to them, if you go along with
them, there are wondrous surprises.''

Jack Kadden contributed reporting.

Advertisement

\protect\hyperlink{after-bottom}{Continue reading the main story}

\hypertarget{site-index}{%
\subsection{Site Index}\label{site-index}}

\hypertarget{site-information-navigation}{%
\subsection{Site Information
Navigation}\label{site-information-navigation}}

\begin{itemize}
\tightlist
\item
  \href{https://help.nytimes3xbfgragh.onion/hc/en-us/articles/115014792127-Copyright-notice}{©~2020~The
  New York Times Company}
\end{itemize}

\begin{itemize}
\tightlist
\item
  \href{https://www.nytco.com/}{NYTCo}
\item
  \href{https://help.nytimes3xbfgragh.onion/hc/en-us/articles/115015385887-Contact-Us}{Contact
  Us}
\item
  \href{https://www.nytco.com/careers/}{Work with us}
\item
  \href{https://nytmediakit.com/}{Advertise}
\item
  \href{http://www.tbrandstudio.com/}{T Brand Studio}
\item
  \href{https://www.nytimes3xbfgragh.onion/privacy/cookie-policy\#how-do-i-manage-trackers}{Your
  Ad Choices}
\item
  \href{https://www.nytimes3xbfgragh.onion/privacy}{Privacy}
\item
  \href{https://help.nytimes3xbfgragh.onion/hc/en-us/articles/115014893428-Terms-of-service}{Terms
  of Service}
\item
  \href{https://help.nytimes3xbfgragh.onion/hc/en-us/articles/115014893968-Terms-of-sale}{Terms
  of Sale}
\item
  \href{https://spiderbites.nytimes3xbfgragh.onion}{Site Map}
\item
  \href{https://help.nytimes3xbfgragh.onion/hc/en-us}{Help}
\item
  \href{https://www.nytimes3xbfgragh.onion/subscription?campaignId=37WXW}{Subscriptions}
\end{itemize}
