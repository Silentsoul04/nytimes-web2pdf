Sections

SEARCH

\protect\hyperlink{site-content}{Skip to
content}\protect\hyperlink{site-index}{Skip to site index}

\href{https://www.nytimes3xbfgragh.onion/section/business}{Business}

\href{https://myaccount.nytimes3xbfgragh.onion/auth/login?response_type=cookie\&client_id=vi}{}

\href{https://www.nytimes3xbfgragh.onion/section/todayspaper}{Today's
Paper}

\href{/section/business}{Business}\textbar{}Driving Across the Globe to
Fight Child Trafficking

\url{https://nyti.ms/2XjmZue}

\begin{itemize}
\item
\item
\item
\item
\item
\end{itemize}

Advertisement

\protect\hyperlink{after-top}{Continue reading the main story}

Supported by

\protect\hyperlink{after-sponsor}{Continue reading the main story}

Wheels

\hypertarget{driving-across-the-globe-to-fight-child-trafficking}{%
\section{Driving Across the Globe to Fight Child
Trafficking}\label{driving-across-the-globe-to-fight-child-trafficking}}

Renée Brinkerhoff has been on a quest, on almost every continent, to use
racing ``as a platform to do something about'' a global scourge. Next
stop, Antarctica.

\includegraphics{https://static01.graylady3jvrrxbe.onion/images/2020/08/03/business/02wheels1-print/02wheels-1-articleLarge.jpg?quality=75\&auto=webp\&disable=upscale}

By Mercedes Lilienthal

\begin{itemize}
\item
  Aug. 2, 2020
\item
  \begin{itemize}
  \item
  \item
  \item
  \item
  \item
  \end{itemize}
\end{itemize}

It wasn't until Renée Brinkerhoff's last of four children departed for
college that her life changed forever.

``At the age of 55, I had a realization,'' Ms. Brinkerhoff said. It was
reassuringly simple and dauntingly complex: ``One day, I'm going to race
a car.''

She chose road rallying; competing on a racetrack held little interest.
Road rallies, however, would bring her through countless landscapes with
captivating scenery. She was sold. Still, she was a novice, and she
would have to overcome huge obstacles, ``of which fear was primary,''
she said.

So in 2013, she founded ****
\href{https://www.valkyrieracing.com/}{Valkyrie Racing} with one chief
goal: to break barriers for women. The name ``Valkyrie'' comes from
Norse mythology, where women warriors saved the worthy from the field of
battle and restored their life in Valhalla. Ms. Brinkerhoff thinks of
them as warriors who are strong yet compassionate.

She hardly started small, beginning her racing career in La Carrera
Panamericana, a treacherous 2,000-mile road rally across Mexico
considered one of the most dangerous in the world. She proved she was a
natural, too, reaching the podium in her first try, and on subsequent
entries, too.

\includegraphics{https://static01.graylady3jvrrxbe.onion/images/2020/08/02/business/02wheels-2/02wheels-2-articleLarge.jpg?quality=75\&auto=webp\&disable=upscale}

Her ambitions were larger still. ``We found we had a `voice', though
small,'' said Ms. Brinkerhoff, who is now 64 and lives in Colorado.
``People were listening to our unique story. We thought if we did
something on a bigger scale, a global scale, we could potentially have a
bigger voice. That was the impetus for Project 356 World Rally Tour.''

Her tour would encompass six rallies and one ultimate challenge: a drive
on every continent, all while piloting one vehicle --- her classic
Porsche 356, for which she named the project.

And for her project, she chose the most challenging races on each
continent. Beijing to Paris (otherwise known as Peking to Paris):
crossing so many countries and time zones. The East African Safari
Classic: the most difficult off-tarmac classic car rally in the world.
Each rally has its unique challenges; she wanted to be a groundbreaking
part with each of them.

When she started Project 356, Ms. Brinkerhoff also saw the event as a
way to give back. ``At this same time, we started our philanthropic arm,
\href{https://www.valkyriegives.org/}{Valkyrie Gives},'' she said. ``The
idea was to use our racing as a platform to do something about child
trafficking worldwide.''

``The mission to fight child trafficking found me, I didn't choose it,''
she added. A few critical coincidences led her to this cause. ``It
became apparent to me I was supposed to do everything I could to fight
for these children.''

By chance, Ms. Brinkerhoff met an F.B.I. agent whose job was to track
down peddlers of child pornography. In another chance encounter, she saw
a man on an airport bus viewing an illicit image of a young child on his
phone.

``I do not believe in coincidence, and knew I was being told to do
something about this,'' she said. ``I began researching this crime and
learned child pornography fuels child trafficking and that this was a
massive global problem.''

Image

Driving through the mud in the East African Safari Classic
Rally.Credit...John Bennie

Her adventurous spirit may owe something to her childhood. Her early
years were split between a small beach town in Southern California and
also Southeast Asia during the height of the Cold War. She and her
family lived in Hong Kong during
\href{https://www.nytimes3xbfgragh.onion/2016/05/15/world/asia/china-cultural-revolution-explainer.html}{Mao
Zedong's Cultural Revolution}, and they lived in Laos at the end of the
Vietnam War.

To date, Ms. Brinkerhoff has rallied her way through all but one major
challenge. She has competed in 17 countries on six continents.
\href{https://www.youtube.com/channel/UCElmKq1ePnGrPKPqH1fIqLg}{She will
soon face} her most difficult endeavor --- Antarctica, where she and her
team will race, alone, against extreme elements. The plan is to cover
356 miles on ice, and if successful, notch a land speed record on a blue
ice runway at Union Glacier. Once finished, Ms. Brinkerhoff will have
raced nearly 20,000 miles.

Ms. Brinkerhoff and team have had many challenges along the way. Corrupt
custom agents held the Porsche and its parts until they bribed their way
past. Accidents and broken parts, like a crash at the 2015 Carrera
Panamericana race, kept them on their toes. Delays crossing oceans and
new rally regulations tested their patience.

More pedestrian issues posed hurdles, as well, like racing at elevations
of up to 16,000 feet and competing in new terrain that was different
than what they had prepared for. The tour pushed their skills beyond
capacity. Even language barriers caused hiccups.

Toughest among their challenges was the East African Safari Classic late
last year across Kenya and Tanzania. ``It was the rainiest season in 40
years and the roads, already chosen for their difficulty, were more
treacherous than planned,'' said Ms. Brinkerhoff, who was a novice again
in this kind of terrain. Deep sand, thick mud and treacherous water
crossings made it difficult for all competitors.

Adding to the adversity, the front right steering arm wasn't holding up.
It couldn't withstand the severe terrain and repeatedly failed.
Eventually, the issue was corrected and Ms. Brinkerhoff and her
navigator crossed the finish line.

To date, Valkyrie Gives has raised approximately \$200,000 to help fight
child trafficking. All donations have gone directly to charitable
organizations around the world: Mexico, Australia, Peru, China,
Mongolia, Kazakhstan, Russia, the Netherlands, Kenya, Cambodia, Thailand
and the United States.

Ms. Brinkerhoff said she had also ``participated in undercover
operations to gather evidence for law enforcement to arrest and
prosecute traffickers.'' She once spent two weeks in Southeast Asia
working with the Exodus Road, a nonprofit organization with seven
offices worldwide and a staff of 72 that finds and frees victims of
trafficking. Since the group's inception, it has rescued more than 1,500
victims and aided in the arrest of 600 traffickers and pedophiles, it
says.

Matt Parker, an Exodus Road co-founder and the acting chief executive,
met Ms. Brinkerhoff several years ago at his Colorado Springs office.
``She came to my office and once I heard her remarkable story, we began
a working relationship,'' he said.

``Renée is one of the most genuine people I've ever met, and the world
is full of people who want to see trafficking end,'' Mr. Parker added.
``But I wish it was full of people like Renée who will actually
\emph{do} something about it.''

Image

By Day Four of the East African rally, Valkyrie Racing was getting
accustomed to the unique landscapes and terrain --- including this
wooden bridge crossing.Credit...John Bennie

In November, before the rally in East Africa, Ms. Brinkerhoff and her
team arrived early to meet with a group they had supported in Kenya.
``When we go to a country to race,'' she said, ``we try to visit the
child trafficking charity we have chosen for support.''

The charity this time was Awareness Against Human Trafficking, and
Sophie Otiende, the program consultant, shared a story about two young
girls who had tried to commit suicide just a night before. ``They were
waiting to be repatriated to their country,'' Ms. Otiende told the
group. ``They were losing hope of ever returning home.''

The girls wouldn't be treated without making a cash prepayment to the
hospital. The Valkyrie Gives foundation immediately donated the funds to
get the girls admitted to the hospital.

The rallies are thrilling, Ms. Brinkerhoff said, but making a difference
for children is the most rewarding. ``The memories of the children
around the world that we have been blessed to touch and their innocent
faces will forever be in our minds and hearts,'' she said.

``Since I started racing, I have always believed it was what I was being
called to do,'' she added. ``Faith has allowed me to push through the
many fears I've faced.''

Advertisement

\protect\hyperlink{after-bottom}{Continue reading the main story}

\hypertarget{site-index}{%
\subsection{Site Index}\label{site-index}}

\hypertarget{site-information-navigation}{%
\subsection{Site Information
Navigation}\label{site-information-navigation}}

\begin{itemize}
\tightlist
\item
  \href{https://help.nytimes3xbfgragh.onion/hc/en-us/articles/115014792127-Copyright-notice}{©~2020~The
  New York Times Company}
\end{itemize}

\begin{itemize}
\tightlist
\item
  \href{https://www.nytco.com/}{NYTCo}
\item
  \href{https://help.nytimes3xbfgragh.onion/hc/en-us/articles/115015385887-Contact-Us}{Contact
  Us}
\item
  \href{https://www.nytco.com/careers/}{Work with us}
\item
  \href{https://nytmediakit.com/}{Advertise}
\item
  \href{http://www.tbrandstudio.com/}{T Brand Studio}
\item
  \href{https://www.nytimes3xbfgragh.onion/privacy/cookie-policy\#how-do-i-manage-trackers}{Your
  Ad Choices}
\item
  \href{https://www.nytimes3xbfgragh.onion/privacy}{Privacy}
\item
  \href{https://help.nytimes3xbfgragh.onion/hc/en-us/articles/115014893428-Terms-of-service}{Terms
  of Service}
\item
  \href{https://help.nytimes3xbfgragh.onion/hc/en-us/articles/115014893968-Terms-of-sale}{Terms
  of Sale}
\item
  \href{https://spiderbites.nytimes3xbfgragh.onion}{Site Map}
\item
  \href{https://help.nytimes3xbfgragh.onion/hc/en-us}{Help}
\item
  \href{https://www.nytimes3xbfgragh.onion/subscription?campaignId=37WXW}{Subscriptions}
\end{itemize}
