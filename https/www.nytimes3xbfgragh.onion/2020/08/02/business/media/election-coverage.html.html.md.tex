Sections

SEARCH

\protect\hyperlink{site-content}{Skip to
content}\protect\hyperlink{site-index}{Skip to site index}

\href{https://www.nytimes3xbfgragh.onion/section/business/media}{Media}

\href{https://myaccount.nytimes3xbfgragh.onion/auth/login?response_type=cookie\&client_id=vi}{}

\href{https://www.nytimes3xbfgragh.onion/section/todayspaper}{Today's
Paper}

\href{/section/business/media}{Media}\textbar{}How the Media Could Get
the Election Story Wrong

\url{https://nyti.ms/2XkCXnU}

\begin{itemize}
\item
\item
\item
\item
\item
\end{itemize}

Advertisement

\protect\hyperlink{after-top}{Continue reading the main story}

Supported by

\protect\hyperlink{after-sponsor}{Continue reading the main story}

The media equation

\hypertarget{how-the-media-could-get-the-election-story-wrong}{%
\section{How the Media Could Get the Election Story
Wrong}\label{how-the-media-could-get-the-election-story-wrong}}

We may not know the results for days, and maybe weeks. So it's time to
rethink ``election night.''

\includegraphics{https://static01.graylady3jvrrxbe.onion/images/2020/08/03/business/03bensmith/03bensmith-superJumbo.jpg}

By \href{https://www.nytimes3xbfgragh.onion/by/ben-smith}{Ben Smith}

\begin{itemize}
\item
  Aug. 2, 2020
\item
  \begin{itemize}
  \item
  \item
  \item
  \item
  \item
  \end{itemize}
\end{itemize}

Picture this Thanksgiving: turkey, football (maybe), tenser-than-usual
interactions with relatives. And perhaps a new tradition: finding out
who actually won the presidential election.

The coronavirus crisis means that states like Pennsylvania may be
counting mail-in ballots for weeks, while President Trump tweets false
allegations about fraud. And the last barriers between American
democracy and a deep political crisis may be television news and some
version of that maddening needle on The New York Times website.

I spoke last week to executives, TV hosts and election analysts across
leading American newsrooms, and I was struck by the blithe confidence
among some top managers and hosts, who generally said they've handled
complicated elections before and can do so again. And I was alarmed by
the near panic among some of the people paying the closest attention
---~the analysts and producers trying, and often failing, to get answers
from state election officials about how and when they will count the
ballots and report results.

``The nerds are freaking out,'' said Brandon Finnigan, the founder of
Decision Desk HQ, which delivers election results to media outlets. ``I
don't think it's penetrated enough in the average viewer's mind that
there's not going to be an election night. The usual razzmatazz of a
panel sitting around discussing election results --- that's dead,'' he
said.

The changes the media faces are profound, with technical and political
dimensions.

First, there's already a shift underway from a single-day, in-person
election. In the 2018
midterms,\href{https://www.census.gov/library/stories/2019/04/behind-2018-united-states-midterm-election-turnout.html}{only
60 percent of the votes were cast in person} on Election Day. More votes
will probably be sent in this year by mail or cast in September and
October. That risks coverage misfires: In 2018, cable news commentators
spent\href{https://www.youtube.com/watch?v=T6FQmy-1Cfc}{election night
suggesting that the ``blue wave'' hadn't arrived}. But they were simply
impatient: The Democratic surge showed up when the final California
races were called weeks later. If the 2016 election had been conducted
amid the expected surge in mail-in voting because of the coronavirus
crisis, the Pennsylvania results might not have been counted until
Thanksgiving.

Then, there's the continuing Trump-era political crisis, often driven on
Twitter and Facebook. President Trump last Thursday
again\href{https://www.nytimes3xbfgragh.onion/2020/07/30/us/elections/biden-vs-trump.html}{sought}
to call mail-in voting into question with false claims about fraud. If
you want a glimpse of how this could play out in November, look to 2018,
when Mr.
Trump\href{https://twitter.com/realDonaldTrump/status/1060993836984324096}{tweeted
the suggestion}, ``Call for a new election?'' when the Republican
nominee for Senate in Arizona fell behind as mail ballots were counted.

These are hard challenges. The media specializes in fighting the last
war, and has done a decent job this cycle of avoiding the mistakes of
2016. Reporters are calling out Mr. Trump's falsehoods, showing
skepticism about polls and avoiding turning politics into a sport.

But the American media plays a bizarrely outsize role in American
elections, occupying the place of most countries' national election
commissions.

Here, the media actually assembles the results from 50 states, tabulates
them and declares a victor. And --- we can't really help ourselves ---
the media establishes the narrative to explain what happened. That task
was most memorably mishandled in 2000, when inaccurate calls that George
W. Bush had won Florida led to a wild retraction by Vice President Al
Gore of the concession he had offered to Mr. Bush earlier that evening,
followed by weeks of uncertainty.

The flashy graphics and sober, confident hosts embody a long tradition
of television flimflam. When CBS invented the election night tradition
of dramatic vote projections and official calls in 1952, it outfitted
its set with a blinking, Remington Rand Univac computer. The blinking
device made for a good show. But the computer was a prop, a fake, as the
historian Jill Lepore noted in her
podcast,\href{https://www.thelastarchive.com/}{The Last Archive}.

The TV presentation is always slick, but the underpinnings of
county-by-county electoral systems are baroque and antiquated. And the
pandemic means more people will vote by mail this year, in states with
little experience processing those votes.

``There's a lot of planning for the whiz-bang graphics, and not enough
planning for avoiding undermining trust in the American electoral
system,'' said Brendan Nyhan, a Dartmouth political scientist and one of
the authors of an
April\href{https://www.law.uci.edu/faculty/full-time/hasen/2020ElectionReport.pdf}{report}
on how to run a fair election during the pandemic. ``It's not going to
be great TV, it might not be viral content, but it's the truth.''

Some particularly wonky journalists are trying to lay the groundwork.
NBC's Chuck Todd said in June that he has been having ``major
nightmares'' about the election, and his First Read newsletter has
been\href{https://www.nbcnews.com/politics/meet-the-press/say-goodbye-election-night-hello-election-week-n1228206}{referring}
to ``election week'' instead of Election Day.

But at the highest levels of most news organizations and the big social
media platforms, executives and insiders told me that it simply hasn't
sunk in how different this year is going to be --- and how to prepare
audiences for it.

Though the hosts and news executives I talked to all take preparations
seriously, many seemed to be preparing for this election as they have
for others in the past, and some waved off my alarmism.

``We don't want to create a self-fulfilling prophecy of chaos and
confusion or suggest somehow that that's a preordained outcome,'' said
the president of NBC News, Noah Oppenheim.

Mr. Oppenheim's optimism is a bit hard to justify. The April report on
running a fair election offers two recommendations for the media, which
it's mostly been ignoring. First, undertake an intense campaign to
explain to voters how the process will actually work this year. And
second, teach the public patience.

That's not the media's instinct. CNN did the opposite this February,
when the Iowa caucuses were slow to report results and the network put
on a ``count-up'' clock, impatiently tapping its foot for a result and
signaling that there's something wrong with a slow, careful count.

Another, smaller but important change that many political types suggest:
Get rid of the misleading ``percent of precincts reporting'' measure. In
states like Pennsylvania and Michigan, it would be easy to have 100
percent of precincts reporting their Election Day results --- but have
mail-in votes piled up in a warehouse, uncounted.

There are some encouraging signs. CNN and The Associated Press, among
others, have devoted far more reporting resources than usual to
informing audiences just how elections work and to lowering their
expectations of quick results. Mr. Oppenheim says NBC is doubling the
size of the team that covers election security and misinformation.

``It's always an unfair standard to expect that kind of movie-like
experience on election night,'' said David Scott, deputy managing editor
at the AP.

And CNN's Washington bureau chief, Sam Feist, and the CBS News elections
and surveys director, Anthony Salvanto, both told me they've moved away
from using the percent of precincts reporting measure.

A top Times editor, Steve Duenes, said The Times was considering
alternatives to the single, predictive needle that offered readers false
confidence in 2016, and is looking at a ``range of tools.''

But what the moment calls for, most of all, is patience. And good luck
with that.

Nobody I talked to had any real idea how cable talkers or Twitter
take-mongers would fill hours, days and, possibly, weeks of counting or
how to apply a sober, careful lens to the wild allegations --- rigged
voting machines, mysterious buses of outsiders turning up at poll sites
--- that surface every election night, only to dissolve in the light of
day.

Facebook's chief executive, Mark Zuckerberg, told me in a brief
interview on Saturday that he's planning to brace his audience for the
postelection period. He said the site planned a round of education aimed
at ``getting people ready for the fact that there's a high likelihood
that it takes days or weeks to count this --- and there's nothing wrong
or illegitimate about that.'' And he said that Facebook is considering
new rules regarding premature claims of victory or other statements
about the results. He added that the company's election center will rely
on wire services for definitive results.

It's possible, of course, that Joe Biden will win by a margin so large
that Florida will be called for him early. Barring that, it's tempting
to say responsible voices should keep their mouths shut and switch over
for a few days to
\href{https://www.youtube.com/watch?v=A3zkedeWbYw}{Floor Is Lava}, and
give the nice local volunteers time to count the votes. That, however,
would just cede the conversation to the least responsible, and
conspiratorial, voices.

The Republican secretary of state of Ohio, Frank LaRose, said he hoped
that the time spent waiting for results could become a kind of civics
lesson, with footage of volunteers feeding ballots into machines. Alex
Padilla, the Democratic California secretary of state, suggested that
television companies look to a Hollywood model: ``You can't think of
Election Day as a single movie --- you have to treat it as maybe a
trilogy,'' he said.

He didn't say which movie.

But conveniently, a group of former top government officials called the
Transition Integrity Project actually
\href{https://www.bostonglobe.com/2020/07/25/nation/bipartisan-group-secretly-gathered-game-out-contested-trump-biden-election-it-wasnt-pretty/}{gamed}
four possible scenarios, including one that doesn't look that different
from 2016: a big popular win for Mr. Biden, and a narrow electoral
defeat, presumably reached after weeks of counting the votes in
Pennsylvania. For their war game, they cast John Podesta, who was
Hillary Clinton's campaign chairman, in the role of Mr. Biden. They
expected him, when the votes came in, to concede, just as Mrs. Clinton
had.

But Mr. Podesta, playing Mr. Biden, shocked the organizers by saying he
felt his party wouldn't let him concede. Alleging voter suppression, he
persuaded the governors of Wisconsin and Michigan to send pro-Biden
electors to the Electoral College.

In that scenario, California, Oregon, and Washington then threatened to
secede from the United States if Mr. Trump took office as planned. The
House named Mr. Biden president; the Senate and White House stuck with
Mr. Trump. At that point in the scenario, the nation stopped looking to
the media for cues, and waited to see what the military would do.

Advertisement

\protect\hyperlink{after-bottom}{Continue reading the main story}

\hypertarget{site-index}{%
\subsection{Site Index}\label{site-index}}

\hypertarget{site-information-navigation}{%
\subsection{Site Information
Navigation}\label{site-information-navigation}}

\begin{itemize}
\tightlist
\item
  \href{https://help.nytimes3xbfgragh.onion/hc/en-us/articles/115014792127-Copyright-notice}{©~2020~The
  New York Times Company}
\end{itemize}

\begin{itemize}
\tightlist
\item
  \href{https://www.nytco.com/}{NYTCo}
\item
  \href{https://help.nytimes3xbfgragh.onion/hc/en-us/articles/115015385887-Contact-Us}{Contact
  Us}
\item
  \href{https://www.nytco.com/careers/}{Work with us}
\item
  \href{https://nytmediakit.com/}{Advertise}
\item
  \href{http://www.tbrandstudio.com/}{T Brand Studio}
\item
  \href{https://www.nytimes3xbfgragh.onion/privacy/cookie-policy\#how-do-i-manage-trackers}{Your
  Ad Choices}
\item
  \href{https://www.nytimes3xbfgragh.onion/privacy}{Privacy}
\item
  \href{https://help.nytimes3xbfgragh.onion/hc/en-us/articles/115014893428-Terms-of-service}{Terms
  of Service}
\item
  \href{https://help.nytimes3xbfgragh.onion/hc/en-us/articles/115014893968-Terms-of-sale}{Terms
  of Sale}
\item
  \href{https://spiderbites.nytimes3xbfgragh.onion}{Site Map}
\item
  \href{https://help.nytimes3xbfgragh.onion/hc/en-us}{Help}
\item
  \href{https://www.nytimes3xbfgragh.onion/subscription?campaignId=37WXW}{Subscriptions}
\end{itemize}
