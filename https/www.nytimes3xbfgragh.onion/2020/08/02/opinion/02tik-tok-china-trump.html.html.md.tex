Sections

SEARCH

\protect\hyperlink{site-content}{Skip to
content}\protect\hyperlink{site-index}{Skip to site index}

\href{https://myaccount.nytimes3xbfgragh.onion/auth/login?response_type=cookie\&client_id=vi}{}

\href{https://www.nytimes3xbfgragh.onion/section/todayspaper}{Today's
Paper}

\href{/section/opinion}{Opinion}\textbar{}Microsoft Can Save TikTok ---
if Trump Doesn't Mess It Up

\url{https://nyti.ms/3goczBg}

\begin{itemize}
\item
\item
\item
\item
\item
\item
\end{itemize}

Advertisement

\protect\hyperlink{after-top}{Continue reading the main story}

\href{/section/opinion}{Opinion}

Supported by

\protect\hyperlink{after-sponsor}{Continue reading the main story}

\hypertarget{microsoft-can-save-tiktok--if-trump-doesnt-mess-it-up}{%
\section{Microsoft Can Save TikTok --- if Trump Doesn't Mess It
Up}\label{microsoft-can-save-tiktok--if-trump-doesnt-mess-it-up}}

Microsoft wants to buy the Chinese app, and the administration needs to
get out of the way.

\includegraphics{https://static01.graylady3jvrrxbe.onion/images/2018/08/02/opinion/02swisher/02swisher-thumbLarge.png}

By Kara Swisher

Ms. Swisher covers technology and is a contributing opinion writer.

\begin{itemize}
\item
  Aug. 2, 2020
\item
  \begin{itemize}
  \item
  \item
  \item
  \item
  \item
  \item
  \end{itemize}
\end{itemize}

\includegraphics{https://static01.graylady3jvrrxbe.onion/images/2020/08/02/opinion/02Swisher/02Swisher-articleLarge.jpg?quality=75\&auto=webp\&disable=upscale}

\href{https://cn.nytimes3xbfgragh.onion/opinion/20200804/tik-tok-china-trump/}{阅读简体中文版}\href{https://cn.nytimes3xbfgragh.onion/opinion/20200804/tik-tok-china-trump/zh-hant/}{閱讀繁體中文版}

Yes, President Trump \emph{can} ban TikTok in the United States. The
problem? This won't actually stop the digital deluge from China.

In fact, in the perfect example of someone playing checkers against
someone playing chess, the outcome of this face-off with the Chinese
government seems destined to blow up in Mr. Trump's face, if he doesn't
move carefully going forward.

Let's be clear: while it sounds plausible in this clown car of a White
House that Trump is motivated by revenge, I don't think he's trying to
bring down Sarah Cooper for her eviscerating mimicry of him on TikTok.
And he's also probably not gunning for the K-pop kids who organized a
TikTok prank to order tickets to his campaign rally in Tulsa and then
ghosted.

And I don't think it has much to do with TikTok's becoming an
interesting counterweight to Mr. Trump's favored social media site,
Facebook, which lets him mostly go unleashed. (It was lost on no one
paying attention that Facebook's leader, Mark Zuckerberg, attacked China
and, by extension, TikTok, last week at a congressional hearing on
antitrust, casting himself as the bold Western defender against the
Asian villains.)

In fact, the president is directionally correct in his effort to thwart
China's ambitions to establish internet hegemony. As I wrote in
\href{https://www.nytimes3xbfgragh.onion/2020/07/17/opinion/tiktok-ban-china.html}{a
column} about using a burner phone when I enjoy TikTok, Trump and other
tech executives, like Mr. Zuckerberg, are right to say that China and
the country's tech companies threaten American users when it comes to
security, data and, more important, influence and propaganda.

But how best to deal with that scary and real issue is a very complex
thing. And over the weekend, the signals from the Trump administration
were all over the place. On Friday, Mr. Trump threatened a ban. Then, he
said that TikTok couldn't be bought by Microsoft, which had been in
talks with its owner, ByteDance.

On Sunday morning, administration officials got on the news shows, all
with different messages. By that night, Microsoft, which had pulled back
from talks to buy TikTok, said it would re-engage in a news release that
read like a hostage letter. It praised President Trump effusively.

Even when he's right, Mr. Trump often manages to snatch defeat from the
jaws of victory --- in this case a safe landing for TikTok. He could
make a mess if he doesn't tread carefully.

``The Trump administration needs to look at the next move,'' said Alex
Stamos, who used to be in charge of stopping foreign incursions at both
Facebook and Yahoo and now is director of the Stanford Internet
Observatory. ``They can block economic activity by TikTok in the U.S.,
but we fortunately don't have a Great Firewall in this country. If they
push too hard, ByteDance can focus on providing TikTok as a side-loaded
Android app and a mobile website, both of which would be impossible for
Trump to block.''

Let me translate Mr. Stamos' geekspeak: If there's a will to make a
TikTok dance video go viral, there's a way.

And that's why the United States should support an open internet that
touts democratic values using sophisticated strategy, smart policy and
large investments in research and innovation, as well as some
well-placed cudgels. But it seems that Mr. Trump would, rather than
governing, prefer to make loud declarations to reporters and on Twitter.

Unlike other executive orders he has issued related to tech --- like his
toothless effort to eliminate immunity protections under the
Communications Decency Act --- Mr. Trump's arsenal here has more bite
and includes stronger national security levers. He could, for example,
block TikTok from app stores in the United States. Or he could bar
American businesses from selling goods and services to it without a
license. In this, TikTok would have little recourse other than to cease
operating through its app.

Such pressure is obviously best used to force a sale, as was done with
the gay dating service Grindr, to a United States company. And that's
why the Microsoft deal to acquire TikTok makes a lot of sense. With its
strong tech security chops, Microsoft is one of the handful of U.S.
companies with experience in managing big and complex platforms (besides
the massive Windows and Office franchises, the company also owns
LinkedIn, Skype and Minecraft).

While there are other American tech giants --- Amazon and Apple spring
to mind --- that could also take on the Chinese security threat, a
Microsoft-owned TikTok could also create a healthy and suspicion-free
rival to Facebook in the social media space. And Microsoft would fix
security issues quicker than taking TikTok public as a U.S. company.

``They should take the Microsoft deal,'' Mr. Stamos noted. ``It's the
best outcome for the United States, as Microsoft has one of the best
security teams in tech, as it prevents a fight over the basic freedom of
Americans to use the open Web.''

It remains to be seen whether Mr. Trump will get in the way again, as he
is pushed and pulled by various members of his staff. If he does, it's a
tell --- that the president cares little about actual smart solutions
and more about sound and fury, signifying nothing.

\emph{The Times is committed to publishing}
\href{https://www.nytimes3xbfgragh.onion/2019/01/31/opinion/letters/letters-to-editor-new-york-times-women.html}{\emph{a
diversity of letters}} \emph{to the editor. We'd like to hear what you
think about this or any of our articles. Here are some}
\href{https://help.nytimes3xbfgragh.onion/hc/en-us/articles/115014925288-How-to-submit-a-letter-to-the-editor}{\emph{tips}}\emph{.
And here's our email:}
\href{mailto:letters@NYTimes.com}{\emph{letters@NYTimes.com}}\emph{.}

\emph{Follow The New York Times Opinion section on}
\href{https://www.facebookcorewwwi.onion/nytopinion}{\emph{Facebook}}\emph{,}
\href{http://twitter.com/NYTOpinion}{\emph{Twitter (@NYTopinion)}}
\emph{and}
\href{https://www.instagram.com/nytopinion/}{\emph{Instagram}}\emph{,
and sign up for the}
\href{http://www.nytimes3xbfgragh.onion/newsletters/opiniontoday/}{\emph{Opinion
Today newsletter}}\emph{.}

Advertisement

\protect\hyperlink{after-bottom}{Continue reading the main story}

\hypertarget{site-index}{%
\subsection{Site Index}\label{site-index}}

\hypertarget{site-information-navigation}{%
\subsection{Site Information
Navigation}\label{site-information-navigation}}

\begin{itemize}
\tightlist
\item
  \href{https://help.nytimes3xbfgragh.onion/hc/en-us/articles/115014792127-Copyright-notice}{©~2020~The
  New York Times Company}
\end{itemize}

\begin{itemize}
\tightlist
\item
  \href{https://www.nytco.com/}{NYTCo}
\item
  \href{https://help.nytimes3xbfgragh.onion/hc/en-us/articles/115015385887-Contact-Us}{Contact
  Us}
\item
  \href{https://www.nytco.com/careers/}{Work with us}
\item
  \href{https://nytmediakit.com/}{Advertise}
\item
  \href{http://www.tbrandstudio.com/}{T Brand Studio}
\item
  \href{https://www.nytimes3xbfgragh.onion/privacy/cookie-policy\#how-do-i-manage-trackers}{Your
  Ad Choices}
\item
  \href{https://www.nytimes3xbfgragh.onion/privacy}{Privacy}
\item
  \href{https://help.nytimes3xbfgragh.onion/hc/en-us/articles/115014893428-Terms-of-service}{Terms
  of Service}
\item
  \href{https://help.nytimes3xbfgragh.onion/hc/en-us/articles/115014893968-Terms-of-sale}{Terms
  of Sale}
\item
  \href{https://spiderbites.nytimes3xbfgragh.onion}{Site Map}
\item
  \href{https://help.nytimes3xbfgragh.onion/hc/en-us}{Help}
\item
  \href{https://www.nytimes3xbfgragh.onion/subscription?campaignId=37WXW}{Subscriptions}
\end{itemize}
