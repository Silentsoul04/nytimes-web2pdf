Sections

SEARCH

\protect\hyperlink{site-content}{Skip to
content}\protect\hyperlink{site-index}{Skip to site index}

\href{https://myaccount.nytimes3xbfgragh.onion/auth/login?response_type=cookie\&client_id=vi}{}

\href{https://www.nytimes3xbfgragh.onion/section/todayspaper}{Today's
Paper}

\href{/section/opinion}{Opinion}\textbar{}Biden's Quietly Radical Care
Plan

\url{https://nyti.ms/3fmlXEd}

\begin{itemize}
\item
\item
\item
\item
\item
\end{itemize}

Advertisement

\protect\hyperlink{after-top}{Continue reading the main story}

\href{/section/opinion}{Opinion}

Supported by

\protect\hyperlink{after-sponsor}{Continue reading the main story}

\hypertarget{bidens-quietly-radical-care-plan}{%
\section{Biden's Quietly Radical Care
Plan}\label{bidens-quietly-radical-care-plan}}

The candidate is talking about child care and elder care in the same
breath, and making them part of his economic package. Both changes are
long overdue.

\includegraphics{https://static01.graylady3jvrrxbe.onion/images/2017/08/15/opinion/bryce-covert/bryce-covert-thumbLarge-v2.png}

By Bryce Covert

Contributing Opinion Writer

\begin{itemize}
\item
  Aug. 2, 2020, 6:25 p.m. ET
\item
  \begin{itemize}
  \item
  \item
  \item
  \item
  \item
  \end{itemize}
\end{itemize}

\includegraphics{https://static01.graylady3jvrrxbe.onion/images/2020/08/02/opinion/02Covert/02Covert-articleLarge.jpg?quality=75\&auto=webp\&disable=upscale}

Joe Biden's
\href{https://www.cnbc.com/2020/07/21/biden-to-unveil-775-billion-plan-to-fund-child-care-and-elder-care.html}{recent
policy proposal} to address the country's crisis of care didn't garner
major headlines. There were no haphazardly capitalized Trump tweets in
response, nor congressional Republicans denouncing it as socialism. But
make no mistake: His plan is quietly radical in both its
comprehensiveness and its framing.

Mr. Biden's plan incorporates a lot of ideas that are not his own. His
pledge to give all 3- and 4-year-olds access to preschool? President
Barack Obama
\href{https://obamawhitehouse.archives.gov/the-press-office/2013/02/13/fact-sheet-president-obama-s-plan-early-education-all-americans}{initiated}
an effort to ensure universal preschool in 2013. His promise to help
parents afford child care? It piggybacks on Senator Patty Murray and
Representative Bobby Scott's
\href{https://www.help.senate.gov/imo/media/doc/CCFWFA\%20Fact\%20Sheet\%20116th\%20Congress\%20FINAL.pdf}{Child
Care for Working Families Act.} His argument that caregivers deserve
better pay and more rights? To get there, he says he'd sign the Domestic
Workers Bill of Rights bill
\href{https://www.thenation.com/article/archive/federal-domestic-workers-bill-of-rights-harris-jayapal-labor/}{put
forward} by Senator Kamala Harris and Representative Pramila Jayapal in
2018.

But by bringing all of these pieces together in one place and by talking
about them in the same breath with his other economic policies, he is
pushing this conversation into new territory. No longer is the struggle
to care for our families while earning a living something relegated to
kitchen tables and curtained-off hospital beds. These challenges affect
all of us, rippling throughout the entire economy. And Mr. Biden is the
first presidential candidate to drag them out of the shadows and into
the public conversation in such a sweeping way.

Mr. Biden's plan ties together many kinds of care in one package: care
for young children, care for elderly parents, and care for the sick and
disabled. Each of these tends to get siloed off; as a result, they're
fought for by disparate advocacy groups, whose efforts target varied
solutions. This fragmentation often keeps these constituencies from
coalescing into a single movement.

Elder care is especially neglected in policymaking, despite the fact
that the number of Americans age 65 or older is set to grow by
\href{https://www.census.gov/library/visualizations/2018/comm/historic-first.html}{more
than 90 percent} over the next four decades. Look no further for proof
that we ignore the plight of the elderly than the abysmal way people in
nursing homes have fared in the coronavirus crisis, where deaths
recently passed
\href{https://data.cms.gov/stories/s/COVID-19-Nursing-Home-Data/bkwz-xpvg/}{40,000}.
A
\href{https://www.longtermcarepoll.org/long-term-care-in-america-expectations-and-preferences-for-care-and-caregiving/}{vast
majority} of adults facing old age would prefer to remain in their
homes, but in-home care is usually cost prohibitive or unavailable. Mr.
Biden wants to find ways to make it more accessible and to ensure that
the people who provide this care are paid and treated well.

But it's not just the inclusion of all of these kinds of care --- for
the young, the old and the disabled --- in one platform that breaks new
ground. It's also that Mr. Biden puts these ideas forward as a central
part of his economic package, the heart of his campaign. This is both
smart politics and smart policy.

That adequate child care undergirds the smooth functioning of the rest
of the economy has become increasingly clear over the course of the
pandemic. But it was true long before the virus. The share of American
women in the paid work force has been
\href{http://ftp.iza.org/dp7140.pdf}{falling behind} that in other
developed countries for decades thanks to a lack of investment in child
care.We still haven't made the same conceptual leap with elder care,
however, or care for other family members who aren't children --- though
we should. Those who care for their spouses or parents are also
\href{https://www.urban.org/research/publication/impact-informal-caregiving-older-adults-laborsupply-and-economic-resources}{less
likely to work}, and even those with jobs
\href{http://news.gallup.com/businessjournal/151049/cost-caregiving-economy.aspx}{miss
over a week of work} each year on average because of these
responsibilities; the lost productivity costs more than \$28 billion a
year. Imagine how much more productive we could all be if we could
obtain and afford quality care for our children, parents, spouses and
siblings.

Mr. Biden didn't simply drag out a mom to talk about this issue; he
introduced his proposal by talking about himself. ``I was a single
parent for five years after my wife and daughter were killed and my two
boys were badly injured,''
\href{https://www.cnbc.com/2020/07/21/biden-to-unveil-775-billion-plan-to-fund-child-care-and-elder-care.html}{he
said}, referring to a 1972 car accident. ``We cared for our parents at
the end. My dad was months in hospice in our home, same with my mom.''
He also talked about caring for his son Beau before he died of brain
cancer in 2015. This matters, because child care in particular has
fallen all too often into the culture war over women's roles at home and
at work, neglecting the fact that fathers --- and men generally --- also
have family members who need care.

Care is an issue that comes for almost all of us at some point in our
lives. Even if we don't decide to have children, we may very well find
ourselves tending to a sick spouse or an aging parent. But very few of
us are adequately prepared to shoulder the responsibility all on our
own. And yet we have an economy, and a policy apparatus, that completely
ignores this enormous burden.

While you might try to tell a father struggling to find a child care
spot for his child, let alone pay for it, that he should have thought of
this before he decided to have children --- as if the optimal way to
design our society is to make having children a luxury --- it's hard to
argue that if you can't afford decent and dignified care for your aging
parent, you should have chosen not to have a parent get old. Binding
these things together makes it clear: We all have a responsibility to
make sure the vulnerable are adequately cared for.

With his announcement last week, Mr. Biden turned up the volume on
something usually discussed in the fearful whispers of daughters worried
about their mothers on the verge of dementia, or the hushed arguments
parents have late at night about how to afford preschool. Mr. Biden put
it squarely at the center of what he would do if he enters the White
House. It was a remarkable moment, and one that could change the way we
think about care for the long term.

Bryce Covert is a contributor at The Nation and a contributing Opinion
writer.

\emph{The Times is committed to publishing}
\href{https://www.nytimes3xbfgragh.onion/2019/01/31/opinion/letters/letters-to-editor-new-york-times-women.html}{\emph{a
diversity of letters}} \emph{to the editor. We'd like to hear what you
think about this or any of our articles. Here are some}
\href{https://help.nytimes3xbfgragh.onion/hc/en-us/articles/115014925288-How-to-submit-a-letter-to-the-editor}{\emph{tips}}\emph{.
And here's our email:}
\href{mailto:letters@NYTimes.com}{\emph{letters@NYTimes.com}}\emph{.}

\emph{Follow The New York Times Opinion section on}
\href{https://www.facebookcorewwwi.onion/nytopinion}{\emph{Facebook}}\emph{,}
\href{http://twitter.com/NYTOpinion}{\emph{Twitter (@NYTopinion)}}
\emph{and}
\href{https://www.instagram.com/nytopinion/}{\emph{Instagram}}\emph{.}

Advertisement

\protect\hyperlink{after-bottom}{Continue reading the main story}

\hypertarget{site-index}{%
\subsection{Site Index}\label{site-index}}

\hypertarget{site-information-navigation}{%
\subsection{Site Information
Navigation}\label{site-information-navigation}}

\begin{itemize}
\tightlist
\item
  \href{https://help.nytimes3xbfgragh.onion/hc/en-us/articles/115014792127-Copyright-notice}{©~2020~The
  New York Times Company}
\end{itemize}

\begin{itemize}
\tightlist
\item
  \href{https://www.nytco.com/}{NYTCo}
\item
  \href{https://help.nytimes3xbfgragh.onion/hc/en-us/articles/115015385887-Contact-Us}{Contact
  Us}
\item
  \href{https://www.nytco.com/careers/}{Work with us}
\item
  \href{https://nytmediakit.com/}{Advertise}
\item
  \href{http://www.tbrandstudio.com/}{T Brand Studio}
\item
  \href{https://www.nytimes3xbfgragh.onion/privacy/cookie-policy\#how-do-i-manage-trackers}{Your
  Ad Choices}
\item
  \href{https://www.nytimes3xbfgragh.onion/privacy}{Privacy}
\item
  \href{https://help.nytimes3xbfgragh.onion/hc/en-us/articles/115014893428-Terms-of-service}{Terms
  of Service}
\item
  \href{https://help.nytimes3xbfgragh.onion/hc/en-us/articles/115014893968-Terms-of-sale}{Terms
  of Sale}
\item
  \href{https://spiderbites.nytimes3xbfgragh.onion}{Site Map}
\item
  \href{https://help.nytimes3xbfgragh.onion/hc/en-us}{Help}
\item
  \href{https://www.nytimes3xbfgragh.onion/subscription?campaignId=37WXW}{Subscriptions}
\end{itemize}
