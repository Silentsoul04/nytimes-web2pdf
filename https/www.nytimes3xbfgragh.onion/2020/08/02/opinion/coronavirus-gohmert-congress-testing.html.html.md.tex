Sections

SEARCH

\protect\hyperlink{site-content}{Skip to
content}\protect\hyperlink{site-index}{Skip to site index}

\href{https://myaccount.nytimes3xbfgragh.onion/auth/login?response_type=cookie\&client_id=vi}{}

\href{https://www.nytimes3xbfgragh.onion/section/todayspaper}{Today's
Paper}

\href{/section/opinion}{Opinion}\textbar{}Why Are So Many People on
Capitol Hill Getting Infected?

\url{https://nyti.ms/2BQjtQD}

\begin{itemize}
\item
\item
\item
\item
\item
\item
\end{itemize}

Advertisement

\protect\hyperlink{after-top}{Continue reading the main story}

\href{/section/opinion}{Opinion}

Supported by

\protect\hyperlink{after-sponsor}{Continue reading the main story}

\hypertarget{why-are-so-many-people-on-capitol-hill-getting-infected}{%
\section{Why Are So Many People on Capitol Hill Getting
Infected?}\label{why-are-so-many-people-on-capitol-hill-getting-infected}}

Lawmakers are essential workers. They should start acting like it and
protect themselves and the other people who work for Congress.

By
\href{https://www.nytimes3xbfgragh.onion/interactive/opinion/editorialboard.html}{The
Editorial Board}

The editorial board is a group of opinion journalists whose views are
informed by expertise, research, debate and certain longstanding ****
\href{https://www.nytimes3xbfgragh.onion/interactive/2018/opinion/editorialboard.html}{values}.
It is separate from the newsroom.

\begin{itemize}
\item
  Aug. 2, 2020
\item
  \begin{itemize}
  \item
  \item
  \item
  \item
  \item
  \item
  \end{itemize}
\end{itemize}

\includegraphics{https://static01.graylady3jvrrxbe.onion/images/2020/08/02/opinion/02covid1/merlin_170595843_3e0251a8-5138-400c-b288-cc98229f752e-articleLarge.jpg?quality=75\&auto=webp\&disable=upscale}

Representative Louie Gohmert, Republican of Texas, is not the first
member of Congress to test positive for the coronavirus. At least 14
lawmakers, hailing from both parties and both chambers, are known to
have either tested or been presumed positive for the coronavirus.

But something about the diagnosis of Mr. Gohmert, who has belligerently
flouted public health recommendations such as mask wearing and social
distancing, prompted a
\href{https://www.politico.com/news/2020/07/30/capitol-hill-angry-gohmert-coronavirus-388230}{convulsion
of rage on Capitol Hill}.

From maintenance workers to legislative aides, employees came forward
with anonymous accounts of how the patchwork of precautions --- each
lawmaker's office operates with its own rules --- and cavalier behavior
by some members was endangering the thousands of people who keep the
Capitol complex running. In addition to the lawmakers and members of
their staff who have been infected,
\href{https://www.politico.com/news/2020/07/30/capitol-hill-angry-gohmert-coronavirus-388230}{around
90 workers} in support roles such as the Capitol Police and the
\href{https://www.aoc.gov/about-us/who-we-are}{Architect of the
Capitol}, are known to have contracted the virus.

Many lawmakers are approaching the pandemic with appropriate
seriousness: running skeleton crews in their offices, encouraging masks
and following social distancing guidelines.

Others are not. Some are ignoring public health advice for political
reasons, while others seem to believe the virus cannot touch them.
Representative Clay Higgins, Republican of Louisiana,
\href{https://www.cnn.com/2020/07/29/politics/pelosi-mask-mandate-capitol/index.html}{told
CNN} that wearing a mask was ``part of the dehumanization of the
children of God.''

This sort of denial leads to unnecessary tragedy, as was driven home by
the death of
\href{https://www.nytimes3xbfgragh.onion/2020/07/30/us/politics/herman-cain-dead.html}{Herman
Cain}. The former pizza magnate and Republican presidential candidate
tested positive for the coronavirus nine days after attending President
Trump's June 20 campaign rally in Tulsa, Okla. ---~where he was shown in
a number of photos sitting close to other attendees without a mask.

Elected officials have a particular responsibility both to model
responsible behavior during this pandemic and to take extra precautions
so they don't become super spreaders. Put another way, when a lawmaker
behaves like a ``ding dong,'' as one Republican aide said of Mr.
Gohmert, he puts everyone at risk.

The danger extends far beyond Capitol Hill. Members of Congress have an
essential and unusually public, mobile job. In any given week, hundreds
of members jet back and forth across the country, some to coronavirus
hot spots. Representative
\href{https://thehill.com/homenews/house/509647-texas-rep-kay-granger-self-quarantining-after-sitting-next-to-gohmert-on}{Kay
Granger} is currently self-quarantining after sitting next to a
non-masked Mr. Gohmert on a flight back to Washington from their home
state of Texas on July 26. Representative Raúl Grijalva, Democrat of
Arizona, has been self-quarantining as well, after chairing a hearing on
July 28 that Mr. Gohmert attended, at times unmasked. On Saturday,
\href{https://twitter.com/RepRaulGrijalva/status/1289637013394583553}{Mr.
Grijalva announced} that he had tested positive for the virus. **** Who
knows how many other people Mr. Gohmert may have potentially exposed?

In response to the Gohmert news, congressional leaders reminded members
of the safety protocols already in place. The House speaker, Nancy
Pelosi, went further, tightening the rules on mask wearing to require
face coverings on the chamber floor and inside the House office
buildings. (Mask wearing has been mandatory in
\href{https://www.rollcall.com/2020/06/17/face-masks-required-in-house-committee-hearings-congressional-physician-says/}{committee
hearings} since mid-June.) Those who fail to comply can be denied entry
or
\href{https://www.cnn.com/2020/07/29/politics/pelosi-mask-mandate-capitol/index.html}{removed}
by security.

This is a welcome, if belated, step. But that may not be enough. For one
thing, Mr. Gohmert's experience has led to renewed calls by members and
staff workers to implement a testing regimen on the Hill. That's an
important next step.

In May, as Congress was figuring out how to safely get back up and
running, the idea of testing members was considered as one possible
piece of the puzzle. The White House offered to provide 1,000
rapid-response tests and machines for processing.

Leadership in both chambers
\href{https://www.speaker.gov/newsroom/5220}{rejected the offer}. ``Our
country's testing capacities are continuing to scale up nationwide and
Congress wants to keep directing resources to the front-line facilities
where they can do the most good the most quickly,'' read a joint
statement from Ms. Pelosi and Mitch McConnell, the Senate majority
leader.

Neither leader has changed their position in the intervening months. Mr.
McConnell, who has declined to impose mask requirements for the Senate,
continues to maintain that the current protocols are working.
Understandably, neither leader wants members to look as though they are
affording themselves privileges still not available in most workplaces.

But Congress is not an ordinary workplace. As the attending physician
for the House
\href{https://www.rollcall.com/2020/06/17/face-masks-required-in-house-committee-hearings-congressional-physician-says/}{noted
in June}, ``The Congress has unique risks in that individuals attending
the Congress do not represent a gathering of regional citizens but an
intra-National assembly of individuals traveling from areas of variable
disease activity to assemble in Washington, D.C.''

Senator Roy Blunt, the Missouri Republican, recently characterized
lawmakers, with all their traveling,
``\href{https://www.nytimes3xbfgragh.onion/2020/07/29/us/politics/louie-gohmert-positive-coronavirus-mask.html}{the
perfect petri dish for how you spread a disease}.''

Testing skeptics also cite the logistical challenges of setting up such
a testing program. Who would get tested? Just members? How often would
they be tested? Would it be mandatory? How would it be enforced?

Working out these details would of course be challenging, and no system
would eliminate risk entirely. But one cannot allow the perfect to be
the enemy of the good.

This spring, the House minority leader, Kevin McCarthy, looked into a
preliminary testing plan. One suggestion from the experts he consulted
was to focus on members and staff members scheduled to participate in
hearings or other meetings requiring prolonged contact with others.
Likewise, Hill employees who face a higher risk of exposure probably
ought to be prioritized.

More ambitious plans include testing members every week or two, before
they fly back to their home districts.

Some lawmakers have been publicly pushing testing for months, including
Mr. McCarthy and
\href{https://www.timesfreepress.com/news/local/story/2020/may/06/sen-alexander-congressional-leaders-should-ta/522336/}{Senator
Lamar Alexander}, the Tennessee Republican. Other members are now
joining that call. Steny Hoyer, the House majority leader,
\href{https://www.rollcall.com/2020/07/29/louie-gohmert-coronavirus-capitol/}{said}
in the wake of the Gohmert diagnosis that the question of mandatory
testing ought to be revisited.

Mr. Blunt said he is pursuing the issue as well. ``I've been advocating
for a couple months that we test everybody and start with people who are
traveling,'' he
\href{https://www.rollcall.com/2020/07/29/louie-gohmert-coronavirus-capitol/}{told
reporters}.

Congress members are influential figures, and in this time of crisis
they ought to be leading by example. By taking steps to protect
themselves, their staff members and their constituents, lawmakers can
send a signal about the seriousness of this situation to a confused and
weary public.

\emph{The Times is committed to publishing}
\href{https://www.nytimes3xbfgragh.onion/2019/01/31/opinion/letters/letters-to-editor-new-york-times-women.html}{\emph{a
diversity of letters}} \emph{to the editor. We'd like to hear what you
think about this or any of our articles. Here are some}
\href{https://help.nytimes3xbfgragh.onion/hc/en-us/articles/115014925288-How-to-submit-a-letter-to-the-editor}{\emph{tips}}\emph{.
And here's our email:}
\href{mailto:letters@NYTimes.com}{\emph{letters@NYTimes.com}}\emph{.}

\emph{Follow The New York Times Opinion section on}
\href{https://www.facebookcorewwwi.onion/nytopinion}{\emph{Facebook}}\emph{,}
\href{http://twitter.com/NYTOpinion}{\emph{Twitter (@NYTopinion)}}
\emph{and}
\href{https://www.instagram.com/nytopinion/}{\emph{Instagram}}\emph{.}

Advertisement

\protect\hyperlink{after-bottom}{Continue reading the main story}

\hypertarget{site-index}{%
\subsection{Site Index}\label{site-index}}

\hypertarget{site-information-navigation}{%
\subsection{Site Information
Navigation}\label{site-information-navigation}}

\begin{itemize}
\tightlist
\item
  \href{https://help.nytimes3xbfgragh.onion/hc/en-us/articles/115014792127-Copyright-notice}{©~2020~The
  New York Times Company}
\end{itemize}

\begin{itemize}
\tightlist
\item
  \href{https://www.nytco.com/}{NYTCo}
\item
  \href{https://help.nytimes3xbfgragh.onion/hc/en-us/articles/115015385887-Contact-Us}{Contact
  Us}
\item
  \href{https://www.nytco.com/careers/}{Work with us}
\item
  \href{https://nytmediakit.com/}{Advertise}
\item
  \href{http://www.tbrandstudio.com/}{T Brand Studio}
\item
  \href{https://www.nytimes3xbfgragh.onion/privacy/cookie-policy\#how-do-i-manage-trackers}{Your
  Ad Choices}
\item
  \href{https://www.nytimes3xbfgragh.onion/privacy}{Privacy}
\item
  \href{https://help.nytimes3xbfgragh.onion/hc/en-us/articles/115014893428-Terms-of-service}{Terms
  of Service}
\item
  \href{https://help.nytimes3xbfgragh.onion/hc/en-us/articles/115014893968-Terms-of-sale}{Terms
  of Sale}
\item
  \href{https://spiderbites.nytimes3xbfgragh.onion}{Site Map}
\item
  \href{https://help.nytimes3xbfgragh.onion/hc/en-us}{Help}
\item
  \href{https://www.nytimes3xbfgragh.onion/subscription?campaignId=37WXW}{Subscriptions}
\end{itemize}
