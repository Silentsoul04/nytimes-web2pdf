Sections

SEARCH

\protect\hyperlink{site-content}{Skip to
content}\protect\hyperlink{site-index}{Skip to site index}

\href{https://myaccount.nytimes3xbfgragh.onion/auth/login?response_type=cookie\&client_id=vi}{}

\href{https://www.nytimes3xbfgragh.onion/section/todayspaper}{Today's
Paper}

\href{/section/opinion}{Opinion}\textbar{}When Utility Money Talks

\url{https://nyti.ms/3i3MA2z}

\begin{itemize}
\item
\item
\item
\item
\item
\end{itemize}

Advertisement

\protect\hyperlink{after-top}{Continue reading the main story}

\href{/section/opinion}{Opinion}

Supported by

\protect\hyperlink{after-sponsor}{Continue reading the main story}

\hypertarget{when-utility-money-talks}{%
\section{When Utility Money Talks}\label{when-utility-money-talks}}

Corruption scandals in Ohio and Illinois reveal an unsavory underside to
the politics of energy.

\href{https://www.nytimes3xbfgragh.onion/by/justin-gillis}{\includegraphics{https://static01.graylady3jvrrxbe.onion/images/2018/02/13/opinion/justin-gillis/justin-gillis-thumbLarge-v3.png}}

By \href{https://www.nytimes3xbfgragh.onion/by/justin-gillis}{Justin
Gillis}

Mr. Gillis is a contributing Opinion writer and a former environmental
reporter for The Times.

\begin{itemize}
\item
  Aug. 2, 2020, 12:23 p.m. ET
\item
  \begin{itemize}
  \item
  \item
  \item
  \item
  \item
  \end{itemize}
\end{itemize}

\includegraphics{https://static01.graylady3jvrrxbe.onion/images/2020/08/02/opinion/02gillis-illo/02gillis-illo-articleLarge.jpg?quality=75\&auto=webp\&disable=upscale}

The billion-dollar bailout of one of Ohio's biggest utilities seemed
suspicious from the start. It turns out the F.B.I. was paying attention,
too.

Federal agents recently raided the home of the speaker of the Ohio House
of Representatives, Larry Householder, and charged him with
racketeering. He and his associates are accused of
\href{about:blank}{operating a \$60 million political slush fund} to
elect their candidates, with the money coming from one of the state's
largest electricity companies.

Pause to marvel at that figure: \$60 million. That is a lot of money
washing through state politics, even in a state the size of Ohio.
Prosecutors contend that in return for the cash, Mr. Householder, a
Republican, pushed through a huge bailout of two nuclear plants and
several coal plants that were losing money. He has not commented.

The Ohio arrests came less than a week after the big electricity company
in Illinois, Commonwealth Edison, admitted in federal court
\href{about:blank}{to bribing political figures} in that state and
agreed to pay a \$200 million fine.

No public officials have yet been charged with accepting ComEd's bribes,
but
\href{https://www.npr.org/local/309/2020/07/20/893009659/breaking-down-the-com-ed-patronage-scandal-and-what-s-next-for-michael-madigan}{subpoenas}
have been issued to the offices of Michael J. Madigan, the longtime
speaker of the Illinois House of Representatives. Mr. Madigan, a
Democrat, oversaw passage of legislation sought by the power company,
including measures that effectively allowed it to raise rates to save
its own nuclear power plants. The Illinois governor has called on Mr.
Madigan to resign if charges are proved against him; he denies any
wrongdoing.

Those cases are only the most recent examples in a
\href{https://www.energyandpolicy.org/utility-corruption/}{wave of
utility wrongdoing} that has come to light in recent years across the
nation.

In South Carolina, criminal and civil charges have been filed in the \$9
billion collapse of a nuclear power plant. The Securities and Exchange
Commission has
\href{https://www.wsj.com/articles/sec-charges-south-carolina-companies-executives-in-failed-nuclear-project-case-11582855101}{accused}
former executives of the utility Scana of lying to the public and the
utility's own shareholders about problems with the project, which state
regulators allowed to move forward for years despite exploding costs and
construction snafus. Last month, a former executive of Scana, Stephen
Byrne,
\href{https://www.postandcourier.com/business/former-scana-executive-pleads-guilty-to-charges-tied-to-failed-sc-nuclear-project/article_26e23ca8-c50b-11ea-8377-e7b39854212b.html}{pleaded
guilty} to fraud and agreed to testify for prosecutors as they seek to
bring additional charges.

In New Orleans, the utility Entergy
\href{https://www.nola.com/news/article_f2266bc6-6458-5b14-87d5-c1e628650210.html}{was
caught} hiring actors to show up at City Hall and pretend to be citizens
in favor of a controversial gas-fired power plant; the company was fined
\$5 million. A big Arizona utility, Arizona Public Service, has become
embroiled in repeated political scandals, including pumping millions in
dark money into a
\href{https://www.azcentral.com/story/money/business/energy/2019/03/29/arizona-public-service-admits-spending-millions-2014-corporation-commission-races/3317121002/}{campaign}
to stack the state regulatory board with its lackeys.

Taken together, these and other cases demonstrate that too many power
and gas companies have sought to exercise undue influence over the
governments that nominally control them. Utilities spend lavishly on
campaign contributions, dinners, hunting trips for politicians and more.
They set up fake citizens' groups to support their undertakings. And
they have been known to ply nonprofit community organizations with
``\href{https://www.energyandpolicy.org/strings-attached-how-utilities-use-charitable-giving-to-influence-politics-increase-investor-profits/}{donations}''
to take public stances that favor the utility --- and against the real
interests of the people these organizations ostensibly represent.

The monopoly gas and power companies are lucrative enterprises by their
nature, and their rates are generally under direct government control.
Using money to influence politicians and regulators is nothing new. But
there is reason to be especially alert to it now, because these
companies too often are standing in the way of the switch to clean
energy that the country so desperately needs.

The Ohio case looks to be truly malodorous. Mr. Householder appears to
have won his high office largely because the power company, FirstEnergy,
and its affiliates were funding his political operation under the table,
using a ``nonprofit'' shell corporation that he controlled. That allowed
him to pump huge sums into the campaigns of allied candidates who, after
winning their legislative seats, voted to give him the speakership. Then
they voted in favor of his highest priority, the bailout bill.

Prosecutors claim that \$400,000 of the power company's money went
directly into Mr. Householder's pocket as he was doing the company's
bidding in the Ohio Statehouse. However, much of the money was used to
pay for deceptive advertising to advance the bailout bill and to protect
Mr. Householder and his allies from angry voters. Prosecutors contend
that his operation also hired agents to interfere,
\href{https://www.cleveland.com/open/2019/09/house-bill-6-campaign-worker-charged-over-confrontation-with-repeal-campaign-worker.html}{sometimes
physically}, with a petition drive to repeal the bailout law.

Why was the power company seeking bailouts in the first place? Across
the country, nuclear and coal plants are at risk of closure because they
cannot compete with natural-gas plants and wind and solar farms. The
coal shutdowns are good news for the climate; the nuclear shutdowns are
more problematic, since these plants are among the nation's largest
sources of clean electricity.

Congress could stop the nuclear closures with a big climate bill, but
Mitch McConnell, the Senate majority leader, is a big supporter of the
coal industry and will let no such bill through that body. Until
Congress acts, saving the nuclear plants with state subsidies \emph{may}
be a good idea. But states need to drive a hard bargain in these deals,
paying only as much as the power companies really need and doing so only
after carefully weighing alternatives for cleaning up the grid.

That is not what happened in Ohio. Instead the legislature passed a
malign law that charged ratepayers \$150 million a year to fund the
bailouts, with no credible auditing of how much was really needed. While
the law did throw a minor sop to the solar industry, it also gutted most
of Ohio's standards on energy efficiency and clean energy, which were
weak to begin with.

It is now clear that as the Ohio deal went down, the F.B.I. was all over
the statehouse. Mr. Householder's associates were caught on tape
deciding how to spend the gusher of dark money. Gov. Mike DeWine has
called on the legislature to repeal this crooked bill. In a final act of
disrespect for the people of Ohio, Mr. Householder refused to resign as
speaker; House members were forced to vote him out, which they did on
Thursday, 90 to 0.

For citizens elsewhere, the big message from all these scandals is that
you cannot assume your state government is working in the public
interest as it oversees the energy transition.

The electric and gas companies have huge investments in dirty energy,
and they are fighting to protect those plants from clean competition.
Nuclear plants are nominally on the clean side of the ledger, but
proposals to save them need to be examined with a microscope, since
these can be used to argue against or gut what little clean-energy
legislation we have on the books.

Hearings in Congress are warranted, to see whether new federal laws are
needed to restrain this unethical behavior.

In the meantime, citizens are getting a clearer picture of what they are
up against. They are not just fighting dirty energy --- they are also
fighting the dirty money in politics that keeps it alive.

\emph{The Times is committed to publishing}
\href{https://www.nytimes3xbfgragh.onion/2019/01/31/opinion/letters/letters-to-editor-new-york-times-women.html}{\emph{a
diversity of letters}} \emph{to the editor. We'd like to hear what you
think about this or any of our articles. Here are some}
\href{https://help.nytimes3xbfgragh.onion/hc/en-us/articles/115014925288-How-to-submit-a-letter-to-the-editor}{\emph{tips}}\emph{.
And here's our email:}
\href{mailto:letters@NYTimes.com}{\emph{letters@NYTimes.com}}\emph{.}

\emph{Follow The New York Times Opinion section on}
\href{https://www.facebookcorewwwi.onion/nytopinion}{\emph{Facebook}}\emph{,}
\href{http://twitter.com/NYTOpinion}{\emph{Twitter (@NYTopinion)}}
\emph{and}
\href{https://www.instagram.com/nytopinion/}{\emph{Instagram}}\emph{.}

Advertisement

\protect\hyperlink{after-bottom}{Continue reading the main story}

\hypertarget{site-index}{%
\subsection{Site Index}\label{site-index}}

\hypertarget{site-information-navigation}{%
\subsection{Site Information
Navigation}\label{site-information-navigation}}

\begin{itemize}
\tightlist
\item
  \href{https://help.nytimes3xbfgragh.onion/hc/en-us/articles/115014792127-Copyright-notice}{©~2020~The
  New York Times Company}
\end{itemize}

\begin{itemize}
\tightlist
\item
  \href{https://www.nytco.com/}{NYTCo}
\item
  \href{https://help.nytimes3xbfgragh.onion/hc/en-us/articles/115015385887-Contact-Us}{Contact
  Us}
\item
  \href{https://www.nytco.com/careers/}{Work with us}
\item
  \href{https://nytmediakit.com/}{Advertise}
\item
  \href{http://www.tbrandstudio.com/}{T Brand Studio}
\item
  \href{https://www.nytimes3xbfgragh.onion/privacy/cookie-policy\#how-do-i-manage-trackers}{Your
  Ad Choices}
\item
  \href{https://www.nytimes3xbfgragh.onion/privacy}{Privacy}
\item
  \href{https://help.nytimes3xbfgragh.onion/hc/en-us/articles/115014893428-Terms-of-service}{Terms
  of Service}
\item
  \href{https://help.nytimes3xbfgragh.onion/hc/en-us/articles/115014893968-Terms-of-sale}{Terms
  of Sale}
\item
  \href{https://spiderbites.nytimes3xbfgragh.onion}{Site Map}
\item
  \href{https://help.nytimes3xbfgragh.onion/hc/en-us}{Help}
\item
  \href{https://www.nytimes3xbfgragh.onion/subscription?campaignId=37WXW}{Subscriptions}
\end{itemize}
