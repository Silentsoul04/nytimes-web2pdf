Sections

SEARCH

\protect\hyperlink{site-content}{Skip to
content}\protect\hyperlink{site-index}{Skip to site index}

\href{https://www.nytimes3xbfgragh.onion/section/science}{Science}

\href{https://myaccount.nytimes3xbfgragh.onion/auth/login?response_type=cookie\&client_id=vi}{}

\href{https://www.nytimes3xbfgragh.onion/section/todayspaper}{Today's
Paper}

\href{/section/science}{Science}\textbar{}NASA Astronauts in SpaceX
Capsule Make First Water Landing Since 1975

\url{https://nyti.ms/2DdwHYl}

\begin{itemize}
\item
\item
\item
\item
\item
\end{itemize}

\href{https://www.nytimes3xbfgragh.onion/2020/08/02/science/spacex-astronauts-splashdown.html?action=click\&pgtype=Article\&state=default\&region=TOP_BANNER\&context=storylines_menu}{SpaceX's
Astronaut Trip}

\begin{itemize}
\tightlist
\item
  \href{https://www.nytimes3xbfgragh.onion/2020/08/02/science/spacex-astronauts-splashdown.html?action=click\&pgtype=Article\&state=default\&region=TOP_BANNER\&context=storylines_menu}{`Thanks
  for Flying SpaceX'}
\item
  \href{https://www.nytimes3xbfgragh.onion/2020/05/26/science/spacex-launch-nasa.html?action=click\&pgtype=Article\&state=default\&region=TOP_BANNER\&context=storylines_menu}{Why
  NASA Picked SpaceX}
\item
  \href{https://www.nytimes3xbfgragh.onion/interactive/2020/05/26/science/spacex-nasa.html?action=click\&pgtype=Article\&state=default\&region=TOP_BANNER\&context=storylines_menu}{Inside
  the Capsule}
\item
  \href{https://www.nytimes3xbfgragh.onion/2020/05/27/science/bob-behnken-doug-hurley.html?action=click\&pgtype=Article\&state=default\&region=TOP_BANNER\&context=storylines_menu}{Meet
  the Astronauts}
\end{itemize}

Advertisement

\protect\hyperlink{after-top}{Continue reading the main story}

Supported by

\protect\hyperlink{after-sponsor}{Continue reading the main story}

\hypertarget{nasa-astronauts-in-spacex-capsule-make-first-water-landing-since-1975}{%
\section{NASA Astronauts in SpaceX Capsule Make First Water Landing
Since
1975}\label{nasa-astronauts-in-spacex-capsule-make-first-water-landing-since-1975}}

Bob Behnken and Doug Hurley splashed down on Sunday in the Gulf of
Mexico in the Crew Dragon spacecraft.

\includegraphics{https://static01.graylady3jvrrxbe.onion/images/2020/08/02/video/02vid-spacex-splash/02vid-spacex-splash-videoSixteenByNineJumbo1600.jpg}

\href{https://www.nytimes3xbfgragh.onion/by/kenneth-chang}{\includegraphics{https://static01.graylady3jvrrxbe.onion/images/2018/02/16/multimedia/author-kenneth-chang/author-kenneth-chang-thumbLarge.jpg}}

By \href{https://www.nytimes3xbfgragh.onion/by/kenneth-chang}{Kenneth
Chang}

\begin{itemize}
\item
  Aug. 2, 2020
\item
  \begin{itemize}
  \item
  \item
  \item
  \item
  \item
  \end{itemize}
\end{itemize}

\hypertarget{highlights-from-the-splashdown}{%
\subsubsection{Highlights From the
Splashdown:}\label{highlights-from-the-splashdown}}

\begin{itemize}
\tightlist
\item
  \protect\hyperlink{link-5f1ec4b9}{Astronauts parachute to sea after
  nearly 64-day trip to orbit}
\item
  \protect\hyperlink{link-1125e469}{With a little help, astronauts exit
  the capsule}
\item
  \protect\hyperlink{link-44b2cb75}{Private boats encroached on the
  landing zone as capsule is lifted from sea}
\item
  \protect\hyperlink{link-2ee20506}{What happened as the spacecraft
  began to land?}
\item
  \protect\hyperlink{link-3054d8}{Is it safer to land on water or on
  land?}
\end{itemize}

\hypertarget{astronauts-parachute-to-sea-after-nearly-64-day-trip-to-orbit}{%
\subsection{Astronauts parachute to sea after nearly 64-day trip to
orbit}\label{astronauts-parachute-to-sea-after-nearly-64-day-trip-to-orbit}}

The first astronaut trip to orbit by a private company has splashed down
safely in the Gulf of Mexico. Two astronauts dropped out of orbit in
what was the first water landing by NASA since 1975, when the agency's
crews were still flying in the Apollo modules used for the historic
American moon missions.

Suspended under four giant billowing orange-and-white parachutes, the
Crew Dragon settled into calm waters near Pensacola, Fla. at a gentle
pace of 15 miles per hour. Two small SpaceX boats arrived quickly to
begin the operation to prepare the capsule to be pulled out by the main
recovery ship, where crews will tend to the spacecraft's passengers.

They are two NASA astronauts ---
\href{https://www.nytimes3xbfgragh.onion/2020/05/27/science/bob-behnken-doug-hurley.html}{Robert
L. Behnken and Douglas G. Hurley} --- but the conclusion of their first
voyage of nearly 64 days could be a first step to more people going to
space for a variety of new activities like sightseeing, corporate
research and satellite repair.

This flight of
\href{https://www.nytimes3xbfgragh.onion/interactive/2020/05/26/science/spacex-nasa.html}{the
Crew Dragon capsule was operated by SpaceX}, the rocket company started
by Elon Musk, as part of NASA's efforts to turn over to private
enterprise some things it used to do.

NASA has hired two companies --- SpaceX and Boeing --- to provide
transportation of astronauts to and from the International Space
Station, and SpaceX was the first to be ready to take astronauts to
orbit,
\href{https://www.nytimes3xbfgragh.onion/2020/05/30/science/spacex-nasa-astronauts.html}{launching
Mr. Behnken and Mr. Hurley in May}.

\hypertarget{with-a-little-help-astronauts-exit-the-capsule}{%
\subsection{With a little help, astronauts exit the
capsule}\label{with-a-little-help-astronauts-exit-the-capsule}}

\includegraphics{https://static01.graylady3jvrrxbe.onion/images/2020/08/02/science/02sci-astronauts-chopper/merlin_175248828_3e77d01d-2687-4b20-8ac0-f3fb299a7b7b-articleLarge.jpg?quality=75\&auto=webp\&disable=upscale}

After 4 p.m., the recovery crews opened the hatch and assisted Mr.
Behnken and Mr. Hurley, still in their spacesuits and giving thumbs ups,
to exit the capsule. Mr. Hurley thanked employees of NASA and SpaceX for
their work on the spacecraft and mission.

``You should take a moment to just cherish this day, especially given
all the things that have happened this year,'' Mr. Hurley said as he
exited the Crew Dragon.

He added that he felt ``proud to be a small part of this whole effort,''
to carry people to and from space.

\hypertarget{private-boats-encroached-on-the-landing-zone-as-capsule-is-lifted-from-sea}{%
\subsection{Private boats encroached on the landing zone as capsule is
lifted from
sea}\label{private-boats-encroached-on-the-landing-zone-as-capsule-is-lifted-from-sea}}

\includegraphics{https://static01.graylady3jvrrxbe.onion/images/2020/08/02/multimedia/02xp-boats-pix-sub/02xp-boats-pix-sub-videoSixteenByNine3000.jpg}

Before the recovery ship hoisted the capsule out of the water and
prepared to open the capsule's hatch,
\href{https://www.nytimes3xbfgragh.onion/2020/08/02/us/flag-boat-SpaceX.html}{a
flotilla of small private boats} piloted by onlookers also converged
around the bobbing capsule. At least one of them flew a banner
supporting President Trump.

Personnel aboard SpaceX's boats chased them farther away; there is a
possibility of toxic propellant fumes from the spacecraft thrusters,
which could endanger passengers on other vessels nearby. Detection of
the fumes in fact caused a short delay in opening the hatch to release
the astronauts from the capsule.

``Let's just keep everyone safe, no reason to rush,'' Mr. Hurley said to
mission control of the delay.

\hypertarget{what-happened-as-the-spacecraft-began-to-land}{%
\subsection{What happened as the spacecraft began to
land?}\label{what-happened-as-the-spacecraft-began-to-land}}

On Saturday, the Crew Dragon performed a series of thruster burns to
move away from the space station and then line up with the splashdown
site.

Before leaving orbit on Sunday, the spacecraft jettisoned its bottom
half, known as the trunk, which will no longer be needed. That exposes
the heat shield that protects the capsule and astronauts during
re-entry.

``Oh yeah, we felt it,'' Mr. Hurley said after the maneuver was
confirmed on the ground.

One more thruster burn of about 11 minutes in length was completed after
2 p.m. Eastern time, which will cause the capsule to drop out of orbit,
headed toward its landing site at sea.

Following further maneuvers, the capsule's parachutes brought the
spacecraft to the surface of the Gulf of Mexico.

Image

Doug Hurley, left, and Bob Behnken were suited up on Sunday in
preparation for the splashdown of the Crew Dragon capsule carrying them
home to Earth from the International Space Station.Credit...SpaceX, via
Associated Press

\hypertarget{is-it-safer-to-land-on-water-or-on-land}{%
\subsection{Is it safer to land on water or on
land?}\label{is-it-safer-to-land-on-water-or-on-land}}

Spacecraft can safely return to Earth in either environment.

During the 1960s and 1970s, NASA's Mercury, Gemini and Apollo capsules
all splashed down in the ocean while Soviet capsules all ended their
trips on land. Russia's current Soyuz capsules continue to make ground
landings, as do China's astronaut-carrying Shenzhou capsules.

The last water landing by NASA astronauts occurred in July 1975 at
\href{https://www.nasa.gov/mission_pages/apollo-soyuz/index.html}{the
end of the Apollo-Soyuz mission}, during which an American crew aboard
an Apollo module docked in orbit with two Soviet astronauts aboard a
Soyuz capsule.

While the crew splashed down safely, a problem with the Apollo
spacecraft during re-entry caused
\href{https://history.nasa.gov/apollo/apsoyhist.html}{fumes from rocket
propellant to fill the capsule}, causing breathing and eye problems for
the astronauts.

When Boeing's Starliner capsule begins carrying crews to the space
station, it will return on land, in New Mexico. SpaceX had originally
planned for the Crew Dragon to do ground landings, but decided that
water landings, employed for the earlier version of Dragon for taking
cargo, simplified the development of the capsule. Elon Musk, the founder
of SpaceX, further explained the reasoning on Twitter early on Sunday:

\hypertarget{what-do-astronauts-experience-during-a-water-landing}{%
\subsection{What do astronauts experience during a water
landing?}\label{what-do-astronauts-experience-during-a-water-landing}}

Returning from the free-fall environment of orbit to the normal forces
of gravity on Earth is often disorienting for astronauts. A water
landing adds the possibility of seasickness.

During a news conference on Friday, Mr. Hurley said he had read some of
the reports by the Skylab astronauts. ``There was some challenges post
splashdown,'' he said. ``Folks didn't feel well, and you know, that is
the way it is with a water landing, even if you're not deconditioned
like we're going to be.''

Mr. Hurley acknowledged that vomiting would not be unexpected.

``There are bags if you need them, and we'll have those handy,'' he
said. ``We'll probably have some towels handy as well. And you know, if
that needs to happen, it certainly wouldn't be the first time that
that's happened in a space vehicle.''

But not long after the splash down, he asked SpaceX's mission
controllers to tell flight surgeons monitoring their health that,
``we're doing pretty good so far.''

\hypertarget{what-have-the-astronauts-been-doing-since-they-undocked}{%
\subsection{What have the astronauts been doing since they
undocked?}\label{what-have-the-astronauts-been-doing-since-they-undocked}}

\includegraphics{https://static01.graylady3jvrrxbe.onion/images/2020/08/02/science/02sci-astronauts/merlin_175216431_287c3773-da04-4003-9cbb-1318c4e80f6e-videoSixteenByNineJumbo1600.jpg}

Sleeping mostly.

Following a series of thruster firings to put the spacecraft on track
with the landing site, the astronauts'
\href{https://twitter.com/NASA/status/1289245570565992449}{schedule
included}a full night of rest. The capsule even
\href{https://twitter.com/Commercial_Crew/status/1289744080214212609}{completed
one of its maneuvers} while the astronauts are supposed to be sleeping.

Any return journey that exceeds six hours has to be long enough for the
crew to get some sleep between undocking and splashdown, Daniel Huot, a
NASA spokesman, said in an email.

Otherwise, because of the extended process that leads up to undocking,
the crew would end up working more than 20 hours straight, ``which is
not safe for dynamic operations like water splashdown and recovery,''
Mr. Huot said.

On Sunday morning, the astronauts were greeted with a wake up message
sent from Earth by their two sons.

Just before noon, the
\href{https://www.nytimes3xbfgragh.onion/2020/05/27/fashion/SpaceX-Dragon-Suits.html}{astronauts
began to put their SpaceX spacesuits} back on as they completed
preparations for landing.

\hypertarget{why-is-the-return-trip-an-important-part-of-the-crew-dragons-first-flight}{%
\subsection{Why is the return trip an important part of the Crew
Dragon's first
flight?}\label{why-is-the-return-trip-an-important-part-of-the-crew-dragons-first-flight}}

After launch, re-entry through Earth's atmosphere is the second most
dangerous phase of spaceflight. Friction of air rushing past will heat
the bottom of the capsule to about 3,500 degrees Fahrenheit. A test
flight of the Crew Dragon last year successfully splashed down, so
engineers know the system works.

A successful conclusion to the trip opens the door to more people flying
to space. Some companies have already announced plans to use Crew
Dragons to lift wealthy tourists to orbit.

In the past, NASA astronauts launched on spacecraft like the Saturn 5
moon rocket and the space shuttles that NASA itself operated. After the
retirement of the space shuttles in 2011, NASA had to rely on Russia,
buying seats on the Soyuz capsules for trips to and from orbit.

Under the Obama administration, NASA hired two companies, SpaceX and
Boeing, to build spacecraft to take astronauts to the space station.
NASA financed much of the work to develop the spacecraft but will now
buy rides at fixed prices. For SpaceX, the trip by Mr. Behnken and Mr.
Hurley --- the first launch of astronauts from American soil since the
last space shuttle flight --- was the last major demonstration needed
before NASA officially certifies that the Crew Dragon is ready to begin
regular flights.

\href{https://www.nytimes3xbfgragh.onion/interactive/2020/05/26/science/spacex-nasa.html}{}

\includegraphics{https://static01.graylady3jvrrxbe.onion/images/2020/05/26/us/spacex-nasa-promo-1590499638707/spacex-nasa-promo-1590499638707-articleLarge-v2.jpg}

\hypertarget{now-boarding-spacexs-new-ride-to-orbit-for-nasa-astronauts}{%
\subsection{Now Boarding: SpaceX's New Ride to Orbit for NASA
Astronauts}\label{now-boarding-spacexs-new-ride-to-orbit-for-nasa-astronauts}}

The Crew Dragon launched successfully on Saturday.

\hypertarget{who-are-the-astronauts}{%
\subsection{Who are the astronauts?}\label{who-are-the-astronauts}}

The astronauts are
\href{https://www.nytimes3xbfgragh.onion/2020/05/27/science/bob-behnken-doug-hurley.html}{Robert
L. Behnken and Douglas G. Hurley}, who have been friends and colleagues
since both were selected by NASA to be astronauts in 2000.

Both men have backgrounds as military test pilots and each has flown
twice before on space shuttle missions, although this is the first time
they have worked together on a mission. Mr. Hurley flew on the space
shuttle's final mission in 2011.

In 2015, they were among the astronauts chosen to work with Boeing and
SpaceX on the commercial space vehicles that the companies were
developing. In 2018, they were assigned to the first SpaceX flight.

\hypertarget{what-have-the-astronauts-been-doing-aboard-the-space-station}{%
\subsection{What have the astronauts been doing aboard the space
station?}\label{what-have-the-astronauts-been-doing-aboard-the-space-station}}

Originally, the mission was to last only up to two weeks, but Mr.
Behnken and Mr. Hurley ended up with a longer and busier stay at the
space station. Because of repeated delays by SpaceX and Boeing, NASA
ended up short-handed, with only one astronaut, Christopher J. Cassidy,
aboard the space station when the Crew Dragon and its two passengers
docked.

They stayed two months, helping Mr. Cassidy with space station chores.
Mr. Behnken and Mr. Cassidy performed four spacewalks to complete the
installation of new batteries on the space station. Mr. Hurley helped by
operating the station's robotic arm.

The men have also been contributing to science experiments in low earth
orbit. They assisted in
\href{https://www.nasa.gov/mission_pages/station/research/behnken-hurley-science-scrapbook}{a
study of water droplet formation} in the low gravity environment of the
space station using a shower head, and another that used fruit punch and
foam to look at \href{https://www.youtube.com/watch?v=2Dzx6b6vSK4}{how
to manage fluids in space}. They also
\href{https://www.nasa.gov/mission_pages/station/research/behnken-hurley-science-scrapbook}{helped
install new equipment inside the station} that will be used in future
scientific research.

Mr. Cassidy will remain aboard the station with two Russian astronauts,
Anatoly Ivanishin and Ivan Vagner. All three are to stay
\href{https://www.nasa.gov/sites/default/files/atoms/files/exp-63-summary.pdf}{on
board through October} when another crew of one American and two Russian
astronauts
\href{https://www.nasa.gov/press-release/nasa-astronaut-kate-rubins-crewmates-to-discuss-upcoming-spaceflight}{will
replace them}.

\hypertarget{when-are-the-next-crew-dragon-flights-and-who-will-they-carry}{%
\subsection{When are the next Crew Dragon flights, and who will they
carry?}\label{when-are-the-next-crew-dragon-flights-and-who-will-they-carry}}

The first operational flight of the Crew Dragon will launch no earlier
than late September. It will take three NASA astronauts --- Michael S.
Hopkins, Victor J. Glover and Shannon Walker --- and one Japanese
astronaut, Soichi Noguchi, to the space station.

The second operational flight, tentatively scheduled for February 2021,
will carry two NASA astronauts, Robert S. Kimbrough and K. Megan
McArthur; Akihiko Hoshide of Japan; and Thomas Pesquet of the European
Space Agency.

Ms. McArthur is married to Mr. Behnken.

\hypertarget{which-other-companies-are-building-rides-to-space}{%
\subsection{Which other companies are building rides to
space?}\label{which-other-companies-are-building-rides-to-space}}

SpaceX's counterpart in the commercial crew program, Boeing, will almost
certainly not be able to launch astronauts until next year. An uncrewed
flight last year suffered significant software errors, which prevented
the spacecraft from achieving its primary goal of docking at the space
station, and could have led to a loss of the spacecraft during its
orbital test. Boeing will now repeat the uncrewed test later this year
before putting astronauts aboard.

Closer to Earth, a couple of companies --- Blue Origin and Virgin
Galactic --- are developing spacecraft that provide brief up-and-down
tourist rides to the edge of space, although neither is capable of
making the trip to the space station, or even to orbit.

Advertisement

\protect\hyperlink{after-bottom}{Continue reading the main story}

\hypertarget{site-index}{%
\subsection{Site Index}\label{site-index}}

\hypertarget{site-information-navigation}{%
\subsection{Site Information
Navigation}\label{site-information-navigation}}

\begin{itemize}
\tightlist
\item
  \href{https://help.nytimes3xbfgragh.onion/hc/en-us/articles/115014792127-Copyright-notice}{©~2020~The
  New York Times Company}
\end{itemize}

\begin{itemize}
\tightlist
\item
  \href{https://www.nytco.com/}{NYTCo}
\item
  \href{https://help.nytimes3xbfgragh.onion/hc/en-us/articles/115015385887-Contact-Us}{Contact
  Us}
\item
  \href{https://www.nytco.com/careers/}{Work with us}
\item
  \href{https://nytmediakit.com/}{Advertise}
\item
  \href{http://www.tbrandstudio.com/}{T Brand Studio}
\item
  \href{https://www.nytimes3xbfgragh.onion/privacy/cookie-policy\#how-do-i-manage-trackers}{Your
  Ad Choices}
\item
  \href{https://www.nytimes3xbfgragh.onion/privacy}{Privacy}
\item
  \href{https://help.nytimes3xbfgragh.onion/hc/en-us/articles/115014893428-Terms-of-service}{Terms
  of Service}
\item
  \href{https://help.nytimes3xbfgragh.onion/hc/en-us/articles/115014893968-Terms-of-sale}{Terms
  of Sale}
\item
  \href{https://spiderbites.nytimes3xbfgragh.onion}{Site Map}
\item
  \href{https://help.nytimes3xbfgragh.onion/hc/en-us}{Help}
\item
  \href{https://www.nytimes3xbfgragh.onion/subscription?campaignId=37WXW}{Subscriptions}
\end{itemize}
