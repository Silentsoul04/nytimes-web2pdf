Sections

SEARCH

\protect\hyperlink{site-content}{Skip to
content}\protect\hyperlink{site-index}{Skip to site index}

\href{https://www.nytimes3xbfgragh.onion/section/sports/baseball}{Baseball}

\href{https://myaccount.nytimes3xbfgragh.onion/auth/login?response_type=cookie\&client_id=vi}{}

\href{https://www.nytimes3xbfgragh.onion/section/todayspaper}{Today's
Paper}

\href{/section/sports/baseball}{Baseball}\textbar{}`Everyone Can Do
Better': Baseball Searches for Blame Amid Outbreaks

\url{https://nyti.ms/31aAYUC}

\begin{itemize}
\item
\item
\item
\item
\item
\item
\end{itemize}

Advertisement

\protect\hyperlink{after-top}{Continue reading the main story}

Supported by

\protect\hyperlink{after-sponsor}{Continue reading the main story}

\hypertarget{everyone-can-do-better-baseball-searches-for-blame-amid-outbreaks}{%
\section{`Everyone Can Do Better': Baseball Searches for Blame Amid
Outbreaks}\label{everyone-can-do-better-baseball-searches-for-blame-amid-outbreaks}}

Major League Baseball's commissioner put the onus on players to behave
more responsibly, but some players and health experts see weaknesses in
the league's safety protocols.

\includegraphics{https://static01.graylady3jvrrxbe.onion/images/2020/08/02/sports/02mlb/merlin_175096953_c4eb53d3-68c2-4cda-9770-cbfd9a1f9b8a-articleLarge.jpg?quality=75\&auto=webp\&disable=upscale}

\href{https://www.nytimes3xbfgragh.onion/by/james-wagner}{\includegraphics{https://static01.graylady3jvrrxbe.onion/images/2018/06/13/multimedia/author-james-wagner/author-james-wagner-thumbLarge.jpg}}

By \href{https://www.nytimes3xbfgragh.onion/by/james-wagner}{James
Wagner}

\begin{itemize}
\item
  Published Aug. 2, 2020Updated Aug. 3, 2020
\item
  \begin{itemize}
  \item
  \item
  \item
  \item
  \item
  \item
  \end{itemize}
\end{itemize}

With the Major League Baseball season becoming more precarious seemingly
by the day amid a slow but steady stream of new
\href{https://www.nytimes3xbfgragh.onion/2020/08/03/sports/baseball/mlb-coronavirus-outbreak.html}{coronavirus
cases among the teams}, the league's commissioner, Rob Manfred, issued
something of a rallying cry.

``We are playing,'' Manfred
\href{https://twitter.com/karlravechespn/status/1289628827157581830}{told
ESPN} on Saturday. ``The players need to be better, but I am not a
quitter in general and there is no reason to quit now. We have had to be
fluid, but it is manageable.''

Those words bothered some players in the sport and some health experts
outside it. Two outbreaks --- 20 cases among the Miami Marlins and
\href{https://www.nytimes3xbfgragh.onion/2020/08/01/sports/baseball/coronavirus-cardinals.html}{six
among the St. Louis Cardinals}, as of Sunday afternoon --- less than two
weeks into the season have
\href{https://www.nytimes3xbfgragh.onion/2020/07/30/sports/baseball/phillies-blue-jays-postponed-coronavirus.html}{wreaked
havoc on the schedules of eight teams} and raised questions about
M.L.B.'s protocols and the role of the players' individual
responsibilities in stopping the virus.

In saying the games would go on, Manfred thrust the onus on the players.

``I don't know Rob's situation, and I don't want to put my foot in my
mouth on that one,'' Chicago Cubs pitcher Jon Lester
\href{https://theathletic.com/1968292/2020/08/02/at-a-time-like-this-jon-lester-isnt-going-to-blame-or-second-guess-the-players/}{told
reporters} on Saturday. ``But I do know we --- not only the players, but
families --- are making sacrifices day in and day out. I don't want to
put my foot in my mouth. I guess I'll stop there.''

M.L.B.'s 113-page operating manual for the 2020 season, which was
crafted with input from the players' union, has details on everything
from how a team should travel to proper spacing in the dugout to what to
do if a player tests positive. But it does not explicitly state what
should happen after an outbreak or what the threshold is for postponing
games.

Kathleen Bachynski, an assistant professor of public health at
Muhlenberg College,
\href{https://twitter.com/bachyns/status/1289665507117772800}{took
issue} with Manfred's comments, writing on Twitter that the virus
thrives ``when people insist on sticking with a poor plan to the bitter
end.''

Bachynski said in a phone interview that her biggest concerns were about
the plan itself. She said she was shocked when she read that the M.L.B.
manual did not detail steps for the league and players to follow after
an outbreak.

``You can certainly say you want to encourage safe personal behaviors,''
she said. ``There's nothing wrong with that. But that doesn't make up
for not having a written policy. And the responsibility has to be on the
league to provide safe conditions to play in.

``I just don't think it makes sense to put the burden of all of this on
the players.''

M.L.B. stopped the Cardinals, who registered their first two positive
cases on Friday, from playing the Brewers, and put the brakes on the
Phillies, who have not had a player test positive, for seven days
because of their exposure to the Marlins.

But on July 26, the fourth day of the season,
\href{https://www.nytimes3xbfgragh.onion/2020/07/27/sports/baseball/marlins-game-canceled.html}{the
Marlins played the Phillies} despite knowing they had four players test
positive, a decision that has been questioned by some non-Marlins
players and health experts.

``It doesn't matter how quick the testing turnaround time is if you're
not taking appropriate actions based on the results of those tests,''
Bachynski said.

\includegraphics{https://static01.graylady3jvrrxbe.onion/images/2020/08/02/sports/02mlb-dodgers/merlin_175224090_96532903-004d-4bd5-b09e-9fce66b7fb92-articleLarge.jpg?quality=75\&auto=webp\&disable=upscale}

Since their series against the Marlins a week ago, the Phillies have had
three staff members test positive. But M.L.B. said on Saturday that it
appeared that two of those tests were false positives, and ``it is
unclear if the third individual contracted Covid-19 from Marlins players
and staff based on the timing of the positive test.'' The Phillies are
set to resume play on Monday against
\href{https://www.nytimes3xbfgragh.onion/2020/07/29/sports/baseball/yankees-schedule.html}{the
Yankees}.

``The protocols are a series of little things that people need to do,''
Manfred told
\href{https://apnews.com/eff9d34788720e0b3d6df53c2779fd01}{The
Associated Press} on Saturday. ``We've had some problems. In order to be
better, it's another series of little things. I think it's peer
pressure. I think it's players taking personal responsibility.''

\hypertarget{the-games-resume}{%
\subsubsection{The Games Resume}\label{the-games-resume}}

\hypertarget{sports-and-the-virus}{%
\paragraph{Sports and the Virus}\label{sports-and-the-virus}}

Updated Aug. 4, 2020

Here's what's happening as the world of sports slowly comes back to
life:

\begin{itemize}
\item
  \begin{itemize}
  \tightlist
  \item
    As the virus spreads through baseball,
    \href{https://www.nytimes3xbfgragh.onion/2020/08/03/sports/baseball/mlb-coronavirus-outbreak.html?action=click\&pgtype=Article\&state=default\&region=MAIN_CONTENT_2\&context=storylines_keepup}{so
    does frustration}. Series have been postponed, teams have been
    quarantined and road trips have been rerouted in a season that has
    been defined above all by its precariousness.
  \item
    On all but the two biggest courts, automated line calls
    \href{https://www.nytimes3xbfgragh.onion/2020/08/03/sports/tennis/us-open-hawkeye-line-judges.html?action=click\&pgtype=Article\&state=default\&region=MAIN_CONTENT_2\&context=storylines_keepup}{will
    replace human judges} at the U.S. Open to reduce the number of
    people on site during the pandemic.
  \item
    Mets star Yoenis Cespedes is healthy, but
    \href{https://www.nytimes3xbfgragh.onion/2020/08/02/sports/baseball/Yoenis-cespedes-opt-out-rule.html?action=click\&pgtype=Article\&state=default\&region=MAIN_CONTENT_2\&context=storylines_keepup}{has
    decided to opt out} of the 2020 baseball season for Covid-related
    reasons.
  \end{itemize}
\end{itemize}

He said he also had a ``constructive conversation'' with the players'
union chief, Tony Clark, on Friday.

Several players have decided to opt out of the season after seeing the
virus infiltrate team rosters.
\href{https://www.nytimes3xbfgragh.onion/2020/08/02/sports/baseball/Yoenis-cespedes-opt-out-rule.html}{Yoenis
Cespedes of the Mets on Sunday} became the fourth player to opt out
since the Marlins' outbreak, joining more than a dozen who had made the
decision before opening day.

After news of the Marlins' outbreak surfaced, David Price, the Los
Angeles Dodgers pitcher who opted out of the season before it began,
tweeted: ``Part of the reason I'm at home right now is because players
health wasn't being put first. I can see that hasn't changed.''

Among those who are still playing, there is some acknowledgment that
both protocols and personal behavior can improve.

``Everyone can do better. This is a learning process,'' Yankees
outfielder Giancarlo Stanton said on Sunday. ``We learn things every day
from how to do better with this on both sides. Both sides can be better.
And by the time the season is over, there can still be improvements of
what we could've done better.''

From the start, M.L.B. and the players' union recognized the season's
policies would evolve. The manual's opening page states that it does not
address every aspect of the season's operations and that additional
guidance may come throughout the year. Last week, M.L.B. informed teams
of tightened regulations, including designating a compliance officer for
each club. More changes could be coming.

The Marlins were found to have been lax in following protocols: At least
some of them did not strictly adhere to all of the rules. But players on
many teams have been spotted high-fiving or spitting or getting too
close too often in the dugout --- all in violation of the manual.

Even diligent teams and conscientious players are worried about
contracting the virus unknowingly while in their community or traveling.
The Dodgers have gone above and beyond the M.L.B. rules, deciding as a
team to require all players to wear face coverings in the dugout and
limit when coaches can be there during games,
\href{https://twitter.com/alannarizzo/status/1289371846425550848}{according
to third baseman Justin Turner}.

While high-fives or fist bumps are lower-risk activities and health
experts believe players and staff members are more likely to be infected
away from the stadium, Bachynski said she worried that those smaller
lapses suggested a larger culture of ignoring the rules even away from
the field.

``If your leadership is showing how important it is and you've got the
front office, like we have here, taking it very seriously, then that'll
trickle down to the players taking it seriously,'' said Yankees pitcher
James Paxton, who sits on the players' union executive subcommittee.

Mike Zunino, a catcher for the Tampa Bay Rays who has two young
children, said he thought often about not continuing to play this season
after the Marlins' and Cardinals' outbreaks.

``I'd be lying if I told you it didn't cross your mind every day when
you see positive tests come out,'' he said, adding later: ``I have a lot
of trust in the team here, the guys, we're doing stuff the right way.
It's a real conversation I have every day just to see how the dynamic of
the league is going.''

While he is not opting out, the Boston Red Sox' top pitcher, Eduardo
Rodriguez,
\href{https://www.masslive.com/redsox/2020/08/eduardo-rodriguez-boston-red-sox-starter-out-for-season-with-heart-ailment.html}{will
not play this season} because he is still recovering from myocarditis,
the inflammation of his heart, which he developed after contracting the
virus before the season.

``Hopefully if somebody does test positive, we don't just immediately
point the finger that they're doing something wrong,'' Lester told
reporters on Saturday. ``They could have gone to Target and needed soap
and got it there. Hopefully, we can get away from pointing fingers
immediately to the bad side of things. Hopefully, the real stories come
out, and maybe they're good. And if they are bad, then that sucks ---
it's unfortunate that guys made bad decisions on that.''

Advertisement

\protect\hyperlink{after-bottom}{Continue reading the main story}

\hypertarget{site-index}{%
\subsection{Site Index}\label{site-index}}

\hypertarget{site-information-navigation}{%
\subsection{Site Information
Navigation}\label{site-information-navigation}}

\begin{itemize}
\tightlist
\item
  \href{https://help.nytimes3xbfgragh.onion/hc/en-us/articles/115014792127-Copyright-notice}{©~2020~The
  New York Times Company}
\end{itemize}

\begin{itemize}
\tightlist
\item
  \href{https://www.nytco.com/}{NYTCo}
\item
  \href{https://help.nytimes3xbfgragh.onion/hc/en-us/articles/115015385887-Contact-Us}{Contact
  Us}
\item
  \href{https://www.nytco.com/careers/}{Work with us}
\item
  \href{https://nytmediakit.com/}{Advertise}
\item
  \href{http://www.tbrandstudio.com/}{T Brand Studio}
\item
  \href{https://www.nytimes3xbfgragh.onion/privacy/cookie-policy\#how-do-i-manage-trackers}{Your
  Ad Choices}
\item
  \href{https://www.nytimes3xbfgragh.onion/privacy}{Privacy}
\item
  \href{https://help.nytimes3xbfgragh.onion/hc/en-us/articles/115014893428-Terms-of-service}{Terms
  of Service}
\item
  \href{https://help.nytimes3xbfgragh.onion/hc/en-us/articles/115014893968-Terms-of-sale}{Terms
  of Sale}
\item
  \href{https://spiderbites.nytimes3xbfgragh.onion}{Site Map}
\item
  \href{https://help.nytimes3xbfgragh.onion/hc/en-us}{Help}
\item
  \href{https://www.nytimes3xbfgragh.onion/subscription?campaignId=37WXW}{Subscriptions}
\end{itemize}
