Sections

SEARCH

\protect\hyperlink{site-content}{Skip to
content}\protect\hyperlink{site-index}{Skip to site index}

\href{https://www.nytimes3xbfgragh.onion/section/opinion/sunday}{Sunday
Review}

\href{https://myaccount.nytimes3xbfgragh.onion/auth/login?response_type=cookie\&client_id=vi}{}

\href{https://www.nytimes3xbfgragh.onion/section/todayspaper}{Today's
Paper}

\href{/section/opinion/sunday}{Sunday Review}\textbar{}To Break a Horse,
and a Woman

\href{https://nyti.ms/39N57x3}{https://nyti.ms/39N57x3}

\begin{itemize}
\item
\item
\item
\item
\item
\end{itemize}

Advertisement

\protect\hyperlink{after-top}{Continue reading the main story}

Supported by

\protect\hyperlink{after-sponsor}{Continue reading the main story}

news analysis

\hypertarget{to-break-a-horse-and-a-woman}{%
\section{To Break a Horse, and a
Woman}\label{to-break-a-horse-and-a-woman}}

How do prey animals stay safe in a world out to get them? And how would
I?

\includegraphics{https://static01.graylady3jvrrxbe.onion/images/2020/08/02/opinion/sunday/02nir-print/02nir-superJumbo.jpg}

\href{https://www.nytimes3xbfgragh.onion/by/sarah-maslin-nir}{\includegraphics{https://static01.graylady3jvrrxbe.onion/images/2018/06/13/multimedia/author-sarah-maslin-nir/author-sarah-maslin-nir-thumbLarge.jpg}}

By \href{https://www.nytimes3xbfgragh.onion/by/sarah-maslin-nir}{Sarah
Maslin Nir}

Ms. Nir is a Metro reporter.

\begin{itemize}
\item
  Aug. 2, 2020
\item
  \begin{itemize}
  \item
  \item
  \item
  \item
  \item
  \end{itemize}
\end{itemize}

Horses talk, but not like Mr. Ed. A horse may be trained to respond to
``whoa'' and ``giddy up,'' but if you speak to a horse, you're not
speaking \emph{in} horse. They converse mostly through body language,
and their most visible correspondence is a sign language of ears.

Horses can flip and flick their ears 180 degrees. Ears pricked forward
is a horse's smile. Tipped back can mean boredom or displeasure. Pinned
to the skull, that's fury. Ears akimbo, and a horse is daydreaming,
thinking nothing much at all or maybe everything.

In the herd, horses turn and wheel across a field like larks, guiding
each other with shoulder and flank, ear and eye. Observe the dynamics
--- the solicitousness of the studs, the defiance of the mares, the
submission of the spindly foals. They are communicating nonverbally in a
clear system of gestures.

Monty Roberts, the legendary trainer and best-selling author known as
the ``man who listens to horses,'' calls this system ``Equus.''

``I've tried to stop calling it a language,'' he said to me. ``We think
way too much about words and alphabets and stuff like that,'' he added.
``That's not horses.''

Mr. Roberts, who is in his 80s, spoke to me from Flag Is Up Farms, his
ranch in Solvang, Calif., which he's run since 1966. He trains problem
horses around the world using a technique he calls Join Up, which,
essentially, asks the horse to work with the rider as a member of its
herd, rather than as its master.

I had reached out to Mr. Roberts to learn how horses communicate, how
prey animals stay safe in a world out to get them. But as our
conversation unspooled, I realized I was asking the questions not to
understand horses, my singular obsession since I was 2 years old, but to
understand myself.

Mr. Roberts told me he had a little shadow at his side. It was a West
Coast mule deer he found still wet with amniotic fluid in a stand of
grass, a runt with little chance of surviving.

Raising the fawn deepened Mr. Roberts's understanding of how horses
communicate. He views deer as an equine exemplar; to him, they are the
raw, wild prey creatures hyperattuned to their world for sheer survival
that horses once were --- before domestication bred the edge off. The
Cervidae are Equus at their most elemental.

The deer's name was Benediction.

``I worked with a lady in England named Elizabeth; she's the best namer
of horses I've ever met,'' Mr. Roberts said. ``So I emailed her and
asked what I would name him. She emailed right back. She didn't say `I
suggest,' or `I think.''' She said, ``His name is Benediction.''

I wondered where the segue about naming this little slip of a mule deer
was going. Then he landed the punchline: ``Queen Elizabeth can sure name
a horse.''

He added: ``And a deer.''

\textbf{``What a horse does under compulsion he does blindly,''} wrote
Xenophon, an ancient Greek cavalry master. ``And his performance is no
more beautiful than would be that of a ballet dancer taught by whip and
goad.'' Far better, he wrote, ``that the horse should of his own accord
exhibit his finest airs and paces.''

A pupil of Socrates, Xenophon survived the battles of Sparta and died in
354 B.C.E. His treatise ``On Horsemanship'' is one of the earliest
surviving works on the art of the equine. ``The majesty of men
themselves is best discovered in the graceful handling of such
animals,'' Xenophon wrote. (I'd like to add, ``of women'' too.)

Taming a horse, gentling it, or, crudely, breaking it, involves
messaging more than anything. A horse may be 1,200 pounds --- so huge
that no amount of force a human could use can truly push it around.
Horses are ridable at all, in a way that, say, lions and tigers are not,
in large part, I believe, because equines are prey animals, bound to the
herd. Horses are genetically inclined to accept a boss.

The phrase ``broke to ride'' used to mean the animal's spirit was broken
so the shell left behind would submit to human will. In modern
equitation, the process is something more like recalibration, convincing
a horse that you run the show. Done well, submission is rebranded as
alliance; the mount and rider, a herd of two.

Mr. Roberts has had a lot to do with that shift. When his first book
came out in the 1990s, the radical departure from the prevailing
discipline earned him enemies. There were even death threats, according
to Mr. Roberts, leading to the arrest of at least one person.

``You're telling them that everything that they've done in their life is
wrong,'' Mr. Roberts said, trying to make sense of why he was so hated
for saying simply, \emph{be gentle}.

Through Benediction the fawn, Roberts honed further his understanding of
how horses communicate. ``The ears, the eye, the neck, the lowering of
the head, the licking and chewing the tongue, all of the appendages or
parts of the anatomy of Equus and Cervidae are put to work to let the
others know what the reading of the situation is,'' he said. The point
is to answer one question: ``Is there danger?''

\textbf{For a year of my life, Equus was my language.} Because on
Thanksgiving Day 2010, I became prey.

Dawn had not yet broken in my apartment in the West Village. I slept
fitfully that night, zapped with excitement for the morning: my first
time covering the Macy's Thanksgiving Day Parade for The Times*.* As a
child I had squished myself among the spectators to watch Snoopy and
Popeye float down Central Park West. Later that day, notebook I hand, I
would walk beneath the shadows of my childhood icons.

In my dark bedroom, I felt a different shadow.

Suddenly, I was awake and I was fighting, squalling, kicking my legs as
fists descended onto me over and over again. I was screaming, but my
voice betrayed me, and my throat made no sound. Deep inside me I heard a
truth: ``This is not a fight you can win. Find another way.'' I stopped
fighting. I lay still.

The man in my bedroom smelled like smoke amplified, like a thousand
stubbed-out Marlboros. He wanted cash, jewels, electronics, stuff, he
told me, and instructed me to lie prone as he ransacked my home. I would
survive, I decided, by being the most helpful victim of all time.

Confined to my bed, as he rooted around I told him how to find
everything of any value. I chided him to get a pen and paper from the
kitchen, so he would be sure not to forget my A.T.M. code for whenever
he went to the bank to wipe out my savings. When he discovered the only
valuables in my tiny apartment were a single laptop and a fistful of
costume jewelry, I cracked the New Yorkiest of jokes to appease him:
``You know how Manhattan real estate is --- we spend all our money on
rent!''

The stranger had climbed through my second-story window, detectives
would later tell me. He left through my front door. ``I swear on my
son's life, I won't hurt you,'' he said when he was finished robbing me,
and the lock clicked closed.

That was when I realized he already had.

Alone again, I lay in a pool of my blood; it poured from a four-inch
wound in my leg. The man had stabbed me with a box-cutter in that first
brutal struggle. In the rush of adrenaline, I hadn't even noticed. As I
watched my blood coursing from the wound, I was terrified, but also
elated --- my pulse was proof that I was still alive.

It took the police just a day to trace his prints and catch the man. He
was sentenced to 17 years in prison, but as I limped around the city and
tried to recover, I realized I was also trapped. New York's cacophony
was my childhood lullaby, but suddenly the city was loud, so loud. And
suddenly I was hypervigilant to every sound.

I had become the Cervidae.

Like a deer, my body was listening for him, for box cutters bared.
Air-conditioners whirring were buzzsaws, made scarier because their
white noise blocked me from hearing what else might approach. The grind
and hustle of the metropolis was a predatory screech telling me that
nowhere was safe anymore.

In horses, hyperattunement to their environment keeps them alive. But it
is also why horses can ``hear'' us humans and respond to our bodies,
like the pressure of our heel that says, ``Trot on.''

``Everything they do --- reading your intention through cortisol levels
and pulse rates and adrenaline levels --- relates to that,'' Mr. Roberts
told me many years after my attack.

``Reading that from afar is their way to survive, and they do it better
than any human being ever would,'' he continued. ``Reading it close up
--- a horse can feel the artery in your inner thigh pulse through the
saddle --- is why they can be ridden.''

During those loud days of my life, I found safety among horses, those
quiet beings. How did I learn to trust the world again? It was the same
way a foal learns to stand --- in that it doesn't actually learn. It
just does. It gets up, falls down, gets up, carries on, because it must,
because that is living.

Benediction means a blessing. And as I kept stumbling on, I felt less
afraid, and more keenly the blessing of being here, alive, still, even
if it was just to stagger forward. On Thanksgiving Day the next year, I
walked down Central Park West underneath Snoopy's big helium belly,
notebook in hand. There in the middle of the parade, the city was no
longer so loud.

Unlike a fawn or a foal, I realized that whether to live as prey was a
choice I could make, not one made for me by a stranger in the dark.

This essay is adapted from the author's forthcoming book
``\href{https://www.simonandschuster.com/books/Horse-Crazy/Sarah-Maslin-Nir/9781501196232}{Horse
Crazy: The Story of a Woman and a World in Love With an Animal}.''

Advertisement

\protect\hyperlink{after-bottom}{Continue reading the main story}

\hypertarget{site-index}{%
\subsection{Site Index}\label{site-index}}

\hypertarget{site-information-navigation}{%
\subsection{Site Information
Navigation}\label{site-information-navigation}}

\begin{itemize}
\tightlist
\item
  \href{https://help.nytimes3xbfgragh.onion/hc/en-us/articles/115014792127-Copyright-notice}{©~2020~The
  New York Times Company}
\end{itemize}

\begin{itemize}
\tightlist
\item
  \href{https://www.nytco.com/}{NYTCo}
\item
  \href{https://help.nytimes3xbfgragh.onion/hc/en-us/articles/115015385887-Contact-Us}{Contact
  Us}
\item
  \href{https://www.nytco.com/careers/}{Work with us}
\item
  \href{https://nytmediakit.com/}{Advertise}
\item
  \href{http://www.tbrandstudio.com/}{T Brand Studio}
\item
  \href{https://www.nytimes3xbfgragh.onion/privacy/cookie-policy\#how-do-i-manage-trackers}{Your
  Ad Choices}
\item
  \href{https://www.nytimes3xbfgragh.onion/privacy}{Privacy}
\item
  \href{https://help.nytimes3xbfgragh.onion/hc/en-us/articles/115014893428-Terms-of-service}{Terms
  of Service}
\item
  \href{https://help.nytimes3xbfgragh.onion/hc/en-us/articles/115014893968-Terms-of-sale}{Terms
  of Sale}
\item
  \href{https://spiderbites.nytimes3xbfgragh.onion}{Site Map}
\item
  \href{https://help.nytimes3xbfgragh.onion/hc/en-us}{Help}
\item
  \href{https://www.nytimes3xbfgragh.onion/subscription?campaignId=37WXW}{Subscriptions}
\end{itemize}
