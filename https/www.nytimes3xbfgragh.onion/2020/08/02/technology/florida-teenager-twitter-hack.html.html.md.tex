\href{/section/technology}{Technology}\textbar{}From Minecraft Tricks to
Twitter Hack: A Florida Teen's Troubled Online Path

\url{https://nyti.ms/30n2H58}

\begin{itemize}
\item
\item
\item
\item
\item
\item
\end{itemize}

\includegraphics{https://static01.graylady3jvrrxbe.onion/images/2020/08/02/business/02twitterteen-clark-copy/02twitterteen-clark-copy-articleLarge.jpg?quality=75\&auto=webp\&disable=upscale}

Sections

\protect\hyperlink{site-content}{Skip to
content}\protect\hyperlink{site-index}{Skip to site index}

\hypertarget{from-minecraft-tricks-to-twitter-hack-a-florida-teens-troubled-online-path}{%
\section{From Minecraft Tricks to Twitter Hack: A Florida Teen's
Troubled Online
Path}\label{from-minecraft-tricks-to-twitter-hack-a-florida-teens-troubled-online-path}}

The teenage ``mastermind'' of the recent Twitter breach, who had a
difficult family life, poured his energy into video games and
cryptocurrency.

Graham Ivan Clark's arrest raised questions about how someone so young
could penetrate the defenses of a sophisticated tech company.Credit...

Supported by

\protect\hyperlink{after-sponsor}{Continue reading the main story}

By
\href{https://www.nytimes3xbfgragh.onion/by/nathaniel-popper}{Nathaniel
Popper}, \href{https://www.nytimes3xbfgragh.onion/by/kate-conger}{Kate
Conger} and
\href{https://www.nytimes3xbfgragh.onion/by/kellen-browning}{Kellen
Browning}

\begin{itemize}
\item
  Published Aug. 2, 2020Updated Aug. 4, 2020, 3:38 p.m. ET
\item
  \begin{itemize}
  \item
  \item
  \item
  \item
  \item
  \item
  \end{itemize}
\end{itemize}

\href{https://www.nytimes3xbfgragh.onion/es/2020/08/04/espanol/negocios/joven-florida-hacker-twitter.html}{Leer
en español}

For Graham Ivan Clark, the online mischief-making started early.

By the age of 10, he was playing the video game Minecraft, in part to
escape what he told friends was an unhappy home life. In Minecraft, he
became known as an adept scammer with an explosive temper who cheated
people out of their money, several friends said.

At 15, he joined an online hackers' forum. By 16, he had gravitated to
the world of Bitcoin, appearing to involve himself in a theft of
\$856,000 of the cryptocurrency, though he was never charged for it,
social media and legal records show. On Instagram posts afterward, he
showed up with designer sneakers and a bling-encrusted Rolex.

The teenager's digital misbehavior ended on Friday when
\href{https://www.nytimes3xbfgragh.onion/2020/07/31/technology/twitter-hack-arrest.html}{the
police arrested him} at a Tampa, Fla., apartment. Florida prosecutors
said Mr. Clark, now 17, was the ``mastermind'' of a
\href{https://www.nytimes3xbfgragh.onion/2020/07/17/technology/twitter-hackers-interview.html}{prominent
hack last month}, accusing him of tricking his way into Twitter's
systems and
\href{https://www.nytimes3xbfgragh.onion/2020/07/15/technology/twitter-hack-bill-gates-elon-musk.html}{taking
over the accounts} of some of the world's most famous people, including
Barack Obama, Kanye West and Jeff Bezos.

His arrest raised questions about how someone so young could penetrate
the defenses of what was supposedly one of Silicon Valley's most
sophisticated technology companies. Mr. Clark, who prosecutors said
worked with at least two others to hack
\href{https://www.nytimes3xbfgragh.onion/2020/08/03/technology/ftc-twitter-privacy-violations.html}{Twitter}
but was the leader, is being charged as an adult with 30 felonies.

Millions of teenagers play the same video games and interact in the same
online forums as Mr. Clark. But what emerges in interviews with more
than a dozen people who know him, along with legal documents, online
forensic work and social media archives, is a picture of a youth who had
a strained relationship with his family and who spent much of his life
online becoming skilled at convincing people to give him money, photos
and information.

``He scammed me for a little bit of money when I was just a kid,'' said
Colby Meeds, 19, a Minecraft player who said Mr. Clark stole \$50 from
him in 2016 by offering to sell him a digital cape for a Minecraft
character but not delivering it.

\includegraphics{https://static01.graylady3jvrrxbe.onion/images/2020/08/03/business/03twitterteen-3/merlin_174606657_99a08768-0e18-4af8-afd1-a79cbbedd123-articleLarge.jpg?quality=75\&auto=webp\&disable=upscale}

Reached via a brief video call on Sunday from the Hillsborough County
Jail in Tampa, Mr. Clark appeared in a black sleeveless shirt, his hair
tumbling into his eyes. ``What are your questions?'' he asked, before
pushing back his chair and hanging up. He is scheduled for a virtual
court appearance on Tuesday.

Mr. Clark and his sister grew up in Tampa with their mother, Emiliya
Clark, a Russian immigrant who holds certifications to work as a
facialist and as a real estate broker. Reached at her home, his mother
declined to comment. His father lives in Indiana, according to public
documents; he did not return a request for comment. His parents divorced
when he was 7.

Mr. Clark doted on his dog and didn't like school or have many friends,
said James Xio, who met Mr. Clark online several years ago. He had a
habit of moving between emotional extremes, flying off the handle over
small transgressions, Mr. Xio said.

``He'd get mad mad,'' said Mr. Xio, 18. ``He had a thin patience.''

Abishek Patel, 19, who played Minecraft with Mr. Clark, defended him.
``He has a good heart and always looks out for the people who he cares
about,'' he said.

In 2016, Mr. Clark set up a YouTube channel, according to the social
media monitoring firm SocialBlade. He built an audience of thousands of
fans and became known for playing a violent version of Minecraft called
Hardcore Factions, under user names like ``Open'' and ``OpenHCF.''

But he became even better known for taking money from other Minecraft
players. People can pay for upgrades with the game, like accessories for
their characters.

One tactic used by Mr. Clark was appearing to sell desirable user names
for Minecraft and then not actually providing the buyer with that user
name. He also offered to sell the capes for Minecraft characters, but
sometimes vanished after other players sent him money.

Image

Some of the online profiles tied to Mr. Clark that were banned because
of his online behavior.

Mr. Clark once offered to sell his own Minecraft user name, ``Open,''
said Nick Jerome, 21, a student at Christopher Newport University in
Virginia. The two messaged over Skype and Mr. Jerome, who was then 17,
said he sent about \$100 for the user name because he thought it was
cool. Then Mr. Clark blocked him.

``I was just kind of a dumb teenager, and looking back, there's no way I
should have ever done this,'' Mr. Jerome said. ``Why should I ever have
trusted this dude?''

In late 2016 and early 2017, other Minecraft players produced videos on
YouTube \href{https://www.youtube.com/watch?v=CvrPXbk-BXw}{describing}
how they had lost money or faced online attacks after brushes with Mr.
Clark's alias ``Open.'' In some of those videos, Mr. Clark, who can be
heard using racist and sexist epithets, also talked about being home
schooled while making \$5,000 a month from his Minecraft activities.

Mr. Clark's real identity rarely showed up online. At one point, he
revealed his face and gaming setup online, and some players called him
Graham. His name was also mentioned in
\href{https://twitter.com/opengrahamclark/status/848014008632344576}{a
2017 Twitter post}.

Mr. Clark's interests soon expanded to the video game
\href{https://www.nytimes3xbfgragh.onion/2018/07/25/arts/what-is-fortnite-battle-royale-nyt.html}{Fortnite}
and the lucrative world of cryptocurrencies. He joined an online forum
for hackers, known as OGUsers, and used the screen name Graham\$. His
OGUsers account was registered from the same internet protocol address
in Tampa that had been attached to his Minecraft accounts, according to
research done for The Times by the online forensics firm Echosec.

Mr. Clark described himself on OGUsers as a ``full time crypto trader
dropout'' and said he was ``focused on just making money all around for
everyone.'' Graham\$ was later banned from the community, according to
posts uncovered by Echosec, after the moderators said he failed to pay
Bitcoin to another user who had already sent him money to complete a
transaction.

Image

Mr. Clark's online misbehavior ended on Friday when police arrested him
at his apartment in Tampa, Fla.Credit...Octavio Jones for The New York
Times

Still, Mr. Clark had already harnessed OGUsers to find his way into a
hacker community known for taking over people's phone numbers to access
all of the online accounts attached to the numbers, an
\href{https://www.nytimes3xbfgragh.onion/2017/08/21/business/dealbook/phone-hack-bitcoin-virtual-currency.html}{attack
known as SIM swapping}. The main goal was to drain victims'
cryptocurrency accounts.

In 2019, hackers remotely seized control of the phone of Gregg Bennett,
a tech investor in the Seattle area. Within a few minutes, they had
secured Mr. Bennett's online accounts, including his Amazon and email
accounts, as well as 164 Bitcoins that were worth \$856,000 at the time
and would be worth \$1.8 million today.

Mr. Bennett soon received an extortion note, which he shared with The
Times. It was signed by Scrim, another of Mr. Clark's online aliases,
according to several of his online friends.

``We just want the remainder of the funds in the Bittrex,'' Scrim wrote,
referring to the Bitcoin exchange from which the coins had been taken.
``We are always one step ahead and this is your easiest option.''

In April, the Secret Service seized 100 Bitcoins from Mr. Clark,
according to government forfeiture documents. A few weeks later, Mr.
Bennett received a letter from the Secret Service saying they had
recovered 100 of his Bitcoins, citing the same code that was assigned to
the coins seized from Mr. Clark.

It is unclear whether other people were involved in the incident or what
happened to the remaining 64 Bitcoins.

Mr. Bennett said in an interview that a Secret Service agent told him
that the person with the stolen Bitcoins was not arrested because he was
a minor. The Secret Service did not respond to a request for comment.

By then, Mr. Clark was living in his own apartment in a Tampa condo
complex. He had an expensive gaming setup, a balcony and a view of a
grassy park, according to friends and social media posts.

Two neighbors said that Mr. Clark kept to himself, coming and going at
unusual hours and driving a white BMW 3 Series.

On an Instagram account that has since been taken down, @error, Mr.
Clark also shared videos of himself swaying to rap music in designer
sneakers. He was given a shout-out on Instagram by a jeweler to the
hip-hop elite, with a picture showing that Mr. Clark, as @error, had
purchased a gem-encrusted Rolex.

Mr. Xio, who became close friends with Mr. Clark, said the April run-in
with the Secret Service shook Mr. Clark.

``He knew he was given a second chance,'' Mr. Xio said. ``And he wanted
to work on being as legit as possible.''

But less than two weeks after the Secret Service seizure, prosecutors
said Mr. Clark began working to get inside Twitter. According to a
government affidavit, Mr. Clark convinced a ``Twitter employee that he
was a co-worker in the IT department and had the employee provide
credentials to access the customer service portal.''

For help, Mr. Clark found accomplices on OGUsers, according to the
charging documents. The accomplices offered to broker the sale of
Twitter accounts that had cool user names, like @w, while Mr. Clark
would enter Twitter's systems and change ownership of the accounts,
according to the filings and accounts from the accomplices.

Image

The Twitter hack unfolded on July 15. For help, Mr. Clark found
accomplices on OGUsers, legal filings said.Credit...Jim Wilson/The New
York Times

The hack unfolded on July 15. A few days later, one accomplice, who went
by the name ``lol,''
\href{https://www.nytimes3xbfgragh.onion/2020/07/17/technology/twitter-hackers-interview.html}{told
The Times} that the person they knew as the mastermind began cheating
the customers who wanted to covertly buy the Twitter accounts. The
hacker took the money and handed over the account, but then quickly
reclaimed it by using his access to Twitter's systems to boot out the
client. It was reminiscent of what Mr. Clark had done earlier on
Minecraft.

When Mr. Clark's online acquaintances learned he had been charged with
the hack, several said they were not surprised.

``He never really seemed to care about anyone but himself,'' said Connor
Belcher, a gamer known as @iMakeMcVidz who had previously teamed up on a
separate YouTube channel with Mr. Clark before becoming one of his
online critics.

Susan Jacobson contributed reporting from Tampa, Fla. Sheelagh McNeil
and Jack Begg contributed research.

Advertisement

\protect\hyperlink{after-bottom}{Continue reading the main story}

\hypertarget{site-index}{%
\subsection{Site Index}\label{site-index}}

\hypertarget{site-information-navigation}{%
\subsection{Site Information
Navigation}\label{site-information-navigation}}

\begin{itemize}
\tightlist
\item
  \href{https://help.nytimes3xbfgragh.onion/hc/en-us/articles/115014792127-Copyright-notice}{©~2020~The
  New York Times Company}
\end{itemize}

\begin{itemize}
\tightlist
\item
  \href{https://www.nytco.com/}{NYTCo}
\item
  \href{https://help.nytimes3xbfgragh.onion/hc/en-us/articles/115015385887-Contact-Us}{Contact
  Us}
\item
  \href{https://www.nytco.com/careers/}{Work with us}
\item
  \href{https://nytmediakit.com/}{Advertise}
\item
  \href{http://www.tbrandstudio.com/}{T Brand Studio}
\item
  \href{https://www.nytimes3xbfgragh.onion/privacy/cookie-policy\#how-do-i-manage-trackers}{Your
  Ad Choices}
\item
  \href{https://www.nytimes3xbfgragh.onion/privacy}{Privacy}
\item
  \href{https://help.nytimes3xbfgragh.onion/hc/en-us/articles/115014893428-Terms-of-service}{Terms
  of Service}
\item
  \href{https://help.nytimes3xbfgragh.onion/hc/en-us/articles/115014893968-Terms-of-sale}{Terms
  of Sale}
\item
  \href{https://spiderbites.nytimes3xbfgragh.onion}{Site Map}
\item
  \href{https://help.nytimes3xbfgragh.onion/hc/en-us}{Help}
\item
  \href{https://www.nytimes3xbfgragh.onion/subscription?campaignId=37WXW}{Subscriptions}
\end{itemize}
