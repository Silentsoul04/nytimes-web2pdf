Sections

SEARCH

\protect\hyperlink{site-content}{Skip to
content}\protect\hyperlink{site-index}{Skip to site index}

\href{https://www.nytimes3xbfgragh.onion/section/us}{U.S.}

\href{https://myaccount.nytimes3xbfgragh.onion/auth/login?response_type=cookie\&client_id=vi}{}

\href{https://www.nytimes3xbfgragh.onion/section/todayspaper}{Today's
Paper}

\href{/section/us}{U.S.}\textbar{}Florida's Summer of Dread

\url{https://nyti.ms/30iWIOI}

\begin{itemize}
\item
\item
\item
\item
\item
\end{itemize}

\hypertarget{the-coronavirus-outbreak}{%
\subsubsection{\texorpdfstring{\href{https://www.nytimes3xbfgragh.onion/news-event/coronavirus?name=styln-coronavirus-national\&region=TOP_BANNER\&variant=undefined\&block=storyline_menu_recirc\&action=click\&pgtype=Article\&impression_id=a1a3f0a0-e386-11ea-afce-73cba6f30df3}{The
Coronavirus
Outbreak}}{The Coronavirus Outbreak}}\label{the-coronavirus-outbreak}}

\begin{itemize}
\tightlist
\item
  live\href{https://www.nytimes3xbfgragh.onion/2020/08/20/world/coronavirus-covid.html?name=styln-coronavirus-national\&region=TOP_BANNER\&variant=undefined\&block=storyline_menu_recirc\&action=click\&pgtype=Article\&impression_id=a1a3f0a1-e386-11ea-afce-73cba6f30df3}{Latest
  Updates}
\item
  \href{https://www.nytimes3xbfgragh.onion/interactive/2020/us/coronavirus-us-cases.html?name=styln-coronavirus-national\&region=TOP_BANNER\&variant=undefined\&block=storyline_menu_recirc\&action=click\&pgtype=Article\&impression_id=a1a3f0a2-e386-11ea-afce-73cba6f30df3}{Maps
  and Cases}
\item
  \href{https://www.nytimes3xbfgragh.onion/interactive/2020/science/coronavirus-vaccine-tracker.html?name=styln-coronavirus-national\&region=TOP_BANNER\&variant=undefined\&block=storyline_menu_recirc\&action=click\&pgtype=Article\&impression_id=a1a3f0a3-e386-11ea-afce-73cba6f30df3}{Vaccine
  Tracker}
\item
  \href{https://www.nytimes3xbfgragh.onion/2020/08/19/us/colleges-closing-covid.html?name=styln-coronavirus-national\&region=TOP_BANNER\&variant=undefined\&block=storyline_menu_recirc\&action=click\&pgtype=Article\&impression_id=a1a417b0-e386-11ea-afce-73cba6f30df3}{Colleges
  Closing}
\item
  \href{https://www.nytimes3xbfgragh.onion/live/2020/08/20/business/stock-market-today-coronavirus?name=styln-coronavirus-national\&region=TOP_BANNER\&variant=undefined\&block=storyline_menu_recirc\&action=click\&pgtype=Article\&impression_id=a1a417b1-e386-11ea-afce-73cba6f30df3}{Economy}
\end{itemize}

Advertisement

\protect\hyperlink{after-top}{Continue reading the main story}

Supported by

\protect\hyperlink{after-sponsor}{Continue reading the main story}

MIAMI JOURNAL

\hypertarget{floridas-summer-of-dread}{%
\section{Florida's Summer of Dread}\label{floridas-summer-of-dread}}

The coronavirus had entrenched itself in communities from Pensacola to
Key West, killing more than 7,000 Floridians. Then came Tropical Storm
Isaias.

\includegraphics{https://static01.graylady3jvrrxbe.onion/images/2020/08/02/us/politics/02virus-floridastorm-1/merlin_175209375_a3ff0f8e-a302-461a-a795-c76e4cf3f1ab-articleLarge.jpg?quality=75\&auto=webp\&disable=upscale}

\href{https://www.nytimes3xbfgragh.onion/by/patricia-mazzei}{\includegraphics{https://static01.graylady3jvrrxbe.onion/images/2018/11/28/multimedia/author-patricia-mazzei/author-patricia-mazzei-thumbLarge.png}}

By \href{https://www.nytimes3xbfgragh.onion/by/patricia-mazzei}{Patricia
Mazzei}

\begin{itemize}
\item
  Published Aug. 2, 2020Updated Aug. 3, 2020
\item
  \begin{itemize}
  \item
  \item
  \item
  \item
  \item
  \end{itemize}
\end{itemize}

MIAMI --- The crowded grocery stores, empty shelves and barren streets
of South Florida in the dawning days of the coronavirus pandemic felt
unsettlingly familiar: They resembled the rush of preparations and then
the tense silence that precede a hurricane.

Maybe the tough residents of a state used to dealing with unpredictable
forces of nature would have an edge in handling the deadly coronavirus.
In theory, the people of Florida know a thing or two about how to follow
orders during an emergency and stay at home.

Oh, were we naïve.

The virus has entrenched itself in communities from Pensacola to
\href{https://www.nytimes3xbfgragh.onion/2020/07/31/us/coronavirus-masks-enforcement-key-west.html?referringSource=articleShare}{Key
West}, killing more than 7,000 Floridians. For four consecutive days
last week, the daily number of fatalities broke state records. Florida's
257 deaths on Friday accounted for nearly one-fifth of all of the deaths
attributed to Covid-19 that day in the United States.

With the scourge of virus death came
\href{https://www.nytimes3xbfgragh.onion/2020/08/03/us/isaias-storm-updates.html}{Tropical
Storm Isaias} to stalk the Atlantic Coast. The calendar had barely
turned to August --- too early, in a normal year, to worry much about
storms. But this annus horribilis would not have it any other way.

A public health crisis. An economic calamity, with more than a million
\href{https://www.nytimes3xbfgragh.onion/2020/04/23/us/florida-coronavirus-unemployment.html}{Floridians
out of work} and an unemployment payment system that was one of the
slowest in the country. And now an early debut of hurricane season to
remind the state that the inevitable convergence of the pandemic and the
weather is likely to play out again, and perhaps much more seriously
than this relatively mild storm, before this nightmare season ends.

``It's just kind of been the way 2020's gone so far,'' said Howard
Tipton, the administrator for St. Lucie County, on Florida's Treasure
Coast. ``But we roll with it, right? We don't get to determine the cards
that we're dealt.''

Tropical Storm Isaias (which is written Isaías in Spanish and pronounced
ees-ah-EE-ahs) threatens the entire East Coast all the way up to Maine,
but it is the South that has seen a recent dramatic increase in new
coronavirus cases. Health officials in Georgia, South Carolina and North
Carolina have warned that hospitals could be strained beyond capacity
with the flood of new patients.

Meantime, emergency management officials
\href{https://www.nytimes3xbfgragh.onion/2020/05/24/us/hurricane-pandemic-coronavirus-florida.html}{have
drawn up special plans} to deal with people fleeing or displaced by
storms. To avoid virus exposure in shelters, the first choice is for
coastal residents in homes vulnerable to flooding to stay with relatives
or friends farther inland, being careful to wear masks and remain
socially distant.

``Because of Covid, we feel that you are safer at home,'' said Bill
Johnson, the emergency management director for Palm Beach County.
``Shelters should be considered your last resort.''

\hypertarget{latest-updates-the-coronavirus-outbreak}{%
\section{\texorpdfstring{\href{https://www.nytimes3xbfgragh.onion/2020/08/20/world/coronavirus-covid.html?action=click\&pgtype=Article\&state=default\&region=MAIN_CONTENT_1\&context=storylines_live_updates}{Latest
Updates: The Coronavirus
Outbreak}}{Latest Updates: The Coronavirus Outbreak}}\label{latest-updates-the-coronavirus-outbreak}}

Updated 2020-08-21T07:46:15.883Z

\begin{itemize}
\tightlist
\item
  \href{https://www.nytimes3xbfgragh.onion/2020/08/20/world/coronavirus-covid.html?action=click\&pgtype=Article\&state=default\&region=MAIN_CONTENT_1\&context=storylines_live_updates\#link-68774d88}{Shutdowns,
  warnings and scoldings follow alarming incidents on college campuses.}
\item
  \href{https://www.nytimes3xbfgragh.onion/2020/08/20/world/coronavirus-covid.html?action=click\&pgtype=Article\&state=default\&region=MAIN_CONTENT_1\&context=storylines_live_updates\#link-26b58724}{Biden
  knocks Trump's pandemic response, and outlines a national strategy.}
\item
  \href{https://www.nytimes3xbfgragh.onion/2020/08/20/world/coronavirus-covid.html?action=click\&pgtype=Article\&state=default\&region=MAIN_CONTENT_1\&context=storylines_live_updates\#link-4e542da3}{U.S.
  health agencies announce moves to confront the flu season and
  plummeting child vaccination rates.}
\end{itemize}

\href{https://www.nytimes3xbfgragh.onion/2020/08/20/world/coronavirus-covid.html?action=click\&pgtype=Article\&state=default\&region=MAIN_CONTENT_1\&context=storylines_live_updates}{See
more updates}

More live coverage:
\href{https://www.nytimes3xbfgragh.onion/live/2020/08/20/business/stock-market-today-coronavirus?action=click\&pgtype=Article\&state=default\&region=MAIN_CONTENT_1\&context=storylines_live_updates}{Markets}

Summer in Florida, with its routine thunderstorms, sweaty nights and
unforgiving mosquitoes, is not for the faint of heart. (At least 11
suspected cases of coronavirus in the Florida Keys last month turned out
to be mosquito-borne dengue fever.) Sometimes it feels as though the
season's only rewards are royal poinciana blooms, ripened mangoes and
fewer tourists.

This summer has been made harder by the virus, which brought a sense of
despair and helplessness that seemed especially acute in the days
leading up to Tropical Storm Isaias. The storm goes away. The virus has
not.

``It's really stretching our limits,'' said Kevin Cho, 31, a Florida
National Guard captain and a nurse practitioner who treats Covid-19
patients in the intensive care units of several Miami public hospitals.
Among them have been a doctor, who died, and a fellow nurse, who lived.

Many poor people contracting the disease ``are losing their jobs, and
now they're faced with a hurricane,'' he added. ``How could they prepare
for a hurricane when they have been exhausted of every resource they
have? This hurricane is only going to make things worse.''

In Miami-Dade County, where the coronavirus has hit worse than anywhere
else in Florida, the emergency operations center has been outfitted with
plexiglass desk dividers and fans equipped with ultraviolet lights to
try to kill the virus. Many employees who would normally be in the
building worked from home, at least as long as their internet did not go
out.

``It's not as good as being here,'' said Frank K. Rollason, the county's
emergency management director. ``But right now, it's better than being
here.''

Some South Floridians hurried to supermarkets, gas stations and hardware
stores to stock up on canned food, water bottles and plywood. But
others, unfazed by the relatively weak and disorganized storm, did not
bother. My building in a Miami suburb, which was not in the storm's
direct path, did not even bring in the patio furniture, and my potted
plants remained on the balcony. One neighbor on my street put up window
shutters.

``We usually would be assuming, `This is terrible,' I think, except
we're already so busy assuming that Covid is terrible that we don't have
any room,'' said the humor writer Dave Barry, a fellow veteran of the
Miami press corps whom I have known since we both worked at The Miami
Herald. ``We go through this every year, where we always overreact to
it, and maybe this time we underreact to it. Or maybe this is just 2020
lulling us into: `OK, you guys think you had a hurricane. Now you can
relax!' Then the big hurricane comes.''

Local officials worried that the usual spike in alcohol sales before the
storm would entice people to invite friends and relatives over.

Verdenia C. Baker, the Palm Beach County administrator, warned: ``I know
we've been cooped up. Now we have a storm. And some of us normally would
have hurricane parties. This is not the time.''

Florida's relentless coronavirus surge has been driven by a
\href{https://www.nytimes3xbfgragh.onion/2020/06/26/us/coronavirus-florida-texas-bars-closing.html}{rapid
economic reopening} that exposed people to infection in bars and
\href{https://www.nytimes3xbfgragh.onion/2020/07/06/us/coronavirus-florida-miami.html}{house
parties}. Contact tracers in Miami-Dade County have found that about 30
percent of people who tested positive for the virus were exposed by
someone else in their household, the biggest source of infection after
``don't know.''

The huge growth in case numbers, which is finally starting to dip, came
even though South Florida had locked down earlier and longer than the
rest of the state. Gov. Ron DeSantis, a Republican, has not issued a
statewide mask order, but Miami-Dade County imposed its first facial
covering requirement back in April.

\href{https://www.nytimes3xbfgragh.onion/news-event/coronavirus?action=click\&pgtype=Article\&state=default\&region=MAIN_CONTENT_3\&context=storylines_faq}{}

\hypertarget{the-coronavirus-outbreak-}{%
\subsubsection{The Coronavirus Outbreak
›}\label{the-coronavirus-outbreak-}}

\hypertarget{frequently-asked-questions}{%
\paragraph{Frequently Asked
Questions}\label{frequently-asked-questions}}

Updated August 17, 2020

\begin{itemize}
\item ~
  \hypertarget{why-does-standing-six-feet-away-from-others-help}{%
  \paragraph{Why does standing six feet away from others
  help?}\label{why-does-standing-six-feet-away-from-others-help}}

  \begin{itemize}
  \tightlist
  \item
    The coronavirus spreads primarily through droplets from your mouth
    and nose, especially when you cough or sneeze. The C.D.C., one of
    the organizations using that measure,
    \href{https://www.nytimes3xbfgragh.onion/2020/04/14/health/coronavirus-six-feet.html?action=click\&pgtype=Article\&state=default\&region=MAIN_CONTENT_3\&context=storylines_faq}{bases
    its recommendation of six feet} on the idea that most large droplets
    that people expel when they cough or sneeze will fall to the ground
    within six feet. But six feet has never been a magic number that
    guarantees complete protection. Sneezes, for instance, can launch
    droplets a lot farther than six feet,
    \href{https://jamanetwork.com/journals/jama/fullarticle/2763852}{according
    to a recent study}. It's a rule of thumb: You should be safest
    standing six feet apart outside, especially when it's windy. But
    keep a mask on at all times, even when you think you're far enough
    apart.
  \end{itemize}
\item ~
  \hypertarget{i-have-antibodies-am-i-now-immune}{%
  \paragraph{I have antibodies. Am I now
  immune?}\label{i-have-antibodies-am-i-now-immune}}

  \begin{itemize}
  \tightlist
  \item
    As of right
    now,\href{https://www.nytimes3xbfgragh.onion/2020/07/22/health/covid-antibodies-herd-immunity.html?action=click\&pgtype=Article\&state=default\&region=MAIN_CONTENT_3\&context=storylines_faq}{that
    seems likely, for at least several months.} There have been
    frightening accounts of people suffering what seems to be a second
    bout of Covid-19. But experts say these patients may have a
    drawn-out course of infection, with the virus taking a slow toll
    weeks to months after initial exposure. People infected with the
    coronavirus typically
    \href{https://www.nature.com/articles/s41586-020-2456-9}{produce}
    immune molecules called antibodies, which are
    \href{https://www.nytimes3xbfgragh.onion/2020/05/07/health/coronavirus-antibody-prevalence.html?action=click\&pgtype=Article\&state=default\&region=MAIN_CONTENT_3\&context=storylines_faq}{protective
    proteins made in response to an
    infection}\href{https://www.nytimes3xbfgragh.onion/2020/05/07/health/coronavirus-antibody-prevalence.html?action=click\&pgtype=Article\&state=default\&region=MAIN_CONTENT_3\&context=storylines_faq}{.
    These antibodies may} last in the body
    \href{https://www.nature.com/articles/s41591-020-0965-6}{only two to
    three months}, which may seem worrisome, but that's perfectly normal
    after an acute infection subsides, said Dr. Michael Mina, an
    immunologist at Harvard University. It may be possible to get the
    coronavirus again, but it's highly unlikely that it would be
    possible in a short window of time from initial infection or make
    people sicker the second time.
  \end{itemize}
\item ~
  \hypertarget{im-a-small-business-owner-can-i-get-relief}{%
  \paragraph{I'm a small-business owner. Can I get
  relief?}\label{im-a-small-business-owner-can-i-get-relief}}

  \begin{itemize}
  \tightlist
  \item
    The
    \href{https://www.nytimes3xbfgragh.onion/article/small-business-loans-stimulus-grants-freelancers-coronavirus.html?action=click\&pgtype=Article\&state=default\&region=MAIN_CONTENT_3\&context=storylines_faq}{stimulus
    bills enacted in March} offer help for the millions of American
    small businesses. Those eligible for aid are businesses and
    nonprofit organizations with fewer than 500 workers, including sole
    proprietorships, independent contractors and freelancers. Some
    larger companies in some industries are also eligible. The help
    being offered, which is being managed by the Small Business
    Administration, includes the Paycheck Protection Program and the
    Economic Injury Disaster Loan program. But lots of folks have
    \href{https://www.nytimes3xbfgragh.onion/interactive/2020/05/07/business/small-business-loans-coronavirus.html?action=click\&pgtype=Article\&state=default\&region=MAIN_CONTENT_3\&context=storylines_faq}{not
    yet seen payouts.} Even those who have received help are confused:
    The rules are draconian, and some are stuck sitting on
    \href{https://www.nytimes3xbfgragh.onion/2020/05/02/business/economy/loans-coronavirus-small-business.html?action=click\&pgtype=Article\&state=default\&region=MAIN_CONTENT_3\&context=storylines_faq}{money
    they don't know how to use.} Many small-business owners are getting
    less than they expected or
    \href{https://www.nytimes3xbfgragh.onion/2020/06/10/business/Small-business-loans-ppp.html?action=click\&pgtype=Article\&state=default\&region=MAIN_CONTENT_3\&context=storylines_faq}{not
    hearing anything at all.}
  \end{itemize}
\item ~
  \hypertarget{what-are-my-rights-if-i-am-worried-about-going-back-to-work}{%
  \paragraph{What are my rights if I am worried about going back to
  work?}\label{what-are-my-rights-if-i-am-worried-about-going-back-to-work}}

  \begin{itemize}
  \tightlist
  \item
    Employers have to provide
    \href{https://www.osha.gov/SLTC/covid-19/standards.html}{a safe
    workplace} with policies that protect everyone equally.
    \href{https://www.nytimes3xbfgragh.onion/article/coronavirus-money-unemployment.html?action=click\&pgtype=Article\&state=default\&region=MAIN_CONTENT_3\&context=storylines_faq}{And
    if one of your co-workers tests positive for the coronavirus, the
    C.D.C.} has said that
    \href{https://www.cdc.gov/coronavirus/2019-ncov/community/guidance-business-response.html}{employers
    should tell their employees} -\/- without giving you the sick
    employee's name -\/- that they may have been exposed to the virus.
  \end{itemize}
\item ~
  \hypertarget{what-is-school-going-to-look-like-in-september}{%
  \paragraph{What is school going to look like in
  September?}\label{what-is-school-going-to-look-like-in-september}}

  \begin{itemize}
  \tightlist
  \item
    It is unlikely that many schools will return to a normal schedule
    this fall, requiring the grind of
    \href{https://www.nytimes3xbfgragh.onion/2020/06/05/us/coronavirus-education-lost-learning.html?action=click\&pgtype=Article\&state=default\&region=MAIN_CONTENT_3\&context=storylines_faq}{online
    learning},
    \href{https://www.nytimes3xbfgragh.onion/2020/05/29/us/coronavirus-child-care-centers.html?action=click\&pgtype=Article\&state=default\&region=MAIN_CONTENT_3\&context=storylines_faq}{makeshift
    child care} and
    \href{https://www.nytimes3xbfgragh.onion/2020/06/03/business/economy/coronavirus-working-women.html?action=click\&pgtype=Article\&state=default\&region=MAIN_CONTENT_3\&context=storylines_faq}{stunted
    workdays} to continue. California's two largest public school
    districts --- Los Angeles and San Diego --- said on July 13, that
    \href{https://www.nytimes3xbfgragh.onion/2020/07/13/us/lausd-san-diego-school-reopening.html?action=click\&pgtype=Article\&state=default\&region=MAIN_CONTENT_3\&context=storylines_faq}{instruction
    will be remote-only in the fall}, citing concerns that surging
    coronavirus infections in their areas pose too dire a risk for
    students and teachers. Together, the two districts enroll some
    825,000 students. They are the largest in the country so far to
    abandon plans for even a partial physical return to classrooms when
    they reopen in August. For other districts, the solution won't be an
    all-or-nothing approach.
    \href{https://bioethics.jhu.edu/research-and-outreach/projects/eschool-initiative/school-policy-tracker/}{Many
    systems}, including the nation's largest, New York City, are
    devising
    \href{https://www.nytimes3xbfgragh.onion/2020/06/26/us/coronavirus-schools-reopen-fall.html?action=click\&pgtype=Article\&state=default\&region=MAIN_CONTENT_3\&context=storylines_faq}{hybrid
    plans} that involve spending some days in classrooms and other days
    online. There's no national policy on this yet, so check with your
    municipal school system regularly to see what is happening in your
    community.
  \end{itemize}
\end{itemize}

Dr. Mary Jo Trepka, chairwoman of the epidemiology department at Florida
International University, attributed the contagion in part to Miami's
larger-than-average household sizes and higher poverty rates, as well as
to uneven mask use. The prevalence of the virus is declining so slowly
that it might take until December to get down to a 5 percent positivity
rate, she estimated.

``It's really important that we don't open the tap in any way,'' she
said of the possibility of further reopening. ``I hope we're not going
to be having exposures related to that --- or to any of the future
hurricanes we might face over the next couple of months.''

Florida's shockingly high coronavirus case numbers came after it
initially appeared that the state had weathered the first two months of
the outbreak with success. Instead, after most counties returned to
business and holidays prompted people to hold gatherings, the infections
got out of control.

Gus Perez, 32, whom I met at a party last year, thinks he contracted the
virus three weeks ago, over a weekend on which he hung out with a few
friends and went to an outdoor event late one night at a brewery.

He wore a mask and was careful --- his friend who had leukemia and his
friend's mother had both succumbed earlier to the virus.

``I thought I was very on top of it, and it still got me,'' he said.

The hospitals have not collapsed, but only because they have added
scores of beds, straining doctors and nurses.

The Rev. Maria Anderson, 64, an interfaith Miami hospital chaplain, has
been tending to exhausted medical workers treating Covid-19 patients and
to family members allowed to visit their loved ones shortly before or
after they die.

``I've actually lost track of time,'' she said. ``We're in a timeline
limbo. The end doesn't seem to be in sight, and we have no hope that it
will end.''

Ms. Anderson said that coming home to watch news coverage of political
fighting over masks and the virus has been frustrating, underscoring the
distance between the elected officials making decisions and the
professionals toiling in hospitals every day.

``It's such a huge moral disconnect that state and federal leaders have,
and that's what makes me angry,'' she said. ``I look up and say, `Sorry,
God' --- but the anger is there.''

And now storm season may imperil the tenuous new normal that businesses
have tried to forge as they confront the virus.

When Hurricane Irma lashed Florida in 2017, Mike Beltran kept Ariete,
his restaurant in Miami's Coconut Grove neighborhood, open until the
last minute, cooking rice and black beans for customers. He then worked
out of a borrowed food truck while the electricity was out.

This year, Mr. Beltran, 34, was so consumed by staying afloat amid the
virus, which has forced him to close one of his three restaurants and
lay off some of his staff, that he did not know about Tropical Storm
Isaias until late on Thursday.

``It's like, `Oh, something \emph{else},''' he said. ``I'm just waiting
for the year to be over.''

Advertisement

\protect\hyperlink{after-bottom}{Continue reading the main story}

\hypertarget{site-index}{%
\subsection{Site Index}\label{site-index}}

\hypertarget{site-information-navigation}{%
\subsection{Site Information
Navigation}\label{site-information-navigation}}

\begin{itemize}
\tightlist
\item
  \href{https://help.nytimes3xbfgragh.onion/hc/en-us/articles/115014792127-Copyright-notice}{©~2020~The
  New York Times Company}
\end{itemize}

\begin{itemize}
\tightlist
\item
  \href{https://www.nytco.com/}{NYTCo}
\item
  \href{https://help.nytimes3xbfgragh.onion/hc/en-us/articles/115015385887-Contact-Us}{Contact
  Us}
\item
  \href{https://www.nytco.com/careers/}{Work with us}
\item
  \href{https://nytmediakit.com/}{Advertise}
\item
  \href{http://www.tbrandstudio.com/}{T Brand Studio}
\item
  \href{https://www.nytimes3xbfgragh.onion/privacy/cookie-policy\#how-do-i-manage-trackers}{Your
  Ad Choices}
\item
  \href{https://www.nytimes3xbfgragh.onion/privacy}{Privacy}
\item
  \href{https://help.nytimes3xbfgragh.onion/hc/en-us/articles/115014893428-Terms-of-service}{Terms
  of Service}
\item
  \href{https://help.nytimes3xbfgragh.onion/hc/en-us/articles/115014893968-Terms-of-sale}{Terms
  of Sale}
\item
  \href{https://spiderbites.nytimes3xbfgragh.onion}{Site Map}
\item
  \href{https://help.nytimes3xbfgragh.onion/hc/en-us}{Help}
\item
  \href{https://www.nytimes3xbfgragh.onion/subscription?campaignId=37WXW}{Subscriptions}
\end{itemize}
