Sections

SEARCH

\protect\hyperlink{site-content}{Skip to
content}\protect\hyperlink{site-index}{Skip to site index}

\href{https://www.nytimes3xbfgragh.onion/section/world/americas}{Americas}

\href{https://myaccount.nytimes3xbfgragh.onion/auth/login?response_type=cookie\&client_id=vi}{}

\href{https://www.nytimes3xbfgragh.onion/section/todayspaper}{Today's
Paper}

\href{/section/world/americas}{Americas}\textbar{}Guyana Swears In New
Leader, Resolving Struggle Over Oil Wealth

\url{https://nyti.ms/2PkFIRQ}

\begin{itemize}
\item
\item
\item
\item
\item
\end{itemize}

Advertisement

\protect\hyperlink{after-top}{Continue reading the main story}

Supported by

\protect\hyperlink{after-sponsor}{Continue reading the main story}

\hypertarget{guyana-swears-in-new-leader-resolving-struggle-over-oil-wealth}{%
\section{Guyana Swears In New Leader, Resolving Struggle Over Oil
Wealth}\label{guyana-swears-in-new-leader-resolving-struggle-over-oil-wealth}}

Mohamed Irfaan Ali will manage the flow of billions in new oil revenues,
which are reshaping the small South American country and heightened a
standoff after elections in March.

\includegraphics{https://static01.graylady3jvrrxbe.onion/images/2020/08/02/world/02guyana/merlin_169986186_c1231a6a-b3e4-45de-8a61-b70ca5d3e8b3-articleLarge.jpg?quality=75\&auto=webp\&disable=upscale}

By Nafeeza Yahya-Sakur and Anatoly Kurmanaev

\begin{itemize}
\item
  Aug. 2, 2020
\item
  \begin{itemize}
  \item
  \item
  \item
  \item
  \item
  \end{itemize}
\end{itemize}

GEORGETOWN, Guyana --- Guyana's opposition party has won a bitterly
contested general election, ending a prolonged political standoff that
had crippled investment and heightened ethnic tensions in the small
South American nation.

The opposition candidate Mohamed Irfaan Ali was sworn in on as Guyana's
president on Sunday, shortly after the national electoral commission
said he had beaten the incumbent, David Granger, by just over 15,000
votes, a margin of more than 3 percent. The governing party said it
planned to challenge his victory, alleging fraud.

As president, Mr. Ali will manage billions of dollars in new oil
revenues, which have transformed Guyana, an impoverished former British
colony with fewer than 800,000 people, into the world's fastest-growing
economy this year, despite a slumping global oil market.

Mr. Ali's assumption of office follows five months of political
wrangling between Guyana's two major political parties over the outcome
of the March 2 vote, which exposed deep tensions between Black citizens
and those of South Asian descent. The power struggle has been amplified
by the newfound wealth pouring in from offshore oil fields where
production began in January.

The election dispute brought a 100-day ballot recount, at least half a
dozen court cases and accusations of fraud against both major parties.

After the recount showed Mr. Ali, 40, winning a slim victory, the
governing party's allies in the electoral commission repeatedly tried to
present results that did not match the recount tally approved by
electoral observers.

Members of the governing party also flooded the courts with petitions
for injunctions to block the declaration of results, and the party's
activists threatened to make the country ungovernable if Mr. Ali was
declared winner.

Supporters of both political parties, which are split almost exactly
among ethnic lines, fear they will be excluded from the oil bounty if
their opponents take power.

Growing international pressure and economic pain appeared to persuade
the governing party to cede control. Mr. Granger's intransigence was
condemned by practically all of Guyana's economic partners, including
its usually restrained Caribbean neighbors and the United States, which
revoked visas of top government officials for subverting the elections.

Mr. Granger, 75, continues to assert that a quarter of all the votes
showed signs of irregularities, although the results were endorsed by at
least four major groups of international observers.

In a statement on Sunday, Mr. Granger said that he ``cannot endorse a
flawed report,'' and that he would continue to ``campaign to ensure that
the votes of all Guyanese are accurately recorded.'' He added that he
would challenge the results in court, but stopped short of calling
supporters into the streets.

Guyana's prolonged political standoff had deepened the economic pain of
the pandemic, which is ripping through South America. Despite a steep
rise in oil production, local businessmen said their international
partners had delayed investments and banks had cut credit until the
political crisis was resolved.

``These past five months were very hard: People lost their jobs, money
wasn't circulating,'' said Nasrudeen Alli, a mechanic from a suburb of
the capital, Georgetown. ``I am very optimistic that we can move forward
from this shadow and into the light of betterment for all.''

The legal limbo had prevented Mr. Granger's government from spending the
accumulating oil revenues and mobilizing emergency funds to combat the
coronavirus and the ensuing economic fallout. Without an approved
national budget, Guyana's public spending shrank to a minimum.

The electoral crisis has also brought fears of a return of the ethnic
violence that had plagued Guyana's politics for decades. Mr. Ali's
swearing-in ceremony was greeted cautiously by supporters, who are
mostly of South Asian descent, for fear of provoking retribution from
Mr. Granger's primarily Black followers.

``All I am asking for right now is for us to get back to normalcy, where
we can reap the benefits of a functioning government,'' said Ramsook
Sukdeo, a house painter from a coastal village of Bush Lot.

Advertisement

\protect\hyperlink{after-bottom}{Continue reading the main story}

\hypertarget{site-index}{%
\subsection{Site Index}\label{site-index}}

\hypertarget{site-information-navigation}{%
\subsection{Site Information
Navigation}\label{site-information-navigation}}

\begin{itemize}
\tightlist
\item
  \href{https://help.nytimes3xbfgragh.onion/hc/en-us/articles/115014792127-Copyright-notice}{©~2020~The
  New York Times Company}
\end{itemize}

\begin{itemize}
\tightlist
\item
  \href{https://www.nytco.com/}{NYTCo}
\item
  \href{https://help.nytimes3xbfgragh.onion/hc/en-us/articles/115015385887-Contact-Us}{Contact
  Us}
\item
  \href{https://www.nytco.com/careers/}{Work with us}
\item
  \href{https://nytmediakit.com/}{Advertise}
\item
  \href{http://www.tbrandstudio.com/}{T Brand Studio}
\item
  \href{https://www.nytimes3xbfgragh.onion/privacy/cookie-policy\#how-do-i-manage-trackers}{Your
  Ad Choices}
\item
  \href{https://www.nytimes3xbfgragh.onion/privacy}{Privacy}
\item
  \href{https://help.nytimes3xbfgragh.onion/hc/en-us/articles/115014893428-Terms-of-service}{Terms
  of Service}
\item
  \href{https://help.nytimes3xbfgragh.onion/hc/en-us/articles/115014893968-Terms-of-sale}{Terms
  of Sale}
\item
  \href{https://spiderbites.nytimes3xbfgragh.onion}{Site Map}
\item
  \href{https://help.nytimes3xbfgragh.onion/hc/en-us}{Help}
\item
  \href{https://www.nytimes3xbfgragh.onion/subscription?campaignId=37WXW}{Subscriptions}
\end{itemize}
