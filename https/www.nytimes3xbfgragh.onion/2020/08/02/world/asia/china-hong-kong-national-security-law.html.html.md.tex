Sections

SEARCH

\protect\hyperlink{site-content}{Skip to
content}\protect\hyperlink{site-index}{Skip to site index}

\href{https://www.nytimes3xbfgragh.onion/section/world/asia}{Asia
Pacific}

\href{https://myaccount.nytimes3xbfgragh.onion/auth/login?response_type=cookie\&client_id=vi}{}

\href{https://www.nytimes3xbfgragh.onion/section/todayspaper}{Today's
Paper}

\href{/section/world/asia}{Asia Pacific}\textbar{}`Clean Up This Mess':
The Chinese Thinkers Behind Xi's Hard Line

\href{https://nyti.ms/2PjcvGY}{https://nyti.ms/2PjcvGY}

\begin{itemize}
\item
\item
\item
\item
\item
\item
\end{itemize}

Advertisement

\protect\hyperlink{after-top}{Continue reading the main story}

Supported by

\protect\hyperlink{after-sponsor}{Continue reading the main story}

\hypertarget{clean-up-this-mess-the-chinese-thinkers-behind-xis-hard-line}{%
\section{`Clean Up This Mess': The Chinese Thinkers Behind Xi's Hard
Line}\label{clean-up-this-mess-the-chinese-thinkers-behind-xis-hard-line}}

Chinese academics have been honing the Communist Party's authoritarian
response in Hong Kong, rejecting the liberal ideas of their youth.

\includegraphics{https://static01.graylady3jvrrxbe.onion/images/2020/07/30/world/00china-hardliners-1/merlin_174541701_189de3ad-c3df-49d2-aff4-fd96d8fe02a6-articleLarge.jpg?quality=75\&auto=webp\&disable=upscale}

\href{https://www.nytimes3xbfgragh.onion/by/chris-buckley}{\includegraphics{https://static01.graylady3jvrrxbe.onion/images/2018/10/08/multimedia/author-chris-buckley/author-chris-buckley-thumbLarge.png}}

By \href{https://www.nytimes3xbfgragh.onion/by/chris-buckley}{Chris
Buckley}

\begin{itemize}
\item
  Aug. 2, 2020
\item
  \begin{itemize}
  \item
  \item
  \item
  \item
  \item
  \item
  \end{itemize}
\end{itemize}

\href{https://cn.nytimes3xbfgragh.onion/china/20200803/china-hong-kong-national-security-law/}{阅读简体中文版}\href{https://cn.nytimes3xbfgragh.onion/china/20200803/china-hong-kong-national-security-law/zh-hant/}{閱讀繁體中文版}

HONG KONG --- When Tian Feilong first arrived in Hong Kong as demands
for free elections were on the rise, he said he felt sympathetic toward
a society that seemed to reflect the liberal political ideas he had
studied as a graduate student in Beijing.

Then, as the calls escalated into protests across Hong Kong in 2014, he
increasingly embraced Chinese warnings that freedom could go too far,
threatening national unity. He became an ardent critic of the
demonstrations, and six years later he is a staunch defender of the
\href{https://www.nytimes3xbfgragh.onion/2020/06/30/world/asia/china-critics-security-law-hong-kong.html}{sweeping
national security law}that China has imposed on the former British
colony.

Mr. Tian has joined a tide of Chinese scholars who have turned against
Western-inspired ideas that once flowed in China's universities, instead
promoting the proudly authoritarian worldview ascendant under Xi
Jinping, the Communist Party leader. This cadre of Chinese intellectuals
serve as champions, even official advisers, defending and honing the
party's hardening policies, including the
\href{https://www.nytimes3xbfgragh.onion/2020/07/01/world/asia/hong-kong-security-law-china.html}{rollout
of the security law} in Hong Kong.

``Back when I was weak, I had to totally play by your rules. Now I'm
strong and have confidence, so why can't I lay down my own rules and
values and ideas?'' Mr. Tian, 37, said in an interview, explaining the
prevailing outlook in China. Witnessing the tumult as a visiting scholar
in Hong Kong in 2014, Mr. Tian said, he ``rethought the relationship
between individual freedom and state authority.''

``Hong Kong is, after all, China's Hong Kong,'' he said. ``It's up to
the Communist Party to clean up this mess.''

While China's Communist Party has long nurtured legions of academics to
defend its agenda, these authoritarian thinkers stand out for their
unabashed, often flashily erudite advocacy of one-party rule and
assertive sovereignty, and their turn against the liberal ideas that
many of them once embraced.

\includegraphics{https://static01.graylady3jvrrxbe.onion/images/2020/07/30/world/00china-hardliners-2/merlin_159290796_9e925016-4af7-44c2-b118-143e328785b0-articleLarge.jpg?quality=75\&auto=webp\&disable=upscale}

They portray themselves as fortifying China for an era of deepening
ideological rivalry. They describe the United States as a
\href{http://www.publiclaw.cn/?c=news\&m=view\&id=7852}{dangerous,}overreaching
shambles, even more so in the wake of the coronavirus pandemic. They
\href{http://history.sina.com.cn/his/zl/2015-08-04/1454123837.shtml}{oppose
constitutional fetters} on Communist Party control, arguing that
Western-inspired ideas of the rule of law are a dangerous mirage that
could hobble the party.

They argue that China must reclaim its status as a world power, even as
a new
\href{https://www.readingthechinadream.com/jiang-shigong-empire-and-world-order.html}{kind
of benign empire} displacing the United States. They extol Mr. Xi as a
historic leader, guiding China through a momentous transformation.

A number of these scholars,
sometimes\href{https://www.researchgate.net/publication/326028916_The_Specter_of_Leviathan_A_Critique_of_Chinese_Statism_since_2000}{called
``statists},'' have worked on policy toward Hong Kong, the sole
territory under Chinese rule that has been a stubborn enclave for
pro-democracy defiance of Beijing. Their proposals have fed into China's
increasingly uncompromising line, including the security law, which has
swiftly
\href{https://www.nytimes3xbfgragh.onion/2020/07/01/world/asia/hong-kong-security-law-china.html}{curbed
protests} and
\href{https://www.nytimes3xbfgragh.onion/2020/07/08/world/asia/hong-kong-security-china-media.html}{political
debate}.

``We ignore these voices at our own risk,'' said
\href{https://history.ubc.ca/profile/timothy-cheek/}{Timothy Cheek}, a
historian at the University of British Columbia who helps run
\href{https://www.readingthechinadream.com/}{Reading the China Dream}, a
website that translates works by Chinese thinkers. ``They give voice to
a stream of Chinese political thought that is probably more influential
than liberal thought.''

As well as earnestly citing Mr. Xi's speeches, these academics
\href{http://www.ccps.gov.cn/bkjd/xzglgg/xzglgg2019_08/201909/t20190923_134457.shtml}{draw
on ancient Chinese thinkers} who counseled stern rulership, along with
Western critics of liberal political traditions. Traditional Marxism is
rarely cited; they are proponents of order, not revolution.

Image

The scholars extol the proudly authoritarian worldview ascendant under
Xi Jinping, China's top leader.Credit...Roman Pilipey/EPA, via
Shutterstock

Many of them make respectful nods in their papers to Carl Schmitt, the
German legal theorist who supplied rightist leaders in the 1930s and the
emerging Nazi regime with arguments for extreme executive power in times
of crisis,
\href{https://www.law.cuhk.edu.hk/app/people/prof-ryan-mitchell/}{Ryan
Mitchell}, an assistant professor of law at the Chinese University of
Hong Kong,
\href{https://papers.ssrn.com/sol3/papers.cfm?abstract_id=3400946}{documented
in a recent paper}.

``They've provided the reasoning and justification,'' Fu Hualing, a
\href{https://www.law.hku.hk/academic_staff/professor-fu-hualing/}{professor
of law} at the University of Hong Kong, said of China's new
authoritarian scholars. ``In a way, it's the Carl Schmitt moment here.''

China's ideological landscape was more varied a decade ago, when Mr.
Tian was a graduate student at Peking University, a traditionally more
liberal campus. Censorship was lighter, and universities tolerated
guarded discussion of liberal ideas in classrooms.

Many scholars, including Mr. Tian's dissertation adviser, Zhang Qianfan,
\href{http://www.calaw.cn/article/default.asp?id=1238}{argued that Hong
Kong}, with its robust judicial independence, could inspire similar
steps in mainland China. ``I had also been nurtured by liberal
scholars.'' Mr. Tian said.

Such ideas have gone into sharp retreat since Mr. Xi took power in 2012.
He
\href{https://www.nytimes3xbfgragh.onion/2013/08/20/world/asia/chinas-new-leadership-takes-hard-line-in-secret-memo.html}{began
a drive}to discredit ideas like universal human rights, separation of
powers and other liberal concepts.

Dissenting academics are maligned in the party-run news media and risk
professional ruin. Xu Zhangrun, a law professor at Tsinghua University
in Beijing, was
\href{https://www.nytimes3xbfgragh.onion/2020/07/12/world/asia/xu-zhangrun-china-xi.html}{detained
in July} and
\href{https://chinadigitaltimes.net/2020/07/xi-critic-xu-zhangrun-released-from-detention-fired-from-tsinghua-university/}{dismissed
from his job} after writing a stream of essays condemning the party's
direction under Mr. Xi.

Image

Xu Zhangrun, a law professor at Tsinghua University, was detained by the
police in Beijing in July and lost his job.Credit...The New York Times

The education authorities generously fund pro-party scholars for topics
\href{http://news.sohu.com/20150720/n417161554.shtml}{such as}how to
introduce security laws in Hong Kong. Chinese and foreign foundations
that once supported less orthodox Chinese scholars
\href{https://www.nytimes3xbfgragh.onion/2016/12/29/world/asia/china-foreign-ngo.html}{have
retrenched}because of tightening official restrictions.

More than fear and career rewards have driven this resurgence of
authoritarian ideas in China. The global financial crisis of 2007, and
the United States' floundering response to the coronavirus pandemic,
have reinforced Chinese views that liberal democracies are decaying,
while China has prospered, defying predictions of the collapse of
one-party rule.

``China is actually also following a path that the United States took,
seizing opportunities, developing outward, creating a new world,'' Mr.
Tian said. ``There is even a fervent hope that we'll overtake the West
in another 30 years.''

China's authoritarian academics have
\href{https://jamestown.org/program/toward-a-second-generation-of-ethnic-policies/}{proposed
policies}to assimilate ethnic minorities thoroughly. They have defended
Mr. Xi's
\href{https://www.nytimes3xbfgragh.onion/2018/03/11/world/asia/china-xi-constitution-term-limits.html}{abolition
of a term limit} on the presidency, opening the way for him to stay in
power indefinitely. They have argued that Chinese-style ``rule by law''
is \href{http://www.aisixiang.com/data/77230.html}{inseparable from
rule} by the Communist Party. And more recently they have served as
intellectual warriors in Beijing's efforts to subdue protest in Hong
Kong.

``For them, law becomes a weapon, but it's law that's subordinated to
politics,'' said Sebastian Veg, a professor at the School of Advanced
Studies in Social Sciences in Paris who has studied
the\href{https://www.journals.uchicago.edu/doi/abs/10.1086/702687}{rise
of China's statist thinkers}. ``We've seen that at work in China, and
now it seems to me we're seeing it come to Hong Kong.''

For Hong Kong, these scholars have supplied arguments advancing
Beijing's drive for greater central control.

Under the legal framework that defined Hong Kong's semi-autonomy after
its return to China in 1997, many in the territory assumed that it would
mostly manage its own affairs for decades. Many believed that Hong Kong
lawmakers and leaders would be left to develop national security
legislation, which was required by that framework.

But Mr. Xi's government has pushed back, demanding greater influence.
The authoritarian scholars, familiar with both Mr. Xi's agenda and Hong
Kong law, have distilled those demands into elaborate legal arguments.

Several Beijing law professors earlier served as advisers to the Chinese
government's office in Hong Kong, including Jiang Shigong and Chen
Duanhong, both of Peking University. They declined to be interviewed.

Image

``The survival of the state comes first,'' over individual rights, wrote
Chen Duanhong, professor of law at Peking University, who served as an
adviser to the Chinese government's office in Hong Kong.Credit...Wang
Zhao/Agence France-Presse --- Getty Images

``I don't think they're necessarily setting the party line, but they're
helping to shape it, finding clever ways to put into words and laws what
the party is trying to do,'' said Mr. Mitchell, of the Chinese
University of Hong Kong. ``This is all happening through legislation, so
their ideas matter.''

A
\href{http://english.www.gov.cn/archive/white_paper/2014/08/23/content_281474982986578.htm}{Chinese
government paper} in 2014, which Professor Jiang is
\href{https://tocqueville21.com/le-club/the-restructuring-of-hong-kong-and-the-rise-of-neostatism/}{widely
credited} with helping write, asserted that Beijing had ``comprehensive
jurisdiction'' over Hong Kong, dismissing the idea that China should
stay hands off. The framework that defined Hong Kong's status was
written in the 1980s, when China was still weak and under the sway of
foreign liberal ideas,
\href{https://www.dwnews.com/\%E4\%B8\%AD\%E5\%9B\%BD/59790413/\%E5\%BC\%BA\%E4\%B8\%96\%E5\%8A\%9F\%E4\%B8\%80\%E5\%9B\%BD\%E4\%B8\%A4\%E5\%88\%B6\%E4\%BA\%9F\%E9\%A1\%BB\%E8\%B5\%B0\%E5\%87\%BA\%E9\%82\%93\%E5\%B0\%8F\%E5\%B9\%B3\%E6\%97\%B6\%E4\%BB\%A3?itm_source=universal_search\&itm_campaign=universal_search\&itm_content=\%E5\%BC\%BA\%E4\%B8\%96\%E5\%8A\%9F\&itm_medium=web}{he
later said}.

``They treat Hong Kong as if it were part of the West, and they treat
the West as if it were the entire world.'' Professor Jiang
\href{http://hk.zijing.org/2020/0602/819565.shtml}{recently said} of
Hong Kong's protesters. ``China's rise has not, as some imagined, drawn
Hong Kong society to trust the central authorities.''

After protesters occupied Hong Kong streets in 2014, he and other
scholars \href{https://opinion.huanqiu.com/article/9CaKrnJGSZs}{pressed
the case} that China had the power to impose national security
legislation there, rejecting the idea that such legislation should be
left in the hands of the reluctant Hong Kong authorities.

``The survival of the state comes first, and constitutional law must
serve this fundamental objective,'' Professor Chen, the Peking
University academic,
\href{http://hk.zijing.org/2018/0122/750657.shtml}{wrote in 2018},
citing Mr. Schmitt, the authoritarian German jurist, to make the case
for a security law in Hong Kong.

``When the state is in dire peril,'' Professor Chen wrote, leaders could
set aside the usual constitutional norms, ``in particular provisions for
civic rights, and take all necessary measures.''

Professor Chen submitted an internal study to the party's policymakers
on introducing security legislation for Hong Kong, according to a
\href{http://www.publiclaw.cn/static/upfile/201807251034548539.pdf}{Peking
University report} in 2018, over a year before the party publicly
\href{https://www.nytimes3xbfgragh.onion/2019/10/31/world/asia/hong-kong-protests-china.html}{announced
plans} for such a law.

Since China's legislature
\href{https://www.nytimes3xbfgragh.onion/2020/06/29/world/asia/china-hong-kong-security-law-rules.html}{passed
the security law} in late June, he, Mr. Tian and allied Chinese scholars
have energetically defended it in dozens of articles, interviews and
news conferences. Chinese intellectuals, Mr. Tian suggested, will next
confront worsening relations with the United States.

``We have to choose what side we're on, including us scholars, right?''
he said. ``Sorry, the goal now is not Westernization; it's the great
rejuvenation of the Chinese nation.''

Amber Wang contributed research from Beijing.

Advertisement

\protect\hyperlink{after-bottom}{Continue reading the main story}

\hypertarget{site-index}{%
\subsection{Site Index}\label{site-index}}

\hypertarget{site-information-navigation}{%
\subsection{Site Information
Navigation}\label{site-information-navigation}}

\begin{itemize}
\tightlist
\item
  \href{https://help.nytimes3xbfgragh.onion/hc/en-us/articles/115014792127-Copyright-notice}{©~2020~The
  New York Times Company}
\end{itemize}

\begin{itemize}
\tightlist
\item
  \href{https://www.nytco.com/}{NYTCo}
\item
  \href{https://help.nytimes3xbfgragh.onion/hc/en-us/articles/115015385887-Contact-Us}{Contact
  Us}
\item
  \href{https://www.nytco.com/careers/}{Work with us}
\item
  \href{https://nytmediakit.com/}{Advertise}
\item
  \href{http://www.tbrandstudio.com/}{T Brand Studio}
\item
  \href{https://www.nytimes3xbfgragh.onion/privacy/cookie-policy\#how-do-i-manage-trackers}{Your
  Ad Choices}
\item
  \href{https://www.nytimes3xbfgragh.onion/privacy}{Privacy}
\item
  \href{https://help.nytimes3xbfgragh.onion/hc/en-us/articles/115014893428-Terms-of-service}{Terms
  of Service}
\item
  \href{https://help.nytimes3xbfgragh.onion/hc/en-us/articles/115014893968-Terms-of-sale}{Terms
  of Sale}
\item
  \href{https://spiderbites.nytimes3xbfgragh.onion}{Site Map}
\item
  \href{https://help.nytimes3xbfgragh.onion/hc/en-us}{Help}
\item
  \href{https://www.nytimes3xbfgragh.onion/subscription?campaignId=37WXW}{Subscriptions}
\end{itemize}
