Sections

SEARCH

\protect\hyperlink{site-content}{Skip to
content}\protect\hyperlink{site-index}{Skip to site index}

\href{https://www.nytimes3xbfgragh.onion/section/world/europe}{Europe}

\href{https://myaccount.nytimes3xbfgragh.onion/auth/login?response_type=cookie\&client_id=vi}{}

\href{https://www.nytimes3xbfgragh.onion/section/todayspaper}{Today's
Paper}

\href{/section/world/europe}{Europe}\textbar{}As Genoa Inaugurates New
Bridge, the Feeling Is Bittersweet

\url{https://nyti.ms/30jnt5D}

\begin{itemize}
\item
\item
\item
\item
\item
\end{itemize}

Advertisement

\protect\hyperlink{after-top}{Continue reading the main story}

Supported by

\protect\hyperlink{after-sponsor}{Continue reading the main story}

\hypertarget{as-genoa-inaugurates-new-bridge-the-feeling-is-bittersweet}{%
\section{As Genoa Inaugurates New Bridge, the Feeling Is
Bittersweet}\label{as-genoa-inaugurates-new-bridge-the-feeling-is-bittersweet}}

Built in less than two years to replace the collapsed Morandi bridge,
the new span is already a point of pride for Italy. But residents fear
it will not be enough to revive their aging port city.

\includegraphics{https://static01.graylady3jvrrxbe.onion/images/2020/08/03/world/03genoa-new-bridge-sub2/merlin_175214016_41b4457a-b13a-4da7-9ca0-cc69e81911f1-articleLarge.jpg?quality=75\&auto=webp\&disable=upscale}

By \href{https://www.nytimes3xbfgragh.onion/by/gaia-pianigiani}{Gaia
Pianigiani}

\begin{itemize}
\item
  Aug. 2, 2020
\item
  \begin{itemize}
  \item
  \item
  \item
  \item
  \item
  \end{itemize}
\end{itemize}

GENOA, Italy --- Since the dramatic and deadly
\href{https://www.nytimes3xbfgragh.onion/2018/08/14/world/europe/italy-genoa-bridge-collapse.html}{collapse
of the Morandi bridge} over the Italian port of Genoa two years ago,
builders have worked around the clock, through a judicial investigation
and the coronavirus pandemic, so a new bridge could open on time.

Designed by a native son of the city, the architect Renzo Piano, and
built in a record 15 months, the new Genoa San Giorgio bridge, whose
inauguration is Monday, has become a matter of pride for Genoa and all
of Italy, a symbol of their can-do spirit.

Yet residents and business owners say the accomplishment will hardly
cure the pains of the city, which was shrinking --- economically,
demographically and culturally --- even before the collapse, which
killed 43 people on Aug. 14, 2018.

The loss of one of the city's main arteries and its fastest east-west
connection compounded all those problems, devastating businesses and
paralyzing life. Today many in Genoa are still suffering and lament that
the new bridge will not be enough to overcome the absence of a broad,
long-term vision to revive their city.

Though the government and the company that manages the bridge,
Autostrade per l'Italia, or Highways for Italy, gave aid to dozens of
businesses in the area to help them stay afloat, many had to relocate or
remained cut off from the rest of Genoa.

``I lost 50 percent of my business with the collapse; my patients who
lived across the bridge could no longer get here,'' said Dr. Fabio
Bertoldi, a veterinarian whose office is about 300 yards north of the
bridge.

``Now I even bike to work,'' said Dr. Bertoldi, who lives about 15 miles
away. If he drove, he added, ``It would take me three hours to get here
with the construction on the highway.''

A surge of long-overdue infrastructure work has further snarled traffic.
For many of those who live in Genoa, then, the opening of the new bridge
is at best bittersweet.

\includegraphics{https://static01.graylady3jvrrxbe.onion/images/2020/07/31/world/00Genoa-NewBridge2/merlin_142844628_044251c1-e53a-4919-87e1-89af130aa2b4-articleLarge.jpg?quality=75\&auto=webp\&disable=upscale}

``We are glad for the new bridge, built so fast, and the maintenance on
the highways, but it all also leaves us a bitter feeling,'' said Egle
Possetti, spokeswoman for a group of the victims' families. ``Had they
done it before, our relatives might have been alive.'' Ms. Possetti's
sister, brother-in-law, young nephew and niece were killed when the
bridge fell.

Genoa's unique location --- wedged between mountains and sea in Italy's
northeast --- makes it hard to reach and even harder to navigate. It
takes five hours to reach Genoa by fast train from Rome, almost twice
the time it takes to go from Rome to Milan, which is only about 40 miles
farther north.

As high-speed trains near the port city, they must switch to older,
slower two-way tracks, which are often flooded by the region's violent
thunderstorms.

In recent months, dozens of new maintenance sites have forced officials
to limit road traffic. Drivers endure traffic jams, accompanied by the
metallic sound of drilling, that wind along the city's picturesque
highways, an overlapping series of viaducts with stunning sea views.

After the vault of a tunnel northwest of Genoa partly collapsed last
year, Italy's Transportation Ministry ordered a thorough inspection of
the region's overpasses and bridges. Nearly all had safety problems and
had to be repaired.

``We are prioritizing security,'' Placido Migliorino, the engineer in
charge of highway inspections at the ministry, said in a phone
interview.

In the past four months, Mr. Migliorino has traveled weekly to Genoa to
monitor the progress of the maintenance work.

``About 50 of the galleries are around Genoa, and in some the problems
couldn't be fixed overnight,'' Mr. Migliorino said, referring to the
area's tunnels. ``That's why cars and trucks have limited circulation.''

Mr. Migliorino has also been examining viaducts and says they, too, have
been poorly maintained.

Emanuele Piccardo, an architecture critic, said, ``For a country that
tends to work in emergency mode, constant maintenance is difficult, from
the local to the national level.''

Image

The opening of the new bridge, shown under construction in April, is set
for Monday. Genoa residents are proud, but many have mixed
feelings.Credit...Fabio Bussalino/Getty Images

``You can build the long bypass, or a new bridge,'' said Mr. Piccardo,
the curator of ``The Collapse of Modernity,'' an upcoming book on the
consequences of the Morandi disaster. ``But if you don't rethink
mobility in this narrow valley to make it based on rails instead of
wheels, it's a waste of time.''

Even with the new bridge's opening, Mr. Piccardo expects the outlying
Polcevera River valley to remain greatly disconnected from the city
center because of the heavy traffic.

``Building a bridge is an opportunity, but the valley won't improve just
because of this new infrastructure,'' he said.

The valley's Certosa district, just north of the Morandi viaduct, was
among the hardest hit by the bridge collapse. These days, its residents
mingle in the mornings under trees in the cobblestone piazzas. By
midday, the streets are almost empty as the elderly go indoors to nap.

The area, once home to the working class of Genoa's industrial port, is
an urban sprawl divided by the river that the viaduct crossed. Large
department stores and industries line one bank of the river, the port's
containers and residential neighborhoods the other.

Image

Workers on the bridge in June.Credit...Luca Zennaro/EPA, via
Shutterstock

Certosa residents say that the new bridge will hardly bring them any
closer to the city center. Even Marco Bucci, Genoa's mayor, admitted
after the tragedy that he had never been to the district before.

Residents do not expect the area to be reshaped much, despite a subway
stop and the introduction of a park with a memorial for the victims of
the bridge collapse, designed by the famed Italian architect Stefano
Boeri.

``I don't blame anyone,'' said Paolo Lecca, 68, a retiree, as he looked
at the new bridge's huge working site, where his friends used to live.
``But we don't even have a hospital here. We need to get to Genoa.''

Christian Giannini, 48, who owns a bike store in Certosa, said: ``This
new bridge is beautiful. I just hope they make what is underneath nice,
too.''

Mr. Giannini signed his shop's lease four days before the Morandi
viaduct collapsed. His store overlooks the large boulevard that ran
under it, which was closed for eight months. Children once played soccer
where cars are now parked.

``It was somewhat charming,'' he smiled. ``It reminded me of my
childhood when we drove here to buy the best clothes in town.''

While some businesses relocated during the reconstruction, others closed
for good.

Image

A small memorial for a victim of the Morandi bridge collapse. The
tragedy spurred a wave of infrastructure inspections that revealed broad
safety problems.Credit...Massimo Pinca/Reuters

``I open my store every day, but if people leave here, how are we
supposed to make business?'' asked Marianna Correnti, 61, the owner of a
flower shop in the Certosa district. ``I had many clients in the
apartment buildings that were demolished, and of course they are gone.''

Gian Battista Cassano owns a large scrap center under the bridge and to
the west. His cameras recorded the Morandi falling, and he was among
those to relocate. Mr. Cassano's company navigated the red tape to move
to a space half as large, and to install solar panels there. But it
struggled to pay bills and salaries.

``We were left alone,'' he said.

Because of the congestion, some Polcevera Valley residents don't drive
anymore.

``People need to wake up at night to travel with no traffic,'' said
Teresa Altovino, 49, a health worker shopping at the local market in
Certosa. ``I even stopped going to the beach.''

Ms. Altovino, who was working in the area when the bridge collapsed,
said she walked out of the building that day to see what had happened.
She can still hear the people screaming that morning two years ago.

``The new bridge looks solid, but I won't take it,'' she said. ``I am
too scared.''

Advertisement

\protect\hyperlink{after-bottom}{Continue reading the main story}

\hypertarget{site-index}{%
\subsection{Site Index}\label{site-index}}

\hypertarget{site-information-navigation}{%
\subsection{Site Information
Navigation}\label{site-information-navigation}}

\begin{itemize}
\tightlist
\item
  \href{https://help.nytimes3xbfgragh.onion/hc/en-us/articles/115014792127-Copyright-notice}{©~2020~The
  New York Times Company}
\end{itemize}

\begin{itemize}
\tightlist
\item
  \href{https://www.nytco.com/}{NYTCo}
\item
  \href{https://help.nytimes3xbfgragh.onion/hc/en-us/articles/115015385887-Contact-Us}{Contact
  Us}
\item
  \href{https://www.nytco.com/careers/}{Work with us}
\item
  \href{https://nytmediakit.com/}{Advertise}
\item
  \href{http://www.tbrandstudio.com/}{T Brand Studio}
\item
  \href{https://www.nytimes3xbfgragh.onion/privacy/cookie-policy\#how-do-i-manage-trackers}{Your
  Ad Choices}
\item
  \href{https://www.nytimes3xbfgragh.onion/privacy}{Privacy}
\item
  \href{https://help.nytimes3xbfgragh.onion/hc/en-us/articles/115014893428-Terms-of-service}{Terms
  of Service}
\item
  \href{https://help.nytimes3xbfgragh.onion/hc/en-us/articles/115014893968-Terms-of-sale}{Terms
  of Sale}
\item
  \href{https://spiderbites.nytimes3xbfgragh.onion}{Site Map}
\item
  \href{https://help.nytimes3xbfgragh.onion/hc/en-us}{Help}
\item
  \href{https://www.nytimes3xbfgragh.onion/subscription?campaignId=37WXW}{Subscriptions}
\end{itemize}
