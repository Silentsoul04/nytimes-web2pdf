Sections

SEARCH

\protect\hyperlink{site-content}{Skip to
content}\protect\hyperlink{site-index}{Skip to site index}

\href{https://www.nytimes3xbfgragh.onion/section/world/middleeast}{Middle
East}

\href{https://myaccount.nytimes3xbfgragh.onion/auth/login?response_type=cookie\&client_id=vi}{}

\href{https://www.nytimes3xbfgragh.onion/section/todayspaper}{Today's
Paper}

\href{/section/world/middleeast}{Middle East}\textbar{}Israeli Court
Rebukes Prime Minister's Son Over Harassing Protest Leaders

\url{https://nyti.ms/33lsNYi}

\begin{itemize}
\item
\item
\item
\item
\item
\end{itemize}

Advertisement

\protect\hyperlink{after-top}{Continue reading the main story}

Supported by

\protect\hyperlink{after-sponsor}{Continue reading the main story}

\hypertarget{israeli-court-rebukes-prime-ministers-son-over-harassing-protest-leaders}{%
\section{Israeli Court Rebukes Prime Minister's Son Over Harassing
Protest
Leaders}\label{israeli-court-rebukes-prime-ministers-son-over-harassing-protest-leaders}}

Yair Netanyahu tweeted the addresses and phone numbers of three men who
led protests against his father's administration. All three said they
later received death threats.

\includegraphics{https://static01.graylady3jvrrxbe.onion/images/2020/08/02/world/02israel-netanyahu/02israel-netanyahu-articleLarge.jpg?quality=75\&auto=webp\&disable=upscale}

By Adam Rasgon

\begin{itemize}
\item
  Aug. 2, 2020
\item
  \begin{itemize}
  \item
  \item
  \item
  \item
  \item
  \end{itemize}
\end{itemize}

TEL AVIV --- An Israeli court on Sunday ordered Prime Minister Benjamin
Netanyahu's older son to stop harassing three people helping to lead
protests against his father's administration after he tweeted out their
home addresses and cellphone numbers.

Judge Dorit Feinstein of the Jerusalem Magistrates Court also ordered
Yair Netanyahu, to delete the tweet, which called on his more than
88,000 followers to demonstrate in front of the homes of the protest
leaders.

``I instruct him to refrain for the next six months from harassing the
petitioners in every shape, way and form,'' Judge Feinstein wrote in her
decision.

The ruling came a day after large crowds of protesters across the
country demanded Mr. Netanyahu's ouster, criticizing his handling of the
economic and health problems stemming from the coronavirus and arguing
that he should not be permitted to serve as prime minister while under
indictment on corruption charges.

The judge said in her decision that she was concerned the prime
minister's son would continue to harass the petitioners and infringe on
their privacy, adding that he did not rebuke calls for violence that
were posted in response to his tweet.

One of the protest leaders, Yitzhak Ben Gonen, who represented himself
and the two other petitioners, said that Yair Netanyahu's tweet prompted
incessant phone calls, and that each of the three received death threats
from callers.

``We are very happy about this legal victory, but the threats keep
coming,'' said Mr. Ben Gonen, who is a member of A New Contract, an
anti-Netanyahu group popularly known as ``Crime Minister.'' The group
says Mr. Netanyahu should not be able to serve as prime minister while
on trial for bribery, fraud and breach of trust. The trial got underway
in May, and is still in its initial stages.

Later Sunday, the younger Mr. Netanyahu deleted the tweet, but in a
series of other social media posts, he strongly criticized the court's
ruling, calling it Kafkaesque.

He also contended that courts in Israel would one day ban Israelis from
voting for his father's Likud party and order all those on the political
right placed in ``re-education camps.''

``For a long time, Israel hasn't been a democratic state,'' he wrote in
another post, remarking that a petitioner, Haim Shadmi, had been
recorded speaking about hurling a firebomb at the prime minister's
official residence but that Mr. Shadmi was still permitted to protest
near it. ``There's a law for right-wingers and another law for
left-wingers,'' the Facebook post said.

Yair Netanyahu, 29, is a fierce defender of his father and has a history
of stoking controversy through his social media posts, some of which
even the prime minister has condemned.

In December 2018, the son wrote on Facebook that he wished the deaths of
two Israeli soldiers killed by a Palestinian gunman in the West Bank
would be ``avenged,'' adding, ``There will never be peace with the
monsters in human form known since 1964 as `Palestinians.''' In a
separate post at the time, he wrote that he would prefer an Israel
without Muslim residents.

Facebook removed those posts following a flurry of complaints. It said
they included hate speech and violated its community standards.

In February, Yair Netanyahu posted on Twitter a picture of a young
Israeli, Dana Cassidy, who had been photographed earlier with Benny
Gantz, the leader of the Blue and White party. He also posted
unsubstantiated accusations that Mr. Gantz engaged in extramarital
affairs.

The posts prompted some of the son's followers to spread unfounded
rumors that Ms. Cassidy and Mr. Gantz were having an affair.

Mr. Gantz has repeatedly clashed with the prime minister even though
they are now coalition partners.

And in a tweet in May, the prime minister's son questioned how Dana
Weiss, a well-known Israeli journalist, got her job at Channel 12, a
major Israeli television outlet. That tweet led some of Yair Netanyahu's
followers to make unsubstantiated allegations that Ms. Weiss had sexual
relations with her bosses to get her position.

The son apologized for that tweet --- after Ms. Weiss and Channel 12
threatened to file a lawsuit against him.

Mr. Ben Gonen, the lawyer representing his fellow protest leaders, said
he sensed that the protests were gaining momentum. ``We see young
people, who are very angry about the situation in Israel and determined
to change it,'' he said. ``It's too early to determine if we will change
everything, but I feel that something important and new is happening.''

Advertisement

\protect\hyperlink{after-bottom}{Continue reading the main story}

\hypertarget{site-index}{%
\subsection{Site Index}\label{site-index}}

\hypertarget{site-information-navigation}{%
\subsection{Site Information
Navigation}\label{site-information-navigation}}

\begin{itemize}
\tightlist
\item
  \href{https://help.nytimes3xbfgragh.onion/hc/en-us/articles/115014792127-Copyright-notice}{©~2020~The
  New York Times Company}
\end{itemize}

\begin{itemize}
\tightlist
\item
  \href{https://www.nytco.com/}{NYTCo}
\item
  \href{https://help.nytimes3xbfgragh.onion/hc/en-us/articles/115015385887-Contact-Us}{Contact
  Us}
\item
  \href{https://www.nytco.com/careers/}{Work with us}
\item
  \href{https://nytmediakit.com/}{Advertise}
\item
  \href{http://www.tbrandstudio.com/}{T Brand Studio}
\item
  \href{https://www.nytimes3xbfgragh.onion/privacy/cookie-policy\#how-do-i-manage-trackers}{Your
  Ad Choices}
\item
  \href{https://www.nytimes3xbfgragh.onion/privacy}{Privacy}
\item
  \href{https://help.nytimes3xbfgragh.onion/hc/en-us/articles/115014893428-Terms-of-service}{Terms
  of Service}
\item
  \href{https://help.nytimes3xbfgragh.onion/hc/en-us/articles/115014893968-Terms-of-sale}{Terms
  of Sale}
\item
  \href{https://spiderbites.nytimes3xbfgragh.onion}{Site Map}
\item
  \href{https://help.nytimes3xbfgragh.onion/hc/en-us}{Help}
\item
  \href{https://www.nytimes3xbfgragh.onion/subscription?campaignId=37WXW}{Subscriptions}
\end{itemize}
