Sections

SEARCH

\protect\hyperlink{site-content}{Skip to
content}\protect\hyperlink{site-index}{Skip to site index}

\href{https://www.nytimes3xbfgragh.onion/section/arts}{Arts}

\href{https://myaccount.nytimes3xbfgragh.onion/auth/login?response_type=cookie\&client_id=vi}{}

\href{https://www.nytimes3xbfgragh.onion/section/todayspaper}{Today's
Paper}

\href{/section/arts}{Arts}\textbar{}Adam Max, Patron of Brooklyn
Cultural Institutions, Dies at 62

\url{https://nyti.ms/3i870rd}

\begin{itemize}
\item
\item
\item
\item
\item
\end{itemize}

Advertisement

\protect\hyperlink{after-top}{Continue reading the main story}

Supported by

\protect\hyperlink{after-sponsor}{Continue reading the main story}

\hypertarget{adam-max-patron-of-brooklyn-cultural-institutions-dies-at-62}{%
\section{Adam Max, Patron of Brooklyn Cultural Institutions, Dies at
62}\label{adam-max-patron-of-brooklyn-cultural-institutions-dies-at-62}}

He was chairman of the Brooklyn Academy of Music and a benefactor of St.
Ann's Warehouse. He and his wife also helped create a center for women's
history.

\includegraphics{https://static01.graylady3jvrrxbe.onion/images/2020/08/04/obituaries/03Max1/03Max1-articleLarge.jpg?quality=75\&auto=webp\&disable=upscale}

\href{https://www.nytimes3xbfgragh.onion/by/sam-roberts}{\includegraphics{https://static01.graylady3jvrrxbe.onion/images/2018/02/20/multimedia/author-sam-roberts/author-sam-roberts-thumbLarge.jpg}}

By \href{https://www.nytimes3xbfgragh.onion/by/sam-roberts}{Sam Roberts}

\begin{itemize}
\item
  Aug. 3, 2020
\item
  \begin{itemize}
  \item
  \item
  \item
  \item
  \item
  \end{itemize}
\end{itemize}

Adam Max, an investment manager who became a leading patron of Brooklyn
cultural institutions, notably the \href{https://www.bam.org/}{Brooklyn
Academy of Music}, where he had been board chairman since 2017, died on
July 27 at his home in East Hampton, N.Y. He was 62.

The cause was bile duct cancer, his wife, Diane Max, said.

Mr. Max was captivated by the venerable Brooklyn Academy after he and
Ms. Max were given a gift subscription as a wedding present in 1985. The
couple's gifts to BAM would become the largest by individual donors in
its history, according to the academy (though it declined to divulge the
amount).

Mr. Max went on to be a major benefactor of
\href{https://stannswarehouse.org/}{St. Ann's Warehouse}, the
performance space now in Brooklyn Bridge Park. And he and his wife were
instrumental in the establishment of the New-York Historical Society's
\href{https://www.nyhistory.org/womens-history}{Center for Women's
History}, which opened for research and exhibitions at the society's
headquarters in Manhattan in 2017. The couple also sponsored the Diane
and Adam E. Max Conference on Women's History there.

Originally an aspiring actor, Mr. Max taught children's theater as a
Brown University student in Providence, R.I., performed as a mime during
a gap year with a theater group, and inveighed against British taxes as
a costumed character on a Boston Tea Party tourist attraction ship
before joining \href{https://www.thejordancompany.com/}{the Jordan
Company}, a private New York equity firm, in 1986.

``In a moment of self-realization,'' he once said, ``I decided the role
of enthusiastic audience member suited me best.''

Adam Edward Max was born on July 18, 1958, in Manhattan to Herbert B.
Max, a lawyer, and Dorothy (Seletsky) Max. As a child he attended
performances at the Brooklyn Academy with his mother. After graduating
from the Collegiate School in Manhattan, he earned his bachelor's degree
in literature and society at Brown in 1981.

He married Diane Peterson, a former chairwoman of Planned Parenthood of
New York City, in 1985. She survives him, along with their children,
Hannah, Jonah and Ezra; and his brothers, Daniel and Eric.

\includegraphics{https://static01.graylady3jvrrxbe.onion/images/2020/08/03/obituaries/03Max2/03Max2-articleLarge.jpg?quality=75\&auto=webp\&disable=upscale}

Mr. Max joined BAM's board in 2003 and became vice chairman in 2008.
Katy Clark, the Academy's president, described him as ``voracious in his
appetite for solving problems.''

He was credited with playing a significant role in the opening last fall
of
\href{https://blog.bam.org/2019/03/a-first-look-at-bam-strong.html}{BAM
Strong}, an expansion, including a visual art exhibition space, designed
to unify the academy's facilities, along Fulton Street in the Fort
Greene section.

Mr. Max and his wife headed the Next Stage Campaign, which raised more
than \$30 million for its endowment as BAM celebrated its 150th
anniversary.

But Mr. Max's affinity for BAM was about more than bricks and mortar or
balance sheets. He even appeared briefly in a performance of
\href{https://www.nytimes3xbfgragh.onion/2014/11/07/theater/geoff-sobelles-the-object-lesson-at-bam.html}{Geoff
Sobelle's} ``Home,'' which
\href{https://www.nytimes3xbfgragh.onion/2017/12/07/theater/home-geoff-sobelle-review.html}{The
New York Times} described as a ``hallucinatory performance piece'' at
BAM's Harvey Theater in 2017, when he answered a call from the stage for
volunteers in the audience to join the cast.

Mr. Sobelle said he felt indebted to Mr. Max. ``I've never had anyone
seek me out, to begin a meaningful conversation with me, to actively
take a vested interest in my life as a performing artist, and do all of
that on their own accord,'' he wrote in an email. He added, ``He told me
that he saw in me a path that he hadn't taken.''

Advertisement

\protect\hyperlink{after-bottom}{Continue reading the main story}

\hypertarget{site-index}{%
\subsection{Site Index}\label{site-index}}

\hypertarget{site-information-navigation}{%
\subsection{Site Information
Navigation}\label{site-information-navigation}}

\begin{itemize}
\tightlist
\item
  \href{https://help.nytimes3xbfgragh.onion/hc/en-us/articles/115014792127-Copyright-notice}{©~2020~The
  New York Times Company}
\end{itemize}

\begin{itemize}
\tightlist
\item
  \href{https://www.nytco.com/}{NYTCo}
\item
  \href{https://help.nytimes3xbfgragh.onion/hc/en-us/articles/115015385887-Contact-Us}{Contact
  Us}
\item
  \href{https://www.nytco.com/careers/}{Work with us}
\item
  \href{https://nytmediakit.com/}{Advertise}
\item
  \href{http://www.tbrandstudio.com/}{T Brand Studio}
\item
  \href{https://www.nytimes3xbfgragh.onion/privacy/cookie-policy\#how-do-i-manage-trackers}{Your
  Ad Choices}
\item
  \href{https://www.nytimes3xbfgragh.onion/privacy}{Privacy}
\item
  \href{https://help.nytimes3xbfgragh.onion/hc/en-us/articles/115014893428-Terms-of-service}{Terms
  of Service}
\item
  \href{https://help.nytimes3xbfgragh.onion/hc/en-us/articles/115014893968-Terms-of-sale}{Terms
  of Sale}
\item
  \href{https://spiderbites.nytimes3xbfgragh.onion}{Site Map}
\item
  \href{https://help.nytimes3xbfgragh.onion/hc/en-us}{Help}
\item
  \href{https://www.nytimes3xbfgragh.onion/subscription?campaignId=37WXW}{Subscriptions}
\end{itemize}
