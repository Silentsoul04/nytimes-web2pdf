Sections

SEARCH

\protect\hyperlink{site-content}{Skip to
content}\protect\hyperlink{site-index}{Skip to site index}

\href{https://www.nytimes3xbfgragh.onion/section/business}{Business}

\href{https://myaccount.nytimes3xbfgragh.onion/auth/login?response_type=cookie\&client_id=vi}{}

\href{https://www.nytimes3xbfgragh.onion/section/todayspaper}{Today's
Paper}

\href{/section/business}{Business}\textbar{}The United States Is
Reopening Many of the Wrong Schools

\url{https://nyti.ms/2XqI8TE}

\begin{itemize}
\item
\item
\item
\item
\item
\end{itemize}

\href{https://www.nytimes3xbfgragh.onion/news-event/coronavirus?action=click\&pgtype=Article\&state=default\&region=TOP_BANNER\&context=storylines_menu}{The
Coronavirus Outbreak}

\begin{itemize}
\tightlist
\item
  live\href{https://www.nytimes3xbfgragh.onion/2020/08/03/world/coronavirus-covid-19.html?action=click\&pgtype=Article\&state=default\&region=TOP_BANNER\&context=storylines_menu}{Latest
  Updates}
\item
  \href{https://www.nytimes3xbfgragh.onion/interactive/2020/us/coronavirus-us-cases.html?action=click\&pgtype=Article\&state=default\&region=TOP_BANNER\&context=storylines_menu}{Maps
  and Cases}
\item
  \href{https://www.nytimes3xbfgragh.onion/interactive/2020/science/coronavirus-vaccine-tracker.html?action=click\&pgtype=Article\&state=default\&region=TOP_BANNER\&context=storylines_menu}{Vaccine
  Tracker}
\item
  \href{https://www.nytimes3xbfgragh.onion/2020/08/02/us/covid-college-reopening.html?action=click\&pgtype=Article\&state=default\&region=TOP_BANNER\&context=storylines_menu}{College
  Reopening}
\item
  \href{https://www.nytimes3xbfgragh.onion/live/2020/08/03/business/stock-market-today-coronavirus?action=click\&pgtype=Article\&state=default\&region=TOP_BANNER\&context=storylines_menu}{Economy}
\end{itemize}

Advertisement

\protect\hyperlink{after-top}{Continue reading the main story}

Supported by

\protect\hyperlink{after-sponsor}{Continue reading the main story}

economic View

\hypertarget{the-united-states-is-reopening-many-of-the-wrong-schools}{%
\section{The United States Is Reopening Many of the Wrong
Schools}\label{the-united-states-is-reopening-many-of-the-wrong-schools}}

When it is safe enough to return to school, young children would benefit
the most. Yet financial pressures are pushing colleges to reopen most
rapidly, an economist says.

\includegraphics{https://static01.graylady3jvrrxbe.onion/images/2020/08/03/business/03Virus-View-01/merlin_171513666_77afb6d2-fcff-4e60-afd1-d509f326d13a-articleLarge.jpg?quality=75\&auto=webp\&disable=upscale}

By Susan Dynarski

\begin{itemize}
\item
  Aug. 3, 2020
\item
  \begin{itemize}
  \item
  \item
  \item
  \item
  \item
  \end{itemize}
\end{itemize}

With
\href{https://www.nytimes3xbfgragh.onion/news-event/coronavirus}{coronavirus
cases} spiking in
\href{https://www.nytimes3xbfgragh.onion/interactive/2020/07/09/us/coronavirus-cases-reopening-trends.html}{dozens
of states}, the prospect of anything resembling a normal school year is
fading fast.

\href{https://www.nytimes3xbfgragh.onion/2020/08/03/us/school-closing-coronavirus.html}{Schools
can't safely reopen} if infections are
\href{https://www.nytimes3xbfgragh.onion/interactive/2020/07/31/us/coronavirus-school-reopening-risk.html}{exploding
in the communities} they serve.

But in regions where the pandemic appears to be under control, it is
most important to get the youngest children back into school buildings,
to stop the alarming slide in their learning. Older students, especially
those in college, are better equipped to cope with the difficulties of
online education.

That is the broad consensus among experts on back-to-school priorities.
But, as things stand now, much of the United States is preparing to do
exactly the opposite.

In many towns, college students are more likely than kindergartners to
return to school for in-person instruction. An example is my home of Ann
Arbor, Mich., where schoolchildren will be learning completely online
and university students will be attending at least some classes in
person.

Fifteen of the
\href{https://www.edweek.org/ew/section/multimedia/school-districts-reopening-plans-a-snapshot.html}{20
largest school districts} are expected to be teaching fully online in
the fall, according to data collected by Education Week, a trade
publication focused on education. By contrast, among colleges, the
pattern is reversed.
\href{https://collegecrisis.shinyapps.io/dashboard/}{Just 128} ---
roughly 4 percent --- of the nation's thousands of colleges have
announced plans to hold classes solely online, according to data
gathered by Davidson College.

There are plenty of exceptions. At the elementary and secondary levels,
major school districts like
\href{https://www.edweek.org/ew/section/multimedia/school-districts-reopening-plans-a-snapshot.html}{Hillsborough,
Fla. (which includes Tampa) and Dallas} are planning to have students
attend classes in person. New York City, which has the nation's largest
school system, is trying to
\href{https://www.schools.nyc.gov/school-year-20-21/return-to-school-2020/welcome-to-the-2020-2021-school-year}{have
it both ways}, by having students split their learning between the
classroom and the home.

The Boston campus of the University of Massachusetts plans to
\href{https://www.masslive.com/news/2020/07/here-are-the-fall-reopening-plans-for-the-25-largest-colleges-in-mass.html}{shift
classes completely online}, as do several of Harvard's graduate
programs. But most colleges are welcoming students back to campus this
fall, while enormous public school districts like
\href{https://achieve.lausd.net/site/default.aspx?PageType=3\&DomainID=4\&ModuleInstanceID=4466\&ViewID=6446EE88-D30C-497E-9316-3F8874B3E108\&RenderLoc=0\&FlexDataID=91406\&PageID=1}{Los
Angeles Unified} and
\href{https://www.philasd.org/coronavirus/schoolstart2020/}{Philadelphia}
are expected to have no children in their buildings.

\hypertarget{latest-updates-economy}{%
\section{\texorpdfstring{\href{https://www.nytimes3xbfgragh.onion/live/2020/08/03/business/stock-market-today-coronavirus?action=click\&pgtype=Article\&state=default\&region=MAIN_CONTENT_1\&context=storylines_live_updates}{Latest
Updates:
Economy}}{Latest Updates: Economy}}\label{latest-updates-economy}}

\href{https://www.nytimes3xbfgragh.onion/live/2020/08/03/business/stock-market-today-coronavirus?action=click\&pgtype=Article\&state=default\&region=MAIN_CONTENT_1\&context=storylines_live_updates\#the-chicago-fed-president-says-its-up-to-congress-to-save-the-economy}{10h
ago}

\href{https://www.nytimes3xbfgragh.onion/live/2020/08/03/business/stock-market-today-coronavirus?action=click\&pgtype=Article\&state=default\&region=MAIN_CONTENT_1\&context=storylines_live_updates\#the-chicago-fed-president-says-its-up-to-congress-to-save-the-economy}{The
Chicago Fed president says it's up to Congress to save the economy.}

\href{https://www.nytimes3xbfgragh.onion/live/2020/08/03/business/stock-market-today-coronavirus?action=click\&pgtype=Article\&state=default\&region=MAIN_CONTENT_1\&context=storylines_live_updates\#faa-says-boeing-has-effectively-mitigated-defects-in-the-737-max}{11h
ago}

\href{https://www.nytimes3xbfgragh.onion/live/2020/08/03/business/stock-market-today-coronavirus?action=click\&pgtype=Article\&state=default\&region=MAIN_CONTENT_1\&context=storylines_live_updates\#faa-says-boeing-has-effectively-mitigated-defects-in-the-737-max}{F.A.A.
says Boeing has `effectively mitigated' defects in the 737 Max.}

\href{https://www.nytimes3xbfgragh.onion/live/2020/08/03/business/stock-market-today-coronavirus?action=click\&pgtype=Article\&state=default\&region=MAIN_CONTENT_1\&context=storylines_live_updates\#small-businesses-got-emergency-loans-but-not-what-they-expected}{13h
ago}

\href{https://www.nytimes3xbfgragh.onion/live/2020/08/03/business/stock-market-today-coronavirus?action=click\&pgtype=Article\&state=default\&region=MAIN_CONTENT_1\&context=storylines_live_updates\#small-businesses-got-emergency-loans-but-not-what-they-expected}{Small
businesses got emergency loans, but not what they expected.}

\href{https://www.nytimes3xbfgragh.onion/live/2020/08/03/business/stock-market-today-coronavirus?action=click\&pgtype=Article\&state=default\&region=MAIN_CONTENT_1\&context=storylines_live_updates}{See
more updates}

More live coverage:
\href{https://www.nytimes3xbfgragh.onion/2020/08/03/world/coronavirus-covid-19.html?action=click\&pgtype=Article\&state=default\&region=MAIN_CONTENT_1\&context=storylines_live_updates}{Global}

A broad group of educators and health experts argue that young children
should be the first back in classrooms, partly because if a child
doesn't learn to read, write and handle numbers early in primary school,
she or he may struggle for the rest of her or his life. And as a
practical matter, many young children can't log themselves onto a
computer and learn independently, as university students can.

The educational needs of young children have economic implications. When
children learn at home, they need adults nearby to assist them. Affluent
families can afford to hire a nanny to supervise their children during
the school day, but most people cannot.

Sometimes older siblings will help younger ones, which can interrupt
their own learning. But a lot of parents will have no option but to give
up work to watch their kids, which will hamper the economic recovery and
aggravate inequality in wealth and income. The economic case for getting
young children back to school first is therefore very strong.

But just because something makes sense, doesn't mean it automatically
happens.

For colleges, competition for tuition dollars is pushing them to take
outsize risks to get students back on campus, while for public school
districts, inadequate funding --- combined with the failure of
government to curb the coronavirus --- is keeping them from getting
children safely back into school buildings.

Consider the plight of some American colleges. Most rely on tuition to
survive, though some schools have other revenue streams: Public
institutions get financial support
\href{https://research.collegeboard.org/trends/college-pricing/figures-tables/total-and-student-state-and-local-funding-and-public-enrollment-over-time}{from
their states} and a
\href{https://research.collegeboard.org/pdf/2019-trendsincp-fig-19.pdf}{relatively
small number} of private schools have enormous endowments that generate
substantial income.

Colleges without deep pockets --- those that rely most heavily on
tuition --- are at the greatest risk of extinction in this pandemic.

Many college administrators rightly fear they will lose students to
their competitors if they don't hold out the promise of in-person
classes --- which means opening their campuses. This creates what, in
economics, we call an immense coordination problem. The first colleges
to announce that they are teaching entirely online will risk a plunge in
enrollment and tuition revenue, so relatively few have been willing to
do so.

\includegraphics{https://static01.graylady3jvrrxbe.onion/images/2020/08/03/business/03Virus-View-02/merlin_170388108_9f877d20-218d-4f55-83f5-b730f9c1cc4d-articleLarge.jpg?quality=75\&auto=webp\&disable=upscale}

The colleges that have bucked this trend tend not to face strong
competitive pressures. In California, the chancellor of the large,
statewide community college system recommended in May that these schools
should
\href{https://www.cbs8.com/article/news/local/california/california-community-college-chancellor-endorses-going-online-only-this-fall/509-bab87578-c4be-4f25-bce7-58bdecf57c90}{go
online}. This was a logical move, from an economic standpoint, because
students tend to choose the community college closest to their home, and
that reduces competition --- Each college has market power in its area.
A community college in San Diego doesn't worry about its students
decamping to a school in San Francisco.

Two community college in Los Angeles might compete with each other,
however. But because the community colleges are part of a unified
system,
\href{https://edsource.org/2020/some-california-colleges-decide-to-offer-all-fall-classes-online/630660}{they
can act in concert} and are not driven to risk the public health in
order to compete for students who prefer learning in person.

In contrast, the elite University of California campuses like Berkeley
and \href{https://www.ucla.edu/}{Los Angeles} operate in a highly
competitive environment, vying with each other (and with top private
colleges) for the best students in the world.

Perhaps that's why the University of California schools held out the
prospect of in-person classes
\href{https://www.sacbee.com/news/politics-government/capitol-alert/article244116777.html}{long
after the community colleges}. The University of California, Berkeley
only
\href{https://www.berkeleyside.com/2020/07/21/uc-berkeley-online-covid-19-fall-semester}{announced
in late July} that it would be operating completely online, while
U.C.L.A.
\href{https://newsroom.ucla.edu/releases/preparations-for-the-2020-21-academic-year}{revealed
in mid-June} that it was planning to teach just about one-fifth of its
classes in person or in a hybrid format.

If competition for students is preventing many colleges from making
choices that would benefit the public, constrained public revenue
streams are behind the existential challenge for schools that educate
younger students.

Because of the crisis, budgets are being slashed just as costs are
rising. Schools that hope to reopen safely need a steady supply of masks
and sanitizer; upgraded ventilation systems; more school buses and
drivers; a robust system of testing and tracing; and lots of extra real
estate for social distancing. But the tax revenues that fund schools are
drying up in the economic downturn. Elementary and secondary schools
typically
\href{https://www.pgpf.org/blog/2020/07/pandemic-budget-crunch-could-force-states-to-slash-social-services-education-police-budgets-and-more}{eat
up 22 percent} of state and local budgets. Across the United States,
state and local governments could face shortfalls this year
\href{https://www.forbes.com/sites/lizfarmer/2020/07/14/trump-cant-withhold-education-funding-but-schools-could-face-a-funding-crisis-anyway/\#_blank}{estimated
to total more than \$700 billion}. And
\href{https://www.pgpf.org/blog/2020/07/pandemic-budget-crunch-could-force-states-to-slash-social-services-education-police-budgets-and-more}{nearly
every state has a balanced budget requirement} that precludes using
borrowing to get through this crunch.

Without extensive investment in safety measures, virus outbreaks will
most likely close any school that opens its doors. And with Congress
headed toward its August vacation, it's unlikely that fresh funds will
start flowing to local schools soon. In this environment, it is not
surprising that many districts are deciding not to reopen classrooms,
and that many teachers are
\href{https://www.nytimes3xbfgragh.onion/2020/07/29/us/teacher-union-school-reopening-coronavirus.html?searchResultPosition=1}{protesting
when asked to return}.

The decisions of schools reverberate beyond the classroom, affecting the
broader economy. And, like the pandemic itself, the effects spill across
city and state lines, with the choices in one place constraining the
ability of businesses in another to survive. A pandemic is exactly the
type of interstate, multisector challenge that an active federal
government is built for.

By now, in many places, it's already too late to undertake the enormous
preparation needed to open classrooms to young children this fall. That
could change in the unlikely event of a big and immediate infusion of
federal funding.

This assumes, of course, that the government has succeeded in
suppressing the pandemic \emph{outside} of schools. If the coronavirus
is raging unchecked through a region, no amount of funding can make it
safe for children to be in school.

Susan Dynarski is a professor of education, public policy and economics
at the University of Michigan. Follow her on Twitter @dynarski

Advertisement

\protect\hyperlink{after-bottom}{Continue reading the main story}

\hypertarget{site-index}{%
\subsection{Site Index}\label{site-index}}

\hypertarget{site-information-navigation}{%
\subsection{Site Information
Navigation}\label{site-information-navigation}}

\begin{itemize}
\tightlist
\item
  \href{https://help.nytimes3xbfgragh.onion/hc/en-us/articles/115014792127-Copyright-notice}{©~2020~The
  New York Times Company}
\end{itemize}

\begin{itemize}
\tightlist
\item
  \href{https://www.nytco.com/}{NYTCo}
\item
  \href{https://help.nytimes3xbfgragh.onion/hc/en-us/articles/115015385887-Contact-Us}{Contact
  Us}
\item
  \href{https://www.nytco.com/careers/}{Work with us}
\item
  \href{https://nytmediakit.com/}{Advertise}
\item
  \href{http://www.tbrandstudio.com/}{T Brand Studio}
\item
  \href{https://www.nytimes3xbfgragh.onion/privacy/cookie-policy\#how-do-i-manage-trackers}{Your
  Ad Choices}
\item
  \href{https://www.nytimes3xbfgragh.onion/privacy}{Privacy}
\item
  \href{https://help.nytimes3xbfgragh.onion/hc/en-us/articles/115014893428-Terms-of-service}{Terms
  of Service}
\item
  \href{https://help.nytimes3xbfgragh.onion/hc/en-us/articles/115014893968-Terms-of-sale}{Terms
  of Sale}
\item
  \href{https://spiderbites.nytimes3xbfgragh.onion}{Site Map}
\item
  \href{https://help.nytimes3xbfgragh.onion/hc/en-us}{Help}
\item
  \href{https://www.nytimes3xbfgragh.onion/subscription?campaignId=37WXW}{Subscriptions}
\end{itemize}
