Sections

SEARCH

\protect\hyperlink{site-content}{Skip to
content}\protect\hyperlink{site-index}{Skip to site index}

\href{https://www.nytimes3xbfgragh.onion/section/business}{Business}

\href{https://myaccount.nytimes3xbfgragh.onion/auth/login?response_type=cookie\&client_id=vi}{}

\href{https://www.nytimes3xbfgragh.onion/section/todayspaper}{Today's
Paper}

\href{/section/business}{Business}\textbar{}Small Businesses Got
Emergency Loans, but Not What They Expected

\url{https://nyti.ms/39Ye4Uj}

\begin{itemize}
\item
\item
\item
\item
\item
\item
\end{itemize}

\href{https://www.nytimes3xbfgragh.onion/news-event/coronavirus?action=click\&pgtype=Article\&state=default\&region=TOP_BANNER\&context=storylines_menu}{The
Coronavirus Outbreak}

\begin{itemize}
\tightlist
\item
  live\href{https://www.nytimes3xbfgragh.onion/2020/08/03/world/coronavirus-covid-19.html?action=click\&pgtype=Article\&state=default\&region=TOP_BANNER\&context=storylines_menu}{Latest
  Updates}
\item
  \href{https://www.nytimes3xbfgragh.onion/interactive/2020/us/coronavirus-us-cases.html?action=click\&pgtype=Article\&state=default\&region=TOP_BANNER\&context=storylines_menu}{Maps
  and Cases}
\item
  \href{https://www.nytimes3xbfgragh.onion/interactive/2020/science/coronavirus-vaccine-tracker.html?action=click\&pgtype=Article\&state=default\&region=TOP_BANNER\&context=storylines_menu}{Vaccine
  Tracker}
\item
  \href{https://www.nytimes3xbfgragh.onion/2020/08/02/us/covid-college-reopening.html?action=click\&pgtype=Article\&state=default\&region=TOP_BANNER\&context=storylines_menu}{College
  Reopening}
\item
  \href{https://www.nytimes3xbfgragh.onion/live/2020/08/03/business/stock-market-today-coronavirus?action=click\&pgtype=Article\&state=default\&region=TOP_BANNER\&context=storylines_menu}{Economy}
\end{itemize}

Advertisement

\protect\hyperlink{after-top}{Continue reading the main story}

Supported by

\protect\hyperlink{after-sponsor}{Continue reading the main story}

\hypertarget{small-businesses-got-emergency-loans-but-not-what-they-expected}{%
\section{Small Businesses Got Emergency Loans, but Not What They
Expected}\label{small-businesses-got-emergency-loans-but-not-what-they-expected}}

The S.B.A.'s disaster relief program allows for loans of up to \$2
million. But now they're capped at \$150,000 --- and agency officials
are saying little about why.

\includegraphics{https://static01.graylady3jvrrxbe.onion/images/2020/08/04/business/00sba-disasterloan1/merlin_174739206_54c92148-8b18-476d-8b70-74438b7780d2-articleLarge.jpg?quality=75\&auto=webp\&disable=upscale}

\href{https://www.nytimes3xbfgragh.onion/by/stacy-cowley}{\includegraphics{https://static01.graylady3jvrrxbe.onion/images/2018/10/03/multimedia/author-stacy-cowley/author-stacy-cowley-thumbLarge.png}}

By \href{https://www.nytimes3xbfgragh.onion/by/stacy-cowley}{Stacy
Cowley}

\begin{itemize}
\item
  Aug. 3, 2020
\item
  \begin{itemize}
  \item
  \item
  \item
  \item
  \item
  \item
  \end{itemize}
\end{itemize}

For nearly 70 years, the Small Business Administration's disaster relief
program has helped companies recover from catastrophes including
wildfires, hurricanes and earthquakes. But it has never faced anything
like
\href{https://www.nytimes3xbfgragh.onion/news-event/coronavirus}{the
coronavirus crisis}.

Besieged by more than eight million applicants --- and operating in the
shadow of the hastily assembled
\href{https://www.nytimes3xbfgragh.onion/2020/04/26/business/ppp-small-business-loans.html}{Paycheck
Protection Program} --- the disaster relief effort has given out more
money in the past few months than it had in its entire history.

But the demand has created a problem that is hobbling hundreds of
thousands of applicants: The agency, afraid of running out of cash,
capped its coronavirus loans at a fraction of what companies can
normally borrow --- even though the program has handed out less than
half of the \$360 billion it can lend.

Caroline Keefer, a clothing designer in Los Angeles, had expected to
qualify for a loan of at least \$500,000 based on a complex formula
devised by the agency. But when her loan offer arrived in May, it was
for \$150,000 --- the ceiling the S.B.A. quietly put in place that
month. Qualified companies can usually take loans of
\href{https://www.sba.gov/about-sba/sba-newsroom/press-releases-media-advisories/sba-provide-disaster-assistance-loans-small-businesses-impacted-coronavirus-covid-19}{up
to \$2 million}.

``Without the extra capital, it will be very difficult for us to
survive,'' she wrote in a direct appeal to Jovita Carranza, the agency's
administrator, and James Rivera, the head of the agency's disaster
office.

The limit has crimped Ms. Keefer's efforts to salvage a business that
did \$2 million in sales last year. Her company,
\href{https://www.riverandskycalifornia.com/}{River + Sky}, sells
directly to merchants like boutiques, department stores and hotel spa
shops. In just a few days in March, as
\href{https://www.nytimes3xbfgragh.onion/interactive/2020/us/states-reopen-map-coronavirus.html}{virus
shutdown orders} cascaded throughout the country, nearly \$700,000 in
orders --- all of her spring and summer season --- evaporated. She was
left with a pile of unpaid bills for inventory that she suddenly had no
place to sell.

Six days after she wrote to the agency, representatives there
acknowledged that she had run up against the cap. Officials ``do not
anticipate increasing loans above this amount,'' the representatives
said in an email.

Ms. Keefer is grateful for the help she received, but irked by what she
sees as an arbitrary, poorly explained limit that was put in place after
other businesses got bigger loans early in the crisis. Data released by
the agency last month showed that it had made at least 20,000 disaster
relief loans for more than \$150,000. Its largest was for \$900,000 in
early April.

Nearly 400,000 businesses have run into the \$150,000 limit, according
to
\href{https://www.sba.gov/funding-programs/loans/coronavirus-relief-options/economic-injury-disaster-loans\#section-header-5}{the
agency's data}. S.B.A. representatives declined to comment on the cap or
why it was imposed.

The cap has been just one problem with the disaster program, officially
called the Economic Injury Disaster Loan program. Applicants faced
\href{https://www.nytimes3xbfgragh.onion/2020/04/09/business/smallbusiness/small-business-disaster-loans-coronavirus.html}{long
delays}, confusing procedures and communication lapses. And last
Tuesday, the agency's internal watchdog said hundreds of millions of
dollars handed out through the program
\href{https://www.nytimes3xbfgragh.onion/live/2020/07/28/business/stock-market-today-coronavirus\#thieves-are-targeting-small-business-relief-programs-a-watchdog-says}{may
have been fraudulently obtained}.

\hypertarget{latest-updates-economy}{%
\section{\texorpdfstring{\href{https://www.nytimes3xbfgragh.onion/live/2020/08/03/business/stock-market-today-coronavirus?action=click\&pgtype=Article\&state=default\&region=MAIN_CONTENT_1\&context=storylines_live_updates}{Latest
Updates:
Economy}}{Latest Updates: Economy}}\label{latest-updates-economy}}

\href{https://www.nytimes3xbfgragh.onion/live/2020/08/03/business/stock-market-today-coronavirus?action=click\&pgtype=Article\&state=default\&region=MAIN_CONTENT_1\&context=storylines_live_updates\#the-chicago-fed-president-says-its-up-to-congress-to-save-the-economy}{10h
ago}

\href{https://www.nytimes3xbfgragh.onion/live/2020/08/03/business/stock-market-today-coronavirus?action=click\&pgtype=Article\&state=default\&region=MAIN_CONTENT_1\&context=storylines_live_updates\#the-chicago-fed-president-says-its-up-to-congress-to-save-the-economy}{The
Chicago Fed president says it's up to Congress to save the economy.}

\href{https://www.nytimes3xbfgragh.onion/live/2020/08/03/business/stock-market-today-coronavirus?action=click\&pgtype=Article\&state=default\&region=MAIN_CONTENT_1\&context=storylines_live_updates\#faa-says-boeing-has-effectively-mitigated-defects-in-the-737-max}{11h
ago}

\href{https://www.nytimes3xbfgragh.onion/live/2020/08/03/business/stock-market-today-coronavirus?action=click\&pgtype=Article\&state=default\&region=MAIN_CONTENT_1\&context=storylines_live_updates\#faa-says-boeing-has-effectively-mitigated-defects-in-the-737-max}{F.A.A.
says Boeing has `effectively mitigated' defects in the 737 Max.}

\href{https://www.nytimes3xbfgragh.onion/live/2020/08/03/business/stock-market-today-coronavirus?action=click\&pgtype=Article\&state=default\&region=MAIN_CONTENT_1\&context=storylines_live_updates\#small-businesses-got-emergency-loans-but-not-what-they-expected}{13h
ago}

\href{https://www.nytimes3xbfgragh.onion/live/2020/08/03/business/stock-market-today-coronavirus?action=click\&pgtype=Article\&state=default\&region=MAIN_CONTENT_1\&context=storylines_live_updates\#small-businesses-got-emergency-loans-but-not-what-they-expected}{Small
businesses got emergency loans, but not what they expected.}

\href{https://www.nytimes3xbfgragh.onion/live/2020/08/03/business/stock-market-today-coronavirus?action=click\&pgtype=Article\&state=default\&region=MAIN_CONTENT_1\&context=storylines_live_updates}{See
more updates}

More live coverage:
\href{https://www.nytimes3xbfgragh.onion/2020/08/03/world/coronavirus-covid-19.html?action=click\&pgtype=Article\&state=default\&region=MAIN_CONTENT_1\&context=storylines_live_updates}{Global}

Application hurdles, changing requirements and
\href{https://www.nytimes3xbfgragh.onion/2020/07/27/us/lamborghini-ppp-covid-19.html}{reports
of fraud} also plagued the Paycheck Protection Program, the short-term
relief effort created by the CARES Act that has
\href{https://www.nytimes3xbfgragh.onion/2020/06/30/business/paycheck-protection-program-coronavirus.html}{handed
out \$521 billion in forgivable loans} to cover payroll and other costs.

Image

River + Sky saw almost \$700,000 in orders disappear after the pandemic
struck. Most of its sales are to boutiques, department stores and hotel
spa shops.Credit...Nolwen Cifuentes for The New York Times

Image

Ms. Keefer used some of the disaster loan money to retool her business
so she could sell directly to customers. But she has also had to borrow
more money.Credit...Nolwen Cifuentes for The New York Times

The disaster loan program, a core part of the agency's operations since
it was founded in 1953, is more flexible. The program offers companies
with 500 or fewer employees low-interest loans for terms of up to 30
years, which can be used for nearly any business purpose, including
buying protective equipment and keeping up on debt payments.

Since March, it has
\href{https://www.sba.gov/sites/default/files/2020-07/EIDL\%20COVID-19\%20Loan\%207.27.20-508.pdf}{lent
out \$164 billion} in EIDL (pronounced ``idle'') loans, more than twice
what it previously distributed in its entire existence, to three million
companies. Nearly \$200 billion is currently unused.

More than two million other businesses have been offered loans but have
not yet accepted them, so much of the unused money could still be lent
out. But the agency's ability to forecast how much money it will
distribute may have been complicated by a decision Congress made in
March to speed aid.

As the coronavirus pandemic took hold, Congress increased its
allocations to the agency, enough to support \$360 billion in loans. But
it also set aside another pool of money for the S.B.A. to distribute as
grants to those who applied to the disaster loan program, whether they
received a loan or not. The \$20 billion for those grants --- up to
\$10,000 per applicant ---
\href{https://www.sba.gov/about-sba/sba-newsroom/press-releases-media-advisories/sba-provided-20-billion-small-businesses-and-non-profits-through-economic-injury-disaster-loan}{ran
out last month}.

Any business that wanted the grant was part of the applicant pool, even
if it had no intention of taking a loan. (Applicants have up to 60 days
to make a decision about taking the loan.)

It is not clear what role that uncertainty played in capping loan
amounts, and agency officials have offered little clarity to lawmakers
about the loan limit.

During a House hearing last month, Ms. Carranza was pressed by
representatives from both parties about why the agency had not lifted
the \$150,000 limit. She said she would ``continue assessing it.''

Two senators --- John Cornyn, Republican of Texas, and Jacky Rosen,
Democrat of Nevada ---
\href{https://www.cornyn.senate.gov/content/news/cornyn-introduces-bipartisan-legislation-provide-increased-eidl-loans-and-eidl-advance}{introduced
legislation on July 21} that would provide the agency with billions more
for its disaster loan program and prohibit it from capping loans at less
than \$2 million.

Ms. Rosen said the agency had not explained its ``arbitrary'' caps. The
agency has ``refused to publicly request more financial support for
EIDL, despite small businesses across the country struggling to cover
their operating costs,'' she said.

The cap has left many borrowers with loans that they fear will not be
enough to keep their businesses afloat.

Nicholas Johnson runs \href{http://sucasa-furniture.com/}{Su Casa}, a
furniture retailer with four stores in Maryland and Delaware. After all
his shops were shuttered in March, he calculated that he would need
around \$500,000 to keep the company alive.

He got \$157,000 in April through the Paycheck Protection Program, which
he did not tap into until his stores started reopening in late May and
his staff members began to return. Based on his operating costs and
revenue, he expected to qualify for a \$380,000 disaster loan.

Receiving an offer in May for just \$150,000 was ``like a punch in the
gut,'' he said. He spent many sleepless nights, he said, wondering how
he would fill his projected \$200,000 shortfall.

So far, Mr. Johnson is managing to survive on higher-than-expected sales
from his reopened stores, but he is anticipating rough months ahead.
``My supply chains are all but broken,'' he said. ``At some point,
revenue will taper off again because I won't have anything to sell. I'm
trying to build a buffer, because I know there's more pain to come.''

For some, the cap is a minor impediment: Joy Parisi, the owner of
\href{https://www.paragraphny.com/}{Paragraph}, a writers space with two
locations in New York City, said her disaster loan was enough to give
her breathing room to chip away at unpaid bills and overdue rent.

But others would borrow more from the program if they could. Ms. Keefer
also received a \$48,000 P.P.P. loan, which she is using to pay two
employees, but it did not come close to closing the gap.

With her wholesale business in tatters, she pivoted to consumer sales.
The disaster loan paid off her most urgent bills and allowed her to hire
an agency to improve her retail website. Then she started buying ads on
Facebook and Instagram.

The strategy shift has helped: In June, she more than doubled what she
sold directly in all of last year. But that is still only a sliver of
what she would usually make. And now Ms. Keefer needs cash to start
manufacturing her fall and winter merchandise.

Seeing no other options, she took out an expensive loan from an online
lender. It feels, she said, like a payday loan: ``You have to start
paying it back immediately, and it's like a trap --- you end up
borrowing more just to keep up.''

The cash crunch has forced her to manufacture her clothing in smaller,
more expensive batches; limit her marketing budget; and hold off on
rehiring more workers. If she could borrow more money from the
government, she said, she would immediately spend it on expanding her
company --- exactly the kind of economic activity the government wants
to encourage.

``The EIDL loan is perfect; it's exactly what we need to steady our
ship,'' Ms. Keefer said. ``We just need more of it.''

Advertisement

\protect\hyperlink{after-bottom}{Continue reading the main story}

\hypertarget{site-index}{%
\subsection{Site Index}\label{site-index}}

\hypertarget{site-information-navigation}{%
\subsection{Site Information
Navigation}\label{site-information-navigation}}

\begin{itemize}
\tightlist
\item
  \href{https://help.nytimes3xbfgragh.onion/hc/en-us/articles/115014792127-Copyright-notice}{©~2020~The
  New York Times Company}
\end{itemize}

\begin{itemize}
\tightlist
\item
  \href{https://www.nytco.com/}{NYTCo}
\item
  \href{https://help.nytimes3xbfgragh.onion/hc/en-us/articles/115015385887-Contact-Us}{Contact
  Us}
\item
  \href{https://www.nytco.com/careers/}{Work with us}
\item
  \href{https://nytmediakit.com/}{Advertise}
\item
  \href{http://www.tbrandstudio.com/}{T Brand Studio}
\item
  \href{https://www.nytimes3xbfgragh.onion/privacy/cookie-policy\#how-do-i-manage-trackers}{Your
  Ad Choices}
\item
  \href{https://www.nytimes3xbfgragh.onion/privacy}{Privacy}
\item
  \href{https://help.nytimes3xbfgragh.onion/hc/en-us/articles/115014893428-Terms-of-service}{Terms
  of Service}
\item
  \href{https://help.nytimes3xbfgragh.onion/hc/en-us/articles/115014893968-Terms-of-sale}{Terms
  of Sale}
\item
  \href{https://spiderbites.nytimes3xbfgragh.onion}{Site Map}
\item
  \href{https://help.nytimes3xbfgragh.onion/hc/en-us}{Help}
\item
  \href{https://www.nytimes3xbfgragh.onion/subscription?campaignId=37WXW}{Subscriptions}
\end{itemize}
