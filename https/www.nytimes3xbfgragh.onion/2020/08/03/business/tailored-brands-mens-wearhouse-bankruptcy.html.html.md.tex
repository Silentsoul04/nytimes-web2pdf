Sections

SEARCH

\protect\hyperlink{site-content}{Skip to
content}\protect\hyperlink{site-index}{Skip to site index}

\href{https://www.nytimes3xbfgragh.onion/section/business}{Business}

\href{https://myaccount.nytimes3xbfgragh.onion/auth/login?response_type=cookie\&client_id=vi}{}

\href{https://www.nytimes3xbfgragh.onion/section/todayspaper}{Today's
Paper}

\href{/section/business}{Business}\textbar{}Men's Wearhouse Owner Files
for Bankruptcy

\url{https://nyti.ms/39RFTxK}

\begin{itemize}
\item
\item
\item
\item
\item
\end{itemize}

Advertisement

\protect\hyperlink{after-top}{Continue reading the main story}

Supported by

\protect\hyperlink{after-sponsor}{Continue reading the main story}

\hypertarget{mens-wearhouse-owner-files-for-bankruptcy}{%
\section{Men's Wearhouse Owner Files for
Bankruptcy}\label{mens-wearhouse-owner-files-for-bankruptcy}}

Tailored Brands, known for its clothing chains Men's Wearhouse and JoS.
A. Bank, struggled as the pandemic shut stores and consumer demand for
office attire dropped.

\includegraphics{https://static01.graylady3jvrrxbe.onion/images/2020/06/23/business/00tailored-brands-HFO/merlin_80417494_48f6049a-c9db-44a5-a884-e972884eeed3-articleLarge.jpg?quality=75\&auto=webp\&disable=upscale}

By Gillian Friedman

\begin{itemize}
\item
  Aug. 3, 2020
\item
  \begin{itemize}
  \item
  \item
  \item
  \item
  \item
  \end{itemize}
\end{itemize}

The owner of Men's Wearhouse and JoS. A. Bank, which once dominated the
market for affordable men's suits, filed for bankruptcy protection late
Sunday, as demand plummeted for its corporate clothing with the
coronavirus pandemic keeping America's office workers at home.

The company, Tailored Brands, had about 1,400 stores and 18,000
employees. It had already
\href{https://www.businesswire.com/news/home/20200721005319/en/Tailored-Brands-Announces-Plans-Reduce-Headcount-Close}{announced}
plans in July to eliminate 20 percent of its corporate jobs and close up
to 500 stores, and on Sunday said it planned to use the restructuring
process to slash its debt by at least \$630 million.

``Our enduring commitment to help customers look and feel their best
will allow us to overcome the challenges of Covid-19,'' Dinesh Lathi,
chief executive of Tailored Brands, said in a
\href{https://www.tailoredstronger.com/wp-content/uploads/2020/08/TLRD-Infographic_08022020.pdf}{statement}
accompanying the filing in U.S. Bankruptcy Court for the Southern
District of Texas.

The apparel industry has been hit particularly hard by the pandemic,
prompting bankruptcy filings from retailers like the
\href{https://www.nytimes3xbfgragh.onion/2020/05/07/business/neiman-marcus-bankruptcy.html}{Neiman
Marcus Group},
\href{https://www.nytimes3xbfgragh.onion/2020/05/03/business/j-crew-bankruptcy-coronavirus.html}{J.
Crew}and
\href{https://www.nytimes3xbfgragh.onion/2020/05/15/business/jc-penney-bankruptcy-coronavirus.html}{J.C.
Penney}.
\href{https://www.nytimes3xbfgragh.onion/2020/08/02/business/Lord-and-Taylor-Bankruptcy.html}{Lord
\& Taylor}, once a major presence in America's department stores, and
its owner, Le Tote, filed for bankruptcy several hours before Tailored
Brands on Sunday. The owner of Ann Taylor and Lane Bryant,
\href{https://www.nytimes3xbfgragh.onion/2020/07/23/business/ascena-bankruptcy-ann-taylor-lane-bryant.html}{Ascena
Retail}, which just a few years ago was one of the country's largest
retailers for affordable professional clothing for women, sought Chapter
11 protection on July 23.

Many clothing stores shut their doors during the lockdowns, leading to
unpaid rents and staff furloughs. Brick-and-mortar retailers had already
been struggling to adapt to the rise of e-commerce and changing consumer
behavior.

With millions of Americans unemployed or working from home, and a pause
on proms and weddings, demand has plummeted for Tailored Brands' core
product: men's suits. The company
\href{https://ir.tailoredbrands.com/press-releases/detail/1893/tailored-brands-inc-provides-business-update-reports}{reported}
that in the three months that ended on May 2, net sales fell 60.4
percent from a year earlier.

\href{https://www.nytimes3xbfgragh.onion/2020/07/08/business/brooks-brothers-chapter-11-bankruptcy.html}{Brooks
Brothers}, a more upscale seller of suits and preppy clothes that has
been in business since 1818, also saw demand for its wares crater during
the pandemic. It filed for bankruptcy in early July.

Men's Wearhouse was founded in 1973 by George Zimmer, who became known
for his catchy slogan in TV and radio commercials: ``You're going to
like the way you look. I guarantee it.'' The business, catering to the
common man who wanted to look sharp for work without breaking the bank,
took off. Men's Wearhouse had about 100 stores when it went public in
1992.

``They came out with an inexpensive option that allowed a guy to go in
and buy everything from one place, all at a certain quality and all at a
certain price point,'' said Mark-Evan Blackman, assistant professor and
men's wear specialist at the Fashion Institute of Technology. ``For many
years, they were considered by certain customers to be the only game in
town.''

But the company's troubles predate the pandemic. In 2013, the company
abruptly
\href{https://www.nytimes3xbfgragh.onion/2013/06/20/business/dumping-the-face-and-founder-of-mens-wearhouse.html}{fired
Mr. Zimmer}, the executive chairman, saying he had been unwilling to
cede control to the board and had pushed to sell the company to private
investors against shareholders' interests. In response, Mr. Zimmer
\href{https://blogs.wsj.com/corporate-intelligence/2013/06/26/an-open-letter-from-the-ousted-mens-wearhouse-boss/}{released
a letter}expressing his concerns that the company was heading in the
wrong direction.

Mr. Zimmer, 71, said it was painful to see the company seek bankruptcy
protection after he had invested so much of his life building it. He
attributes the company's downfall to decisions made after his
contentious exit.

``It's a crying shame,'' Mr. Zimmer said in an interview. ``I spent 40
years creating a really neat company, and it only took seven years to
destroy it.''

In 2014, Men's Wearhouse acquired the men's wear company JoS. A. Bank,
forming the parent organization Tailored Brands. The merger was meant to
unite the two retail companies and capture a bigger share of the market
of budget-conscious suit buyers and renters.

\includegraphics{https://static01.graylady3jvrrxbe.onion/images/2020/07/31/business/00virus-tailoredbrands-2/00virus-tailoredbrands-2-articleLarge.jpg?quality=75\&auto=webp\&disable=upscale}

Instead of increasing sales, the merger mired Tailored Brands in debt.
On May 2, the company had long-term debt of \$1.4 billion and \$244.2
million of cash and cash equivalents.

``When you merge two poorly performing companies together and layer on a
lot of debt, it's usually not a recipe for success, and it hasn't
been,'' said Ivan Feinseth, director of research for Tigress Financial
Partners.

The merger was ill conceived, Mr. Feinseth said, because the two
companies had fundamentally different business models and it was
difficult and expensive to consolidate their inventory and
brick-and-mortar locations into one seamless enterprise.

At the same time, Tailored Brands faced other pressures. It struggled to
compete with the rise of fast fashion and the dominance of online
retailers, while saddled with the extensive real estate and operating
costs of maintaining stores.

It was also hurt by the
\href{https://www.nytimes3xbfgragh.onion/2016/05/26/fashion/office-fashion-uniforms.html}{relaxation
of office dress codes}, inspired by tech start-up culture. The casual
workwear trend had such sweeping influence that Goldman Sachs, a leader
in an industry known for its formality, gave its employees the
\href{https://www.nytimes3xbfgragh.onion/2019/03/06/business/dealbook/goldman-sachs-dress-code.html}{green
light} in 2019 to wear casual clothes.

``Fifteen years ago, every guy had a suit in his closet, whether he was
a plumber or a middle management administrator,'' said Mr. Blackman of
the Fashion Institute of Technology. ``That's no longer the case.
Tailored garments have been hurting for a very long time.''

Tailored Brands, like most of the retailers that have filed for
bankruptcy during the pandemic, plans to stay in business and use the
Chapter 11 filing to cut down on debt and close stores. To sustain its
business, Tailored Brands will have to reinvent its business model and
significantly improve its online presence, said Anthony Campagna, global
director of fundamental research at ISS EVA, an analytics firm.

``There is a place in the market you can sell lower-tier men's
clothing,'' he said. ``It's just a matter of positioning it correctly.''

Advertisement

\protect\hyperlink{after-bottom}{Continue reading the main story}

\hypertarget{site-index}{%
\subsection{Site Index}\label{site-index}}

\hypertarget{site-information-navigation}{%
\subsection{Site Information
Navigation}\label{site-information-navigation}}

\begin{itemize}
\tightlist
\item
  \href{https://help.nytimes3xbfgragh.onion/hc/en-us/articles/115014792127-Copyright-notice}{©~2020~The
  New York Times Company}
\end{itemize}

\begin{itemize}
\tightlist
\item
  \href{https://www.nytco.com/}{NYTCo}
\item
  \href{https://help.nytimes3xbfgragh.onion/hc/en-us/articles/115015385887-Contact-Us}{Contact
  Us}
\item
  \href{https://www.nytco.com/careers/}{Work with us}
\item
  \href{https://nytmediakit.com/}{Advertise}
\item
  \href{http://www.tbrandstudio.com/}{T Brand Studio}
\item
  \href{https://www.nytimes3xbfgragh.onion/privacy/cookie-policy\#how-do-i-manage-trackers}{Your
  Ad Choices}
\item
  \href{https://www.nytimes3xbfgragh.onion/privacy}{Privacy}
\item
  \href{https://help.nytimes3xbfgragh.onion/hc/en-us/articles/115014893428-Terms-of-service}{Terms
  of Service}
\item
  \href{https://help.nytimes3xbfgragh.onion/hc/en-us/articles/115014893968-Terms-of-sale}{Terms
  of Sale}
\item
  \href{https://spiderbites.nytimes3xbfgragh.onion}{Site Map}
\item
  \href{https://help.nytimes3xbfgragh.onion/hc/en-us}{Help}
\item
  \href{https://www.nytimes3xbfgragh.onion/subscription?campaignId=37WXW}{Subscriptions}
\end{itemize}
