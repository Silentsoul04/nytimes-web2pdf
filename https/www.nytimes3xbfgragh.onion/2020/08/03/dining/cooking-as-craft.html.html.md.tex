Sections

SEARCH

\protect\hyperlink{site-content}{Skip to
content}\protect\hyperlink{site-index}{Skip to site index}

\href{https://www.nytimes3xbfgragh.onion/section/food}{Food}

\href{https://myaccount.nytimes3xbfgragh.onion/auth/login?response_type=cookie\&client_id=vi}{}

\href{https://www.nytimes3xbfgragh.onion/section/todayspaper}{Today's
Paper}

\href{/section/food}{Food}\textbar{}Cooking as Craft

\url{https://nyti.ms/2XmQGuC}

\begin{itemize}
\item
\item
\item
\item
\item
\end{itemize}

Advertisement

\protect\hyperlink{after-top}{Continue reading the main story}

Supported by

\protect\hyperlink{after-sponsor}{Continue reading the main story}

\href{/column/what-to-cook}{What to Cook}

\hypertarget{cooking-as-craft}{%
\section{Cooking as Craft}\label{cooking-as-craft}}

\includegraphics{https://static01.graylady3jvrrxbe.onion/images/2019/07/22/dining/aw-tomato-and-peach-salad-with-whipped-goat-cheese/merlin_157887927_dd555ee7-3af4-4efc-b3aa-dff8cbaba8bf-articleLarge.jpg?quality=75\&auto=webp\&disable=upscale}

By \href{https://www.nytimes3xbfgragh.onion/by/sam-sifton}{Sam Sifton}

\begin{itemize}
\item
  Aug. 3, 2020
\item
  \begin{itemize}
  \item
  \item
  \item
  \item
  \item
  \end{itemize}
\end{itemize}

Good morning. ``Manual competence'' is the ideal Matthew Crawford speaks
up for in his excellent 2010 book,
``\href{https://www.penguinrandomhouse.com/books/301618/shop-class-as-soulcraft-by-matthew-b-crawford/}{Shop
Class as Soulcraft: An Inquiry Into the Value of Work}.'' Crawford's a
mechanic and a philosopher. He has no truck with craftsmanship or
artistry. He speaks instead for the importance of trade work --- ``work
that is meaningful because it is genuinely useful,'' he writes. He finds
that work engaging intellectually, and he finds in it an argument that
celebrates responsibility and personal agency. (He's a philosopher,
remember.)

I think that's right, and I think it's right about cooking especially.
The work that we do in the kitchen is at its best an exhibition of
manual competence --- making bread dough or flipping omelets or dicing
carrots --- and a salute to the responsibility we take on in feeding
ourselves and our families. There's no question it's good to be able to
work with your hands to make something useful like dinner. And it is
meaningful, too, both for you and those you serve.

So practice and learn, if you haven't already.
\href{https://cooking.nytimes3xbfgragh.onion/guides/11-how-to-roast-chicken}{Learn
to roast chicken}. Learn
\href{https://cooking.nytimes3xbfgragh.onion/guides/24-how-to-make-pancakes}{how
to make pancakes} and
\href{https://cooking.nytimes3xbfgragh.onion/guides/49-how-to-make-rice}{how
to make perfect rice}.
\href{https://cooking.nytimes3xbfgragh.onion/guides/7-how-to-grill}{Learn
how to grill}. Manual competence in the kitchen will pay dividends for
the rest of your life.

And if you're a scratch cook already? Cook something new. I like this
\href{https://cooking.nytimes3xbfgragh.onion/recipes/1021294-ginger-lime-chicken}{ginger-lime
chicken} number, for instance. And this
\href{https://cooking.nytimes3xbfgragh.onion/recipes/1020367-tomato-and-peach-salad-with-whipped-goat-cheese}{tomato
and peach salad with whipped goat cheese} (above) as well. Make
\href{https://cooking.nytimes3xbfgragh.onion/recipes/1016866-pork-schnitzel-with-quick-pickles}{pork
schnitzel with quick pickles}, and you'll end up schnitzeling
everything, pickling everything, too.

I hope you'll make
\href{https://cooking.nytimes3xbfgragh.onion/recipes/1014721-shakshuka-with-feta}{shakshuka
with feta} at some point soon, and
\href{https://cooking.nytimes3xbfgragh.onion/recipes/1014868-wild-salmon-with-green-sauce}{wild
salmon with green sauce} as well. Take those zucchinis from the farmers'
market, make them into
\href{https://cooking.nytimes3xbfgragh.onion/recipes/1017522-olive-oil-zucchini-bread}{olive
oil zucchini bread}. Take the early corn and make
\href{https://cooking.nytimes3xbfgragh.onion/recipes/1020354-shrimp-linguine-with-herbs-corn-and-arugula}{shrimp
linguine with herbs, corn and arugula}.

This is all labor, but still some distance from digging ditches. You can
even have a cocktail while you work, as with this recipe
\href{https://cooking.nytimes3xbfgragh.onion/recipes/1020019-pasta-with-chickpeas-and-a-negroni}{for
pasta with chickpeas and a Negroni}.

\href{https://cooking.nytimes3xbfgragh.onion/recipes/1020080-baked-spinach-artichoke-pasta}{Baked
spinach-artichoke pasta}?
\href{https://cooking.nytimes3xbfgragh.onion/recipes/1020543-mayo-marinated-chicken-with-chimichurri}{Mayo-marinated
chicken with chimichurri}? There are thousands and thousands of recipes
to work on waiting for you on
\href{https://cooking.nytimes3xbfgragh.onion/}{NYT Cooking}. It's true
you need a subscription to access them all, and to use the features of
our site and apps. Subscriptions support our work. They allow it to
continue. If you haven't already, will you please
\href{https://www.nytimes3xbfgragh.onion/subscription/cooking.html?campaignId=6XQHR}{subscribe
today}?

And we'll be standing by in case something goes sideways with your
cooking or our code. Just reach out to us at
\href{mailto:cookingcare@NYTimes.com}{\nolinkurl{cookingcare@NYTimes.com}}.
Someone will get back to you, promise.

Now, it's nothing to do with preserving peaches or grilling tautog, but
Joe Coscarelli's latest
``\href{https://www.nytimes3xbfgragh.onion/2020/07/30/arts/music/phoebe-bridgers-kyoto.html}{Diary
of a Song},'' featuring Phoebe Bridgers and
``\href{https://www.youtube.com/watch?v=Tw0zYd0eIlk}{Kyoto},'' is a
master class in new-format journalism, and really fascinating to boot.
And while you're at it, here's
``\href{https://www.youtube.com/watch?v=WJ9-xN6dCW4}{I Know the End},''
which features the primal scream Bridgers talks about in Joe's story.

Also in The Times, Malia Wollan has the inside scoop on
\href{https://www.nytimes3xbfgragh.onion/2020/07/28/magazine/how-to-stop-biting-your-nails.html}{how
to stop biting your nails}.

You bet I'm going to read Natasha Trethewey's new memoir,
``\href{https://www.nytimes3xbfgragh.onion/2020/07/27/books/review-memorial-drive-memoir-natasha-trethewey.html}{Memorial
Drive}.''

Finally, to end where we started, of course
\href{https://www.youtube.com/watch?v=xdGky1JZovg}{Matthew Crawford has
a TED Talk}. It's about making things and fixing things. You might take
that in as well. And I'll see you on Wednesday.

Advertisement

\protect\hyperlink{after-bottom}{Continue reading the main story}

\hypertarget{site-index}{%
\subsection{Site Index}\label{site-index}}

\hypertarget{site-information-navigation}{%
\subsection{Site Information
Navigation}\label{site-information-navigation}}

\begin{itemize}
\tightlist
\item
  \href{https://help.nytimes3xbfgragh.onion/hc/en-us/articles/115014792127-Copyright-notice}{©~2020~The
  New York Times Company}
\end{itemize}

\begin{itemize}
\tightlist
\item
  \href{https://www.nytco.com/}{NYTCo}
\item
  \href{https://help.nytimes3xbfgragh.onion/hc/en-us/articles/115015385887-Contact-Us}{Contact
  Us}
\item
  \href{https://www.nytco.com/careers/}{Work with us}
\item
  \href{https://nytmediakit.com/}{Advertise}
\item
  \href{http://www.tbrandstudio.com/}{T Brand Studio}
\item
  \href{https://www.nytimes3xbfgragh.onion/privacy/cookie-policy\#how-do-i-manage-trackers}{Your
  Ad Choices}
\item
  \href{https://www.nytimes3xbfgragh.onion/privacy}{Privacy}
\item
  \href{https://help.nytimes3xbfgragh.onion/hc/en-us/articles/115014893428-Terms-of-service}{Terms
  of Service}
\item
  \href{https://help.nytimes3xbfgragh.onion/hc/en-us/articles/115014893968-Terms-of-sale}{Terms
  of Sale}
\item
  \href{https://spiderbites.nytimes3xbfgragh.onion}{Site Map}
\item
  \href{https://help.nytimes3xbfgragh.onion/hc/en-us}{Help}
\item
  \href{https://www.nytimes3xbfgragh.onion/subscription?campaignId=37WXW}{Subscriptions}
\end{itemize}
