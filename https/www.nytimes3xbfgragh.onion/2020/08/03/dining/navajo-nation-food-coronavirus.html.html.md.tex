Sections

SEARCH

\protect\hyperlink{site-content}{Skip to
content}\protect\hyperlink{site-index}{Skip to site index}

\href{https://www.nytimes3xbfgragh.onion/section/food}{Food}

\href{https://myaccount.nytimes3xbfgragh.onion/auth/login?response_type=cookie\&client_id=vi}{}

\href{https://www.nytimes3xbfgragh.onion/section/todayspaper}{Today's
Paper}

\href{/section/food}{Food}\textbar{}For the Navajo Nation, a Fight for
Better Food Gains New Urgency

\url{https://nyti.ms/33nbMgk}

\begin{itemize}
\item
\item
\item
\item
\item
\item
\end{itemize}

\href{https://www.nytimes3xbfgragh.onion/spotlight/at-home?action=click\&pgtype=Article\&state=default\&region=TOP_BANNER\&context=at_home_menu}{At
Home}

\begin{itemize}
\tightlist
\item
  \href{https://www.nytimes3xbfgragh.onion/2020/08/03/well/family/the-benefits-of-talking-to-strangers.html?action=click\&pgtype=Article\&state=default\&region=TOP_BANNER\&context=at_home_menu}{Talk:
  To Strangers}
\item
  \href{https://www.nytimes3xbfgragh.onion/2020/08/01/at-home/coronavirus-make-pizza-on-a-grill.html?action=click\&pgtype=Article\&state=default\&region=TOP_BANNER\&context=at_home_menu}{Make:
  Grilled Pizza}
\item
  \href{https://www.nytimes3xbfgragh.onion/2020/07/31/arts/television/goldbergs-abc-stream.html?action=click\&pgtype=Article\&state=default\&region=TOP_BANNER\&context=at_home_menu}{Watch:
  'The Goldbergs'}
\item
  \href{https://www.nytimes3xbfgragh.onion/interactive/2020/at-home/even-more-reporters-editors-diaries-lists-recommendations.html?action=click\&pgtype=Article\&state=default\&region=TOP_BANNER\&context=at_home_menu}{Explore:
  Reporters' Google Docs}
\end{itemize}

Advertisement

\protect\hyperlink{after-top}{Continue reading the main story}

Supported by

\protect\hyperlink{after-sponsor}{Continue reading the main story}

\hypertarget{for-the-navajo-nation-a-fight-for-better-food-gains-new-urgency-}{%
\section{For the Navajo Nation, a Fight for Better Food Gains New
Urgency
}\label{for-the-navajo-nation-a-fight-for-better-food-gains-new-urgency-}}

As the pandemic has brought home the importance of the global movement
for food sovereignty, members are planting and sharing.

\includegraphics{https://static01.graylady3jvrrxbe.onion/images/2020/08/05/dining/30Navajo1/merlin_174907557_fa5e6075-b20f-41ba-867c-2dc7b20a4da9-articleLarge.jpg?quality=75\&auto=webp\&disable=upscale}

By \href{https://www.nytimes3xbfgragh.onion/by/amelia-nierenberg}{Amelia
Nierenberg}

\begin{itemize}
\item
  Aug. 3, 2020
\item
  \begin{itemize}
  \item
  \item
  \item
  \item
  \item
  \item
  \end{itemize}
\end{itemize}

THE NAVAJO NATION --- When Summer Brown lived in Phoenix, she had no
problem finding fresh produce. If the Sprouts supermarket near her home
didn't have what she was looking for, she would just drive somewhere
else.

This winter, Ms. Brown, an enrolled member of the Navajo Nation, moved
back to her childhood home in Cornfields, Ariz., to start a
\href{https://www.lotusandlayneleather.com/}{small business} as a
leatherworker. Now, healthy food is harder to find for her two children,
Paisley, 6, and Landon, 7. The entire Nation, which stretches 27,000
square miles across Arizona, New Mexico and Utah, has fewer than 15
grocery stores.

``The pickings are kind of slim here,'' said Ms. Brown, 31. ``It's a lot
of processed foods, and I try not to feed my family that.''

Even before the coronavirus pandemic
\href{https://www.nytimes3xbfgragh.onion/2020/05/11/us/coronavirus-native-americans-indian-country.html}{dealt
an exceptionally brutal blow to the Navajo} --- who call themselves the
Diné, which means ``the People'' --- Ms. Brown wanted to grow her own
food. She spent last winter collecting seeds from Indigenous seed banks
and researching Indigenous methods. Her small garden is already feeding
her family, and she is looking forward to the fall harvest.

Her backyard garden isn't meant just to replace a trip to the grocery
store. Ms. Brown is part of a movement for food sovereignty,
\href{https://foodsecurecanada.org/who-we-are/what-food-sovereignty}{a
global effort to give people control of their food supply and
nutrition}. It is a public health endeavor, an economic reclamation, an
environmental protest and for many, a spiritual quest. Gardeners aim to
grow
\href{https://www.nytimes3xbfgragh.onion/2020/04/13/dining/native-americans-coronavirus.html}{healthy
foods} that are connected to their traditions, and to revive old methods
of cultivation.

\includegraphics{https://static01.graylady3jvrrxbe.onion/images/2020/08/05/dining/30Navajo2/merlin_174907428_999b04c6-31b4-408f-99f1-be6450356bb1-articleLarge.jpg?quality=75\&auto=webp\&disable=upscale}

``I want to show the whole Navajo Nation, and even off the reservation,
that you can live with the earth and you don't have to rely so much on
the outside to feed yourself,'' Ms. Brown said. ``We have all this land.
We should be able to just go outside and get our food.''

The small gardens and cornfields rising across the Nation (which the
Diné call the Dinétah) are attempts to correct legacies of historical
wrongs. Once, the Diné were prosperous gardeners, hunters and stewards
of the land. Then the United States government
\href{https://www.nytimes3xbfgragh.onion/2020/05/13/opinion/sunday/navajo-nation-coronavirus.html}{colonized
the land and displaced} the Diné in the mid-1800s, during what is now
known as
\href{https://americanindian.si.edu/nk360/navajo/long-walk/long-walk.cshtml}{the
Long Walk}, to an internment camp at Fort Sumner, N.M. Livestock were
killed off. Fields were trampled. And some orchards were lost forever.

Those and other attempts to
\href{https://www.firstnations.org/wp-content/uploads/publication-attachments/Dine_Policy_Institute_Food_Sovereignty_Report.pdf}{divorce
the Diné from their land and ancestral foodways} have also left them
vulnerable to the pandemic. Across the United States, Indigenous nations
have suffered outbreaks that often appear to be more devastating than
those in surrounding cities. But
\href{https://www.nytimes3xbfgragh.onion/2020/07/30/us/native-americans-coronavirus-data.html}{data
gaps, population fluidity and under-testing make the scope of infections
hard to quantify}.

The Diné have weathered
\href{https://www.reuters.com/article/us-health-coronavirus-usa-navajo/facing-arizona-surge-navajos-reimpose-virus-curfew-idUSKBN23O3R4}{curfews}
and high rates of infection. The Nation is one of the hardest hit areas
in the United States: 9,019 people have tested positive and 454 have
died of Covid-19, as of July 30, according to
\href{https://www.ndoh.navajo-nsn.gov/COVID-19}{the Navajo Department of
Health}.

Many households do not have running water, at a time when hand washing
is critical. Many multigenerational families live together in compounds,
which makes social distancing impossible. And for the Diné and many
other Indigenous nations, the public health crises caused by food
inequality are generations old.

Image

Nate Etsitty teaches young gardeners, and cultivates Indigenous seeds
using traditional methods.Credit...John Burcham for The New York Times

``This virus has really shown how fragile and even unreliable the system
really is, and how quickly everything can collapse,'' said Nate Etsitty,
40, a food sovereignty advocate who has been helping Ms. Brown with her
garden. ``That's what is driving more people to be partially more
self-sustained.''

After seeing food shortages during the pandemic, many Diné have started
gardens. Normally, they would work communally, but social distancing has
required some innovations. This year, Mx. Etsitty (who uses
gender-neutral pronouns and titles) has been helping first-time
gardeners through the complex processes from afar. Other experienced
gardeners, inundated with requests for help, are recording videos.

Felix Earle, 43, one of Mx. Etsitty's closest friends, has been advising
gardeners growing Indigenous seeds. In 2015, he found a handful of white
corn kernels in a jar, 35 years after his grandmother hid them for
safekeeping. He named the strain ``Grandma Helen's Corn.'' Its kernels
look like little white teeth, perfect and round.

This year, Mr. Earle, a fashion designer, planted his biggest crop ever.
Across his property, stalks of corn are rising, almost 1,000 in all. He
turned his discovery into a business, Red Earth Gardens,~and gives
kernels to interested members of the Nation. This year, for the first
time, he ran out.

``It took a deadly virus to make people realize just how important this
is, how important it is to grow your own food,'' he said.

Image

Felix Earle helped grow a cornfield on land his maternal and paternal
grandparents once walked in Ganado, Ariz. He and Mx. Etsitty worked to
rehabilitate the land, Red Point Farms, with the assistance of the
nonprofit Black Mesa Water Coalition and his father's clan, Waters
Edge.Credit...John Burcham for The New York Times

A global Indigenous food sovereignty movement began well before the
pandemic. Some point to 2002 as a transitional moment, when
representatives from Indigenous nations in 28 countries gathered at Lake
Atitlán, in Panajachel, Guatemala, to write
\href{https://www.iitc.org/wp-content/uploads/2013/07/FINAL_Atitlan-Declaration-Food-Security_Apr25_ENGL.pdf}{a
declaration of the right to food}.

``We know that food is medicine, but it can also make us more sick,''
said Denisa Livingston, a leading Diné community health advocate and the
Slow Food International Indigenous Councilor of the Global North. ``It
can contaminate our health and well-being.''

With limited sources of income to pay for the gasoline for long trips to
faraway stores, families buy cheap bulk foods with a long shelf life,
instead of fresh produce. Today, the Diné have high rates of diabetes,
heart disease and obesity, conditions that pose a higher risk of
complications or death from Covid-19.

Many Diné also receive
\href{https://gardenwarriorsgoodseeds.com/2020/05/03/native-food-systems-in-the-time-of-covid-19/}{federal
food benefits}. ``You've got to stretch those funds, and the cheapest
out there is junk food,'' said Artie Yazzie, a community gardener, who
grows produce for his neighbors.

Image

``People come in here and pick whatever they want,'' Mr. Yazzie said.
``I just leave a sign.''Credit...John Burcham for The New York Times

``We have these big old thirst-busters that cost 60 cents, and three
people can share it,'' he added, using a popular term for soda. ``So
we're going to go ahead and buy that instead of the \$3 water or milk.''

Some programs are working to get fresh produce to Diné children. The
\href{https://www.copeprogram.org/}{Community Outreach and Patient
Empowerment} program, a nonprofit health partnership, provides vouchers
for families with young children that are good for buying only fruits,
vegetables and traditional foods. The amount, depending on family size,
can go up to \$35 a week.

Image

A watermelon plant in the Teesto Community Garden.Credit...John Burcham
for The New York Times

But when children turn 5, they are usually no longer eligible for
vouchers. Dr. Lydia Kim, a pediatrician with the Indian Health Services,
sees the damage that unreliable access to food and water can wreak on
her patients. Some who struggle with obesity drink five or six sodas a
day.

With help from community partners, Dr. Kim started the Shiprock Area
Food Access Coalition in 2018 to increase access to healthy food and
water. She asked managers of a local grocery to move produce to the
front of the store, so more shoppers might see and buy them. The
managers told her their hands were tied, she said, because soft-drink
companies owned the displays near the entrance. (The store manager did
not respond to multiple texts seeking comment.)

Dr. Kim said she has seen many more sugary-drink displays in grocery
stores in Shiprock, N.M., than in those outside the reservation.
``There's predatory targeting and a complete lack of choice.''

Ms. Livingston and members of the
\href{https://dineadvocacy.org/?fbclid=IwAR17Oomqz5BgHuhAVh693s1hPGHkWJIrRUMzaej_G393ohQe-Y2O5gCvqH0}{Diné
Community Advocacy Alliance} lobbied the Navajo Nation government to
pass the
\href{https://www.tax.navajo-nsn.gov/Navajo\%20Taxes/Regulations/Junk\%20Food\%20Tax\%20Regulations\%20\%201-14-15.pdf}{Healthy
Diné Nation Act of 2014}. It lead to an elimination of a tax on healthy
foods, while imposing a 2 percent tax on unhealthy foods, like soda.

Image

Some gardens at schools and senior centers have been closed since March.
The Teesto Community Garden, which Mr. Yazzie tends, has remained opened
through the pandemic.Credit...John Burcham for The New York Times

The tax revenues help fund community wellness and food sovereignty
projects like the Teesto Community Garden, in the Arizona section of the
Nation. Last year, Mr. Yazzie, 39, the community gardener who oversaw
the garden as a seasonal employee, would borrow the municipal pickup
truck to fill an 1,800-gallon tank with water and irrigate the garden.
In the summer, he goes through about 3,000 gallons each week.

For much of this year's planting season, the truck was tied up
delivering emergency provisions to families in need. But for Mr. Yazzie,
planting was still a priority.

At first, he recruited his cousin to help fill a 300-gallon barrel each
week. They contributed part of their federal stimulus checks to repair a
600-gallon water tank and trailer, burning through expensive gas as they
drove back and forth.

Image

Mr. Earle keeps corn pollen in a pouch for his morning
prayers.Credit...John Burcham for The New York Times

Now Mr. Yazzie's garden is flourishing. The grit and determination of
growers and gardeners like him keeps the food sovereignty movement
alive.

Every morning, Mr. Earle, the fashion designer, climbs a hill near his
home and turns toward the sunrise to pray.

Mr. Earle started seeing reddened sunsets in January, which he and other
spiritual leaders interpreted as a warning of what was to come. Long
before coronavirus infection rates started climbing across the country,
he started praying for protection.

Facing east as night becomes day, he thinks of his grandmother Helen
Cornfields. In her old age, she would look out over the once-barren
land.

``I wonder if we planted anything, if it would grow,'' he said she once
mused. Now, he has his answer.

\emph{Follow} \href{https://twitter.com/nytfood}{\emph{NYT Food on
Twitter}} \emph{and}
\href{https://www.instagram.com/nytcooking/}{\emph{NYT Cooking on
Instagram}}\emph{,}
\href{https://www.facebookcorewwwi.onion/nytcooking/}{\emph{Facebook}}\emph{,}
\href{https://www.youtube.com/nytcooking}{\emph{YouTube}} \emph{and}
\href{https://www.pinterest.com/nytcooking/}{\emph{Pinterest}}\emph{.}
\href{https://www.nytimes3xbfgragh.onion/newsletters/cooking}{\emph{Get
regular updates from NYT Cooking, with recipe suggestions, cooking tips
and shopping advice}}\emph{.}

Advertisement

\protect\hyperlink{after-bottom}{Continue reading the main story}

\hypertarget{site-index}{%
\subsection{Site Index}\label{site-index}}

\hypertarget{site-information-navigation}{%
\subsection{Site Information
Navigation}\label{site-information-navigation}}

\begin{itemize}
\tightlist
\item
  \href{https://help.nytimes3xbfgragh.onion/hc/en-us/articles/115014792127-Copyright-notice}{©~2020~The
  New York Times Company}
\end{itemize}

\begin{itemize}
\tightlist
\item
  \href{https://www.nytco.com/}{NYTCo}
\item
  \href{https://help.nytimes3xbfgragh.onion/hc/en-us/articles/115015385887-Contact-Us}{Contact
  Us}
\item
  \href{https://www.nytco.com/careers/}{Work with us}
\item
  \href{https://nytmediakit.com/}{Advertise}
\item
  \href{http://www.tbrandstudio.com/}{T Brand Studio}
\item
  \href{https://www.nytimes3xbfgragh.onion/privacy/cookie-policy\#how-do-i-manage-trackers}{Your
  Ad Choices}
\item
  \href{https://www.nytimes3xbfgragh.onion/privacy}{Privacy}
\item
  \href{https://help.nytimes3xbfgragh.onion/hc/en-us/articles/115014893428-Terms-of-service}{Terms
  of Service}
\item
  \href{https://help.nytimes3xbfgragh.onion/hc/en-us/articles/115014893968-Terms-of-sale}{Terms
  of Sale}
\item
  \href{https://spiderbites.nytimes3xbfgragh.onion}{Site Map}
\item
  \href{https://help.nytimes3xbfgragh.onion/hc/en-us}{Help}
\item
  \href{https://www.nytimes3xbfgragh.onion/subscription?campaignId=37WXW}{Subscriptions}
\end{itemize}
