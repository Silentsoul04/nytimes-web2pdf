\href{/section/health}{Health}\textbar{}`The Biggest Monster' Is
Spreading. And It's Not the Coronavirus.

\url{https://nyti.ms/33ha3cp}

\begin{itemize}
\item
\item
\item
\item
\item
\item
\end{itemize}

\href{https://www.nytimes3xbfgragh.onion/news-event/coronavirus?action=click\&pgtype=Article\&state=default\&region=TOP_BANNER\&context=storylines_menu}{The
Coronavirus Outbreak}

\begin{itemize}
\tightlist
\item
  live\href{https://www.nytimes3xbfgragh.onion/2020/08/04/world/coronavirus-cases.html?action=click\&pgtype=Article\&state=default\&region=TOP_BANNER\&context=storylines_menu}{Latest
  Updates}
\item
  \href{https://www.nytimes3xbfgragh.onion/interactive/2020/us/coronavirus-us-cases.html?action=click\&pgtype=Article\&state=default\&region=TOP_BANNER\&context=storylines_menu}{Maps
  and Cases}
\item
  \href{https://www.nytimes3xbfgragh.onion/interactive/2020/science/coronavirus-vaccine-tracker.html?action=click\&pgtype=Article\&state=default\&region=TOP_BANNER\&context=storylines_menu}{Vaccine
  Tracker}
\item
  \href{https://www.nytimes3xbfgragh.onion/2020/08/02/us/covid-college-reopening.html?action=click\&pgtype=Article\&state=default\&region=TOP_BANNER\&context=storylines_menu}{College
  Reopening}
\item
  \href{https://www.nytimes3xbfgragh.onion/live/2020/08/04/business/stock-market-today-coronavirus?action=click\&pgtype=Article\&state=default\&region=TOP_BANNER\&context=storylines_menu}{Economy}
\end{itemize}

\includegraphics{https://static01.graylady3jvrrxbe.onion/images/2020/08/04/science/00VIRUS-GLOBAL-franyuti1/merlin_174091377_bd123d23-7d76-4af5-b50f-3ea3a28ecc3f-articleLarge.jpg?quality=75\&auto=webp\&disable=upscale}

Sections

\protect\hyperlink{site-content}{Skip to
content}\protect\hyperlink{site-index}{Skip to site index}

GLOBAL HEALTH

\hypertarget{the-biggest-monster-is-spreading-and-its-not-the-coronavirus}{%
\section{`The Biggest Monster' Is Spreading. And It's Not the
Coronavirus.}\label{the-biggest-monster-is-spreading-and-its-not-the-coronavirus}}

Tuberculosis kills 1.5 million people each year. Lockdowns and
supply-chain disruptions threaten progress against the disease as well
as H.I.V. and malaria.

Dr. Giorgio Franyuti is usually in the remote jungles of Mexico
diagnosing T.B. But since the pandemic, he has worked at a makeshift
hospital treating coronavirus patients in Mexico City.Credit...Meghan
Dhaliwal for The New York Times

Supported by

\protect\hyperlink{after-sponsor}{Continue reading the main story}

By
\href{https://www.nytimes3xbfgragh.onion/by/apoorva-mandavilli}{Apoorva
Mandavilli}

\begin{itemize}
\item
  Aug. 3, 2020
\item
  \begin{itemize}
  \item
  \item
  \item
  \item
  \item
  \item
  \end{itemize}
\end{itemize}

It begins with a mild fever and malaise, followed by a painful cough and
shortness of breath. The infection prospers in crowds, spreading to
people in close reach. Containing an outbreak requires contact tracing,
as well as isolation and treatment of the sick for weeks or months.

This insidious disease has touched every part of the globe. It is
tuberculosis, the biggest infectious-disease killer worldwide, claiming
1.5 million lives each year.

Until this year, TB and its deadly allies, H.I.V. and malaria, were on
the run. The toll from each disease over the previous decade was at its
nadir in 2018, the last year for which data are available.

Yet now, as the coronavirus pandemic spreads around the world, consuming
global health resources, these perennially neglected adversaries are
making a comeback.

``Covid-19 risks derailing all our efforts and taking us back to where
we were 20 years ago,'' said Dr. Pedro L. Alonso, the director of the
World Health Organization's global malaria program.

It's not just that the coronavirus has diverted scientific attention
from TB, H.I.V. and malaria. The lockdowns, particularly across parts of
Africa, Asia and Latin America, have raised insurmountable barriers to
patients who must travel to obtain diagnoses or drugs, according to
interviews with more than two dozen public health officials, doctors and
patients worldwide.

Fear of the coronavirus and the shuttering of clinics have kept away
many patients struggling with H.I.V., TB and malaria, while restrictions
on air and sea travel have severely limited delivery of medications to
the hardest-hit regions.

About 80 percent of tuberculosis, H.I.V. and malaria programs worldwide
have
\href{https://www.theglobalfund.org/en/covid-19/news/2020-06-17-global-fund-survey-majority-of-hiv-tb-and-malaria-programs-face-disruptions-as-a-result-of-covid-19/}{reported
disruptions} in services, and one in four people living with H.I.V. have
reported problems with gaining access to medications, according to U.N.
AIDS. Interruptions or delays in treatment may lead to drug resistance,
already a formidable problem in many countries.

\includegraphics{https://static01.graylady3jvrrxbe.onion/images/2020/07/23/science/00VIRUS-GLOBAL6/merlin_172023063_d8d1e991-50c4-4988-a828-f4be5fb5b26e-articleLarge.jpg?quality=75\&auto=webp\&disable=upscale}

In India, home to about
\href{https://www.who.int/tb/publications/global_report/gtbr2018_main_text_28Feb2019.pdf}{27
percent}of the world's TB cases, diagnoses have dropped
\href{https://reports.nikshay.in/Reports/TBNotification}{by nearly 75
percent} since the pandemic began. In Russia, H.I.V. clinics have been
\href{https://www.unaids.org/en/resources/presscentre/featurestories/2020/may/20200514_russian-federation-covid19}{repurposed
for coronavirus} testing.

Malaria season has begun in West Africa, which has 90 percent of malaria
deaths in the world, but the normal strategies for prevention ---
distribution of insecticide-treated bed nets and spraying with
pesticides ---
\href{https://www.who.int/emergencies/diseases/novel-coronavirus-2019/question-and-answers-hub/q-a-detail/malaria-and-the-covid-19-pandemic}{have
been curtailed} because of lockdowns.

According to one
\href{http://www.stoptb.org/assets/documents/news/Modeling\%20Report_1\%20May\%202020_FINAL.pdf}{estimate},
a three-month lockdown across different parts of the world and a gradual
return to normal over 10 months could result in an additional 6.3
million cases of tuberculosis and 1.4 million deaths from it.

A six-month disruption of antiretroviral therapy may lead to more
than\href{https://www.who.int/news-room/detail/11-05-2020-the-cost-of-inaction-covid-19-related-service-disruptions-could-cause-hundreds-of-thousands-of-extra-deaths-from-hiv}{500,000
additional deaths} from illnesses related to H.I.V., according to the
W.H.O. Another model by the W.H.O. predicted that in the worst-case
scenario, deaths from malaria
\href{https://www.who.int/publications/i/item/the-potential-impact-of-health-service-disruptions-on-the-burden-of-malaria}{could
double to 770,000} per year.

Several public health experts, some close to tears, warned that if the
current trends continue, the coronavirus is likely to set back years,
perhaps decades, of painstaking progress against TB, H.I.V. and malaria.

The Global Fund, a public-private partnership to fight these diseases,
estimates that mitigating this damage will
\href{https://www.theglobalfund.org/en/news/2020-06-24-global-fund-covid-19-report-deaths-from-hiv-tb-and-malaria-could-almost-double-in-12-months-unless-urgent-action-is-taken/}{require
at least \$28.5 billion}, a sum that is unlikely to materialize.

\hypertarget{delays-in-diagnosis}{%
\subsection{Delays in diagnosis}\label{delays-in-diagnosis}}

Image

With most private clinics closed, patients with H.I.V., TB and malaria
have few places to go for the kind of medical care offered at this
Doctors Without Borders clinic in Nairobi.Credit...Brian
Inganga/Associated Press

If history is any guide, the coronavirus's impact on the poor will be
felt long after the pandemic is over. The socioeconomic crisis in
Eastern Europe in the early 1990s, for example, led to the highest rates
in the world of a kind of TB that was resistant to multiple drugs, a
dubious distinction the region holds even today.

\hypertarget{latest-updates-global-coronavirus-outbreak}{%
\section{\texorpdfstring{\href{https://www.nytimes3xbfgragh.onion/2020/08/04/world/coronavirus-cases.html?action=click\&pgtype=Article\&state=default\&region=MAIN_CONTENT_1\&context=storylines_live_updates}{Latest
Updates: Global Coronavirus
Outbreak}}{Latest Updates: Global Coronavirus Outbreak}}\label{latest-updates-global-coronavirus-outbreak}}

Updated 2020-08-04T19:54:55.182Z

\begin{itemize}
\tightlist
\item
  \href{https://www.nytimes3xbfgragh.onion/2020/08/04/world/coronavirus-cases.html?action=click\&pgtype=Article\&state=default\&region=MAIN_CONTENT_1\&context=storylines_live_updates\#link-4825b93}{Public
  and private schools in Maryland and elsewhere are divided over
  in-person instruction.}
\item
  \href{https://www.nytimes3xbfgragh.onion/2020/08/04/world/coronavirus-cases.html?action=click\&pgtype=Article\&state=default\&region=MAIN_CONTENT_1\&context=storylines_live_updates\#link-4d1eafa8}{N.Y.C.'s
  health commissioner resigns after clashing with the mayor over the
  virus.}
\item
  \href{https://www.nytimes3xbfgragh.onion/2020/08/04/world/coronavirus-cases.html?action=click\&pgtype=Article\&state=default\&region=MAIN_CONTENT_1\&context=storylines_live_updates\#link-6b644638}{`Long
  days, long nights': Washington prepares for a prolonged fight over
  virus relief.}
\end{itemize}

\href{https://www.nytimes3xbfgragh.onion/2020/08/04/world/coronavirus-cases.html?action=click\&pgtype=Article\&state=default\&region=MAIN_CONTENT_1\&context=storylines_live_updates}{See
more updates}

More live coverage:
\href{https://www.nytimes3xbfgragh.onion/live/2020/08/04/business/stock-market-today-coronavirus?action=click\&pgtype=Article\&state=default\&region=MAIN_CONTENT_1\&context=storylines_live_updates}{Markets}

The starting point in this ruinous chain of events is a failure to
diagnose: The longer a person goes undiagnosed, and the later treatment
begins, the more likely an infectious disease is to spread, sicken and
kill.

For malaria, a short delay in diagnosis can swiftly turn fatal,
sometimes within just 36 hours of a spiking fever. ``It's one of those
diseases where we cannot afford to wait,'' Dr. Alonso said.

Apprehensive about malaria's rise in West Africa, the W.H.O. is now
considering giving entire populations antimalarial drugs --- a strategy
of last resort used during the Ebola epidemic in West Africa and the
Boko Haram insurgency.

Across sub-Saharan Africa, fewer women are coming to clinics for H.I.V.
diagnosis. A six-month disruption in access to drugs that prevent
H.I.V.-positive pregnant women from passing the infection to their
babies in utero could
\href{https://reliefweb.int/report/world/estimation-potential-impact-covid-19-responses-hiv-epidemic-analysis-using-goals-model}{increase
H.I.V. infections in children} by as much as 139 percent in Uganda and
162 percent in Malawi, according to U.N. AIDS.

Diminishing diagnostic capacity may have the greatest effect on TB,
leading to dire consequences for households because, like the
coronavirus, the bacterium spreads most efficiently in indoor air and
among people in close contact.

Each person with TB can spread the disease to
\href{https://www.who.int/news-room/fact-sheets/detail/tuberculosis}{another
15 individuals over a year}, sharply raising the possibility of people
infected while indoors spreading it among their communities once
lockdowns end. The prospect is especially worrisome in densely populated
places with high rates of TB, such as the favelas of Rio de Janeiro or
the townships of South Africa.

``The more you leave undiagnosed and untreated, the more you will have
next year and the year after,'' said Dr. Lucica Ditiu, who heads the
Stop TB Partnership, an international consortium of 1,700 groups
fighting the disease.

The infrastructure built to diagnose H.I.V. and TB has been a boon for
many countries grappling with the coronavirus. GeneXpert, the tool used
to detect genetic material from the TB bacteria and from H.I.V., can
also amplify RNA from the coronavirus for diagnosis.

But now most clinics are using the machines only to look for the
coronavirus. Prioritizing the coronavirus over TB is ``very stupid from
a public health perspective,'' Dr. Ditiu said. ``You should actually be
smart and do both.''

In country after country, the pandemic has resulted in
\href{https://www.medrxiv.org/content/10.1101/2020.04.28.20079582v1}{sharp
drops in diagnoses} of TB: a 70 percent decline in Indonesia, 50 percent
in Mozambique and
\href{https://www.nicd.ac.za/wp-content/uploads/2020/05/Impact-of-Covid-19-interventions-on-TB-testing-in-South-Africa-10-May-2020.pdf}{South
Africa}, and 20 percent in China, according to the W.H.O.

Image

Dr. Giorgio Franyuti said many patients with TB at a makeshift hospital
in Mexico City were being misdiagnosed with Covid-19.Credit...Meghan
Dhaliwal for The New York Times

In late May in Mexico, as coronavirus infections climbed, TB diagnoses
recorded by the government
\href{https://www.gob.mx/salud/acciones-y-programas/direccion-general-de-epidemiologia-boletin-epidemiologico}{fell
to 263 cases} from 1,097 the same week last year.

Dr. Giorgio Franyuti, the executive director of Medical Impact, an
advocacy group based in Mexico, normally works in the country's remote
jungles, diagnosing and treating TB in the Lacandon people. Unable to
travel there during the pandemic, he has worked at a makeshift army
hospital treating Covid-19 patients in Mexico City.

There, he has seen nine patients with a sputum-filled cough ---
characteristic of TB --- that began months earlier but who were presumed
to have Covid-19. The patients later contracted the coronavirus in the
hospital and became seriously ill. At least four have died.

``Nobody is testing for TB at any facility,'' he said. ``The mind of
clinicians in Mexico, as well as decision makers, is stuck with
Covid-19.''

``TB is the biggest monster of them all. If we're talking about deaths
and pandemics, 10 million cases a year,'' he said, Covid doesn't compare
yet to that toll.

India went into lockdown on March 24, and the government directed public
hospitals to focus on Covid-19. Many hospitals
\href{https://timesofindia.indiatimes.com/india/how-covid-war-is-hurting-indias-non-covid-patients/articleshow/74949121.cms}{shuttered
outpatient services} for other diseases.

The impact on TB diagnoses was immediate: The
\href{https://reports.nikshay.in/Reports/TBNotification}{number of new
cases} recorded by the Indian government between March 25 and June 19
was 60,486, compared with 179,792 during the same period in 2019.

The pandemic is also shrinking the supply of diagnostic tests for these
killers as companies turn to making more expensive tests to detect the
coronavirus. Cepheid, the California-based manufacturer of TB diagnostic
tests, has pivoted to making tests for the coronavirus. Companies that
make diagnostic tests for malaria are doing the same, according to Dr.
Catharina Boehme, the chief executive of the Foundation for Innovative
New Diagnostics.

Coronavirus tests are much more lucrative, at about \$10, compared with
18 cents for a rapid malaria test.

These companies ``have tremendous demand for Covid right now,'' said Dr.
Madhukar Pai, the director of the McGill International TB Centre in
Montreal. ``I can't imagine diseases of poverty getting any attention in
this space.''

\hypertarget{treatment-interruptions}{%
\subsection{Treatment interruptions}\label{treatment-interruptions}}

Image

Thomas Wuoto had to borrow H.I.V. medications from his wife and went
without any for 10 days during the lockdown in Nairobi, putting him at
risk of developing drug resistance.~Credit...Khadija Farah for The New
York Times

The pandemic has hindered the availability of drugs for H.I.V., TB and
malaria worldwide by interrupting supply chains, diverting manufacturing
capacity and imposing physical barriers for patients who must travel to
distant clinics to pick up the medications.

And these shortages are forcing some patients to ration their
medications, endangering their health. In Indonesia, the official policy
is to provide a month's supply of drugs at a time to H.I.V. patients,
but antiretroviral therapy has lately been hard to come by outside of
Jakarta.

Even in the city, some people are stretching a month's supply to two,
said ``Davi'' Sepi Maulana Ardiansyah, an activist with the group Inti
Muda.

Mr. Ardiansyah has done so himself, although he knows it has jeopardized
his well-being. ``This pandemic and this unavailability of the medicines
is really impacting our mental health and also our health,'' he said.

During the lockdown in Nairobi, Thomas Wuoto, who has H.I.V., borrowed
antiretroviral pills from his wife, who also is infected. As a volunteer
educator for H.I.V., Mr. Wuoto knew only too well that he was risking
drug resistance by mixing or skipping medications. When he finally made
it to the Mbagathi County Hospital, he had gone 10 days without his
H.I.V. medicines, the first time since 2002 that he had missed his
therapy.

People with H.I.V. and TB who skip medication are likely to get sicker
in the short term. In the long term, there's an even more worrisome
consequence: a rise in drug-resistant forms of these diseases. Already
drug-resistant TB is such a threat that patients are closely monitored
during treatment --- a practice that has mostly been suspended during
the pandemic.

According to the W.H.O., at least 121 countries have reported a drop in
TB patients visiting clinics since the pandemic began, threatening
hard-fought gains.

\href{https://www.nytimes3xbfgragh.onion/news-event/coronavirus?action=click\&pgtype=Article\&state=default\&region=MAIN_CONTENT_3\&context=storylines_faq}{}

\hypertarget{the-coronavirus-outbreak-}{%
\subsubsection{The Coronavirus Outbreak
›}\label{the-coronavirus-outbreak-}}

\hypertarget{frequently-asked-questions}{%
\paragraph{Frequently Asked
Questions}\label{frequently-asked-questions}}

Updated August 4, 2020

\begin{itemize}
\item ~
  \hypertarget{i-have-antibodies-am-i-now-immune}{%
  \paragraph{I have antibodies. Am I now
  immune?}\label{i-have-antibodies-am-i-now-immune}}

  \begin{itemize}
  \tightlist
  \item
    As of right
    now,\href{https://www.nytimes3xbfgragh.onion/2020/07/22/health/covid-antibodies-herd-immunity.html?action=click\&pgtype=Article\&state=default\&region=MAIN_CONTENT_3\&context=storylines_faq}{that
    seems likely, for at least several months.} There have been
    frightening accounts of people suffering what seems to be a second
    bout of Covid-19. But experts say these patients may have a
    drawn-out course of infection, with the virus taking a slow toll
    weeks to months after initial exposure. People infected with the
    coronavirus typically
    \href{https://www.nature.com/articles/s41586-020-2456-9}{produce}
    immune molecules called antibodies, which are
    \href{https://www.nytimes3xbfgragh.onion/2020/05/07/health/coronavirus-antibody-prevalence.html?action=click\&pgtype=Article\&state=default\&region=MAIN_CONTENT_3\&context=storylines_faq}{protective
    proteins made in response to an
    infection}\href{https://www.nytimes3xbfgragh.onion/2020/05/07/health/coronavirus-antibody-prevalence.html?action=click\&pgtype=Article\&state=default\&region=MAIN_CONTENT_3\&context=storylines_faq}{.
    These antibodies may} last in the body
    \href{https://www.nature.com/articles/s41591-020-0965-6}{only two to
    three months}, which may seem worrisome, but that's perfectly normal
    after an acute infection subsides, said Dr. Michael Mina, an
    immunologist at Harvard University. It may be possible to get the
    coronavirus again, but it's highly unlikely that it would be
    possible in a short window of time from initial infection or make
    people sicker the second time.
  \end{itemize}
\item ~
  \hypertarget{im-a-small-business-owner-can-i-get-relief}{%
  \paragraph{I'm a small-business owner. Can I get
  relief?}\label{im-a-small-business-owner-can-i-get-relief}}

  \begin{itemize}
  \tightlist
  \item
    The
    \href{https://www.nytimes3xbfgragh.onion/article/small-business-loans-stimulus-grants-freelancers-coronavirus.html?action=click\&pgtype=Article\&state=default\&region=MAIN_CONTENT_3\&context=storylines_faq}{stimulus
    bills enacted in March} offer help for the millions of American
    small businesses. Those eligible for aid are businesses and
    nonprofit organizations with fewer than 500 workers, including sole
    proprietorships, independent contractors and freelancers. Some
    larger companies in some industries are also eligible. The help
    being offered, which is being managed by the Small Business
    Administration, includes the Paycheck Protection Program and the
    Economic Injury Disaster Loan program. But lots of folks have
    \href{https://www.nytimes3xbfgragh.onion/interactive/2020/05/07/business/small-business-loans-coronavirus.html?action=click\&pgtype=Article\&state=default\&region=MAIN_CONTENT_3\&context=storylines_faq}{not
    yet seen payouts.} Even those who have received help are confused:
    The rules are draconian, and some are stuck sitting on
    \href{https://www.nytimes3xbfgragh.onion/2020/05/02/business/economy/loans-coronavirus-small-business.html?action=click\&pgtype=Article\&state=default\&region=MAIN_CONTENT_3\&context=storylines_faq}{money
    they don't know how to use.} Many small-business owners are getting
    less than they expected or
    \href{https://www.nytimes3xbfgragh.onion/2020/06/10/business/Small-business-loans-ppp.html?action=click\&pgtype=Article\&state=default\&region=MAIN_CONTENT_3\&context=storylines_faq}{not
    hearing anything at all.}
  \end{itemize}
\item ~
  \hypertarget{what-are-my-rights-if-i-am-worried-about-going-back-to-work}{%
  \paragraph{What are my rights if I am worried about going back to
  work?}\label{what-are-my-rights-if-i-am-worried-about-going-back-to-work}}

  \begin{itemize}
  \tightlist
  \item
    Employers have to provide
    \href{https://www.osha.gov/SLTC/covid-19/standards.html}{a safe
    workplace} with policies that protect everyone equally.
    \href{https://www.nytimes3xbfgragh.onion/article/coronavirus-money-unemployment.html?action=click\&pgtype=Article\&state=default\&region=MAIN_CONTENT_3\&context=storylines_faq}{And
    if one of your co-workers tests positive for the coronavirus, the
    C.D.C.} has said that
    \href{https://www.cdc.gov/coronavirus/2019-ncov/community/guidance-business-response.html}{employers
    should tell their employees} -\/- without giving you the sick
    employee's name -\/- that they may have been exposed to the virus.
  \end{itemize}
\item ~
  \hypertarget{should-i-refinance-my-mortgage}{%
  \paragraph{Should I refinance my
  mortgage?}\label{should-i-refinance-my-mortgage}}

  \begin{itemize}
  \tightlist
  \item
    \href{https://www.nytimes3xbfgragh.onion/article/coronavirus-money-unemployment.html?action=click\&pgtype=Article\&state=default\&region=MAIN_CONTENT_3\&context=storylines_faq}{It
    could be a good idea,} because mortgage rates have
    \href{https://www.nytimes3xbfgragh.onion/2020/07/16/business/mortgage-rates-below-3-percent.html?action=click\&pgtype=Article\&state=default\&region=MAIN_CONTENT_3\&context=storylines_faq}{never
    been lower.} Refinancing requests have pushed mortgage applications
    to some of the highest levels since 2008, so be prepared to get in
    line. But defaults are also up, so if you're thinking about buying a
    home, be aware that some lenders have tightened their standards.
  \end{itemize}
\item ~
  \hypertarget{what-is-school-going-to-look-like-in-september}{%
  \paragraph{What is school going to look like in
  September?}\label{what-is-school-going-to-look-like-in-september}}

  \begin{itemize}
  \tightlist
  \item
    It is unlikely that many schools will return to a normal schedule
    this fall, requiring the grind of
    \href{https://www.nytimes3xbfgragh.onion/2020/06/05/us/coronavirus-education-lost-learning.html?action=click\&pgtype=Article\&state=default\&region=MAIN_CONTENT_3\&context=storylines_faq}{online
    learning},
    \href{https://www.nytimes3xbfgragh.onion/2020/05/29/us/coronavirus-child-care-centers.html?action=click\&pgtype=Article\&state=default\&region=MAIN_CONTENT_3\&context=storylines_faq}{makeshift
    child care} and
    \href{https://www.nytimes3xbfgragh.onion/2020/06/03/business/economy/coronavirus-working-women.html?action=click\&pgtype=Article\&state=default\&region=MAIN_CONTENT_3\&context=storylines_faq}{stunted
    workdays} to continue. California's two largest public school
    districts --- Los Angeles and San Diego --- said on July 13, that
    \href{https://www.nytimes3xbfgragh.onion/2020/07/13/us/lausd-san-diego-school-reopening.html?action=click\&pgtype=Article\&state=default\&region=MAIN_CONTENT_3\&context=storylines_faq}{instruction
    will be remote-only in the fall}, citing concerns that surging
    coronavirus infections in their areas pose too dire a risk for
    students and teachers. Together, the two districts enroll some
    825,000 students. They are the largest in the country so far to
    abandon plans for even a partial physical return to classrooms when
    they reopen in August. For other districts, the solution won't be an
    all-or-nothing approach.
    \href{https://bioethics.jhu.edu/research-and-outreach/projects/eschool-initiative/school-policy-tracker/}{Many
    systems}, including the nation's largest, New York City, are
    devising
    \href{https://www.nytimes3xbfgragh.onion/2020/06/26/us/coronavirus-schools-reopen-fall.html?action=click\&pgtype=Article\&state=default\&region=MAIN_CONTENT_3\&context=storylines_faq}{hybrid
    plans} that involve spending some days in classrooms and other days
    online. There's no national policy on this yet, so check with your
    municipal school system regularly to see what is happening in your
    community.
  \end{itemize}
\end{itemize}

``This is really difficult to digest,'' Dr. Ditiu said. ``It took a lot
of work to arrive where we are. We were not at the peak of the mountain,
but we were away from the base. But then an avalanche came and pushed us
back to the bottom.''

The lockdowns in many places were imposed so swiftly that drug stocks
were rapidly depleted. Mexico already had expired drugs in its supply,
but that problem has been
\href{https://www.reuters.com/article/us-health-coronavirus-lgbt-aids/no-medicine-no-food-coronavirus-restrictions-amplify-health-risks-to-lgbt-people-with-hiv-idUSKBN22W28G}{exacerbated
by the pandemic}, according to Dr. Franyuti.

In Brazil, H.I.V. and TB drugs are purchased and distributed by the
ministry of health. But the coronavirus is racing through the country,
and distribution of these treatments has become increasingly difficult
as health care workers try to cope with the pandemic's toll.

``It's a big logistical challenge to have municipalities have higher
stock so they can supply,'' said Dr. Betina Durovni, a senior scientist
at the Fiocruz Foundation, a research institute in Brazil.

Image

Some coronavirus patients in Tabatinga, Brazil, are airlifted to Manaus,
about 1,000 miles away, for treatment, but many more are
missed.Credit...Bruno Kelly/Reuters

Even if governments are prepared, with some help from big aid agencies,
to buy drugs months in advance, the global supply may soon run out.

The pandemic has severely restricted international transport, hindering
the availability not just of chemical ingredients and raw materials, but
also of packaging supplies.

``The disruption of supply chains is really something that worries me
--- for H.I.V., for TB, for malaria,'' said Dr. Carlos del Rio, chair of
the scientific advisory board of the President's Emergency Plan for AIDS
Relief.

The hype over chloroquine as a potential treatment for the coronavirus
has led to hoarding of the drug in some countries like Myanmar,
depleting its global stocks.

More than 80 percent of the global supply of antiretroviral drugs comes
from just eight Indian companies. The cost of these alone
\href{https://www.unaids.org/en/resources/presscentre/pressreleaseandstatementarchive/2020/june/20200622_availability-and-cost-of-antiretroviral-medicines}{could
rise by \$225 million} per year because of shortfalls in supplies and
work force, transport disruptions and currency fluctuations, according
to U.N. AIDS.

There is also a real risk that Indian companies will turn to more
profitable medicines, or will not be able to meet the global demand
because migrant workers have deserted cities as the coronavirus spreads.

The Indian government may even decide
\href{https://theprint.in/health/india-could-ban-export-of-anti-tb-drugs-as-lockdown-hits-production/406119/}{not
to export TB medicines}, saving its supply for its own citizens.

``We're very dependent on a few key developers or manufacturers for all
of the drugs around the world, and that needs to be diversified,'' said
Dr. Meg Doherty, who directs H.I.V. programs at the W.H.O. ``If you had
more locally developed drug depots or drug manufacturers, it would be
closer to the point of need.''

Aid organizations and governments are trying to mitigate some of the
damage by stretching supplies and stockpiling medications. In June, the
W.H.O.
changed\href{https://www.who.int/news-room/detail/15-06-2020-who-urges-countries-to-enable-access-to-fully-oral-drug-resistant-tb-treatment-regimens}{its
recommendation} for treatment of drug-resistant TB. Instead of 20 months
of injections, patients may now take pills for nine to 11 months. The
change means patients don't have to travel to clinics, increasingly
closed by lockdowns.

More than half of 144 countries surveyed by the W.H.O. said they have
opted to give patients H.I.V. drugs sufficient to last for at least
three months --- six months, in the case of a few countries like South
Sudan --- in order to limit their trips to hospitals. But it's unclear
how successful those efforts have been.

In some countries, such as the Philippines, advocacy groups have set up
depots for patients to pick up antiretroviral pills or arrange to
\href{https://www.unaids.org/en/resources/presscentre/featurestories/2020/april/20200408_philippines}{drop
them off at patients' homes}.

In a few nations, like South Africa, most patients already pick up
medications from community centers rather than from hospitals, said Dr.
Salim S. Abdool Karim, a global health expert in South Africa and the
chair of a government advisory committee on Covid-19. ``That has been an
important advantage in a way.''

\hypertarget{what-are-we-not-doing-right}{%
\subsection{`What are we not doing
right?'}\label{what-are-we-not-doing-right}}

Image

The Sally Mugabe Central Hospital in Harare, where Dr. Tapiwa Mungofa
works, has closed its outpatient department, where TB and H.I.V.
patients received their medications.Credit...Cynthia R. Matonhodze for
The New York Times

The pandemic has exposed deep fissures in the health care systems of
many countries.

In Zimbabwe, staffs in public hospitals were working reduced shifts even
before the pandemic, because the government could not afford to pay
their full salaries. Some hospitals like the Sally Mugabe Central
Hospital in Harare --- which was
\href{https://www.newsday.co.zw/2020/02/sally-mugabe-hospital-operating-at-45-capacity/}{operating
at half capacity} because of water shortages and other problems --- have
since closed their outpatient departments, where TB and H.I.V. patients
received their medications.

``Hospitals are functioning in an emergency mode,'' said Dr. Tapiwa
Mungofa, a physician at the Sally Mugabe Hospital.

The situation is no better in KwaZulu-Natal, which has the highest
prevalence of H.I.V. in South Africa. Dr. Zolelwa Sifumba was a teenager
when she saw images of skeletal patients dying of AIDS. Over the past
few years in KwaZulu-Natal, she is again seeing patients with full-blown
AIDS.

``We're seeing people come in at the stage where they're kind of on
death's door,'' she said. ``What are we not doing right?''

Some remote parts of the world are being decimated by the coronavirus
--- but their very remoteness makes the pandemic's impact on these other
big infectious killers impossible to measure.

The town of Tabatinga in Amazonas, the biggest state in Brazil, is more
than 1,000 miles from the closest city with an I.C.U., Manaus. The
government has been using airplanes to transport coronavirus patients to
Manaus, but many cases are being missed, said Dr. Marcelo
Cordeiro-Santos, a researcher at the Tropical Medicine Foundation in
Manaus.

Hospitals are giving chloroquine to people with Covid-19, at the
recommendation of the Brazilian ministry of health, even though evidence
now suggests it does not help and may even be harmful.

Chloroquine is also a crucial malaria medicine, and its indiscriminate
use now may lead to resistance to the drug, Dr. Cordeiro-Santos warned
--- with possibly dire consequences for those infected in the future.
But he also said it's possible that widespread distribution of
chloroquine may help protect residents of Amazonas from malaria.

Other experts said they hope the coronavirus pandemic brings some silver
linings.

Aid agencies have long recommended that countries buy drugs in bulk and
provide several months' supply at a time to their citizens. Some
governments are considering doing so now for H.I.V., according to Dr.
Doherty at the W.H.O.

Health care providers are also embracing video and phone calls to
counsel and treat patients, which many people find far easier than
traveling to distant clinics.

``Sometimes systems are tough to change,'' said Dr. del Rio, ``but I
think there's nothing better than a crisis to change the system,
right?''

\emph{Lynsey Chutel contributed reporting from Johannesburg.}

Image

Some remote areas in Brazil are being decimated by the coronavirus ---
but their remoteness makes the impact on other big infectious killers
impossible to measure.~Credit...Tarso Sarraf/Agence France-Presse ---
Getty Images

Advertisement

\protect\hyperlink{after-bottom}{Continue reading the main story}

\hypertarget{site-index}{%
\subsection{Site Index}\label{site-index}}

\hypertarget{site-information-navigation}{%
\subsection{Site Information
Navigation}\label{site-information-navigation}}

\begin{itemize}
\tightlist
\item
  \href{https://help.nytimes3xbfgragh.onion/hc/en-us/articles/115014792127-Copyright-notice}{©~2020~The
  New York Times Company}
\end{itemize}

\begin{itemize}
\tightlist
\item
  \href{https://www.nytco.com/}{NYTCo}
\item
  \href{https://help.nytimes3xbfgragh.onion/hc/en-us/articles/115015385887-Contact-Us}{Contact
  Us}
\item
  \href{https://www.nytco.com/careers/}{Work with us}
\item
  \href{https://nytmediakit.com/}{Advertise}
\item
  \href{http://www.tbrandstudio.com/}{T Brand Studio}
\item
  \href{https://www.nytimes3xbfgragh.onion/privacy/cookie-policy\#how-do-i-manage-trackers}{Your
  Ad Choices}
\item
  \href{https://www.nytimes3xbfgragh.onion/privacy}{Privacy}
\item
  \href{https://help.nytimes3xbfgragh.onion/hc/en-us/articles/115014893428-Terms-of-service}{Terms
  of Service}
\item
  \href{https://help.nytimes3xbfgragh.onion/hc/en-us/articles/115014893968-Terms-of-sale}{Terms
  of Sale}
\item
  \href{https://spiderbites.nytimes3xbfgragh.onion}{Site Map}
\item
  \href{https://help.nytimes3xbfgragh.onion/hc/en-us}{Help}
\item
  \href{https://www.nytimes3xbfgragh.onion/subscription?campaignId=37WXW}{Subscriptions}
\end{itemize}
