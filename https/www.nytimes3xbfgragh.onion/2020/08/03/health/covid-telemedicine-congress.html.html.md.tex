Sections

SEARCH

\protect\hyperlink{site-content}{Skip to
content}\protect\hyperlink{site-index}{Skip to site index}

\href{https://www.nytimes3xbfgragh.onion/section/health}{Health}

\href{https://myaccount.nytimes3xbfgragh.onion/auth/login?response_type=cookie\&client_id=vi}{}

\href{https://www.nytimes3xbfgragh.onion/section/todayspaper}{Today's
Paper}

\href{/section/health}{Health}\textbar{}Is Telemedicine Here to Stay?

\url{https://nyti.ms/33jkP1N}

\begin{itemize}
\item
\item
\item
\item
\item
\item
\end{itemize}

\href{https://www.nytimes3xbfgragh.onion/spotlight/at-home?action=click\&pgtype=Article\&state=default\&region=TOP_BANNER\&context=at_home_menu}{At
Home}

\begin{itemize}
\tightlist
\item
  \href{https://www.nytimes3xbfgragh.onion/2020/08/03/well/family/the-benefits-of-talking-to-strangers.html?action=click\&pgtype=Article\&state=default\&region=TOP_BANNER\&context=at_home_menu}{Talk:
  To Strangers}
\item
  \href{https://www.nytimes3xbfgragh.onion/2020/08/01/at-home/coronavirus-make-pizza-on-a-grill.html?action=click\&pgtype=Article\&state=default\&region=TOP_BANNER\&context=at_home_menu}{Make:
  Grilled Pizza}
\item
  \href{https://www.nytimes3xbfgragh.onion/2020/07/31/arts/television/goldbergs-abc-stream.html?action=click\&pgtype=Article\&state=default\&region=TOP_BANNER\&context=at_home_menu}{Watch:
  'The Goldbergs'}
\item
  \href{https://www.nytimes3xbfgragh.onion/interactive/2020/at-home/even-more-reporters-editors-diaries-lists-recommendations.html?action=click\&pgtype=Article\&state=default\&region=TOP_BANNER\&context=at_home_menu}{Explore:
  Reporters' Google Docs}
\end{itemize}

Advertisement

\protect\hyperlink{after-top}{Continue reading the main story}

Supported by

\protect\hyperlink{after-sponsor}{Continue reading the main story}

\hypertarget{is-telemedicine-here-to-stay}{%
\section{Is Telemedicine Here to
Stay?}\label{is-telemedicine-here-to-stay}}

The answer largely depends on whether Medicare and private health
insurers will adequately cover virtual doctor visits once coronavirus
outbreaks subside.

\includegraphics{https://static01.graylady3jvrrxbe.onion/images/2020/07/23/science/00VIRUS-TELEHEALTH1/00VIRUS-TELEHEALTH1-articleLarge-v2.jpg?quality=75\&auto=webp\&disable=upscale}

\href{https://www.nytimes3xbfgragh.onion/by/reed-abelson}{\includegraphics{https://static01.graylady3jvrrxbe.onion/images/2018/07/16/multimedia/author-reed-abelson/author-reed-abelson-thumbLarge.png}}

By \href{https://www.nytimes3xbfgragh.onion/by/reed-abelson}{Reed
Abelson}

\begin{itemize}
\item
  Aug. 3, 2020
\item
  \begin{itemize}
  \item
  \item
  \item
  \item
  \item
  \item
  \end{itemize}
\end{itemize}

Telemedicine is having its moment. Over the last few months, millions of
people have relied on video or telephone calls to talk to their doctors.
But as the pandemic moves across the United States, and eventually
recedes in some places, how long will the moment last?

While patients used virtual visits to avoid overcrowded and potentially
infectious doctor's offices or emergency rooms, many are returning to
face-to-face appointments in cities where the threat has subsided.

And insurance payments for telehealth services, especially at full cost,
may only be temporary.

Medicare's coverage of a broad range of services is slated to end when
the coronavirus no longer poses a public health emergency. Private
insurers, which followed the federal government's lead, could revert to
paying doctors for virtual visits at a fraction of the cost for
traditional visits, if anything at all.

Some of the nation's biggest insurers, like UnitedHealthcare and Anthem,
say they haven't decided beyond September or October on whether to
extend the policies they adopted that allowed for coverage in lieu of
doctors' visits during the coronavirus crisis.

``The concern everyone in the industry has is that reimbursement is in
jeopardy,'' said Dr. Mia Levy, the director of the cancer center at Rush
University Medical Center in Chicago, which treated patients virtually
during the height of the pandemic. ``Because of telehealth, we were able
to stay actively engaged with our patients,'' she said.

While there is broad bipartisan support for telehealth coverage,
Congress would have to pass specific legislation to make some of
Medicare's changes permanent.

``Reversing course would be a mistake,'' said Seema Verma, the
administrator for the federal program, which reimbursed doctors the same
for virtual visits, including those over the telephone, as for in-person
ones and relaxed rules about who can use telemedicine.

About
\href{https://www.healthaffairs.org/do/10.1377/hblog20200715.454789/full/}{nine
million people under traditional Medicare}used telemedicine services
during the early months of the crisis. Early data does not show wide
variations in use by race or ethnicity.

``It was really a no-brainer for us,'' Ms. Verma said.

And spending on telemedicine services during the first peak of the
coronavirus pandemic in the United States underscores the demand. In
addition to federal spending through Medicare, nearly \$4 billion was
billed nationally for telehealth visits during March and April, compared
to less than \$60 million for the same two months of 2019, according to
FAIR Health, a nonprofit group that analyzes private health insurance
claims.

But to convince insurers they should continue paying for virtual care,
doctors must demonstrate they can move beyond treating simple
respiratory infections to caring for patients with chronic conditions
like depression or diabetes. ``From the perspective of managing the cost
and quality, there's a lot we don't know about telemedicine,'' said Dr.
Rahul Rajkumar, the chief medical officer at Blue Cross Blue Shield of
North Carolina.

BlueCross BlueShield of Tennessee
\href{https://bcbstnews.com/pressreleases/bluecross-making-in-network-telehealth-services-permanent/}{says
it is the first major insurer} to make coverage of telehealth services
permanent, but it has not yet determined how much it will eventually pay
for the care. A few insurers, including Cigna and the Blue Cross plan in
North Carolina, said they will continue to cover telehealth services at
pandemic levels through the end of the year.

``We need to give providers time to get more comfortable,'' said Dr.
Scott Josephs, the chief medical officer for Cigna. To make remote
medicine successful and worthwhile, doctors and medical groups need to
invest in technology and train staff. ``If they don't have the time,
they won't make the investments,'' he said.

\includegraphics{https://static01.graylady3jvrrxbe.onion/images/2020/07/23/science/00VIRUS-TELEHEALTH2/merlin_170137515_416199cc-d0d6-428c-b307-0139a1f40972-articleLarge.jpg?quality=75\&auto=webp\&disable=upscale}

The biggest hurdle to widespread adoption by both the government and
insurers is the potential cost.

Lawmakers are reluctant to pass any bill that would significantly add to
Medicare's budget, with the government already spending a total of some
\$750 billion a year.

And private insurers see telemedicine as a way to save them money, said
Sabrina Corlette, a research professor at Georgetown University, who
helped author
\href{https://www.rwjf.org/en/library/research/2020/06/the-covid-19-pandemic-insurer-insights-into-challenges-implications-and-lessons-learned.html?cid=xem_other_unpd_ini:moni_dte:20200629_des:insurer\%20insight}{a
recent report} on how the companies responded to the pandemic. ``Unless
they are required to by the states or federal government, a lot of
carriers will try to reimburse less for telehealth than an in-person
visit,'' she said.

For those at risk, telemedicine is particularly valuable. When a fever
sent Susan Varak, 45, who has breast cancer, to the emergency room
during the height of Chicago's outbreak in April, she felt as if she
were ``walking into this war zone,'' she said, because she was so
terrified of catching the virus.

She appreciates she still can see her oncologist remotely. ``I don't
think it's absolutely necessary to be face-to-face every couple of
weeks,'' she said.

Other patients like the convenience. David Collins, 67, didn't have a
choice when he had a 20-minute video visit in March to rule out a
diagnosis of coronavirus.
\href{https://www.nytimes3xbfgragh.onion/2020/03/11/health/telemedicine-coronavirus.html}{Like
many practices during the pandemic}, the Kelsey-Seybold Clinic, a large
physician group in Houston, was not allowing most patients to come in.

``I loved it because it saved me a lot of time.'' he said, adding ``I'd
much rather do that than drive across town and look for parking.''

But, a few months later, he didn't hesitate to go to the clinic for his
checkup. ``There's a little more hands-on required,'' he explained, like
getting a physical exam and having his blood pressure taken. Not
everything can be done virtually, he said. ``If you break your arm, an
e-visit isn't going to help you at all,'' he said.

After seeing about 90 percent of its patients virtually, Kelsey-Seybold
has ``almost flip-flopped back,'' said Dr. Donnie Aga, an internist who
oversees telehealth for the group. Most patients seem to prefer an
in-person appointment. ``You could really see that people missed coming
in,'' he said.

With coronavirus cases now at epidemic levels in Texas, the clinic wants
to shift to dividing visits to half virtual, half in person. ``You've
got to have a balance, for sure,'' Dr. Aga said.

But how doctors and insurers can do that is still unknown.

``We need to see where the equilibrium ends up,'' said Dr. Andrea
Gelzer, the corporate chief medical officer for AmeriHealth Caritas, a
Medicaid managed care company. ``If the total number of visits far
exceeds pre-Covid, I don't think that's sustainable,'' she said.
Additional visits that do not improve patients' health will only result
in higher costs.

Image

Dr. Donnie Aga oversees telehealth at Kelsey-Seybold in Houston, which
has seen telemedicine rise and fall in waves since the pandemic
began.~``You've got to have a balance, for sure,'' he
said.Credit...Callaghan O'Hare for The New York Times

Doctors have to be more discriminating about which patients to see
remotely, said Rita Numerof, a health care consultant.

Telemedicine ``was a solution to an immediate problem,'' she said, and
doctors did not have clear criteria about who should be seen, under what
circumstances and for which conditions.

Many in Congress are already convinced that Medicare should continue the
current coverage. ``The Covid-19 pandemic has been a trial by fire, but
the experience to date has made clear that the health care system is
ready for broader access to telehealth on a permanent basis,'' said Sen.
Ron Wyden of Oregon, a Democrat who introduced legislation earlier this
month.

On Thursday, Sen. Lamar Alexander of Tennessee, a Republican and chair
of the Senate health committee, introduced the Telehealth Modernization
Act, which would also make some changes permanent. The experience of the
previous four months ``will likely mean that hundreds of millions of
physician-patient visits will be remote or online that were in-person
before,'' he said.

Since May, nearly 20 telemedicine bills have been brought to the House
floor and about the same number in the Senate, said Miranda Franco, a
senior policy adviser for the law firm Holland \& Knight. She thinks
legislation will be passed by the end of the year.

While some lawmakers favor permanently expanding Medicare payment for a
broad range of telemedicine services, others are concerned about the
technology's cost and potential for fraud. ``Now you're talking about
reimbursing services we haven't reimbursed before,'' Ms. Franco said.

Some patients say telemedicine is not a substitute for in-person care.
Jorge Cueto, who is in his mid-20s, said a virtual visit is often an
additional step before going to the doctor's office for, say, a sore
throat.

``It's another fee, it's another gating mechanism,'' he said.

His parents, who are not fluent in English, prefer going to the doctor's
office because they find it easier to communicate in person, he said,
and they have difficulty setting up video calls. ``I don't think they
would be willing opt for telehealth if they weren't required to do it,''
Mr. Cueto said.

Others may not have access to a computer or smartphone to connect for
video visits, and insurers are particularly wary of doctors charging for
phone calls to follow up on lab results or tell someone to come to the
office.

Even patients who have cellphones may not be able to afford a lengthy
consultation, Dr. Levy said. She and her colleagues discovered some
people stopped answering their phones at the end of the month because
they had run out of minutes. ``That was very eye-opening to us,'' she
said.

Some proponents argue the goal of telemedicine should not be to lower
health care costs over all. One of its main benefits is improving
patients' access to care, said Dr. Ateev Mehrotra, a professor of health
care policy at Harvard Medical School, adding that it would be foolish
to expect savings if more people also get treatment. ``Those don't
reconcile,'' he said.

Insurers should evaluate whether telemedicine is more effective for
treating conditions like depression than it is for, say, cancer. They
could then make those distinctions in reimbursing for virtual visits, he
said, just as they do for different prescription drugs.

``There should be no single telemedicine policy,'' Dr. Mehrotra said.

Advertisement

\protect\hyperlink{after-bottom}{Continue reading the main story}

\hypertarget{site-index}{%
\subsection{Site Index}\label{site-index}}

\hypertarget{site-information-navigation}{%
\subsection{Site Information
Navigation}\label{site-information-navigation}}

\begin{itemize}
\tightlist
\item
  \href{https://help.nytimes3xbfgragh.onion/hc/en-us/articles/115014792127-Copyright-notice}{©~2020~The
  New York Times Company}
\end{itemize}

\begin{itemize}
\tightlist
\item
  \href{https://www.nytco.com/}{NYTCo}
\item
  \href{https://help.nytimes3xbfgragh.onion/hc/en-us/articles/115015385887-Contact-Us}{Contact
  Us}
\item
  \href{https://www.nytco.com/careers/}{Work with us}
\item
  \href{https://nytmediakit.com/}{Advertise}
\item
  \href{http://www.tbrandstudio.com/}{T Brand Studio}
\item
  \href{https://www.nytimes3xbfgragh.onion/privacy/cookie-policy\#how-do-i-manage-trackers}{Your
  Ad Choices}
\item
  \href{https://www.nytimes3xbfgragh.onion/privacy}{Privacy}
\item
  \href{https://help.nytimes3xbfgragh.onion/hc/en-us/articles/115014893428-Terms-of-service}{Terms
  of Service}
\item
  \href{https://help.nytimes3xbfgragh.onion/hc/en-us/articles/115014893968-Terms-of-sale}{Terms
  of Sale}
\item
  \href{https://spiderbites.nytimes3xbfgragh.onion}{Site Map}
\item
  \href{https://help.nytimes3xbfgragh.onion/hc/en-us}{Help}
\item
  \href{https://www.nytimes3xbfgragh.onion/subscription?campaignId=37WXW}{Subscriptions}
\end{itemize}
