Sections

SEARCH

\protect\hyperlink{site-content}{Skip to
content}\protect\hyperlink{site-index}{Skip to site index}

\href{https://www.nytimes3xbfgragh.onion/section/reader-center}{Times
Insider}

\href{https://myaccount.nytimes3xbfgragh.onion/auth/login?response_type=cookie\&client_id=vi}{}

\href{https://www.nytimes3xbfgragh.onion/section/todayspaper}{Today's
Paper}

\href{/section/reader-center}{Times Insider}\textbar{}Letting Their
Cameras Transport You

\url{https://nyti.ms/33mDoSO}

\begin{itemize}
\item
\item
\item
\item
\item
\item
\end{itemize}

\href{https://www.nytimes3xbfgragh.onion/spotlight/at-home?action=click\&pgtype=Article\&state=default\&region=TOP_BANNER\&context=at_home_menu}{At
Home}

\begin{itemize}
\tightlist
\item
  \href{https://www.nytimes3xbfgragh.onion/2020/07/28/books/time-for-a-literary-road-trip.html?action=click\&pgtype=Article\&state=default\&region=TOP_BANNER\&context=at_home_menu}{Take:
  A Literary Road Trip}
\item
  \href{https://www.nytimes3xbfgragh.onion/2020/07/29/magazine/bored-with-your-home-cooking-some-smoky-eggplant-will-fix-that.html?action=click\&pgtype=Article\&state=default\&region=TOP_BANNER\&context=at_home_menu}{Cook:
  Smoky Eggplant}
\item
  \href{https://www.nytimes3xbfgragh.onion/2020/07/27/travel/moose-michigan-isle-royale.html?action=click\&pgtype=Article\&state=default\&region=TOP_BANNER\&context=at_home_menu}{Look
  Out: For Moose}
\item
  \href{https://www.nytimes3xbfgragh.onion/interactive/2020/at-home/even-more-reporters-editors-diaries-lists-recommendations.html?action=click\&pgtype=Article\&state=default\&region=TOP_BANNER\&context=at_home_menu}{Explore:
  Reporters' Obsessions}
\end{itemize}

Advertisement

\protect\hyperlink{after-top}{Continue reading the main story}

Supported by

\protect\hyperlink{after-sponsor}{Continue reading the main story}

Times Insider

\hypertarget{letting-their-cameras-transport-you}{%
\section{Letting Their Cameras Transport
You}\label{letting-their-cameras-transport-you}}

With some of the finest photojournalists as your guide, The World
Through a Lens series offers immersive escapes.

\href{https://www.nytimes3xbfgragh.onion/by/stephen-hiltner}{\includegraphics{https://static01.graylady3jvrrxbe.onion/images/2018/06/13/multimedia/author-stephen-hiltner/author-stephen-hiltner-thumbLarge-v2.jpg}}\href{https://www.nytimes3xbfgragh.onion/by/phaedra-brown}{\includegraphics{https://static01.graylady3jvrrxbe.onion/images/2019/03/01/multimedia/author-phaedra-brown/author-phaedra-brown-thumbLarge.png}}

By \href{https://www.nytimes3xbfgragh.onion/by/stephen-hiltner}{Stephen
Hiltner} and
\href{https://www.nytimes3xbfgragh.onion/by/phaedra-brown}{Phaedra
Brown}

\begin{itemize}
\item
  Aug. 3, 2020
\item
  \begin{itemize}
  \item
  \item
  \item
  \item
  \item
  \item
  \end{itemize}
\end{itemize}

\href{https://www.nytimes3xbfgragh.onion/series/times-insider}{\emph{Times
Insider}} \emph{explains who we are and what we do, and delivers
behind-the-scenes insights into how our journalism comes together.}

For those of us lucky enough to have avoided the coronavirus itself, the
most noticeable change in our daily reality might be our relative lack
of mobility. Traveling to the grocery store can feel like an odyssey.
Traveling internationally can feel like a
\href{https://www.nytimes3xbfgragh.onion/2020/07/28/travel/future-travel-bucket-list-coronavirus.html}{distant
dream}.

Of course, the loss of our collective ability to travel pales in
comparison to the
\href{https://www.nytimes3xbfgragh.onion/interactive/2020/05/24/us/us-coronavirus-deaths-100000.html}{many}
\href{https://www.nytimes3xbfgragh.onion/interactive/2020/03/27/world/europe/coronavirus-italy-bergamo.html}{other}
\href{https://www.nytimes3xbfgragh.onion/2020/04/22/world/africa/coronavirus-hunger-crisis.html}{existential}
\href{https://www.nytimes3xbfgragh.onion/interactive/2020/world/coronavirus-health-care-workers.html}{crises}
\href{https://www.nytimes3xbfgragh.onion/2020/05/04/nyregion/coronavirus-nyc-burials-muslim.html}{posed}
\href{https://www.nytimes3xbfgragh.onion/2020/07/01/nyregion/Coronavirus-hospitals.html}{by}
\href{https://www.nytimes3xbfgragh.onion/2020/06/22/nyregion/nyc-evictions-moratorium-coronavirus.html}{this}
\href{https://www.nytimes3xbfgragh.onion/2020/07/23/business/economy/unemployment-economy-coronavirus.html}{pandemic}.
But there's no denying that, for many of us, our inability to venture
out --- beyond the borders of our homes, our states, our countries ---
has made the past several months (and the prospect of many months to
come) all the more challenging.

So, back in March, as travel restrictions began to lock people in place
all around the world, we editors on the Travel desk launched a new
visual series to help readers cope. We called it
\href{https://www.nytimes3xbfgragh.onion/column/the-world-through-a-lens}{The
World Through a Lens}. The idea was to showcase some of our planet's
most beautiful and intriguing places, and to introduce readers to
aspects of global culture --- whether
\href{https://www.nytimes3xbfgragh.onion/2020/05/12/travel/italy-saffron.html}{agricultural}
or
\href{https://www.nytimes3xbfgragh.onion/2020/04/22/travel/a-glimpse-inside-the-secluded-world-of-a-georgian-convent.html}{religious}
or
\href{https://www.nytimes3xbfgragh.onion/2020/04/15/travel/colombia-lost-city-ciudad-perdida.html}{historical}
--- that, in these months of stasis, might otherwise have remained
hidden from our view.

\includegraphics{https://static01.graylady3jvrrxbe.onion/images/2020/08/03/travel/31insider-trv-1/13travel-vermont-1-mobileMasterAt3x-v2.jpg}\includegraphics{https://static01.graylady3jvrrxbe.onion/images/2020/08/03/travel/31insider-trv-2/15travel-colombia-04-mobileMasterAt3x.jpg}\includegraphics{https://static01.graylady3jvrrxbe.onion/images/2020/08/03/travel/31insider-trv/27travel-michigan-09-mobileMasterAt3x.jpg}

Caleb Kenna

The photographer Caleb Kenna specializes in aerial photography. This
shot was taken in Goshen, Vt., using a drone.

Stephen Hiltner, an editor on the Travel team, photographed and wrote
about Colombia's ancient Ciudad Perdida. The trail to Ciudad Perdida
follows the Buritaca River, whose waters offer trekkers a chance to cool
off during unforgivingly hot days.

Tony Cenicola captured this shot of a bull moose at the Rock Harbor
Campground on Michigan's Isle Royale.

Our goal with this series is slightly different from that of our typical
Travel fare. Instead of inspiring travel among our readers or describing
the travels of our contributors, we're aiming to \emph{approximate}
elements of travel itself --- to provide a kind of virtual travel
substitute that soothes, transports and distracts.

But escapism isn't the only objective. At its best, travel can transform
us: It can awaken us to the
\href{https://www.nytimes3xbfgragh.onion/2020/06/24/travel/dolomites-italy-hut-hiking.html}{restorative
power of nature}; it can broaden our ability to
\href{https://www.nytimes3xbfgragh.onion/2020/06/08/travel/lens-sheep-island-maine.html}{understand
and appreciate dissimilar cultures}; it can help us become more
empathetic to
\href{https://www.nytimes3xbfgragh.onion/2020/06/15/travel/gujarat-india-portraits.html}{people
whose lives fall outside the scope of our day-to-day routines}. These,
too, are things we hope the series can provide, especially at a time
when such transformations aren't available to many of us via direct
experience.

To achieve all that, we've tried to create immersive visual experiences;
every story in the series is driven by images. (Instead of assigning
photographers to shoot new work, we are relying on photojournalists with
previously shot, and unpublished, portfolios.) We've also tried to
create a more intuitive and symbiotic relationship between images and
text. (In most cases, the journalist who took the pictures is also the
one who wrote the copy.) And as a general rule, we've avoided
``service'' information: no hotel plugs or restaurant reviews, no
recommended itineraries. The focus is on the people and the places
themselves, as seen and captured by some of the best photojournalists in
the world.

In this week's installment, we carry you to rural Patagonia to observe
some of the region's remote schoolhouses. As with our other destinations
(dip into the
\href{https://www.nytimes3xbfgragh.onion/column/the-world-through-a-lens}{four-month
archive} to see more), we hope you enjoy --- and are transformed by ---
the virtual journey.

Advertisement

\protect\hyperlink{after-bottom}{Continue reading the main story}

\hypertarget{site-index}{%
\subsection{Site Index}\label{site-index}}

\hypertarget{site-information-navigation}{%
\subsection{Site Information
Navigation}\label{site-information-navigation}}

\begin{itemize}
\tightlist
\item
  \href{https://help.nytimes3xbfgragh.onion/hc/en-us/articles/115014792127-Copyright-notice}{©~2020~The
  New York Times Company}
\end{itemize}

\begin{itemize}
\tightlist
\item
  \href{https://www.nytco.com/}{NYTCo}
\item
  \href{https://help.nytimes3xbfgragh.onion/hc/en-us/articles/115015385887-Contact-Us}{Contact
  Us}
\item
  \href{https://www.nytco.com/careers/}{Work with us}
\item
  \href{https://nytmediakit.com/}{Advertise}
\item
  \href{http://www.tbrandstudio.com/}{T Brand Studio}
\item
  \href{https://www.nytimes3xbfgragh.onion/privacy/cookie-policy\#how-do-i-manage-trackers}{Your
  Ad Choices}
\item
  \href{https://www.nytimes3xbfgragh.onion/privacy}{Privacy}
\item
  \href{https://help.nytimes3xbfgragh.onion/hc/en-us/articles/115014893428-Terms-of-service}{Terms
  of Service}
\item
  \href{https://help.nytimes3xbfgragh.onion/hc/en-us/articles/115014893968-Terms-of-sale}{Terms
  of Sale}
\item
  \href{https://spiderbites.nytimes3xbfgragh.onion}{Site Map}
\item
  \href{https://help.nytimes3xbfgragh.onion/hc/en-us}{Help}
\item
  \href{https://www.nytimes3xbfgragh.onion/subscription?campaignId=37WXW}{Subscriptions}
\end{itemize}
