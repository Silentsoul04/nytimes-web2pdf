Sections

SEARCH

\protect\hyperlink{site-content}{Skip to
content}\protect\hyperlink{site-index}{Skip to site index}

\href{https://myaccount.nytimes3xbfgragh.onion/auth/login?response_type=cookie\&client_id=vi}{}

\href{https://www.nytimes3xbfgragh.onion/section/todayspaper}{Today's
Paper}

\href{/section/opinion}{Opinion}\textbar{}Death of a Cat

\url{https://nyti.ms/33lFEte}

\begin{itemize}
\item
\item
\item
\item
\item
\item
\end{itemize}

Advertisement

\protect\hyperlink{after-top}{Continue reading the main story}

\href{/section/opinion}{Opinion}

Supported by

\protect\hyperlink{after-sponsor}{Continue reading the main story}

\hypertarget{death-of-a-cat}{%
\section{Death of a Cat}\label{death-of-a-cat}}

For weeks, I have been trying to understand my own tears in the presence
of a dying creature I did not love.

\href{https://www.nytimes3xbfgragh.onion/by/margaret-renkl}{\includegraphics{https://static01.graylady3jvrrxbe.onion/images/2017/04/08/opinion/margaret-renkl/margaret-renkl-thumbLarge-v2.png}}

By \href{https://www.nytimes3xbfgragh.onion/by/margaret-renkl}{Margaret
Renkl}

Contributing Opinion Writer

\begin{itemize}
\item
  Aug. 3, 2020
\item
  \begin{itemize}
  \item
  \item
  \item
  \item
  \item
  \item
  \end{itemize}
\end{itemize}

\includegraphics{https://static01.graylady3jvrrxbe.onion/images/2020/07/31/opinion/31renklWeb/31renklWeb-articleLarge.jpg?quality=75\&auto=webp\&disable=upscale}

NASHVILLE --- I first saw it in bits --- two paws here, the tip of a
tail there --- on grainy, black-and-white images taken by our backyard
trail camera. Normally that device captures only the wild animals you'd
expect to find in a first-ring suburb: opossums, mainly, but also
rabbits and raccoons and rat snakes, sometimes an owl, now and then a
fox. I once saw a bobcat slink across our street, but it's never shown
up in the trail-cam photos. My heart lifted when I first saw a clearly
feline haunch in one of the pictures, but the scale was all wrong for a
bobcat. It was just an ordinary house cat, prowling in my backyard.

Soon the cat was showing up in the daytime, too, apparently drawn by the
leftover bits of lunches left around the construction site two doors
down from our house. He was a ragged, battle-scarred tom, thin but not
emaciated, with one eye that didn't open all the way. A feral cat, not
someone's cherished pet.

As we passed my husband's car in the driveway one morning, my skittish
rescue dog darted away from the tires, spooked by something under the
car. I squatted down for a look. The feral cat hissed at me.

Let me just say it, flat out: If I owned a gun, I swear I ** would have
shot that cat. I would have chased that hissing cat out from under the
car without a thought and shot it as it fled.

I wish you could see all the baby birds in our yard. Baby towhees and
robins, baby redbirds and bluebirds. Two broods of house finches and two
broods of house wrens. One morning, an ungainly fledgling crow sat
nearby and rustled its feathers, yelling for food like a giant nestling.
A fuzzy screech owl chick spent an entire day staring wide-eyed at me
from a tree near our back deck while its mother dozed beside it.

Image

Two fledgling bluebirds and their mother in the author's front
yard.Credit...Margaret Renkl

Image

A pair of goldfinches in the author's pollinator
garden.Credit...Margaret Renkl

We've had baby downy woodpeckers and red-bellied woodpeckers, baby
white-breasted nuthatches and northern flickers. All summer long, they
have been taking their wobbly maiden flights from branch to fence post,
from cherry laurel to holly hedge. They try to land on the birdbath and
miss altogether. They fly to the ground to catch an insect and then
can't figure out how to take off again.

Our bug-averse neighbors
\href{https://www.nytimes3xbfgragh.onion/2020/05/18/opinion/lawn-pesticides-insect-extinction.html?searchResultPosition=1}{poison
every crawling or flying thing in sight}, but my husband and I have
spent the last 25 years trying to make a haven here for wildlife,
including the insects that pollinate flowers and feed reptiles and
birds. Maybe you can picture these adorable baby birds and this
flowering, insect-friendly yard. If so, you might forgive me for losing
my mind a bit when the feral cat showed up.

I was thinking of the first nest the bluebirds built this spring, the
one in which not a single baby survived. I was thinking of the gravid
broadhead skink who would lie on our stoop every afternoon, warming her
egg-swollen body in the sun. She disappeared one day to lay her eggs and
guard her nest, I assumed, but now I wasn't sure. I was thinking of the
chipmunk who lives in a tunnel under our stoop and of the little screech
owl, its feet holding down some small prey, its eyes glowing in the
infrared light of our trail camera.

The more I thought about those vulnerable creatures, already crowded out
by construction and starved out by insecticides, the angrier I got at
the feral tom. In truth, I would never kill a cat, but I can surely hate
one with a murderous rage. A person who has spent a quarter-century
trying to create an oasis for wildlife can go a little mad when a cat
shows up in the photos on her trail camera.

\includegraphics{https://static01.graylady3jvrrxbe.onion/images/2020/08/03/opinion/03renkl5/merlin_175171125_d6d86fbe-651f-4bb6-b2a3-3e164aba9012-articleLarge.jpg?quality=75\&auto=webp\&disable=upscale}

Cats are domestic animals that don't belong outdoors. In the natural
world, including my own backyard, animals kill one another every day,
but there is ample scientific evidence that predation by cats
\href{https://www.cnn.com/2020/03/12/world/pet-cats-disrupt-wildlife-scn/index.html}{is
not part of any natural order}. ``House cats have a two- to 10-times
larger impact on wildlife than wild predators,'' Roland Kays, a
zoologist at the North Carolina Museum of Natural Sciences, told CNN
earlier this year.

The effect is even more devastating with feral cats, descendants of
house cats that are not socialized to human beings: In Australia, feral
cats have already been
\href{https://www.nytimes3xbfgragh.onion/2019/04/25/magazine/australia-cat-killing.html?searchResultPosition=8}{the
driving force in the extinction of 22 species}. From an ecological
standpoint, the \href{http://www.iucngisd.org/gisd/}{domestic cat,
whether feral or a free-range pet, is an invasive species}, every year
killing
\href{https://www.fws.gov/southeast/pubs/keep-cats-indoors.pdf}{billions
of birds and mammals} already imperiled by habitat loss, pesticides and
climate change.

The morning after I didn't kill the feral cat, a child from the
neighborhood came to get me, hoping I could help a sick cat she'd found
in her family's driveway. A few weeks earlier, this child had taken an
infant deer mouse out of her dog's mouth, and I had kept it alive
overnight, long enough to drive it safely to
\href{https://waldenspuddle.org/}{Walden's Puddle}, a wildlife rescue
organization. There it would be raised with other baby mice and then set
free to fulfill its own role in the natural cycle of things. My young
friend was hoping for another miracle.

There would be no second miracle. When we got to her house, the
embattled tomcat was lying in the driveway, his limbs twitching, his
eyes unseeing, his hindquarters resting in a pool of urine. From time to
time his neck would arch, and his mouth would pull back in a grimace.
Our feral cat was in agony. Our feral cat was dying, and his suffering
broke my heart.

Later, when Animal Control picked up his body, we learned that the cat
had been poisoned. Most likely a neighbor had set out rat poison, and
the cat had caught and eaten the dying rodent.

Cats are killers, but they are not apex predators. In this neighborhood,
there are coyotes, great horned owls, at least one camera-shy bobcat.
Any one of them could handily kill a cat, but our feral cat didn't lose
his life to a hungry owl or coyote or bobcat. He died because a human
being was too squeamish to set the kind of trap that leaves behind a
corpse. There are mousetraps that kill quickly and painlessly, and those
traps don't weaponize the mouse, turning it into a poison-delivery
system for predators, but such traps do require people to face what they
are doing: taking the life of another creature.

For weeks I have been trying to understand my own tears in the presence
of a dying cat I did not love. It's hard not to feel connected to a
living thing in a state of suffering. In
\href{https://www.theatlantic.com/magazine/archive/1948/01/death-pig/309203/}{``Death
of a Pig,''} E.B. White writes, ``He had evidently become precious to
me, not that he represented a distant nourishment in a hungry time, but
that he had suffered in a suffering world.'' I understand that essay
now.

In the weeks since the tomcat's terrible death, I have thought a lot
about the danger outdoor cats pose to the natural world and also about
the danger the natural world poses to cats. Most of all I have been
thinking about the way human beings, the deadliest predators of all,
keep finding new ways to destroy everything that sustains the planet
that sustains us. A hungry animal cannot be faulted for killing to eat.
A feral cat --- like a house cat allowed to roam outdoors --- is not an
evil creature. Like the poisoned mouse and the poisoned insects and all
the other animals crowded out by development, it is simply a creature
that has been failed by human beings.

My young neighbor came to me for help. At 10, she was sure that someone
my age would know what to do about a dying cat. I did not know what to
do, and I couldn't tell her that the true miracle would not have been
the saving of a doomed cat's life. I couldn't tell her that the true
miracle will never come until human beings have finally learned to live
a better way: in concert with the natural world, and not in domination.

Margaret Renkl is a contributing opinion writer who covers flora, fauna,
politics and culture in the American South. She is the author of the
book ``\href{https://milkweed.org/book/late-migrations}{Late Migrations:
A Natural History of Love and Loss}.''

\emph{The Times is committed to publishing}
\href{https://www.nytimes3xbfgragh.onion/2019/01/31/opinion/letters/letters-to-editor-new-york-times-women.html}{\emph{a
diversity of letters}} \emph{to the editor. We'd like to hear what you
think about this or any of our articles. Here are some}
\href{https://help.nytimes3xbfgragh.onion/hc/en-us/articles/115014925288-How-to-submit-a-letter-to-the-editor}{\emph{tips}}\emph{.
And here's our email:}
\href{mailto:letters@NYTimes.com}{\emph{letters@NYTimes.com}}\emph{.}

\emph{Follow The New York Times Opinion section on}
\href{https://www.facebookcorewwwi.onion/nytopinion}{\emph{Facebook}}\emph{,}
\href{http://twitter.com/NYTOpinion}{\emph{Twitter (@NYTopinion)}}
\emph{and}
\href{https://www.instagram.com/nytopinion/}{\emph{Instagram}}\emph{.}

Advertisement

\protect\hyperlink{after-bottom}{Continue reading the main story}

\hypertarget{site-index}{%
\subsection{Site Index}\label{site-index}}

\hypertarget{site-information-navigation}{%
\subsection{Site Information
Navigation}\label{site-information-navigation}}

\begin{itemize}
\tightlist
\item
  \href{https://help.nytimes3xbfgragh.onion/hc/en-us/articles/115014792127-Copyright-notice}{©~2020~The
  New York Times Company}
\end{itemize}

\begin{itemize}
\tightlist
\item
  \href{https://www.nytco.com/}{NYTCo}
\item
  \href{https://help.nytimes3xbfgragh.onion/hc/en-us/articles/115015385887-Contact-Us}{Contact
  Us}
\item
  \href{https://www.nytco.com/careers/}{Work with us}
\item
  \href{https://nytmediakit.com/}{Advertise}
\item
  \href{http://www.tbrandstudio.com/}{T Brand Studio}
\item
  \href{https://www.nytimes3xbfgragh.onion/privacy/cookie-policy\#how-do-i-manage-trackers}{Your
  Ad Choices}
\item
  \href{https://www.nytimes3xbfgragh.onion/privacy}{Privacy}
\item
  \href{https://help.nytimes3xbfgragh.onion/hc/en-us/articles/115014893428-Terms-of-service}{Terms
  of Service}
\item
  \href{https://help.nytimes3xbfgragh.onion/hc/en-us/articles/115014893968-Terms-of-sale}{Terms
  of Sale}
\item
  \href{https://spiderbites.nytimes3xbfgragh.onion}{Site Map}
\item
  \href{https://help.nytimes3xbfgragh.onion/hc/en-us}{Help}
\item
  \href{https://www.nytimes3xbfgragh.onion/subscription?campaignId=37WXW}{Subscriptions}
\end{itemize}
