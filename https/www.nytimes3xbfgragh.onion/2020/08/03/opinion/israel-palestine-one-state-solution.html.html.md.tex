Sections

SEARCH

\protect\hyperlink{site-content}{Skip to
content}\protect\hyperlink{site-index}{Skip to site index}

\href{https://myaccount.nytimes3xbfgragh.onion/auth/login?response_type=cookie\&client_id=vi}{}

\href{https://www.nytimes3xbfgragh.onion/section/todayspaper}{Today's
Paper}

\href{/section/opinion}{Opinion}\textbar{}The Siren Song of `One State'

\url{https://nyti.ms/3i2eWdp}

\begin{itemize}
\item
\item
\item
\item
\item
\item
\end{itemize}

Advertisement

\protect\hyperlink{after-top}{Continue reading the main story}

\href{/section/opinion}{Opinion}

Supported by

\protect\hyperlink{after-sponsor}{Continue reading the main story}

\hypertarget{the-siren-song-of-one-state}{%
\section{The Siren Song of `One
State'}\label{the-siren-song-of-one-state}}

A hopeless plan could dash hopes for a peaceful resolution between
Israelis and Palestinians.

\href{https://www.nytimes3xbfgragh.onion/by/bret-stephens}{\includegraphics{https://static01.graylady3jvrrxbe.onion/images/2017/08/27/insider/bretstephens/bretstephens-thumbLarge-v6.png}}

By \href{https://www.nytimes3xbfgragh.onion/by/bret-stephens}{Bret
Stephens}

Opinion Columnist

\begin{itemize}
\item
  Aug. 3, 2020
\item
  \begin{itemize}
  \item
  \item
  \item
  \item
  \item
  \item
  \end{itemize}
\end{itemize}

\includegraphics{https://static01.graylady3jvrrxbe.onion/images/2020/08/03/opinion/03stephensWeb/03stephensWeb-articleLarge.jpg?quality=75\&auto=webp\&disable=upscale}

Amos Oz, the Israeli writer who was also a founder of the Peace Now
movement, was once asked by a Norwegian journalist why Jews and
Palestinians couldn't just live as equal citizens in a single state. Oz
countered by asking why Norway and Sweden couldn't just merge into a
single state, too, as they had been for most of the 19th century.

``Clearly, Mr. Oz,'' the journalist replied, ``you know nothing about
the Swedes!''

I heard Oz tell this story many years ago, so it might have been a
Swedish journalist talking about Norwegians. But the point is the same:
If Norwegians don't want to share a state with Swedes, if Scots may not
want to share a state with the English, or Catalans with Spaniards, then
how can anyone imagine Israelis and Palestinians, with rivers of blood
between them, joining hands in a common political enterprise?

The idea is utopian in theory and would be disastrous in practice. It
has no support among Jewish Israelis or Israeli-Arab leaders. As for
Palestinians,
\href{http://pcpsr.org/sites/default/files/Poll\%2076\%20English\%20press\%20release\%20\%D9\%8D_June\%202020.pdf}{a
recent poll finds that}, when given a choice of political alternatives,
only six percent support it.

Peter Beinart, however,
\href{https://www.nytimes3xbfgragh.onion/2020/07/08/opinion/israel-annexation-two-state-solution.html}{endorses
it,} and he seeks to start a movement on the left.

I won't argue here with Beinart on the big picture or the details of his
unworkable and unoriginal plan. (The Libyan dictator Muammar el-Qaddafi
proposed the same thing in
\href{https://www.nytimes3xbfgragh.onion/2009/01/22/opinion/22qaddafi.html}{a
Times Op-Ed in 2009}.) But it's important to point out the types of
damage even a feckless proposal creates, provided it attracts a critical
mass of support. Three points stand out.

The first is the damage to the hopes of a peaceful two-state settlement.
Israelis have been most amenable to territorial concessions when they
felt reasonably confident that Americans understood their security
predicament (narrow borders, mortal enemies, ambivalent friends) and
believed in the moral necessity of a Jewish state. It's why the George
W. Bush administration achieved more
\href{https://www.nytimes3xbfgragh.onion/2005/08/14/nyregion/pain-of-israels-withdrawal-from-gaza-strip-is-felt-by-american.html}{in
terms of territorial withdrawals} by hugging Israel close than Barack
Obama's administration did by
\href{https://www.wsj.com/articles/how-obama-abandoned-israel-1434409772}{deliberately
trying to ``maintain daylight''} between Washington and Jerusalem.

This is not old history. Benjamin Netanyahu came close this summer to
unilaterally annexing large parts of the West Bank,
\href{https://www.washingtonpost.com/outlook/im-an-ardent-zionist-but-israels-annexation-makes-no-sense/2020/06/25/f949e6a4-b59e-11ea-a8da-693df3d7674a_story.html}{partly
on the view} that the delegitimization of Israel meant that it should
take what it can get while it can get it. The more Israel is ostracized
because it's a Jewish state, the less amenable it will be to make
concessions of any sort. Far from creating pressure on Israel to make
way for a Palestinian state, as some advocates of a one-state solution
fancy, Beinart and his fellow travelers are unwitting handmaidens to the
Israeli right-wingers they claim to despise.

Next there is the damage to the Palestinians. Scores of Palestinians
were killed in 2018 and 2019 in a long series of border
\href{https://www.theguardian.com/world/ng-interactive/2019/mar/29/a-year-of-bloodshed-at-gaza-border-protests}{clashes}
in Gaza, purportedly to demand their ``right of return'' to pre-1967
Israel. That demand, as left-leaning Israeli writers Adi Schwartz and
Einat Wilf demonstrate in their convincing and essential book, ``The War
of Return,'' has been the central obstacle to reaching a peace
settlement. Unlike the surrender of settlements (which Israel repeatedly
proved willing to abandon in Sinai and Gaza for the sake of peace), it
is the one demand no Israeli government can concede if it means to
preserve the country's Jewish character.

Anyone who demands that Israel withdraw from part or all of the West
Bank needs to be equally forceful in demanding that Palestinians abandon
this so-called right. One-state advocates achieve the precise opposite:
They foster a crippling fantasy that the right of return need never be
conceded because eventually Israel will be pressured into dissolving
itself. That will never happen, but chances for peace will be missed in
the future, as they were in the past, so long as the fantasy survives.

The final bit of damage is to the American Jewish community. For
decades, the opinions and advice of American Jews mattered to Israel.
But if the views of a significant segment of American Jewish opinion are
soon to harden into a moralizing anti-Zionism, it will only persuade
Israelis to reciprocate with indifference and contempt. Whatever else
advocates of a one-state solution think they are doing, they are
withdrawing from any meaningful dialogue with Israelis about the future
of a Jewish homeland.

It used to be that Israelis depended on a secure and thriving American
Jewry to help stand up their fragile state. Today it is
\href{https://www.tabletmag.com/sections/arts-letters/articles/the-collapse}{American
Jewry that is fragile}, threatened by dwindling cultural influence,
stagnant demographic trends, increasing alienation from the Democratic
Party and abiding discomfort with the G.O.P., and rising anti-Semitism
---
\href{https://www.nytimes3xbfgragh.onion/2019/02/08/opinion/sunday/israel-progressive-anti-semitism.html}{sometimes
masked as anti-Zionism} --- from across the political spectrum.

Should American Jews start looking for the exits --- just as every other
Diaspora community in history has done,
\href{https://www.nationalgeographic.com/history/2019/11/french-jews-fleeing-country/}{and
continues to do} --- they will be grateful to find a Jewish state that
resisted the siren song of ``one state.''

\emph{The Times is committed to publishing}
\href{https://www.nytimes3xbfgragh.onion/2019/01/31/opinion/letters/letters-to-editor-new-york-times-women.html}{\emph{a
diversity of letters}} \emph{to the editor. We'd like to hear what you
think about this or any of our articles. Here are some}
\href{https://help.nytimes3xbfgragh.onion/hc/en-us/articles/115014925288-How-to-submit-a-letter-to-the-editor}{\emph{tips}}\emph{.
And here's our email:}
\href{mailto:letters@NYTimes.com}{\emph{letters@NYTimes.com}}\emph{.}

\emph{Follow The New York Times Opinion section on}
\href{https://www.facebookcorewwwi.onion/nytopinion}{\emph{Facebook}}\emph{,}
\href{http://twitter.com/NYTOpinion}{\emph{Twitter (@NYTopinion)}}
\emph{and}
\href{https://www.instagram.com/nytopinion/}{\emph{Instagram}}\emph{.}

Advertisement

\protect\hyperlink{after-bottom}{Continue reading the main story}

\hypertarget{site-index}{%
\subsection{Site Index}\label{site-index}}

\hypertarget{site-information-navigation}{%
\subsection{Site Information
Navigation}\label{site-information-navigation}}

\begin{itemize}
\tightlist
\item
  \href{https://help.nytimes3xbfgragh.onion/hc/en-us/articles/115014792127-Copyright-notice}{©~2020~The
  New York Times Company}
\end{itemize}

\begin{itemize}
\tightlist
\item
  \href{https://www.nytco.com/}{NYTCo}
\item
  \href{https://help.nytimes3xbfgragh.onion/hc/en-us/articles/115015385887-Contact-Us}{Contact
  Us}
\item
  \href{https://www.nytco.com/careers/}{Work with us}
\item
  \href{https://nytmediakit.com/}{Advertise}
\item
  \href{http://www.tbrandstudio.com/}{T Brand Studio}
\item
  \href{https://www.nytimes3xbfgragh.onion/privacy/cookie-policy\#how-do-i-manage-trackers}{Your
  Ad Choices}
\item
  \href{https://www.nytimes3xbfgragh.onion/privacy}{Privacy}
\item
  \href{https://help.nytimes3xbfgragh.onion/hc/en-us/articles/115014893428-Terms-of-service}{Terms
  of Service}
\item
  \href{https://help.nytimes3xbfgragh.onion/hc/en-us/articles/115014893968-Terms-of-sale}{Terms
  of Sale}
\item
  \href{https://spiderbites.nytimes3xbfgragh.onion}{Site Map}
\item
  \href{https://help.nytimes3xbfgragh.onion/hc/en-us}{Help}
\item
  \href{https://www.nytimes3xbfgragh.onion/subscription?campaignId=37WXW}{Subscriptions}
\end{itemize}
