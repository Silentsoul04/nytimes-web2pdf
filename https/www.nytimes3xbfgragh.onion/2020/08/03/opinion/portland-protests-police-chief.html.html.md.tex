Sections

SEARCH

\protect\hyperlink{site-content}{Skip to
content}\protect\hyperlink{site-index}{Skip to site index}

\href{https://myaccount.nytimes3xbfgragh.onion/auth/login?response_type=cookie\&client_id=vi}{}

\href{https://www.nytimes3xbfgragh.onion/section/todayspaper}{Today's
Paper}

\href{/section/opinion}{Opinion}\textbar{}I'm the Police Chief in
Portland. Violence Isn't the Answer.

\url{https://nyti.ms/2DxuIOi}

\begin{itemize}
\item
\item
\item
\item
\item
\end{itemize}

Advertisement

\protect\hyperlink{after-top}{Continue reading the main story}

\href{/section/opinion}{Opinion}

Supported by

\protect\hyperlink{after-sponsor}{Continue reading the main story}

\hypertarget{im-the-police-chief-in-portland-violence-isnt-the-answer}{%
\section{I'm the Police Chief in Portland. Violence Isn't the
Answer.}\label{im-the-police-chief-in-portland-violence-isnt-the-answer}}

As a Black man and a public servant, I see that spectacle is drowning
out the voices that need to be heard to make positive change.

By Chuck Lovell

Mr. Lovell is the chief of the Police Bureau in Portland, Ore.

\begin{itemize}
\item
  Aug. 3, 2020
\item
  \begin{itemize}
  \item
  \item
  \item
  \item
  \item
  \end{itemize}
\end{itemize}

\includegraphics{https://static01.graylady3jvrrxbe.onion/images/2020/08/03/opinion/03Lovell1/merlin_175146375_96b5f246-7b90-4ab9-8501-cd53c0c0c636-articleLarge.jpg?quality=75\&auto=webp\&disable=upscale}

After the horrendous killing of George Floyd, people in Portland, Ore.,
joined with thousands across the country in demonstrations to address
police reform and widespread systemic racism. The leaders of the
Portland Police Bureau denounced this tragic death, and we reiterated
our willingness to engage in reforms.

But Portland has now faced weeks of extreme difficulties and drew
intense national attention after federal officers were deployed here.

As police officers, our duty is to uphold the rights of anyone to
assemble peacefully and engage in free speech. But over the months of
protests, a concerning dynamic developed. People protested peacefully,
while others engaged in dangerous activities that could have resulted in
injury and even death.

\includegraphics{https://static01.graylady3jvrrxbe.onion/images/2020/08/03/opinion/03Lovell2/03Lovell2-articleLarge.jpg?quality=75\&auto=webp\&disable=upscale}

The night of May 29 was a pivotal moment for our city. Hundreds of
people, in a coordinated effort, attacked the Justice Center, which
includes our Central Precinct station and the Multnomah County Detention
Center. They broke into the building, destroyed the first-floor interior
and lit fires. Afterward, there was looting and destruction downtown.

Yet in the following weeks, thousands of people demonstrated peacefully
in an awesome expression of First Amendment rights. The Police Bureau
had little to no interaction with members of this group, because they
did not allow criminal activity to impede their message.

As a Black man and a public servant, I have a unique perspective. I
agree with a local pastor, E.D. Mondainé, who
\href{https://www.washingtonpost.com/opinions/2020/07/23/portlands-protests-were-supposed-be-about-black-lives-now-theyre-white-spectacle/}{stated}
these ``spectacles'' are drowning out the voices that need to be heard
to make positive change. This violence is doing nothing to further the
Black Lives Matter movement.

On one night, for example, individuals screwed the doors of our North
Precinct station shut, barricaded other entrances and lit the station on
fire with people inside. Nearby businesses, owned by people of color,
were damaged and looted. On other nights, there were multiple attempts
to breach the Justice Center. Other law enforcement facilities were
targeted, including the union building, which was broken into and had
fires set within.

Violence and destruction have also been directed at the federal
courthouse, which sits next to the Justice Center. The president sent
additional federal agents to Portland, and our city became national
news.

Gov. Kate Brown recently
\href{https://www.nytimes3xbfgragh.onion/2020/07/29/us/protests-portland-federal-withdrawal.html}{negotiated}
an agreement with the Department of Homeland Security to remove federal
agents in stages and have them replaced by Oregon State Police. The
governor acknowledged that this might not be the end of the violence. I
hope she is wrong.

Portland is a beautiful and vibrant city with smart, progressive people,
and I am hopeful we can come together to move beyond the unrest and
refocus on critical issues.

The Portland Police Bureau remains committed to protecting life and
responding to events as appropriate. I am proud of our efforts in
extreme circumstances few in the country have faced.

During these events, our agency has responded to assaults, stabbings,
shootings, people with guns and the stockpiling of explosives. Shooting
off commercial-grade fireworks and mortars is not peaceful protest. We
are fortunate that no one has been killed.

There are those who say the Portland police have not done enough to
quell violence. I ask them to come speak with our officers, who have
been responding for two months to protests. They have served with
professionalism, courage and resiliency through an extraordinary time.
Many have been injured and some have received threats of violence to
themselves or their families. They would prefer to return to regular
patrol and investigative duties and see peace in our community.

The voices of victims are not heard as well. Because of the protests,
officers have not been able to respond to 911 calls or have been delayed
for hours. Investigators' cases lie on their desks as they work nights
to process arrests. We have seen an alarming increase in shootings and
homicides. We need to redirect our focus to preventing and solving these
crimes that are taking a hugely disproportionate number of minority
lives.

I have said frequently that the Portland Police Bureau is committed to
reform. We are a progressive agency and have demonstrated our
willingness to change over the past eight years. Working with the
Department of Justice, we have made significant changes to our policies
and training. The Portland Police Bureau's policy on the use of deadly
force is more restrictive than state and federal law.

We recently enhanced our Community Engagement Unit to help build trust
and legitimacy with the communities we serve. We have also developed
several advisory councils that help the Police Bureau make decisions
with the benefit of a diverse set of inputs.

The Portland Police Bureau has had an equity and inclusion office for
over five years. I recently changed the organizational structure to have
it report directly to me, to ensure we are prioritizing its work.

I have confidence in our community and the people who have dedicated
their lives to building relationships with police. They have stood up
and said no more violence. I stand with them with a servant's heart,
committed to being leaders in police reform.

Chuck Lovell is the chief of the Police Bureau in Portland, Ore.

\emph{The Times is committed to publishing}
\href{https://www.nytimes3xbfgragh.onion/2019/01/31/opinion/letters/letters-to-editor-new-york-times-women.html}{\emph{a
diversity of letters}} \emph{to the editor. We'd like to hear what you
think about this or any of our articles. Here are some}
\href{https://help.nytimes3xbfgragh.onion/hc/en-us/articles/115014925288-How-to-submit-a-letter-to-the-editor}{\emph{tips}}\emph{.
And here's our email:}
\href{mailto:letters@NYTimes.com}{\emph{letters@NYTimes.com}}\emph{.}

\emph{Follow The New York Times Opinion section on}
\href{https://www.facebookcorewwwi.onion/nytopinion}{\emph{Facebook}}\emph{,}
\href{http://twitter.com/NYTOpinion}{\emph{Twitter (@NYTopinion)}}
\emph{and}
\href{https://www.instagram.com/nytopinion/}{\emph{Instagram}}\emph{.}

Advertisement

\protect\hyperlink{after-bottom}{Continue reading the main story}

\hypertarget{site-index}{%
\subsection{Site Index}\label{site-index}}

\hypertarget{site-information-navigation}{%
\subsection{Site Information
Navigation}\label{site-information-navigation}}

\begin{itemize}
\tightlist
\item
  \href{https://help.nytimes3xbfgragh.onion/hc/en-us/articles/115014792127-Copyright-notice}{©~2020~The
  New York Times Company}
\end{itemize}

\begin{itemize}
\tightlist
\item
  \href{https://www.nytco.com/}{NYTCo}
\item
  \href{https://help.nytimes3xbfgragh.onion/hc/en-us/articles/115015385887-Contact-Us}{Contact
  Us}
\item
  \href{https://www.nytco.com/careers/}{Work with us}
\item
  \href{https://nytmediakit.com/}{Advertise}
\item
  \href{http://www.tbrandstudio.com/}{T Brand Studio}
\item
  \href{https://www.nytimes3xbfgragh.onion/privacy/cookie-policy\#how-do-i-manage-trackers}{Your
  Ad Choices}
\item
  \href{https://www.nytimes3xbfgragh.onion/privacy}{Privacy}
\item
  \href{https://help.nytimes3xbfgragh.onion/hc/en-us/articles/115014893428-Terms-of-service}{Terms
  of Service}
\item
  \href{https://help.nytimes3xbfgragh.onion/hc/en-us/articles/115014893968-Terms-of-sale}{Terms
  of Sale}
\item
  \href{https://spiderbites.nytimes3xbfgragh.onion}{Site Map}
\item
  \href{https://help.nytimes3xbfgragh.onion/hc/en-us}{Help}
\item
  \href{https://www.nytimes3xbfgragh.onion/subscription?campaignId=37WXW}{Subscriptions}
\end{itemize}
