Sections

SEARCH

\protect\hyperlink{site-content}{Skip to
content}\protect\hyperlink{site-index}{Skip to site index}

\href{https://www.nytimes3xbfgragh.onion/section/realestate}{Real
Estate}

\href{https://myaccount.nytimes3xbfgragh.onion/auth/login?response_type=cookie\&client_id=vi}{}

\href{https://www.nytimes3xbfgragh.onion/section/todayspaper}{Today's
Paper}

\href{/section/realestate}{Real Estate}\textbar{}The Challenge: Finding
a Roommate During a Pandemic

\url{https://nyti.ms/2DqJkzh}

\begin{itemize}
\item
\item
\item
\item
\item
\end{itemize}

\href{https://www.nytimes3xbfgragh.onion/spotlight/at-home?action=click\&pgtype=Article\&state=default\&region=TOP_BANNER\&context=at_home_menu}{At
Home}

\begin{itemize}
\tightlist
\item
  \href{https://www.nytimes3xbfgragh.onion/2020/08/03/well/family/the-benefits-of-talking-to-strangers.html?action=click\&pgtype=Article\&state=default\&region=TOP_BANNER\&context=at_home_menu}{Talk:
  To Strangers}
\item
  \href{https://www.nytimes3xbfgragh.onion/2020/08/01/at-home/coronavirus-make-pizza-on-a-grill.html?action=click\&pgtype=Article\&state=default\&region=TOP_BANNER\&context=at_home_menu}{Make:
  Grilled Pizza}
\item
  \href{https://www.nytimes3xbfgragh.onion/2020/07/31/arts/television/goldbergs-abc-stream.html?action=click\&pgtype=Article\&state=default\&region=TOP_BANNER\&context=at_home_menu}{Watch:
  'The Goldbergs'}
\item
  \href{https://www.nytimes3xbfgragh.onion/interactive/2020/at-home/even-more-reporters-editors-diaries-lists-recommendations.html?action=click\&pgtype=Article\&state=default\&region=TOP_BANNER\&context=at_home_menu}{Explore:
  Reporters' Google Docs}
\end{itemize}

Advertisement

\protect\hyperlink{after-top}{Continue reading the main story}

Supported by

\protect\hyperlink{after-sponsor}{Continue reading the main story}

renters

\hypertarget{the-challenge-finding-a-roommate-during-a-pandemic}{%
\section{The Challenge: Finding a Roommate During a
Pandemic}\label{the-challenge-finding-a-roommate-during-a-pandemic}}

With social distancing required, two new roommates discovered, it's a
little like marrying someone before you meet.

\includegraphics{https://static01.graylady3jvrrxbe.onion/images/2020/08/03/realestate/03renters1/merlin_175089951_ad28cb14-0ee4-44c8-82eb-944ad75dd910-articleLarge.jpg?quality=75\&auto=webp\&disable=upscale}

\href{https://www.nytimes3xbfgragh.onion/by/kim-velsey}{\includegraphics{https://static01.graylady3jvrrxbe.onion/images/2018/06/14/multimedia/author-kim-velsey/author-kim-velsey-thumbLarge.png}}

By \href{https://www.nytimes3xbfgragh.onion/by/kim-velsey}{Kim Velsey}

\begin{itemize}
\item
  Aug. 3, 2020
\item
  \begin{itemize}
  \item
  \item
  \item
  \item
  \item
  \end{itemize}
\end{itemize}

Most people spent the spring trying to stay as far away from strangers
as possible. But for renters like Elizabeth Merritt and Michael Daly,
who needed to find roommates in the midst of the pandemic, moving in
with someone they had never met was part of the reality of living in New
York, global health crisis or not.

At any other time, an in-person meetup would be standard before agreeing
to become roommates. And under normal circumstances, two busy
professionals sharing an apartment might cross paths only on evenings
and weekends. But the pandemic meant they would be spending most of
their waking hours in the same space.

``It is a little like marrying someone before you meet,'' said Ms.
Merritt, 26, who met Mr. Daly, 25, on the Facebook group Gypsy Housing
in late March and moved into a Bushwick, Brooklyn, two-bedroom with him
in late June. ``You're sharing a toilet and depending on them to pay
rent.''

Ms. Merritt had been living in a Bedford-Stuyvesant one-bedroom that
cost \$1,950 a month. Her lease ended in May, and while a pandemic
wasn't an ideal time to move, she had planned to stay in the one-bedroom
only for a year, a time-limited splurge after a stint living with five
roommates, two cats and a dog.

\includegraphics{https://static01.graylady3jvrrxbe.onion/images/2020/08/03/realestate/03renters2/merlin_175089993_f94796c4-1166-44a9-840b-a5feb517dff8-articleLarge.jpg?quality=75\&auto=webp\&disable=upscale}

``I thought it would be nice to try living alone, but I couldn't really
afford it,'' said Ms. Merritt, who sells technical textiles for a
company in the garment district. She had managed to cover the cost, but
money had been uncomfortably tight, and the prospect of a raise seemed
remote, given the pandemic. She had also turned 26, which meant that she
had been booted off her parents' health insurance plan and had to buy
into the company plan, which tacked a few hundred dollars onto her
monthly expenses.

As for Mr. Daly, a public relations associate who moved to New York in
October, he had hoped to stay in the three-bedroom Bushwick sublet where
he had been living for a few months. But then he learned that both
roommates were planning to move out when the lease expired at the end of
June.

\begin{center}\rule{0.5\linewidth}{\linethickness}\end{center}

\$2,500 \textbar{} Bushwick

\hypertarget{elizabeth-merritt-26-and-michael-daly-25}{%
\subsection{Elizabeth Merritt, 26, and Michael Daly,
25}\label{elizabeth-merritt-26-and-michael-daly-25}}

\textbf{Occupations:} Ms. Merritt works in sales for a technical
textiles manufacturer; Mr. Daly is an associate at public relations
firm.\\
\textbf{The sofa:} Mr. Daly's sofa wouldn't fit up the stairs, so they
bought a smaller one online, but even that was a tight fit. ``It had to
go very close to the oven,'' Ms. Merritt said, pointing out the somewhat
strange juxtaposition. ``But that's just a New York thing.''\\
\textbf{Their neighborhood:} ``If and when we can entertain, it will be
great,'' Ms. Merritt said. ``We're in this wedge where the J, Z and M
meet. We're really close to Mood Ring, Birdy's, Happyfun Hideaway.
Should things resume, we'll be ready.''

\begin{center}\rule{0.5\linewidth}{\linethickness}\end{center}

Image

Mr. Daly and Ms. Merritt liked that the bedrooms were on opposite sides
of the apartment, maximizing privacy. ``It's kind of like we have our
own wings,'' Ms. Merritt said.Credit...Katherine Marks for The New York
Times

At first, Mr. Daly thought he would apply for his apartment with two new
roommates. A friend wanted one of the rooms, and after Zooming with Ms.
Merritt --- they also started following one another on social media ---
they decided that she would be a good fit for the third room. But then
the friend, who had been worried about signing a lease given the
economic uncertainty, decided to play it safe and moved back in with his
parents.

After several weeks of trying to find a third roommate on Gypsy Housing
--- the odds were not in their favor, they noted, as people were
flooding the site with sublets and lease takeovers --- Ms. Merritt and
Mr. Daly decided it would be easier to look for a two-bedroom.

They agreed on a budget --- \$2,500 or less --- and a neighborhood:
Bushwick, by the JMZ train line, which worked for their commutes, if
they ever went back to the office. Ms. Merritt's landlord also agreed to
extend her lease by a month so that she and Mr. Daly could look for a
July 1 move-in.

The pair met in person for the first time in May, when they went to look
at an apartment. Mr. Daly said he wasn't worried that they would clash
in person. ``We'd spent so much time talking together at that point,''
he said.

``And we were both wearing masks, so I felt like we were both taking
coronavirus seriously,'' Ms. Merritt said.

They were also relieved to find they were aligned in their apartment
tastes: After touring the space, they agreed that it was serviceable,
but nothing special.

The broker mentioned that he had another apartment nearby, a
\$2,500-a-month two-bedroom with a lofted mezzanine that had come back
on the market after the landlord discovered the would-be tenants hadn't
been paying rent at their previous apartment. As it was somewhat
dubiously listed as a three-bedroom, it had never come up in their
searches.

They took to the space immediately.

``We liked that it had high ceilings and more character. If we
entertained, we felt like we'd be a little bit more proud of this
apartment,'' Ms. Merritt said. ``And our bedrooms could not be farther
apart --- it's kind of like we have our own wings.''

``We didn't want to share a wall,'' Mr. Daly said. And Ms. Merritt
recalled that she once ``had one roommate that I could hear breathe,
just breathe. That freaked me out, having no feeling of privacy.''

The only real downside was that the landlord wanted a June 20 move-in
and wouldn't budge, which meant paying double rent for 10 days. That,
and the bedrooms had no closets: Instead, there were small, wall-mounted
open wardrobes --- a few rows of shelving with a rack below to hang
clothing on. (Mr. Daly likes the feature; Ms. Merritt prefers garment
racks.)

They decided to go for it anyway. The mezzanine over the living room,
laundry in the building and the shared backyard helped tip the scales.

So far, things have gone fairly well, with a few minor glitches. The
movers couldn't fit Mr. Daly's sofa up the building stairs, so he had to
sell it in a hurry, as it was stuck in the lobby. The roommates ordered
a new one from Overstock, but it took a while to arrive, so they spent
most of the last month in their rooms, as common-space seating options
were limited to the metal bar stools.

But even now that the new sofa has arrived, they find that they spend
most of the day working in their rooms.

``We don't really get in each other's way,'' Ms. Merritt said. ``And we
are home 100 percent of the time, but neither of us cooks much.''

``I eat a lot of soup. I'm not in the kitchen a lot on weekdays,'' Mr.
Daly said.

``I like soup, too. Cereal, oatmeal, maybe a salad, and we both eat a
lot of chips,'' Ms. Merritt said. ``Mostly grab-and-go stuff, even
though we're always here.''

As for the housewarming, it will have to wait. ``I do fantasize, but I'm
not sure when it will happen,'' Ms. Merritt said.

At least there's time. ``I want to be here for two years, at least,''
she said.

For weekly email updates on residential real estate news,
\href{http://www.nytimes3xbfgragh.onion/newsletters/realestate/}{sign up
here}. Follow us on Twitter:
\href{https://twitter.com/nytrealestate}{@nytrealestate}.

Advertisement

\protect\hyperlink{after-bottom}{Continue reading the main story}

\hypertarget{site-index}{%
\subsection{Site Index}\label{site-index}}

\hypertarget{site-information-navigation}{%
\subsection{Site Information
Navigation}\label{site-information-navigation}}

\begin{itemize}
\tightlist
\item
  \href{https://help.nytimes3xbfgragh.onion/hc/en-us/articles/115014792127-Copyright-notice}{©~2020~The
  New York Times Company}
\end{itemize}

\begin{itemize}
\tightlist
\item
  \href{https://www.nytco.com/}{NYTCo}
\item
  \href{https://help.nytimes3xbfgragh.onion/hc/en-us/articles/115015385887-Contact-Us}{Contact
  Us}
\item
  \href{https://www.nytco.com/careers/}{Work with us}
\item
  \href{https://nytmediakit.com/}{Advertise}
\item
  \href{http://www.tbrandstudio.com/}{T Brand Studio}
\item
  \href{https://www.nytimes3xbfgragh.onion/privacy/cookie-policy\#how-do-i-manage-trackers}{Your
  Ad Choices}
\item
  \href{https://www.nytimes3xbfgragh.onion/privacy}{Privacy}
\item
  \href{https://help.nytimes3xbfgragh.onion/hc/en-us/articles/115014893428-Terms-of-service}{Terms
  of Service}
\item
  \href{https://help.nytimes3xbfgragh.onion/hc/en-us/articles/115014893968-Terms-of-sale}{Terms
  of Sale}
\item
  \href{https://spiderbites.nytimes3xbfgragh.onion}{Site Map}
\item
  \href{https://help.nytimes3xbfgragh.onion/hc/en-us}{Help}
\item
  \href{https://www.nytimes3xbfgragh.onion/subscription?campaignId=37WXW}{Subscriptions}
\end{itemize}
