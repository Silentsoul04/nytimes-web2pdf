Sections

SEARCH

\protect\hyperlink{site-content}{Skip to
content}\protect\hyperlink{site-index}{Skip to site index}

\href{https://www.nytimes3xbfgragh.onion/section/science}{Science}

\href{https://myaccount.nytimes3xbfgragh.onion/auth/login?response_type=cookie\&client_id=vi}{}

\href{https://www.nytimes3xbfgragh.onion/section/todayspaper}{Today's
Paper}

\href{/section/science}{Science}\textbar{}There Are Two Ways Out of a
Frog. This Beetle Chose the Back Door.

\url{https://nyti.ms/3gq3V5k}

\begin{itemize}
\item
\item
\item
\item
\item
\item
\end{itemize}

Advertisement

\protect\hyperlink{after-top}{Continue reading the main story}

Supported by

\protect\hyperlink{after-sponsor}{Continue reading the main story}

Trilobites

\hypertarget{there-are-two-ways-out-of-a-frog-this-beetle-chose-the-back-door}{%
\section{There Are Two Ways Out of a Frog. This Beetle Chose the Back
Door.}\label{there-are-two-ways-out-of-a-frog-this-beetle-chose-the-back-door}}

A researcher fed beetles to frogs. The encounter did not end as
expected.

\includegraphics{https://static01.graylady3jvrrxbe.onion/images/2020/08/03/science/03TB-BEETLE1/merlin_175260999_c4908271-f4de-4d3e-97af-bd0255f15125-articleLarge.jpg?quality=75\&auto=webp\&disable=upscale}

By
\href{https://www.nytimes3xbfgragh.onion/by/katherine-j--wu}{Katherine
J. Wu}

\begin{itemize}
\item
  Aug. 3, 2020
\item
  \begin{itemize}
  \item
  \item
  \item
  \item
  \item
  \item
  \end{itemize}
\end{itemize}

\href{https://www.nytimes3xbfgragh.onion/es/2020/08/04/espanol/ciencia-y-tecnologia/escarabajo-excremento-rana.html}{Leer
en español}

It's a familiar story: Predator hunts prey. Predator catches prey.
Predator gulps down prey.

Usually, that's it. But the water scavenger beetle Regimbartia attenuata
says, ``Not today.'' After getting swallowed by a frog, this plucky
little insect can scuttle down the amphibian's gut and force it to poop
--- emerging slightly soiled, but very much alive.

The bug's transit through the digestive tract can last as briefly as six
minutes, a measly fraction of the two or more days it typically takes
for a frog to fully digest and defecate its dinner, according to a study
published Monday in
\href{http://dx.doi.org/10.1016/j.cub.2020.06.026}{Current Biology}.

``This is a weirdly wonderful behavior that I hadn't heard about
before,'' said Carla Bardua, an evolutionary biologist at London's
Natural History Museum who wasn't involved in the study. ``That a little
beetle can actively swim through a digestive system is peculiar and
amazing.''

Shinji Sugiura, a biologist at Kobe University in Japan, has been
cataloging the
\href{https://royalsocietypublishing.org/doi/10.1098/rsbl.2017.0647}{strange
shenanigans} of \href{https://peerj.com/articles/5942/}{insects and
their predators}
\href{https://onlinelibrary.wiley.com/doi/abs/10.1111/ens.12423}{for
years}. Some bugs, for instance, goad toads into
\href{https://www.nytimes3xbfgragh.onion/2018/02/06/science/bombardier-beetle-toad-vomit.html}{puking
them back up} after they've been gobbled.

``Insect morphologies and behaviors always inspire me,'' Dr. Sugiura
said in an email, adding that he's particularly keen on defenses against
predators that seem ``unimaginable.''

After noticing that Regimbartia beetles and frogs frequent the same
paddy fields in Japan, Dr. Sugiura brought together one specimen of each
in the lab, expecting the insect would be spit out. Instead, it rocketed
out the other end of the digestive tract --- a fecal feat that Dr.
Sugiura managed to capture on film.

Eager to test the behavior's limits, Dr. Sugiura repeated his
experiments with five species of insect-munching frogs in the lab. A
whopping 90 percent of the beetles they swallowed made it out the other
end alive, all within six hours of being gulped down.

Beetles of other species didn't fare quite as well and were excreted as
corpses after a couple days in amphibio*.* Dead Regimbartia took days
too, hinting that their living counterparts were actively engineering
their great escapes. Without watching the bugs in action from within the
frogs' bowels, Dr. Sugiura can't say for sure what their strategy is.
But when he immobilized the beetles' legs with wax, they died a slow
digestive death.

``That was smoking gun evidence that they are using their legs,'' said
Nora Moskowitz, who studies frog digestion at Stanford University but
wasn't involved in the study.

Dr. Sugiura thinks Regimbartia beetles may use their legs to brace
themselves and crawl through the gut, which can stretch several inches
--- an arduous journey for a four- or five-millimeter-long beetle. When
they reach the end of that tunnel, the insects may be able to tickle
open the cloacal sphincter, the ring of muscle that drawstrings the
frog's rear end shut, expelling themselves in a flood of feces.

A trek through this passage probably isn't trivial, said Aurora
Alvarez-Buylla, a frog researcher at Stanford University who wasn't
involved in the study. Because frogs swallow their prey whole, their
digestive juices have to be potent. ``You're dealing with a chemical and
acidic environment that is built to pull things apart and break them
down,'' she said.

But as far as Dr. Sugiura could tell, the insects were entirely unfazed
by their tortuous trip through the tract. Once liberated, they simply
extracted themselves from the dung and swam happily onward. Months
later, some of the bugs were still kicking about as if the traumatic
encounter had never even happened.

The insects' resilient outer casing, or exoskeleton, might help. But a
few repeat trips down a frog's gullet could eventually take a toll, Dr.
Sugiura said. More experiments are needed, he said, to understand how it
all comes out in the end.

The frogs, too, seemed to depart the encounter unscathed. According to
Dr. Sugiura, amphibian waste is often studded with the hard body parts
of prey.

``However,'' he said, ``I do not want to eat this beetle if I'm a
frog.''

Advertisement

\protect\hyperlink{after-bottom}{Continue reading the main story}

\hypertarget{site-index}{%
\subsection{Site Index}\label{site-index}}

\hypertarget{site-information-navigation}{%
\subsection{Site Information
Navigation}\label{site-information-navigation}}

\begin{itemize}
\tightlist
\item
  \href{https://help.nytimes3xbfgragh.onion/hc/en-us/articles/115014792127-Copyright-notice}{©~2020~The
  New York Times Company}
\end{itemize}

\begin{itemize}
\tightlist
\item
  \href{https://www.nytco.com/}{NYTCo}
\item
  \href{https://help.nytimes3xbfgragh.onion/hc/en-us/articles/115015385887-Contact-Us}{Contact
  Us}
\item
  \href{https://www.nytco.com/careers/}{Work with us}
\item
  \href{https://nytmediakit.com/}{Advertise}
\item
  \href{http://www.tbrandstudio.com/}{T Brand Studio}
\item
  \href{https://www.nytimes3xbfgragh.onion/privacy/cookie-policy\#how-do-i-manage-trackers}{Your
  Ad Choices}
\item
  \href{https://www.nytimes3xbfgragh.onion/privacy}{Privacy}
\item
  \href{https://help.nytimes3xbfgragh.onion/hc/en-us/articles/115014893428-Terms-of-service}{Terms
  of Service}
\item
  \href{https://help.nytimes3xbfgragh.onion/hc/en-us/articles/115014893968-Terms-of-sale}{Terms
  of Sale}
\item
  \href{https://spiderbites.nytimes3xbfgragh.onion}{Site Map}
\item
  \href{https://help.nytimes3xbfgragh.onion/hc/en-us}{Help}
\item
  \href{https://www.nytimes3xbfgragh.onion/subscription?campaignId=37WXW}{Subscriptions}
\end{itemize}
