Sections

SEARCH

\protect\hyperlink{site-content}{Skip to
content}\protect\hyperlink{site-index}{Skip to site index}

\href{https://www.nytimes3xbfgragh.onion/section/smarter-living}{Smarter
Living}

\href{https://myaccount.nytimes3xbfgragh.onion/auth/login?response_type=cookie\&client_id=vi}{}

\href{https://www.nytimes3xbfgragh.onion/section/todayspaper}{Today's
Paper}

\href{/section/smarter-living}{Smarter Living}\textbar{}4 Simple Ways to
Take Control of Your Coronavirus Budget

\url{https://nyti.ms/2BT1sBg}

\begin{itemize}
\item
\item
\item
\item
\item
\end{itemize}

\href{https://www.nytimes3xbfgragh.onion/news-event/coronavirus?action=click\&pgtype=Article\&state=default\&region=TOP_BANNER\&context=storylines_menu}{The
Coronavirus Outbreak}

\begin{itemize}
\tightlist
\item
  live\href{https://www.nytimes3xbfgragh.onion/2020/08/04/world/coronavirus-cases.html?action=click\&pgtype=Article\&state=default\&region=TOP_BANNER\&context=storylines_menu}{Latest
  Updates}
\item
  \href{https://www.nytimes3xbfgragh.onion/interactive/2020/us/coronavirus-us-cases.html?action=click\&pgtype=Article\&state=default\&region=TOP_BANNER\&context=storylines_menu}{Maps
  and Cases}
\item
  \href{https://www.nytimes3xbfgragh.onion/interactive/2020/science/coronavirus-vaccine-tracker.html?action=click\&pgtype=Article\&state=default\&region=TOP_BANNER\&context=storylines_menu}{Vaccine
  Tracker}
\item
  \href{https://www.nytimes3xbfgragh.onion/2020/08/02/us/covid-college-reopening.html?action=click\&pgtype=Article\&state=default\&region=TOP_BANNER\&context=storylines_menu}{College
  Reopening}
\item
  \href{https://www.nytimes3xbfgragh.onion/live/2020/08/04/business/stock-market-today-coronavirus?action=click\&pgtype=Article\&state=default\&region=TOP_BANNER\&context=storylines_menu}{Economy}
\end{itemize}

Advertisement

\protect\hyperlink{after-top}{Continue reading the main story}

Supported by

\protect\hyperlink{after-sponsor}{Continue reading the main story}

\hypertarget{4-simple-ways-to-take-control-of-your-coronavirus-budget}{%
\section{4 Simple Ways to Take Control of Your Coronavirus
Budget}\label{4-simple-ways-to-take-control-of-your-coronavirus-budget}}

The pandemic has changed the way we budget. Here's some advice to get
back on track.

\includegraphics{https://static01.graylady3jvrrxbe.onion/images/2020/08/03/smarter-living/04sl-virus-budgetimage/04sl-virus-budgetimage-articleLarge.png?quality=75\&auto=webp\&disable=upscale}

By \href{https://www.nytimes3xbfgragh.onion/by/taylor-tepper}{Taylor
Tepper}

\begin{itemize}
\item
  Aug. 3, 2020
\item
  \begin{itemize}
  \item
  \item
  \item
  \item
  \item
  \end{itemize}
\end{itemize}

Budgeting is never an easy task, but keeping tabs on your money has
become especially difficult over the past few months. About 30 million
Americans are
\href{https://www.nytimes3xbfgragh.onion/live/2020/07/23/business/stock-market-today-coronavirus\#coin-shortage-united-states-mint}{collecting
jobless benefits}, and more than a million new people
\href{https://www.dol.gov/ui/data.pdf}{file for unemployment} each week.

In response, the federal government amped up unemployment benefits,
granted loan deferrals and sent out a one-off stimulus jolt that kept
finances afloat until states started to open back up. But Congress most
likely won't extend the
\href{https://www.nytimes3xbfgragh.onion/interactive/2020/07/24/business/economy/600-unemployment-benefits.html}{\$600
unemployment lifeline}, and those affected will need to adjust, once
again, to a harsh financial reality.

Too often we abandon our financial plans, or avoid planning altogether,
because budgets can be complicated and our lives are unpredictable. But
there are simple actions you can take to get your financial budgets in
order. Here's a four-step, totally achievable plan that can help you
attain a sense of control.

\hypertarget{step-1-cancel-something}{%
\subsection{Step 1: Cancel something}\label{step-1-cancel-something}}

Sometimes the task of monitoring and planning your saving and spending
feels too big, and the trick is to break down what you're trying to
achieve into smaller parts before starting with the easiest one.

This strategy, called ``small victories'' or ``snowballing,'' is best
known for dealing with credit card debt. You want to go after the card
with the smallest balance, rather than the lowest interest rate, to
garner positive momentum,
\href{https://www.nber.org/papers/w20125.pdf}{according to research}.

\hypertarget{latest-updates-global-coronavirus-outbreak}{%
\section{\texorpdfstring{\href{https://www.nytimes3xbfgragh.onion/2020/08/04/world/coronavirus-cases.html?action=click\&pgtype=Article\&state=default\&region=MAIN_CONTENT_1\&context=storylines_live_updates}{Latest
Updates: Global Coronavirus
Outbreak}}{Latest Updates: Global Coronavirus Outbreak}}\label{latest-updates-global-coronavirus-outbreak}}

Updated 2020-08-04T21:54:31.017Z

\begin{itemize}
\tightlist
\item
  \href{https://www.nytimes3xbfgragh.onion/2020/08/04/world/coronavirus-cases.html?action=click\&pgtype=Article\&state=default\&region=MAIN_CONTENT_1\&context=storylines_live_updates\#link-2daa96b5}{As
  talks drag on, McConnell signals openness to jobless aid extension
  that Republicans have opposed.}
\item
  \href{https://www.nytimes3xbfgragh.onion/2020/08/04/world/coronavirus-cases.html?action=click\&pgtype=Article\&state=default\&region=MAIN_CONTENT_1\&context=storylines_live_updates\#link-1228a480}{Novavax
  sees encouraging results from two studies of its experimental
  vaccine.}
\item
  \href{https://www.nytimes3xbfgragh.onion/2020/08/04/world/coronavirus-cases.html?action=click\&pgtype=Article\&state=default\&region=MAIN_CONTENT_1\&context=storylines_live_updates\#link-4825b93}{Public
  and private schools in Maryland and elsewhere are divided over
  in-person instruction.}
\end{itemize}

\href{https://www.nytimes3xbfgragh.onion/2020/08/04/world/coronavirus-cases.html?action=click\&pgtype=Article\&state=default\&region=MAIN_CONTENT_1\&context=storylines_live_updates}{See
more updates}

More live coverage:
\href{https://www.nytimes3xbfgragh.onion/live/2020/08/04/business/stock-market-today-coronavirus?action=click\&pgtype=Article\&state=default\&region=MAIN_CONTENT_1\&context=storylines_live_updates}{Markets}

But the principle can be applied to your budget more broadly. To earn a
small win, comb through your credit card and bank statements to find one
small recurring charge you no longer need and cancel it. Consider that
TV streaming service you don't actually watch, for instance, or ax a
travel or airline credit card with a large annual fee. You can even try
calling the bank to downgrade your credit card to the no-fee version.
(Speaking with a real person almost always yields better results than
sending an email or chatting with a bot.)

\hypertarget{step-2-establish-your-spending-number}{%
\subsection{Step 2: Establish your spending
number}\label{step-2-establish-your-spending-number}}

No matter your income, you probably don't enjoy setting a budget. In
fact, just one in three households endeavor to track how much they earn
or spend in a given month, per a 2013
\href{https://news.gallup.com/poll/162872/one-three-americans-prepare-detailed-household-budget.aspx}{Gallup
survey}.

A simple trick is to recast how you think of a budget. Rather than
accounting for every dollar spent and arranging expenditures by category
(dining out, home improvement, etc.), most people would be better off if
they were to set a number they could safely spend each month on
everything, no matter where or how.

Simplifi,
\href{https://www.nytimes3xbfgragh.onion/wirecutter/money/best-budgeting-apps-and-tools/?utm_source=nytimes\&utm_medium=referral\&utm_campaign=wcsmarterliving\&utm_content=covid-budget}{Wirecutter's
favorite budgeting app}, offers exactly such a service with its Spending
Plan. Once you connect your accounts, the app estimates how much you'll
earn in a month and then subtracts how much you owe in recurring costs
(like debt payments or your Netflix account). From there, you enter how
much you want to save, and Simplifi helps calculate a custom amount to
spend for the rest of the month without going into debt. Spending
Watchlists also let you keep track of categories where you tend to
overspend. While the app costs \$3 per month for a subscription (or \$30
for a year), we think its interface and features are worth it for people
looking to take charge of their finances.

\hypertarget{step-3-create-a-reasonable-emergency-fund}{%
\subsection{Step 3: Create a reasonable emergency
fund}\label{step-3-create-a-reasonable-emergency-fund}}

A safe-to-spend number won't work well for everyone, especially if you
don't have a stable, predictable paycheck.

Take the couple identified as Becky and Jeremy Moore (their real names
were changed for privacy's sake), whom the authors Jonathan Morduch and
Rachel Schneider followed for months for their book
``\href{https://www.usfinancialdiaries.org/book}{The Financial
Diaries}.'' Jeremy, an Ohio-based long-haul truck mechanic, earned most
of his salary in the summer and winter months when the weather was bad
and trucks would get beat up. When it was more temperate, he took home
as little as \$300 a week. And the couple is not alone: About half of
households, according to a
\href{https://www.pewtrusts.org/~/media/Assets/2015/01/FSM_Balance_Sheet_Report.pdf}{2015
Pew Charitable Trust study}, see at least a 25 percent gain or loss of
income from one year to the next.

Financial planners would say the Moores should simply dip into their
emergency fund (a pot of money consisting of six months' worth of
essential expenses in a savings account) when times get tight. But this
commitment can run into the tens of thousands and is a nearly impossible
feat for families with volatile earnings, and difficult even for higher
earning households.

If you can't save up for an emergency fund, try just saving a regular
percentage of your paycheck (or any windfall, like a tax refund) in a
checking or savings account to help you smooth over income spikes and
dips throughout the year.

\href{https://www.nytimes3xbfgragh.onion/news-event/coronavirus?action=click\&pgtype=Article\&state=default\&region=MAIN_CONTENT_3\&context=storylines_faq}{}

\hypertarget{the-coronavirus-outbreak-}{%
\subsubsection{The Coronavirus Outbreak
›}\label{the-coronavirus-outbreak-}}

\hypertarget{frequently-asked-questions}{%
\paragraph{Frequently Asked
Questions}\label{frequently-asked-questions}}

Updated August 4, 2020

\begin{itemize}
\item ~
  \hypertarget{i-have-antibodies-am-i-now-immune}{%
  \paragraph{I have antibodies. Am I now
  immune?}\label{i-have-antibodies-am-i-now-immune}}

  \begin{itemize}
  \tightlist
  \item
    As of right
    now,\href{https://www.nytimes3xbfgragh.onion/2020/07/22/health/covid-antibodies-herd-immunity.html?action=click\&pgtype=Article\&state=default\&region=MAIN_CONTENT_3\&context=storylines_faq}{that
    seems likely, for at least several months.} There have been
    frightening accounts of people suffering what seems to be a second
    bout of Covid-19. But experts say these patients may have a
    drawn-out course of infection, with the virus taking a slow toll
    weeks to months after initial exposure. People infected with the
    coronavirus typically
    \href{https://www.nature.com/articles/s41586-020-2456-9}{produce}
    immune molecules called antibodies, which are
    \href{https://www.nytimes3xbfgragh.onion/2020/05/07/health/coronavirus-antibody-prevalence.html?action=click\&pgtype=Article\&state=default\&region=MAIN_CONTENT_3\&context=storylines_faq}{protective
    proteins made in response to an
    infection}\href{https://www.nytimes3xbfgragh.onion/2020/05/07/health/coronavirus-antibody-prevalence.html?action=click\&pgtype=Article\&state=default\&region=MAIN_CONTENT_3\&context=storylines_faq}{.
    These antibodies may} last in the body
    \href{https://www.nature.com/articles/s41591-020-0965-6}{only two to
    three months}, which may seem worrisome, but that's perfectly normal
    after an acute infection subsides, said Dr. Michael Mina, an
    immunologist at Harvard University. It may be possible to get the
    coronavirus again, but it's highly unlikely that it would be
    possible in a short window of time from initial infection or make
    people sicker the second time.
  \end{itemize}
\item ~
  \hypertarget{im-a-small-business-owner-can-i-get-relief}{%
  \paragraph{I'm a small-business owner. Can I get
  relief?}\label{im-a-small-business-owner-can-i-get-relief}}

  \begin{itemize}
  \tightlist
  \item
    The
    \href{https://www.nytimes3xbfgragh.onion/article/small-business-loans-stimulus-grants-freelancers-coronavirus.html?action=click\&pgtype=Article\&state=default\&region=MAIN_CONTENT_3\&context=storylines_faq}{stimulus
    bills enacted in March} offer help for the millions of American
    small businesses. Those eligible for aid are businesses and
    nonprofit organizations with fewer than 500 workers, including sole
    proprietorships, independent contractors and freelancers. Some
    larger companies in some industries are also eligible. The help
    being offered, which is being managed by the Small Business
    Administration, includes the Paycheck Protection Program and the
    Economic Injury Disaster Loan program. But lots of folks have
    \href{https://www.nytimes3xbfgragh.onion/interactive/2020/05/07/business/small-business-loans-coronavirus.html?action=click\&pgtype=Article\&state=default\&region=MAIN_CONTENT_3\&context=storylines_faq}{not
    yet seen payouts.} Even those who have received help are confused:
    The rules are draconian, and some are stuck sitting on
    \href{https://www.nytimes3xbfgragh.onion/2020/05/02/business/economy/loans-coronavirus-small-business.html?action=click\&pgtype=Article\&state=default\&region=MAIN_CONTENT_3\&context=storylines_faq}{money
    they don't know how to use.} Many small-business owners are getting
    less than they expected or
    \href{https://www.nytimes3xbfgragh.onion/2020/06/10/business/Small-business-loans-ppp.html?action=click\&pgtype=Article\&state=default\&region=MAIN_CONTENT_3\&context=storylines_faq}{not
    hearing anything at all.}
  \end{itemize}
\item ~
  \hypertarget{what-are-my-rights-if-i-am-worried-about-going-back-to-work}{%
  \paragraph{What are my rights if I am worried about going back to
  work?}\label{what-are-my-rights-if-i-am-worried-about-going-back-to-work}}

  \begin{itemize}
  \tightlist
  \item
    Employers have to provide
    \href{https://www.osha.gov/SLTC/covid-19/standards.html}{a safe
    workplace} with policies that protect everyone equally.
    \href{https://www.nytimes3xbfgragh.onion/article/coronavirus-money-unemployment.html?action=click\&pgtype=Article\&state=default\&region=MAIN_CONTENT_3\&context=storylines_faq}{And
    if one of your co-workers tests positive for the coronavirus, the
    C.D.C.} has said that
    \href{https://www.cdc.gov/coronavirus/2019-ncov/community/guidance-business-response.html}{employers
    should tell their employees} -\/- without giving you the sick
    employee's name -\/- that they may have been exposed to the virus.
  \end{itemize}
\item ~
  \hypertarget{should-i-refinance-my-mortgage}{%
  \paragraph{Should I refinance my
  mortgage?}\label{should-i-refinance-my-mortgage}}

  \begin{itemize}
  \tightlist
  \item
    \href{https://www.nytimes3xbfgragh.onion/article/coronavirus-money-unemployment.html?action=click\&pgtype=Article\&state=default\&region=MAIN_CONTENT_3\&context=storylines_faq}{It
    could be a good idea,} because mortgage rates have
    \href{https://www.nytimes3xbfgragh.onion/2020/07/16/business/mortgage-rates-below-3-percent.html?action=click\&pgtype=Article\&state=default\&region=MAIN_CONTENT_3\&context=storylines_faq}{never
    been lower.} Refinancing requests have pushed mortgage applications
    to some of the highest levels since 2008, so be prepared to get in
    line. But defaults are also up, so if you're thinking about buying a
    home, be aware that some lenders have tightened their standards.
  \end{itemize}
\item ~
  \hypertarget{what-is-school-going-to-look-like-in-september}{%
  \paragraph{What is school going to look like in
  September?}\label{what-is-school-going-to-look-like-in-september}}

  \begin{itemize}
  \tightlist
  \item
    It is unlikely that many schools will return to a normal schedule
    this fall, requiring the grind of
    \href{https://www.nytimes3xbfgragh.onion/2020/06/05/us/coronavirus-education-lost-learning.html?action=click\&pgtype=Article\&state=default\&region=MAIN_CONTENT_3\&context=storylines_faq}{online
    learning},
    \href{https://www.nytimes3xbfgragh.onion/2020/05/29/us/coronavirus-child-care-centers.html?action=click\&pgtype=Article\&state=default\&region=MAIN_CONTENT_3\&context=storylines_faq}{makeshift
    child care} and
    \href{https://www.nytimes3xbfgragh.onion/2020/06/03/business/economy/coronavirus-working-women.html?action=click\&pgtype=Article\&state=default\&region=MAIN_CONTENT_3\&context=storylines_faq}{stunted
    workdays} to continue. California's two largest public school
    districts --- Los Angeles and San Diego --- said on July 13, that
    \href{https://www.nytimes3xbfgragh.onion/2020/07/13/us/lausd-san-diego-school-reopening.html?action=click\&pgtype=Article\&state=default\&region=MAIN_CONTENT_3\&context=storylines_faq}{instruction
    will be remote-only in the fall}, citing concerns that surging
    coronavirus infections in their areas pose too dire a risk for
    students and teachers. Together, the two districts enroll some
    825,000 students. They are the largest in the country so far to
    abandon plans for even a partial physical return to classrooms when
    they reopen in August. For other districts, the solution won't be an
    all-or-nothing approach.
    \href{https://bioethics.jhu.edu/research-and-outreach/projects/eschool-initiative/school-policy-tracker/}{Many
    systems}, including the nation's largest, New York City, are
    devising
    \href{https://www.nytimes3xbfgragh.onion/2020/06/26/us/coronavirus-schools-reopen-fall.html?action=click\&pgtype=Article\&state=default\&region=MAIN_CONTENT_3\&context=storylines_faq}{hybrid
    plans} that involve spending some days in classrooms and other days
    online. There's no national policy on this yet, so check with your
    municipal school system regularly to see what is happening in your
    community.
  \end{itemize}
\end{itemize}

This type of emergency fund is easier to achieve than the traditional
six-month fund for a few reasons. You don't need to save as much money,
because the goal is to level off temporary income dips rather than cover
half a year of necessary spending.

And according to Sharif Muhammad, a New Jersey-based certified financial
planner, since you're saving for something tangible (lower-earning
months that you know will arrive), rather than an undefined, unknown
terrible event, you may find motivation easier to come by.

\hypertarget{step-4-save-the-money-that-youre-not-spending}{%
\subsection{Step 4: Save the money that you're not
spending}\label{step-4-save-the-money-that-youre-not-spending}}

With vacations canceled, bars closed and gyms out of the question,
you're probably spending less money than you were pre-Covid. Most people
have pared back their swiping.
\href{https://www.businessinsider.com/economic-outlook-credit-card-spending-data-recovery-stalling-recession-jpmorgan-2020-7}{Chase
credit card customers}, for instance, are charging about 10 percent less
than they did a year ago. Fewer available ways to spend money can
ultimately be a good thing. Nearly
\href{https://www.comparecards.com/blog/credit-card-confidence-index/}{70
percent of credit card} borrowers, according to CompareCards.com, are
totally confident they'll pay off their balance in full this month, an
almost 10 percentage point increase from this time last year.

Of course, skipping all the ways you used to spend money --- on flights,
boozy dinners or nights at the movies --- only helps your bottom line if
you continue to spend prudently. Sure, you're saving money by not going
out, but use that money on groceries rather than shelling out \$50 on
Wednesday night Grubhub. You might not be going on vacation this summer,
but stash that money away rather than blow it on a luxury purchase.

With nowhere to go and no one to see, spending less and saving more ---
always an aspiration --- now feels entirely possible.

\emph{Interested in learning more about the best things to buy and how
to use them? Visit}
\href{https://www.nytimes3xbfgragh.onion/wirecutter/?utm_source=nytimes\&utm_medium=referral\&utm_campaign=wcsmarterliving\&utm_content=covid-budget}{\emph{Wirecutter}}
\emph{where you can read the latest}
\href{https://www.nytimes3xbfgragh.onion/wirecutter/reviews?utm_source=nytimes\&utm_medium=referral\&utm_campaign=wcsmarterliving\&utm_content=covid-budget}{\emph{reviews}}
\emph{and find}
\href{https://www.nytimes3xbfgragh.onion/wirecutter/deals/?utm_source=nytimes\&utm_medium=referral\&utm_campaign=wcsmarterliving\&utm_content=covid-budget}{\emph{daily
deals}}\emph{.}

Advertisement

\protect\hyperlink{after-bottom}{Continue reading the main story}

\hypertarget{site-index}{%
\subsection{Site Index}\label{site-index}}

\hypertarget{site-information-navigation}{%
\subsection{Site Information
Navigation}\label{site-information-navigation}}

\begin{itemize}
\tightlist
\item
  \href{https://help.nytimes3xbfgragh.onion/hc/en-us/articles/115014792127-Copyright-notice}{©~2020~The
  New York Times Company}
\end{itemize}

\begin{itemize}
\tightlist
\item
  \href{https://www.nytco.com/}{NYTCo}
\item
  \href{https://help.nytimes3xbfgragh.onion/hc/en-us/articles/115015385887-Contact-Us}{Contact
  Us}
\item
  \href{https://www.nytco.com/careers/}{Work with us}
\item
  \href{https://nytmediakit.com/}{Advertise}
\item
  \href{http://www.tbrandstudio.com/}{T Brand Studio}
\item
  \href{https://www.nytimes3xbfgragh.onion/privacy/cookie-policy\#how-do-i-manage-trackers}{Your
  Ad Choices}
\item
  \href{https://www.nytimes3xbfgragh.onion/privacy}{Privacy}
\item
  \href{https://help.nytimes3xbfgragh.onion/hc/en-us/articles/115014893428-Terms-of-service}{Terms
  of Service}
\item
  \href{https://help.nytimes3xbfgragh.onion/hc/en-us/articles/115014893968-Terms-of-sale}{Terms
  of Sale}
\item
  \href{https://spiderbites.nytimes3xbfgragh.onion}{Site Map}
\item
  \href{https://help.nytimes3xbfgragh.onion/hc/en-us}{Help}
\item
  \href{https://www.nytimes3xbfgragh.onion/subscription?campaignId=37WXW}{Subscriptions}
\end{itemize}
