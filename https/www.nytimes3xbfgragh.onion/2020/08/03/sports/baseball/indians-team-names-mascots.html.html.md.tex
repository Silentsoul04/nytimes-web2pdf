Sections

SEARCH

\protect\hyperlink{site-content}{Skip to
content}\protect\hyperlink{site-index}{Skip to site index}

\href{/section/sports/baseball}{Baseball}\textbar{}Why One Team Named
the Indians Won't Be Changing Its Name

\url{https://nyti.ms/30nLOHw}

\begin{itemize}
\item
\item
\item
\item
\item
\item
\end{itemize}

\includegraphics{https://static01.graylady3jvrrxbe.onion/images/2020/08/04/sports/04Mascot-print-1/31mascot-spokane2-print-articleLarge-v3.jpg?quality=75\&auto=webp\&disable=upscale}

\hypertarget{why-one-team-named-the-indians-wont-be-changing-its-name}{%
\section{Why One Team Named the Indians Won't Be Changing Its
Name}\label{why-one-team-named-the-indians-wont-be-changing-its-name}}

A minor league team in Spokane, Wash., has steadfastly stood by its
nickname with the support of the local Native American community.

Credit...Ruth Fremson/The New York Times

Supported by

\protect\hyperlink{after-sponsor}{Continue reading the main story}

\href{https://www.nytimes3xbfgragh.onion/by/david-waldstein}{\includegraphics{https://static01.graylady3jvrrxbe.onion/images/2018/02/20/multimedia/author-david-waldstein/author-david-waldstein-thumbLarge.jpg}}

By \href{https://www.nytimes3xbfgragh.onion/by/david-waldstein}{David
Waldstein}

\begin{itemize}
\item
  Aug. 3, 2020
\item
  \begin{itemize}
  \item
  \item
  \item
  \item
  \item
  \item
  \end{itemize}
\end{itemize}

Many sports teams using names and mascots invoking Native Americans do
so over the longstanding and
\href{https://www.nytimes3xbfgragh.onion/2020/07/10/sports/football/washington-redskins-name-change-mascots.html}{strenuous
objections} of people who say it is racist. Some teams, after years of
stubborn refusal, have
\href{https://www.nytimes3xbfgragh.onion/2020/07/23/sports/football/washington-football-team-name-logo.html}{recently
relented}, like the Washington Football Team of the N.F.L., which
abandoned its nickname last month.

Then there are the Spokane Indians.

The minor league team in Washington State has been collaborating with
the Spokane Tribe of Indians in what it hopes is a respectful manner of
honoring the local Indigenous population.

Can that be done? Some say it is not possible, but the Spokane Indians
may be as close to an understanding as any team has come.

``They came and listened to the elders, and that is what really
developed the relationship over time,'' said Carol Evans, the chairwoman
of the Spokane Tribal Council, ``and it has grown like a family
partnership unit, where we have a lot of respect for one another.''

\includegraphics{https://static01.graylady3jvrrxbe.onion/images/2020/08/04/sports/04Mascot-print-2/merlin_175141332_2c17ed4b-1f11-436a-a9a3-242b0b6f391a-articleLarge.jpg?quality=75\&auto=webp\&disable=upscale}

But Suzan Shown Harjo, an advocate for Native American rights who has
led the fight against Indigenous team names and mascots in sports for
decades, said no matter the good intent, the name should still be
changed.

``There is no such thing as respectable treatment of any mascot or team
name that has a native theme in sports,'' she said. ``There is just no
such thing, no matter how you package it.''

The Cleveland Indians of Major League Baseball said they were
\href{https://www.nytimes3xbfgragh.onion/2020/07/03/sports/baseball/cleveland-indians-name-change.html}{having
discussions} about the ``best path forward'' regarding their name.

The Spokane Indians have had such talks, too, even volunteering to
abandon the name 14 years ago but eventually deciding to keep it with
the support of tribal leadership.

The Spokane Indians were founded in 1903 and are now a Class A affiliate
of the Texas Rangers (an awkward and painful historical connection
because the original Texas Rangers, a law enforcement division, were
known to hunt down Native Americans). Decades ago, the Spokane team logo
featured a grotesque caricature of a Native person. But there are no
longer any such depictions associated with the team.

Since 2006, the team has actively engaged with the Spokane Tribe, many
of whom live on a reservation about 40 miles from the city of Spokane.
Meetings are held with government leaders at least once a year and the
team has made several changes and innovations to their uniform design,
stadium exhibits and cultural outreach programs, based on
recommendations from the Spokane leaders.

Image

The word ``Spokane'' is written in the Salish language on Spokane Indian
team jerseys.Credit...Tyler Tjomsland/The Spokesman-Review, via
Associated Press

The current uniform has Sp'q'n'i' emblazoned on the front. It is the
spelling in Salish, the local Native American language, for Spokane,
which is pronounced Spo-ka-NEE, according to Evans. One of the jerseys
hangs at the Baseball Hall of Fame in Cooperstown, N.Y., and is said to
be the first example of Native American language on a professional
baseball uniform.

The team adopted a mascot dressed as a trout, a traditional food source
of the Spokane people, in part to raise awareness for redband trout
conservation in the area. The logo includes a feather inspired by the
art of a member of the Spokane community and one version has Salish
words on it. Some signs in the stadium, like for the team store, the
concession stand and the restrooms, are in both English and Salish.

All of it was done in consultation with the Spokane people, said Evans
and Otto Klein, a senior vice president and part owner of the team.

Image

The Salish language is used throughout the Spokane Indians
ballpark.Credit...Ruth Fremson/The New York Times

``In the early conversations, we had everything on the table, including
a name change,'' Klein said. ``The partnership you see today is where it
ended up, and we are very proud of it from our side.''

Klein said the team has joined in an effort to restock the local rivers
with salmon, the Spokane Tribe's historical food source until the
construction of dams in the 20th century cut off the supply.

With the Spokane team, the straightforward narratives that often apply
to teams with Native nicknames are upended by seeming paradoxes and
nuance. While the team has endeavored to erase all Native American
imagery, one image still exists on a scoreboard, but it is an
advertisement for the Spokane Tribe of Indians, and it depicts a
traditional member of the Spokane nation in headdress.

On the reservation, the nickname of the public Wellpinit High School is
Redskins, arguably the most offensive nickname used by sports teams.
(Evans hopes the younger generations will one day change the name, which
has often gotten a
\href{https://www.spokesman.com/stories/2014/jul/02/for-wellpinit-students-redskins-a-source-of-pride/}{shrug}
from members of the community.)

John Schleppi, a sports history professor at the University of Dayton in
Ohio, has researched the team's unique relationship with the Spokane
people. He said he may never have discovered the rich history and
traditions of the Spokane people if it were not for a visit to the
stadium in 2006.

Near the entrance of the stadium, a series of exhibits showcased the
culture and history of the Spokane nation, the people who had been
pushed off the very land on which the stadium sits, more than a century
ago.

Image

Cultural information is displayed outside the ballpark, Avista
Stadium.~Credit...Ruth Fremson/The New York Times

``I learned about the various ceremonies and what certain practices
meant,'' he said from his home in Dayton. ``I remember there was a
display on lifestyles and fishing methods and one on language and
culture.''

Klein said the team, which is not playing this year after the minor
league season was canceled because of the pandemic, also helps finance
charitable efforts on the reservation, including a fund for children,
and they are helping to rebuild the local baseball field.

But Harjo said donations from teams to local Native groups could be used
to induce the endorsement of local groups. She pointed to contributions
made by Florida State University to the Seminole people (Florida State's
teams are known as the Seminoles).

``It always makes me sad to hear that Native people, especially tribal
leaders, have been sold a bill of goods when it comes to stereotypes,''
she said. ``There really is no such thing as a good stereotype.''

Stephanie Fryberg, a professor of psychology at the University of
Michigan, is a member of the Tulalip Nation in Western Washington State,
a group that is part of the Coastal Salish people. Her research has
shown that an overwhelming majority of Native Americans who are engaged
in cultural practices are offended by Native sports teams names and
logos.

She said that as a scientist, she would need more data to fully evaluate
the Spokane situation --- including surveys of people's feelings and
opinions --- but on the surface, she said there appears to be a
respectful approach by the team.

``This seems to be a different story,'' Fryberg said, contrasting the
Spokane situation with many other more contentious team names and
mascots. ``I would still like to change the name, but I think there is a
place for specific Native names. The goal isn't to get rid of them
completely, but to use them appropriately. You can't use a mascot
appropriately.''

Even if the team and its fans demonstrate proper respect, she said, fans
of opposing teams might feel emboldened to hurl racist language and
gestures against them.

But to add to the collegial relationship between the team and the
Spokane people, she called on the team to apologize for earlier use of
the racist logos, even though they were abandoned decades before the
current group owned the team. Evans, the chairwoman of the Spokane
council, agreed that would be a helpful gesture.

Evans also used to favor a name change, but her nation's leadership
persuaded her otherwise.

``We want them to keep the name because of how they approached us,'' she
said. ``They listened to the elders, and that is what really developed
the relationship over time and it has grown into like a family
partnership unit where we have a lot of respect for one another.''

Advertisement

\protect\hyperlink{after-bottom}{Continue reading the main story}

\hypertarget{site-index}{%
\subsection{Site Index}\label{site-index}}

\hypertarget{site-information-navigation}{%
\subsection{Site Information
Navigation}\label{site-information-navigation}}

\begin{itemize}
\tightlist
\item
  \href{https://help.nytimes3xbfgragh.onion/hc/en-us/articles/115014792127-Copyright-notice}{©~2020~The
  New York Times Company}
\end{itemize}

\begin{itemize}
\tightlist
\item
  \href{https://www.nytco.com/}{NYTCo}
\item
  \href{https://help.nytimes3xbfgragh.onion/hc/en-us/articles/115015385887-Contact-Us}{Contact
  Us}
\item
  \href{https://www.nytco.com/careers/}{Work with us}
\item
  \href{https://nytmediakit.com/}{Advertise}
\item
  \href{http://www.tbrandstudio.com/}{T Brand Studio}
\item
  \href{https://www.nytimes3xbfgragh.onion/privacy/cookie-policy\#how-do-i-manage-trackers}{Your
  Ad Choices}
\item
  \href{https://www.nytimes3xbfgragh.onion/privacy}{Privacy}
\item
  \href{https://help.nytimes3xbfgragh.onion/hc/en-us/articles/115014893428-Terms-of-service}{Terms
  of Service}
\item
  \href{https://help.nytimes3xbfgragh.onion/hc/en-us/articles/115014893968-Terms-of-sale}{Terms
  of Sale}
\item
  \href{https://spiderbites.nytimes3xbfgragh.onion}{Site Map}
\item
  \href{https://help.nytimes3xbfgragh.onion/hc/en-us}{Help}
\item
  \href{https://www.nytimes3xbfgragh.onion/subscription?campaignId=37WXW}{Subscriptions}
\end{itemize}
