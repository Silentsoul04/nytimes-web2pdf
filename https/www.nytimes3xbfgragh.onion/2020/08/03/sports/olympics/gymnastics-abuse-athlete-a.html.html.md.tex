Sections

SEARCH

\protect\hyperlink{site-content}{Skip to
content}\protect\hyperlink{site-index}{Skip to site index}

\href{https://www.nytimes3xbfgragh.onion/section/sports/olympics}{Olympics}

\href{https://myaccount.nytimes3xbfgragh.onion/auth/login?response_type=cookie\&client_id=vi}{}

\href{https://www.nytimes3xbfgragh.onion/section/todayspaper}{Today's
Paper}

\href{/section/sports/olympics}{Olympics}\textbar{}Gymnasts Worldwide
Push Back on Their Sport's Culture of Abuse

\url{https://nyti.ms/3k5NYmX}

\begin{itemize}
\item
\item
\item
\item
\item
\item
\end{itemize}

Advertisement

\protect\hyperlink{after-top}{Continue reading the main story}

Supported by

\protect\hyperlink{after-sponsor}{Continue reading the main story}

\hypertarget{gymnasts-worldwide-push-back-on-their-sports-culture-of-abuse}{%
\section{Gymnasts Worldwide Push Back on Their Sport's Culture of
Abuse}\label{gymnasts-worldwide-push-back-on-their-sports-culture-of-abuse}}

On Instagram and other social networks, gymnasts have tagged posts with
\#GymnastAlliance to share their own experiences in the wake of a new
documentary that highlights verbal and physical abuse by coaches.

\includegraphics{https://static01.graylady3jvrrxbe.onion/images/2020/08/03/sports/03gymnastics1-print/03gymnastics1-print-articleLarge-v2.jpg?quality=75\&auto=webp\&disable=upscale}

By \href{https://www.nytimes3xbfgragh.onion/by/juliet-macur}{Juliet
Macur}

\begin{itemize}
\item
  Aug. 3, 2020
\item
  \begin{itemize}
  \item
  \item
  \item
  \item
  \item
  \item
  \end{itemize}
\end{itemize}

A culture in gymnastics that has tolerated coaches belittling,
manipulating and in some cases physically abusing young athletes is
being challenged by Olympians and other gymnasts around the world after
an uprising in the United States.

Many current and former competitors, emboldened by their American peers,
have broken their silence in recent weeks against treatment they say
created mental scars on girls that lasted well into adulthood.

\href{https://www.itv.com/news/2020-07-07/girl-8-describes-bullying-by-gymnastics-coaches-as-review-is-pledged}{One
gymnast, who is just 8 years old}, said a coach tied her wrists to a
horizontal bar when she was 7 and ignored her as she cried out in pain.

At a time when the Tokyo Olympics would be in session, had they not been
\href{https://www.nytimes3xbfgragh.onion/2020/03/24/sports/olympics/coronavirus-summer-olympics-postponed.html}{postponed
until 2021} by
\href{https://www.nytimes3xbfgragh.onion/news-event/coronavirus}{the
coronavirus pandemic}, gymnasts have been sharing horrific stories of
coaches body-shaming them, stifling their emotions, using corporal
punishment on them and forcing them to train with injuries, using the
pursuit of medals as a way to rationalize shameful behavior.

Chloe Gilliland, 29, a former member of the Australian national team,
recalled her coaches telling her that
\href{https://7news.com.au/sport/gymnastics/how-fat-shaming-drove-aussie-gymnast-to-breaking-point-c-1186590}{she
was ``a bad child'' and ``a danger''} to her own body because she was
too heavy. At 17, she thought of killing herself because, she said on
Instagram, ``I felt like it was easier to end my own life than to give
in to what they wanted me to be.''

Catherine Lyons, 19, once a top junior competitor for Britain, said
coaches would hit her and harass her about her weight, and when she was
7 or 8 she would cry so hard that coaches would shut her inside a
cupboard until she composed herself. Later, she said, she learned she
had post-traumatic stress disorder because of the treatment.
\href{https://www.itv.com/news/2020-07-22/truth-gives-power-to-abused-gymnasts}{She
told ITV News}: ``I wasn't worth anything. I wasn't a human. I was a
commodity rather than a child.''

Other gymnasts have simply said on social media, ``I am one of them.''

The stories from gymnasts in all levels of the sport are part of a
coordinated effort, similar to the \#MeToo movement, calling for the
sport's leaders to eradicate existing norms that in reality are not
normal at all.

``I was told many times that gymnasts should be seen and not heard
because the sport is all about being the good little gymnast,'' said
Lisa Mason, a 2000 British Olympian who was among the gymnasts to speak
out recently.

Mason, 38, said her coaches threw shoes at her and scratched her when
she did not perform perfectly. Once she was made to stay on the uneven
bars until her hands were blistered and ripped, only for a coach to pin
her hands down and pour rubbing alcohol into her raw wounds, she said
Friday in an interview.

Looking back, Mason called the atmosphere ``insane,'' especially because
gymnasts who start training seriously at a very young age are often left
alone with coaches.

``So many of us are done with normalizing the abuse that we were told
was needed to make champions,'' Mason said. ``We want change, and it's
incredible that so many of us are coming together to demand it.''

The gymnasts have made an impact. National gymnastics federations in
Britain, Australia, the Netherlands and Belgium have begun
investigations or requested inquiries into alleged abuse, with the Dutch
federation saying it would suspend its entire women's national team
program as it looked for answers.

\includegraphics{https://static01.graylady3jvrrxbe.onion/images/2020/08/03/sports/03JPgymnastics1-print/merlin_175172106_0adf4067-f9ec-4d44-9036-0d0be1d75e89-articleLarge.jpg?quality=75\&auto=webp\&disable=upscale}

\href{https://www.telegraph.co.uk/gymnastics/2020/07/08/culture-fear-permeating-whole-gymnastics-has-stop/}{Jennifer
Pinches, a 2012 British Olympian}, said on Friday that this most recent
global push to change the sport was sparked by
\href{https://www.nytimes3xbfgragh.onion/2020/06/24/movies/athlete-a-review.html}{a
Netflix documentary, ``Athlete A,''} which depicts the harrowing
training of American elite gymnasts and the cover-ups that surrounded
\href{https://www.nytimes3xbfgragh.onion/2020/03/03/sports/olympics/biles-gymnastics-settlement-nassar.html?searchResultPosition=3}{U.S.A.
Gymnastics's sexual abuse scandal} involving
\href{https://www.nytimes3xbfgragh.onion/video/us/100000005698488/larry-nassar-gymnastics-sex-abuse.html?searchResultPosition=7}{its
longtime doctor, Lawrence G. Nassar}.

``It was a tipping point that enabled this movement to happen,'' said
Pinches, who saw the film when it was released in late June and
persuaded more than 30 British gymnasts, including Mason, to post
\href{https://www.instagram.com/p/CCBs9R1DPFh/}{a joint statement} on
social media condemning ``the culture that didn't put athlete health and
well-being first and allowed Nassar to act.'' Mason, who had already
posted a scathing Instagram commentary on the sport's culture, suggested
that the group use the hashtag \#GymnastAlliance. Now, there have been
more than 700 posts with related tags on Instagram alone.

Back in the United States,
\href{https://www.nytimes3xbfgragh.onion/2017/03/30/opinion/sunday/how-gymnastics-culture-breeds-sexual-abuse.html}{Jennifer
Sey, a 1986 national champion} and one of the producers of
\href{https://www.rogerebert.com/reviews/athlete-a-movie-review-2020}{``Athlete
A,''} was stunned that the film prompted so many athletes to come
forward about their experiences, especially when athletes had so many
previous chances to speak up.

Over the past 25 years, there have been several books about abuse in the
sport,
\href{https://www.npr.org/templates/story/story.php?storyId=90105904}{including
one by Sey in 2008}. There also were
\href{https://www.nytimes3xbfgragh.onion/video/sports/100000005686677/gymnasts-confront-larry-nassar-over-sexual-abuse.html?searchResultPosition=8}{Nassar's
sentencing hearings in 2018}, during which
\href{https://www.nytimes3xbfgragh.onion/2018/01/24/sports/rachael-denhollander-nassar-gymnastics.html?searchResultPosition=28}{more
than 150 girls and women spoke out} about his molesting them. Many said
the sport's culture of fear and silence made them vulnerable to Nassar's
abuse.

In April, Maggie Haney, who coached the Olympian Laurie Hernandez and
other gymnasts,
\href{https://www.nytimes3xbfgragh.onion/2020/05/01/sports/maggie-haney-gymnastics-abuse.html}{was
suspended for eight years by U.S.A. Gymnastics} for verbally abusing and
mistreating athletes. It was among the first times --- if not the first
--- that
\href{https://www.nytimes3xbfgragh.onion/2020/04/29/sports/gymnastics-coach-banned-maggie-haney.html}{a
top American coach had been punished} for that kind of abuse.

Yet gymnasts saw this moment in 2020 as a perfect time to bring
worldwide change to nonsexual abuse in the sport.

``This time, you can't say the accusations are against just one bad
apple, one bad coach or one bad system, and then dismiss them,'' Sey
said. ``I think a lot of these women who came forward continued to
suffer because of their coaches' cruel treatment, and they don't want to
suffer anymore. They don't want future generations to suffer, either.''

Image

Jennifer Sey, at age 16, competing during the McDonald's 1985 Gymnastics
Challenge. Sey said she was surprised so many athletes came forward
after the film ``Athlete A'' because there had been other opportunities
to speak out.Credit...Doug Pizac/Associated Press

The timing makes sense to Cheryl Cooky, an associate professor at Purdue
and sociologist who studies gender and sports.

``People have been more reflective of their lives during this pandemic,
and for athletes, it's not a time for the Olympics or a time to just
focus solely on your sport,'' she said. ``They can take time to step
back and think about some important philosophical or existential
questions.''

In speaking up, some athletes have been forced to process some dark
memories.

``Athlete A'' prompted Olivia Vivian,
\href{https://www.youtube.com/watch?v=1WZPvwQf9AU}{an Australian star}
of the reality competition ``Ninja Warrior'' and a 2008 Olympian, to
surface troubling memories of her gymnastics career that, according to
an Instagram post, she had packed away ``deep, deep down.''

It was the first time she thought about the culture of fear in
gymnastics, she said Sunday in a telephone interview, and she was
overwhelmed because she always considered the mistreatment to be normal.
She said her coaches often yelled and threatened her, saying the gym
would lose funding if she didn't make the world team. She recalled them
using her as an example to younger girls of what it takes to be a hard
worker, bringing them to her so they could see her torn-up, bleeding
hands. They would say, ``You need to push through like Olivia pushes
through,'' she said.

The pressure and stress broke Vivian after the 2008 Olympics, she said,
and it took the positivity of gymnastics at Oregon State and the ``Ninja
Warrior'' community to build her back up. Still, since ``Athlete A,''
Vivian has turned to a psychologist to work through her sadness and
guilt over not speaking up sooner so she could have saved other
gymnasts.

``It's going to take a bit of time for me to work through this,'' Vivian
said. ``But I'm glad I told my story for the sake of the sport and maybe
give other gymnasts the strength to stand up.''

One person familiar with the abuse gymnasts like Vivian have described
is Gerrit Beltman --- because he used to coach that way. In 2013,
Beltman was the subject of a book that accused him of tyrannical
behavior while coaching the Dutch national team. He has also coached
elite teams in Canada, Belgium and Singapore.

But in late June, as part of
\href{https://www.noordhollandsdagblad.nl/cnt/dmf20200724_98276713/turncoach-gerrit-beltman-verbreekt-stilzwijgen-over-zijn-onmenselijke-trainingsmethoden-ik-mishandelde-en-vernederde-jonge-turnsters-om-medailles-te-w?utm_source=google\&utm_medium=organic}{an
investigation} by the Dutch newspaper Noordhollands Dagblad, Beltman
apologized for his actions and said he was ``ashamed.'' Ten current and
former Dutch national team gymnasts
\href{https://nltimes.nl/2020/07/25/top-gymnastics-coach-acknowledges-history-abuse-mistreatment}{came
forward with accusations} of abuse in that report.

``It was never my conscious intention to beat them, to yell at them, to
hurt their feelings, to belittle them, to gag them or make constant
derogatory remarks about their weight,'' Beltman said in Dutch. ``But it
did happen. I went too far because I thought it was the only way to
instill a winning mentality in them.''

He said he copied techniques from colleagues and previous coaches who
had delivered many champions. When coaching in Canada, he said in the
newspaper report, he realized that his style needed to change.

Now Beltman says a shift in the sport is ``a matter of bitter urgency,''
and his daughter Reina agrees. She was an alternate on the Dutch Olympic
team for the 2016 Rio Games and coached by her father when she was
young. She was among many Dutch gymnasts who posted on social media
about past abuse, later explaining in an interview that she has
struggled with low self-confidence as a result. When asked about her
father's coaching, she declined to comment.

Reina Beltman did say, however, that her father's decision to speak
about his past means there is hope for the sport.

``The best thing that could happen right now is for coaches to be honest
and apologize for what they've done, so the gymnasts and the sport can
move on,'' she said. ``We just have to have this conversation and, of
course, changes won't be made over one night, but it's a beautiful thing
to take responsibility and try to make things better.''

The next step, said Mason, the British Olympian, is for the sport to
create a way for athletes to report abuse to an independent organization
without fear of retribution by their federation or coaches, who have the
power to keep them off top teams, including a national or Olympic team.
Many athletes who recently came forward told Mason they did not trust
leaders in the sport to investigate coaches.

It is no wonder they are skeptical: Some athletes and leaders still
doubt that emotional abuse is real.

Liubov Charkashyna, a former Belarusian rhythmic gymnast, is the
president of the athletes' commission at the International Gymnastics
Federation, the world's governing body for the sport. One of her jobs is
to listen to athletes' concerns and fight for athletes' rights inside
the federation. But
\href{http://eng.gymnovosti.com/charkashyna-on-sexual-abuse-in-sports-its-america-they-know-how-to-make-money/}{in
2019}, she said the number of women who said they were abused by Nassar
was exaggerated because some women were just trying to get revenge on
their former coaches or earn money.

And less than two weeks ago, the former gymnast Svetlana Khorkina of
Russia, a three-time Olympian and 20-time medal winner at the world
championships whose voice is still powerful in the sport,
\href{https://www.sport-express.ru/artistic-gymnastics/reviews/chto-dumaet-olimpiyskaya-chempionka-svetlana-horkina-o-harrasmente-v-sportivnoy-gimnastike-1694484/}{told
Sport Express} that the recent wave of abuse allegations is silly.

``There are certain things in the profession that are necessary for
success,'' she said in Russian to the sports newspaper, and asked why
the gymnasts just didn't quit the sport, leave the coaches or file
lawsuits. ``It's just self-promoting, ugh!''

Athletes hesitated out of fear and uncertainty because abuse was so
normalized, Mason said. She said she was disappointed, but not
surprised, to hear Khorkina's comments because ``when you tell a
difficult truth there will always be people trying to discredit you.''

Still, Mason expects even more gymnasts to call for changes in the
sport. She said she had heard from many gymnasts training for the Tokyo
Games who would not come forward until the Olympics were completed. They
do not want to jeopardize their chances of being chosen for their team,
she said.

Those competitors are just some of the many women who have reached out
to Mason since she first posted about ``Athlete A,'' to share their
experiences and vent. They say they no longer feel so scared and alone.

``We have a great chance at changing the sport because so many of us are
finally being heard,'' Mason said. ``There's so many of us and we're so
loud that you can't ignore us.''

Advertisement

\protect\hyperlink{after-bottom}{Continue reading the main story}

\hypertarget{site-index}{%
\subsection{Site Index}\label{site-index}}

\hypertarget{site-information-navigation}{%
\subsection{Site Information
Navigation}\label{site-information-navigation}}

\begin{itemize}
\tightlist
\item
  \href{https://help.nytimes3xbfgragh.onion/hc/en-us/articles/115014792127-Copyright-notice}{©~2020~The
  New York Times Company}
\end{itemize}

\begin{itemize}
\tightlist
\item
  \href{https://www.nytco.com/}{NYTCo}
\item
  \href{https://help.nytimes3xbfgragh.onion/hc/en-us/articles/115015385887-Contact-Us}{Contact
  Us}
\item
  \href{https://www.nytco.com/careers/}{Work with us}
\item
  \href{https://nytmediakit.com/}{Advertise}
\item
  \href{http://www.tbrandstudio.com/}{T Brand Studio}
\item
  \href{https://www.nytimes3xbfgragh.onion/privacy/cookie-policy\#how-do-i-manage-trackers}{Your
  Ad Choices}
\item
  \href{https://www.nytimes3xbfgragh.onion/privacy}{Privacy}
\item
  \href{https://help.nytimes3xbfgragh.onion/hc/en-us/articles/115014893428-Terms-of-service}{Terms
  of Service}
\item
  \href{https://help.nytimes3xbfgragh.onion/hc/en-us/articles/115014893968-Terms-of-sale}{Terms
  of Sale}
\item
  \href{https://spiderbites.nytimes3xbfgragh.onion}{Site Map}
\item
  \href{https://help.nytimes3xbfgragh.onion/hc/en-us}{Help}
\item
  \href{https://www.nytimes3xbfgragh.onion/subscription?campaignId=37WXW}{Subscriptions}
\end{itemize}
