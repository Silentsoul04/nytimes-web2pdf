Sections

SEARCH

\protect\hyperlink{site-content}{Skip to
content}\protect\hyperlink{site-index}{Skip to site index}

\href{https://www.nytimes3xbfgragh.onion/section/technology}{Technology}

\href{https://myaccount.nytimes3xbfgragh.onion/auth/login?response_type=cookie\&client_id=vi}{}

\href{https://www.nytimes3xbfgragh.onion/section/todayspaper}{Today's
Paper}

\href{/section/technology}{Technology}\textbar{}This Tool Could Protect
Your Photos From Facial Recognition

\url{https://nyti.ms/31g6Gjh}

\begin{itemize}
\item
\item
\item
\item
\item
\item
\end{itemize}

Advertisement

\protect\hyperlink{after-top}{Continue reading the main story}

Supported by

\protect\hyperlink{after-sponsor}{Continue reading the main story}

\hypertarget{this-tool-could-protect-your-photos-from-facial-recognition}{%
\section{This Tool Could Protect Your Photos From Facial
Recognition}\label{this-tool-could-protect-your-photos-from-facial-recognition}}

Researchers at the University of Chicago want you to be able to post
selfies without worrying that the next Clearview AI will use them to
identify you.

\includegraphics{https://static01.graylady3jvrrxbe.onion/images/2020/07/31/business/31fawkes4/merlin_175009401_d9273e2a-5fbf-4229-8e98-44a2cce6b72b-articleLarge.jpg?quality=75\&auto=webp\&disable=upscale}

\href{https://www.nytimes3xbfgragh.onion/by/kashmir-hill}{\includegraphics{https://static01.graylady3jvrrxbe.onion/images/2020/07/24/business/author-hill-kashmir/author-hill-kashmir-thumbLarge-v2.png}}

By \href{https://www.nytimes3xbfgragh.onion/by/kashmir-hill}{Kashmir
Hill}

\begin{itemize}
\item
  Published Aug. 3, 2020Updated Aug. 4, 2020, 12:19 a.m. ET
\item
  \begin{itemize}
  \item
  \item
  \item
  \item
  \item
  \item
  \end{itemize}
\end{itemize}

In recent years, companies have been
\href{https://onezero.medium.com/this-simple-facial-recognition-search-engine-can-track-you-down-across-the-internet-518c7129e454}{prowling}
the web for public photos associated with people's names that they can
use to build enormous databases of faces and
\href{https://www.nytimes3xbfgragh.onion/interactive/2019/10/11/technology/flickr-facial-recognition.html}{improve
their facial recognition systems}, adding to a growing sense that
personal privacy is being lost, bit by digital bit.

A start-up called Clearview AI, for example, scraped
\href{https://www.nytimes3xbfgragh.onion/2020/01/18/technology/clearview-privacy-facial-recognition.html}{billions
of online photos} to build a tool for the police that could lead them
from a face to a Facebook account, revealing a person's identity.

Now researchers are trying to foil those systems. A team of computer
engineers at the University of Chicago has developed a tool that
disguises photos with pixel-level changes that confuse facial
recognition systems.

Named \href{http://sandlab.cs.uchicago.edu/fawkes/}{Fawkes} in honor of
the
\href{https://www.nytimes3xbfgragh.onion/2019/11/05/opinion/guy-fawkes-day-v-for-vendetta.html}{Guy
Fawkes mask} favored by protesters worldwide, the software was made
available to developers on the researchers' website last month. After
being discovered by
\href{https://news.ycombinator.com/item?id=23917337}{Hacker News}, it
has been downloaded more than 50,000 times. The researchers are working
on a free app version for noncoders, which they hope to make available
soon.

The software is not intended to be just a one-off tool for
privacy-loving individuals. If deployed across millions of images, it
would be a broadside against facial recognition systems, poisoning the
accuracy of the so-called data sets they gather from the web.

``Our goal is to make Clearview go away,'' said Ben Zhao, a professor of
computer science at the University of Chicago.

Fawkes converts an image --- or ``cloaks'' it, in the researchers'
parlance --- by subtly altering some of the features that facial
recognition systems depend on when they construct a person's face print.
In a \href{https://arxiv.org/pdf/2002.08327.pdf}{research paper},
reported earlier by
\href{https://onezero.medium.com/this-filter-makes-your-photos-invisible-to-facial-recognition-a26929b5ccf}{OneZero},
the team describes ``cloaking'' photos of the actress Gwyneth Paltrow
using the actor Patrick Dempsey's face, so that a system learning what
Ms. Paltrow looks like based on those photos would start associating her
with some of the features of Mr. Dempsey's face.

The changes, usually subtle and not perceptible to the naked eye, would
prevent the system from recognizing Ms. Paltrow when presented with a
real, uncloaked photo of her. In testing, the researchers were able to
fool facial recognition systems from Amazon, Microsoft and the Chinese
tech company Megvii.

\includegraphics{https://static01.graylady3jvrrxbe.onion/images/2020/07/27/business/00fawkes3/merlin_175009404_3d4454a7-1d31-400b-ba2e-15e2b2b25d94-articleLarge.jpg?quality=75\&auto=webp\&disable=upscale}

To test the tool, I asked the team to cloak some images of my family and
me. I then uploaded the originals and the cloaked images to Facebook to
see if they fooled the social network's
\href{https://www.facebookcorewwwi.onion/help/122175507864081}{facial
recognition system}. It worked: Facebook tagged me in the original photo
but did not recognize me in the cloaked version.

However, the changes to the photos were noticeable to the naked eye. In
the altered images, I looked ghoulish, my 3-year-old daughter sprouted
what looked like facial hair, and my husband appeared to have a black
eye.

The researchers had a few explanations for this. One is that the
software is designed to match you with the face template of someone who
looks as much unlike you as possible, pulling from
\href{http://www.robots.ox.ac.uk/~vgg/data/vgg_face2/}{a database} of
celebrity faces. That usually ends up being a person of the opposite
sex, which leads to obvious problems.

``Women get mustaches, and guys get extra eyelashes or eye shadow,'' Mr.
Zhao said. He is enthusiastic about what he calls ``privacy armor'' and
previously helped design
\href{https://www.nytimes3xbfgragh.onion/2020/02/14/technology/alexa-jamming-bracelet-privacy-armor.html}{a
bracelet that stops smart speakers} from overhearing conversations.

Image

Left, our reporter's original images and, right, the ``cloaked''
versions.

The team says it plans to tweak the software so that it will no longer
subtly change the sex of users.

The other issue is that my experiment wasn't what the tool was designed
to do, so Shawn Shan, a Ph.D. student at the University of Chicago who
is one of the creators of the Fawkes software, made the changes to my
photos as extreme as possible to ensure that it worked. Fawkes isn't
intended to keep a facial recognition system like Facebook's from
recognizing someone in a single photo. It's trying to more broadly
corrupt facial recognition systems, performing an algorithmic attack
called data poisoning.

The researchers said that, ideally, people would start cloaking all the
images they uploaded. That would mean a company like Clearview that
scrapes those photos wouldn't be able to create a functioning database,
because an unidentified photo of you from the real world wouldn't match
the template of you that Clearview would have built over time from your
online photos.

But Clearview's chief executive, Hoan Ton-That, ran a version of my
Facebook experiment on the Clearview app and said the technology did not
interfere with his system. In fact, he said, his company could use
images cloaked by Fawkes to improve its ability to make sense of altered
images.

``There are billions of unmodified photos on the internet, all on
different domain names,'' Mr. Ton-That said. ``In practice, it's almost
certainly too late to perfect a technology like Fawkes and deploy it at
scale.''

Image

Hoan Ton-That, the chief executive of Clearview AI, using the Clearview
smart phone application.Credit...Amr Alfiky for The New York Times

Other experts were also skeptical that Fawkes would work. Joseph Atick,
a facial recognition pioneer
\href{https://www.nytimes3xbfgragh.onion/2014/05/18/technology/never-forgetting-a-face.html}{who
has come to regret} the surveillance society he helped to create, said
the volume of images of ourselves that we had already made available
would be too hard to overcome.

``The cat is out of the bag. We're out there,'' Dr. Atick said. ``While
I encourage this type of research, I'm highly skeptical this is a
solution to solve the problem that we're faced with.''

Dr. Atick thinks that only lawmakers can ensure that people have a right
to facial anonymity. No such federal law is on the horizon, though
Democratic senators did recently propose a
\href{https://www.markey.senate.gov/news/press-releases/senators-markey-and-merkley-and-reps-jayapal-pressley-to-introduce-legislation-to-ban-government-use-of-facial-recognition-other-biometric-technology}{ban
on government use of facial recognition}.

``I personally think that no matter which approach you use, you lose,''
said Emily Wenger, a Ph.D. student who helped create Fawkes. ``You can
have these technological solutions, but it's a cat-and-mouse game. And
you can have a law, but there will always be illegal actors.''

Ms. Wenger thinks ``a two-prong approach'' is needed, where individuals
have technological tools and a privacy law to protect themselves.

Elizabeth Joh, a law professor at the University of California, Davis,
has written about tools like Fawkes as
``\href{https://papers.ssrn.com/sol3/papers.cfm?abstract_id=2285095}{privacy
protests},'' where individuals want to thwart surveillance but not for
criminal reasons. She has repeatedly seen what she called a ``tired
rubric" of surveillance, then countersurveillance and then
anti-countersurveillance, as new monitoring technologies are introduced.

``People are feeling a sense of privacy exhaustion,'' Ms. Joh said.
``There are too many ways that our conventional sense of privacy is
being exploited in real life and online.''

For Fawkes to have an immediate effect, we would need all the photos of
ourselves that we had already posted to be cloaked overnight. That could
happen if a huge platform that maintains an enormous number of online
images decided to roll out Fawkes systemwide.

A platform like Facebook's adopting Fawkes would prevent a future
Clearview from scraping its users' images to identify them. ``They could
say, `Give us your real photos, we'll cloak them, and then we'll share
them with the world so you'll be protected,''' Mr. Zhao said.

Jay Nancarrow, a Facebook spokesman, did not rule out that possibility
when asked for comment. ``As part of our efforts to protect people's
privacy, we have a dedicated team exploring this type of technology and
other methods of preventing photo misuse,'' Mr. Nancarrow said.

``I'm actually interning on that exact team at Facebook right now,''
said the Fawkes co-creator Mr. Shan.

Advertisement

\protect\hyperlink{after-bottom}{Continue reading the main story}

\hypertarget{site-index}{%
\subsection{Site Index}\label{site-index}}

\hypertarget{site-information-navigation}{%
\subsection{Site Information
Navigation}\label{site-information-navigation}}

\begin{itemize}
\tightlist
\item
  \href{https://help.nytimes3xbfgragh.onion/hc/en-us/articles/115014792127-Copyright-notice}{©~2020~The
  New York Times Company}
\end{itemize}

\begin{itemize}
\tightlist
\item
  \href{https://www.nytco.com/}{NYTCo}
\item
  \href{https://help.nytimes3xbfgragh.onion/hc/en-us/articles/115015385887-Contact-Us}{Contact
  Us}
\item
  \href{https://www.nytco.com/careers/}{Work with us}
\item
  \href{https://nytmediakit.com/}{Advertise}
\item
  \href{http://www.tbrandstudio.com/}{T Brand Studio}
\item
  \href{https://www.nytimes3xbfgragh.onion/privacy/cookie-policy\#how-do-i-manage-trackers}{Your
  Ad Choices}
\item
  \href{https://www.nytimes3xbfgragh.onion/privacy}{Privacy}
\item
  \href{https://help.nytimes3xbfgragh.onion/hc/en-us/articles/115014893428-Terms-of-service}{Terms
  of Service}
\item
  \href{https://help.nytimes3xbfgragh.onion/hc/en-us/articles/115014893968-Terms-of-sale}{Terms
  of Sale}
\item
  \href{https://spiderbites.nytimes3xbfgragh.onion}{Site Map}
\item
  \href{https://help.nytimes3xbfgragh.onion/hc/en-us}{Help}
\item
  \href{https://www.nytimes3xbfgragh.onion/subscription?campaignId=37WXW}{Subscriptions}
\end{itemize}
