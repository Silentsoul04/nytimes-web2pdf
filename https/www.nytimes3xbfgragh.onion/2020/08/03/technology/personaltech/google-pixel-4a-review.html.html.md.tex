Sections

SEARCH

\protect\hyperlink{site-content}{Skip to
content}\protect\hyperlink{site-index}{Skip to site index}

\href{https://www.nytimes3xbfgragh.onion/section/technology/personaltech}{Personal
Tech}

\href{https://myaccount.nytimes3xbfgragh.onion/auth/login?response_type=cookie\&client_id=vi}{}

\href{https://www.nytimes3xbfgragh.onion/section/todayspaper}{Today's
Paper}

\href{/section/technology/personaltech}{Personal Tech}\textbar{}Google
Pixel 4A Review: At \$350, a Win for Those on a Budget

\url{https://nyti.ms/39Tuc9r}

\begin{itemize}
\item
\item
\item
\item
\item
\item
\end{itemize}

Advertisement

\protect\hyperlink{after-top}{Continue reading the main story}

Supported by

\protect\hyperlink{after-sponsor}{Continue reading the main story}

tech fix

\hypertarget{google-pixel-4a-review-at-350-a-win-for-those-on-a-budget}{%
\section{Google Pixel 4A Review: At \$350, a Win for Those on a
Budget}\label{google-pixel-4a-review-at-350-a-win-for-those-on-a-budget}}

Yes, you can now pay less than \$399 for a smartphone --- and it won't
stink.

\includegraphics{https://static01.graylady3jvrrxbe.onion/images/2020/08/04/business/03techfix1/merlin_175172151_1161db2d-0ad4-4dfb-ba9d-acded59f4da7-articleLarge.jpg?quality=75\&auto=webp\&disable=upscale}

\href{https://www.nytimes3xbfgragh.onion/by/brian-x-chen}{\includegraphics{https://static01.graylady3jvrrxbe.onion/images/2018/02/16/multimedia/author-brian-x-chen/author-brian-x-chen-thumbLarge.jpg}}

By \href{https://www.nytimes3xbfgragh.onion/by/brian-x-chen}{Brian X.
Chen}

\begin{itemize}
\item
  Aug. 3, 2020
\item
  \begin{itemize}
  \item
  \item
  \item
  \item
  \item
  \item
  \end{itemize}
\end{itemize}

People generally adore Google for bringing useful technology to the
masses at an aggressively low price, if not free.

For those who like that, I have happy news: Google is getting really,
really good at the price cutting, while still bringing quality, with its
smartphones.

The evidence? The Pixel 4A smartphone, which the company introduced on
Monday.

With this latest device, Google took the best stuff from
its\href{https://www.nytimes3xbfgragh.onion/2019/10/21/technology/personaltech/pixel-4-review.html}{\$800
high-end Pixel 4} --- specifically, an excellent camera system and a
large, bright screen --- and squeezed it into the body of a \$350
device. That's \$50 less than
\href{https://www.nytimes3xbfgragh.onion/2019/05/07/technology/personaltech/pixel-3a.html}{last
year's Pixel 3A}, which was Google's first budget phone.

The Pixel 4A is cheaper than high-end devices largely because it lacks
the frills in fancy phones, like wireless charging and a face scanner.
But for what you pay, it's a great value. Its camera quality and bright
screen are on a par with many of the best smartphones out there.

Like the
\href{https://www.nytimes3xbfgragh.onion/2020/05/06/technology/personaltech/apple-iphone-se-review.html}{\$399
iPhone SE}, which Apple released in April, the Pixel 4A is a sign that
you no longer need to pay through the nose for a great smartphone with a
nice camera.

Carolina Milanesi, a consumer technology analyst for Creative
Strategies, a consulting firm, said Google's prowess in software was
helping bring down the price of premium smartphone technology.

``It's been a long time coming for this market, which has always been
focused on the high end because of the dominance of Samsung and the
iPhone,'' she said.

I tested the Pixel 4A for two weeks, including comparing it against the
budget iPhone. Here's what I found.

\hypertarget{focusing-on-what-we-care-about}{%
\subsection{Focusing on What We Care
About}\label{focusing-on-what-we-care-about}}

Google, which prides itself on being driven by data, surveyed its
customers to find out
\href{https://www.statista.com/chart/5995/the-most-wanted-smartphone-features/}{what
they wanted in a smartphone}. A large screen, long battery life, lots of
storage and great camera quality were among the features that people
cared most about.

For the most part, that's what you get with the Pixel 4A. Let's run
through its features:

\begin{itemize}
\item
  The screen measures 5.81 diagonal inches and relies on OLED (organic
  light-emitting diode) technology, a newer type of display that can be
  made thinner, lighter and brighter with better color accuracy and
  contrast than its predecessor, LCD, or liquid crystal display. For a
  device in this price range, this is a very roomy, attractive display,
  with nice colors and deep shadow detail.

  Its colors aren't as accurate as the screen on Apple's \$999 iPhone 11
  Pro, which is nearly identical in size, but at this price, I can't
  complain.
\item
  The 4A has a reasonably long battery life, though your mileage may
  depend on what you do. On a typical workday juggling emails, calls and
  texts, I had more than 30 percent battery life left in the evening.
  But on weekends, when I had downtime, playing games and watching
  videos with the 4A drained the battery in a few hours. This is typical
  of smartphones.
\item
  The 4A comes in only one model with a generous amount of storage: 128
  gigabytes, which is enough for most casual users to store thousands of
  photos and lots of apps and games. This was a wise decision by Google.
  Whenever I talk to people about their tech problems, running out of
  storage on their phone is a frequent complaint.
\item
  Pixel phones are famous for their camera system, which has software
  features powered by artificial intelligence and machine learning. The
  Pixel 4A has Google's most important camera features, including
  Portrait mode, also known as the bokeh effect, which puts a picture's
  main subject in sharp focus while gently blurring the background. It
  also offers Night Sight, which makes photos taken in low light look as
  if they had been shot in normal conditions, without a flash.
\end{itemize}

The 4A lacks some features that were in the Pixel 4 --- such as wireless
charging and a face scanner for unlocking the phone. That's not a
dealbreaker for a device this cheap.

But Google left out one big feature that does matter: water resistance.
That would save a phone that was accidentally dunked in a toilet or left
out in a storm. So it was disappointing not to have it because
durability was another feature that people wanted most in their
smartphones.

The company left out water resistance because it put priority on other
features while reducing the price from last year's Pixel 3A, a Google
spokesman said.

\hypertarget{pixel-versus-iphone}{%
\subsection{Pixel Versus iPhone}\label{pixel-versus-iphone}}

The 4A's strengths and weaknesses are best illustrated in a side-by-side
comparison with Apple's iPhone SE. Here's a rundown of where each one
came out on top.

\hypertarget{camera}{%
\subsubsection{\texorpdfstring{\textbf{Camera}}{Camera}}\label{camera}}

Image

A daylight photo shot with the Pixel 4A.Credit...Brian X. Chen

Image

A daylight photo shot with the iPhone SE.Credit...Brian X. Chen

\textbf{Winner: Pixel 4A.} Though both phone cameras produced very
clear, satisfying photos in bright light, the Pixel's camera was
slightly better than the iPhone's camera. That was partly because of the
Pixel's ability to take photos in low light, a feature that is useful in
some situations, like a family photo inside a dimly lit restaurant.

Image

A low-light photo shot with the Pixel 4A.Credit...Brian X. Chen

Image

A low-light photo shot with the iPhone SE.Credit...Brian X. Chen

\hypertarget{screen}{%
\subsubsection{Screen}\label{screen}}

\textbf{Winner: Pixel 4A.} The Google phone's screen is notably larger
than the iPhone's, which measures 4.7 diagonal inches and uses older,
dimmer-looking LCD screen technology.

\hypertarget{power}{%
\subsubsection{Power}\label{power}}

\textbf{Winner: Pixel 4A.} On average, after a day of heavy use, the
Pixel 4A had plenty of battery left by the evening. The iPhone SE's
battery was usually almost empty by supper time. (Unlike the Pixel 4A,
the iPhone SE does have wireless charging, which
\href{https://www.nytimes3xbfgragh.onion/2018/10/03/technology/personaltech/wireless-charging-pros-cons.html}{is
not a must-have feature.})

\hypertarget{durability}{%
\subsubsection{Durability}\label{durability}}

\textbf{Winner: iPhone SE.} The Apple phone is water and dust resistant.
That's a major benefit for budget-conscious consumers who want their
phone to last many years.

\hypertarget{performance}{%
\subsubsection{Performance}\label{performance}}

\textbf{Winner: iPhone SE.} Gaming and web browsing are noticeably
smoother on the budget iPhone. The Apple handset uses the same mobile
computing processor as the one found in Apple's high-end phones, which
is the fastest on the market. People who plan to use their phone heavily
for work tasks may prefer the iPhone.

\hypertarget{storage}{%
\subsubsection{Storage}\label{storage}}

\textbf{Winner: Pixel 4A.} The iPhone SE starts at \$399 with 64
gigabytes, a modest amount of storage that can be quickly gobbled up if
you take lots of photos. To get an iPhone SE that matches the 128
gigabytes in the Pixel 4A, you would have to shell out \$449.

\hypertarget{the-bottom-line}{%
\subsection{The Bottom Line}\label{the-bottom-line}}

Both the budget Pixel and Apple phones are excellent values. Some may
prefer the Pixel 4A for its richer set of features, while others may
choose the iPhone for its greater longevity.

What's clear is that the tech giants are now aggressively competing to
offer bang-for-the-buck phones. For several years, especially after the
introduction of the
\href{https://www.nytimes3xbfgragh.onion/2017/11/01/technology/personaltech/apple-iphone-x-review.html}{\$999
iPhone X in 2017}, it felt as if phone prices were only going up. It's a
relief that the Pixel 4A is not only better than its predecessor but
also cheaper.

So why would you spend \$800 for the Pixel 4 if the Pixel 4A, for less
than half the price, is also very good minus some superfluous features?

The clear answer: Only gadget enthusiasts should have to pay a premium
for cutting-edge tech.

The rest of us now have the option to save.

Advertisement

\protect\hyperlink{after-bottom}{Continue reading the main story}

\hypertarget{site-index}{%
\subsection{Site Index}\label{site-index}}

\hypertarget{site-information-navigation}{%
\subsection{Site Information
Navigation}\label{site-information-navigation}}

\begin{itemize}
\tightlist
\item
  \href{https://help.nytimes3xbfgragh.onion/hc/en-us/articles/115014792127-Copyright-notice}{©~2020~The
  New York Times Company}
\end{itemize}

\begin{itemize}
\tightlist
\item
  \href{https://www.nytco.com/}{NYTCo}
\item
  \href{https://help.nytimes3xbfgragh.onion/hc/en-us/articles/115015385887-Contact-Us}{Contact
  Us}
\item
  \href{https://www.nytco.com/careers/}{Work with us}
\item
  \href{https://nytmediakit.com/}{Advertise}
\item
  \href{http://www.tbrandstudio.com/}{T Brand Studio}
\item
  \href{https://www.nytimes3xbfgragh.onion/privacy/cookie-policy\#how-do-i-manage-trackers}{Your
  Ad Choices}
\item
  \href{https://www.nytimes3xbfgragh.onion/privacy}{Privacy}
\item
  \href{https://help.nytimes3xbfgragh.onion/hc/en-us/articles/115014893428-Terms-of-service}{Terms
  of Service}
\item
  \href{https://help.nytimes3xbfgragh.onion/hc/en-us/articles/115014893968-Terms-of-sale}{Terms
  of Sale}
\item
  \href{https://spiderbites.nytimes3xbfgragh.onion}{Site Map}
\item
  \href{https://help.nytimes3xbfgragh.onion/hc/en-us}{Help}
\item
  \href{https://www.nytimes3xbfgragh.onion/subscription?campaignId=37WXW}{Subscriptions}
\end{itemize}
