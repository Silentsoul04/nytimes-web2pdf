Sections

SEARCH

\protect\hyperlink{site-content}{Skip to
content}\protect\hyperlink{site-index}{Skip to site index}

\href{https://www.nytimes3xbfgragh.onion/section/technology}{Technology}

\href{https://myaccount.nytimes3xbfgragh.onion/auth/login?response_type=cookie\&client_id=vi}{}

\href{https://www.nytimes3xbfgragh.onion/section/todayspaper}{Today's
Paper}

\href{/section/technology}{Technology}\textbar{}With TikTok, `Uncool'
Microsoft Aims for the Love of Tweens

\href{https://nyti.ms/3hTY2xA}{https://nyti.ms/3hTY2xA}

\begin{itemize}
\item
\item
\item
\item
\item
\item
\end{itemize}

Advertisement

\protect\hyperlink{after-top}{Continue reading the main story}

Supported by

\protect\hyperlink{after-sponsor}{Continue reading the main story}

\hypertarget{with-tiktok-uncool-microsoft-aims-for-the-love-of-tweens}{%
\section{With TikTok, `Uncool' Microsoft Aims for the Love of
Tweens}\label{with-tiktok-uncool-microsoft-aims-for-the-love-of-tweens}}

Buying one of the largest and most influential social networks in the
country could change the perception of the tech giant.

\includegraphics{https://static01.graylady3jvrrxbe.onion/images/2020/08/03/business/03microsoft1/merlin_169492596_88048f5c-3a77-4706-9082-19d123949da8-articleLarge.jpg?quality=75\&auto=webp\&disable=upscale}

\href{https://www.nytimes3xbfgragh.onion/by/karen-weise}{\includegraphics{https://static01.graylady3jvrrxbe.onion/images/2019/04/11/multimedia/author-karen-weise/author-karen-weise-thumbLarge.png}}

By \href{https://www.nytimes3xbfgragh.onion/by/karen-weise}{Karen Weise}

\begin{itemize}
\item
  Aug. 3, 2020
\item
  \begin{itemize}
  \item
  \item
  \item
  \item
  \item
  \item
  \end{itemize}
\end{itemize}

SEATTLE --- Maybe money can buy love. Or, at least,
\href{https://www.nytimes3xbfgragh.onion/2020/08/03/us/navy-seal-museum-kaepernick.html}{Microsoft}
would like to find out.

On Sunday, the tech giant
\href{https://blogs.microsoft.com/blog/2020/08/02/microsoft-to-continue-discussions-on-potential-tiktok-purchase-in-the-united-states/}{announced
in a blog post} that it would continue holding talks to buy TikTok in
the United States, opening up the possibility of using its financial
might to buy the fickle infatuation of tweens.

Microsoft made the announcement after its chief executive, Satya
Nadella, talked with President Trump, who had threatened to ban the
hugely popular social media company, which is owned by China's
ByteDance, because of national security fears. On Monday,
\href{https://www.nytimes3xbfgragh.onion/2020/08/03/technology/trump-tiktok-microsoft.html?action=click\&module=Top\%20Stories\&pgtype=Homepage}{Mr.
Trump said} that TikTok would shut down in the United States on Sept. 15
unless Microsoft or another company bought it.

A deal for TikTok --- whose app featuring video clips has become wildly
popular among young smartphone users --- could give Microsoft, a company
best known for databases and operating systems, control of one of the
largest and most influential social networks in the country.

``Microsoft is viewed as your grandpa's company, and it is trying to
change that,'' said Dan Ives, a managing director and an analyst at
Wedbush Securities. ``Microsoft goes from an uncool company to many
under 25 to potentially as hip as TikTok if they get this done.''

Microsoft declined to comment beyond its blog post.

Under Mr. Nadella, who became chief executive in early 2014, Microsoft
has turned the company into a giant of cloud computing. It has a market
value above \$1.5 trillion and more than \$130 billion in cash.

But Microsoft has not had a clear path to serving young consumers beyond
its video gaming business. The company is behind the Xbox video game
console and owns Minecraft, the hugely popular building game. As social
media grew into enormous consumer businesses --- Facebook alone is worth
more than \$720 billion --- Microsoft largely missed out.

Under Mr. Nadella, Microsoft's biggest acquisitions have been of online
communities, those whose networks hooked users but also required a
complex cloud-computing infrastructure that Microsoft could provide.

In 2014, it bought the Swedish company that created Minecraft for \$2.5
billion, and in 2016, it acquired LinkedIn, the professional social
network, for \$26.2 billion. In 2018, Microsoft bought GitHub, an online
network for software developers, for \$7.5 billion.

If history is a guide, teens and tweens may not need to worry that
Microsoft will mess up the product they love. With those recent
acquisitions, Microsoft increased the financial and tech resources at
the companies but largely let them run independently.

When Christopher Wanstrath, co-founder of GitHub, discussed the deal in
a call with investors after it was announced, he said the way Microsoft
handled Minecraft and LinkedIn showed him ``how serious they are about
growing new businesses while maintaining their independence and
identity.''

Analysts say that patient approach has been successful. Until the
coronavirus pandemic, LinkedIn had been growing faster under Microsoft's
ownership than before it was acquired.

TikTok, with its fun artificial intelligence tools and more than 100
million users in the United States, would most likely fit that pattern.
To be successful, its owner would need to run the technology and sustain
its online community. The deal being discussed involves purchasing
TikTok offices in the United States, Canada, Australia and New Zealand.
ByteDance, the parent company of TikTok, would continue to own the
social media app's offices in Beijing.

\includegraphics{https://static01.graylady3jvrrxbe.onion/images/2020/08/03/business/03microsoft3/merlin_171386274_c93d8e9b-6971-4094-b774-f04c0ff69273-articleLarge.jpg?quality=75\&auto=webp\&disable=upscale}

Microsoft also offers something other giant tech rivals cannot: peace in
Washington. Amazon, Facebook and Google have to tread carefully amid
antitrust scrutiny, making a blockbuster deal with obvious political
baggage unappealing. Other than its gaming business, Microsoft is mainly
in business markets.

And it has strategically navigated Mr. Trump's Washington. While the
company has criticized some of the president's policies, such as
immigration restrictions, it won a high-profile,
\href{https://www.nytimes3xbfgragh.onion/2019/10/25/technology/dod-jedi-contract.html}{\$10
billion contract} to provide cloud computing to the Pentagon, which many
analysts had expected Amazon to get. Amazon is contesting the award,
saying it was politically motivated.

Microsoft has
\href{https://www.nytimes3xbfgragh.onion/2019/01/23/business/china-microsoft-bing.html}{gingerly
tried to court China for more than a decade}, with only modest success.
Its search engine, Bing, was one of the last remaining portals to the
global internet, in part because the service directed users in China to
state media accounts on disputed topics like the Dalai Lama. Even so,
Bing
\href{https://www.nytimes3xbfgragh.onion/2019/01/24/technology/microsoft-bing-china.html}{was
taken down} briefly last year. While Microsoft Windows and Office are
common in China, they are largely pirated copies; the country accounts
for
\href{https://www.nytimes3xbfgragh.onion/2020/02/26/technology/microsoft-coronavirus-earnings.html}{less
than 2 percent} of Microsoft's revenue.

Mr. Nadella's conversation with Mr. Trump appeared to assuage the
president's worries about TikTok's security, at least temporarily. The
concern of the American government is that the app potentially gives the
Chinese government access to American user data. Microsoft's statement
said it would ``ensure that all private data of TikTok's American users
is transferred to and remains in the United States.''

On Sunday evening, the news of continuing talks quickly garnered praise
from Republican lawmakers.

Senator Lindsey Graham of South Carolina, who
\href{https://www.nytimes3xbfgragh.onion/2020/08/02/business/economy/trump-tiktok-china-national-security.html}{spoke
with Microsoft officials a few times} about the deal,
\href{https://twitter.com/LindseyGrahamSC/status/1290083628261584896}{tweeted}
that the talks between TikTok and Microsoft were ``a win-win in the
making.'' And Representative Kevin McCarthy of California, the House
minority leader,
\href{https://twitter.com/GOPLeader/status/1290037004797804544}{said} it
``would protect Americans' data, increase competition and secure a
creative platform for millions of users.''

Microsoft, said Mr. Ives, the analyst, ``realized there is a window in
the clouds that just opened up, and they have to fly right through.''

Steve Lohr contributed reporting from New York.

Advertisement

\protect\hyperlink{after-bottom}{Continue reading the main story}

\hypertarget{site-index}{%
\subsection{Site Index}\label{site-index}}

\hypertarget{site-information-navigation}{%
\subsection{Site Information
Navigation}\label{site-information-navigation}}

\begin{itemize}
\tightlist
\item
  \href{https://help.nytimes3xbfgragh.onion/hc/en-us/articles/115014792127-Copyright-notice}{©~2020~The
  New York Times Company}
\end{itemize}

\begin{itemize}
\tightlist
\item
  \href{https://www.nytco.com/}{NYTCo}
\item
  \href{https://help.nytimes3xbfgragh.onion/hc/en-us/articles/115015385887-Contact-Us}{Contact
  Us}
\item
  \href{https://www.nytco.com/careers/}{Work with us}
\item
  \href{https://nytmediakit.com/}{Advertise}
\item
  \href{http://www.tbrandstudio.com/}{T Brand Studio}
\item
  \href{https://www.nytimes3xbfgragh.onion/privacy/cookie-policy\#how-do-i-manage-trackers}{Your
  Ad Choices}
\item
  \href{https://www.nytimes3xbfgragh.onion/privacy}{Privacy}
\item
  \href{https://help.nytimes3xbfgragh.onion/hc/en-us/articles/115014893428-Terms-of-service}{Terms
  of Service}
\item
  \href{https://help.nytimes3xbfgragh.onion/hc/en-us/articles/115014893968-Terms-of-sale}{Terms
  of Sale}
\item
  \href{https://spiderbites.nytimes3xbfgragh.onion}{Site Map}
\item
  \href{https://help.nytimes3xbfgragh.onion/hc/en-us}{Help}
\item
  \href{https://www.nytimes3xbfgragh.onion/subscription?campaignId=37WXW}{Subscriptions}
\end{itemize}
