Sections

SEARCH

\protect\hyperlink{site-content}{Skip to
content}\protect\hyperlink{site-index}{Skip to site index}

\href{https://www.nytimes3xbfgragh.onion/section/theater}{Theater}

\href{https://myaccount.nytimes3xbfgragh.onion/auth/login?response_type=cookie\&client_id=vi}{}

\href{https://www.nytimes3xbfgragh.onion/section/todayspaper}{Today's
Paper}

\href{/section/theater}{Theater}\textbar{}The (Virtual) Theatrical
Fringe Moves Front and Center

\url{https://nyti.ms/2Dgxo35}

\begin{itemize}
\item
\item
\item
\item
\item
\item
\end{itemize}

Advertisement

\protect\hyperlink{after-top}{Continue reading the main story}

Supported by

\protect\hyperlink{after-sponsor}{Continue reading the main story}

critic's notebook

\hypertarget{the-virtual-theatrical-fringe-moves-front-and-center}{%
\section{The (Virtual) Theatrical Fringe Moves Front and
Center}\label{the-virtual-theatrical-fringe-moves-front-and-center}}

Expanding content and experimenting with form, the avant-garde finds a
congenial new home online, as two recent offerings demonstrate.

\includegraphics{https://static01.graylady3jvrrxbe.onion/images/2020/08/04/arts/03farmscrounger-notebook-4/merlin_175262493_88d58a34-1b1e-4dd1-87a6-75756ba29c80-articleLarge.jpg?quality=75\&auto=webp\&disable=upscale}

\href{https://www.nytimes3xbfgragh.onion/by/jesse-green}{\includegraphics{https://static01.graylady3jvrrxbe.onion/images/2018/02/16/multimedia/author-jesse-green/author-jesse-green-thumbLarge.jpg}}

By \href{https://www.nytimes3xbfgragh.onion/by/jesse-green}{Jesse Green}

\begin{itemize}
\item
  Aug. 3, 2020
\item
  \begin{itemize}
  \item
  \item
  \item
  \item
  \item
  \item
  \end{itemize}
\end{itemize}

As the pandemic has sent most art forms scurrying into mole holes, some
have had to adapt more than others. In the theater, the change has been
especially pronounced, amounting to a complete upside-down flip: Big
players can't squeeze themselves into the new accommodations, but little
ones feel right at home.

So the experimental, the fringe and the avant-garde, which have never
depended on lavish funding or popular attention, are not only surviving
in their online digs but also, at least by contrast, thriving. My diet
as a theater critic has likewise flipped. Temporarily relieved of
Broadway jukebox musicals and 16-week revivals, I am taking in more
offbeat stuff than ever, and finding more to like in it.

Or more to learn from it, at least.

Last week alone, I watched a slapstick adaptation of
\href{https://www.nytimes3xbfgragh.onion/2020/07/31/theater/the-7th-voyage-of-egon-tichy-review.html}{a
science fiction classic staged in a closet} by Theater in Quarantine, a
\href{http://manualcinema.com/watch/}{nearly wordless shadow-puppet
show} from Manual Cinema and
\href{https://www.nytimes3xbfgragh.onion/2020/07/26/theater/in-these-uncertain-times-review.html}{a
Zoom pandemic drama} devised by Source Material. On Saturday I added an
in-your-face autobiographical memoir about a wheelchair and a multimedia
memory play produced, in part, by an academic think tank.

Mind you, offbeat material, by its nature, is not always as coherent or
as charming as the polished commercial kind. Nor is everything offbeat
always so fresh; though some of what I saw was making its debut via
livestream, some of it was recorded. (Manual Cinema's show,
``\href{https://www.nytimes3xbfgragh.onion/2017/01/06/theater/lula-del-ray-a-spectral-parade-of-fantastical-images.html}{Lula
del Ray},'' is part of
\href{https://www.nytimes3xbfgragh.onion/2020/07/23/theater/manual-cinema-puppets-retrospective.html}{the
company's 10th anniversary retrospective}.) Furthermore, like moles
everywhere, online experimental productions can be hard to catch: They
pop up pretty suddenly and disappear fast.

But it's worth the effort to track them down because they are doing the
important work that the avant-garde has always done, only now on your
computer and often for free. They push theater toward new content and
forms.

\includegraphics{https://static01.graylady3jvrrxbe.onion/images/2020/08/03/arts/03farmscrounger-notebook-2/merlin_175262487_42cdba76-1043-46f2-b085-ad0f4063619a-articleLarge.jpg?quality=75\&auto=webp\&disable=upscale}

Accessibility is a key theme in Athena Stevens's play ``Scrounger,''
which was recorded in front of a live audience in January at the 50-seat
\href{https://finboroughtheatre.co.uk}{Finborough Theater} in London.
(\href{https://www.youtube.com/watch?v=1o8jvP5djME}{The recording can be
streamed on YouTube through 7 p.m. Eastern on Aug. 3, and again on Aug.
31.}) It is hardly experimental in concept: It's basically a monologue,
performed by Stevens, with Leigh Quinn assisting in a variety of small
roles. But it is, unfortunately, avant-garde in its concerns, advancing
a woefully belated discussion about
\href{https://www.nytimes3xbfgragh.onion/2020/01/13/theater/under-the-radar-festival-neurodiversity.html}{making
people with disabilities welcome in the theater} both as audiences and
artists.

Stevens, \href{https://www.athenastevens.com/}{a Chicago-born playwright
now living in England}, has cerebral palsy. ``Scrounger,'' directed by
Lily McLeish, is the galling true story of how she was kicked off a
British Airways flight from London to Glasgow because the plane crew
could not fit her motorized wheelchair into the cargo hold --- and
subsequently ``trashed'' it. Neither the regulations of the European
Union nor a social media campaign that drew thousands of supporters
seemed to help; she remained all but trapped in her apartment for weeks.

It says a lot about Stevens's mordant tone that she named the play (and
her character) for the insult online haters flung at her, as if she were
perpetrating a scam instead of trying to regain her independent
mobility. Also part of her strategy is a total lack of ingratiation, not
only to authorities, but also to her boyfriend and even to the audience.
She begins the play by congratulating us for being so ``delightfully
progressive'' as to watch ``a borderline freak show'' in order to brag
about it later.

``You'll say how hard it was to watch me,'' she predicts witheringly,
``but you stuck through the awkward moments because you are a good
person waiting for a poor outsider like me to be a vibrator to your
ego.''

That part was in fact hard to watch, and hear; the captioning was of
little help in interpreting some of Stevens's speech. (It often rendered
``Scrounger'' as ``grandeur.'') But the difficulty was apt: Disability
or not, people don't readily understand one another. And there was no
mistaking the impact of the play's shift from anecdote to insight, as
all of Scrounger's paths to recourse reach a dead end. ``We can't see
what we want in the law, what we think ought to be there when it really
isn't,'' she says mournfully.

Mainstream works won't typically tell you, as ``Scrounger'' does, that
optimism is just another privilege only a few can afford. But
experimental works by Black playwrights, now beginning to reach wide
audiences, have been showing us that for a while. Stevens, who is white,
says she has been influenced by writers of color like
\href{https://www.nytimes3xbfgragh.onion/2014/11/23/magazine/branden-jacobs-jenkins-isnt-writing-about-race.html}{Branden
Jacobs-Jenkins} and
\href{https://www.theguardian.com/stage/2019/apr/27/arinze-kene-interview-misty-death-of-a-salesman}{Arinzé
Kene}, at least in figuring out how do write about disability
\href{https://www.larktheatre.org/blog/athena-stevens-box-ticking-narrative-control/}{without
writing a ``disability play.''} What I think she's missed from those
playwrights, at least in ``Scrounger,'' is the form-busting spirit that
often accompanies the expansion of content.

Image

A screenshot from Virginia Grise's ``a farm for meme,'' which mixes live
film and puppetry.

Virginia Grise's ``a farm for meme'' has the form-busting down. A
20-minute meditation on growth, death and rebirth, it mixes box puppets,
shadow play, live film and archival footage into a gorgeous
mise-en-scène that feels theatrical in its purposefully homemade
aesthetic. Arms are made of red construction paper; flower stems, of
measuring tapes.

Like Grise's earlier work, including
\href{https://artsbeat.blogs.nytimes3xbfgragh.onion/2010/02/23/yale-drama-series-prize-awarded/}{the
award-winning ``blu,''} the story itself is poetic and symbolic ---
perhaps too much so, even in such a short piece. But it is anchored in
the reality of the 14-acre
\href{https://www.southcentralfarm.org/about-us}{South Central Farm}
that arose after the 1992 uprisings in Los Angeles. As a child draws his
memories of that Eden after it is bulldozed, a woman (Marlene Beltran)
tells the story of the queer family and vibrant Chicano culture that
flourished among its walnut trees.

Directed (``virtually'') by Elena Araoz, ``a farm for meme'' was
produced by a consortium that includes
\href{https://www.caramiatheatre.org/}{Cara Mía Theater} of North Texas
and \href{https://www.sociallydistantperformance.com/}{Innovations in
Socially Distant Performance}, an academic research program Araoz leads
at Princeton University. We are going to need those innovations, though
it's hard to say whether all of the ones I heard about in a companion
seminar are viable. Plays on eBay? Ghost audiences?

But that's the point, really. Ideas bubble up from underground during
crises, thanks to people who couldn't find a platform before. Now that
they're suggesting new kinds of stories and new ways of telling them,
can we hope that, soon enough, the rest of the theater will start
putting them together?

Advertisement

\protect\hyperlink{after-bottom}{Continue reading the main story}

\hypertarget{site-index}{%
\subsection{Site Index}\label{site-index}}

\hypertarget{site-information-navigation}{%
\subsection{Site Information
Navigation}\label{site-information-navigation}}

\begin{itemize}
\tightlist
\item
  \href{https://help.nytimes3xbfgragh.onion/hc/en-us/articles/115014792127-Copyright-notice}{©~2020~The
  New York Times Company}
\end{itemize}

\begin{itemize}
\tightlist
\item
  \href{https://www.nytco.com/}{NYTCo}
\item
  \href{https://help.nytimes3xbfgragh.onion/hc/en-us/articles/115015385887-Contact-Us}{Contact
  Us}
\item
  \href{https://www.nytco.com/careers/}{Work with us}
\item
  \href{https://nytmediakit.com/}{Advertise}
\item
  \href{http://www.tbrandstudio.com/}{T Brand Studio}
\item
  \href{https://www.nytimes3xbfgragh.onion/privacy/cookie-policy\#how-do-i-manage-trackers}{Your
  Ad Choices}
\item
  \href{https://www.nytimes3xbfgragh.onion/privacy}{Privacy}
\item
  \href{https://help.nytimes3xbfgragh.onion/hc/en-us/articles/115014893428-Terms-of-service}{Terms
  of Service}
\item
  \href{https://help.nytimes3xbfgragh.onion/hc/en-us/articles/115014893968-Terms-of-sale}{Terms
  of Sale}
\item
  \href{https://spiderbites.nytimes3xbfgragh.onion}{Site Map}
\item
  \href{https://help.nytimes3xbfgragh.onion/hc/en-us}{Help}
\item
  \href{https://www.nytimes3xbfgragh.onion/subscription?campaignId=37WXW}{Subscriptions}
\end{itemize}
