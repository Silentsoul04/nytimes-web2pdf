Sections

SEARCH

\protect\hyperlink{site-content}{Skip to
content}\protect\hyperlink{site-index}{Skip to site index}

\href{/section/travel}{Travel}\textbar{}A Visit to 5 of Patagonia's Most
Remote Schoolhouses

\url{https://nyti.ms/31dLaeT}

\begin{itemize}
\item
\item
\item
\item
\item
\item
\end{itemize}

\includegraphics{https://static01.graylady3jvrrxbe.onion/images/2020/08/03/travel/03travel-patagonia-01/03travel-patagonia-01-articleLarge-v2.jpg?quality=75\&auto=webp\&disable=upscale}

The World Through a Lens

\hypertarget{a-visit-to-5-of-patagonias-most-remote-schoolhouses}{%
\section{A Visit to 5 of Patagonia's Most Remote
Schoolhouses}\label{a-visit-to-5-of-patagonias-most-remote-schoolhouses}}

Community life within the isolated villages of southern Chile is
facilitated in part by an unlikely source: a network of rural schools.

Students play soccer in front of the rural school in Seno Obstrucción, a
coastal village south of Puerto Natales. (The school's lone teacher
resides in the house on the left.)Credit...

Supported by

\protect\hyperlink{after-sponsor}{Continue reading the main story}

Photographs and Text by Andria Hautamaki

\begin{itemize}
\item
  Aug. 3, 2020
\item
  \begin{itemize}
  \item
  \item
  \item
  \item
  \item
  \item
  \end{itemize}
\end{itemize}

\emph{At the onset of the coronavirus pandemic, with travel restrictions
in place worldwide, we launched a new series ---}
\href{https://www.nytimes3xbfgragh.onion/column/the-world-through-a-lens}{\emph{The
World Through a Lens}} \emph{--- in which photojournalists help
transport you, virtually, to some of our planet's most beautiful and
intriguing places. This week, Andria Hautamaki shares a collection of
images from rural Patagonia.}

\begin{center}\rule{0.5\linewidth}{\linethickness}\end{center}

Known for its soaring, glacier-capped Andean peaks and its labyrinth of
fjords, Magallanes ---~in southernmost Patagonia --- is Chile's largest
but second least populated region. Daily existence here requires
tenacity and resilience,~and community life within the isolated villages
is facilitated in part by an unlikely source: a network of rural
schools.

Last year, more than 275,000 Chilean students attended one of the
country's rural schools. Half of these schools were led by a sole
teacher who instructed multiple grade levels inside a single room. Many
of the schools also include gymnasiums, libraries, cafeterias or
computer rooms --- resources that benefit the broader community.

\includegraphics{https://static01.graylady3jvrrxbe.onion/images/2020/08/03/travel/03travel-patagonia-13/03travel-patagonia-13-articleLarge.jpg?quality=75\&auto=webp\&disable=upscale}

Image

Icy winds whip across the hills surrounding Cerro Dorotea.

After coordinating with local educational authorities and teachers, and
with the blessing of the students' parents and guardians, I spent over a
month last year traveling to five such schools.

The areas I visited exemplify the diversity of lifestyle found in the
remote corners of Chilean Patagonia. Some of the communities rely on
ranching or peat extraction, some on fishing, some on tourism. Many of
the schools themselves are off-the-grid, powered by a combination of
diesel generators and wind or solar energy. When winter temperatures dip
below freezing, gravity-fed water systems and water pipes can ice over
for days or even weeks. Wood-burning stoves heat many of the classrooms.

Image

Romina Pizarro Fuenzalida, 33, redirects a student's attention while
managing multiple grades in a single classroom in Puerto Edén. Desks are
arranged to face separate whiteboards, each corresponding to a different
level of coursework.

Image

José Miguel Zuñiga Negué, 13, skips across a rocky outcropping~on an
island visible from~the school~in Puerto Edén.

Getting to these areas can be an arduous undertaking. Villa Puerto Edén,
for example, a remote hamlet on Wellington Island, was a 27-hour trip
from the mainland. Because of the staggered schedule of the ferry, I had
to wait ten days to catch the next boat out.

Image

The rural school Miguel Montecinos, the red and white structure at
center-right, is accessible only by sea. Situated in Puerto Edén, the
building has five classrooms --- but only two are currently utilized to
accommodate the school's 16 students.

Reaching Pampa Guanaco, a hamlet on the archipelago of Tierra del Fuego,
also required a ferry ride --- in addition to a three-hour drive on a
gravel road.

At the end of my stay, when it came time to return to mainland Chile,
the port had closed because of high winds. I was left stranded on a
blustery shore, surrounded by white-capped waves. Several hours later,
the wind let up just enough for the ferry to resume crossing the Strait
of Magellan.

Image

Cynthia Viani, 31, teaches in a multigrade classroom in Pampa Guanaco.
Ms. Viani's daughter, Matilda, left, is one of the school's three
students.

Image

Sebastián Calderón Calderón, 13, helps his grandfather, Don Hernando,
castrate a bull calf after school. The family lives on Estancia Vicuña
in Tierra del Fuego; Sebastián attends the school in Pampa Guanaco.

Image

Belén Huichapani Troncoso, 7, attends school in Pampa Guanaco and lives
on a sheep and cattle ranch named Estancia Bella Vista.

Classroom instruction in these rural communities is provided from
preschool through eighth grade. After eighth grade, students must
relocate during the school year to the closest city, which could be
hours, or days away. There, they often live with either a parent or
extended family, or even a host family, in order to attend high school.

Image

The rural school Pampa Guanaco has just three students; the entire
school population is pictured here.

Image

Matilda Aguayo Viani, 10, reads a library book while walking home from
school. The town currently has seven year-round residents.

Cynthia Almonacid Molinet, who is 36, teaches 11 students ---
representing six distinct grade levels --- at a school in Cerro Guido.

The school sits on a livestock ranch in view of
\href{https://www.conaf.cl/parques/parque-nacional-torres-del-paine/}{Torres
del Paine National Park} and is often surrounded by gauchos on
horseback. Classroom instruction is sometimes interrupted by the
pattering of hooves from passing flocks of woolly sheep.

Image

Cynthia Almonacid Molinet, 36, enters students' scores in a book used to
track their progress across each grade level. She instructs 11 students
--- representing six distinct grade levels --- at a school in Cerro
Guido.

Image

Students eat lunch in the Cerro Guido cafeteria. A cook prepares
breakfast and lunch each day for the school's students, instructor and
classroom aide.

Image

The school in Cerro Guido, center, includes a gymnasium, right, which is
also utilized by the community. Students arrive and depart at the bus
stop, front left. A pile of chopped wood waits to heat the buildings.
Ranch outbuildings and the long shed for shearing sheep can be seen in
the distance.

One of the many advantages of the rural educational environment ``is
being able to study the parts of a plant and then being able to go out
into the environment to find concrete examples in nature,'' Ms.
Almonacid said.

Image

Constanza Muñoz Curinao, 8, from Puerto Edén, helps her family
gather~junquillo, a fibrous plant that will later be used to weave
artisanal crafts, which are then sold to tourists.

Image

Benjamín Sanchez Hernandez, 7, with his pony, Spirit un Corcel
Indomable. Benjamín is a student at Cerro Guido; he and Spirit represent
Estancia Cerro Guido at an endurance horse race.

Image

Sebastián Calderón Calderón looks forward to the after-school hours he
spends outdoors on Estancia Vicuña. (The ranch is about 15 minutes from
the school building in Pampa Guanaco.) Working with livestock and riding
a four-wheeler is his favorite part of the day, he said.

But working in small, isolated communities is also uniquely demanding.
``Teachers who are in rural schools must enjoy living in extreme
areas,'' Ms. Almonacid said, adding that the management of multigrade
classrooms --- with students at a variety of levels and abilities --- is
a constant struggle.

Image

Karla Vargas Zúñiga, 33, is the teacher in Seno Obstrucción. The school
has four primary school students. Two additional kindergarten students
visit in the mornings.

The coronavirus pandemic has upended educational routines all around the
globe, and many schools in Chile have pivoted to remote learning. But
rural Chilean schools face particularly difficult challenges now, not
the least of which is the lack of consistent internet and
telecommunication networks.

``Not all students have access to the internet, a computer or a
telephone,'' Ms. Almonacid said. ``And parents, due to limited
schooling, find it difficult to help their children with their
homework.''

Image

José Miguel Zuñiga Negué, center, a student in Puerto Edén, spends time
with his two older sisters, Deisy and Cecilia, in his family's kitchen.
Fresh bread rises in the heat above the wood-burning stove; yerba mate
tea is poured to share; and the children's father, Hugo, savors an
afternoon snack of homemade bread and jam.

Image

Krishna Alexsandra Muñoz Torres, 9, visits her aunt, Camila Navarro,
after school. They snack on~nalca, or Chilean rhubarb, a plant native to
the region. (The stalks can be eaten fresh or cooked.)

Enrollment in schools here fluctuates from year to year --- as students
graduate, or as families come or go. But rural schools continue to serve
aspiring marine biologists, artisan boat builders, bilingual tour guides
and veterinarians. And despite the difficulties introduced by the
coronavirus pandemic, these students' professional dreams will continue
to be kindled by daily contact with the natural world, coupled with the
freedom to embrace curiosity and creative problem-solving.

Image

In Puerto Edén, backpacks are hung by grade. (The boat pictured on the
shelf is~No Te Rindas, or Don't Give Up, and is owned by a former
student's parent.)

Image

Using their tennis shoes to skate across a frozen pond, all four
students from Seno Obstrucción play together during morning recess.

\begin{center}\rule{0.5\linewidth}{\linethickness}\end{center}

\href{https://www.ahowdyphoto.com/}{\emph{Andria Hautamaki}} \emph{is an
American photojournalist and writer who lives in southern Chile. You can
follow her work on}
\href{https://www.instagram.com/ahowdyphoto/}{\emph{Instagram}}\emph{.}

\emph{This story was supported by a grant from the photography festival}
\href{https://www.eyesonmainstreetwilson.com/}{\emph{Eyes on Main
Street}}\emph{.}

\emph{\textbf{Follow New York Times Travel}} \emph{on}
\href{https://www.instagram.com/nytimestravel/}{\emph{Instagram}}\emph{,}
\href{https://twitter.com/nytimestravel}{\emph{Twitter}} \emph{and}
\href{https://www.facebookcorewwwi.onion/nytimestravel/}{\emph{Facebook}}\emph{.
And}
\href{https://www.nytimes3xbfgragh.onion/newsletters/traveldispatch}{\emph{sign
up for our weekly Travel Dispatch newsletter}} \emph{to receive expert
tips on traveling smarter and inspiration for your next vacation.}

Advertisement

\protect\hyperlink{after-bottom}{Continue reading the main story}

\hypertarget{site-index}{%
\subsection{Site Index}\label{site-index}}

\hypertarget{site-information-navigation}{%
\subsection{Site Information
Navigation}\label{site-information-navigation}}

\begin{itemize}
\tightlist
\item
  \href{https://help.nytimes3xbfgragh.onion/hc/en-us/articles/115014792127-Copyright-notice}{©~2020~The
  New York Times Company}
\end{itemize}

\begin{itemize}
\tightlist
\item
  \href{https://www.nytco.com/}{NYTCo}
\item
  \href{https://help.nytimes3xbfgragh.onion/hc/en-us/articles/115015385887-Contact-Us}{Contact
  Us}
\item
  \href{https://www.nytco.com/careers/}{Work with us}
\item
  \href{https://nytmediakit.com/}{Advertise}
\item
  \href{http://www.tbrandstudio.com/}{T Brand Studio}
\item
  \href{https://www.nytimes3xbfgragh.onion/privacy/cookie-policy\#how-do-i-manage-trackers}{Your
  Ad Choices}
\item
  \href{https://www.nytimes3xbfgragh.onion/privacy}{Privacy}
\item
  \href{https://help.nytimes3xbfgragh.onion/hc/en-us/articles/115014893428-Terms-of-service}{Terms
  of Service}
\item
  \href{https://help.nytimes3xbfgragh.onion/hc/en-us/articles/115014893968-Terms-of-sale}{Terms
  of Sale}
\item
  \href{https://spiderbites.nytimes3xbfgragh.onion}{Site Map}
\item
  \href{https://help.nytimes3xbfgragh.onion/hc/en-us}{Help}
\item
  \href{https://www.nytimes3xbfgragh.onion/subscription?campaignId=37WXW}{Subscriptions}
\end{itemize}
