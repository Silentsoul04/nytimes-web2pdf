Sections

SEARCH

\protect\hyperlink{site-content}{Skip to
content}\protect\hyperlink{site-index}{Skip to site index}

\href{https://www.nytimes3xbfgragh.onion/section/us}{U.S.}

\href{https://myaccount.nytimes3xbfgragh.onion/auth/login?response_type=cookie\&client_id=vi}{}

\href{https://www.nytimes3xbfgragh.onion/section/todayspaper}{Today's
Paper}

\href{/section/us}{U.S.}\textbar{}As the Coronavirus Comes to School, a
Tough Choice: When to Close

\url{https://nyti.ms/2Djsdj5}

\begin{itemize}
\item
\item
\item
\item
\item
\end{itemize}

\href{https://www.nytimes3xbfgragh.onion/news-event/coronavirus?action=click\&pgtype=Article\&state=default\&region=TOP_BANNER\&context=storylines_menu}{The
Coronavirus Outbreak}

\begin{itemize}
\tightlist
\item
  live\href{https://www.nytimes3xbfgragh.onion/2020/08/03/world/coronavirus-covid-19.html?action=click\&pgtype=Article\&state=default\&region=TOP_BANNER\&context=storylines_menu}{Latest
  Updates}
\item
  \href{https://www.nytimes3xbfgragh.onion/interactive/2020/us/coronavirus-us-cases.html?action=click\&pgtype=Article\&state=default\&region=TOP_BANNER\&context=storylines_menu}{Maps
  and Cases}
\item
  \href{https://www.nytimes3xbfgragh.onion/interactive/2020/science/coronavirus-vaccine-tracker.html?action=click\&pgtype=Article\&state=default\&region=TOP_BANNER\&context=storylines_menu}{Vaccine
  Tracker}
\item
  \href{https://www.nytimes3xbfgragh.onion/2020/08/02/us/covid-college-reopening.html?action=click\&pgtype=Article\&state=default\&region=TOP_BANNER\&context=storylines_menu}{College
  Reopening}
\item
  \href{https://www.nytimes3xbfgragh.onion/live/2020/08/03/business/stock-market-today-coronavirus?action=click\&pgtype=Article\&state=default\&region=TOP_BANNER\&context=storylines_menu}{Economy}
\end{itemize}

Advertisement

\protect\hyperlink{after-top}{Continue reading the main story}

Supported by

\protect\hyperlink{after-sponsor}{Continue reading the main story}

\hypertarget{as-the-coronavirus-comes-to-school-a-tough-choice-when-to-close}{%
\section{As the Coronavirus Comes to School, a Tough Choice: When to
Close}\label{as-the-coronavirus-comes-to-school-a-tough-choice-when-to-close}}

As schools in the South and the Midwest reopen this week, officials must
decide what steps to take as staff members and students test positive.

\includegraphics{https://static01.graylady3jvrrxbe.onion/images/2020/08/03/us/03VIRUS-SCHOOLS-corinth/merlin_175033581_42fb4b24-0f58-473e-b4ff-1626dbdbfce8-articleLarge.jpg?quality=75\&auto=webp\&disable=upscale}

\href{https://www.nytimes3xbfgragh.onion/by/sarah-mervosh}{\includegraphics{https://static01.graylady3jvrrxbe.onion/images/2018/07/18/multimedia/author-sarah-mervosh/author-sarah-mervosh-thumbLarge-v3.png}}\href{https://www.nytimes3xbfgragh.onion/by/shawn-hubler}{\includegraphics{https://static01.graylady3jvrrxbe.onion/images/2020/06/05/reader-center/author-shawn-hubler/author-shawn-hubler-thumbLarge.png}}

By \href{https://www.nytimes3xbfgragh.onion/by/sarah-mervosh}{Sarah
Mervosh} and
\href{https://www.nytimes3xbfgragh.onion/by/shawn-hubler}{Shawn Hubler}

\begin{itemize}
\item
  Aug. 3, 2020
\item
  \begin{itemize}
  \item
  \item
  \item
  \item
  \item
  \end{itemize}
\end{itemize}

More than 200 employees have been barred from work in Georgia's largest
school district. A high school in Indiana had to shift to online
learning after just two days. And students in Mississippi were forced to
quarantine after classmates tested positive for the coronavirus during
the first week of classes.

The new academic year is off to a chaotic start as schools open in some
parts of the country while infections continue to rage. Already in the
South and the Midwest, students and teachers have brought the virus to
school with them, triggering quarantines, delayed openings and temporary
shutdowns as positive tests roll in.

With the first schools open barely a week, one question is quickly
arising: How many positive cases should it take to close down again?

It is a question to which education leaders have received vague,
sometimes conflicting answers from state and local officials, with
widely varying standards in different parts of the country.

In New York City, two cases in unrelated classrooms would be enough to
trigger a temporary closure of an entire school, which could be extended
for two weeks. But California's rules say its schools should shut down
if 5 percent of the staff and students test positive, which in large
schools could mean scores of cases. Mississippi guidelines say that
schools ``may consider dismissal'' if
\href{https://msachieves.mdek12.org/wp-content/uploads/2020/07/MSDH-Guidelines-for-K-12-Reopening-During-COVID-19.pdf}{at
least three classrooms} have simultaneous outbreaks.

Federal recommendations also leave the decisions largely up to schools,
saying that ``a single case of Covid-19 in a school would not likely
warrant closing the entire school.''

The uncertainty means that superintendents and other administrators are
being asked to make decisions for which they are often ill-equipped,
said Noelle Ellerson Ng, an associate executive director at AASA, the
School Superintendents Association.

``Do districts know enough? They don't,'' she said, adding, ``They're
not experts on pandemics.''

The rash of positive cases during the first week of school foreshadows a
stop-and-start year in which students and staff members may have to
bounce between instruction in the classroom and remotely at home because
of infections and quarantines.

Many of the nation's largest school districts have already announced
that they will begin the school year with remote learning because
infections in their communities are too widespread. That includes Los
Angeles, the nation's second-largest district, which
\href{https://www.latimes.com/california/story/2020-08-03/lausd-teachers-reach-tentative-online-learning-pact}{reached
a deal on Monday} with the local teachers' union that would establish a
remote learning schedule that resembles a typical school day more
closely than home instruction did in the spring.

In Maryland, Gov. Larry Hogan
\href{https://twitter.com/GovLarryHogan/status/1290330304830246912}{issued
an emergency order} counteracting Montgomery County's health department,
which said on Friday that all private schools needed to
\href{https://www.washingtonpost.com/local/education/montgomery-county-health-officials-tell-private-schools-to-start-school-online/2020/08/01/64552b9e-d3fd-11ea-9038-af089b63ac21_story.html}{start
the year remotely}, just as public schools in the region were planning.

\hypertarget{latest-updates-global-coronavirus-outbreak}{%
\section{\texorpdfstring{\href{https://www.nytimes3xbfgragh.onion/2020/08/03/world/coronavirus-covid-19.html?action=click\&pgtype=Article\&state=default\&region=MAIN_CONTENT_1\&context=storylines_live_updates}{Latest
Updates: Global Coronavirus
Outbreak}}{Latest Updates: Global Coronavirus Outbreak}}\label{latest-updates-global-coronavirus-outbreak}}

Updated 2020-08-04T07:33:06.428Z

\begin{itemize}
\tightlist
\item
  \href{https://www.nytimes3xbfgragh.onion/2020/08/03/world/coronavirus-covid-19.html?action=click\&pgtype=Article\&state=default\&region=MAIN_CONTENT_1\&context=storylines_live_updates\#link-4547638f}{Fauci
  defends Birx after she is criticized by Trump.}
\item
  \href{https://www.nytimes3xbfgragh.onion/2020/08/03/world/coronavirus-covid-19.html?action=click\&pgtype=Article\&state=default\&region=MAIN_CONTENT_1\&context=storylines_live_updates\#link-15e7f995}{Trump
  derides Democrats as lawmakers and administration officials try to
  break stimulus impasse.}
\item
  \href{https://www.nytimes3xbfgragh.onion/2020/08/03/world/coronavirus-covid-19.html?action=click\&pgtype=Article\&state=default\&region=MAIN_CONTENT_1\&context=storylines_live_updates\#link-e5a2cda}{The
  deadline for 2020 census counting has been moved up by a month.}
\end{itemize}

\href{https://www.nytimes3xbfgragh.onion/2020/08/03/world/coronavirus-covid-19.html?action=click\&pgtype=Article\&state=default\&region=MAIN_CONTENT_1\&context=storylines_live_updates}{See
more updates}

More live coverage:
\href{https://www.nytimes3xbfgragh.onion/live/2020/08/03/business/stock-market-today-coronavirus?action=click\&pgtype=Article\&state=default\&region=MAIN_CONTENT_1\&context=storylines_live_updates}{Markets}

Montgomery County is home to some of the nation's most prestigious
private schools, including St. Andrew's Episcopal School, attended by
President Trump's youngest child. Mr. Hogan, a Republican, said the
county's closure order was overly broad and ``inconsistent with the
powers intended to be delegated to the county health officer.''

Mr. Trump, who has pressured schools to reopen and threatened to
withhold federal funding from those that do not teach in person, renewed
his call on Monday, tweeting
\href{https://twitter.com/realDonaldTrump/status/1290257055534551043}{first}
``Open the Schools!'' and
\href{https://twitter.com/realDonaldTrump/status/1290488364362170371}{later}
``OPEN THE SCHOOLS!!!''

But some public health experts doubt that schools that open classrooms
in hard-hit parts of the country will be able to avoid shutdowns for
long.

``It's simply not possible,'' said Dr. Peter Hotez, dean for the
National School of Tropical Medicine at Baylor College of Medicine in
Houston, adding that ``there's just too much virus transmission'' across
a significant portion of the United States, particularly the Sun Belt,
for schools there to avoid major outbreaks.

``As soon as you open classrooms, within two weeks, teachers and
students will get sick, bus drivers will get sick, and staff will get
sick,'' Dr. Hotez said. ``And all it's going to take is one teacher
admitted to the hospital in the school district and that's it, it's
going to be lights out and no one will show up to work.''

While school will not start in some parts of the country for several
weeks,
\href{https://www.pewresearch.org/fact-tank/2019/08/14/back-to-school-dates-u-s/}{about
14 percent of American children} typically go back by the first full
week of August, mostly in the South and parts of the Midwest.

It took no time at all for the virus to appear in hallways and
classrooms after schools started in Indiana on Thursday. At Greenfield
Central Junior High School, outside Indianapolis, a student received a
positive test
\href{https://www.nytimes3xbfgragh.onion/2020/08/01/us/schools-reopening-indiana-coronavirus.html}{on
the very first day of classes}, and students who had been in close
contact were told to quarantine for 14 days.

In Elwood, Ind., a community of about 8,000 in the central part of the
state, the superintendent of the Elwood Community School Corporation
sent out a note on Saturday thanking students and parents for
\href{https://www.facebookcorewwwi.onion/ElwoodCommunitySchools/photos/pcb.1875944365880857/1875944205880873/?type=3\&theater}{``a
great first two days of school!''} But the optimistic tone quickly gave
way: Several staff members had tested positive for the virus, he wrote,
and one employee at the high school had potentially exposed other staff
members.

Students in seventh through 12th grades are now spending this week
learning online. Officials plan to return to in-classroom learning as
soon as next week.

That is the right response, according to Dr. Troy Abbott, president of
the Board of Health for Madison County, which includes much of Elwood.
\href{https://www.indystar.com/story/news/education/2020/08/02/indiana-schools-reopening-how-we-know-districts-safe/5558849002/}{Without
state requirements,} he said, the county has been recommending that
schools stay open unless they exceed an average of 24 cases per day over
a seven-day period.

``We don't have a vaccine, and I don't know that we should wait around
for a vaccine,'' Dr. Abbott said, citing
\href{https://www.nytimes3xbfgragh.onion/2020/06/30/us/coronavirus-schools-reopening-guidelines-aap.html}{recommendations
from the American Academy of Pediatrics} that it is important for
children to be physically present in school. He also noted that younger
people are less likely to get seriously sick from the coronavirus.

\href{https://www.nytimes3xbfgragh.onion/news-event/coronavirus?action=click\&pgtype=Article\&state=default\&region=MAIN_CONTENT_3\&context=storylines_faq}{}

\hypertarget{the-coronavirus-outbreak-}{%
\subsubsection{The Coronavirus Outbreak
›}\label{the-coronavirus-outbreak-}}

\hypertarget{frequently-asked-questions}{%
\paragraph{Frequently Asked
Questions}\label{frequently-asked-questions}}

Updated August 3, 2020

\begin{itemize}
\item ~
  \hypertarget{im-a-small-business-owner-can-i-get-relief}{%
  \paragraph{I'm a small-business owner. Can I get
  relief?}\label{im-a-small-business-owner-can-i-get-relief}}

  \begin{itemize}
  \tightlist
  \item
    The
    \href{https://www.nytimes3xbfgragh.onion/article/small-business-loans-stimulus-grants-freelancers-coronavirus.html?action=click\&pgtype=Article\&state=default\&region=MAIN_CONTENT_3\&context=storylines_faq}{stimulus
    bills enacted in March} offer help for the millions of American
    small businesses. Those eligible for aid are businesses and
    nonprofit organizations with fewer than 500 workers, including sole
    proprietorships, independent contractors and freelancers. Some
    larger companies in some industries are also eligible. The help
    being offered, which is being managed by the Small Business
    Administration, includes the Paycheck Protection Program and the
    Economic Injury Disaster Loan program. But lots of folks have
    \href{https://www.nytimes3xbfgragh.onion/interactive/2020/05/07/business/small-business-loans-coronavirus.html?action=click\&pgtype=Article\&state=default\&region=MAIN_CONTENT_3\&context=storylines_faq}{not
    yet seen payouts.} Even those who have received help are confused:
    The rules are draconian, and some are stuck sitting on
    \href{https://www.nytimes3xbfgragh.onion/2020/05/02/business/economy/loans-coronavirus-small-business.html?action=click\&pgtype=Article\&state=default\&region=MAIN_CONTENT_3\&context=storylines_faq}{money
    they don't know how to use.} Many small-business owners are getting
    less than they expected or
    \href{https://www.nytimes3xbfgragh.onion/2020/06/10/business/Small-business-loans-ppp.html?action=click\&pgtype=Article\&state=default\&region=MAIN_CONTENT_3\&context=storylines_faq}{not
    hearing anything at all.}
  \end{itemize}
\item ~
  \hypertarget{what-are-my-rights-if-i-am-worried-about-going-back-to-work}{%
  \paragraph{What are my rights if I am worried about going back to
  work?}\label{what-are-my-rights-if-i-am-worried-about-going-back-to-work}}

  \begin{itemize}
  \tightlist
  \item
    Employers have to provide
    \href{https://www.osha.gov/SLTC/covid-19/standards.html}{a safe
    workplace} with policies that protect everyone equally.
    \href{https://www.nytimes3xbfgragh.onion/article/coronavirus-money-unemployment.html?action=click\&pgtype=Article\&state=default\&region=MAIN_CONTENT_3\&context=storylines_faq}{And
    if one of your co-workers tests positive for the coronavirus, the
    C.D.C.} has said that
    \href{https://www.cdc.gov/coronavirus/2019-ncov/community/guidance-business-response.html}{employers
    should tell their employees} -\/- without giving you the sick
    employee's name -\/- that they may have been exposed to the virus.
  \end{itemize}
\item ~
  \hypertarget{should-i-refinance-my-mortgage}{%
  \paragraph{Should I refinance my
  mortgage?}\label{should-i-refinance-my-mortgage}}

  \begin{itemize}
  \tightlist
  \item
    \href{https://www.nytimes3xbfgragh.onion/article/coronavirus-money-unemployment.html?action=click\&pgtype=Article\&state=default\&region=MAIN_CONTENT_3\&context=storylines_faq}{It
    could be a good idea,} because mortgage rates have
    \href{https://www.nytimes3xbfgragh.onion/2020/07/16/business/mortgage-rates-below-3-percent.html?action=click\&pgtype=Article\&state=default\&region=MAIN_CONTENT_3\&context=storylines_faq}{never
    been lower.} Refinancing requests have pushed mortgage applications
    to some of the highest levels since 2008, so be prepared to get in
    line. But defaults are also up, so if you're thinking about buying a
    home, be aware that some lenders have tightened their standards.
  \end{itemize}
\item ~
  \hypertarget{what-is-school-going-to-look-like-in-september}{%
  \paragraph{What is school going to look like in
  September?}\label{what-is-school-going-to-look-like-in-september}}

  \begin{itemize}
  \tightlist
  \item
    It is unlikely that many schools will return to a normal schedule
    this fall, requiring the grind of
    \href{https://www.nytimes3xbfgragh.onion/2020/06/05/us/coronavirus-education-lost-learning.html?action=click\&pgtype=Article\&state=default\&region=MAIN_CONTENT_3\&context=storylines_faq}{online
    learning},
    \href{https://www.nytimes3xbfgragh.onion/2020/05/29/us/coronavirus-child-care-centers.html?action=click\&pgtype=Article\&state=default\&region=MAIN_CONTENT_3\&context=storylines_faq}{makeshift
    child care} and
    \href{https://www.nytimes3xbfgragh.onion/2020/06/03/business/economy/coronavirus-working-women.html?action=click\&pgtype=Article\&state=default\&region=MAIN_CONTENT_3\&context=storylines_faq}{stunted
    workdays} to continue. California's two largest public school
    districts --- Los Angeles and San Diego --- said on July 13, that
    \href{https://www.nytimes3xbfgragh.onion/2020/07/13/us/lausd-san-diego-school-reopening.html?action=click\&pgtype=Article\&state=default\&region=MAIN_CONTENT_3\&context=storylines_faq}{instruction
    will be remote-only in the fall}, citing concerns that surging
    coronavirus infections in their areas pose too dire a risk for
    students and teachers. Together, the two districts enroll some
    825,000 students. They are the largest in the country so far to
    abandon plans for even a partial physical return to classrooms when
    they reopen in August. For other districts, the solution won't be an
    all-or-nothing approach.
    \href{https://bioethics.jhu.edu/research-and-outreach/projects/eschool-initiative/school-policy-tracker/}{Many
    systems}, including the nation's largest, New York City, are
    devising
    \href{https://www.nytimes3xbfgragh.onion/2020/06/26/us/coronavirus-schools-reopen-fall.html?action=click\&pgtype=Article\&state=default\&region=MAIN_CONTENT_3\&context=storylines_faq}{hybrid
    plans} that involve spending some days in classrooms and other days
    online. There's no national policy on this yet, so check with your
    municipal school system regularly to see what is happening in your
    community.
  \end{itemize}
\item ~
  \hypertarget{is-the-coronavirus-airborne}{%
  \paragraph{Is the coronavirus
  airborne?}\label{is-the-coronavirus-airborne}}

  \begin{itemize}
  \tightlist
  \item
    The coronavirus
    \href{https://www.nytimes3xbfgragh.onion/2020/07/04/health/239-experts-with-one-big-claim-the-coronavirus-is-airborne.html?action=click\&pgtype=Article\&state=default\&region=MAIN_CONTENT_3\&context=storylines_faq}{can
    stay aloft for hours in tiny droplets in stagnant air}, infecting
    people as they inhale, mounting scientific evidence suggests. This
    risk is highest in crowded indoor spaces with poor ventilation, and
    may help explain super-spreading events reported in meatpacking
    plants, churches and restaurants.
    \href{https://www.nytimes3xbfgragh.onion/2020/07/06/health/coronavirus-airborne-aerosols.html?action=click\&pgtype=Article\&state=default\&region=MAIN_CONTENT_3\&context=storylines_faq}{It's
    unclear how often the virus is spread} via these tiny droplets, or
    aerosols, compared with larger droplets that are expelled when a
    sick person coughs or sneezes, or transmitted through contact with
    contaminated surfaces, said Linsey Marr, an aerosol expert at
    Virginia Tech. Aerosols are released even when a person without
    symptoms exhales, talks or sings, according to Dr. Marr and more
    than 200 other experts, who
    \href{https://academic.oup.com/cid/article/doi/10.1093/cid/ciaa939/5867798}{have
    outlined the evidence in an open letter to the World Health
    Organization}.
  \end{itemize}
\end{itemize}

``Our hospitals are not overworked, they are not overwhelmed, we have
not been busy for over a month, and getting kids back to school is a
very reasonable option,'' he said.

\includegraphics{https://static01.graylady3jvrrxbe.onion/images/2020/08/03/us/03VIRUS-SCHOOLS-greenfield/merlin_175172043_10e25aaa-a280-427f-bf4b-b0b4bff0e6dd-articleLarge.jpg?quality=75\&auto=webp\&disable=upscale}

But some public health experts are skeptical that it is safe to reopen
schools in areas where the virus is spreading widely, or where local
officials do not have the means to test and trace people quickly.

Even if children are less likely to get seriously sick, ``every child
that I know lives in a home with an adult,'' said Dr. Harry Heiman, a
clinical associate professor of health policy and behavioral sciences at
Georgia State University. ``The idea that you can safely reopen schools
and not in fact worsen spread is not based on science,'' he added.
``It's based on wishful thinking.''

In some cases, the virus has arrived in schools even before the
students.

In Gwinnett County, Ga., the largest school system in the state,
teachers returned to work on Wednesday in preparation for starting
classes remotely on Aug. 12. But as of Thursday, about 260 employees had
been excluded from work because they tested positive or had potentially
been exposed to the virus.

Sloan Roach, a spokeswoman for the school district, said a majority of
cases had been attributed to community spread. ``We have people who have
called in to report who have not been at school or work,'' she said,
adding that positive cases were to be expected.

Gwinnett County has seen about 4,000 new infections confirmed within the
last two weeks.

About 55 miles away, a district in Pickens County, Ga., faced a similar
problem after staff members who gathered for training at an elementary
school showed coronavirus symptoms. The district, which serves about
4,400 students, delayed the start of school for two weeks so all of that
school's employees could get tested.

``We are just being overcautious,'' the superintendent, Rick Townsend,
said on Monday, which was supposed to be the first day of school. He
said he made the decision to delay classes until Aug. 17 in consultation
with local health officials.

At Corinth High School in northern Mississippi, students are filing into
classrooms according to seating charts to limit their contacts with
others. They eat breakfast and lunch at their desks. English and math
classes are taught in big open spaces, like the cafeteria.

Still, at least three students have tested positive for the virus since
school started last week, and about 40 are in quarantine.

``I've been in the business over 40 years --- I have never experienced
anything like this,'' said Lee Childress, the district's superintendent.
``It's kind of like drinking out of a fire hose because it's happening
so fast.''

Still, after a summer of preparation, he said he felt comfortable
proceeding with the regular start date.

``It doesn't matter if you open schools in July, August, September or
October,'' he said. ``It's something that every school is going to have
to address.''

Advertisement

\protect\hyperlink{after-bottom}{Continue reading the main story}

\hypertarget{site-index}{%
\subsection{Site Index}\label{site-index}}

\hypertarget{site-information-navigation}{%
\subsection{Site Information
Navigation}\label{site-information-navigation}}

\begin{itemize}
\tightlist
\item
  \href{https://help.nytimes3xbfgragh.onion/hc/en-us/articles/115014792127-Copyright-notice}{©~2020~The
  New York Times Company}
\end{itemize}

\begin{itemize}
\tightlist
\item
  \href{https://www.nytco.com/}{NYTCo}
\item
  \href{https://help.nytimes3xbfgragh.onion/hc/en-us/articles/115015385887-Contact-Us}{Contact
  Us}
\item
  \href{https://www.nytco.com/careers/}{Work with us}
\item
  \href{https://nytmediakit.com/}{Advertise}
\item
  \href{http://www.tbrandstudio.com/}{T Brand Studio}
\item
  \href{https://www.nytimes3xbfgragh.onion/privacy/cookie-policy\#how-do-i-manage-trackers}{Your
  Ad Choices}
\item
  \href{https://www.nytimes3xbfgragh.onion/privacy}{Privacy}
\item
  \href{https://help.nytimes3xbfgragh.onion/hc/en-us/articles/115014893428-Terms-of-service}{Terms
  of Service}
\item
  \href{https://help.nytimes3xbfgragh.onion/hc/en-us/articles/115014893968-Terms-of-sale}{Terms
  of Sale}
\item
  \href{https://spiderbites.nytimes3xbfgragh.onion}{Site Map}
\item
  \href{https://help.nytimes3xbfgragh.onion/hc/en-us}{Help}
\item
  \href{https://www.nytimes3xbfgragh.onion/subscription?campaignId=37WXW}{Subscriptions}
\end{itemize}
