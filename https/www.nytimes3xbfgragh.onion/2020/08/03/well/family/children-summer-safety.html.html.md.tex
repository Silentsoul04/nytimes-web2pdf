Sections

SEARCH

\protect\hyperlink{site-content}{Skip to
content}\protect\hyperlink{site-index}{Skip to site index}

\href{https://www.nytimes3xbfgragh.onion/section/well/family}{Family}

\href{https://myaccount.nytimes3xbfgragh.onion/auth/login?response_type=cookie\&client_id=vi}{}

\href{https://www.nytimes3xbfgragh.onion/section/todayspaper}{Today's
Paper}

\href{/section/well/family}{Family}\textbar{}In This Pandemic Summer,
Don't Forget About Kids' Other Risks

\url{https://nyti.ms/3hZTIwM}

\begin{itemize}
\item
\item
\item
\item
\item
\item
\end{itemize}

\href{https://www.nytimes3xbfgragh.onion/news-event/coronavirus?action=click\&pgtype=Article\&state=default\&region=TOP_BANNER\&context=storylines_menu}{The
Coronavirus Outbreak}

\begin{itemize}
\tightlist
\item
  live\href{https://www.nytimes3xbfgragh.onion/2020/08/03/world/coronavirus-covid-19.html?action=click\&pgtype=Article\&state=default\&region=TOP_BANNER\&context=storylines_menu}{Latest
  Updates}
\item
  \href{https://www.nytimes3xbfgragh.onion/interactive/2020/us/coronavirus-us-cases.html?action=click\&pgtype=Article\&state=default\&region=TOP_BANNER\&context=storylines_menu}{Maps
  and Cases}
\item
  \href{https://www.nytimes3xbfgragh.onion/interactive/2020/science/coronavirus-vaccine-tracker.html?action=click\&pgtype=Article\&state=default\&region=TOP_BANNER\&context=storylines_menu}{Vaccine
  Tracker}
\item
  \href{https://www.nytimes3xbfgragh.onion/2020/08/02/us/covid-college-reopening.html?action=click\&pgtype=Article\&state=default\&region=TOP_BANNER\&context=storylines_menu}{College
  Reopening}
\item
  \href{https://www.nytimes3xbfgragh.onion/live/2020/08/03/business/stock-market-today-coronavirus?action=click\&pgtype=Article\&state=default\&region=TOP_BANNER\&context=storylines_menu}{Economy}
\end{itemize}

Advertisement

\protect\hyperlink{after-top}{Continue reading the main story}

Supported by

\protect\hyperlink{after-sponsor}{Continue reading the main story}

The Checkup

\hypertarget{in-this-pandemic-summer-dont-forget-about-kids-other-risks}{%
\section{In This Pandemic Summer, Don't Forget About Kids' Other
Risks}\label{in-this-pandemic-summer-dont-forget-about-kids-other-risks}}

Reinforcing summer safety with some of the special twists and dangers of
this dangerous and twisty time.

\includegraphics{https://static01.graylady3jvrrxbe.onion/images/2020/08/03/well/03klass-summer/03klass-summer-articleLarge.jpg?quality=75\&auto=webp\&disable=upscale}

By \href{https://www.nytimes3xbfgragh.onion/by/perri-klass-md}{Perri
Klass, M.D.}

\begin{itemize}
\item
  Aug. 3, 2020
\item
  \begin{itemize}
  \item
  \item
  \item
  \item
  \item
  \item
  \end{itemize}
\end{itemize}

I'm not sure whether to call this the good news or the bad news, but
Covid-19 is not the only thing that parents need to think about right
now.

Summer regularly brings up a set of safety concerns for children, and
that is as true this year as any other. Of course, this year is very
different, and when I talked to pediatric emergency room specialists
around the country, they reinforced summer safety advice, while noting
some of the special twists and dangers of this dangerous and twisty
time.

``Everyone has cabin fever, and they want to get out and have a good
time,'' said Dr. Mark Zonfrillo, an associate professor of emergency
medicine and pediatrics at the Warren Alpert Medical School of Brown
University. And even as many parents may feel that it's all they can do
to enforce social distancing and mask-wearing, it's important to
remember the safety measures from the before times as well.

Dr. Maya Haasz, an attending physician in the pediatric emergency room
at Children's Hospital Colorado and an assistant professor at University
of Colorado School of Medicine, said they are seeing injuries that
reflect a summer of individual activity rather than team sports. Kids
are out riding their bikes and their scooters, she said, but not always
wearing helmets. ``We're seeing more significant head injuries,'' she
said.

{[}\textbf{\href{https://www.nytimes3xbfgragh.onion/newsletters/well-family}{\emph{Sign
up for the Well Family newsletter}}}{]}

And because some parents are still scared to go to hospitals, injured
children are sometimes not coming in immediately. The delay can be
painful for the child and problematic for the doctors, for example if a
laceration is more than a day old, and can't be safely sewn up.

``We're doing a tremendous amount to keep the hospital safe,'' Dr. Haasz
said. ``You are not at risk of getting Covid in the hospital.''

But to help keep kids out of the emergency room, remember the sunscreen
and the bike helmets and the adult supervision for kids in the water.

\hypertarget{latest-updates-global-coronavirus-outbreak}{%
\section{\texorpdfstring{\href{https://www.nytimes3xbfgragh.onion/2020/08/03/world/coronavirus-covid-19.html?action=click\&pgtype=Article\&state=default\&region=MAIN_CONTENT_1\&context=storylines_live_updates}{Latest
Updates: Global Coronavirus
Outbreak}}{Latest Updates: Global Coronavirus Outbreak}}\label{latest-updates-global-coronavirus-outbreak}}

Updated 2020-08-04T07:33:06.428Z

\begin{itemize}
\tightlist
\item
  \href{https://www.nytimes3xbfgragh.onion/2020/08/03/world/coronavirus-covid-19.html?action=click\&pgtype=Article\&state=default\&region=MAIN_CONTENT_1\&context=storylines_live_updates\#link-4547638f}{Fauci
  defends Birx after she is criticized by Trump.}
\item
  \href{https://www.nytimes3xbfgragh.onion/2020/08/03/world/coronavirus-covid-19.html?action=click\&pgtype=Article\&state=default\&region=MAIN_CONTENT_1\&context=storylines_live_updates\#link-15e7f995}{Trump
  derides Democrats as lawmakers and administration officials try to
  break stimulus impasse.}
\item
  \href{https://www.nytimes3xbfgragh.onion/2020/08/03/world/coronavirus-covid-19.html?action=click\&pgtype=Article\&state=default\&region=MAIN_CONTENT_1\&context=storylines_live_updates\#link-e5a2cda}{The
  deadline for 2020 census counting has been moved up by a month.}
\end{itemize}

\href{https://www.nytimes3xbfgragh.onion/2020/08/03/world/coronavirus-covid-19.html?action=click\&pgtype=Article\&state=default\&region=MAIN_CONTENT_1\&context=storylines_live_updates}{See
more updates}

More live coverage:
\href{https://www.nytimes3xbfgragh.onion/live/2020/08/03/business/stock-market-today-coronavirus?action=click\&pgtype=Article\&state=default\&region=MAIN_CONTENT_1\&context=storylines_live_updates}{Markets}

\hypertarget{water-safety}{%
\subsection{Water Safety}\label{water-safety}}

\href{https://www.nytimes3xbfgragh.onion/2019/07/22/well/family/drowning-children-water-safety.html}{Drowning}
is the leading cause of death for children from 1 to 4 and it can happen
silently and swiftly, leaving behind devastated families and regrets
that never go away.

Dr. Maneesha Agarwal, a pediatric emergency physician and assistant
professor at Emory in Atlanta, said that in 2018, 443 children from 1 to
4 died from drowning, and that it kills about 1,000 children of all ages
every year. There are two peaks in age, she said, first the toddlers and
young children who accidentally gain access to a body of water, and then
the adolescents, the risk-takers, ``who might be horsing around,
sneaking into pools.''

With the pandemic, children may not be going to community pools, where
there would be lifeguards, and the home pool market has been booming.
``A lot of people are getting new
\href{https://www.healthychildren.org/English/safety-prevention/at-play/Pages/Pool-Dangers-Drowning-Prevention-When-Not-Swimming-Time.aspx}{pools}
and first-time pools, so with that comes a responsibility for not only
proper barriers and pool gates, but also proper supervision in an era of
distraction,'' Dr. Zonfrillo said.

Parents need to think about layers of safety, Dr. Agarwal said, such as
having a four-foot tall fence around the entire pool, but also alarms.
Parental supervision is key. ``We recommend for younger children and not
experienced swimmers that they should always be within arm's reach,''
Dr. Agarwal said. Parents should not assume they can rely on a
lifeguard, who will have many swimmers to watch.

Even kiddie pools and shallow bodies of water can be dangerous, Dr.
Zonfrillo said: ``A toddler can drown in just a few inches of water.''

\hypertarget{trampolines}{%
\subsection{Trampolines}\label{trampolines}}

If you have a trampoline, supervise children carefully, follow all
safety instructions, and make sure there is only one child on the
trampoline at a time. Trampoline sales have gone up in the pandemic, and
doctors have been very concerned about
\href{https://www.nytimes3xbfgragh.onion/2020/06/19/well/family/coronavirus-shutdown-children-injuries.html?searchResultPosition=1}{trampoline-related
fractures} and trips to the emergency room. ``A bunch of kids on a
trampoline can really cause a lot of injury,'' Dr. Agarwal said.

\hypertarget{bikes-scooters-and-atvs}{%
\subsection{Bikes, Scooters and ATVs}\label{bikes-scooters-and-atvs}}

Be mindful of
\href{https://www.nytimes3xbfgragh.onion/2019/06/10/well/family/children-bike-scooter-safety.html?searchResultPosition=3}{bike
safety}, be vigilant about helmets. And remember that kids can get badly
injured on scooters and on
\href{https://www.healthychildren.org/English/safety-prevention/at-play/Pages/ATV-Safety-Rules.aspx}{all-terrain
vehicles}, or ATVs. ** ATVs are very common, especially in rural
communities, Dr. Agarwal said, and nationally, about four children are
seen in an emergency department every hour with ATV injuries. She
recently treated a child who had taken ``every single precaution,'' she
said. ``He was on a designated ATV recreational area, he had a helmet,
he was supervised, he had no passengers --- and yet he still managed to
roll over his ATV on himself.'' Bottom line: Although she understands
their appeal, Dr. Agarwal said, ``Don't put your kid on an ATV.''

\hypertarget{sun}{%
\subsection{Sun}\label{sun}}

Take the
\href{https://www.nytimes3xbfgragh.onion/2019/07/15/well/family/shielding-kids-from-the-sun-isnt-just-about-sunscreen.html?searchResultPosition=2}{summer
sun} seriously: Keep children in the shade as much as possible, use hats
and protective clothing in addition to sunscreen. Apply lots of
sunscreen, reapply it every couple of hours, and after children go in
the water. Make sure children stay hydrated, especially if they're
exercising. Children who are engaged in athletics should start hydrating
before they go out to practice, Dr. Agarwal said, and if they haven't
been practicing during the shutdown, they should ease back in, and be
particularly careful about hydration and heat exposure when they go back
to practicing.

\href{https://www.nytimes3xbfgragh.onion/news-event/coronavirus?action=click\&pgtype=Article\&state=default\&region=MAIN_CONTENT_3\&context=storylines_faq}{}

\hypertarget{the-coronavirus-outbreak-}{%
\subsubsection{The Coronavirus Outbreak
›}\label{the-coronavirus-outbreak-}}

\hypertarget{frequently-asked-questions}{%
\paragraph{Frequently Asked
Questions}\label{frequently-asked-questions}}

Updated August 3, 2020

\begin{itemize}
\item ~
  \hypertarget{im-a-small-business-owner-can-i-get-relief}{%
  \paragraph{I'm a small-business owner. Can I get
  relief?}\label{im-a-small-business-owner-can-i-get-relief}}

  \begin{itemize}
  \tightlist
  \item
    The
    \href{https://www.nytimes3xbfgragh.onion/article/small-business-loans-stimulus-grants-freelancers-coronavirus.html?action=click\&pgtype=Article\&state=default\&region=MAIN_CONTENT_3\&context=storylines_faq}{stimulus
    bills enacted in March} offer help for the millions of American
    small businesses. Those eligible for aid are businesses and
    nonprofit organizations with fewer than 500 workers, including sole
    proprietorships, independent contractors and freelancers. Some
    larger companies in some industries are also eligible. The help
    being offered, which is being managed by the Small Business
    Administration, includes the Paycheck Protection Program and the
    Economic Injury Disaster Loan program. But lots of folks have
    \href{https://www.nytimes3xbfgragh.onion/interactive/2020/05/07/business/small-business-loans-coronavirus.html?action=click\&pgtype=Article\&state=default\&region=MAIN_CONTENT_3\&context=storylines_faq}{not
    yet seen payouts.} Even those who have received help are confused:
    The rules are draconian, and some are stuck sitting on
    \href{https://www.nytimes3xbfgragh.onion/2020/05/02/business/economy/loans-coronavirus-small-business.html?action=click\&pgtype=Article\&state=default\&region=MAIN_CONTENT_3\&context=storylines_faq}{money
    they don't know how to use.} Many small-business owners are getting
    less than they expected or
    \href{https://www.nytimes3xbfgragh.onion/2020/06/10/business/Small-business-loans-ppp.html?action=click\&pgtype=Article\&state=default\&region=MAIN_CONTENT_3\&context=storylines_faq}{not
    hearing anything at all.}
  \end{itemize}
\item ~
  \hypertarget{what-are-my-rights-if-i-am-worried-about-going-back-to-work}{%
  \paragraph{What are my rights if I am worried about going back to
  work?}\label{what-are-my-rights-if-i-am-worried-about-going-back-to-work}}

  \begin{itemize}
  \tightlist
  \item
    Employers have to provide
    \href{https://www.osha.gov/SLTC/covid-19/standards.html}{a safe
    workplace} with policies that protect everyone equally.
    \href{https://www.nytimes3xbfgragh.onion/article/coronavirus-money-unemployment.html?action=click\&pgtype=Article\&state=default\&region=MAIN_CONTENT_3\&context=storylines_faq}{And
    if one of your co-workers tests positive for the coronavirus, the
    C.D.C.} has said that
    \href{https://www.cdc.gov/coronavirus/2019-ncov/community/guidance-business-response.html}{employers
    should tell their employees} -\/- without giving you the sick
    employee's name -\/- that they may have been exposed to the virus.
  \end{itemize}
\item ~
  \hypertarget{should-i-refinance-my-mortgage}{%
  \paragraph{Should I refinance my
  mortgage?}\label{should-i-refinance-my-mortgage}}

  \begin{itemize}
  \tightlist
  \item
    \href{https://www.nytimes3xbfgragh.onion/article/coronavirus-money-unemployment.html?action=click\&pgtype=Article\&state=default\&region=MAIN_CONTENT_3\&context=storylines_faq}{It
    could be a good idea,} because mortgage rates have
    \href{https://www.nytimes3xbfgragh.onion/2020/07/16/business/mortgage-rates-below-3-percent.html?action=click\&pgtype=Article\&state=default\&region=MAIN_CONTENT_3\&context=storylines_faq}{never
    been lower.} Refinancing requests have pushed mortgage applications
    to some of the highest levels since 2008, so be prepared to get in
    line. But defaults are also up, so if you're thinking about buying a
    home, be aware that some lenders have tightened their standards.
  \end{itemize}
\item ~
  \hypertarget{what-is-school-going-to-look-like-in-september}{%
  \paragraph{What is school going to look like in
  September?}\label{what-is-school-going-to-look-like-in-september}}

  \begin{itemize}
  \tightlist
  \item
    It is unlikely that many schools will return to a normal schedule
    this fall, requiring the grind of
    \href{https://www.nytimes3xbfgragh.onion/2020/06/05/us/coronavirus-education-lost-learning.html?action=click\&pgtype=Article\&state=default\&region=MAIN_CONTENT_3\&context=storylines_faq}{online
    learning},
    \href{https://www.nytimes3xbfgragh.onion/2020/05/29/us/coronavirus-child-care-centers.html?action=click\&pgtype=Article\&state=default\&region=MAIN_CONTENT_3\&context=storylines_faq}{makeshift
    child care} and
    \href{https://www.nytimes3xbfgragh.onion/2020/06/03/business/economy/coronavirus-working-women.html?action=click\&pgtype=Article\&state=default\&region=MAIN_CONTENT_3\&context=storylines_faq}{stunted
    workdays} to continue. California's two largest public school
    districts --- Los Angeles and San Diego --- said on July 13, that
    \href{https://www.nytimes3xbfgragh.onion/2020/07/13/us/lausd-san-diego-school-reopening.html?action=click\&pgtype=Article\&state=default\&region=MAIN_CONTENT_3\&context=storylines_faq}{instruction
    will be remote-only in the fall}, citing concerns that surging
    coronavirus infections in their areas pose too dire a risk for
    students and teachers. Together, the two districts enroll some
    825,000 students. They are the largest in the country so far to
    abandon plans for even a partial physical return to classrooms when
    they reopen in August. For other districts, the solution won't be an
    all-or-nothing approach.
    \href{https://bioethics.jhu.edu/research-and-outreach/projects/eschool-initiative/school-policy-tracker/}{Many
    systems}, including the nation's largest, New York City, are
    devising
    \href{https://www.nytimes3xbfgragh.onion/2020/06/26/us/coronavirus-schools-reopen-fall.html?action=click\&pgtype=Article\&state=default\&region=MAIN_CONTENT_3\&context=storylines_faq}{hybrid
    plans} that involve spending some days in classrooms and other days
    online. There's no national policy on this yet, so check with your
    municipal school system regularly to see what is happening in your
    community.
  \end{itemize}
\item ~
  \hypertarget{is-the-coronavirus-airborne}{%
  \paragraph{Is the coronavirus
  airborne?}\label{is-the-coronavirus-airborne}}

  \begin{itemize}
  \tightlist
  \item
    The coronavirus
    \href{https://www.nytimes3xbfgragh.onion/2020/07/04/health/239-experts-with-one-big-claim-the-coronavirus-is-airborne.html?action=click\&pgtype=Article\&state=default\&region=MAIN_CONTENT_3\&context=storylines_faq}{can
    stay aloft for hours in tiny droplets in stagnant air}, infecting
    people as they inhale, mounting scientific evidence suggests. This
    risk is highest in crowded indoor spaces with poor ventilation, and
    may help explain super-spreading events reported in meatpacking
    plants, churches and restaurants.
    \href{https://www.nytimes3xbfgragh.onion/2020/07/06/health/coronavirus-airborne-aerosols.html?action=click\&pgtype=Article\&state=default\&region=MAIN_CONTENT_3\&context=storylines_faq}{It's
    unclear how often the virus is spread} via these tiny droplets, or
    aerosols, compared with larger droplets that are expelled when a
    sick person coughs or sneezes, or transmitted through contact with
    contaminated surfaces, said Linsey Marr, an aerosol expert at
    Virginia Tech. Aerosols are released even when a person without
    symptoms exhales, talks or sings, according to Dr. Marr and more
    than 200 other experts, who
    \href{https://academic.oup.com/cid/article/doi/10.1093/cid/ciaa939/5867798}{have
    outlined the evidence in an open letter to the World Health
    Organization}.
  \end{itemize}
\end{itemize}

\hypertarget{heat-stroke}{%
\subsection{Heat Stroke}\label{heat-stroke}}

Heat stroke is always a worry, especially
\href{https://www.kidsandcars.org/how-kids-get-hurt/heat-stroke/}{vehicular
heat stroke}, which happens when small children are left in cars. Many
doctors were worried that the pandemic might put children at additional
risk, if parents who are reluctant to take them into stores leave them
in vehicles. ``In hot temperatures, the temperature in the car can rise
within minutes,'' Dr. Zonfrillo said. Ideally, parents should leave
children at home while they do errands.

This year, Dr. Agarwal said, the numbers are actually looking a little
better. For the past two years, over 50 children a year have died from
heat stroke; there have been 11 deaths so far this year.

\hypertarget{guns}{%
\subsection{Guns}\label{guns}}

Be aware of the danger posed by
\href{https://well.blogs.nytimes3xbfgragh.onion/2013/01/07/keeping-guns-away-from-children/}{firearms}
that are not properly secured and stored. Firearms are not specifically
a summer risk, but this is a summer of children not going to camp, and
home injuries loom large. Firearm sales have increased in the pandemic,
and Dr. Agarwal says that pediatric emergency doctors are worrying over
these new owners especially, and whether they are storing the guns
safely --- they should be stored unloaded, locked up in a gun safe or
with a trigger lock, and with the ammunition locked up in a separate
location. ``I want to encourage all parents to
\href{https://well.blogs.nytimes3xbfgragh.onion/2016/06/24/ask-well-gun-storage-and-children/}{ask
about the presence of unsecured firearms} in any home where children go
to visit,'' Dr. Agarwal said.

\hypertarget{household-injuries}{%
\subsection{Household Injuries}\label{household-injuries}}

Other at-home injuries to avoid in an at-home summer include poisonings
and falls, especially
\href{https://www.nsc.org/home-safety/safety-topics/child-safety/window-safety\#:~:text=Falls\%20from\%20windows\%20are\%20more,to\%20go\%20to\%20the\%20hospital.}{falls
from windows}. A window screen alone is not a sufficient protection. Be
aware that even the substances you're using to protect your children can
be
\href{https://www.nytimes3xbfgragh.onion/2020/06/22/health/fda-Eskbiochem-toxic-hand-sanitizer-virus.html?searchResultPosition=1}{toxic};
the pandemic has meant increased
\href{https://www.healthychildren.org/English/health-issues/conditions/COVID-19/Pages/Keep-Hand-Sanitizer-Out-of-Childrens-Reach.aspx\#:~:text=Children\%20and\%20adults\%20also\%20have,after\%20repeated\%20use\%20on\%20skin.}{poisoning
incidents involving hand sanitizers}. Dr. Zonfrillo emphasized the
importance of what he called ``re-child-proofing the home, based on the
child's developmental age.''

\hypertarget{mental-health}{%
\subsection{Mental Health}\label{mental-health}}

Social isolation may be taking a toll, especially on children who suffer
from anxiety or depression, and on those who may not have been able to
get help and therapy virtually. On the other hand, as children begin to
interact more, whether in person or virtually, and even start school
again, doctors worry about bullying.

Dr. Haasz said this is a time for ``really keeping an eye out for
friends and family who have mental health concerns.'' To help keep
vulnerable children safe, it's again important to be sure the home is
injury-proofed, she said: ``Lock up anything they could use to hurt
themselves, even seemingly benign medications like Tylenol and
Benadryl.'' And if you're worried about your child's mental health, she
said, bring it up.

``You are not going to harm your child by asking them questions,'' she
said. ``If you are concerned about them, talk to them. If you're
concerned they're going to hurt themselves, bring them into the
hospital.''

\emph{Dr. Perri Klass is the author of the forthcoming book
``}\href{https://www.amazon.com/Good-Time-Be-Born-Children/dp/0393609995/ref=tmm_hrd_swatch_0?_encoding=UTF8\&qid=\&sr=}{\emph{A
Good Time to Be Born}}\emph{: How Science and Public Health Gave
Children a Future,'' on how our world has been transformed by the
radical decline of infant and child mortality.}

Advertisement

\protect\hyperlink{after-bottom}{Continue reading the main story}

\hypertarget{site-index}{%
\subsection{Site Index}\label{site-index}}

\hypertarget{site-information-navigation}{%
\subsection{Site Information
Navigation}\label{site-information-navigation}}

\begin{itemize}
\tightlist
\item
  \href{https://help.nytimes3xbfgragh.onion/hc/en-us/articles/115014792127-Copyright-notice}{©~2020~The
  New York Times Company}
\end{itemize}

\begin{itemize}
\tightlist
\item
  \href{https://www.nytco.com/}{NYTCo}
\item
  \href{https://help.nytimes3xbfgragh.onion/hc/en-us/articles/115015385887-Contact-Us}{Contact
  Us}
\item
  \href{https://www.nytco.com/careers/}{Work with us}
\item
  \href{https://nytmediakit.com/}{Advertise}
\item
  \href{http://www.tbrandstudio.com/}{T Brand Studio}
\item
  \href{https://www.nytimes3xbfgragh.onion/privacy/cookie-policy\#how-do-i-manage-trackers}{Your
  Ad Choices}
\item
  \href{https://www.nytimes3xbfgragh.onion/privacy}{Privacy}
\item
  \href{https://help.nytimes3xbfgragh.onion/hc/en-us/articles/115014893428-Terms-of-service}{Terms
  of Service}
\item
  \href{https://help.nytimes3xbfgragh.onion/hc/en-us/articles/115014893968-Terms-of-sale}{Terms
  of Sale}
\item
  \href{https://spiderbites.nytimes3xbfgragh.onion}{Site Map}
\item
  \href{https://help.nytimes3xbfgragh.onion/hc/en-us}{Help}
\item
  \href{https://www.nytimes3xbfgragh.onion/subscription?campaignId=37WXW}{Subscriptions}
\end{itemize}
