Sections

SEARCH

\protect\hyperlink{site-content}{Skip to
content}\protect\hyperlink{site-index}{Skip to site index}

\href{https://www.nytimes3xbfgragh.onion/section/world/middleeast}{Middle
East}

\href{https://myaccount.nytimes3xbfgragh.onion/auth/login?response_type=cookie\&client_id=vi}{}

\href{https://www.nytimes3xbfgragh.onion/section/todayspaper}{Today's
Paper}

\href{/section/world/middleeast}{Middle East}\textbar{}Israel Says It
Hit Bombers on Syrian Boundary

\url{https://nyti.ms/3i3nHEc}

\begin{itemize}
\item
\item
\item
\item
\item
\end{itemize}

Advertisement

\protect\hyperlink{after-top}{Continue reading the main story}

Supported by

\protect\hyperlink{after-sponsor}{Continue reading the main story}

\hypertarget{israel-says-it-hit-bombers-on-syrian-boundary}{%
\section{Israel Says It Hit Bombers on Syrian
Boundary}\label{israel-says-it-hit-bombers-on-syrian-boundary}}

The military released grainy, edited footage of the ambush of what it
said were four militants planting explosives in the area, amid
heightened tensions along Israel's northern frontiers.

\includegraphics{https://static01.graylady3jvrrxbe.onion/images/2020/08/03/world/03israel-border1a/merlin_175254969_e866868b-9fa8-45a2-b739-d9e6d0dcd58e-articleLarge.jpg?quality=75\&auto=webp\&disable=upscale}

\href{https://www.nytimes3xbfgragh.onion/by/isabel-kershner}{\includegraphics{https://static01.graylady3jvrrxbe.onion/images/2018/10/12/multimedia/author-isabel-kershner/author-isabel-kershner-thumbLarge.png}}

By \href{https://www.nytimes3xbfgragh.onion/by/isabel-kershner}{Isabel
Kershner}

\begin{itemize}
\item
  Aug. 3, 2020
\item
  \begin{itemize}
  \item
  \item
  \item
  \item
  \item
  \end{itemize}
\end{itemize}

JERUSALEM --- Israeli troops ambushed four militants as they planted
bombs along Israel's boundary with Syria around midnight, the Israeli
military said early Monday, amid heightened tensions along Israel's
frontiers.

Late Monday, the military said its fighter jets, attack helicopters and
aircraft had struck military targets in southern Syria belonging to the
Syrian Armed Forces in response to the attempt to lay explosives the
night before.

The militants, whose identity or affiliation were not immediately known,
were assumed to have been killed on the spot by a combination of ground
and air fire, said Lt. Col. Jonathan Conricus, an army spokesman, adding
that troops were scanning the area.

No casualties were reported on the Israeli side of the lines, where
special forces had been lying in wait, aided by observation teams and
permanent cameras, after the military noticed irregular activity at
night during the past week, Colonel Conricus said.

While the prevailing assessment in Israel was that a small and local
pro-Iranian group was probably behind the attempt to lay the explosives,
Israel said it held Syria responsible for any aggression from its
territory.

Israel rarely acknowledges specific strikes on Syrian soil when the
targets are Iranian weapons stores or shipments. But this time the
military said in a statement that it had struck observation posts and
intelligence-collection systems, antiaircraft artillery facilities and
command and control systems in Syrian Army bases.

Syria's state-run news agency, SANA, quoted an unidentified military
source as saying Israeli drones hit some of their positions near the
boundary, resulting in material losses only.

Israeli forces have been on elevated alert in the north in recent weeks,
and particularly over the past week, after Israel said it had
\href{https://www.nytimes3xbfgragh.onion/2020/07/27/world/middleeast/israel-hezbollah-lebanon-fighting.html?searchResultPosition=1}{thwarted
a raid} by a Hezbollah ``terrorist squad'' armed with assault rifles in
a disputed area along its northern border with Lebanon.

The latest event took place near Tel Fares, in the southern Golan
Heights, in the same spot at which Israel had operated a field hospital
to provide medical care for Syrian civilians during the Syrian civil war
under the so-called
\href{https://www.nytimes3xbfgragh.onion/2017/07/20/world/middleeast/israel-syria-humanitarian-aid.html}{Operation
Good Neighbor}, which is now defunct.

The location is within Israeli-controlled territory, just across from a
reinforced steel security fence but west of the old, internationally
recognized cease-fire line that forms Israel's boundary with Syria.

The military issued
\href{https://spokesperson.gincher.net/releases/gZ61jQcawD3d1zbmZg-4-}{a
32-second, grainy, edited video} showing four silhouetted figures
approaching the fence then being struck down by fire as they retreated
at an amble over the rough terrain.

Colonel Conricus said that the target of the improvised explosive
devices that the militants had planted would probably have been Israeli
soldiers who occasionally patrol the area, where there is an Israeli
military post that is not permanently staffed.

Israel held Syria responsible for any hostile activity emanating from
its territory, he said. Syria did not immediately comment on the Israeli
reports.

\includegraphics{https://static01.graylady3jvrrxbe.onion/images/2020/08/03/world/03israel-border3/merlin_175234347_ef60ee00-7419-4941-97a7-c29e237f95f2-articleLarge.jpg?quality=75\&auto=webp\&disable=upscale}

No special restrictions were placed on Israeli civilians living in the
Golan Heights or northern Israel, indicating that the military was not
anticipating any immediate retaliation.

The events overnight came after what the Israeli military said was a
thwarted raid by Hezbollah, the Lebanese militant organization, last
Monday.

Israel said it had repelled the squad with small arms, tank and
artillery fire, and that the men had fled back into Lebanon, while
firing back, but Hezbollah denied that account. Hezbollah said that
weapons had only been fired from the Israeli side.

Israel had cast the raid as an expected, and failed, retaliation attempt
from Hezbollah for the killing of one of its operatives in a strike in
Syria about a week before that was attributed to Israel. Hezbollah said
that retaliation for the killing in Syria was still coming.

The Israeli military did not release any video of the thwarted raid,
raising questions about what exactly had happened. Israeli analysts said
that the military had deliberately refrained from killing the Hezbollah
squad, because it did not want to inflame the situation, and that both
sides were calibrating their actions with that in mind.

The military did not explain why the Hezbollah squad was allowed to
escape but said its decisions stemmed from a range of strategic
considerations.

The decision not to release video or evidence of findings on the ground
from that episode may have been intended to avoid embarrassment on both
sides: for Hezbollah, in the form of evidence of a failed operation; and
for Israel, in the sight of the militants fleeing unharmed.

Prime Minister Benjamin Netanyahu of Israel and his defense minister,
Benny Gantz, both warned at the time that Israel would act forcefully to
prevent any entrenchment by Iran or its proxies along its northern
borders and was prepared for any scenario.

Iran and Hezbollah became involved in Syria to support the government of
President Bashar al-Assad in its battle against insurgents. Israel has
accused Iran and its proxies of trying to establish forward bases
against Israel across the lines in the Golan Heights and of transferring
advanced weapons to Hezbollah in Lebanon.

Image

A section of the cease-fire line between Syria and Israel in the Golan
Heights.Credit...Atef Safadi/EPA, via Shutterstock

Colonel Conricus said early Monday that Israel could not confirm any
link between the group hit in the latest attack with Hezbollah or with
the Iranians, adding that many smaller jihadist factions were active in
the area, on Syrian soil.

On July 24, Israel carried out airstrikes near its border with Syria on
targets belonging to the Syrian Army after munitions had been fired
toward the Israeli-controlled part of the Golan Heights, according to
the Israeli military. The Syrian military reported that two of its
members had been wounded.

Adam Rasgon contributed reporting.

Advertisement

\protect\hyperlink{after-bottom}{Continue reading the main story}

\hypertarget{site-index}{%
\subsection{Site Index}\label{site-index}}

\hypertarget{site-information-navigation}{%
\subsection{Site Information
Navigation}\label{site-information-navigation}}

\begin{itemize}
\tightlist
\item
  \href{https://help.nytimes3xbfgragh.onion/hc/en-us/articles/115014792127-Copyright-notice}{©~2020~The
  New York Times Company}
\end{itemize}

\begin{itemize}
\tightlist
\item
  \href{https://www.nytco.com/}{NYTCo}
\item
  \href{https://help.nytimes3xbfgragh.onion/hc/en-us/articles/115015385887-Contact-Us}{Contact
  Us}
\item
  \href{https://www.nytco.com/careers/}{Work with us}
\item
  \href{https://nytmediakit.com/}{Advertise}
\item
  \href{http://www.tbrandstudio.com/}{T Brand Studio}
\item
  \href{https://www.nytimes3xbfgragh.onion/privacy/cookie-policy\#how-do-i-manage-trackers}{Your
  Ad Choices}
\item
  \href{https://www.nytimes3xbfgragh.onion/privacy}{Privacy}
\item
  \href{https://help.nytimes3xbfgragh.onion/hc/en-us/articles/115014893428-Terms-of-service}{Terms
  of Service}
\item
  \href{https://help.nytimes3xbfgragh.onion/hc/en-us/articles/115014893968-Terms-of-sale}{Terms
  of Sale}
\item
  \href{https://spiderbites.nytimes3xbfgragh.onion}{Site Map}
\item
  \href{https://help.nytimes3xbfgragh.onion/hc/en-us}{Help}
\item
  \href{https://www.nytimes3xbfgragh.onion/subscription?campaignId=37WXW}{Subscriptions}
\end{itemize}
