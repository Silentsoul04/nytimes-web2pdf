Sections

SEARCH

\protect\hyperlink{site-content}{Skip to
content}\protect\hyperlink{site-index}{Skip to site index}

\href{https://www.nytimes3xbfgragh.onion/section/arts/music}{Music}

\href{https://myaccount.nytimes3xbfgragh.onion/auth/login?response_type=cookie\&client_id=vi}{}

\href{https://www.nytimes3xbfgragh.onion/section/todayspaper}{Today's
Paper}

\href{/section/arts/music}{Music}\textbar{}A Black Pianist Helped Birth
Bossa Nova. His Story Is Rarely Told.

\url{https://nyti.ms/3kbh3gP}

\begin{itemize}
\item
\item
\item
\item
\item
\end{itemize}

Advertisement

\protect\hyperlink{after-top}{Continue reading the main story}

Supported by

\protect\hyperlink{after-sponsor}{Continue reading the main story}

\hypertarget{a-black-pianist-helped-birth-bossa-nova-his-story-is-rarely-told}{%
\section{A Black Pianist Helped Birth Bossa Nova. His Story Is Rarely
Told.}\label{a-black-pianist-helped-birth-bossa-nova-his-story-is-rarely-told}}

Johnny Alf has always been revered by Antônio Carlos Jobim and João
Gilberto, but his legacy remains obscure, even among Brazilians.

\includegraphics{https://static01.graylady3jvrrxbe.onion/images/2020/07/28/arts/00alf1/merlin_175044852_f2214432-832a-435a-b0d5-394264c39b2c-articleLarge.jpg?quality=75\&auto=webp\&disable=upscale}

By Beatriz Miranda

\begin{itemize}
\item
  Aug. 4, 2020, 10:49 a.m. ET
\item
  \begin{itemize}
  \item
  \item
  \item
  \item
  \item
  \end{itemize}
\end{itemize}

\href{https://www.youtube.com/watch?v=Zt6JuURjuzg}{João Gilberto's
landmark ``Chega de Saudade''} is widely considered bossa nova's first
album. But about seven years before its 1959 release, a Brazilian
musician known as Johnny Alf composed ``Rapaz de Bem.'' The song
incorporated several elements that became hallmarks of the genre: a
linear melody, a gentle way of singing, a series of unconventional key
changes, a rhythmic dissociation of drums and bass.

Alf, a pianist, composer and singer, dared to blend classical, popular,
foreign and local music references into his own creative process. The
source of his inspiration lies in the music of Chopin, Debussy, Nat King
Cole, Stan Kenton, and the Brazilian notables Custódio Mesquita and
Francisco Alves. His music attracted the most avant-garde ears to the
piano bars of the Copacabana neighborhood in Rio, where he regularly
performed in the early '50s.

Antônio Carlos Jobim, who was also known as Tom, and Gilberto --- now
known as two of the most famous names in bossa nova --- were among the
faithful habitués who were stunned by songs like
``\href{https://www.youtube.com/watch?v=c0rJQ02Fjyo}{Rapaz de Bem},''
one of Alf's first professional compositions and, the radio host and
music producer Ramalho Neto argues in the 1965 book ``Historinha do
Desafinado,'' the actual first bossa nova song.

It's been 10 years since
\href{https://www.nytimes3xbfgragh.onion/2010/03/12/arts/music/12alf.html}{Alf's
death from cancer} at 80. In his six-decade career, he composed more
than 80 songs, some recorded with giants like
\href{https://www.youtube.com/watch?v=Sr20erfxSrk}{Caetano Veloso} and
\href{https://www.youtube.com/watch?v=Z9CfAkqVKvs}{Chico Buarque}. But
while his pivotal role is well-accepted by scholars, artists and
critics, why isn't he more widely known to listeners as a pioneer of one
of the most appreciated Brazilian genres around the world?

Like his music, Alf's story was complex. He was born Alfredo José da
Silva to a poor Black housemaid. His father died when he was 3, and he
owed much of his musical education to his mother's employers~--- an
upper-middle-class family who paid for his piano lessons and helped
raise him, though they later strongly disapproved of his nightclub
career. (They expected Alf to become an accountant or an English
teacher.)

But music was intrinsic to Alf's life long before his debut in the clubs
of Copacabana. When he was only a high school student, he was invited to
play the piano at the Brazilian-United States Institute (IBEU), in Rio's
downtown. Sessions there earned him his first radio gig and generated
his artist name: ``Johnny'' was suggested by a fellow student for being
a popular name in America, and ``Alf'' was his nickname at the school.

In 1949, Alf, a connoisseur of American music, joined the Sinatra-Farney
Fan Club, a place to appreciate, discuss and play the music of Frank
Sinatra and Dick Farney, a Brazilian pianist and singer inspired by the
American star. It was here that Alf's music began to flourish, and his
sound developed a striking modernity. According to José Domingos
Raffaelli, a music critic and former Sinatra-Farney attendee, Alf used
to perform for hours as listeners begged him to play on. In 1952, thanks
to a recommendation from Farney himself, Alf got his first job at a
Copacabana piano bar called Cantina do César.

Three years later, Alf moved to São Paulo with the promise of making a
better living. By then he was an independent musician in his mid-20s
with no family support. In 1958, when bossa nova was gaining momentum in
Rio, Alf was 270 miles away ---~far enough to miss out on the movement.

But distance wasn't the only barrier Alf faced. João Carlos Rodrigues,
the author of ``Johnny Alf: Duas ou Três Coisas Que Você Não Sabe''
(``Johnny Alf: Two or Three Things You Don't Know''), believes the
sophistication of Alf's music was a significant obstacle, because the
radio and record companies preferred a more ``easily digestible'' music.

Marcos Napolitano, a social history professor at the University of São
Paulo who researches music movements in Brazil, agreed. ``It is
undeniable that Alf's work was more subtle, intimate and
sophisticated,'' he said in an email interview, ``apart from him being a
timid, standoffish performer.''

Alf's personality offstage --- where he boldly insisted on creative
freedom ---~also limited him. ``No one called me to record because I
only did what I wanted,'' he told Rodrigues.

Nelson Valencia, who managed Alf's career for over 20 years, said in a
phone interview that Alf didn't naturally chase opportunities, but
rather waited for them to appear. He added that his client also could
have pushed harder to explore the growing Japanese and American markets.

In 1963, Alf did record an album in English (with Jobim's compositions)
that has never been released. In the late '70s, when Sarah Vaughan, one
of his greatest idols, invited him to tour the United States, Alf was
warned to refuse it by his mãe de santo --- the priestess of Umbanda,
the Afro-Brazilian religion to which he was devoted --- and he didn't
go.

Alf was also Black, poor and gay in an industry aiming to attract
affluent white audiences to the genre. Napolitano said that in the late
'50s, the music business saw in bossa nova a chance to compete with
American rock 'n' roll, and Valencia acknowledged that race and class
inequalities restricted Alf's rise: ``There was a movement to promote
Tom Jobim, who was rich, white, young, good-looking,'' he said, and
added that with Alf's abilities, ``maybe he was someone who could
overshadow Tom Jobim.''

Alf's talent, Rodrigues said, could also have been a problem for Aloysio
de Oliveira, an influential bossa nova producer who was particularly
interested in promoting Jobim and Gilberto in the United States. Alf was
excluded from the landmark
``\href{https://www.youtube.com/watch?v=JfVPnYSLmvA}{Bossa Nova at
Carnegie Hall}'' concert, organized by Oliveira and held in New York in
1962.

Alaíde Costa, Alf's favorite singer, said that racism in bossa nova has
always been veiled, to the point where Black people, like her and Alf,
didn't even realize they were facing discrimination. ``When the movement
began, I was already a professional. I used to be invited to the
meetings because I could help the movement somehow,'' she said in a
phone interview. ``But when the bossa nova boomed, I felt I was no
longer necessary.'' She added that she regrets not having been part of
the movement in a larger way.

While Alf never achieved stardom, the genre's legends always saw him as
a musical guru. Jobim not only used to call him Genialf (a combination
of ``genius'' and ``Alf''), but was so inspired by ``Rapaz de Bem'' that
he composed
``\href{https://www.youtube.com/watch?v=OUK5ay_8G-w}{Desafinado
(Off-Key)},'' one of bossa nova's most famous songs. When Gilberto
realized that Alf's unusual syncopations reminded him of the beat of the
tamborim (a samba percussion instrument), he said he had finally found
what he was looking for.

Alf, however, has remained a few steps removed from fame. He was,
Valencia, said, ``his own hero and own villain.''

Advertisement

\protect\hyperlink{after-bottom}{Continue reading the main story}

\hypertarget{site-index}{%
\subsection{Site Index}\label{site-index}}

\hypertarget{site-information-navigation}{%
\subsection{Site Information
Navigation}\label{site-information-navigation}}

\begin{itemize}
\tightlist
\item
  \href{https://help.nytimes3xbfgragh.onion/hc/en-us/articles/115014792127-Copyright-notice}{©~2020~The
  New York Times Company}
\end{itemize}

\begin{itemize}
\tightlist
\item
  \href{https://www.nytco.com/}{NYTCo}
\item
  \href{https://help.nytimes3xbfgragh.onion/hc/en-us/articles/115015385887-Contact-Us}{Contact
  Us}
\item
  \href{https://www.nytco.com/careers/}{Work with us}
\item
  \href{https://nytmediakit.com/}{Advertise}
\item
  \href{http://www.tbrandstudio.com/}{T Brand Studio}
\item
  \href{https://www.nytimes3xbfgragh.onion/privacy/cookie-policy\#how-do-i-manage-trackers}{Your
  Ad Choices}
\item
  \href{https://www.nytimes3xbfgragh.onion/privacy}{Privacy}
\item
  \href{https://help.nytimes3xbfgragh.onion/hc/en-us/articles/115014893428-Terms-of-service}{Terms
  of Service}
\item
  \href{https://help.nytimes3xbfgragh.onion/hc/en-us/articles/115014893968-Terms-of-sale}{Terms
  of Sale}
\item
  \href{https://spiderbites.nytimes3xbfgragh.onion}{Site Map}
\item
  \href{https://help.nytimes3xbfgragh.onion/hc/en-us}{Help}
\item
  \href{https://www.nytimes3xbfgragh.onion/subscription?campaignId=37WXW}{Subscriptions}
\end{itemize}
