Sections

SEARCH

\protect\hyperlink{site-content}{Skip to
content}\protect\hyperlink{site-index}{Skip to site index}

\href{https://www.nytimes3xbfgragh.onion/section/arts}{Arts}

\href{https://myaccount.nytimes3xbfgragh.onion/auth/login?response_type=cookie\&client_id=vi}{}

\href{https://www.nytimes3xbfgragh.onion/section/todayspaper}{Today's
Paper}

\href{/section/arts}{Arts}\textbar{}Will Superblue Be the `Infinity
Room' Writ Large?

\url{https://nyti.ms/30rLZSj}

\begin{itemize}
\item
\item
\item
\item
\item
\item
\end{itemize}

Advertisement

\protect\hyperlink{after-top}{Continue reading the main story}

Supported by

\protect\hyperlink{after-sponsor}{Continue reading the main story}

\hypertarget{will-superblue-be-the-infinity-room-writ-large}{%
\section{Will Superblue Be the `Infinity Room' Writ
Large?}\label{will-superblue-be-the-infinity-room-writ-large}}

Art objects are a bore. People want multisensory ``experiences,'' the
more immersive the better. With JR, James Turrell, teamLab and more, a
new business venture funded by Marc Glimcher and Laurene Powell Jobs
hopes to deliver.

\includegraphics{https://static01.graylady3jvrrxbe.onion/images/2020/08/05/arts/04SUPERBLUE3/merlin_175177776_3b59712b-3803-4023-ad60-0eef78280def-articleLarge.jpg?quality=75\&auto=webp\&disable=upscale}

By Frank Rose

\begin{itemize}
\item
  Aug. 4, 2020Updated 12:54 p.m. ET
\item
  \begin{itemize}
  \item
  \item
  \item
  \item
  \item
  \item
  \end{itemize}
\end{itemize}

You could say the whole thing started with an argument over whether to
sell tickets.

It was late 2015, and Pace, the blue-chip New York gallery with outposts
in London and Beijing, was planning to open its new space in Silicon
Valley with a show by the Japanese art collective
\href{https://www.nytimes3xbfgragh.onion/2016/02/04/t-magazine/art/teamlab-living-digital-space-future-parks-pace-gallery-california.html}{teamLab}.
It seemed a good fit: The whole point of Pace Art + Technology, as the
new venue was called, was to bring art to the tech crowd, and teamLab's
wildly colorful, highly kinetic electronic environments are an immersive
celebration of art, science, technology and nature. But then the teamLab
people said, You're going to sell tickets, right?

Marc Glimcher, Pace's chief executive, was taken aback. ``I said, you
can't sell tickets,'' he recalled recently.

Why not? they asked.

Because, he replied, ``art galleries sell art, not tickets.''

Then you're telling us that you're only in business to sell art to the
ultrarich.

``No, no!''---and yet, he realized, they had a point: Art isn't always a
commodity. So they ended up selling tickets --- but more important, Mr.
Glimcher said last week, ``That was the kernel of a disruptive idea.''

This ``disruptive idea'' is poised to reach fruition with the
announcement Tuesday of a new venture that aims ** to reinvent how art
is shown. Superblue, as it is called, will open a series of experiential
art centers (EACs for short) that won't sell precious objects, as
conventional galleries do. They'll present art experiences: deep dives
into all-encompassing works by such artists as
\href{https://www.nytimes3xbfgragh.onion/2014/09/25/arts/design/jr-brings-ellis-islands-abandoned-hospital-to-life.html}{JR},
the French photographer who focuses on issues like migration,
displacement and imprisonment; and James Turrell, the celebrated Light
and Space artist whose massive
\href{https://www.nytimes3xbfgragh.onion/2013/06/21/arts/design/james-turrell-plays-with-color-at-the-guggenheim.html}{installation
at the Guggenheim Museum} seven years ago was described in The New York
Times as ``a meditative spectacle.''

\includegraphics{https://static01.graylady3jvrrxbe.onion/images/2020/08/05/arts/04superblue7/merlin_175177788_098ad61d-9732-4236-8e28-7b40e68e7d25-articleLarge.jpg?quality=75\&auto=webp\&disable=upscale}

Superblue plans to open first in Miami in December, in a formerly
abandoned industrial building across the street from the Rubell Museum,
one of the premier contemporary art collections.

At 50,000 square feet it will be big enough to present multiple artists
at once, and its shows will stay up for as long as 18 months, far longer
than a gallery exhibition. Ultimately, Mr. Glimcher, Superblue's
chairman, and \href{https://www.youtube.com/watch?v=_aC5tn6jKzk}{Christy
MacLear}, the cultural entrepreneur and strategist who will be its chief
executive, expect to open several such centers in the United States,
Europe and Asia. It's no accident that their chief operating officer,
Marcy Davis, comes from Cirque du Soleil, the troupe that disrupted the
circus --- before itself being pushed into bankruptcy
\href{https://www.nytimes3xbfgragh.onion/2020/05/17/world/canada/cirque-du-soleil-coronavirus-debt.html}{in
the wake of the coronavirus.}

\href{https://www.nytimes3xbfgragh.onion/2020/05/17/world/canada/cirque-du-soleil-coronavirus-debt.html}{Superblue
won't be the first imme}rsive art enterprise. A company called
\href{https://www.nytimes3xbfgragh.onion/2018/03/14/arts/artechouse-washington-dc-museum.html}{Artechouse}
has set up smaller-scale experiential art spaces in Washington, Miami
Beach and New York's Chelsea Market, and teamLab went on to partner with
a Japanese real-estate developer in
\href{https://borderless.teamlab.art/}{teamLab Borderless}, an enormous
showcase on Tokyo's waterfront that drew 2.3 million people in its first
year, more of them from the United States than from any other country
beyond Japan itself.

Both Artechouse and teamLab Borderless charge an admission fee, and
Superblue will do the same --- for something under \$40 in Miami, with
the artists sharing the proceeds. ``You could say it's an evolution of
patronage from the collector to the public,'' Ms. MacLear said. ``From
the collector owning the work, to the public engaging directly with the
artist.''

Image

Es Devlin's ``Room 2022,'' an immersive installation in 2017 set inside
a Miami Beach hotel.Credit...Eugene Gologursky/Getty Images for American
Express Platinum

Beyond sharing ticket revenue with its artists, Superblue expects to
commission them to create new works, offering direct financial support
to get them started and in other cases helping them land commissions for
public artworks from cities, festivals and the like. The tab for such
works can run into the millions --- witness
\href{https://illuminate.org/projects/the-bay-lights/}{``The Bay
Lights,''} Leo Villareal's monumental light installation on the San
Francisco-Oakland Bay Bridge, which cost \$8 million to install in 2013
and another \$4 million to make permanent in 2016, not counting the
electric bill. Mr. Villareal is part of the Superblue stable --- ``we're
loosely calling it a network,'' Ms. MacLear said --- and like several
others, including JR, teamLab and Mr. Turrell, he is also represented by
Pace. Others are not, and according to Ms. MacLear the two businesses
are separate. How much it's all expected to cost appears to be a closely
guarded secret.

Immersive art installations have been around since at least the late
'50s, when emerging pop artists like Jim Dine and Claes Oldenburg set up
anarchic ``environments'' in the basement of the
\href{https://www.judson.org/arts}{Judson Memorial Church} in Greenwich
Village. By the mid-60s, Robert Rauschenberg and Andy Warhol were
teaming up with engineers at AT\&T's Bell Labs to create tech-infused
works that were meant to be experienced rather than admired. More
recently,
\href{https://www.nytimes3xbfgragh.onion/2013/12/02/arts/design/yayoi-kusamas-mirrored-room-at-david-zwirner-gallery.html}{Yayoi
Kusama's ``Infinity Mirrored Room''} at David Zwirner Gallery in Chelsea
and
\href{https://www.nytimes3xbfgragh.onion/video/arts/design/100000002254021/rain-room-at-the-barbican-center.html}{Random
International's ``Rain Room''} at the Museum of Modern Art had people
standing in line for hours to experience a few moments of --- was it
art? No matter. ``It felt like eternity,'' one Kusama fan told The
Times, referring not to the nearly three-hour wait he'd endured in
wintry weather but to the 45 seconds he got to spend amid the twinkling
lights of her walk-in-closet-size simulation of infinity.

Image

``Universe of Water Particles on a Rock Where People Gather,'' a kinetic
installation at teamLab Borderless, the group's showcase on the Tokyo
waterfront, attracted 2.3 million people.Credit...teamLab and Pace
Gallery

Yet even mega-galleries like Zwirner or Pace are ill-equipped to handle
such work. Their stock in trade is art that sells for seven figures or
more by the likes of Jeff Koons, Alexander Calder, Chuck Close, David
Hockney, Mark Rothko and Julian Schnabel. All the same, Mr. Glimcher
said, ``we would tend toward the installation and away from the object,
knowing that money was going out the door and not in the door. The way
we were going to get money back in the door was by selling paintings.''

Fortunately for Mr. Glimcher, collectors are still buying paintings ---
or they were until the coronavirus struck. (Like other galleries,
\href{https://www.artnews.com/art-news/news/pace-gallery-lay-offs-1202695286/}{Pace
has laid off}a substantial contingent of workers, and the immense new
headquarters it opened in Chelsea last September was closed for months
and has only recently reopened by appointment only.)

But for broader audiences, and younger people in particular, art objects
are no longer the draw they once were. This is part of a much bigger
shift in favor of immersive experiences and against consumerism in
general. For well over a decade, trend-watchers have noted a growing
preference for experiences over things. ``And this is doubly true with
the coronavirus,'' said B. Joseph Pine II, co-author with James H.
Gilmore of
\href{https://www.google.com/books/edition/The_Experience_Economy/5hs-tyRrSXMC?hl=en\&gbpv=0\&kptab=getbook}{``The
Experience Economy,''} a book that declared the crux of business today
to no longer be goods or services but experiences. The pandemic, he
added, ``makes us sit back and think, what makes us happy? What does
life mean? We've got enough stuff.''

For Pace, the road to experiences ran through Silicon Valley, where Mr.
Glimcher --- whose father, Arne Glimcher, now 82, founded Pace 60 years
ago and built it into a powerhouse --- forged a dense web of
connections. The most critical of these was the billionaire activist
Laurene Powell Jobs, the widow and heir of Apple co-founder Steve Jobs,
who joined Mr. Glimcher in funding the new venture through the Emerson
Collective, her investment vehicle for social change. Another key figure
was Mollie Dent-Brocklehurst, now a co-founder of Superblue, who set up
Pace's London outpost in 2010 and later headed Future/Pace, the
gallery's first attempt to break into experiential art. This eventually
led to an initiative called PaceX that was the immediate forerunner of
Superblue.

``Laurene told me, It isn't going to be called PaceX --- sorry,'' Mr.
Glimcher said. ``We're breaking all the rules, and we've got to come up
with something new.'' All this rule-breaking put him in mind of the
\href{https://www.nytimes3xbfgragh.onion/2018/10/24/arts/design/franz-marc-august-macke-neue-galerie.html}{Blue
Rider}, the radical art movement that sprang up in Germany before World
War I. One of his younger employees heard that and came back with
Superblue.

``I said, That's the worst thing I ever heard,'' Mr. Glimcher recalled,
``but I couldn't get it out of my head.'' He mentioned it to his
meditation teacher, Thom Knoles, a one-time protégé of Maharishi Mahesh
Yogi, who introduced Transcendental Meditation to the Beatles in the
'60s. According to Mr. Glimcher, Mr. Knoles told him that the word
Krishna is Sanskrit for ``superblue'' --- a liberal translation,
apparently, but no matter. Superblue it would be.

And the artists? ``I loved, loved, loved the idea,'' said JR, reached by
phone as he was returning from a shoot in the low-income suburbs of
Paris. ``I started right away working and brainstorming.'' Ralph Nauta
and Lonneke Gordijn of Studio Drift feel much the same. ``Luckily now
there's this energy and movement going on where our work fits in,'' said
Mr. Nauta, adding that ``there is no market'' for the kind of art they
make --- like the 300 light-bearing drones they sent swooping out across
central Rotterdam in May in celebration of health and freedom.

``I've never sold anything,'' said
\href{https://esdevlin.com/work/louis-vuitton-fw2020}{Es Devlin}, the
London-based Superblue artist who designed the astonishing set for Sam
Mendes's acclaimed production of ``The Lehman Trilogy''--- an
800-square-foot rotating glass box in which the rise and fall of an
American dynasty transpires as if in a giant bauble. More recently she
staged a choir made up of people dressed as portrait subjects from the
Louvre for the Louis Vuitton fall/winter women's fashion show, with
music by Bryce Dessner of the National. She doesn't have gallery
representation either, working with a manager instead.

``It's pretty clear I'm a theater animal,'' she added --- so at least
she has no problem with selling tickets. Will Superblue work? ``We'll
know more after we've done it,'' she concluded. ``Ask me this question
in a year.''

\emph{Frank Rose is the author of ``The Art of Immersion'' and faculty
director of Columbia University's executive education seminar in
Strategic Storytelling.}

Advertisement

\protect\hyperlink{after-bottom}{Continue reading the main story}

\hypertarget{site-index}{%
\subsection{Site Index}\label{site-index}}

\hypertarget{site-information-navigation}{%
\subsection{Site Information
Navigation}\label{site-information-navigation}}

\begin{itemize}
\tightlist
\item
  \href{https://help.nytimes3xbfgragh.onion/hc/en-us/articles/115014792127-Copyright-notice}{©~2020~The
  New York Times Company}
\end{itemize}

\begin{itemize}
\tightlist
\item
  \href{https://www.nytco.com/}{NYTCo}
\item
  \href{https://help.nytimes3xbfgragh.onion/hc/en-us/articles/115015385887-Contact-Us}{Contact
  Us}
\item
  \href{https://www.nytco.com/careers/}{Work with us}
\item
  \href{https://nytmediakit.com/}{Advertise}
\item
  \href{http://www.tbrandstudio.com/}{T Brand Studio}
\item
  \href{https://www.nytimes3xbfgragh.onion/privacy/cookie-policy\#how-do-i-manage-trackers}{Your
  Ad Choices}
\item
  \href{https://www.nytimes3xbfgragh.onion/privacy}{Privacy}
\item
  \href{https://help.nytimes3xbfgragh.onion/hc/en-us/articles/115014893428-Terms-of-service}{Terms
  of Service}
\item
  \href{https://help.nytimes3xbfgragh.onion/hc/en-us/articles/115014893968-Terms-of-sale}{Terms
  of Sale}
\item
  \href{https://spiderbites.nytimes3xbfgragh.onion}{Site Map}
\item
  \href{https://help.nytimes3xbfgragh.onion/hc/en-us}{Help}
\item
  \href{https://www.nytimes3xbfgragh.onion/subscription?campaignId=37WXW}{Subscriptions}
\end{itemize}
