Sections

SEARCH

\protect\hyperlink{site-content}{Skip to
content}\protect\hyperlink{site-index}{Skip to site index}

\href{https://www.nytimes3xbfgragh.onion/section/arts/television}{Television}

\href{https://myaccount.nytimes3xbfgragh.onion/auth/login?response_type=cookie\&client_id=vi}{}

\href{https://www.nytimes3xbfgragh.onion/section/todayspaper}{Today's
Paper}

\href{/section/arts/television}{Television}\textbar{}Sam Jay: A Comic
Who Belongs to No Camp

\url{https://nyti.ms/3i3tE3V}

\begin{itemize}
\item
\item
\item
\item
\item
\item
\end{itemize}

\href{https://www.nytimes3xbfgragh.onion/spotlight/at-home?action=click\&pgtype=Article\&state=default\&region=TOP_BANNER\&context=at_home_menu}{At
Home}

\begin{itemize}
\tightlist
\item
  \href{https://www.nytimes3xbfgragh.onion/2020/08/03/well/family/the-benefits-of-talking-to-strangers.html?action=click\&pgtype=Article\&state=default\&region=TOP_BANNER\&context=at_home_menu}{Talk:
  To Strangers}
\item
  \href{https://www.nytimes3xbfgragh.onion/2020/08/01/at-home/coronavirus-make-pizza-on-a-grill.html?action=click\&pgtype=Article\&state=default\&region=TOP_BANNER\&context=at_home_menu}{Make:
  Grilled Pizza}
\item
  \href{https://www.nytimes3xbfgragh.onion/2020/07/31/arts/television/goldbergs-abc-stream.html?action=click\&pgtype=Article\&state=default\&region=TOP_BANNER\&context=at_home_menu}{Watch:
  'The Goldbergs'}
\item
  \href{https://www.nytimes3xbfgragh.onion/interactive/2020/at-home/even-more-reporters-editors-diaries-lists-recommendations.html?action=click\&pgtype=Article\&state=default\&region=TOP_BANNER\&context=at_home_menu}{Explore:
  Reporters' Google Docs}
\end{itemize}

Advertisement

\protect\hyperlink{after-top}{Continue reading the main story}

Supported by

\protect\hyperlink{after-sponsor}{Continue reading the main story}

On Comedy

\hypertarget{sam-jay-a-comic-who-belongs-to-no-camp}{%
\section{Sam Jay: A Comic Who Belongs to No
Camp}\label{sam-jay-a-comic-who-belongs-to-no-camp}}

With a deadpan glare, the stand-up takes shots at everyone and won't say
something funny just for a laugh. Now she's on the cusp of breaking out,
thanks to a Netflix special.

\includegraphics{https://static01.graylady3jvrrxbe.onion/images/2020/08/05/arts/04sam-jay1/04sam-jay1-articleLarge-v2.jpg?quality=75\&auto=webp\&disable=upscale}

\href{https://www.nytimes3xbfgragh.onion/by/jason-zinoman}{\includegraphics{https://static01.graylady3jvrrxbe.onion/images/2018/12/10/multimedia/author-jason-zinoman/author-jason-zinoman-thumbLarge.png}}

By \href{https://www.nytimes3xbfgragh.onion/by/jason-zinoman}{Jason
Zinoman}

\begin{itemize}
\item
  Aug. 4, 2020Updated 12:44 p.m. ET
\item
  \begin{itemize}
  \item
  \item
  \item
  \item
  \item
  \item
  \end{itemize}
\end{itemize}

The stand-up comic \href{https://samjaycomic.com/}{Sam Jay} calls
herself a late bloomer.

``Got my period late, got my titties late,'' she said in her Long Island
City apartment, chuckling as she counted off on her fingers more
evidence: She didn't discover her homosexuality until her mid-20s. And
while she had tried stand-up once before, she didn't really get into
comedy until 29.

Comedy is like tennis: It helps to start young, which is why so many of
its superstars (Eddie Murphy, Adam Sandler, Dave Chappelle) began their
careers when they were teenagers. But Jay, long a fan and student of
stand-up, provides a good argument for the benefits of holding off. ``I
needed the confidence of a life behind me to know what I wanted to
say,'' she said.

Now 38, she is one of the most exciting provocateurs in comedy right now
because she tackles the hottest button issues without dogma. She belongs
to no camp. She takes shots at President Trump, but also tells you that
America is not better than him. Her takes on \#MeToo, the racism of
``white man ambition'' and transgender identity reflect an eccentric
mind working through issues, her startlingly funny jackhammer punch
lines emerging from a deadpan glare.

Her comedy is tough to pigeonhole: profane and heady, aiming for belly
laughs, but never seeming desperate for them. Jay's thoughtful bits have
a cynical edge, one earned from a few lifetimes of tragedy already. Her
late start gave her career an urgency that has led to this moment, when
she's on the verge of a major breakout.

Her debut hour, \href{https://www.youtube.com/watch?v=D2EcvFfzu-o}{``3
in the Morning,''} which premiered Tuesday on Netflix, is one of the
last specials to be shot before the pandemic. Jay filmed it Feb. 22 and
pointedly tried to dig into the big issues of the day, but inevitably it
suddenly looks less topical, since she doesn't address the virus, the
death of George Floyd or any protests. On her seventh-floor balcony,
where we socially distanced, I asked through a mask if this absence
bothered her. Jay paused, letting the silence sit long enough for it to
be a little uncomfortable --- a place she's perfectly comfortable in.

As the question about topicality hung there, she went to get a drink and
by the time she returned, she suggested --- on the advice of her
girlfriend, Yanise Monet Vincent --- we go inside because it was
scorching hot. By the time we sat down, Jay conceded some regret:
``Damn,'' she said, with gravel in her voice. ``I hope my special
doesn't come out and people think, as a Black person, I just chose to
not speak on this.''

If Jay has a stage persona distinct from her offstage one, I couldn't
detect it after talking to her for three hours. She projects a steely
presence that periodically shifts into jarring vulnerability. Discussing
the transition from performing every night at clubs to going weeks
without leaving the apartment, she said she appreciated the break from
the grind, but news of police violence had sometimes sent her spiraling
into hopelessness: ``There's definitely days when me and my girl are
just sad, when I've just cried all day.''

\includegraphics{https://static01.graylady3jvrrxbe.onion/images/2020/08/05/arts/04sam-jay2/merlin_175173609_bcf28411-26dc-4be0-a2f4-9275f9ce1e60-articleLarge.jpg?quality=75\&auto=webp\&disable=upscale}

At the core of her new special is an idea that true freedom means
standing out from the crowd. She starts with a self-deprecating joke
about how despite being gay she used to have sex with men as a way to
fit in with her friends. And then she describes how she felt out of step
with white lesbians, before moving into a celebration of
\href{https://www.cheatsheet.com/entertainment/these-are-the-most-bizarre-things-jaden-smith-ever-said.html/}{Jaden
Smith}. ``Finally we got a weird Black kid,'' she says, listing his
eccentricities, including dressing as a shoeless Batman and singing
about the stars. Marveling at his choices, she delivers her punchline
with awe: ``That's Martin Luther King's dream.''

More so than most comics, she has a healthy skepticism of audience
response, seeing dangers in pandering to it. ``You can't just be up
there for them because once they control you, they will lead you
anywhere,'' she said.

As an experiment, Jay even hosted several stand-up shows
\href{https://thestandnyc.com/shows/show/1129/20191108-1159-pm-nycf-presents-black-out-comedy-in-the-dark-with-sam-jay}{completely
in the dark} as a way in part to liberate comics from feedback. (``You
can stop caring about them because you can't see them.'')

Growing up in the projects in Boston, Jay said, she was a weird Black
kid. She was alert to how listening to Foo Fighters and Alice in Chains
in her room could be viewed with suspicion by her hip-hop-loving
brothers. ``I had a friend who liked `Seinfeld' and we used to make fun
of him so bad,'' she said, pointing out that in her set you were
supposed to prefer
\href{https://theundefeated.com/features/martin-lawrences-martin-20-years-later/}{``Martin.''}
``There's lines in the sand, bro. But I always hung out with kids like
that.''

Jay was only 16 when her mother died of lupus (her father had already
died). They were close. Her mother regularly took her to plays and
museums, and was a comedy fan as well (especially of George Carlin); she
also taught Jay to think for herself. But when her mother got sick and
became bedridden, Jay said, she responded by growing distant, until one
day her mother confronted her. ``Are you mad at me?'' she asked her. Jay
told her no.

``She said you don't have to lie,'' Jay said. ``It's OK to be mad at me.
I'm your mother and I'm supposed to do certain things, so don't feel
guilty. I cried and she held me. Within six or seven months, she was
dead.''

Jay paused to consider what her life would be like if her mother had not
said that. ``Who knows what trajectory that would have set me on? She
had that foresight to relieve me of that,'' she said, getting choked up.
``She was an impeccable parent.''

When Jay herself received a lupus diagnosis at age 20, she became
obsessed with and terrified by death, a fear that abated only recently
in part due to some professional success, but also what she described as
thinking through it. ``When I'm bothered by something like that, I need
to confront it mentally,'' she said. ``I need to sit in it. And I
remember talking to my aunt, who said: The more you try to control
something like that the less you will. The real control is acceptance.''

Jay's special, which has the barreling energy and intimacy of a bar
fight, only appears more political than personal. It's just as much
about her attitude toward death and her relationship with her
girlfriend. Jay ends with a story from her childhood about extreme
maternal love that serves as a kind of tribute.

Jay began performing after working a series of jobs (Starbucks, Best
Buy) she had no passion for. ``Stand-up wasn't risky at all,'' she said.
``I was working in the mail room and the highest thing I could become
was head in charge of mail. You have to love something enough to work
hard and stand-up is the only thing I love like that. It was the only
option.''

She knew early on what kind of comic she wanted to be. ``I don't say
anything just to be funny,'' she said, adding that she had to believe
it. She also didn't like segues between jokes or even callbacks, even
though both are common and effective. She thought they seemed fake.

Jak Knight, another comic who is a good friend, said of Jay with
admiration, ``There's no moving her unless she moves herself.''

The first time I saw Jay onstage, she told a couple of jokes to a
sparse, lethargic club crowd, shook her head and without a note of
anger, walked off the stage. ``I don't like to feel like the help up
there,'' she said reflecting on the episode. Audience members are the
customers, she conceded, but she does not believe every crowd is the
same. ``Comics put that on themselves,'' she says. ```If the audience
isn't good, it's my fault.' Nah. Sometimes they just suck.''

Jay even hesitated when she was offered a job as a writer on
\href{https://www.nbc.com/saturday-night-live/cast/sam-jay-292972}{``Saturday
Night Live,''} since she had no experience or particular ambition in
sketch comedy. When asked if people advised her to do it, her girlfriend
in the next room shouted, ``Yes!,'' then laughed. Along with her vital
counsel, Jay also said what informed her decision was recalling the time
in high school when Oberlin College tried to recruit her but she said no
because it seemed ``too whitey-tighty.'' She regretted it. She joined
``S.N.L.'' in 2017, and while adjusting has been difficult, frequently
getting sketches cut (``I live in the cut zone''), she also wrote
several pieces that were broadcast, including
\href{https://www.nbc.com/saturday-night-live/video/black-jeopardy-velvet-jones/4088748}{a
Velvet Jones one for Eddie Murphy}.

When we spoke over Zoom a week and half later, the thought occurred to
me that maybe the fact that Jay's special does not mention the most
recent current events is actually on brand, another example of her
refusal to follow the crowd. When I suggested this, she didn't seem
impressed, shifting subjects.

If she could shoot it now, what would she say about the recent protests?
Jay was torn. She said she would probably go at it from a couple of
angles, since she's inspired by the young people in the streets, but
also hates that the answer for some people is still in throwing ``a
Molotov cocktail.'' Then she thought some more and shifted course,
saying that sometimes she saw white society as an abusive boyfriend and
Black people just need to move out.

``I am at a point where I don't even believe in therapy for Black people
because you can't progress when you're still living with your abuser,''
she said, sounding like she was working out a bit for a set that isn't
likely to happen anytime soon. ``There's no situation where a woman's
getting beat by her husband and they're like: `Go to therapy.''' Then
she added, ``Just get out of the house.''

Advertisement

\protect\hyperlink{after-bottom}{Continue reading the main story}

\hypertarget{site-index}{%
\subsection{Site Index}\label{site-index}}

\hypertarget{site-information-navigation}{%
\subsection{Site Information
Navigation}\label{site-information-navigation}}

\begin{itemize}
\tightlist
\item
  \href{https://help.nytimes3xbfgragh.onion/hc/en-us/articles/115014792127-Copyright-notice}{©~2020~The
  New York Times Company}
\end{itemize}

\begin{itemize}
\tightlist
\item
  \href{https://www.nytco.com/}{NYTCo}
\item
  \href{https://help.nytimes3xbfgragh.onion/hc/en-us/articles/115015385887-Contact-Us}{Contact
  Us}
\item
  \href{https://www.nytco.com/careers/}{Work with us}
\item
  \href{https://nytmediakit.com/}{Advertise}
\item
  \href{http://www.tbrandstudio.com/}{T Brand Studio}
\item
  \href{https://www.nytimes3xbfgragh.onion/privacy/cookie-policy\#how-do-i-manage-trackers}{Your
  Ad Choices}
\item
  \href{https://www.nytimes3xbfgragh.onion/privacy}{Privacy}
\item
  \href{https://help.nytimes3xbfgragh.onion/hc/en-us/articles/115014893428-Terms-of-service}{Terms
  of Service}
\item
  \href{https://help.nytimes3xbfgragh.onion/hc/en-us/articles/115014893968-Terms-of-sale}{Terms
  of Sale}
\item
  \href{https://spiderbites.nytimes3xbfgragh.onion}{Site Map}
\item
  \href{https://help.nytimes3xbfgragh.onion/hc/en-us}{Help}
\item
  \href{https://www.nytimes3xbfgragh.onion/subscription?campaignId=37WXW}{Subscriptions}
\end{itemize}
