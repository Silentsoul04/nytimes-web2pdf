Sections

SEARCH

\protect\hyperlink{site-content}{Skip to
content}\protect\hyperlink{site-index}{Skip to site index}

\href{https://www.nytimes3xbfgragh.onion/section/books}{Books}

\href{https://myaccount.nytimes3xbfgragh.onion/auth/login?response_type=cookie\&client_id=vi}{}

\href{https://www.nytimes3xbfgragh.onion/section/todayspaper}{Today's
Paper}

\href{/section/books}{Books}\textbar{}In `Luster,' a Young Woman Moves
in With Her Lover --- and His Family

\url{https://nyti.ms/3i6LH9h}

\begin{itemize}
\item
\item
\item
\item
\item
\end{itemize}

Advertisement

\protect\hyperlink{after-top}{Continue reading the main story}

Supported by

\protect\hyperlink{after-sponsor}{Continue reading the main story}

\href{/column/books-of-the-times}{Books of The Times}

\hypertarget{in-luster-a-young-woman-moves-in-with-her-lover--and-his-family}{%
\section{In `Luster,' a Young Woman Moves in With Her Lover --- and His
Family}\label{in-luster-a-young-woman-moves-in-with-her-lover--and-his-family}}

By \href{https://www.nytimes3xbfgragh.onion/by/parul-sehgal}{Parul
Sehgal}

\begin{itemize}
\item
  Aug. 4, 2020
\item
  \begin{itemize}
  \item
  \item
  \item
  \item
  \item
  \end{itemize}
\end{itemize}

\includegraphics{https://static01.graylady3jvrrxbe.onion/images/2020/08/05/books/04BOOKLEILANI1/04BOOKLEILANI1-articleLarge.jpg?quality=75\&auto=webp\&disable=upscale}

Buy Book ▾

\begin{itemize}
\tightlist
\item
  \href{https://www.amazon.com/gp/search?index=books\&tag=NYTBSREV-20\&field-keywords=Luster+Raven+Leilani}{Amazon}
\item
  \href{https://du-gae-books-dot-nyt-du-prd.appspot.com/buy?title=Luster\&author=Raven+Leilani}{Apple
  Books}
\item
  \href{https://www.anrdoezrs.net/click-7990613-11819508?url=https\%3A\%2F\%2Fwww.barnesandnoble.com\%2Fw\%2F\%3Fean\%3D9780374194321}{Barnes
  and Noble}
\item
  \href{https://www.anrdoezrs.net/click-7990613-35140?url=https\%3A\%2F\%2Fwww.booksamillion.com\%2Fp\%2FLuster\%2FRaven\%2BLeilani\%2F9780374194321}{Books-A-Million}
\item
  \href{https://bookshop.org/a/3546/9780374194321}{Bookshop}
\item
  \href{https://www.indiebound.org/book/9780374194321?aff=NYT}{Indiebound}
\end{itemize}

When you purchase an independently reviewed book through our site, we
earn an affiliate commission.

You may know of the hemline theory --- the idea that skirt lengths
fluctuate with the stock market, rising in boom times and growing longer
in recessions.

Perhaps publishing has a parallel; call it the blurb theory. The more
strained our circumstances, the more manic the publicity machine, the
more breathless and orotund the advance praise. Blurbers (and critics)
speak with a reverent quiver of \emph{this moment,} anointing every
other book its guide, every second writer its essential voice.

Take ``Luster,'' by Raven Leilani, perhaps the summer's most touted
debut. It's a book that has been so feverishly praised for its boldness,
humor and sexual frankness that I was a little crushed to find instead a
perfectly agreeable if uneven first novel --- brisk and pleasantly
pulpy, hobbled occasionally by some seriously mangled prose and pat
psychology.

Edie, the protagonist of ``Luster,'' would never have allowed this to
happen. She's a spectacularly disabused and skeptical 23-year-old Black
woman, an assistant book editor, wise to the industry's excesses. Not
that she's especially engaged with her 9-to-5; she marshals her real
creative energy for sleeping with her male colleagues. She's a sexual
adventuress of a joyless sort, not looking for love so much as a kind of
annihilation, an escape from the self --- ``ecstatic rutting and cushy
ether of the void,'' as she puts it.

\emph{{[} ``Luster'' was one of our most anticipated books of the
month.}
\href{https://www.nytimes3xbfgragh.onion/2020/07/30/books/new-august-books.html}{\emph{See
the full list}}\emph{. {]}}

When we meet her, she's become infatuated with Eric, a much older white
man in an open marriage. The snaky plot takes us from their online
courtship to Edie moving in with him and his wife, Rebecca. It's a tense
ménage, initiated by Rebecca --- a woman of ``freaky competence'' who
takes a chilly interest in Edie and hopes she might counsel the couple's
12-year-old adopted daughter, Akila, who is lonely and isolated, one of
the few Black children in a white suburb. It's the strange attraction
between Edie and Rebecca --- ``two magnets of identical charge'' ---
that propels the story.

The sex in the book, marked by a light sadomasochistic edge --- Edie
pines to be punched, to be dominated --- has attracted enthusiastic
admiration. Sex has a way of getting all the attention; in this case, it
obscures that, page by page, this is less a story about coupling than it
is one about work. The spikiest, funniest scenes send up corporate life,
with all its feints at inclusion and its complacent racism.

In one scene, Edie browses the latest ``diversity'' offerings from her
publishing house: They include ``a slave narrative about a mixed-race
house girl fighting for a piece of her father's estate; a slave
narrative about a runaway's friendship with the white schoolteacher who
selflessly teaches her how to read; a slave narrative about a tragic
mulatto who raises the dead with her magic chitlin pies; a domestic
drama about a Black maid who, like Schrödinger's cat, is both alive and
dead.''

Image

Raven Leilani, author of the debut novel ``Luster.''Credit...Nina Subin

There is nothing on offer like ``Luster'' --- the story of a Black woman
who is neither heroic nor unduly tragic. Edie's a little depressed, a
little perplexing to herself, a little needy. She's thrilled to hear
that Eric finds her aloof: ``Aloof is a casual lean, a choice. It is not
a girl in Bushwick, licking clean a can of tuna.'' She is destructive
but tender, ravenous for experience but deeply vulnerable --- and often
wickedly funny: ``There are times I interact with kids and recall my
abortion fondly.''

{[} \emph{Read}
\href{https://www.nytimes3xbfgragh.onion/2020/07/31/books/raven-leilani-luster.html}{\emph{our
profile}} \emph{of Raven Leilani.} {]}

Even when her competitive side is awakened --- with her colleague Aria,
for example, the only other Black woman on staff, one who has mastered
``that unthreatening aw-shucks shtick for all the professional whites''
--- she shows generosity and a desire for kinship. ``Even as we look at
each other through borrowed faces, we see each other. I see her hunger,
and she sees mine.''

A blurry feeling settled over me as I kept encountering rhyming
descriptions and plot points. Edie's moments of connection with women
take identical forms. She shares a cigarette with each of them, and
cares for their hair, or tries to. She tenderly dyes Rebecca's hair, and
she teaches Akila how to care for hers, traveling back to Brooklyn to
buy the oil and conditioners unavailable in the white suburbs. Other
details repeat: Rebecca monitors Akila's weight in a way that reminds
Edie of her own mother: ``We were bonded in our mutual hatred of our
bodies.''

It's a book full of pairs, relationships that mirror one another. It is,
in fact, a book gleaming with actual mirrors. There are countless
references to Edie gazing at her reflection --- in dressing rooms, at
the bathroom at a party. ``Sometimes the face I see doesn't feel like
mine,'' she says. That tension creeps into the painting she pursues in
her spare time; she's unable to complete a self-portrait.

This is the governing terror Edie describes in her life, and in the
lives of the Black women in the novel: If you are hypersurveilled but
unseen, can you lose sight of your own face? Your own desires?

The reader, though, perhaps sees Edie too clearly. Narrative causality
flows a little too neatly, the back story filters in to explain Edie as
a culmination of her upbringing --- her father's philandering, her
mother's addiction and drive to oblivion. It's strange, perhaps, to
crave more privacy for a fictional character, but I wanted it for Edie.
I wanted more mystery, for her to resist being so neatly summed up. In a
word she might use, I wanted agency for her, but this story is
interested in inheritance, hence those echoes and doublings. Why do the
gestures repeat? We use what is available to us, the tools and habits of
consolation; we use what we have been taught, on ourselves and on
others.

One of Edie and Eric's few shared interests is disco, the pleasure of
which Edie explains slightly apologetically: It's the ``too much,'' she
says, it's ``the horn section and the cheese.'' So it is with
``Luster.'' Your enjoyment will depend on a tolerance for run-on
sentences that strain painfully for profundity (``for a moment I
consider the possibility of God as a chaotic, amorphous evil who made
autoimmune disease but gave us miraculous genitals to cope'' begins one
such example), for odd, often indecipherable metaphors. When Edie is
discovered snooping in Akila's room, she describes seeing the
embarrassment on the young girl's face ``like seeing an Olive Garden
commercial after having already plowed through two bowls of
fettuccine.''

The dialogue is flat, mostly expository with an interesting repetition.
The characters frame their impatience with Edie --- her transgressions,
her need --- as a generational divide, prefiguring, perhaps, how this
book might be read. Novels by young writers tend to attract a strange
sort of attention --- more anthropological than literary. \emph{What
does the work of Sally Rooney tell us about ``millennial precarity''?
What does Kristen Roupenian's ``Cat Person'' reveal about gender and
power?} It's not that fiction can't be a rich mine of evidence, but such
analyses often go against how and why most people read. Do we go to Ann
Petry for a sociological snapshot of midcentury Harlem life? ``Anna
Karenina'' for hot takes on the sexual politics of the Russian landed
gentry? We don't go to novels because they are timely, but because they
feel \emph{timeless} in their treatment of consciousness and emotional
life. We don't want characters to exist as a bundle of symptoms but as
full personalities in their rich confusion and ambivalence.

Edie could tell you all that. She enacts it; she'd flinch from being
lazily lumped in with any trend or cohort. ``Why does it have to be my
generation?'' she learns to respond. ``Why can't it be me,
specifically?''

Advertisement

\protect\hyperlink{after-bottom}{Continue reading the main story}

\hypertarget{site-index}{%
\subsection{Site Index}\label{site-index}}

\hypertarget{site-information-navigation}{%
\subsection{Site Information
Navigation}\label{site-information-navigation}}

\begin{itemize}
\tightlist
\item
  \href{https://help.nytimes3xbfgragh.onion/hc/en-us/articles/115014792127-Copyright-notice}{©~2020~The
  New York Times Company}
\end{itemize}

\begin{itemize}
\tightlist
\item
  \href{https://www.nytco.com/}{NYTCo}
\item
  \href{https://help.nytimes3xbfgragh.onion/hc/en-us/articles/115015385887-Contact-Us}{Contact
  Us}
\item
  \href{https://www.nytco.com/careers/}{Work with us}
\item
  \href{https://nytmediakit.com/}{Advertise}
\item
  \href{http://www.tbrandstudio.com/}{T Brand Studio}
\item
  \href{https://www.nytimes3xbfgragh.onion/privacy/cookie-policy\#how-do-i-manage-trackers}{Your
  Ad Choices}
\item
  \href{https://www.nytimes3xbfgragh.onion/privacy}{Privacy}
\item
  \href{https://help.nytimes3xbfgragh.onion/hc/en-us/articles/115014893428-Terms-of-service}{Terms
  of Service}
\item
  \href{https://help.nytimes3xbfgragh.onion/hc/en-us/articles/115014893968-Terms-of-sale}{Terms
  of Sale}
\item
  \href{https://spiderbites.nytimes3xbfgragh.onion}{Site Map}
\item
  \href{https://help.nytimes3xbfgragh.onion/hc/en-us}{Help}
\item
  \href{https://www.nytimes3xbfgragh.onion/subscription?campaignId=37WXW}{Subscriptions}
\end{itemize}
