Sections

SEARCH

\protect\hyperlink{site-content}{Skip to
content}\protect\hyperlink{site-index}{Skip to site index}

\href{https://www.nytimes3xbfgragh.onion/section/books/review}{Book
Review}

\href{https://myaccount.nytimes3xbfgragh.onion/auth/login?response_type=cookie\&client_id=vi}{}

\href{https://www.nytimes3xbfgragh.onion/section/todayspaper}{Today's
Paper}

\href{/section/books/review}{Book Review}\textbar{}What Do America's
Racial Problems Have in Common With India and Nazi Germany?

\url{https://nyti.ms/31k0dDZ}

\begin{itemize}
\item
\item
\item
\item
\item
\end{itemize}

Advertisement

\protect\hyperlink{after-top}{Continue reading the main story}

Supported by

\protect\hyperlink{after-sponsor}{Continue reading the main story}

nonfiction

\hypertarget{what-do-americas-racial-problems-have-in-common-with-india-and-nazi-germany}{%
\section{What Do America's Racial Problems Have in Common With India and
Nazi
Germany?}\label{what-do-americas-racial-problems-have-in-common-with-india-and-nazi-germany}}

\includegraphics{https://static01.graylady3jvrrxbe.onion/images/2020/08/09/books/review/09Appiah-COVER-Sub01/09Appiah-COVER-Sub01-articleLarge.jpg?quality=75\&auto=webp\&disable=upscale}

Buy Book ▾

\begin{itemize}
\tightlist
\item
  \href{https://www.amazon.com/gp/search?index=books\&tag=NYTBSREV-20\&field-keywords=Caste+Isabel+Wilkerson}{Amazon}
\item
  \href{https://du-gae-books-dot-nyt-du-prd.appspot.com/buy?title=Caste\&author=Isabel+Wilkerson}{Apple
  Books}
\item
  \href{https://www.anrdoezrs.net/click-7990613-11819508?url=https\%3A\%2F\%2Fwww.barnesandnoble.com\%2Fw\%2F\%3Fean\%3D9780593230251}{Barnes
  and Noble}
\item
  \href{https://www.anrdoezrs.net/click-7990613-35140?url=https\%3A\%2F\%2Fwww.booksamillion.com\%2Fp\%2FCaste\%2FIsabel\%2BWilkerson\%2F9780593230251}{Books-A-Million}
\item
  \href{https://bookshop.org/a/3546/9780593230251}{Bookshop}
\item
  \href{https://www.indiebound.org/book/9780593230251?aff=NYT}{Indiebound}
\end{itemize}

When you purchase an independently reviewed book through our site, we
earn an affiliate commission.

By Kwame Anthony Appiah

\begin{itemize}
\item
  Aug. 4, 2020
\item
  \begin{itemize}
  \item
  \item
  \item
  \item
  \item
  \end{itemize}
\end{itemize}

\textbf{CASTE}\\
\textbf{The Origins of Our Discontents}\\
By Isabel Wilkerson

Almost three decades ago, when she was a national correspondent for this
newspaper, Isabel Wilkerson set out to write a piece about Chicago's
``Magnificent Mile,'' an upscale commercial stretch where some New York
retailers were planning to open branches. When the last of her interview
subjects arrived at his showroom, where she was waiting patiently, she
tried to introduce herself. But the man, looking harried, brushed past
her. He didn't have time to talk, he said. He was running late for an
important appointment with a New York Times reporter. After Wilkerson
explained that she was the reporter in question, the man asked her to
produce identification, and even then he turned her away, doubtful that
the Black woman in front of him could be the Times reporter of that
name. Recalling the incident in ``Caste: The Origins of Our
Discontents,'' Wilkerson writes mordantly, ``This was the first time I
had ever been accused of impersonating myself.''

Conversations about race in America have reached fever pitch. We hear
talk of white privilege and fragility, of implicit bias, of racism in
structural and systemic varieties. There are loud and urgent calls for
racial justice, and many Americans of all races seem to agree that what
W. E. B. Du Bois, more than a century ago, termed ``the problem of the
color line'' remains a troubling feature of this country now. From
listening in on today's conversations, the only uncontroversial
conclusion one can arrive at is that the effects of race in America, as
elsewhere in the world, are complicated and continuing.

Sometimes, when we're faced with a perplexing, manifold social reality,
it can help to have a compact theory. That's the task Wilkerson has set
herself here, surveying our jostling ideas about the relation of
African-Americans to European-Americans. Our racial order, she argues,
is a system of caste --- a hierarchical structure of hereditary status.

Wilkerson, a Pulitzer Prize-winning journalist and the author of a
marvelous chronicle of the Great Migration titled
``\href{https://www.nytimes3xbfgragh.onion/2010/09/09/books/09wilkerson.html?searchResultPosition=4https://www.nytimes3xbfgragh.onion/2010/09/05/books/review/Oshinsky-t.html?searchResultPosition=18}{The
Warmth of Other Suns},'' is an elegant and persuasive writer. She has,
in particular, a masterly command of the complex extended metaphor ---
as when, at the start of the book, she analogizes the regular return of
``hatred and tribalism'' in societies to the anthrax released from the
Siberian permafrost by the summer's heat in 2016, entering the air as
virulent as it was when consigned to the earth during World War II.
Indeed, the book is itself devoted to an extended analogy: She wants us
to see America's enduring resistance to Black equality through the prism
of the caste system of India. In a secondary line of analysis, she notes
that the Nazis' anti-Semitic regime borrowed ideas and practices from
the legal structures created in the Jim Crow South and invites us to
see, in some of the horrors of the Holocaust, echoes of America's caste
order.

An analogy seeks to illuminate one thing by showing what it has in
common with something else. Of course, as Bishop Butler sagely observed
a couple of centuries ago, ``Every thing is what it is, and not another
thing.'' Race is and is not like Indian caste. The anti-Semitism of the
Nazis is and is not like American anti-Black racism. And the story of
race in America as a kind of caste system requires Wilkerson to tell us
a great deal about caste in modern India --- with the Dalits as
``outcastes'' in relation to the main system of the priests, rulers,
merchants and tradesmen --- while inevitably stinting on its complexity.
The analogy nevertheless offers potent illumination.

\includegraphics{https://static01.graylady3jvrrxbe.onion/images/2020/08/09/books/review/09Appiah-jp/09Appiah-jp-articleLarge.jpg?quality=75\&auto=webp\&disable=upscale}

In fact, you could say that the analogy with race is built into the very
origins of the idea of caste. As Wilkerson points out, when the
Portuguese first used their word ``casta'' to describe Indian social
structure, they were repurposing a term that had been applied in the
Iberian Peninsula to lineages of people defined by descent. (The word
shares its etymology with ``chaste,'' because a pure lineage was defined
as breeding only within the group.) If race is like caste, perhaps that
is because the concept of caste descends, originally, from that of race.

Certainly the race-as-caste model has an eminent scholarly history. Back
in the 1930s, Lloyd Warner, then a professor of anthropology and
sociology at the University of Chicago, introduced a caste-and-class
diagram that became foundational to much American social science on the
subject of race. His student Allison Davis, who was one of the
pre-eminent African-American social scientists of the mid-20th century
(and who, along with his wife, Elizabeth, gets a chapter in this book),
co-wrote a classic work about race in a Southern town titled ``Deep
South: A Social Anthropological Study of Caste and Class'' and published
in 1941. People on the subcontinent recognized the commonalities, too.
Wilkerson tells us that B. R. Ambedkar, an intellectual leader of the
Dalit Buddhist movement, which organized India's so-called untouchables
and sought what he famously called ``the annihilation of caste,'' wrote
to Du Bois in 1946, ``There is so much similarity between the position
of the untouchables in India and of the position of the Negroes in
America.''

In chapter after chapter, Wilkerson brings out suggestive similarities
in the treatment of Dalits in India, African-Americans in the United
States and Jews in Nazi Germany. Lower-caste members are dehumanized and
stigmatized, kept in their place through cruelty and terror, and
forbidden to intermarry with members of the higher castes. Privileges
are arrogated to the high caste. Pollution comes from contact with the
low caste. When, in the 1960s, a Black civil rights activist sought to
``integrate'' a pool by swimming a lap, Wilkerson tells us, it was
subsequently drained and entirely refilled to appease its white users.

What distinguishes Wilkerson is her grasp of the power of individual
narratives to illustrate such general ideas, allowing her to tell us
what these abstract notions have meant in the lived experience of
ordinary people both of the higher castes (white Americans, Brahmins and
``Aryan'' Germans) and of the lowest (African-Americans, Dalits and
Jews). The dexterity with which she combines larger historical
descriptions with vignettes from particular lives, recounted with the
skill of a veteran reporter, will be familiar to readers of ``The Warmth
of Other Suns.''

Any theoretical model is like a camera that brings a view of its subject
into sharp focus while leaving other features of the landscape blurred
or out of the frame. Its value comes from what, given its focal
distance, it is able to capture. For Wilkerson, the power of the caste
model is that it situates race as merely its ``faithful servant,'' and
overt racism as merely a subset of its malign workings. In her account,
the retailer who is certain that a Black woman cannot be the Times
reporter scheduled to interview him may be ``casteist'' --- invested in
or content with the hierarchy --- without necessarily being racist, in
the sense of having antipathy toward members of the stigmatized race.

At the same time, a focus on social status can blur the economic story
--- the way
\href{https://www.nytimes3xbfgragh.onion/interactive/2019/08/14/magazine/racial-wealth-gap.html}{Black
material disadvantage} is perpetuated through disparities in human and
financial capital, as measured by income, wealth and education.
Wilkerson complains about the way ``news outlets feed audiences a diet
of inner-city crime and poverty so out of proportion to the numbers,''
given that only 22 percent of African-Americans live in poverty. But
should the media focus less on that economic disadvantage? The average
white family, according to the Brookings Institution, has a
\href{https://www.brookings.edu/blog/up-front/2020/02/27/examining-the-black-white-wealth-gap/}{net
worth that's nearly 10 times as great} as that of the average Black
family. Wilkerson mentions the role of the Federal Housing
Administration's denial of mortgages to Blacks and its insistence on
segregation in generating these disparities. But she doesn't follow
Allison Davis's example and attend to the intertwining of class and
caste. The place of Black workers in the American economy is surely part
of the racial story, and it's notable that the word ``capitalism''
doesn't appear in Wilkerson's book. Low-status jobs are generally
low-income jobs; both income and status matter. Nor can we turn to the
caste model in explaining the centrality of Black people to American
popular culture.

Still, ``Caste'' will spur readers to think and to feel in equal
measure. Its vivid stories about the mistreatment of Black Americans by
government and law and in everyday social life --- from the violence of
the slave plantation to the terror of lynchings to the routines of
discourtesy and worse that are still a common experience for so many ---
retain their ability to appall and unsettle, to prompt flashes of
indignation and moments of sorrow. The result is a book that is at once
beautifully written and painful to read. Many Black Americans, I wager,
will find little about our country to surprise them here, though there
will be much to interest them in seeing America's race problem situated
within a global context. If Americans who are not Black fail to
recognize our country in this careful indictment, however, they should
probably ask themselves whether Wilkerson is showing them truths they
need to face up to. And we do need to face up to them if we are to bring
the hopes for racial justice echoing in our streets closer to reality
--- and advance the hard work of annihilating caste.

Advertisement

\protect\hyperlink{after-bottom}{Continue reading the main story}

\hypertarget{site-index}{%
\subsection{Site Index}\label{site-index}}

\hypertarget{site-information-navigation}{%
\subsection{Site Information
Navigation}\label{site-information-navigation}}

\begin{itemize}
\tightlist
\item
  \href{https://help.nytimes3xbfgragh.onion/hc/en-us/articles/115014792127-Copyright-notice}{©~2020~The
  New York Times Company}
\end{itemize}

\begin{itemize}
\tightlist
\item
  \href{https://www.nytco.com/}{NYTCo}
\item
  \href{https://help.nytimes3xbfgragh.onion/hc/en-us/articles/115015385887-Contact-Us}{Contact
  Us}
\item
  \href{https://www.nytco.com/careers/}{Work with us}
\item
  \href{https://nytmediakit.com/}{Advertise}
\item
  \href{http://www.tbrandstudio.com/}{T Brand Studio}
\item
  \href{https://www.nytimes3xbfgragh.onion/privacy/cookie-policy\#how-do-i-manage-trackers}{Your
  Ad Choices}
\item
  \href{https://www.nytimes3xbfgragh.onion/privacy}{Privacy}
\item
  \href{https://help.nytimes3xbfgragh.onion/hc/en-us/articles/115014893428-Terms-of-service}{Terms
  of Service}
\item
  \href{https://help.nytimes3xbfgragh.onion/hc/en-us/articles/115014893968-Terms-of-sale}{Terms
  of Sale}
\item
  \href{https://spiderbites.nytimes3xbfgragh.onion}{Site Map}
\item
  \href{https://help.nytimes3xbfgragh.onion/hc/en-us}{Help}
\item
  \href{https://www.nytimes3xbfgragh.onion/subscription?campaignId=37WXW}{Subscriptions}
\end{itemize}
