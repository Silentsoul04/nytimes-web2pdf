Sections

SEARCH

\protect\hyperlink{site-content}{Skip to
content}\protect\hyperlink{site-index}{Skip to site index}

\href{https://www.nytimes3xbfgragh.onion/section/books/review}{Book
Review}

\href{https://myaccount.nytimes3xbfgragh.onion/auth/login?response_type=cookie\&client_id=vi}{}

\href{https://www.nytimes3xbfgragh.onion/section/todayspaper}{Today's
Paper}

\href{/section/books/review}{Book Review}\textbar{}`Life of a Klansman,'
by Edward Ball: An Excerpt

\url{https://nyti.ms/3k3kZ38}

\begin{itemize}
\item
\item
\item
\item
\item
\end{itemize}

Advertisement

\protect\hyperlink{after-top}{Continue reading the main story}

Supported by

\protect\hyperlink{after-sponsor}{Continue reading the main story}

\hypertarget{life-of-a-klansman-by-edward-ball-an-excerpt}{%
\section{`Life of a Klansman,' by Edward Ball: An
Excerpt}\label{life-of-a-klansman-by-edward-ball-an-excerpt}}

Buy Book ▾

\begin{itemize}
\tightlist
\item
  \href{https://www.amazon.com/gp/search?index=books\&tag=NYTBSREV-20\&field-keywords=Life+of+a+Klansman\%3A+A+Family+History+in+White+Supremacy+Edward+Ball}{Amazon}
\item
  \href{https://du-gae-books-dot-nyt-du-prd.appspot.com/buy?title=Life+of+a+Klansman\%3A+A+Family+History+in+White+Supremacy\&author=Edward+Ball}{Apple
  Books}
\item
  \href{https://www.anrdoezrs.net/click-7990613-11819508?url=https\%3A\%2F\%2Fwww.barnesandnoble.com\%2Fw\%2F\%3Fean\%3D9780374186326}{Barnes
  and Noble}
\item
  \href{https://www.anrdoezrs.net/click-7990613-35140?url=https\%3A\%2F\%2Fwww.booksamillion.com\%2Fp\%2FLife\%2Bof\%2Ba\%2BKlansman\%253A\%2BA\%2BFamily\%2BHistory\%2Bin\%2BWhite\%2BSupremacy\%2FEdward\%2BBall\%2F9780374186326}{Books-A-Million}
\item
  \href{https://bookshop.org/a/3546/9780374186326}{Bookshop}
\item
  \href{https://www.indiebound.org/book/9780374186326?aff=NYT}{Indiebound}
\end{itemize}

When you purchase an independently reviewed book through our site, we
earn an affiliate commission.

Aug. 4, 2020

\begin{itemize}
\item
\item
\item
\item
\item
\end{itemize}

The middle of the week is good for an attack, for the surprise. It is
March 4, 1873, in the city of New Orleans, a Tuesday night. About 9:00
p.m., a man called Polycarp Constant Lecorgne emerges from his house by
the levee of the Mississippi River. He is a forty-one-year-old
carpenter. Constant Lecorgne and his wife, Gabrielle Duchemin, live in a
neighborhood called Bouligny. They have six children, and Gabrielle is
pregnant with another. Gabrielle and the children remain in the house
when Constant leaves for the night. He carries a gun, probably a
revolver. The U.S. Army confiscated most of the long guns years ago.

The newspapers tell much of the story. The Catholic Messenger. The New
Orleans Republican. The Times, the Picayune. Newspapers tell a crisp
story, and court records say more.

At home, the family of Constant Lecorgne speaks French, their first
language. French is a tongue of preference, as it is for about
one-quarter of the city, black as well as white. French is the language
of Creoles, English the language of most business and politics. The
family's house by the levee of the river is a rental. Constant and
Gabrielle once lived in a house they owned, but ten years ago they lost
it, along with all their money. They can no longer afford to buy.
Constant is a ship carpenter who works on the barges and steamers, the
passenger boats and freighters that ply the Mississippi. The house is
close to his work, a stone's throw from the water.

\emph{{[} Return to the review of}
\href{https://www.nytimes3xbfgragh.onion/2020/08/04/books/review/life-of-a-klansman-edward-ball.html}{\emph{``Life
of a Klansman.''}} \emph{{]}}

Constant has brothers and sisters, five of them. All have families, all
live nearby. People named Lecorgne are scattered through Bouligny. The
neighborhood of Bouligny lies three miles upstream on the Mississippi
from the old center of New Orleans. It is a square mile of clapboard
houses and workshops on the shoulder of the river, a place the Lecorgnes
regard as theirs. Before they start to move away from Bouligny, which
eventually occurs during the 1940s, the family lives in this part of New
Orleans for one hundred years.

The Lecorgne who carries a gun leaves the rented house on Valmont Street
and makes his way east some blocks through Bouligny. Constant meets
others. A cousin by marriage named Ernest Livaudais, who is a musician,
good on trumpet. He was the bugler in his company during the Civil War,
which ended some years ago. Tonight, Livaudais does not carry a horn,
but a gun. Constant and Ernest Livaudais continue downriver and join
another man, Joseph Guillotte. The carpenter and the bugler defer to
Guillotte. He is the leader of tonight's action, a raid on Precinct 7,
stationhouse of the Metropolitan Police.

Guillotte, Livaudais, and Lecorgne: these three are the French fingers
of the gang. They speak French to one another, and to other Creoles.
They speak English to the people they call les Américains, ``the
Americans.'' Creoles are French-speaking natives of Louisiana, white or
black. The English-speaking are les Américains, people who came to the
city after the United States bought Louisiana, in the early 1800s. The
Americans have grown to three-quarters of the population since then, and
they dominate the Creoles. Constant and the others dislike being
dominated, but it is their portion.

They rendezvous with more men, about thirty. Half of them Americans,
half of them Creoles, all of them white.

The gang moves in the direction of Lawrence Square, an acre of green at
the middle of Bouligny. At its edge is police precinct 7, a two-story
garrison. Court papers say the men have ``guns, muskets, pistols,
swords, bayonets, and other warlike instruments.'' Their muskets are
single-load rifles they managed to hide when the U.S. Army, the goddamn
Yankees, ordered every house in New Orleans to surrender its weapons,
after the war. Lawrence Square looks handsome. A big church called St.
Stephen overlooks the square, as do a town hall and food markets. The
main street, Napoleon Avenue, runs past St. Stephen, and streetcars on
railroad tracks rumble past every half hour.

Constant Lecorgne and his comrades come to Berlin Street, on the
southeast corner of the square, and there they find their target.

Tonight, probably, the gang does not wear hoods. Chances are that no one
wears a Ku-klux robe. Costumes like hoods and robes are good for the
parishes, the rural parts outside New Orleans. The parishes are what
people in Louisiana call their counties. It is there, in the black
villages, that a man must take steps to disguise himself. To bring rough
justice to the doors of les nègres, ``the blacks,'' a man needs
camouflage. But tonight is not a night ride with clubs and ropes and
whips. Tonight, a hood would get in the way. This is the first strike of
an insurrection, and the costume of the Ku-klux, and the usual tools, do
not fit the job.

Almost everybody in the gang is a soldier. A few years back, everyone
fought in the other insurrection, the one to make up the Confederate
States of America. The Confederate States was the slave nation that died
on the birthing table during the Civil War. The white South calls the
fight the ``war between the states,'' or the ``lost cause.'' It ended in
1865, eight years ago. The black South and the Northern states call it
``the rebellion.'' Eventually everyone will agree to call it the Civil
War.

The men are veterans, they know tactics. Constant Lecorgne was a second
lieutenant in the Eighteenth Louisiana Infantry during the war. Joseph
Guillotte, leader of tonight's assault, went with the Twenty-Second
Infantry. Ernest Livaudais fought with the Thirtieth Louisiana Regiment,
along with another man in the raid, Kendrick Chandler.

The newspapers call them ``Ku-kluxers.'' The men think of themselves as
guerrillas or vigilantes---they are vigilant in bringing order to a
disrupted world. Last year the same men went with the so-called
Louisiana Legion. And before that, the gang belonged to a group called
the Knights of the White Camellia. The guerrillas put on and take off
names like their costumes.

In this raid, the gang calls itself the ``McEnery Militia.'' They are in
the fight for a man named John McEnery, a politician. John McEnery ran
for governor last fall, and whites say he should have won, had the other
side not stolen the election. The McEnery Militia says it is taking back
power from the coloreds and the carpetbaggers. They are taking it back
from the U.S. Army. The army is the occupier, the carpetbaggers are the
thieves, and les nègres are the lackeys of both. The McEnery Militia
wants change. They want to return things to the way they were. If they
take the target, the garrison, the rest will come.

\emph{{[} Return to the review of}
\href{https://www.nytimes3xbfgragh.onion/2020/08/04/books/review/life-of-a-klansman-edward-ball.html}{\emph{``Life
of a Klansman.''}} \emph{{]}}

Advertisement

\protect\hyperlink{after-bottom}{Continue reading the main story}

\hypertarget{site-index}{%
\subsection{Site Index}\label{site-index}}

\hypertarget{site-information-navigation}{%
\subsection{Site Information
Navigation}\label{site-information-navigation}}

\begin{itemize}
\tightlist
\item
  \href{https://help.nytimes3xbfgragh.onion/hc/en-us/articles/115014792127-Copyright-notice}{©~2020~The
  New York Times Company}
\end{itemize}

\begin{itemize}
\tightlist
\item
  \href{https://www.nytco.com/}{NYTCo}
\item
  \href{https://help.nytimes3xbfgragh.onion/hc/en-us/articles/115015385887-Contact-Us}{Contact
  Us}
\item
  \href{https://www.nytco.com/careers/}{Work with us}
\item
  \href{https://nytmediakit.com/}{Advertise}
\item
  \href{http://www.tbrandstudio.com/}{T Brand Studio}
\item
  \href{https://www.nytimes3xbfgragh.onion/privacy/cookie-policy\#how-do-i-manage-trackers}{Your
  Ad Choices}
\item
  \href{https://www.nytimes3xbfgragh.onion/privacy}{Privacy}
\item
  \href{https://help.nytimes3xbfgragh.onion/hc/en-us/articles/115014893428-Terms-of-service}{Terms
  of Service}
\item
  \href{https://help.nytimes3xbfgragh.onion/hc/en-us/articles/115014893968-Terms-of-sale}{Terms
  of Sale}
\item
  \href{https://spiderbites.nytimes3xbfgragh.onion}{Site Map}
\item
  \href{https://help.nytimes3xbfgragh.onion/hc/en-us}{Help}
\item
  \href{https://www.nytimes3xbfgragh.onion/subscription?campaignId=37WXW}{Subscriptions}
\end{itemize}
