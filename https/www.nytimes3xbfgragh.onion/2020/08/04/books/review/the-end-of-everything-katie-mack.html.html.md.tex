Sections

SEARCH

\protect\hyperlink{site-content}{Skip to
content}\protect\hyperlink{site-index}{Skip to site index}

\href{https://www.nytimes3xbfgragh.onion/section/books/review}{Book
Review}

\href{https://myaccount.nytimes3xbfgragh.onion/auth/login?response_type=cookie\&client_id=vi}{}

\href{https://www.nytimes3xbfgragh.onion/section/todayspaper}{Today's
Paper}

\href{/section/books/review}{Book Review}\textbar{}This Is How It All
Ends

\href{https://nyti.ms/2XmiSOc}{https://nyti.ms/2XmiSOc}

\begin{itemize}
\item
\item
\item
\item
\item
\end{itemize}

Advertisement

\protect\hyperlink{after-top}{Continue reading the main story}

Supported by

\protect\hyperlink{after-sponsor}{Continue reading the main story}

Nonfiction

\hypertarget{this-is-how-it-all-ends}{%
\section{This Is How It All Ends}\label{this-is-how-it-all-ends}}

\includegraphics{https://static01.graylady3jvrrxbe.onion/images/2020/07/29/books/review/Gleick1/Gleick1-articleLarge.jpg?quality=75\&auto=webp\&disable=upscale}

By \href{https://www.nytimes3xbfgragh.onion/by/james-gleick}{James
Gleick}

\begin{itemize}
\item
  Aug. 4, 2020
\item
  \begin{itemize}
  \item
  \item
  \item
  \item
  \item
  \end{itemize}
\end{itemize}

\textbf{THE END OF EVERYTHING}\\
\textbf{(Astrophysically Speaking)}\\
By Katie Mack

Not to give anything away, but ``in about five billion years, the sun
will swell to its red giant phase, engulf the orbit of Mercury and
perhaps Venus, and leave the Earth a charred, lifeless, magma-covered
rock.'' That's how Katie Mack \emph{starts} her story. It's downhill
from there.

Many books have been written about our cosmic origins: the creation of
the universe 13.8 billion years ago; the Big Bang and all that followed.
The denouement, presumably tens of billions of years away, remains
comparatively mysterious. How does it all end? For that matter,
\emph{does} it all end, or can we keep on in our merry way indefinitely?
In ``The End of Everything: (Astrophysically Speaking),'' Mack, a
theoretical cosmologist at North Carolina State University, attempts to
answer what might seem the most remote of scientific questions.

``Some say the world will end in fire, / Some say in ice.'' She gives
Robert Frost his due. She doesn't bother with T. S. Eliot: ``not with a
bang but a whimper.'' Traditional speculation about the end times, the
end of days, comes from religion, where it is called eschatology.
Apocalypse, doomsday, Judgment Day --- all this, for theologians,
provides a way of thinking about the meaning or purpose of existence.
Our destiny, if only we could know it, might provide some reason for why
we're here.

Does the arc of the moral universe bend toward justice? Does it bend
anywhere at all? When Martin Luther King Jr. said it did, he was
paraphrasing the 19th-century abolitionist minister Theodore Parker, who
also said: ``My eye reaches but little ways. I cannot calculate the
curve and complete the figure by the experience of sight.''

\includegraphics{https://static01.graylady3jvrrxbe.onion/images/2020/07/29/books/review/Gleick2/Gleick2-articleLarge.jpg?quality=75\&auto=webp\&disable=upscale}

The eye of astrophysics reaches a great deal farther now. Cosmologists
calculate the curve and complete the figure by employing a potent
arsenal of instruments and methodologies. Optical, radio, X-ray and
gamma ray telescopes on mountaintops and in space, underground neutrino
detectors and gravitational-wave observatories extend our sight to the
edges of the universe. But what that really means is that they extend
our sight into the past.

This is perhaps the most basic fact of cosmology, sometimes taken for
granted, and Mack explains it elegantly. Telescope users have a window
into time. Light travels at finite speed, so everything comes to us with
its own time delay. We see the sun not as it is now but as it was eight
minutes ago. All we can know of a galaxy 10 billion light-years away is
what it looked like 10 billion years ago, when the universe was young.
``We can look even farther back,'' she writes, ``and see matter swirling
into supermassive black holes in a universe less than 500 million years
old, when starlight had only just begun to penetrate the darkness
between galaxies.''

``The End of Everything'' is a pleasure. Mack's style is personal and
often funny as she guides us along a cosmic timeline studded with
scientific esoterica and mystery. Most of what astronomers know comes
not from seeing but from deduction --- complex ladders of logic,
building upon one another. Black holes were first inferred and then
understood in more and more detail and now can just barely be said to
have been ``seen.'' Two of the most crucial components of the current
picture of the universe, dark matter and dark energy, are so-called
precisely because they are invisible. Scientists say they can see the
Big Bang, and they can, in a way, but what they mean is that they can
detect via radio telescopes a faint microwave radiation coming from all
directions in space --- the remnant of a time when the entire universe
was a fast-growing fireball.

Astrophysicists' techniques for looking backward through the eons can
let them peer into the future, too. The equations of physics run forward
as well as back. A picture begins to emerge of the far-future evolution
of the universe --- more than one picture, painted in broad strokes by
cosmologists with conflicting theories. Mack organizes the current
thinking into a handful of scenarios, some more plausible than others.

One is the ``Big Crunch.'' We know the universe has been expanding since
the Big Bang. That is to say, \emph{space itself} is expanding:
Galaxies, stars and all other things in the cosmos move farther and
farther apart. It's possible that the expansion will eventually slow,
stop and reverse itself, like a ball thrown up in the air that then
comes back down. And then? Catastrophe. High-energy particle jets and
radiation from stars condense and ignite a conflagration. ``Nuclear
explosions tear through stellar atmospheres, ripping apart the stars and
filling space with hot plasma,'' Mack says. ``At this point, things are
really very bad.'' You can tell she's enjoying this.

Alternatively, the expansion keeps on going until everything attenuates
and fades into nothingness. This cosmic endgame is the one known as
``heat death.'' You've heard of entropy: the inexorable tendency toward
disorder described by the second law of thermodynamics. It's entropy
that does us in. This scenario is ``a slow and agonizing one,'' Mack
says, ``marked by increasing isolation, inexorable decay and an
eons-long fade into darkness.'' Everything tends toward equilibrium, and
equilibrium means death. Stars burn out, galaxies fade into darkness,
even black holes evaporate. This notion has been with us since the
development of thermodynamics in the 19th century. H. G. Wells
visualized it this way in ``The Time Machine'': ``It would be hard to
convey the stillness of it. \ldots{} The darkness thickened. \ldots{}
All else was rayless obscurity. \ldots{} A horror of this great darkness
came on me.''

Other possibilities involve dark energy, a still poorly understood
business that seems to be the dominant component of our universe. A
dark-energy apocalypse could ``tear apart the very fabric of reality,
rendering any thinking creatures in the cosmos helpless as they watch
their universe being ripped open around them,'' Mack says. Some paths to
destruction arise from theories that involve parallel universes lurking
in extra dimensions. A so-called ``ekpyrotic'' scenario imagines
collisions of ``branes,'' three-dimensional universes ordinarily
invisible to one another. At the fringes, the cosmological theories with
the best jargon and cleverest names are often the most speculative.

Forty years ago, when much of this science was new, the physicist
Freeman Dyson complained that some of his colleagues felt it was
``disreputable'' to study our universe's destiny. He urged them to do it
anyway. ``If our analysis of the long-range future leads us to raise
questions related to the ultimate meaning and purpose of life,'' he
wrote, ``then let us examine these questions boldly and without
embarrassment.''

This might seem like the wrong time for a book peering billions of years
into the future to examine the ultimate doom and destruction. We have
doom and destruction of our own to worry about, arriving faster and
faster. These days many people wake up wondering if we'll make it past
November. Plague is rampant. The Arctic Circle is on fire. Still, I
found it helpful --- not reassuring, certainly, but mind-expanding ---
to be reminded of our place in a vast cosmos. Mack puts it this way:
``When we ask the question, `Can this all really go on forever?,' we are
implicitly validating our own existence, extending it indefinitely into
the future, taking stock and examining our legacy.''

It seems safe to say, though, that any meaning and purpose will have to
be found in ourselves, not in the stars. The cosmic end times will bring
no day of judgment, no redemption. All we can expect is the total
obliteration of whatever universe remains and any intelligence that
still abides there.

Advertisement

\protect\hyperlink{after-bottom}{Continue reading the main story}

\hypertarget{site-index}{%
\subsection{Site Index}\label{site-index}}

\hypertarget{site-information-navigation}{%
\subsection{Site Information
Navigation}\label{site-information-navigation}}

\begin{itemize}
\tightlist
\item
  \href{https://help.nytimes3xbfgragh.onion/hc/en-us/articles/115014792127-Copyright-notice}{©~2020~The
  New York Times Company}
\end{itemize}

\begin{itemize}
\tightlist
\item
  \href{https://www.nytco.com/}{NYTCo}
\item
  \href{https://help.nytimes3xbfgragh.onion/hc/en-us/articles/115015385887-Contact-Us}{Contact
  Us}
\item
  \href{https://www.nytco.com/careers/}{Work with us}
\item
  \href{https://nytmediakit.com/}{Advertise}
\item
  \href{http://www.tbrandstudio.com/}{T Brand Studio}
\item
  \href{https://www.nytimes3xbfgragh.onion/privacy/cookie-policy\#how-do-i-manage-trackers}{Your
  Ad Choices}
\item
  \href{https://www.nytimes3xbfgragh.onion/privacy}{Privacy}
\item
  \href{https://help.nytimes3xbfgragh.onion/hc/en-us/articles/115014893428-Terms-of-service}{Terms
  of Service}
\item
  \href{https://help.nytimes3xbfgragh.onion/hc/en-us/articles/115014893968-Terms-of-sale}{Terms
  of Sale}
\item
  \href{https://spiderbites.nytimes3xbfgragh.onion}{Site Map}
\item
  \href{https://help.nytimes3xbfgragh.onion/hc/en-us}{Help}
\item
  \href{https://www.nytimes3xbfgragh.onion/subscription?campaignId=37WXW}{Subscriptions}
\end{itemize}
