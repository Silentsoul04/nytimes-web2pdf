Sections

SEARCH

\protect\hyperlink{site-content}{Skip to
content}\protect\hyperlink{site-index}{Skip to site index}

\href{https://www.nytimes3xbfgragh.onion/section/business}{Business}

\href{https://myaccount.nytimes3xbfgragh.onion/auth/login?response_type=cookie\&client_id=vi}{}

\href{https://www.nytimes3xbfgragh.onion/section/todayspaper}{Today's
Paper}

\href{/section/business}{Business}\textbar{}Distilleries Raced to Make
Hand Sanitizer for the Pandemic. No Longer.

\url{https://nyti.ms/2XrDIfc}

\begin{itemize}
\item
\item
\item
\item
\item
\item
\end{itemize}

\href{https://www.nytimes3xbfgragh.onion/news-event/coronavirus?action=click\&pgtype=Article\&state=default\&region=TOP_BANNER\&context=storylines_menu}{The
Coronavirus Outbreak}

\begin{itemize}
\tightlist
\item
  live\href{https://www.nytimes3xbfgragh.onion/2020/08/04/world/coronavirus-cases.html?action=click\&pgtype=Article\&state=default\&region=TOP_BANNER\&context=storylines_menu}{Latest
  Updates}
\item
  \href{https://www.nytimes3xbfgragh.onion/interactive/2020/us/coronavirus-us-cases.html?action=click\&pgtype=Article\&state=default\&region=TOP_BANNER\&context=storylines_menu}{Maps
  and Cases}
\item
  \href{https://www.nytimes3xbfgragh.onion/interactive/2020/science/coronavirus-vaccine-tracker.html?action=click\&pgtype=Article\&state=default\&region=TOP_BANNER\&context=storylines_menu}{Vaccine
  Tracker}
\item
  \href{https://www.nytimes3xbfgragh.onion/2020/08/02/us/covid-college-reopening.html?action=click\&pgtype=Article\&state=default\&region=TOP_BANNER\&context=storylines_menu}{College
  Reopening}
\item
  \href{https://www.nytimes3xbfgragh.onion/live/2020/08/04/business/stock-market-today-coronavirus?action=click\&pgtype=Article\&state=default\&region=TOP_BANNER\&context=storylines_menu}{Economy}
\end{itemize}

Advertisement

\protect\hyperlink{after-top}{Continue reading the main story}

Supported by

\protect\hyperlink{after-sponsor}{Continue reading the main story}

\hypertarget{distilleries-raced-to-make-hand-sanitizer-for-the-pandemic-no-longer}{%
\section{Distilleries Raced to Make Hand Sanitizer for the Pandemic. No
Longer.}\label{distilleries-raced-to-make-hand-sanitizer-for-the-pandemic-no-longer}}

Even though coronavirus cases have surged again, craft distilleries say
the business of making the disinfectant has become more difficult.

\includegraphics{https://static01.graylady3jvrrxbe.onion/images/2020/07/28/business/00virus-distilleries1/merlin_174824706_9513ed78-510a-479c-84bb-6771cf818c25-articleLarge.jpg?quality=75\&auto=webp\&disable=upscale}

By \href{https://www.nytimes3xbfgragh.onion/by/kellen-browning}{Kellen
Browning}

\begin{itemize}
\item
  Aug. 4, 2020, 5:00 a.m. ET
\item
  \begin{itemize}
  \item
  \item
  \item
  \item
  \item
  \item
  \end{itemize}
\end{itemize}

As the
\href{https://www.nytimes3xbfgragh.onion/news-event/coronavirus?action=click\&pgtype=Article\&state=default\&module=styln-coronavirus\&region=TOP_BANNER\&context=storylines_menu}{coronavirus
pandemic} shuttered bars and restaurants in March, Phil McDaniel's craft
distillery in St. Augustine, Fla., stopped producing bourbon. Then he
realized there was one alcohol-based product he could make that people
would still clamor for: hand sanitizer.

His \href{http://staugustinedistillery.com/}{St. Augustine Distillery}
soon churned out the first of what became 10,000 gallons of the
disinfectant. With sanitizer in short supply nationwide, he quickly sold
and donated most of the supplies to hospitals and emergency responders
along Florida's northeastern coast.

``In the beginning, it was just unbelievable, the sort of frenetic
demand that was out there,'' said Mr. McDaniel, 62. ``It was so
gratifying to us to be able to come in and help.''

But
\href{https://www.nytimes3xbfgragh.onion/interactive/2020/world/coronavirus-maps.html?action=click\&module=Top\%20Stories\&pgtype=Homepage}{as
virus cases have spiked again} in Florida and other states, Mr. McDaniel
said he had no plans to make more sanitizer. That's because the early
demand he experienced tailed off in June when large brands like Purell
were able to pump out more product. The price for sanitizer, which had
hovered at \$50 a gallon, plunged to around \$15 a gallon. Today, he
still has about 1,000 gallons of it, spread between 250-gallon square
totes of finished product and 50-gallon drums of ingredients, sitting in
a warehouse.

Mr. McDaniel is one of more than 800 craft distillers across the United
States who leapt into action to help in the first wave of the pandemic,
urged on by federal agencies, but who are now hesitant to invest more
time and money into those efforts. With demand for sanitizer
fluctuating, distillers have faced unforeseen costs and excess supplies
that they could not get rid of.

\includegraphics{https://static01.graylady3jvrrxbe.onion/images/2020/07/28/business/00virus-distilleries2/merlin_175038075_c2418889-c06a-4714-936f-e11485da6256-articleLarge.jpg?quality=75\&auto=webp\&disable=upscale}

At the same time, the economics of the \$3.2 billion craft beverage
industry have deteriorated. Craft distilleries are often fragile,
mom-and-pop operations with fewer than a dozen employees and their
owners' life savings invested in the business. Now
\href{https://www.nytimes3xbfgragh.onion/2020/04/23/dining/drinks/craft-distillers-coronavirus.html}{battered
by months of lost liquor sales}, many cannot afford to spend more on
making sanitizer when all they really want to do is get back to making
whiskey to survive.

Their conundrum shows how life has become more complicated as the
pandemic has persisted. What had been a no-brainer good Samaritan
decision to help local communities and nurture a new business has
instead devolved into a messy financial calculus as the hardships of the
crisis continue piling up.

``It feels a little bit like no good deed is going unpunished right
now,'' said Spencer Whelan, the director of the Texas Whiskey
Association, a trade group representing some of the state's distillers.

The hand sanitizer industry has long been dominated by large companies
like Clorox and Gojo Industries, the maker of Purell. As registered drug
manufacturers, they are subject to rigorous Food and Drug Administration
inspections and reviews that allow them to produce the disinfectant.

\hypertarget{latest-updates-economy}{%
\section{\texorpdfstring{\href{https://www.nytimes3xbfgragh.onion/live/2020/08/04/business/stock-market-today-coronavirus?action=click\&pgtype=Article\&state=default\&region=MAIN_CONTENT_1\&context=storylines_live_updates}{Latest
Updates:
Economy}}{Latest Updates: Economy}}\label{latest-updates-economy}}

\href{https://www.nytimes3xbfgragh.onion/live/2020/08/04/business/stock-market-today-coronavirus?action=click\&pgtype=Article\&state=default\&region=MAIN_CONTENT_1\&context=storylines_live_updates\#the-ad-giant-publicis-has-parted-ways-with-an-executive-over-his-virus-tweets}{41m
ago}

\href{https://www.nytimes3xbfgragh.onion/live/2020/08/04/business/stock-market-today-coronavirus?action=click\&pgtype=Article\&state=default\&region=MAIN_CONTENT_1\&context=storylines_live_updates\#the-ad-giant-publicis-has-parted-ways-with-an-executive-over-his-virus-tweets}{The
ad giant Publicis has `parted ways' with an executive over his virus
tweets.}

\href{https://www.nytimes3xbfgragh.onion/live/2020/08/04/business/stock-market-today-coronavirus?action=click\&pgtype=Article\&state=default\&region=MAIN_CONTENT_1\&context=storylines_live_updates\#nbcuniversal-to-cut-about-10-percent-of-its-work-force}{2h
ago}

\href{https://www.nytimes3xbfgragh.onion/live/2020/08/04/business/stock-market-today-coronavirus?action=click\&pgtype=Article\&state=default\&region=MAIN_CONTENT_1\&context=storylines_live_updates\#nbcuniversal-to-cut-about-10-percent-of-its-work-force}{NBCUniversal
to cut about 10 percent of its work force.}

\href{https://www.nytimes3xbfgragh.onion/live/2020/08/04/business/stock-market-today-coronavirus?action=click\&pgtype=Article\&state=default\&region=MAIN_CONTENT_1\&context=storylines_live_updates\#loans-are-harder-to-get-even-as-interest-rates-are-low}{3h
ago}

\href{https://www.nytimes3xbfgragh.onion/live/2020/08/04/business/stock-market-today-coronavirus?action=click\&pgtype=Article\&state=default\&region=MAIN_CONTENT_1\&context=storylines_live_updates\#loans-are-harder-to-get-even-as-interest-rates-are-low}{Loans
are harder to get, even as interest rates are low.}

\href{https://www.nytimes3xbfgragh.onion/live/2020/08/04/business/stock-market-today-coronavirus?action=click\&pgtype=Article\&state=default\&region=MAIN_CONTENT_1\&context=storylines_live_updates}{See
more updates}

More live coverage:
\href{https://www.nytimes3xbfgragh.onion/2020/08/04/world/coronavirus-cases.html?action=click\&pgtype=Article\&state=default\&region=MAIN_CONTENT_1\&context=storylines_live_updates}{Global}

But the market was shaken up in March when hand sanitizer became scarce
after being snapped up by hospitals, emergency medical workers and
\href{https://www.nytimes3xbfgragh.onion/2020/03/14/technology/coronavirus-purell-wipes-amazon-sellers.html}{profiteers}.
That month, the F.D.A. and the Alcohol and Tobacco Tax and Trade Bureau
issued temporary guidance allowing businesses to start making sanitizer
without needing to undergo the typical inspections and oversight.

For distillers, it turned out that making hand sanitizer was fairly
straightforward. They already had the distilled spirits permits needed
to handle ethanol, the type of alcohol used both for liquor and as a
sanitizer ingredient. They also had the tanks, blending and bottling
equipment necessary for production.

The prospect of replacing liquor revenue with sanitizer sales piqued the
interest of Jonathan Eagan, a co-owner of the
\href{https://azdistilling.com/}{Arizona Distilling Company} in Tempe,
Ariz. He spent \$50,000 on alcohol to produce the disinfectant in the
spring, and said he quickly sold enough of it to make up for two months
in lost liquor sales.

Image

Purell hand sanitizer at a restaurant in Tulsa, Okla. The early demand
that distilleries experienced for sanitizer tailed off in June when
large brands like Purell were able to ramp up production.Credit...Joseph
Rushmore for The New York Times

That money was crucial, given that bars, restaurants and tours --- the
distillers' main sources of income --- were hobbled. In an
\href{https://www.distilledspirits.org/news/new-survey-shows-covid-19-creating-severe-financial-hardships-for-craft-distillers/}{April
survey} of 118 distilleries in 35 states by the Distilled Spirits
Council, an advocacy organization, respondents said 43 percent of their
employees had been laid off and sales had plummeted 64 percent.
Two-thirds said they didn't expect to be able to stay open in another
six months.

But even as distillers ramped up sanitizer production, that lifeline
also started petering out. As panic-buying of the disinfectant leveled
off and production among larger companies stabilized, ``the business
just kind of dried up'' in the last few weeks, said Mr. Eagan.

Now as Arizona deals with
\href{https://www.nytimes3xbfgragh.onion/interactive/2020/07/09/us/coronavirus-cases-reopening-trends.html}{new
virus cases}, much of his remaining 1,000 gallons of sanitizer has sat
idle. He blamed the changing demand on the ``vacillating'' by officials
over the pandemic's severity and their ``flip-flopping'' over which
businesses could reopen --- and stay open.

``The bigger frustrating issue is the fits and starts,'' he said. ``A
lot of these bars and restaurants don't know if they're open today, if
they're open tomorrow.''

To help distillers, advocacy groups like the Distilled Spirits Council
have lobbied Congress to provide
\href{https://www.ttb.gov/alcohol/craft-beverage-modernization-and-tax-reform-cbmtra}{economic
relief}. They also want the F.D.A. to specify how long it will allow
sanitizer production by distillers to continue, to give the businesses
some certainty.

Some states, like California and many of those in New England, have also
temporarily suspended laws that prohibit distilleries from directly
shipping alcohol to consumers. In states where those rules haven't been
changed, some distillers said their willingness to make and donate hand
sanitizer during a crisis merited a reprieve from the shipping
restrictions.

Distillers ``have absolutely done their civic duty,'' said Mr. McDaniel,
who is also president of the Florida Distillers' Guild, an advocacy
group. Now ``they're all on life support.''

For some distilleries weighing whether to continue sanitizer production,
the decision was easy: no way.

Barry Butler, the owner of
\href{https://www.tarponspringsdistillery.net/}{Tarpon Springs
Distillery} in Tarpon Springs, Fla., had teamed up with a nearby rum
distillery to give away about 15,000 gallons of sanitizer and had made
\$40,000 by selling 10,000 gallons more. But when demand plummeted in
June, he returned to producing moonshine and ouzo, a Greek liquor.

``It kept the lights on, it kept the guys working and employed when we
were shut down for tours and tasting,'' Mr. Butler said of making
sanitizer. But ``as a long-term economic solution for a distillery, it's
not a way to make money.''

Mr. Butler said he dealt with equipment problems, too. The F.D.A.'s
\href{https://www.fda.gov/media/136118/download}{sanitizer} guidelines
require distillers to add a bittering agent like the compound Bitrix to
ensure people don't try to drink the finished product.

Bitrix is so strong, Mr. Butler said, that any distillery equipment used
to make sanitizer was ruined. ``Anything you use for that is now dead
forever,'' he said, adding that he threw away stainless steel bottling
hoses that had been contaminated.

Image

A Dark Door Spirits co-owner, Matt Allen, with large containers of hand
sanitizer at the distillery's warehouse in Tampa.Credit...Eve Edelheit
for The New York Times

Not all distillers have given up on sanitizer. Matt Allen, co-owner of
\href{http://darkdoorspirits.com/}{Dark Door Spirits} in Tampa, said he
also began making sanitizer in the spring and has sold and given away at
least 20,000 gallons to eight cities, 16 counties, the Federal Aviation
Administration and the Postal Service.

When demand declined, he began working to lock in contracts with local
hospitals to buy a guaranteed amount of product going forward.

``We just need some sort of gauge of long-term commitment,'' Mr. Allen
said.

Mr. McDaniel's St. Augustine Distillery, which opened in 2014, is known
for its Florida cane vodka, rum made from Florida molasses and a variety
of bourbons, which are aged for three or more years. One of the state's
larger craft distilleries, it has attracted tens of thousands of
tourists every year to its headquarters, which once housed an ice
manufacturing plant.

The business, which initially lost money, has been profitable since the
bourbon finished aging nearly four years ago, Mr. McDaniel said. But
when the pandemic hit, the packed tasting room and tours became just a
memory. In April, Mr. McDaniel slashed advertising and furloughed about
15 of his more than 45 employees.

``It's terrifying,'' he said.

Mr. McDaniel said his sanitizer made enough to cover the cost of what he
donated and a little more. In June, when demand dropped off, he stopped
making it.

St. Augustine Distillery is now producing bourbon again --- but it is
seeing only half its normal amount of liquor sales.

``At the end of the day, our core business is making really great
alcohol,'' Mr. McDaniel said. ``To be able to get back to business and
have demand for that and to sell it profitably is what we're all looking
for right now.''

Advertisement

\protect\hyperlink{after-bottom}{Continue reading the main story}

\hypertarget{site-index}{%
\subsection{Site Index}\label{site-index}}

\hypertarget{site-information-navigation}{%
\subsection{Site Information
Navigation}\label{site-information-navigation}}

\begin{itemize}
\tightlist
\item
  \href{https://help.nytimes3xbfgragh.onion/hc/en-us/articles/115014792127-Copyright-notice}{©~2020~The
  New York Times Company}
\end{itemize}

\begin{itemize}
\tightlist
\item
  \href{https://www.nytco.com/}{NYTCo}
\item
  \href{https://help.nytimes3xbfgragh.onion/hc/en-us/articles/115015385887-Contact-Us}{Contact
  Us}
\item
  \href{https://www.nytco.com/careers/}{Work with us}
\item
  \href{https://nytmediakit.com/}{Advertise}
\item
  \href{http://www.tbrandstudio.com/}{T Brand Studio}
\item
  \href{https://www.nytimes3xbfgragh.onion/privacy/cookie-policy\#how-do-i-manage-trackers}{Your
  Ad Choices}
\item
  \href{https://www.nytimes3xbfgragh.onion/privacy}{Privacy}
\item
  \href{https://help.nytimes3xbfgragh.onion/hc/en-us/articles/115014893428-Terms-of-service}{Terms
  of Service}
\item
  \href{https://help.nytimes3xbfgragh.onion/hc/en-us/articles/115014893968-Terms-of-sale}{Terms
  of Sale}
\item
  \href{https://spiderbites.nytimes3xbfgragh.onion}{Site Map}
\item
  \href{https://help.nytimes3xbfgragh.onion/hc/en-us}{Help}
\item
  \href{https://www.nytimes3xbfgragh.onion/subscription?campaignId=37WXW}{Subscriptions}
\end{itemize}
