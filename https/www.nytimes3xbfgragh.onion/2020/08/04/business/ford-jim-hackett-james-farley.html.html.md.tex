Sections

SEARCH

\protect\hyperlink{site-content}{Skip to
content}\protect\hyperlink{site-index}{Skip to site index}

\href{https://www.nytimes3xbfgragh.onion/section/business}{Business}

\href{https://myaccount.nytimes3xbfgragh.onion/auth/login?response_type=cookie\&client_id=vi}{}

\href{https://www.nytimes3xbfgragh.onion/section/todayspaper}{Today's
Paper}

\href{/section/business}{Business}\textbar{}Ford, Struggling in a
Changing Industry, Replaces Its C.E.O.

\href{https://nyti.ms/2DaZptb}{https://nyti.ms/2DaZptb}

\begin{itemize}
\item
\item
\item
\item
\item
\end{itemize}

Advertisement

\protect\hyperlink{after-top}{Continue reading the main story}

Supported by

\protect\hyperlink{after-sponsor}{Continue reading the main story}

\hypertarget{ford-struggling-in-a-changing-industry-replaces-its-ceo}{%
\section{Ford, Struggling in a Changing Industry, Replaces Its
C.E.O.}\label{ford-struggling-in-a-changing-industry-replaces-its-ceo}}

Jim Hackett, who failed to impress Wall Street, will be replaced by
James Farley, an auto industry veteran who started his career at Toyota.

\includegraphics{https://static01.graylady3jvrrxbe.onion/images/2020/08/04/business/04-markets-brf-ford/04-markets-brf-ford-articleLarge.jpg?quality=75\&auto=webp\&disable=upscale}

By \href{https://www.nytimes3xbfgragh.onion/by/neal-e-boudette}{Neal E.
Boudette}

\begin{itemize}
\item
  Aug. 4, 2020
\item
  \begin{itemize}
  \item
  \item
  \item
  \item
  \item
  \end{itemize}
\end{itemize}

Three years ago, Ford Motor brought in a new chief executive, Jim
Hackett, to streamline the company's inner workings and raise profits.

Now, after he achieved mixed results, the company is again turning to a
new boss, hoping he can accelerate the process and finish the job.

On Tuesday, Ford said Mr. Hackett will retire on Oct. 1 and will be
succeeded by James D. Farley Jr., whose promotion to
\href{https://www.nytimes3xbfgragh.onion/2020/02/07/business/ford-motor-jim-farley.html}{chief
operating officer in February} had fueled speculation that Mr. Hackett's
tenure was nearing an end.

``I am very grateful to Jim Hackett for all he has done to modernize
Ford and prepare us to compete and win in the future,'' said William
Clay Ford Jr., Ford's executive chairman.
\href{https://www.nytimes3xbfgragh.onion/topic/company/ford-motor-company}{The
company}, he added, is becoming ``much more nimble.''

Mr. Farley will take the titles of president and C.E.O., and join Ford's
board of directors, the company said.

Mr. Hackett, a former chief executive of Steelcase, an office furniture
manufacturer that is much smaller and less complex, was named to
\href{https://www.nytimes3xbfgragh.onion/2017/05/22/business/jim-hackett-ford.html}{the
top job at Ford} in May 2017, as the company's business was slumping
under his predecessor, Mark Fields. Mr. Hackett promised to revitalize
Ford's operations and steer the company toward vehicles that would
generate profits --- pickup trucks and sport-utility vehicles --- and
invest in emerging technologies like electric and self-driving vehicles.

The company is starting to introduce some of the models developed under
Mr. Hackett, including a redesigned F-150 pickup truck and a new series
of S.U.V.s under its dormant Bronco brand. The Mustang Mach E,
\href{https://www.nytimes3xbfgragh.onion/2019/11/17/business/ford-mustang-electric-suv.html}{an
electric S.U.V. styled to resemble the storied sports car}, has
generated lots of buzz and is set to go into production later this year.

``We have lots of work ahead of us to complete our mission, but thanks
to Jim, we are a very different company today than we were three years
ago,'' Mr. Ford said in a conference call to discuss the leadership
change.

Mr. Hackett is credited with eliminating money-losing cars from Ford's
North American lineup in favor of more profitable pickups and S.U.V.s.
He formed alliances
\href{https://www.nytimes3xbfgragh.onion/2019/07/05/business/ford-vw-self-driving-electric-cars.html}{with
Volkswagen}, the
\href{https://www.nytimes3xbfgragh.onion/2019/10/01/business/ford-india.html}{Indian
automaker Mahindra} and Rivian, a start-up working on electric trucks
\href{https://www.nytimes3xbfgragh.onion/2019/04/24/business/ford-rivian-investment.html}{in
which Ford has invested}. Mr. Hackett also accelerated plans to develop
electric vehicles.

\includegraphics{https://static01.graylady3jvrrxbe.onion/images/2020/08/04/business/04ford2/merlin_168532413_28ed4e05-a899-4dac-a9d6-43380313250d-articleLarge.jpg?quality=75\&auto=webp\&disable=upscale}

But so far, the turnaround has had little effect on the company's bottom
line and stock price. Ford's
\href{https://s23.q4cdn.com/799033206/files/doc_financials/quarterly/2019/q4/Ford-4Q2019-Earnings-Press-Release.pdf}{profits
fell in 2018 and 2019}, dropping to \$47 million last year. This year,
the pandemic has hammered its business, and the company lost \$876
million in the first half of the year.

``Making the C.E.O. change now could be viewed as Ford is ready to move
faster, more aggressively,'' Joseph Spak, an analyst with RBC Capital
Markets, wrote in a note to clients. ``Farley's persona and style may be
what is needed for that.''

Wall Street analysts have criticized Mr. Hackett for stopping short of
presenting a full turnaround plan with detailed financial goals and
timetables. Ford had planned to do so early in his tenure, but changed
course and presented only broad targets and revealed its plans piecemeal
as it rolled out specific initiatives and projects.

Ford shares were trading at about \$11 when Mr. Hackett arrived. The
stock closed at \$6.86 on Tuesday, up 2.5 percent.

Mr. Hackett ``faced challenges in technology changes and current
operations without technology or auto industry experience,'' said Erik
Gordon, a business professor at the University of Michigan who follows
the auto industry.

Investors value Ford at about \$27 billion, just one-tenth the market
capitalization of
\href{https://www.nytimes3xbfgragh.onion/topic/company/tesla-motors-inc}{Tesla,
the electric automaker} that makes far fewer cars and has been around
only since 2003.

Mr. Farley, 58, joined Ford in 2007 from Toyota Motor, where he played a
key role in launching the company's Lexus luxury brand. Since arriving
at Ford, he has held a variety of jobs, including running the company's
marketing, its European operations and a new business strategy group.

Mr. Farley's temperament is very different from Mr. Hackett's. The
current chief executive has a cerebral, grandfatherly presence and is
known for writing long companywide emails about his thoughts on Ford and
its future. Mr. Farley is known as an intense and animated personality.

In a conference call, Mr. Farley noted the automaker has a history going
back 117 years. Over more than a century, Ford has competed with and
been closely compared to its two Detroit rivals, General Motors and the
company now known as Fiat Chrysler.

But given the technological changes and disruption now rippling through
the auto industry, Mr. Farley said Ford's competitors today include
companies like Amazon, Tesla, Toyota and Baidu, a Chinese
artificial-intelligence technology company that is working on
self-driving cars.

Mr. Farley said his first priority would be to ensure a smooth
transition. He added that he is optimistic about the company's prospects
now that it has introduced the new F-150, Bronco S.U.V. and the Mustang
Mach E. Ford is hoping the Mustang Mach E will emerge as a serious
challenger to Tesla, which
\href{https://www.nytimes3xbfgragh.onion/2020/07/22/business/tesla-electric-car-audi-polestar.html}{has
faced little serious competition}. The company is also taking aim at
Jeep, a Fiat Chrysler brand, with its new Bronco, for which it has taken
150,000 reservations.

On Mr. Farley's list of tasks are raising Ford's profit margin in North
America to 10 percent or more, cutting costs and reviving the company's
sales in Europe, China and South America.

``I'm inspired by the momentum we are building,'' Mr. Farley said in the
conference call. ``To fulfill our mission, we need to swing for the
fences.''

Advertisement

\protect\hyperlink{after-bottom}{Continue reading the main story}

\hypertarget{site-index}{%
\subsection{Site Index}\label{site-index}}

\hypertarget{site-information-navigation}{%
\subsection{Site Information
Navigation}\label{site-information-navigation}}

\begin{itemize}
\tightlist
\item
  \href{https://help.nytimes3xbfgragh.onion/hc/en-us/articles/115014792127-Copyright-notice}{©~2020~The
  New York Times Company}
\end{itemize}

\begin{itemize}
\tightlist
\item
  \href{https://www.nytco.com/}{NYTCo}
\item
  \href{https://help.nytimes3xbfgragh.onion/hc/en-us/articles/115015385887-Contact-Us}{Contact
  Us}
\item
  \href{https://www.nytco.com/careers/}{Work with us}
\item
  \href{https://nytmediakit.com/}{Advertise}
\item
  \href{http://www.tbrandstudio.com/}{T Brand Studio}
\item
  \href{https://www.nytimes3xbfgragh.onion/privacy/cookie-policy\#how-do-i-manage-trackers}{Your
  Ad Choices}
\item
  \href{https://www.nytimes3xbfgragh.onion/privacy}{Privacy}
\item
  \href{https://help.nytimes3xbfgragh.onion/hc/en-us/articles/115014893428-Terms-of-service}{Terms
  of Service}
\item
  \href{https://help.nytimes3xbfgragh.onion/hc/en-us/articles/115014893968-Terms-of-sale}{Terms
  of Sale}
\item
  \href{https://spiderbites.nytimes3xbfgragh.onion}{Site Map}
\item
  \href{https://help.nytimes3xbfgragh.onion/hc/en-us}{Help}
\item
  \href{https://www.nytimes3xbfgragh.onion/subscription?campaignId=37WXW}{Subscriptions}
\end{itemize}
