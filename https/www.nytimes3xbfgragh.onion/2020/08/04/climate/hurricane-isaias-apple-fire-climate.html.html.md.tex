Sections

SEARCH

\protect\hyperlink{site-content}{Skip to
content}\protect\hyperlink{site-index}{Skip to site index}

\href{https://www.nytimes3xbfgragh.onion/section/climate}{Climate}

\href{https://myaccount.nytimes3xbfgragh.onion/auth/login?response_type=cookie\&client_id=vi}{}

\href{https://www.nytimes3xbfgragh.onion/section/todayspaper}{Today's
Paper}

\href{/section/climate}{Climate}\textbar{}Hurricane, Fire, Covid-19:
Disasters Expose the Hard Reality of Climate Change

\href{https://nyti.ms/3kburBB}{https://nyti.ms/3kburBB}

\begin{itemize}
\item
\item
\item
\item
\item
\item
\end{itemize}

\href{https://www.nytimes3xbfgragh.onion/section/climate?action=click\&pgtype=Article\&state=default\&region=TOP_BANNER\&context=storylines_menu}{Climate
and Environment}

\begin{itemize}
\tightlist
\item
  \href{https://www.nytimes3xbfgragh.onion/2020/07/30/climate/sea-level-inland-floods.html?action=click\&pgtype=Article\&state=default\&region=TOP_BANNER\&context=storylines_menu}{Rising
  Seas}
\item
  \href{https://www.nytimes3xbfgragh.onion/interactive/2020/climate/trump-environment-rollbacks.html?action=click\&pgtype=Article\&state=default\&region=TOP_BANNER\&context=storylines_menu}{Trump's
  Changes}
\item
  \href{https://www.nytimes3xbfgragh.onion/interactive/2020/04/19/climate/climate-crash-course-1.html?action=click\&pgtype=Article\&state=default\&region=TOP_BANNER\&context=storylines_menu}{Climate
  101}
\item
  \href{https://www.nytimes3xbfgragh.onion/interactive/2018/08/30/climate/how-much-hotter-is-your-hometown.html?action=click\&pgtype=Article\&state=default\&region=TOP_BANNER\&context=storylines_menu}{Is
  Your Hometown Hotter?}
\item
  \href{https://www.nytimes3xbfgragh.onion/newsletters/climate-change?action=click\&pgtype=Article\&state=default\&region=TOP_BANNER\&context=storylines_menu}{Newsletter}
\end{itemize}

Advertisement

\protect\hyperlink{after-top}{Continue reading the main story}

Supported by

\protect\hyperlink{after-sponsor}{Continue reading the main story}

\hypertarget{hurricane-fire-covid-19-disasters-expose-the-hard-reality-of-climate-change}{%
\section{Hurricane, Fire, Covid-19: Disasters Expose the Hard Reality of
Climate
Change}\label{hurricane-fire-covid-19-disasters-expose-the-hard-reality-of-climate-change}}

Twin emergencies on two coasts this week --- Hurricane Isaias and the
Apple Fire --- offer a preview of life in a warming world and the steady
danger of overlapping disasters.

\includegraphics{https://static01.graylady3jvrrxbe.onion/images/2020/08/04/climate/04CLI-DISASTERS1/merlin_175218708_8db171ff-e25f-4763-9915-6e1332d38bf2-articleLarge.jpg?quality=75\&auto=webp\&disable=upscale}

By
\href{https://www.nytimes3xbfgragh.onion/by/christopher-flavelle}{Christopher
Flavelle} and
\href{https://www.nytimes3xbfgragh.onion/by/henry-fountain}{Henry
Fountain}

\begin{itemize}
\item
  Aug. 4, 2020
\item
  \begin{itemize}
  \item
  \item
  \item
  \item
  \item
  \item
  \end{itemize}
\end{itemize}

A low-grade
\href{https://www.nytimes3xbfgragh.onion/2020/08/04/us/isaias-storm-updates.html}{hurricane
that is slowly scraping along the East Coast}. A wildfire in California
that has led to evacuation orders for 8,000 people. And in both places,
as well as everywhere between, a
\href{https://www.nytimes3xbfgragh.onion/interactive/2020/world/coronavirus-maps.html}{pandemic}
that keeps worsening.

The daily morning briefing from the Federal Emergency Management Agency,
usually a dry document full of acronyms and statistics, has begun to
resemble the setup for a disaster movie. But rather than a freak
occurrence, experts say that the pair of hazards bracketing the country
this week offers a preview of life under climate change: a relentless
grind of overlapping disasters, major or minor.

The coronavirus pandemic has further
\href{https://www.nytimes3xbfgragh.onion/2020/05/22/climate/fema-volunteer-disaster-response.html}{exposed
flaws} in the nation's defenses, including
\href{https://www.nytimes3xbfgragh.onion/2019/10/26/climate/building-codes-secret-deal.html}{weak
construction standards} in vulnerable areas,
\href{https://newrepublic.com/article/158486/towns-arent-equipped-handle-climate-emergencies}{underfunded}
government agencies, and racial and income
\href{https://www.nytimes3xbfgragh.onion/2020/05/17/climate/pollution-poverty-coronavirus.html}{disparities}
that put some communities
\href{https://www.nytimes3xbfgragh.onion/2020/07/24/climate/houston-flooding-race.html}{at
greater risk}. Experts argue that the country must fundamentally rethink
how it prepares for similar disasters as the effects of global warming
accelerate.

``State and local governments already stretched with Covid responses
must now stretch even further,'' said Lisa Anne Hamilton, adaptation
program director at the Georgetown Climate Center in Washington. Better
planning and preparation are crucial, she added, as the frequency and
intensity of disasters increase.

Hurricane Isaias made landfall in the Carolinas on Monday evening, its
75 mile-an-hour winds driving a storm surge as great as five feet. By
Tuesday afternoon, downgraded to a tropical storm, Isaias had pushed
north to the Mid-Atlantic states and the Northeast. Flash flooding was
reported in Pennsylvania, and damaging winds left more than 1.2 million
people in New Jersey and New York without power. The storm also spawned
tornadoes, including one that killed two people in North Carolina.

\includegraphics{https://static01.graylady3jvrrxbe.onion/images/2020/08/04/climate/04CLI-DISASTERS2/merlin_175291140_f68b5d37-f0da-4c88-87d4-75570a6f556d-articleLarge.jpg?quality=75\&auto=webp\&disable=upscale}

Isaias makes nine named storms in the Atlantic so far this year,
something that has never before happened this early in the hurricane
season, which runs from June 1 to Nov. 30. Forecasters had predicted
\href{https://www.nytimes3xbfgragh.onion/2020/05/21/climate/hurricane-season-2020-noaa.html}{an
active season,} given warm ocean waters and other conditions, but 2020
is on track to be one of the busiest ever. It follows three years of
devastating hurricanes, starting with Hurricane Harvey, Irma and Maria
in 2017, then Florence and Michael in 2018 and Dorian in 2019.

``Climate change is tough for people to grasp, but attribution studies
continue to find its DNA in today's tropical systems, heat waves,
droughts and rainstorms,'' said Marshall Shepherd, a professor of
atmospheric sciences and geography at the University of Georgia and
director of its atmospheric sciences program.

For hurricanes, warmer oceans provide more energy, making them stronger.
And warmer air holds more moisture, so
\href{https://www.nytimes3xbfgragh.onion/2019/07/11/climate/hurricane-tropical-storms.html}{the
storms bring more rain}.

\href{https://www.nytimes3xbfgragh.onion/section/climate?action=click\&pgtype=Article\&state=default\&region=MAIN_CONTENT_1\&context=storylines_keepup}{}

\hypertarget{climate-and-environment-}{%
\subsubsection{Climate and Environment
›}\label{climate-and-environment-}}

\hypertarget{keep-up-on-the-latest-climate-news}{%
\paragraph{Keep Up on the Latest Climate
News}\label{keep-up-on-the-latest-climate-news}}

Updated Aug. 4, 2020

Here's what you need to know about the latest climate change news this
week:

\begin{itemize}
\item
  \begin{itemize}
  \tightlist
  \item
    \href{https://www.nytimes3xbfgragh.onion/2020/08/04/climate/hurricane-isaias-apple-fire-climate.html?action=click\&pgtype=Article\&state=default\&region=MAIN_CONTENT_1\&context=storylines_keepup}{Twin
    emergencies on two coasts this week} --- Hurricane Isaias and the
    Apple Fire --- offer a preview of life in a warming world and the
    steady danger of overlapping disasters.
  \item
    \href{https://www.nytimes3xbfgragh.onion/2020/07/30/climate/bangladesh-floods.html?action=click\&pgtype=Article\&state=default\&region=MAIN_CONTENT_1\&context=storylines_keepup}{Floods
    in}\href{https://www.nytimes3xbfgragh.onion/2020/07/30/climate/bangladesh-floods.html?action=click\&pgtype=Article\&state=default\&region=MAIN_CONTENT_1\&context=storylines_keepup}{Bangladesh}
    are punishing the people least responsible for climate change.
  \item
    The E.P.A. inspector general plans to investigate whether a rollback
    of fuel efficiency standards
    \href{https://www.nytimes3xbfgragh.onion/2020/07/27/climate/trump-fuel-efficiency-rule.html?action=click\&pgtype=Article\&state=default\&region=MAIN_CONTENT_1\&context=storylines_keepup}{violated
    government rules}.
  \end{itemize}
\end{itemize}

``Climate change shifts us into an era of sustained elevated risk from
extreme weather and climate events,'' Dr. Shepherd said.

Isaias has captured much of the public's attention, but it's far from
the only natural disaster facing the country. In Southern California,
firefighters were struggling Tuesday to contain a wildfire in the San
Bernardino Mountains 80 miles east of Los Angeles. It had spread rapidly
in the rugged terrain after first being reported on Friday.

Called the Apple Fire, it has burned 27,000 acres so far, though it
remains much smaller than other recent fires in the state. The largest,
the Mendocino Complex Fire in 2018, burned nearly half a million acres.
The disastrous
\href{https://www.nytimes3xbfgragh.onion/interactive/2019/07/31/magazine/paradise-camp-fire-california.html}{Camp
Fire of 2018}, which burned 150,000 acres and killed 85 people, barely
makes the Top 20 list.

``At a certain point in California's history, 20,000 acres would have
been a pretty big fire,'' said Daniel Swain, a climate scientist at the
University of California, Los Angeles. However, the warming climate and
shifting precipitation patterns have lengthened the state's fire season
and contributed to an increase in larger fires.

The fires can grow more rapidly --- in a matter of hours or days --- as
a result of warming that has made vegetation drier and more likely to
ignite.

So far there are no reports of casualties from the Apple Fire. But there
is concern downwind, in Nevada and other states, as smoke from the
wildfire is
\href{https://rapidrefresh.noaa.gov/hrrr/HRRRsmoke/jsloopLocalDiskDateDomainZipTZA.cgi?dsKeys=hrrr_smoke_jet:\&runTime=2020080314\&plotName=trc1_t4sfc\&fcstInc=60\&numFcsts=49\&model=hrrr\&ptitle=HRRR-Smoke\%20Model\%20Fields\%20-\%20Experimental\&maxFcstLen=48\&fcstStrLen=-1\&resizePlot=1\&domain=t4}{carried
eastward}. In Las Vegas, Clark County air-quality officials issued a
two-day smoke advisory, urging people with respiratory problems to stay
indoors.

Wildfire smoke contains high amounts of soot and other fine particles
that can
\href{https://www.nytimes3xbfgragh.onion/2019/10/24/climate/california-wildfires-climate-change.html}{aggravate
asthma and other respiratory problems}.

Amid the coronavirus pandemic, there is heightened concern that the
smoke, while not necessarily increasing the rate of infection, can make
cases of Covid-19, the disease caused by the virus, worse, said Dr. John
Balmes, a professor of medicine at the University of California, San
Francisco. There is strong evidence from studies of influenza and other
viruses that smoke can increase the risk of deep-lung infections like
pneumonia, which occurs in severe cases of Covid-19.

Image

A coronavirus testing center in New York City said it would close as
Isaias approached.Credit...Shannon Stapleton/Reuters

The combination of tropical storms, wildfires and other disasters,
coming after
\href{https://www.nytimes3xbfgragh.onion/2020/05/20/us/michigan-flooding-dams-midland.html}{months}
of
\href{https://www.nytimes3xbfgragh.onion/2020/04/13/us/tornado-storm-south.html}{prior}
\href{https://www.nytimes3xbfgragh.onion/2020/01/11/us/winter-storm-tornadoes.html}{disasters}
and the struggle to deal with the pandemic, has taken a growing toll on
the nation's disaster response system. Part of the problem is that more
frequent disasters make it harder to recover, according to Samantha
Montano, an assistant professor of emergency management at Massachusetts
Maritime Academy.

``What makes climate change so insidious is that it alters hazards, like
flooding, just enough to turn what otherwise could have been just an
emergency into a disaster, and disasters into catastrophes,'' Dr.
Montano said. ``Not only does this lead to more damage but also traps
people in a cycle of recovery.''

Coping with that change, she said, means that governments have to spend
more money before a storm or wildfire hits, reinforcing homes and
infrastructure, rather than just trying to build better afterward. And
local emergency departments need increased funding as their jobs expand.

When state and local governments can't keep up with the need,
responsibility falls to FEMA. But the agency risks being overwhelmed,
according to Brock Long, who was FEMA's administrator during the
hurricanes and wildfires of 2017 and 2018.

``The current business model for the Federal Emergency Management
Agency, and the expectations placed upon it by the public and Congress,
are unrealistic at this point,'' said Mr. Long, who is now executive
chairman of Hagerty Consulting, which advises companies and governments
on dealing with disasters.

That toll can be measured in the minutia of FEMA's daily briefings.

Three years ago, before Hurricane Harvey marked the beginning of a
string of record natural catastrophes, FEMA was managing 27 major
disasters around the country, with a staff of slightly more than 10,000
people. As of Tuesday, the agency was handling about twice as many
disasters, not counting its pandemic response in every state and five
territories, despite a staff increase of just one-third. And the country
has yet to reach peak hurricane season.

In a statement, Lizzie Litzow, FEMA's press secretary, said the agency
continues to help states hit by natural disasters.

``FEMA is well positioned with thousands of personnel in the field
supporting existing operations, thousands more ready to support emergent
disaster operations and more personnel joining the agency through
virtual onboarding every two weeks,'' Ms. Litzow said.

Image

Damage in Southport, N.C., from Isaias early Tuesday.Credit...WECT-TV,
via Associated Press

But the real solution, Mr. Long said, isn't a bigger FEMA. Rather, local
governments have to impose tougher building codes and restrictions in
\href{https://www.bloomberg.com/news/articles/2018-11-14/california-s-wildfire-epidemic-blamed-on-bad-building-decisions?sref=UBrhZ1ro}{vulnerable
areas}, which home builders often oppose for fear of increased costs. If
cities and towns had better building codes, he said, fewer people would
need to evacuate their homes, reducing their exposure to the
coronavirus.

``Mass evacuation has become a man-made disaster, because we failed to
put proper residential codes or building codes in place,'' Mr. Long
said. ``We have a severe case of hazard amnesia.''

Juan Declet-Barreto, a social scientist at the Union of Concerned
Scientists who works on climate vulnerability, laid part of the blame
with President Trump for difficulties in disaster response. The
president, he said, has
\href{https://www.nytimes3xbfgragh.onion/2020/07/09/climate/trump-hurricane-dorian-noaa.html}{politicized}
the work of scientific agencies like the Centers for Disease Control and
Prevention and the
\href{https://www.nytimes3xbfgragh.onion/2019/09/13/climate/trump-noaa-hurricane-tweet.html}{National
Weather Service} that Americans rely on to navigate disasters, and has
tried to
\href{https://www.sciencemag.org/news/2020/02/trump-s-new-budget-cuts-all-favored-few-science-programs}{cut
their budgets}.

``They need to be well funded,'' Dr. Declet-Barreto said. ``They need to
be allowed to do their work.''

The twin disasters of climate change and the pandemic have something
else in common, he said, in addition to the failures of the Trump
administration to respond to them. Both disasters have
disproportionately hurt minorities.

``We shouldn't be romanticizing some sort of pre-Covid ideal state. We
did not live in that,'' Dr. Declet-Barreto said. ``These threats that we
are living through are going to continue to expose the inequalities that
already exist.''

Advertisement

\protect\hyperlink{after-bottom}{Continue reading the main story}

\hypertarget{site-index}{%
\subsection{Site Index}\label{site-index}}

\hypertarget{site-information-navigation}{%
\subsection{Site Information
Navigation}\label{site-information-navigation}}

\begin{itemize}
\tightlist
\item
  \href{https://help.nytimes3xbfgragh.onion/hc/en-us/articles/115014792127-Copyright-notice}{©~2020~The
  New York Times Company}
\end{itemize}

\begin{itemize}
\tightlist
\item
  \href{https://www.nytco.com/}{NYTCo}
\item
  \href{https://help.nytimes3xbfgragh.onion/hc/en-us/articles/115015385887-Contact-Us}{Contact
  Us}
\item
  \href{https://www.nytco.com/careers/}{Work with us}
\item
  \href{https://nytmediakit.com/}{Advertise}
\item
  \href{http://www.tbrandstudio.com/}{T Brand Studio}
\item
  \href{https://www.nytimes3xbfgragh.onion/privacy/cookie-policy\#how-do-i-manage-trackers}{Your
  Ad Choices}
\item
  \href{https://www.nytimes3xbfgragh.onion/privacy}{Privacy}
\item
  \href{https://help.nytimes3xbfgragh.onion/hc/en-us/articles/115014893428-Terms-of-service}{Terms
  of Service}
\item
  \href{https://help.nytimes3xbfgragh.onion/hc/en-us/articles/115014893968-Terms-of-sale}{Terms
  of Sale}
\item
  \href{https://spiderbites.nytimes3xbfgragh.onion}{Site Map}
\item
  \href{https://help.nytimes3xbfgragh.onion/hc/en-us}{Help}
\item
  \href{https://www.nytimes3xbfgragh.onion/subscription?campaignId=37WXW}{Subscriptions}
\end{itemize}
