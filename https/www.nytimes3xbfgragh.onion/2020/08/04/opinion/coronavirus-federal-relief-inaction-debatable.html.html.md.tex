Sections

SEARCH

\protect\hyperlink{site-content}{Skip to
content}\protect\hyperlink{site-index}{Skip to site index}

\href{https://myaccount.nytimes3xbfgragh.onion/auth/login?response_type=cookie\&client_id=vi}{}

\href{https://www.nytimes3xbfgragh.onion/section/todayspaper}{Today's
Paper}

\href{/section/opinion}{Opinion}\textbar{}Inaction and a Blame Game in
the Throes of a Calamity

\url{https://nyti.ms/3fnn4U8}

\begin{itemize}
\item
\item
\item
\item
\item
\end{itemize}

Advertisement

\protect\hyperlink{after-top}{Continue reading the main story}

\href{/section/opinion}{Opinion}

Supported by

\protect\hyperlink{after-sponsor}{Continue reading the main story}

\hypertarget{inaction-and-a-blame-game-in-the-throes-of-a-calamity}{%
\section{Inaction and a Blame Game in the Throes of a
Calamity}\label{inaction-and-a-blame-game-in-the-throes-of-a-calamity}}

Federal relief has ended, and the need grows worse.

\href{https://www.nytimes3xbfgragh.onion/by/talmon-joseph-smith}{\includegraphics{https://static01.graylady3jvrrxbe.onion/images/2020/07/22/opinion/Talmon-Smith/Talmon-Smith-thumbLarge.png}}

By
\href{https://www.nytimes3xbfgragh.onion/by/talmon-joseph-smith}{Talmon
Joseph Smith}

Mr. Smith is a staff editor.

\begin{itemize}
\item
  Aug. 4, 2020
\item
  \begin{itemize}
  \item
  \item
  \item
  \item
  \item
  \end{itemize}
\end{itemize}

\includegraphics{https://static01.graylady3jvrrxbe.onion/images/2020/08/08/opinion/0804debatable/0804debatable-articleLarge.jpg?quality=75\&auto=webp\&disable=upscale}

\emph{This article is part of the Debatable newsletter. You can}
\href{https://www.nytimes3xbfgragh.onion/newsletters/debatable}{\emph{sign
up here}} \emph{to receive it on Tuesdays and Thursdays.}

On Tuesday, as congressional leaders and Treasury Secretary Steven
Mnuchin headed into yet another round of negotiations over a new
pandemic relief package, the sides remain locked in a stalemate, or a
game of chicken.

``Every day the script is the same,'' Senator
\href{https://www.bloomberg.com/news/articles/2020-08-04/pelosi-mnuchin-signal-first-halting-progress-on-stimulus-deal}{Mitch
McConnell, the majority leader, said}. ``And the script is we had a
pleasant conversation, but we don't feel like making a deal.''

In the meantime, tens of millions of Americans are being battered
financially, if not physically, by the virus. Whatever agreement
Congress and the Trump White House come to is likely to be the last
major item they tackle before this November's presidential election.
Rarely in Washington do the policy stakes get higher.

So, via Slack, I reached out to my colleague Binyamin Appelbaum, a
member of the editorial board and a former Washington correspondent for
The Times, who has covered economic debates for years, to get a firmer
grip on what's going on, what may happen next and how the emergency aid
that has been enacted so far could shape policymaking in the coming
years.

\begin{center}\rule{0.5\linewidth}{\linethickness}\end{center}

\textbf{T.J.S.:} So, Binya, the country has just gone through its worst
economic quarter in over half a century. States and cities are
experiencing budget shortfalls just as bad or worse than the crunch they
faced during the Great Recession. And emergency unemployment insurance
for as many as 30 million Americans has expired. Oh, and the pandemic is
surging through the country again.

Without putting too fine a point on it --- though feel free --- what is
Congress doing?

\textbf{B.A.:} Well, we probably shouldn't talk about Congress as a
unit. The original package of coronavirus aid, passed in March, has
mostly run out. Democrats, who control the House, passed a perfectly
serviceable extension and expansion of those aid programs in May.
Republicans, who control the Senate, insisted until recently that they
didn't see the need for another bill, so they ignored it. Senator Mitch
McConnell, the majority leader, now says he'd like to do something but
a) his plan is much narrower than what Democrats favor and b) a lot of
Senate Republicans are not onboard and c) the White House can't seem to
make up its mind.

\textbf{T.J.S.:} What, if any, workable theories are there for the
Senate's inaction so far?

\textbf{B.A.:} I think some Republicans genuinely don't understand how
the economy works. They've taken legitimate observations about the
limits of government's powers and turned them into religious principles
about the futility of any government intervention. Others, including
McConnell, may simply be playing a game of chicken, trying to see how
much Democrats are willing to concede on Republican priorities ---
notably, legal protections for businesses whose workers get sick.

The most cynical interpretation --- which isn't to say that it's wrong
--- is that Republicans expect to lose power in November. During the
early years of the Obama administration, Republicans sabotaged economic
stimulus efforts because they thought it would help them return to
power. And it worked --- although only at the expense of millions of
Americans.

\textbf{T.J.S.:} Of course, there is a certain someone --- President
Trump --- who has every reason, if he wants to do all he can to get
re-elected, to push Mr. McConnell and other Senate Republicans into
passing a package of moderately generous emergency aid extensions
through November. But by Trumpian standards, he seems to have barely
made a peep.

\textbf{B.A.:} Trump does seem to be sabotaging his own re-election
prospects. I'm not going to try to climb inside the president's head,
but I would note that it can be misleading to think of the presidency as
a single person sitting there making autonomous decisions. A president
is encrusted in advisers who filter what he hears and who inform what he
decides. So the reasons for the Senate's inaction may well apply to the
White House too.

\textbf{T.J.S.:} If only we could climb inside the president's head.

Moving on from the palace intrigue, I recently read
\href{https://www.barrons.com/articles/the-best-part-of-the-cares-act-is-dying-it-should-live-again-51596067632}{a
fascinating essay in Barron's} by Matthew Zeitlin. He observes how the
\$600 weekly payment that Congress approved for the unemployed --- many
of whom are now bringing in more than they did from their previous jobs
--- has provided an informal case study of government aid. As he puts
it, ``even with a limited run, the benefit has provided a wealth of
data, insight, and experience that should reshape how we think about
unemployment benefits and social policy.''

Because of the benefits these payments have provided to the economy, he
says, ``they --- or something like them --- could be a permanent feature
of U.S. economic policy.''

It's definitely bold, and only something elected Democrats, and those to
their left, would even consider. But vote-counting aside, what do you
make of the ideas he covers in the piece, like wage subsidies or a mini
universal basic income on the merits?

\textbf{B.A.:} It's one of the few things that has gone right since the
crisis started. For me, the \$600 payments really underscore that we
need two different kinds of unemployment benefits, one for normal times
and one for bad times. In normal times, the state only replaces a
portion of lost wages because we want to encourage people to go back to
work. In bad times, when work is hard to find, it makes sense to give
people more money. And we could switch automatically between those two
kinds of programs based on the level of unemployment.

The details, though, get a little tricky. The average weekly
unemployment benefit is a little less than \$400. Congress added the
\$600 because the combination replaces the wages of an average worker.
That meant a lot of people were making more than they did while they
were working, which drove Republicans crazy. But they acquiesced because
the Labor Department told them states didn't have the necessary systems
to calculate actual wage replacement.

But there's another big problem. Benefits vary widely by state. In
Florida, the maximum is \$275; in Massachusetts, it's \$873. If you move
to a system where the federal government is covering the difference
between state benefits and full wage replacement, you're basically
subsidizing Florida's stinginess.

I do have reservations about viewing expanded unemployment benefits as a
proof-of-concept for U.B.I. The rationale for providing people with
money during economic downturns is very different than the rationale for
providing a constant subsidy. This probably isn't the place to get into
the details.

\textbf{T.J.S.:} All fair points. In the spirit of this being a
relatively informal place, not fit for all the details, here's a
question based purely on speculation: Assuming the polls hold relatively
steady and President Trump loses re-election, what is the first economic
policy that a hypothetical Joe Biden administration should pursue?

\textbf{B.A.:} My answer is baby bonds, actually. The basic idea is that
every kid gets a \$2,000 government savings account when they're born,
and then additional annual contributions based on family income. To me,
one of the defining challenges facing our society is to give kids a
better chance to succeed. Right now, it matters way too much how much
money your parents have --- and those wealth inequalities really map
onto racial lines.

Giving every kid a nest egg is a very powerful idea: It would improve
every kid's opportunities. It's like Social Security, but for the
beginning of life.

\textbf{T.J.S.:} Indeed. I think that's a good place to leave it.
Thanks, Binya.

\textbf{B.A.:} Thank you.

\emph{Do you have a point of view we missed? Email us at}
\href{mailto:debatable@NYTimes.com}{\emph{debatable@NYTimes.com}}\emph{.
Please note your name, age and location in your response, which may be
included in the next newsletter.}

\begin{center}\rule{0.5\linewidth}{\linethickness}\end{center}

\hypertarget{more-on-congress-pandemic-response}{%
\subsubsection{MORE ON CONGRESS' PANDEMIC
RESPONSE}\label{more-on-congress-pandemic-response}}

``\href{https://www.nytimes3xbfgragh.onion/2020/07/30/opinion/mitch-mcconnell-coronavirus-economy.html}{Mitch
McConnell Could Rescue Millions. What Is He Waiting For?}'' \emph{{[}The
New York Times{]}}

``\href{https://www.bloomberg.com/news/articles/2020-08-04/pelosi-mnuchin-signal-first-halting-progress-on-stimulus-deal}{Virus
Relief Talks Resume With Little Movement on Biggest Issues}''
\emph{{[}Bloomberg{]}}

``\href{https://www.nytimes3xbfgragh.onion/2020/07/22/opinion/sunday/unemployment-supplement-congress.html}{If
Congress Cuts Jobless Benefits, `It Would Be a Disaster}''' \emph{{[}The
New York Times{]}}

``\href{https://www.nytimes3xbfgragh.onion/2020/08/01/business/economy/housing-overcrowding-coronavirus.html}{12
People in a 3-Bedroom House, Then the Virus Entered the Equation}''
\emph{{[}The New York Times{]}}

\begin{center}\rule{0.5\linewidth}{\linethickness}\end{center}

\hypertarget{what-youre-saying}{%
\subsubsection{WHAT YOU'RE SAYING}\label{what-youre-saying}}

\emph{Here's what readers had to say about the last debate:}
\href{https://www.nytimes3xbfgragh.onion/2020/07/30/opinion/fascism-us.html}{\emph{Fascism:
A Concern}}\emph{.}

Peter from Truro, Mass.: ``I feel one part of my understanding of
fascism that seems left out is the corporate/business cooperation that
certainly existed in European fascism in the '30s and '40s. Hitler
needed cooperation from many German and U.S. corporations to support his
existence and war effort. Our corporations certainly have incredible
influence through donations and are rewarded with business-friendly
policies now and historically.''

Erik from Chicago: ``A fascist politician or leader is one who is a
product of a nation with extreme capitalism (wealth inequality) who
demonizes minorities to gain or maintain power.

``In a sense, fascism is the next evolution of extreme capitalism. This
is not 1939, so fascism today will be a little different than fascism in
the past. For example authoritarians of all stripes today maintain
trappings of democracy; Russia, Iran and North Korea, for example, all
have sham elections.

``Let's not sugarcoat it by debating how it may or may not be fascism.
The takeaway of having the fascism debate at all is that we're living in
incredibly extreme times.''

Advertisement

\protect\hyperlink{after-bottom}{Continue reading the main story}

\hypertarget{site-index}{%
\subsection{Site Index}\label{site-index}}

\hypertarget{site-information-navigation}{%
\subsection{Site Information
Navigation}\label{site-information-navigation}}

\begin{itemize}
\tightlist
\item
  \href{https://help.nytimes3xbfgragh.onion/hc/en-us/articles/115014792127-Copyright-notice}{©~2020~The
  New York Times Company}
\end{itemize}

\begin{itemize}
\tightlist
\item
  \href{https://www.nytco.com/}{NYTCo}
\item
  \href{https://help.nytimes3xbfgragh.onion/hc/en-us/articles/115015385887-Contact-Us}{Contact
  Us}
\item
  \href{https://www.nytco.com/careers/}{Work with us}
\item
  \href{https://nytmediakit.com/}{Advertise}
\item
  \href{http://www.tbrandstudio.com/}{T Brand Studio}
\item
  \href{https://www.nytimes3xbfgragh.onion/privacy/cookie-policy\#how-do-i-manage-trackers}{Your
  Ad Choices}
\item
  \href{https://www.nytimes3xbfgragh.onion/privacy}{Privacy}
\item
  \href{https://help.nytimes3xbfgragh.onion/hc/en-us/articles/115014893428-Terms-of-service}{Terms
  of Service}
\item
  \href{https://help.nytimes3xbfgragh.onion/hc/en-us/articles/115014893968-Terms-of-sale}{Terms
  of Sale}
\item
  \href{https://spiderbites.nytimes3xbfgragh.onion}{Site Map}
\item
  \href{https://help.nytimes3xbfgragh.onion/hc/en-us}{Help}
\item
  \href{https://www.nytimes3xbfgragh.onion/subscription?campaignId=37WXW}{Subscriptions}
\end{itemize}
