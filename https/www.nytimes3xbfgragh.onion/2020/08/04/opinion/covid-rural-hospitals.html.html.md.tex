Sections

SEARCH

\protect\hyperlink{site-content}{Skip to
content}\protect\hyperlink{site-index}{Skip to site index}

\href{https://myaccount.nytimes3xbfgragh.onion/auth/login?response_type=cookie\&client_id=vi}{}

\href{https://www.nytimes3xbfgragh.onion/section/todayspaper}{Today's
Paper}

\href{/section/opinion}{Opinion}\textbar{}`If I Hadn't Been Transferred,
I Would Have Died'

\url{https://nyti.ms/2DaW25s}

\begin{itemize}
\item
\item
\item
\item
\item
\item
\end{itemize}

Advertisement

\protect\hyperlink{after-top}{Continue reading the main story}

\href{/section/opinion}{Opinion}

Supported by

\protect\hyperlink{after-sponsor}{Continue reading the main story}

\hypertarget{if-i-hadnt-been-transferred-i-would-have-died}{%
\section{`If I Hadn't Been Transferred, I Would Have
Died'}\label{if-i-hadnt-been-transferred-i-would-have-died}}

Conversations about death from Covid-19 revolve around patient
characteristics. But hospitals matter, too.

By Daniela J. Lamas

Dr. Lamas is a critical care doctor.

\begin{itemize}
\item
  Aug. 4, 2020
\item
  \begin{itemize}
  \item
  \item
  \item
  \item
  \item
  \item
  \end{itemize}
\end{itemize}

\includegraphics{https://static01.graylady3jvrrxbe.onion/images/2020/08/04/opinion/04Lamas/04Lamas-articleLarge.jpg?quality=75\&auto=webp\&disable=upscale}

There she was. After more than three weeks on the ventilator, after
battling weakness and delirium on the general medical floor and a stay
at the long-term rehab hospital where she rebuilt the strength to walk
again, my patient had made it home. The dark shadows beneath her eyes
were fading. Her skin was tanned. The persistent shortness of breath had
finally abated, and she had recently run four miles to commemorate four
months since she was diagnosed with Covid-19.

Four months. I closed my eyes and found myself once again in those early
days of the pandemic, clustered outside her room with a team of doctors
and nurses. Nearly two weeks in, she still needed high levels of support
from the ventilator and we were starting to talk about the impossible
decisions we might face if her lungs never improved. But we waited,
because this was a new virus and we did not know its course, and because
we had the resources to do so. And now there she was, in clinic ---
months later, doing far better than I would have predicted.

I have been surprised by similar recoveries in the past weeks. People we
thought could die, or at least end up significantly impaired, have made
it home. But there is something troubling about this, too. It is clear
to me that there was no one specific therapy that determined the
outcomes of our sickest coronavirus patients in the intensive care unit.

On the contrary. While even the best possible treatment couldn't save
everyone, those who survived did so because of meticulous critical care,
which requires a combination of resources and competency that is only
available to a minority of hospitals in this country. And now, even as
we race toward the hope of a magic bullet for this virus, we must openly
acknowledge that disparity --- and work to address it.

Since the beginning of this crisis, conversations about death from
Covid-19 have revolved around patient characteristics --- men are more
likely to die than women, as are people who are older or obese, or those
with co-morbidities. But we now know that the hospital matters, too.

In
\href{https://jamanetwork.com/journals/jamainternalmedicine/fullarticle/2768602}{a
large study} that was recently published in the journal JAMA Internal
Medicine, a team of researchers examined hospital mortality rates in
more than 2,200 critically ill coronavirus patients in 65 hospitals
throughout the country. Their findings? Patients admitted to hospitals
with fewer than 50 I.C.U. beds --- smaller hospitals --- were more than
three times more likely to die than patients admitted to larger
hospitals.

Though they were not able to study factors like staffing and hospital
strain, these likely contributed. In fact, a recent
\href{https://www.nytimes3xbfgragh.onion/2020/07/01/nyregion/Coronavirus-hospitals.html}{investigative
piece in The Times} examined mortality data for hospitals in New York
City --- and found that at the peak of the pandemic, patients at some
community hospitals (with lower staffing and worse equipment) were three
times more likely to die as patients in medical centers in the
wealthiest areas.

Knowing firsthand what it requires to keep critically ill Covid-19
patients alive, this does not surprise me. Though the public has largely
focused on new treatments --- with excitement and controversy swirling
around remdesivir and dexamethasone and convalescent plasma --- none of
these are any use without the people and systems to deliver critical
care, a laborious and resource-intensive process.

In the I.C.U., we must interpret and react to each indicator. Our nurses
are frequently at the bedside, attuned to the most minute change. We
make constant small tweaks to the ventilator and to our medications to
support blood pressure. Though it looks passive in a way --- a comatose
patient in a bed --- and is not at all glamorous, critical care is an
immensely active process.

We are all familiar with the images of Covid-19 patients lying on their
chests, and we know that prone positioning saves lives. But the simple
act of turning a critically ill patient is physically strenuous and, if
done hastily, treacherous. Breathing tubes and intravenous lines can
become dislodged. The head must be repositioned every two hours.

At my hospital, during the height of the pandemic, we formed a dedicated
``prone team'' of respiratory and physical therapists who were available
24 hours a day. This spared the bedside nurses and kept patients as safe
as possible. Even so, breathing tubes became kinked, and on at least one
occasion, we had to urgently replace a breathing tube --- a risky
procedure. This is why in some hospitals, prone positioning might not
have been offered at all. Indeed, the JAMA study found rates of prone
positioning to range from just under 5 percent at one hospital to nearly
80 percent at another. Patients would have suffered as a result.

Anyone who has cared for a coronavirus patient knows how quickly they
can crash. Thick mucus blocks airways and endotracheal tubes. Oxygen
levels plummet. Heart rhythms go haywire. As a doctor, I'll admit that
we are rarely the first to intervene in these moments of crisis.
Instead, we rely on nurses and respiratory therapists. More times than I
would like to count, I have watched with gratitude as their
interventions --- suctioning, repositioning a breathing tube, increasing
the dose of medications to raise blood pressure --- avert certain
disaster. It is humbling to realize that had our nurses been spread too
thin, these relatively small events would have turned catastrophic.

Perhaps most importantly, because we had the resources to do so, we were
able to give our patients time for their lungs to recover. I think of
one man, a father, so sick that he was dependent not just on the
ventilator but also on a heart-lung bypass machine. These machines, and
the staff who know how to manage them, are a truly limited resource.
Large academic centers have five of them, maybe 10. Some community
hospitals do not have any.

This man had been on the machine for weeks, encountering one
complication after another. He bled, we stopped blood thinners, and then
surgeons had to rush in overnight to replace a part of the machine when
it clotted off. There seemed to be no way out. But then, even as we
prepared to say \emph{enough}, his lungs started to improve. I remember
standing outside his room one overnight, amazed, as his stiff lungs
began to work with the ventilator once again.

He has now left the hospital. On the night of his return home, his son
sent me a note: ``Finally family is back, and that is the best feeling
of this world.''

You might say he was lucky. But so were we. He was able to return home
not because of any 11th-hour save on our part, but because we were able
to watch and wait. And we could only afford to do so because here in
Boston, we were busy but never underwater. Of course, we made mistakes,
miscalculations and errors in judgment as we learned about this new
disease. But we were in a privileged position. It could have been far
worse. And as the pandemic tears through rural areas of the country with
even less access to resource-rich hospitals, I am worried that the
inequities of this virus will only become more entrenched.

Just as we devote resources to finding a vaccine, we must also devote
resources to helping hospitals deliver high-quality critical care. Maybe
that will mean better allocating the resources we do have through a more
robust, coordinated system of hospital-to-hospital patient transfers
within each region. Maybe it means creating something akin to dedicated
coronavirus centers of excellence throughout the country, with certain
core competencies. Maybe it will mean expanding the reach of experienced
critical care hospitals through telehealth. This will not be easy. But
as this virus will be with us for the foreseeable future, it is our duty
to try.

As the video visit with my patient ended that day, she reminded me that
she had been transferred to us from a small hospital in the western part
of our state. ``If I hadn't been transferred, I would have died,'' she
said. I paused, reflecting on that. What had we done for her, really? We
had never enrolled her in a clinical trial. There was no mystery
diagnosis to be solved, no high-risk procedure performed. We simply did
our best to minimize damage to her lungs and keep her other organs
functioning while we waited.

Which makes it even more painful to admit that she might be right.

Daniela J. Lamas is a critical care doctor at Brigham and Women's
Hospital in Boston.

\emph{The Times is committed to publishing}
\href{https://www.nytimes3xbfgragh.onion/2019/01/31/opinion/letters/letters-to-editor-new-york-times-women.html}{\emph{a
diversity of letters}} \emph{to the editor. We'd like to hear what you
think about this or any of our articles. Here are some}
\href{https://help.nytimes3xbfgragh.onion/hc/en-us/articles/115014925288-How-to-submit-a-letter-to-the-editor}{\emph{tips}}\emph{.
And here's our email:}
\href{mailto:letters@NYTimes.com}{\emph{letters@NYTimes.com}}\emph{.}

\emph{Follow The New York Times Opinion section on}
\href{https://www.facebookcorewwwi.onion/nytopinion}{\emph{Facebook}}\emph{,}
\href{http://twitter.com/NYTOpinion}{\emph{Twitter (@NYTopinion)}}
\emph{and}
\href{https://www.instagram.com/nytopinion/}{\emph{Instagram}}\emph{.}

Advertisement

\protect\hyperlink{after-bottom}{Continue reading the main story}

\hypertarget{site-index}{%
\subsection{Site Index}\label{site-index}}

\hypertarget{site-information-navigation}{%
\subsection{Site Information
Navigation}\label{site-information-navigation}}

\begin{itemize}
\tightlist
\item
  \href{https://help.nytimes3xbfgragh.onion/hc/en-us/articles/115014792127-Copyright-notice}{©~2020~The
  New York Times Company}
\end{itemize}

\begin{itemize}
\tightlist
\item
  \href{https://www.nytco.com/}{NYTCo}
\item
  \href{https://help.nytimes3xbfgragh.onion/hc/en-us/articles/115015385887-Contact-Us}{Contact
  Us}
\item
  \href{https://www.nytco.com/careers/}{Work with us}
\item
  \href{https://nytmediakit.com/}{Advertise}
\item
  \href{http://www.tbrandstudio.com/}{T Brand Studio}
\item
  \href{https://www.nytimes3xbfgragh.onion/privacy/cookie-policy\#how-do-i-manage-trackers}{Your
  Ad Choices}
\item
  \href{https://www.nytimes3xbfgragh.onion/privacy}{Privacy}
\item
  \href{https://help.nytimes3xbfgragh.onion/hc/en-us/articles/115014893428-Terms-of-service}{Terms
  of Service}
\item
  \href{https://help.nytimes3xbfgragh.onion/hc/en-us/articles/115014893968-Terms-of-sale}{Terms
  of Sale}
\item
  \href{https://spiderbites.nytimes3xbfgragh.onion}{Site Map}
\item
  \href{https://help.nytimes3xbfgragh.onion/hc/en-us}{Help}
\item
  \href{https://www.nytimes3xbfgragh.onion/subscription?campaignId=37WXW}{Subscriptions}
\end{itemize}
