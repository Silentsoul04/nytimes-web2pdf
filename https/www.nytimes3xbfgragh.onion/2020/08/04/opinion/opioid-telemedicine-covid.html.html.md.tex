Sections

SEARCH

\protect\hyperlink{site-content}{Skip to
content}\protect\hyperlink{site-index}{Skip to site index}

\href{https://myaccount.nytimes3xbfgragh.onion/auth/login?response_type=cookie\&client_id=vi}{}

\href{https://www.nytimes3xbfgragh.onion/section/todayspaper}{Today's
Paper}

\href{/section/opinion}{Opinion}\textbar{}Using Telemedicine to Treat
Opioid Addiction

\url{https://nyti.ms/33o9k97}

\begin{itemize}
\item
\item
\item
\item
\item
\end{itemize}

Advertisement

\protect\hyperlink{after-top}{Continue reading the main story}

\href{/section/opinion}{Opinion}

Supported by

\protect\hyperlink{after-sponsor}{Continue reading the main story}

Fixes

\hypertarget{using-telemedicine-to-treat-opioid-addiction}{%
\section{Using Telemedicine to Treat Opioid
Addiction}\label{using-telemedicine-to-treat-opioid-addiction}}

Getting medication long meant seeing a licensed provider. Now a strategy
for evading Covid-19 makes treatment available via the web.

\includegraphics{https://static01.graylady3jvrrxbe.onion/images/2019/02/13/opinion/tina-rosenberg/tina-rosenberg-thumbLarge-v2.png}

By Tina Rosenberg

Ms. Rosenberg is an author and a former editorial writer for The New
York Times.

\begin{itemize}
\item
  Aug. 4, 2020, 5:00 a.m. ET
\item
  \begin{itemize}
  \item
  \item
  \item
  \item
  \item
  \end{itemize}
\end{itemize}

\includegraphics{https://static01.graylady3jvrrxbe.onion/images/2020/08/04/opinion/04Fixes2/merlin_174571176_9112759a-6edf-4f19-981a-f42d6f93a807-articleLarge.jpg?quality=75\&auto=webp\&disable=upscale}

Covid-19 has made life much harder for people with opioid addiction. But
the response to the virus has also revealed a way forward that could
radically expand effective treatment and reduce overdose deaths.

Until now, getting effective treatment depended on where you lived.
Forty percent of American counties --- much of Appalachia, for example
--- have \href{https://oig.hhs.gov/oei/reports/oei-12-17-00240.asp}{no
providers licensed to prescribe buprenorphine}, the most successful
treatment so far.

But the pandemic has made it possible to see a licensed provider from
home, and that could make buprenorphine treatment available anywhere.

Michelle (she asked me to not use her family name) is 57, lives near
Wilkes-Barre, Pa., and works from home as a customer service
representative. Her computer allows her to live a good life --- after
eight years on heroin that followed many years on other drugs.

Over the internet, she sees a psychiatric nurse practitioner, Roseanna
Melle, who offers light counseling and prescribes the widely used drug
Suboxone --- a combination of buprenorphine and the overdose reversal
drug naloxone. It blocks her cravings and prevents withdrawal symptoms,
but doesn't get her high. She feels \ldots{} normal.

Before starting telemedicine in April, Michelle got Suboxone at a local
addiction medicine clinic. ``It was a revolving door --- sometimes
standing room only,'' she said. ``Who wouldn't want to just do your
appointment in the comfort and privacy of your own home?''

Robert, 30, another of Ms. Melle's patients, from nearby Scranton, said:
``Home treatment lessens the shame for me. I don't have to worry what
doctors around here think. My Suboxone --- it's just a medication. I
don't think about it. I just take it and go about my day.''

Ms. Melle is one of two --- soon to be four --- providers at a new
telemedicine company in Pennsylvania called Ophelia. It's one of several
companies started in the last few years that prescribe Suboxone: Bicycle
Health, Bright Heart Health, Workit Health, PursueCare, Boulder Care.
Each is slightly different and they operate in different states.

The science is
\href{https://www.ncbi.nlm.nih.gov/books/NBK534504/}{unequivocal}: The
only effective treatment for opioid use disorder is what is called
``medication-assisted treatment.'' Medication makes patients
\href{https://www.ncbi.nlm.nih.gov/books/NBK534504/}{far more successful
in treatment and less likely to overdose.}

But at least 80 percent of people who could benefit from it don't
receive it. Some are deterred by the stigma still attached to taking
Suboxone. But likely more important is the shortage of local providers.
In 2016, the Obama administration increased the number of providers and
allowed them to treat more patients. It's still far from enough.

So people buy Suboxone from their drug dealer. ``I bought it on the
black market --- a lot of people do,'' Michelle said. ``They sometimes
try it because they can't get their drug of choice. Or they're thinking
about getting clean, and they don't have insurance and don't want to go
through the red tape.''

Ophelia's medical director, Arthur Robin Williams, an addiction
psychiatrist and assistant professor at Columbia University said, ``It
is easier for people to get the dangerous drugs than to get the
treatment for addiction.''

Some clinics have been using telemedicine for the last few years, but
patients still faced many barriers to treatment. The first visit had to
be in person --- which meant that access to treatment still depended on
where you lived. Doctors were paid a pittance for telehealth
appointments, so few doctors offered them. A patient could get only a
week's supply of buprenorphine at a time.

Advocates for treatment have campaigned to remove these barriers ---
unsuccessfully until Covid-19.

The pandemic has led to
\href{https://www.samhsa.gov/sites/default/files/faqs-for-oud-prescribing-and-dispensing.pdf}{regulatory
changes}: Treatment can now be entirely virtual, including the first
appointment. Medicare now pays providers the same for a video
appointment as a conventional one --- many insurers and Medicaid
programs have followed. Patients can get a month's prescription for
buprenorphine instead of just a week's.

These changes are temporary, but everyone I talked to wanted them made
permanent.

``This has just catapulted through this crisis,'' said Allegra Schorr, a
Manhattan doctor and the president of Compa, a New York State coalition
of medication-assisted treatment providers and advocates. ``Now
everybody's doing it,'' she said. ``Within this environment, it
certainly seems to be working.''

Prevention Point Philadelphia is among the largest harm reduction
centers in the country. It offers syringe exchange, medical care, social
services --- and now, food. The majority of its patients are without
homes, and most suffer from multiple mental and physical illnesses.

Prevention Point has offered medication-assisted treatment for 12 years
and now treats 268 patients --- many out of a mobile van. ``We try to
wipe out any barriers,'' said Silvana Mazzella, associate executive
director.

\includegraphics{https://static01.graylady3jvrrxbe.onion/images/2020/08/04/opinion/04Fixes1/04Fixes1-articleLarge.jpg?quality=75\&auto=webp\&disable=upscale}

In mid-March, Prevention Point started prescribing Suboxone through
telemedicine. But many of its patients can't do the ``tele'' part. They
don't have phones or have no-data phones and can't afford the airtime
for an appointment. Prevention Point has given patients some donated
phones with data and minutes of usage included, and is seeking more.

Ms. Mazzella said telemedicine has helped the patients who can use it.
``It's a reduction of the hassle, wait times, anxiety and fear of
withdrawal in a waiting room,'' she said. But the switch has also
changed the foundations of treatment,'' she added. ``We have moved to
more of a harm reduction model. We have taken away drug screens and
things that feel punitive, things patients must do to prove they're a
good patient. We've removed the stigma and the power dynamic that
typically exists in a clinic. We are putting the same level of trust in
patients as you would with diabetes or hypertension.''

She said telemedicine patients have proven more likely to fill their
prescriptions than patients who had appeared in person in the past.

Although Prevention Point doesn't do video drug tests, other practices
do. Ophelia sends Robert a kit. On camera, he unseals the box, which has
test strips built in to identify buprenorphine and 11 other drugs. Then
he fills a tube with saliva to be tested for the presence of opioids,
fits the tube into the box so that the test strips can work and shows
Ms. Melle the result.

Ophelia markets to consumers. It has 90 patients, all self-pay while the
company waits to be accepted by insurance plans and Medicaid.

Some other telehealth companies offer therapy. Ophelia doesn't, but
neither do most doctors who prescribe Suboxone. Zack Gray, Ophelia's
founder, explained: ``We want to make the barrier as low as possible for
people to opt into treatment --- then give them the choice to opt into
therapy.''

Dr. Williams, the medical director, said: ``When I first heard about
Ophelia, I was terrified. I thought the idea was ludicrous.'' He worried
about doing proper clinical management over video.

Both Robert and Michelle, by contrast, said that their previous doctors
had left them feeling rushed and judged, and that Ms. Melle knows them
far better.

``Bringing this into the home may not be quite as robust and nuanced as
in-person care,'' but it is necessary, Dr. Williams said. ``The U.S. has
40,000 to 45,000 people dying from opioid overdoses every year,'' he
continued. ``We haven't done anything at the level needed, and it's been
20 years.''

\emph{To receive alerts for Fixes columns, sign up}
\href{http://eepurl.com/ABIxL}{\emph{here.}}

Tina Rosenberg is a co-founder of the
\href{http://solutionsjournalism.org}{Solutions Journalism Network},
which supports rigorous reporting about responses to social problems.
She won a Pulitzer Prize for her book
``\href{http://www.randomhouse.com/catalog/display.pperl?isbn=9780679744993}{The
Haunted Land: Facing Europe's Ghosts After Communism},'' and is the
author, most recently, of
``\href{http://books.wwnorton.com/books/Join-the-Club}{Join the Club:
How Peer Pressure Can Transform the World}'' and the World War II spy
story e-book
\href{https://www.goodreads.com/book/show/16124470-d-for-deception}{``D
for Deception.''}

\emph{The Times is committed to publishing}
\href{https://www.nytimes3xbfgragh.onion/2019/01/31/opinion/letters/letters-to-editor-new-york-times-women.html}{\emph{a
diversity of letters}} \emph{to the editor. We'd like to hear what you
think about this or any of our articles. Here are some}
\href{https://help.nytimes3xbfgragh.onion/hc/en-us/articles/115014925288-How-to-submit-a-letter-to-the-editor}{\emph{tips}}\emph{.
And here's our email:}
\href{mailto:letters@NYTimes.com}{\emph{letters@NYTimes.com}}\emph{.}

\emph{Follow The New York Times Opinion section on}
\href{https://www.facebookcorewwwi.onion/nytopinion}{\emph{Facebook}}\emph{,}
\href{http://twitter.com/NYTOpinion}{\emph{Twitter (@NYTopinion)}}
\emph{and}
\href{https://www.instagram.com/nytopinion/}{\emph{Instagram}}\emph{.}

Advertisement

\protect\hyperlink{after-bottom}{Continue reading the main story}

\hypertarget{site-index}{%
\subsection{Site Index}\label{site-index}}

\hypertarget{site-information-navigation}{%
\subsection{Site Information
Navigation}\label{site-information-navigation}}

\begin{itemize}
\tightlist
\item
  \href{https://help.nytimes3xbfgragh.onion/hc/en-us/articles/115014792127-Copyright-notice}{©~2020~The
  New York Times Company}
\end{itemize}

\begin{itemize}
\tightlist
\item
  \href{https://www.nytco.com/}{NYTCo}
\item
  \href{https://help.nytimes3xbfgragh.onion/hc/en-us/articles/115015385887-Contact-Us}{Contact
  Us}
\item
  \href{https://www.nytco.com/careers/}{Work with us}
\item
  \href{https://nytmediakit.com/}{Advertise}
\item
  \href{http://www.tbrandstudio.com/}{T Brand Studio}
\item
  \href{https://www.nytimes3xbfgragh.onion/privacy/cookie-policy\#how-do-i-manage-trackers}{Your
  Ad Choices}
\item
  \href{https://www.nytimes3xbfgragh.onion/privacy}{Privacy}
\item
  \href{https://help.nytimes3xbfgragh.onion/hc/en-us/articles/115014893428-Terms-of-service}{Terms
  of Service}
\item
  \href{https://help.nytimes3xbfgragh.onion/hc/en-us/articles/115014893968-Terms-of-sale}{Terms
  of Sale}
\item
  \href{https://spiderbites.nytimes3xbfgragh.onion}{Site Map}
\item
  \href{https://help.nytimes3xbfgragh.onion/hc/en-us}{Help}
\item
  \href{https://www.nytimes3xbfgragh.onion/subscription?campaignId=37WXW}{Subscriptions}
\end{itemize}
