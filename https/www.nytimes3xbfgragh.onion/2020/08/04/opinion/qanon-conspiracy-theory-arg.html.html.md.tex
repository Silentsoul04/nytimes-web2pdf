Sections

SEARCH

\protect\hyperlink{site-content}{Skip to
content}\protect\hyperlink{site-index}{Skip to site index}

\href{https://myaccount.nytimes3xbfgragh.onion/auth/login?response_type=cookie\&client_id=vi}{}

\href{https://www.nytimes3xbfgragh.onion/section/todayspaper}{Today's
Paper}

\href{/section/opinion}{Opinion}\textbar{}How QAnon Creates a Dangerous
Alternate Reality

\url{https://nyti.ms/2C2vumi}

\begin{itemize}
\item
\item
\item
\item
\item
\item
\end{itemize}

Advertisement

\protect\hyperlink{after-top}{Continue reading the main story}

\href{/section/opinion}{Opinion}

Supported by

\protect\hyperlink{after-sponsor}{Continue reading the main story}

\hypertarget{how-qanon-creates-a-dangerous-alternate-reality}{%
\section{How QAnon Creates a Dangerous Alternate
Reality}\label{how-qanon-creates-a-dangerous-alternate-reality}}

Game designer Adrian Hon says the conspiracy theory parallels the
immersive worlds of alternate reality games.

\href{https://www.nytimes3xbfgragh.onion/by/charlie-warzel}{\includegraphics{https://static01.graylady3jvrrxbe.onion/images/2019/03/15/opinion/charlie-warzel/charlie-warzel-thumbLarge-v3.png}}

By \href{https://www.nytimes3xbfgragh.onion/by/charlie-warzel}{Charlie
Warzel}

Opinion writer at large

\begin{itemize}
\item
  Aug. 4, 2020
\item
  \begin{itemize}
  \item
  \item
  \item
  \item
  \item
  \item
  \end{itemize}
\end{itemize}

\includegraphics{https://static01.graylady3jvrrxbe.onion/images/2020/08/04/opinion/04warzel1/04warzel1-articleLarge.jpg?quality=75\&auto=webp\&disable=upscale}

In 2019, the F.B.I. cited QAnon as one of the dangerous conspiracy
theories posing domestic terrorist threats to the United States and
cited past incitements of violence from its adherents. Despite its
fringe origins, the conspiracy movement continues to grow in troubling
ways. QAnon-supporting candidates are running for office in surprising
numbers (Media Matters' Alex Kaplan
\href{https://www.mediamatters.org/qanon-conspiracy-theory/qanon-may-be-coming-congress-and-journalists-need-be-ready}{reports
that} ``at least 14 candidates made it out of primaries to the ballot in
November or to primary runoffs.'') The movement has been tacitly
embraced by
\href{https://www.washingtonpost.com/politics/how-the-trump-campaign-came-to-court-qanon-the-online-conspiracy-movement-identified-by-the-fbi-as-a-violent-threat/2020/08/01/dd0ea9b4-d1d4-11ea-9038-af089b63ac21_story.html?utm_campaign=wp_politics_am\&utm_medium=email\&utm_source=newsletter\&wpisrc=nl_politics}{President
Trump and his re-election campaign}, who've amplified QAnon accounts and
even some of their memes.

For those who haven't paid attention to the community since the early
days, the movement's growing popularity is alarming and often confusing.
Some have compared it to a
\href{https://www.theatlantic.com/magazine/archive/2020/06/qanon-nothing-can-stop-what-is-coming/610567/}{budding
religion}. Personally, the phenomenon has always struck me as
\href{https://www.buzzfeednews.com/article/charliewarzel/behind-the-alarming-rise-of-the-online-vigilante-detective}{a
dark iteration of vigilante investigations that grew popular} on message
boards in the 2010s --- citizen journalism gone wrong.

\includegraphics{https://static01.graylady3jvrrxbe.onion/images/2020/08/04/opinion/04warzel2/merlin_157571772_92b61242-a040-43c5-93db-6ba987028cc9-articleLarge.jpg?quality=75\&auto=webp\&disable=upscale}

Perhaps the best explanation I've heard for the movement's popularity
comes from Adrian Hon, the chief executive of the gaming company Six to
Start and a designer of alternate reality games or ARGs. Unlike video
games, alternate reality games aren't played on a console --- they use
the world as their storytelling platform. There's no one particular
medium. The story takes place in real time and seems to exist in the
world. So game designers hide clues and puzzles in websites, apps and
even newspaper advertisements. It's a bit like a networked treasure hunt
that turns the world around you into a game.

For Mr. Hon, that phenomenon resembled the dynamics governing QAnon. In
a viral
\href{https://mssv.net/2020/08/02/what-args-can-teach-us-about-qanon/}{Twitter
thread and follow-up post}, he argued that ``QAnon pushes the same
buttons that ARGs do, whether by intention or by coincidence. In both
cases, `do your research' leads curious onlookers to a cornucopia of
brain-tingling information.''

I spoke to Mr. Hon about why QAnon is so popular, if it can be stopped
and what, if anything, we can learn from the movement's rise. The
conversation has been edited for length and clarity.

\textbf{What are some of the most striking similarities you see between
alternate reality games and QAnon, besides being sprawling and complex?}

Alternate reality games incorporate the internet and websites, real
world interactions, advertisements in newspapers, smartphone apps, any
medium we can get ahold of in order to produce the most immersive story
possible. I saw the parallel with QAnon for two reasons. QAnon is a
uniquely 21st century conspiracy theory. There have been others but
QAnon was born on forums like 4chan and 8chan, and the way that people
interact with it initially is so purely online. But the effects bleed
into the real world much like an alternate reality game.

Image

Medical equipment workers at a company in Allentown, Penn., showing
their support of QAnon and President Trump during Trump's visit to the
there in May.Credit...Carlos Barria/Reuters

But specifically what caught my eye is that almost everyone who
discovers QAnon uses a phrase like, ``I did my research.'' I kept
hearing that and I couldn't get it out of my head. This research is,
basically, typing things into Google but when they do, they go down the
rabbit hole. They open a fascinating fantasy world of secret wars and
cabals and Hillary Clinton controlling things, and it offers convenient
explanations for things that feel inexplicable or wrong about the world.
It reminded me specifically of how people get to alternate reality
games. Through these research rabbit holes.

\textbf{There's a phenomenon you mention in these games called ``This is
Not a Game.'' Can you explain that?}

``This is Not a Game'' is the idea that the game is more enjoyable for
players if we try and avoid to break the suspension of disbelief as much
as possible. This came to the fore with a game called, The Beast.
Microsoft was behind it but nobody knew for a long time in. It started
with a cryptic message on a movie poster, which, if you Googled it, led
you a fake blog, which led to other websites and email addresses and
more. Those playing knew it wasn't real but the design made every effort
to seem like it was.

\textbf{And you note that, in order to prolong the suspension of
disbelief, you never admit any designer error, you just add to the story
line with more complexity like it was the plan all along?}

Sometimes as a designer you will change something in the game on the fly
based on how people are playing it. Sometimes their instincts and
suggestions add depth to the game and so you quickly rewrite. You can
see that happen with QAnon. New theories and tangents appear at dead
ends.

\textbf{In your piece you suggest that alternate reality games ``reward
active discovery, the drawing of connections between clues, the
delicious sensation of a hunch that pays off after hours or days of
work.'' How does QAnon do that?}

There are a certain type of people who are attracted to alternate
reality games and they are quite devoted. They like puzzle solving in
the same way people like murder mysteries or crossword puzzles. As game
designers we encourage that mind-set. We provide extremely difficult
tasks that only 1 in 1,000 people could solve. And we do that because
that one person who \emph{can} solve it will feel like a hero because
this weird talent they have is put to use. Alternate reality game
designers like to reward its community for niche skills.

This is at play in QAnon. Many people feel alienated and left behind by
the world. There's something about QAnon like ARGs that reward and
involve people for being who they are. They create a community that lets
people show off their ``research'' skills and those people become
incredibly valuable to the community.

Image

Supporters of President Trump in line for a campaign rally in Johnson
City, Tenn., in 2018.Credit...Sean Rayford/Getty Images

\textbf{Unlike lots of immersive games, QAnon blurs the lines between
the writers/creators/players. Why is that important?}

In a classical game or story you have a very distinct difference between
those writing or designing and those playing. With QAnon you have this
figure {[}Q{]} who has a stable identity and plants the seed frequently.
But there's just so many theories involved in the greater QAnon universe
that are only tangentially related to the figure of Q. And you see this
---
\href{https://www.vice.com/en_us/article/v7gz53/the-conspiracy-singularity-has-arrived}{QAnon
has absorbed every other conspiracy theory.} What would happen if Q
stopped posting content forever? Would it die out? Maybe. But maybe not.
And the reason it's unclear is because so many people in the community
have essentially built out their own theories and story lines and
generated their own massive followings.

\textbf{Do you think that this feeling of community makes QAnon that
much more resilient?}

There's a real Darwinian process in these communities. Just tons of
people with ideas and so many forums with up-voting and sharing. Nine
hundred and 99 out of 1,000 theories are utterly bonkers but one might
hit, maybe because it is slightly harder to disprove or a bit more
compelling and it will immediately win out. It's a collaborative fiction
built on wild speculation that hardens into reality.

\textbf{Trying to moderate online communities likely won't stop QAnon
from spreading in social media comments or private chat groups or
unmoderated forums. And the only way to stop people from mistaking
speculation from fact is for them to want to stop. This isn't really a
question but that strikes me as very, very dark stuff.}

Don't get me wrong, it is good thing that Twitter is trying to ban
accounts and viral conversations around QAnon. It helps reduce the
spread. But the reason this is so dangerous is that the little rabbit
holes that take you deeper into QAnon are everywhere. A YouTube video
might lead you to a Wikipedia page that takes you to another video. Each
one is maybe harmless but the combined effect might draw you into the
world. And it's already so robust as a community that if people want to
talk about QAnon they'll find the ways.

\textbf{You write: ``QAnon fills the void of information that states
have created'' and that the choice to go on vigilante truth missions is
rooted in a decline in trust in institutions. Can we reverse that?}

This isn't new but if you look at the roots of why people are drawn to
conspiratorial thinking, it's because people have reason to believe
there is a conspiracy behind how the world works. They feel lost. That
lots of information is hidden from them or that important decisions have
been made in ways they don't understand. They'll prefer to believe
something from a forum that caters toward their biases and is easier to
read and consume than news coverage or from reading a dull 1,000 page
pdf from a government website explaining complex policy decisions.

\textbf{But lots of conspiracy ``researching'' draws upon complicated
primary sources --- it's just often a misguided or dangerous
interpretation.}

That's true. And that's where the collaborative, investigative ARG
element comes in.

\textbf{Specifically you mean that, because so many people have access
to unbelievable amounts of information online, there's an expectation
that all information ought to be discoverable, if you just search hard
enough?}

Exactly. A lot of it is about a lack of trust. But also a lack of
comfort with ambiguity. In reality, the answer to most hard questions
is, `It's complicated.' But people want definitive answers. Many of
these theories provide that feeling for people. When really everyone
needs to be a bit more comfortable with ambiguity.

Image

A QAnon banner hanging on the side of a building in rural Georgia in
June.Credit...Audra Melton for The New York Times

\textbf{Where do you think QAnon goes from here? Some have suggested it
feels more like a nascent religion than a conspiracy theory. Does that
sound right to you?}

I am not sure. I don't know if QAnon is quite comparable to a religion.
What concerns me is that right now it is quite easy for bad actors to
see what has worked with QAnon and try to copy it. There's a playbook
now. Any groups can hijack it. It wouldn't take a lot to try and
replicate it.

\textbf{But, despite all this, you still say you're hopeful for the
future? Why?}

The reason I'm optimistic is not that I think QAnon will disappear in a
year but that something like QAnon is proof that people care and people
like being involved in pursuit of truth. In QAnon that care and pursuit
are dangerously twisted. But it gives people who feel unwelcome in lots
of places a sense of purpose. You can make projects and build community
that harnesses that positively. The same way bad actors can look at
QAnon and find a playbook, so can good actors. We can find similar ways
to motivate alienated people in a more constructive way. At least I hope
so.

\emph{The Times is committed to publishing}
\href{https://www.nytimes3xbfgragh.onion/2019/01/31/opinion/letters/letters-to-editor-new-york-times-women.html}{\emph{a
diversity of letters}} \emph{to the editor. We'd like to hear what you
think about this or any of our articles. Here are some}
\href{https://help.nytimes3xbfgragh.onion/hc/en-us/articles/115014925288-How-to-submit-a-letter-to-the-editor}{\emph{tips}}\emph{.
And here's our
email:}\href{mailto:letters@NYTimes.com}{\emph{letters@NYTimes.com}}\emph{.}

\emph{Follow The New York Times Opinion section on}
\href{https://www.facebookcorewwwi.onion/nytopinion}{\emph{Facebook}}\emph{,}
\href{http://twitter.com/NYTOpinion}{\emph{Twitter (@NYTopinion)}}
\emph{and}
\href{https://www.instagram.com/nytopinion/}{\emph{Instagram}}\emph{.}

Advertisement

\protect\hyperlink{after-bottom}{Continue reading the main story}

\hypertarget{site-index}{%
\subsection{Site Index}\label{site-index}}

\hypertarget{site-information-navigation}{%
\subsection{Site Information
Navigation}\label{site-information-navigation}}

\begin{itemize}
\tightlist
\item
  \href{https://help.nytimes3xbfgragh.onion/hc/en-us/articles/115014792127-Copyright-notice}{©~2020~The
  New York Times Company}
\end{itemize}

\begin{itemize}
\tightlist
\item
  \href{https://www.nytco.com/}{NYTCo}
\item
  \href{https://help.nytimes3xbfgragh.onion/hc/en-us/articles/115015385887-Contact-Us}{Contact
  Us}
\item
  \href{https://www.nytco.com/careers/}{Work with us}
\item
  \href{https://nytmediakit.com/}{Advertise}
\item
  \href{http://www.tbrandstudio.com/}{T Brand Studio}
\item
  \href{https://www.nytimes3xbfgragh.onion/privacy/cookie-policy\#how-do-i-manage-trackers}{Your
  Ad Choices}
\item
  \href{https://www.nytimes3xbfgragh.onion/privacy}{Privacy}
\item
  \href{https://help.nytimes3xbfgragh.onion/hc/en-us/articles/115014893428-Terms-of-service}{Terms
  of Service}
\item
  \href{https://help.nytimes3xbfgragh.onion/hc/en-us/articles/115014893968-Terms-of-sale}{Terms
  of Sale}
\item
  \href{https://spiderbites.nytimes3xbfgragh.onion}{Site Map}
\item
  \href{https://help.nytimes3xbfgragh.onion/hc/en-us}{Help}
\item
  \href{https://www.nytimes3xbfgragh.onion/subscription?campaignId=37WXW}{Subscriptions}
\end{itemize}
