Sections

SEARCH

\protect\hyperlink{site-content}{Skip to
content}\protect\hyperlink{site-index}{Skip to site index}

\href{https://myaccount.nytimes3xbfgragh.onion/auth/login?response_type=cookie\&client_id=vi}{}

\href{https://www.nytimes3xbfgragh.onion/section/todayspaper}{Today's
Paper}

\href{/section/opinion}{Opinion}\textbar{}Trump Doesn't Need the Most
Votes. What if He Doesn't Even Want Them?

\href{https://nyti.ms/33oVDXK}{https://nyti.ms/33oVDXK}

\begin{itemize}
\item
\item
\item
\item
\item
\item
\end{itemize}

Advertisement

\protect\hyperlink{after-top}{Continue reading the main story}

\href{/section/opinion}{Opinion}

Supported by

\protect\hyperlink{after-sponsor}{Continue reading the main story}

\hypertarget{trump-doesnt-need-the-most-votes-what-if-he-doesnt-even-want-them}{%
\section{Trump Doesn't Need the Most Votes. What if He Doesn't Even Want
Them?}\label{trump-doesnt-need-the-most-votes-what-if-he-doesnt-even-want-them}}

Government of the minority, chosen by a minority, on behalf of a
minority, is not what Lincoln had in mind at Gettysburg.

\href{https://www.nytimes3xbfgragh.onion/column/jamelle-bouie}{\includegraphics{https://static01.graylady3jvrrxbe.onion/images/2019/01/24/opinion/jamelle-bouie/jamelle-bouie-thumbLarge-v3.png}}

By
\href{https://www.nytimes3xbfgragh.onion/column/jamelle-bouie}{Jamelle
Bouie}

Opinion Columnist

\begin{itemize}
\item
  Aug. 4, 2020
\item
  \begin{itemize}
  \item
  \item
  \item
  \item
  \item
  \item
  \end{itemize}
\end{itemize}

\includegraphics{https://static01.graylady3jvrrxbe.onion/images/2020/08/04/opinion/04bouie/merlin_114069913_00b3f249-1710-471a-89bb-9679af440615-articleLarge.jpg?quality=75\&auto=webp\&disable=upscale}

\hypertarget{listen-to-this-op-ed}{%
\subsubsection{Listen to This Op-Ed}\label{listen-to-this-op-ed}}

Audio Recording by Audm

\emph{To hear more audio stories from publishers like The New York
Times,
download}\href{https://www.audm.com/?utm_source=nytmag\&utm_medium=embed\&utm_campaign=left_behind_draper}{**}\href{https://www.audm.com/?utm_source=nytopinion\&utm_medium=embed\&utm_campaign=trump_votes_want}{\emph{Audm
for iPhone or Android}}\emph{.}

Nearly everyone involved in reporting on, analyzing or forecasting the
upcoming presidential election agrees that Donald Trump could win
another term in office. But no one save his most dedicated sycophants
thinks he could do so with a majority of the public on his side. We have
accepted, as a matter of course, that Trump could be constitutionally
re-elected through the Electoral College, but not democratically
selected by the voting public.

That's how he won in 2016, and the reason is straightforward. Enough of
the president's base is concentrated in swing states like Florida,
Michigan, Pennsylvania and Wisconsin. Because of that fact, he can lose
by
\href{https://www.nbcnews.com/politics/2020-election/how-trump-could-lose-5-million-votes-still-win-2020-n1031601}{as
many as five million votes} and still win an Electoral College majority.

As much as this contradicts our democratic expectations, you can imagine
a scenario where, aware of his minority position, Trump governed with an
eye toward consensus and popular legitimacy. The Electoral College
misfire would have been a problem, but not a dangerous one. Instead,
President Trump and his allies embraced this plainly anti-democratic
feature of our political system to liberate themselves from majoritarian
politics and coalition building. It's not just that they \emph{can} win
with a plurality, but that they intend to, with no interest in
persuading the majority of American voters and no concern for the
consequences of that choice.

It was clear from the start of his administration that Trump saw his
Electoral College advantage as license for an intentionally divisive
style of politics, stoking anger and racial prejudice whenever it seemed
politically advantageous. He bases key governing decisions on whether he
won a state or group of states in the previous election. If the United
States does not have a national strategy for the pandemic, it is at
least in part because --- as a report in Vanity Fair suggests --- the
administration originally believed the problem was
\href{https://www.vanityfair.com/news/2020/07/how-jared-kushners-secret-testing-plan-went-poof-into-thin-air}{restricted
to ``blue'' states.}

All of this has obviously carried over into the president's re-election
campaign. Trump has made no attempt to win a majority of voters, no
effort to bring a skeptical public to his side. Instead, he has directed
his energy toward suppressing opposition in hopes of winning by
technical knockout for a second time. His chief target right now is the
United States Postal Service, whose operation, it almost goes without
saying, is critical for the success of mail-in voting.

Because it lowers the barrier to participation and encourages modestly
higher turnout, Trump sees vote-by-mail as a threat. ``MAIL-IN VOTING
WILL LEAD TO MASSIVE FRAUD AND ABUSE,''
\href{https://twitter.com/realdonaldtrump/status/1266172570983940101?s=21}{he
tweeted} several months ago in a typical attack. ``IT WILL ALSO LEAD TO
THE END OF OUR GREAT REPUBLICAN PARTY. WE CAN NEVER LET THIS TRAGEDY
BEFALL OUR NATION.''

Similarly, after Nevada approved a plan on Monday to send
\href{https://www.cnn.com/2020/08/03/politics/nevada-mail-ballots-registered-voters/index.html}{mail-in
ballots to all active voters} in November,
\href{https://twitter.com/realdonaldtrump/status/1290250416278532096?s=21}{Trump
denounced} the plan as an ``illegal late night coup'' that will ``make
it impossible for Republicans to win the state.''

As if to make his attacks reality, Trump has taken steps to undercut the
Postal Service. His newly installed postmaster general --- Louis DeJoy,
\href{https://www.nytimes3xbfgragh.onion/2020/07/31/us/politics/trump-usps-mail-delays.html}{a
major campaign donor} --- has
\href{https://www.washingtonpost.com/business/2020/07/14/postal-service-trump-dejoy-delay-mail/}{imposed
new rules} that greatly reduce the flow of mail. In some parts of the
country, \href{https://t.co/MPqSHlXP1P}{like Philadelphia}, mail
collection and delivery has slowed down considerably. For a critical
city in a critical swing state, this is deeply concerning.

The Republican Party is all in on the effort to keep the anti-Trump
public from casting anti-Trump votes. The Republican National Committee
has
\href{https://www.nytimes3xbfgragh.onion/2020/05/18/us/Voting-republicans-trump.html}{established
a program} to ``protect the vote'' by monitoring polling places,
challenging voters deemed suspicious and blocking efforts to expand
vote-by-mail or relax voting restrictions. The Trump campaign, likewise,
is suing to shape mail-in voting in a way that might give the president
a strategic advantage. In Pennsylvania, for example, it wants
\href{https://www.inquirer.com/politics/election/trump-campaign-lawsuit-pennsylvania-mail-ballots-20200629.html}{to
keep voters} from using officially designated drop boxes for their
ballots, forcing them to go through the mail system.

There are still other ways in which Trump is trying to optimize for
minoritarian victories. On Monday, his Census Bureau announced it would
\href{https://www.npr.org/2020/08/03/898548910/census-cut-short-a-month-rushes-to-finish-all-counting-efforts-by-sept-30}{end
all counting efforts a month early}, in order to ``accelerate completion
of data collection and apportionment counts.'' It's a last-minute change
that threatens the accuracy of the census, and there's a strong chance
that any undercount will disadvantage Black and immigrant communities,
robbing them of resources and representation that will go, instead, to
whiter and more rural areas. This won't affect the upcoming election,
but it would shape American politics for the next decade in the
Republican Party's favor.

If all of this succeeds --- if he sabotages voting just enough to eke
out another Electoral College victory --- then Trump will be the first
president since the advent of a presidential ``popular vote'' to win two
terms without also winning the most votes. It would be the third such
misfire since 2000, another instance in which Democrats won the largest
share of voters without winning power.

Yes, everyone knows the rules of American presidential elections. But
those rules survived, in part, because this divergence was extremely
rare. Before the 2000 election, it had happened only three times: 1824,
1876 and 1888. The Electoral College may not have been the most modern
way to conduct a national election, but its outcomes did not
consistently violate our democratic intuitions, our collective
expectation that one person equals one vote.

For Trump to win, again, without winning the most votes would shatter
whatever remaining faith millions of Americans have in the political
system. Our simmering legitimacy crisis would almost certainly heat to a
boil. After such an outcome, how could you say this was a democracy? How
could you say, if you prefer the terminology of the 18th century
instead, that this was a republic?

It is true our system was meant to hedge against the ``tyranny of the
majority.'' But that's why it has multiple and overlapping spheres of
representation. The goal was balance, not a system where the arbitrary
distribution of voters could meet the abuse of power to produce an
almost permanent advantage for one side over the other. \emph{That} is
the tyranny of the minority, which is just another way of saying
tyranny.

\emph{The Times is committed to publishing}
\href{https://www.nytimes3xbfgragh.onion/2019/01/31/opinion/letters/letters-to-editor-new-york-times-women.html}{\emph{a
diversity of letters}} \emph{to the editor. We'd like to hear what you
think about this or any of our articles. Here are some}
\href{https://help.nytimes3xbfgragh.onion/hc/en-us/articles/115014925288-How-to-submit-a-letter-to-the-editor}{\emph{tips}}\emph{.
And here's our email:}
\href{mailto:letters@NYTimes.com}{\emph{letters@NYTimes.com}}\emph{.}

\emph{Follow The New York Times Opinion section on}
\href{https://www.facebookcorewwwi.onion/nytopinion}{\emph{Facebook}}\emph{,}
\href{http://twitter.com/NYTOpinion}{\emph{Twitter (@NYTopinion)}}
\emph{and}
\href{https://www.instagram.com/nytopinion/}{\emph{Instagram}}\emph{.}

Advertisement

\protect\hyperlink{after-bottom}{Continue reading the main story}

\hypertarget{site-index}{%
\subsection{Site Index}\label{site-index}}

\hypertarget{site-information-navigation}{%
\subsection{Site Information
Navigation}\label{site-information-navigation}}

\begin{itemize}
\tightlist
\item
  \href{https://help.nytimes3xbfgragh.onion/hc/en-us/articles/115014792127-Copyright-notice}{©~2020~The
  New York Times Company}
\end{itemize}

\begin{itemize}
\tightlist
\item
  \href{https://www.nytco.com/}{NYTCo}
\item
  \href{https://help.nytimes3xbfgragh.onion/hc/en-us/articles/115015385887-Contact-Us}{Contact
  Us}
\item
  \href{https://www.nytco.com/careers/}{Work with us}
\item
  \href{https://nytmediakit.com/}{Advertise}
\item
  \href{http://www.tbrandstudio.com/}{T Brand Studio}
\item
  \href{https://www.nytimes3xbfgragh.onion/privacy/cookie-policy\#how-do-i-manage-trackers}{Your
  Ad Choices}
\item
  \href{https://www.nytimes3xbfgragh.onion/privacy}{Privacy}
\item
  \href{https://help.nytimes3xbfgragh.onion/hc/en-us/articles/115014893428-Terms-of-service}{Terms
  of Service}
\item
  \href{https://help.nytimes3xbfgragh.onion/hc/en-us/articles/115014893968-Terms-of-sale}{Terms
  of Sale}
\item
  \href{https://spiderbites.nytimes3xbfgragh.onion}{Site Map}
\item
  \href{https://help.nytimes3xbfgragh.onion/hc/en-us}{Help}
\item
  \href{https://www.nytimes3xbfgragh.onion/subscription?campaignId=37WXW}{Subscriptions}
\end{itemize}
