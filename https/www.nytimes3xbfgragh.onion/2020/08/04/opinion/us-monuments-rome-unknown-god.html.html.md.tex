Sections

SEARCH

\protect\hyperlink{site-content}{Skip to
content}\protect\hyperlink{site-index}{Skip to site index}

\href{https://myaccount.nytimes3xbfgragh.onion/auth/login?response_type=cookie\&client_id=vi}{}

\href{https://www.nytimes3xbfgragh.onion/section/todayspaper}{Today's
Paper}

\href{/section/opinion}{Opinion}\textbar{}We Need a Monument to the
Unknown America

\url{https://nyti.ms/3i5c13x}

\begin{itemize}
\item
\item
\item
\item
\item
\end{itemize}

Advertisement

\protect\hyperlink{after-top}{Continue reading the main story}

\href{/section/opinion}{Opinion}

Supported by

\protect\hyperlink{after-sponsor}{Continue reading the main story}

THE STONE

\hypertarget{we-need-a-monument-to-the-unknown-america}{%
\section{We Need a Monument to the Unknown
America}\label{we-need-a-monument-to-the-unknown-america}}

A memorial makes a statement about who is worth preserving. Why not a
sacred and beautiful nothingness?

By Ed Simon

Mr. Simon is an editor and writer.

\begin{itemize}
\item
  Aug. 4, 2020
\item
  \begin{itemize}
  \item
  \item
  \item
  \item
  \item
  \end{itemize}
\end{itemize}

\includegraphics{https://static01.graylady3jvrrxbe.onion/images/2020/08/04/opinion/04stone/04stone-articleLarge.jpg?quality=75\&auto=webp\&disable=upscale}

In Rome's Palatine Museum, built atop the excavation site of Emperor
Domitian's palace and not far from imperial residences of Tiberius and
Augustus, there is a curious bit of masonry that was rediscovered in
1820. This artifact is exhibited alongside terra cotta heads of Jupiter
and marble sculptures of Apollo, all of the grand detritus of a lost
civilization housed in a former monastery. At the corner of one of the
museum's galleries sits this short altar chiseled from granite that's
slightly worn from the elements and framed with a modest curled design
where a statue might sit. No idol is framed by the base; no statue of
strong Jupiter or beautiful Apollo sits atop the altar. It is rather
simply dedicated to Agnostos Theos \emph{---} the ``Unknown God.'' Where
a deity would normally be present is rather a sacred and beautiful
nothingness.

These altars were not uncommon in the ancient Mediterranean. The most
famous reference to them is in a homily delivered by the Apostle Paul in
the Book of Acts, in which he claims that an altar to the Unknown God on
the Areopagus Hill in Athens was actually in honor of Christ (the
Athenians, ripe for conversion, Paul thought, had yet to fully
understand that). Arguments have been made that the altars honored the
undepictable God of the Jews, or that they're a bit of pragmatic
hedge-betting for any deities who should be worshiped but that have
escaped the attention of the otherwise pious. Regardless of the exact
intent of their makers, the altars for the Unknown God present an
arresting memorial --- the blessing of a lacunae, the sacralization of
an absence.

The altar to the Unknown God also provides an opportunity for us to
think about memorialization and memory. This has been an iconoclastic
summer, as there has been welcome progress in taking down statues
dedicated to Confederate generals, as well as the beginning of a more
nuanced discussion of figures like the founding fathers. Contrary to the
retrograde claim that statues are simply ``history,'' a memorial is
always an argument, and rarely an ambivalent one.

For example, the statue of Augustus of Primo Porta, housed in the
Vatican Museums not far from the Palatine Hill, makes a very specific
argument. Almost seven feet tall, the emperor is represented with an
arresting beauty, the breastplate depicting his diplomatic and military
victories that resulted in Pax Romana, ** and symbols associated with
Venus and Mars that conflate Augustus with divinity. Augustus of Primo
Porta's sculptors made a claim about their subject just as the designers
of the statues of Robert E. Lee, Stonewall Jackson and Jefferson Davis
that line Monument Avenue in Richmond, Va., made an argument about their
subject.

How we determine who is deserving of honor and think about
representation is often complicated by the altar. The Augustus of Primo
Porto's argument is about the glory of imperial Rome and the divinity of
the emperor; the Confederate memorials of Monument Avenue make an
erroneous claim about the righteousness of the ``Lost Cause.'' The altar
to the Unknown God's claim is different --- there's a humility in its
design, an ambiguity in its meaning and even an uncertainty in its
subject. The historian Alain Besançon explains in ``The Forbidden Image:
An Intellectual History of Iconoclasm'' that the tradition of avoiding
images of God are born from ``two incoercible facts about our nature:
first that we must look toward the divine \ldots{} and, second, that
representing it is futile, sacrilegious, inconceivable.'' The sculptors
of statues and memorials promise us gods and heroes, but those who
created the altar at the Palatine Museum achieved a far greater
verisimilitude regarding those subjects, and they did it by depicting
nothing.

Any statue --- be it of Christopher Columbus or Abraham Lincoln, Thomas
Jefferson or Frederick Douglass --- makes a statement about who is worth
preserving in bronze. To make a statue of a recognizable subject is to
venture something about that subject, but an altar for nothing argues
that its subject is so all-encompassing, so universal, so grand that it
can't be circumscribed in mere stone. Christian theologians speak of a
methodology called ``apophasis,'' whereby the divine is discussed in
terms of what can't be known about it, where definition is always
deferred and thought can't be put into mere words. In some ways the
altar to the Unknown God is a form of apophatic sculpture, whereby
nonexistence is a heftier material than steel or bronze.

I'd also suggest that the apophatic is precisely the mode in which we
should think of our national future, for if God is a lofty subject, then
so are ``freedom,'' ``democracy'' and ``America.'' We are better served
in memorializing the contradictory, conflicted and hypocritical history
of this nation not in monumental equestrian statues punctuating traffic
roundabouts but in a humbler idiom that paradoxically expresses an
inherent greatness in all the more remarkable of a way.

This would be nothing new for the United States as a nation. One of the
most popular memorials on the National Mall already considers its
subject in an apophatic way. When Maya Lin's Vietnam Veterans Memorial
was completed in 1982, it was met with great controversy. It is an
ascending V-shaped wall of polished black granite inscribed with the
names of the more than 58,000 Americans who died in the Vietnam War. Lin
intended there to be no statues of men atop tanks, no sculptures of
brave soldiers traipsing through the jungle (the inclusion, against her
wishes, of just such a depiction at the edge of the memorial speaks to
the controversy, despite its popularity).

Unlike Lincoln's Parthenon or Washington's obelisk, Lin's memorial
ventured no argument other than the sheer unimaginable depth of the
war's toll. Abstraction has been used to great effect in other modernist
memorials, such as the Oklahoma City National Memorial, the Flight 93
National Memorial in Stoystown, Pa., and especially the National
Memorial for Peace and Justice in Montgomery, Ala., where each of 805
hanging steel rectangles represents an American county where a lynching
is known to have taken place.

Like the altars to the Unknown God, these memorials make their arguments
in subtler and ultimately more powerful ways than does a statue of a man
astride a horse. They tell us something about language and image and the
ways in which some things are so huge that they defy traditional
description. Imagine an altar to the Unknown America doing something
similar, where a potential vision of a country that has yet to exist
could be gestured toward in sacred silence; where an expansive
understanding of America that mourns its past without mistaking villains
for heroes allows for the possibility of redemption. Not in a mythic
great past but in a utopian future that has yet to be lived, but that
can be imagined in every individual soul.

Ed Simon is a staff writer at The Millions. His latest book is
``\href{https://www.johnhuntpublishing.com/zer0-books/our-books/printed-in-utopia}{Printed
in Utopia: The Renaissance's Radicalism}.''

\emph{\textbf{Now in print:}}
\emph{``}\href{http://bitly.com/1MW2kN3}{\emph{Modern Ethics in 77
Arguments}}\emph{'' and ``}\href{http://bitly.com/1MW2kN3}{\emph{The
Stone Reader: Modern Philosophy in 133 Arguments}}\emph{,'' with essays
from the series, edited by Peter Catapano and Simon Critchley, published
by Liveright Books.}

\emph{The Times is committed to publishing}
\href{https://www.nytimes3xbfgragh.onion/2019/01/31/opinion/letters/letters-to-editor-new-york-times-women.html}{\emph{a
diversity of letters}} \emph{to the editor. We'd like to hear what you
think about this or any of our articles. Here are some}
\href{https://help.nytimes3xbfgragh.onion/hc/en-us/articles/115014925288-How-to-submit-a-letter-to-the-editor}{\emph{tips}}\emph{.
And here's our email:}
\href{mailto:letters@NYTimes.com}{\emph{letters@NYTimes.com}}\emph{.}

\emph{Follow The New York Times Opinion section on}
\href{https://www.facebookcorewwwi.onion/nytopinion}{\emph{Facebook}}\emph{,}
\href{http://twitter.com/NYTOpinion}{\emph{Twitter (@NYTopinion)}}
\emph{and}
\href{https://www.instagram.com/nytopinion/}{\emph{Instagram}}\emph{.}

Advertisement

\protect\hyperlink{after-bottom}{Continue reading the main story}

\hypertarget{site-index}{%
\subsection{Site Index}\label{site-index}}

\hypertarget{site-information-navigation}{%
\subsection{Site Information
Navigation}\label{site-information-navigation}}

\begin{itemize}
\tightlist
\item
  \href{https://help.nytimes3xbfgragh.onion/hc/en-us/articles/115014792127-Copyright-notice}{©~2020~The
  New York Times Company}
\end{itemize}

\begin{itemize}
\tightlist
\item
  \href{https://www.nytco.com/}{NYTCo}
\item
  \href{https://help.nytimes3xbfgragh.onion/hc/en-us/articles/115015385887-Contact-Us}{Contact
  Us}
\item
  \href{https://www.nytco.com/careers/}{Work with us}
\item
  \href{https://nytmediakit.com/}{Advertise}
\item
  \href{http://www.tbrandstudio.com/}{T Brand Studio}
\item
  \href{https://www.nytimes3xbfgragh.onion/privacy/cookie-policy\#how-do-i-manage-trackers}{Your
  Ad Choices}
\item
  \href{https://www.nytimes3xbfgragh.onion/privacy}{Privacy}
\item
  \href{https://help.nytimes3xbfgragh.onion/hc/en-us/articles/115014893428-Terms-of-service}{Terms
  of Service}
\item
  \href{https://help.nytimes3xbfgragh.onion/hc/en-us/articles/115014893968-Terms-of-sale}{Terms
  of Sale}
\item
  \href{https://spiderbites.nytimes3xbfgragh.onion}{Site Map}
\item
  \href{https://help.nytimes3xbfgragh.onion/hc/en-us}{Help}
\item
  \href{https://www.nytimes3xbfgragh.onion/subscription?campaignId=37WXW}{Subscriptions}
\end{itemize}
