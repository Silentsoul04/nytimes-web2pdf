Sections

SEARCH

\protect\hyperlink{site-content}{Skip to
content}\protect\hyperlink{site-index}{Skip to site index}

\href{https://www.nytimes3xbfgragh.onion/section/style}{Style}

\href{https://myaccount.nytimes3xbfgragh.onion/auth/login?response_type=cookie\&client_id=vi}{}

\href{https://www.nytimes3xbfgragh.onion/section/todayspaper}{Today's
Paper}

\href{/section/style}{Style}\textbar{}The Anonymous Professor Who Wasn't

\url{https://nyti.ms/2XrHlBA}

\begin{itemize}
\item
\item
\item
\item
\item
\end{itemize}

\href{https://www.nytimes3xbfgragh.onion/news-event/coronavirus?action=click\&pgtype=Article\&state=default\&region=TOP_BANNER\&context=storylines_menu}{The
Coronavirus Outbreak}

\begin{itemize}
\tightlist
\item
  live\href{https://www.nytimes3xbfgragh.onion/2020/08/04/world/coronavirus-cases.html?action=click\&pgtype=Article\&state=default\&region=TOP_BANNER\&context=storylines_menu}{Latest
  Updates}
\item
  \href{https://www.nytimes3xbfgragh.onion/interactive/2020/us/coronavirus-us-cases.html?action=click\&pgtype=Article\&state=default\&region=TOP_BANNER\&context=storylines_menu}{Maps
  and Cases}
\item
  \href{https://www.nytimes3xbfgragh.onion/interactive/2020/science/coronavirus-vaccine-tracker.html?action=click\&pgtype=Article\&state=default\&region=TOP_BANNER\&context=storylines_menu}{Vaccine
  Tracker}
\item
  \href{https://www.nytimes3xbfgragh.onion/2020/08/02/us/covid-college-reopening.html?action=click\&pgtype=Article\&state=default\&region=TOP_BANNER\&context=storylines_menu}{College
  Reopening}
\item
  \href{https://www.nytimes3xbfgragh.onion/live/2020/08/04/business/stock-market-today-coronavirus?action=click\&pgtype=Article\&state=default\&region=TOP_BANNER\&context=storylines_menu}{Economy}
\end{itemize}

Advertisement

\protect\hyperlink{after-top}{Continue reading the main story}

Supported by

\protect\hyperlink{after-sponsor}{Continue reading the main story}

\hypertarget{the-anonymous-professor-who-wasnt}{%
\section{The Anonymous Professor Who
Wasn't}\label{the-anonymous-professor-who-wasnt}}

A professor at Arizona State University does not exist.

\includegraphics{https://static01.graylady3jvrrxbe.onion/images/2020/08/04/fashion/04McLaughlin/04McLaughlin-articleLarge.jpg?quality=75\&auto=webp\&disable=upscale}

By
\href{https://www.nytimes3xbfgragh.onion/by/jonah-engel-bromwich}{Jonah
Engel Bromwich} and
\href{https://www.nytimes3xbfgragh.onion/by/ezra-marcus}{Ezra Marcus}

\begin{itemize}
\item
  Aug. 4, 2020
\item
  \begin{itemize}
  \item
  \item
  \item
  \item
  \item
  \end{itemize}
\end{itemize}

An anonymous anthropology professor remained outspoken about fairness in
academia even as she suffered for months with coronavirus.

``This person was a scientist who got Covid because they'd been forced
to teach,'' said Michael Eisen, a fly geneticist at the University of
California, Berkeley, who had interacted on Twitter with the professor
for years. ``It wasn't the first person I knew who got Covid --- but for
a lot of people it was one of the first people they knew who got it.''

He said that he had continued to exchange messages with the person
running the account through June and that this person frequently
discussed a difficult recovery.

Then BethAnn McLaughlin, another Twitter connection, announced on July
31 that the anonymous professor had died from complications of the
virus.

Just a few days later, both the account of the anonymous professor and
of Ms. McLaughlin
\href{https://www.buzzfeednews.com/article/peteraldhous/bethann-mclaughlin-twitter-suspension-fake-covid-death}{were
suspended for Twitter policies} that, among other things, bar the
coordination of fake accounts.

The same day, Gerardo Gonzalez, a spokesman for Arizona State
University, where the anonymous Twitter user was supposedly a professor,
described the anonymous account as a ``hoax.''

The account had posted inaccurate information about the school, he said.
``We also have had no one, such as a family member or friend, report a
death to anyone at the university,'' he added.

Among scientists and academics, the shock of mourning was already laced
with suspicion. Enough of them had unpleasant interactions with the
combative account and were troubled by its inconsistencies and seeming
about-turns.

``You have these internal alarms that are like, `Oh, I don't trust
you,''' said Julie Libarkin, the head of the Geocognition Research
Laboratory at Michigan State University. ``Kind of the same as when I
worked with BethAnn.''

\hypertarget{my-actions-are-inexcusable}{%
\subsection{`My Actions Are
Inexcusable'}\label{my-actions-are-inexcusable}}

On Tuesday, Ms. McLaughlin gave a statement to The New York Times
through her lawyer.

``I take full responsibility for my involvement in creating the
@sciencing\_bi Twitter account,'' it said. ``My actions are inexcusable.
I apologize without reservation to all the people I hurt.''

The anonymous account, @Sciencing\_Bi, was an active participant in the
corner of Science Twitter that frequently discusses issues of sexual
misconduct in the sciences. It claimed on at least
\href{https://twitter.com/IsabelOtt/status/1290073135396528129/photo/1}{one
occasion} to have grown up in Alabama, to have ``fled the south because
of their oppression of queer folk,'' and to have attended Catholic
school. The account began to pointedly make reference to being Native
American and, earlier this year, began to identify as Hopi.

\hypertarget{latest-updates-global-coronavirus-outbreak}{%
\section{\texorpdfstring{\href{https://www.nytimes3xbfgragh.onion/2020/08/04/world/coronavirus-cases.html?action=click\&pgtype=Article\&state=default\&region=MAIN_CONTENT_1\&context=storylines_live_updates}{Latest
Updates: Global Coronavirus
Outbreak}}{Latest Updates: Global Coronavirus Outbreak}}\label{latest-updates-global-coronavirus-outbreak}}

Updated 2020-08-05T07:44:16.891Z

\begin{itemize}
\tightlist
\item
  \href{https://www.nytimes3xbfgragh.onion/2020/08/04/world/coronavirus-cases.html?action=click\&pgtype=Article\&state=default\&region=MAIN_CONTENT_1\&context=storylines_live_updates\#link-762df92}{As
  talks drag on, McConnell signals openness to jobless aid extension,
  and negotiators agree on a deadline.}
\item
  \href{https://www.nytimes3xbfgragh.onion/2020/08/04/world/coronavirus-cases.html?action=click\&pgtype=Article\&state=default\&region=MAIN_CONTENT_1\&context=storylines_live_updates\#link-1228a480}{Novavax
  sees encouraging results from two studies of its experimental
  vaccine.}
\item
  \href{https://www.nytimes3xbfgragh.onion/2020/08/04/world/coronavirus-cases.html?action=click\&pgtype=Article\&state=default\&region=MAIN_CONTENT_1\&context=storylines_live_updates\#link-794484ed}{Mississippians
  must now wear masks in public, governor says.}
\end{itemize}

\href{https://www.nytimes3xbfgragh.onion/2020/08/04/world/coronavirus-cases.html?action=click\&pgtype=Article\&state=default\&region=MAIN_CONTENT_1\&context=storylines_live_updates}{See
more updates}

More live coverage:
\href{https://www.nytimes3xbfgragh.onion/live/2020/08/04/business/stock-market-today-coronavirus?action=click\&pgtype=Article\&state=default\&region=MAIN_CONTENT_1\&context=storylines_live_updates}{Markets}

Since 2016, it has posted often about issues around social justice in
the sciences, with a focus on activism and research about sexual
harassment.

It was also active in the career of Ms. McLaughlin, a neuroscientist.
(News of the relationship between the Twitter accounts
\href{https://heavy.com/news/2020/08/sciencing_bi-bethann-mclaughlin-asu/}{was
reported by Heavy.com},
\href{https://www.sciencemag.org/news/2020/08/twitter-account-embattled-metoostem-founder-suspended}{Science}
and the
\href{https://www.chronicle.com/article/did-the-founder-of-metoostem-create-a-fake-online-persona-and-then-kill-it-off}{Chronicle
of Higher Education}.) It was key in promoting a
\href{https://www.change.org/p/vanderbilt-don-t-fire-prof-bethann-mclaughlin-for-standing-against-sexual-harassment}{petition}
that called for Ms. McLaughlin to be given tenure at Vanderbilt
University. She was not given tenure in 2017, a decision she said was
influenced by her having testified against a former Vanderbilt professor
accused of sexual harassment. Her effort to reverse that decision was
unsuccessful in 2019, and
\href{https://www.sciencemag.org/news/2019/07/metoostem-founder-out-vanderbilt}{she
left the university that summer.}

On one occasion, the account responded to someone asking Ms. McLaughlin
for information about Vanderbilt with
\href{https://twitter.com/IsabelOtt/status/1290101201397329921/photo/1}{extensive
details} about the university's salary structure.

In April, @Sciencing\_Bi began to undergo a drama that belonged solely
to her, announcing the coronavirus diagnosis in a tweet. It was Ms.
McLaughlin who announced that the anonymous professor had died.

``I was pretty shocked,'' said Erica Smith, a postdoctoral researcher at
Indiana University. ``I had never had particularly great experiences
with @Sciencing\_Bi, but I thought that she was a whole real human who
had just died. I was surprised by how hard it hit me. I ate a pint of
ice cream about it.''

Ms. McLaughlin came across as particularly distraught. She mourned
@Sciencing\_Bi in a long thread, paying testament to her humanity and
toughness.

``She was supposed to get Hopi talisman for health as gifts for us but
she ran out,'' Ms. McLaughlin tweeted. ``God. The irony of running out
of health talisman.'' She also said that she and the person behind the
account had been planning on getting matching tattoos in the Hopi
language.

Ms. McLaughlin has prompted particular frustration and disgust by posing
as a Hopi woman, right as the coronavirus has
\href{https://www.nytimes3xbfgragh.onion/2020/04/09/us/coronavirus-navajo-nation.html}{caused
disproportionate harm} to
\href{https://www.nytimes3xbfgragh.onion/2020/07/30/us/native-americans-coronavirus-data.html}{Indigenous
communities in the United States}.

``There are millions who want to be us,'' said Jacqueline Keeler, a
writer and the editor of
\href{https://www.pollennationmagazine.com/}{Pollen Nation}, a
Native-led magazine. ``These people are centering themselves in our
issues, they are heading Native American departments, they are telling
Native students what they can and can't study --- it's to protect their
own position. And so it does change our ability to advocate for
ourselves when we are constantly being replaced by frauds, white people
or other people of different backgrounds pretending to be us.''

\hypertarget{it-just-became-obvious-to-me}{%
\subsection{`It Just Became Obvious to
Me'}\label{it-just-became-obvious-to-me}}

The first time Mr. Eisen heard from the account was in defense of Ms.
McLaughlin. ``The fact that @Sci-Bi was saying all these things about
BethAnn, saying that BethAnn had helped her, it didn't make me trust
BethAnn --- but it made me less willing to publicly criticize her
because I thought that public criticism would be felt by the people she
was helping,'' he said. ``Who turned out to be fake.''

Melissa Bates, a professor at the University of Iowa, was among those
invited to a Zoom memorial via a Twitter thread. Ms. Bates said on
Twitter that when she showed up on the Zoom, it was just her, Ms.
McLaughlin, Mr. Eisen and another man.

Afterward, Mr. Eisen began to search for any evidence that
@Sciencing\_Bi had been a real person. He could not find any.

``The combination of the weird things that were happening on the call
and looking at the tweets and seeing how much they circled BethAnn, it
just became obvious to me,'' he said. ```Oh, this is BethAnn.'''

Ms. McLaughlin first began to make waves among those concerned about
sexual harassment in the sciences in May 2018. She
\href{https://www.change.org/p/national-academy-of-sciences-remove-sexual-harassers-from-national-academy-of-sciences}{wrote
and circulated a petition that month} calling for the National Academy
of Sciences to revoke the membership of those who had been punished for
sexual harassment, retaliation and assault.

In June 2018, she and Ms. Libarkin started a website, MeTooSTEM, which
quickly garnered attention, as women posted a series of largely
anonymous stories there about being harassed while working in science,
technology, engineering and math (STEM).

In the same month, Ms. McLaughlin further raised her profile when she
\href{https://www.buzzfeednews.com/article/juliareinstein/rate-my-professors-hotness-chili-pepper-sexist?bfsource=relatedmanual}{used
Twitter} to successfully pressure the website ratemyprofessor.com to
drop its system of chili peppers used to rank the ``hotness'' of
academics.

In October 2018, Ms. McLaughlin, who had begun to make decisions for the
organization without informing her colleagues, created a fund-raiser for
MeTooSTEM on GoFundMe. It eventually raised more than \$79,000.

Ms. McLaughlin's colleagues at MeTooSTEM were already feeling
uncomfortable with her leadership at that point, and were made
particularly uneasy by the GoFundMe.

\href{https://www.nytimes3xbfgragh.onion/news-event/coronavirus?action=click\&pgtype=Article\&state=default\&region=MAIN_CONTENT_3\&context=storylines_faq}{}

\hypertarget{the-coronavirus-outbreak-}{%
\subsubsection{The Coronavirus Outbreak
›}\label{the-coronavirus-outbreak-}}

\hypertarget{frequently-asked-questions}{%
\paragraph{Frequently Asked
Questions}\label{frequently-asked-questions}}

Updated August 4, 2020

\begin{itemize}
\item ~
  \hypertarget{i-have-antibodies-am-i-now-immune}{%
  \paragraph{I have antibodies. Am I now
  immune?}\label{i-have-antibodies-am-i-now-immune}}

  \begin{itemize}
  \tightlist
  \item
    As of right
    now,\href{https://www.nytimes3xbfgragh.onion/2020/07/22/health/covid-antibodies-herd-immunity.html?action=click\&pgtype=Article\&state=default\&region=MAIN_CONTENT_3\&context=storylines_faq}{that
    seems likely, for at least several months.} There have been
    frightening accounts of people suffering what seems to be a second
    bout of Covid-19. But experts say these patients may have a
    drawn-out course of infection, with the virus taking a slow toll
    weeks to months after initial exposure. People infected with the
    coronavirus typically
    \href{https://www.nature.com/articles/s41586-020-2456-9}{produce}
    immune molecules called antibodies, which are
    \href{https://www.nytimes3xbfgragh.onion/2020/05/07/health/coronavirus-antibody-prevalence.html?action=click\&pgtype=Article\&state=default\&region=MAIN_CONTENT_3\&context=storylines_faq}{protective
    proteins made in response to an
    infection}\href{https://www.nytimes3xbfgragh.onion/2020/05/07/health/coronavirus-antibody-prevalence.html?action=click\&pgtype=Article\&state=default\&region=MAIN_CONTENT_3\&context=storylines_faq}{.
    These antibodies may} last in the body
    \href{https://www.nature.com/articles/s41591-020-0965-6}{only two to
    three months}, which may seem worrisome, but that's perfectly normal
    after an acute infection subsides, said Dr. Michael Mina, an
    immunologist at Harvard University. It may be possible to get the
    coronavirus again, but it's highly unlikely that it would be
    possible in a short window of time from initial infection or make
    people sicker the second time.
  \end{itemize}
\item ~
  \hypertarget{im-a-small-business-owner-can-i-get-relief}{%
  \paragraph{I'm a small-business owner. Can I get
  relief?}\label{im-a-small-business-owner-can-i-get-relief}}

  \begin{itemize}
  \tightlist
  \item
    The
    \href{https://www.nytimes3xbfgragh.onion/article/small-business-loans-stimulus-grants-freelancers-coronavirus.html?action=click\&pgtype=Article\&state=default\&region=MAIN_CONTENT_3\&context=storylines_faq}{stimulus
    bills enacted in March} offer help for the millions of American
    small businesses. Those eligible for aid are businesses and
    nonprofit organizations with fewer than 500 workers, including sole
    proprietorships, independent contractors and freelancers. Some
    larger companies in some industries are also eligible. The help
    being offered, which is being managed by the Small Business
    Administration, includes the Paycheck Protection Program and the
    Economic Injury Disaster Loan program. But lots of folks have
    \href{https://www.nytimes3xbfgragh.onion/interactive/2020/05/07/business/small-business-loans-coronavirus.html?action=click\&pgtype=Article\&state=default\&region=MAIN_CONTENT_3\&context=storylines_faq}{not
    yet seen payouts.} Even those who have received help are confused:
    The rules are draconian, and some are stuck sitting on
    \href{https://www.nytimes3xbfgragh.onion/2020/05/02/business/economy/loans-coronavirus-small-business.html?action=click\&pgtype=Article\&state=default\&region=MAIN_CONTENT_3\&context=storylines_faq}{money
    they don't know how to use.} Many small-business owners are getting
    less than they expected or
    \href{https://www.nytimes3xbfgragh.onion/2020/06/10/business/Small-business-loans-ppp.html?action=click\&pgtype=Article\&state=default\&region=MAIN_CONTENT_3\&context=storylines_faq}{not
    hearing anything at all.}
  \end{itemize}
\item ~
  \hypertarget{what-are-my-rights-if-i-am-worried-about-going-back-to-work}{%
  \paragraph{What are my rights if I am worried about going back to
  work?}\label{what-are-my-rights-if-i-am-worried-about-going-back-to-work}}

  \begin{itemize}
  \tightlist
  \item
    Employers have to provide
    \href{https://www.osha.gov/SLTC/covid-19/standards.html}{a safe
    workplace} with policies that protect everyone equally.
    \href{https://www.nytimes3xbfgragh.onion/article/coronavirus-money-unemployment.html?action=click\&pgtype=Article\&state=default\&region=MAIN_CONTENT_3\&context=storylines_faq}{And
    if one of your co-workers tests positive for the coronavirus, the
    C.D.C.} has said that
    \href{https://www.cdc.gov/coronavirus/2019-ncov/community/guidance-business-response.html}{employers
    should tell their employees} -\/- without giving you the sick
    employee's name -\/- that they may have been exposed to the virus.
  \end{itemize}
\item ~
  \hypertarget{should-i-refinance-my-mortgage}{%
  \paragraph{Should I refinance my
  mortgage?}\label{should-i-refinance-my-mortgage}}

  \begin{itemize}
  \tightlist
  \item
    \href{https://www.nytimes3xbfgragh.onion/article/coronavirus-money-unemployment.html?action=click\&pgtype=Article\&state=default\&region=MAIN_CONTENT_3\&context=storylines_faq}{It
    could be a good idea,} because mortgage rates have
    \href{https://www.nytimes3xbfgragh.onion/2020/07/16/business/mortgage-rates-below-3-percent.html?action=click\&pgtype=Article\&state=default\&region=MAIN_CONTENT_3\&context=storylines_faq}{never
    been lower.} Refinancing requests have pushed mortgage applications
    to some of the highest levels since 2008, so be prepared to get in
    line. But defaults are also up, so if you're thinking about buying a
    home, be aware that some lenders have tightened their standards.
  \end{itemize}
\item ~
  \hypertarget{what-is-school-going-to-look-like-in-september}{%
  \paragraph{What is school going to look like in
  September?}\label{what-is-school-going-to-look-like-in-september}}

  \begin{itemize}
  \tightlist
  \item
    It is unlikely that many schools will return to a normal schedule
    this fall, requiring the grind of
    \href{https://www.nytimes3xbfgragh.onion/2020/06/05/us/coronavirus-education-lost-learning.html?action=click\&pgtype=Article\&state=default\&region=MAIN_CONTENT_3\&context=storylines_faq}{online
    learning},
    \href{https://www.nytimes3xbfgragh.onion/2020/05/29/us/coronavirus-child-care-centers.html?action=click\&pgtype=Article\&state=default\&region=MAIN_CONTENT_3\&context=storylines_faq}{makeshift
    child care} and
    \href{https://www.nytimes3xbfgragh.onion/2020/06/03/business/economy/coronavirus-working-women.html?action=click\&pgtype=Article\&state=default\&region=MAIN_CONTENT_3\&context=storylines_faq}{stunted
    workdays} to continue. California's two largest public school
    districts --- Los Angeles and San Diego --- said on July 13, that
    \href{https://www.nytimes3xbfgragh.onion/2020/07/13/us/lausd-san-diego-school-reopening.html?action=click\&pgtype=Article\&state=default\&region=MAIN_CONTENT_3\&context=storylines_faq}{instruction
    will be remote-only in the fall}, citing concerns that surging
    coronavirus infections in their areas pose too dire a risk for
    students and teachers. Together, the two districts enroll some
    825,000 students. They are the largest in the country so far to
    abandon plans for even a partial physical return to classrooms when
    they reopen in August. For other districts, the solution won't be an
    all-or-nothing approach.
    \href{https://bioethics.jhu.edu/research-and-outreach/projects/eschool-initiative/school-policy-tracker/}{Many
    systems}, including the nation's largest, New York City, are
    devising
    \href{https://www.nytimes3xbfgragh.onion/2020/06/26/us/coronavirus-schools-reopen-fall.html?action=click\&pgtype=Article\&state=default\&region=MAIN_CONTENT_3\&context=storylines_faq}{hybrid
    plans} that involve spending some days in classrooms and other days
    online. There's no national policy on this yet, so check with your
    municipal school system regularly to see what is happening in your
    community.
  \end{itemize}
\end{itemize}

``We were about to get in front of a crowd of people and say: `Give us
your money,''' said Ms. Smith. She didn't know how the money would be
used, but did remember thinking: ``We're too broke to participate in
white-collar crime.''

Former colleagues of Ms. McLaughlin at MeTooSTEM said they did not know
where that money ultimately went. A 2019 report from MeTooSTEM said that
the money, along with other donations, had ``provided free services for
18 months to over 500 clients.''

In November 2018, the Massachusetts Institute of Technology's Media Lab
awarded its
\href{http://news.mit.edu/2019/call-for-nominations-media-lab-disobedience-award-0529}{Disobedience
Award} to Ms. McLaughlin; Tarana Burke, a founder of the \#MeToo
movement; and Sherry Marts, who left academia after being harassed by a
colleague in her graduate lab. The award recognizes ``ethical,
nonviolent acts of disobedience'' and comes with \$250,000, which that
year was split among the three recipients.

As Ms. McLaughlin received more public recognition, her colleagues at
MeTooSTEM began to leave the organization, accusing her of frequent
verbal abuse and citing the dysfunction plaguing the organization. By
May 2019, seven members had resigned, according to a
\href{https://www.buzzfeednews.com/article/peteraldhous/metoostem-sexual-harassment-science?bfsource=relatedmanual}{report
in BuzzFeed} at the time.

Deanna Arsala recently received a doctorate from the University of
Illinois at Chicago and was one of the few women of color at MeTooSTEM.
She said that @Sciencing\_Bi had claimed to know details about
leadership meetings that included only BethAnn and a few other people.
``I remember thinking, `Is this BethAnn?''' Ms. Arsala said.

Ms. Arsala left the organization in part because she and another
colleague, also a woman of color, felt that white leadership did not
prioritize what they had to say.

Kathryn Clancy, an anthropology professor at the University of Illinois
who was also involved in the MeTooSTEM movement and who is white, said
that even as Ms. McLaughlin's leadership issues brought bad publicity,
her issues with race had gone largely overlooked, even as women of color
inside and outside the organization had tried to get others to see them.

In February of 2020,
\href{https://www.buzzfeednews.com/article/peteraldhous/bethann-mclaughlin-metoostem-resignation-harassment}{another
BuzzFeed report}detailed further strife and resignations at MeTooSTEM.
Activists including Ms. Marts
\href{https://twitter.com/sherrymarts/status/1231059406328074240?ref_src=twsrc\%5Etfw\%7Ctwcamp\%5Etweetembed\%7Ctwterm\%5E1231059406328074240\%7Ctwgr\%5E\&ref_url=https\%3A\%2F\%2Fwww.buzzfeednews.com\%2Farticle\%2Fpeteraldhous\%2Fbethann-mclaughlin-metoostem-resignation-harassment}{publicly
distanced themselves} from the embattled MeTooSTEM leader; they even
officially resigned from the hashtag.

\hypertarget{its-so-easy-to-mislead-people}{%
\subsection{`It's So Easy to Mislead
People'}\label{its-so-easy-to-mislead-people}}

In retrospect, the symbiosis of the two accounts makes sense to those
who were aware of both for years. As Ms. McLaughlin lost some
credibility, the account gained it.

And Ms. McLaughlin's invention of the character behind @Sciencing\_Bi
was not as unusual as it may seem --- nor would it have been unusual for
her to have killed off the account.

Dr. Marc Feldman is a psychiatrist
\href{https://www.nytimes3xbfgragh.onion/1999/07/20/science/a-great-pretender-now-faces-the-truth-of-illness.html}{who
studies factitious disorders} in which a person acts as if they, or a
loved one, have a disease. He specializes in what he calls Munchausen by
internet, in which such deceptions take place online, and said he hears
about a new case every couple of weeks.

``I think it happens online more than offline these days because it's so
easy to mislead people via social media,'' Dr. Feldman said. He added
that Covid-19 had been a boon for those with such disorders. ``Nobody
wants to be near a Covid-19 sufferer so they say, `We can't meet,''' he
said. ``There's no way to arrange a face-to-face meeting.''

In her statement, Ms. McLaughlin said: ``As I've reflected on my actions
the last few days, it's become clear to me that I need mental health
treatment, which I'm pursuing now. My failures are mine alone, so I'm
stepping away from all activities with MeTooSTEM to ensure that it isn't
unfairly criticized for my actions.''

``These Pretendians are better at it than we are, because they don't
carry our specific trauma," Ms. Keeler said. And: ``They are ethnic
opportunists and that is a colonial endeavor that has been going on for
centuries in this hemisphere.''

Even the lack of tenure of the pretend professor served a purpose,
giving the account a perfectly good reason for remaining anonymous.

``This is a good thing about Science Twitter, that it gives people who
are marginalized a voice,'' Mr. Eisen said. ``It sucks that BethAnn took
advantage of that.''

Advertisement

\protect\hyperlink{after-bottom}{Continue reading the main story}

\hypertarget{site-index}{%
\subsection{Site Index}\label{site-index}}

\hypertarget{site-information-navigation}{%
\subsection{Site Information
Navigation}\label{site-information-navigation}}

\begin{itemize}
\tightlist
\item
  \href{https://help.nytimes3xbfgragh.onion/hc/en-us/articles/115014792127-Copyright-notice}{©~2020~The
  New York Times Company}
\end{itemize}

\begin{itemize}
\tightlist
\item
  \href{https://www.nytco.com/}{NYTCo}
\item
  \href{https://help.nytimes3xbfgragh.onion/hc/en-us/articles/115015385887-Contact-Us}{Contact
  Us}
\item
  \href{https://www.nytco.com/careers/}{Work with us}
\item
  \href{https://nytmediakit.com/}{Advertise}
\item
  \href{http://www.tbrandstudio.com/}{T Brand Studio}
\item
  \href{https://www.nytimes3xbfgragh.onion/privacy/cookie-policy\#how-do-i-manage-trackers}{Your
  Ad Choices}
\item
  \href{https://www.nytimes3xbfgragh.onion/privacy}{Privacy}
\item
  \href{https://help.nytimes3xbfgragh.onion/hc/en-us/articles/115014893428-Terms-of-service}{Terms
  of Service}
\item
  \href{https://help.nytimes3xbfgragh.onion/hc/en-us/articles/115014893968-Terms-of-sale}{Terms
  of Sale}
\item
  \href{https://spiderbites.nytimes3xbfgragh.onion}{Site Map}
\item
  \href{https://help.nytimes3xbfgragh.onion/hc/en-us}{Help}
\item
  \href{https://www.nytimes3xbfgragh.onion/subscription?campaignId=37WXW}{Subscriptions}
\end{itemize}
