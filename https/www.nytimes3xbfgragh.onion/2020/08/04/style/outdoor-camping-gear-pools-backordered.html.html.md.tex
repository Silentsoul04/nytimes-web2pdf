Sections

SEARCH

\protect\hyperlink{site-content}{Skip to
content}\protect\hyperlink{site-index}{Skip to site index}

\href{https://www.nytimes3xbfgragh.onion/section/style}{Style}

\href{https://myaccount.nytimes3xbfgragh.onion/auth/login?response_type=cookie\&client_id=vi}{}

\href{https://www.nytimes3xbfgragh.onion/section/todayspaper}{Today's
Paper}

\href{/section/style}{Style}\textbar{}The New Panic Buys: Kayaks, Pools,
Tents and Trampolines

\url{https://nyti.ms/31k07w7}

\begin{itemize}
\item
\item
\item
\item
\item
\item
\end{itemize}

\href{https://www.nytimes3xbfgragh.onion/spotlight/at-home?action=click\&pgtype=Article\&state=default\&region=TOP_BANNER\&context=at_home_menu}{At
Home}

\begin{itemize}
\tightlist
\item
  \href{https://www.nytimes3xbfgragh.onion/2020/08/03/well/family/the-benefits-of-talking-to-strangers.html?action=click\&pgtype=Article\&state=default\&region=TOP_BANNER\&context=at_home_menu}{Talk:
  To Strangers}
\item
  \href{https://www.nytimes3xbfgragh.onion/2020/08/01/at-home/coronavirus-make-pizza-on-a-grill.html?action=click\&pgtype=Article\&state=default\&region=TOP_BANNER\&context=at_home_menu}{Make:
  Grilled Pizza}
\item
  \href{https://www.nytimes3xbfgragh.onion/2020/07/31/arts/television/goldbergs-abc-stream.html?action=click\&pgtype=Article\&state=default\&region=TOP_BANNER\&context=at_home_menu}{Watch:
  'The Goldbergs'}
\item
  \href{https://www.nytimes3xbfgragh.onion/interactive/2020/at-home/even-more-reporters-editors-diaries-lists-recommendations.html?action=click\&pgtype=Article\&state=default\&region=TOP_BANNER\&context=at_home_menu}{Explore:
  Reporters' Google Docs}
\end{itemize}

Advertisement

\protect\hyperlink{after-top}{Continue reading the main story}

Supported by

\protect\hyperlink{after-sponsor}{Continue reading the main story}

\hypertarget{the-new-panic-buys-kayaks-pools-tents-and-trampolines}{%
\section{The New Panic Buys: Kayaks, Pools, Tents and
Trampolines}\label{the-new-panic-buys-kayaks-pools-tents-and-trampolines}}

Americans seeking getaways (or staycations) in the pandemic are putting
outdoor retailers in a supply crunch.

\includegraphics{https://static01.graylady3jvrrxbe.onion/images/2020/08/06/fashion/03BACKORDER1/merlin_172496982_de7aa074-951d-4362-ba4c-5fe9eb17553e-articleLarge.jpg?quality=75\&auto=webp\&disable=upscale}

By \href{https://www.nytimes3xbfgragh.onion/by/john-herrman}{John
Herrman}

\begin{itemize}
\item
  Aug. 4, 2020
\item
  \begin{itemize}
  \item
  \item
  \item
  \item
  \item
  \item
  \end{itemize}
\end{itemize}

In January, the Outdoor Foundation, the nonprofit arm of the Outdoor
Industry Association, shared some gloomy statistics. According to its
\href{https://outdoorindustry.org/resource/2019-outdoor-participation-report/}{annual
survey}, fewer than 20 percent of Americans were recreating outside at
least once a week, working out to an estimated billion fewer outdoor
activities a year compared to a decade earlier. Just under half of the
population, the organization estimated, was not participating in outdoor
recreation at all.

``Outdoor participation,'' the survey also found, still skewed slightly
male and more than slightly white. Overall, a ``historical downward
trend'' indicated that ``Americans will likely continue spending less
time outdoors'' due in part to ``work and family demands as well as
technology and cost of entry.''

As many of these Americans would soon find out, preferring not to go
outside is quite a different thing from being told to stay inside. As
official and de facto lockdowns dragged on, and something like a
consensus about the relative safety of outdoor activity emerged, an
indoor country found itself suddenly, and rather urgently, drawn out.

One way to tell the story of the pandemic is through its shortages.
Early and ongoing P.P.E. shortages underscored the threat of the virus
and how woefully unprepared the U.S. was to confront it. Subsequent
stockpiling ---~of toilet paper, canned food, weapons --- revealed the
strange and complicated ways Americans process disasters. Facing
widespread business closures and stay-at-home orders, customers looking
to buy home fitness equipment, grooming supplies or video game consoles
ran up against shortages and backorders.

Already, summer is oozing by, and the uncertainty of autumn is creeping
closer. Outdoor pools and trampolines were an early warning sign. Bait
and tackle spiked in April. Now, it's a mad rush for outdoor gear. All
of it.

``For us, it's been a bit of a mirror of the human condition,'' said
Paul Calandrella, a merchandising manager at REI. First, he said, people
came for survival gear: freeze-dried food, water filtration and backup
power supplies. ``They surprised us out of the gate.''

\includegraphics{https://static01.graylady3jvrrxbe.onion/images/2020/08/06/fashion/03BACKORDER2/merlin_141725751_aeb974b8-b764-4e32-b179-2d7200dc4394-articleLarge.jpg?quality=75\&auto=webp\&disable=upscale}

First came the bike rush, which caused shortages that began in March and
have extended, for some brands, into next year. Next to run low were
kayaks and stand-up paddle boards. Soon, running shoes were picked over.
Sales at outdoor retailers were up almost across the board, with a few
exceptions in specialized categories (indoor climbing gear, for
example). Next up: camping gear. In the space of a few months, outdoor
retailers went from solving a marketing problem to wresting with supply
chain problems.

``We're probably in the early days of feeling some constraints on tents,
sleeping bags and camping furniture,'' said Mr. Calandrella. ``One place
that really caught us off guard was backpacking food,'' he said.

Freeze-dried food and camp fuel suppliers, already cleaned out by
prepper-minded customers months before, were confronted with
``extraordinary levels of demand'' from actual backpackers looking to
escape into the wilderness. ``It was whiplash.''

Mountain House, a popular freeze-dried food brand, has a warning on its
website: ``Due to increased demand, many of our products are out of
stock.'' An upscale competitor, Good To-Go, founded by the decorated
chef Jennifer Scism and her husband, David Koorits, has paused sales
through its website while it tries to catch up with orders from
retailers. The company has increased its staff by more than 10 percent;
after an early spring surge followed by a brief lull, August orders are
five times what they were last year.

Some sought-after items suggest that people may also be camping in their
own backyards. Hammocks, after a few mysteriously slow years, are back
in high demand. Powered coolers are hard to keep in stock. Good To-Go
has seen particular demand for some of its less eclectic foods, such as
chili and pasta. ``Food that would also be kid friendly,'' Mr. Koorits
said.

Moosejaw, the outdoor retailer acquired by Walmart in 2017, has been
looking for new suppliers for tents and camp furniture, two categories
where sales more than doubled this year; new birders have taken popular
binocular models out of stock. Customer moods have changed accordingly.

``Generally speaking, with these more considered purchases, people they
do their homework, and they'll come to us having already picked out a
product,'' said Eoin Comerford, Moosejaw's C.E.O. ``With some of these
supply chain constraints, people are coming to us and saying, `do you
have a bike?'''

``Back in March, we didn't have a picture of how it would turn out,''
said Helen Johnson-Leipold, the chairman and chief executive of Johnson
Outdoors, which owns the camping gear brand Eureka, a backpacking stove
company Jetboil and the boat brands Old Town and Ocean Kayak, among
others. ``We had more cancellations than we had new orders,'' she said.
At the same time, in response to the spread of Covid-19, Johnson
Outdoors was keeping many office and factory workers at home.

By the end of April, however, interest had resumed. ``Usually, with
significant orders we have a little bit of lead time,'' Mrs.
Johnson-Leipold said. Not this time. Demand for boats ---~tandem kayaks
stood out as a surprising example --- was ``through the roof,'' with
orders soon outpacing production. People were fishing more, or taking it
up for the first time. Later, especially as public lands began to
reopen, demand for camping gear followed; Mrs. Johnson-Leipold suspects
that, if current trends continue, Eureka and Jetboil could soon be
dealing with backorders, too.

For much of the outdoor recreation industry, this is generally good
news: lots of people buying lots of things, potentially finding hobbies
or forming habits that will last for years. ``The participation rate
will grow,'' said Mrs. Johnson-Leipold.

Outdoor recreation is, of course, seasonal, and nobody is quite sure
what's going to happen when winter comes. REI, for example, is
reconsidering what its stores will look like come fall, when customers
would normally expect to see store displays filled with gear for ski and
snowboard outings that, this year, could be risky or impossible.

Shortages also have consequences.~A lot of outdoor gear is produced
overseas, ordered a year ahead and sold seasonally, meaning some
sold-out items won't reappear until 2021. As with bikes, however, many
of the first products to sell out are more affordable and practical
options ---~a stark reminder that some of the industry's underlying
problems are the same as they were in January.

``The dampening effect is real,'' said Kenji Haroutunian, a consultant
and diversity and inclusion advocate for the outdoor industry. Outside
is free, as plenty of enthusiasts will insist, but outdoor gear can be
expensive, even in the best of times. As Covid-19 continues to spread,
many formerly social activities have been reduced to individual or
family pursuits, leaving would-be participants without access to the
people and communities that could otherwise welcome them. ``Indoor
climbing, for example, is more diverse than outdoor,'' said Mr.
Haroutunian, but gyms are closed, ``and that's cutting off a chain of
new climbers.''

Mr. Haroutunian is cautiously optimistic. Close-to-home recreation is on
the rise around diverse urban centers. Lots of outdoors companies are
hiring, and he's heard that larger general interest retailers are
ramping up investments in recreational gear, some of it more affordable
than what you might find at REI. It's an opportunity to be seized or
blown. ``When times are good, people forget that you have to invest in
maintaining the future,'' he said.

Advertisement

\protect\hyperlink{after-bottom}{Continue reading the main story}

\hypertarget{site-index}{%
\subsection{Site Index}\label{site-index}}

\hypertarget{site-information-navigation}{%
\subsection{Site Information
Navigation}\label{site-information-navigation}}

\begin{itemize}
\tightlist
\item
  \href{https://help.nytimes3xbfgragh.onion/hc/en-us/articles/115014792127-Copyright-notice}{©~2020~The
  New York Times Company}
\end{itemize}

\begin{itemize}
\tightlist
\item
  \href{https://www.nytco.com/}{NYTCo}
\item
  \href{https://help.nytimes3xbfgragh.onion/hc/en-us/articles/115015385887-Contact-Us}{Contact
  Us}
\item
  \href{https://www.nytco.com/careers/}{Work with us}
\item
  \href{https://nytmediakit.com/}{Advertise}
\item
  \href{http://www.tbrandstudio.com/}{T Brand Studio}
\item
  \href{https://www.nytimes3xbfgragh.onion/privacy/cookie-policy\#how-do-i-manage-trackers}{Your
  Ad Choices}
\item
  \href{https://www.nytimes3xbfgragh.onion/privacy}{Privacy}
\item
  \href{https://help.nytimes3xbfgragh.onion/hc/en-us/articles/115014893428-Terms-of-service}{Terms
  of Service}
\item
  \href{https://help.nytimes3xbfgragh.onion/hc/en-us/articles/115014893968-Terms-of-sale}{Terms
  of Sale}
\item
  \href{https://spiderbites.nytimes3xbfgragh.onion}{Site Map}
\item
  \href{https://help.nytimes3xbfgragh.onion/hc/en-us}{Help}
\item
  \href{https://www.nytimes3xbfgragh.onion/subscription?campaignId=37WXW}{Subscriptions}
\end{itemize}
