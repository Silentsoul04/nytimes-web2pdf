Sections

SEARCH

\protect\hyperlink{site-content}{Skip to
content}\protect\hyperlink{site-index}{Skip to site index}

\href{https://www.nytimes3xbfgragh.onion/section/technology}{Technology}

\href{https://myaccount.nytimes3xbfgragh.onion/auth/login?response_type=cookie\&client_id=vi}{}

\href{https://www.nytimes3xbfgragh.onion/section/todayspaper}{Today's
Paper}

\href{/section/technology}{Technology}\textbar{}When Taming Big Tech
Goes Wrong

\href{https://nyti.ms/2DcEm9A}{https://nyti.ms/2DcEm9A}

\begin{itemize}
\item
\item
\item
\item
\item
\end{itemize}

Advertisement

\protect\hyperlink{after-top}{Continue reading the main story}

Supported by

\protect\hyperlink{after-sponsor}{Continue reading the main story}

on tech

\hypertarget{when-taming-big-tech-goes-wrong}{%
\section{When Taming Big Tech Goes
Wrong}\label{when-taming-big-tech-goes-wrong}}

What the United States can learn from Europe's efforts to restrain
America's tech giants.

\includegraphics{https://static01.graylady3jvrrxbe.onion/images/2020/08/04/business/04ontech/04ontech-articleLarge.jpg?quality=75\&auto=webp\&disable=upscale}

\href{https://www.nytimes3xbfgragh.onion/by/shira-ovide}{\includegraphics{https://static01.graylady3jvrrxbe.onion/images/2020/03/18/reader-center/author-shira-ovide/author-shira-ovide-thumbLarge-v2.png}}

By \href{https://www.nytimes3xbfgragh.onion/by/shira-ovide}{Shira Ovide}

\begin{itemize}
\item
  Aug. 4, 2020
\item
  \begin{itemize}
  \item
  \item
  \item
  \item
  \item
  \end{itemize}
\end{itemize}

\emph{This article is part of the On Tech newsletter. You can}
\href{https://www.nytimes3xbfgragh.onion/newsletters/signup/OT}{\emph{sign
up here}} \emph{to receive it weekdays.}

What if the U.S. government tries to restrain the power of America's
tech superstars, and it doesn't work?

That's essentially what has happened so far in Europe, where some
regulators and lawmakers have been ahead of the United States in putting
guardrails on Google, Amazon, Facebook and Apple.

I talked with my colleague
\href{https://www.nytimes3xbfgragh.onion/by/adam-satariano}{Adam
Satariano}, who writes about technology in Europe, about what happens
when government officials decide they want to do \emph{something} about
Big Tech, but struggle to figure out what that something should be.

\textbf{Shira: What's been Europe's approach to the American tech
superpowers?}

\textbf{Adam:} Europe started doing antitrust investigations several
years ago. But the resulting lawsuits and regulation haven't done much.

Google is the best example. Investigations into allegations of
anti-competitive tactics took years. The European Union then
\href{https://www.nytimes3xbfgragh.onion/2018/07/18/technology/google-eu-android-fine.html}{fined}
Google
\href{https://www.nytimes3xbfgragh.onion/2017/06/27/technology/eu-google-fine.html}{several}
billion
\href{https://www.nytimes3xbfgragh.onion/2019/03/20/business/google-fine-advertising.html}{dollars}
for breaking the law. But critics say that Google wasn't forced to
change much, so this
\href{https://www.nytimes3xbfgragh.onion/2019/11/11/business/europe-technology-antitrust-regulation.html}{did
little to restore competition}.

\textbf{What can American lawmakers and regulators learn from Europe's
mistakes?}

Speed is crucial, otherwise regulators are fighting yesterday's battles.

There was also a problem with enforcement. Europe passed a highly-touted
privacy law called General Data Protection Regulation, but it's been
\href{https://www.nytimes3xbfgragh.onion/2020/04/27/technology/GDPR-privacy-law-europe.html}{a
flop at limiting data collection} by the biggest tech companies. Much of
the policing of the law was left to Ireland, which simply doesn't have
the resources to keep up.

\textbf{It sounds as if the lesson is that you can believe there's a
problem with tech companies' power, but it's hard to craft an
appropriate response.}

Without a doubt. The authorities in Europe are
\href{https://www.nytimes3xbfgragh.onion/2020/07/30/technology/europe-new-phase-tech-amazon-apple-facebook-google.html}{changing
their tactics now}. They're looking at rewriting laws to directly target
the way big tech companies do business: How they use data or box out
rivals, and ways they give their products preferential treatment.

But after a few years covering this in Europe, I have become much more
suspect about these efforts at regulation. There's a lot of potential
for unintended consequences.

\textbf{What have been the unintended consequences?}

A big one is the ripple effects of several countries, especially
Germany, making new rules against hate speech online. The worry is that
\href{https://www.nytimes3xbfgragh.onion/2019/05/06/technology/europe-tech-censorship.html}{gave
cover to countries to enact censorship laws}. Turkey cited Germany's
hate speech law in enacting its new social media restrictions. Another
law,
\href{https://www.nytimes3xbfgragh.onion/2019/09/23/technology/right-to-be-forgotten-law-europe.html}{known
as the right to be forgotten}, has also raised concerns about limiting
the availability of information online.

\textbf{Does this mean we should be skeptical about any U.S. government
action against Big Tech?}

Maybe. But it's still been interesting to watch this growing realization
among the authorities on both sides of the Atlantic about the size of
these companies and the influence they hold over their democratic
societies and how they operate, including their communication systems,
information flow and commerce.

You can sense this feeling that lawmakers think they must do something,
but aren't quite sure what that is yet. The next 12 to 18 months are
going to be extraordinary.

\emph{Hello, dear readers! Several of you had asked about differences in
how the United States and Europe handle regulation of technology. This
conversation gets the ball rolling on that subject. Please keep the
questions coming. You can reach us at
\href{mailto:ontech@NYTimes.com}{\nolinkurl{ontech@NYTimes.com}}.}

\begin{center}\rule{0.5\linewidth}{\linethickness}\end{center}

\hypertarget{twitters-blunder-is-a-symptom-of-our-broken-internet}{%
\subsection{Twitter's blunder is a symptom of our broken
internet}\label{twitters-blunder-is-a-symptom-of-our-broken-internet}}

Twitter is in hot water with the U.S. government. Let me explain why you
should be angry.

The Federal Trade Commission is
\href{https://www.nytimes3xbfgragh.onion/2020/08/03/technology/ftc-twitter-privacy-violations.html}{investigating
Twitter} and may fine the company up to \$250 million for using people's
phone numbers in ways that users didn't expect, my colleague Kate Conger
wrote.

This stems from Twitter's
\href{https://help.twitter.com/en/information-and-ads}{disclosure} in
October that the phone number you might have listed as a
\href{https://www.nytimes3xbfgragh.onion/2017/08/08/technology/personaltech/protecting-your-accounts-by-text-or-app.html}{second
form of account verification} --- in addition to a password --- might
also have been used to target advertisements at you. This is not good.

Phone numbers used for security purposes shouldn't be repurposed for
Twitter to make money. Twitter said this was a mistake, but it hasn't
said how long this practice had gone on or how it made this error.
(Facebook had
\href{https://www.washingtonpost.com/technology/2019/07/23/facebook-deceived-users-about-way-it-used-phone-numbers-facial-recognition-ftc-allege-complaint/}{gotten
in trouble for the same thing}.)

There are two interrelated problems revealed by what might seem like
Twitter's minor flub.

First is the problem of companies ignoring common sense. People who
enter a phone number to keep impostors out of their Twitter accounts do
not expect that number to be used to target ads. Period. Therefore,
companies should absolutely not do this.

Two, Twitter's mistake is a symptom of the broader scourge of online
surveillance. Internet advertising
\href{https://www.nytimes3xbfgragh.onion/2020/07/15/technology/just-collect-less-data-period.html}{is
an arms race waged with our data}, and that leads to ever more intrusive
harvesting and use of our information.

This is not only a concern for tin-foil-hat privacy paranoiacs.
Companies like Facebook and Google have a big leg up over all other
companies that sell digital ads because they have more and better data
than anyone else. Every other company is tempted to play catch up by
resorting to increasingly creepy ways of collecting user data.

That might include Twitter, or apps we've never heard of
\href{https://www.nytimes3xbfgragh.onion/interactive/2018/12/10/business/location-data-privacy-apps.html}{that
gather our location information and sell it without our knowledge to
other data-hungry companies}. All the tech problems are connected.

\begin{center}\rule{0.5\linewidth}{\linethickness}\end{center}

\hypertarget{before-we-go-}{%
\subsection{Before we go \ldots{}}\label{before-we-go-}}

\begin{itemize}
\item
  \textbf{I suspect people will have} \emph{\textbf{feelings}}
  \textbf{about this:} Stacey Steinberg, a children's rights lawyer and
  photographer, says that parents sharing information, including photos,
  about their children online may unwittingly be providing fodder for
  bullying or child abuse. These risks have
  \href{https://www.nytimes3xbfgragh.onion/2020/08/04/well/family/parents-social-media-privacy.html}{made
  her think twice before she shares information} about her kids on
  social media.
\item
  \textbf{Maybe this newsletter should just be memes?} On Instagram and
  other apps, news organizations and activists have captivated
  audiences' attention by posting less text and more charts, graphics
  and other
  \href{https://www.axios.com/instagram-information-coronavirus-george-floyd-706d1ad3-9d6d-48d4-b085-3469a0d93b56.html}{easy-to-digest
  ways to convey news and information}, according to Axios. This is a
  pattern that younger people latched onto a long time ago: Quick-scan
  visuals like memes are
  \href{https://www.nytimes3xbfgragh.onion/2019/12/24/style/finance-memes.html}{a
  great way to explain complicated things}.
\item
  \textbf{The online forum that traces the internet's past five years:}
  Wired has a
  \href{https://www.wired.com/story/the-hate-fueled-rise-of-rthe-donald-and-its-epic-takedown/}{fascinating
  look} at a Reddit group about President Trump that became one of the
  most popular and divisive gatherings online. The article tells how
  this group's story encapsulated the evolution of internet forums in
  molding or policing people's personal interactions and beliefs.
\end{itemize}

\hypertarget{hugs-to-this}{%
\subsubsection{Hugs to this}\label{hugs-to-this}}

Check out these \href{https://vimeo.com/424875558}{ethereal cuttlefish
embryos bobbing around in their eggs}. Yes, those tiny black dots are
the cuttlefish babies' eyes. (Thank you to an On Tech reader, Dr. Julie
Drawbridge, for suggesting this video.)

\begin{center}\rule{0.5\linewidth}{\linethickness}\end{center}

\emph{We want to hear from you. Tell us what you think of this
newsletter and what else you'd like us to explore. You can reach us at}
\href{mailto:ontech@NYTimes.com?subject=On\%20Tech\%20Feedback}{\emph{ontech@NYTimes.com.}}
**

\emph{If you don't already get this newsletter in your inbox,}
\href{https://www.nytimes3xbfgragh.onion/newsletters/signup/OT}{\emph{please
sign up here}}\emph{.}

Advertisement

\protect\hyperlink{after-bottom}{Continue reading the main story}

\hypertarget{site-index}{%
\subsection{Site Index}\label{site-index}}

\hypertarget{site-information-navigation}{%
\subsection{Site Information
Navigation}\label{site-information-navigation}}

\begin{itemize}
\tightlist
\item
  \href{https://help.nytimes3xbfgragh.onion/hc/en-us/articles/115014792127-Copyright-notice}{©~2020~The
  New York Times Company}
\end{itemize}

\begin{itemize}
\tightlist
\item
  \href{https://www.nytco.com/}{NYTCo}
\item
  \href{https://help.nytimes3xbfgragh.onion/hc/en-us/articles/115015385887-Contact-Us}{Contact
  Us}
\item
  \href{https://www.nytco.com/careers/}{Work with us}
\item
  \href{https://nytmediakit.com/}{Advertise}
\item
  \href{http://www.tbrandstudio.com/}{T Brand Studio}
\item
  \href{https://www.nytimes3xbfgragh.onion/privacy/cookie-policy\#how-do-i-manage-trackers}{Your
  Ad Choices}
\item
  \href{https://www.nytimes3xbfgragh.onion/privacy}{Privacy}
\item
  \href{https://help.nytimes3xbfgragh.onion/hc/en-us/articles/115014893428-Terms-of-service}{Terms
  of Service}
\item
  \href{https://help.nytimes3xbfgragh.onion/hc/en-us/articles/115014893968-Terms-of-sale}{Terms
  of Sale}
\item
  \href{https://spiderbites.nytimes3xbfgragh.onion}{Site Map}
\item
  \href{https://help.nytimes3xbfgragh.onion/hc/en-us}{Help}
\item
  \href{https://www.nytimes3xbfgragh.onion/subscription?campaignId=37WXW}{Subscriptions}
\end{itemize}
