Sections

SEARCH

\protect\hyperlink{site-content}{Skip to
content}\protect\hyperlink{site-index}{Skip to site index}

\href{https://www.nytimes3xbfgragh.onion/section/us}{U.S.}

\href{https://myaccount.nytimes3xbfgragh.onion/auth/login?response_type=cookie\&client_id=vi}{}

\href{https://www.nytimes3xbfgragh.onion/section/todayspaper}{Today's
Paper}

\href{/section/us}{U.S.}\textbar{}With Census Count Finishing Early,
Fears of a Skewed Tally Rise

\url{https://nyti.ms/2XuiayB}

\begin{itemize}
\item
\item
\item
\item
\item
\end{itemize}

\href{https://www.nytimes3xbfgragh.onion/news-event/coronavirus?action=click\&pgtype=Article\&state=default\&region=TOP_BANNER\&context=storylines_menu}{The
Coronavirus Outbreak}

\begin{itemize}
\tightlist
\item
  live\href{https://www.nytimes3xbfgragh.onion/2020/08/04/world/coronavirus-cases.html?action=click\&pgtype=Article\&state=default\&region=TOP_BANNER\&context=storylines_menu}{Latest
  Updates}
\item
  \href{https://www.nytimes3xbfgragh.onion/interactive/2020/us/coronavirus-us-cases.html?action=click\&pgtype=Article\&state=default\&region=TOP_BANNER\&context=storylines_menu}{Maps
  and Cases}
\item
  \href{https://www.nytimes3xbfgragh.onion/interactive/2020/science/coronavirus-vaccine-tracker.html?action=click\&pgtype=Article\&state=default\&region=TOP_BANNER\&context=storylines_menu}{Vaccine
  Tracker}
\item
  \href{https://www.nytimes3xbfgragh.onion/2020/08/02/us/covid-college-reopening.html?action=click\&pgtype=Article\&state=default\&region=TOP_BANNER\&context=storylines_menu}{College
  Reopening}
\item
  \href{https://www.nytimes3xbfgragh.onion/live/2020/08/04/business/stock-market-today-coronavirus?action=click\&pgtype=Article\&state=default\&region=TOP_BANNER\&context=storylines_menu}{Economy}
\end{itemize}

Advertisement

\protect\hyperlink{after-top}{Continue reading the main story}

Supported by

\protect\hyperlink{after-sponsor}{Continue reading the main story}

\hypertarget{with-census-count-finishing-early-fears-of-a-skewed-tally-rise}{%
\section{With Census Count Finishing Early, Fears of a Skewed Tally
Rise}\label{with-census-count-finishing-early-fears-of-a-skewed-tally-rise}}

With 60 million households still uncounted, the bureau said it would
wrap up the survey a month early. Critics called it a bald move to
politicize the count in favor of Republicans.

\includegraphics{https://static01.graylady3jvrrxbe.onion/images/2020/08/04/us/04CENUS/merlin_175075221_67d12f76-904e-4011-838b-bb8359329694-articleLarge.jpg?quality=75\&auto=webp\&disable=upscale}

By \href{https://www.nytimes3xbfgragh.onion/by/michael-wines}{Michael
Wines} and
\href{https://www.nytimes3xbfgragh.onion/by/richard-fausset}{Richard
Fausset}

\begin{itemize}
\item
  Aug. 4, 2020
\item
  \begin{itemize}
  \item
  \item
  \item
  \item
  \item
  \end{itemize}
\end{itemize}

WASHINGTON --- With the Trump administration's decision to end the 2020
census count four weeks early, the Census Bureau now has to accomplish
what officials have said it cannot do: accurately count the nation's
hardest-to-reach residents --- nearly four of every 10 households --- in
just six weeks.

The result is both a logistical challenge of enormous proportions that
must take place in the middle of a pandemic, and yet another political
crisis for the census, historically a nonpartisan enterprise. The
announcement, which came Monday evening, immediately generated sharp
criticism.

On Tuesday, four former directors of the Census Bureau issued a
statement warning that an earlier deadline would ``result in seriously
incomplete enumerations in many areas across our country,'' and urged
the administration to restore the lost weeks. The directors, who served
under Democratic and Republican presidents, also urged Congress to
assemble a trusted body of experts to develop standards for assessing
the quality of the bureau's population totals.

\href{https://sdcclearinghouse.files.wordpress.com/2020/08/2020-census-timeline-july-2020.pdf}{A
similar plea was issued} on Tuesday by an official network of agencies
and nonprofit institutions that act as liaisons between the Census
Bureau and state governments, helping them use population data to make
policies.

``The credibility of the U.S. Census Bureau as the gold standard of data
in the United States will be undermined by rushing an incomplete census
count to meet deadlines,'' a letter from the group stated.

The Census Bureau, which had earlier set and planned on an April 2021
deadline because of the coronavirus pandemic, said the change was needed
to meet a federal deadline to get the numbers to President Trump by the
end of the year. But Democratic lawmakers said the change reflected a
deliberate attempt to undercount groups that tend to support their
party.

Representative Steny H. Hoyer, Democrat of Maryland and the House
majority leader, said on Tuesday that the change was an attempt to
undercount poor communities with large numbers of immigrants and ethnic
minorities and called the shortened schedule ``yet another example of
this administration's blatant assault on our Constitution and our
democracy.''

Federal law requires the Census Bureau to send population totals to the
president by Dec. 31 of every census year. But the pandemic forced
census officials in April to rewrite that timeline, pushing delivery of
population totals to April 2021. The House approved the new deadline in
May, but the Republican-controlled Senate has not followed suit,
apparently at Mr. Trump's behest.

The Constitution requires a count of all residents, but Mr. Trump has
long made clear his desire to have population counts of citizens, not
all residents. The president ordered the Census Bureau last month to
produce a state-by-state count of undocumented immigrants so he could
remove undocumented residents from census totals before he sends them to
Congress for use in reapportioning the House. Several lawsuits have
argued that would be unconstitutional.

An end-of-year delivery of population figures could provide a different
avenue for Mr. Trump to remove undocumented immigrants --- by not
counting them in the first place. And delaying the totals until next
year, as had been planned, would open the possibility that the totals
would go to a new president and Congress.

\hypertarget{latest-updates-global-coronavirus-outbreak}{%
\section{\texorpdfstring{\href{https://www.nytimes3xbfgragh.onion/2020/08/04/world/coronavirus-cases.html?action=click\&pgtype=Article\&state=default\&region=MAIN_CONTENT_1\&context=storylines_live_updates}{Latest
Updates: Global Coronavirus
Outbreak}}{Latest Updates: Global Coronavirus Outbreak}}\label{latest-updates-global-coronavirus-outbreak}}

Updated 2020-08-05T07:58:24.076Z

\begin{itemize}
\tightlist
\item
  \href{https://www.nytimes3xbfgragh.onion/2020/08/04/world/coronavirus-cases.html?action=click\&pgtype=Article\&state=default\&region=MAIN_CONTENT_1\&context=storylines_live_updates\#link-762df92}{As
  talks drag on, McConnell signals openness to jobless aid extension,
  and negotiators agree on a deadline.}
\item
  \href{https://www.nytimes3xbfgragh.onion/2020/08/04/world/coronavirus-cases.html?action=click\&pgtype=Article\&state=default\&region=MAIN_CONTENT_1\&context=storylines_live_updates\#link-1228a480}{Novavax
  sees encouraging results from two studies of its experimental
  vaccine.}
\item
  \href{https://www.nytimes3xbfgragh.onion/2020/08/04/world/coronavirus-cases.html?action=click\&pgtype=Article\&state=default\&region=MAIN_CONTENT_1\&context=storylines_live_updates\#link-794484ed}{Mississippians
  must now wear masks in public, governor says.}
\end{itemize}

\href{https://www.nytimes3xbfgragh.onion/2020/08/04/world/coronavirus-cases.html?action=click\&pgtype=Article\&state=default\&region=MAIN_CONTENT_1\&context=storylines_live_updates}{See
more updates}

More live coverage:
\href{https://www.nytimes3xbfgragh.onion/live/2020/08/04/business/stock-market-today-coronavirus?action=click\&pgtype=Article\&state=default\&region=MAIN_CONTENT_1\&context=storylines_live_updates}{Markets}

Mr. Trump had tried earlier to achieve a similar objective by adding a
question on citizenship to the census, but the Supreme Court rejected
that effort last year after an extended legal battle.

Some of his most ardent Republican backers have supported his efforts
aimed at not counting undocumented immigrants.

``Instead of working with Republicans to conduct credible oversight of
2020 census operations to ensure that all Americans are counted,
Democrats are instead choosing to recklessly reduce census participation
by resorting to lies and scare tactics,'' Representative James R. Comer
of Kentucky, a Republican who is the ranking member on the Oversight
Committee, said in a recent statement.

But the campaign has
\href{https://www.msn.com/en-us/news/politics/trump-s-census-memo-puts-some-republicans-in-catch-22/ar-BB17gKkQ}{placed
some Republicans in a bind} of sorts, since not counting undocumented
immigrants in major Republican states like Texas and Florida probably
would cause them to lose seats in the House in the next reapportionment.

``It's sort of a Catch-22 because it would hurt Florida'' to the extent
that the state has undocumented immigrants, Senator Marco Rubio,
Republican of Florida, said last week in an interview with MSNBC. ``But
at the same time, it dilutes the representation of people that are here
legally and eligible to vote.''

The decision to shorten the counting schedule is a U-turn from the
bureau's statement months ago that the pandemic
\href{https://www.nytimes3xbfgragh.onion/2020/04/13/us/census-coronavirus-delay.html?searchResultPosition=12}{had
made it necessary} to ask for more time to complete the count. And
census experts have said that shortening the time frame would wreak
havoc with efforts to reach the very hardest-to-count households that
have long been flagged as most likely to be missed in this year's tally.

``This is a whole systemic attack on the census for political gain,''
Julie Menin, the census director for New York City, said. ``There's an
intentional attempt here to basically steal the census --- to politicize
this census to gain Republican seats across the country.''

So far this year, nearly 63 percent of households have voluntarily
completed census surveys, either online, by mail or by telephone. Exact
comparisons to the 2010 census are not possible because this year's
response deadline has not passed, but during that census the
\href{https://www.politifact.com/factchecks/2020/jul/31/stacey-abrams/fact-checking-stacey-abrams-comparison-census-resp/}{peak
response rate was slightly higher, at 66.5 percent}. In part because of
population growth, however, the bureau this year has a considerably
larger number of households to track down and count --- 60 million,
compared with 47 million in 2010.

The schedule change announced on Monday primarily affects the count of
those 60 million households, but it also compresses the time left for
tallying other groups, including homeless people and residents of
nursing homes and dormitories.

All of those counts normally would be completed this month, and some
well before that, but the bureau said in April that it was giving itself
until Oct. 31 to complete the count.

The latest
\href{https://www.nytimes3xbfgragh.onion/2020/07/28/us/trump-census.html}{schedule
change} will move that deadline up by one month, to Sept. 30. The effect
is to shorten to six weeks what had been a 10-week period reserved for
completing the count, so that the data can be compiled and processed in
time to deliver population totals by year's end.

In recent months, however, some senior census officials have been clear
that they believe that timeline is not feasible.

``We can't do that anymore,'' the census official leading field
operations for the count, Tim Olson,
\href{https://www.youtube.com/watch?v=F6IyJMtDDgY\&feature=youtu.be\&t=4688}{told
a Native American organization} during a webinar in May. The associate
director of the census, Albert E. Fontenot Jr., echoed that last month,
saying ``we are past the window of being able to get those counts'' by
year's end.

Some state and local officials in areas with diverse, hard-to-count
populations reacted with dismay.

``Oh God,'' said Shameka S. Reynolds, the mayor of Lithonia, Ga.
``Covid-19 is on the rise in our county, and it was already hard, to be
honest with you, in previous years. It was hard trying to get people to
get their mailings in and do it.''

\href{https://www.nytimes3xbfgragh.onion/news-event/coronavirus?action=click\&pgtype=Article\&state=default\&region=MAIN_CONTENT_3\&context=storylines_faq}{}

\hypertarget{the-coronavirus-outbreak-}{%
\subsubsection{The Coronavirus Outbreak
›}\label{the-coronavirus-outbreak-}}

\hypertarget{frequently-asked-questions}{%
\paragraph{Frequently Asked
Questions}\label{frequently-asked-questions}}

Updated August 4, 2020

\begin{itemize}
\item ~
  \hypertarget{i-have-antibodies-am-i-now-immune}{%
  \paragraph{I have antibodies. Am I now
  immune?}\label{i-have-antibodies-am-i-now-immune}}

  \begin{itemize}
  \tightlist
  \item
    As of right
    now,\href{https://www.nytimes3xbfgragh.onion/2020/07/22/health/covid-antibodies-herd-immunity.html?action=click\&pgtype=Article\&state=default\&region=MAIN_CONTENT_3\&context=storylines_faq}{that
    seems likely, for at least several months.} There have been
    frightening accounts of people suffering what seems to be a second
    bout of Covid-19. But experts say these patients may have a
    drawn-out course of infection, with the virus taking a slow toll
    weeks to months after initial exposure. People infected with the
    coronavirus typically
    \href{https://www.nature.com/articles/s41586-020-2456-9}{produce}
    immune molecules called antibodies, which are
    \href{https://www.nytimes3xbfgragh.onion/2020/05/07/health/coronavirus-antibody-prevalence.html?action=click\&pgtype=Article\&state=default\&region=MAIN_CONTENT_3\&context=storylines_faq}{protective
    proteins made in response to an
    infection}\href{https://www.nytimes3xbfgragh.onion/2020/05/07/health/coronavirus-antibody-prevalence.html?action=click\&pgtype=Article\&state=default\&region=MAIN_CONTENT_3\&context=storylines_faq}{.
    These antibodies may} last in the body
    \href{https://www.nature.com/articles/s41591-020-0965-6}{only two to
    three months}, which may seem worrisome, but that's perfectly normal
    after an acute infection subsides, said Dr. Michael Mina, an
    immunologist at Harvard University. It may be possible to get the
    coronavirus again, but it's highly unlikely that it would be
    possible in a short window of time from initial infection or make
    people sicker the second time.
  \end{itemize}
\item ~
  \hypertarget{im-a-small-business-owner-can-i-get-relief}{%
  \paragraph{I'm a small-business owner. Can I get
  relief?}\label{im-a-small-business-owner-can-i-get-relief}}

  \begin{itemize}
  \tightlist
  \item
    The
    \href{https://www.nytimes3xbfgragh.onion/article/small-business-loans-stimulus-grants-freelancers-coronavirus.html?action=click\&pgtype=Article\&state=default\&region=MAIN_CONTENT_3\&context=storylines_faq}{stimulus
    bills enacted in March} offer help for the millions of American
    small businesses. Those eligible for aid are businesses and
    nonprofit organizations with fewer than 500 workers, including sole
    proprietorships, independent contractors and freelancers. Some
    larger companies in some industries are also eligible. The help
    being offered, which is being managed by the Small Business
    Administration, includes the Paycheck Protection Program and the
    Economic Injury Disaster Loan program. But lots of folks have
    \href{https://www.nytimes3xbfgragh.onion/interactive/2020/05/07/business/small-business-loans-coronavirus.html?action=click\&pgtype=Article\&state=default\&region=MAIN_CONTENT_3\&context=storylines_faq}{not
    yet seen payouts.} Even those who have received help are confused:
    The rules are draconian, and some are stuck sitting on
    \href{https://www.nytimes3xbfgragh.onion/2020/05/02/business/economy/loans-coronavirus-small-business.html?action=click\&pgtype=Article\&state=default\&region=MAIN_CONTENT_3\&context=storylines_faq}{money
    they don't know how to use.} Many small-business owners are getting
    less than they expected or
    \href{https://www.nytimes3xbfgragh.onion/2020/06/10/business/Small-business-loans-ppp.html?action=click\&pgtype=Article\&state=default\&region=MAIN_CONTENT_3\&context=storylines_faq}{not
    hearing anything at all.}
  \end{itemize}
\item ~
  \hypertarget{what-are-my-rights-if-i-am-worried-about-going-back-to-work}{%
  \paragraph{What are my rights if I am worried about going back to
  work?}\label{what-are-my-rights-if-i-am-worried-about-going-back-to-work}}

  \begin{itemize}
  \tightlist
  \item
    Employers have to provide
    \href{https://www.osha.gov/SLTC/covid-19/standards.html}{a safe
    workplace} with policies that protect everyone equally.
    \href{https://www.nytimes3xbfgragh.onion/article/coronavirus-money-unemployment.html?action=click\&pgtype=Article\&state=default\&region=MAIN_CONTENT_3\&context=storylines_faq}{And
    if one of your co-workers tests positive for the coronavirus, the
    C.D.C.} has said that
    \href{https://www.cdc.gov/coronavirus/2019-ncov/community/guidance-business-response.html}{employers
    should tell their employees} -\/- without giving you the sick
    employee's name -\/- that they may have been exposed to the virus.
  \end{itemize}
\item ~
  \hypertarget{should-i-refinance-my-mortgage}{%
  \paragraph{Should I refinance my
  mortgage?}\label{should-i-refinance-my-mortgage}}

  \begin{itemize}
  \tightlist
  \item
    \href{https://www.nytimes3xbfgragh.onion/article/coronavirus-money-unemployment.html?action=click\&pgtype=Article\&state=default\&region=MAIN_CONTENT_3\&context=storylines_faq}{It
    could be a good idea,} because mortgage rates have
    \href{https://www.nytimes3xbfgragh.onion/2020/07/16/business/mortgage-rates-below-3-percent.html?action=click\&pgtype=Article\&state=default\&region=MAIN_CONTENT_3\&context=storylines_faq}{never
    been lower.} Refinancing requests have pushed mortgage applications
    to some of the highest levels since 2008, so be prepared to get in
    line. But defaults are also up, so if you're thinking about buying a
    home, be aware that some lenders have tightened their standards.
  \end{itemize}
\item ~
  \hypertarget{what-is-school-going-to-look-like-in-september}{%
  \paragraph{What is school going to look like in
  September?}\label{what-is-school-going-to-look-like-in-september}}

  \begin{itemize}
  \tightlist
  \item
    It is unlikely that many schools will return to a normal schedule
    this fall, requiring the grind of
    \href{https://www.nytimes3xbfgragh.onion/2020/06/05/us/coronavirus-education-lost-learning.html?action=click\&pgtype=Article\&state=default\&region=MAIN_CONTENT_3\&context=storylines_faq}{online
    learning},
    \href{https://www.nytimes3xbfgragh.onion/2020/05/29/us/coronavirus-child-care-centers.html?action=click\&pgtype=Article\&state=default\&region=MAIN_CONTENT_3\&context=storylines_faq}{makeshift
    child care} and
    \href{https://www.nytimes3xbfgragh.onion/2020/06/03/business/economy/coronavirus-working-women.html?action=click\&pgtype=Article\&state=default\&region=MAIN_CONTENT_3\&context=storylines_faq}{stunted
    workdays} to continue. California's two largest public school
    districts --- Los Angeles and San Diego --- said on July 13, that
    \href{https://www.nytimes3xbfgragh.onion/2020/07/13/us/lausd-san-diego-school-reopening.html?action=click\&pgtype=Article\&state=default\&region=MAIN_CONTENT_3\&context=storylines_faq}{instruction
    will be remote-only in the fall}, citing concerns that surging
    coronavirus infections in their areas pose too dire a risk for
    students and teachers. Together, the two districts enroll some
    825,000 students. They are the largest in the country so far to
    abandon plans for even a partial physical return to classrooms when
    they reopen in August. For other districts, the solution won't be an
    all-or-nothing approach.
    \href{https://bioethics.jhu.edu/research-and-outreach/projects/eschool-initiative/school-policy-tracker/}{Many
    systems}, including the nation's largest, New York City, are
    devising
    \href{https://www.nytimes3xbfgragh.onion/2020/06/26/us/coronavirus-schools-reopen-fall.html?action=click\&pgtype=Article\&state=default\&region=MAIN_CONTENT_3\&context=storylines_faq}{hybrid
    plans} that involve spending some days in classrooms and other days
    online. There's no national policy on this yet, so check with your
    municipal school system regularly to see what is happening in your
    community.
  \end{itemize}
\end{itemize}

Ms. Reynolds, who became mayor of the small, majority-Black suburban
city this year, said about 44 percent of residents had responded to the
census. ``Now we no longer have until October, so it's shrinking the
time, and it's kind of messing me up,'' she said. ``Now we've got to get
creative.''

Lithonia is in DeKalb County, a sprawling expanse of 760,000 people east
of Atlanta that is typical of the places that are hardest to get an
accurate population count. The local chamber of commerce
\href{https://www.dekalbchamber.org/why-dekalb/}{says it is the most
ethnically diverse county in the Southeast}, with at least 64 languages
spoken.

The stakes are tremendous: The county's chief executive, Michael
Thurmond, has estimated that with a full count, DeKalb County could
receive \$1.8 billion per year in federal funding over the next 10
years. A serious undercount could leave billions of dollars on the table
for public safety, public health, immunizations, Head Start programs,
summer jobs programs and more.

``It's just disheartening,'' Larry Johnson, a DeKalb County
commissioner, said on Tuesday. All three officials are Democrats.

The Census Bureau said in its announcement that it planned to mount ``a
robust field data collection operation'' to meet the new deadline, and
that it would be able to complete the 2020 census in a short time
``without sacrificing completeness.'' Beyond saying it would hire more
people and give its army of door-knockers awards for extra work, it has
offered few details of how it plans to meet the new goal.

But outside experts, including directors of past censuses, have said the
bureau would be forced to use statistical techniques and notoriously
inaccurate administrative records to make educated guesses about who
lived where, especially in low-income areas.

Serious inaccuracies would not only affect numbers used to reapportion
and redraw political districts, but also would skew the baseline that
will be used to allot trillions of dollars in federal grants and other
aid to the states until the next census in 2030.

Representative Carolyn B. Maloney of New York, the Democratic chairwoman
of the House Oversight Committee that has jurisdiction over the census,
said in
\href{https://oversight.house.gov/sites/democrats.oversight.house.gov/files/2020-08-04.CBM\%20to\%20Dillingham\%20re\%20Transcribed\%20Interviews.pdf}{a
letter sent Tuesday} to Steven Dillingham, the Census Bureau director,
that she would summon career Census Bureau experts to testify about the
impact of the change.

The controversy comes atop a series of political struggles over the
census.

Less than a month after Mr. Trump was sworn in, Commerce Secretary
Wilbur L. Ross Jr. began exploring the addition of a question on
citizenship to the 2020 survey, an action that both outside critics and
the bureau's own experts warned would deter noncitizens and immigrants
from responding. A court battle over the issue raged for two years
before the Supreme Court invalidated the question last summer, calling
its addition a pretext for some other unstated goal. The furor over Mr.
Trump's order not to count undocumented immigrants played out last
month.

In July, the Trump administration also added two political appointees to
Census Bureau positions just one level beneath that of the director, Mr.
Dillingham, a huge departure for the bureau.

Outside experts who deal regularly with the Census Bureau say its
operations have become progressively more opaque in recent months.
Within the last day, the bureau removed from
\href{https://2020census.gov}{its main 2020 census website} a link that
directed outsiders to a sheaf of documents detailing its plans for the
head count, although
\href{https://www.census.gov/newsroom/press-kits/2019/2020-operations.html?\#}{the
same link} remains on a less prominent page
\href{https://2020census.gov/en/news-events/operational-adjustments-covid-19.html}{explaining
the bureau's response to the pandemic.}

In the end, the success of the census will depend on millions of
connections, made or missed. Less time will not help make them.

Roger Alexander, 32, a resident of Marietta, Ga., was in downtown
Lithonia on Tuesday finishing up work on a hip-hop-themed video
encouraging people to register and vote. Mr. Alexander said he had heard
that the decennial count was being taken: ``I feel like I've seen
something about the census trying to get it together,'' he said.

Mr. Alexander had not filled out a form yet but said that he would
eventually do so --- ``if it is something convenient for me to do.''

Michael Wines reported from Washington, and Richard Fausset from
Lithonia, Ga.

Advertisement

\protect\hyperlink{after-bottom}{Continue reading the main story}

\hypertarget{site-index}{%
\subsection{Site Index}\label{site-index}}

\hypertarget{site-information-navigation}{%
\subsection{Site Information
Navigation}\label{site-information-navigation}}

\begin{itemize}
\tightlist
\item
  \href{https://help.nytimes3xbfgragh.onion/hc/en-us/articles/115014792127-Copyright-notice}{©~2020~The
  New York Times Company}
\end{itemize}

\begin{itemize}
\tightlist
\item
  \href{https://www.nytco.com/}{NYTCo}
\item
  \href{https://help.nytimes3xbfgragh.onion/hc/en-us/articles/115015385887-Contact-Us}{Contact
  Us}
\item
  \href{https://www.nytco.com/careers/}{Work with us}
\item
  \href{https://nytmediakit.com/}{Advertise}
\item
  \href{http://www.tbrandstudio.com/}{T Brand Studio}
\item
  \href{https://www.nytimes3xbfgragh.onion/privacy/cookie-policy\#how-do-i-manage-trackers}{Your
  Ad Choices}
\item
  \href{https://www.nytimes3xbfgragh.onion/privacy}{Privacy}
\item
  \href{https://help.nytimes3xbfgragh.onion/hc/en-us/articles/115014893428-Terms-of-service}{Terms
  of Service}
\item
  \href{https://help.nytimes3xbfgragh.onion/hc/en-us/articles/115014893968-Terms-of-sale}{Terms
  of Sale}
\item
  \href{https://spiderbites.nytimes3xbfgragh.onion}{Site Map}
\item
  \href{https://help.nytimes3xbfgragh.onion/hc/en-us}{Help}
\item
  \href{https://www.nytimes3xbfgragh.onion/subscription?campaignId=37WXW}{Subscriptions}
\end{itemize}
