Sections

SEARCH

\protect\hyperlink{site-content}{Skip to
content}\protect\hyperlink{site-index}{Skip to site index}

\href{https://www.nytimes3xbfgragh.onion/section/politics}{Politics}

\href{https://myaccount.nytimes3xbfgragh.onion/auth/login?response_type=cookie\&client_id=vi}{}

\href{https://www.nytimes3xbfgragh.onion/section/todayspaper}{Today's
Paper}

\href{/section/politics}{Politics}\textbar{}Trading Concessions on
Recovery Plan, Negotiators Set Week's End Deadline for a Deal

\url{https://nyti.ms/2DjD2Sc}

\begin{itemize}
\item
\item
\item
\item
\item
\end{itemize}

Advertisement

\protect\hyperlink{after-top}{Continue reading the main story}

Supported by

\protect\hyperlink{after-sponsor}{Continue reading the main story}

\hypertarget{trading-concessions-on-recovery-plan-negotiators-set-weeks-end-deadline-for-a-deal}{%
\section{Trading Concessions on Recovery Plan, Negotiators Set Week's
End Deadline for a
Deal}\label{trading-concessions-on-recovery-plan-negotiators-set-weeks-end-deadline-for-a-deal}}

White House officials and top Democrats reported some movement in their
talks on a pandemic relief package as the talks reached a crisis point
on Capitol Hill.

\includegraphics{https://static01.graylady3jvrrxbe.onion/images/2020/08/04/us/politics/04dc-virus-cong/merlin_175309938_0696afcb-d343-4aea-b5c1-8fd0ba2769de-articleLarge.jpg?quality=75\&auto=webp\&disable=upscale}

\href{https://www.nytimes3xbfgragh.onion/by/emily-cochrane}{\includegraphics{https://static01.graylady3jvrrxbe.onion/images/2018/11/28/multimedia/author-emily-cochrane/author-emily-cochrane-thumbLarge-v3.png}}\href{https://www.nytimes3xbfgragh.onion/by/nicholas-fandos}{\includegraphics{https://static01.graylady3jvrrxbe.onion/images/2018/11/06/multimedia/author-nicholas-fandos/author-nicholas-fandos-thumbLarge-v2.png}}

By \href{https://www.nytimes3xbfgragh.onion/by/emily-cochrane}{Emily
Cochrane} and
\href{https://www.nytimes3xbfgragh.onion/by/nicholas-fandos}{Nicholas
Fandos}

\begin{itemize}
\item
  Aug. 4, 2020
\item
  \begin{itemize}
  \item
  \item
  \item
  \item
  \item
  \end{itemize}
\end{itemize}

WASHINGTON --- The White House and congressional Democrats on Tuesday
crept toward an elusive compromise on an economic recovery plan,
agreeing to an end-of-the-week deadline to seal a deal that would
restore expired jobless aid for tens of millions of Americans even as
they remained far apart on the contours of the package.

In a sign that Republicans may be willing to offer a major concession
amid deep divisions in their ranks about supporting any additional
federal relief, Senator Mitch McConnell, Republican of Kentucky and the
majority leader, signaled that he might be willing to accept the
extension of \$600-per-week unemployment payments --- something many in
his party oppose --- if it would yield a compromise.

Later, after a meeting with Mark Meadows, the White House chief of
staff, and Steven Mnuchin, the Treasury secretary, top Democrats
indicated there had been progress toward a potential deal.

``They made some concessions, which we appreciated; we made some
concessions, which they appreciated,'' Senator Chuck Schumer of New
York, the Democratic leader, said after the 90-minute meeting, which
Speaker Nancy Pelosi of California hosted in her Capitol Hill suite.
``We're still far away on a lot of the important issues, but we're
continuing to go at it.''

The apparent movement came as talks on the latest round of federal
\href{https://www.nytimes3xbfgragh.onion/interactive/2020/us/coronavirus-us-cases.html}{pandemic}
relief were reaching a crisis point on Capitol Hill, with the
unemployment payments already expired, the looming lapse at week's end
of a popular small-business loan program that has kept many businesses
afloat, and the Senate set to start its summer recess on Friday.

A number of significant policy differences remain, and officials
involved in the negotiations cautioned that there was little time to
meet their agreed-upon timeline for a final deal, which would pave the
way for votes on the package next week. Even if they were able to do so
and President Trump signed a recovery bill by the end of next week, it
could take weeks for tens of millions of workers to begin receiving the
supplemental benefits again.

It was unclear what concessions either side made in the meeting, though
Mr. Mnuchin said he and Mr. Meadows had proposed an agreement to revive
and extend a federal eviction moratorium through the end of the year.
Republicans omitted the measure from their \$1 trillion relief package,
but Mr. Trump has said it was a top priority.

``We're going to work around the clock the next few days to see if we
can bridge the issues,'' Mr. Mnuchin said. ``Some issues we've been able
to agree on, some significant issues are still open.''

The terms of the eviction moratorium remain unresolved, however, because
the offer did not include additional assistance for homeowners and
renters of the kind included in the Democrats' \$3 trillion plan,
according to two people familiar with the negotiations, who asked to
speak on the condition of anonymity to disclose details of a private
meeting.

``We know that the devil is in the details --- so are the angels,'' Ms.
Pelosi said after the session with the administration officials and Mr.
Schumer.

The four plan to reconvene on Wednesday to continue talks.

\includegraphics{https://static01.graylady3jvrrxbe.onion/images/2020/08/04/us/politics/04dc-virus-cong2/merlin_175311039_99b078a1-955e-45e2-b42f-a7aadaa2a09e-articleLarge.jpg?quality=75\&auto=webp\&disable=upscale}

At the White House, Mr. Trump continued to dangle the possibility that
he could circumvent Congress and take executive action to try to halt
evictions nationwide and to suspend the payroll tax, an idea that
Republicans dropped from their proposal amid wide opposition. Mr.
Meadows said that remained a possibility, though he added that he hoped
the negotiations would produce a deal that would make such action
unnecessary.

It is far from clear that the president has the power to make either
move unilaterally, but his deputies appeared to be using the possibility
as a negotiating tactic with Democrats and to simply get around the
objections within Mr. Trump's own party on the payroll tax issue.

``We want to take care of the eviction problem,'' the president said
Tuesday during a briefing at the White House. ``People are being evicted
unfairly. It's not their fault. It's China's fault.''

He noted that there was progress being made on Capitol Hill, but
continued to attack Democrats even as his top advisers were seeking a
compromise with them.

At the Capitol, Mr. McConnell conceded that many of his fellow
Republicans, who are pressing to cut the jobless aid, ``may not vote for
a package'' if it includes Democrats' demand to extend the \$600 weekly
benefit through year's end. But, in a nod to the economic and political
stakes of the negotiations, which are unfolding three months before a
general election, he said Congress had no choice but to push through an
aid measure.

``The American people, in the end, need help, and wherever this thing
settles between the president of the United States and his team that has
to sign it into law, and the Democrat not-insignificant minority in the
Senate and majority in the House is something I am prepared to support
--- even if I have some problems with certain parts of it,'' Mr.
McConnell said.

The halting progress came as Senate Republicans, who have not yet acted
on a relief package and have not been a part of the negotiations with
Democrats, appeared to be losing patience with the process as the clock
ticked down to their scheduled summer recess.

``How do you think it looks for us to be back home when this is
unresolved?'' Senator John Cornyn, Republican of Texas, who is facing
re-election, told reporters. ``This is the most important thing we need
to be doing.''

Senator Lisa Murkowski, Republican of Alaska, said she had begun to
reach out to individual Democrats to see if rank-and-file lawmakers
could help facilitate a breakthrough.

``As I look to where we are, I feel the same discouragement, despair ---
despair that we're here on the fourth of August,'' she said. ``We don't
have answers for people right now. So, we're going to keep working.''

Senator Marco Rubio, Republican of Florida, noted that even as the
chamber had failed to find a way to act on major legislation like the
recovery bill, Mr. McConnell had kept up a steady pace of judicial
confirmations.

``We're going to have to move toward each other to get something done,''
Mr. Rubio said.

``At some point,'' he added, chuckling, ``we run out of judges.''

Jim Tankersley contributed reporting.

Advertisement

\protect\hyperlink{after-bottom}{Continue reading the main story}

\hypertarget{site-index}{%
\subsection{Site Index}\label{site-index}}

\hypertarget{site-information-navigation}{%
\subsection{Site Information
Navigation}\label{site-information-navigation}}

\begin{itemize}
\tightlist
\item
  \href{https://help.nytimes3xbfgragh.onion/hc/en-us/articles/115014792127-Copyright-notice}{©~2020~The
  New York Times Company}
\end{itemize}

\begin{itemize}
\tightlist
\item
  \href{https://www.nytco.com/}{NYTCo}
\item
  \href{https://help.nytimes3xbfgragh.onion/hc/en-us/articles/115015385887-Contact-Us}{Contact
  Us}
\item
  \href{https://www.nytco.com/careers/}{Work with us}
\item
  \href{https://nytmediakit.com/}{Advertise}
\item
  \href{http://www.tbrandstudio.com/}{T Brand Studio}
\item
  \href{https://www.nytimes3xbfgragh.onion/privacy/cookie-policy\#how-do-i-manage-trackers}{Your
  Ad Choices}
\item
  \href{https://www.nytimes3xbfgragh.onion/privacy}{Privacy}
\item
  \href{https://help.nytimes3xbfgragh.onion/hc/en-us/articles/115014893428-Terms-of-service}{Terms
  of Service}
\item
  \href{https://help.nytimes3xbfgragh.onion/hc/en-us/articles/115014893968-Terms-of-sale}{Terms
  of Sale}
\item
  \href{https://spiderbites.nytimes3xbfgragh.onion}{Site Map}
\item
  \href{https://help.nytimes3xbfgragh.onion/hc/en-us}{Help}
\item
  \href{https://www.nytimes3xbfgragh.onion/subscription?campaignId=37WXW}{Subscriptions}
\end{itemize}
