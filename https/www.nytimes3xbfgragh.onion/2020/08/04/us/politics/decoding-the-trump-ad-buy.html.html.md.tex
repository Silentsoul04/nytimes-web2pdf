Sections

SEARCH

\protect\hyperlink{site-content}{Skip to
content}\protect\hyperlink{site-index}{Skip to site index}

\href{https://www.nytimes3xbfgragh.onion/section/politics}{Politics}

\href{https://myaccount.nytimes3xbfgragh.onion/auth/login?response_type=cookie\&client_id=vi}{}

\href{https://www.nytimes3xbfgragh.onion/section/todayspaper}{Today's
Paper}

\href{/section/politics}{Politics}\textbar{}Decoding the Trump Ad Buy

\url{https://nyti.ms/3guJdBi}

\begin{itemize}
\item
\item
\item
\item
\item
\end{itemize}

\begin{itemize}
\item
  \href{https://www.nytimes3xbfgragh.onion/interactive/2020/08/04/us/elections/results-arizona-kansas-michigan-missouri-primaries.html?action=click\&pgtype=Article\&state=default\&region=TOP_BANNER\&context=storylines_menu}{Latest
  Results}
\item
  \href{https://www.nytimes3xbfgragh.onion/article/biden-vice-president-2020.html?action=click\&pgtype=Article\&state=default\&region=TOP_BANNER\&context=storylines_menu}{Biden's
  V.P. Search}
\item
  \href{https://www.nytimes3xbfgragh.onion/interactive/2020/07/24/us/politics/trump-biden-campaign-donors.html?action=click\&pgtype=Article\&state=default\&region=TOP_BANNER\&context=storylines_menu}{Map
  of Donations}
\item
  \href{https://www.nytimes3xbfgragh.onion/interactive/2020/us/elections/delegate-count-primary-results.html?action=click\&pgtype=Article\&state=default\&region=TOP_BANNER\&context=storylines_menu}{Delegate
  Count}
\item
  \href{https://www.nytimes3xbfgragh.onion/interactive/2019/us/politics/2020-presidential-candidates.html?action=click\&pgtype=Article\&state=default\&region=TOP_BANNER\&context=storylines_menu}{The
  Candidates}
\item
  \href{https://www.nytimes3xbfgragh.onion/newsletters/politics?action=click\&pgtype=Article\&state=default\&region=TOP_BANNER\&context=storylines_menu}{Politics
  Newsletter}
\end{itemize}

Advertisement

\protect\hyperlink{after-top}{Continue reading the main story}

Supported by

\protect\hyperlink{after-sponsor}{Continue reading the main story}

On Politics Media Watch

\hypertarget{decoding-the-trump-ad-buy}{%
\section{Decoding the Trump Ad Buy}\label{decoding-the-trump-ad-buy}}

After going dark on the TV airwaves last week, the Trump campaign has
returned, targeting early voting states.

\href{https://www.nytimes3xbfgragh.onion/by/nick-corasaniti}{\includegraphics{https://static01.graylady3jvrrxbe.onion/images/2018/06/13/multimedia/author-nick-corasaniti/author-nick-corasaniti-thumbLarge-v2.png}}

By \href{https://www.nytimes3xbfgragh.onion/by/nick-corasaniti}{Nick
Corasaniti}

\begin{itemize}
\item
  Aug. 4, 2020
\item
  \begin{itemize}
  \item
  \item
  \item
  \item
  \item
  \end{itemize}
\end{itemize}

\emph{Hi. Welcome to}
\href{https://nl.nytimes3xbfgragh.onion/f/a/zarSo8aO-mBP32V6vtzLuw~~/AAAAAQA~/RgRfpIqEP0TSaHR0cHM6Ly93d3cubnl0aW1lcy5jb20vc3BvdGxpZ2h0L29uLXBvbGl0aWNzP3RlPTEmbmw9b24tcG9saXRpY3Mgd2l0aCBsaXNhIGxlcmVyJmVtYz1lZGl0X2NuXzIwMTkxMTA1P2NhbXBhaWduX2lkPTU2Jmluc3RhbmNlX2lkPTEzNjQxJnNlZ21lbnRfaWQ9MTg1NDcmdXNlcl9pZD1iNTA4ZWUwOTdhYWY0NWE5N2IxYmEwZTc5Y2QwNjg0ZSZyZWdpX2lkPTYzMjU0MTc0VwNueXRCCgAehAXCXTVeoxVSH25pY2hvbGFzLmNvcmFzYW5pdGlAbnl0aW1lcy5jb21YBAAAAAA~}{\emph{On
Politics}}\emph{, your guide to the day in national politics. I'm Nick
Corasaniti, your host on Tuesdays for our coverage of all things media
and messaging.}

\href{https://www.nytimes3xbfgragh.onion/newsletters/politics?module=inline}{\emph{Sign
up here}} \emph{to get On Politics in your inbox every weekday.}

\includegraphics{https://static01.graylady3jvrrxbe.onion/images/2020/08/04/us/politics/04onpolitics-pm/merlin_175276041_e18fad70-18ca-41e9-8430-c0e6a0591a59-articleLarge.jpg?quality=75\&auto=webp\&disable=upscale}

For six days last week, there was not a single Trump campaign ad on any
television across the country. While it was an unusual time to go
completely dark on the TV airwaves, with less than 100 days left in the
election, the Trump team said it was part of a necessary review by its
new campaign manager,
\href{https://www.nytimes3xbfgragh.onion/article/bill-stepien.html}{Bill
Stepien}, who took the reins in mid-July.

What returned on Monday was a more streamlined ad campaign in four
battleground states: North Carolina, Georgia, Florida and Arizona.

Why those four? Early voting.

Though campaigns have always made an extra effort to advertise in states
that start voting earlier in the calendar, the pace of early voting this
year is expected to accelerate exponentially amid the coronavirus
pandemic. As a safer method of voting compared with showing up at
crowded Election Day polling locations, early voting --- both in-person
and through mail-in ballots --- has set records in nearly every primary
election since March.

For the general election, North Carolina, Georgia, Florida and Arizona
are some of the earliest states to begin voting, with Georgia and
Arizona starting as soon as the first week of October. Because many are
expected to cast early ballots, what was once a low drum beat of August
political advertising is likely to grow into a more cacophonous cascade
of ads in early states.

With its new advertising strategy, the Trump campaign is returning to an
older message: fighting the threat of ``socialism.'' Gone are the
scattershot attacks on Joseph R. Biden Jr.'s relationship with China, or
his age, or his position on defunding the police. Returning in one of
President Trump's new ads are the familiar faces of Republican boogeymen
and women like Senator Bernie Sanders of Vermont and Representative
Alexandria Ocasio-Cortez of New York, as the Trump campaign attempts to
paint Mr. Biden as beholden to the far-left of his party.

\hypertarget{latest-updates-2020-election}{%
\section{\texorpdfstring{\href{https://www.nytimes3xbfgragh.onion/2020/08/04/us/elections/primary-election-michigan-arizona-kansas.html?action=click\&pgtype=Article\&state=default\&region=MAIN_CONTENT_1\&context=storylines_live_updates}{Latest
Updates: 2020
Election}}{Latest Updates: 2020 Election}}\label{latest-updates-2020-election}}

Updated 2020-08-05T03:23:56.561Z

\begin{itemize}
\tightlist
\item
  \href{https://www.nytimes3xbfgragh.onion/2020/08/04/us/elections/primary-election-michigan-arizona-kansas.html?action=click\&pgtype=Article\&state=default\&region=MAIN_CONTENT_1\&context=storylines_live_updates\#link-3924dd44}{Two
  G.O.P. Senate primaries offer --- what else? --- a test of loyalty to
  Trump.}
\item
  \href{https://www.nytimes3xbfgragh.onion/2020/08/04/us/elections/primary-election-michigan-arizona-kansas.html?action=click\&pgtype=Article\&state=default\&region=MAIN_CONTENT_1\&context=storylines_live_updates\#link-62a8e06b}{The
  military-style uniforms of federal agents who responded to the unrest
  in Portland will be replaced.}
\item
  \href{https://www.nytimes3xbfgragh.onion/2020/08/04/us/elections/primary-election-michigan-arizona-kansas.html?action=click\&pgtype=Article\&state=default\&region=MAIN_CONTENT_1\&context=storylines_live_updates\#link-32b39e33}{President
  Trump is suddenly a big supporter of mail-in voting --- in Florida.}
\end{itemize}

\href{https://www.nytimes3xbfgragh.onion/2020/08/04/us/elections/primary-election-michigan-arizona-kansas.html?action=click\&pgtype=Article\&state=default\&region=MAIN_CONTENT_1\&context=storylines_live_updates}{See
more updates}

If that feels familiar, it is: The Trump campaign ran an extensive ad
campaign during the impeachment process, often
\href{https://www.nytimes3xbfgragh.onion/2019/10/28/us/politics/impeachment-ads-facebook.html}{casting
it as a far-left conspiracy}, with television and digital ads full of
imagery of prominent progressive figures like Mr. Sanders, Ms.
Ocasio-Cortez and Representative Ilhan Omar of Minnesota.

The new ad, which also features familiar hard-line immigration
positions, appears to be a recognition that the previous attempts to
\href{https://www.nytimes3xbfgragh.onion/2020/07/10/us/biden-trump.html}{define
Mr. Biden had largely failed}. The Trump campaign had spent more than
\$30 million in attack ads that mischaracterized Mr. Biden's position on
defunding the police, but the former vice president continued to climb
in key battleground state polling.

Of course, just as notable are the early states the Trump campaign is
not targeting in this initial return to the airwaves.

The campaign continues to stay off the air in Michigan, a state Mr.
Trump won by less than 11,000 votes against Hillary Clinton in 2016 but
where he has trailed Mr. Biden in most statewide polls. Michigan votes
in mid-September.

Trump ads are also off the air in Pennsylvania, a state he won by less
than 45,000 votes in 2016. But there, Mr. Biden is also leading,
according to recent polling. Though Pennsylvania does not have expansive
in-person early voting, a new law allows voters in some counties to
\href{https://www.inquirer.com/politics/election/pennsylvania-early-voting-absentee-ballots-20200113.html}{request
a mail-in ballot as early as mid-September}.

And while the Trump campaign has publicly claimed Minnesota as
competitive --- a state that Mr. Trump lost by just 44,000 votes in 2016
but hasn't voted for a Republican presidential candidate since 1972 ---
it is not currently going back on air in the state, which also starts
voting in mid-September.

\begin{center}\rule{0.5\linewidth}{\linethickness}\end{center}

\textbf{Drop us a line!}

Image

\emph{We want to hear from our readers. Have a question? We'll try to
answer it. Have a comment? We're all ears. Email us at}
\href{mailto:onpolitics@NYTimes.com}{\emph{onpolitics@NYTimes.com}}\emph{.}

\begin{center}\rule{0.5\linewidth}{\linethickness}\end{center}

\hypertarget{ad-of-the-week-meddling-in-kansas}{%
\subsection{Ad of the week: Meddling in
Kansas?}\label{ad-of-the-week-meddling-in-kansas}}

The \href{https://twitter.com/greggiroux/status/1288089119529283585}{ads
of a mysterious new super PAC called Sunflower State} first appeared on
television screens in Kansas in mid-July, with a message assailing
Representative Roger Marshall, one of the Republican candidates for
Senate in Kansas, as it offered backhanded compliments to one of his
opponents in the primary, Kris Kobach, the former Kansas secretary of
state.

Over the past three weeks, Sunflower State has run \$2.5 million worth
of television ads attacking Mr. Marshall, according to Advertising
Analytics, an ad tracking firm. Little is known about the group, though
what is publicly available links it to Democrats: It uses the same
advertising buyer as a pro-Biden super PAC, and it shares the same bank
as Senate Majority PAC, another top Democratic group.

\textbf{The message:} After Mr. Trump popularized derision of Washington
insiderdom as ``the swamp,'' the phrase quickly became a common tagline
for conservatives who wanted to attack mainstream Democrats. In this ad,
the phrase is weaponized against Mr. Marshall, a two-term congressman,
along with other clichéd takedowns like ``special-interest lover.''

For Mr. Kobach, the praise is short but effusive: ``the pro-Trump
conservative leader.''

\textbf{The takeaway:} Democratic strategists have argued that if Mr.
Kobach, who led Mr. Trump's ill-fated election integrity commission and
lost the 2018 governor's race to a Democrat, wins the primary today, the
normally deeply red Kansas Senate seat could flip.

So while the Democratic-linked Sunflower State won't run a full-on
pro-Kobach ad, it has no issue running anti-Marshall ads, a politician
it clearly disagrees with. It's a tactic known as meddling, when a group
linked to one party runs ads in the other party's primary in an attempt
to help a preferred candidate win. Of course, it could backfire --- Mr.
Trump won Kansas by more than 20 percentage points in 2016 --- and the
\$2.5 million the super PAC spent to help elect Mr. Kobach could become
a six-year black eye for Democrats.

\begin{center}\rule{0.5\linewidth}{\linethickness}\end{center}

\emph{Thanks for reading. On Politics is your guide to the political
news cycle, delivering clarity from the chaos.}

\emph{On Politics is also available as a newsletter.}
\href{https://www.nytimes3xbfgragh.onion/newsletters/politics}{\emph{Sign
up here}} \emph{to get it delivered to your inbox.}

\emph{Is there anything you think we're missing? Anything you want to
see more of? We'd love to hear from you. Email us at}
\href{mailto:onpolitics@NYTimes.com}{\emph{onpolitics@NYTimes.com}}\emph{.}

\hypertarget{our-2020-election-guide}{%
\section{Our 2020 Election Guide}\label{our-2020-election-guide}}

Updated Aug. 4, 2020

\begin{itemize}
\item
  \begin{center}\rule{0.5\linewidth}{\linethickness}\end{center}

  \hypertarget{the-latest}{%
  \subsection{The Latest}\label{the-latest}}

  \begin{itemize}
  \tightlist
  \item
    Kris Kobach, a polarizing figure in Kansas politics,
    \href{https://www.nytimes3xbfgragh.onion/2020/08/04/us/politics/kobach-tlaib.html?action=click\&pgtype=Article\&state=default\&region=BELOW_MAIN_CONTENT\&context=storylines_guide}{lost
    the Senate primary there}, relieving G.O.P. officials who feared he
    could jeopardize a safe seat.
  \end{itemize}
\item
  \begin{center}\rule{0.5\linewidth}{\linethickness}\end{center}

  \hypertarget{bidens-vp-search}{%
  \subsection{Biden's V.P. Search}\label{bidens-vp-search}}

  \begin{itemize}
  \tightlist
  \item
    \href{https://www.nytimes3xbfgragh.onion/article/biden-vice-president-2020.html?action=click\&pgtype=Article\&state=default\&region=BELOW_MAIN_CONTENT\&context=storylines_guide}{Here
    are 13 women} who have been under consideration to be Joe Biden's
    running mate, and why each might be chosen --- and might not be.
  \end{itemize}
\item
  \begin{center}\rule{0.5\linewidth}{\linethickness}\end{center}

  \hypertarget{keep-up-with-our-coverage}{%
  \subsection{Keep Up With Our
  Coverage}\label{keep-up-with-our-coverage}}

  \begin{itemize}
  \tightlist
  \item
    Get an
    \href{https://www.nytimes3xbfgragh.onion/newsletters/politics?action=click\&pgtype=Article\&state=default\&region=BELOW_MAIN_CONTENT\&context=storylines_guide}{email}
    recapping the day's news
  \end{itemize}

  \begin{itemize}
  \tightlist
  \item
    Download our mobile app on
    \href{https://apps.apple.com/us/app/nytimes/id284862083?ls=1\&mat_click_id=5c79ae7455014fd1bd66b5610c05b8f2-20191112-16948\&referrer=mat_click_id\%3D5c79ae7455014fd1bd66b5610c05b8f2-20191112-16948\%26link_click_id\%3D722930677036718082}{iOS}
    and
    \href{http://a.localytics.com/android?id=com.nytimes.android\&referrer=utm_source\%3Dother_nyt_mobile_web\%26utm_medium\%3DWeb\%2520page\%26utm_term\%3DGeneral\%2520Mobile\%2520Page\%26utm_campaign\%3DNYT\%2520Mobile\%2520General\%2520Page}{Android}
    and turn on Breaking News and Politics alerts
  \end{itemize}
\end{itemize}

Advertisement

\protect\hyperlink{after-bottom}{Continue reading the main story}

\hypertarget{site-index}{%
\subsection{Site Index}\label{site-index}}

\hypertarget{site-information-navigation}{%
\subsection{Site Information
Navigation}\label{site-information-navigation}}

\begin{itemize}
\tightlist
\item
  \href{https://help.nytimes3xbfgragh.onion/hc/en-us/articles/115014792127-Copyright-notice}{©~2020~The
  New York Times Company}
\end{itemize}

\begin{itemize}
\tightlist
\item
  \href{https://www.nytco.com/}{NYTCo}
\item
  \href{https://help.nytimes3xbfgragh.onion/hc/en-us/articles/115015385887-Contact-Us}{Contact
  Us}
\item
  \href{https://www.nytco.com/careers/}{Work with us}
\item
  \href{https://nytmediakit.com/}{Advertise}
\item
  \href{http://www.tbrandstudio.com/}{T Brand Studio}
\item
  \href{https://www.nytimes3xbfgragh.onion/privacy/cookie-policy\#how-do-i-manage-trackers}{Your
  Ad Choices}
\item
  \href{https://www.nytimes3xbfgragh.onion/privacy}{Privacy}
\item
  \href{https://help.nytimes3xbfgragh.onion/hc/en-us/articles/115014893428-Terms-of-service}{Terms
  of Service}
\item
  \href{https://help.nytimes3xbfgragh.onion/hc/en-us/articles/115014893968-Terms-of-sale}{Terms
  of Sale}
\item
  \href{https://spiderbites.nytimes3xbfgragh.onion}{Site Map}
\item
  \href{https://help.nytimes3xbfgragh.onion/hc/en-us}{Help}
\item
  \href{https://www.nytimes3xbfgragh.onion/subscription?campaignId=37WXW}{Subscriptions}
\end{itemize}
