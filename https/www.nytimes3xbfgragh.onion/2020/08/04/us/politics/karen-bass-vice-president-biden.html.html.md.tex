Sections

SEARCH

\protect\hyperlink{site-content}{Skip to
content}\protect\hyperlink{site-index}{Skip to site index}

\href{https://www.nytimes3xbfgragh.onion/section/politics}{Politics}

\href{https://myaccount.nytimes3xbfgragh.onion/auth/login?response_type=cookie\&client_id=vi}{}

\href{https://www.nytimes3xbfgragh.onion/section/todayspaper}{Today's
Paper}

\href{/section/politics}{Politics}\textbar{}From Outsider to Insider:
Karen Bass's Unexpected Journey to Power

\href{https://nyti.ms/3i6QveR}{https://nyti.ms/3i6QveR}

\begin{itemize}
\item
\item
\item
\item
\item
\item
\end{itemize}

\begin{itemize}
\item
  \href{https://www.nytimes3xbfgragh.onion/interactive/2020/08/04/us/elections/results-arizona-kansas-michigan-missouri-primaries.html?action=click\&pgtype=Article\&state=default\&region=TOP_BANNER\&context=storylines_menu}{Latest
  Results}
\item
  \href{https://www.nytimes3xbfgragh.onion/article/biden-vice-president-2020.html?action=click\&pgtype=Article\&state=default\&region=TOP_BANNER\&context=storylines_menu}{Biden's
  V.P. Search}
\item
  \href{https://www.nytimes3xbfgragh.onion/interactive/2020/07/24/us/politics/trump-biden-campaign-donors.html?action=click\&pgtype=Article\&state=default\&region=TOP_BANNER\&context=storylines_menu}{Map
  of Donations}
\item
  \href{https://www.nytimes3xbfgragh.onion/interactive/2020/us/elections/delegate-count-primary-results.html?action=click\&pgtype=Article\&state=default\&region=TOP_BANNER\&context=storylines_menu}{Delegate
  Count}
\item
  \href{https://www.nytimes3xbfgragh.onion/interactive/2019/us/politics/2020-presidential-candidates.html?action=click\&pgtype=Article\&state=default\&region=TOP_BANNER\&context=storylines_menu}{The
  Candidates}
\item
  \href{https://www.nytimes3xbfgragh.onion/newsletters/politics?action=click\&pgtype=Article\&state=default\&region=TOP_BANNER\&context=storylines_menu}{Politics
  Newsletter}
\end{itemize}

Advertisement

\protect\hyperlink{after-top}{Continue reading the main story}

Supported by

\protect\hyperlink{after-sponsor}{Continue reading the main story}

\hypertarget{from-outsider-to-insider-karen-basss-unexpected-journey-to-power}{%
\section{From Outsider to Insider: Karen Bass's Unexpected Journey to
Power}\label{from-outsider-to-insider-karen-basss-unexpected-journey-to-power}}

At every step in her political career, the California congresswoman had
to be coaxed to run for a higher office. Now she's a top contender to be
Joe Biden's running mate.

\includegraphics{https://static01.graylady3jvrrxbe.onion/images/2020/08/03/us/politics/03bass1/merlin_175261452_1fcbd46a-2ce4-48e4-a5ec-fa29f351163d-articleLarge.jpg?quality=75\&auto=webp\&disable=upscale}

\href{https://www.nytimes3xbfgragh.onion/by/adam-nagourney}{\includegraphics{https://static01.graylady3jvrrxbe.onion/images/2018/02/20/multimedia/author-adam-nagourney/author-adam-nagourney-thumbLarge-v3.png}}\href{https://www.nytimes3xbfgragh.onion/by/jennifer-medina}{\includegraphics{https://static01.graylady3jvrrxbe.onion/images/2018/02/20/multimedia/author-jennifer-medina/author-jennifer-medina-thumbLarge-v3.png}}

By \href{https://www.nytimes3xbfgragh.onion/by/adam-nagourney}{Adam
Nagourney} and
\href{https://www.nytimes3xbfgragh.onion/by/jennifer-medina}{Jennifer
Medina}

\begin{itemize}
\item
  Aug. 4, 2020
\item
  \begin{itemize}
  \item
  \item
  \item
  \item
  \item
  \item
  \end{itemize}
\end{itemize}

LOS ANGELES --- In 2008, as California confronted its most severe fiscal
crisis since the Great Depression, the center of power in the state
capital was a tent. It was set up in the courtyard outside the office of
Gov. Arnold Schwarzenegger, an invitation-only retreat with folding
chairs, a fake grass floor and ashtrays. Mr. Schwarzenegger, his aides
and California's most influential players --- mostly white men --- went
there to smoke expensive cigars and talk politics and legislative deals.

But not Karen Bass.

Newly elected as the speaker of the State Assembly, Ms. Bass felt apart
from that club the moment she stepped through the canvas flaps at the
invitation of Mr. Schwarzenegger, a Republican.

```I guess you probably don't want to be doing business here,''' the
governor told her, as Ms. Bass recalled in a recent interview. ``He
didn't know what to do. And that was right. I didn't want to be in the
tent.''

Here she was, a liberal Democrat and the first Black woman to lead a
statehouse in the nation's history, one of the three most powerful
elected leaders of California in a moment of fiscal peril, and a wholly
different kind of player navigating among men in a thick haze of cigar
smoke.

``They didn't know what to make of me,'' she said.

\includegraphics{https://static01.graylady3jvrrxbe.onion/images/2020/08/03/us/politics/03bass2/03bass2-articleLarge.jpg?quality=75\&auto=webp\&disable=upscale}

Today Ms. Bass, a 66-year-old congresswoman and head of the
Congressional Black Caucus, is among the leading candidates being
considered for vice president by
\href{https://www.nytimes3xbfgragh.onion/interactive/2020/us/elections/joe-biden.html}{Joseph
R. Biden Jr.} But in some ways she is still an outsider, making her way
in an overwhelmingly male political culture epitomized 12 years ago by
Mr. Schwarzenegger, the shoot-'em-up action movie star, and today by the
power circles in Washington, where the presidency and vice presidency
remain all-male redoubts. She is barely known to many Americans, even to
many leading Democrats, who --- like the men in that tent in Sacramento
--- are now trying to figure out what to make of her.

And of all the contenders under consideration, none offer quite the
contrast with Mr. Biden --- in the story of her life and the story of
her politics --- as the Democratic congresswoman from California.

Ms. Bass grew up in a Black middle-class neighborhood in central Los
Angeles; Mr. Biden spent his first years in predominantly white,
middle-class Scranton, Pa. Mr. Biden was elected to his first public
office in 1970, when he was 27. Ms. Bass was 51 when she was first
elected to the Assembly.

Ms. Bass was liberal in a state known as one of the most liberal in the
nation. Mr. Biden is the face of the moderate wing of the Democratic
Party. One of Ms. Bass's earliest causes, when she was a community
organizer confronting crime, drugs and poverty, was opposing
\href{https://www.nytimes3xbfgragh.onion/2019/06/25/us/joe-biden-crime-laws.html}{the
very criminal justice law that Mr. Biden had championed}, with its
mandatory jail terms.

At every step in her political career, Ms. Bass had to be coaxed to run
for a higher office. Mr. Biden has been running for president for nearly
half a century.

\emph{{[}Sign up}
\href{https://www.nytimes3xbfgragh.onion/newsletters/california-today}{\emph{for
California Today}}\emph{, our daily newsletter from the Golden
State.{]}}

In many ways, Ms. Bass could help Mr. Biden against
\href{https://www.nytimes3xbfgragh.onion/interactive/2020/us/elections/donald-trump.html}{President
Trump}. She would make history as the first Black woman to become a
major party's candidate for vice president. She talks more about
conciliation than confrontation and does not have as many detractors as
some of her rivals, including Senator Elizabeth Warren of Massachusetts
and Senator Kamala Harris of California.

Image

Ms. Bass with other House Democrats on Capitol Hill in June, before the
vote for the George Floyd Justice in Policing Act.~Credit...Anna
Moneymaker for The New York Times

Ms. Bass's advocacy of liberal positions on criminal justice, poverty,
housing, foster care and welfare could provide a Biden-Bass ticket a
lift with progressive and younger voters. She would arguably be the most
liberal politician chosen for vice president since Jimmy Carter tapped
Walter F. Mondale as his running mate in 1976.

But her record of espousing progressive causes, lack of experience as a
national campaigner and still largely unexplored record in public life
also offer targets for Mr. Trump.

\hypertarget{latest-updates-2020-election}{%
\section{\texorpdfstring{\href{https://www.nytimes3xbfgragh.onion/2020/08/04/us/elections/primary-election-michigan-arizona-kansas.html?action=click\&pgtype=Article\&state=default\&region=MAIN_CONTENT_1\&context=storylines_live_updates}{Latest
Updates: 2020
Election}}{Latest Updates: 2020 Election}}\label{latest-updates-2020-election}}

Updated 2020-08-05T03:23:56.561Z

\begin{itemize}
\tightlist
\item
  \href{https://www.nytimes3xbfgragh.onion/2020/08/04/us/elections/primary-election-michigan-arizona-kansas.html?action=click\&pgtype=Article\&state=default\&region=MAIN_CONTENT_1\&context=storylines_live_updates\#link-3924dd44}{Two
  G.O.P. Senate primaries offer --- what else? --- a test of loyalty to
  Trump.}
\item
  \href{https://www.nytimes3xbfgragh.onion/2020/08/04/us/elections/primary-election-michigan-arizona-kansas.html?action=click\&pgtype=Article\&state=default\&region=MAIN_CONTENT_1\&context=storylines_live_updates\#link-62a8e06b}{The
  military-style uniforms of federal agents who responded to the unrest
  in Portland will be replaced.}
\item
  \href{https://www.nytimes3xbfgragh.onion/2020/08/04/us/elections/primary-election-michigan-arizona-kansas.html?action=click\&pgtype=Article\&state=default\&region=MAIN_CONTENT_1\&context=storylines_live_updates\#link-32b39e33}{President
  Trump is suddenly a big supporter of mail-in voting --- in Florida.}
\end{itemize}

\href{https://www.nytimes3xbfgragh.onion/2020/08/04/us/elections/primary-election-michigan-arizona-kansas.html?action=click\&pgtype=Article\&state=default\&region=MAIN_CONTENT_1\&context=storylines_live_updates}{See
more updates}

In the 1970s, Ms. Bass joined the
\href{https://www.theatlantic.com/politics/archive/2020/07/karen-bass-cuba-venceremos-brigade/614662/}{Venceremos
Brigade}, a group of young leftists working on construction projects in
Fidel Castro's Cuba. When Castro
\href{https://www.nytimes3xbfgragh.onion/2016/11/26/world/americas/fidel-castro-dies.html}{died
in 2016}, she put out a
\href{https://bass.house.gov/media-center/press-releases/rep-bass-statement-passing-fidel-castro}{statement}
saying that ``the passing of the Comandante en Jefe is a great loss to
the people of Cuba.'' Those positions could allow Mr. Trump to continue
to attack the Democratic ticket as socialist, and could be damaging with
many Cubans in Florida and moderate voters across the Midwest.

Ms. Bass said she had made a mistake calling Castro ``Comandante en
Jefe,'' a description widely detested by Cuban exiles. ``Wouldn't do
that again,'' she said Sunday on NBC's ``Meet the Press.'' ``Talked
immediately to my colleagues from Florida and realized that that was
something that just shouldn't have been said.''

She said she had gone to Cuba to help the Cuban people by building
houses and later to recruit doctors for the United States. ``Now, that
doesn't excuse the fact that I know the Castro regime has been a brutal
regime to its people,'' she said.

Chapters of her career that might have gone unnoticed, even as the
leader of the Assembly or a prominent member of Congress, are now coming
to the forefront with the scrutiny that comes with being a potential
candidate for vice president.

In one example, Ms. Bass
\href{https://www.scientology.org/scientology-today/church-openings/church-of-scientology-los-angeles.html}{spoke
effusively} about the Church of Scientology, the organization
scrutinized over cultlike practices and alleged abuse of members, at the
opening of its world headquarters in Los Angeles in 2010. Over the
weekend, she
\href{https://twitter.com/KarenBassTweets/status/1289574623424573442?s=20}{tweeted}
a statement explaining why she had attended (it was a big project in her
district) and added, ``Just so you all know, I proudly worship at First
New Christian Fellowship Baptist Church in South LA.''

Back in her days as a street organizer, she had never thought about
running for office --- not for the Assembly, not for speaker and not for
Congress.

And certainly not for vice president.

``Oh God, no!'' she said. ``There isn't anything in my background that
would make you think that I was planning to run for office, period.''

As Mr. Biden closes in on making his choice, Ms. Bass has become a
classic inside player, calling party leaders, giving interviews and
joining Mr. Biden for a virtual fund-raiser where, in a preview of
sorts, she attacked Mr. Trump. Yet she never forgot the lessons she
learned in those first days as Assembly speaker.

``When a woman is in a role that's new, people underestimate,'' she
said. ``You need to know that the minute I got the job, the question
was, was I really going to last and be a speaker? Is she real, does she
really have any power?''

\hypertarget{i-felt-like-i-had-failed-a-generation}{%
\subsection{`I felt like I had failed a
generation'}\label{i-felt-like-i-had-failed-a-generation}}

As she grew up in 1960s Los Angeles, Ms. Bass watched the civil rights
movement unfold on the nightly news with her father, volunteering to
walk precincts for Robert F. Kennedy, who was assassinated not far from
her home.

After graduating from the physician assistant program at the University
of Southern California, Ms. Bass worked in an emergency room and taught
on the faculty of the college. As she saw more evidence of the crack
epidemic that was starting to ravage what was then called South Central
Los Angeles, she started a group called the Community Coalition for
Substance Abuse Prevention and Treatment, with the help of a federal
grant.

Image

Ms. Bass was the executive director of the Community Coalition in South
Central Los Angeles, where she worked to close down businesses like
liquor stores and replace them with schools and other
businesses.Credit...Ken Hively/Los Angeles Times, via Getty Images

While elected officials were beginning to enact harsh legal punishments,
Ms. Bass viewed addiction as a public health crisis --- hardly a popular
view at the time. In meetings in living rooms and community centers, Ms.
Bass argued that there were responses to the epidemic other than the
aggressive policing that was widespread in Los Angeles, such as curbing
the number of liquor stores in the area. The Community Coalition would
become one of the largest and most influential advocacy groups in the
city.

``What she saw quickly was that this was not going to be a time-limited
campaign to stop government from criminalizing people --- she realized
this was going on at a national level and set about to build an
organization that was really going to be focused on local reality,''
said Marqueece Harris-Dawson, who took over the organization from Ms.
Bass and is now a councilman from the area.

What made her work notable, Mr. Harris-Dawson said, was her emphasis on
bringing attention to local residents. When television cameras showed
up, she would often step aside and defer to the people who lived there,
her attitude shaped in part by seeing the civil rights movement lose
momentum after the assassinations of the Rev. Dr. Martin Luther King Jr.
and Robert Kennedy.

For young organizers on staff, she offered specific lessons: Talk to
everyone you can meet in the community, don't rely on rhetoric and
theories learned in college or activist circles, always have the
conversation with someone who has expressed disagreement with you. She
instructed them to shop in local grocery stores and strike up
conversations with residents while waiting in line.

At the time, South Los Angeles was starting to shift from being a
largely African-American neighborhood to one where Latinos eventually
became the majority. The neighborhood would soon go through turmoil
after a video captured members of the Los Angeles Police Department
beating Rodney King after a high-speed police chase in 1991.

Image

Smoke rising from a shopping center burned by rioters in Los Angeles in
1992 after four police officers were acquitted in the beating of Rodney
King.Credit...Paul Sakuma/Associated Press

Ms. Bass said she initially thought the widely seen recording of the
beating would drive sweeping change --- ``finally, there was evidence of
what we had been talking about for years,'' she said. She, like many
others in the city, was stunned when the officers were acquitted in 1992
and disheartened when the verdict was met with violence and riots, many
in her neighborhood.

``I felt like I had failed a generation, that they were so angry and
that we hadn't gotten change,'' she said.

In the weeks after the uprising, Ms. Bass went block by block surveying
the damage, which had ravaged large areas of the city. As officials set
about plans for rebuilding, she focused on keeping out the kinds of
businesses that residents had long complained were too prevalent, such
as liquor stores and motels, and which she said contributed to the
proliferation of drugs in the community.

By 1994, there was clear evidence of how much the Community Coalition
was going against the grain with its theories for dealing with crime.
And it came from Washington: a crime bill that would set mandatory
minimum sentences for possession of crack cocaine.

One of the main champions of the bill was Mr. Biden.

``The crime bill was aimed at the war on drugs, and that went against
the very reason she started Community Coalition,'' said Mark
Ridley-Thomas, a Los Angeles County supervisor and longtime friend of
Ms. Bass's. ``She knew that the declaration of a war on drugs was to
declare war on human beings who were suffering from a range of
addictions, and that was the wrong approach.''

Ms. Bass initially thought that she could be more effective dealing with
the ills of her city --- drugs, crime, poverty --- in private
organizations free of the strains of electoral politics.

Image

Ms. Bass with fellow activists in 1992. As she worked for the Community
Coalition for Substance Abuse Prevention and Treatment, she began to
attain local prominence.~Credit...Community Coalition

But during those years in Los Angeles, Ms. Bass achieved a level of
prominence and prestige in the community and with local officials, who
began to consult her on issues like appointments to the Police
Department and education policy.

Then, in 2004, pressed by elected officials whom she had been working
with, she agreed to run for the State Assembly. The outside player was
ready to move to the inside.

\hypertarget{crafting-a-way-out-of-fiscal-crisis}{%
\subsection{Crafting a way out of fiscal
crisis}\label{crafting-a-way-out-of-fiscal-crisis}}

California was sliding into crisis --- a \$41 billion deficit, larger
than the entire budget of many states --- when Ms. Bass was elected
speaker in early 2008. For the next two years, she and other legislative
leaders struggled with Mr. Schwarzenegger over a punishing package of
spending cuts and tax increases to rescue California from its worst
fiscal crisis since the Depression.

It also was a time of a tragedy in her life: In 2006, her only daughter
and her son-in-law were killed in a car crash, an echo of the loss Mr.
Biden has experienced; his wife and daughter were killed in an
automobile accident in 1972.

``I had a choice as to whether to go back to work or hide,'' Ms. Bass
said. ``I had to embrace the grief and wrap myself in grief. People
don't know how to deal with death and people avoid you, people don't
know what to say. Parents who have lost their children, you join a club
that you didn't ask to be a part of.''

The collapse of the national economy and housing market had hammered the
state's finances. Workers were forced to take furloughs. Vendors were
paid with i.o.u.s, and tax refunds were delayed.

``It was the worst budget time in the history of California by far,''
said Michael C. Genest, who was Mr. Schwarzenegger's top financial
adviser. ``The major dynamic was that Republicans wanted to cut,
Democrats wanted to raise taxes. And the reality was we could never cut
enough to balance the budget. And we could never raise taxes enough to
balance the budget.''

Democrats held a substantial majority in the Assembly and the Senate.
But passing a budget in California at the time required a vote by
two-thirds of members of each chamber. As a result, Republican leaders
had significant power in the negotiations.

John A. Pérez, who succeeded Ms. Bass as speaker, said the dynamic of a
minority party holding the votes needed to pass legislation greatly
complicated her negotiations with the Senate and governor.

``But then she had the hard work of delivering the votes in the
caucus,'' he said. ``And nobody goes to office to make cuts on programs
they hold near and dear.''

Antonio R. Villaraigosa, who was the mayor of Los Angeles at the time,
said he had pushed her for speaker because he thought she had the
political skills to navigate a particularly difficult time in
Sacramento.

``I thought we needed a leader who could work across the aisles,'' he
said. ``She's unabashedly a progressive, but she's also practical and
she knows how to work with Republicans.''

It was during that period that she honed the political and legislative
skill that have drawn the attention of Mr. Biden --- the drive for
consensus, a quiet if firm manner, an ability to work with people on the
other side of the political spectrum.

``I would make the decisions and go tell everybody: OK, this is the
decision we made,'' said Fabian Núñez, who was speaker before Ms. Bass.
``She was a coalition builder. She wanted the buy-in.''

To this day, Ms. Bass remembers the unlikely alliance with Mr.
Schwarzenegger and how they navigated the budget crisis from vastly
different ideological perspectives. She would joke morosely when it came
time for the daily walk through the Capitol to Mr. Schwarzenegger's
office, past the clusters of reporters and lobbyists.

``She used to refer to our daily negotiations as going to Gitmo,'' ****
said Darrell Steinberg, who was the Democratic leader of the Senate at
the time and is now the mayor of Sacramento. ```Gotta go to Gitmo.'''

Image

Mr. Schwarzenegger talking with Ms. Bass at a budget meeting in his
Capitol office in Sacramento in January 2009.Credit...Rich
Pedroncelli/Associated Press

The
\href{https://www.nytimes3xbfgragh.onion/2009/02/20/us/20california.html}{budget
deal agreed to} in February 2009 forced sharp concessions from Democrats
and Republicans: \$14.8 billion in cuts from health care, education,
public transit and other social services; \$12.5 billion in tax
increases;\$5.4 billion in borrowing; and various accounting gimmicks,
including reliance on an infusion of federal aid. And as the state
economy kept deteriorating, and voters refused to approve some of the
taxes, California was back in the red by spring, forcing Ms. Bass to
support even more budget cuts.

``She was unflappable,'' Mr. Steinberg said. ``It was high-stress. And
we were making decisions that we did not want to make. Deep cuts to
things that we cherished.''

Ms. Bass found herself having to go against years of her advocacy by
yielding to Mr. Schwarzenegger and Republicans on issues like foster
care and welfare.

``It was very devastating personally,'' she said. ``The whole reason I
wanted to run for office to begin with was to expand the social safety
net.''

As Ms. Bass finds herself in contention for a position she had never
sought or considered, a critical question has emerged: Is the ability to
seek consensus versus confrontation, which has served her so well over
the decades, what is required for a vice-presidential candidate in
today's brutal political environment?

The No. 2 person on the ticket is often expected to carry the burden of
delivering searing attacks. Mr. Biden's running mate is likely to face
particular scrutiny from Mr. Trump,
\href{https://www.nytimes3xbfgragh.onion/2020/07/10/us/biden-trump.html}{who
has struggled to find an effective line of attack} against the former
vice president.

That said, Ms. Bass's nonthreatening style and ability to make alliances
could prove to be a tonic in Washington in a post-Trump era. She counts
among her friends an old fellow legislator from Sacramento, Kevin
McCarthy, the House minority leader and Trump ally.

``People use the word nice about her, if you notice,'' Mr. Villaraigosa
said. ``But I think a better word is --- she knows how to stand up for
what she believes in, without being difficult.''

Ms. Bass said that she was glad to hear herself described as nice ``as
opposed to mean'' and that with the nation reeling from an epidemic,
confrontation politics seemed petty. But, she added, ``Nice doesn't mean
I wouldn't fight.''

\hypertarget{our-2020-election-guide}{%
\section{Our 2020 Election Guide}\label{our-2020-election-guide}}

Updated Aug. 4, 2020

\begin{itemize}
\item
  \begin{center}\rule{0.5\linewidth}{\linethickness}\end{center}

  \hypertarget{the-latest}{%
  \subsection{The Latest}\label{the-latest}}

  \begin{itemize}
  \tightlist
  \item
    Kris Kobach, a polarizing figure in Kansas politics,
    \href{https://www.nytimes3xbfgragh.onion/2020/08/04/us/politics/kobach-tlaib.html?action=click\&pgtype=Article\&state=default\&region=BELOW_MAIN_CONTENT\&context=storylines_guide}{lost
    the Senate primary there}, relieving G.O.P. officials who feared he
    could jeopardize a safe seat.
  \end{itemize}
\item
  \begin{center}\rule{0.5\linewidth}{\linethickness}\end{center}

  \hypertarget{bidens-vp-search}{%
  \subsection{Biden's V.P. Search}\label{bidens-vp-search}}

  \begin{itemize}
  \tightlist
  \item
    \href{https://www.nytimes3xbfgragh.onion/article/biden-vice-president-2020.html?action=click\&pgtype=Article\&state=default\&region=BELOW_MAIN_CONTENT\&context=storylines_guide}{Here
    are 13 women} who have been under consideration to be Joe Biden's
    running mate, and why each might be chosen --- and might not be.
  \end{itemize}
\item
  \begin{center}\rule{0.5\linewidth}{\linethickness}\end{center}

  \hypertarget{keep-up-with-our-coverage}{%
  \subsection{Keep Up With Our
  Coverage}\label{keep-up-with-our-coverage}}

  \begin{itemize}
  \tightlist
  \item
    Get an
    \href{https://www.nytimes3xbfgragh.onion/newsletters/politics?action=click\&pgtype=Article\&state=default\&region=BELOW_MAIN_CONTENT\&context=storylines_guide}{email}
    recapping the day's news
  \end{itemize}

  \begin{itemize}
  \tightlist
  \item
    Download our mobile app on
    \href{https://apps.apple.com/us/app/nytimes/id284862083?ls=1\&mat_click_id=5c79ae7455014fd1bd66b5610c05b8f2-20191112-16948\&referrer=mat_click_id\%3D5c79ae7455014fd1bd66b5610c05b8f2-20191112-16948\%26link_click_id\%3D722930677036718082}{iOS}
    and
    \href{http://a.localytics.com/android?id=com.nytimes.android\&referrer=utm_source\%3Dother_nyt_mobile_web\%26utm_medium\%3DWeb\%2520page\%26utm_term\%3DGeneral\%2520Mobile\%2520Page\%26utm_campaign\%3DNYT\%2520Mobile\%2520General\%2520Page}{Android}
    and turn on Breaking News and Politics alerts
  \end{itemize}
\end{itemize}

Advertisement

\protect\hyperlink{after-bottom}{Continue reading the main story}

\hypertarget{site-index}{%
\subsection{Site Index}\label{site-index}}

\hypertarget{site-information-navigation}{%
\subsection{Site Information
Navigation}\label{site-information-navigation}}

\begin{itemize}
\tightlist
\item
  \href{https://help.nytimes3xbfgragh.onion/hc/en-us/articles/115014792127-Copyright-notice}{©~2020~The
  New York Times Company}
\end{itemize}

\begin{itemize}
\tightlist
\item
  \href{https://www.nytco.com/}{NYTCo}
\item
  \href{https://help.nytimes3xbfgragh.onion/hc/en-us/articles/115015385887-Contact-Us}{Contact
  Us}
\item
  \href{https://www.nytco.com/careers/}{Work with us}
\item
  \href{https://nytmediakit.com/}{Advertise}
\item
  \href{http://www.tbrandstudio.com/}{T Brand Studio}
\item
  \href{https://www.nytimes3xbfgragh.onion/privacy/cookie-policy\#how-do-i-manage-trackers}{Your
  Ad Choices}
\item
  \href{https://www.nytimes3xbfgragh.onion/privacy}{Privacy}
\item
  \href{https://help.nytimes3xbfgragh.onion/hc/en-us/articles/115014893428-Terms-of-service}{Terms
  of Service}
\item
  \href{https://help.nytimes3xbfgragh.onion/hc/en-us/articles/115014893968-Terms-of-sale}{Terms
  of Sale}
\item
  \href{https://spiderbites.nytimes3xbfgragh.onion}{Site Map}
\item
  \href{https://help.nytimes3xbfgragh.onion/hc/en-us}{Help}
\item
  \href{https://www.nytimes3xbfgragh.onion/subscription?campaignId=37WXW}{Subscriptions}
\end{itemize}
