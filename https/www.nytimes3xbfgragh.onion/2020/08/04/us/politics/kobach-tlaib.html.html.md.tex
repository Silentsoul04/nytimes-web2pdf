Sections

SEARCH

\protect\hyperlink{site-content}{Skip to
content}\protect\hyperlink{site-index}{Skip to site index}

\href{https://www.nytimes3xbfgragh.onion/section/politics}{Politics}

\href{https://myaccount.nytimes3xbfgragh.onion/auth/login?response_type=cookie\&client_id=vi}{}

\href{https://www.nytimes3xbfgragh.onion/section/todayspaper}{Today's
Paper}

\href{/section/politics}{Politics}\textbar{}Kris Kobach Loses Kansas
Senate Primary, Easing Republican Worries

\href{https://nyti.ms/3frasv7}{https://nyti.ms/3frasv7}

\begin{itemize}
\item
\item
\item
\item
\item
\end{itemize}

Advertisement

\protect\hyperlink{after-top}{Continue reading the main story}

Supported by

\protect\hyperlink{after-sponsor}{Continue reading the main story}

\hypertarget{kris-kobach-loses-kansas-senate-primary-easing-republican-worries}{%
\section{Kris Kobach Loses Kansas Senate Primary, Easing Republican
Worries}\label{kris-kobach-loses-kansas-senate-primary-easing-republican-worries}}

In Missouri, Cori Bush, a progressive activist, upset the 10-term
incumbent William Lacy Clay Jr., marking another major step forward for
the progressive movement in its bid to threaten centrist officials.

\includegraphics{https://static01.graylady3jvrrxbe.onion/images/2020/08/04/us/politics/04election-ledeall-marshall-sub/merlin_175315569_88f0675c-2295-444e-a659-0c9a83ca54fa-articleLarge.jpg?quality=75\&auto=webp\&disable=upscale}

\href{https://www.nytimes3xbfgragh.onion/by/astead-w-herndon}{\includegraphics{https://static01.graylady3jvrrxbe.onion/images/2018/09/14/us/author-head-astead/author-head-astead-thumbLarge-v2.png}}\href{https://www.nytimes3xbfgragh.onion/by/katie-glueck}{\includegraphics{https://static01.graylady3jvrrxbe.onion/images/2020/01/29/reader-center/author-katie-glueck/author-katie-glueck-thumbLarge.png}}

By \href{https://www.nytimes3xbfgragh.onion/by/astead-w-herndon}{Astead
W. Herndon} and
\href{https://www.nytimes3xbfgragh.onion/by/katie-glueck}{Katie Glueck}

\begin{itemize}
\item
  Published Aug. 4, 2020Updated Aug. 5, 2020, 12:57 a.m. ET
\item
  \begin{itemize}
  \item
  \item
  \item
  \item
  \item
  \end{itemize}
\end{itemize}

Kansas Republicans on Tuesday soundly rejected the Senate bid of Kris W.
Kobach, a polarizing figure in state politics and a staunch ally of
\href{https://www.nytimes3xbfgragh.onion/interactive/2020/us/elections/donald-trump.html}{President
Trump}'s, choosing instead to nominate a conservative congressman who
was the preferred choice of party leaders.

Mr. Kobach was defeated in the primary by Representative Roger Marshall,
The Associated Press reported, a major relief to G.O.P. officials in
Kansas and Washington who had worried that Mr. Kobach would uniquely
jeopardize the seat in the general election and would be a thorn in the
side of party leadership if he won. Mr. Marshall will face State Senator
Barbara Bollier, a former Republican herself who switched parties, in
November.

In Missouri, a progressive activist, Cori Bush,
\href{https://www.nytimes3xbfgragh.onion/2020/08/05/us/politics/cori-bush-missouri-william-lacy-clay.html}{pulled
off a stunning upset} against the longtime incumbent William Lacy Clay
Jr., The A.P. reported, marking a turning point for the progressive
movement in its bid to threaten more centrist elected officials. If she
wins in November, Ms. Bush, a nurse who was a local leader in the Black
Lives Matter movement in St. Louis, would be the first person outside
the Clay family to represent the seat in more than 50 years.

Ms. Bush, 44, would also become the first Black woman to represent
Missouri in Congress.

In a closely watched Democratic House race in Michigan, Representative
Rashida Tlaib was facing a tough challenge from her 2018 rival, Brenda
Jones, with little of the vote having been reported late in the evening.
Ms. Tlaib is a member of the progressive ``squad'' of women of color who
helped shape the party's House majority.

Mr. Kobach, a former Kansas secretary of state known for his hard-line
views on immigration and voting rights, was seen by party leaders as an
especially weak potential general election candidate, even in a state
that has not sent a Democrat to the Senate in 88 years. In the 2018
governor's race, Mr. Kobach lost to Laura Kelly, a Democrat, and heading
into this week's contest, Senate Republican polling showed that nearly
30 percent of Republican primary voters indicated they would support Ms.
Bollier in the general election if Mr. Kobach were the nominee.

Early results indicated that Mr. Kobach lost counties he had won handily
in the 2018 primary, and in some places he lost last cycle, the margins
of defeat were bigger this time. A rival candidate,
\href{https://www.kansascity.com/news/politics-government/article244374532.html}{Bob
Hamilton}, a businessman who started a successful plumbing company and
has lent his own campaign several million dollars, also took some
counties Mr. Kobach had won in the 2018 primary. (His slogan: ``Send in
a plumber to drain the swamp.'')

It is possible that the race could still be in play this fall, as
Republicans confront a challenging political landscape shaped by
disapproval of Mr. Trump's leadership during the coronavirus crisis. But
Republicans and Democrats alike expected the state to be much more
competitive if Mr. Kobach had won the nomination.

\includegraphics{https://static01.graylady3jvrrxbe.onion/images/2020/08/04/us/politics/04election-ledeall3/merlin_171905724_5d9215fb-0509-4d26-8003-4ca3a46bf78a-articleLarge.jpg?quality=75\&auto=webp\&disable=upscale}

Kansas was one of several states, including Missouri, Michigan and
Arizona, holding some of the last remaining primaries before November's
general election. It was a new test of the mail-in voting systems that
many states are relying on during the coronavirus pandemic. The lack of
immediate results in some places was yet another precursor of what is
likely to unfold in November, when the reliance on absentee voting
systems could delay results past Election Day.

That dynamic was evident on Tuesday in New York City, where, six weeks
after Primary Day, the Board of Elections
\href{https://www.nytimes3xbfgragh.onion/2020/08/04/nyregion/maloney-torres-ny-congressional-races.html}{delivered
long-awaited victories to two Democrats}: Ritchie Torres, a 32-year-old
New York City councilman, who won a 12-way Democratic primary for a
soon-to-be open House seat, and Representative Carolyn B. Maloney, a
longtime incumbent. The expansive use of vote-by-mail in New York was
viewed by some as a test of whether the nation is ready for November.

The contests nationwide on Tuesday were a microcosm of several political
themes the parties are confronting, including the embrace of Republican
candidates fashioned in the style of Mr. Trump and the left-wing push to
unseat more centrist House Democrats.

On the Republican side, the Kansas Senate race in particular offered
another reminder that the party divisions that existed before Mr. Trump
won will persist even after he leaves office. That includes the
disagreement between deeply conservative activists, who are skeptical of
Washington and approve of the type of white identity politics Mr. Trump
has embraced, and the party's traditional establishment --- many members
of which have argued that such messaging hurts the party long-term.

One Republican House member, Representative Steve Watkins of Kansas,
fell to a primary challenger, Jake LaTurner. Mr. Watkins had been
\href{https://www.nytimes3xbfgragh.onion/2020/07/14/us/steve-watkins-voter-fraud-kansas.html}{charged
with four counts of voter fraud} last month, which capped off an
embattled two years in Congress after he was elected in 2018. Mr.
Watkins reportedly listed a UPS store in Topeka as his official
residence on a change-of-address form for voter registration in 2019.

The success in Missouri of Ms. Bush, shows a new pathway for the
left-wing efforts to remake the House Democratic caucus. Since 2018,
progressives have found some success in heavily Democratic districts
with a white incumbent and a majority-minority population, a pathway
executed by successful House challengers like Alexandria Ocasio-Cortez
in New York, Ayanna Pressley in Massachusetts and Jamaal Bowman in New
York, who coupled the insurgent message of ideological change with an
argument about racial representation.

Ms. Bush is the first example of that wing defeating a Black or Latino
members of the party's establishment. Earlier this year, other longtime
Black caucus members in Ohio and New York also easily defeated
challengers, and some members of the Congressional Black Caucus crowed
that the party's left wing could not threaten them. That is no longer
true.

Ms. Bush also won without the full backing of the progressive apparatus.
Though she received support from political groups such as Justice
Democrats and Sunrise Movement, and from Senator Bernie Sanders of
Vermont, she did not have the full-throated support of other figures
including Ms. Ocasio-Cortez.

Ms. Bush had sought to reframe the race in the wake of the recent
protests across the country over police brutality and racism, arguing
that she was attuned to the needs of Black voters, and that Mr. Clay had
not made a sufficient impact during his years in Congress. Ms. Bush
leaned on the region's recent history of activism stemming from the
protests in Ferguson, Mo., that kicked off the Black Lives Matter
movement in 2014.

``Tonight, Missouri's 1st has decided that an incremental approach isn't
going to work any longer,'' Ms. Bush said after her win.

Missouri voters also approved the expansion of Medicaid to more than
200,000 low-income adults, a break from the Trump Administration, which
has tried to repeal the Affordable Care Act.

States handled the election activity Tuesday with moderate success, as
Americans continued to show a degree of **** comfort with mail-in and
absentee voting systems even as Mr. Trump and his allies have sought to
sow distrust. In Michigan, more than 1.6 million voters had turned in an
absentee ballot by Tuesday evening, according to election officials, a
sizable portion of the total electorate.

The contests unfolded at a moment of extraordinary turmoil in the
nation, capping a summer defined by a pandemic and economic crisis, as
well as a national outcry over racism and police brutality. And on both
sides of the aisle, the races tested enthusiasm for voting amid a public
health crisis.

In Detroit, Corlette Selman, 59, a hair stylist wearing a Black Lives
Matter mask, said she felt as if she were voting for her life on
Tuesday.

Image

Corlette Selman in front of a line of people waiting to vote outside the
Detroit Department of Elections.Credit...Elaine Cromie for The New York
Times

``What's most important for me is to get the proper people in place to
take over the Senate, to maintain the House and to get us a new
president, because we can't live like this anymore,'' she said.

In Kansas, a statewide race for a Democrat is always an uphill battle.
But after the moderate Kansas City suburbs sent a Democrat to Congress
in 2018, and as Mr. Trump faces a backlash even in red states,
Republican strategists had grown increasingly uneasy about the contest
over all --- though many observers' fears were especially concentrated
on the prospect of a Kobach nomination.

``The Republican majority in this country may well go through Kansas and
we are the backstop,'' Mr. Marshall said on Tuesday,
\href{https://www.kansascity.com/news/politics-government/election/article244717957.html}{according
to} The Kansas City Star. Mr. Kobach, in his concession speech, promised
to ``do everything I can'' to help Republicans hold the seat, the Star
reported.

Mr. Kobach, who has run for office
\href{https://www.nytimes3xbfgragh.onion/2017/06/13/magazine/the-man-behind-trumps-voter-fraud-obsession.html}{multiple
times}, has long been a controversial figure in Kansas. He has
cultivated a devoted conservative following but has also alienated more
centrist Republicans.

Throughout the race, he sought to paint his lead primary rival, Mr.
Marshall, as too moderate and insufficiently supportive of the
president. Mr. Marshall, who is in fact deeply conservative especially
on social issues, fought those characterizations at every turn while the
Senate Republican leadership implored Mr. Trump to endorse Mr. Marshall
and block Mr. Kobach. The president did not do so, fueling tensions
between Capitol Hill and the White House.

On Tuesday evening, Mr. Trump spoke with Mr. Marshall, and the
congressman
\href{https://twitter.com/RogerMarshallMD/status/1290840935807549440?s=20}{put
the president on speakerphone} at an election night gathering.

``Well, I want to congratulate everybody and Roger, that's an incredible
race,'' Mr. Trump said, pledging his ``total support.'' ``Now we have to
win the one on Nov. 3. We have to win a couple of them on Nov. 3, come
to think.''

Mr. Marshall was not the original top choice of party leaders, who had
hoped that Secretary of State Mike Pompeo, a former Kansas congressman,
would enter the race.

When Mr. Pompeo declined to run, top Senate Republicans rallied around
Mr. Marshall, as did a range of influential organizations, a list that
included the U.S. Chamber of Commerce, the Kansas Farm Bureau and
several anti-abortion groups. The National Republican Senatorial
Committee also quietly led a voter contact effort called ``Operation
Scorched Prairie'' aimed at boosting Mr. Marshall, according to a person
familiar with the effort, making 2.3 million unique voter contacts over
text and calls in the final six days of the race.

In Arizona, Joe Arpaio, the bellicose former sheriff who gained
international celebrity for his hardline immigration policies, was in a
dead heat with his former chief deputy in the Republican primary for his
old job, with many mail-in ballots still to be counted.

Luke Broadwater, Nick Corasaniti and Kathleen Gray contributed
reporting.

Advertisement

\protect\hyperlink{after-bottom}{Continue reading the main story}

\hypertarget{site-index}{%
\subsection{Site Index}\label{site-index}}

\hypertarget{site-information-navigation}{%
\subsection{Site Information
Navigation}\label{site-information-navigation}}

\begin{itemize}
\tightlist
\item
  \href{https://help.nytimes3xbfgragh.onion/hc/en-us/articles/115014792127-Copyright-notice}{©~2020~The
  New York Times Company}
\end{itemize}

\begin{itemize}
\tightlist
\item
  \href{https://www.nytco.com/}{NYTCo}
\item
  \href{https://help.nytimes3xbfgragh.onion/hc/en-us/articles/115015385887-Contact-Us}{Contact
  Us}
\item
  \href{https://www.nytco.com/careers/}{Work with us}
\item
  \href{https://nytmediakit.com/}{Advertise}
\item
  \href{http://www.tbrandstudio.com/}{T Brand Studio}
\item
  \href{https://www.nytimes3xbfgragh.onion/privacy/cookie-policy\#how-do-i-manage-trackers}{Your
  Ad Choices}
\item
  \href{https://www.nytimes3xbfgragh.onion/privacy}{Privacy}
\item
  \href{https://help.nytimes3xbfgragh.onion/hc/en-us/articles/115014893428-Terms-of-service}{Terms
  of Service}
\item
  \href{https://help.nytimes3xbfgragh.onion/hc/en-us/articles/115014893968-Terms-of-sale}{Terms
  of Sale}
\item
  \href{https://spiderbites.nytimes3xbfgragh.onion}{Site Map}
\item
  \href{https://help.nytimes3xbfgragh.onion/hc/en-us}{Help}
\item
  \href{https://www.nytimes3xbfgragh.onion/subscription?campaignId=37WXW}{Subscriptions}
\end{itemize}
