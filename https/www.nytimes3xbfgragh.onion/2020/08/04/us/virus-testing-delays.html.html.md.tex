Sections

SEARCH

\protect\hyperlink{site-content}{Skip to
content}\protect\hyperlink{site-index}{Skip to site index}

\href{https://www.nytimes3xbfgragh.onion/section/us}{U.S.}

\href{https://myaccount.nytimes3xbfgragh.onion/auth/login?response_type=cookie\&client_id=vi}{}

\href{https://www.nytimes3xbfgragh.onion/section/todayspaper}{Today's
Paper}

\href{/section/us}{U.S.}\textbar{}`It's Like Having No Testing':
Coronavirus Test Results Are Still Delayed

\url{https://nyti.ms/2C27W0Q}

\begin{itemize}
\item
\item
\item
\item
\item
\end{itemize}

\href{https://www.nytimes3xbfgragh.onion/news-event/coronavirus?action=click\&pgtype=Article\&state=default\&region=TOP_BANNER\&context=storylines_menu}{The
Coronavirus Outbreak}

\begin{itemize}
\tightlist
\item
  live\href{https://www.nytimes3xbfgragh.onion/2020/08/04/world/coronavirus-cases.html?action=click\&pgtype=Article\&state=default\&region=TOP_BANNER\&context=storylines_menu}{Latest
  Updates}
\item
  \href{https://www.nytimes3xbfgragh.onion/interactive/2020/us/coronavirus-us-cases.html?action=click\&pgtype=Article\&state=default\&region=TOP_BANNER\&context=storylines_menu}{Maps
  and Cases}
\item
  \href{https://www.nytimes3xbfgragh.onion/interactive/2020/science/coronavirus-vaccine-tracker.html?action=click\&pgtype=Article\&state=default\&region=TOP_BANNER\&context=storylines_menu}{Vaccine
  Tracker}
\item
  \href{https://www.nytimes3xbfgragh.onion/2020/08/02/us/covid-college-reopening.html?action=click\&pgtype=Article\&state=default\&region=TOP_BANNER\&context=storylines_menu}{College
  Reopening}
\item
  \href{https://www.nytimes3xbfgragh.onion/live/2020/08/04/business/stock-market-today-coronavirus?action=click\&pgtype=Article\&state=default\&region=TOP_BANNER\&context=storylines_menu}{Economy}
\end{itemize}

Advertisement

\protect\hyperlink{after-top}{Continue reading the main story}

Supported by

\protect\hyperlink{after-sponsor}{Continue reading the main story}

\hypertarget{its-like-having-no-testing-coronavirus-test-results-are-still-delayed}{%
\section{`It's Like Having No Testing': Coronavirus Test Results Are
Still
Delayed}\label{its-like-having-no-testing-coronavirus-test-results-are-still-delayed}}

A shortage of chemicals needed to test for the virus is part of what is
slowing turnaround times.

\includegraphics{https://static01.graylady3jvrrxbe.onion/images/2020/08/04/us/04VIRUS-TESTING-mass/merlin_174443316_57fadf4c-cbfd-45c0-b35b-8a68c9821b14-articleLarge.jpg?quality=75\&auto=webp\&disable=upscale}

By \href{https://www.nytimes3xbfgragh.onion/by/sarah-mervosh}{Sarah
Mervosh} and
\href{https://www.nytimes3xbfgragh.onion/by/manny-fernandez}{Manny
Fernandez}

\begin{itemize}
\item
  Aug. 4, 2020
\item
  \begin{itemize}
  \item
  \item
  \item
  \item
  \item
  \end{itemize}
\end{itemize}

Frustrated by a nationwide testing backlog, the governors of six states
took the unusual step of banding together on Tuesday to reduce the
turnaround time for coronavirus test results from days to minutes.

The agreement, by three Republican governors and three Democratic
governors, was called the first interstate testing compact of its kind.
The six states --- Louisiana, Maryland, Massachusetts, Michigan, Ohio
and Virginia --- agreed to work with the Rockefeller Foundation and two
U.S. manufacturers of rapid tests to buy three million tests.

The bipartisan plan highlights the depth of the testing problems in the
United States more than six months into the pandemic.

The United States is testing about 755,000 people a day, up from about
640,000 per day a month ago, and far more than in April and May,
\href{https://covidtracking.com/data/us-daily}{according to the Covid
Tracking Project}. But numbers alone do not tell the whole story. With
testing chemicals and other equipment in short supply, and a surge in
coronavirus cases nationwide leading to skyrocketing demands, many
Americans are still having to wait many days for results, effectively
rendering those tests useless.

Most people who are tested for the virus do not receive results within
the 24 to 48 hours recommended by public health experts to effectively
stall the virus's spread and quickly conduct contact tracing, according
to a
\href{https://covidstates.net/COVID19\%20CONSORTIUM\%20REPORT\%208\%20TEST\%20JULY\%202020.pdf}{new
national survey} by researchers from Harvard University, Northeastern
University, Northwestern University and Rutgers University.

The survey --- representing 19,000 people from 50 states and Washington,
D.C., who responded to an online questionnaire last month --- found
lengthy wait times among those who had been tested for the virus, about
18 percent of all respondents. People who had been tested for the virus
in July reported an average wait time of about four days. That is about
the same wait time for those who reported taking a test in April. Over
all, about 10 percent of people reported waiting 10 days or more.

Respondents in a vast majority of states reported a median turnaround
time of at least three days, including residents of California, Florida,
Texas and other hot spots.

\hypertarget{latest-updates-global-coronavirus-outbreak}{%
\section{\texorpdfstring{\href{https://www.nytimes3xbfgragh.onion/2020/08/04/world/coronavirus-cases.html?action=click\&pgtype=Article\&state=default\&region=MAIN_CONTENT_1\&context=storylines_live_updates}{Latest
Updates: Global Coronavirus
Outbreak}}{Latest Updates: Global Coronavirus Outbreak}}\label{latest-updates-global-coronavirus-outbreak}}

Updated 2020-08-05T07:58:24.076Z

\begin{itemize}
\tightlist
\item
  \href{https://www.nytimes3xbfgragh.onion/2020/08/04/world/coronavirus-cases.html?action=click\&pgtype=Article\&state=default\&region=MAIN_CONTENT_1\&context=storylines_live_updates\#link-762df92}{As
  talks drag on, McConnell signals openness to jobless aid extension,
  and negotiators agree on a deadline.}
\item
  \href{https://www.nytimes3xbfgragh.onion/2020/08/04/world/coronavirus-cases.html?action=click\&pgtype=Article\&state=default\&region=MAIN_CONTENT_1\&context=storylines_live_updates\#link-1228a480}{Novavax
  sees encouraging results from two studies of its experimental
  vaccine.}
\item
  \href{https://www.nytimes3xbfgragh.onion/2020/08/04/world/coronavirus-cases.html?action=click\&pgtype=Article\&state=default\&region=MAIN_CONTENT_1\&context=storylines_live_updates\#link-794484ed}{Mississippians
  must now wear masks in public, governor says.}
\end{itemize}

\href{https://www.nytimes3xbfgragh.onion/2020/08/04/world/coronavirus-cases.html?action=click\&pgtype=Article\&state=default\&region=MAIN_CONTENT_1\&context=storylines_live_updates}{See
more updates}

More live coverage:
\href{https://www.nytimes3xbfgragh.onion/live/2020/08/04/business/stock-market-today-coronavirus?action=click\&pgtype=Article\&state=default\&region=MAIN_CONTENT_1\&context=storylines_live_updates}{Markets}

The survey also found disparities across racial groups, an indication
that people who are hit hardest by the pandemic are also having to wait
longer for test results. Black people surveyed reported an average wait
time for results of five days, and Hispanic respondents reported an
average wait time of 4.6 days, compared with 3.9 days for white people.

``Testing is just not quick enough,'' said Matthew A. Baum, a professor
of public policy at Harvard University and one of the researchers in the
group, which found that wait times were ``strikingly similar'' across
the country. ``This is an enormously widespread problem.''

The challenge, experts say, is a basic issue of supply and demand. About
half of all coronavirus tests are conducted by large-scale commercial
laboratory companies like LabCorp and Quest Diagnostics, which are
racing to turn around swabs while competing with a global market. There
is also a critical shortage of certain equipment and supplies, including
reagents, the chemical ingredients needed to detect whether the
coronavirus is present in a sample.

Federal officials argue that long wait times are unusual, citing figures
that more than 80 percent of tests are completed within three days. With
plans to reduce the share of tests done by commercial labs, shifting
more responsibility to hospitals and other facilities where tests can be
done in as little as 15 minutes, officials say that turnaround times are
expected to improve soon. Still, labs around the country have been
feeling the pressure for weeks.

In Brunswick, Ga., lab technicians are working around the clock.
Machines churn out results in 50 minutes to four and a half hours, but
there are so many orders that the labs cannot keep up. Regular shipments
of chemicals needed to test for the virus do not last even a week, so
pathologists have begun to carefully dole out their supplies.

``We literally ration tests,'' said Dr. Patrick Godbey, the president of
the College of American Pathologists and the director of two labs in the
Brunswick area. He estimated that for every test his labs are able to
perform, they have to send three to national commercial laboratory
companies.

``Why keep a backlog?'' he said. ``Tomorrow is going to be just as
busy.''

In California, Dr. Amir Jamali, an orthopedic surgeon, had to order
swabs from eBay in order to test surgery patients at his practice in the
Bay Area. He paid about \$50 for a bundle of five, which he said was far
above the normal price. But now, Dr. Jamali said, the effort has been
largely for naught because turnaround times at local labs are inching
longer by the day.

``We used to be able to get it within 24 and 48 hours,'' Dr. Jamali
said. ``Almost every week it's increasing to three day, five days, seven
days, 10 days.'' In that time, he fears his patients could have gone to
a party or a barbecue, potentially exposing other people, as well as his
staff and other patients. Some days, he has had to cancel surgery rather
than risk exposure.

``It's like having no testing,'' he said.

\includegraphics{https://static01.graylady3jvrrxbe.onion/images/2020/08/04/us/04VIRUS-TESTING-calif/merlin_175013205_0fe57767-941a-4dbd-a430-70abba858af1-articleLarge.jpg?quality=75\&auto=webp\&disable=upscale}

The problem spans much of the country. In Alabama, the health department
recently put out a warning that the turnaround time for test results has
ballooned to an average of seven days.

``A test result that comes back in seven or eight days is worthless for
everybody --- it shouldn't even be counted,'' said Dr. Amesh Adalja, a
senior scholar at the Johns Hopkins University Center for Health
Security and a physician in Pittsburgh. ``It's not a test in any kind of
effective manner because it's not actionable.''

Despite a shortage of the chemical reagents, Dr. Adalja said he
typically has no problem getting a quick test for hospitalized patients,
who are given priority. But people who go to a drive-through clinic or
walk into a CVS may have to wait much longer for their tests to come
through the system.

\href{https://www.nytimes3xbfgragh.onion/news-event/coronavirus?action=click\&pgtype=Article\&state=default\&region=MAIN_CONTENT_3\&context=storylines_faq}{}

\hypertarget{the-coronavirus-outbreak-}{%
\subsubsection{The Coronavirus Outbreak
›}\label{the-coronavirus-outbreak-}}

\hypertarget{frequently-asked-questions}{%
\paragraph{Frequently Asked
Questions}\label{frequently-asked-questions}}

Updated August 4, 2020

\begin{itemize}
\item ~
  \hypertarget{i-have-antibodies-am-i-now-immune}{%
  \paragraph{I have antibodies. Am I now
  immune?}\label{i-have-antibodies-am-i-now-immune}}

  \begin{itemize}
  \tightlist
  \item
    As of right
    now,\href{https://www.nytimes3xbfgragh.onion/2020/07/22/health/covid-antibodies-herd-immunity.html?action=click\&pgtype=Article\&state=default\&region=MAIN_CONTENT_3\&context=storylines_faq}{that
    seems likely, for at least several months.} There have been
    frightening accounts of people suffering what seems to be a second
    bout of Covid-19. But experts say these patients may have a
    drawn-out course of infection, with the virus taking a slow toll
    weeks to months after initial exposure. People infected with the
    coronavirus typically
    \href{https://www.nature.com/articles/s41586-020-2456-9}{produce}
    immune molecules called antibodies, which are
    \href{https://www.nytimes3xbfgragh.onion/2020/05/07/health/coronavirus-antibody-prevalence.html?action=click\&pgtype=Article\&state=default\&region=MAIN_CONTENT_3\&context=storylines_faq}{protective
    proteins made in response to an
    infection}\href{https://www.nytimes3xbfgragh.onion/2020/05/07/health/coronavirus-antibody-prevalence.html?action=click\&pgtype=Article\&state=default\&region=MAIN_CONTENT_3\&context=storylines_faq}{.
    These antibodies may} last in the body
    \href{https://www.nature.com/articles/s41591-020-0965-6}{only two to
    three months}, which may seem worrisome, but that's perfectly normal
    after an acute infection subsides, said Dr. Michael Mina, an
    immunologist at Harvard University. It may be possible to get the
    coronavirus again, but it's highly unlikely that it would be
    possible in a short window of time from initial infection or make
    people sicker the second time.
  \end{itemize}
\item ~
  \hypertarget{im-a-small-business-owner-can-i-get-relief}{%
  \paragraph{I'm a small-business owner. Can I get
  relief?}\label{im-a-small-business-owner-can-i-get-relief}}

  \begin{itemize}
  \tightlist
  \item
    The
    \href{https://www.nytimes3xbfgragh.onion/article/small-business-loans-stimulus-grants-freelancers-coronavirus.html?action=click\&pgtype=Article\&state=default\&region=MAIN_CONTENT_3\&context=storylines_faq}{stimulus
    bills enacted in March} offer help for the millions of American
    small businesses. Those eligible for aid are businesses and
    nonprofit organizations with fewer than 500 workers, including sole
    proprietorships, independent contractors and freelancers. Some
    larger companies in some industries are also eligible. The help
    being offered, which is being managed by the Small Business
    Administration, includes the Paycheck Protection Program and the
    Economic Injury Disaster Loan program. But lots of folks have
    \href{https://www.nytimes3xbfgragh.onion/interactive/2020/05/07/business/small-business-loans-coronavirus.html?action=click\&pgtype=Article\&state=default\&region=MAIN_CONTENT_3\&context=storylines_faq}{not
    yet seen payouts.} Even those who have received help are confused:
    The rules are draconian, and some are stuck sitting on
    \href{https://www.nytimes3xbfgragh.onion/2020/05/02/business/economy/loans-coronavirus-small-business.html?action=click\&pgtype=Article\&state=default\&region=MAIN_CONTENT_3\&context=storylines_faq}{money
    they don't know how to use.} Many small-business owners are getting
    less than they expected or
    \href{https://www.nytimes3xbfgragh.onion/2020/06/10/business/Small-business-loans-ppp.html?action=click\&pgtype=Article\&state=default\&region=MAIN_CONTENT_3\&context=storylines_faq}{not
    hearing anything at all.}
  \end{itemize}
\item ~
  \hypertarget{what-are-my-rights-if-i-am-worried-about-going-back-to-work}{%
  \paragraph{What are my rights if I am worried about going back to
  work?}\label{what-are-my-rights-if-i-am-worried-about-going-back-to-work}}

  \begin{itemize}
  \tightlist
  \item
    Employers have to provide
    \href{https://www.osha.gov/SLTC/covid-19/standards.html}{a safe
    workplace} with policies that protect everyone equally.
    \href{https://www.nytimes3xbfgragh.onion/article/coronavirus-money-unemployment.html?action=click\&pgtype=Article\&state=default\&region=MAIN_CONTENT_3\&context=storylines_faq}{And
    if one of your co-workers tests positive for the coronavirus, the
    C.D.C.} has said that
    \href{https://www.cdc.gov/coronavirus/2019-ncov/community/guidance-business-response.html}{employers
    should tell their employees} -\/- without giving you the sick
    employee's name -\/- that they may have been exposed to the virus.
  \end{itemize}
\item ~
  \hypertarget{should-i-refinance-my-mortgage}{%
  \paragraph{Should I refinance my
  mortgage?}\label{should-i-refinance-my-mortgage}}

  \begin{itemize}
  \tightlist
  \item
    \href{https://www.nytimes3xbfgragh.onion/article/coronavirus-money-unemployment.html?action=click\&pgtype=Article\&state=default\&region=MAIN_CONTENT_3\&context=storylines_faq}{It
    could be a good idea,} because mortgage rates have
    \href{https://www.nytimes3xbfgragh.onion/2020/07/16/business/mortgage-rates-below-3-percent.html?action=click\&pgtype=Article\&state=default\&region=MAIN_CONTENT_3\&context=storylines_faq}{never
    been lower.} Refinancing requests have pushed mortgage applications
    to some of the highest levels since 2008, so be prepared to get in
    line. But defaults are also up, so if you're thinking about buying a
    home, be aware that some lenders have tightened their standards.
  \end{itemize}
\item ~
  \hypertarget{what-is-school-going-to-look-like-in-september}{%
  \paragraph{What is school going to look like in
  September?}\label{what-is-school-going-to-look-like-in-september}}

  \begin{itemize}
  \tightlist
  \item
    It is unlikely that many schools will return to a normal schedule
    this fall, requiring the grind of
    \href{https://www.nytimes3xbfgragh.onion/2020/06/05/us/coronavirus-education-lost-learning.html?action=click\&pgtype=Article\&state=default\&region=MAIN_CONTENT_3\&context=storylines_faq}{online
    learning},
    \href{https://www.nytimes3xbfgragh.onion/2020/05/29/us/coronavirus-child-care-centers.html?action=click\&pgtype=Article\&state=default\&region=MAIN_CONTENT_3\&context=storylines_faq}{makeshift
    child care} and
    \href{https://www.nytimes3xbfgragh.onion/2020/06/03/business/economy/coronavirus-working-women.html?action=click\&pgtype=Article\&state=default\&region=MAIN_CONTENT_3\&context=storylines_faq}{stunted
    workdays} to continue. California's two largest public school
    districts --- Los Angeles and San Diego --- said on July 13, that
    \href{https://www.nytimes3xbfgragh.onion/2020/07/13/us/lausd-san-diego-school-reopening.html?action=click\&pgtype=Article\&state=default\&region=MAIN_CONTENT_3\&context=storylines_faq}{instruction
    will be remote-only in the fall}, citing concerns that surging
    coronavirus infections in their areas pose too dire a risk for
    students and teachers. Together, the two districts enroll some
    825,000 students. They are the largest in the country so far to
    abandon plans for even a partial physical return to classrooms when
    they reopen in August. For other districts, the solution won't be an
    all-or-nothing approach.
    \href{https://bioethics.jhu.edu/research-and-outreach/projects/eschool-initiative/school-policy-tracker/}{Many
    systems}, including the nation's largest, New York City, are
    devising
    \href{https://www.nytimes3xbfgragh.onion/2020/06/26/us/coronavirus-schools-reopen-fall.html?action=click\&pgtype=Article\&state=default\&region=MAIN_CONTENT_3\&context=storylines_faq}{hybrid
    plans} that involve spending some days in classrooms and other days
    online. There's no national policy on this yet, so check with your
    municipal school system regularly to see what is happening in your
    community.
  \end{itemize}
\end{itemize}

Based on federal guidance, LabCorp said it was prioritizing hospitalized
patients and nursing home tests. ``All other testing for patients are
performed in the order in which they are received,'' a spokeswoman said
in a statement. As of July 30, the company reported that its turnaround
time for test results was two to three days from when the sample was
picked up.

Quest Diagnostics, another major commercial lab, is
\href{https://newsroom.questdiagnostics.com/COVIDTestingUpdates/}{reporting
a turnaround time of five days} for all patients, and two for patients
who were the highest priority.

James Davis, executive vice president of general diagnostics for Quest,
said that the company had recently ramped up testing by acquiring its
own testing process and ingredients, and that it was already seeing
improvements in turnaround times. But he said there was only so much
that can be done because a majority of about 150,000 tests that Quest
performs each day rely on automated testing machines from two companies
that must use reagent kits from those companies, similar to a Honda car
that needed a replacement part made by Honda.

``Right now, we live hand to mouth,'' he said of the chemicals. ``I need
a shipment on Monday to make sure all of my labs run through Wednesday,
and then on Wednesday, I need another shipment,'' he said, adding,
``That's how tight the supply chain is.''

The six-state agreement announced on Tuesday showed how the lack of a
federal testing program has left municipalities and states to fend for
themselves. The Trump administration has provided new support to
hard-hit regions by providing free coronavirus testing in cities through
a ``surge testing'' program announced last month. But the bulk of
government-sponsored testing has been provided by cities, counties and
states that hire third-party contractors such as Quest and LabCorp. As a
result, the length of the delay varies between states, and within them.

In Texas, officials in San Antonio said it was taking 24 to 36 hours to
get test results back at a free government-run testing site at Freeman
Coliseum. The wait was longer at Legacy Community Health clinics in
Houston and the surrounding Gulf Coast region, where test results were
on a two-day to five-day turnaround time.

On Tuesday, officials in Harris County, which includes Houston, said
they were increasing testing capacity and shortening lab turnaround
times as part of federal ``surge testing.'' The temporary support, which
lasts until 30,000 testing samples are reached at two free testing
sites, will have test results in three to five business days.

The compact was negotiated by the Rockefeller Foundation and Gov. Larry
Hogan of Maryland, a Republican, during the final days of his tenure as
chairman of the National Governors Association. Instead of each of the
six states separately ordering thousands of tests, the group of states
is instead in discussions with the two U.S. manufacturers of antigen
tests, the Quidel Corporation and Becton, Dickinson \& Company, to buy
three million tests, or 500,000 tests per state.

Unlike the more readily available tests that use polymerase chain
reaction, or P.C.R., antigen tests can rapidly determine whether a
person has been infected by the virus by detecting fragments of virus in
a sample. The tests will deliver results in 15 to 20 minutes, the
governors said. Still, scientists have said the tests can frequently
miss infections. Tests from both Becton Dickinson and Quidel
\href{https://www.nytimes3xbfgragh.onion/2020/07/06/health/fast-coronavirus-tests.html}{could
produce false negative results} between 15 and 20 percent of the time.

The Trump administration's testing czar, Adm. Brett P. Giroir,
\href{https://www.nytimes3xbfgragh.onion/2020/07/31/us/politics/trump-coronavirus-testing.html}{told
Congress last week} that ``turnaround times are definitely improving.''
But Admiral Giroir, a doctor and the assistant secretary for health,
testified that getting test results within two to three days ``is not a
possible benchmark we can achieve today.''

Nicholas Bogel-Burroughs contributed reporting.

Advertisement

\protect\hyperlink{after-bottom}{Continue reading the main story}

\hypertarget{site-index}{%
\subsection{Site Index}\label{site-index}}

\hypertarget{site-information-navigation}{%
\subsection{Site Information
Navigation}\label{site-information-navigation}}

\begin{itemize}
\tightlist
\item
  \href{https://help.nytimes3xbfgragh.onion/hc/en-us/articles/115014792127-Copyright-notice}{©~2020~The
  New York Times Company}
\end{itemize}

\begin{itemize}
\tightlist
\item
  \href{https://www.nytco.com/}{NYTCo}
\item
  \href{https://help.nytimes3xbfgragh.onion/hc/en-us/articles/115015385887-Contact-Us}{Contact
  Us}
\item
  \href{https://www.nytco.com/careers/}{Work with us}
\item
  \href{https://nytmediakit.com/}{Advertise}
\item
  \href{http://www.tbrandstudio.com/}{T Brand Studio}
\item
  \href{https://www.nytimes3xbfgragh.onion/privacy/cookie-policy\#how-do-i-manage-trackers}{Your
  Ad Choices}
\item
  \href{https://www.nytimes3xbfgragh.onion/privacy}{Privacy}
\item
  \href{https://help.nytimes3xbfgragh.onion/hc/en-us/articles/115014893428-Terms-of-service}{Terms
  of Service}
\item
  \href{https://help.nytimes3xbfgragh.onion/hc/en-us/articles/115014893968-Terms-of-sale}{Terms
  of Sale}
\item
  \href{https://spiderbites.nytimes3xbfgragh.onion}{Site Map}
\item
  \href{https://help.nytimes3xbfgragh.onion/hc/en-us}{Help}
\item
  \href{https://www.nytimes3xbfgragh.onion/subscription?campaignId=37WXW}{Subscriptions}
\end{itemize}
