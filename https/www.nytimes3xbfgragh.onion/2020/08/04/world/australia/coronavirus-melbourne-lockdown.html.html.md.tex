Sections

SEARCH

\protect\hyperlink{site-content}{Skip to
content}\protect\hyperlink{site-index}{Skip to site index}

\href{https://www.nytimes3xbfgragh.onion/section/world/australia}{Australia}

\href{https://myaccount.nytimes3xbfgragh.onion/auth/login?response_type=cookie\&client_id=vi}{}

\href{https://www.nytimes3xbfgragh.onion/section/todayspaper}{Today's
Paper}

\href{/section/world/australia}{Australia}\textbar{}What Lockdown 2.0
Looks Like: Harsher Rules, Deeper Confusion

\url{https://nyti.ms/2PmCYn5}

\begin{itemize}
\item
\item
\item
\item
\item
\item
\end{itemize}

\href{https://www.nytimes3xbfgragh.onion/news-event/coronavirus?action=click\&pgtype=Article\&state=default\&region=TOP_BANNER\&context=storylines_menu}{The
Coronavirus Outbreak}

\begin{itemize}
\tightlist
\item
  live\href{https://www.nytimes3xbfgragh.onion/2020/08/04/world/coronavirus-cases.html?action=click\&pgtype=Article\&state=default\&region=TOP_BANNER\&context=storylines_menu}{Latest
  Updates}
\item
  \href{https://www.nytimes3xbfgragh.onion/interactive/2020/us/coronavirus-us-cases.html?action=click\&pgtype=Article\&state=default\&region=TOP_BANNER\&context=storylines_menu}{Maps
  and Cases}
\item
  \href{https://www.nytimes3xbfgragh.onion/interactive/2020/science/coronavirus-vaccine-tracker.html?action=click\&pgtype=Article\&state=default\&region=TOP_BANNER\&context=storylines_menu}{Vaccine
  Tracker}
\item
  \href{https://www.nytimes3xbfgragh.onion/2020/08/02/us/covid-college-reopening.html?action=click\&pgtype=Article\&state=default\&region=TOP_BANNER\&context=storylines_menu}{College
  Reopening}
\item
  \href{https://www.nytimes3xbfgragh.onion/live/2020/08/04/business/stock-market-today-coronavirus?action=click\&pgtype=Article\&state=default\&region=TOP_BANNER\&context=storylines_menu}{Economy}
\end{itemize}

Advertisement

\protect\hyperlink{after-top}{Continue reading the main story}

Supported by

\protect\hyperlink{after-sponsor}{Continue reading the main story}

\hypertarget{what-lockdown-20-looks-like-harsher-rules-deeper-confusion}{%
\section{What Lockdown 2.0 Looks Like: Harsher Rules, Deeper
Confusion}\label{what-lockdown-20-looks-like-harsher-rules-deeper-confusion}}

Melbourne, Australia's second-largest city, is becoming a case study in
handling a second wave of infections. There are lots of unanswered
questions.

\includegraphics{https://static01.graylady3jvrrxbe.onion/images/2020/08/04/world/04virus-melbourne-1/merlin_175223976_a61872fb-30c4-464a-8d65-ffc51cd30c21-articleLarge.jpg?quality=75\&auto=webp\&disable=upscale}

\href{https://www.nytimes3xbfgragh.onion/by/damien-cave}{\includegraphics{https://static01.graylady3jvrrxbe.onion/images/2018/10/08/multimedia/author-damien-cave/author-damien-cave-thumbLarge.png}}

By \href{https://www.nytimes3xbfgragh.onion/by/damien-cave}{Damien Cave}

\begin{itemize}
\item
  Aug. 4, 2020Updated 12:21 p.m. ET
\item
  \begin{itemize}
  \item
  \item
  \item
  \item
  \item
  \item
  \end{itemize}
\end{itemize}

Australia's second-largest city, Melbourne, is grappling with a
spiraling coronavirus outbreak in a country that once thought it had the
pandemic beat.

It has now imposed some of the toughest restrictions in the world,
offering a preview of what many urban dwellers elsewhere could confront
in coming weeks and months.

Melbourne, and the rest of Australia, was once a shining victor,
believing the virus to have been vanquished as of late June. But
assumptions made about hotel guests turned out to be the weak link.

A
\href{https://www.theage.com.au/national/victoria/how-hotel-quarantine-let-covid-19-out-of-the-bag-in-victoria-20200703-p558og.html}{breakdown
in the quarantine program for hotels}, which was contracted out to
private security, meant that returning travelers passed the virus to
hotel security guards, who carried the contagion into their
neighborhoods.

The spread continued even after Melbourne started a so-called Stage 3
lockdown in early July --- until recently, the highest level of
restrictions --- with no large gatherings and most people working from
home. Officials grew increasingly angry as they discovered that the
perception of a problem solved had produced complacency.

As officials cast about for ways to break the chain of infections, the
city has developed a confounding matrix of hefty fines for disobedience
to the lockdown, with minor exceptions for everything from romantic
partners to home building, and endless versions of the question: So,
wait, can I \_\_\_\_?

Restaurant owners are wondering about food delivery after an 8 p.m.
curfew began on Sunday night. Teenagers are asking if their boyfriends
and girlfriends count as essential partners. Can animal shelter
volunteers walk dogs at night? Are house cleaners essential for those
struggling with their mental health? Can people who have been tested
exercise outside?

``This is such a weird, scary, bizarro time that we live in,'' said
Tessethia Von Tessle Roberts, 25, a student in Melbourne who admits to
having hit a breaking point a few days ago, when her washing machine
broke.

``Our health care workers are hustling around the clock to keep us
alive,'' she said. ``Our politicians are as scared as we are, but they
have to pretend like they have a better idea than we do of what's going
to happen next.''

Pandemic lockdowns, never easy, are getting ever more confusing and
contentious as they evolve in the face of second and third rounds of
outbreaks that have exhausted both officials and residents. With success
against the virus as fleeting as the breeze, the new waves of
restrictions feel to many like a bombing raid that just won't end.

\hypertarget{latest-updates-global-coronavirus-outbreak}{%
\section{\texorpdfstring{\href{https://www.nytimes3xbfgragh.onion/2020/08/04/world/coronavirus-cases.html?action=click\&pgtype=Article\&state=default\&region=MAIN_CONTENT_1\&context=storylines_live_updates}{Latest
Updates: Global Coronavirus
Outbreak}}{Latest Updates: Global Coronavirus Outbreak}}\label{latest-updates-global-coronavirus-outbreak}}

Updated 2020-08-04T18:55:17.683Z

\begin{itemize}
\tightlist
\item
  \href{https://www.nytimes3xbfgragh.onion/2020/08/04/world/coronavirus-cases.html?action=click\&pgtype=Article\&state=default\&region=MAIN_CONTENT_1\&context=storylines_live_updates\#link-4825b93}{Public
  and private schools in Maryland and elsewhere are divided over
  in-person instruction.}
\item
  \href{https://www.nytimes3xbfgragh.onion/2020/08/04/world/coronavirus-cases.html?action=click\&pgtype=Article\&state=default\&region=MAIN_CONTENT_1\&context=storylines_live_updates\#link-4d1eafa8}{N.Y.C.'s
  health commissioner resigns after clashing with the mayor over the
  virus.}
\item
  \href{https://www.nytimes3xbfgragh.onion/2020/08/04/world/coronavirus-cases.html?action=click\&pgtype=Article\&state=default\&region=MAIN_CONTENT_1\&context=storylines_live_updates\#link-6b644638}{`Long
  days, long nights': Washington prepares for a prolonged fight over
  virus relief.}
\end{itemize}

\href{https://www.nytimes3xbfgragh.onion/2020/08/04/world/coronavirus-cases.html?action=click\&pgtype=Article\&state=default\&region=MAIN_CONTENT_1\&context=storylines_live_updates}{See
more updates}

More live coverage:
\href{https://www.nytimes3xbfgragh.onion/live/2020/08/04/business/stock-market-today-coronavirus?action=click\&pgtype=Article\&state=default\&region=MAIN_CONTENT_1\&context=storylines_live_updates}{Markets}

For some places, risk calculations can change overnight. In Hong Kong,
officials banned daytime dining in restaurants last month, only to
\href{https://www.nytimes3xbfgragh.onion/2020/07/31/world/asia/hongkong-dining-lunch-coronavirus.html}{reverse
themselves a day later} after an outcry. Schools in some cities are
opening and closing like screen doors in summer.

In many areas where the virus has retreated and then resurged, the
future looks like a long, complicated haul. Leaders are reaching for
their own metaphors to try to explain it.

\includegraphics{https://static01.graylady3jvrrxbe.onion/images/2020/08/04/world/04virus-melbourne-2/04virus-melbourne-2-articleLarge.jpg?quality=75\&auto=webp\&disable=upscale}

In California, Gov. Gavin Newsom has compared his opening and shutting
of businesses to a
\href{https://www.nytimes3xbfgragh.onion/2020/06/30/us/coronavirus-california-cases-rising.html}{``dimmer
switch.''} Dan Andrews, the premier in Victoria, the state of which
Melbourne is the capital, has repeatedly referred to ``pilot light
mode'' for industries like construction and meatpacking, which have been
ordered to temporarily reduce their work forces.

Whatever the metaphor, the situation is bleak.

In Melbourne, a city of five million that is considered a capital of
food and culture, traffic data showed people
\href{https://www.theage.com.au/national/victoria/lockdown-2-0-traffic-worse-compared-with-april-but-less-than-pre-covid-20200801-p55hlq.html}{driving
more in July} than they had during the first Stage 3 lockdown, in March
and April. Even worse, almost nine out of 10 people with Covid-19 had
not been tested or isolated when they first felt sick, Mr. Andrews, the
state's top leader, said in late July. And 53 percent had not
quarantined while waiting for their test results.

``That means people have felt unwell and just gone about their
business,'' Mr. Andrews said.

He made face masks mandatory the next day, on July 22.

Still, infections have continued to rise. They peaked at 753 new cases
on July 30, and have hovered around 500 a day ever since, with the death
toll in Victoria now standing at 147, after 11 deaths were recorded on
Monday.

Those figures have paved the way for a Stage 4 lockdown --- what
officials are calling a ``shock and awe'' attack on the virus --- that
will last at least six weeks.

Image

People exercising at Albert Park Lake in Melbourne on
Tuesday.~Credit...Daniel Pockett/Getty Images

Overwhelming force, with precision, seems to be the goal. The chief
modelers of the pandemic response in Australia have found that the virus
can be suppressed only if more than 70 percent of the population abides
by social distancing guidelines and other public health rules.

Mr. Andrews said the new restrictions would take 250,000 more people out
of their routines, in the hopes of reaching the necessary threshold.

So retail stores will be closed. Schools will return to at-home
instruction. Restaurants will be takeout or delivery only. Child-care
centers will be available only for permitted workers.

Those restrictions are already well understood. The rules requiring more
explanation are tied to the curfew and industries that have to cut back.

Large-scale construction projects of more than three stories, for
example, will have to reduce their on-site work force by 75 percent, and
workers will not be able to work at more than one location. Small-scale
construction cannot have more than five workers.

All of which sounds clear. But does a bathroom renovation, for example,
amount to home building in an apartment with one bathroom? And what
about fixing things that break, like Ms. Von Tessle Roberts's washing
machine?

Some businesses, like cleaning services, are already emailing customers
to say they think they can do some work, for people who pay through
welfare or who need help for mental health reasons. But, like many
others, they are still seeking official clarification.

Image

The premier of Victoria, Daniel Andrews, has the difficult job of
explaining the many coronavirus restrictions to the
public.Credit...David Crosling/EPA, via Shutterstock

Mr. Andrews, a Labor politician sometimes described as
\href{https://www.theage.com.au/national/victoria/the-strange-feeling-i-have-when-i-watch-daniel-andrews-20200710-p55av5.html}{awkward}
and paternal, has become the dad everyone needs answers from. He now
oversees, under the lockdown rules, what may be the country's most
intrusive bureaucracy since its days as a penal colony.

On Tuesday, he answered questions from reporters about dog-walking
(allowed after curfew, sort of, only near home) and other subjects of
great confusion at a news conference in Melbourne.

\href{https://www.nytimes3xbfgragh.onion/news-event/coronavirus?action=click\&pgtype=Article\&state=default\&region=MAIN_CONTENT_3\&context=storylines_faq}{}

\hypertarget{the-coronavirus-outbreak-}{%
\subsubsection{The Coronavirus Outbreak
›}\label{the-coronavirus-outbreak-}}

\hypertarget{frequently-asked-questions}{%
\paragraph{Frequently Asked
Questions}\label{frequently-asked-questions}}

Updated August 4, 2020

\begin{itemize}
\item ~
  \hypertarget{i-have-antibodies-am-i-now-immune}{%
  \paragraph{I have antibodies. Am I now
  immune?}\label{i-have-antibodies-am-i-now-immune}}

  \begin{itemize}
  \tightlist
  \item
    As of right
    now,\href{https://www.nytimes3xbfgragh.onion/2020/07/22/health/covid-antibodies-herd-immunity.html?action=click\&pgtype=Article\&state=default\&region=MAIN_CONTENT_3\&context=storylines_faq}{that
    seems likely, for at least several months.} There have been
    frightening accounts of people suffering what seems to be a second
    bout of Covid-19. But experts say these patients may have a
    drawn-out course of infection, with the virus taking a slow toll
    weeks to months after initial exposure. People infected with the
    coronavirus typically
    \href{https://www.nature.com/articles/s41586-020-2456-9}{produce}
    immune molecules called antibodies, which are
    \href{https://www.nytimes3xbfgragh.onion/2020/05/07/health/coronavirus-antibody-prevalence.html?action=click\&pgtype=Article\&state=default\&region=MAIN_CONTENT_3\&context=storylines_faq}{protective
    proteins made in response to an
    infection}\href{https://www.nytimes3xbfgragh.onion/2020/05/07/health/coronavirus-antibody-prevalence.html?action=click\&pgtype=Article\&state=default\&region=MAIN_CONTENT_3\&context=storylines_faq}{.
    These antibodies may} last in the body
    \href{https://www.nature.com/articles/s41591-020-0965-6}{only two to
    three months}, which may seem worrisome, but that's perfectly normal
    after an acute infection subsides, said Dr. Michael Mina, an
    immunologist at Harvard University. It may be possible to get the
    coronavirus again, but it's highly unlikely that it would be
    possible in a short window of time from initial infection or make
    people sicker the second time.
  \end{itemize}
\item ~
  \hypertarget{im-a-small-business-owner-can-i-get-relief}{%
  \paragraph{I'm a small-business owner. Can I get
  relief?}\label{im-a-small-business-owner-can-i-get-relief}}

  \begin{itemize}
  \tightlist
  \item
    The
    \href{https://www.nytimes3xbfgragh.onion/article/small-business-loans-stimulus-grants-freelancers-coronavirus.html?action=click\&pgtype=Article\&state=default\&region=MAIN_CONTENT_3\&context=storylines_faq}{stimulus
    bills enacted in March} offer help for the millions of American
    small businesses. Those eligible for aid are businesses and
    nonprofit organizations with fewer than 500 workers, including sole
    proprietorships, independent contractors and freelancers. Some
    larger companies in some industries are also eligible. The help
    being offered, which is being managed by the Small Business
    Administration, includes the Paycheck Protection Program and the
    Economic Injury Disaster Loan program. But lots of folks have
    \href{https://www.nytimes3xbfgragh.onion/interactive/2020/05/07/business/small-business-loans-coronavirus.html?action=click\&pgtype=Article\&state=default\&region=MAIN_CONTENT_3\&context=storylines_faq}{not
    yet seen payouts.} Even those who have received help are confused:
    The rules are draconian, and some are stuck sitting on
    \href{https://www.nytimes3xbfgragh.onion/2020/05/02/business/economy/loans-coronavirus-small-business.html?action=click\&pgtype=Article\&state=default\&region=MAIN_CONTENT_3\&context=storylines_faq}{money
    they don't know how to use.} Many small-business owners are getting
    less than they expected or
    \href{https://www.nytimes3xbfgragh.onion/2020/06/10/business/Small-business-loans-ppp.html?action=click\&pgtype=Article\&state=default\&region=MAIN_CONTENT_3\&context=storylines_faq}{not
    hearing anything at all.}
  \end{itemize}
\item ~
  \hypertarget{what-are-my-rights-if-i-am-worried-about-going-back-to-work}{%
  \paragraph{What are my rights if I am worried about going back to
  work?}\label{what-are-my-rights-if-i-am-worried-about-going-back-to-work}}

  \begin{itemize}
  \tightlist
  \item
    Employers have to provide
    \href{https://www.osha.gov/SLTC/covid-19/standards.html}{a safe
    workplace} with policies that protect everyone equally.
    \href{https://www.nytimes3xbfgragh.onion/article/coronavirus-money-unemployment.html?action=click\&pgtype=Article\&state=default\&region=MAIN_CONTENT_3\&context=storylines_faq}{And
    if one of your co-workers tests positive for the coronavirus, the
    C.D.C.} has said that
    \href{https://www.cdc.gov/coronavirus/2019-ncov/community/guidance-business-response.html}{employers
    should tell their employees} -\/- without giving you the sick
    employee's name -\/- that they may have been exposed to the virus.
  \end{itemize}
\item ~
  \hypertarget{should-i-refinance-my-mortgage}{%
  \paragraph{Should I refinance my
  mortgage?}\label{should-i-refinance-my-mortgage}}

  \begin{itemize}
  \tightlist
  \item
    \href{https://www.nytimes3xbfgragh.onion/article/coronavirus-money-unemployment.html?action=click\&pgtype=Article\&state=default\&region=MAIN_CONTENT_3\&context=storylines_faq}{It
    could be a good idea,} because mortgage rates have
    \href{https://www.nytimes3xbfgragh.onion/2020/07/16/business/mortgage-rates-below-3-percent.html?action=click\&pgtype=Article\&state=default\&region=MAIN_CONTENT_3\&context=storylines_faq}{never
    been lower.} Refinancing requests have pushed mortgage applications
    to some of the highest levels since 2008, so be prepared to get in
    line. But defaults are also up, so if you're thinking about buying a
    home, be aware that some lenders have tightened their standards.
  \end{itemize}
\item ~
  \hypertarget{what-is-school-going-to-look-like-in-september}{%
  \paragraph{What is school going to look like in
  September?}\label{what-is-school-going-to-look-like-in-september}}

  \begin{itemize}
  \tightlist
  \item
    It is unlikely that many schools will return to a normal schedule
    this fall, requiring the grind of
    \href{https://www.nytimes3xbfgragh.onion/2020/06/05/us/coronavirus-education-lost-learning.html?action=click\&pgtype=Article\&state=default\&region=MAIN_CONTENT_3\&context=storylines_faq}{online
    learning},
    \href{https://www.nytimes3xbfgragh.onion/2020/05/29/us/coronavirus-child-care-centers.html?action=click\&pgtype=Article\&state=default\&region=MAIN_CONTENT_3\&context=storylines_faq}{makeshift
    child care} and
    \href{https://www.nytimes3xbfgragh.onion/2020/06/03/business/economy/coronavirus-working-women.html?action=click\&pgtype=Article\&state=default\&region=MAIN_CONTENT_3\&context=storylines_faq}{stunted
    workdays} to continue. California's two largest public school
    districts --- Los Angeles and San Diego --- said on July 13, that
    \href{https://www.nytimes3xbfgragh.onion/2020/07/13/us/lausd-san-diego-school-reopening.html?action=click\&pgtype=Article\&state=default\&region=MAIN_CONTENT_3\&context=storylines_faq}{instruction
    will be remote-only in the fall}, citing concerns that surging
    coronavirus infections in their areas pose too dire a risk for
    students and teachers. Together, the two districts enroll some
    825,000 students. They are the largest in the country so far to
    abandon plans for even a partial physical return to classrooms when
    they reopen in August. For other districts, the solution won't be an
    all-or-nothing approach.
    \href{https://bioethics.jhu.edu/research-and-outreach/projects/eschool-initiative/school-policy-tracker/}{Many
    systems}, including the nation's largest, New York City, are
    devising
    \href{https://www.nytimes3xbfgragh.onion/2020/06/26/us/coronavirus-schools-reopen-fall.html?action=click\&pgtype=Article\&state=default\&region=MAIN_CONTENT_3\&context=storylines_faq}{hybrid
    plans} that involve spending some days in classrooms and other days
    online. There's no national policy on this yet, so check with your
    municipal school system regularly to see what is happening in your
    community.
  \end{itemize}
\end{itemize}

Thanking those who complied with the new rules and scolding those who
did not, he announced that no one in self-isolation would now be allowed
to exercise outdoors. A door-knocking campaign to check in on 3,000
people who had Covid-19 found that 800 of them were not at home.

All 800 have been referred to the Victoria police for investigation. The
fine for violators going forward, he said, will be 4,957 Australian
dollars, \$3,532.

Working, even legally, will also become trickier. Other than, say,
hospital workers with formal identification, everyone traveling for a
job deemed essential during the lockdown must carry a formal document
--- a work permit signed by the employer and employee.

For Cara Devine, who works at a wine store that closes at 8 p.m., that
means carrying a government form with her everywhere, and hoping that
the police recognize her task as essential when she heads home after the
curfew. But she also worried about the Uber drivers who take her back
and forth.

Image

Shoppers outside a Melbourne Costco on Sunday.Credit...William
West/Agence France-Presse --- Getty Images

``Even before the newest restrictions, I've had two Uber drivers being
really late picking up from the shop because they got stopped by the
police, taking about an hour out of their work time,'' she said.

The police are already confronting opposition. On at least four
occasions in the last week, they reported having to smash the windows of
cars and pull people out after they refused to provide a name and
address at a police checkpoint. The Victoria police commissioner, Shane
Patton, said a 38-year-old woman had also been charged with assault
after attacking a police officer who had stopped her for not wearing a
face mask.

Some criminologists are questioning whether the harsher enforcement will
help. Mostly, though, Melburnians are just trying to endure.

Walking to get groceries, Peter Barnes, 56, said he welcomed the
stricter rules, though he admitted his city was starting to feel like
George Orwell's ``1984,'' with the heavy hand of the state around every
corner.

Those focused solely on the economics, he said, should remember the
obvious: ``You can't hire a corpse. Very bad employment prospects for
people who are dead.''

By Monday night, the city seemed to be in listening mode. The streets
were emptying out, silent in hibernation.

``It's like a Sunday in the 1950s,'' said Mark Rubbo, the owner of
Readings, Melbourne's largest independent bookstore. He also noted that
people were stocking up again on books through online orders, with a
memoir called
\href{https://www.readings.com.au/products/32836563/the-happiest-man-on-earth}{``The
Happiest Man on Earth,''} about a Holocaust survivor, becoming a runaway
hit.

Ms. Von Tessle Roberts has found another solution, perhaps just as
likely to grow in popularity:
\href{https://www.facebookcorewwwi.onion/events/2738223623055680/}{Stand
on your front porch and scream}. That's the name she has given to an
event she posted on Facebook, set for Friday at 7 p.m. By Tuesday
afternoon, 70,000 people had expressed an interest in joining her
collective shout in anguish.

``Yelling is great,'' she said. ``It's less dehydrating than crying.''

Image

Melbourne's central business district on Sunday night.~Credit...Erik
Anderson/EPA, via Shutterstock

\begin{center}\rule{0.5\linewidth}{\linethickness}\end{center}

Besha Rodell and Yan Zhuang contributed reporting from Melbourne, and
Livia Albeck-Ripka from Cairns, Australia.

Advertisement

\protect\hyperlink{after-bottom}{Continue reading the main story}

\hypertarget{site-index}{%
\subsection{Site Index}\label{site-index}}

\hypertarget{site-information-navigation}{%
\subsection{Site Information
Navigation}\label{site-information-navigation}}

\begin{itemize}
\tightlist
\item
  \href{https://help.nytimes3xbfgragh.onion/hc/en-us/articles/115014792127-Copyright-notice}{©~2020~The
  New York Times Company}
\end{itemize}

\begin{itemize}
\tightlist
\item
  \href{https://www.nytco.com/}{NYTCo}
\item
  \href{https://help.nytimes3xbfgragh.onion/hc/en-us/articles/115015385887-Contact-Us}{Contact
  Us}
\item
  \href{https://www.nytco.com/careers/}{Work with us}
\item
  \href{https://nytmediakit.com/}{Advertise}
\item
  \href{http://www.tbrandstudio.com/}{T Brand Studio}
\item
  \href{https://www.nytimes3xbfgragh.onion/privacy/cookie-policy\#how-do-i-manage-trackers}{Your
  Ad Choices}
\item
  \href{https://www.nytimes3xbfgragh.onion/privacy}{Privacy}
\item
  \href{https://help.nytimes3xbfgragh.onion/hc/en-us/articles/115014893428-Terms-of-service}{Terms
  of Service}
\item
  \href{https://help.nytimes3xbfgragh.onion/hc/en-us/articles/115014893968-Terms-of-sale}{Terms
  of Sale}
\item
  \href{https://spiderbites.nytimes3xbfgragh.onion}{Site Map}
\item
  \href{https://help.nytimes3xbfgragh.onion/hc/en-us}{Help}
\item
  \href{https://www.nytimes3xbfgragh.onion/subscription?campaignId=37WXW}{Subscriptions}
\end{itemize}
