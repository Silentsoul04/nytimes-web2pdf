Sections

SEARCH

\protect\hyperlink{site-content}{Skip to
content}\protect\hyperlink{site-index}{Skip to site index}

\href{https://www.nytimes3xbfgragh.onion/section/world/australia}{Australia}

\href{https://myaccount.nytimes3xbfgragh.onion/auth/login?response_type=cookie\&client_id=vi}{}

\href{https://www.nytimes3xbfgragh.onion/section/todayspaper}{Today's
Paper}

\href{/section/world/australia}{Australia}\textbar{}3 Men Marooned in
the Pacific Are Rescued After Writing SOS in the Sand

\href{https://nyti.ms/2XmQFa0}{https://nyti.ms/2XmQFa0}

\begin{itemize}
\item
\item
\item
\item
\item
\end{itemize}

Advertisement

\protect\hyperlink{after-top}{Continue reading the main story}

Supported by

\protect\hyperlink{after-sponsor}{Continue reading the main story}

\hypertarget{3-men-marooned-in-the-pacific-are-rescued-after-writing-sos-in-the-sand}{%
\section{3 Men Marooned in the Pacific Are Rescued After Writing SOS in
the
Sand}\label{3-men-marooned-in-the-pacific-are-rescued-after-writing-sos-in-the-sand}}

Three days after their boat ran out of fuel and drifted off course in
Micronesia, the men were found in good condition after a plane saw their
giant plea for help spelled out on the sand.

\includegraphics{https://static01.graylady3jvrrxbe.onion/images/2020/08/04/world/04sos-pacific-1/04sos-pacific-1-videoSixteenByNine3000.jpg}

\href{https://www.nytimes3xbfgragh.onion/by/elian-peltier}{\includegraphics{https://static01.graylady3jvrrxbe.onion/images/2019/07/03/reader-center/author-elian-peltier/165383d8b7284129a185b6ca96e2a52e-thumbLarge.png}}

By \href{https://www.nytimes3xbfgragh.onion/by/elian-peltier}{Elian
Peltier}

\begin{itemize}
\item
  Aug. 4, 2020
\item
  \begin{itemize}
  \item
  \item
  \item
  \item
  \item
  \end{itemize}
\end{itemize}

Here's a useful tip: If you ever find yourself stuck on an uninhabited
island in the Pacific, it turns out that writing SOS in giant letters on
the sand works.

At least, it did this past weekend for three men whose small boat had
run out of fuel and drifted off course among the hundreds of islands and
atolls of Micronesia.

The men, said to be mariners from the Micronesian island of Pulap, were
reported missing on Friday after setting off for home from the Puluwat
atoll, about 25 miles away, a day earlier. In a joint operation,
aircraft dispatched by the American authorities in Guam and a ship sent
by the Australian military combed the area in a search for the sailors,
\href{https://www.dvidshub.net/news/375188/coast-guard-partners-rescue-three-stranded-mariners-island-federated-states-micronesia}{the
U.S. Coast Guard said}.

On Sunday afternoon, one of the American aircraft was finishing the
final leg of the day's patrol when crew members saw the scrawled letters
and a blue-and-white vessel on the sand of a tiny uninhabited atoll
called Pikelot. Lt. Col. Jason Palmeira-Yen, the pilot of the aircraft,
said he was reaching the end of the planned search grid when he turned
the plane to avoid a rain shower.

``That's when we looked down and saw an island, so we decide to check it
out and that's when we saw SOS and a boat right next to it on the
beach,'' he said.

An Australian reconnaissance helicopter later delivered food and water
to the marooned men, before a Micronesian vessel picked them up on
Monday and took them back home to Pulap.

The authorities said that the men's 23-foot-long boat had drifted off
course after it ran out of fuel, and eventually reached Pikelot, about
118 miles west of their departure point on Puluwat.

\includegraphics{https://static01.graylady3jvrrxbe.onion/images/2020/08/04/world/04sos-pacific-2/merlin_175285194_7f8d682d-9e8c-4d4f-aa24-7f9b9f3010eb-articleLarge.jpg?quality=75\&auto=webp\&disable=upscale}

Capt. Terry Morrison, the commanding officer of the Canberra, the
Australian military ship that helped search for the men, said, ``I am
proud of the response and professionalism of all on board as we fulfill
our obligation to contribute to the safety of life at sea wherever we
are in the world,'' according to
\href{https://news.defence.gov.au/international/hmas-canberra-assists-search-and-rescue}{a
statement from the Australian Defense Ministry}.

Writing huge letters in the sand has proved helpful to travelers
stranded in the Pacific in the past. In 2016, three men whose boat was
overturned in Micronesian waters swam two miles to reach a tiny island,
from which they were
\href{https://twitter.com/USCGHawaiiPac/status/718703857837019137?ref_src=twsrc\%5Etfw\%7Ctwcamp\%5Etweetembed\%7Ctwterm\%5E718703857837019137\%7Ctwgr\%5E\&ref_url=https\%3A\%2F\%2Fwww.theguardian.com\%2Fus-news\%2F2016\%2Fapr\%2F09\%2Fmen-rescued-help-palm-leaves-island-pacific}{rescued
by the U.S. Coast Guard} after writing ``HELP'' in the sand.

``This isn't the set of Castaway\ldots'' the U.S. Coast Guard Hawaii
Pacific
\href{https://www.facebookcorewwwi.onion/USCG.Hawaii.Pacific/posts/986794331411733}{wrote
at the time}, a reference to the movie in which Tom Hanks plays a man
stranded on an uninhabited island after a plane crash.

\href{https://www.bbc.com/news/world-asia-37203796}{Later that year},
two other people who had been missing for a week were rescued from a
Micronesian island after they wrote SOS in the sand.

The American and Australian authorities said cooperation had been
crucial to the weekend rescue.

``Through coordination with multiple response organizations, we were
able to save three members of our community and bring them back home to
their families,'' said Capt. Christopher Chase, commander of the U.S.
Coast Guard Sector Guam.

Advertisement

\protect\hyperlink{after-bottom}{Continue reading the main story}

\hypertarget{site-index}{%
\subsection{Site Index}\label{site-index}}

\hypertarget{site-information-navigation}{%
\subsection{Site Information
Navigation}\label{site-information-navigation}}

\begin{itemize}
\tightlist
\item
  \href{https://help.nytimes3xbfgragh.onion/hc/en-us/articles/115014792127-Copyright-notice}{©~2020~The
  New York Times Company}
\end{itemize}

\begin{itemize}
\tightlist
\item
  \href{https://www.nytco.com/}{NYTCo}
\item
  \href{https://help.nytimes3xbfgragh.onion/hc/en-us/articles/115015385887-Contact-Us}{Contact
  Us}
\item
  \href{https://www.nytco.com/careers/}{Work with us}
\item
  \href{https://nytmediakit.com/}{Advertise}
\item
  \href{http://www.tbrandstudio.com/}{T Brand Studio}
\item
  \href{https://www.nytimes3xbfgragh.onion/privacy/cookie-policy\#how-do-i-manage-trackers}{Your
  Ad Choices}
\item
  \href{https://www.nytimes3xbfgragh.onion/privacy}{Privacy}
\item
  \href{https://help.nytimes3xbfgragh.onion/hc/en-us/articles/115014893428-Terms-of-service}{Terms
  of Service}
\item
  \href{https://help.nytimes3xbfgragh.onion/hc/en-us/articles/115014893968-Terms-of-sale}{Terms
  of Sale}
\item
  \href{https://spiderbites.nytimes3xbfgragh.onion}{Site Map}
\item
  \href{https://help.nytimes3xbfgragh.onion/hc/en-us}{Help}
\item
  \href{https://www.nytimes3xbfgragh.onion/subscription?campaignId=37WXW}{Subscriptions}
\end{itemize}
