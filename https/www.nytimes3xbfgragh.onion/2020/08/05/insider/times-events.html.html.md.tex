Sections

SEARCH

\protect\hyperlink{site-content}{Skip to
content}\protect\hyperlink{site-index}{Skip to site index}

\href{https://www.nytimes3xbfgragh.onion/section/reader-center}{Times
Insider}

\href{https://myaccount.nytimes3xbfgragh.onion/auth/login?response_type=cookie\&client_id=vi}{}

\href{https://www.nytimes3xbfgragh.onion/section/todayspaper}{Today's
Paper}

\href{/section/reader-center}{Times Insider}\textbar{}Taking Times
Journalism to Another Stage

\url{https://nyti.ms/2PmcZfE}

\begin{itemize}
\item
\item
\item
\item
\item
\item
\end{itemize}

Advertisement

\protect\hyperlink{after-top}{Continue reading the main story}

Supported by

\protect\hyperlink{after-sponsor}{Continue reading the main story}

Times Insider

\hypertarget{taking-times-journalism-to-another-stage}{%
\section{Taking Times Journalism to Another
Stage}\label{taking-times-journalism-to-another-stage}}

With online programming that has included virtual workouts, a discussion
on protesters and a night with Broadway stars, the Times Events team has
expanded its scope to reach new audiences.

\includegraphics{https://static01.graylady3jvrrxbe.onion/images/2020/08/05/insider/05insider/05insider-articleLarge.jpg?quality=75\&auto=webp\&disable=upscale}

By \href{https://www.nytimes3xbfgragh.onion/by/nancy-coleman}{Nancy
Coleman}

\begin{itemize}
\item
  Aug. 5, 2020, 2:24 a.m. ET
\item
  \begin{itemize}
  \item
  \item
  \item
  \item
  \item
  \item
  \end{itemize}
\end{itemize}

\href{https://www.nytimes3xbfgragh.onion/series/times-insider}{\emph{Times
Insider}} \emph{explains who we are and what we do, and delivers
behind-the-scenes insights into how our journalism comes together.}

Taking in stories from The New York Times usually means combing through
the website or poring over the paper. But it can also mean
\href{https://timesevents.nytimes3xbfgragh.onion/thejoyofgymlessworkouts}{working
out} with a fitness trainer, watching
\href{https://timesevents.nytimes3xbfgragh.onion/kidscooking}{Ted Allen
cook a meal} or hearing Patti LuPone belt out a Broadway tune.

The Times has long supplemented its journalism with in-person events:
live interviews and performances with cultural mainstays; festivals and
conferences on food or business; and conversations about the news.

But since mid-March, when the pandemic began to shut down large
gatherings and performances across New York and elsewhere, the
\href{https://timesevents.nytimes3xbfgragh.onion/}{Times Events} team
has increased and broadened its offerings. The number of events --- all
of which are now virtual --- jumped from three or four a month before
the pandemic to, at times, that many in a single day. On Thursday, one
event will focus on the
\href{https://timesevents.nytimes3xbfgragh.onion/dealbookdebrief0806}{call
for racial justice in corporate America}; another will be an
\href{https://timesevents.nytimes3xbfgragh.onion/astorywritingclasswithcurtissittenfeld}{interactive
class} on writing short fiction.

``The pandemic has offered challenges, but when approached the right
way, they become opportunities,'' said Beth Weinstein, the team's senior
director of programming. ``We've taken on a lot. I think we pushed the
team really hard to offer a lot of events. I think we were trying to see
what was possible.''

Many of the events in the team's new digital-focused chapter, by nature
of the moment, have pivoted to subjects and activities surrounding the
coronavirus crisis:
\href{https://timesevents.nytimes3xbfgragh.onion/parenting}{parenting in
quarantine}, meditating on YouTube with Jewel, a philosophy-based
discussion titled, ``Can We Laugh Yet?''

Others have centered on non-pandemic news, like conversations about
nationwide protests and unrest. There have been sessions on cooking at
home and discussions on
\href{https://timesevents.nytimes3xbfgragh.onion/bookreviewlive}{summer
reading lists}. And new series have emerged, including a continuing
exploration from the Culture desk on Broadway's interrupted season.

``Talking heads are a thing of the past,'' Ms. Weinstein said. ``We
really want our events to be an energetic representation --- a living
representation --- of The New York Times. **** We want them to spark
conversation, spark imagination, to be interdisciplinary, to be
surprising, to be interactive in the way that the paper has become.''

Those digital programming efforts have enabled the team to find new
audiences during a time when people are isolated and looking to connect
with others. One of the last in-person events in early March
accommodated around 250 people, with hundreds more on the wait list, Ms.
Weinstein said. Now there is room for thousands of others --- especially
audiences joining from outside New York or even outside the country ---
to watch online.

``There are a lot of people who don't necessarily think that The New
York Times is for them,'' Ms. Weinstein said. ``The events do a really
great job at building relationships, at creating intimacy.''

The programming team --- a group in the newsroom consisting of Claudine
Hutton, Rachel Karpf and Kelsey Fowler in addition to Ms. Weinstein ---
collaborates with journalists in different departments to talk about
stories and ideas they want to convey, and how best to communicate those
to an audience. Once those events begin to take shape, the group works
with The Times's business operations to execute the programs. The
planning process varies widely --- some events take weeks; others are
reactive to the news and can come together in days.

In a way, Ms. Weinstein said, the conversion to a completely digital
slate of events has allowed for more experimentation in how the team
presents stories. As protests against racism reached a fever pitch in
early June, three reporters covering race across the country convened
for a virtual panel,
\href{https://timesevents.nytimes3xbfgragh.onion/race}{``America,
Inflamed.''} The format lent itself to more of a multimedia production:
The event included a clip from a related Visual Investigations piece as
well as photographs from several cities.

``Our stories about the unrest and protests took readers inside an
emotionally charged line of coverage,'' said Jamie Stockwell, a deputy
National editor at The Times who hosted the event. ``Bringing it to life
in a panel discussion was an immediate way to go behind the coverage and
into the personalities, ambitions and thoughts the reporters have when
out in the field, but more important, to truly have a conversation with
our readers.''

The team will most likely move away from the pace it set early in the
pandemic, Ms. Weinstein said, to focus on fewer, bigger events and
possibly a virtual festival. One upcoming
\href{https://timesevents.nytimes3xbfgragh.onion/thejoyofgymlessworkouts}{event
on Aug. 18} --- the 100th anniversary of women earning the right to vote
--- will present a new play online, commissioned by The Times and
adapted from the book
\href{https://www.nytimes3xbfgragh.onion/2020/07/24/books/finish-the-fight-excerpt.html}{``Finish
the Fight!''} by Times journalists.

``We're trying to create expressions of New York Times journalism ---
whether that's theatrical journalism, or whether that's continuing the
story,'' Ms. Weinstein said. ``Sometimes it's entirely appropriate for
it to be two people sitting in conversation. But sometimes a little
extra is needed to bring an idea to life.''

Advertisement

\protect\hyperlink{after-bottom}{Continue reading the main story}

\hypertarget{site-index}{%
\subsection{Site Index}\label{site-index}}

\hypertarget{site-information-navigation}{%
\subsection{Site Information
Navigation}\label{site-information-navigation}}

\begin{itemize}
\tightlist
\item
  \href{https://help.nytimes3xbfgragh.onion/hc/en-us/articles/115014792127-Copyright-notice}{©~2020~The
  New York Times Company}
\end{itemize}

\begin{itemize}
\tightlist
\item
  \href{https://www.nytco.com/}{NYTCo}
\item
  \href{https://help.nytimes3xbfgragh.onion/hc/en-us/articles/115015385887-Contact-Us}{Contact
  Us}
\item
  \href{https://www.nytco.com/careers/}{Work with us}
\item
  \href{https://nytmediakit.com/}{Advertise}
\item
  \href{http://www.tbrandstudio.com/}{T Brand Studio}
\item
  \href{https://www.nytimes3xbfgragh.onion/privacy/cookie-policy\#how-do-i-manage-trackers}{Your
  Ad Choices}
\item
  \href{https://www.nytimes3xbfgragh.onion/privacy}{Privacy}
\item
  \href{https://help.nytimes3xbfgragh.onion/hc/en-us/articles/115014893428-Terms-of-service}{Terms
  of Service}
\item
  \href{https://help.nytimes3xbfgragh.onion/hc/en-us/articles/115014893968-Terms-of-sale}{Terms
  of Sale}
\item
  \href{https://spiderbites.nytimes3xbfgragh.onion}{Site Map}
\item
  \href{https://help.nytimes3xbfgragh.onion/hc/en-us}{Help}
\item
  \href{https://www.nytimes3xbfgragh.onion/subscription?campaignId=37WXW}{Subscriptions}
\end{itemize}
