\href{/section/sports/golf}{Golf}\textbar{}A Memorable Golf Course
Honors a Forgettable President

\url{https://nyti.ms/30r5HNZ}

\begin{itemize}
\item
\item
\item
\item
\item
\end{itemize}

\includegraphics{https://static01.graylady3jvrrxbe.onion/images/2020/08/05/sports/05golf-harding-1/merlin_175174263_f9fc6aeb-cdd4-439e-b116-6ef492df98d3-articleLarge.jpg?quality=75\&auto=webp\&disable=upscale}

Sections

\protect\hyperlink{site-content}{Skip to
content}\protect\hyperlink{site-index}{Skip to site index}

\hypertarget{a-memorable-golf-course-honors-a-forgettable-president}{%
\section{A Memorable Golf Course Honors a Forgettable
President}\label{a-memorable-golf-course-honors-a-forgettable-president}}

This week's P.G.A. Championship site is named for Warren G. Harding. The
course's reputation has fared better than his did.

Credit...Associated Press

Supported by

\protect\hyperlink{after-sponsor}{Continue reading the main story}

\href{https://www.nytimes3xbfgragh.onion/by/john-branch}{\includegraphics{https://static01.graylady3jvrrxbe.onion/images/2019/03/01/multimedia/author-john-branch/author-john-branch-thumbLarge.png}}

By \href{https://www.nytimes3xbfgragh.onion/by/john-branch}{John Branch}

\begin{itemize}
\item
  Aug. 5, 2020, 3:00 a.m. ET
\item
  \begin{itemize}
  \item
  \item
  \item
  \item
  \item
  \end{itemize}
\end{itemize}

SAN FRANCISCO --- Had
\href{https://www.whitehouse.gov/about-the-white-house/presidents/warren-g-harding/}{President
Warren G. Harding} not been bedridden in those midsummer days of 1923,
he might have left his eighth-floor suite at the Palace Hotel and headed
to the southwestern edge of the city, where a new golf course named Lake
Merced Golf Links was under construction.

Harding was a golfer, after all, and he had traveled thousands of miles
by train and ship on his summer-long ``Voyage of Understanding.'' It was
intended to be three months and 15,000 miles of publicity stops. He had
played golf only days before, in Vancouver, British Columbia.

But Harding never left the Palace Hotel alive again.

Whatever good will that he hoped to engender on his trip, whatever hopes
he had of winning re-election the next year, ended two and a half years
into his largely forgotten presidency.

``Most historians rank Harding as the worst of all American
presidents,'' according to the
\href{https://millercenter.org/about}{University of Virginia's Miller
Center}, a nonpartisan think tank devoted to presidential history.

But Harding was at the right place at the right time to be linked,
improbably, to something as distant as a major professional golf
tournament in 2020.

The first major men's golf championship of the coronavirus era has
arrived, belatedly, in the form of this week's P.G.A. Championship --- a
fan-free, made-for-television event scheduled to begin Thursday,
starring Tiger Woods, Rory McIlroy, the two-time defending champion
Brooks Koepka and the world No. 1 Justin Thomas, among about 150 others.

Yet the name that promises to be uttered most across the global airwaves
this week is Harding.

\includegraphics{https://static01.graylady3jvrrxbe.onion/images/2020/08/05/sports/055golf-harding-6-02/055golf-harding-6-02-articleLarge.jpg?quality=75\&auto=webp\&disable=upscale}

The golf course being constructed seven miles from the president's
deathbed 97 years ago was soon christened Harding Park, during a bygone
era when naming things for presidents was done with little debate or
consideration.

Harding Park became one of golf's great, enduring layouts, curiously
named for the most disrespected of American presidents.

That this most somber of summers should be linked, even tenuously, to
Harding and the strangeness of 1923 feels about right.

Harding, a former Republican senator from Ohio, had little in the way of
a platform when he ran for the White House in 1920 other than a ``return
to normalcy.''
\href{https://www.whitehouse.gov/about-the-white-house/presidents/warren-g-harding/}{He
was viewed}as a tax-cutting, anti-immigrant nationalist who, in the wake
of World War I, did not want the United States to be part of the postwar
League of Nations. He was rumored to
\href{https://www.nytimes3xbfgragh.onion/2014/07/13/magazine/letters-warren-g-harding.html}{have
had affairs} (and at least
\href{https://www.nytimes3xbfgragh.onion/2015/08/13/us/dna-is-said-to-solve-a-mystery-of-warren-hardings-love-life.html}{one
out-of-wedlock child}) and was soon surrounded by scandal throughout his
administration. (Curious? Start by Googling ``Teapot Dome.'') He was not
considered a deep thinker and was prone to rambling. He liked to play
golf.

Image

President Harding, left, was an avid golfer who played his final holes
in Vancouver shortly before his death.Credit...Associated Press

Still, he and his running mate, Calvin Coolidge, won 60.3 percent of the
popular vote. The Democratic candidate, James M. Cox (with Franklin D.
Roosevelt as the vice-presidential nominee), earned just 34.1 percent.

Harding's administration coincided with a boom in American golf and
golf-course construction. The game's popularity hit hard in San
Francisco, which had deemed its small, peninsula setting too valuable to
hold cemeteries
(\href{https://www.nytimes3xbfgragh.onion/2016/02/06/sports/football/the-town-of-colma-where-san-franciscos-dead-live.html}{it
had been kicking out the dead for years}), but perfect for golf,
especially amid the oceanside dunes near Lake Merced.

Alister MacKenzie, the famed Scottish course architect, called it ``the
finest golfing territory I have seen in America.''

Several prestigious (and still private) courses were constructed near
one another within a few years, including the
\href{https://www.linksmagazine.com/classic-course-san-francisco-golf-club/}{San
Francisco Golf Club}, \href{https://www.calclub.org/\#course}{California
Golf Club of San Francisco} (which later moved a few miles south) and
\href{https://www.lmgc.org/Default.aspx?p=dynamicmodule\&pageid=106\&ssid=100120\&vnf=1\#:~:text=Founded\%20in\%201922\%2C\%20the\%20club,met\%20to\%20begin\%20organizational\%20efforts.}{Lake
Merced Golf Club}.

Image

Harding Park in 1938.Credit...Bo Links

\href{https://www.olyclub.com/golf-course-tours/}{The Olympic Club},
which has held five United States Opens and is the scheduled site of the
2028 P.G.A. Championship and the 2032 Ryder Cup, opened two courses in
1924.

Across Lake Merced from the Olympic Club, on what might be the best
property of them all, a design by Willie Watson and Sam Whiting was
nearly two years from opening when President Harding's train rolled into
San Francisco in the summer of 1923.

About six weeks after Harding's death, San Francisco's Harding Memorial
Committee decided that the best way to honor the president was with a
half-built golf course. Lake Merced Golf Links became Harding Park.

``Nothing could be more appropriate as a tribute to President Harding,
as he was a great lover of outdoor recreation,'' Herbert Fleishhacker, a
prominent businessman and the head of the city's parks, said at the
time.

The course built a better reputation. It quickly played host to national
amateur championships and became a home course for the
\href{http://sfgolfchampionship.com/about-the-san-francisco-city-championship/}{San
Francisco City Championship}, whose winners included Ken Venturi, whose
parents ran the Harding Park pro shop for years. Harding became a
regular stop on the P.G.A. Tour in the 1960s. Winners included Venturi,
Gary Player, Billy Casper, Gene Littler and Chi-Chi Rodriguez.

But Harding Park's prestige was frayed by municipal budgets and neglect.
The P.G.A. Tour left, and for most of the next 30 years, the course was
loved for its history more than its condition. During the 1998 U.S. Open
at the Olympic Club, Harding Park's fairways served as parking lots.

That embarrassing fate stirred a push
to\href{https://www.golfdigest.com/story/golf_harding_park_jaime_diaz}{rescue
the course's faded glory}, led by Sandy Tatum, a San Francisco lawyer
and former U.S.G.A. president.

Image

Arnold Palmer at the 1963 San Francisco International Open at Harding
Park. Within a decade, the course was falling into
disrepair.Credit...Associated Press

Rehabilitated and reopened in 2003, and still a municipal course
(September tee times for city residents can be had for as little as
\$64), Harding Park attracted big events again. That included the
W.G.C.-American Express Championship in 2005 (won by
\href{https://www.youtube.com/watch?v=j6YSXKHxLYo}{Tiger Woods in a
playoff over John Daly}) and the Presidents Cup in 2009 (won by the
United States).

This week's P.G.A. Championship is Harding Park's first major. Among all
the famous names to be uttered worldwide during its four rounds will be
that of a long-ago president, probably without explanation or context.
Thankfully, the course is in much better shape than Harding was.

He had left Washington in June, in a 10-car train filled with cabinet
members, dignitaries, staff members, reporters and the first lady,
Florence Harding.

It was part campaign trip, part summer vacation. Harding's speech in St.
Louis was one of the first presidential addresses to be broadcast live
on radio. The train then rolled through Kansas City, Denver and Salt
Lake City, among other cities, as it carried the president west.
Newspapers covered every mile of it.

Image

Harding. left, and his wife, Florence, had a whirlwhind tour of the
West, including a visit with the governor of the Alaska Territory, Scott
C. Bone.Credit...Library of Congress/Corbis, via VCG, via Getty Images

Harding, 57, was a tireless tourist, a master of the photo opportunity.
He visited Yellowstone and Zion national parks, where hemorrhoids made
misery of a horseback ride. He visited a mine, drove a wheat binder,
attended a commemoration of the Oregon Trail and, eventually, boarded a
U.S. Navy transport ship in Seattle
\href{http://www.sitnews.us/Kiffer/PresidentialVisitor/081615_harding.html}{to
tour the Alaskan territory}.

He gleefully watched as the ship's gunners fired shells into the Taku
Glacier to create giant splashes from chunks of falling ice.

Then everything turned south.

Harding may have been sickened by bad crab in Sitka. He complained
privately of abdominal pain and may have caught pneumonia. His golf
outing in Vancouver was limited to a few holes. (His visit made a
lasting impact; there remains a
\href{https://www.vancouverhistory.ca/archives_harding.htm}{Harding
memorial in the city's Stanley Park}.)

Ominously, the Navy ship accidentally rammed an American destroyer in
Puget Sound fog. Harding mustered the energy to give a speech in
Seattle, but a weekend in Portland, Ore., was scrapped so the entourage
could hurry to San Francisco to get the president medical attention and
rest.

Image

Harding died days after he arrived in San Francisco.Credit...Underwood
Archives/Getty Images

Harding was taken to the
\href{https://www.marriott.com/hotels/travel/sfolc-palace-hotel-a-luxury-collection-hotel-san-francisco/}{Palace
Hotel} and placed in the presidential suite.

``He was weak and ill but his condition had improved to an encouraging
degree,'' The New York Times reported as Harding arrived on July 29.

The nation waited for news. Harding's doctor, a homeopath named Charles
Sawyer, provided updates. For a couple of days, Harding was in ``grave''
condition. Other physicians, including the Stanford president Ray Lyman
Wilbur, were called in.

``President Harding seems to have passed the crisis,'' The Times
reported atop its front page on Aug. 1. ``His physicians will not say
that he is out of danger, but even through their cautious comment it is
plainly perceptible that they believe that there is little chance of a
recurrence of the dangerous symptoms which threatened the life of the
President last night.''

That made the Aug. 3 headline in the San Francisco Chronicle so
stunning.

``HARDING DEAD'' it read in huge letters.

Florence Harding had been reading aloud to her husband an article from
The Saturday Evening Post about him entitled, ``A Calm View of a Calm
Man.''

``Oh, that's good,'' the president supposedly said about one passage.
``Go on.''

Those were his last words, apparently.

``Suddenly and without warning, a shudder passed over the President's
body,'' the Chronicle reported. ``He raised one arm, but not a word came
from his lips. The arm dropped back and the President lay still.''

Image

The President Harding Suite at the Palace Hotel. Harding died there in
1923.Credit...San Francisco Historical Photograph Collection/San
Francisco Public Library

Florence Harding screamed. Officials rushed in, led by the Secretary of
Commerce Herbert Hoover. Vice President Coolidge, at home in Vermont,
was sworn in overnight.

The cause of Harding's death was initially ruled ``apoplexy,'' but
rumors spread --- about poor care from his doctors, about his many other
ailments. Some
\href{https://www.mercurynews.com/2018/09/04/bay-area-history-on-95th-anniversary-of-president-warren-g-hardings-death-san-francisco-man-renews-story-of-poisoning/}{still
believe that the first lady poisoned her husband}, maybe as revenge for
his infidelities. She refused to allow an autopsy.

Most historians today think he had a heart attack. Harding's body was
taken from the Palace Hotel, placed aboard a train back to Washington,
D.C., and
\href{https://www.ohiohistory.org/visit/museum-and-site-locator/warren-g-harding-home-memorial}{interred
in his hometown, Marion, Ohio}.

There are not a lot of things named for Harding these days --- schools,
mostly, and
\href{https://www.nytimes3xbfgragh.onion/2019/03/27/realestate/harding-township-nj-a-historic-place-that-feels-like-the-country.html}{Harding
Township}, N.J., among them --- though there is a
\href{https://www.golf.lacity.org/course_harding/}{Harding Golf Course
in Los Angeles} that opened in 1923. (It is next to Wilson Golf Course,
opened in 1927, and named for Harding's presidential predecessor, who
died about six months after Harding.)

In San Francisco, the biggest recognition of Harding's place in city
history is at a now-famous golf course that he never played or visited.
But it does help keep his name alive, rarely more than this week.

Advertisement

\protect\hyperlink{after-bottom}{Continue reading the main story}

\hypertarget{site-index}{%
\subsection{Site Index}\label{site-index}}

\hypertarget{site-information-navigation}{%
\subsection{Site Information
Navigation}\label{site-information-navigation}}

\begin{itemize}
\tightlist
\item
  \href{https://help.nytimes3xbfgragh.onion/hc/en-us/articles/115014792127-Copyright-notice}{©~2020~The
  New York Times Company}
\end{itemize}

\begin{itemize}
\tightlist
\item
  \href{https://www.nytco.com/}{NYTCo}
\item
  \href{https://help.nytimes3xbfgragh.onion/hc/en-us/articles/115015385887-Contact-Us}{Contact
  Us}
\item
  \href{https://www.nytco.com/careers/}{Work with us}
\item
  \href{https://nytmediakit.com/}{Advertise}
\item
  \href{http://www.tbrandstudio.com/}{T Brand Studio}
\item
  \href{https://www.nytimes3xbfgragh.onion/privacy/cookie-policy\#how-do-i-manage-trackers}{Your
  Ad Choices}
\item
  \href{https://www.nytimes3xbfgragh.onion/privacy}{Privacy}
\item
  \href{https://help.nytimes3xbfgragh.onion/hc/en-us/articles/115014893428-Terms-of-service}{Terms
  of Service}
\item
  \href{https://help.nytimes3xbfgragh.onion/hc/en-us/articles/115014893968-Terms-of-sale}{Terms
  of Sale}
\item
  \href{https://spiderbites.nytimes3xbfgragh.onion}{Site Map}
\item
  \href{https://help.nytimes3xbfgragh.onion/hc/en-us}{Help}
\item
  \href{https://www.nytimes3xbfgragh.onion/subscription?campaignId=37WXW}{Subscriptions}
\end{itemize}
