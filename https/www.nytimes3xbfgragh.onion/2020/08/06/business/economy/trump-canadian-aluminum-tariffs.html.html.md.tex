Sections

SEARCH

\protect\hyperlink{site-content}{Skip to
content}\protect\hyperlink{site-index}{Skip to site index}

\href{https://www.nytimes3xbfgragh.onion/section/business/economy}{Economy}

\href{https://myaccount.nytimes3xbfgragh.onion/auth/login?response_type=cookie\&client_id=vi}{}

\href{https://www.nytimes3xbfgragh.onion/section/todayspaper}{Today's
Paper}

\href{/section/business/economy}{Economy}\textbar{}Trump Reinstates
Tariff on Canadian Aluminum

\url{https://nyti.ms/30BwBTw}

\begin{itemize}
\item
\item
\item
\item
\item
\end{itemize}

Advertisement

\protect\hyperlink{after-top}{Continue reading the main story}

Supported by

\protect\hyperlink{after-sponsor}{Continue reading the main story}

\hypertarget{trump-reinstates-tariff-on-canadian-aluminum}{%
\section{Trump Reinstates Tariff on Canadian
Aluminum}\label{trump-reinstates-tariff-on-canadian-aluminum}}

The tariff, which prompted dollar-for-dollar retaliation from Canada,
was imposed just one month after a new trade deal between the two
countries went into effect.

\includegraphics{https://static01.graylady3jvrrxbe.onion/images/2020/09/06/multimedia/06canadatarriffs-sub/06canadatarriffs-sub-articleLarge.jpg?quality=75\&auto=webp\&disable=upscale}

\href{https://www.nytimes3xbfgragh.onion/by/ana-swanson}{\includegraphics{https://static01.graylady3jvrrxbe.onion/images/2018/12/10/multimedia/author-ana-swanson/author-ana-swanson-thumbLarge.png}}\href{https://www.nytimes3xbfgragh.onion/by/ian-austen}{\includegraphics{https://static01.graylady3jvrrxbe.onion/images/2019/07/18/reader-center/author-ian-austen/author-ian-austen-thumbLarge.png}}

By \href{https://www.nytimes3xbfgragh.onion/by/ana-swanson}{Ana Swanson}
and \href{https://www.nytimes3xbfgragh.onion/by/ian-austen}{Ian Austen}

\begin{itemize}
\item
  Aug. 6, 2020
\item
  \begin{itemize}
  \item
  \item
  \item
  \item
  \item
  \end{itemize}
\end{itemize}

WASHINGTON --- President Trump announced Thursday that he was reimposing
a 10 percent tariff on Canadian aluminum to help struggling American
producers, a step that is likely to incite retaliation and worsen ties
with Canada just one month after the countries'
\href{https://www.nytimes3xbfgragh.onion/2020/07/01/business/economy/usmca-takes-effect.html}{new
trade deal went into effect}.

Speaking at a Whirlpool factory in Clyde, Ohio, Mr. Trump said that he
had signed a proclamation earlier on Thursday that would reimpose the
levy on Canada, accusing the country of ``taking advantage of us as
usual.''

``To be a strong nation, America must be a manufacturing nation and not
be led by a bunch of fools,'' the president said. ``That means
protecting our national industrial base.''

Mr. Trump imposed tariffs on steel and aluminum from
\href{https://www.nytimes3xbfgragh.onion/2018/05/31/us/politics/trump-aluminum-steel-tariffs.html}{Canada,
Mexico and the European Union in early 2018}, prompting those countries
to respond with their own tariffs on American goods. The levies on
imports from Canada and Mexico were not lifted
\href{https://www.nytimes3xbfgragh.onion/2019/05/17/us/politics/china-auto-tariffs-donald-trump.html}{until
the following year,} when the countries reached an agreement as part of
the negotiations toward a new North American trade deal.

But the United States retained the right to reinstate them if it
observed a spike in metal imports, which Mr. Trump cited on Thursday.

``My administration agreed to lift those tariffs in return for a promise
from the Canadian government that its aluminum industry would not flood
our country with exports and kill all our aluminum jobs, which is
exactly what they did,'' Mr. Trump said Thursday. ``Canadian aluminum
producers have broken that commitment.''

On Thursday evening, Prime Minister Justin Trudeau announced Canada's
\href{https://twitter.com/JustinTrudeau/status/1291521260044931073}{response
via Twitter}. ``In response to the American tariffs announced today,
Canada will impose countermeasures that will include dollar-for-dollar
retaliatory tariffs,'' he wrote. ``We will always stand up for our
aluminum workers. We did so in 2018 and we will stand up for them again
now.''

The deputy prime minister, Chrystia Freeland, issued a
\href{https://deputypm.canada.ca/en/news/statements/2020/08/06/statement-deputy-prime-minister-us-tariffs-imports-canadian-aluminum}{pointed
statement}, as well. ``In the time of a global pandemic and an economic
crisis,'' Ms. Freeland said, ``the last thing Canadian and American
workers need is new tariffs that will raise costs for manufacturers and
consumers, impede the free flow of trade, and hurt provincial and state
economies.''

She also rejected Mr. Trump's national security justification for the
measure. ``Canadian aluminum strengthens U.S. national security and has
done so for decades through unparalleled cooperation between our two
countries,'' she said in the statement.

For months, American and Canadian officials
\href{https://www.nytimes3xbfgragh.onion/2020/06/23/business/economy/usmca-canada-aluminum-tariffs.html}{have
debated} whether Canada's rising imports violate that agreement or
constitute a surge. Imports of Canadian aluminum have risen since the
tariffs were lifted last year, but they remain below levels seen within
the last few years.

The American aluminum industry has struggled to compete in recent years
with producers in countries like China, Russia, Iceland, the United Arab
Emirates and Canada that offer generous state subsidies or benefit from
cheap electricity. Today, only a handful of American aluminum smelters,
which make raw aluminum out of bauxite, still operate.

Supporters of the tariffs say that they have helped to revive American
production, but that imports from Canada and the economic slump that
accompanied the pandemic had once again thrown the industry into
disarray. In April, the aluminum giant Alcoa idled a smelter in
Ferndale, Wash., saying that production there was ``uncompetitive.''

Two American companies with domestic aluminum capacity, Century Aluminum
and Magnitude 7 Metals, have lobbied intensely for the tariffs to be
reimposed. In a statement Thursday, Michael Bless, the chief executive
of Century Aluminum, said the move ``demonstrates this administration's
continued dedication to restoring the U.S. aluminum industry'' and
``helps to secure continued domestic production of this vital strategic
material.''

But the rest of the aluminum industry, which has operations spread
around the globe, including in Canada, has fought against the measure.
The multitude of industries that use aluminum to make products including
cars, beer cans and washing machines, have also argued against the
levies, saying they increase their costs and make their products less
competitive globally. Even Whirlpool, the appliance maker where Mr.
Trump made his announcement on Thursday, has seen its
\href{https://www.wsj.com/articles/whirlpool-wanted-washing-machine-tariffs-it-didnt-plan-for-a-trade-war-1531757621}{costs
for raw materials rise} as a result of the metal levies.

In June, executives from more than 15 of the world's
\href{https://www.nytimes3xbfgragh.onion/2020/06/23/business/economy/usmca-canada-aluminum-tariffs.html}{largest
aluminum companies}, including Alcoa, Constellium and Novelis, sent a
letter to the Trump administration arguing against the tariffs.

``Fully 97 percent of U.S. aluminum industry jobs are in
mid-and-downstream production and processing,'' the letter read. ``These
jobs depend on a mix of domestic and imported primary aluminum,
including from countries like Canada.''

Jim McGreevy, the chief executive of the Beer Institute, a trade
association of beer producers and importers, said his group strongly
opposed the decision.

``Since the implementation of aluminum tariffs in 2018, the American
beverage industries have paid more than \$582 million in tariffs,'' he
said. ``Increased aluminum premiums due to tariffs increase the cost of
beer production and force brewers to make difficult business decisions
--- especially amidst a global pandemic that has reduced overall sales
while simultaneously increasing demand for aluminum cans.''

In a statement, Myron Brilliant, the executive vice president of the
U.S. Chamber of Commerce, called the move ``a step in the wrong
direction'' and urged the administration to reconsider.

``These tariffs will raise costs for American manufacturers, are opposed
by most U.S. aluminum producers and will draw retaliation against U.S.
exports --- just as they did before,'' he said.

Mr. Trump's metal tariffs have been imposed with the legal rationale of
protecting American national security. That has particularly irked
Canadians, who count themselves as America's closest ally.

``It's totally misguided, it's the wrong thing at the wrong time,'' said
Jean Simard, the president and chief executive of the Aluminium
Association of Canada. ``We're still in a Covid-related downturn.''

Mr. Simard said that shipments of basic aluminum ingots to the United
States from Canada had risen after automakers and other importers of
more sophisticated aluminum closed their factories because of the
pandemic. But he said that with renewed manufacturing in the United
States, the market was rebalancing.

According to Mr. Simard's group, exports of basic aluminum ingots from
Canada declined 16 percent in June and fell 40 percent last month.

The tariff will most likely cause American industries that use aluminum
to turn to China and Russia for their supplies, Mr. Simard said.
Currently, according to Mr. Simard, the United States aluminum
production capacity can meet only about one-sixth of the country's
consumption of the metal.

The Canadian aluminum industry, Mr. Simard said, will push the Canadian
government to apply tariffs on American-made products made from
aluminum, including beer cans and bicycles, for which there are Canadian
substitutes.

``We can drink Canadian beer out of Canadian cans,'' Mr. Simard said.

Advertisement

\protect\hyperlink{after-bottom}{Continue reading the main story}

\hypertarget{site-index}{%
\subsection{Site Index}\label{site-index}}

\hypertarget{site-information-navigation}{%
\subsection{Site Information
Navigation}\label{site-information-navigation}}

\begin{itemize}
\tightlist
\item
  \href{https://help.nytimes3xbfgragh.onion/hc/en-us/articles/115014792127-Copyright-notice}{©~2020~The
  New York Times Company}
\end{itemize}

\begin{itemize}
\tightlist
\item
  \href{https://www.nytco.com/}{NYTCo}
\item
  \href{https://help.nytimes3xbfgragh.onion/hc/en-us/articles/115015385887-Contact-Us}{Contact
  Us}
\item
  \href{https://www.nytco.com/careers/}{Work with us}
\item
  \href{https://nytmediakit.com/}{Advertise}
\item
  \href{http://www.tbrandstudio.com/}{T Brand Studio}
\item
  \href{https://www.nytimes3xbfgragh.onion/privacy/cookie-policy\#how-do-i-manage-trackers}{Your
  Ad Choices}
\item
  \href{https://www.nytimes3xbfgragh.onion/privacy}{Privacy}
\item
  \href{https://help.nytimes3xbfgragh.onion/hc/en-us/articles/115014893428-Terms-of-service}{Terms
  of Service}
\item
  \href{https://help.nytimes3xbfgragh.onion/hc/en-us/articles/115014893968-Terms-of-sale}{Terms
  of Sale}
\item
  \href{https://spiderbites.nytimes3xbfgragh.onion}{Site Map}
\item
  \href{https://help.nytimes3xbfgragh.onion/hc/en-us}{Help}
\item
  \href{https://www.nytimes3xbfgragh.onion/subscription?campaignId=37WXW}{Subscriptions}
\end{itemize}
