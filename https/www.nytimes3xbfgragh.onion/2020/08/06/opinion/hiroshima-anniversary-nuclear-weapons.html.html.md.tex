Sections

SEARCH

\protect\hyperlink{site-content}{Skip to
content}\protect\hyperlink{site-index}{Skip to site index}

\href{https://myaccount.nytimes3xbfgragh.onion/auth/login?response_type=cookie\&client_id=vi}{}

\href{https://www.nytimes3xbfgragh.onion/section/todayspaper}{Today's
Paper}

\href{/section/opinion}{Opinion}\textbar{}The World Can Still Be
Destroyed in a Flash

\href{https://nyti.ms/3gFsmvC}{https://nyti.ms/3gFsmvC}

\begin{itemize}
\item
\item
\item
\item
\item
\item
\end{itemize}

Advertisement

\protect\hyperlink{after-top}{Continue reading the main story}

\href{/section/opinion}{Opinion}

Supported by

\protect\hyperlink{after-sponsor}{Continue reading the main story}

\hypertarget{the-world-can-still-be-destroyed-in-a-flash}{%
\section{The World Can Still Be Destroyed in a
Flash}\label{the-world-can-still-be-destroyed-in-a-flash}}

It seems that the United States is plunging into a new arms race without
learning the lessons of the last.

By
\href{https://www.nytimes3xbfgragh.onion/interactive/opinion/editorialboard.html}{The
Editorial Board}

The editorial board is a group of opinion journalists whose views are
informed by expertise, research, debate and certain longstanding ****
\href{https://www.nytimes3xbfgragh.onion/interactive/2018/opinion/editorialboard.html}{values}.
It is separate from the newsroom.

\begin{itemize}
\item
  Aug. 6, 2020
\item
  \begin{itemize}
  \item
  \item
  \item
  \item
  \item
  \item
  \end{itemize}
\end{itemize}

\includegraphics{https://static01.graylady3jvrrxbe.onion/images/2020/08/06/opinion/06nuke-still/06nuke-still-mediumSquareAt3X.jpg}

The nuclear weapons dropped over
\href{https://www.nytimes3xbfgragh.onion/2020/08/06/world/asia/hiroshima-nagasaki-japan-photos.html}{Hiroshima
and Nagasaki} 75 years ago this week wreaked a devastation never before
seen in human warfare. Yet they were firecrackers compared with the
nuclear weapons that were soon developed --- bombs, warheads, shells,
torpedoes and other devices capable of vaporizing the human race in an
apocalyptic flash.

For decades, that thought cast a pall of acute anxiety over America and
the world. Whether because of that fear, a strategy of effective
deterrence, chance or all the above, the United States remains the only
country to have used nuclear weapons in combat. With the end of the Cold
War, anxiety around nuclear war has receded. Most people probably are
not aware that a harrowing and expensive new arms race is now underway.

Today Americans are more likely to identify climate change as the
greatest man-made threat to the planet. Last year, in the
\href{https://www.chapman.edu/wilkinson/research-centers/babbie-center/survey-american-fears.aspx\#:~:text=Results\%20from\%20the\%202019\%20Chapman,warfare\%20for\%20the\%202019\%20survey.}{list
of what Americans fear} compiled annually by Chapman University, ``North
Korea using nuclear weapons'' and ``Nuclear weapons attack'' ranked 27
and 29, far below ``Corrupt government officials'' (No. 1) or
``Pollution of oceans, rivers and lakes'' (No. 2).

Yet even with the Cold War long over and stockpiles of nuclear weapons
in the Russian and American arsenals sharply reduced through a series of
nuclear arms treaties, to fewer than 6,000 warheads each, there are no
grounds for complacency. The world can still be destroyed in a flash.

Nine states have nuclear weapons --- the United States, Russia, Britain,
France, China, India, Pakistan, Israel and North Korea.
\href{https://www.nytimes3xbfgragh.onion/2020/08/01/world/asia/iran-khamenei-us-sanctions.html}{Iran's
nuclear program} has been the focus of intense concern for years, and
Saudi Arabia has vowed that if Iran develops a nuclear weapon, it will
follow suit. Consider also that two men have the power to unleash a
nuclear barrage entirely on their own --- President Trump and Vladimir
Putin, the Russian president, who are both working assiduously on
modernizing their arsenals.

Mr. Trump has
\href{https://foreignpolicy.com/2020/07/23/trump-china-russia-new-arms-control-agreement-start/}{said
he is working} on a new arms control agreement with Russia and is
seeking to include China in the talks. But his administration has always
found it easier to tear up treaties than to sign them, especially if the
result in any way restrains the United States. As the special envoy for
arms control,
\href{https://www.reuters.com/article/uk-usa-armscontrol/u-s-prepared-to-spend-russia-china-into-oblivion-to-win-nuclear-arms-race-u-s-envoy-idUSKBN22X2LS}{Marshall
Billingslea}, boasted in May, ``We know how to win these races, and we
know how to spend the adversary into oblivion.''

Before the coronavirus pandemic put millions of Americans out of work,
spending so much money on new doomsday weapons was profligate. Now, it
seems morally indefensible. This week, the
\href{https://homelandprepnews.com/stories/53270-report-examines-nuclear-weapons-modernization-costs/}{Government
Accountability Office} said that, without changes, the Pentagon's
nuclear weapons modernization effort is on track to surpass its \$1.2
trillion price tag over the next three decades. It seems as though the
United States is plunging into a new nuclear arms race with Russia and
China without having learned the lessons of the last one.

When briefed by the military in 2017 on the levels to which American and
Russian nuclear arsenals had been reduced through arms treaties, Mr.
Trump
\href{https://www.cnbc.com/2017/10/11/trump-reportedly-wanted-nearly-10-times-more-nuclear-weapons.html}{reportedly}
demanded that the United States increase its nuclear stockpile tenfold.
According to some reports, this was what prompted the secretary of state
at the time, Rex Tillerson, to call the president a ``moron.''

Mr. Trump
\href{https://www.nytimes3xbfgragh.onion/2019/02/01/us/politics/trump-inf-nuclear-treaty.html}{withdrew
the United States} from the Intermediate-Range Nuclear Forces Treaty and
the Iran nuclear deal, and he has not yet extended the New START accord,
the only agreement still in place limiting American and Russian nuclear
forces, which was signed by President Barack Obama and expires in early
February. In addition, the Trump administration was
\href{https://www.cnn.com/2020/06/24/politics/trump-administration-nuclear-test/index.html}{recently
reported} to be thinking of breaking the 28-year-old moratorium on
nuclear testing.

The 75th anniversary of Hiroshima is a good time to revive serious
public concern about nuclear weapons. The pandemic may leave little room
for other fears, but public health and economic recovery should not have
to compete for resources with a needless and enormously expensive new
arms race. As Jessica Mathews, former president of the Carnegie
Endowment for International Peace, writes in the current issue of
\href{https://www.nybooks.com/contributors/jessica-t-mathews/}{The New
York Review of Books}, it would be good for the five original nuclear
powers --- the United States, Russia, Britain, France and China --- to
formally endorse the principle set forth by Presidents Ronald Reagan and
Mikhail Gorbachev at their 1985 summit, that ``a nuclear war cannot be
won and must never be fought.''

Above all, the wrenching images of scorched rubble where Hiroshima had
stood ought to be cause for serious reflection on what nuclear weapons
do --- and what they cannot do.

\emph{The Times is committed to publishing}
\href{https://www.nytimes3xbfgragh.onion/2019/01/31/opinion/letters/letters-to-editor-new-york-times-women.html}{\emph{a
diversity of letters}} \emph{to the editor. We'd like to hear what you
think about this or any of our articles. Here are some}
\href{https://help.nytimes3xbfgragh.onion/hc/en-us/articles/115014925288-How-to-submit-a-letter-to-the-editor}{\emph{tips}}\emph{.
And here's our email:}
\href{mailto:letters@NYTimes.com}{\emph{letters@NYTimes.com}}\emph{.}

\emph{Follow The New York Times Opinion section on}
\href{https://www.facebookcorewwwi.onion/nytopinion}{\emph{Facebook}}\emph{,}
\href{http://twitter.com/NYTOpinion}{\emph{Twitter (@NYTopinion)}}
\emph{and}
\href{https://www.instagram.com/nytopinion/}{\emph{Instagram}}\emph{.}

Advertisement

\protect\hyperlink{after-bottom}{Continue reading the main story}

\hypertarget{site-index}{%
\subsection{Site Index}\label{site-index}}

\hypertarget{site-information-navigation}{%
\subsection{Site Information
Navigation}\label{site-information-navigation}}

\begin{itemize}
\tightlist
\item
  \href{https://help.nytimes3xbfgragh.onion/hc/en-us/articles/115014792127-Copyright-notice}{©~2020~The
  New York Times Company}
\end{itemize}

\begin{itemize}
\tightlist
\item
  \href{https://www.nytco.com/}{NYTCo}
\item
  \href{https://help.nytimes3xbfgragh.onion/hc/en-us/articles/115015385887-Contact-Us}{Contact
  Us}
\item
  \href{https://www.nytco.com/careers/}{Work with us}
\item
  \href{https://nytmediakit.com/}{Advertise}
\item
  \href{http://www.tbrandstudio.com/}{T Brand Studio}
\item
  \href{https://www.nytimes3xbfgragh.onion/privacy/cookie-policy\#how-do-i-manage-trackers}{Your
  Ad Choices}
\item
  \href{https://www.nytimes3xbfgragh.onion/privacy}{Privacy}
\item
  \href{https://help.nytimes3xbfgragh.onion/hc/en-us/articles/115014893428-Terms-of-service}{Terms
  of Service}
\item
  \href{https://help.nytimes3xbfgragh.onion/hc/en-us/articles/115014893968-Terms-of-sale}{Terms
  of Sale}
\item
  \href{https://spiderbites.nytimes3xbfgragh.onion}{Site Map}
\item
  \href{https://help.nytimes3xbfgragh.onion/hc/en-us}{Help}
\item
  \href{https://www.nytimes3xbfgragh.onion/subscription?campaignId=37WXW}{Subscriptions}
\end{itemize}
