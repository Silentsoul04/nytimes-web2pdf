\href{/section/opinion/sunday}{Sunday Review}\textbar{}`Racism Makes a
Liar of God'

\url{https://nyti.ms/30AeehG}

\begin{itemize}
\item
\item
\item
\item
\item
\item
\end{itemize}

\includegraphics{https://static01.graylady3jvrrxbe.onion/images/2020/08/09/opinion/05liz2/merlin_175181958_f5f8f51f-d21a-422e-a97a-13ae8be1bba6-articleLarge.jpg?quality=75\&auto=webp\&disable=upscale}

Sections

\protect\hyperlink{site-content}{Skip to
content}\protect\hyperlink{site-index}{Skip to site index}

\href{/section/opinion}{Opinion}

\hypertarget{racism-makes-a-liar-of-god}{%
\section{`Racism Makes a Liar of
God'}\label{racism-makes-a-liar-of-god}}

How the American Catholic Church is wrestling with the Black Lives
Matter movement.

Gloria Purvis hosts a popular Catholic radio show, ``Morning Glory,''
and a limited television series, ``Authentically Free at
Last.''Credit...Johnathon Kelso for The New York Times

Supported by

\protect\hyperlink{after-sponsor}{Continue reading the main story}

\href{https://www.nytimes3xbfgragh.onion/by/elizabeth-bruenig}{\includegraphics{https://static01.graylady3jvrrxbe.onion/images/2020/01/24/opinion/elizabeth-bruenig/elizabeth-bruenig-thumbLarge.png}}

By
\href{https://www.nytimes3xbfgragh.onion/by/elizabeth-bruenig}{Elizabeth
Bruenig}

Ms. Bruenig is an Opinion writer.

\begin{itemize}
\item
  Aug. 6, 2020
\item
  \begin{itemize}
  \item
  \item
  \item
  \item
  \item
  \item
  \end{itemize}
\end{itemize}

In 1963, when 250,000 demonstrators gathered at the Lincoln Memorial and
heard the Rev. Dr. Martin Luther King Jr.'s ``I have a dream'' speech,
they did so under the prayerful invocation of Archbishop Patrick O'Boyle
of Washington. He called for the Holy Spirit to open the eyes of
Christians to the injustice of racial discrimination, condemned violence
and praised the activists who had possessed the courage to go forth,
like Moses, in search of a beautiful country.

Five decades later, these hopes seem in many respects unfulfilled. About
\href{https://www.pewforum.org/religious-landscape-study/}{one in five}
Americans identify as Catholic, and as of 2018,
roughly\href{https://www.prri.org/research/partisan-polarization-dominates-trump-era-findings-from-the-2018-american-values-survey/}{six
in 10}white Catholics felt that police killings of Black men were
isolated incidents rather than evidence of a profound and lethal bias.
Prominent Catholic commentators, including Bill O'Reilly and Father
Dwight Longenecker,
\href{https://www.billoreilly.com/b/Bills-Column-on-the-BLM-Organization:-Its-News-to-Us/-20554200728065000.html?dest=/mobile/blogdetail.jsp}{fear}
and
\href{https://www.crisismagazine.com/2020/blacks-lives-matter-but-so-does-the-truth}{reject}
the Black Lives Matter movement.

\includegraphics{https://static01.graylady3jvrrxbe.onion/images/2020/08/09/opinion/05liz1/merlin_174769998_30a6f755-c3c9-48f6-88b0-20695f9843fb-articleLarge.jpg?quality=75\&auto=webp\&disable=upscale}

American Catholic unease with Black Lives Matter has been particularly
noticeable during the protests over the killing of George Floyd. Statues
commemorating Junipero Serra, a Spanish monk responsible for founding
several of California's Catholic missions in the early days of European
colonization, have been torn down by protesters outraged by what they
say was Father Serra's eager participation in the conquest of North
America, including the
\href{https://www.nytimes3xbfgragh.onion/2015/09/30/us/attack-on-statue-of-new-saint-junipero-serra-digs-up-old-conflicts.html}{torture,
enslavement and murder} of some of the Native Americans he intended to
convert --- accusations disputed by many Catholics.

Other religious statues, too, have been damaged by protesters. Coupled
with the vandalism of a handful of Catholic churches along with a slew
of ordinary buildings, the attacks on statuary have sparked fury among
conservative Catholics, confirming what they perhaps already believed:
that racial justice movements --- or at least this particular one ---
are antithetical to the Christian faith, rooted in Marxism and atheism.

A Catholic anti-abortion activist, Abby Johnson,
\href{https://twitter.com/AbbyJohnson/status/1277590055490633730}{tweeted}
in June: ``The Catholic Church is burning. And everyday, liberal
Catholics continue to throw matches on Her with sacrilegious nonsense
like this,'' in reference to an icon showing Mr. Floyd as a Jesus
figure, dying in his mother's arms.

Andrew Sullivan, a Catholic writer,
\href{https://twitter.com/sullydish/status/1287057974834462720}{argued}
in July that Black Lives Matter and Christianity are ``fundamentally
incompatible world views.''

In a July 5
\href{https://dioceseofspokane.org/news/statement-on-catholic-charities-eastern-washington-the-church-and-racism}{statement},
Bishop Thomas A. Daly of Spokane, Wash., wrote: ``BLM is in conflict
with Church teaching regarding marriage, family and the sanctity of
life. Moreover, it is disturbing that BLM has not vocally condemned the
recent violence that has torn apart so many cities.''

Image

A cardboard cutout of Pope Francis at the Basilica of the National
Shrine of the Immaculate Conception in Washington.Credit...Johnathon
Kelso for The New York Times

Steady in the midst of this supposed conflict between faith and
anti-racism efforts is Gloria Purvis. She is a Black Catholic --- a
designation lonely enough even without intrafaith political strife, as
only
\href{https://www.pewforum.org/religious-landscape-study/religious-tradition/catholic/}{3
percent} of American Catholics are Black. Ms. Purvis co-hosts a popular
Catholic radio show, ``Morning Glory'', and a limited television series,
``Authentically Free at Last.''

After the murder of Mr. Floyd, Ms. Purvis denounced his killing and the
many killings of Black men and women by the police that had come before.

``I said I thought racism was demonic,'' she told me over a recent
dinner at a Washington bistro. In the weeks following Mr. Floyd's death,
``Morning Glory'' featured episodes devoted to saints who resisted
racism in their lifetimes, the impact of racial discrimination on
society at large and the reality of systemic racism itself.

Her comments set off a wave of recrimination via tweets and emails from
indignant listeners.

``Racism makes a liar of God,'' she told me. ``It says not everyone is
made in his image. What a horrible lie from the pit of hell.''

Image

A crucifix depicting a Black Jesus that belongs to Gloria
Purvis.Credit...Johnathon Kelso for The New York Times

Image

Prayer cards for sale in the gift shop at the Basilica of the National
Shrine of the Immaculate Conception in Washington.Credit...Johnathon
Kelso for The New York Times

Her radio program
was\href{https://osvnews.com/2020/06/26/ewtns-largest-radio-affiliate-drops-morning-glory-as-host-gloria-purvis-continues-to-speak-out-about-racism/}{dropped}
in June by Guadalupe Radio Network, a Catholic station based in Midland,
Texas. After outcry on social media, the network released
a\href{https://www.grnonline.com/en/events/southeast}{statement}
claiming that Ms. Purvis's show had temporarily been suspended not for
her remarks on racism but because the network had detected ``a spirit of
contention growing among the hosts.'' Guadalupe Radio Network did not
respond to a request for comment.

Ms. Purvis didn't buy the explanation: There had always been occasional,
friendly disagreements between the show's hosts, but it had never been
an issue before. Ms. Purvis told me the network has neither reinstated
her program nor offered any explanation of when or if it plans to air it
again. She still believes the show was suspended because of her explicit
condemnation of police killings of Black people and her impassioned
exhortations against racism.

I asked Ms. Purvis about the toppled statues and the church vandalism,
which have been
\href{https://newdailycompass.com/en/the-senseless-hatred-of-st-junipero-serra}{raised}\href{https://www.washingtontimes.com/news/2020/jul/15/black-lives-matter-protesters-turn-rage-churches-r/}{repeatedly}
as evidence of the imagined conflict between Christianity and today's
anti-racism movement.

She sighed. It isn't that she dismisses sacred sites or representations
of the saints; in fact, she told me, she credits a visit to the grotto
where Our Lady of Lourdes is believed to have appeared with the birth of
her daughter, after a 15-year struggle with infertility. And she was
present when Pope Francis canonized Father Serra during the pontiff's
first visit to the United States. But she wishes it were possible to
stipulate without incurring rancor that objects of piety have their
place in the order of things.

Image

Ms. Purvis wants to see a sincere confronting of anti-Black racism
within the Catholic Church.Credit...Johnathon Kelso for The New York
Times

``In the Catholic world, we're pro-life, right?'' she said. ``But we
were so quick to forget about a man killed in the street in favor of
things that can be rebuilt or replaced. This injustice that happened to
George Floyd seemed to evaporate as soon as money or property came into
it.''

After she spoke out about Mr. Floyd's death, Ms. Purvis was inundated
with videos sent by her fellow faithful, condemning Mr. Floyd with an
\href{https://www.snopes.com/news/2020/06/12/george-floyd-criminal-record/}{exaggerated
version} of his criminal record.

``I thought: Any Catholic who can watch that and not be bothered by it
is missing something in their faith,'' Ms. Purvis said. Mr. Floyd, she
said, ``had a right to life. But he also had a right to a natural
death.''

That this foundational principle could be overlooked in the name of
icons seemed to exhaust and dispirit her.

``I don't think a lot of people realize racism is a sin,'' she said.
``Having these discussions makes people uncomfortable.''

It should not be so difficult for so many Christians to affirm that yes,
Black lives matter, without conditions or complaints. ``We are being
called to love our neighbor,'' Ms. Purvis observed, ``and my God, my
God, we are failing.''

\includegraphics{https://static01.graylady3jvrrxbe.onion/images/2020/08/06/autossell/06Bruenig-twitter-thumb/06Bruenig-twitter-thumb-videoSixteenByNineJumbo1600.jpg}

Ms. Purvis maintains hope for the future. She wants to see a sincere
reckoning with anti-Black racism within the church. ``We need to name
it,'' she said, ``and say: Yes, we have sinned; yes, religious orders
owned slaves; we did not speak out in the abolition movement; we pushed
some people even in the celebration of Mass to the side or to the back,
so they could only receive our Lord when others were done.'' That much
and more is necessary.

This month, Americans will march on Washington in commemoration of the
original march on the capital for civil rights and in hopes of reviving
and redoubling efforts to achieve racial equality.

A diverse group of Catholics including clergy and laypeople --- myself
among them --- have
\href{https://www.catholicsocialaction.com/open_letter_to_all_bishops}{prepared
a letter} exhorting our bishops to join us at this march, to fulfill the
hope laid out for Christians in the first epistle of John: ``Let us
love, not in word or speech, but in truth and action.''

Elizabeth Bruenig (@ebruenig) is an Opinion writer.

\emph{The Times is committed to publishing}
\href{https://www.nytimes3xbfgragh.onion/2019/01/31/opinion/letters/letters-to-editor-new-york-times-women.html}{\emph{a
diversity of letters}} \emph{to the editor. We'd like to hear what you
think about this or any of our articles. Here are some}
\href{https://help.nytimes3xbfgragh.onion/hc/en-us/articles/115014925288-How-to-submit-a-letter-to-the-editor}{\emph{tips}}\emph{.
And here's our email:}
\href{mailto:letters@NYTimes.com}{\emph{letters@NYTimes.com}}\emph{.}

\emph{Follow The New York Times Opinion section on}
\href{https://www.facebookcorewwwi.onion/nytopinion}{\emph{Facebook}}\emph{,}
\href{http://twitter.com/NYTOpinion}{\emph{Twitter (@NYTopinion)}}
\emph{and}
\href{https://www.instagram.com/nytopinion/}{\emph{Instagram}}\emph{.}

Advertisement

\protect\hyperlink{after-bottom}{Continue reading the main story}

\hypertarget{site-index}{%
\subsection{Site Index}\label{site-index}}

\hypertarget{site-information-navigation}{%
\subsection{Site Information
Navigation}\label{site-information-navigation}}

\begin{itemize}
\tightlist
\item
  \href{https://help.nytimes3xbfgragh.onion/hc/en-us/articles/115014792127-Copyright-notice}{©~2020~The
  New York Times Company}
\end{itemize}

\begin{itemize}
\tightlist
\item
  \href{https://www.nytco.com/}{NYTCo}
\item
  \href{https://help.nytimes3xbfgragh.onion/hc/en-us/articles/115015385887-Contact-Us}{Contact
  Us}
\item
  \href{https://www.nytco.com/careers/}{Work with us}
\item
  \href{https://nytmediakit.com/}{Advertise}
\item
  \href{http://www.tbrandstudio.com/}{T Brand Studio}
\item
  \href{https://www.nytimes3xbfgragh.onion/privacy/cookie-policy\#how-do-i-manage-trackers}{Your
  Ad Choices}
\item
  \href{https://www.nytimes3xbfgragh.onion/privacy}{Privacy}
\item
  \href{https://help.nytimes3xbfgragh.onion/hc/en-us/articles/115014893428-Terms-of-service}{Terms
  of Service}
\item
  \href{https://help.nytimes3xbfgragh.onion/hc/en-us/articles/115014893968-Terms-of-sale}{Terms
  of Sale}
\item
  \href{https://spiderbites.nytimes3xbfgragh.onion}{Site Map}
\item
  \href{https://help.nytimes3xbfgragh.onion/hc/en-us}{Help}
\item
  \href{https://www.nytimes3xbfgragh.onion/subscription?campaignId=37WXW}{Subscriptions}
\end{itemize}
