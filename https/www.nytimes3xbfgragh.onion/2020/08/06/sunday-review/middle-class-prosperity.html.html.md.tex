Sections

SEARCH

\protect\hyperlink{site-content}{Skip to
content}\protect\hyperlink{site-index}{Skip to site index}

\href{https://www.nytimes3xbfgragh.onion/pages/opinion/index.html\#sundayreview}{Sunday
Review}

\href{https://myaccount.nytimes3xbfgragh.onion/auth/login?response_type=cookie\&client_id=vi}{}

\href{https://www.nytimes3xbfgragh.onion/section/todayspaper}{Today's
Paper}

\href{/pages/opinion/index.html}{Sunday Review}\textbar{}The Real Reason
the American Economy Boomed After World War II

\url{https://nyti.ms/2XGEu89}

\begin{itemize}
\item
\item
\item
\item
\item
\item
\end{itemize}

Advertisement

\protect\hyperlink{after-top}{Continue reading the main story}

Supported by

\protect\hyperlink{after-sponsor}{Continue reading the main story}

NEWS ANALYSIS

\hypertarget{the-real-reason-the-american-economy-boomed-after-world-war-ii}{%
\section{The Real Reason the American Economy Boomed After World War
II}\label{the-real-reason-the-american-economy-boomed-after-world-war-ii}}

How expanding opportunity for women, immigrants and nonwhite workers
helped everyone --- and why we need to do so again.

\includegraphics{https://static01.graylady3jvrrxbe.onion/images/2020/08/09/opinion/sunday/09tankersley/09tankersley-articleLarge.jpg?quality=75\&auto=webp\&disable=upscale}

\href{https://www.nytimes3xbfgragh.onion/by/jim-tankersley}{\includegraphics{https://static01.graylady3jvrrxbe.onion/images/2018/10/19/multimedia/author-jim-tankersley/author-jim-tankersley-thumbLarge.png}}

By \href{https://www.nytimes3xbfgragh.onion/by/jim-tankersley}{Jim
Tankersley}

Mr. Tankersley covers economic policy in the Washington bureau of The
Times.

\begin{itemize}
\item
  Aug. 6, 2020
\item
  \begin{itemize}
  \item
  \item
  \item
  \item
  \item
  \item
  \end{itemize}
\end{itemize}

The United States long reserved its most lucrative occupations for an
elite class of white men. Those men held power by selling everyone else
a myth: The biggest threat to workers like you are workers who do not
look like you. ** Again and again, they told working-class white men
that they were losing out on good jobs to women, nonwhite men and
immigrants.

It was, and remains, a politically potent lie. It is undercut by the
real story of how America engineered its Golden Era of shared prosperity
--- the great middle-class expansion in the decades after World War II.

Americans deserve to know the truth about that Golden Era, which was not
the whitewashed, ``Leave It to Beaver'' tale that so many people have
been led to believe. They deserve to know who built the middle class and
can actually rebuild it, for all workers, no matter their race or gender
or hometown.

We need to hear it now, as our nation is immersed in a pandemic
recession and a summer of protests demanding equality, and as American
workers struggle to shake off decades of sluggish wage growth. We need
to hear it because it is a beacon of hope in a bleak time for our
economy, but more important because the lies that elite white men peddle
about workers in conflict have made the economy worse for everyone, for
far too long.

The hopeful truth is that when Americans band together to force open the
gates of opportunity for women, for Black men, for the groups that have
long been oppressed in our economy, everyone gets ahead.

I have spent my career as an economics reporter consumed by the
questions of how America might revive the Golden Era of the middle class
that boomed after World War II. I have searched for the secret to
restoring prosperity for the sons of lumber-mill workers in my home
county, where the timber industry crashed in the 1980s, or the
burned-out factories along the Ohio River, where I chased politicians in
the early 2000s who were promising --- and failing --- to bring the good
jobs back.

The old jobs are not coming back. What I have learned over time is that
our best hope to create a new wave of good ones is to invest in the
groups of Americans who were responsible for the success of our economy
at the time it worked best for working people.

The economy thrived after World War II in large part because America
made it easier for people who had been previously shut out of economic
opportunity --- women, minority groups, immigrants --- to enter the work
force and climb the economic ladder, to make better use of their talents
and potential. In 1960, \href{http://klenow.com/HHJK.pdf}{cutting-edge
research} from economists at the University of Chicago and Stanford
University has documented, more than half of Black men in America worked
as janitors, freight handlers or something similar. Only 2 percent of
women and Black men worked in what economists call ``high-skill'' jobs
that pay high wages, like engineering or law. Ninety-four percent of
doctors in the United States were white men.

That disparity was by design. It protected white male elites. Everyone
else was barred entry to top professions by overt discrimination,
inequality of schooling, social convention and, often, the law itself.
They were devalued as humans and as workers. (Slavery was the greatest
devaluation, but the gates of opportunity remained closed to most
enslaved Americans and their descendants through Emancipation and its
aftermath.)

Women and nonwhite men gradually chipped away at those barriers, in fits
and starts. They seized opportunities, like a war effort creating a need
for workers to replace the men being sent abroad to fight. They
protested and bled and died for civil rights. And when they won
victories, it wasn't just for them, or even for people like them. They
generated economic gains that helped everyone.

The Chicago and Stanford economists calculated that the simple, radical
act of reducing discrimination against those groups was responsible for
more than 40 percent of the country's per-worker economic growth after
1960. It's the reason the country could sustain rapid growth with low
unemployment, yielding rising wages for everyone, including white men
without college degrees.

America's ruling elites did not learn from that success. The aggressive
expansion of opportunity that had driven economic gains was choked off
by a backlash to social progress in the 1970s and '80s. The white men
who ran the country declared victory over discrimination far too early,
consigning the economy to slower growth. Sustained shared prosperity was
replaced by widening inequality, lost jobs and decades of disappointing
income growth for workers of all races.

In important ways, much of the work of breaking down discrimination
stalled soon after the passage of the Civil Rights Act in 1964. ``It was
fundamentally over by the time of the Reagan presidency,'' William A.
Darity Jr., a Duke University economist who is one of his profession's
most accomplished researchers on racial discrimination, told me. Over
the past several decades, some barriers to advancement for women and
nonwhite men have grown back. New ones have grown up beside them.

A host of studies illustrate this. A
\href{http://web.utk.edu/~mwanamak/Intergen_NBER.pdf}{recent and
devastating one} is co-authored by a University of Tennessee economic
historian, Marianne Wanamaker, who served a year in the White House on
President Trump's Council of Economic Advisers. She and a co-worker went
back to Reconstruction and measured how much easier it was for the sons
of poor white men to climb the economic ladder than the sons of poor
Black men.

In terms of economic mobility, they found, the penalty for being born
Black is the same today as it was in the 1870s.

Women have made more progress in recent decades than Black men, but they
are nowhere close to equality. They still earn less for the same work,
and they are still blocked by harassment, discrimination and policies
from reaching the same heights as white men in many of America's most
important industries.

Take Silicon Valley. In 2018, venture capitalists in the United States
distributed \$131 billion to start-up businesses, hoping to seed the
next Google or Tesla. That money went to nearly 9,000 companies. Just
over 2 percent of them were founded entirely by women. Another 12
percent had at least one female founder. The rest, 86 percent, were
founded entirely by men.

The statistics show tragedy. They also show opportunity. If America can
once again tear down barriers to advancement, it can tap a geyser of
entrepreneurship, productivity and talent, which could by itself produce
the strong growth and low unemployment that historically drive up wages
for the working class, including working-class white men.

If you want to know where the new good jobs will come from --- those
that will help millions of Americans climb back into the middle class
--- this is where you should look, to the great untapped talent of
America's women, of its Black men, of the highly skilled immigrants that
study after study show to be catalysts of innovation and job creation.

That is not the appeal that populist politicians make to working-class
white men, who have been rocked by globalization and automation and the
greed of the governing class. But it should be.

All Americans have a stake in the protests for equality they see every
night on the news. Working-class white men, like the guys I went to high
school with, have a bond with the Black men, the immigrants and the
women of all races who have taken to the streets.

The real story of America today is this: If you want to restore the
greatness of an economy that doesn't work for you or your children the
way that it used to, those women and men are your best shot at
salvation. Their progress will lift you up.

Jim Tankersley covers economic policy in the Washington bureau of The
Times. He is the author of
``\href{https://www.publicaffairsbooks.com/titles/jim-tankersley/the-riches-of-this-land/9781541767836/}{The
Riches of This Land: The Untold, True Story of America's Middle
Class},'' from which this essay is adapted.

\emph{The Times is committed to publishing}
\href{https://www.nytimes3xbfgragh.onion/2019/01/31/opinion/letters/letters-to-editor-new-york-times-women.html}{\emph{a
diversity of letters}} \emph{to the editor. We'd like to hear what you
think about this or any of our articles. Here are some}
\href{https://help.nytimes3xbfgragh.onion/hc/en-us/articles/115014925288-How-to-submit-a-letter-to-the-editor}{\emph{tips}}\emph{.
And here's our email:}
\href{mailto:letters@NYTimes.com}{\emph{letters@NYTimes.com}}\emph{.}

\emph{Follow The New York Times Opinion section on}
\href{https://www.facebookcorewwwi.onion/nytopinion}{\emph{Facebook}}\emph{,}
\href{http://twitter.com/NYTOpinion}{\emph{Twitter (@NYTopinion)}}
\emph{and}
\href{https://www.instagram.com/nytopinion/}{\emph{Instagram}}\emph{.}

Advertisement

\protect\hyperlink{after-bottom}{Continue reading the main story}

\hypertarget{site-index}{%
\subsection{Site Index}\label{site-index}}

\hypertarget{site-information-navigation}{%
\subsection{Site Information
Navigation}\label{site-information-navigation}}

\begin{itemize}
\tightlist
\item
  \href{https://help.nytimes3xbfgragh.onion/hc/en-us/articles/115014792127-Copyright-notice}{©~2020~The
  New York Times Company}
\end{itemize}

\begin{itemize}
\tightlist
\item
  \href{https://www.nytco.com/}{NYTCo}
\item
  \href{https://help.nytimes3xbfgragh.onion/hc/en-us/articles/115015385887-Contact-Us}{Contact
  Us}
\item
  \href{https://www.nytco.com/careers/}{Work with us}
\item
  \href{https://nytmediakit.com/}{Advertise}
\item
  \href{http://www.tbrandstudio.com/}{T Brand Studio}
\item
  \href{https://www.nytimes3xbfgragh.onion/privacy/cookie-policy\#how-do-i-manage-trackers}{Your
  Ad Choices}
\item
  \href{https://www.nytimes3xbfgragh.onion/privacy}{Privacy}
\item
  \href{https://help.nytimes3xbfgragh.onion/hc/en-us/articles/115014893428-Terms-of-service}{Terms
  of Service}
\item
  \href{https://help.nytimes3xbfgragh.onion/hc/en-us/articles/115014893968-Terms-of-sale}{Terms
  of Sale}
\item
  \href{https://spiderbites.nytimes3xbfgragh.onion}{Site Map}
\item
  \href{https://help.nytimes3xbfgragh.onion/hc/en-us}{Help}
\item
  \href{https://www.nytimes3xbfgragh.onion/subscription?campaignId=37WXW}{Subscriptions}
\end{itemize}
