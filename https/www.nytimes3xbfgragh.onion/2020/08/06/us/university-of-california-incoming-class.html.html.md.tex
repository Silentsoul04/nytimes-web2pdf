Sections

SEARCH

\protect\hyperlink{site-content}{Skip to
content}\protect\hyperlink{site-index}{Skip to site index}

\href{https://www.nytimes3xbfgragh.onion/section/us}{U.S.}

\href{https://myaccount.nytimes3xbfgragh.onion/auth/login?response_type=cookie\&client_id=vi}{}

\href{https://www.nytimes3xbfgragh.onion/section/todayspaper}{Today's
Paper}

\href{/section/us}{U.S.}\textbar{}New Students Show Diversity of U.C.'s
Freshman Class

\url{https://nyti.ms/33wSTYg}

\begin{itemize}
\item
\item
\item
\item
\item
\end{itemize}

Advertisement

\protect\hyperlink{after-top}{Continue reading the main story}

Supported by

\protect\hyperlink{after-sponsor}{Continue reading the main story}

California Today

\hypertarget{new-students-show-diversity-of-ucs-freshman-class}{%
\section{New Students Show Diversity of U.C.'s Freshman
Class}\label{new-students-show-diversity-of-ucs-freshman-class}}

Thursday: Meet more students in the historic incoming class at the
University of California. Also: School reopening becomes more
complicated.

\href{https://www.nytimes3xbfgragh.onion/by/jill-cowan}{\includegraphics{https://static01.graylady3jvrrxbe.onion/images/2018/12/10/multimedia/author-jill-cowan/author-jill-cowan-thumbLarge.png}}

By \href{https://www.nytimes3xbfgragh.onion/by/jill-cowan}{Jill Cowan}

\begin{itemize}
\item
  Aug. 6, 2020
\item
  \begin{itemize}
  \item
  \item
  \item
  \item
  \item
  \end{itemize}
\end{itemize}

\includegraphics{https://static01.graylady3jvrrxbe.onion/images/2020/08/06/us/06daviscatoday/merlin_173814255_6b234d33-08c3-46a1-8392-ca3eb4f7ecdc-articleLarge.jpg?quality=75\&auto=webp\&disable=upscale}

\emph{Good morning.}

\textbf{For the first time in the university's history, Chicano or
Latino students made up the biggest group of Californians}
\textbf{\href{https://www.ucop.edu/institutional-research-academic-planning/_files/factsheets/2020/fall-2020-admission-table-2-1.pdf}{admitted
to the freshman class of the University of California, at 36 percent}.}

\textbf{Yesterday, we introduced you to two of those students. And
today, we're introducing you to a couple more:}

\hypertarget{erick-tobar}{%
\subsection{Erick Tobar}\label{erick-tobar}}

Image

A family photo. From left, Ashley Tobar, Juan Tobar, Kennedy Tobar,
Marta Batres, Dora Tobar, Marta Tobar and Erick Tobar.Credit...Dora
Tobar

Six years ago, Dora and Erick Tobar --- now 17 and 21 --- moved to the
Bay Area after living in Guatemala.

Their father, whom they lived with while their mother lived in the San
Rafael area and cleaned houses, had been killed not long before.

``It was the hardest day I ever lived,'' Mr. Tobar said. He remembers
vividly hearing gunshots as he headed to an uncle's house, followed by
the horror of learning that his father and uncle had been gunned down.

At first, the siblings stayed in Guatemala with relatives. But it became
too difficult, so their mother brought them to join her.

In the United States, Mr. Tobar said school was a top priority, but when
their mother had to stop working for health reasons, it was up to them
to keep the family afloat.

For Mr. Tobar, that meant working at a carwash and an elder care home
while finishing high school and, more recently, attending community
college.

\emph{{[}Read the}
\href{https://www.nytimes3xbfgragh.onion/2020/08/05/us/university-california-latino-students.html?smid=tw-share}{\emph{first
part of this series}}\emph{.{]}}

There wasn't much time to hang out with friends or for extracurricular
activities.

This year, Mr. Tobar will start school at U.C. Davis.

``I want to get into a U.C.,'' he recalled thinking, ``so I can prove to
myself I have the capacity to do it.''

He wants to become a certified public accountant. Eventually, he said,
he wants to continue on to graduate degrees and start businesses, like a
cafe with his brother.

His sister, Ms. Tobar, credited her older siblings for helping her take
advantage of opportunities to volunteer and travel.

In high school, Ms. Tobar visited Arizona for a solo backpacking trip
and helped build houses in the Dominican Republic.

She got into several colleges, including U.C. Merced. But Ms. Tobar will
instead attend San Francisco State University, because it has a criminal
justice major.

``As a Latina, I want to show women they can also work as a police
officer,'' she said.

Ms. Tobar said that she understood why protests erupted calling for
justice for Black and, in California especially, Latino victims of
police violence.

\emph{{[}Read about}
\href{https://www.nytimes3xbfgragh.onion/2020/06/09/us/ca-defund-police.html}{\emph{calls
to defund the police}} \emph{in California.{]}}

But she said she believed that ``there are officers who want to take
care of us and make sure we're safe.''

Ultimately, Ms. Tobar said she planned to become an agent for the F.B.I.
before starting her own security company.

She was inspired, she said, in part by the loss of her father.

Both Tobar siblings said their mother has watched their ascent proudly.

\hypertarget{dalia-rangel-perez}{%
\subsection{Dalia Rangel Perez}\label{dalia-rangel-perez}}

Image

Dalia Rangel Perez celebrating her graduation.Credit...Dalia Rangel
Perez

Recently, it's been hot in Thermal, the small Coachella Valley community
where Dalia Rangel Perez, 18, has been gearing up to start her career at
U.C. Berkeley.

``The weather is around 120 right now,'' she said late last month,
speaking from the trailer where she lives with her family.

She thinks about her parents, who are essential farmworkers, and others
like them.

``It's incredibly difficult,'' she said. ``It's heart wrenching.''

Ms. Rangel Perez grew up with the language of farmworker activism in her
home, she said --- her father, brother and sister all work with United
Farm Workers --- and it's shaped her desire to become an immigration
lawyer.

Her family immigrated from Mexicali when she was 3.

At first, they lived in a labor camp not unlike the ones where many
farmworkers still live, packed together and vulnerable.

Ms. Rangel Perez said her memories are vague of her first home in
California, but she recalled ``the whites of the walls,'' and that there
was ``nothing except for a mattress,'' for her family of five.

Later, when they lived in a one-room trailer, she remembered roaches
almost like pets, everyone sleeping on the same bed, ``our skin being
kind of like a marshmallow being melted into each other,'' when it was
summer.

``I hated it,'' she said.

Ms. Rangel Perez said she's sad she won't be able to get to campus soon
to start her first semester. She plans to stay with her sister in
Bakersfield to study, since she doesn't have internet access at home in
Thermal.

She still hasn't been to Berkeley. And although she's nervous about
being in a new environment, Ms. Rangel Perez looks forward to seeing it
for herself.

\emph{(This article is part of the}
\href{https://www.nytimes3xbfgragh.onion/column/california-today}{\emph{California
Today}} \emph{newsletter.}
\href{https://www.nytimes3xbfgragh.onion/newsletters/california-today}{\emph{Sign
up}} \emph{to get it by email.)}

\begin{center}\rule{0.5\linewidth}{\linethickness}\end{center}

\hypertarget{heres-what-else-to-know-today}{%
\subsection{Here's what else to know
today}\label{heres-what-else-to-know-today}}

Image

A reminder outside Theodore Judah Elementary School in Sacramento in
April.Credit...Rich Pedroncelli/Associated Press

\begin{itemize}
\tightlist
\item
  \textbf{Schools in the Sacramento region and across the state are
  weighing whether to take advantage of a complex waiver process} that
  would allow some elementary schools in some counties to reopen for
  in-person instructions. One extra confounding factor is that ---
  thanks to a technical problem --- the state
  \href{https://ktla.com/news/california/no-counties-will-be-removed-added-from-covid-19-watchlist-amid-californias-data-issues/}{is
  not adding or taking counties off its county monitoring list}.
  {[}\href{https://www.sacbee.com/news/coronavirus/article244717802.html}{The
  Sacramento Bee}{]}
\end{itemize}

\begin{itemize}
\tightlist
\item
  \textbf{A small group of pro-charter-school parents in Orange County
  rallied} in front of the Santa Ana teachers' union, calling on the
  state to reopen physical classrooms.
  {[}\href{https://www.ocregister.com/2020/08/04/these-o-c-parents-have-a-message-for-gov-newsom-teachers-unions-open-up-the-schools/}{The
  Orange County Register}{]}
\end{itemize}

\emph{Read about the details of the state's plan for schools.
{[}}\href{https://www.nytimes3xbfgragh.onion/2020/07/17/us/california-schools-reopening-newsom.html}{\emph{The
New York Times}}\emph{{]}}

\emph{Also,} \emph{\textbf{please help us understand how you'll be
teaching in the fall.}} \emph{If you're a teacher, what will your remote
classroom look like? Are you preparing to get back into school
eventually? Please fill out this form.
{[}}\href{https://www.nytimes3xbfgragh.onion/2020/08/05/reader-center/teachers-show-us-how-the-coronavirus-is-changing-your-classroom.html?referringSource=articleShare}{\emph{The
New York Times}}\emph{{]}}

\begin{itemize}
\item
  \textbf{Small businesses in ethnic enclaves like South Los Angeles,
  Chinatown and Boyle Heights are some of the most endangered} in the
  coronavirus economic crisis. If they close, it could accelerate
  gentrification.
  {[}\href{https://www.washingtonpost.com/business/2020/07/31/ethnic-enclaves-gentrification-coronavirus/?arc404=true}{The
  Washington Post{]}}
\item
  \textbf{California lawmakers are scrambling to prevent an eviction
  catastrophe.} They don't have much time before some protections are
  set to end.
  {[}\href{https://www.sfchronicle.com/politics/article/Rent-is-coming-due-in-California-Two-weeks-to-15452787.php}{The
  San Francisco Chronicle}{]}
\item
  And more than half of the members of the State Legislature
  \textbf{signed on to a call to Gov. Gavin Newsom to immediately pay
  unemployment benefits to Californians} caught in what the lawmakers
  described as a broken system: ``Millions of our constituents have had
  no income for months.''
  {[}\href{https://www.latimes.com/california/story/2020-08-05/california-legislature-edd-changes-unemployment-benefits-newsom}{The
  Los Angeles Times}{]}
\item
  \textbf{Following reports of restriction-flouting mansion parties},
  including a high-profile
  \href{https://www.nytimes3xbfgragh.onion/2020/08/04/us/la-mansion-party-shooting.html}{one
  that ended in a deadly shooting}, Mayor Eric Garcetti of Los Angeles
  said he'd authorize the city to shut off power and water services to
  violators.
  {[}\href{https://www.latimes.com/california/story/2020-08-05/residents-holding-large-parties-may-have-their-water-and-power-shut-off-garcetti-says}{The
  Los Angeles Times}{]}
\item
  The timing couldn't be more on the nose: \textbf{With the
  Chinese-owned TikTok mired in a geopolitical morass, Facebook-owned
  Instagram released Reels,} a direct competitor that allows users to
  create and share 15-second videos.
  {[}\href{https://www.nytimes3xbfgragh.onion/2020/08/05/technology/tiktok-facebook-instagram-reels.html}{The
  New York Times}{]}
\item
  The idea of the lone culinary auteur is eroding, as the food world
  confronts toxic kitchen culture. Now, Tejal Rao, The Times's
  California restaurant critic, wonders if \textbf{there are ways to
  acknowledge the work of everyone who makes a restaurant}.
  {[}\href{https://www.nytimes3xbfgragh.onion/2020/08/04/dining/chef-restaurant-culture.html}{The
  New York Times}{]}
\end{itemize}

\begin{center}\rule{0.5\linewidth}{\linethickness}\end{center}

\emph{California Today goes live at 6:30 a.m. Pacific time weekdays.
Tell us what you want to see:}
\href{mailto:CAtoday@NYTimes.com}{\emph{CAtoday@NYTimes.com}}\emph{.
Were you forwarded this email?}
\href{https://www.nytimes3xbfgragh.onion/newsletters/california-today?module=inline}{\emph{Sign
up for California Today here}} \emph{and}
\href{https://www.nytimes3xbfgragh.onion/column/california-today}{\emph{read
every edition online here}}\emph{.}

\emph{Jill Cowan grew up in Orange County, went to school at U.C.
Berkeley and has reported all over the state, including the Bay Area,
Bakersfield and Los Angeles --- but she always wants to see more. Follow
along here or on}
\href{https://twitter.com/JillCowan}{\emph{Twitter}}\emph{.}

\emph{California Today is edited by Julie Bloom, who grew up in Los
Angeles and graduated from U.C. Berkeley.}

Advertisement

\protect\hyperlink{after-bottom}{Continue reading the main story}

\hypertarget{site-index}{%
\subsection{Site Index}\label{site-index}}

\hypertarget{site-information-navigation}{%
\subsection{Site Information
Navigation}\label{site-information-navigation}}

\begin{itemize}
\tightlist
\item
  \href{https://help.nytimes3xbfgragh.onion/hc/en-us/articles/115014792127-Copyright-notice}{©~2020~The
  New York Times Company}
\end{itemize}

\begin{itemize}
\tightlist
\item
  \href{https://www.nytco.com/}{NYTCo}
\item
  \href{https://help.nytimes3xbfgragh.onion/hc/en-us/articles/115015385887-Contact-Us}{Contact
  Us}
\item
  \href{https://www.nytco.com/careers/}{Work with us}
\item
  \href{https://nytmediakit.com/}{Advertise}
\item
  \href{http://www.tbrandstudio.com/}{T Brand Studio}
\item
  \href{https://www.nytimes3xbfgragh.onion/privacy/cookie-policy\#how-do-i-manage-trackers}{Your
  Ad Choices}
\item
  \href{https://www.nytimes3xbfgragh.onion/privacy}{Privacy}
\item
  \href{https://help.nytimes3xbfgragh.onion/hc/en-us/articles/115014893428-Terms-of-service}{Terms
  of Service}
\item
  \href{https://help.nytimes3xbfgragh.onion/hc/en-us/articles/115014893968-Terms-of-sale}{Terms
  of Sale}
\item
  \href{https://spiderbites.nytimes3xbfgragh.onion}{Site Map}
\item
  \href{https://help.nytimes3xbfgragh.onion/hc/en-us}{Help}
\item
  \href{https://www.nytimes3xbfgragh.onion/subscription?campaignId=37WXW}{Subscriptions}
\end{itemize}
