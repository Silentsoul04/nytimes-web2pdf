Sections

SEARCH

\protect\hyperlink{site-content}{Skip to
content}\protect\hyperlink{site-index}{Skip to site index}

\href{https://www.nytimes3xbfgragh.onion/section/business/economy}{Economy}

\href{https://myaccount.nytimes3xbfgragh.onion/auth/login?response_type=cookie\&client_id=vi}{}

\href{https://www.nytimes3xbfgragh.onion/section/todayspaper}{Today's
Paper}

\href{/section/business/economy}{Economy}\textbar{}Millions of Evictions
Are a Sharper Threat as Government Support Ends

\url{https://nyti.ms/31sBugU}

\begin{itemize}
\item
\item
\item
\item
\item
\end{itemize}

\href{https://www.nytimes3xbfgragh.onion/news-event/coronavirus?action=click\&pgtype=Article\&state=default\&region=TOP_BANNER\&context=storylines_menu}{The
Coronavirus Outbreak}

\begin{itemize}
\tightlist
\item
  live\href{https://www.nytimes3xbfgragh.onion/2020/08/07/world/covid-19-news.html?action=click\&pgtype=Article\&state=default\&region=TOP_BANNER\&context=storylines_menu}{Latest
  Updates}
\item
  \href{https://www.nytimes3xbfgragh.onion/interactive/2020/us/coronavirus-us-cases.html?action=click\&pgtype=Article\&state=default\&region=TOP_BANNER\&context=storylines_menu}{Maps
  and Cases}
\item
  \href{https://www.nytimes3xbfgragh.onion/interactive/2020/science/coronavirus-vaccine-tracker.html?action=click\&pgtype=Article\&state=default\&region=TOP_BANNER\&context=storylines_menu}{Vaccine
  Tracker}
\item
  \href{https://www.nytimes3xbfgragh.onion/interactive/2020/world/coronavirus-tips-advice.html?action=click\&pgtype=Article\&state=default\&region=TOP_BANNER\&context=storylines_menu}{F.A.Q.}
\item
  \href{https://www.nytimes3xbfgragh.onion/live/2020/08/07/business/stock-market-today-coronavirus?action=click\&pgtype=Article\&state=default\&region=TOP_BANNER\&context=storylines_menu}{Markets
  \& Economy}
\end{itemize}

Advertisement

\protect\hyperlink{after-top}{Continue reading the main story}

Supported by

\protect\hyperlink{after-sponsor}{Continue reading the main story}

\hypertarget{millions-of-evictions-are-a-sharper-threat-as-government-support-ends}{%
\section{Millions of Evictions Are a Sharper Threat as Government
Support
Ends}\label{millions-of-evictions-are-a-sharper-threat-as-government-support-ends}}

Without more federal aid for workers, experts are expecting the largest
disruption to the housing market since the Depression.

\includegraphics{https://static01.graylady3jvrrxbe.onion/images/2020/08/08/business/08virus-eviction1/merlin_175408071_6b9bf3e6-34e1-48fc-b87a-c02c575fb2a3-articleLarge.jpg?quality=75\&auto=webp\&disable=upscale}

\href{https://www.nytimes3xbfgragh.onion/by/conor-dougherty}{\includegraphics{https://static01.graylady3jvrrxbe.onion/images/2018/07/27/multimedia/author-conor-dougherty/author-conor-dougherty-thumbLarge.png}}

By \href{https://www.nytimes3xbfgragh.onion/by/conor-dougherty}{Conor
Dougherty}

\begin{itemize}
\item
  Aug. 7, 2020, 6:17 p.m. ET
\item
  \begin{itemize}
  \item
  \item
  \item
  \item
  \item
  \end{itemize}
\end{itemize}

For the 108 million people who live in a rental home or apartment, Aug.
1 was a grim milestone. It marked the first time rent was due after much
of the nation's economic response to the coronavirus had expired.

The lapse of
\href{https://www.nytimes3xbfgragh.onion/2020/08/06/business/coronavirus-jobs-report-trump-congress-stimulus.html}{expanded
unemployment benefits} and federal, state and local eviction moratoriums
is forcing
\href{https://www.latimes.com/california/story/2020-08-05/california-legislative-leaders-judicial-council-delay-lifting-eviction-moratorium-coronavirus}{lawmakers}
to figure out how to extend those protections. It has also left experts
resorting to natural disaster metaphors
(``\href{https://www.nytimes3xbfgragh.onion/2020/05/27/us/coronavirus-evictions-renters.html}{avalanche},''
``\href{https://www.nbcnews.com/business/business-news/tsunami-evictions-coming-warn-housing-advocates-n1233965}{tsunami}'')
to describe the scale of potential evictions.

Unlike the U.S. economy, which was enjoying the longest expansion on
record, housing --- specifically rental housing --- was troubled before
the virus hit, with problems going back decades. A little under four
million evictions are filed each year, one in four tenant households
spends
\href{https://www.jchs.harvard.edu/sites/default/files/Harvard_JCHS_Americas_Rental_Housing_2020.pdf}{about
half its pretax income} on rent, and each night some
\href{https://www.whitehouse.gov/wp-content/uploads/2019/09/The-State-of-Homelessness-in-America.pdf}{200,000
people} sleep in their cars, on streets or under bridges.

Those were the statistics in good times. Now, with
\href{https://www.nytimes3xbfgragh.onion/2020/08/07/world/coronavirus-covid-19.html}{unemployment
above 10 percent} and projected to stay there through at least next
year, tens of millions of households could be at risk of eviction
\href{https://nlihc.org/sites/default/files/The_Eviction_Crisis_080720.pdf}{in
the coming months}. Even if only a fraction of those evictions actually
take place, it would still be several times the current pace and the
biggest disruption in rental housing in decades.

Whatever the final tally, it is increasingly clear that if the Great
Recession was personified by empty subdivisions and foreclosed
homeowners, the enduring symbol of coronavirus, with its
disproportionate impact on hourly workers, is likely to be a laid-off
tenant struggling to keep an
\href{https://www.nytimes3xbfgragh.onion/2020/08/01/business/economy/housing-overcrowding-coronavirus.html}{overcrowded
apartment}.

``The United States is on the brink of an eviction crisis of
unprecedented magnitude,'' said Emily A. Benfer, a professor at Wake
Forest University School of Law.

That is, of course, a projection --- and so far, government efforts to
hold back a wave of displacement have been effective. About
\href{https://bfi.uchicago.edu/working-paper/2020-62/}{two-thirds} of
the workers eligible for extended unemployment protections could make
more than they did when they were employed, allowing tens of millions of
tenants to shelter in place while paying their monthly bills.

\hypertarget{latest-updates-the-coronavirus-outbreak-and-the-economy}{%
\section{\texorpdfstring{\href{https://www.nytimes3xbfgragh.onion/live/2020/08/07/business/stock-market-today-coronavirus?action=click\&pgtype=Article\&state=default\&region=MAIN_CONTENT_1\&context=storylines_live_updates}{Latest
Updates: The Coronavirus Outbreak and the
Economy}}{Latest Updates: The Coronavirus Outbreak and the Economy}}\label{latest-updates-the-coronavirus-outbreak-and-the-economy}}

\href{https://www.nytimes3xbfgragh.onion/live/2020/08/07/business/stock-market-today-coronavirus?action=click\&pgtype=Article\&state=default\&region=MAIN_CONTENT_1\&context=storylines_live_updates\#the-publisher-of-the-onion-jezebel-and-other-websites-lays-off-15-employees}{32m
ago}

\href{https://www.nytimes3xbfgragh.onion/live/2020/08/07/business/stock-market-today-coronavirus?action=click\&pgtype=Article\&state=default\&region=MAIN_CONTENT_1\&context=storylines_live_updates\#the-publisher-of-the-onion-jezebel-and-other-websites-lays-off-15-employees}{The
publisher of The Onion, Jezebel and other websites lays off 15
employees.}

\href{https://www.nytimes3xbfgragh.onion/live/2020/08/07/business/stock-market-today-coronavirus?action=click\&pgtype=Article\&state=default\&region=MAIN_CONTENT_1\&context=storylines_live_updates\#canada-outlines-its-response-to-the-new-us-aluminum-tariff}{5h
ago}

\href{https://www.nytimes3xbfgragh.onion/live/2020/08/07/business/stock-market-today-coronavirus?action=click\&pgtype=Article\&state=default\&region=MAIN_CONTENT_1\&context=storylines_live_updates\#canada-outlines-its-response-to-the-new-us-aluminum-tariff}{Canada
outlines its response to the new U.S. aluminum tariff.}

\href{https://www.nytimes3xbfgragh.onion/live/2020/08/07/business/stock-market-today-coronavirus?action=click\&pgtype=Article\&state=default\&region=MAIN_CONTENT_1\&context=storylines_live_updates\#laid-off-rehired-laid-off-again-and-now-without-a-weekly-600-boost}{8h
ago}

\href{https://www.nytimes3xbfgragh.onion/live/2020/08/07/business/stock-market-today-coronavirus?action=click\&pgtype=Article\&state=default\&region=MAIN_CONTENT_1\&context=storylines_live_updates\#laid-off-rehired-laid-off-again-and-now-without-a-weekly-600-boost}{Laid
off, rehired, laid off again, and now without a weekly \$600 boost.}

\href{https://www.nytimes3xbfgragh.onion/live/2020/08/07/business/stock-market-today-coronavirus?action=click\&pgtype=Article\&state=default\&region=MAIN_CONTENT_1\&context=storylines_live_updates}{See
more updates}

More live coverage:
\href{https://www.nytimes3xbfgragh.onion/2020/08/07/world/covid-19-news.html?action=click\&pgtype=Article\&state=default\&region=MAIN_CONTENT_1\&context=storylines_live_updates}{Global}

Renters who didn't receive unemployment pay were largely covered by the
various eviction moratoriums that, while not relieving their debts, had
at least granted them a reprieve. The federal moratorium alone, passed
as part of the CARES Act in March,
\href{https://www.frbatlanta.org/community-development/publications/partners-update/2020/covid-19-publications/200616-housing-policy-impact-federal-eviction-protection-coverage-and-the-need-for-better-data.aspx}{covered
between 28.1 percent and 45.6 percent} of rental units.

On Friday, after talks between the Trump administration and Democrats
effectively stalled, advisers to President Trump said they would
recommend that he extend the moratorium through an executive order.

The moratoriums were supposed to be emergency measures to give tenants
some relief until the virus subsided and the economy returned to health.

Except that didn't happen. The virus continues to surge around the
country, and parents are unsure when schools will reopen. Each week
\href{https://www.nytimes3xbfgragh.onion/2020/08/06/business/economy/unemployment-claims.html}{more
than a million} laid-off employees continue to file for unemployment
insurance, while temporary layoffs are becoming permanent
\href{https://www.nytimes3xbfgragh.onion/2020/05/21/business/economy/coronavirus-unemployment-claims.html}{job
losses}.

Landlords hold that the most extreme predictions of evictions are
overblown. For starters, the limited data available suggests that most
tenants have
\href{https://www.nytimes3xbfgragh.onion/2020/05/31/business/economy/coronavirus-rent-landlords-tenants.html}{stayed
current} on their bills. Also, property owners, facing rising vacancies
and
\href{https://www.zillow.com/research/april-2020-market-report-27100/}{falling
rents}, are increasingly working out rent cuts and extended payment
plans.

Still, put all the numbers together, and it becomes clear that renters
were struggling before the pandemic, they've been
\href{https://www.jchs.harvard.edu/blog/how-much-assistance-would-it-take-to-help-renters-affected-by-covid-19/}{hit
harder} by the virus and job losses, and the rental market is likely to
be more challenging even after the economy recovers.

Image

Ebbets Field, a housing complex in Brooklyn. Eviction moratoriums have
protected people who might not have received other government
benefits.Credit...Scott Heins/Getty Images

Image

An apartment building in Oakland, Calif. Expanded unemployment benefits
have helped millions of people stay current on their
bills.Credit...Felix Uribe for The New York Times

\hypertarget{the-pandemic-response-had-problems-but-mostly-succeeded}{%
\subsection{The pandemic response had problems, but mostly
succeeded.}\label{the-pandemic-response-had-problems-but-mostly-succeeded}}

In the wake of the pandemic, 43 states and Washington, D.C., enacted
some kind of eviction moratorium, according to Ms. Benfer. On top of
that were various local measures, along with the
\href{https://nlihc.org/federal-moratoriums}{federal eviction
moratorium}, which covered subsidized housing and rental properties with
loans backed by Fannie Mae and Freddie Mac.

While these measures were of
\href{https://evictionlab.org/covid-policy-scorecard/\#scorecard-app}{varying
length and strength} --- and many, including the federal ban, had little
to no enforcement mechanism --- together the patchwork served to halt or
slow evictions for a majority of renters. Only seven states ---
Arkansas, Georgia, Missouri, Ohio, Oklahoma, South Dakota and Wyoming
--- never issued a statewide stay on evictions, and even in those
states, the federal rules should have protected at least a third of
renter households.

Just as important as those protections were the federal unemployment and
stimulus payments. After all, most renters do not have eviction problems
if they stay current on their bills, and with help from the \$1,200
stimulus payments and \$600 in extended unemployment that came with the
CARES Act,
\href{https://www.nmhc.org/research-insight/nmhc-rent-payment-tracker/}{many
of them have}.

\hypertarget{that-help-is-ending-and-renters-are-slipping-off-the-cliff}{%
\subsection{That help is ending, and renters are slipping off the
cliff.}\label{that-help-is-ending-and-renters-are-slipping-off-the-cliff}}

Benjamin Schenk, a San Diego landlord who operates 30 units in two
buildings, is one of the many property owners who have been surprised by
the high number of tenants paying their rent in the early months of the
pandemic. In March he was talking with his lenders about how he might
restructure his loans in anticipation of nonpayments, only to make it to
August with payment rates close to 100 percent, which he attributes to
the CARES Act.

But people are now falling behind. Though it will take until mid-month
to get a true sense of how bad August will be, several tenants who lost
their jobs stopped paying rent in the first few days. ``The aid that
folks are relying on has dried up and not a lot of places are hiring,''
Mr. Schenk said.

While there's no comprehensive data on rent payments, a
\href{https://www.nmhc.org/research-insight/nmhc-rent-payment-tracker/}{weekly
tracker} from the National Multifamily Housing Council that covers about
11 million units has started slipping. In the Census Bureau's most
recent
\href{https://www.census.gov/data/tables/2020/demo/hhp/hhp12.html}{Pulse
Survey}, for the week of July 16 to 21, just under one in five renters
said they were unable to pay July's rent on time, while one in three
were unsure they could make August payments.

The threat to small landlords is also a threat to tenants.
\href{https://www.hud.gov/press/press_releases_media_advisories/HUD_No_20_071}{About
40 percent} of the nation's 48.2 million rental units are owned by
``mom-and-pop'' operators who tend to have a limited financial cushion.
Since much of the nation's affordable housing consists of small
apartment buildings and single-family homes if these smaller landlords
go under many of their units could be ``lost.'' Some would become
owner-occupied housing. Others will get acquired by larger investors who
plan renovations and rent increases --- compounding a longstanding
affordable housing shortage.

\hypertarget{evictions-have-piled-up-and-are-now-resuming}{%
\subsection{Evictions have piled up and are now
resuming.}\label{evictions-have-piled-up-and-are-now-resuming}}

Evictions, meted out by local courts, are difficult to tally nationwide.
For now, new filings are depressed compared with historical averages,
according to a \href{https://evictionlab.org/eviction-tracking/}{survey
of a dozen cities} by Princeton University's Eviction Lab. But they have
\href{https://www.nytimes3xbfgragh.onion/2020/07/23/business/evictions-moratorium-cares-act.html}{resumed
around the country}, and are likely to grow.

There is a difference between an eviction filing, which is the start of
a legal process, and an actual eviction, in which a tenant is removed.
According to Eviction Lab, there were 3.7 million such filings in 2016,
about one million of which led to an eviction --- a figure that
undercounts displacement.

Many tenants leave after a threat of eviction or the first sign of a
filing. Others leave after a landlord turns off utilities or changes the
locks. Even for tenants who are never taken from their home by a
sheriff, behind every filing is severe stress and tattered credit that
makes it harder to find a new place. Beyond that is the uncountable
number of families whose rent was raised beyond their means and who left
before missing a payment.

So even if there are only a million formal evictions a year, the number
of people who are displaced is probably several times that, and likely
to grow.

While homelessness would almost certainly increase with a spike in
evictions, this doesn't necessarily mean shelters will fill up or
encampments will pop up on every street. Tenants, in particular,
families, often exhaust every available option ---~living in weekly
hotels and illegal garages, staying with friends or piling in with
multiple roommates --- before they end up in the shelter system or the
streets.

Steve Noggle, 43, was evicted from his apartment in Annville, Pa., this
week. He received just five weeks of extended unemployment benefits even
though he lost his restaurant job four months ago. He has been sleeping
on his sister's couch since Monday. ``I don't like having to be here,
it's a burden on everybody, especially because I can't contribute
anything financially,'' he said. ``I'm just hoping I can get a job as
soon as possible.''

Gillian Friedman contributed reporting.

Advertisement

\protect\hyperlink{after-bottom}{Continue reading the main story}

\hypertarget{site-index}{%
\subsection{Site Index}\label{site-index}}

\hypertarget{site-information-navigation}{%
\subsection{Site Information
Navigation}\label{site-information-navigation}}

\begin{itemize}
\tightlist
\item
  \href{https://help.nytimes3xbfgragh.onion/hc/en-us/articles/115014792127-Copyright-notice}{©~2020~The
  New York Times Company}
\end{itemize}

\begin{itemize}
\tightlist
\item
  \href{https://www.nytco.com/}{NYTCo}
\item
  \href{https://help.nytimes3xbfgragh.onion/hc/en-us/articles/115015385887-Contact-Us}{Contact
  Us}
\item
  \href{https://www.nytco.com/careers/}{Work with us}
\item
  \href{https://nytmediakit.com/}{Advertise}
\item
  \href{http://www.tbrandstudio.com/}{T Brand Studio}
\item
  \href{https://www.nytimes3xbfgragh.onion/privacy/cookie-policy\#how-do-i-manage-trackers}{Your
  Ad Choices}
\item
  \href{https://www.nytimes3xbfgragh.onion/privacy}{Privacy}
\item
  \href{https://help.nytimes3xbfgragh.onion/hc/en-us/articles/115014893428-Terms-of-service}{Terms
  of Service}
\item
  \href{https://help.nytimes3xbfgragh.onion/hc/en-us/articles/115014893968-Terms-of-sale}{Terms
  of Sale}
\item
  \href{https://spiderbites.nytimes3xbfgragh.onion}{Site Map}
\item
  \href{https://help.nytimes3xbfgragh.onion/hc/en-us}{Help}
\item
  \href{https://www.nytimes3xbfgragh.onion/subscription?campaignId=37WXW}{Subscriptions}
\end{itemize}
