Sections

SEARCH

\protect\hyperlink{site-content}{Skip to
content}\protect\hyperlink{site-index}{Skip to site index}

\href{https://www.nytimes3xbfgragh.onion/section/business/economy}{Economy}

\href{https://myaccount.nytimes3xbfgragh.onion/auth/login?response_type=cookie\&client_id=vi}{}

\href{https://www.nytimes3xbfgragh.onion/section/todayspaper}{Today's
Paper}

\href{/section/business/economy}{Economy}\textbar{}Trump's Orders on
WeChat and TikTok Are Uncertain. That May Be the Point.

\url{https://nyti.ms/3adzDk0}

\begin{itemize}
\item
\item
\item
\item
\item
\end{itemize}

Advertisement

\protect\hyperlink{after-top}{Continue reading the main story}

Supported by

\protect\hyperlink{after-sponsor}{Continue reading the main story}

\hypertarget{trumps-orders-on-wechat-and-tiktok-are-uncertain-that-may-be-the-point}{%
\section{Trump's Orders on WeChat and TikTok Are Uncertain. That May Be
the
Point.}\label{trumps-orders-on-wechat-and-tiktok-are-uncertain-that-may-be-the-point}}

The restrictions on the two Chinese-owned apps followed a familiar model
for other policy announcements on China from the Trump administration.

\includegraphics{https://static01.graylady3jvrrxbe.onion/images/2020/08/08/business/07jpDC-ORDERs-print/merlin_175392885_12569975-dca2-460a-83de-8e4453ec2632-articleLarge.jpg?quality=75\&auto=webp\&disable=upscale}

\href{https://www.nytimes3xbfgragh.onion/by/ana-swanson}{\includegraphics{https://static01.graylady3jvrrxbe.onion/images/2018/12/10/multimedia/author-ana-swanson/author-ana-swanson-thumbLarge.png}}

By \href{https://www.nytimes3xbfgragh.onion/by/ana-swanson}{Ana Swanson}

\begin{itemize}
\item
  Aug. 7, 2020
\item
  \begin{itemize}
  \item
  \item
  \item
  \item
  \item
  \end{itemize}
\end{itemize}

\href{https://cn.nytimes3xbfgragh.onion/business/20200810/trump-executive-order-tiktok-wechat/}{阅读简体中文版}\href{https://cn.nytimes3xbfgragh.onion/business/20200810/trump-executive-order-tiktok-wechat/zh-hant/}{閱讀繁體中文版}

WASHINGTON --- President Trump's sudden decision late Thursday to
\href{https://www.nytimes3xbfgragh.onion/2020/08/06/technology/trump-wechat-tiktok-china.html}{restrict
two popular Chinese social media services} from the United States has
created confusion about how broad the bans on doing business with China
could ultimately be.

That confusion may be part of the point.

Citing national security concerns, the Trump administration announced
that it would bar people and property within U.S. jurisdictions from
carrying out ``transactions''
\href{https://www.nytimes3xbfgragh.onion/2020/08/07/business/trump-china-wechat-tiktok.html}{with
WeChat} and
\href{https://www.nytimes3xbfgragh.onion/2020/08/01/technology/tiktok-trump-microsoft-bytedance-china-ban.html}{TikTok},
the two Chinese-owned apps, after 45 days. But the White House did not
define what those transactions included, leaving companies bewildered
about whether they may be forced to fundamentally change their business
within a matter of weeks.

Stoking this kind of uncertainty is something that the Trump
administration has not been apologetic about in the past. Some White
House advisers see it as a feature rather than a bug of their policy
process, arguing that the risk of further crackdowns will dissuade
American companies from operating in China.

That, they said, is a good thing because Chinese policies like
\href{https://www.state.gov/chinas-military-civil-fusion-strategy-poses-a-risk-to-national-security/}{``civil-military
fusion''} have undermined the ability of both Chinese and American
companies to operate independently in China.

``Mobile apps like TikTok and WeChat that collect your personal or
business information and that can track, surveil or monitor your
movements put you and your family in the cross hairs of an Orwellian
regime,'' Peter Navarro, the White House director of trade and
manufacturing policy, said in an interview. He posed a question to the
mothers of America, ``It's 10 p.m. Does the Chinese Communist Party know
where your children are at?''

Mr. Navarro acknowledged that some multinationals might oppose the
measures, but said that ``the American public is tired of the corporate
greed that, before the Age of Trump, sent our jobs overseas and now
endangers our national security and privacy.''

Critics countered that the Trump administration's unpredictable actions
threaten to compromise the secure business environment that the United
States is known for, in which rule of law prevails and the government
rarely interferes in the market.

``The government inserting this much uncertainty into the business
landscape and into the user landscape is deeply problematic,'' said Matt
Perault, a professor of Duke University's Center for Science \&
Technology Policy.

On Friday, TikTok, which is owned by
\href{https://www.nytimes3xbfgragh.onion/2018/10/29/technology/bytedance-app-funding-china.html}{Chinese
internet conglomerate ByteDance}, said in a statement that it was
``shocked by the recent executive order, which was issued without any
due process.'' It said it had sought to work with the U.S. government
for nearly a year but instead found the White House ``paid no attention
to facts, dictated terms of an agreement without going through standard
legal processes, and tried to insert itself into negotiations between
private businesses.''

A spokesman for Tencent, the parent company of WeChat, which is
\href{https://www.nytimes3xbfgragh.onion/video/technology/100000004574648/china-internet-wechat.html}{widely
used in China} and around the world as a messaging and payments app,
said it was ``reviewing the executive order to get a full
understanding.''

The Trump administration has steadily ramped up its actions in a broader
economic and geopolitical fight with China, starting with a trade war
that put tariffs on hundreds of billions of dollars of Chinese products
in 2018 and 2019. It also introduced restrictions on other kinds of
Chinese technology, including clamping down on exports to the Chinese
telecom giant Huawei.

The sudden, vaguely worded order from the White House on Thursday night,
which came without further explanation or a media briefing, followed a
familiar model for some of the other policy announcements on China from
the Trump administration. Many have left multinational companies in
suspense for days or weeks about the specifics.

With policy moves like tariffs and export controls, the Trump
administration wielded uncertainty as a source of leverage, using it to
frighten companies into compliance and leaving themselves room to back
down or escalate the situation.

The executive orders on WeChat and TikTok leave the determination of
what constitutes a ``transaction'' up to the secretary of commerce,
Wilbur Ross. According to the language of the orders, Mr. Ross will make
that determination in 45 days, meaning it would not be clear to
businesses what will be included in the ban until it actually goes into
effect.

``It may be that it's won't be nearly as bad as people might fear,''
said Jason M. Waite, a partner at the law firm Alston \& Bird, adding
that the administration might discover legal or practical concerns with
putting the order in place in the interim. ``It is a 45-day surprise.''

People familiar with the deliberations said administration officials
clearly intended to target the presence of WeChat and TikTok on the
Google and Apple app stores, cutting off downloads and updates for the
Chinese apps. It is unclear if the restrictions could affect other parts
of the Chinese companies' sprawling portfolios and business dealings,
particularly for Tencent.

The order appears to bar transactions with Tencent or its subsidiaries
that are specifically related to WeChat. That suggests it would not
affect Tencent's sprawling investment relationships and business
dealings with companies like Tesla; the Snapchat owner Snap;
\href{https://www.nba.com/article/2019/07/28/nba-tencent-announce-partnership-expansion-release}{the}National
Basketball Association; Activision Blizzard, the maker of video game
World of Warcraft; and Epic Games, the maker of Fortnite.

But many American companies, including
\href{https://usa.visa.com/about-visa/newsroom/press-releases.releaseId.16681.html}{Visa},
\href{https://newsroom.mastercard.com/asia-pacific/press-releases/mastercard-works-with-tencent-and-ant-financial-to-enable-new-payments-choices-across-china-for-international-visitors/}{Mastercard}
and
\href{https://stories.starbucks.com/stories/2016/starbucks-tencent-partnership/}{Starbucks},
have more direct partnerships with WeChat in China to use its payment
platform and e-commerce functions. Whether those kinds of activities
would be barred in China or around the world, or whether phone makers
like Apple would be allowed to sell mobile phones installed with WeChat,
remain up in the air.

``The Trump administration has left itself a lot of wiggle room in terms
of what is covered, how quickly prohibitions will be carried out, and
how the order will be enforced,'' said Scott Kennedy, a China expert at
the Center for Strategic and International Relations.

Other Chinese tech companies could find themselves as the next target of
the Trump administration. U.S. officials viewed the executive orders on
TikTok and WeChat as a template that could be applied to other Chinese
companies, and some have discussed whether services like Alibaba's
Alipay pose a similar national security concern, according to people
with knowledge of the matter.

``There's definitely a chilling effect,'' said Samm Sacks, a fellow in
cybersecurity policy and China's digital economy fellow at New America,
a think tank. But she said that companies like Alibaba and Tencent had
long understood the risks of operating in the United States.

``This latest move may have come as a surprise, but their real growth
strategies have never focused in the U.S.,'' she said. ``They've always
known it was a hostile environment.''

Advertisement

\protect\hyperlink{after-bottom}{Continue reading the main story}

\hypertarget{site-index}{%
\subsection{Site Index}\label{site-index}}

\hypertarget{site-information-navigation}{%
\subsection{Site Information
Navigation}\label{site-information-navigation}}

\begin{itemize}
\tightlist
\item
  \href{https://help.nytimes3xbfgragh.onion/hc/en-us/articles/115014792127-Copyright-notice}{©~2020~The
  New York Times Company}
\end{itemize}

\begin{itemize}
\tightlist
\item
  \href{https://www.nytco.com/}{NYTCo}
\item
  \href{https://help.nytimes3xbfgragh.onion/hc/en-us/articles/115015385887-Contact-Us}{Contact
  Us}
\item
  \href{https://www.nytco.com/careers/}{Work with us}
\item
  \href{https://nytmediakit.com/}{Advertise}
\item
  \href{http://www.tbrandstudio.com/}{T Brand Studio}
\item
  \href{https://www.nytimes3xbfgragh.onion/privacy/cookie-policy\#how-do-i-manage-trackers}{Your
  Ad Choices}
\item
  \href{https://www.nytimes3xbfgragh.onion/privacy}{Privacy}
\item
  \href{https://help.nytimes3xbfgragh.onion/hc/en-us/articles/115014893428-Terms-of-service}{Terms
  of Service}
\item
  \href{https://help.nytimes3xbfgragh.onion/hc/en-us/articles/115014893968-Terms-of-sale}{Terms
  of Sale}
\item
  \href{https://spiderbites.nytimes3xbfgragh.onion}{Site Map}
\item
  \href{https://help.nytimes3xbfgragh.onion/hc/en-us}{Help}
\item
  \href{https://www.nytimes3xbfgragh.onion/subscription?campaignId=37WXW}{Subscriptions}
\end{itemize}
