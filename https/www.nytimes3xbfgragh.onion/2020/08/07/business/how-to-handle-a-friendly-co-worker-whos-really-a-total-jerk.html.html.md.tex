\href{/section/business}{Business}\textbar{}How to Handle a `Friendly'
Co-Worker Who's Really a Total Jerk

\href{https://nyti.ms/3a5X6DJ}{https://nyti.ms/3a5X6DJ}

\begin{itemize}
\item
\item
\item
\item
\item
\item
\end{itemize}

\includegraphics{https://static01.graylady3jvrrxbe.onion/images/2020/08/09/business/09Workfriend/merlin_175289196_8fae26c8-6101-4b0d-afe2-d90bc5a4cebe-articleLarge.jpg?quality=75\&auto=webp\&disable=upscale}

Sections

\protect\hyperlink{site-content}{Skip to
content}\protect\hyperlink{site-index}{Skip to site index}

work Friend

\hypertarget{how-to-handle-a-friendly-co-worker-whos-really-a-total-jerk}{%
\section{How to Handle a `Friendly' Co-Worker Who's Really a Total
Jerk}\label{how-to-handle-a-friendly-co-worker-whos-really-a-total-jerk}}

A woman has a misogynist at her job. A 56-year-old despairs about his
prospects. And employees are asked to pay for their own diversity
training.

Credit...Margeaux Walter for The New York Times

Supported by

\protect\hyperlink{after-sponsor}{Continue reading the main story}

\href{https://www.nytimes3xbfgragh.onion/column/roxane-gay}{\includegraphics{https://static01.graylady3jvrrxbe.onion/images/2015/03/16/opinion/Gay-Roxane-circular/Gay-Roxane-circular-thumbLarge-v3.png}}

By \href{https://www.nytimes3xbfgragh.onion/column/roxane-gay}{Roxane
Gay}

\begin{itemize}
\item
  Aug. 7, 2020, 12:10 p.m. ET
\item
  \begin{itemize}
  \item
  \item
  \item
  \item
  \item
  \item
  \end{itemize}
\end{itemize}

\emph{Send questions about the office, money, careers and work-life
balance to}
\href{mailto:workfriend@NYTimes.com}{\nolinkurl{workfriend@NYTimes.com}}\emph{.
Include your name and location, or a request to remain anonymous.
Letters may be edited.}

\hypertarget{the-banality-of-misogyny}{%
\subsection{The Banality of Misogyny}\label{the-banality-of-misogyny}}

\begin{quote}
\emph{I work with a man who is considered ``nice'' and ``friendly.'' He
always asks how I'm doing and shows interest. He also once mansplained
to me ``when a female is or is not being intentionally sexually
provocative.'' He often shares memes on social media making fun of women
that are very attractive, which to him means that they're dumb. Once he
said, ``Well, all that feminism stuff is B.S. anyways.'' He is also an
anti-masker.}

\emph{He offends me to my core, but I cannot afford to lose my job. I
have only worked here for three months, and have already seen our (white
American male) boss fire four competent women for minor offenses, and so
I don't say anything when this co-worker ticks me off. I don't know
whether this working environment is toxic, or if I'm just too sensitive.
Help! What would you do?}

\emph{--- Anonymous, Hong Kong}
\end{quote}

A great many women work with a ``nice,'' ``friendly'' man who abuses his
position and their well-mannered tolerance of his nonsense. You have my
sympathies. It is exhausting, pretending to find boorish misogynists
charming or clever when, in fact, they are banal and unoriginal.

You are not too sensitive. His behavior is inappropriate at best. And
any time a man says feminism is B.S., he is plainly communicating
exactly who he is --- a total jerk. When you cannot afford to lose your
job, your options for dealing with a guy like this are limited,
especially given the company's pattern of firing women for minor issues.
You are dealing with a toxic work environment, but only you know how
much toxicity you can tolerate. Listen: Life is short. Stop talking to
this man. Stop giving him your time and energy. Playing along with him
is not part of your job. He is a sexist pig looking for attention from
women because he suffers from profound self-loathing. Or whatever --- I
don't really care what his problem is, and neither should you. Document
any instance when he engages in unacceptable workplace behavior. If he
crosses a line you cannot abide, you can and should report him to Human
Resources or your management team. You need your job, but you have
rights. Your dignity matters, and so does your peace of mind.

\begin{center}\rule{0.5\linewidth}{\linethickness}\end{center}

\hypertarget{56-angry-and-out-of-work}{%
\subsection{56, Angry, and Out of Work}\label{56-angry-and-out-of-work}}

\begin{quote}
\emph{I've been out of work for almost a year and a half, but my spouse
is still employed. While financially we are OK, the mental anguish is
real. After applying for and being rejected from more than 100 jobs,
humiliation, anger and raging self-doubt plague me. I find it hard to
talk to friends because my news ultimately ends with failure. I worry
about getting sick with my minimal, high-deductible insurance. (I am 56
--- and, yes, ageism is real and a widely accepted business practice.)}
\emph{I feel awful for depriving my family, especially my wife, who was
looking forward to an early and financially solid retirement. And yet I
know I am lucky.}

\emph{Given the wide-ranging personal and societal consequences of
unemployment, should employers give added weight to an unemployed
person's application?}

\emph{--- Anonymous}
\end{quote}

Millions are in a similar position, living in economic peril, one health
crisis away from bankruptcy, because for some inexplicable reason,
Americans are resistant to single-payer health care and prefer having
their right to good health tied to the precarity of employment. We know
about the economic consequences of unemployment, but far less attention
is paid to the emotional toll it takes --- anxiety, depression,
substance abuse, suicidal ideation. I hope you have a robust support
system, and I hope you allow them to be there for you. It's clear you
also understand that you are relatively fortunate, by way of your wife's
employment, but that is a small consolation when you can't find your way
back into the job market.

You aren't depriving your family. You can provide for them in other ways
--- taking over more of the domestic responsibilities, engaging in
active parenting. These things are work, too.

I'm supposed to say that employers should hire the best people for any
given job, but right now, it feels like there are more ``best people''
seeking work than there are jobs. And there are all kinds of systemic
biases that privilege certain kinds of ``best people'' over others. What
additional criteria should employers use now? Should the unemployed be
prioritized? What about people with families to support? What about
people who are taking care of sick parents, or with significant student
debt? Do employers address systemic bias and hire the best people who
will also create more equitable representation in the workplace? There
are no easy answers --- because as we know, fairness and capitalism are
largely incompatible.

I hear the anguish in every word of your letter and I am sorry for what
you're going through. Prolonged unemployment is incredibly demoralizing.
And you're right --- age discrimination is pervasive, so you're carrying
a heavy burden. It might be useful to have a professional résumé doctor
take a look at your application package and LinkedIn profile to see how
you might reposition yourself. If you haven't already, let your
professional and personal networks know you're looking for work. Expand
your search into adjacent fields where your skill set would be useful.

You have a right to your despair and anger, but I hope you don't let
these feelings consume you. I hope you find an amazing job where you can
thrive professionally, and in the meantime, I wish you the very best.

\begin{center}\rule{0.5\linewidth}{\linethickness}\end{center}

\hypertarget{minor-reparations}{%
\subsection{Minor Reparations}\label{minor-reparations}}

\begin{quote}
\emph{I work on a small team within a large city's government. We are
doing work around equity, and one of the first things is to hold speaker
events to educate ourselves.}

\emph{Because it's government, we don't have a budget for this. Our
budget is allocated well in advance and there's nothing we can shuffle
around. One of the members of the organizing sub-team said we should ask
employees who attend these events to contribute personally to pay
speakers. I'm deeply uncomfortable with asking people to pay for things
associated with work. Am I wrong to object?}

\emph{---~Anonymous}
\end{quote}

Kudos to your team for their willingness to do the work of expanding and
improving their thinking and efforts around diversity, equity and
inclusion. Public speaking is labor that deserves to be compensated, but
it is absolutely unacceptable that your team members should be spending
their own money on this. You are not at all wrong to object. It is
ridiculous that the most feasible solution here is for your staff
members to assume their employer's financial obligations. I suppose
that's a reflection of how governments all over this country, including
the federal government, are shirking their responsibilities and hoping
--- if they care at all --- that right-minded individuals will take up
the slack.

I do not believe there is nothing in the budget that can be shuffled; I
believe there is nothing your organization is \emph{willing} to shuffle.
When an organization truly wants to find money for something they
prioritize, they find the money. If they aren't going to treat work
around equity as a priority, you and your fellow employees don't need to
pay the bill. There are other things you can do --- reading groups,
discussions and the like. But mostly, you need to hold your management
accountable. This is their responsibility, not yours.

\href{http://www.roxanegay.com/}{\emph{Roxane Gay}} \emph{is the author,
most recently, of ``Hunger'' and a contributing opinion writer. Write to
her at}
\href{mailto:workfriend@NYTimes.com}{\emph{workfriend@NYTimes.com}}.

Advertisement

\protect\hyperlink{after-bottom}{Continue reading the main story}

\hypertarget{site-index}{%
\subsection{Site Index}\label{site-index}}

\hypertarget{site-information-navigation}{%
\subsection{Site Information
Navigation}\label{site-information-navigation}}

\begin{itemize}
\tightlist
\item
  \href{https://help.nytimes3xbfgragh.onion/hc/en-us/articles/115014792127-Copyright-notice}{©~2020~The
  New York Times Company}
\end{itemize}

\begin{itemize}
\tightlist
\item
  \href{https://www.nytco.com/}{NYTCo}
\item
  \href{https://help.nytimes3xbfgragh.onion/hc/en-us/articles/115015385887-Contact-Us}{Contact
  Us}
\item
  \href{https://www.nytco.com/careers/}{Work with us}
\item
  \href{https://nytmediakit.com/}{Advertise}
\item
  \href{http://www.tbrandstudio.com/}{T Brand Studio}
\item
  \href{https://www.nytimes3xbfgragh.onion/privacy/cookie-policy\#how-do-i-manage-trackers}{Your
  Ad Choices}
\item
  \href{https://www.nytimes3xbfgragh.onion/privacy}{Privacy}
\item
  \href{https://help.nytimes3xbfgragh.onion/hc/en-us/articles/115014893428-Terms-of-service}{Terms
  of Service}
\item
  \href{https://help.nytimes3xbfgragh.onion/hc/en-us/articles/115014893968-Terms-of-sale}{Terms
  of Sale}
\item
  \href{https://spiderbites.nytimes3xbfgragh.onion}{Site Map}
\item
  \href{https://help.nytimes3xbfgragh.onion/hc/en-us}{Help}
\item
  \href{https://www.nytimes3xbfgragh.onion/subscription?campaignId=37WXW}{Subscriptions}
\end{itemize}
