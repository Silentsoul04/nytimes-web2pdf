Sections

SEARCH

\protect\hyperlink{site-content}{Skip to
content}\protect\hyperlink{site-index}{Skip to site index}

\href{https://www.nytimes3xbfgragh.onion/section/nyregion}{New York}

\href{https://myaccount.nytimes3xbfgragh.onion/auth/login?response_type=cookie\&client_id=vi}{}

\href{https://www.nytimes3xbfgragh.onion/section/todayspaper}{Today's
Paper}

\href{/section/nyregion}{New York}\textbar{}N.Y. Schools Can Reopen,
Cuomo Says, in Contrast With Much of U.S.

\url{https://nyti.ms/3iieNCF}

\begin{itemize}
\item
\item
\item
\item
\item
\end{itemize}

\href{https://www.nytimes3xbfgragh.onion/news-event/coronavirus?action=click\&pgtype=Article\&state=default\&region=TOP_BANNER\&context=storylines_menu}{The
Coronavirus Outbreak}

\begin{itemize}
\tightlist
\item
  live\href{https://www.nytimes3xbfgragh.onion/2020/08/07/world/covid-19-news.html?action=click\&pgtype=Article\&state=default\&region=TOP_BANNER\&context=storylines_menu}{Latest
  Updates}
\item
  \href{https://www.nytimes3xbfgragh.onion/interactive/2020/us/coronavirus-us-cases.html?action=click\&pgtype=Article\&state=default\&region=TOP_BANNER\&context=storylines_menu}{Maps
  and Cases}
\item
  \href{https://www.nytimes3xbfgragh.onion/interactive/2020/science/coronavirus-vaccine-tracker.html?action=click\&pgtype=Article\&state=default\&region=TOP_BANNER\&context=storylines_menu}{Vaccine
  Tracker}
\item
  \href{https://www.nytimes3xbfgragh.onion/interactive/2020/world/coronavirus-tips-advice.html?action=click\&pgtype=Article\&state=default\&region=TOP_BANNER\&context=storylines_menu}{F.A.Q.}
\item
  \href{https://www.nytimes3xbfgragh.onion/live/2020/08/07/business/stock-market-today-coronavirus?action=click\&pgtype=Article\&state=default\&region=TOP_BANNER\&context=storylines_menu}{Markets
  \& Economy}
\end{itemize}

Advertisement

\protect\hyperlink{after-top}{Continue reading the main story}

Supported by

\protect\hyperlink{after-sponsor}{Continue reading the main story}

\hypertarget{ny-schools-can-reopen-cuomo-says-in-contrast-with-much-of-us}{%
\section{N.Y. Schools Can Reopen, Cuomo Says, in Contrast With Much of
U.S.}\label{ny-schools-can-reopen-cuomo-says-in-contrast-with-much-of-us}}

In a long-awaited announcement, the governor says schools can welcome
back students if the rate of infection in their communities remains low.

\includegraphics{https://static01.graylady3jvrrxbe.onion/images/2020/08/06/nyregion/00nyvirus-schoolsHFO/merlin_175370388_5e723ec8-9174-47aa-950a-c8012dab604f-articleLarge.jpg?quality=75\&auto=webp\&disable=upscale}

\href{https://www.nytimes3xbfgragh.onion/by/eliza-shapiro}{\includegraphics{https://static01.graylady3jvrrxbe.onion/images/2018/12/28/multimedia/author-eliza-shapiro/author-eliza-shapiro-thumbLarge.png}}

By \href{https://www.nytimes3xbfgragh.onion/by/eliza-shapiro}{Eliza
Shapiro}

\begin{itemize}
\item
  Aug. 7, 2020Updated 5:53 p.m. ET
\item
  \begin{itemize}
  \item
  \item
  \item
  \item
  \item
  \end{itemize}
\end{itemize}

Schools across New York can reopen for in-person instruction this fall,
Gov. Andrew M. Cuomo said Friday, solidifying
\href{https://www.nytimes3xbfgragh.onion/2020/08/05/nyregion/nyc-schools-reopening.html}{New
York's status as one of the few states in America that has a virus
transmission rate low enough}to bring children back into classrooms ---
not only in its rural communities but also in the country's biggest
city.

Just a few months after New York became a global epicenter of the
pandemic, the governor opened the door for millions of students across
the state to return to classrooms, even as most public school students
in the country will start the school year remotely.

But Mr. Cuomo's announcement does not guarantee that school buildings in
the state's more than 700 local districts will actually reopen in the
coming weeks. It is now up to local politicians and superintendents to
decide whether to reopen, and how to do so. Their in-person reopening
plans must also be approved by the state's education and health
departments in the coming weeks.

Under the governor's announcement, schools can decide to open as long as
they are in a region where the average rate of positive coronavirus
tests is below 5 percent over a two- week period. That threshold was
recommended by the World Health Organization to begin general reopening
\href{https://www.nytimes3xbfgragh.onion/2020/07/14/us/coronavirus-schools-fall.html}{and
has recently been adopted by some school districts}.

Most of the state, including New York City, has maintained a positivity
rate of about 1 percent. New York calculates its test positivity rate
based on reporting from all the state's counties. Mayor Bill de Blasio
has said schools can only open here if the positivity rate is below 3
percent.

``All schools can open,'' Mr. Cuomo said during a news conference.

``If anyone can open schools, we can open schools,'' he said, adding,
``we have the best infection rate in the country.''

Governors in other states, including Ron DeSantis of Florida and Greg
Abbott of Texas, have gone beyond Mr. Cuomo by actively encouraging
school districts in their states to reopen. But the pandemic is raging
in those states, prompting officials in major districts like Miami-Dade
and Houston to announce that they would start the school year
remote-only.

If New York City and most of the state's other districts reopen, it will
be a remarkable turnaround from the spring, when hundreds of people were
dying a day from the virus and the state's hospital capacity was
stretched to a breaking point.

The city has recently logged fewer than 200 reported cases per day,
although lags in test results could compromise that data.

Though Mr. Cuomo
\href{https://www.nytimes3xbfgragh.onion/2020/04/12/nyregion/schools-cuomo-de-blasio-nyc-coronavirus.html}{has
tried to assert his power over school closures throughout the pandemic,
in some cases contradicting Mr. de Blasio on key decisions}, he has
signaled that his role in the debate over reopening for the fall will be
limited to setting the threshold for a safe reopening, and unilaterally
shutting down schools if that threshold is reached.

Mr. Cuomo frequently celebrates the state's transformation from a global
epicenter of the virus to one of the safest places in the country in
terms of transmission levels, and has received accolades for his
management of the crisis. New York's test positivity rate is now among
the lowest in the nation; the rate in states like Florida where there
has been enormous resistance to reopening schools reached as high as 20
percent last month.

The school reopening debate, however, presents the governor with a
political conundrum from which it might be difficult to emerge
unscathed.

If the city does reopen schools, it could alienate him from educators
and the teachers' union, a crucial ally. But if the city halts or delays
its opening plan, it could leave over 1 million families in the lurch
\href{https://www.nytimes3xbfgragh.onion/2020/07/10/nyregion/nyc-school-daycare-reopening.html}{over
child care}, and
\href{https://www.nytimes3xbfgragh.onion/2020/04/16/nyregion/special-education-coronavirus-nyc.html}{hundreds
of thousands of low-income children, homeless children, and students
with disabilities without in-person learning for months to come}.

Mr. Cuomo acknowledged those difficulties on Friday, saying that he had
been ``deluged'' with calls from parents and teachers who have concerns
about reopening.

``If the teachers don't come back, then you can't really open the
schools,'' he said. ``If the parents don't send their students, then
you're not really opening the schools.''

\hypertarget{the-coronavirus-outbreak}{%
\subsubsection{The Coronavirus
Outbreak}\label{the-coronavirus-outbreak}}

\hypertarget{back-to-school}{%
\paragraph{Back to School}\label{back-to-school}}

Updated Aug. 7, 2020

The latest highlights as the first students return to U.S. schools.

\begin{itemize}
\item
  \begin{itemize}
  \tightlist
  \item
    Schools are open in parts of the country --- and some are
    \href{https://www.nytimes3xbfgragh.onion/2020/08/03/us/school-closing-coronavirus.html?action=click\&pgtype=Article\&state=default\&region=MAIN_CONTENT_2\&context=storylines_keepup}{already
    closing again}. Students have already faced altered classrooms and
    emergency quarantines. We spoke to some to see what
    \href{https://www.nytimes3xbfgragh.onion/2020/08/06/us/coronavirus-students.html?action=click\&pgtype=Article\&state=default\&region=MAIN_CONTENT_2\&context=storylines_keepup}{school
    is like in the age of Covid-19.}
  \item
    Photos of a crowded high school hallway evoked outrage on social
    media. The student who took them
    \href{https://www.nytimes3xbfgragh.onion/2020/08/06/us/north-paulding-high-school-coronavirus-georgia.html?action=click\&pgtype=Article\&state=default\&region=MAIN_CONTENT_2\&context=storylines_keepup}{says
    she was suspended}.
  \item
    Faced with remote learning or socially distanced classroom options,
    some parents of rising kindergartners
    \href{https://www.nytimes3xbfgragh.onion/2020/07/23/parenting/school-opening-kindergarten-coronavirus.html?action=click\&pgtype=Article\&state=default\&region=MAIN_CONTENT_2\&context=storylines_keepup}{are
    considering holding their kids back.}
  \item
    Teachers,
    \href{https://www.nytimes3xbfgragh.onion/2020/08/05/reader-center/teachers-show-us-how-the-coronavirus-is-changing-your-classroom.html?action=click\&pgtype=Article\&state=default\&region=MAIN_CONTENT_2\&context=storylines_keepup}{show
    us your classrooms}! We want to see how educators are preparing to
    keep themselves and students safe if their schools open this fall.
  \end{itemize}
\end{itemize}

Union leaders who represent teachers in New York City and the rest of
the state have raised alarms about reopening, saying they do not believe
it is currently safe to do in-person instruction in at least some parts
of the state. Mr. Cuomo has a political alliance with Michael Mulgrew,
the president of the city's powerful United Federation of Teachers,
which represents New York City teachers.

Though it is illegal for teachers to strike in New York, Mr. Mulgrew
recently hinted at potential legal action against the city if it reopens
schools and the union does not believe adequate safety precautions are
in place.

On Friday, Mr. Mulgrew responded to the governor's announcement with a
terse statement: ``As Governor Cuomo noted, parents and teachers must be
confident that schools are safe before they can reopen. In New York City
that is still an open question.''

Teachers over 65 and those with pre-existing conditions are eligible for
medical exemptions that will allow them to work from home.

``I don't think you want to get into a legal battle with the teachers,''
Mr. Cuomo said Friday, adding, ``teachers have to feel safe, they can't
teach if they don't feel safe.''

The governor directed districts to publicly post their plans for testing
teachers and students after the school year begins --- a demand from the
teachers' union that New York City has not released significant details
on. The governor said all districts should release information about
where students and teachers can get tested and how quickly test results
can be provided.

Mr. Cuomo also asked all districts to post their protocols for when
someone in a school tests positive. Mr. de Blasio outlined the city's
plan for that last week: Just two cases in different classrooms of the
same school could force its closing for two weeks.

And Mr. Cuomo said school districts must hold more virtual meetings with
parents and teachers throughout August to answer questions.

Mr. Cuomo is leaving most of the other details about how to actually
reopen safely to individual school districts, which have spent the
summer creating reopening plans to be approved by the State Education
Department. Districts across the state are tentatively planning to
reopen late in August or early next month. New York City is scheduled to
start school on Sept. 10.

The challenges facing all districts are myriad, but especially so in New
York City, the nation's largest school system, and the only major
district in America currently planning to reopen even part-time.

New York City and other districts across the state are still finalizing
strategies that allow for social distancing in school buildings, trying
to find enough nurses to staff school buildings, and upgrading or
replacing ventilation systems in classrooms. Some New York City
buildings are over a century old and lack working windows.

Districts are also scrambling to improve their remote instruction plans,
since educators did not have much time to perfect online learning when
schools shuttered suddenly in the spring.

Most of the state's districts have submitted their reopening plans
already, Mr. Cuomo said, though about 50 of the proposals were still
incomplete. New York City was granted a two-week extension on their
plan, and the final version was submitted Friday.

Districts across the state are mandating mask use and social distancing,
and many have proposed different hybrid models that would involve
cohorts of children cycling in and out of buildings on alternating days.
Some small districts are planning to bring students back into buildings
full-time.

Other cities across the state have taken different approaches. Buffalo's
superintendent has
already\href{https://www.wgrz.com/article/news/education/state-to-buffalo-schools-180-day-requirement-for-student-instruction-stands/71-75983566-9d86-4b01-8463-621ce0f86d0b}{indicated
that he might prefer to delay the start of in-person learning until
October}. Mr. Cuomo said Friday that districts could decide when to
reopen buildings as long as their infection rate stayed low. In
Syracuse,
\href{https://www.syracuse.com/schools/2020/08/central-ny-school-reopening-plans-are-finally-in-none-are-the-same.html}{most
high school students will likely learn remotely}.

\href{https://www.nytimes3xbfgragh.onion/2020/07/29/us/teacher-union-school-reopening-coronavirus.html}{Many
teachers} and parents across the state have expressed alarm about
returning to school buildings as the virus has spiked in other states.
But some families say they are desperate for schools and child care
centers to open so that they can return to work. About 75 percent of New
York City students are low-income and many of their parents are
essential workers or employees who cannot work from home.

Some local officials, including Mr. de Blasio, have indicated that they
may not make a final decision whether to reopen, even part-time, until
late this month or even early next month.

New York City's system, with 1.1 million children and 1,800 schools, is
planning to open on hybrid model, in which children report to school one
to three days a week and learn online the rest of the time.

Advertisement

\protect\hyperlink{after-bottom}{Continue reading the main story}

\hypertarget{site-index}{%
\subsection{Site Index}\label{site-index}}

\hypertarget{site-information-navigation}{%
\subsection{Site Information
Navigation}\label{site-information-navigation}}

\begin{itemize}
\tightlist
\item
  \href{https://help.nytimes3xbfgragh.onion/hc/en-us/articles/115014792127-Copyright-notice}{©~2020~The
  New York Times Company}
\end{itemize}

\begin{itemize}
\tightlist
\item
  \href{https://www.nytco.com/}{NYTCo}
\item
  \href{https://help.nytimes3xbfgragh.onion/hc/en-us/articles/115015385887-Contact-Us}{Contact
  Us}
\item
  \href{https://www.nytco.com/careers/}{Work with us}
\item
  \href{https://nytmediakit.com/}{Advertise}
\item
  \href{http://www.tbrandstudio.com/}{T Brand Studio}
\item
  \href{https://www.nytimes3xbfgragh.onion/privacy/cookie-policy\#how-do-i-manage-trackers}{Your
  Ad Choices}
\item
  \href{https://www.nytimes3xbfgragh.onion/privacy}{Privacy}
\item
  \href{https://help.nytimes3xbfgragh.onion/hc/en-us/articles/115014893428-Terms-of-service}{Terms
  of Service}
\item
  \href{https://help.nytimes3xbfgragh.onion/hc/en-us/articles/115014893968-Terms-of-sale}{Terms
  of Sale}
\item
  \href{https://spiderbites.nytimes3xbfgragh.onion}{Site Map}
\item
  \href{https://help.nytimes3xbfgragh.onion/hc/en-us}{Help}
\item
  \href{https://www.nytimes3xbfgragh.onion/subscription?campaignId=37WXW}{Subscriptions}
\end{itemize}
