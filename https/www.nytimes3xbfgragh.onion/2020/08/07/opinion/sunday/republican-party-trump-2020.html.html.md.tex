\href{/section/opinion/sunday}{Sunday Review}\textbar{}Where Do
Republicans Go From Here?

\url{https://nyti.ms/2Cdq1cs}

\begin{itemize}
\item
\item
\item
\item
\item
\item
\end{itemize}

\includegraphics{https://static01.graylady3jvrrxbe.onion/images/2020/08/09/opinion/sunday/09brooks/09brooks-articleLarge.jpg?quality=75\&auto=webp\&disable=upscale}

Sections

\protect\hyperlink{site-content}{Skip to
content}\protect\hyperlink{site-index}{Skip to site index}

\href{/section/opinion}{Opinion}

\hypertarget{where-do-republicans-go-from-here}{%
\section{Where Do Republicans Go From
Here?}\label{where-do-republicans-go-from-here}}

The party looks brain-dead at every spot Trump touches. But off in the
corners, there's a lot of intellectual ferment.

Credit...Tim Enthoven

Supported by

\protect\hyperlink{after-sponsor}{Continue reading the main story}

\href{https://www.nytimes3xbfgragh.onion/by/david-brooks}{\includegraphics{https://static01.graylady3jvrrxbe.onion/images/2018/04/03/opinion/david-brooks/david-brooks-thumbLarge-v2.png}}

By \href{https://www.nytimes3xbfgragh.onion/by/david-brooks}{David
Brooks}

Opinion Columnist

\begin{itemize}
\item
  Aug. 7, 2020
\item
  \begin{itemize}
  \item
  \item
  \item
  \item
  \item
  \item
  \end{itemize}
\end{itemize}

Jonathan V. Last thinks President Trump is here forever. Last, the
editor of The Bulwark, a conservative site that's been hostile to Trump,
argues that if Trump loses in November, he'll claim he was cheated out
of the election. He'll force other Republicans to back up his claim.
He'll get a TV show, hold rallies, be coy about running again in 2024.

He'll still be the center of everything Republican. Ambitious
Republicans will have to lash themselves to the husk of the dying czar
if they want to have any future in the party. The whole party will go
Trump-crazed and brain-dead for another four years.

I salute Last for coming up with a post-2020 scenario even more
pessimistic than my own!

My guess is that if Trump gets crushed in the election, millions of
Republicans will decide they never liked that loser and jerk anyway.
He'll get relegated to whatever bargain basement they are using to hold
Sarah Palin. But something will remain: Trumpism.

The basic Trump worldview --- on immigration, trade, foreign policy,
etc. --- will shape the G.O.P. for decades, the way the basic Reagan
worldview did for decades. A thousand smarter conservatives will be
building a new party after 2020, but one that builds from the framework
Trump established.

I think Trumpism will survive Trump because the history of the modern
Republican Party is the history of paradigm shifts.

If you came of age with conservative values and around Republican
politics in the 1980s and 1990s, you lived within a certain Ronald
Reagan-Margaret Thatcher paradigm. It was about limiting government,
spreading democracy abroad, building dynamic free markets at home and
cultivating people with vigorous virtues --- people who are energetic,
upright, entrepreneurial, independent-minded, loyal to friends and
strong against foes.

\includegraphics{https://static01.graylady3jvrrxbe.onion/images/2020/08/07/opinion/07brooks1/07brooks1-articleLarge.jpg?quality=75\&auto=webp\&disable=upscale}

For decades conservatives were happy to live in that paradigm. But as
years went by many came to see its limits. It was so comprehensively
anti-government that it had no way to use government to solve common
problems. It was so focused on cultivating strong individuals that it
had no language to cultivate a sense of community and belonging. So, if
you were right of center, you leapt. You broke from the Reagan paradigm
and tried to create a new, updated conservative paradigm.

My own leap came early. On Sept. 15, 1997, William Kristol and I
\href{https://www.wsj.com/articles/SB874276753849168000}{wrote a piece}
for The Wall Street Journal on what we called National Greatness
Conservatism. We argued that the G.O.P. had become too anti-government.
``How can Americans love their nation if they hate its government?'' we
asked. Only a return to the robust American nationalism of Alexander
Hamilton, Henry Clay and Theodore Roosevelt would do: ambitious national
projects, infrastructure, federal programs to increase social mobility.

The closest National Greatness Conservatism came to influencing the
party was John McCain's 2000 presidential bid. He was defeated by a man,
George W. Bush, who made his own leap, to Compassionate Conservatism.
(You know somebody has made a paradigm leap when he or she starts adding
some modifying word or phrase before ``Conservatism.'') This was an
attempt to meld Catholic social teaching to conservatism.

Image

John McCain campaigning in New Hampshire in 2000.Credit...Henny Ray
Abrams/Agence France-Presse --- Getty Images

There were many other leaps over the decades. Sam's Club Republicans,
led by Reihan Salam and my Times colleague Ross Douthat, pointed a way
to link the G.O.P. to working-class concerns. Front Porch Republicans
celebrated small towns and local communities. The Reformicons tried to
use government to build strong families and neighborhoods. The Niskanen
Center is an entire think tank for people who have leapt from
libertarianism.

Most actual Republican politicians rejected all of this. They stuck,
mostly through dumb inertia, to an anti-government zombie Reaganism
decades after Reagan was dead and even though the nation's problems were
utterly different from what they were when he was alive. Year after
year, G.O.P. politicians clung to a dead paradigm, ran the same
anti-Washington campaigns and had no positive governing philosophy once
they got there.

Steve Bannon's leap finally did what none of us could do. Donald Trump
and Bannon took a low-rent strand of conservatism --- class-based ethnic
nationalism --- that had always been locked away in the basement of the
American right, and overturned the Reagan paradigm.

Image

In 2016, Steve Bannon showed he understood what motivated Republican
voters.~Credit...Stephen Crowley/The New York Times

Bannon and Trump got the emotions right. They understood that Republican
voters were no longer motivated by a sense of hope and opportunity; they
were motivated by a sense of menace, resentment and fear. At base, many
Republicans felt they were being purged from their own country --- by
the educated elite, by multiculturalism, by militant secularism.

During the 2016 presidential campaign, Trump and Bannon discarded the
Republican orthodox --- entitlement reform, fiscal restraint, free
trade, comprehensive immigration reform. They embraced a European-style
blood-and-soil conservatism. Close off immigration. Close trade. We have
nothing to offer the world and should protect ourselves from its
dangers.

It would have been interesting if Trump had governed as a big-government
populist. But he tossed Bannon out and handed power to Jared Kushner and
a bunch of old men locked in the Reagan paradigm. We got bigotry,
incompetence and tax cuts for the wealthy.

Image

President Trump with Jared Kushner at the White House in
2017.Credit...Tom Brenner/The New York Times

But by defeating the Reagan paradigm, Trump and Bannon gave permission
to a lot of Republican politicians to make their own leaps. Over the
last three years, it's been interesting to watch a series of Republican
officeholders break free from old orthodoxies and begin to think afresh.
You could see their eyes get wider: \emph{Suddenly I can think for
myself. The range of possibilities is wider than I thought it was.}

Their newfound liberation didn't extend to crossing Trump, but because
the president's political vision isn't exactly what you'd call fleshed
out, there's a lot of running room within his paradigm.

The post-2020, post-Trump Republican future is contained in those leaps.
And that future is embodied by a small group of Republican senators in
their 40s, including Marco Rubio, Josh Hawley, Tom Cotton and Ben Sasse.
They all came of age when Reaganism was already in the rearview mirror.
Though populist, three of them have advanced degrees from Harvard or
Yale. They are not particularly close to one another. They may be joined
by a common experience, but they are divided by ambition.

Each has a different vision of where the country should go, but they
start with certain common Trumpian premises:

\textbf{Everything is not OK.} The free market is not working well.
Wages are stagnant. Too much power is in the hands of the corporate
elites. Middle America is getting screwed. Finance capitalism is
unbalanced. American society is in abject decline. If Reaganism was
``Let's be free,'' the new mood is ``Take control.''

\textbf{Economic libertarianism is not the answer}. Free markets alone
won't solve our problems. G.D.P. growth alone is not the be-all and
end-all of politics. We need policies to shore up the conservative units
of society --- family, neighborhood, faith, nation. We need policies
that build solidarity, not just liberty.

\textbf{The working class is the heart of the Republican Party.} Once,
businesspeople and entrepreneurs were at the center of the Republican
imagination. Now it's clear that the party needs to stop catering to the
corporate class and start focusing on the shop owners, the plumbers, the
salaried workers. It needs to emphasize the dignity of work and honor
those who are not trying to make millions, not looking for handouts, but
just want to build middle-class lives in a stable social order. In
Britain, the Conservative Party has built a majority around the working
class, and that's what Republicans need to do here.

\textbf{China changes everything.} The rise of a 1.4-billion-person
authoritarian superpower means that free trade no longer works because
the Chinese are not playing by the same rules. The U.S. government
cannot just stand back and let China control the new technologies.
``Republicans are going to have to get used to the idea of industrial
policy to counter China, at least in a few key industries,'' Mike
Gallagher, a rising star among House Republicans, told me.

\textbf{The managerial class betrays America.} Many of the post-Reagan
positions seem like steps to the left. But these Republicans combine a
greater willingness to use government with a greater hostility to the
managerial class. The solution to too much corporate power is not
handing power to Elizabeth Warren and a cloud of federal regulators.
There's a difference between empowering workers and empowering the
Washington elite.

From these common premises the four senators go off in different
directions.

Image

Senator Marco Rubio espouses ``common-good capitalism.''Credit...Erin
Schaff/The New York Times

Rubio bases his vision in Catholic social teaching. A year ago, he wrote
an essay for First Things titled,
\href{https://www.firstthings.com/web-exclusives/2019/08/what-economics-is-for}{``What
Economics Is For,''} arguing that the purpose of markets is not growth
but allowing each person to find dignity in work. He followed that up
with a
\href{https://www.rubio.senate.gov/public/_cache/files/6d09ae19-8df3-4755-b301-795154a68c59/C58480B07D02452574C5DB8D603803EF.final---cua-speech-11.5.19.pdf}{speech
at Catholic University} calling for ``common-good capitalism'' (remember
what I said about modifying phrases) in which he criticized contemporary
capitalism for its obsessive focus on maximizing shareholder value.

His basic position **** is that American capitalism has become too much
about finance. It needs to be balanced toward manufacturing. He, too,
supports a ``pro-American industrial policy'' to meet the Chinese
challenge.

Hawley is the most populist of the group. His core belief is that
middle-class Americans have been betrayed by their elites on every level
--- political elites, cultural elites, financial elites. The modern
leadership class has one set of values --- globalization,
cosmopolitanism --- and the Middle Americans have another set ---
family, home, rootedness, nation. Corporate elites have concentrated so
much power that they now crush the yeomen masses.

Image

Senator Josh Hawley says America's elites have betrayed the middle
class.Credit...Doug Mills/The New York Times

Last November, Hawley
\href{https://www.hawley.senate.gov/senator-hawleys-speech-rethinking-americas-foreign-policy-consensus}{gave
a speech} in which he sought to overturn the last 70 years of Republican
foreign policy. He contended that the right had erred in trying to
spread American values abroad. ``Imperial domination violates our
principles and it threatens our character. Our aim must be to prevent
imperialism, not to exercise it; to stop domination, not foster it,'' he
said.

Cotton has a less developed political vision but a more developed
attitude: hawkishness. Whether it's China, the left, immigration or Big
Tech, Cotton is hawkish. He sees a world threatened by disorder and
gravitates toward the toughest positions in order to ward off threat. He
is the most vocal foe of the Chinese ``pariah state.'' He wants sharp
reductions in legal immigration.

Image

Senator Tom Cotton is an unabashed hawk.~Credit...Erin Schaff for The
New York Times

Sasse is the most sociological of the crew. He is a Tocquevillian
localist, who notes that most normal Americans go days without thinking
of national politics. His vision is centered on the small associations
--- neighborhood groups, high school football teams, churches and
community centers --- where people find their greatest joys,
satisfactions and supports. Government's job, he says, is to ``create a
framework of ordered liberty'' so that people can make their family and
neighborhood the center of their lives.

He is the most suspicious of government and politics today. ``I think
politicians are arsonists,'' he told me over the phone last month. ``The
main thing the G.O.P. does is try to light the Democrats on fire, and
the main thing the Democrats do is light the Republicans on fire. That's
why there's so little trust in politics.''

Image

Senator Ben Sasse describes politicians as arsonists.Credit...Gabriella
Demczuk for The New York Times

Behind these public figures there is a posse of policy wonks and
commentators supporting a new Working-Class Republicanism, including
Oren Cass, Henry Olsen, J.D. Vance, Michael Brendan Dougherty, Saagar
Enjeti, Samuel Hammond and, in his own way, Tucker Carlson.

Cass, for example, has created a new think tank, the American Compass,
to push the G.O.P. in a post-Trump direction. Cass, a former adviser to
Mitt Romney, argues that free-market economists pay too much attention
to G.D.P. growth. What matters is the kind of growth and whether it
allows people to lead stable lives. He says there's too much emphasis on
consumption. People should be seen as producers, and government should
create the kind of jobs that allow people to earn dignity through work.

He says the core of the economy is the industrial economy:
manufacturing, transportation, infrastructure --- making things in the
physical world. ``Investment in our economy has completely discounted
the making of stuff,'' he told me in a recent interview. ``You have a
V.C. industry that goes entirely to software. Private equity financial
flow is about buying and trading companies.'' Government needs to engage
in ``predistribution,'' to steer investment to manufacturing, and also
to those Middle American parts of the country that are currently left
out.

``The American labor force cannot be changed into what the economy
wants,'' Cass says. ``We have to change the economy to what the American
labor force can be successful in.''

Image

A line of people in June seeking help in Kentucky with their
unemployment claims.~Credit...Bryan Woolston/Reuters

The intellectual future of conservatism will be wrestled over at a
series of forums at the Center for Social, Cultural and Constitutional
Studies at the American Enterprise Institute that are being organized by
Yuval Levin, a scholar there. Right now, the various factions are
exchanging sarcastic one-liners on Twitter. Levin is bringing the
players together. ``People should be talking to each other, not about
each other,'' he told me.

Levin thinks the prevailing post-Trump viewpoints define the problem too
much in economic terms. The crucial problem, he argues, is not economic;
it's social: alienation. Millions of Americans don't feel part of
anything they can trust. They feel no one is looking out for them. Trump
was a false answer to their desire for social solidarity, but the desire
can be a force for good.

``What's needed,'' Levin says, ``is not just to expand economic
conservatism beyond growth to also prioritize family, community and
nation, but also to expand social conservatism beyond sexual ethics and
religious liberty to prioritize family, community and nation. The
coalition can be a powerful political force again if its different wings
converge on these priorities, without each giving up on its longstanding
aims.''

The Republican Party looks completely brain-dead at every spot Trump
directly reaches. Off in the corners, though, there's a lot of
intellectual ferment on the right. But if there is one thing I've
learned over the decades, it is never to underestimate the staying power
of the dead Reagan paradigm.

The Wall Street Journal editorial page stands as a vigilant guardian of
the corpse, eager to rebut all dissenters. The former U.N. ambassador
Nikki Haley and Senator Pat Toomey of Pennsylvania are staunch defenders
of Minimal-Government Conservatism. Senator Ted Cruz seems to be
positioning himself for a 2024 presidential run that seeks to
triangulate all the pre-Trump and pro-Trump versions of the party into
one stew.

And if Joe Biden defeats Trump and begins legislating, as seems more and
more likely, there's also the possibility that Republicans will abandon
any positive vision and revert to being a simple anti-government party
--- a party of opposition to whatever Biden is doing.

But over the long term, some version of Working-Class Republicanism will
redefine the G.O.P. In the first place, that's where Republican voters
are. When push comes to shove, Republican politicians are going to
choose their voters over their donor class.

Second, the working-class emphasis is the only way out of the
demographic doom loop. If the party sticks with its old white high
school-educated base, it will die. They just aren't making enough old
white men. To have any shot of surviving as a major party, the G.O.P.
has to build a cross-racial alliance among working-class whites,
working-class Hispanics and some working-class Blacks.

None of this works unless Republicans can deracialize their appeal ---
by which I mean they must stop pandering to the racists in the party and
stop presenting themselves and seeing themselves as the party of white
people --- and wage a class struggle between diverse workers in their
coalition and the highly educated coastal manager and professional class
in the Democratic coalition.

Rubio, Hawley, Sasse and Cotton are inching toward a G.O.P. future. What
are the odds they'll succeed? They've got to be way under 50-50.

\emph{The Times is committed to publishing}
\href{https://www.nytimes3xbfgragh.onion/2019/01/31/opinion/letters/letters-to-editor-new-york-times-women.html}{\emph{a
diversity of letters}} \emph{to the editor. We'd like to hear what you
think about this or any of our articles. Here are some}
\href{https://help.nytimes3xbfgragh.onion/hc/en-us/articles/115014925288-How-to-submit-a-letter-to-the-editor}{\emph{tips}}\emph{.
And here's our email:}
\href{mailto:letters@NYTimes.com}{\emph{letters@NYTimes.com}}\emph{.}

\emph{Follow The New York Times Opinion section on}
\href{https://www.facebookcorewwwi.onion/nytopinion}{\emph{Facebook}}\emph{,}
\href{http://twitter.com/NYTOpinion}{\emph{Twitter (@NYTopinion)}}
\emph{and}
\href{https://www.instagram.com/nytopinion/}{\emph{Instagram}}\emph{.}

Advertisement

\protect\hyperlink{after-bottom}{Continue reading the main story}

\hypertarget{site-index}{%
\subsection{Site Index}\label{site-index}}

\hypertarget{site-information-navigation}{%
\subsection{Site Information
Navigation}\label{site-information-navigation}}

\begin{itemize}
\tightlist
\item
  \href{https://help.nytimes3xbfgragh.onion/hc/en-us/articles/115014792127-Copyright-notice}{©~2020~The
  New York Times Company}
\end{itemize}

\begin{itemize}
\tightlist
\item
  \href{https://www.nytco.com/}{NYTCo}
\item
  \href{https://help.nytimes3xbfgragh.onion/hc/en-us/articles/115015385887-Contact-Us}{Contact
  Us}
\item
  \href{https://www.nytco.com/careers/}{Work with us}
\item
  \href{https://nytmediakit.com/}{Advertise}
\item
  \href{http://www.tbrandstudio.com/}{T Brand Studio}
\item
  \href{https://www.nytimes3xbfgragh.onion/privacy/cookie-policy\#how-do-i-manage-trackers}{Your
  Ad Choices}
\item
  \href{https://www.nytimes3xbfgragh.onion/privacy}{Privacy}
\item
  \href{https://help.nytimes3xbfgragh.onion/hc/en-us/articles/115014893428-Terms-of-service}{Terms
  of Service}
\item
  \href{https://help.nytimes3xbfgragh.onion/hc/en-us/articles/115014893968-Terms-of-sale}{Terms
  of Sale}
\item
  \href{https://spiderbites.nytimes3xbfgragh.onion}{Site Map}
\item
  \href{https://help.nytimes3xbfgragh.onion/hc/en-us}{Help}
\item
  \href{https://www.nytimes3xbfgragh.onion/subscription?campaignId=37WXW}{Subscriptions}
\end{itemize}
