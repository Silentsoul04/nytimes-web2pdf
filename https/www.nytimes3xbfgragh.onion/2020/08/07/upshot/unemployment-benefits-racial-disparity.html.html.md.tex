Sections

SEARCH

\protect\hyperlink{site-content}{Skip to
content}\protect\hyperlink{site-index}{Skip to site index}

\href{https://myaccount.nytimes3xbfgragh.onion/auth/login?response_type=cookie\&client_id=vi}{}

\href{https://www.nytimes3xbfgragh.onion/section/todayspaper}{Today's
Paper}

\href{/section/upshot}{The Upshot}\textbar{}Why Black Workers Will Hurt
the Most if Congress Doesn't Extend Jobless Benefits

\url{https://nyti.ms/2DGUhgc}

\begin{itemize}
\item
\item
\item
\item
\item
\item
\end{itemize}

Advertisement

\protect\hyperlink{after-top}{Continue reading the main story}

Upshot

Supported by

\protect\hyperlink{after-sponsor}{Continue reading the main story}

\hypertarget{why-black-workers-will-hurt-the-most-if-congress-doesnt-extend-jobless-benefits}{%
\section{Why Black Workers Will Hurt the Most if Congress Doesn't Extend
Jobless
Benefits}\label{why-black-workers-will-hurt-the-most-if-congress-doesnt-extend-jobless-benefits}}

An extra \$600 a week smoothed out sharp differences in benefits among
states, and among the people who lived in them.

\href{https://www.nytimes3xbfgragh.onion/by/emily-badger}{\includegraphics{https://static01.graylady3jvrrxbe.onion/images/2018/02/16/multimedia/author-emily-badger/author-emily-badger-thumbLarge-v2.png}}\href{https://www.nytimes3xbfgragh.onion/by/alicia-parlapiano}{\includegraphics{https://static01.graylady3jvrrxbe.onion/images/2018/12/10/multimedia/author-alicia-parlapiano/author-alicia-parlapiano-thumbLarge.png}}\href{https://www.nytimes3xbfgragh.onion/by/quoctrung-bui}{\includegraphics{https://static01.graylady3jvrrxbe.onion/images/2018/06/13/multimedia/author-quoctrung-bui/author-quoctrung-bui-thumbLarge-v2.png}}

By \href{https://www.nytimes3xbfgragh.onion/by/emily-badger}{Emily
Badger},
\href{https://www.nytimes3xbfgragh.onion/by/alicia-parlapiano}{Alicia
Parlapiano} and
\href{https://www.nytimes3xbfgragh.onion/by/quoctrung-bui}{Quoctrung
Bui}

\begin{itemize}
\item
  Aug. 7, 2020Updated 11:11 a.m. ET
\item
  \begin{itemize}
  \item
  \item
  \item
  \item
  \item
  \item
  \end{itemize}
\end{itemize}

\includegraphics{https://static01.graylady3jvrrxbe.onion/images/2020/08/07/upshot/07up-benefit/merlin_175341492_21cfac67-81ab-432a-bc22-fd38a5f98be3-articleLarge.jpg?quality=75\&auto=webp\&disable=upscale}

When Congress expanded unemployment insurance this year to meet the
staggering economic toll of the pandemic, it had one less-noticed
effect: It made America's fractured jobless benefits system more fair.

Starting in April, the federal government provided \$600 weekly payments
to unemployed workers in addition to state jobless benefits, smoothing
\href{https://www.nytimes3xbfgragh.onion/interactive/2020/04/23/business/economy/unemployment-benefits-stimulus-coronavirus.html}{sharp
differences between more and less generous states}. It also broadly
expanded who qualified, removing barriers for lower-wage, seasonal and
gig workers, who are typically excluded from aid. All of this had the
added effect of reducing racial disparities in unemployment benefits
that have for decades disadvantaged Black workers in particular.

Now, with the \$600 payments expired as of the end of July and with
congressional leaders and the White House
\href{https://www.nytimes3xbfgragh.onion/2020/08/05/us/politics/mcconnell-stimulus-talks-coronavirus.html?action=click\&module=Top\%20Stories\&pgtype=Homepage}{debating
whether to extend them}, Black workers stand to be hurt the most if they
fail to reach a deal.

This is in large part because Black workers disproportionately live in
states with the lowest benefit levels and
\href{https://www.nytimes3xbfgragh.onion/2020/04/30/upshot/unemployment-state-restrictions-pandemic.html}{the
highest barriers to receiving them}. Without the \$600 federal payments,
the most an unemployed worker in Florida or Alabama can receive is \$275
a week. Workers still covered under the expanded gig worker categories
would potentially get even less.

``It's just a pretty straightforward fact that one of the biggest
problems facing unemployed Black workers is that they live in places
with particularly inadequate unemployment insurance systems,'' said
Jared Bernstein, a senior fellow at the Center on Budget and Policy
Priorities, who served as chief economist to Joe Biden when he was vice
president.

The geographic pattern is not as stark for Hispanic workers,
\href{https://www.pewresearch.org/hispanic/2020/08/04/coronavirus-economic-downturn-has-hit-latinos-especially-hard/}{who
have also been hit hard in the coronavirus recession}. But they make up
more than a quarter of workers in two states with maximum weekly
benefits of less than \$300 --- Florida and Arizona.

Among Black workers, almost one in four live in just three states:
Florida, Georgia and Texas. And nearly 60 percent over all live in the
South, in states that tend to put the interests of businesses ahead of
those of workers, and where race itself has historically been
inseparable from policy decisions about the safety net.

Unemployment insurance in America was originally devised during the
Great Depression as a compromise between Northern Democrats who wanted
to expand worker aid and Southern Democrats
\href{https://www.nytimes3xbfgragh.onion/2005/08/28/books/review/when-affirmative-action-was-white-uncivil-rights.html}{who
didn't want to empower Black workers}. The resulting system --- a
network of state programs rather than a single federal one like Social
Security --- explicitly excluded domestic and agricultural workers. And
the states were given wide control that they retain today over how much
a worker has to earn to qualify for the program, how generous the
benefits are and how onerous the requirements.

The same pattern has persisted in heavily state-controlled programs
\href{https://www.urban.org/research/publication/why-does-cash-welfare-depend-where-you-live}{like
welfare}: The larger a state's Black population, the less generous its
benefits.

``Yesterday's racist system becomes today's incidental structural
racism,'' said Kathryn Edwards, an economist at the RAND Corporation.

She has found that the geographic concentration of Black workers in
stingier states means that the average maximum unemployment benefit a
Black worker in America can receive per week
\href{https://www.rand.org/blog/2020/07/the-racial-disparity-in-unemployment-benefits.html}{is
about \$40 less} than the average maximum benefit a white worker can
get. That number might sound small, but Ms. Edwards points out that it
adds up over 26 weeks of unemployment to a median rent payment in many
states, or nearly the size of a \$1,200 pandemic stimulus check.

If policymakers wanted to reduce racial disparities in what seem like
race-neutral unemployment programs, William Spriggs, a Howard University
economist, said they would want to do precisely the two things Congress
did: expand the categories of covered workers, and increase the benefits
they receive.

``What I did not anticipate fully and was shocked by,'' Mr. Spriggs
said, ``was the South is also bad about running these programs.''

The most complicated part of the federal expansion was the entirely new
program, called
\href{https://www.dol.gov/coronavirus/unemployment-insurance\#:~:text=The\%20new\%20law\%20creates\%20the,)\%2C\%20PEUC\%2C\%20PUA\%2C\%20Extended}{Pandemic
Unemployment Assistance}, meant for workers who wouldn't normally
qualify for state unemployment. This is the benefit that covers Uber
drivers, self-employed hair stylists, and tipped servers or part-time
retail workers whose reported earnings were too low to qualify normally.

Georgia and Florida were among the last states to begin making payments
through that additional program (Florida's labor force also has one of
the
\href{https://bfi.uchicago.edu/wp-content/uploads/BFI_White-Paper_Koustas_4.2020.pdf}{highest
shares of self-employed workers}). And last week, Gov. Ron DeSantis of
Florida publicly acknowledged that the state's
\href{https://www.nytimes3xbfgragh.onion/2020/04/23/us/florida-coronavirus-unemployment.html}{deeply
troubled unemployment system} introduced under the previous governor,
Rick Scott, had been set up
\href{https://miami.cbslocal.com/2020/08/04/exclusive-governor-ron-desantis-acknowledges-florida-unemployment-system-designed-frustrate/}{to
frustrate workers and make as few payments as possible}.

Mr. Spriggs fears that racial disparities are embedded in these delays
in receiving benefits, too. And because unemployment has remained
stubbornly high for Blacks
\href{https://www.nytimes3xbfgragh.onion/interactive/2020/05/13/upshot/coronavirus-america-job-losses-slowing-tracker.html}{in
surveys} even as it has fallen more for other groups, Black workers are
likelier to face longer spells of unemployment without the added
benefits.

All of these choices in the unemployment system are layered on top of
\href{https://www.nytimes3xbfgragh.onion/2020/06/01/business/economy/black-workers-inequality-economic-risks.html}{racial
disparities that exist in the economy} even during better times. Black
workers have less wealth to cushion them when they lose income. And they
tend to experience unemployment longer, as they face discrimination
finding work again. Initial evidence already suggests that Black workers
were
\href{https://eml.berkeley.edu/~jrothst/workingpapers/bartik_bertrand_etal_july212020.pdf}{less
likely to be rehired in May and June} as some businesses reopened.

``We've come to grips with the fact that Black lives are devalued as it
relates to engagement with law enforcement,'' said Darrick Hamilton, who
leads the Kirwan Institute for the Study of Race and Ethnicity at Ohio
State. ``So why is it a leap of faith to believe that that devaluation
would not be limited to law enforcement?''

It exists as well, he said, in how the economy values Black workers.

The initial shock of the pandemic hit workers wherever they happened to
be in the economy, particularly in jobs requiring close contact. But now
the labor market is rebalancing, a process that plays out to the
advantage of white workers, Mr. Spriggs said. He expects things will
settle where they invariably do: with Black workers having about twice
the unemployment rate of white ones.

As he watches that happen, Mr. Spriggs takes issue with familiar
arguments
\href{https://thehill.com/homenews/sunday-talk-shows/510172-mnuchin-on-600-unemployment-benefit-we-cant-be-paying-people-more}{made
by White House officials} and
\href{https://www.nbcnews.com/politics/congress/600-federal-unemployment-benefit-ends-month-gop-senators-say-enough-n1232588}{congressional
Republicans} that they don't want to give workers money to sit at home.

``It stretches credulity, in the worst labor market ever, and with a
record number of Americans unemployed, for someone to suggest that, `Oh,
if I give these people money, the big problem is they won't work,''' Mr.
Spriggs said. ``Their characterization of workers as inherently lazy,
that's a dog whistle to me.''

Representative Bobby Scott, a Democrat from Virginia and the chairman of
the House Committee on Education and Labor, said he hears in those
arguments something else.

``What I hear are things that are not true,'' he said. There just aren't
jobs for many people to go back to. And workers generally can't refuse
to return to work and still keep their unemployment benefits. ``It's
hard to analyze the statement in terms of impact or philosophy,'' Mr.
Scott said, ``if you start off with an understanding that it's not true
to begin with.''

Advertisement

\protect\hyperlink{after-bottom}{Continue reading the main story}

\hypertarget{site-index}{%
\subsection{Site Index}\label{site-index}}

\hypertarget{site-information-navigation}{%
\subsection{Site Information
Navigation}\label{site-information-navigation}}

\begin{itemize}
\tightlist
\item
  \href{https://help.nytimes3xbfgragh.onion/hc/en-us/articles/115014792127-Copyright-notice}{©~2020~The
  New York Times Company}
\end{itemize}

\begin{itemize}
\tightlist
\item
  \href{https://www.nytco.com/}{NYTCo}
\item
  \href{https://help.nytimes3xbfgragh.onion/hc/en-us/articles/115015385887-Contact-Us}{Contact
  Us}
\item
  \href{https://www.nytco.com/careers/}{Work with us}
\item
  \href{https://nytmediakit.com/}{Advertise}
\item
  \href{http://www.tbrandstudio.com/}{T Brand Studio}
\item
  \href{https://www.nytimes3xbfgragh.onion/privacy/cookie-policy\#how-do-i-manage-trackers}{Your
  Ad Choices}
\item
  \href{https://www.nytimes3xbfgragh.onion/privacy}{Privacy}
\item
  \href{https://help.nytimes3xbfgragh.onion/hc/en-us/articles/115014893428-Terms-of-service}{Terms
  of Service}
\item
  \href{https://help.nytimes3xbfgragh.onion/hc/en-us/articles/115014893968-Terms-of-sale}{Terms
  of Sale}
\item
  \href{https://spiderbites.nytimes3xbfgragh.onion}{Site Map}
\item
  \href{https://help.nytimes3xbfgragh.onion/hc/en-us}{Help}
\item
  \href{https://www.nytimes3xbfgragh.onion/subscription?campaignId=37WXW}{Subscriptions}
\end{itemize}
