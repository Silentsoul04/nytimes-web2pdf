Sections

SEARCH

\protect\hyperlink{site-content}{Skip to
content}\protect\hyperlink{site-index}{Skip to site index}

\href{https://www.nytimes3xbfgragh.onion/section/us}{U.S.}

\href{https://myaccount.nytimes3xbfgragh.onion/auth/login?response_type=cookie\&client_id=vi}{}

\href{https://www.nytimes3xbfgragh.onion/section/todayspaper}{Today's
Paper}

\href{/section/us}{U.S.}\textbar{}Abolish the Police? Those Who Survived
the Chaos in Seattle Aren't So Sure

\href{https://nyti.ms/2DDznOW}{https://nyti.ms/2DDznOW}

\begin{itemize}
\item
\item
\item
\item
\item
\item
\end{itemize}

\href{https://www.nytimes3xbfgragh.onion/news-event/george-floyd-protests-minneapolis-new-york-los-angeles?action=click\&pgtype=Article\&state=default\&region=TOP_BANNER\&context=storylines_menu}{Race
and America}

\begin{itemize}
\tightlist
\item
  \href{https://www.nytimes3xbfgragh.onion/interactive/2020/07/03/us/george-floyd-protests-crowd-size.html?action=click\&pgtype=Article\&state=default\&region=TOP_BANNER\&context=storylines_menu}{Black
  Lives Matter Movement}
\item
  \href{https://www.nytimes3xbfgragh.onion/interactive/2020/06/28/us/i-cant-breathe-police-arrest.html?action=click\&pgtype=Article\&state=default\&region=TOP_BANNER\&context=storylines_menu}{History
  of `I Can't Breathe'}
\item
  \href{https://www.nytimes3xbfgragh.onion/interactive/2020/06/10/upshot/black-lives-matter-attitudes.html?action=click\&pgtype=Article\&state=default\&region=TOP_BANNER\&context=storylines_menu}{How
  Public Opinion Shifted}
\item
  \href{https://www.nytimes3xbfgragh.onion/interactive/2020/07/16/us/black-lives-matter-protests-louisville-breonna-taylor.html?action=click\&pgtype=Article\&state=default\&region=TOP_BANNER\&context=storylines_menu}{45
  Days in Louisville}
\end{itemize}

Advertisement

\protect\hyperlink{after-top}{Continue reading the main story}

Supported by

\protect\hyperlink{after-sponsor}{Continue reading the main story}

\hypertarget{abolish-the-police-those-who-survived-the-chaos-in-seattle-arent-so-sure}{%
\section{Abolish the Police? Those Who Survived the Chaos in Seattle
Aren't So
Sure}\label{abolish-the-police-those-who-survived-the-chaos-in-seattle-arent-so-sure}}

What is it like when a city abandons a neighborhood and the police
vanish? Business owners describe a harrowing experience of calling for
help and being left all alone.

\includegraphics{https://static01.graylady3jvrrxbe.onion/images/2020/09/03/us/03CHAZ1/merlin_174979071_1ce12ee2-37f8-44d8-8a9c-d593f10b64ca-articleLarge.jpg?quality=75\&auto=webp\&disable=upscale}

By \href{https://www.nytimes3xbfgragh.onion/by/nellie-bowles}{Nellie
Bowles}

\begin{itemize}
\item
  Aug. 7, 2020Updated 6:00 p.m. ET
\item
  \begin{itemize}
  \item
  \item
  \item
  \item
  \item
  \item
  \end{itemize}
\end{itemize}

SEATTLE --- Faizel Khan was being told by the news media and his own
mayor that the protests in his hometown were peaceful, with
``\href{https://www.realclearpolitics.com/video/2020/06/12/seattle_mayor_durkan_chaz_has_a_block_party_atmosphere_could_turn_into_summer_of_love.html}{a
block party atmosphere}.''

But that was not what he saw through the windows of his Seattle coffee
shop. He saw encampments overtaking the sidewalks. He saw roving bands
of masked protesters smashing windows and looting.

Young white men wielding guns would harangue customers as well as Mr.
Khan, a gay man of Middle Eastern descent who moved here from Texas so
he could more comfortably be out. To get into his coffee shop, he
sometimes had to seek the permission of self-appointed armed guards to
cross a border they had erected.

``They barricaded us all in here,'' Mr. Khan said. ``And they were
sitting in lawn chairs with guns.''

For 23 days in June, about six blocks in the city's Capitol Hill
neighborhood were claimed by left-wing demonstrators and declared
police-free. Protesters hailed it as liberation --- from police
oppression, from white supremacy --- and a catalyst for a national
movement.

In the wake of the killing of George Floyd by the Minneapolis police,
the Black Lives Matter movement is calling to defund the police, arguing
that the criminal justice system is inherently racist.

Leaders in many progressive cities are listening. In New York City,
Mayor Bill de Blasio has announced a plan to
\href{https://www.nytimes3xbfgragh.onion/2020/06/30/nyregion/nypd-budget.html}{shift
\$1 billion out of the police budget}. The Minneapolis City Council is
\href{https://www.nytimes3xbfgragh.onion/2020/06/07/us/minneapolis-police-abolish.html}{pitching
a major reduction}, and the Seattle City Council is pushing for a 50
percent cut to Police Department funding. (The mayor said that plan goes
too far.)

Some even call for
``\href{https://www.nytimes3xbfgragh.onion/2020/06/12/opinion/sunday/floyd-abolish-defund-police.html}{abolishing
the police}'' altogether and closing down precincts, which is what
happened in Seattle.

That has left small-business owners as lonely voices in progressive
areas, arguing that police officers are necessary and that cities cannot
function without a robust public safety presence. In Minneapolis,
Seattle and Portland, Ore., many of those business owners consider
themselves progressive, and in interviews they express support for the
Black Lives Matter movement. But they also worry that their businesses,
already debilitated by the coronavirus pandemic, will struggle to
survive if police departments and city governments cannot protect them.

On Capitol Hill, business crashed as the Seattle police refused to
respond to calls to the area. Officers did not retake the region until
July 1, after four shootings,
\href{https://www.nytimes3xbfgragh.onion/2020/06/22/us/seattle-autonomous-zone-chop-shootings.html}{including
two fatal ones}.

Now a group of local businesses owners --- including a locksmith, the
owner of a tattoo parlor, a mechanic, the owners of a Mexican restaurant
and Mr. Khan --- is
\href{https://www.seattletimes.com/seattle-news/capitol-hill-residents-and-businesses-sue-city-of-seattle-for-failing-to-disband-chop/}{suing
the city}. The lawsuit claims that ``Seattle's unprecedented decision to
abandon and close off an entire city neighborhood, leaving it unchecked
by the police, unserved by fire and emergency health services, and
inaccessible to the public'' resulted in enormous property damage and
lost revenue.

\includegraphics{https://static01.graylady3jvrrxbe.onion/images/2020/09/03/us/03CHAZ4/merlin_174979347_d7f2bc3f-d248-4708-8821-994a9e359c47-articleLarge.jpg?quality=75\&auto=webp\&disable=upscale}

The Seattle lawsuit --- and interviews with shop owners in cities like
Portland and Minneapolis --- underscores a key question: Can businesses
still rely on local governments, which are now rethinking the role of
the police, to keep them safe? The issue is especially tense in Seattle,
where the city government not only permitted the establishment of a
police-free zone, but provided infrastructure like concrete barriers and
portable toilets to sustain it.

The economic losses that businesses suffered during the recent tumult
are significant: One community relief fund in Minneapolis, where early
protests included vandalism and arson, has raised \$9 million for
businesses along the Lake Street corridor, a largely
\href{http://www2.minneapolismn.gov/ncr/engagement/WCMS1P-143470}{Latino
and East African business district}. ``We asked the small businesses
what they needed to cover the damage that insurance wasn't paying, and
the gap was around \$200 million,'' said Allison Sharkey, the executive
director of the Lake Street Council, which is organizing the fund. Her
own office, between a crafts market and \href{https://www.diw-mn.org/}{a
Native American support center}, was burned down in the protests.

Some small businesses have resorted to posting
\href{https://www.gofundme.com/f/1101-e-lake-street-minneapolis-mn}{GoFundMe
pleas} for donations online.

Many are nervous about speaking out lest they lend ammunition to a
conservative critique of the Black Lives Matter movement. In Portland,
Elizabeth Snow McDougall, the owner of Stevens-Ness legal printers,
emphasized her support for the cause before describing the damage done
to her business.

``One window broken, then another, then another, then another. Garbage
to clean off the sidewalk in front of the store every morning. Urine to
wash out of our doorway alcove. Graffiti to remove,'' Ms. McDougall
wrote in an email. ``Costs to board up and later we'll have costs to
repair.''

The impact of the occupation on Cafe Argento, Mr. Khan's coffee shop on
Capitol Hill, has been devastating. Very few people braved the
barricades set up by the armed occupiers to come in for his coffee and
breakfast sandwiches. Cars coming to pick up food orders would turn
around. At two points, he and his workers felt scared and called 911.
``They said they would not come into CHOP,'' said Mr. Khan, referring to
one of the names that protesters gave to the occupied Capitol Hill area.
``It was lawless.''

He had to start chipping in for private security, a hard thing to do
when his business had already been hurt by the coronavirus.

But he considers himself lucky --- and he was. Even weeks after the
protests, blocks of his previously bustling neighborhood remained
boarded up and covered in shattered glass. Many business owners are
scared to speak out, Mr. Khan said, because of worries that they would
be targeted further.

One mid-July morning in the neighborhood, workers in orange vests were
mopping off the sidewalks and power-spraying graffiti off the sides of
buildings. Two window repair guys said they had their hands full for
weeks. Shattered street lamps were being unscrewed and replaced.

A confusing array of security teams wandered around, armed with handguns
and rifles. Some wore official-looking private security uniforms. Others
wore casual clothes and lanyards identifying their affiliation with
Black Lives Matter. A third group wore all black with no identifying
labels and declined to name their group affiliation.

Image

Rick Hearns displays two handguns while patrolling Capitol Hill as a
private security guard. His badge reads ``Black Lives Matter Community
Patrol.''Credit...Grant Hindsley for The New York Times

When a tall man in a trench coat and hiking boots walked over to
question Mr. Khan, the man spread his coat open, revealing several
pistols on harnesses around his chest and waist. He presented a badge on
a lanyard that read ``Black Lives Matter Community Patrol.''

His name is Rick Hearns and he identified himself as a longtime security
guard and mover who is now a Black Lives Matter community guard, in
charge of several others. Local merchants pay for his protection, he
said as he handed out his business card. (Mr. Khan said he and his
neighbors are now paying thousands of dollars a month for protection
from Iconic Global, a Washington State-based private security
contractor.)

Mr. Hearns has had bad experiences with the police in his own life. He
says he wants police reform, but he was appalled by the violent tactics
and rhetoric he witnessed during the occupation.

He blamed the destruction and looting on ``opportunists,'' but also said
that much of the damage on Capitol Hill came from a distinct contingent
of violent, armed white activists. ``It's antifa,'' he said. ``They
don't want to see the progress we've made. They want chaos.''

Many of the business owners on Capitol Hill agreed: Much of the violence
they saw and the intimidation of their patrons came from a group these
business owners identified as antifa, which they distinguished from the
Black Lives Matter movement. ``The idea of taking up the Black movement
and turning it into a white occupation, it's white privilege in its
finest definition,'' Mr. Khan said. ``And that's what they did.''

Antifa, which stands for anti-fascist, is
\href{https://www.nytimes3xbfgragh.onion/article/what-antifa-trump.html}{a
radical, leaderless leftist political movement} that uses
\href{https://newrepublic.com/article/154110/antifa-arming-trump-crackdown}{armed,
violent protest} as a method to create what supporters say is a more
just and equitable country. They have
\href{https://www.theatlantic.com/magazine/archive/2017/09/the-rise-of-the-violent-left/534192/}{a
strong presence in the Pacific Northwest}, including the current
protests in Portland.

Image

Police officers did not retake the Capitol Hill Occupied Protest zone,
where demonstrators graffitied walls and broke windows, until after four
shootings had occurred there.Credit...Grant Hindsley for The New York
Times

When the occupation in Seattle started in early June, Mayor Jenny Durkan
seemed almost amused. ``We could have the Summer of Love,'' she said.

After President Trump took aim at the governor of Washington State and
Seattle's mayor
\href{https://twitter.com/realDonaldTrump/status/1271142274416562176}{on
June 11}, Ms. Durkan
\href{https://twitter.com/MayorJenny/status/1271226494858129409}{defended
the occupation on Twitter} as ``a peaceful expression of our community's
collective grief and their desire to build a better world,'' she wrote,
pointing to the ``food trucks, spaghetti potlucks, teach-ins, and
movies.''

The lawsuit by the small-business owners, filed by the firm Calfo Eakes
on June 24, seizes on such language, pointing out that the city knew
what was happening and provided material support for the occupation.

Matthew Ploszaj, a Capitol Hill resident, is one of the complainants. He
said his apartment building, blocks from Mr. Khan's shop, was broken
into four times during the occupation. The Seattle Police were called
each time and never came to his apartment, according to Mr. Ploszaj.
When he and another resident called the police after one burglary, they
told him to meet them outside the occupation zone, about eight blocks
away. He and other residents spent nights at a friend's house outside
the area during the height of the protests.

The employees of Bergman's Lock and Key say they were followed by
demonstrators with baseball bats. Cure Cocktail, a local bar and
charcuterie, said its workers were asked by protesters to pledge loyalty
to the movement: ``Are you for the CHOP or are you for the police?''
they were asked, according to the lawsuit.

The business owners also found that trying to get help from the Seattle
Police, who declined to comment for this article, made them targets of
activists.

Across from Cafe Argento is a funky old auto repair shop called Car
Tender run by John McDermott, a big soft-spoken man. On June 14, Mr.
McDermott was driving his wife home from their anniversary dinner when
he received a call from a neighbor who saw someone trying to break into
his shop.

Mr. McDermott and his 27-year-old son, Mason, raced over. A man who was
inside the shop, Mr. McDermott said, had emptied the cash drawer and was
in the midst of setting the building on fire. Mr. McDermott said he and
his son wrestled the man down and planned to hold him until the police
arrived. But officers never showed up. A group of several hundred
protesters did, according to Mr. McDermott, breaking down the chain-link
fence around his shop and claiming that Mr. McDermott had kidnapped the
man.

``They started coming across the fence --- you see all these beautiful
kids, a mob but kids --- and they have guns and are pointing them at you
and telling you they're going to kill you,'' Mr. McDermott said.
``Telling me I'm the K.K.K. I'm not the K.K.K.''

The demonstrators were livestreaming the confrontation. Mr. McDermott's
wife watched, frantically calling anyone she could think of to go help
him.

Later, Mr. McDermott's photo and shop address appeared on a website
called Cop Blaster, whose stated aim is to track police brutality but
also has galleries of what it calls ``Snitches'' and ``Cop Callers.''
The McDermotts were categorized as both of those things on the website,
which warned they should ``keep their mouths shut.''

Image

John McDermott said the police did not respond when his auto repair shop
on Capitol Hill was being robbed by someone trying to set the building
on fire.Credit...Grant Hindsley for The New York Times

Many of the listings include names and addresses of people who are said
to have called the police. Since the Cop Blaster post went up, Mr.
McDermott's shop has received so many harassing phone calls and messages
that some employees have had to take time off.

A block away is Bill Donner, the owner of Richmark Label, who let police
officers use the roof of his factory to monitor the demonstration.
Inside, his company had spent 50 years making labels for products like
whiskey, soaps and natural beef jerky. Many days during the occupation,
Mr. Donner, who said he was in favor of police reform, had to negotiate
with the occupiers of the zone for access to his factory.

Twice, he called 911 and was told that the police would not be coming
into the area.

The experience of the small-business owners seems a universe away from
the rhetoric of Seattle's politicians. As the violence turned deadly,
Councilwoman Kshama Sawant, who represents Capitol Hill, defended the
protesters' use of their own armed guards instead of the police.

``Elected committees of self defense have historically played vital
roles during general strikes, occupations and in mass movements, in
order for the working class and marginalized people to defend themselves
and carry out necessary functions in place of the forces of the state,''
\href{https://council.seattle.gov/2020/06/20/statement-on-the-shooting-at-the-capitol-hill-organized-protest/}{she
wrote}. She has called for the local police precinct to be permanently
placed under ``community control.''

When the mayor did send in police officers to end the occupation after
the shootings, Ms. Sawant
\href{https://twitter.com/cmkshama/status/1278848527460757505?lang=en}{wrote
on Twitter}, ``Shame on Mayor Jenny Durkan for deploying Seattle police
yesterday in a brutal attack against peaceful Black Lives Matter
protesters \& homeless neighbors at the Capitol Hill Organized
Protest.''

Many protesters who remained in early July were milling around a small
tent encampment on a lawn at Seattle Central College, some with rifles
slung over their shoulders. The smell of weed drifted through. The
streets were full of moving trucks.

The crowds were gone, but every now and then, the demonstrators gave
speeches about the importance of disbanding the police. Sometimes the
activists spoke about what went wrong with the occupation. One young
woman on a bullhorn argued to passers-by that the police left too
quickly and that a sustainable police-free region would have to be built
more slowly.

These days, storefronts in the neighborhood remain boarded up, covered
in Black Lives Matter signs and graffiti. Demonstrators still hold
evening protests, albeit smaller and quieter than before. But the
businesses remain on edge.

``This is an ongoing crisis,'' Mr. Donner said on Tuesday. ``Protesters
are apparently staying until they get some of what they want. No one
knows what level of city cooperation will be enough for them.''

But the area is slowly going back to its old normal. The park and
playing fields have been cleared, and police officers have returned to
the streets. An apartment building that opened earlier this summer is
finally attracting prospective tenants.

A spokeswoman for Mayor Durkan did not comment on the lawsuit but
acknowledged frustrations from small businesses.

``Many who live and work in Capitol Hill and other parts of the city
continue to witness daily protests that are rightly demanding an end to
systemic racism,'' she wrote. ``In some circumstances, businesses and
residents have faced property destruction in the last two months.''

She encouraged the businesses to file claims.

Advertisement

\protect\hyperlink{after-bottom}{Continue reading the main story}

\hypertarget{site-index}{%
\subsection{Site Index}\label{site-index}}

\hypertarget{site-information-navigation}{%
\subsection{Site Information
Navigation}\label{site-information-navigation}}

\begin{itemize}
\tightlist
\item
  \href{https://help.nytimes3xbfgragh.onion/hc/en-us/articles/115014792127-Copyright-notice}{©~2020~The
  New York Times Company}
\end{itemize}

\begin{itemize}
\tightlist
\item
  \href{https://www.nytco.com/}{NYTCo}
\item
  \href{https://help.nytimes3xbfgragh.onion/hc/en-us/articles/115015385887-Contact-Us}{Contact
  Us}
\item
  \href{https://www.nytco.com/careers/}{Work with us}
\item
  \href{https://nytmediakit.com/}{Advertise}
\item
  \href{http://www.tbrandstudio.com/}{T Brand Studio}
\item
  \href{https://www.nytimes3xbfgragh.onion/privacy/cookie-policy\#how-do-i-manage-trackers}{Your
  Ad Choices}
\item
  \href{https://www.nytimes3xbfgragh.onion/privacy}{Privacy}
\item
  \href{https://help.nytimes3xbfgragh.onion/hc/en-us/articles/115014893428-Terms-of-service}{Terms
  of Service}
\item
  \href{https://help.nytimes3xbfgragh.onion/hc/en-us/articles/115014893968-Terms-of-sale}{Terms
  of Sale}
\item
  \href{https://spiderbites.nytimes3xbfgragh.onion}{Site Map}
\item
  \href{https://help.nytimes3xbfgragh.onion/hc/en-us}{Help}
\item
  \href{https://www.nytimes3xbfgragh.onion/subscription?campaignId=37WXW}{Subscriptions}
\end{itemize}
