Sections

SEARCH

\protect\hyperlink{site-content}{Skip to
content}\protect\hyperlink{site-index}{Skip to site index}

\href{/section/us}{U.S.}\textbar{}In a Wistful Age, Farmers Find a New
Angle: Chore TV

\url{https://nyti.ms/3knBUgT}

\begin{itemize}
\item
\item
\item
\item
\item
\item
\end{itemize}

\includegraphics{https://static01.graylady3jvrrxbe.onion/images/2020/08/07/us/07farmerinfluencer1/merlin_175042608_2a4dde30-9e9a-43b9-a4ad-6577fd58351c-articleLarge.jpg?quality=75\&auto=webp\&disable=upscale}

The Great Read

\hypertarget{in-a-wistful-age-farmers-find-a-new-angle-chore-tv}{%
\section{In a Wistful Age, Farmers Find a New Angle: Chore
TV}\label{in-a-wistful-age-farmers-find-a-new-angle-chore-tv}}

It's hard for small farmers to earn a living selling crops. Enter the
``farmer-influencer,'' who can earn more by streaming farm life, in all
its comforting monotony, to a growing online audience.

Credit...Hilary Swift for The New York Times

Supported by

\protect\hyperlink{after-sponsor}{Continue reading the main story}

\href{https://www.nytimes3xbfgragh.onion/by/ellen-barry}{\includegraphics{https://static01.graylady3jvrrxbe.onion/images/2018/10/08/multimedia/author-ellen-barry/author-ellen-barry-thumbLarge.png}}

By \href{https://www.nytimes3xbfgragh.onion/by/ellen-barry}{Ellen Barry}

\begin{itemize}
\item
  Aug. 7, 2020Updated 2:46 p.m. ET
\item
  \begin{itemize}
  \item
  \item
  \item
  \item
  \item
  \item
  \end{itemize}
\end{itemize}

PEACHAM, Vt. --- The sweet smell of hay rose off the earth on a recent
evening, as Morgan Gold strode across his farmyard in heavy boots. He
crossed the paddock, scanning for new eggs, water levels, infected peck
wounds, rips in the fence line.

But mainly --- let's be honest --- he was looking for content.

Though Mr. Gold sells poultry and eggs from
\href{https://www.youtube.com/channel/UCjl3zDun9SazYI0UWWu6_1A}{his duck
farm}\textbf{\href{https://www.youtube.com/channel/UCjl3zDun9SazYI0UWWu6_1A}{}}\href{https://www.youtube.com/channel/UCjl3zDun9SazYI0UWWu6_1A}{in
Vermont's northeast corner}, most of what he produces as a farmer is,
well, entertainment.

Mr. Gold, who is short and stocky, with the good-natured ease of a
standup comedian, does his chores while carrying a digital camera in one
hand and murmuring into a microphone.

Then, twice a week, like clockwork, he
\href{https://www.youtube.com/watch?v=XU-ylxJ55HQ\&t=130s}{posts a short
video on YouTube} about his exploits as a neophyte farmer, often
highlighting
\href{https://www.youtube.com/watch?v=mYmBQ-gv9gE\&t=20s}{failures} or
\href{https://www.youtube.com/watch?v=Kv2Q9jq8kxw}{pratfalls}. Keeping a
close eye on analytics, he has boosted his YouTube audiences high enough
to provide a steady advertising revenue of around \$2,500 to \$4,000 a
month, about eight times what he earns from selling farm products.

\includegraphics{https://static01.graylady3jvrrxbe.onion/images/2020/08/07/us/07farmerinfluencer6/merlin_175042551_c693c73a-105c-475f-8b22-7fe606a085d7-articleLarge.jpg?quality=75\&auto=webp\&disable=upscale}

This part of New England is rocky, hilly and isolated, and generations
of small farmers have cast about for new ways to scrape out a living:
the sleigh rides, the alpacas, the therapy ponies, the pick-your-own
hemp. It is a new thing, though, to make farm life into reality TV.

Mr. Gold, 40, has learned the hard way --- he tried to take a month off
last winter --- that any gap in his YouTube publication schedule results
in a steep drop-off in audience. So he keeps a running list of themes
that could be fodder for future videos. It reads, in part:

Should I Feed My Dog Eggs?

Don't Trust This Duck

My Homestead Is a Dumpster Fire

What Does My Guard Dog Do All Day?

He has learned, through trial and error, what works with an audience.
The sheepdog-mounted GoPro didn't work. (``People were like, 10 seconds
and I was puking,'' said his wife, Allison Ebrahimi Gold.) Slow,
sumptuous drone footage of his sun-dappled 150 acres, land porn for
wistful cubicle dwellers --- that definitely works.

Character development works, as demonstrated by Mr. Gold's
\href{https://www.youtube.com/watch?v=gwWKzq4WY3g}{most popular video},
``Our Freakishly Huge Duck (This Is Not NORMAL),'' which, as he would
put it, blew the doors off. ** Slow-motion footage of waggling goose
butts, set to a bouncy, whimsical orchestral soundtrack, works.

Image

Allison Ebrahimi Gold collects eggs at Gold Shaw Farm in Peacham,
Vt.Credit...Hilary Swift for The New York Times

But few things compel audiences, he came to realize, more than a
real-life setback. He came to this realization last summer
\href{https://www.youtube.com/watch?v=ym817n5S2eI\&feature=youtu.be}{when
a mink broke into his duck hutch}, leaving its interior spattered with
eggs and blood and feathers.

``It was one of the most depressing days of my life,'' he said, adding,
``but at the same time, I'm thinking, `How is the audience going to
react to this sort of thing?'''

The next videos, which featured freaky night-vision footage of the
offending mink, helped boost Mr. Gold's YouTube audience toward the
100,000-viewer threshold. And it helped him understand his own place in
the universe of farmer-influencers, which tilts heavily toward the
how-to genre.

``The storytelling part is what I'm good at,'' he said. ``I'm not that
good at the farming part.''

It is a paradox that the less financially viable small farming becomes,
the more that Americans want to experience it firsthand.

This idea is as old as the dude ranch; video streaming of farm life is
only the most recent iteration. Amy Fewell, the founder of Homesteaders
of America, said the number of farmers who earn substantial income off
YouTube channels is steadily climbing, and now stands at around 50.

Some of them earn money through product endorsement deals, like Al
Lumnah, who posts videos five days a week from
\href{https://www.youtube.com/c/DoingItGreen/about}{his farm in
Littleton, N.H}.

It's a lot of work: Mr. Lumnah wakes up at 3:30 a.m. so he can edit the
previous day's footage in time to post new video at 6 a.m., which his
210,000 regular viewers, who are scattered as far as Cambodia and India,
have come to expect. ``People will say, it's lunchtime here in
Ukraine,'' Mr. Lumnah said.

Others, like
\href{https://www.youtube.com/c/JustinRhodesVlog/featured}{Justin
Rhodes}, a farmer in North Carolina, have parlayed a giant YouTube
audience into a dues-paying membership enterprise --- he has 2,000 fans
who pay annual fees of up to \$249 for private instruction and direct
communication, via text message. ``We don't sell a single farm
product,'' Mr. Rhodes said. ``Our farm product is education and
entertainment.''

Mr. Gold, who moved to Vermont and started his YouTube channel four
years ago, has not reached that point. He still has a full-time job, as
a marketing executive for an insurance company, and so far has refused
the endorsement deals. He has built up his flocks of chicken, geese and
ducks to 100, and is hoping to **** add cows next spring.

Image

Mr. Gold has built up his flocks of chicken, geese and ducks to
100.Credit...Hilary Swift for The New York Times

He's certainly captured the interest of the farmers who surround him in
Peacham, said Tom Galinat, a neighbor, who runs his family's 550-acre
farm.

Farmers here struggle to eke out a living from a rocky, uneven soil and
hostile climate, and they are astounded --- in some cases a little
jealous --- to discover that Mr. Gold is internet famous, he said.

``He's found a way to way to monetize farming with less physical
labor,'' Mr. Galinat said. ``Some guys are like, this is silly, since
he's farming 20 ducks. But at the same time, he's making more than other
farmers who have 500 acres of land.''

But Mr. Galinat, who is also Peacham's town clerk, counts himself among
\href{https://www.foxfirerevivalfarm.com/}{a younger generation of
farmers} who are learning from Mr. Gold.

``He has taught me I am no longer selling hay, I am selling a
lifestyle,'' he said. ``He's really selling himself --- his emotions,
his opinions, his downfalls, his successes. Boom! That's it, that's the
way forward.''

As Mr. Gold's audience has grown, he has at times been taken aback by
the enthusiasm.

Several dozen viewers **** have driven all the way to Peacham and
knocked on his door, hoping to buy eggs or talk about ducks, something
his wife described as ``really distressing.''

``Morgan is so vulnerable on film,'' she said, ``that people assume they
know us as people.''

Most of it is nice, though. Viewers send handcrafted accessories for his
outbuildings, like a plaque that says, in elaborate lettering, ``Ye Olde
Quack House.'' When one of the Golds' barn cats was hit by a car
recently, **** at least 50 viewers offered cash to cover her medical
bills.

Samier Elrasoul, a nursing student in Howell, Mich., is so devoted to
Mr. Gold's videos that he got a vanity license plate reading QUACKN, in
honor of the catchphrase --- ``Release the Quacken!'' --- that Mr. Gold
exclaims when he frees his ducks from their hutch in the morning.

Mr. Elrasoul, 34, says the videos inspire him because he, too, has a
dead-end job --- he works as a supervisor at Starbucks --- and he, too,
harbors a dream of changing his life.

``Seeing some guy just like me, just dropping everything and doing what
he's passionate about, was very encouraging to see,'' he said. ``I'm
like, wow, he's living his dream.''

Image

Twice a week, Mr. Gold posts a short episode about life on the
farm.Credit...Hilary Swift for The New York Times

For others, Mr. Gold's farm has provided a haven in a difficult time.
Charlotte Schmoll, who is 6 and lives in Portland, Ore., spent days at
the beginning of lockdown watching Mr. Gold's videos over and over. She
announced last month that she, too, plans to raise ducks in Vermont.

``One of the questions that comes up when we watch shows is, `Is this
real? Did this happen?''' said her mother, Julie Schmoll. ``That's one
of the things she liked about Mr. Rogers, and maybe she likes about the
duck farmer, that he is also quote-unquote true, or real.''

Mr. Gold does wonder, sometimes, about what it means, in the long term,
to make his life into a story. When the cat was hit by a car, he found
himself reflexively converting the event into a script, and stopped to
ask himself who he was becoming.

``It's like, how much is the experience and how much is the packaging of
the experience, and how do you distinguish between the two,'' he said.
``Because you almost go, `I had a duck die, let me think about the first
act here, and the second act.'''

And still, the show goes on. Late on a recent evening, Mr. Gold was
putting finishing touches on a video about
\href{https://www.youtube.com/watch?v=JAX2ARoZoks}{his dog, Toby}, who
has never quite grown into his intended role as a duck herder.

Early drafts of the video had focused on how much the dog had improved.

Image

The Golds with their dog, Toby, a reluctant duck herder.Credit...Hilary
Swift for The New York Times

But there was something dishonest about that, Mr. Gold realized that
evening, as he and Ms. Gold flung themselves around the paddock, trying
to catch birds with string nets, while the dog looked on placidly,
thumping his tail.

Now, in the gathering dark, Mr. Gold was rewriting the ending to one
that emphasized \href{https://www.youtube.com/watch?v=EHMZ6YmiJ8U}{his
acceptance of the dog's true nature.}

It's always difficult to bring closure to a video, Ms. Gold said. It was
almost 9 o'clock, and she was hoping to go inside.

``You have to create an end,'' she said. ``Because the truth is, we do
this every day, so there's not really an end.''

But Mr. Gold, for his part, was pleased.

``I love it when a story has a good moral,'' he said.

Advertisement

\protect\hyperlink{after-bottom}{Continue reading the main story}

\hypertarget{site-index}{%
\subsection{Site Index}\label{site-index}}

\hypertarget{site-information-navigation}{%
\subsection{Site Information
Navigation}\label{site-information-navigation}}

\begin{itemize}
\tightlist
\item
  \href{https://help.nytimes3xbfgragh.onion/hc/en-us/articles/115014792127-Copyright-notice}{©~2020~The
  New York Times Company}
\end{itemize}

\begin{itemize}
\tightlist
\item
  \href{https://www.nytco.com/}{NYTCo}
\item
  \href{https://help.nytimes3xbfgragh.onion/hc/en-us/articles/115015385887-Contact-Us}{Contact
  Us}
\item
  \href{https://www.nytco.com/careers/}{Work with us}
\item
  \href{https://nytmediakit.com/}{Advertise}
\item
  \href{http://www.tbrandstudio.com/}{T Brand Studio}
\item
  \href{https://www.nytimes3xbfgragh.onion/privacy/cookie-policy\#how-do-i-manage-trackers}{Your
  Ad Choices}
\item
  \href{https://www.nytimes3xbfgragh.onion/privacy}{Privacy}
\item
  \href{https://help.nytimes3xbfgragh.onion/hc/en-us/articles/115014893428-Terms-of-service}{Terms
  of Service}
\item
  \href{https://help.nytimes3xbfgragh.onion/hc/en-us/articles/115014893968-Terms-of-sale}{Terms
  of Sale}
\item
  \href{https://spiderbites.nytimes3xbfgragh.onion}{Site Map}
\item
  \href{https://help.nytimes3xbfgragh.onion/hc/en-us}{Help}
\item
  \href{https://www.nytimes3xbfgragh.onion/subscription?campaignId=37WXW}{Subscriptions}
\end{itemize}
