Sections

SEARCH

\protect\hyperlink{site-content}{Skip to
content}\protect\hyperlink{site-index}{Skip to site index}

\href{https://www.nytimes3xbfgragh.onion/section/us}{U.S.}

\href{https://myaccount.nytimes3xbfgragh.onion/auth/login?response_type=cookie\&client_id=vi}{}

\href{https://www.nytimes3xbfgragh.onion/section/todayspaper}{Today's
Paper}

\href{/section/us}{U.S.}\textbar{}Lost Summer: How Schools Missed a
Chance to Fix Remote Learning

\url{https://nyti.ms/2XEn2kS}

\begin{itemize}
\item
\item
\item
\item
\item
\item
\end{itemize}

\href{https://www.nytimes3xbfgragh.onion/news-event/coronavirus?action=click\&pgtype=Article\&state=default\&region=TOP_BANNER\&context=storylines_menu}{The
Coronavirus Outbreak}

\begin{itemize}
\tightlist
\item
  live\href{https://www.nytimes3xbfgragh.onion/2020/08/07/world/covid-19-news.html?action=click\&pgtype=Article\&state=default\&region=TOP_BANNER\&context=storylines_menu}{Latest
  Updates}
\item
  \href{https://www.nytimes3xbfgragh.onion/interactive/2020/us/coronavirus-us-cases.html?action=click\&pgtype=Article\&state=default\&region=TOP_BANNER\&context=storylines_menu}{Maps
  and Cases}
\item
  \href{https://www.nytimes3xbfgragh.onion/interactive/2020/science/coronavirus-vaccine-tracker.html?action=click\&pgtype=Article\&state=default\&region=TOP_BANNER\&context=storylines_menu}{Vaccine
  Tracker}
\item
  \href{https://www.nytimes3xbfgragh.onion/interactive/2020/world/coronavirus-tips-advice.html?action=click\&pgtype=Article\&state=default\&region=TOP_BANNER\&context=storylines_menu}{F.A.Q.}
\item
  \href{https://www.nytimes3xbfgragh.onion/live/2020/08/07/business/stock-market-today-coronavirus?action=click\&pgtype=Article\&state=default\&region=TOP_BANNER\&context=storylines_menu}{Markets
  \& Economy}
\end{itemize}

Advertisement

\protect\hyperlink{after-top}{Continue reading the main story}

Supported by

\protect\hyperlink{after-sponsor}{Continue reading the main story}

\hypertarget{lost-summer-how-schools-missed-a-chance-to-fix-remote-learning}{%
\section{Lost Summer: How Schools Missed a Chance to Fix Remote
Learning}\label{lost-summer-how-schools-missed-a-chance-to-fix-remote-learning}}

Education leaders spent months preparing to reopen classrooms. But with
online learning set to continue for millions of students this fall,
schools must catch up with reality.

\includegraphics{https://static01.graylady3jvrrxbe.onion/images/2020/08/07/us/07virus-lostsummer1/merlin_174847689_c980ca80-f75b-47c2-9851-5c954ced82a6-articleLarge.jpg?quality=75\&auto=webp\&disable=upscale}

\href{https://www.nytimes3xbfgragh.onion/by/dana-goldstein}{\includegraphics{https://static01.graylady3jvrrxbe.onion/images/2018/06/12/multimedia/author-dana-goldstein/author-dana-goldstein-thumbLarge.png}}

By \href{https://www.nytimes3xbfgragh.onion/by/dana-goldstein}{Dana
Goldstein}

\begin{itemize}
\item
  Aug. 7, 2020Updated 3:17 p.m. ET
\item
  \begin{itemize}
  \item
  \item
  \item
  \item
  \item
  \item
  \end{itemize}
\end{itemize}

With some combination of optimism, anxiety and wishful thinking, many
educators spent their summers planning, in minute detail, how to safely
reopen classrooms. Teachers stocked up on sanitation supplies as
superintendents took a crash course in epidemiology and studied supply
chain logistics for portable air filters.

But with the pandemic now surging across a wide swath of the country,
many of those plans have been shelved, and a different reality has
emerged for the nation's exhausted and stir-crazy families: Millions of
American children will spend their fall once again learning in front of
laptop screens.

In places where schools haven't
\href{https://www.nytimes3xbfgragh.onion/2020/08/06/us/coronavirus-students.html}{already
reopened} --- in some,
\href{https://www.nytimes3xbfgragh.onion/2020/08/03/us/school-closing-coronavirus.html}{to
just as quickly close again} --- educators are spending the little time
they have left before the new academic year moving to focus more fully
on improving online instruction, which
\href{https://www.nytimes3xbfgragh.onion/2020/06/05/us/coronavirus-education-lost-learning.html}{failed
to reach and engage} many children in the spring, leading to growing
achievement gaps by income and race.

But what if they had started much sooner?

``Clearly, lots of time was lost, and it's unfortunate,'' said Morgan
Polikoff, a professor of education at the University of Southern
California. ``We're not going to have a ton of in-person instruction in
this entire school year,'' he predicted,
\href{https://www.nytimes3xbfgragh.onion/2020/08/03/us/school-closing-coronavirus.html}{echoing
some public health experts}.

When schools abruptly moved to remote learning in March, the majority of
districts did not require teachers to provide any live instruction, and
many interactions between educators and students took place over email
and text message.

Lessons were often passive, revolving around online videos and
worksheets. Students without access to computers and wifi, or without a
tech-savvy adult to help them, fell behind, in some cases with
consequences that are
\href{https://www.nytimes3xbfgragh.onion/2020/06/05/us/coronavirus-education-lost-learning.html}{expected
to be long-lasting}. An alarming number of students
\href{https://www.nytimes3xbfgragh.onion/2020/04/06/us/coronavirus-schools-attendance-absent.html}{didn't
show up for remote learning} at all.

To avoid a repeat, Dr. Polikoff said, ``We should be pouring absolutely
everything we can into making the online experience as good as
possible.''

Instead, American educators spent the late spring and summer planning to
welcome students back to classrooms. They did so at a time when the
outbreak appeared to be under control, and the message from politicians,
pediatricians and many parents was clear: Reopening schools, in person,
was a priority to restart the economy and educate children.

But as districts stocked up on face shields and disinfectant sprays,
many governors were allowing bars and restaurants to reopen. Delays in
providing virus test results grew. And local leaders were left to figure
out the details of running schools during a public health crisis largely
on their own.

Most of the country's largest districts settled on some mix of in-person
and remote learning, but they have been forced to shelve those plans in
recent weeks as virus cases soar in their communities. Of the nation's
15 largest school systems, only two, New York City and Hawaii, now plan
to start the year with open classrooms, even part-time. On Friday, Gov.
Andrew M. Cuomo of New York
\href{https://www.nytimes3xbfgragh.onion/2020/08/07/nyregion/cuomo-schools-reopening.html}{cleared
the way for schools} in the state to choose to reopen, but the decision
remains in the hands of local officials.

\includegraphics{https://static01.graylady3jvrrxbe.onion/images/2020/08/07/us/07virus-lostsummer2/merlin_174557727_e04b99cc-cb5d-4bab-8222-c10b5a51de65-articleLarge.jpg?quality=75\&auto=webp\&disable=upscale}

President Trump and his education secretary, Betsy DeVos, have
threatened to withhold federal funding from schools that do not teach in
person but have provided no detailed recommendations on how to do so
safely. Nor has Congress provided the billions that
\href{https://www.nytimes3xbfgragh.onion/2020/07/09/us/schools-reopening-trump.html}{local
officials say they need} to cover the costs of retrofitting buildings
for pandemic conditions.

Governors and state education departments, too, focused more on the hope
for an in-person reopening than the chance that online eduction could
continue.

Only
\href{https://www.crpe.org/thelens/states-must-take-decisive-action-avert-coming-education-crisis}{11
state reopening plans} require districts to put in place specific
strategies for delivering a quality education remotely, according to
research from the Center on Reinventing Public Education, a think tank
and advocacy group. Just a few states --- such as Maine, Connecticut and
South Carolina --- have taken a centralized approach to making sure
students have access to computers and internet connections in their
homes.

Local control of education is a centuries-old American tradition, but
one that may not be well suited to emergency planning during a pandemic.

``Every system, every school had to think through this, often in
isolation,'' said Jamila Newman of TNTP, formerly the New Teacher
Project, a nonprofit that provides consulting services for districts
across the country on staffing and student achievement. ``Districts were
spending the lion's share of their time focused on an in-person plan or
a hybrid plan that feels incredibly untenable in most places.''

\href{https://www.crpe.org/current-research/covid-19-school-closures}{Fewer
than half of districts} offered summer professional development programs
for educators, or have a plan in place to continue coaching them on
remote teaching practices during the school year, according to a
database of more than 100 local plans, including those for the nation's
30 largest school systems, compiled by the Center on Reinventing Public
Education.

\hypertarget{the-coronavirus-outbreak}{%
\subsubsection{The Coronavirus
Outbreak}\label{the-coronavirus-outbreak}}

\hypertarget{back-to-school}{%
\paragraph{Back to School}\label{back-to-school}}

Updated Aug. 7, 2020

The latest highlights as the first students return to U.S. schools.

\begin{itemize}
\item
  \begin{itemize}
  \tightlist
  \item
    Schools are open in parts of the country --- and some are
    \href{https://www.nytimes3xbfgragh.onion/2020/08/03/us/school-closing-coronavirus.html?action=click\&pgtype=Article\&state=default\&region=MAIN_CONTENT_2\&context=storylines_keepup}{already
    closing again}. Students have already faced altered classrooms and
    emergency quarantines. We spoke to some to see what
    \href{https://www.nytimes3xbfgragh.onion/2020/08/06/us/coronavirus-students.html?action=click\&pgtype=Article\&state=default\&region=MAIN_CONTENT_2\&context=storylines_keepup}{school
    is like in the age of Covid-19.}
  \item
    Photos of a crowded high school hallway evoked outrage on social
    media. The student who took them
    \href{https://www.nytimes3xbfgragh.onion/2020/08/06/us/north-paulding-high-school-coronavirus-georgia.html?action=click\&pgtype=Article\&state=default\&region=MAIN_CONTENT_2\&context=storylines_keepup}{says
    she was suspended}.
  \item
    Faced with remote learning or socially distanced classroom options,
    some parents of rising kindergartners
    \href{https://www.nytimes3xbfgragh.onion/2020/07/23/parenting/school-opening-kindergarten-coronavirus.html?action=click\&pgtype=Article\&state=default\&region=MAIN_CONTENT_2\&context=storylines_keepup}{are
    considering holding their kids back.}
  \item
    Teachers,
    \href{https://www.nytimes3xbfgragh.onion/2020/08/05/reader-center/teachers-show-us-how-the-coronavirus-is-changing-your-classroom.html?action=click\&pgtype=Article\&state=default\&region=MAIN_CONTENT_2\&context=storylines_keepup}{show
    us your classrooms}! We want to see how educators are preparing to
    keep themselves and students safe if their schools open this fall.
  \end{itemize}
\end{itemize}

In many parts of the country, collective bargaining agreements between
districts and teachers play a large role in determining how services are
administered. Part of the summer was spent
\href{https://www.nytimes3xbfgragh.onion/2020/07/29/us/teacher-union-school-reopening-coronavirus.html}{renegotiating
emergency work agreements} created for the pandemic, with the assumption
that teachers would be in classrooms at least part of the time.

But in recent weeks, those discussions have shifted quickly from
focusing on social distancing and ventilation to
\href{https://www.nytimes3xbfgragh.onion/2020/07/29/us/teacher-union-school-reopening-coronavirus.html}{tense
debates} over how many hours of live interaction teachers will be
expected to provide to students, and whether they should use their
vacant classrooms as television sets, of sorts, for video instruction.

Ms. Newman said the continued uncertainty over when students might
return to schools could hamper the ability of educators to provide
quality remote learning. In some regions, she suggested, it might make
sense to consider the possibility of spending the entire school year
online.

``People won't necessarily be able to get back into the comfortable
cadence of instruction,'' she said, ``if there's always this hope that
right around the corner --- next week, next month --- we can transition
back.''

There are places where educators have already forged ahead in the past
months with creative plans for remote learning.

Image

Back to school supplies at a Walmart store in Encinitas,
Calif.Credit...Mike Blake/Reuters

During the spring, KIPP New Jersey, a charter school network of 15
schools that overwhelmingly serves Black and Latino children from
low-income families, did not require its teachers to provide live
instruction. But it is planning big changes for the fall semester,
which, pending state approval, will be online until October at earliest.

Students in kindergarten, first grade and second grade will begin the
day with a 30-minute social-emotional check-in with teachers and peers,
followed by 90 minutes of live instruction in reading, writing and math.
For academics, they will be placed into small groups of 10 students.

The network has learned that limiting class sizes may be even more
important online than in the physical classroom, said its chief
executive, Ryan Hill. On Zoom, for example, it is helpful for a teacher
to be able to see all of their students' faces at once, instead of
having to scroll through multiple screens.

``I do think one benefit of us staying remote is we can get really good
at remote instruction,'' Mr. Hill said. ``And I think there is a lot of
promise in it.''

Los Angeles Unified, the nation's second-largest school system, decided
last month that it would start the school year online. It has
\href{http://laschoolboard.org/sites/default/files/IAU\%20Report\%202020\%200707\%20-\%20Student\%20Engagement\%20Online\%20During\%20Closures.pdf}{published
data} showing that when schools closed in March, about a third of middle
and high school students did not actively participate each week in
online learning ---
\href{https://www.nytimes3xbfgragh.onion/2020/04/06/us/coronavirus-schools-attendance-absent.html}{a
national problem}.

The district plans to use staff members such as bus drivers to help make
phone calls to track down students who do not turn in assignments or
show up for online classes.

Another major concern is how schools will facilitate socialization and
warmth between students and teachers. When schools suddenly shut down
this spring, children continued to interact online with the classmates
and teachers they knew and loved; that won't be the case for many in a
new academic year.

To address that, the Esperanza Community School in Tempe, Ariz., plans
to keep most students with their teachers from last year, according to
Hannah Wysong, a teacher leader at the school.

Esperanza is a public school led by a collective of teachers, so it is
able to operate outside of the typical bureaucracy that can make it hard
to embrace nontraditional practices, Ms. Wysong said.

In Odessa, Texas, the Ector County school district, which plans to open
this month under a hybrid model that combines remote and in-person
schooling, has come up with a plan to provide internet access to its
rural population, some of whom live in areas not reached by broadband
providers.

During days when students are slated for remote learning, they will be
able to come to school buildings to access devices and internet in a
supervised, socially distanced space.

As difficult as remote learning was for many teachers and students in
the spring, it could be just as tough for parents and caregivers who
juggled their jobs and basic child care while trying to help students
navigate a seemingly endless array of apps and logins.

Miami-Dade County, the nation's fourth largest district, is trying to
address that problem through
\href{https://parentacademymiami.com/}{online training} for parents.

Shelley Gonzalez, the mother of a rising second grader, participated in
several sessions this summer. During one, she learned about the
importance of giving her son, Luis, breaks for ``sunlight activities,''
like basketball or playing with LEGOs, that allow him to recharge.

She was also introduced to a tool called ``Immersive Reader'' that helps
Luis, who has been diagnosed with attention deficit-hyperactivity
disorder and vision problems, focus on specific passages he is supposed
to read, by reducing distractions on the screen.

``I am hoping this teaches parents and kids you can learn anywhere,''
Ms. Gonzalez said. ``You don't have to be sitting in a classroom.''

Advertisement

\protect\hyperlink{after-bottom}{Continue reading the main story}

\hypertarget{site-index}{%
\subsection{Site Index}\label{site-index}}

\hypertarget{site-information-navigation}{%
\subsection{Site Information
Navigation}\label{site-information-navigation}}

\begin{itemize}
\tightlist
\item
  \href{https://help.nytimes3xbfgragh.onion/hc/en-us/articles/115014792127-Copyright-notice}{©~2020~The
  New York Times Company}
\end{itemize}

\begin{itemize}
\tightlist
\item
  \href{https://www.nytco.com/}{NYTCo}
\item
  \href{https://help.nytimes3xbfgragh.onion/hc/en-us/articles/115015385887-Contact-Us}{Contact
  Us}
\item
  \href{https://www.nytco.com/careers/}{Work with us}
\item
  \href{https://nytmediakit.com/}{Advertise}
\item
  \href{http://www.tbrandstudio.com/}{T Brand Studio}
\item
  \href{https://www.nytimes3xbfgragh.onion/privacy/cookie-policy\#how-do-i-manage-trackers}{Your
  Ad Choices}
\item
  \href{https://www.nytimes3xbfgragh.onion/privacy}{Privacy}
\item
  \href{https://help.nytimes3xbfgragh.onion/hc/en-us/articles/115014893428-Terms-of-service}{Terms
  of Service}
\item
  \href{https://help.nytimes3xbfgragh.onion/hc/en-us/articles/115014893968-Terms-of-sale}{Terms
  of Sale}
\item
  \href{https://spiderbites.nytimes3xbfgragh.onion}{Site Map}
\item
  \href{https://help.nytimes3xbfgragh.onion/hc/en-us}{Help}
\item
  \href{https://www.nytimes3xbfgragh.onion/subscription?campaignId=37WXW}{Subscriptions}
\end{itemize}
