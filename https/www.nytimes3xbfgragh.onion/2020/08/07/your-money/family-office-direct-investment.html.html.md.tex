Sections

SEARCH

\protect\hyperlink{site-content}{Skip to
content}\protect\hyperlink{site-index}{Skip to site index}

\href{https://www.nytimes3xbfgragh.onion/section/your-money}{Your Money}

\href{https://myaccount.nytimes3xbfgragh.onion/auth/login?response_type=cookie\&client_id=vi}{}

\href{https://www.nytimes3xbfgragh.onion/section/todayspaper}{Today's
Paper}

\href{/section/your-money}{Your Money}\textbar{}More Wealthy Families
Are Throwing a Lifeline to Distressed Businesses

\url{https://nyti.ms/3fJ4OF2}

\begin{itemize}
\item
\item
\item
\item
\item
\end{itemize}

\hypertarget{the-coronavirus-outbreak}{%
\subsubsection{\texorpdfstring{\href{https://www.nytimes3xbfgragh.onion/news-event/coronavirus?name=styln-coronavirus-national\&region=TOP_BANNER\&variant=undefined\&block=storyline_menu_recirc\&action=click\&pgtype=Article\&impression_id=1470dab0-e384-11ea-87e8-83b655896aaa}{The
Coronavirus
Outbreak}}{The Coronavirus Outbreak}}\label{the-coronavirus-outbreak}}

\begin{itemize}
\tightlist
\item
  live\href{https://www.nytimes3xbfgragh.onion/2020/08/20/world/coronavirus-covid.html?name=styln-coronavirus-national\&region=TOP_BANNER\&variant=undefined\&block=storyline_menu_recirc\&action=click\&pgtype=Article\&impression_id=1470dab1-e384-11ea-87e8-83b655896aaa}{Latest
  Updates}
\item
  \href{https://www.nytimes3xbfgragh.onion/interactive/2020/us/coronavirus-us-cases.html?name=styln-coronavirus-national\&region=TOP_BANNER\&variant=undefined\&block=storyline_menu_recirc\&action=click\&pgtype=Article\&impression_id=1470dab2-e384-11ea-87e8-83b655896aaa}{Maps
  and Cases}
\item
  \href{https://www.nytimes3xbfgragh.onion/interactive/2020/science/coronavirus-vaccine-tracker.html?name=styln-coronavirus-national\&region=TOP_BANNER\&variant=undefined\&block=storyline_menu_recirc\&action=click\&pgtype=Article\&impression_id=1470dab3-e384-11ea-87e8-83b655896aaa}{Vaccine
  Tracker}
\item
  \href{https://www.nytimes3xbfgragh.onion/2020/08/19/us/colleges-closing-covid.html?name=styln-coronavirus-national\&region=TOP_BANNER\&variant=undefined\&block=storyline_menu_recirc\&action=click\&pgtype=Article\&impression_id=1470dab4-e384-11ea-87e8-83b655896aaa}{Colleges
  Closing}
\item
  \href{https://www.nytimes3xbfgragh.onion/live/2020/08/20/business/stock-market-today-coronavirus?name=styln-coronavirus-national\&region=TOP_BANNER\&variant=undefined\&block=storyline_menu_recirc\&action=click\&pgtype=Article\&impression_id=1470dab5-e384-11ea-87e8-83b655896aaa}{Economy}
\end{itemize}

Advertisement

\protect\hyperlink{after-top}{Continue reading the main story}

Supported by

\protect\hyperlink{after-sponsor}{Continue reading the main story}

Wealth Matters

\hypertarget{more-wealthy-families-are-throwing-a-lifeline-to-distressed-businesses}{%
\section{More Wealthy Families Are Throwing a Lifeline to Distressed
Businesses}\label{more-wealthy-families-are-throwing-a-lifeline-to-distressed-businesses}}

The pandemic has created an opportunity for family offices to take a
more hands-on approach by investing directly in companies that need
capital.

\includegraphics{https://static01.graylady3jvrrxbe.onion/images/2020/08/08/business/07Wealth-01/07Wealth-01-articleLarge.jpg?quality=75\&auto=webp\&disable=upscale}

By \href{https://www.nytimes3xbfgragh.onion/by/paul-sullivan}{Paul
Sullivan}

\begin{itemize}
\item
  Aug. 7, 2020
\item
  \begin{itemize}
  \item
  \item
  \item
  \item
  \item
  \end{itemize}
\end{itemize}

During the pandemic, wealthy families have continued to use their
investment pools, known as family offices, to gain access to the type of
high-return opportunities once reserved for institutional investors. But
they are taking a more hands-on role in those financial decisions.

These family offices have chosen to bypass private equity and venture
capital funds --- which have high minimum investments and sizable fees
--- to invest directly in companies, either by themselves or with other
significantly wealthy families,
\href{https://www.fintrx.com/fintrx-charles-schwab-2020-family-office-report}{a
report} released on Friday found. And they are taking a greater
entrepreneurial role in their investments, which would not have been
possible if they had put their money into large funds for other people
to manage.

Direct investment in businesses began to rise after the last recession,
climbing 206 percent from 2010 to 2015. Last year, it grew 11 percent.

Now, half of all family offices in the world make direct investments in
companies, according to the report, which was released by Fintrx, a data
and research company, and sponsored by Charles Schwab's family office
arm. That number jumped to 83 percent for single-family offices,
compared with offices that serve multiple families, with many of those
investments focused on areas where the family had originally made its
money.

``Family offices add value in times of crisis,'' said Russ D'Argento,
founder and chief executive of Fintrx. ``That's a big component of how
they stand out and can be different from other fund structures.''

To be sure, the wealthy are able to invest differently in the pandemic
from everyone else. Nearly
\href{https://www.nytimes3xbfgragh.onion/live/2020/08/06/business/stock-market-today-coronavirus/new-state-jobless-claims-decline-but-exceed-one-million-for-the-20th-week}{1.2
million workers filed new claims} for state unemployment benefits last
week, the Labor Department reported on Thursday. The tally has exceeded
one million claims for 20 consecutive weeks, extraordinarily high by
historical standards. And millions of Americans are worried about paying
their bills and not being evicted from their homes.

For the wealthiest families, however, what has been an economic and
health crisis for others is an opportunity to make money by throwing a
financial lifeline to distressed businesses. The stock market may have
rebounded quickly as investors looked past growing hot spots around the
country, but these family offices are betting that the public markets
are overvalued and that more predictable and steadier returns are to be
had through private investments.

\hypertarget{latest-updates-the-coronavirus-outbreak}{%
\section{\texorpdfstring{\href{https://www.nytimes3xbfgragh.onion/2020/08/20/world/coronavirus-covid.html?action=click\&pgtype=Article\&state=default\&region=MAIN_CONTENT_1\&context=storylines_live_updates}{Latest
Updates: The Coronavirus
Outbreak}}{Latest Updates: The Coronavirus Outbreak}}\label{latest-updates-the-coronavirus-outbreak}}

Updated 2020-08-21T07:46:15.883Z

\begin{itemize}
\tightlist
\item
  \href{https://www.nytimes3xbfgragh.onion/2020/08/20/world/coronavirus-covid.html?action=click\&pgtype=Article\&state=default\&region=MAIN_CONTENT_1\&context=storylines_live_updates\#link-68774d88}{Shutdowns,
  warnings and scoldings follow alarming incidents on college campuses.}
\item
  \href{https://www.nytimes3xbfgragh.onion/2020/08/20/world/coronavirus-covid.html?action=click\&pgtype=Article\&state=default\&region=MAIN_CONTENT_1\&context=storylines_live_updates\#link-26b58724}{Biden
  knocks Trump's pandemic response, and outlines a national strategy.}
\item
  \href{https://www.nytimes3xbfgragh.onion/2020/08/20/world/coronavirus-covid.html?action=click\&pgtype=Article\&state=default\&region=MAIN_CONTENT_1\&context=storylines_live_updates\#link-4e542da3}{U.S.
  health agencies announce moves to confront the flu season and
  plummeting child vaccination rates.}
\end{itemize}

\href{https://www.nytimes3xbfgragh.onion/2020/08/20/world/coronavirus-covid.html?action=click\&pgtype=Article\&state=default\&region=MAIN_CONTENT_1\&context=storylines_live_updates}{See
more updates}

More live coverage:
\href{https://www.nytimes3xbfgragh.onion/live/2020/08/20/business/stock-market-today-coronavirus?action=click\&pgtype=Article\&state=default\&region=MAIN_CONTENT_1\&context=storylines_live_updates}{Markets}

``There's been an incredible recovery in the stock market, but how do I
commit more to the public markets when I'm looking at these valuations
and it's still a rocky road ahead?'' said Eric Becker, who made his
wealth by investing in health care companies and more recently founded
Cresset Capital, a multifamily office.

The interest in direct investments has grown in the pandemic. There was
a short pause when the initial stay-at-home orders were issued, but
interest has begun to rise again, especially among newer family offices,
or those formed in the last five years that still retain their
entrepreneurial natures.

``Once the data started coming in and talk about therapeutics began,
people started seeing a path toward normalcy,'' Mr. Becker said.
``Whether it's two years, a year, six months out, it didn't matter. They
could see that path, whether it was distressed companies or companies
that just needed some capital.''

Image

``Family offices add value in times of crisis,'' said Russ D'Argento,
founder and chief executive of Fintrx.Credit...Fintrx

The Fintrx report found that families generally invested in industries
similar to those in which they had made their initial wealth. Technology
led the way, with tech-funded family offices committing 82 percent of
their direct investments to tech companies. Real estate families were
second with more than two-thirds of their investments in real estate.

``Families initially invest in the same areas where they have
experience,'' said Paul Ferguson, managing director of the Schwab
Advisor Family Office, which sponsored the Fintrx report. He added that
family offices could help preserve private businesses through their
investments.

``They have a lot of capital to invest, and they're in a pretty unique
position because of the long-term nature of their investing,'' he said.
``This is where their patient capital is very important.''

Simply having wealth, of course, does not make someone a good investor.
Direct investing has its critics, who say the strategy is far riskier
than its proponents admit.

For one, private equity firms have trillions of dollars sitting in their
funds and are already looking for deals. So when someone with an
investment opportunity approaches a family worth hundreds of millions of
dollars, alarm bells go off.

``We're always wary when someone pitches us an idea,'' said Paul Karger,
a co-founder and managing partner of TwinFocus, which works with 40
families that collectively have put \$7 billion at the firm. ``The first
question is, how did this Texas oil deal miss everyone in Texas and end
up on our doorstep in Back Bay Boston?''

Having sufficient selection options is something that even large private
equity funds have to consider; many of them examine hundreds of deals
before investing in one. But when it comes to direct investments, Mr.
Karger said, he also is wary if professional investors are not part of
the deal.

Mr. Karger said his firm advised most families to participate in direct
investing through deals in commercial real estate --- like apartments
--- that are easier to value and come loaded with tax advantages. He
also counsels his clients to invest in deals through private equity
funds and focus on the funds' performance, not on the fees they charge,
which are typically a 2 percent management fee and a 20 percent cut on
an investment's return.

\href{https://www.nytimes3xbfgragh.onion/news-event/coronavirus?action=click\&pgtype=Article\&state=default\&region=MAIN_CONTENT_3\&context=storylines_faq}{}

\hypertarget{the-coronavirus-outbreak-}{%
\subsubsection{The Coronavirus Outbreak
›}\label{the-coronavirus-outbreak-}}

\hypertarget{frequently-asked-questions}{%
\paragraph{Frequently Asked
Questions}\label{frequently-asked-questions}}

Updated August 17, 2020

\begin{itemize}
\item ~
  \hypertarget{why-does-standing-six-feet-away-from-others-help}{%
  \paragraph{Why does standing six feet away from others
  help?}\label{why-does-standing-six-feet-away-from-others-help}}

  \begin{itemize}
  \tightlist
  \item
    The coronavirus spreads primarily through droplets from your mouth
    and nose, especially when you cough or sneeze. The C.D.C., one of
    the organizations using that measure,
    \href{https://www.nytimes3xbfgragh.onion/2020/04/14/health/coronavirus-six-feet.html?action=click\&pgtype=Article\&state=default\&region=MAIN_CONTENT_3\&context=storylines_faq}{bases
    its recommendation of six feet} on the idea that most large droplets
    that people expel when they cough or sneeze will fall to the ground
    within six feet. But six feet has never been a magic number that
    guarantees complete protection. Sneezes, for instance, can launch
    droplets a lot farther than six feet,
    \href{https://jamanetwork.com/journals/jama/fullarticle/2763852}{according
    to a recent study}. It's a rule of thumb: You should be safest
    standing six feet apart outside, especially when it's windy. But
    keep a mask on at all times, even when you think you're far enough
    apart.
  \end{itemize}
\item ~
  \hypertarget{i-have-antibodies-am-i-now-immune}{%
  \paragraph{I have antibodies. Am I now
  immune?}\label{i-have-antibodies-am-i-now-immune}}

  \begin{itemize}
  \tightlist
  \item
    As of right
    now,\href{https://www.nytimes3xbfgragh.onion/2020/07/22/health/covid-antibodies-herd-immunity.html?action=click\&pgtype=Article\&state=default\&region=MAIN_CONTENT_3\&context=storylines_faq}{that
    seems likely, for at least several months.} There have been
    frightening accounts of people suffering what seems to be a second
    bout of Covid-19. But experts say these patients may have a
    drawn-out course of infection, with the virus taking a slow toll
    weeks to months after initial exposure. People infected with the
    coronavirus typically
    \href{https://www.nature.com/articles/s41586-020-2456-9}{produce}
    immune molecules called antibodies, which are
    \href{https://www.nytimes3xbfgragh.onion/2020/05/07/health/coronavirus-antibody-prevalence.html?action=click\&pgtype=Article\&state=default\&region=MAIN_CONTENT_3\&context=storylines_faq}{protective
    proteins made in response to an
    infection}\href{https://www.nytimes3xbfgragh.onion/2020/05/07/health/coronavirus-antibody-prevalence.html?action=click\&pgtype=Article\&state=default\&region=MAIN_CONTENT_3\&context=storylines_faq}{.
    These antibodies may} last in the body
    \href{https://www.nature.com/articles/s41591-020-0965-6}{only two to
    three months}, which may seem worrisome, but that's perfectly normal
    after an acute infection subsides, said Dr. Michael Mina, an
    immunologist at Harvard University. It may be possible to get the
    coronavirus again, but it's highly unlikely that it would be
    possible in a short window of time from initial infection or make
    people sicker the second time.
  \end{itemize}
\item ~
  \hypertarget{im-a-small-business-owner-can-i-get-relief}{%
  \paragraph{I'm a small-business owner. Can I get
  relief?}\label{im-a-small-business-owner-can-i-get-relief}}

  \begin{itemize}
  \tightlist
  \item
    The
    \href{https://www.nytimes3xbfgragh.onion/article/small-business-loans-stimulus-grants-freelancers-coronavirus.html?action=click\&pgtype=Article\&state=default\&region=MAIN_CONTENT_3\&context=storylines_faq}{stimulus
    bills enacted in March} offer help for the millions of American
    small businesses. Those eligible for aid are businesses and
    nonprofit organizations with fewer than 500 workers, including sole
    proprietorships, independent contractors and freelancers. Some
    larger companies in some industries are also eligible. The help
    being offered, which is being managed by the Small Business
    Administration, includes the Paycheck Protection Program and the
    Economic Injury Disaster Loan program. But lots of folks have
    \href{https://www.nytimes3xbfgragh.onion/interactive/2020/05/07/business/small-business-loans-coronavirus.html?action=click\&pgtype=Article\&state=default\&region=MAIN_CONTENT_3\&context=storylines_faq}{not
    yet seen payouts.} Even those who have received help are confused:
    The rules are draconian, and some are stuck sitting on
    \href{https://www.nytimes3xbfgragh.onion/2020/05/02/business/economy/loans-coronavirus-small-business.html?action=click\&pgtype=Article\&state=default\&region=MAIN_CONTENT_3\&context=storylines_faq}{money
    they don't know how to use.} Many small-business owners are getting
    less than they expected or
    \href{https://www.nytimes3xbfgragh.onion/2020/06/10/business/Small-business-loans-ppp.html?action=click\&pgtype=Article\&state=default\&region=MAIN_CONTENT_3\&context=storylines_faq}{not
    hearing anything at all.}
  \end{itemize}
\item ~
  \hypertarget{what-are-my-rights-if-i-am-worried-about-going-back-to-work}{%
  \paragraph{What are my rights if I am worried about going back to
  work?}\label{what-are-my-rights-if-i-am-worried-about-going-back-to-work}}

  \begin{itemize}
  \tightlist
  \item
    Employers have to provide
    \href{https://www.osha.gov/SLTC/covid-19/standards.html}{a safe
    workplace} with policies that protect everyone equally.
    \href{https://www.nytimes3xbfgragh.onion/article/coronavirus-money-unemployment.html?action=click\&pgtype=Article\&state=default\&region=MAIN_CONTENT_3\&context=storylines_faq}{And
    if one of your co-workers tests positive for the coronavirus, the
    C.D.C.} has said that
    \href{https://www.cdc.gov/coronavirus/2019-ncov/community/guidance-business-response.html}{employers
    should tell their employees} -\/- without giving you the sick
    employee's name -\/- that they may have been exposed to the virus.
  \end{itemize}
\item ~
  \hypertarget{what-is-school-going-to-look-like-in-september}{%
  \paragraph{What is school going to look like in
  September?}\label{what-is-school-going-to-look-like-in-september}}

  \begin{itemize}
  \tightlist
  \item
    It is unlikely that many schools will return to a normal schedule
    this fall, requiring the grind of
    \href{https://www.nytimes3xbfgragh.onion/2020/06/05/us/coronavirus-education-lost-learning.html?action=click\&pgtype=Article\&state=default\&region=MAIN_CONTENT_3\&context=storylines_faq}{online
    learning},
    \href{https://www.nytimes3xbfgragh.onion/2020/05/29/us/coronavirus-child-care-centers.html?action=click\&pgtype=Article\&state=default\&region=MAIN_CONTENT_3\&context=storylines_faq}{makeshift
    child care} and
    \href{https://www.nytimes3xbfgragh.onion/2020/06/03/business/economy/coronavirus-working-women.html?action=click\&pgtype=Article\&state=default\&region=MAIN_CONTENT_3\&context=storylines_faq}{stunted
    workdays} to continue. California's two largest public school
    districts --- Los Angeles and San Diego --- said on July 13, that
    \href{https://www.nytimes3xbfgragh.onion/2020/07/13/us/lausd-san-diego-school-reopening.html?action=click\&pgtype=Article\&state=default\&region=MAIN_CONTENT_3\&context=storylines_faq}{instruction
    will be remote-only in the fall}, citing concerns that surging
    coronavirus infections in their areas pose too dire a risk for
    students and teachers. Together, the two districts enroll some
    825,000 students. They are the largest in the country so far to
    abandon plans for even a partial physical return to classrooms when
    they reopen in August. For other districts, the solution won't be an
    all-or-nothing approach.
    \href{https://bioethics.jhu.edu/research-and-outreach/projects/eschool-initiative/school-policy-tracker/}{Many
    systems}, including the nation's largest, New York City, are
    devising
    \href{https://www.nytimes3xbfgragh.onion/2020/06/26/us/coronavirus-schools-reopen-fall.html?action=click\&pgtype=Article\&state=default\&region=MAIN_CONTENT_3\&context=storylines_faq}{hybrid
    plans} that involve spending some days in classrooms and other days
    online. There's no national policy on this yet, so check with your
    municipal school system regularly to see what is happening in your
    community.
  \end{itemize}
\end{itemize}

``There's a cost to doing something right,'' he said. ``If you pay
peanuts, you get monkeys.''

But Mr. D'Argento countered that family offices had become more
sophisticated and had hired experienced investors in the last five
years.

``They're starting to look, feel and act like institutions,'' he said.
``With that added talent comes the ability to do more deals. It's less
of a cottage industry.''

Another risk in a moment like this, when there are many distressed
companies looking for investors, is missing an opportunity to leverage
the ``family alpha,'' or the operating knowledge that a family has in
the area from which its wealth came, said Kristi Kuechler, managing
director of client relations at Vernal Point Advisors, a multifamily
office.

\includegraphics{https://static01.graylady3jvrrxbe.onion/images/2020/08/08/business/08Wealth-02/07Wealth-02-articleLarge.jpg?quality=75\&auto=webp\&disable=upscale}

``There are families who have as much knowledge of a sector as a private
equity firm,'' Ms. Kuechler said.

When those families join with other families who made their wealth in
different sectors, they could end up with shared family alpha. Or they
could end up with less of an advantage than they think. That's where the
selection of partner families is crucial, particularly now when there
are many more opportunities.

``What's happened after this incredible increase in direct investment is
people feel they need to diversify their private investment portfolio as
if it were an asset class,'' she said. ``Now, they're co-investing
alongside other families, so they're not exploiting their families'
distinct edge.''

That edge in understanding what businesses may be struggling but are
still viable is important in the pandemic, she said. Many wealthy
families went through their own difficult business cycles at some point.

``It does feel like families want a more tangible investment that they
can see on the ground, and that can be a really attractive
opportunity,'' Ms. Kuechler said. ``Families are both quite cautious in
investing cash, but they're also feeling pretty opportunistic with
companies that look like they can navigate this particular uncertainty
quite well.''

That's good for a family office's returns, but it's just as good for
that small business that can be saved by the investment.

Advertisement

\protect\hyperlink{after-bottom}{Continue reading the main story}

\hypertarget{site-index}{%
\subsection{Site Index}\label{site-index}}

\hypertarget{site-information-navigation}{%
\subsection{Site Information
Navigation}\label{site-information-navigation}}

\begin{itemize}
\tightlist
\item
  \href{https://help.nytimes3xbfgragh.onion/hc/en-us/articles/115014792127-Copyright-notice}{©~2020~The
  New York Times Company}
\end{itemize}

\begin{itemize}
\tightlist
\item
  \href{https://www.nytco.com/}{NYTCo}
\item
  \href{https://help.nytimes3xbfgragh.onion/hc/en-us/articles/115015385887-Contact-Us}{Contact
  Us}
\item
  \href{https://www.nytco.com/careers/}{Work with us}
\item
  \href{https://nytmediakit.com/}{Advertise}
\item
  \href{http://www.tbrandstudio.com/}{T Brand Studio}
\item
  \href{https://www.nytimes3xbfgragh.onion/privacy/cookie-policy\#how-do-i-manage-trackers}{Your
  Ad Choices}
\item
  \href{https://www.nytimes3xbfgragh.onion/privacy}{Privacy}
\item
  \href{https://help.nytimes3xbfgragh.onion/hc/en-us/articles/115014893428-Terms-of-service}{Terms
  of Service}
\item
  \href{https://help.nytimes3xbfgragh.onion/hc/en-us/articles/115014893968-Terms-of-sale}{Terms
  of Sale}
\item
  \href{https://spiderbites.nytimes3xbfgragh.onion}{Site Map}
\item
  \href{https://help.nytimes3xbfgragh.onion/hc/en-us}{Help}
\item
  \href{https://www.nytimes3xbfgragh.onion/subscription?campaignId=37WXW}{Subscriptions}
\end{itemize}
