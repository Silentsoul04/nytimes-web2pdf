Sections

SEARCH

\protect\hyperlink{site-content}{Skip to
content}\protect\hyperlink{site-index}{Skip to site index}

\href{/section/us}{U.S.}\textbar{}`Christianity Will Have Power'

\url{https://nyti.ms/2PBIZfJ}

\begin{itemize}
\item
\item
\item
\item
\item
\end{itemize}

\includegraphics{https://static01.graylady3jvrrxbe.onion/images/2020/08/06/us/00EVANGELICAL-siouxcenter-alt/merlin_175262430_c25a4ece-871f-4861-ac4e-68359c6307b3-mobileMasterAt3x.jpg}\includegraphics{https://static01.graylady3jvrrxbe.onion/images/2020/08/09/us/00EVANGELICAL-dordt/merlin_175262328_952e5c34-b14d-48fc-9318-0b8afb462b8b-mobileMasterAt3x.jpg}\includegraphics{https://static01.graylady3jvrrxbe.onion/images/2020/08/06/us/00TRUMPEVANGELICAL-maurice/merlin_175338117_3531ec2b-826a-417b-b6d1-f36a222ef7b1-mobileMasterAt3x.jpg}

In January 2016, Donald J. Trump gave a campaign speech at a small
Christian college in Sioux Center, Iowa.

Standing in front of a three-story pipe organ, he said, ``I have the
most loyal people.''

``I could stand in the middle of Fifth Avenue and shoot somebody and I
wouldn't lose any voters, OK?''

But he said something else that day. And his intended audience heard
him.

\hypertarget{christianity-will-have-power}{%
\section{`Christianity Will Have
Power'}\label{christianity-will-have-power}}

Donald Trump made a promise to white evangelical Christians, whose
support can seem mystifying to the outside observer.

\href{https://www.nytimes3xbfgragh.onion/by/elizabeth-dias}{\includegraphics{https://static01.graylady3jvrrxbe.onion/images/2018/11/06/multimedia/author-elizabeth-dias/author-elizabeth-dias-thumbLarge.png}}

By \href{https://www.nytimes3xbfgragh.onion/by/elizabeth-dias}{Elizabeth
Dias}

Elizabeth Dias covers religion for The New York Times.

Photographs and Video by Jenn Ackerman and Tim Gruber

\href{https://www.nytimes3xbfgragh.onion/es/2020/08/13/espanol/estados-unidos/trump-cristianos-evangelicos.html}{Leer
en español}

\begin{center}\rule{0.5\linewidth}{\linethickness}\end{center}

SIOUX CENTER, Iowa --- They walked to the sanctuary in the frozen
silence before dawn, footsteps crunching over the snow. Soon, hundreds
joined in line. It was January 2016, and the unlikely Republican
front-runner, Donald J. Trump, had come to town.

He was the boastful, thrice-married, foul-mouthed star of ``The
Apprentice.'' They were one of the most conservative Christian
communities in the nation, with 19 churches in a town of about 7,500
people.

Many were skeptical, and came to witness the spectacle for themselves. A
handful stood in silent protest. But when the doors opened and the pews
filled, Mr. Trump's fans welcomed him by chanting his name. A man waved
a ``Silent Majority Stands With Trump'' sign. A woman pointed a lone
pink fingernail up to the sky.

In his dark suit and red tie, Mr. Trump stood in front of a
three-story-tall pipe organ and waved his arms in time with their
shouts: Trump, Trump, Trump.

The 67-minute speech Mr. Trump gave that day at Dordt University, a
Christian college in Sioux Center, would become infamous, instantly
covered on cable news and to this day still invoked by his critics. But
the line that gained notoriety --- the promise that he ``could stand in
the middle of Fifth Avenue and shoot somebody'' and ``wouldn't lose any
voters'' --- overshadowed another message that morning.

``I will tell you, Christianity is under tremendous siege, whether we
want to talk about it or we don't want to talk about it,'' Mr. Trump
said.

Christians make up the overwhelming majority of the country, he said.
And then he slowed slightly to stress each next word: ``And yet we don't
exert the power that we should have.''

If he were elected president, he promised, that would change. He raised
a finger.

``Christianity will have power,'' he said. ``If I'm there, you're going
to have plenty of power, you don't need anybody else. You're going to
have somebody representing you very, very well. Remember that.''

Nine days later, the Iowa caucuses kicked off the most polarizing road
to the White House in memory. Mr. Trump largely lost the evangelicals of
Sioux County that day: Only 11 percent of Republicans caucused for him.
But when November came, they stood by him en masse: 81 percent of the
county voted for him. And so did 81 percent of white evangelical voters
nationwide.

Now, this group could be Mr. Trump's best chance at re-election. The
president's response to the coronavirus pandemic has battered his
political standing: He has trailed Joseph R. Biden Jr., the presumptive
Democratic nominee, by nearly double digits for a month in national
polls. Even among white evangelicals, his approval rating has dipped
slightly. But
\href{https://www.pewresearch.org/fact-tank/2020/07/01/white-evangelical-approval-of-trump-slips-but-eight-in-ten-say-they-would-vote-for-him/}{82
percent} say they intend to vote for him, according to the Pew Research
Center.

To the outside observer, the relationship between white evangelical
Christians and Donald Trump can seem mystifying.

From the start it appeared an impossible contradiction. Evangelicals for
years have defined themselves as the values voters, people who prized
the Bible and sexual morality --- and loving your neighbor as yourself
--- above all.

Donald Trump was the opposite. He bragged about assaulting women. He got
divorced, twice. He built a career off gambling. He cozied up to bigots.
He rarely went to church. He refused to ask for forgiveness.

It is a contradiction that has held for four years. They stood by him
when
\href{https://www.nytimes3xbfgragh.onion/2017/01/27/us/politics/trump-syrian-refugees.html}{he
shut out Muslim refugees}. When he separated children from their parents
at the border. When he issued brash insults over social media. When he
uttered falsehoods as if they were true. When he was impeached.

Theories, and rationalizations, abound:

That evangelical support was purely transactional.

That they saw him as their best chance in decades to end legalized
abortion.

That the opportunity to nominate conservative justices to the Supreme
Court was paramount.

That they hated Hillary Clinton, or felt torn to pick the lesser of two
evils.

That they held their noses and voted, hoping he would advance their
policy priorities and accomplish their goals.

But beneath all this, there is another explanation. One that is more raw
and fundamental.

Evangelicals did not support Mr. Trump in spite of who he is. They
supported him because of who he is, and because of who they are. He is
their protector, the bully who is on their side, the one who offered
safety amid their fears that their country as they know it, and their
place in it, is changing, and changing quickly. White straight married
couples with children who go to church regularly are no longer the
American mainstream. An entire way of life, one in which their values
were dominant, could be headed for extinction. And Mr. Trump offered to
restore them to power, as though they have not been in power all along.

``You are always only one generation away from losing Christianity,''
said Micah Schouten, who was born and raised in Sioux Center, recalling
something a former pastor used to say. ``If you don't teach it to your
children it ends. It stops right there.''

Ultimately Mr. Trump recognized something, said Lisa Burg, a longtime
resident of nearby Orange City. It is a reason she thinks people will
still support him in November.

``The one group of people that people felt like they could dis and mock
and put down had become the Christian. Just the middle-class,
middle-American Christians,'' Ms. Burg said. ``That was the one group
left that you could just totally put down and call deplorable. And he
recognized that, You know what? Yeah, it's OK that we have our set of
values, too. I think people finally said, `Yes, we finally have somebody
that's willing to say we're not bad, we need to have a voice too.'''

Explained Jason Mulder, who runs a small design company in Sioux Center:
``I feel like on the coasts, in some of the cities and stuff, they look
down on us in rural America. You know, we are a bunch of hicks, and
don't know anything. They don't understand us the same way we don't
understand them. So we don't want them telling us how to live our
lives.''

He added: ``You joke that we don't get it, well, you don't get it
either. We are not speaking the same language.''

The speech in Sioux Center symbolized why there has been so much
confusion about evangelical support for Mr. Trump. From the beginning,
the outside world focused on the comment about shooting someone on Fifth
Avenue. Those in the town, though, ultimately heard something else
entirely. What mattered was not just what Mr. Trump said. It was where
he said it. And to whom.

And so to understand the relationship, one has to go back to Jan. 23,
2016. One has to hear the speech at Dordt the way the evangelical
community heard it.

\includegraphics{https://static01.graylady3jvrrxbe.onion/images/2020/08/09/us/00EVANGELICAL-netherlands-reformed/00EVANGELICAL-netherlands-reformed-articleLarge.jpg?quality=75\&auto=webp\&disable=upscale}

\hypertarget{we-started-out-as-a-christian-nation}{%
\subsection{`We started out as a Christian
nation'}\label{we-started-out-as-a-christian-nation}}

The day Mr. Trump spoke at Dordt, Rob Driesen sat in the very front. He
supported Ted Cruz at the time. But now, four years later, his eyes
light up when he talks about Mr. Trump.

\hypertarget{latest-updates-2020-election}{%
\section{\texorpdfstring{\href{https://www.nytimes3xbfgragh.onion/live/2020/08/19/us/dnc-convention-election?action=click\&pgtype=Article\&state=default\&region=MAIN_CONTENT_1\&context=storylines_live_updates}{Latest
Updates: 2020
Election}}{Latest Updates: 2020 Election}}\label{latest-updates-2020-election}}

\href{https://www.nytimes3xbfgragh.onion/live/2020/08/19/us/dnc-convention-election?action=click\&pgtype=Article\&state=default\&region=MAIN_CONTENT_1\&context=storylines_live_updates\#night-3-featured-more-policy-a-focus-on-women-and-a-full-throated-rejection-of-trump-by-his-predecessor}{7h
ago}

\href{https://www.nytimes3xbfgragh.onion/live/2020/08/19/us/dnc-convention-election?action=click\&pgtype=Article\&state=default\&region=MAIN_CONTENT_1\&context=storylines_live_updates\#night-3-featured-more-policy-a-focus-on-women-and-a-full-throated-rejection-of-trump-by-his-predecessor}{Night
3 featured more policy, a focus on women and a full-throated rejection
of Trump by his predecessor.}

\href{https://www.nytimes3xbfgragh.onion/live/2020/08/19/us/dnc-convention-election?action=click\&pgtype=Article\&state=default\&region=MAIN_CONTENT_1\&context=storylines_live_updates\#trump-live-tweeted-obamas-speech-tonight-hell-appear-on-fox-news-right-before-bidens-tomorrow}{9h
ago}

\href{https://www.nytimes3xbfgragh.onion/live/2020/08/19/us/dnc-convention-election?action=click\&pgtype=Article\&state=default\&region=MAIN_CONTENT_1\&context=storylines_live_updates\#trump-live-tweeted-obamas-speech-tonight-hell-appear-on-fox-news-right-before-bidens-tomorrow}{Trump
live-tweeted Obama's speech tonight. He'll appear on Fox News right
before Biden's tomorrow.}

\href{https://www.nytimes3xbfgragh.onion/live/2020/08/19/us/dnc-convention-election?action=click\&pgtype=Article\&state=default\&region=MAIN_CONTENT_1\&context=storylines_live_updates\#advocates-for-domestic-violence-survivors-praised-biden-in-a-video}{9h
ago}

\href{https://www.nytimes3xbfgragh.onion/live/2020/08/19/us/dnc-convention-election?action=click\&pgtype=Article\&state=default\&region=MAIN_CONTENT_1\&context=storylines_live_updates\#advocates-for-domestic-violence-survivors-praised-biden-in-a-video}{Advocates
for domestic violence survivors praised Biden in a video.}

\href{https://www.nytimes3xbfgragh.onion/live/2020/08/19/us/dnc-convention-election?action=click\&pgtype=Article\&state=default\&region=MAIN_CONTENT_1\&context=storylines_live_updates}{See
more updates}

He brought out two photographs, framed, one of him and Mr. Trump, and
one of him with Mike Pence before he became vice president.

``I guess the biggest concern for me is trying to keep our country the
way it was. Conservative. The values. For us, I mean, this is as good as
it gets. We can do whatever we want,'' said Mr. Driesen, 56, sitting at
his kitchen table this spring with his wife, Cheryl, 52. Next to them, a
family motto was painted on the wall in gold and black lettering:
``Home, Where Your Story Begins.''

He gestured to his front door. ``You don't lock the doors,'' he said.
``I never take the keys out of the car.''

He thought back to Mr. Trump's speech. ``There was one gaffe he kind of
got in trouble for. What was it? Because there were a bunch of things he
said.'' He paused a while. ``I can't distinctly remember, but I just
remember there was one thing, and that was the news for 10 days after
that. Something about --- I wish I could remember. I can't.''

``You know how things can sound bad,'' he said. ``He can get away with
it. People seemed to like it.''

Mr. Driesen works for the utility company, and his wife is a nurse. They
have raised their five children in the area, where they grew up. Mr.
Driesen's grandmother's grandparents were among the first Protestant
immigrants to come to Iowa from the Netherlands in the late 1800s. They
were among hundreds of families looking for economic opportunity, and a
place to worship without interference from the Dutch government. The
immigrants called their first colony Pella, after the place where
first-century Christians fled to avoid persecution. Their second colony,
which would include Sioux Center, settled on land that had been home to
the Yankton Sioux, before the U.S. government had forced them west.

Church is still what really holds the community together. A day earlier,
on Sunday, the Driesens had gone to services in the morning and at
night. They unplugged the router and turned off their cellphones. They
read the Bible. Sioux Center was quiet on Sundays, when it is easier to
name what is open --- the Pizza Hut, the Culver's, the Walmart --- than
what is not.

Mr. Driesen spoke of the policies that were important to him, all the
usual conservative issues. Small government. Ending abortion. Judges who
share his political views. ``Traditional families,'' he said.

``Unfortunately,
\href{https://www.nytimes3xbfgragh.onion/2011/03/24/us/24divorce.html}{there's
just more divorce} than there used to be,'' he said. ``There's more
cohabitating. I think it is detrimental to the family. I just think kids
do better in a two-parent home, with a mom and a dad.''

Image

Framed photos of Rob Driesen with Vice President Mike Pence and
President Trump are displayed at his home in Ireton, Iowa.

Image

Republican presidential campaign signs hang in Mr. Driesen's garage. ``I
feel like we are safe for four more years,'' he said.

Image

Cheryl Driesen, second from left, and Mr. Driesen pray with their
children before dinner.

His wife had been quiet, letting him do the talking. She did not go to
Mr. Trump's speech, and politics were not her thing; often the men
around here were more vocal than their wives about supporting the
president. Now she spoke up.

``The religious part is huge for us, as we see religious freedoms being
taken away,'' Ms. Driesen said. ``If you don't believe in homosexuality
or something, you lose your business because of it. And that's a core
part of your faith. Whereas I see Trump as defending that. He's actually
made that executive order to put the Bibles back in the public schools.
That is something very worrisome and dear to us, our religious
freedom.''

She remembered how when her mother was a child about 20 miles north, the
public school still started the day with prayer. But when she was
growing up, it stopped. Her church, Netherlands Reformed, started a
private Christian school in Rock Valley, and so she went there instead.

They send their children to that same school, which still has some of
the same teachers.

``We don't know any different,'' Mr. Driesen said. ``For a lot of people
around here, that's just what you do. You have the same classmates all
the way through. And it holds the community together.'' His siblings
left the area for a while, but then they came back.

They want the Christian education for their children ``so we don't have
to have them indoctrinated with all these different things,'' he said.
``We are free to teach them our values.''

``So far,'' Ms. Driesen clarified. ``That's where we see Trump as a key
figure to keep that freedom.''

She paused. ``It's almost like it is a reverse intolerance. If you have
somebody that's maybe on the liberal side, they say that we are
intolerant of them. But it is inverse intolerant if we can't live out
our faith.''

She worried that the school might be forced to let in students who were
not Christian, or hire teachers who were gay.

``Silly things. Just let the boys go in the boys' bathroom and the girls
go in the girls','' he said. ``It's just something you'd think is never
going to happen, and nowadays it could. And it probably will.''

``Just hope nobody turns it upside down,'' he said.

``But we feel like we are in a little area where we are protected yet,''
she said. ``We are afraid of losing that, I guess.''

Every day, Mr. Driesen said, they pray. He wakes up and prays for his
family, and for safety at his job at Rural Electric Cooperative. Often
he would pray that when he hooked up a transformer it would not blow up.

They want America to be a Christian nation for their children. ``We
started out as a Christian nation,'' she said.

``You can't make people do these things,'' he said. ``But you can try to
protect what you've got, you might say.''

He thought about November, and felt confident Mr. Trump would win. He
sees Trump flags all over as he drives. Something has shifted in the
country, he said, and he is looking ahead to who might even come after
Mr. Trump.

``I feel like we are safe for four more years,'' he said. ``You know. So
that's a good feeling.''

Image

A barn outside of Orange City. Many evangelicals worry that Christian
values are disappearing from America.

\hypertarget{he-will-vanquish-all-our-foes}{%
\subsection{`He will vanquish all our
foes'}\label{he-will-vanquish-all-our-foes}}

Micah Schouten cannot remember exactly why he did not go to hear Mr.
Trump that morning. Probably it was just too cold, or maybe he was
working.

As a child he dreamed of being a farmer like his father, but land was
too expensive. Now he worked at a cattle reproduction company --- or, as
he explained with a smile, ``I.V.F. for cows.''

At the time, he supported Ben Carson. But Mr. Trump was a celebrity, and
Dordt University, 10 minutes down the road, was Mr. Schouten's alma
mater. The school was named for a major church assembly in 1618 and 1619
that declared salvation was only for God's chosen ones, and expelled
from Dutch territory anyone who disagreed. Its students are ``Dordt
Defenders,'' represented by a knight in gray armor, wielding a sword
like a cross.

So that night, after his three children went to bed, Mr. Schouten pulled
up YouTube to hear it for himself.

Soon Mr. Trump made him laugh. The candidate bashed the media. He said
the thing about shooting someone on Fifth Avenue. But the thing Mr.
Schouten remembered most was that he defended Christianity.

Mr. Schouten, 36, is proud of his town and during a tour pointed out a
community hospital and water park for children. Asked about the growing
Latino population in Sioux Center, he drove to an area he did not know
well and pointed out a trailer park where he said new arrivals, many of
them Latino workers, live.

When he was a child, he said, the public school students were almost
entirely white, and now about half of the kindergartners are Hispanic.
He noticed that many of the Latinos in town were Catholic, and that they
worked or shopped on Sunday, which was traditionally a time of rest in
Sioux Center.

``You can't find a single white person to milk cows or do any of that
stuff,'' he said. ``They know how to work hard. They don't mind working
those 12-hour shifts.''

On a Sunday in March, Mr. Schouten worshiped at United Reformed Church
with neighbors he has known for years. They all knew the harmonies by
heart. They were one choir, in sync on yellow quilted pews.

They sang: ``I will praise my dear Redeemer, his triumphant power I'll
tell, how the victory he giveth over sin and death and hell.''

They prayed: ``With our God we shall be valiant, he will vanquish all
our foes.''

The pastor spoke to a sea of white parishioners: ``God's standard
requires absolute, total, perfect, obedience.''

The Schoutens' oldest daughter, who was 11, took careful notes in her
journal.

When the service ended, the church served cookies. Mr. Schouten caught
up with some friends, all fathers in their 30s wearing blue collared
shirts and khaki pants.

``Trump's an outsider, like the rest of us,'' he said. ``We might not
respect Trump, but we still love the guy for who he is.''

``Is he a man of integrity? Absolutely not,'' he went on. ``Does he
stand up for some of our moral Christian values? Yes.''

The guys agreed. ``I'm not going to say he's a Christian, but he just
doesn't attack us,'' his friend Jason Mulder said.

Mr. Schouten's wife, Caryn, had walked over with the other wives. After
the election of President Barack Obama, the country seemed to undergo a
cultural shift, she said. ``It was dangerous to voice your
Christianity,'' she said. ``Because we were viewed as bigots, as racists
--- we were labeled as the haters and the ones who are causing all the
derision and all of the problems in America. Blame it on the white
believers.''

None of them said they had wanted to vote for Mr. Trump, but they did
--- ``When he was the last option,'' Heather Hoogendoorn said. The group
laughed.

But they agreed it would be easier to vote for him this time. Before, it
was hard to know what he would be like as president. Now they knew, and
they liked the results: Supreme Court justices, conservative judges,
including a Dordt graduate now on the U.S. Court of Appeals for the
Eighth Circuit, and growing clout for the anti-abortion movement.

``Obama wanted to take my assault rifle, he wanted to take out all the
high-capacity magazines,'' Mr. Schouten said. ``It just ---''

``--- felt like your freedoms kept getting taken from you,'' said
Heather's husband, Paul, finishing the sentence for him.

When the Schoutens got home, Caryn, 36, scooped a chip into sour cream
dip and plopped into a chair in her living room.

She spoke of her concern about sex trafficking. She had seen posts on
Facebook about mothers being followed to their cars if they went
shopping at Target in Sioux City, almost an hour away.

``I'm safe when I'm here. I'm not afraid when I'm here,'' she said.

They thought about the lives they want for their children, and why they
send them to a Christian elementary school. ``We hope our kids
eventually find a Christian spouse, and that exposes them to other kids
of like-mindedness,'' her husband said. The two of them met through
their rival Christian high schools.

Image

Micah and Caryn Schouten, who live in Sioux Center, send their children
to a Christian school.

Image

The Schoutens' older daughter reads the family's daily devotion.

Image

The Schoutens' younger daughter rides her bike in their front yard.

People seem to get married younger around here than they do in corporate
America, Mr. Schouten said. ``It's fairly common for women to go to
Dordt to get their M.R.S. degree, their Mrs. degree,'' he said.

When she was younger, his wife said, she used to say she would leave
Sioux County. She remembered the shock of traveling to Europe in high
school and seeing ``men in full drag'' for the first time.

``We have life very easy, it is laid back, it is like-minded people. And
it's just, I like the bubble,'' she said. ``I like not worrying about
sending them outside to play, or whose house they are going to if they
are going to the neighbors a few houses down, they might not go to the
same church, they might not hold all the same beliefs, but I trust them.
I don't know, maybe that is naïve.''

The years of the Obama presidency were confusing to her. She said she
heard talk of giving freedoms to gay people and members of minority
groups. But to her it felt like her freedoms were being taken away. And
that she was turning into the minority.

``I do not love Trump. I think Trump is good for America as a country. I
think Trump is going to restore our freedoms, where we spent eight
years, if not more, with our freedoms slowly being taken away under the
guise of giving freedoms to all,'' she said. ``Caucasian-Americans are
becoming a minority. Rapidly.''

She explained what she meant. ``If you are a hard-working
Caucasian-American, your rights are being limited because you are seen
as against all the races or against women,'' she said. ``Or there are
people who think that because we have conservative values and we value
the family and I value submitting to my husband, I must be against
women's rights.''

Her voice grew strong. ``I would say it takes a stronger woman to submit
to a man than to want to rule over him. And I would argue that point to
the death,'' she said.

She felt freer as she spoke. ``Mike Pence is a wonderful gentleman,''
she said. ``This is probably a very bad analogy, but I'd say he is like
the very supportive, submissive wife to Trump. He does the hard work,
and the husband gets the glory.''

She turned to her husband. ``Let's be real, Micah, do you have any clue
what goes on in our children's lives on a daily basis? No.'' They
laughed.

``Pence you can picture as your father, as your dad,'' he said.

But Mr. Biden as president really worried her: ``Biden is a few fries
short of a Happy Meal.''

Image

Wayside Chapel sits outside of Sioux Center, one of the nation's most
conservative Christian communities.

\hypertarget{theyre-not-hispanic}{%
\subsection{`They're not Hispanic'}\label{theyre-not-hispanic}}

Jesús Alvarado first came to the area a few months after Mr. Trump did,
and he was busy, preparing to start a church. It would be the first
Hispanic church in nearby Orange City --- one of just a few emerging in
the region.

He was commuting from an hour away, and had heard about the speech like
most people did, when the sound bite hit the headlines. All he really
remembered was thinking that Mr. Trump sounded like Hugo Chávez, the
former Venezuelan strongman.

Twenty years ago, less than 3 percent of Sioux County was Hispanic. Now,
that figure has nearly quadrupled, largely as the pork and dairy
industries have relied on Hispanic workers.

Most Hispanic migrants who come to the area are Catholic, but many
convert to evangelicalism, as he did, Mr. Alvarado said in his office at
Nueva Esperanza Iglesia, or New Hope Church. They kept a low profile,
especially the ones without the right papers. At first even he had
trouble finding them. Mostly they seemed to stick to work, home and the
grocery store.

``There's fear in the people,'' he said. ``The fear, the fear of losing
everything ---'' His unfinished sentence hung in the air. The lights in
the main fellowship hall were off.

Mr. Alvarado, 64, remembered how he ran away from home in Mexico when he
was 13. His mother had died when he was an infant, he said, and his aunt
and uncle could not pay for him to get an education. He found
agricultural work wherever he could, in New Mexico, California, Texas,
Colorado. At the time, he was undocumented. He met his wife when they
were both being detained on a bus. She was dressed for a dance, he
remembered, and three days later, on Valentine's Day, they got married.

When he was detained another time, he said, a Hispanic pastor spoke to
the judge on his behalf, reducing his sentence. He prayed on the side of
the road and devoted his life to God, and eventually got U.S.
citizenship. He began to start churches --- this one was his sixth.

He and his wife were renting a farmhouse and taking care of four of
their 13 grandchildren. He thought of how wonderful it was to raise them
here. The whole community --- the schools, the businesses --- is
evangelical-minded, he said, and the attitude toward immigrants has
grown more welcoming. One of his church members had called it ``a piece
of heaven for us.''

He appreciated that Mr. Trump defended Christians. But he had another
conviction: ``We should welcome foreigners, immigrants.''

``Doing things like dividing the family, I don't think that is very
Christian,'' he said. ``And building walls, instead of helping people
with medicine, food, especially old people getting sick for not having
enough income.''

Image

Jesús Alvarado, a pastor at Nueva Esperanza Iglesia in Orange City.

Image

``We should welcome foreigners, immigrants,'' Mr. Alvarado said.

Image

Mr. Alvarado leads a Spanish service for about 70 people after an Anglo
congregation's two services.

He does not talk about Mr. Trump with the white Christians around him.
His church has now joined an existing Anglo church, he said, under the
leadership of its pastor. Mr. Alvarado leads a Spanish service on Sunday
afternoons for about 70 people, after the Anglo congregation finishes
its two morning services.

``Maybe they know, that they realize that he is kind of persecuting
Hispanics, so they won't talk very much about that in front of me. I
won't, the same thing, I won't tell them my opinion,'' he said.

He grew quiet when he thought of why he believed that the white
evangelical community around him supported Mr. Trump. Then he spoke as
if it were obvious.

``They're not Hispanic,'' he said. ``They have not been living what we
have been going through.''

``They have to make their own decisions. I understand their point of
view,'' he went on. ``For them, the benefit is that he is pro-Christian.
Which is one of the things I like about him.''

He shared their worry about the disappearance of Christian values in
America, he said, and he was especially concerned about the future of
religious freedom.

``Our freedom has been under attack, that's the way I see it,'' he said.
``This country was based and built on God-fearful leaders, and changing
that is going to change one of the reasons why this country started, and
the thing that everybody loves about this country. A lot of people are
coming here because of the freedom.''

He will not tell his congregation which candidate he will vote for.
Politics, he said, is just not something they talk openly about.

\includegraphics{https://static01.graylady3jvrrxbe.onion/images/2020/08/08/autossell/Screen-Shot-2020-08-06-at-4/Screen-Shot-2020-08-06-at-4-videoSixteenByNineJumbo1600.png}

\hypertarget{the-line-to-lafayette}{%
\subsection{The line to Lafayette}\label{the-line-to-lafayette}}

It is deep into summer now. The pandemic has killed 160,000 people
nationwide. Thousands have taken to the streets to protest the police
killings of Black people. In Sioux Center, where the Black population is
less than 1 percent, feelings about Mr. Trump remain largely unchanged.

Only three people in the county are reported to have died of the
coronavirus. There was an outbreak of cases at the pork processing
plant. Churches have mostly reopened. The closest thing to a protest was
a walk for justice in Orange City.

``People in my circles, you don't really hear about racism, so I guess I
don't know too much about it,'' Mr. Driesen said of the protests. ``When
I see the pictures, I thought they all should be at work, being
productive citizens.''

``I still think he is going to blow Biden away,'' he said of Mr. Trump.

Ms. Schouten remembered a song she taught her children, called ``Jesus
Loves the Little Children.'' She quoted the lyrics, which have been sung
in churches for generations but would be considered racially insensitive
today: ``Red and yellow, black and white, all are precious in his
sight.''

``We are making this huge issue of white versus Black, Black Lives
Matter. All lives matter,'' she said. ``There are more deaths from
abortion than there are from corona, but we are not fighting that
battle.''

``We are picking and choosing who matters and who doesn't,'' she said.
``They say they are being picked on, when we are all being picked on in
one shape or form.''

The Trump era has revealed the complete fusion of evangelical
Christianity and conservative politics, even as white evangelical
Christianity continues to decline as a share of the national population.
There are some signs of fraying at the edges of the coalition, among
some women and young people. If even a small fraction turns away from
Mr. Trump, it could make the difference to his re-election.

But even if he loses in November, mainstream evangelical Christianity
has made plain its deepest impulses and exposed where the majority of
its believers pledge allegiance.

There is a straight line from that day at Dordt four years ago to a
recent scene at a chapel in Washington, where armed officers tear-gassed
peaceful protesters in Lafayette Square and shot them with rubber
pellets. They were
\href{https://www.nytimes3xbfgragh.onion/2020/06/02/us/politics/trump-walk-lafayette-square.html}{clearing
the way for Mr. Trump} to march from the White House to St. John's
Episcopal Church and hold up a Bible, a declaration of Christian power.

``We have the greatest country in the world,'' he said. ``We're going to
keep it nice and safe.''

It was another instantly infamous moment, covered by cable news and
decried by Democrats as an unseemly photo op. But in Sioux Center, many
evangelicals once again received a different message, one that echoed
the words uttered by a long-shot presidential candidate in a sanctuary
on a cold winter morning.

``To me it was like, that's great. Trump is recognizing the Bible, we
are one nation under God,'' Mr. Schouten said. ``He is willing to stand
out there and take a picture of it for the country to see.''

He added: ``Trump was standing up for Christianity.''

\hypertarget{our-2020-election-guide}{%
\section{Our 2020 Election Guide}\label{our-2020-election-guide}}

Updated Aug. 20, 2020

\begin{itemize}
\item
  \begin{center}\rule{0.5\linewidth}{\linethickness}\end{center}

  \hypertarget{convention-recap}{%
  \subsection{Convention Recap}\label{convention-recap}}

  \begin{itemize}
  \tightlist
  \item
    Joe Biden accepted the Democratic nomination, urging Americans to
    have faith that they could
    \href{https://www.nytimes3xbfgragh.onion/2020/08/20/us/politics/Joe-Biden-accepts-democratic-nomination.html?action=click\&pgtype=Article\&state=default\&region=BELOW_MAIN_CONTENT\&context=storylines_guide}{``overcome
    this season of darkness.''}
  \end{itemize}
\item
  \begin{center}\rule{0.5\linewidth}{\linethickness}\end{center}

  \hypertarget{news-analysis}{%
  \subsection{News Analysis}\label{news-analysis}}

  \begin{itemize}
  \tightlist
  \item
    Looming over Mr. Biden's nomination was the ever-present shadow of
    another man who's poised to dominate the campaign:
    \href{https://www.nytimes3xbfgragh.onion/2020/08/20/us/politics/biden-dnc-speech-trump.html?action=click\&pgtype=Article\&state=default\&region=BELOW_MAIN_CONTENT\&context=storylines_guide}{Donald
    J. Trump}.
  \end{itemize}
\item
  \begin{center}\rule{0.5\linewidth}{\linethickness}\end{center}

  \hypertarget{keep-up-with-our-coverage}{%
  \subsection{Keep Up With Our
  Coverage}\label{keep-up-with-our-coverage}}

  \begin{itemize}
  \tightlist
  \item
    Get an
    \href{https://www.nytimes3xbfgragh.onion/newsletters/politics?action=click\&pgtype=Article\&state=default\&region=BELOW_MAIN_CONTENT\&context=storylines_guide}{email}
    recapping the day's news
  \end{itemize}

  \begin{itemize}
  \tightlist
  \item
    Download our mobile app on
    \href{https://apps.apple.com/us/app/nytimes/id284862083?ls=1\&mat_click_id=5c79ae7455014fd1bd66b5610c05b8f2-20191112-16948\&referrer=mat_click_id\%3D5c79ae7455014fd1bd66b5610c05b8f2-20191112-16948\%26link_click_id\%3D722930677036718082}{iOS}
    and
    \href{http://a.localytics.com/android?id=com.nytimes.android\&referrer=utm_source\%3Dother_nyt_mobile_web\%26utm_medium\%3DWeb\%2520page\%26utm_term\%3DGeneral\%2520Mobile\%2520Page\%26utm_campaign\%3DNYT\%2520Mobile\%2520General\%2520Page}{Android}
    and turn on Breaking News and Politics alerts
  \end{itemize}
\end{itemize}

Advertisement

\protect\hyperlink{after-bottom}{Continue reading the main story}

\hypertarget{site-index}{%
\subsection{Site Index}\label{site-index}}

\hypertarget{site-information-navigation}{%
\subsection{Site Information
Navigation}\label{site-information-navigation}}

\begin{itemize}
\tightlist
\item
  \href{https://help.nytimes3xbfgragh.onion/hc/en-us/articles/115014792127-Copyright-notice}{©~2020~The
  New York Times Company}
\end{itemize}

\begin{itemize}
\tightlist
\item
  \href{https://www.nytco.com/}{NYTCo}
\item
  \href{https://help.nytimes3xbfgragh.onion/hc/en-us/articles/115015385887-Contact-Us}{Contact
  Us}
\item
  \href{https://www.nytco.com/careers/}{Work with us}
\item
  \href{https://nytmediakit.com/}{Advertise}
\item
  \href{http://www.tbrandstudio.com/}{T Brand Studio}
\item
  \href{https://www.nytimes3xbfgragh.onion/privacy/cookie-policy\#how-do-i-manage-trackers}{Your
  Ad Choices}
\item
  \href{https://www.nytimes3xbfgragh.onion/privacy}{Privacy}
\item
  \href{https://help.nytimes3xbfgragh.onion/hc/en-us/articles/115014893428-Terms-of-service}{Terms
  of Service}
\item
  \href{https://help.nytimes3xbfgragh.onion/hc/en-us/articles/115014893968-Terms-of-sale}{Terms
  of Sale}
\item
  \href{https://spiderbites.nytimes3xbfgragh.onion}{Site Map}
\item
  \href{https://help.nytimes3xbfgragh.onion/hc/en-us}{Help}
\item
  \href{https://www.nytimes3xbfgragh.onion/subscription?campaignId=37WXW}{Subscriptions}
\end{itemize}
