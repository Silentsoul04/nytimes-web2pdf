Sections

SEARCH

\protect\hyperlink{site-content}{Skip to
content}\protect\hyperlink{site-index}{Skip to site index}

\href{https://myaccount.nytimes3xbfgragh.onion/auth/login?response_type=cookie\&client_id=vi}{}

\href{https://www.nytimes3xbfgragh.onion/section/todayspaper}{Today's
Paper}

The Rising Photographer Inspired by Her Home Country

\url{https://nyti.ms/3fIZzVL}

\begin{itemize}
\item
\item
\item
\item
\item
\end{itemize}

Advertisement

\protect\hyperlink{after-top}{Continue reading the main story}

Supported by

\protect\hyperlink{after-sponsor}{Continue reading the main story}

T Presents

\hypertarget{the-rising-photographer-inspired-by-her-home-country}{%
\section{The Rising Photographer Inspired by Her Home
Country}\label{the-rising-photographer-inspired-by-her-home-country}}

The Jamaica-born, London-based Amber Pinkerton makes stylish pictures
that lead with her subjects' humanity.

\includegraphics{https://static01.graylady3jvrrxbe.onion/images/2020/08/10/t-magazine/art/Tadobe-slide-CF9I/Tadobe-slide-CF9I-articleLarge.jpg?quality=75\&auto=webp\&disable=upscale}

By \href{https://www.nytimes3xbfgragh.onion/by/tariro-mzezewa}{Tariro
Mzezewa}

\begin{itemize}
\item
  Aug. 10, 2020
\item
  \begin{itemize}
  \item
  \item
  \item
  \item
  \item
  \end{itemize}
\end{itemize}

Amber Pinkerton never intended for her work to be a commentary on
racism, classism or colorism, but perhaps more than any other medium,
photography has an uncanny ability to reflect our reality back to us in
ways both beautiful and stark. Pinkerton grew up in a creative family in
what she describes as a comfortable ``bubble'' within uptown Kingston,
Jamaica. As a child, she would watch her mother making mosaics in the
courtyard of their home, her aunts sewing clothes and her older brother
sketching anime characters.

By the age of 5, she knew that she, too, would become some kind of
artist, but it wasn't for another eight years that she began taking
photos of her (mostly light-skinned or white) friends and her
surroundings --- quick snapshots of everyday life captured on that most
immediate of devices, a smartphone. Recognizing her interest, her
parents gave her a Nikon D3000 camera as a 13th birthday present. Over
the next five years or so, Pinkerton, now 22, would regularly borrow
clothes and accessories from her parents' closet (a green turtleneck, a
beige double-breasted blazer), \emph{usually} with their permission, and
fashion some of her own, cutting hair ribbons out of paper and once
repurposing a black-and-white shag rug as a faux fur coat. She would
then drive to the top of the hill in her neighborhood and stage
exuberant D.I.Y. photo shoots, capturing her models (friends and
acquaintances who struck her as particularly stylish) in these looks as
they approximated poses from the editorials of the fashion magazines
they all pored over. ``I had no real intention when I started out,''
Pinkerton said recently over the phone from London. ``I mimicked until I
found my own voice.'' Nonetheless, she shared the images **** on
Facebook and quickly gained a following.

Image

Old i-D covers, ranging from the 1980s to the 2000s, that Pinkerton got
from a photographer she used to assist, and that now hang on a wall in
her bedroom.Credit...Will Sanders

Image

Pinkerton's desk, with an ever-growing stack of magazines she's
collected since moving to London in 2016 and often turns to for
reference.Credit...Will Sanders

At 19, she moved to London to study filmmaking, an experience that led
her to see her home in a new light. ``The socio-economic climate in
England **** was totally different from what I was coming from, and I
developed this keen interest in the dynamics between people,'' she said.
``Once I analyzed the society I grew up in, I became more aware of how
the people in the lower class, who are predominantly dark-skinned, are
looked down upon **** and never represented.'' A legacy of British
colonial rule, which lasted, nearly contiguously, from the mid-17th
century all the way to 1962, colorism in Jamaica, as it is across much
of the Caribbean, is so embedded in the society that those in power
typically have light skin and dark hair, and skin-lightening products
are popular drugstore purchases --- but there is a growing resistance to
these age-old biases to which Pinkerton has proudly added her voice.

\includegraphics{https://static01.graylady3jvrrxbe.onion/images/2020/08/10/t-magazine/tmag-adobe-thumbnail-slide-X5ZO/tmag-adobe-thumbnail-slide-X5ZO-mediumSquareAt3X.jpg}

Today, Pinkerton shows darker-skinned people in a way many locals have
never seen: within an art context. She selects backdrops that are
vividly evocative of the country's architectural and design vernacular
--- the white wrought-iron fence of a residential compound, the bright
chartreuse wall of a vacant house --- without revealing enough to allow
a viewer to make assumptions about who someone is based on where they
are. Rather, it is Pinkerton's sitters, who typically appear at ease and
return her gaze, staring directly into the camera, who seem to hold the
power.

Last year, Pinkerton photographed a portfolio of rising models
represented by the Kingston-based agency Saint International
\href{https://i-d.vice.com/en_uk/article/ne8mzd/saint-international-agency-amber-pinkerton-jamaican-models}{for
i-D} magazine, focusing on her subjects' faces and personalities. She
considers herself both a fashion and a documentary photographer, and the
images demonstrate her sensitivity not only to glamour and fantasy but
also to lives as they are actually lived. Pinkerton strikes a similar
balance in ``\href{https://www.instagram.com/p/B-w8tzeHKUw/}{Girls Next
Door},'' a series she shot in London for Dazed magazine earlier this
year. In one image, a woman in an extravagantly ruffled pink blouse sits
before a brightly iced birthday cake, looking bored, as if awaiting her
long-overdue guests; in another, the subject lounges, in a '70s-style
orange swimsuit, across a bed in a pastel-colored room that immediately
suggests the house of a parent or older relative.

\includegraphics{https://static01.graylady3jvrrxbe.onion/images/2020/08/10/t-magazine/tmag-adobe-thumbnail-slide-USXD/tmag-adobe-thumbnail-slide-USXD-mediumSquareAt3X.jpg}

These days, Pinkerton makes it back to Jamaica several times a year.
**** ``It's a paradise, but it also has deeper issues,'' she said.
``Every time I go, I see things differently.'' She also receives mixed
reviews of her work when she's there, with some viewers asking
point-blank why she doesn't photograph lighter-skinned subjects. ``Some
people are stuck in their prejudices,'' she said, ``but I've learned not
to let them change my vision.'' And while her images have been embraced
for their stylishness and heart abroad, that doesn't negate the fact
that representations of Blackness in international fashion magazines can
be fraught in their own ways. (Last month, for instance, the
photographer Annie Leibovitz and the mostly white editors of American
Vogue were
\href{https://www.huffpost.com/entry/simone-biles-vogue-black-photographers_n_5f09613ac5b6480493d05283?guccounter=1}{criticized}
for failing to properly light the gymnast Simone Biles in a cover
image.) Pinkerton's life experience may not mirror her subjects'
exactly, but she is a Jamaican woman invested in the culture she is
depicting.

As she contemplates the year ahead, she hopes to host educational ****
photography workshops around Jamaica and is looking forward to having
her first solo show, at London's
\href{https://www.aliceblackart.com/}{Alice Black Gallery}. For now,
though, she's hunkered down in London and thinking, once again, about
how she interacts with the world around her. If photography has been
able to telegraph so much about a place she thought she knew, what power
might other mediums hold? ``My work will continue to change,'' she said.
``I'm still young.''

\hypertarget{t-presents-15-creative-women-for-our-time}{%
\subsubsection{\texorpdfstring{\href{https://www.nytimes3xbfgragh.onion/interactive/2020/08/10/t-magazine/creative-women-designers-artists-chefs.html}{T
Presents: 15 Creative Women for Our
Time}}{T Presents: 15 Creative Women for Our Time}}\label{t-presents-15-creative-women-for-our-time}}

\href{https://www.nytimes3xbfgragh.onion/section/t-magazine}{}

\href{https://www.nytimes3xbfgragh.onion/2020/08/10/t-magazine/priya-ahluwalia-fashion-menswear.html}{\includegraphics{https://static01.graylady3jvrrxbe.onion/newsgraphics/2020/06/17/tmag-adobe/assets/images/ahluwalia-460.jpg}}

Priya Ahluwalia

Fashion Designer

\href{https://www.nytimes3xbfgragh.onion/2020/08/10/t-magazine/alice-cicolini-jewelry-art.html}{\includegraphics{https://static01.graylady3jvrrxbe.onion/newsgraphics/2020/06/17/tmag-adobe/assets/images/cicolini-460.jpg}}

Alice Cicolini

Jewelry Designer

\href{https://nytimes3xbfgragh.onion/2020/08/10/t-magazine/sonya-clark-flags-art.html}{\includegraphics{https://static01.graylady3jvrrxbe.onion/newsgraphics/2020/06/17/tmag-adobe/assets/images/clark-460.jpg}}

Sonya Clark

Artist

\href{https://www.nytimes3xbfgragh.onion/2020/08/10/t-magazine/pierre-davis-no-sesso.html}{\includegraphics{https://static01.graylady3jvrrxbe.onion/newsgraphics/2020/06/17/tmag-adobe/assets/images/davis-460.jpg}}

Pierre Davis

Fashion Designer

\href{https://www.nytimes3xbfgragh.onion/2020/08/10/t-magazine/paria-farzaneh-fashion-menswear.html}{\includegraphics{https://static01.graylady3jvrrxbe.onion/newsgraphics/2020/06/17/tmag-adobe/assets/images/farzaneh-460.jpg}}

Paria Farzaneh

Fashion Designer

\href{https://www.nytimes3xbfgragh.onion/2020/08/10/t-magazine/elizabeth-garouste-interior-design.html}{\includegraphics{https://static01.graylady3jvrrxbe.onion/newsgraphics/2020/06/17/tmag-adobe/assets/images/garouste-460.jpg}}

Elizabeth Garouste

Furniture Designer and Artist

\href{https://www.nytimes3xbfgragh.onion/2020/08/10/t-magazine/jatovia-gary-film.html}{\includegraphics{https://static01.graylady3jvrrxbe.onion/newsgraphics/2020/06/17/tmag-adobe/assets/images/gary-460.jpg}}

Ja'Tovia Gary

Artist and Filmmaker

\href{https://www.nytimes3xbfgragh.onion/2020/08/10/t-magazine/aiko-hachisuka-art-sculpture.html}{\includegraphics{https://static01.graylady3jvrrxbe.onion/newsgraphics/2020/06/17/tmag-adobe/assets/images/hachisuka-460.jpg}}

Aiko Hachisuka

Artist

\href{https://www.nytimes3xbfgragh.onion/2020/08/10/t-magazine/juliana-huxtable.html}{\includegraphics{https://static01.graylady3jvrrxbe.onion/newsgraphics/2020/06/17/tmag-adobe/assets/images/huxtable-460.jpg}}

Juliana Huxtable

Artist

\href{https://www.nytimes3xbfgragh.onion/2020/08/10/t-magazine/mariam-kamara-architect-design.html}{\includegraphics{https://static01.graylady3jvrrxbe.onion/newsgraphics/2020/06/17/tmag-adobe/assets/images/kamara-460.jpg}}

Mariam Kamara

Architect

\href{https://www.nytimes3xbfgragh.onion/2020/08/10/t-magazine/sophia-moreno-bunge-floral-design.html}{\includegraphics{https://static01.graylady3jvrrxbe.onion/newsgraphics/2020/06/17/tmag-adobe/assets/images/bunge-460.jpg}}

Sophia Moreno-Bunge

Floral Designer

\href{https://www.nytimes3xbfgragh.onion/2020/08/10/t-magazine/marina-moscone-fashion-design.html}{\includegraphics{https://static01.graylady3jvrrxbe.onion/newsgraphics/2020/06/17/tmag-adobe/assets/images/moscone-460.jpg}}

Marina Moscone

Fashion Designer

\href{https://www.nytimes3xbfgragh.onion/2020/08/10/t-magazine/amber-pinkerton-photography.html}{\includegraphics{https://static01.graylady3jvrrxbe.onion/newsgraphics/2020/06/17/tmag-adobe/assets/images/pinkerton-460.jpg}}

Amber Pinkerton

Photographer

\href{https://www.nytimes3xbfgragh.onion/2020/08/10/t-magazine/sonoko-sakai-chef-cooking-soba.html}{\includegraphics{https://static01.graylady3jvrrxbe.onion/newsgraphics/2020/06/17/tmag-adobe/assets/images/sakai-460.jpg}}

Sonoko Sakai

Cookbook Author and Food Activist

\href{https://www.nytimes3xbfgragh.onion/2020/08/10/t-magazine/daniela-soto-innes-cooking-chef.html}{\includegraphics{https://static01.graylady3jvrrxbe.onion/newsgraphics/2020/06/17/tmag-adobe/assets/images/ines-460.jpg}}

Daniela Soto-Innes

Chef

Advertisement

\protect\hyperlink{after-bottom}{Continue reading the main story}

\hypertarget{site-index}{%
\subsection{Site Index}\label{site-index}}

\hypertarget{site-information-navigation}{%
\subsection{Site Information
Navigation}\label{site-information-navigation}}

\begin{itemize}
\tightlist
\item
  \href{https://help.nytimes3xbfgragh.onion/hc/en-us/articles/115014792127-Copyright-notice}{©~2020~The
  New York Times Company}
\end{itemize}

\begin{itemize}
\tightlist
\item
  \href{https://www.nytco.com/}{NYTCo}
\item
  \href{https://help.nytimes3xbfgragh.onion/hc/en-us/articles/115015385887-Contact-Us}{Contact
  Us}
\item
  \href{https://www.nytco.com/careers/}{Work with us}
\item
  \href{https://nytmediakit.com/}{Advertise}
\item
  \href{http://www.tbrandstudio.com/}{T Brand Studio}
\item
  \href{https://www.nytimes3xbfgragh.onion/privacy/cookie-policy\#how-do-i-manage-trackers}{Your
  Ad Choices}
\item
  \href{https://www.nytimes3xbfgragh.onion/privacy}{Privacy}
\item
  \href{https://help.nytimes3xbfgragh.onion/hc/en-us/articles/115014893428-Terms-of-service}{Terms
  of Service}
\item
  \href{https://help.nytimes3xbfgragh.onion/hc/en-us/articles/115014893968-Terms-of-sale}{Terms
  of Sale}
\item
  \href{https://spiderbites.nytimes3xbfgragh.onion}{Site Map}
\item
  \href{https://help.nytimes3xbfgragh.onion/hc/en-us}{Help}
\item
  \href{https://www.nytimes3xbfgragh.onion/subscription?campaignId=37WXW}{Subscriptions}
\end{itemize}
