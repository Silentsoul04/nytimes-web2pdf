Sections

SEARCH

\protect\hyperlink{site-content}{Skip to
content}\protect\hyperlink{site-index}{Skip to site index}

\href{/section/travel}{Travel}\textbar{}On the Shores of Cape Cod, Where
the Oyster Is Their World

\url{https://nyti.ms/2DUcF55}

\begin{itemize}
\item
\item
\item
\item
\item
\item
\end{itemize}

\href{https://www.nytimes3xbfgragh.onion/spotlight/at-home?action=click\&pgtype=Article\&state=default\&region=TOP_BANNER\&context=at_home_menu}{At
Home}

\begin{itemize}
\tightlist
\item
  \href{https://www.nytimes3xbfgragh.onion/2020/08/14/dining/lobster-salad-recipe.html?action=click\&pgtype=Article\&state=default\&region=TOP_BANNER\&context=at_home_menu}{Make:
  Lobster Salad}
\item
  \href{https://www.nytimes3xbfgragh.onion/2020/08/15/at-home/coronavirus-at-home-quick-exercises.html?action=click\&pgtype=Article\&state=default\&region=TOP_BANNER\&context=at_home_menu}{Sneak
  In: Exercise}
\item
  \href{https://www.nytimes3xbfgragh.onion/interactive/2020/at-home/even-more-reporters-editors-diaries-lists-recommendations.html?action=click\&pgtype=Article\&state=default\&region=TOP_BANNER\&context=at_home_menu}{See:
  Reporters' Obsessions}
\item
  \href{https://www.nytimes3xbfgragh.onion/2020/08/15/at-home/coronavirus-fall-patio-furniture.html?action=click\&pgtype=Article\&state=default\&region=TOP_BANNER\&context=at_home_menu}{Deck
  Out: Your Porch}
\end{itemize}

\includegraphics{https://static01.graylady3jvrrxbe.onion/images/2020/08/15/travel/15travel-capecod1/merlin_175078722_594596fc-f1ba-4c5c-93d1-7323a33241ce-articleLarge.jpg?quality=75\&auto=webp\&disable=upscale}

The World Through a Lens

\hypertarget{on-the-shores-of-cape-cod-where-the-oyster-is-their-world}{%
\section{On the Shores of Cape Cod, Where the Oyster Is Their
World}\label{on-the-shores-of-cape-cod-where-the-oyster-is-their-world}}

Pollution, development and overharvesting have greatly diminished
America's natural oyster habitat. Aquaculture and adaptable farmers have
changed the game.

Flip bags containing oysters in Cape Cod Bay. The bags roll around with
the tide, tumbling the oysters and producing a hard, rounded
shell.Credit...

Supported by

\protect\hyperlink{after-sponsor}{Continue reading the main story}

Photographs and Text by Randy Harris

\begin{itemize}
\item
  Published Aug. 10, 2020Updated Aug. 12, 2020
\item
  \begin{itemize}
  \item
  \item
  \item
  \item
  \item
  \item
  \end{itemize}
\end{itemize}

\emph{At the onset of the coronavirus pandemic, with travel restrictions
in place worldwide, we launched a series ---}
\href{https://www.nytimes3xbfgragh.onion/column/the-world-through-a-lens}{\emph{The
World Through a Lens}} \emph{--- in which photojournalists help
transport you, virtually, to some of our planet's most beautiful and
intriguing places. This week, Randy Harris shares a collection of images
from the shores of Cape Cod.}

\begin{center}\rule{0.5\linewidth}{\linethickness}\end{center}

When I first met Chris Crobar, he was a half mile from the shore, on the
tidal flats that stretch far out into Cape Cod Bay. It was 5 a.m., and I
was out for a walk at low tide. From a distance, I saw what looked like
little black sails in the water.

Chris was a spectacle: alone with his boat and table in the middle of
the bay --- like an artist with his easel, painting a fiery sunrise. He
stood there fastidiously scraping the barnacles off his oysters, then
tossing them back into the cages where they'll sit for a couple of years
on the floor of the bay.

\includegraphics{https://static01.graylady3jvrrxbe.onion/images/2020/08/15/travel/15travel-capecod2/merlin_175077660_2db5e63b-d30e-461d-84e6-a9f033087e24-articleLarge.jpg?quality=75\&auto=webp\&disable=upscale}

Image

Sizing oysters for the market.

Before the arrival of the Europeans, the native people of Cape Cod, the
Nauset tribe, had an abundant supply of oysters. Crassostrea virginica,
known as the American oyster (or the eastern, Wellfleet, Atlantic or
Virginia oyster), was naturally flush in coastal areas and estuaries,
where the rivers meet the sea. Oyster reefs were America's coral reefs;
oysters filtered the water --- some adult oysters can filter 50 gallons
a day --- and fed a range of other sea life.

In the 18th and 19th centuries, disease, overharvesting and new
technologies --- including developments in dredging, canning and
transportation --- transformed America's oyster industry. The native
oyster population in Wellfleet, Mass., for example, was nearly depleted.
To cope, locals began seeding the waters with young oysters imported
from the Chesapeake.

In his
\href{https://play.google.com/books/reader?id=F6ATAAAAYAAJ\&pg=GBS.PA24}{1881
report}, Ernest Ingersoll wrote that ``a hundred thousand bushels of the
bivalves once grew fat'' along the Wellfleet waterfront. ``Now, a little
experimental propagation, of the value of a few hundred dollars, and
about 6,000 bushels of bedded oysters from Virginia, worth perhaps
\$5,000 when sold, form the total active business.''

Image

A salt marsh near Hatch Beach, at low tide.

Today, America's natural oyster habitat remains
\href{https://www.fisheries.noaa.gov/national/habitat-conservation/oyster-reef-habitat\#:~:text=Oysters\%20live\%20in\%20salty\%20or,other\%20marine\%20animals\%20and\%20plants.}{greatly
diminished}. Estimates in the last decade have placed the wild oyster
population at some places in the country at just
\href{https://www.outsideonline.com/2398648/oyster-farming-environment}{one
percent of historical levels}.

Image

In some places in America, the wild oyster population is at a mere one
percent of historical levels.

Modern-day aquaculture has changed the game. A company called
\href{https://www.archatchery.com/}{Aquacultural Research Corporation},
founded around 1960, produces shellfish seed ---~tiny juvenile oysters,
also called ``spat'' --- and sells it to local shellfish farmers. Chris
Crobar is one of those farmers. A native of Eastham, Mass., he grew up
working on his father's clam boat. Today, he sets his hours not by the
clock, but by the tide. He has been raising oysters for five years on
his two-acre farm in the Cape Cod Bay --- just beyond First Encounter
Beach, the location of the first meeting between the Nauset people and
the Pilgrims.

Image

A seagull attempts to break open a quahog by dropping it from the air.

Cape Cod is as unique as its oysters. Depending on the location, the
high tidal waters flush the oysters with a varying mix of freshwater and
saltwater. This helps create nuanced flavors.

Wellfleet, which juts farther out into the bay, is famous throughout the
world for its briny oyster. In Barnstable, Chatham and Orleans, the
fresh tidal water and sweet marsh algae combine to create a sweet and
earthy flavor profile. Eastham oysters are known for being both mildly
briny and earthy.

Image

With oyster knife in hand, Chris scrapes barnacles off an oyster.

Paul Wittenstein, the general manager of A.R.C., explained how the
hatchery produces its seed: In midwinter, the hatchery places adult
shellfish in warm water that's rich with algae, which causes the
shellfish to spawn. The hatchery then catches the eggs, hatches them and
grows them in their tanks until spring, when they're moved into the
A.R.C.'s nursery system. From there, they continue to grow before being
measured, counted and sold to farmers.

Image

Chris buys oyster seed --- measuring 15 to 20 millimeters --- from
Aquacultural Research Corporation.

After obtaining his oyster seeds from A.R.C., Chris initially grows them
in mesh bags, using the French rack-and-bag system. Later, the oysters
are transferred into metal grow-out cages, where they sit on the bottom
of the bay from one-and-a-half to two years~---~until they reach 2.5 to
3 inches, to be marketed as ``petite'' or ``cocktail'' oysters, or 3
inches or more to be marketed as ``select'' oysters. (He also digs for
wild
\href{http://www.edc.uri.edu/restoration/html/gallery/invert/razor.htm}{razor
clams} and
\href{https://www.edc.uri.edu/restoration/html/gallery/invert/quahog.htm}{quahogs}.)

Image

An array of oyster bags.

Image

Chris attaches a bag of seed to a rack, using what's called the French
rack-and-bag-system.

Lately there has been a surge in aquaculture farming, especially with
oysters. But yields can vary significantly from year to year. This past
winter was mild, with little to no ice. Seed did well. Many farmers were
expecting a good year. But with water temperatures increasing to over 80
degrees at high tide, both algae blooms and crashes can result, leaving
the oysters with nothing to eat.

Oyster farming, in other words, has always been an unpredictable
business. And the coronavirus pandemic has hit the industry hard: With
fewer people dining out, farmers are sitting on their inventory. Some
have feared a
\href{https://www.ecori.org/aquaculture/2020/5/5/with-restaurants-shuttered-oyster-farmers-face-market-collapse}{collapse
of the market}.

Image

The sun rises along a trail at Cape Cod National Seashore.

But Chris is more hopeful. ``It's important to me to be optimistic about
the future,'' he said. ``For now we have to keep planting and raising
shellfish, hoping that things will eventually get back to normal.''

``Fishermen adapt,'' he added, ``and always find a way to keep moving
forward.''

\begin{center}\rule{0.5\linewidth}{\linethickness}\end{center}

\href{https://www.iamrandyharris.com/}{\emph{Randy Harris}} \emph{is a
photographer based in New York. You can follow his work on}
\href{https://www.instagram.com/iamrandyharris/}{\emph{Instagram}}\emph{.}

\emph{\textbf{Follow New York Times Travel}} \emph{on}
\href{https://www.instagram.com/nytimestravel/}{\emph{Instagram}}\emph{,}
\href{https://twitter.com/nytimestravel}{\emph{Twitter}} \emph{and}
\href{https://www.facebookcorewwwi.onion/nytimestravel/}{\emph{Facebook}}\emph{.
And}
\href{https://www.nytimes3xbfgragh.onion/newsletters/traveldispatch}{\emph{sign
up for our weekly Travel Dispatch newsletter}} \emph{to receive expert
tips on traveling smarter and inspiration for your next vacation.}

Advertisement

\protect\hyperlink{after-bottom}{Continue reading the main story}

\hypertarget{site-index}{%
\subsection{Site Index}\label{site-index}}

\hypertarget{site-information-navigation}{%
\subsection{Site Information
Navigation}\label{site-information-navigation}}

\begin{itemize}
\tightlist
\item
  \href{https://help.nytimes3xbfgragh.onion/hc/en-us/articles/115014792127-Copyright-notice}{©~2020~The
  New York Times Company}
\end{itemize}

\begin{itemize}
\tightlist
\item
  \href{https://www.nytco.com/}{NYTCo}
\item
  \href{https://help.nytimes3xbfgragh.onion/hc/en-us/articles/115015385887-Contact-Us}{Contact
  Us}
\item
  \href{https://www.nytco.com/careers/}{Work with us}
\item
  \href{https://nytmediakit.com/}{Advertise}
\item
  \href{http://www.tbrandstudio.com/}{T Brand Studio}
\item
  \href{https://www.nytimes3xbfgragh.onion/privacy/cookie-policy\#how-do-i-manage-trackers}{Your
  Ad Choices}
\item
  \href{https://www.nytimes3xbfgragh.onion/privacy}{Privacy}
\item
  \href{https://help.nytimes3xbfgragh.onion/hc/en-us/articles/115014893428-Terms-of-service}{Terms
  of Service}
\item
  \href{https://help.nytimes3xbfgragh.onion/hc/en-us/articles/115014893968-Terms-of-sale}{Terms
  of Sale}
\item
  \href{https://spiderbites.nytimes3xbfgragh.onion}{Site Map}
\item
  \href{https://help.nytimes3xbfgragh.onion/hc/en-us}{Help}
\item
  \href{https://www.nytimes3xbfgragh.onion/subscription?campaignId=37WXW}{Subscriptions}
\end{itemize}
