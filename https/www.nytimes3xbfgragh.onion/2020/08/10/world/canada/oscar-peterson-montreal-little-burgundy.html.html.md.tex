Sections

SEARCH

\protect\hyperlink{site-content}{Skip to
content}\protect\hyperlink{site-index}{Skip to site index}

\href{https://www.nytimes3xbfgragh.onion/section/world/canada}{Canada}

\href{https://myaccount.nytimes3xbfgragh.onion/auth/login?response_type=cookie\&client_id=vi}{}

\href{https://www.nytimes3xbfgragh.onion/section/todayspaper}{Today's
Paper}

\href{/section/world/canada}{Canada}\textbar{}Should Montreal Subway
Honor Polarizing Priest or Jazz Genius?

\url{https://nyti.ms/3ac9cv5}

\begin{itemize}
\item
\item
\item
\item
\item
\end{itemize}

Advertisement

\protect\hyperlink{after-top}{Continue reading the main story}

Supported by

\protect\hyperlink{after-sponsor}{Continue reading the main story}

Montreal Dispatch

\hypertarget{should-montreal-subway-honor-polarizing-priest-or-jazz-genius}{%
\section{Should Montreal Subway Honor Polarizing Priest or Jazz
Genius?}\label{should-montreal-subway-honor-polarizing-priest-or-jazz-genius}}

A fight to rename a metro stop pits those who want to celebrate a
revered Black musician against those who want to retain the name of a
priest who espoused equal rights for Francophone Quebecers --- and also
anti-Semitism.

\includegraphics{https://static01.graylady3jvrrxbe.onion/images/2020/08/06/world/00Montreal-Dispatch/00Montreal-Dispatch-articleLarge.jpg?quality=75\&auto=webp\&disable=upscale}

\href{https://www.nytimes3xbfgragh.onion/by/dan-bilefsky}{\includegraphics{https://static01.graylady3jvrrxbe.onion/images/2019/01/10/multimedia/author-dan-bilefsky/author-dan-bilefsky-thumbLarge.png}}

By \href{https://www.nytimes3xbfgragh.onion/by/dan-bilefsky}{Dan
Bilefsky}

\begin{itemize}
\item
  Aug. 10, 2020
\item
  \begin{itemize}
  \item
  \item
  \item
  \item
  \item
  \end{itemize}
\end{itemize}

MONTREAL --- Steps away from Montreal's historically Black neighborhood
of Little Burgundy, the handsome gray-stone house where the Canadian
jazz virtuoso Oscar Peterson grew up sits conspicuously empty.

There is no city plaque on the house designating it a landmark, nor any
street named after Mr. Peterson, a dazzling, finger-flying pianist and
20th-century musical giant whom Duke Ellington called
\href{https://www.amazon.ae/Oscar-Peterson-Maharaja-Keyboard-Biography/dp/1599860481}{``the
maharajah of the keyboard}.''

But Naveed Hussain, a 36-year-old nurse, thinks something more should be
done to honor the musician --- while removing what he views as an
anachronistic blight, close by the musician's childhood home.

Inspired by a global reckoning in support of Black rights, he
\href{https://www.change.org/p/ville-de-montr\%C3\%A9al-change-lionel-groulx-metro-station-and-rename-it-oscar-peterson-metro-station?recruiter=20620186\&utm_source=share_petition\&utm_medium=facebook\&utm_campaign=psf_combo_share_abi\&utm_term=signature_receipt\&recruited_by_id=5f8fd560-f07c-012f-fc7c-4040aa777426\&utm_content=fht-22876605-en-ca\%3Av12}{wants}
to rename the Lionel-Groulx subway station, which memorializes
\href{https://www.thecanadianencyclopedia.ca/en/article/lionel-adolphe-groulx}{a
polarizing Roman Catholic priest} and historian who championed the
rights of Francophone Quebecers in English-dominated Canada, but who
also
\href{http://faculty.marianopolis.edu/c.belanger/QuebecHistory/docs/jews/Anti-SemitismandQuebecNationalists.html}{espoused
virulent anti-Semitism and fascist sympathies}.

``The metros and monuments in this city are irrelevant to our current
times and glorify imperialists and conquerors and, in the case of Lionel
Groulx, someone who suggested certain immigrants didn't have a place in
Quebec society,'' Mr. Hussain said. ``Oscar Peterson is a symbol of
unity.''

\includegraphics{https://static01.graylady3jvrrxbe.onion/images/2020/08/07/world/07montreal-dispatch-add/merlin_175426887_58e7ce50-53f2-41e4-82ee-5272d19f54f3-articleLarge.jpg?quality=75\&auto=webp\&disable=upscale}

Mr. Hussain, who lived for a time in Little Burgundy, said he was
fighting for nothing less than the soul of the city, eager for the
subway station to reflect the contributions of Canadians of color.

But his push has given rise to a backlash and a counterpetition by those
who contend that Mr. Groulx deserves his place in the city's pantheon.

Especially for some older, French-speaking residents, Mr. Groulx is a
towering figure of the first half of the 20th century whose insistence
on equality for Francophone Quebecers deserves to be remembered.

``Leave the metro alone --- it is a thank you for what this man gave to
Quebec,'' said Annie Roux, 60, a life coach and astrologer who has lived
next door to the station for several years.

The naming skirmish has become an emblem of a long-simmering cultural
battle over the toponymy of a city that was colonized by both France and
Britain, where street names honoring 19th-century British monarchs sit
alongside grand boulevards renamed after 20th-century Québécois
nationalists.

Mr. Hussain said it was a fitting tribute to Montreal's multiculturalism
that someone like him, a Canadian-Muslim with Pakistani roots, was
challenging the celebration of a man with anti-Semitic views in order to
honor
\href{https://www.banffcentre.ca/oscar-peterson-canadian-jazz-legend}{a
Black jazz great} who
\href{https://www.grammy.com/grammys/artists/oscar-peterson/13621}{won
eight Grammy Awards} before
\href{https://www.nytimes3xbfgragh.onion/2007/12/25/arts/25peterson.html}{his
death in 2007} at 82.

Image

A mural of Daisy Peterson Sweeney, a piano teacher and pillar of the
community, and Oscar Peterson's sister, in Little
Burgundy.Credit...Nasuna Stuart-Ulin for The New York Times

Mr. Hussain's petition to rename the subway station (which hyphenates
Mr. Groulx's first and last name) has received nearly 25,000 signatures
--- although some commentators on the counterpetition criticized a
``witch hunt'' against major figures of the past.

His campaign had also pitted modernizers against conservationists and
scholars who argue that the names of subways, streets and statues should
be preserved as historical records.

Cities across the world, from
\href{https://www.nytimes3xbfgragh.onion/2020/06/14/world/europe/Bristol-Colston-statue-slavery.html}{Bristol,
England}, to
\href{https://www.nytimes3xbfgragh.onion/2020/07/26/us/politics/lexington-va-confederate-generals.html}{Lexington,
Va}., have been rethinking their identities amid calls to remove
monuments honoring historical figures who advocated slavery or held
racist views. In Canada, demands have been growing to topple statues
honoring
\href{https://www.nytimes3xbfgragh.onion/2017/08/28/world/americas/canada-john-a-macdonald-kingston.html}{John
A. Macdonald}, Canada's first prime minister, because of his role in
repressing Indigenous people.

On a recent day in Little Burgundy, once known as ``The Harlem of The
North,'' local residents lamented that the social history of Black
Quebecers was noticeably absent or underplayed in Quebec's history
books, popular culture and urban spaces, and overshadowed by the
struggle of white, French-speaking Quebecers for their own rights.

``If you are brown or Black in Quebec, you are seen as the Other,'' said
Charlene Hunte, head of outreach at
\href{http://www.unionunitedchurchmtl.ca/}{the Union United Church},
which Mr. Peterson attended. Montreal, she observed, didn't have a
single subway station named after women or ethnic minorities. ``Black
history is being erased,'' she said.

Image

Charlene Hunte outside the Union United Church, which Oscar Peterson
attended, and where her four children were baptized.Credit...Nasuna
Stuart-Ulin for The New York Times

In Quebec, a majority Francophone province, Mr. Hussain's attempt to
rebrand the popular transport hub has intensified enduring debates over
language, memory and the legacy of colonialism.

Writing in Le Devoir, a leading Montreal-based newspaper,
\href{https://professeurs.uqam.ca/professeur/tellier.luc-n/}{Luc-Normand
Tellier}\textbf{,} an emeritus professor of urban studies at
\href{https://uqam.ca/}{Université du Québec à Montréal}, argued that
Lionel-Groulx station should keep its name because of the pivotal role
its namesake played in shaping the French identity of Quebec.

He suggested renaming the city's McGill subway station after Mr.
Peterson, since
\href{https://www.mcgill.ca/about/history/meet-james-mcgill}{James
McGill,} an 18th-century Scottish businessman whose name adorns McGill
University, owned six Black slaves. ``Such a gesture would, at once,
underline how slavery was intolerable while honoring the contribution of
Blacks to Montreal society,'' he wrote.

\href{https://twitter.com/rphilpotinfo?lang=en}{Robin Philpot}, a
prominent Quebec writer,
\href{https://www.ledevoir.com/opinion/idees/581602/le-combat-pour-la-francisation-de-montreal-de-groulx-a-landry}{argued}
that Montreal should guard against a longstanding drive by the British
conquerors of Quebec and their descendants to Anglicize the names of
streets and bridges in the city.

Image

A game of soccer in Little Burgundy.Credit...Nasuna Stuart-Ulin for The
New York Times

For others, like \href{https://www.youtube.com/watch?v=jc0tDCt5JbY}{Eric
Scott}, who made a documentary film about anti-Semitism in Quebec, the
support to keep the station's name reflected an unwillingness to
acknowledge ``pro-fascist sympathies'' in Quebec in the 1930s through
the postwar period.

Experts agree that Mr. Groulx was a divisive figure who had expressed
anti-Semitic views. But those views, they say, weren't his central
preoccupation and needed to be examined within the context of the
prevailing social mores of his times.

Youssef Amane, a spokesman for Montreal's mayor, Valérie Plante, said
there was a moratorium on renaming subway stations. He noted, however,
that Mr. Peterson had been honored with a park in Little Burgundy as
well as a mural.

``There is more to be done to honor the contributions of the Black
community,'' Mr. Amane added.

In the late 19th century, Black migrants from across Canada, the United
States and the Caribbean came to Little Burgundy to work.

During Prohibition in the United States, the area became a center for
jazz, with boogie-woogie rhythms and booze both flowing freely.

Today, blocky social housing is being gradually supplanted by upmarket
restaurants, designer condominiums and other signs of gentrification.
The Negro Community Center, the heart of the community since 1927, was
demolished a few years ago and is now an empty lot.

Image

A mural of Mr. Peterson overlooking a protest march in Montreal calling
to revive Little Burgundy's historic Negro Community
Center.Credit...Nasuna Stuart-Ulin for The New York Times

\href{https://www.youtube.com/watch?v=JB2gqbO8_i0}{Oliver Jones}, 86, a
celebrated Canadian jazz pianist from Little Burgundy, who was mentored
by Mr. Peterson and went on to play Carnegie Hall, said renaming the
station after his old friend would help cement the neighborhood's
legacy.

``Wherever I have traveled, whether in Puerto Rico or China or
Australia, everyone knows the name Oscar Peterson,'' he said.

He said Mr. Peterson, the son of a West Indian immigrant who worked as a
sleeping car railway porter, was deeply shaped by Little Burgundy.
During the height of its popularity from 1930 to 1950, people across the
racial divide flocked to the famous Black-owned
\href{https://www.thecanadianencyclopedia.ca/en/article/rockheads-paradise}{Rockhead's
Paradise jazz club} to hear him and other jazz greats like Billie
Holiday, Ella Fitzgerald and Louis Armstrong play.

``It was the best show in town,'' Mr. Jones said.

Mr. Peterson has received many accolades, including being honored with a
Canadian postage stamp. But if there was any reluctance to rename a
station after him in Montreal, Mr. Jones said, it was probably because
he was an Anglophone and had moved to Ontario.

``Maybe they just picked the wrong station since Lionel Groulx is a hero
for French Canadians,'' he said.

\href{https://www.mcgill.ca/tcpsych/faculty/myrnalashley}{Myrna
Lashley}, an expert on race relations and assistant professor at McGill
University, supports renaming the station after Oscar Peterson. But she
argued that cities struggling with what to do with statues and other
memorials that can't be renamed should install explanatory plaques to
educate the public about past wrongs.

``We can't start tearing everything down or we will have nothing left,''
she said.

As for Mr. Hussain, he's not backing down.

``Imagine if we were talking about Celine Dion, the city would be
tearing up an entire street to honor her,'' he said. ``The same should
happen for Oscar Peterson.''

Nasuna Stuart-Ulin contributed reporting.

Advertisement

\protect\hyperlink{after-bottom}{Continue reading the main story}

\hypertarget{site-index}{%
\subsection{Site Index}\label{site-index}}

\hypertarget{site-information-navigation}{%
\subsection{Site Information
Navigation}\label{site-information-navigation}}

\begin{itemize}
\tightlist
\item
  \href{https://help.nytimes3xbfgragh.onion/hc/en-us/articles/115014792127-Copyright-notice}{©~2020~The
  New York Times Company}
\end{itemize}

\begin{itemize}
\tightlist
\item
  \href{https://www.nytco.com/}{NYTCo}
\item
  \href{https://help.nytimes3xbfgragh.onion/hc/en-us/articles/115015385887-Contact-Us}{Contact
  Us}
\item
  \href{https://www.nytco.com/careers/}{Work with us}
\item
  \href{https://nytmediakit.com/}{Advertise}
\item
  \href{http://www.tbrandstudio.com/}{T Brand Studio}
\item
  \href{https://www.nytimes3xbfgragh.onion/privacy/cookie-policy\#how-do-i-manage-trackers}{Your
  Ad Choices}
\item
  \href{https://www.nytimes3xbfgragh.onion/privacy}{Privacy}
\item
  \href{https://help.nytimes3xbfgragh.onion/hc/en-us/articles/115014893428-Terms-of-service}{Terms
  of Service}
\item
  \href{https://help.nytimes3xbfgragh.onion/hc/en-us/articles/115014893968-Terms-of-sale}{Terms
  of Sale}
\item
  \href{https://spiderbites.nytimes3xbfgragh.onion}{Site Map}
\item
  \href{https://help.nytimes3xbfgragh.onion/hc/en-us}{Help}
\item
  \href{https://www.nytimes3xbfgragh.onion/subscription?campaignId=37WXW}{Subscriptions}
\end{itemize}
