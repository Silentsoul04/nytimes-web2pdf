Sections

SEARCH

\protect\hyperlink{site-content}{Skip to
content}\protect\hyperlink{site-index}{Skip to site index}

\href{https://www.nytimes3xbfgragh.onion/section/health}{Health}

\href{https://myaccount.nytimes3xbfgragh.onion/auth/login?response_type=cookie\&client_id=vi}{}

\href{https://www.nytimes3xbfgragh.onion/section/todayspaper}{Today's
Paper}

\href{/section/health}{Health}\textbar{}`This Is All Beyond Stupid.'
Experts Worry About Russia's Rushed Vaccine

\url{https://nyti.ms/30L1qFa}

\begin{itemize}
\item
\item
\item
\item
\item
\item
\end{itemize}

\hypertarget{the-coronavirus-outbreak}{%
\subsubsection{\texorpdfstring{\href{https://www.nytimes3xbfgragh.onion/news-event/coronavirus?name=styln-coronavirus-national\&region=TOP_BANNER\&block=storyline_menu_recirc\&action=click\&pgtype=Article\&impression_id=75bc9470-efba-11ea-81ab-99b37689be52\&variant=undefined}{The
Coronavirus
Outbreak}}{The Coronavirus Outbreak}}\label{the-coronavirus-outbreak}}

\begin{itemize}
\tightlist
\item
  live\href{https://www.nytimes3xbfgragh.onion/2020/09/05/world/coronavirus-covid.html?name=styln-coronavirus-national\&region=TOP_BANNER\&block=storyline_menu_recirc\&action=click\&pgtype=Article\&impression_id=75bc9471-efba-11ea-81ab-99b37689be52\&variant=undefined}{Latest
  Updates}
\item
  \href{https://www.nytimes3xbfgragh.onion/interactive/2020/us/coronavirus-us-cases.html?name=styln-coronavirus-national\&region=TOP_BANNER\&block=storyline_menu_recirc\&action=click\&pgtype=Article\&impression_id=75bc9472-efba-11ea-81ab-99b37689be52\&variant=undefined}{Maps
  and Cases}
\item
  \href{https://www.nytimes3xbfgragh.onion/interactive/2020/science/coronavirus-vaccine-tracker.html?name=styln-coronavirus-national\&region=TOP_BANNER\&block=storyline_menu_recirc\&action=click\&pgtype=Article\&impression_id=75bc9473-efba-11ea-81ab-99b37689be52\&variant=undefined}{Vaccine
  Tracker}
\item
  \href{https://www.nytimes3xbfgragh.onion/2020/09/02/your-money/eviction-moratorium-covid.html?name=styln-coronavirus-national\&region=TOP_BANNER\&block=storyline_menu_recirc\&action=click\&pgtype=Article\&impression_id=75bc9474-efba-11ea-81ab-99b37689be52\&variant=undefined}{Eviction
  Moratorium}
\item
  \href{https://www.nytimes3xbfgragh.onion/interactive/2020/09/02/magazine/food-insecurity-hunger-us.html?name=styln-coronavirus-national\&region=TOP_BANNER\&block=storyline_menu_recirc\&action=click\&pgtype=Article\&impression_id=75bc9475-efba-11ea-81ab-99b37689be52\&variant=undefined}{American
  Hunger}
\end{itemize}

Advertisement

\protect\hyperlink{after-top}{Continue reading the main story}

Supported by

\protect\hyperlink{after-sponsor}{Continue reading the main story}

\hypertarget{this-is-all-beyond-stupid-experts-worry-about-russias-rushed-vaccine}{%
\section{`This Is All Beyond Stupid.' Experts Worry About Russia's
Rushed
Vaccine}\label{this-is-all-beyond-stupid-experts-worry-about-russias-rushed-vaccine}}

Vaccines are among the safest medical products in the world --- but only
because of the intense rigor of the clinical trials that test their
safety and effectiveness.

\includegraphics{https://static01.graylady3jvrrxbe.onion/images/2020/08/11/science/11VACCINE-SAFETY1/merlin_175576047_d3096dfc-440e-4924-b7f4-894aeaf3f0c7-articleLarge.jpg?quality=75\&auto=webp\&disable=upscale}

\href{https://www.nytimes3xbfgragh.onion/by/carl-zimmer}{\includegraphics{https://static01.graylady3jvrrxbe.onion/images/2018/06/12/multimedia/author-carl-zimmer/author-carl-zimmer-thumbLarge.png}}

By \href{https://www.nytimes3xbfgragh.onion/by/carl-zimmer}{Carl Zimmer}

\begin{itemize}
\item
  Aug. 11, 2020
\item
  \begin{itemize}
  \item
  \item
  \item
  \item
  \item
  \item
  \end{itemize}
\end{itemize}

\href{https://www.nytimes3xbfgragh.onion/es/2020/08/12/espanol/ciencia-y-tecnologia/vacuna-rusia-coronavirus.html}{Leer
en
español}\href{https://cn.nytimes3xbfgragh.onion/world/20200813/russia-covid-19-vaccine-safety/}{阅读简体中文版}\href{https://cn.nytimes3xbfgragh.onion/world/20200813/russia-covid-19-vaccine-safety/zh-hant/}{閱讀繁體中文版}

When Vladimir Putin announced Tuesday that Russia had approved a
coronavirus vaccine --- with no evidence from large-scale clinical
trials --- vaccine experts were worried.

``I think it's really scary. It's really risky,'' said Daniel Salmon,
the director of the Institute for Vaccine Safety at Johns Hopkins
University.

Dr. Salmon and other experts said that Russia is taking a dangerous step
by jumping ahead of so-called Phase 3 trials, which can determine that
the vaccine works better than a placebo and doesn't cause harm to some
people who get it.

Unlike experimental drugs given to the sick, vaccines are intended to be
given to masses of healthy people. So they must clear a high bar of
safety standards. If hundreds of millions of people get a vaccine, even
a rare side effect could crop up in thousands of people.

Over the course of the past century, researchers have developed
increasingly powerful ways to test vaccines for safety and
effectiveness. Some of those lessons were learned the hard way, when a
new vaccine caused some harm. But
\href{https://www.nytimes3xbfgragh.onion/2019/06/18/health/you-should-get-vaccinated.html}{vaccines
are now among the safest medical products in the world} thanks to the
intense rigor of the clinical trials tracking their safety and
effectiveness.

This testing typically begins before a single person has received a new
vaccine, when researchers inject it into mice or monkeys to see how they
respond.

If those animal studies turn out well, researchers then enlist a few
dozen volunteers for a Phase 1 trial, in which all volunteers get the
experimental vaccine.

Doctors typically keep these volunteers under observation to make sure
they don't have any immediate negative reactions, and to see whether
they make antibodies against a pathogen. It's not uncommon for people to
feel achiness in their muscles or even a mild fever, but these mild
symptoms typically don't last long.

If Phase 1 trials do not turn up serious safety problems, then
researchers usually move to a Phase 2 trial, in which they inject
hundreds of people and make more detailed observations.

The first clinical trials on coronavirus vaccines started in March, and
\href{https://www.nytimes3xbfgragh.onion/interactive/2020/science/coronavirus-vaccine-tracker.html}{now
there are 29 underway}, with more to launch soon. Companies such as
AstraZeneca, Moderna, Novavax and Pfizer are beginning to share
optimistic early results: So far, they have only detected mild or
moderate symptoms and no severe side effects. Volunteers have also
produced antibodies to the coronavirus, in some cases more than are
produced by people who have recovered from an infection.

\includegraphics{https://static01.graylady3jvrrxbe.onion/images/2020/08/11/science/11VACCINE-SAFETY2/11VACCINE-SAFETY2-articleLarge.jpg?quality=75\&auto=webp\&disable=upscale}

But no matter how promising these early results, Phase 3 trials can
fail.

The timing of Russia's announcement makes it ``very unlikely that they
have sufficient data about the efficacy of the product,'' said Natalie
Dean, a biostatistician and infectious disease expert at the University
of Florida who has
\href{https://www.nytimes3xbfgragh.onion/2020/08/03/opinion/sunday/coronavirus-vaccine-efficacy-trials.html}{warned
against rushing the vaccine-approval process}. Dr. Dean noted that even
vaccines that have produced promising data from early trials in humans
have \href{https://www.fda.gov/media/102332/download}{flopped at later
stages}.

In a large, randomized control trial, researchers give the vaccine or a
placebo to tens of thousands of people, and wait for them to encounter
the virus in the real world.

\hypertarget{latest-updates-the-coronavirus-outbreak}{%
\section{\texorpdfstring{\href{https://www.nytimes3xbfgragh.onion/2020/09/04/world/covid-19-coronavirus.html?action=click\&pgtype=Article\&state=default\&region=MAIN_CONTENT_1\&context=storylines_live_updates}{Latest
Updates: The Coronavirus
Outbreak}}{Latest Updates: The Coronavirus Outbreak}}\label{latest-updates-the-coronavirus-outbreak}}

Updated 2020-09-05T12:05:40.998Z

\begin{itemize}
\tightlist
\item
  \href{https://www.nytimes3xbfgragh.onion/2020/09/04/world/covid-19-coronavirus.html?action=click\&pgtype=Article\&state=default\&region=MAIN_CONTENT_1\&context=storylines_live_updates\#link-1654f6ad}{Research
  connects vaping to a higher chance of catching the virus --- and
  suffering its worst effects.}
\item
  \href{https://www.nytimes3xbfgragh.onion/2020/09/04/world/covid-19-coronavirus.html?action=click\&pgtype=Article\&state=default\&region=MAIN_CONTENT_1\&context=storylines_live_updates\#link-52e4198a}{Another
  college football game won't be played as planned.}
\item
  \href{https://www.nytimes3xbfgragh.onion/2020/09/04/world/covid-19-coronavirus.html?action=click\&pgtype=Article\&state=default\&region=MAIN_CONTENT_1\&context=storylines_live_updates\#link-181cef0}{Pharmaceutical
  companies plan a joint pledge on safety standards as they move
  vaccines to the marketplace.}
\end{itemize}

\href{https://www.nytimes3xbfgragh.onion/2020/09/04/world/covid-19-coronavirus.html?action=click\&pgtype=Article\&state=default\&region=MAIN_CONTENT_1\&context=storylines_live_updates}{See
more updates}

More live coverage:
\href{https://www.nytimes3xbfgragh.onion/live/2020/09/04/business/stock-market-today-coronavirus?action=click\&pgtype=Article\&state=default\&region=MAIN_CONTENT_1\&context=storylines_live_updates}{Markets}

``Then you wait to see, do they get sick or not. Do they die or not?''
said Dr. Steven Black, a vaccine expert with the Task Force for Global
Health. If a vaccine is effective, fewer vaccinated volunteers will get
sick than the ones who received the placebo.

The Russian researchers have not yet begun that crucial test.

In June, the Gamaleya Research Institute of Epidemiology and
Microbiology at the Health Ministry of the Russian Federation
\href{https://clinicaltrials.gov/ct2/show/NCT04437875?term=Gamaleya\&draw=2\&rank=2}{registered}
a combined Phase 1 and 2 trial on a vaccine called Gam-COVID-Vac Lyo.
The researchers planned to test it on 38 volunteers.

They said that the vaccine was made from an adenovirus --- a harmless
cold virus --- carrying a coronavirus gene, similar to what AstraZeneca
and Johnson \& Johnson are using in their vaccines. The technology is
still relatively new: The first adenovirus vaccine for any disease was
approved for Ebola in June.

Since then, Russian officials have claimed that they would be moving the
vaccine quickly into manufacturing. Mr. Putin's announcement on Tuesday
made it official. Yet the institute has never published its Phase 1 and
2 trial data.

At Mr. Putin's announcement, Russia's Minister of Health, Mikhail
Murashko, \href{https://www.interfax.ru/russia/721241}{declared} that
``all the volunteers developed high titers of antibodies to COVID-19. At
the same time, none of them had serious complications of immunization.''

That is the sort of result you'd expect from a Phase 1 trial. It doesn't
tell you if the vaccine actually works.

``This is all beyond stupid,'' said John Moore, a virologist at Weill
Cornell Medical College in New York City. ``Putin doesn't have a
vaccine, he's just making a political statement.''

Image

President Vladimir Putin of Russia meeting with the minister of health,
Mikhail Murashko, in January.Credit...Sputnik, via Reuters

On Tuesday, the Russian institute put up a
\href{https://sputnikvaccine.com/about-vaccine/clinical-trials/}{website}
claiming that a Phase 3 trial would begin the next day involving more
than 2,000 people in Russia as well as the United Arab Emirates, Saudi
Arabia, Brazil and Mexico.

All other Phase 3 trials of coronavirus vaccines currently underway are
more than ten times larger than that, with 30,000 volunteers apiece.

Dr. Nicole Lurie, a former assistant secretary for preparedness and
response at the U.S. Department of Health and Human Services and
currently an adviser at the Coalition for Epidemic Preparedness
Innovations, said the lesson that the U.S. government should draw from
Mr. Putin's announcement is clear.

``This is exactly the situation that Americans expect our government to
avoid,'' she said.

\hypertarget{a-faster-process}{%
\subsection{A faster process}\label{a-faster-process}}

Along with determining whether the vaccine protects people, Phase 3
trials can reveal uncommon side effects that may not have shown up in
the comparatively small number of volunteers who enrolled in the earlier
phases.

\href{https://www.nytimes3xbfgragh.onion/news-event/coronavirus?action=click\&pgtype=Article\&state=default\&region=MAIN_CONTENT_3\&context=storylines_faq}{}

\hypertarget{the-coronavirus-outbreak-}{%
\subsubsection{The Coronavirus Outbreak
›}\label{the-coronavirus-outbreak-}}

\hypertarget{frequently-asked-questions}{%
\paragraph{Frequently Asked
Questions}\label{frequently-asked-questions}}

Updated September 4, 2020

\begin{itemize}
\item ~
  \hypertarget{what-are-the-symptoms-of-coronavirus}{%
  \paragraph{What are the symptoms of
  coronavirus?}\label{what-are-the-symptoms-of-coronavirus}}

  \begin{itemize}
  \tightlist
  \item
    In the beginning, the coronavirus
    \href{https://www.nytimes3xbfgragh.onion/article/coronavirus-facts-history.html?action=click\&pgtype=Article\&state=default\&region=MAIN_CONTENT_3\&context=storylines_faq\#link-6817bab5}{seemed
    like it was primarily a respiratory illness}~--- many patients had
    fever and chills, were weak and tired, and coughed a lot, though
    some people don't show many symptoms at all. Those who seemed
    sickest had pneumonia or acute respiratory distress syndrome and
    received supplemental oxygen. By now, doctors have identified many
    more symptoms and syndromes. In April,
    \href{https://www.nytimes3xbfgragh.onion/2020/04/27/health/coronavirus-symptoms-cdc.html?action=click\&pgtype=Article\&state=default\&region=MAIN_CONTENT_3\&context=storylines_faq}{the
    C.D.C. added to the list of early signs}~sore throat, fever, chills
    and muscle aches. Gastrointestinal upset, such as diarrhea and
    nausea, has also been observed. Another telltale sign of infection
    may be a sudden, profound diminution of one's
    \href{https://www.nytimes3xbfgragh.onion/2020/03/22/health/coronavirus-symptoms-smell-taste.html?action=click\&pgtype=Article\&state=default\&region=MAIN_CONTENT_3\&context=storylines_faq}{sense
    of smell and taste.}~Teenagers and young adults in some cases have
    developed painful red and purple lesions on their fingers and toes
    --- nicknamed ``Covid toe'' --- but few other serious symptoms.
  \end{itemize}
\item ~
  \hypertarget{why-is-it-safer-to-spend-time-together-outside}{%
  \paragraph{Why is it safer to spend time together
  outside?}\label{why-is-it-safer-to-spend-time-together-outside}}

  \begin{itemize}
  \tightlist
  \item
    \href{https://www.nytimes3xbfgragh.onion/2020/05/15/us/coronavirus-what-to-do-outside.html?action=click\&pgtype=Article\&state=default\&region=MAIN_CONTENT_3\&context=storylines_faq}{Outdoor
    gatherings}~lower risk because wind disperses viral droplets, and
    sunlight can kill some of the virus. Open spaces prevent the virus
    from building up in concentrated amounts and being inhaled, which
    can happen when infected people exhale in a confined space for long
    stretches of time, said Dr. Julian W. Tang, a virologist at the
    University of Leicester.
  \end{itemize}
\item ~
  \hypertarget{why-does-standing-six-feet-away-from-others-help}{%
  \paragraph{Why does standing six feet away from others
  help?}\label{why-does-standing-six-feet-away-from-others-help}}

  \begin{itemize}
  \tightlist
  \item
    The coronavirus spreads primarily through droplets from your mouth
    and nose, especially when you cough or sneeze. The C.D.C., one of
    the organizations using that measure,
    \href{https://www.nytimes3xbfgragh.onion/2020/04/14/health/coronavirus-six-feet.html?action=click\&pgtype=Article\&state=default\&region=MAIN_CONTENT_3\&context=storylines_faq}{bases
    its recommendation of six feet}~on the idea that most large droplets
    that people expel when they cough or sneeze will fall to the ground
    within six feet. But six feet has never been a magic number that
    guarantees complete protection. Sneezes, for instance, can launch
    droplets a lot farther than six feet,
    \href{https://jamanetwork.com/journals/jama/fullarticle/2763852}{according
    to a recent study}. It's a rule of thumb: You should be safest
    standing six feet apart outside, especially when it's windy. But
    keep a mask on at all times, even when you think you're far enough
    apart.
  \end{itemize}
\item ~
  \hypertarget{i-have-antibodies-am-i-now-immune}{%
  \paragraph{I have antibodies. Am I now
  immune?}\label{i-have-antibodies-am-i-now-immune}}

  \begin{itemize}
  \tightlist
  \item
    As of right
    now,\href{https://www.nytimes3xbfgragh.onion/2020/07/22/health/covid-antibodies-herd-immunity.html?action=click\&pgtype=Article\&state=default\&region=MAIN_CONTENT_3\&context=storylines_faq}{~that
    seems likely, for at least several months.}~There have been
    frightening accounts of people suffering what seems to be a second
    bout of Covid-19. But experts say these patients may have a
    drawn-out course of infection, with the virus taking a slow toll
    weeks to months after initial exposure.~People infected with the
    coronavirus typically
    \href{https://www.nature.com/articles/s41586-020-2456-9}{produce}~immune
    molecules called antibodies, which are
    \href{https://www.nytimes3xbfgragh.onion/2020/05/07/health/coronavirus-antibody-prevalence.html?action=click\&pgtype=Article\&state=default\&region=MAIN_CONTENT_3\&context=storylines_faq}{protective
    proteins made in response to an
    infection}\href{https://www.nytimes3xbfgragh.onion/2020/05/07/health/coronavirus-antibody-prevalence.html?action=click\&pgtype=Article\&state=default\&region=MAIN_CONTENT_3\&context=storylines_faq}{.
    These antibodies may}~last in the body
    \href{https://www.nature.com/articles/s41591-020-0965-6}{only two to
    three months}, which may seem worrisome, but that's~perfectly normal
    after an acute infection subsides, said Dr. Michael Mina, an
    immunologist at Harvard University. It may be possible to get the
    coronavirus again, but it's highly unlikely that it would be
    possible in a short window of time from initial infection or make
    people sicker the second time.
  \end{itemize}
\item ~
  \hypertarget{what-are-my-rights-if-i-am-worried-about-going-back-to-work}{%
  \paragraph{What are my rights if I am worried about going back to
  work?}\label{what-are-my-rights-if-i-am-worried-about-going-back-to-work}}

  \begin{itemize}
  \tightlist
  \item
    Employers have to provide
    \href{https://www.osha.gov/SLTC/covid-19/standards.html}{a safe
    workplace}~with policies that protect everyone equally.
    \href{https://www.nytimes3xbfgragh.onion/article/coronavirus-money-unemployment.html?action=click\&pgtype=Article\&state=default\&region=MAIN_CONTENT_3\&context=storylines_faq}{And
    if one of your co-workers tests positive for the coronavirus, the
    C.D.C.}~has said that
    \href{https://www.cdc.gov/coronavirus/2019-ncov/community/guidance-business-response.html}{employers
    should tell their employees}~-\/- without giving you the sick
    employee's name -\/- that they may have been exposed to the virus.
  \end{itemize}
\end{itemize}

Just because someone gets sick or dies after getting a vaccine, however,
doesn't necessarily show that the vaccine was the culprit. By comparing
large groups of people who received the vaccine versus the placebo,
researchers can identify unusual clusters of cases in the vaccinated
participants.

Along the way, vaccine developers share these results in reports to
government regulators and in peer-reviewed papers for scientific
journals. Outside experts then evaluate the data from Phase 3 trials and
give their recommendation to the F.D.A., which then decides whether to
approve a vaccine for widespread use.

``It's not enough for me to say I have a great product,'' said Dr.
Salmon. ``Before you use it, you need other people to really look at the
data and be convinced that the benefits outweigh the risks.''

And even after a vaccine is licensed, researchers still keep an eye on
it to make sure it's safe. As millions of people get a vaccine, even
rarer side effects may emerge over time. It's also possible that certain
groups of people, such as children or the elderly, turn out to face
risks from a vaccine that weren't immediately clear from the Phase 3
trials.

Regulators can then make adjustments to the vaccine --- changing the
dose, for example --- to make it safer.

In July, a team of researchers at Tel Aviv University reviewed licensed
vaccines in the United States over the past 20 years and
\href{https://doi.org/10.7326/M20-2726}{concluded} they were ``safe,
with no important post-approval safety issues.''

Image

Elena Smolyarchuk, left, a chief researcher at Sechenov University, with
volunteers in the vaccine study in Moscow last month.Credit...Yuri
Kochetkov/EPA, via Shutterstock

Putting in safeguards slows the development of vaccines. In recent
years, new outbreaks such as Ebola, SARS and pandemic flu strains have
spurred vaccine makers to look for ways to speed the process without
sacrificing safety.

Now, in the midst of the Covid-19 pandemic, they're putting those ideas
into practice.

One way to safely accelerate vaccine trials is for regulators to prepare
in advance to analyze each batch of data, so that they can cut down the
time between trials. Vaccine manufacturers have already been
demonstrating to regulators that they can make coronavirus vaccines
safely on an industrial scale, long before the vaccines themselves have
made it through clinical trials.

But researchers are still figuring out how SARS-CoV-2, the name of the
virus that causes Covid-19, makes us sick and evades the immune system.

Adding to the complexity, vaccine makers are testing out just about
every technology they can for a Covid-19 vaccine. Some of the
experimental vaccines are based on old designs, but others have never
been approved for use in humans for any disease.

Dr. Black and his colleagues have been working with CEPI, a nonprofit
organization that is accelerating the development of vaccines, on a new
set of safety procedures for some Covid-19 vaccines, including those
developed by AstraZeneca, CureVac and Novavax.

The researchers have come up with a set of potential medical
complications that vaccine trials should pay particular attention to.
They have addressed the possibility that the vaccine could actually make
people prone to worse cases of Covid-19, for example. Fortunately, the
research so far shows no sign that this is happening.

CEPI is coordinating the sharing of data among vaccine developers. By
pooling the safety data from different vaccine developers, Dr. Black
said, CEPI will be able to detect rare side effects that they might not
have even considered as possible risks.

\emph{Andrew Kramer and Katherine J. Wu contributed reporting.}

\href{https://www.nytimes3xbfgragh.onion/interactive/2020/science/coronavirus-vaccine-tracker.html}{}

\includegraphics{https://static01.graylady3jvrrxbe.onion/images/2020/09/03/us/coronavirus-vaccine-tracker-promo-1599144009786/coronavirus-vaccine-tracker-promo-1599144009786-articleLarge.png}

\hypertarget{coronavirus-vaccine-tracker}{%
\subsection{Coronavirus Vaccine
Tracker}\label{coronavirus-vaccine-tracker}}

A look at all the vaccines that have reached trials in humans.

Advertisement

\protect\hyperlink{after-bottom}{Continue reading the main story}

\hypertarget{site-index}{%
\subsection{Site Index}\label{site-index}}

\hypertarget{site-information-navigation}{%
\subsection{Site Information
Navigation}\label{site-information-navigation}}

\begin{itemize}
\tightlist
\item
  \href{https://help.nytimes3xbfgragh.onion/hc/en-us/articles/115014792127-Copyright-notice}{©~2020~The
  New York Times Company}
\end{itemize}

\begin{itemize}
\tightlist
\item
  \href{https://www.nytco.com/}{NYTCo}
\item
  \href{https://help.nytimes3xbfgragh.onion/hc/en-us/articles/115015385887-Contact-Us}{Contact
  Us}
\item
  \href{https://www.nytco.com/careers/}{Work with us}
\item
  \href{https://nytmediakit.com/}{Advertise}
\item
  \href{http://www.tbrandstudio.com/}{T Brand Studio}
\item
  \href{https://www.nytimes3xbfgragh.onion/privacy/cookie-policy\#how-do-i-manage-trackers}{Your
  Ad Choices}
\item
  \href{https://www.nytimes3xbfgragh.onion/privacy}{Privacy}
\item
  \href{https://help.nytimes3xbfgragh.onion/hc/en-us/articles/115014893428-Terms-of-service}{Terms
  of Service}
\item
  \href{https://help.nytimes3xbfgragh.onion/hc/en-us/articles/115014893968-Terms-of-sale}{Terms
  of Sale}
\item
  \href{https://spiderbites.nytimes3xbfgragh.onion}{Site Map}
\item
  \href{https://help.nytimes3xbfgragh.onion/hc/en-us}{Help}
\item
  \href{https://www.nytimes3xbfgragh.onion/subscription?campaignId=37WXW}{Subscriptions}
\end{itemize}
