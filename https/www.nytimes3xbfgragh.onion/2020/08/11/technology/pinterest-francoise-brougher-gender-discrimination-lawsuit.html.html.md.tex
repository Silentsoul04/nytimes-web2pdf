Sections

SEARCH

\protect\hyperlink{site-content}{Skip to
content}\protect\hyperlink{site-index}{Skip to site index}

\href{https://www.nytimes3xbfgragh.onion/section/technology}{Technology}

\href{https://myaccount.nytimes3xbfgragh.onion/auth/login?response_type=cookie\&client_id=vi}{}

\href{https://www.nytimes3xbfgragh.onion/section/todayspaper}{Today's
Paper}

\href{/section/technology}{Technology}\textbar{}Pinterest Accused of
Gender Bias in Suit by Former No. 2 Executive

\url{https://nyti.ms/3akXwGb}

\begin{itemize}
\item
\item
\item
\item
\item
\end{itemize}

Advertisement

\protect\hyperlink{after-top}{Continue reading the main story}

Supported by

\protect\hyperlink{after-sponsor}{Continue reading the main story}

\hypertarget{pinterest-accused-of-gender-bias-in-suit-by-former-no-2-executive}{%
\section{Pinterest Accused of Gender Bias in Suit by Former No. 2
Executive}\label{pinterest-accused-of-gender-bias-in-suit-by-former-no-2-executive}}

Françoise Brougher, Pinterest's former chief operating officer, said she
was fired after speaking up about mistreatment.

\includegraphics{https://static01.graylady3jvrrxbe.onion/images/2020/08/11/business/00pinterest1/merlin_175585533_2c158e4b-3d75-48ec-9902-c050ed507b52-articleLarge.jpg?quality=75\&auto=webp\&disable=upscale}

\href{https://www.nytimes3xbfgragh.onion/by/erin-griffith}{\includegraphics{https://static01.graylady3jvrrxbe.onion/images/2019/06/18/reader-center/author-erin-griffith/author-erin-griffith-thumbLarge.png}}

By \href{https://www.nytimes3xbfgragh.onion/by/erin-griffith}{Erin
Griffith}

\begin{itemize}
\item
  Aug. 11, 2020
\item
  \begin{itemize}
  \item
  \item
  \item
  \item
  \item
  \end{itemize}
\end{itemize}

SAN FRANCISCO --- In April, Françoise Brougher, the chief operating
officer of Pinterest and its top female executive, abruptly left the
company with little explanation.

In a lawsuit filed on Tuesday, Ms. Brougher accused the \$21 billion
company, which makes virtual pinboards, of firing her after she
complained about sexist treatment. In her suit, which was filed in San
Francisco Superior Court, Ms. Brougher said she had been left out of
important meetings, was given gendered feedback, was paid less than her
male peers when she joined the company, and ultimately was let go for
speaking up about it.

``Gender discrimination at the C-level suite may be a little more
subtle, but it's very insidious and real,'' Ms. Brougher, 54, said in an
interview. ``When men speak out, they get rewarded. When women speak
out, they get fired.''

Pinterest was reviewing the lawsuit, a company spokeswoman said. ``Our
employees are incredibly important to us,'' she said, adding that the
company was committed to advancing its culture so ``all of our employees
feel included and supported.'' Pinterest is conducting an independent
review regarding its culture, policies and practices, she added.

Ms. Brougher is one of the most prominent female tech executives to file
a gender discrimination suit against her onetime employer since the
venture capitalist
\href{https://www.nytimes3xbfgragh.onion/2015/02/23/technology/ellen-pao-suit-against-kleiner-perkins-heads-to-trial-with-big-potential-implications.html}{Ellen
Pao sued her firm}, Kleiner Perkins Caufield \& Byers, in 2012. The new
lawsuit suggests that bias against women in Silicon Valley has
persisted, even after tech's
\href{https://www.nytimes3xbfgragh.onion/2017/06/30/technology/women-entrepreneurs-speak-out-sexual-harassment.html}{culture
of sexual harassment} of female executives and entrepreneurs became part
of the \#MeToo movement.

Ms. Brougher's lawsuit follows a gender discrimination lawsuit last
month against Carta, a financial technology start-up, by its former vice
president for marketing, Emily Kramer. Ms. Kramer accused Carta of
paying her less than her male peers and said the company retaliated
against her for speaking up about gender equality and diversity.

A Carta spokeswoman said, ``Gender inequality in the workplace is a real
and systemic problem, particularly in Silicon Valley, however, the
allegations in this case are unfounded.''

\includegraphics{https://static01.graylady3jvrrxbe.onion/images/2020/08/11/business/00pinterest3/merlin_153683982_f9e74336-15fe-438b-913c-4459dea3d310-articleLarge.jpg?quality=75\&auto=webp\&disable=upscale}

Ms. Brougher's suit adds to the scrutiny of Pinterest, which has a large
audience of female users. In recent months, the company, based in San
Francisco, has also been criticized by some of its former Black
employees over racial discrimination. In June, two of them,
\href{https://twitter.com/IfeomaOzoma/status/1272546213322080258}{Ifeoma
Ozoma} and
\href{https://twitter.com/erikashimizu/status/1272547227177713664}{Aerica
Shimizu Banks}, tweeted about racist and sexist comments, pay inequities
and retaliation they experienced at the company. They quit in May.

Ms. Brougher is well known in Silicon Valley. She previously led the
business side of the financial technology company Square and worked in a
variety of positions on Google's advertising business. She joined
Pinterest in 2018 as chief operating officer and was responsible for the
company's revenue, with roughly half of the 2,000 employees reporting to
her.

When Pinterest
\href{https://www.nytimes3xbfgragh.onion/2019/03/22/technology/pinterest-ipo.html}{filed
to go public} in 2019, Ms. Brougher learned that she was paid less than
her male peers and that her equity grants were ``backloaded,'' meaning
most of them vested after several years, while her executive male peers'
grants were not, according to the lawsuit. After complaining, her
compensation was adjusted.

Ms. Brougher said she was not invited on the ``road show'' to talk to
investors for
\href{https://www.nytimes3xbfgragh.onion/2019/04/18/technology/pinterest-stock.html}{Pinterest's
initial public offering}. She was also not invited to board meetings
after the company went public, though members of her team were sometimes
invited to those meetings without her knowledge, the lawsuit said. (She
was not a member of the board.)

Ms. Brougher described a culture of ``constant exclusion,'' where
decisions were frequently made in unofficial capacities, or ``the
meeting after the meeting.''

``When you are brought in as a No. 2, you are expected to advise the
C.E.O.,'' she said. ``But when you are not in the meeting where the
decisions are made and don't have the context, it makes your job
harder.''

Image

Ben Silbermann, chief executive of Pinterest.Credit...Anastasiia Sapon
for The New York Times

Ms. Brougher said Pinterest's chief financial officer, Todd Morgenfeld,
asked her at one point, ``What is your job anyway?'' in front of peers,
according to the lawsuit. Mr. Morgenfeld also offered Ms. Brougher
formal feedback that she viewed as sexist, according to the lawsuit.
When she confronted him about it on a video call, he raised his voice
and hung up on her, the suit said.

Ben Silbermann, Pinterest's chief executive, was dismissive of Ms.
Brougher's concerns about Mr. Morgenfeld, comparing it to a domestic
dispute, according to the suit. Human resources treated the complaint as
a legal matter, the suit said.

In April, soon after the heated conversation with Mr. Morgenfeld, Ms.
Brougher was terminated, according to the suit.

``I was told I wasn't collaborating enough,'' she said. Pinterest asked
her to announce that leaving was her decision and she declined, she
said.

Ms. Brougher's law firm, Rudy, Exelrod, Zieff \& Lowe, also represented
Ms. Pao.

Advertisement

\protect\hyperlink{after-bottom}{Continue reading the main story}

\hypertarget{site-index}{%
\subsection{Site Index}\label{site-index}}

\hypertarget{site-information-navigation}{%
\subsection{Site Information
Navigation}\label{site-information-navigation}}

\begin{itemize}
\tightlist
\item
  \href{https://help.nytimes3xbfgragh.onion/hc/en-us/articles/115014792127-Copyright-notice}{©~2020~The
  New York Times Company}
\end{itemize}

\begin{itemize}
\tightlist
\item
  \href{https://www.nytco.com/}{NYTCo}
\item
  \href{https://help.nytimes3xbfgragh.onion/hc/en-us/articles/115015385887-Contact-Us}{Contact
  Us}
\item
  \href{https://www.nytco.com/careers/}{Work with us}
\item
  \href{https://nytmediakit.com/}{Advertise}
\item
  \href{http://www.tbrandstudio.com/}{T Brand Studio}
\item
  \href{https://www.nytimes3xbfgragh.onion/privacy/cookie-policy\#how-do-i-manage-trackers}{Your
  Ad Choices}
\item
  \href{https://www.nytimes3xbfgragh.onion/privacy}{Privacy}
\item
  \href{https://help.nytimes3xbfgragh.onion/hc/en-us/articles/115014893428-Terms-of-service}{Terms
  of Service}
\item
  \href{https://help.nytimes3xbfgragh.onion/hc/en-us/articles/115014893968-Terms-of-sale}{Terms
  of Sale}
\item
  \href{https://spiderbites.nytimes3xbfgragh.onion}{Site Map}
\item
  \href{https://help.nytimes3xbfgragh.onion/hc/en-us}{Help}
\item
  \href{https://www.nytimes3xbfgragh.onion/subscription?campaignId=37WXW}{Subscriptions}
\end{itemize}
