Sections

SEARCH

\protect\hyperlink{site-content}{Skip to
content}\protect\hyperlink{site-index}{Skip to site index}

\href{https://www.nytimes3xbfgragh.onion/section/science}{Science}

\href{https://myaccount.nytimes3xbfgragh.onion/auth/login?response_type=cookie\&client_id=vi}{}

\href{https://www.nytimes3xbfgragh.onion/section/todayspaper}{Today's
Paper}

\href{/section/science}{Science}\textbar{}Making Sense of `One of the
Most Baffling Animals That Ever Lived'

\url{https://nyti.ms/2XW93Xs}

\begin{itemize}
\item
\item
\item
\item
\item
\item
\end{itemize}

Advertisement

\protect\hyperlink{after-top}{Continue reading the main story}

Supported by

\protect\hyperlink{after-sponsor}{Continue reading the main story}

Trilobites

\hypertarget{making-sense-of-one-of-the-most-baffling-animals-that-ever-lived}{%
\section{Making Sense of `One of the Most Baffling Animals That Ever
Lived'}\label{making-sense-of-one-of-the-most-baffling-animals-that-ever-lived}}

Important mysteries have been solved about a reptile with a giraffe-like
neck that hunted prey 242 million years ago.

\includegraphics{https://static01.graylady3jvrrxbe.onion/images/2020/08/12/science/12TB-REPTILE1/12TB-REPTILE1-articleLarge.jpg?quality=75\&auto=webp\&disable=upscale}

By Asher Elbein

\begin{itemize}
\item
  Published Aug. 12, 2020Updated Aug. 14, 2020
\item
  \begin{itemize}
  \item
  \item
  \item
  \item
  \item
  \item
  \end{itemize}
\end{itemize}

Nearly 250 million years ago, a very odd reptile patrolled the
shorelines and coves of the Triassic Alps. Called Tanystropheus, it had
a toothy head and a body echoing that of modern monitor lizards. But
between them stretched a horizontal, giraffe-like neck.

The question of how this 20-foot creature used that nine-foot neck has
bedeviled paleontologists for over 100 years, and it is seen as ``one of
the most baffling animals that ever lived,'' said Stephan Spiekman, a
paleontologist at the University of Zurich, in Switzerland*.* ``How
could this animal even breathe or swallow? And then there is the
evolutionary question: Why on earth did this animal evolve this
ridiculously long neck?''

But
\href{https://www.cell.com/current-biology/pdfExtended/S0960-9822(20)31017-4}{research
published last week in Current Biology}, including a new reconstruction
of its skull, shows evidence that its body was primed for an aquatic
hunting strategy and that the creature came in two varieties: regular
and miniature-size.

Tanystropheus was initially described in the 1850s, based on a few
tubelike bones. Only in the 1930s, when more complete fossils emerged
from the Monte San Giorgio in Switzerland, did scientists realize they
were looking at neck vertebrae from a strange reptile whose way of life
they couldn't figure out.

\includegraphics{https://static01.graylady3jvrrxbe.onion/images/2020/08/12/science/12TB-REPTILE3/merlin_175376196_7e06b35c-7084-4c99-8fa2-db7e2d390550-articleLarge.jpg?quality=75\&auto=webp\&disable=upscale}

It took decades until the
\href{https://www.palass.org/publications/palaeontology-journal/archive/31/4/article_pp997-1011}{paleontologist
Karl Tschanz}showed in 1988 that ribs underneath the neck vertebrae
interlocked, forming a horizontal and extremely stiff neck. That
suggested an aquatic lifestyle, Mr. Spiekman said, because such an
unbending neck would have made life on land inconvenient. But
paleontologists continued to argue whether Tanystropheus ** actively
pursued underwater prey or perched onshore, using its long neck like a
fishing pole.

To make matters more confounding, digs had found multiple skeletons of
smaller Tanystropheus ** on Monte San Giorgio. If they belonged to
juveniles, as some suggested, why did they have different teeth?

Mr. Spiekman's team sought answers first by CT scanning a specimen of
Tanystropheus' ** head from a Zurich museum, and reconstructing it,
which proved difficult because ``all the bones were jumbled together,
and because the skull of Tanystropheus ** is very different from other
reptiles in many respects.''

``I very clearly remember the day the model was finished and I was the
first to see the face of this animal after 242 million years,'' he said.

The reconstructed skull revealed several aquatic adaptations: nostrils
positioned on the top of the snout, like a crocodile, and long, curved
fangs. Instead of pursuing prey actively, Mr. Spiekman said, it probably
ambushed them in murky water, lunging forward with its long neck to snap
up fish.

Image

Smaller specimens of~Tanystropheus turned out to be a separate species,
not juveniles of the larger specimens.Credit...Beat Scheffold, PIMUZ,
UZH

To test whether the bones of the smaller Tanystropheus belonged to
juveniles or a separate species, the team studied thin sections of bone
prepared by Mr. Spiekman's supervisor and co-author, Torsten Scheyer. A
close look at the little bones' interior revealed clear signs of a fully
grown adult. That meant that two distinct species of Tanystropheus **
were ** coexisting in the same waters: one large, one mini.

The two closely related animals seem to have gone after different types
of prey, the team reports, in an example of the phenomenon known as
niche partitioning. The larger animal --- newly named Tanystropheus
hydroides --- used its spiked teeth for hunting fish and squid; the
smaller species' teeth point toward a diet of marine invertebrates such
as shrimp.

With two mysteries solved, Mr. Spiekman and his team hope to take a
fresh look at the biomechanics of the jaws, and that long, strange neck.

``People always thought that Tanystropheus was an evolutionary dead
end,'' he said. ``But the fact that Tanystropheus evolved into different
species with very different lifestyles indicates that Tanystropheus and
its neck were quite successful in evolutionary terms.''

Advertisement

\protect\hyperlink{after-bottom}{Continue reading the main story}

\hypertarget{site-index}{%
\subsection{Site Index}\label{site-index}}

\hypertarget{site-information-navigation}{%
\subsection{Site Information
Navigation}\label{site-information-navigation}}

\begin{itemize}
\tightlist
\item
  \href{https://help.nytimes3xbfgragh.onion/hc/en-us/articles/115014792127-Copyright-notice}{©~2020~The
  New York Times Company}
\end{itemize}

\begin{itemize}
\tightlist
\item
  \href{https://www.nytco.com/}{NYTCo}
\item
  \href{https://help.nytimes3xbfgragh.onion/hc/en-us/articles/115015385887-Contact-Us}{Contact
  Us}
\item
  \href{https://www.nytco.com/careers/}{Work with us}
\item
  \href{https://nytmediakit.com/}{Advertise}
\item
  \href{http://www.tbrandstudio.com/}{T Brand Studio}
\item
  \href{https://www.nytimes3xbfgragh.onion/privacy/cookie-policy\#how-do-i-manage-trackers}{Your
  Ad Choices}
\item
  \href{https://www.nytimes3xbfgragh.onion/privacy}{Privacy}
\item
  \href{https://help.nytimes3xbfgragh.onion/hc/en-us/articles/115014893428-Terms-of-service}{Terms
  of Service}
\item
  \href{https://help.nytimes3xbfgragh.onion/hc/en-us/articles/115014893968-Terms-of-sale}{Terms
  of Sale}
\item
  \href{https://spiderbites.nytimes3xbfgragh.onion}{Site Map}
\item
  \href{https://help.nytimes3xbfgragh.onion/hc/en-us}{Help}
\item
  \href{https://www.nytimes3xbfgragh.onion/subscription?campaignId=37WXW}{Subscriptions}
\end{itemize}
