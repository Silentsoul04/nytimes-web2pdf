Sections

SEARCH

\protect\hyperlink{site-content}{Skip to
content}\protect\hyperlink{site-index}{Skip to site index}

\href{https://www.nytimes3xbfgragh.onion/section/health}{Health}

\href{https://myaccount.nytimes3xbfgragh.onion/auth/login?response_type=cookie\&client_id=vi}{}

\href{https://www.nytimes3xbfgragh.onion/section/todayspaper}{Today's
Paper}

\href{/section/health}{Health}\textbar{}You Probably Won't Catch the
Coronavirus From Frozen Food

\url{https://nyti.ms/30RSVIs}

\begin{itemize}
\item
\item
\item
\item
\item
\end{itemize}

\hypertarget{the-coronavirus-outbreak}{%
\subsubsection{\texorpdfstring{\href{https://www.nytimes3xbfgragh.onion/news-event/coronavirus?name=styln-coronavirus-national\&region=TOP_BANNER\&variant=undefined\&block=storyline_menu_recirc\&action=click\&pgtype=Article\&impression_id=ba5ce1b0-e38b-11ea-a3fc-61e873d18389}{The
Coronavirus
Outbreak}}{The Coronavirus Outbreak}}\label{the-coronavirus-outbreak}}

\begin{itemize}
\tightlist
\item
  live\href{https://www.nytimes3xbfgragh.onion/2020/08/20/world/coronavirus-covid.html?name=styln-coronavirus-national\&region=TOP_BANNER\&variant=undefined\&block=storyline_menu_recirc\&action=click\&pgtype=Article\&impression_id=ba5ce1b1-e38b-11ea-a3fc-61e873d18389}{Latest
  Updates}
\item
  \href{https://www.nytimes3xbfgragh.onion/interactive/2020/us/coronavirus-us-cases.html?name=styln-coronavirus-national\&region=TOP_BANNER\&variant=undefined\&block=storyline_menu_recirc\&action=click\&pgtype=Article\&impression_id=ba5d08c0-e38b-11ea-a3fc-61e873d18389}{Maps
  and Cases}
\item
  \href{https://www.nytimes3xbfgragh.onion/interactive/2020/science/coronavirus-vaccine-tracker.html?name=styln-coronavirus-national\&region=TOP_BANNER\&variant=undefined\&block=storyline_menu_recirc\&action=click\&pgtype=Article\&impression_id=ba5d08c1-e38b-11ea-a3fc-61e873d18389}{Vaccine
  Tracker}
\item
  \href{https://www.nytimes3xbfgragh.onion/2020/08/19/us/colleges-closing-covid.html?name=styln-coronavirus-national\&region=TOP_BANNER\&variant=undefined\&block=storyline_menu_recirc\&action=click\&pgtype=Article\&impression_id=ba5d08c2-e38b-11ea-a3fc-61e873d18389}{Colleges
  Closing}
\item
  \href{https://www.nytimes3xbfgragh.onion/live/2020/08/20/business/stock-market-today-coronavirus?name=styln-coronavirus-national\&region=TOP_BANNER\&variant=undefined\&block=storyline_menu_recirc\&action=click\&pgtype=Article\&impression_id=ba5d08c3-e38b-11ea-a3fc-61e873d18389}{Economy}
\end{itemize}

Advertisement

\protect\hyperlink{after-top}{Continue reading the main story}

Supported by

\protect\hyperlink{after-sponsor}{Continue reading the main story}

\hypertarget{you-probably-wont-catch-the-coronavirus-from-frozen-food}{%
\section{You Probably Won't Catch the Coronavirus From Frozen
Food}\label{you-probably-wont-catch-the-coronavirus-from-frozen-food}}

Reports that the virus was detected in a trans-continental shipment of
frozen chicken wings sparked concerns online. But experts aren't
worried.

\includegraphics{https://static01.graylady3jvrrxbe.onion/images/2020/08/13/science/13VIRUS-FROZEN/merlin_175668282_af6b4103-e39a-4c00-b900-e03a2722b552-articleLarge.jpg?quality=75\&auto=webp\&disable=upscale}

\href{https://www.nytimes3xbfgragh.onion/by/katherine-j--wu}{\includegraphics{https://static01.graylady3jvrrxbe.onion/images/2020/08/11/reader-center/author-katherine-j-wu/author-katherine-j-wu-thumbLarge.png}}

By
\href{https://www.nytimes3xbfgragh.onion/by/katherine-j--wu}{Katherine
J. Wu}

\begin{itemize}
\item
  Aug. 13, 2020
\item
  \begin{itemize}
  \item
  \item
  \item
  \item
  \item
  \end{itemize}
\end{itemize}

Amid a
\href{https://twitter.com/business/status/1293767758107336704}{flurry of
concern} over reports that frozen chicken wings imported to China from
Brazil had tested positive for the coronavirus, experts said on Thursday
that the likelihood of catching the virus from food --- especially
frozen, packaged food --- is exceedingly low.

``This means somebody probably handled those chicken wings who might
have had the virus,'' said Angela Rasmussen, a virologist at Columbia
University. ``But it doesn't mean, `Oh my god, nobody buy any chicken
wings because they're contaminated.'''

Guidelines from the Centers for Disease Control and Prevention
\href{https://www.cdc.gov/coronavirus/2019-ncov/daily-life-coping/food-and-COVID-19.html}{maintain}
that ``there is no evidence to suggest that handling food or consuming
food is associated with Covid-19.'' The main route the virus is known to
take from person to person is through spray from sneezing, coughing,
speaking or even breathing.

``I make no connection between this and any fear that this is the cause
of any long-distance transmission events,'' said C. Brandon Ogbunu, a
disease ecologist at Yale University. When the virus crosses
international boundaries, it's almost certainly chauffeured by people,
rather than the commercial products they ship.

The chicken wings were screened on Wednesday in Shenzhen's Longgang
district, where officials have been testing imports for the presence of
coronavirus genetic material, or RNA. Several samples taken from the
outer packaging of
\href{http://med.china.com.cn/content/pid/196876/tid/3}{frozen seafood},
some of which had been shipped in from Ecuador, recently tested positive
for virus RNA in China's Anhui, Shaanxi and Shandong provinces as well.

\hypertarget{latest-updates-the-coronavirus-outbreak}{%
\section{\texorpdfstring{\href{https://www.nytimes3xbfgragh.onion/2020/08/20/world/coronavirus-covid.html?action=click\&pgtype=Article\&state=default\&region=MAIN_CONTENT_1\&context=storylines_live_updates}{Latest
Updates: The Coronavirus
Outbreak}}{Latest Updates: The Coronavirus Outbreak}}\label{latest-updates-the-coronavirus-outbreak}}

Updated 2020-08-21T07:46:15.883Z

\begin{itemize}
\tightlist
\item
  \href{https://www.nytimes3xbfgragh.onion/2020/08/20/world/coronavirus-covid.html?action=click\&pgtype=Article\&state=default\&region=MAIN_CONTENT_1\&context=storylines_live_updates\#link-68774d88}{Shutdowns,
  warnings and scoldings follow alarming incidents on college campuses.}
\item
  \href{https://www.nytimes3xbfgragh.onion/2020/08/20/world/coronavirus-covid.html?action=click\&pgtype=Article\&state=default\&region=MAIN_CONTENT_1\&context=storylines_live_updates\#link-26b58724}{Biden
  knocks Trump's pandemic response, and outlines a national strategy.}
\item
  \href{https://www.nytimes3xbfgragh.onion/2020/08/20/world/coronavirus-covid.html?action=click\&pgtype=Article\&state=default\&region=MAIN_CONTENT_1\&context=storylines_live_updates\#link-4e542da3}{U.S.
  health agencies announce moves to confront the flu season and
  plummeting child vaccination rates.}
\end{itemize}

\href{https://www.nytimes3xbfgragh.onion/2020/08/20/world/coronavirus-covid.html?action=click\&pgtype=Article\&state=default\&region=MAIN_CONTENT_1\&context=storylines_live_updates}{See
more updates}

More live coverage:
\href{https://www.nytimes3xbfgragh.onion/live/2020/08/20/business/stock-market-today-coronavirus?action=click\&pgtype=Article\&state=default\&region=MAIN_CONTENT_1\&context=storylines_live_updates}{Markets}

Laboratory procedures that search for RNA also form the basis of most of
the coronavirus tests performed in people. But RNA is only a proxy for
the presence of the virus, which can leave behind bits of its genetic
material even after it has been destroyed, Dr. Ogbunu said. ``This is
just detecting the signature that the virus has been there at some
point,'' he said.

To prove that a dangerous, viable virus persists on food or packaging,
researchers would need to isolate the microbe and show in a lab that it
can still replicate. These experiments are logistically challenging and
require specially trained personnel, and aren't a part of the typical
testing pipeline.

After samples taken from the surface of the meat came up positive,
officials performed similar tests on several people whom they suspected
had come into contact with the product. They also tested a slew of other
packaged goods. All samples analyzed so far have been negative for
coronavirus RNA, according to a
\href{https://mp.weixin.qq.com/s?__biz=MjM5MDA2MDMwMA==\&mid=2650327724\&idx=1\&sn=82d67142b538e55e23c5d6914dbed753\&chksm=be46f0cf893179d9364006e687f78ae90c6d006b7e04b51d2a70d8a001ee3834b5f79ff232b5\&mpshare=1\&scene=1\&srcid=08134LeG4RM5VcAGUq13yBIf\&sharer_sharetime=1597277606241\&sharer_shareid=8188a18791dc8accdfc7b34b5c91b124\&exportkey=AQDw3kArVmi\%2BlFG4VWI6zg8\%3D\&pass_ticket=Hn\%2FCVGb9sV73pzEUJyB9evfQPBKCR8Q4s6FKHLeyil9HDixUw7G2ZZP3JZlW0noN\&wx_header=0\#rd}{statement}
released by the Shenzhen Epidemic Prevention and Control Headquarters
Office.

But the same statement cautioned consumers about imported frozen
products, and early reports of the news sparked
\href{https://twitter.com/business/status/1293767758107336704}{alarm on
social media}. In New Zealand, where a new outbreak has set off another
lockdown, officials are
\href{https://www.nytimes3xbfgragh.onion/2020/08/13/world/asia/new-zealand-coronavirus-lockdown-elimination.html}{tentatively
exploring the possibility} that the virus might have reentered the
country via frozen products imported from abroad.

Both Dr. Ogbunu and Dr. Rasmussen said that an extraordinarily unusual
series of events would need to occur for the virus to be transmitted via
a frozen meat product. Depending on where the virus originated, it would
need to endure a potentially cross-continental journey in a frozen state
--- likely melting and refreezing at least once along the way --- then
find its way onto someone's bare hands, en route to the nose or mouth.

Even more unlikely is the scenario that a virus could linger on food
after being heated, survive being swallowed into the ultra-acidic human
digestive tract, then set up shop in the airway.

``The risks of that happening are incredibly small,'' Dr. Rasmussen
said.

Some viruses might be able to weather such an onerous pilgrimage. But
the coronavirus probably isn't one of them because it's a so-called
enveloped virus, shrouded in a fragile outer shell that's vulnerable to
all sorts of environmental disturbances, including extreme changes in
temperature.

Viruses are often frozen in laboratories that maintain stocks of
pathogens for experiments. But virologists must monitor that process
carefully to avoid destroying the vulnerable bugs.

\href{https://www.nytimes3xbfgragh.onion/news-event/coronavirus?action=click\&pgtype=Article\&state=default\&region=MAIN_CONTENT_3\&context=storylines_faq}{}

\hypertarget{the-coronavirus-outbreak-}{%
\subsubsection{The Coronavirus Outbreak
›}\label{the-coronavirus-outbreak-}}

\hypertarget{frequently-asked-questions}{%
\paragraph{Frequently Asked
Questions}\label{frequently-asked-questions}}

Updated August 17, 2020

\begin{itemize}
\item ~
  \hypertarget{why-does-standing-six-feet-away-from-others-help}{%
  \paragraph{Why does standing six feet away from others
  help?}\label{why-does-standing-six-feet-away-from-others-help}}

  \begin{itemize}
  \tightlist
  \item
    The coronavirus spreads primarily through droplets from your mouth
    and nose, especially when you cough or sneeze. The C.D.C., one of
    the organizations using that measure,
    \href{https://www.nytimes3xbfgragh.onion/2020/04/14/health/coronavirus-six-feet.html?action=click\&pgtype=Article\&state=default\&region=MAIN_CONTENT_3\&context=storylines_faq}{bases
    its recommendation of six feet} on the idea that most large droplets
    that people expel when they cough or sneeze will fall to the ground
    within six feet. But six feet has never been a magic number that
    guarantees complete protection. Sneezes, for instance, can launch
    droplets a lot farther than six feet,
    \href{https://jamanetwork.com/journals/jama/fullarticle/2763852}{according
    to a recent study}. It's a rule of thumb: You should be safest
    standing six feet apart outside, especially when it's windy. But
    keep a mask on at all times, even when you think you're far enough
    apart.
  \end{itemize}
\item ~
  \hypertarget{i-have-antibodies-am-i-now-immune}{%
  \paragraph{I have antibodies. Am I now
  immune?}\label{i-have-antibodies-am-i-now-immune}}

  \begin{itemize}
  \tightlist
  \item
    As of right
    now,\href{https://www.nytimes3xbfgragh.onion/2020/07/22/health/covid-antibodies-herd-immunity.html?action=click\&pgtype=Article\&state=default\&region=MAIN_CONTENT_3\&context=storylines_faq}{that
    seems likely, for at least several months.} There have been
    frightening accounts of people suffering what seems to be a second
    bout of Covid-19. But experts say these patients may have a
    drawn-out course of infection, with the virus taking a slow toll
    weeks to months after initial exposure. People infected with the
    coronavirus typically
    \href{https://www.nature.com/articles/s41586-020-2456-9}{produce}
    immune molecules called antibodies, which are
    \href{https://www.nytimes3xbfgragh.onion/2020/05/07/health/coronavirus-antibody-prevalence.html?action=click\&pgtype=Article\&state=default\&region=MAIN_CONTENT_3\&context=storylines_faq}{protective
    proteins made in response to an
    infection}\href{https://www.nytimes3xbfgragh.onion/2020/05/07/health/coronavirus-antibody-prevalence.html?action=click\&pgtype=Article\&state=default\&region=MAIN_CONTENT_3\&context=storylines_faq}{.
    These antibodies may} last in the body
    \href{https://www.nature.com/articles/s41591-020-0965-6}{only two to
    three months}, which may seem worrisome, but that's perfectly normal
    after an acute infection subsides, said Dr. Michael Mina, an
    immunologist at Harvard University. It may be possible to get the
    coronavirus again, but it's highly unlikely that it would be
    possible in a short window of time from initial infection or make
    people sicker the second time.
  \end{itemize}
\item ~
  \hypertarget{im-a-small-business-owner-can-i-get-relief}{%
  \paragraph{I'm a small-business owner. Can I get
  relief?}\label{im-a-small-business-owner-can-i-get-relief}}

  \begin{itemize}
  \tightlist
  \item
    The
    \href{https://www.nytimes3xbfgragh.onion/article/small-business-loans-stimulus-grants-freelancers-coronavirus.html?action=click\&pgtype=Article\&state=default\&region=MAIN_CONTENT_3\&context=storylines_faq}{stimulus
    bills enacted in March} offer help for the millions of American
    small businesses. Those eligible for aid are businesses and
    nonprofit organizations with fewer than 500 workers, including sole
    proprietorships, independent contractors and freelancers. Some
    larger companies in some industries are also eligible. The help
    being offered, which is being managed by the Small Business
    Administration, includes the Paycheck Protection Program and the
    Economic Injury Disaster Loan program. But lots of folks have
    \href{https://www.nytimes3xbfgragh.onion/interactive/2020/05/07/business/small-business-loans-coronavirus.html?action=click\&pgtype=Article\&state=default\&region=MAIN_CONTENT_3\&context=storylines_faq}{not
    yet seen payouts.} Even those who have received help are confused:
    The rules are draconian, and some are stuck sitting on
    \href{https://www.nytimes3xbfgragh.onion/2020/05/02/business/economy/loans-coronavirus-small-business.html?action=click\&pgtype=Article\&state=default\&region=MAIN_CONTENT_3\&context=storylines_faq}{money
    they don't know how to use.} Many small-business owners are getting
    less than they expected or
    \href{https://www.nytimes3xbfgragh.onion/2020/06/10/business/Small-business-loans-ppp.html?action=click\&pgtype=Article\&state=default\&region=MAIN_CONTENT_3\&context=storylines_faq}{not
    hearing anything at all.}
  \end{itemize}
\item ~
  \hypertarget{what-are-my-rights-if-i-am-worried-about-going-back-to-work}{%
  \paragraph{What are my rights if I am worried about going back to
  work?}\label{what-are-my-rights-if-i-am-worried-about-going-back-to-work}}

  \begin{itemize}
  \tightlist
  \item
    Employers have to provide
    \href{https://www.osha.gov/SLTC/covid-19/standards.html}{a safe
    workplace} with policies that protect everyone equally.
    \href{https://www.nytimes3xbfgragh.onion/article/coronavirus-money-unemployment.html?action=click\&pgtype=Article\&state=default\&region=MAIN_CONTENT_3\&context=storylines_faq}{And
    if one of your co-workers tests positive for the coronavirus, the
    C.D.C.} has said that
    \href{https://www.cdc.gov/coronavirus/2019-ncov/community/guidance-business-response.html}{employers
    should tell their employees} -\/- without giving you the sick
    employee's name -\/- that they may have been exposed to the virus.
  \end{itemize}
\item ~
  \hypertarget{what-is-school-going-to-look-like-in-september}{%
  \paragraph{What is school going to look like in
  September?}\label{what-is-school-going-to-look-like-in-september}}

  \begin{itemize}
  \tightlist
  \item
    It is unlikely that many schools will return to a normal schedule
    this fall, requiring the grind of
    \href{https://www.nytimes3xbfgragh.onion/2020/06/05/us/coronavirus-education-lost-learning.html?action=click\&pgtype=Article\&state=default\&region=MAIN_CONTENT_3\&context=storylines_faq}{online
    learning},
    \href{https://www.nytimes3xbfgragh.onion/2020/05/29/us/coronavirus-child-care-centers.html?action=click\&pgtype=Article\&state=default\&region=MAIN_CONTENT_3\&context=storylines_faq}{makeshift
    child care} and
    \href{https://www.nytimes3xbfgragh.onion/2020/06/03/business/economy/coronavirus-working-women.html?action=click\&pgtype=Article\&state=default\&region=MAIN_CONTENT_3\&context=storylines_faq}{stunted
    workdays} to continue. California's two largest public school
    districts --- Los Angeles and San Diego --- said on July 13, that
    \href{https://www.nytimes3xbfgragh.onion/2020/07/13/us/lausd-san-diego-school-reopening.html?action=click\&pgtype=Article\&state=default\&region=MAIN_CONTENT_3\&context=storylines_faq}{instruction
    will be remote-only in the fall}, citing concerns that surging
    coronavirus infections in their areas pose too dire a risk for
    students and teachers. Together, the two districts enroll some
    825,000 students. They are the largest in the country so far to
    abandon plans for even a partial physical return to classrooms when
    they reopen in August. For other districts, the solution won't be an
    all-or-nothing approach.
    \href{https://bioethics.jhu.edu/research-and-outreach/projects/eschool-initiative/school-policy-tracker/}{Many
    systems}, including the nation's largest, New York City, are
    devising
    \href{https://www.nytimes3xbfgragh.onion/2020/06/26/us/coronavirus-schools-reopen-fall.html?action=click\&pgtype=Article\&state=default\&region=MAIN_CONTENT_3\&context=storylines_faq}{hybrid
    plans} that involve spending some days in classrooms and other days
    online. There's no national policy on this yet, so check with your
    municipal school system regularly to see what is happening in your
    community.
  \end{itemize}
\end{itemize}

``The act of freezing and unfreezing is a kind of violent thermodynamic
process,'' Dr. Ogbunu said. ``A virus, for all its toughness and
robustness, is a very delicate instrument of infection.''

The C.D.C. has noted that ``it is possible'' that the coronavirus can
spread through contaminated surfaces, including food or food packaging.
But that's not known to be among the main ways the virus gets around.

If you don't want get infected, avoiding direct contact with other
people is probably a better use of your time, Dr. Ogbunu said.

``Yes, we should continue to wash our hands and be mindful of surfaces
where a lot of individuals are,'' he said. ``But it's close proximity to
others that can really facilitate transmission.''

Advertisement

\protect\hyperlink{after-bottom}{Continue reading the main story}

\hypertarget{site-index}{%
\subsection{Site Index}\label{site-index}}

\hypertarget{site-information-navigation}{%
\subsection{Site Information
Navigation}\label{site-information-navigation}}

\begin{itemize}
\tightlist
\item
  \href{https://help.nytimes3xbfgragh.onion/hc/en-us/articles/115014792127-Copyright-notice}{©~2020~The
  New York Times Company}
\end{itemize}

\begin{itemize}
\tightlist
\item
  \href{https://www.nytco.com/}{NYTCo}
\item
  \href{https://help.nytimes3xbfgragh.onion/hc/en-us/articles/115015385887-Contact-Us}{Contact
  Us}
\item
  \href{https://www.nytco.com/careers/}{Work with us}
\item
  \href{https://nytmediakit.com/}{Advertise}
\item
  \href{http://www.tbrandstudio.com/}{T Brand Studio}
\item
  \href{https://www.nytimes3xbfgragh.onion/privacy/cookie-policy\#how-do-i-manage-trackers}{Your
  Ad Choices}
\item
  \href{https://www.nytimes3xbfgragh.onion/privacy}{Privacy}
\item
  \href{https://help.nytimes3xbfgragh.onion/hc/en-us/articles/115014893428-Terms-of-service}{Terms
  of Service}
\item
  \href{https://help.nytimes3xbfgragh.onion/hc/en-us/articles/115014893968-Terms-of-sale}{Terms
  of Sale}
\item
  \href{https://spiderbites.nytimes3xbfgragh.onion}{Site Map}
\item
  \href{https://help.nytimes3xbfgragh.onion/hc/en-us}{Help}
\item
  \href{https://www.nytimes3xbfgragh.onion/subscription?campaignId=37WXW}{Subscriptions}
\end{itemize}
