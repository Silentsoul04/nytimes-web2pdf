Sections

SEARCH

\protect\hyperlink{site-content}{Skip to
content}\protect\hyperlink{site-index}{Skip to site index}

\href{https://www.nytimes3xbfgragh.onion/spotlight/podcasts}{Podcasts}

\href{https://myaccount.nytimes3xbfgragh.onion/auth/login?response_type=cookie\&client_id=vi}{}

\href{https://www.nytimes3xbfgragh.onion/section/todayspaper}{Today's
Paper}

\href{/spotlight/podcasts}{Podcasts}\textbar{}Episode Four: `Here's
Another Fun Thing You Can Do'

\url{https://nyti.ms/2FfGJc5}

\begin{itemize}
\item
\item
\item
\item
\item
\end{itemize}

Advertisement

\protect\hyperlink{after-top}{Continue reading the main story}

transcript

Back to Nice White Parents

bars

0:00/50:38

-50:38

transcript

\hypertarget{episode-four-heres-another-fun-thing-you-can-do}{%
\subsection{Episode Four: `Here's Another Fun Thing You Can
Do'}\label{episode-four-heres-another-fun-thing-you-can-do}}

\hypertarget{reported-by-chana-joffe-walt-produced-by-julie-snyder-edited-by-sarah-koenig-neil-drumming-and-ira-glass-editorial-consulting-by-eve-l-ewing-and-sound-mix-by-stowe-nelson}{%
\subsubsection{Reported by Chana Joffe-Walt; produced by Julie Snyder;
edited by Sarah Koenig, Neil Drumming and Ira Glass; editorial
consulting by Eve L. Ewing; and sound mix by Stowe
Nelson}\label{reported-by-chana-joffe-walt-produced-by-julie-snyder-edited-by-sarah-koenig-neil-drumming-and-ira-glass-editorial-consulting-by-eve-l-ewing-and-sound-mix-by-stowe-nelson}}

\hypertarget{is-it-possible-to-limit-the-power-of-white-parents}{%
\paragraph{Is it possible to limit the power of white
parents?}\label{is-it-possible-to-limit-the-power-of-white-parents}}

Thursday, August 13th, 2020

\begin{itemize}
\tightlist
\item
  speaker\\
  ``Nice White Parents'' is brought to you by Serial Productions, a New
  York Times Company.
\end{itemize}

chana joffe-walt

I want to tell you about another old film I found during my research.
It's from 1951. We see a housewife, a white woman --- everyone in this
film is white. She's sitting in her living room with some neighbors.
They're here to solve a problem. {[}OLD MOVIE PLAYING{]}

\begin{itemize}
\tightlist
\item
  archived recording\\
  A chain broke on a swing in a public playground, relatively
  unimportant, but a child might have been hurt.
\end{itemize}

chana joffe-walt

They sit on couches with notepads, deep in discussion. They will solve
this problem together. In another scene, a machinist in California
approaches his bus, the factory owner, with a request from the workers.

\begin{itemize}
\tightlist
\item
  archived recording\\
  I'd like to show you the new pension plan that we built. I thought we
  had discussed the pension plan previously.
\end{itemize}

chana joffe-walt

They had. The discussion requires listening, debating and waving your
arms a lot, which they do in the film.

\begin{itemize}
\tightlist
\item
  archived recording\\
  Not yet friends, they may never like each other. But they'll sweat it
  out together. The problem is mutual. Much is involved. Developed
  within each citizen is the Democratic spirit, the Democratic method.
\end{itemize}

chana joffe-walt

Where were they taught the Democratic method, you might ask? Public
school, where the housewives taught to problem solve for the safety of
the community's children, public school. This film was made by the
National Education Association. It's a 25-minute promotional film that
spends almost no time inside schools. Instead, it's all about the
purpose of public schools, how they prepare us to live together as
citizens. We see Americans use their public school training in everyday
life, when they sit with their neighbors, debate their bosses, when they
go shopping, and drive a car, buy a house. We are all part of a grand
play, interdependent --- the senator, the homemaker, the factory worker.

\begin{itemize}
\tightlist
\item
  archived recording\\
  And Fred Gorman, the farmer of Pennsylvania, are his decisions
  important? They are, if the nation wants to eat.
\end{itemize}

chana joffe-walt

Fred the farmer has a nameless wife, whom we see now standing next to
him. Fred's wife is trying to resolve a problem. The neighbors want to
build a drainage system into a pond. The lowest land for the pond is an
orchard that belongs to Fred and his nameless wife. The wife understands
that to prevent further flooding, she and Fred will need to sacrifice
for the greater good. Fred is not so sure.

\begin{itemize}
\tightlist
\item
  archived recording\\
  If we don't do something to help, our land is going to get like
  theirs, and you know it. I don't like the idea of losing those trees.
  There's the problem in a nutshell, a tough one to crack, his land and
  his neighbors needs. A dictator could solve it for Fred, but he
  prefers to do his own thinking.
\end{itemize}

chana joffe-walt

Luckily, Fred has the tools to do it.

\begin{itemize}
\tightlist
\item
  archived recording\\
  Those tools are sharpened in the schools of America.
\end{itemize}

chana joffe-walt

And thank goodness, because the stakes are high.

\begin{itemize}
\tightlist
\item
  archived recording\\
  Problems every day, and the way they are solved determines the way the
  country functions.
\end{itemize}

chana joffe-walt

This vision of public schools, the same one laid out 100 years earlier
by the founder of American public schools Horace Mann is that America
and democracy cannot survive without public education. We need common
schools where rich and poor come together to solve problems, generate
fellow feeling. Public schools, the great equalizer. But I have made my
way through the history of one modern American public school. And from
what I can see, white parents are standing in the way of achieving this
vision. Our schools are not an equalizing force, because white parents
take them over and hoard resources. We're not learning how to live
together as one society because white parents flee or cordon themselves
off in special gifted programs. Even when we're not in the school
building, funding and attention still slide our way. So I don't see how
it's possible to have equal public schools, common schools that serve
every child, unless we limit the power of white parents. But how do we
do that? In all my reporting around this one school building from 2015
all the way back to the beginning, I've never seen that happen. And then
I did. From Serial Productions, I'm Chana Joffe-Walt. This is ``Nice
White Parents,'' a series about the most powerful force in public
schools: White people.

Recently, I've come across two examples of schools that seem to be
suppressing the power of white parents, two examples I found in the very
last place I expected, in the I.S. 293 building, one upstairs and one
downstairs. So today's episode, what does it look like to limit the
power of white parents in schools? And does it work? Does it lead to an
equal education for everyone?

I'm going to start downstairs. Eight years ago, the city put a charter
school in the basement if I.S. 293. It's called Success Academy. That
year I spent following the new white parents upstairs at the School For
International Studies, I would occasionally see Success Academy kids
around the building in orange and blue uniforms. It was always a little
startling because Success Academy is an elementary school. So they look
tiny in a building full of middle and high schoolers. But mostly, the
Success kids stood out because of the way they moved through the halls.

\begin{itemize}
\tightlist
\item
  speaker\\
  They walk in single forms like they're in the Army. It's so weird. If
  they don't walk in single form, they stop the whole line.
\end{itemize}

chana joffe-walt

Denagee is one of the many students from SIS upstairs, who is eager to
tell me about the charter school and its rituals, their silent,
controlled lines.

\begin{itemize}
\item
  denagee\\
  It's like a sense of the Children of the Corn. It creeps me out.
\item
  chris\\
  Or Storm Troopers or something.
\end{itemize}

chana joffe-walt

That's his friend Chris saying Storm Troopers. Sometimes they'll hear
Success teachers say, make a bubble in your mouth. And then a line of
six-year-olds will close their lips and fill their cheeks up with air,
that way nobody's talking. Chris and Denagee told me they look like
puffer fish.

\begin{itemize}
\tightlist
\item
  denagee\\
  I remember kindergarten very vividly. And I know if I was to have my
  face in a puffer fish, I would automatically just start making all
  types of sounds and stuff, you know?
\end{itemize}

chana joffe-walt

And they don't?

\begin{itemize}
\item
  denagee\\
  No. That's what's so weird.

  It helps me think, or it makes me think about --- what really happens
  inside of the classrooms with them to be coming out like that?
\end{itemize}

chana joffe-walt

The year I was reporting at SIS, The New York Times published a video
that showed a particular and alarming moment inside one of the Success
classrooms. It was secretly recorded by an assistant teacher, who leaked
it, and it went viral. You see a group of first-graders gathered in a
circle on a polka-dot rug, sitting legs crossed, hands in their laps.
And a teacher is asking one girl to correct a math problem she got
wrong.

\begin{itemize}
\tightlist
\item
  speaker\\
  You cut or you split. So count it again, making sure you're counting
  correctly.
\end{itemize}

chana joffe-walt

The girl does not respond. The teacher leans in and repeats.

\begin{itemize}
\tightlist
\item
  speaker\\
  Count.
\end{itemize}

chana joffe-walt

The girl whimpers, or says something so quiet you can't hear. The entire
class is watching. It's silent, intense. The teacher is visibly upset,
picks up the child's paper, and rips it in half, points an angry finger
to the side of the room.

\begin{itemize}
\tightlist
\item
  speaker\\
  Go to the calm down chair and sit.
\end{itemize}

chana joffe-walt

She goes. The teacher turns to the rest of the circle.

\begin{itemize}
\tightlist
\item
  speaker\\
  There's nothing that infuriates me more than when you don't do what's
  on your paper. Somebody come up and show me how she should have
  counted to get her answer that was 1 and a split.
\end{itemize}

chana joffe-walt

A boy rushes over to do it correctly. But the teacher is not done
publicly reprimanding the girl, who's now sitting to the side of the
classroom in the calm down chair.

\begin{itemize}
\tightlist
\item
  speaker\\
  Thank you. Do not go back to your seat and show me one thing and then
  don't do it here. You're confusing everybody. Very upset and very
  disappointed.
\end{itemize}

chana joffe-walt

The teacher is white. The girl whose work she just ripped up is not. The
whole thing is hard to watch. When this video came out, the student was
living at a homeless shelter with her mom. Success suspended this
teacher, but didn't fire her. Instead, at a press conference, the C.E.O
reprimanded The New York Times for not understanding that this teacher
was having a bad day. When I asked about this incident, the C.E.O of
Success told me the teacher's behavior was unacceptable. Teachers are
not allowed to yell at kids. But it was not a fireable offense. She says
the teacher made a mistake.

I have always been skeptical of Success Academy. Success has a
reputation for being harsh and punitive. Especially unnerving, to me at
least, is that they're harsh, punitive approach is deployed in schools
across the city that are almost entirely BBlack and brown. Success
students are generally kids of color from working class or poor
families. The intense focus on policing kids' bodies, on test prep
drills, frequent use of suspensions --- you don't see that in majority
white schools. I've never seen a line of uniformed, white students
walking through the halls of a public school building with their mouths
in bubbles, or being told to quote show urgency when they dawdle
unpacking their book bags or eating lunch, except for here. This
particular Success Academy in the basement of I.S. 293 is integrated. A
quarter of the student body is white. And it's the first school I saw
putting limits on the power of white parents.

Success Academy is the city's largest charter school network, 47
schools, elementary, middle, and one high school. They get public
funding, like all charter schools. Success Academy also gets private
funding. The state oversees charters, like Success. But it isn't run by
the state or the city. It's run by a private organization. And Success
is a choice school. That means families opt in to Success. The C.E.O, a
woman named Eva Moskowitz, opened her first 40 something schools in
largely working class Black and brown neighborhoods where she imagined
families would want a new school option. Then about a decade ago,
Moskowitz decided she wanted to open an integrated school, a new Success
Academy that was racially integrated and economically diverse. She
needed a school building where integration was possible, where perhaps,
half a century earlier, a group of white families pushed for a
strategically located fringe school building between two racially
segregated neighborhoods. And this is how Success Academy wound up here,
in the old I.S. 293 building because of yet another plan to integrate.
Only this time, it worked. White parents opted in. The way families at
this Success Academy --- it's called Success Academy Cobble Hill ---
tend to come from advantage, just like the white parents upstairs at
SIS. They're upper middle class and rich, doctors and lawyers, corporate
accountants, people who walk into most public schools with a lot of
power. But the influence I'd seen white parents wield upstairs at SIS,
that didn't seem to be the case downstairs. I found that confusing. Do
you have a PTA? We have a parent council so it's very similar to a PTA
this is Alissa Bishop, the principal of Success, Cobble Hill. The parent
council is not that similar to a PTA though, because in the very next
sentence, Principal Bishop told me that the parent council is not
allowed to raise money. This, I assumed, was probably difficult for
parents who are accustomed to fundraising for their kids' schools. Have
you had parents who want to raise money, who come to you and are like, I
want to --- I want this thing to happen, and I want to raise the money
for it.

\begin{itemize}
\item
  alissa bishop\\
  Not anything like that. I have had parents come to me and say I want
  to do a coat drive. They want to donate. We do that stuff throughout
  the entire year. But I've never had anyone approach me about donating
  money.
\item
  speaker\\
  Oh, wow.
\item
  chana joffe-walt\\
  Really? No parents have been like I want to do a fundraiser for x, and
  you have to be like, that's not a thing that we do?
\item
  alissa bishop\\
  No. I've never had that. Principal Bishop looks over at the PR person,
  who's come from Success headquarters to supervise this interview. The
  PR person shakes her head. No, parents don't raise money.
\item
  chana joffe-walt\\
  And what if somebody did want to raise money for the school --- a
  parent wanted to raise money?
\item
  alissa bishop\\
  Yeah, we don't.
\item
  speaker\\
  It's against the policy.
\item
  alissa bishop\\
  We don't raise money.
\end{itemize}

chana joffe-walt

Principal Bishop looks over to the PR person again, as if to say, am I
not being clear with this chick? Why isn't she getting it? But I seem to
be unable to stop myself from listing all the things I've seen white
advantaged parents demand in public schools.

\begin{itemize}
\item
  chana joffe-walt\\
  If parents were like, we want this to be a dual language French
  school, and we can help fund it ---
\item
  alissa bishop\\
  We don't have --- our curriculum is network-based. We're given
  curriculum. We don't have a language curriculum in our elementary
  schools.
\item
  chana joffe-walt\\
  Or, if parents were like, we want there to be less math, or a
  different kind of math, or we want there to be a film program, a film
  program, or whatever, any of those things. Parents are like, we want
  ---
\item
  alissa bishop\\
  Yeah, this is our model. It's our model across all of our schools. No
  changes.
\end{itemize}

chana joffe-walt

The C.E.O. of Success Academy, Eva Moskowitz, followed up later to tell
me, if parents want to give money, they can. But it will be distributed
evenly across all of our schools. We can't have our Cobble Hill families
getting more than our families in Harlem.

Here's what I started to understand about how Success Academy was
limiting the power of white parents. Success was limiting the power of
white parents by limiting the power of all parents. I met a dad named
Travius Sharpe outside the building one day, a Black guy who grew up in
Brooklyn. His son Ethan is at Success.

\begin{itemize}
\item
  travius sharpe\\
  We actually get graded.
\item
  chana joffe-walt\\
  You get graded?
\item
  travius sharpe\\
  We get graded as the parents. We get an email saying, this is what
  your progress is saying.
\item
  chana joffe-walt\\
  You get a grade, like an A, B, C?
\item
  travius sharpe\\
  They get like a meeting expectations or ---
\item
  chana joffe-walt\\
  Not.
\item
  travius sharpe\\
  --- not.
\end{itemize}

chana joffe-walt

Upstairs, SIS tripped over itself to meet the demands of new, white
parents. Downstairs, all parents at Success Academy are being graded.
Even day to day, the Success principal and teachers make sure to remind
parents when they're falling down on the job.

\begin{itemize}
\item
  speaker\\
  So we running a little late. Why is Ethan late? It's your fault why
  he's late. I think one day, I was late. And she texted and said, Ethan
  is not here yet. Any reason why? And I felt like I wasn't the parent
  at that point. That's their time. But it keeps you on your toes and
  stuff like that.
\item
  chana joffe-walt\\
  You felt like you weren't the parent.
\item
  speaker\\
  I wasn't the parent. I wasn't the parent. And I felt like I was just
  dropping this kid off to his parents.
\end{itemize}

chana joffe-walt

One day, in the cafeteria, I met a white mom named Sarah Stanich. Sarah
is a financial advisor. Her son's in fourth grade. And she was telling
me she likes the school, even though --- and then Sarah lowered her
voice, pointed at her boy, and said, he's been suspended.

\begin{itemize}
\item
  sarah stanich\\
  He's been suspended. And I was not happy about that. And I definitely
  never had that experience when I was a kid. But ---
\item
  chana joffe-walt\\
  How old was he when that happened?
\item
  sarah stanich\\
  Well, it's happened more than once, embarrassingly.

  So kind of young. Maybe third grade, or maybe even second the first
  time it happened.
\item
  chana joffe-walt\\
  How many times has he been suspended?
\item
  sarah stanich\\
  A few times, a few times, probably three --- but pushing, fighting.
  And he's really not a fighter, but they're boys, and --- and sometimes
  I think it's kind of harsh. They're young kids. And I know that that's
  a complaint about suspensions in the schools. But on the other hand,
  he had warnings, and it wasn't that I think that his teachers had
  given him given him space and slack in other areas. I have no
  lingering anger about it.
\item
  chana joffe-walt\\
  Overall.
\item
  sarah stanich\\
  Yeah. Overall, I've been I feel very lucky to have been able to be a
  part of this community and be part of this school. Sarah later wrote
  me to say her kid was actually suspended four times that year. I've
  reported on discipline in schools and the use of suspensions a lot.
  I've talked to many mothers of children who have been suspended. Not
  one of them has been white. Black kids are suspended in New York City
  schools at five times the rate of white kids. After I met Sarah, I
  double checked the numbers for the 2017 school year, just to be sure.
  In the regular New York City public schools that same year, not
  Success or other charters, but the traditional, public elementary
  schools, that year, there were 327 suspensions for non-white kids. For
  white kids, there were only nine. I was so surprised after meeting
  Sarah when I left the building I called two people, who know a lot
  about education to say this is what's happening at Success Academy,
  Cobble Hill. White boys are being suspended, rich, white boys. And
  they couldn't believe it either. One of them, Noliwe Rooks, a
  professor at Cornell, said, well, well, how's that for equality?

  So white parents can't raise money, they can't ask for special
  programs, and their kids get suspended. Why are they suddenly OK with
  equality? I interviewed lots of Success parents.
\item
  suzanne gigliotti\\
  We did get a flyer. They put them on the doors. They put them on the
  doors in the neighborhood.
\end{itemize}

chana joffe-walt

Suzanne Gigliotti saw the flyer for Success when her son was in
preschool. So she looked into it and every other possible school option
she had.

\begin{itemize}
\item
  suzanne gigliotti\\
  It was in our neighborhood. But more importantly, we toured so many
  schools, public, private, parochial. We were slated for 58, which is
  an excellent school. And we did get in there. But Success was head and
  above any school I'd seen, just the level of excellence. And yeah,
  nothing matched it. The test scores --- almost every parent I spoke
  with said they were initially drawn to Success Academy because of the
  excellent test scores. If your measure of success in school is
  standardized tests --- and at Success Academy, it is --- this is one
  of the best schools in the city. The scores are truly remarkable.
  Success Academy students perform twice as well on state tests as
  regular New York City public school kids. The vast majority of Success
  kids pass the tests, 95\%, 97\%. In your average city public schools,
  it's less than half. And even more impressive, to me at least, is that
  the kids at Success are doing well on tests no matter if they're poor,
  or rich, or Black, or Latino, or Asian, or white. This is the problem
  that decades of public education reforms have tried to address, the
  achievement gap. Success Academy was pulling off, not only an
  integrated school, but an equal integrated school that was closing the
  achievement gap.

  The way Success achieves equality though, some things give me pause.
\item
  speaker\\
  What's my first expectation? Lock your hands. Track Kamira. The first
  expectation is read ---
\end{itemize}

chana joffe-walt

Last year, I went into the Success classrooms.

\begin{itemize}
\tightlist
\item
  speaker\\
  Send Kamira some love. Give Kamira two claps. {[}CLAPPING{]} My
  expectation is that ---
\end{itemize}

chana joffe-walt

I didn't see any teachers reprimanding kids or ripping up work, like the
one in the video. What I did see were teachers, who issued a constant
wall of verbal directions, where to look, what to do, how to sit,
delivered in the same and consistent, neutral tone. When a teacher calls
on someone, she gives a direction to the class to track the speaker,
look at the person speaking. Meanwhile, a second teacher roams and
hovers, issuing reminders.

\begin{itemize}
\tightlist
\item
  speaker\\
  Lock your hands. Track Shana. Liam's hands are locked tracking Shana.
  Lydia's hands are locked tracking Shana. Colin's hands are ---
\end{itemize}

chana joffe-walt

Shana answers correctly.

\begin{itemize}
\tightlist
\item
  speaker\\
  Nice job, Shana. Nice job, Shana. Scanning for another friend on the
  carpet, who looks so professional, lock your hands, track Zoe.
\end{itemize}

chana joffe-walt

Success achieves equality, at least in part, through utter uniformity.
Every Success Academy across the city uses identical methods, identical
curricula, and identical classrooms. The kids sit on the same polka-dot
carpet, hands locked in their lap, same signs on the wall, singing the
same chants. Even the teachers look the same. They're almost all young
white women in cotton dresses and ballet flats just out of college,
sometimes the same college. I know this because the classrooms are named
after teachers alma maters. And there are three Penn State classrooms.

\begin{itemize}
\tightlist
\item
  speaker\\
  How does it go? We are Penn State. Yeah, we pull our weight Yeah, we
  cannot wait. Yeah, to graduate. Roll call. It's uh-huh. And then
  Shabooya, sha-sha-shabooya, role call. Shabooya, sha-sha-shabooya.
\end{itemize}

chana joffe-walt

Education people talk a lot about the difference between equality and
equity to a point that I believe is tiresome. But I thought about this
difference a lot at Success. Equality means everyone gets the same
thing. Equity means everyone gets what they need. Success is equal.
Everyone is treated the same. But kids are never all the same. Some kids
are chatty in the hallway, or need a minute to think before answering a
question. Some kids have a million bucks at home, and some kids don't. A
Black girl might respond differently than a white girl to being
reprimanded by a white teacher. A single parent with two jobs might have
a harder time getting their kid to school on time than, say, a
stay-at-home mom with a partner. One of the main criticisms of Success
Academy from public education advocates is that Success doesn't actually
serve all students, that it has excellent test scores because it serves
a select group of students. Kids who don't test well, or can't sit
still, they're weeded out of the school. Success Academy vehemently
denies this. They point out that they make special accommodations for
kids with special needs, and they note that they don't get to choose
students because kids get spots in their schools by random lottery. And
that's true. But it's also true that lots of parents don't apply to the
lottery because they know the school's culture and the demands it makes
of families won't work for them. And plenty of kids who do end up at
Success don't last long. Maybe they get held back a grade or they're
suspended. A civil rights complaint filed on behalf of more than a dozen
families alleges their children were regularly removed from class and
suspended, seven, 10, 13 times at Success Academy. Most of those
families eventually left the school.

I had a thought walking through Success. I suspected that the strict
classroom control was partly what made white parents feel comfortable at
Success Academy. I'm speculating here. None of the white parents I spoke
with told me they chose Success because the school polices Black and
brown students so well. And I don't believe this is a conscious thought
for anyone. But I do know that white parents bring plenty of unconscious
biases to public schools with Black and brown kids, fears that the
classrooms will be chaotic, or not challenging, that the kids will be
disorderly or threatening. White parents worry that our kids will be
harmed. Success Academy completely controls for these fears. Everyone
gets excellent test scores. There's no room for misbehavior, no risk of
disruption because there are no idle moments. If 30 children need to
move from their desks to the rug, it sounds like this.

\begin{itemize}
\tightlist
\item
  speaker\\
  On your bottom, on the black line in five, four, three, two, one.
\end{itemize}

chana joffe-walt

Every kid is on their bottom, hands locked, eyes tracking the teacher,
except for one boy. He gets a correction.

Success operates on the principle that with rigor and discipline
uniformly applied, all students will achieve equally well. It's a
tempting vision, especially coming from upstairs, where the power of
white parents seem to have no bounds. But equality does not necessarily
shift the balance of power. White parents aren't running the show here,
but Success is run by a white C.E.O and a board that includes
millionaire hedge fund managers --- sorry, billionaire hedge fund
managers. The board of trustees is listed on the success website. And
the bios include Maverick Capital, Redwood Capital, Glenview Capital,
Cumulus Media, Morgan Stanley, Facebook, Arnold \& Porter. This is not
exactly a disruption to the social order, is all I'm saying. You can
limit the day-to-day influence of white parents. But still, rich white
people control the agenda, the priorities, and the money.

Back in 2015, the year of the white influx and SIS, toward the end of
that school year, I was talking to Imee Hernandez one day. She was the
PTA co-president of SIS. And Imee told me watching all those white
parents come and take over, it was almost like watching tumbleweed move
along in the wind. It was so quiet. That's how they moved through here,
she said, picking up power as they went.

\begin{itemize}
\tightlist
\item
  imee hernandez\\
  Like a tumbleweed, it starts really soft and slow, and it keeps just
  picking up speed and getting bigger. So it's really soft and slow. But
  it's getting bigger. It's not like an avalanche that comes at you.
  It's just tumbling along very slowly. So it's very light. You don't
  feel it coming at you.
\end{itemize}

chana joffe-walt

Back then, Imee told me there's no stopping it. She's worried she
couldn't protect what she loved about her school.

\begin{itemize}
\item
  chana joffe-walt\\
  If you were right, and the worst case scenario happens, what does that
  look like in a year or two?
\item
  imee hernandez\\
  That there's no more color in this school.

  Then there's no more community, which I really hope I'm wrong. That's
  my biggest fear. Then I would question if my daughter's coming back. I
  really would.
\end{itemize}

chana joffe-walt

Imee feared that each year, more and more white families come into SIS
until it just became like the other segregated middle schools, where all
the white parents fought to enroll their kids. Against the repetition of
history, Imee was wrong. What happened at SIS was nothing like she or I
expected. That's up next when we go back upstairs.

This past spring, a Black teacher at Success Academy named Fabiola St
Hillaire publicly criticized the C.E.O. for not taking a stand after the
murder of George Floyd, or acknowledging the effect police violence was
having on the families and communities Success serves. After that, more
staff, families, and alumni raised alarms about Success, calling some of
its practices racist and abusive, its discipline policies, the way white
staff and leadership speak to kids and parents of color. In response,
the C.E.O. apologized, and Success has released a plan that commits to
mandatory bias and sensitivity training for staff. The plan says they
will create an Equity Team and review their culture, their relationships
with staff, and families, and kids with quote, ``an attention and
sensitivity to race.'' I read this plan and thought, huh, there is a
school that's already doing many of these things, right in the same
building, right upstairs.

\begin{itemize}
\tightlist
\item
  speaker\\
  OK, welcome. Hello. Thank you.
\end{itemize}

chana joffe-walt

It's September, 2019. I'm back at SIS. It's been four years since the
French gala and the drama with the PTA. Rob, the dad, who fundraises,
he's not here anymore. His son finished middle school. Imee is still
here. Her daughter is a junior in high school. And a new crop of sixth
graders and their families are settling into the auditorium.

\begin{itemize}
\tightlist
\item
  speaker\\
  Welcome to PA Chaz. Welcome. Please, find a seat for me. Thank you.
\end{itemize}

chana joffe-walt

The school is no longer called SIS, the School for International
Studies. It's now BHS, the Boerum Hill School for International Studies.
They changed the name, again. BHS has a new principal, Nicole
Lanzillatto. She gets up on stage, and the staff cheers. Miss
Lanzillatto welcomes the new families to BHS.

\begin{itemize}
\tightlist
\item
  nicole lanzillatto\\
  Any school is a microcosm of the world, and we are blessed with
  beautiful diversity.
\end{itemize}

chana joffe-walt

Miss Lanzillatto lists the ways the school reflects the world, race,
ethnicity, language, gender.

\begin{itemize}
\tightlist
\item
  nicole lanzillatto\\
  We are an extraordinarily diverse community. And it's a beautiful
  thing, and we fight for it, and we work on it.
\end{itemize}

chana joffe-walt

Miss Lanzillatto says BHS is going for true equity. She says the word
equity three times in this welcome speech. Miss Lanzillatto is white,
chatty, well-liked, with black hair that's styled straight up. The hair
is really Miss Lanzillatto's defining feature. Picture boy band
pompadour. She's worked here most of her professional career. The year
white families arrived at SIS, Miss Lanzillatto was the assistant
principal. She won't say anything bad about that year. It was a learning
experience. It's a process her predecessor, Ms. Juman, talks about it
the same way. Remember? Principals --- diplomatic. They're careful not
to place blame, but both of them said after that year, it was clear they
needed to intervene. One of the first things Miss Lanzillatto did as
principal was request special permission to reserve 40\% of the seats
for kids who get free and reduced price lunch. The majority of kids who
get free and reduced price lunch are kids of color. And Miss Lanzillatto
didn't want the school to flip. She didn't want Black and brown kids to
get pushed out. The assistant principal told me they wanted to make sure
the school did not become colonized. Some things here have changed. They
got rid of the foundation, the Brooklyn World Project Rob and the other
white parents had created. They scrapped some of the French programming,
hired more teachers and staff of color. And one of the most striking
changes I noticed --- spend 10 minutes of the school, and you can't not
notice --- Miss Lanzillatto is talking directly and constantly about
race and equity. She told me everyone here needs to be on alert for
racist habits and ideas. They need to aggressively address them,
whenever they pop up, in the cafeteria, in the classroom.

\begin{itemize}
\tightlist
\item
  nicole lanzillatto\\
  There's a conversation happening in the school around the smart
  classes and the non-smart classes. Let's talk about it where is that
  coming from. So I think it's really about being a beast. I think it's
  about everything we do coming back to it.
\end{itemize}

chana joffe-walt

Coming back to equity. I could not get over how much time and energy the
school puts into ensuring equity, not equality, equity. It's almost like
the obsessive focus Success puts on making sure everything is the same
is exactly matched by the obsessive focus BHS just puts on recognizing
everyone is not the same. BHS formed an Equity Committee of staff and
students a few years ago. They looked for bias in the curriculum, in the
signs on their walls and the books on their shelves. They analyzed
achievement data, discipline data, where they could clearly see that the
school punished Black boys more harshly than other students. So they
revamped their entire approach to discipline, created a restorative
Justice Department. They applied for grants to help pay for this to
train their teachers on implicit bias and then train them again. They
brought in experts.

\begin{itemize}
\tightlist
\item
  speaker\\
  And here are some things that I look for in transition. So how do kids
  engage with each other? Is it verbal engagement? Is it non-verbal
  engagement?
\end{itemize}

chana joffe-walt

Last fall, I watched two equity consultants, Cornelius and Kass Minor,
show a group of BHS teachers how to observe racial dynamics in their
school. This involved teachers walking around in a huddle with
clipboards, taking diligent notes as kids walk through the hallways.

\begin{itemize}
\tightlist
\item
  speaker\\
  One fun lens to look at --- and I'm just naming things out --- I often
  ask, what are boys doing? What are girls doing? What are Black
  students doing? What are students of color doing?
\end{itemize}

chana joffe-walt

Mr. Minor is full of fun things the teachers should look for.

\begin{itemize}
\tightlist
\item
  speaker\\
  Here's another fun thing to do, just because we're out here. I do
  drive-bys in the hallway, where I walk by classroom windows, and I
  look in.
\end{itemize}

chana joffe-walt

They all take turns peering through the small window of a classroom
door. They take more notes. Later, the teachers meet as a group. And one
teacher, Stacy Ann Manswell, explains her observations from a math
class.

\begin{itemize}
\tightlist
\item
  stacy ann manswell\\
  And then in the math classroom that we were in, something that stood
  out to me --- so there was two white males, white female, Black male.
  And I'm walking around. And Black male, he was finished. And he
  finished early, waiting for his peers to do the Think Right Pair
  Share. And when the timer went off, the girl, the white girl he was
  sitting next to, he looked to her, but she looked to the two white
  boys. And they formed the pair. So it was like, now she had to work
  with him. But she was sort of looking for the other two boys for
  validation for what this boy was saying. So my teacher self is like,
  OK, does this child not participate in class, and she doesn't trust
  that he knows what he's doing, or is it because she doesn't see him
  because he's a Black boy and she figures he's not capable?
\end{itemize}

chana joffe-walt

The teachers talked about this moment in depth, what it might mean, what
messages the kids were picking up in their school about race, about
who's important, who's bad, who's smart. And it's not just the staff.
The administration is telling white parents that their mere presence in
the school does not make it integrated. They have to work at making this
place fair.

\begin{itemize}
\tightlist
\item
  meghan casey\\
  So this is our agenda today. We're going to start with a reflection,
  and we're going to get into how we talk about race with our young
  people.
\end{itemize}

chana joffe-walt

One Saturday morning, a group of two dozen parents gathered in the BHS
library for something called Family Academy. This event was open to
everyone, but mostly white parents showed up. And many of them shared
that they had never really talked about race very much when they were
growing up.

\begin{itemize}
\tightlist
\item
  meghan casey\\
  Show of hands, if race was not talked about, or only minimally talked
  about, or sort of avoided in some way. So just looking around the
  room, it's about half --- no, about 60\% of us.
\end{itemize}

chana joffe-walt

Assistant Principal Meghan Casey walks went through a workshop on race
and racism in America and child development. I think about how just a
few years ago, the buzzword in this very school was diversity. Everyone
is all about celebrating diversity. But now, Meghan Casey tells this
room of parents diversity is not the goal. Having a diverse school does
not mean we have an integrated school. We need to work on that to get to
an integrated school. She says they surveyed BHS students last year,
asking them about their experiences.

\begin{itemize}
\tightlist
\item
  meghan casey\\
  And our white kids overall said it feels like I'm in a Benetton ad,
  and it's so diverse, and lovely, and I'm not experiencing racism, or
  racial bias, or implicit bias here at school. It's great. And our kids
  of color were saying, they feel less loved, less seen. They talked ---
  though they didn't use this language, they talked about stereotype
  threat, they talked about implicit bias. They talked about moments
  with white peers that were uncomfortable, where a friendship felt a
  little strained. And it was clear to them that their white friend just
  didn't --- did not have bad intentions, loved them, good friend, but
  didn't know the harm that they were creating, and just didn't have the
  same knowledge base that they had about race and about racial
  consciousness. I want to just make sure, because it's for whatever
  reason --- I don't know why --- sometimes we think that things are
  better than they are. I just wanted to come back to our students. They
  are reporting that this is urgent, and we need to continue to deal
  with it. And it's not a Benetton ad, even if some of our kiddos think
  it is.
\end{itemize}

chana joffe-walt

It's a little jarring to hear school leaders telling parents, even
though everything looks OK, it's not. Principal Lanzillatto says she
knows it can be hard to hear some of this stuff.

\begin{itemize}
\tightlist
\item
  nicole lanzilltto\\
  And some people are going to feel pissed off about it, and some people
  do. And that means some people are going to leave the room feeling
  like they're being blamed. But at the end of the day, this is about
  kids. This is about serving kids and including families and
  communities. What else is the point of the school, right? That's the
  whole point of a school.
\end{itemize}

chana joffe-walt

Is that the point of a school? When Miss Lanzillatto said this, I got
stuck on the phrase. What is the point of a public school? We don't seem
to have any kind of unified vision. Maybe there was one back when they
made that old film about public schools teaching us about democracy and
how to live together. But we don't have a shared vision now. What we
have is choice. You can choose your vision for a public school. You can
go to the test score school, like Success Academy, or the racial justice
school, like BHS. There is no city policy that says every school needs
to be integrated and equitable. It's up to us. If we want that, we can
choose it. For families with the most power, the most choices, that
means we get to choose. Do we want to play fair or not? At BHS, families
were choosing equity, white advantaged families. I didn't see anyone
leave the room at that parents workshop, or seem upset, or blamed at
all. The parents I met at BHS of all races were pretty happy with the
school. They seemed bought in. Meanwhile, the test scores at be adjust
have improved dramatically. There's still an achievement gap, but it
seems to be closing. Black boys are no longer being disciplined at much
higher rates than everyone else. And the kids seem happy, warm, and
confident, and adept at talking about things like race and power. One
day though, I heard a rumor. It was going around the high school. Kids
were saying the PTA was stealing money from the high school and giving
it to the middle school. I heard it first in the library from a group of
10th graders. They said the PTA had taken \$1,500 to create a garden,
and they were pissed. Later, I heard it again from a tenth grader named
Farzana. And it wasn't \$1,500 anymore.

\begin{itemize}
\tightlist
\item
  farzana\\
  Yeah, so they just received \$15,000 for gardening. What else can that
  \$15,000 be used for so much more?
\end{itemize}

chana joffe-walt

This was meaningful because the BHS middle school is much whiter and
larger than the high school. And despite all the focus on racial equity
for the past few years, the PTA leadership at BHS is now almost all
white, a lot of middle school parents, which has not escaped the notice
of students, who have been encouraged by their school to notice such
things and call them out. A girl named Paola told me we have to keep
watching them because there's no one there representing us. My mom
works. She can't go to PTA meetings.

\begin{itemize}
\item
  paola\\
  It's just very unfair, that the fact that your mom can be in the PTA
  and make all these rules, and be like, no. We want the money from
  middle school.
\item
  jeremiah\\
  Yeah. They're like this all power thing that's above everybody's head
  that can just take this money and do this. You know what I mean?
\end{itemize}

chana joffe-walt

That's Jeremiah jumping in. Jeremiah is a kid who jumps in. He's the guy
you go to if you're feeling angry about something unjust, and what you
want more than anything is someone who will feel just as angry as you
do. Jeremiah tells Paola this is ridiculous. I'm going to go to the PTA
and just tell them straight up.

\begin{itemize}
\item
  jeremiah\\
  You guys need to stop taking, stop taking money from this to put in
  their middle school programs. You know what I mean? It's just too
  much.
\item
  paola\\
  Your middle school already has it enough. Why do you want more?
\end{itemize}

chana joffe-walt

I wasn't sure they had the details exactly right. But I did think, yeah,
here we go again. The mostly white PTA probably is manipulating where
money goes. So I looked into it, and it wasn't true. The PTA did not
steal money from the high school. It did get money for a garden, but it
was grant money, not regular PTA money. Plus, the garden is mostly for
the culinary program, which mostly serves the high school. Jeremiah
texted me a few days after we spoke to say, sorry to bother you, but I
think I might have been a little too critical of the school. Is it
possible to do a follow-up interview? He was mad at himself and his
friends for believing the rumor. He was mad that he said it to me and
looked stupid.

\begin{itemize}
\tightlist
\item
  jeremiah\\
  I think there was some leftover feelings. Honestly, I can't even say
  because ---
\end{itemize}

chana joffe-walt

What do you mean by leftover feelings?

\begin{itemize}
\tightlist
\item
  jeremiah\\
  Because that's been the understanding for five years. You know what I
  mean? It's always been that.
\end{itemize}

chana joffe-walt

It's always been that. It took me a while to get Jeremiah to say more
about what he meant by that. Jeremiah is 15 years old. When he was in
third grade, the city closed his mostly Black school --- called it
failing. His mom, a Black woman, fought the school closing as hard as
she could --- went to every meeting. It happened anyway. The city put a
charter school in the building. And it also opened a new small school
designed to appeal to the newly gentrified neighborhood. It had a global
studies curriculum and a dual language Spanish program. Jeremiah went
there third through fifth grade. Then he went to SIS for middle school,
the year the white kids came in. Suddenly, his science class was
sometimes taught in French. The after school programs he wanted to go
to, also French, which he didn't love, for obvious reasons.

\begin{itemize}
\tightlist
\item
  jeremiah\\
  Because I can't speak French. So that was pretty annoying.
\end{itemize}

chana joffe-walt

Right, Jeremiah, a Black kid, believed a rumor that white parents in the
PTA were stealing from him and his classmates because he understands
that this is how schools work. He has leftover feelings. Jeremiah likes
the new BHS, and he says it does feel more integrated and more equal. I
told him about some of the white parents I had been meeting at the
school, who seemed truly committed to integration.

\begin{itemize}
\item
  jeremiah\\
  I think that for white moms just think --- I think its popular now.
  It's like yoga. It's like, oh, yeah, integration. It's cool now. It's
  a new thing.
\item
  chana joffe-walt\\
  And what do you make of that?
\item
  speaker\\
  Yeah. You're a part of it. Thanks, but are you just --- do you
  genuinely care, or is it everyone's doing it? When it's not beneficial
  to the white families, it's going to be changed. And history repeats
  itself. So when this integration isn't beneficial, then it'll go right
  back to where it was before.
\item
  chana joffe-walt\\
  History repeats itself is a very central thesis of my story.
\item
  jeremiah\\
  Yeah. It's just truth for life.

  When integration is not helpful, it's going to become segregated
  again.
\end{itemize}

chana joffe-walt

That's probably true. White parents are opting in to be at BHS right
now. But they can just as easily opt out. Historically, they have.

When this school building first opened its doors years ago, Black and
Puerto Rican parents were demanding integrated equal schools city wide
for everybody. They weren't asking for one curated school or a small
network of schools where people could integrate, if they wanted to. They
were asking the Board of Education to have a plan for all schools. They
were asking for things to go differently than they have for all of
history. Next time, on ``Nice White Parents,'' things go differently.

``Nice White Parents'' is produced by Julie Snyder and me, with editing
on this episode from Sarah Koenig and Ira Glass. Neil Drumming is our
managing editor. Eve L. Ewing is our editorial consultant. Fact checking
and research by Ben Phelan. Additional reporting from Emmanuel Dzotsi,
Jessica Lussenhop and Alvin Melathe. Music supervision and mixing by
Stowe Nelson with production help from Aviva DeKornfeld. Our director of
operations is Seth Lind. Julie Whitaker is our digital manager. Finance
management by Cassie Howley, and production management by Frances
Swanson. Original music for ``Nice White Parents'' is by The Bad Plus
with additional music written and performed by Matt McGinley. Film
clips, courtesy of the National Education Association and C-span Video
Library. Special thanks to Tina Priceman, Johanna Miller, Leonie
Haimson, Jill Cysner, Clayton Harding, Kate Taylor and Ana Espada. At
The New York Times, thank you to Kelly Doe and Jason Fujikuni. And at
Studio Rodrigo, thanks to Khoi Uong, Becki Choe, Nick Emrich and
Christina No.

``Nice White Parents'' is produced by Serial Productions, a New York
Times Company.

\href{https://www.nytimes3xbfgragh.onion/column/nice-white-parents}{\includegraphics{https://static01.graylady3jvrrxbe.onion/images/2020/07/21/podcasts/nice-white-parents-album-art/nice-white-parents-album-art-square320.jpg}Nice
White Parents}

\hypertarget{episode-four-heres-another-fun-thing-you-can-do-1}{%
\section{Episode Four: `Here's Another Fun Thing You Can
Do'}\label{episode-four-heres-another-fun-thing-you-can-do-1}}

\hypertarget{is-it-possible-to-limit-the-power-of-white-parents-1}{%
\subsection{Is it possible to limit the power of white
parents?}\label{is-it-possible-to-limit-the-power-of-white-parents-1}}

Reported by Chana Joffe-Walt; produced by Julie Snyder; edited by Sarah
Koenig, Neil Drumming and Ira Glass; editorial consulting by Eve L.
Ewing; and sound mix by Stowe Nelson

Transcript

transcript

Back to Nice White Parents

bars

0:00/50:38

-0:00

transcript

\hypertarget{episode-four-heres-another-fun-thing-you-can-do-2}{%
\subsection{Episode Four: `Here's Another Fun Thing You Can
Do'}\label{episode-four-heres-another-fun-thing-you-can-do-2}}

\hypertarget{reported-by-chana-joffe-walt-produced-by-julie-snyder-edited-by-sarah-koenig-neil-drumming-and-ira-glass-editorial-consulting-by-eve-l-ewing-and-sound-mix-by-stowe-nelson-1}{%
\subsubsection{Reported by Chana Joffe-Walt; produced by Julie Snyder;
edited by Sarah Koenig, Neil Drumming and Ira Glass; editorial
consulting by Eve L. Ewing; and sound mix by Stowe
Nelson}\label{reported-by-chana-joffe-walt-produced-by-julie-snyder-edited-by-sarah-koenig-neil-drumming-and-ira-glass-editorial-consulting-by-eve-l-ewing-and-sound-mix-by-stowe-nelson-1}}

\hypertarget{is-it-possible-to-limit-the-power-of-white-parents-2}{%
\paragraph{Is it possible to limit the power of white
parents?}\label{is-it-possible-to-limit-the-power-of-white-parents-2}}

Thursday, August 13th, 2020

\begin{itemize}
\tightlist
\item
  speaker\\
  ``Nice White Parents'' is brought to you by Serial Productions, a New
  York Times Company.
\end{itemize}

chana joffe-walt

I want to tell you about another old film I found during my research.
It's from 1951. We see a housewife, a white woman --- everyone in this
film is white. She's sitting in her living room with some neighbors.
They're here to solve a problem. {[}OLD MOVIE PLAYING{]}

\begin{itemize}
\tightlist
\item
  archived recording\\
  A chain broke on a swing in a public playground, relatively
  unimportant, but a child might have been hurt.
\end{itemize}

chana joffe-walt

They sit on couches with notepads, deep in discussion. They will solve
this problem together. In another scene, a machinist in California
approaches his bus, the factory owner, with a request from the workers.

\begin{itemize}
\tightlist
\item
  archived recording\\
  I'd like to show you the new pension plan that we built. I thought we
  had discussed the pension plan previously.
\end{itemize}

chana joffe-walt

They had. The discussion requires listening, debating and waving your
arms a lot, which they do in the film.

\begin{itemize}
\tightlist
\item
  archived recording\\
  Not yet friends, they may never like each other. But they'll sweat it
  out together. The problem is mutual. Much is involved. Developed
  within each citizen is the Democratic spirit, the Democratic method.
\end{itemize}

chana joffe-walt

Where were they taught the Democratic method, you might ask? Public
school, where the housewives taught to problem solve for the safety of
the community's children, public school. This film was made by the
National Education Association. It's a 25-minute promotional film that
spends almost no time inside schools. Instead, it's all about the
purpose of public schools, how they prepare us to live together as
citizens. We see Americans use their public school training in everyday
life, when they sit with their neighbors, debate their bosses, when they
go shopping, and drive a car, buy a house. We are all part of a grand
play, interdependent --- the senator, the homemaker, the factory worker.

\begin{itemize}
\tightlist
\item
  archived recording\\
  And Fred Gorman, the farmer of Pennsylvania, are his decisions
  important? They are, if the nation wants to eat.
\end{itemize}

chana joffe-walt

Fred the farmer has a nameless wife, whom we see now standing next to
him. Fred's wife is trying to resolve a problem. The neighbors want to
build a drainage system into a pond. The lowest land for the pond is an
orchard that belongs to Fred and his nameless wife. The wife understands
that to prevent further flooding, she and Fred will need to sacrifice
for the greater good. Fred is not so sure.

\begin{itemize}
\tightlist
\item
  archived recording\\
  If we don't do something to help, our land is going to get like
  theirs, and you know it. I don't like the idea of losing those trees.
  There's the problem in a nutshell, a tough one to crack, his land and
  his neighbors needs. A dictator could solve it for Fred, but he
  prefers to do his own thinking.
\end{itemize}

chana joffe-walt

Luckily, Fred has the tools to do it.

\begin{itemize}
\tightlist
\item
  archived recording\\
  Those tools are sharpened in the schools of America.
\end{itemize}

chana joffe-walt

And thank goodness, because the stakes are high.

\begin{itemize}
\tightlist
\item
  archived recording\\
  Problems every day, and the way they are solved determines the way the
  country functions.
\end{itemize}

chana joffe-walt

This vision of public schools, the same one laid out 100 years earlier
by the founder of American public schools Horace Mann is that America
and democracy cannot survive without public education. We need common
schools where rich and poor come together to solve problems, generate
fellow feeling. Public schools, the great equalizer. But I have made my
way through the history of one modern American public school. And from
what I can see, white parents are standing in the way of achieving this
vision. Our schools are not an equalizing force, because white parents
take them over and hoard resources. We're not learning how to live
together as one society because white parents flee or cordon themselves
off in special gifted programs. Even when we're not in the school
building, funding and attention still slide our way. So I don't see how
it's possible to have equal public schools, common schools that serve
every child, unless we limit the power of white parents. But how do we
do that? In all my reporting around this one school building from 2015
all the way back to the beginning, I've never seen that happen. And then
I did. From Serial Productions, I'm Chana Joffe-Walt. This is ``Nice
White Parents,'' a series about the most powerful force in public
schools: White people.

Recently, I've come across two examples of schools that seem to be
suppressing the power of white parents, two examples I found in the very
last place I expected, in the I.S. 293 building, one upstairs and one
downstairs. So today's episode, what does it look like to limit the
power of white parents in schools? And does it work? Does it lead to an
equal education for everyone?

I'm going to start downstairs. Eight years ago, the city put a charter
school in the basement if I.S. 293. It's called Success Academy. That
year I spent following the new white parents upstairs at the School For
International Studies, I would occasionally see Success Academy kids
around the building in orange and blue uniforms. It was always a little
startling because Success Academy is an elementary school. So they look
tiny in a building full of middle and high schoolers. But mostly, the
Success kids stood out because of the way they moved through the halls.

\begin{itemize}
\tightlist
\item
  speaker\\
  They walk in single forms like they're in the Army. It's so weird. If
  they don't walk in single form, they stop the whole line.
\end{itemize}

chana joffe-walt

Denagee is one of the many students from SIS upstairs, who is eager to
tell me about the charter school and its rituals, their silent,
controlled lines.

\begin{itemize}
\item
  denagee\\
  It's like a sense of the Children of the Corn. It creeps me out.
\item
  chris\\
  Or Storm Troopers or something.
\end{itemize}

chana joffe-walt

That's his friend Chris saying Storm Troopers. Sometimes they'll hear
Success teachers say, make a bubble in your mouth. And then a line of
six-year-olds will close their lips and fill their cheeks up with air,
that way nobody's talking. Chris and Denagee told me they look like
puffer fish.

\begin{itemize}
\tightlist
\item
  denagee\\
  I remember kindergarten very vividly. And I know if I was to have my
  face in a puffer fish, I would automatically just start making all
  types of sounds and stuff, you know?
\end{itemize}

chana joffe-walt

And they don't?

\begin{itemize}
\item
  denagee\\
  No. That's what's so weird.

  It helps me think, or it makes me think about --- what really happens
  inside of the classrooms with them to be coming out like that?
\end{itemize}

chana joffe-walt

The year I was reporting at SIS, The New York Times published a video
that showed a particular and alarming moment inside one of the Success
classrooms. It was secretly recorded by an assistant teacher, who leaked
it, and it went viral. You see a group of first-graders gathered in a
circle on a polka-dot rug, sitting legs crossed, hands in their laps.
And a teacher is asking one girl to correct a math problem she got
wrong.

\begin{itemize}
\tightlist
\item
  speaker\\
  You cut or you split. So count it again, making sure you're counting
  correctly.
\end{itemize}

chana joffe-walt

The girl does not respond. The teacher leans in and repeats.

\begin{itemize}
\tightlist
\item
  speaker\\
  Count.
\end{itemize}

chana joffe-walt

The girl whimpers, or says something so quiet you can't hear. The entire
class is watching. It's silent, intense. The teacher is visibly upset,
picks up the child's paper, and rips it in half, points an angry finger
to the side of the room.

\begin{itemize}
\tightlist
\item
  speaker\\
  Go to the calm down chair and sit.
\end{itemize}

chana joffe-walt

She goes. The teacher turns to the rest of the circle.

\begin{itemize}
\tightlist
\item
  speaker\\
  There's nothing that infuriates me more than when you don't do what's
  on your paper. Somebody come up and show me how she should have
  counted to get her answer that was 1 and a split.
\end{itemize}

chana joffe-walt

A boy rushes over to do it correctly. But the teacher is not done
publicly reprimanding the girl, who's now sitting to the side of the
classroom in the calm down chair.

\begin{itemize}
\tightlist
\item
  speaker\\
  Thank you. Do not go back to your seat and show me one thing and then
  don't do it here. You're confusing everybody. Very upset and very
  disappointed.
\end{itemize}

chana joffe-walt

The teacher is white. The girl whose work she just ripped up is not. The
whole thing is hard to watch. When this video came out, the student was
living at a homeless shelter with her mom. Success suspended this
teacher, but didn't fire her. Instead, at a press conference, the C.E.O
reprimanded The New York Times for not understanding that this teacher
was having a bad day. When I asked about this incident, the C.E.O of
Success told me the teacher's behavior was unacceptable. Teachers are
not allowed to yell at kids. But it was not a fireable offense. She says
the teacher made a mistake.

I have always been skeptical of Success Academy. Success has a
reputation for being harsh and punitive. Especially unnerving, to me at
least, is that they're harsh, punitive approach is deployed in schools
across the city that are almost entirely BBlack and brown. Success
students are generally kids of color from working class or poor
families. The intense focus on policing kids' bodies, on test prep
drills, frequent use of suspensions --- you don't see that in majority
white schools. I've never seen a line of uniformed, white students
walking through the halls of a public school building with their mouths
in bubbles, or being told to quote show urgency when they dawdle
unpacking their book bags or eating lunch, except for here. This
particular Success Academy in the basement of I.S. 293 is integrated. A
quarter of the student body is white. And it's the first school I saw
putting limits on the power of white parents.

Success Academy is the city's largest charter school network, 47
schools, elementary, middle, and one high school. They get public
funding, like all charter schools. Success Academy also gets private
funding. The state oversees charters, like Success. But it isn't run by
the state or the city. It's run by a private organization. And Success
is a choice school. That means families opt in to Success. The C.E.O, a
woman named Eva Moskowitz, opened her first 40 something schools in
largely working class Black and brown neighborhoods where she imagined
families would want a new school option. Then about a decade ago,
Moskowitz decided she wanted to open an integrated school, a new Success
Academy that was racially integrated and economically diverse. She
needed a school building where integration was possible, where perhaps,
half a century earlier, a group of white families pushed for a
strategically located fringe school building between two racially
segregated neighborhoods. And this is how Success Academy wound up here,
in the old I.S. 293 building because of yet another plan to integrate.
Only this time, it worked. White parents opted in. The way families at
this Success Academy --- it's called Success Academy Cobble Hill ---
tend to come from advantage, just like the white parents upstairs at
SIS. They're upper middle class and rich, doctors and lawyers, corporate
accountants, people who walk into most public schools with a lot of
power. But the influence I'd seen white parents wield upstairs at SIS,
that didn't seem to be the case downstairs. I found that confusing. Do
you have a PTA? We have a parent council so it's very similar to a PTA
this is Alissa Bishop, the principal of Success, Cobble Hill. The parent
council is not that similar to a PTA though, because in the very next
sentence, Principal Bishop told me that the parent council is not
allowed to raise money. This, I assumed, was probably difficult for
parents who are accustomed to fundraising for their kids' schools. Have
you had parents who want to raise money, who come to you and are like, I
want to --- I want this thing to happen, and I want to raise the money
for it.

\begin{itemize}
\item
  alissa bishop\\
  Not anything like that. I have had parents come to me and say I want
  to do a coat drive. They want to donate. We do that stuff throughout
  the entire year. But I've never had anyone approach me about donating
  money.
\item
  speaker\\
  Oh, wow.
\item
  chana joffe-walt\\
  Really? No parents have been like I want to do a fundraiser for x, and
  you have to be like, that's not a thing that we do?
\item
  alissa bishop\\
  No. I've never had that. Principal Bishop looks over at the PR person,
  who's come from Success headquarters to supervise this interview. The
  PR person shakes her head. No, parents don't raise money.
\item
  chana joffe-walt\\
  And what if somebody did want to raise money for the school --- a
  parent wanted to raise money?
\item
  alissa bishop\\
  Yeah, we don't.
\item
  speaker\\
  It's against the policy.
\item
  alissa bishop\\
  We don't raise money.
\end{itemize}

chana joffe-walt

Principal Bishop looks over to the PR person again, as if to say, am I
not being clear with this chick? Why isn't she getting it? But I seem to
be unable to stop myself from listing all the things I've seen white
advantaged parents demand in public schools.

\begin{itemize}
\item
  chana joffe-walt\\
  If parents were like, we want this to be a dual language French
  school, and we can help fund it ---
\item
  alissa bishop\\
  We don't have --- our curriculum is network-based. We're given
  curriculum. We don't have a language curriculum in our elementary
  schools.
\item
  chana joffe-walt\\
  Or, if parents were like, we want there to be less math, or a
  different kind of math, or we want there to be a film program, a film
  program, or whatever, any of those things. Parents are like, we want
  ---
\item
  alissa bishop\\
  Yeah, this is our model. It's our model across all of our schools. No
  changes.
\end{itemize}

chana joffe-walt

The C.E.O. of Success Academy, Eva Moskowitz, followed up later to tell
me, if parents want to give money, they can. But it will be distributed
evenly across all of our schools. We can't have our Cobble Hill families
getting more than our families in Harlem.

Here's what I started to understand about how Success Academy was
limiting the power of white parents. Success was limiting the power of
white parents by limiting the power of all parents. I met a dad named
Travius Sharpe outside the building one day, a Black guy who grew up in
Brooklyn. His son Ethan is at Success.

\begin{itemize}
\item
  travius sharpe\\
  We actually get graded.
\item
  chana joffe-walt\\
  You get graded?
\item
  travius sharpe\\
  We get graded as the parents. We get an email saying, this is what
  your progress is saying.
\item
  chana joffe-walt\\
  You get a grade, like an A, B, C?
\item
  travius sharpe\\
  They get like a meeting expectations or ---
\item
  chana joffe-walt\\
  Not.
\item
  travius sharpe\\
  --- not.
\end{itemize}

chana joffe-walt

Upstairs, SIS tripped over itself to meet the demands of new, white
parents. Downstairs, all parents at Success Academy are being graded.
Even day to day, the Success principal and teachers make sure to remind
parents when they're falling down on the job.

\begin{itemize}
\item
  speaker\\
  So we running a little late. Why is Ethan late? It's your fault why
  he's late. I think one day, I was late. And she texted and said, Ethan
  is not here yet. Any reason why? And I felt like I wasn't the parent
  at that point. That's their time. But it keeps you on your toes and
  stuff like that.
\item
  chana joffe-walt\\
  You felt like you weren't the parent.
\item
  speaker\\
  I wasn't the parent. I wasn't the parent. And I felt like I was just
  dropping this kid off to his parents.
\end{itemize}

chana joffe-walt

One day, in the cafeteria, I met a white mom named Sarah Stanich. Sarah
is a financial advisor. Her son's in fourth grade. And she was telling
me she likes the school, even though --- and then Sarah lowered her
voice, pointed at her boy, and said, he's been suspended.

\begin{itemize}
\item
  sarah stanich\\
  He's been suspended. And I was not happy about that. And I definitely
  never had that experience when I was a kid. But ---
\item
  chana joffe-walt\\
  How old was he when that happened?
\item
  sarah stanich\\
  Well, it's happened more than once, embarrassingly.

  So kind of young. Maybe third grade, or maybe even second the first
  time it happened.
\item
  chana joffe-walt\\
  How many times has he been suspended?
\item
  sarah stanich\\
  A few times, a few times, probably three --- but pushing, fighting.
  And he's really not a fighter, but they're boys, and --- and sometimes
  I think it's kind of harsh. They're young kids. And I know that that's
  a complaint about suspensions in the schools. But on the other hand,
  he had warnings, and it wasn't that I think that his teachers had
  given him given him space and slack in other areas. I have no
  lingering anger about it.
\item
  chana joffe-walt\\
  Overall.
\item
  sarah stanich\\
  Yeah. Overall, I've been I feel very lucky to have been able to be a
  part of this community and be part of this school. Sarah later wrote
  me to say her kid was actually suspended four times that year. I've
  reported on discipline in schools and the use of suspensions a lot.
  I've talked to many mothers of children who have been suspended. Not
  one of them has been white. Black kids are suspended in New York City
  schools at five times the rate of white kids. After I met Sarah, I
  double checked the numbers for the 2017 school year, just to be sure.
  In the regular New York City public schools that same year, not
  Success or other charters, but the traditional, public elementary
  schools, that year, there were 327 suspensions for non-white kids. For
  white kids, there were only nine. I was so surprised after meeting
  Sarah when I left the building I called two people, who know a lot
  about education to say this is what's happening at Success Academy,
  Cobble Hill. White boys are being suspended, rich, white boys. And
  they couldn't believe it either. One of them, Noliwe Rooks, a
  professor at Cornell, said, well, well, how's that for equality?

  So white parents can't raise money, they can't ask for special
  programs, and their kids get suspended. Why are they suddenly OK with
  equality? I interviewed lots of Success parents.
\item
  suzanne gigliotti\\
  We did get a flyer. They put them on the doors. They put them on the
  doors in the neighborhood.
\end{itemize}

chana joffe-walt

Suzanne Gigliotti saw the flyer for Success when her son was in
preschool. So she looked into it and every other possible school option
she had.

\begin{itemize}
\item
  suzanne gigliotti\\
  It was in our neighborhood. But more importantly, we toured so many
  schools, public, private, parochial. We were slated for 58, which is
  an excellent school. And we did get in there. But Success was head and
  above any school I'd seen, just the level of excellence. And yeah,
  nothing matched it. The test scores --- almost every parent I spoke
  with said they were initially drawn to Success Academy because of the
  excellent test scores. If your measure of success in school is
  standardized tests --- and at Success Academy, it is --- this is one
  of the best schools in the city. The scores are truly remarkable.
  Success Academy students perform twice as well on state tests as
  regular New York City public school kids. The vast majority of Success
  kids pass the tests, 95\%, 97\%. In your average city public schools,
  it's less than half. And even more impressive, to me at least, is that
  the kids at Success are doing well on tests no matter if they're poor,
  or rich, or Black, or Latino, or Asian, or white. This is the problem
  that decades of public education reforms have tried to address, the
  achievement gap. Success Academy was pulling off, not only an
  integrated school, but an equal integrated school that was closing the
  achievement gap.

  The way Success achieves equality though, some things give me pause.
\item
  speaker\\
  What's my first expectation? Lock your hands. Track Kamira. The first
  expectation is read ---
\end{itemize}

chana joffe-walt

Last year, I went into the Success classrooms.

\begin{itemize}
\tightlist
\item
  speaker\\
  Send Kamira some love. Give Kamira two claps. {[}CLAPPING{]} My
  expectation is that ---
\end{itemize}

chana joffe-walt

I didn't see any teachers reprimanding kids or ripping up work, like the
one in the video. What I did see were teachers, who issued a constant
wall of verbal directions, where to look, what to do, how to sit,
delivered in the same and consistent, neutral tone. When a teacher calls
on someone, she gives a direction to the class to track the speaker,
look at the person speaking. Meanwhile, a second teacher roams and
hovers, issuing reminders.

\begin{itemize}
\tightlist
\item
  speaker\\
  Lock your hands. Track Shana. Liam's hands are locked tracking Shana.
  Lydia's hands are locked tracking Shana. Colin's hands are ---
\end{itemize}

chana joffe-walt

Shana answers correctly.

\begin{itemize}
\tightlist
\item
  speaker\\
  Nice job, Shana. Nice job, Shana. Scanning for another friend on the
  carpet, who looks so professional, lock your hands, track Zoe.
\end{itemize}

chana joffe-walt

Success achieves equality, at least in part, through utter uniformity.
Every Success Academy across the city uses identical methods, identical
curricula, and identical classrooms. The kids sit on the same polka-dot
carpet, hands locked in their lap, same signs on the wall, singing the
same chants. Even the teachers look the same. They're almost all young
white women in cotton dresses and ballet flats just out of college,
sometimes the same college. I know this because the classrooms are named
after teachers alma maters. And there are three Penn State classrooms.

\begin{itemize}
\tightlist
\item
  speaker\\
  How does it go? We are Penn State. Yeah, we pull our weight Yeah, we
  cannot wait. Yeah, to graduate. Roll call. It's uh-huh. And then
  Shabooya, sha-sha-shabooya, role call. Shabooya, sha-sha-shabooya.
\end{itemize}

chana joffe-walt

Education people talk a lot about the difference between equality and
equity to a point that I believe is tiresome. But I thought about this
difference a lot at Success. Equality means everyone gets the same
thing. Equity means everyone gets what they need. Success is equal.
Everyone is treated the same. But kids are never all the same. Some kids
are chatty in the hallway, or need a minute to think before answering a
question. Some kids have a million bucks at home, and some kids don't. A
Black girl might respond differently than a white girl to being
reprimanded by a white teacher. A single parent with two jobs might have
a harder time getting their kid to school on time than, say, a
stay-at-home mom with a partner. One of the main criticisms of Success
Academy from public education advocates is that Success doesn't actually
serve all students, that it has excellent test scores because it serves
a select group of students. Kids who don't test well, or can't sit
still, they're weeded out of the school. Success Academy vehemently
denies this. They point out that they make special accommodations for
kids with special needs, and they note that they don't get to choose
students because kids get spots in their schools by random lottery. And
that's true. But it's also true that lots of parents don't apply to the
lottery because they know the school's culture and the demands it makes
of families won't work for them. And plenty of kids who do end up at
Success don't last long. Maybe they get held back a grade or they're
suspended. A civil rights complaint filed on behalf of more than a dozen
families alleges their children were regularly removed from class and
suspended, seven, 10, 13 times at Success Academy. Most of those
families eventually left the school.

I had a thought walking through Success. I suspected that the strict
classroom control was partly what made white parents feel comfortable at
Success Academy. I'm speculating here. None of the white parents I spoke
with told me they chose Success because the school polices Black and
brown students so well. And I don't believe this is a conscious thought
for anyone. But I do know that white parents bring plenty of unconscious
biases to public schools with Black and brown kids, fears that the
classrooms will be chaotic, or not challenging, that the kids will be
disorderly or threatening. White parents worry that our kids will be
harmed. Success Academy completely controls for these fears. Everyone
gets excellent test scores. There's no room for misbehavior, no risk of
disruption because there are no idle moments. If 30 children need to
move from their desks to the rug, it sounds like this.

\begin{itemize}
\tightlist
\item
  speaker\\
  On your bottom, on the black line in five, four, three, two, one.
\end{itemize}

chana joffe-walt

Every kid is on their bottom, hands locked, eyes tracking the teacher,
except for one boy. He gets a correction.

Success operates on the principle that with rigor and discipline
uniformly applied, all students will achieve equally well. It's a
tempting vision, especially coming from upstairs, where the power of
white parents seem to have no bounds. But equality does not necessarily
shift the balance of power. White parents aren't running the show here,
but Success is run by a white C.E.O and a board that includes
millionaire hedge fund managers --- sorry, billionaire hedge fund
managers. The board of trustees is listed on the success website. And
the bios include Maverick Capital, Redwood Capital, Glenview Capital,
Cumulus Media, Morgan Stanley, Facebook, Arnold \& Porter. This is not
exactly a disruption to the social order, is all I'm saying. You can
limit the day-to-day influence of white parents. But still, rich white
people control the agenda, the priorities, and the money.

Back in 2015, the year of the white influx and SIS, toward the end of
that school year, I was talking to Imee Hernandez one day. She was the
PTA co-president of SIS. And Imee told me watching all those white
parents come and take over, it was almost like watching tumbleweed move
along in the wind. It was so quiet. That's how they moved through here,
she said, picking up power as they went.

\begin{itemize}
\tightlist
\item
  imee hernandez\\
  Like a tumbleweed, it starts really soft and slow, and it keeps just
  picking up speed and getting bigger. So it's really soft and slow. But
  it's getting bigger. It's not like an avalanche that comes at you.
  It's just tumbling along very slowly. So it's very light. You don't
  feel it coming at you.
\end{itemize}

chana joffe-walt

Back then, Imee told me there's no stopping it. She's worried she
couldn't protect what she loved about her school.

\begin{itemize}
\item
  chana joffe-walt\\
  If you were right, and the worst case scenario happens, what does that
  look like in a year or two?
\item
  imee hernandez\\
  That there's no more color in this school.

  Then there's no more community, which I really hope I'm wrong. That's
  my biggest fear. Then I would question if my daughter's coming back. I
  really would.
\end{itemize}

chana joffe-walt

Imee feared that each year, more and more white families come into SIS
until it just became like the other segregated middle schools, where all
the white parents fought to enroll their kids. Against the repetition of
history, Imee was wrong. What happened at SIS was nothing like she or I
expected. That's up next when we go back upstairs.

This past spring, a Black teacher at Success Academy named Fabiola St
Hillaire publicly criticized the C.E.O. for not taking a stand after the
murder of George Floyd, or acknowledging the effect police violence was
having on the families and communities Success serves. After that, more
staff, families, and alumni raised alarms about Success, calling some of
its practices racist and abusive, its discipline policies, the way white
staff and leadership speak to kids and parents of color. In response,
the C.E.O. apologized, and Success has released a plan that commits to
mandatory bias and sensitivity training for staff. The plan says they
will create an Equity Team and review their culture, their relationships
with staff, and families, and kids with quote, ``an attention and
sensitivity to race.'' I read this plan and thought, huh, there is a
school that's already doing many of these things, right in the same
building, right upstairs.

\begin{itemize}
\tightlist
\item
  speaker\\
  OK, welcome. Hello. Thank you.
\end{itemize}

chana joffe-walt

It's September, 2019. I'm back at SIS. It's been four years since the
French gala and the drama with the PTA. Rob, the dad, who fundraises,
he's not here anymore. His son finished middle school. Imee is still
here. Her daughter is a junior in high school. And a new crop of sixth
graders and their families are settling into the auditorium.

\begin{itemize}
\tightlist
\item
  speaker\\
  Welcome to PA Chaz. Welcome. Please, find a seat for me. Thank you.
\end{itemize}

chana joffe-walt

The school is no longer called SIS, the School for International
Studies. It's now BHS, the Boerum Hill School for International Studies.
They changed the name, again. BHS has a new principal, Nicole
Lanzillatto. She gets up on stage, and the staff cheers. Miss
Lanzillatto welcomes the new families to BHS.

\begin{itemize}
\tightlist
\item
  nicole lanzillatto\\
  Any school is a microcosm of the world, and we are blessed with
  beautiful diversity.
\end{itemize}

chana joffe-walt

Miss Lanzillatto lists the ways the school reflects the world, race,
ethnicity, language, gender.

\begin{itemize}
\tightlist
\item
  nicole lanzillatto\\
  We are an extraordinarily diverse community. And it's a beautiful
  thing, and we fight for it, and we work on it.
\end{itemize}

chana joffe-walt

Miss Lanzillatto says BHS is going for true equity. She says the word
equity three times in this welcome speech. Miss Lanzillatto is white,
chatty, well-liked, with black hair that's styled straight up. The hair
is really Miss Lanzillatto's defining feature. Picture boy band
pompadour. She's worked here most of her professional career. The year
white families arrived at SIS, Miss Lanzillatto was the assistant
principal. She won't say anything bad about that year. It was a learning
experience. It's a process her predecessor, Ms. Juman, talks about it
the same way. Remember? Principals --- diplomatic. They're careful not
to place blame, but both of them said after that year, it was clear they
needed to intervene. One of the first things Miss Lanzillatto did as
principal was request special permission to reserve 40\% of the seats
for kids who get free and reduced price lunch. The majority of kids who
get free and reduced price lunch are kids of color. And Miss Lanzillatto
didn't want the school to flip. She didn't want Black and brown kids to
get pushed out. The assistant principal told me they wanted to make sure
the school did not become colonized. Some things here have changed. They
got rid of the foundation, the Brooklyn World Project Rob and the other
white parents had created. They scrapped some of the French programming,
hired more teachers and staff of color. And one of the most striking
changes I noticed --- spend 10 minutes of the school, and you can't not
notice --- Miss Lanzillatto is talking directly and constantly about
race and equity. She told me everyone here needs to be on alert for
racist habits and ideas. They need to aggressively address them,
whenever they pop up, in the cafeteria, in the classroom.

\begin{itemize}
\tightlist
\item
  nicole lanzillatto\\
  There's a conversation happening in the school around the smart
  classes and the non-smart classes. Let's talk about it where is that
  coming from. So I think it's really about being a beast. I think it's
  about everything we do coming back to it.
\end{itemize}

chana joffe-walt

Coming back to equity. I could not get over how much time and energy the
school puts into ensuring equity, not equality, equity. It's almost like
the obsessive focus Success puts on making sure everything is the same
is exactly matched by the obsessive focus BHS just puts on recognizing
everyone is not the same. BHS formed an Equity Committee of staff and
students a few years ago. They looked for bias in the curriculum, in the
signs on their walls and the books on their shelves. They analyzed
achievement data, discipline data, where they could clearly see that the
school punished Black boys more harshly than other students. So they
revamped their entire approach to discipline, created a restorative
Justice Department. They applied for grants to help pay for this to
train their teachers on implicit bias and then train them again. They
brought in experts.

\begin{itemize}
\tightlist
\item
  speaker\\
  And here are some things that I look for in transition. So how do kids
  engage with each other? Is it verbal engagement? Is it non-verbal
  engagement?
\end{itemize}

chana joffe-walt

Last fall, I watched two equity consultants, Cornelius and Kass Minor,
show a group of BHS teachers how to observe racial dynamics in their
school. This involved teachers walking around in a huddle with
clipboards, taking diligent notes as kids walk through the hallways.

\begin{itemize}
\tightlist
\item
  speaker\\
  One fun lens to look at --- and I'm just naming things out --- I often
  ask, what are boys doing? What are girls doing? What are Black
  students doing? What are students of color doing?
\end{itemize}

chana joffe-walt

Mr. Minor is full of fun things the teachers should look for.

\begin{itemize}
\tightlist
\item
  speaker\\
  Here's another fun thing to do, just because we're out here. I do
  drive-bys in the hallway, where I walk by classroom windows, and I
  look in.
\end{itemize}

chana joffe-walt

They all take turns peering through the small window of a classroom
door. They take more notes. Later, the teachers meet as a group. And one
teacher, Stacy Ann Manswell, explains her observations from a math
class.

\begin{itemize}
\tightlist
\item
  stacy ann manswell\\
  And then in the math classroom that we were in, something that stood
  out to me --- so there was two white males, white female, Black male.
  And I'm walking around. And Black male, he was finished. And he
  finished early, waiting for his peers to do the Think Right Pair
  Share. And when the timer went off, the girl, the white girl he was
  sitting next to, he looked to her, but she looked to the two white
  boys. And they formed the pair. So it was like, now she had to work
  with him. But she was sort of looking for the other two boys for
  validation for what this boy was saying. So my teacher self is like,
  OK, does this child not participate in class, and she doesn't trust
  that he knows what he's doing, or is it because she doesn't see him
  because he's a Black boy and she figures he's not capable?
\end{itemize}

chana joffe-walt

The teachers talked about this moment in depth, what it might mean, what
messages the kids were picking up in their school about race, about
who's important, who's bad, who's smart. And it's not just the staff.
The administration is telling white parents that their mere presence in
the school does not make it integrated. They have to work at making this
place fair.

\begin{itemize}
\tightlist
\item
  meghan casey\\
  So this is our agenda today. We're going to start with a reflection,
  and we're going to get into how we talk about race with our young
  people.
\end{itemize}

chana joffe-walt

One Saturday morning, a group of two dozen parents gathered in the BHS
library for something called Family Academy. This event was open to
everyone, but mostly white parents showed up. And many of them shared
that they had never really talked about race very much when they were
growing up.

\begin{itemize}
\tightlist
\item
  meghan casey\\
  Show of hands, if race was not talked about, or only minimally talked
  about, or sort of avoided in some way. So just looking around the
  room, it's about half --- no, about 60\% of us.
\end{itemize}

chana joffe-walt

Assistant Principal Meghan Casey walks went through a workshop on race
and racism in America and child development. I think about how just a
few years ago, the buzzword in this very school was diversity. Everyone
is all about celebrating diversity. But now, Meghan Casey tells this
room of parents diversity is not the goal. Having a diverse school does
not mean we have an integrated school. We need to work on that to get to
an integrated school. She says they surveyed BHS students last year,
asking them about their experiences.

\begin{itemize}
\tightlist
\item
  meghan casey\\
  And our white kids overall said it feels like I'm in a Benetton ad,
  and it's so diverse, and lovely, and I'm not experiencing racism, or
  racial bias, or implicit bias here at school. It's great. And our kids
  of color were saying, they feel less loved, less seen. They talked ---
  though they didn't use this language, they talked about stereotype
  threat, they talked about implicit bias. They talked about moments
  with white peers that were uncomfortable, where a friendship felt a
  little strained. And it was clear to them that their white friend just
  didn't --- did not have bad intentions, loved them, good friend, but
  didn't know the harm that they were creating, and just didn't have the
  same knowledge base that they had about race and about racial
  consciousness. I want to just make sure, because it's for whatever
  reason --- I don't know why --- sometimes we think that things are
  better than they are. I just wanted to come back to our students. They
  are reporting that this is urgent, and we need to continue to deal
  with it. And it's not a Benetton ad, even if some of our kiddos think
  it is.
\end{itemize}

chana joffe-walt

It's a little jarring to hear school leaders telling parents, even
though everything looks OK, it's not. Principal Lanzillatto says she
knows it can be hard to hear some of this stuff.

\begin{itemize}
\tightlist
\item
  nicole lanzilltto\\
  And some people are going to feel pissed off about it, and some people
  do. And that means some people are going to leave the room feeling
  like they're being blamed. But at the end of the day, this is about
  kids. This is about serving kids and including families and
  communities. What else is the point of the school, right? That's the
  whole point of a school.
\end{itemize}

chana joffe-walt

Is that the point of a school? When Miss Lanzillatto said this, I got
stuck on the phrase. What is the point of a public school? We don't seem
to have any kind of unified vision. Maybe there was one back when they
made that old film about public schools teaching us about democracy and
how to live together. But we don't have a shared vision now. What we
have is choice. You can choose your vision for a public school. You can
go to the test score school, like Success Academy, or the racial justice
school, like BHS. There is no city policy that says every school needs
to be integrated and equitable. It's up to us. If we want that, we can
choose it. For families with the most power, the most choices, that
means we get to choose. Do we want to play fair or not? At BHS, families
were choosing equity, white advantaged families. I didn't see anyone
leave the room at that parents workshop, or seem upset, or blamed at
all. The parents I met at BHS of all races were pretty happy with the
school. They seemed bought in. Meanwhile, the test scores at be adjust
have improved dramatically. There's still an achievement gap, but it
seems to be closing. Black boys are no longer being disciplined at much
higher rates than everyone else. And the kids seem happy, warm, and
confident, and adept at talking about things like race and power. One
day though, I heard a rumor. It was going around the high school. Kids
were saying the PTA was stealing money from the high school and giving
it to the middle school. I heard it first in the library from a group of
10th graders. They said the PTA had taken \$1,500 to create a garden,
and they were pissed. Later, I heard it again from a tenth grader named
Farzana. And it wasn't \$1,500 anymore.

\begin{itemize}
\tightlist
\item
  farzana\\
  Yeah, so they just received \$15,000 for gardening. What else can that
  \$15,000 be used for so much more?
\end{itemize}

chana joffe-walt

This was meaningful because the BHS middle school is much whiter and
larger than the high school. And despite all the focus on racial equity
for the past few years, the PTA leadership at BHS is now almost all
white, a lot of middle school parents, which has not escaped the notice
of students, who have been encouraged by their school to notice such
things and call them out. A girl named Paola told me we have to keep
watching them because there's no one there representing us. My mom
works. She can't go to PTA meetings.

\begin{itemize}
\item
  paola\\
  It's just very unfair, that the fact that your mom can be in the PTA
  and make all these rules, and be like, no. We want the money from
  middle school.
\item
  jeremiah\\
  Yeah. They're like this all power thing that's above everybody's head
  that can just take this money and do this. You know what I mean?
\end{itemize}

chana joffe-walt

That's Jeremiah jumping in. Jeremiah is a kid who jumps in. He's the guy
you go to if you're feeling angry about something unjust, and what you
want more than anything is someone who will feel just as angry as you
do. Jeremiah tells Paola this is ridiculous. I'm going to go to the PTA
and just tell them straight up.

\begin{itemize}
\item
  jeremiah\\
  You guys need to stop taking, stop taking money from this to put in
  their middle school programs. You know what I mean? It's just too
  much.
\item
  paola\\
  Your middle school already has it enough. Why do you want more?
\end{itemize}

chana joffe-walt

I wasn't sure they had the details exactly right. But I did think, yeah,
here we go again. The mostly white PTA probably is manipulating where
money goes. So I looked into it, and it wasn't true. The PTA did not
steal money from the high school. It did get money for a garden, but it
was grant money, not regular PTA money. Plus, the garden is mostly for
the culinary program, which mostly serves the high school. Jeremiah
texted me a few days after we spoke to say, sorry to bother you, but I
think I might have been a little too critical of the school. Is it
possible to do a follow-up interview? He was mad at himself and his
friends for believing the rumor. He was mad that he said it to me and
looked stupid.

\begin{itemize}
\tightlist
\item
  jeremiah\\
  I think there was some leftover feelings. Honestly, I can't even say
  because ---
\end{itemize}

chana joffe-walt

What do you mean by leftover feelings?

\begin{itemize}
\tightlist
\item
  jeremiah\\
  Because that's been the understanding for five years. You know what I
  mean? It's always been that.
\end{itemize}

chana joffe-walt

It's always been that. It took me a while to get Jeremiah to say more
about what he meant by that. Jeremiah is 15 years old. When he was in
third grade, the city closed his mostly Black school --- called it
failing. His mom, a Black woman, fought the school closing as hard as
she could --- went to every meeting. It happened anyway. The city put a
charter school in the building. And it also opened a new small school
designed to appeal to the newly gentrified neighborhood. It had a global
studies curriculum and a dual language Spanish program. Jeremiah went
there third through fifth grade. Then he went to SIS for middle school,
the year the white kids came in. Suddenly, his science class was
sometimes taught in French. The after school programs he wanted to go
to, also French, which he didn't love, for obvious reasons.

\begin{itemize}
\tightlist
\item
  jeremiah\\
  Because I can't speak French. So that was pretty annoying.
\end{itemize}

chana joffe-walt

Right, Jeremiah, a Black kid, believed a rumor that white parents in the
PTA were stealing from him and his classmates because he understands
that this is how schools work. He has leftover feelings. Jeremiah likes
the new BHS, and he says it does feel more integrated and more equal. I
told him about some of the white parents I had been meeting at the
school, who seemed truly committed to integration.

\begin{itemize}
\item
  jeremiah\\
  I think that for white moms just think --- I think its popular now.
  It's like yoga. It's like, oh, yeah, integration. It's cool now. It's
  a new thing.
\item
  chana joffe-walt\\
  And what do you make of that?
\item
  speaker\\
  Yeah. You're a part of it. Thanks, but are you just --- do you
  genuinely care, or is it everyone's doing it? When it's not beneficial
  to the white families, it's going to be changed. And history repeats
  itself. So when this integration isn't beneficial, then it'll go right
  back to where it was before.
\item
  chana joffe-walt\\
  History repeats itself is a very central thesis of my story.
\item
  jeremiah\\
  Yeah. It's just truth for life.

  When integration is not helpful, it's going to become segregated
  again.
\end{itemize}

chana joffe-walt

That's probably true. White parents are opting in to be at BHS right
now. But they can just as easily opt out. Historically, they have.

When this school building first opened its doors years ago, Black and
Puerto Rican parents were demanding integrated equal schools city wide
for everybody. They weren't asking for one curated school or a small
network of schools where people could integrate, if they wanted to. They
were asking the Board of Education to have a plan for all schools. They
were asking for things to go differently than they have for all of
history. Next time, on ``Nice White Parents,'' things go differently.

``Nice White Parents'' is produced by Julie Snyder and me, with editing
on this episode from Sarah Koenig and Ira Glass. Neil Drumming is our
managing editor. Eve L. Ewing is our editorial consultant. Fact checking
and research by Ben Phelan. Additional reporting from Emmanuel Dzotsi,
Jessica Lussenhop and Alvin Melathe. Music supervision and mixing by
Stowe Nelson with production help from Aviva DeKornfeld. Our director of
operations is Seth Lind. Julie Whitaker is our digital manager. Finance
management by Cassie Howley, and production management by Frances
Swanson. Original music for ``Nice White Parents'' is by The Bad Plus
with additional music written and performed by Matt McGinley. Film
clips, courtesy of the National Education Association and C-span Video
Library. Special thanks to Tina Priceman, Johanna Miller, Leonie
Haimson, Jill Cysner, Clayton Harding, Kate Taylor and Ana Espada. At
The New York Times, thank you to Kelly Doe and Jason Fujikuni. And at
Studio Rodrigo, thanks to Khoi Uong, Becki Choe, Nick Emrich and
Christina No.

``Nice White Parents'' is produced by Serial Productions, a New York
Times Company.

Previous

More episodes ofNice White Parents

\href{https://www.nytimes3xbfgragh.onion/2020/08/13/podcasts/nice-white-parents-school.html?action=click\&module=audio-series-bar\&region=header\&pgtype=Article}{\includegraphics{https://static01.graylady3jvrrxbe.onion/images/2020/07/30/podcasts/30nwp-art/nice-white-parents-album-art-thumbLarge.jpg}}

August 13, 2020~~•~ 50:38Episode Four: `Here's Another Fun Thing You Can
Do'

\href{https://www.nytimes3xbfgragh.onion/2020/08/06/podcasts/episode-three-this-is-our-school-how-dare-you.html?action=click\&module=audio-series-bar\&region=header\&pgtype=Article}{\includegraphics{https://static01.graylady3jvrrxbe.onion/images/2020/07/30/podcasts/30nwp-art/nice-white-parents-album-art-thumbLarge.jpg}}

August 6, 2020~~•~ 46:55Episode Three: `This Is Our School, How Dare
You?'

\href{https://www.nytimes3xbfgragh.onion/2020/07/30/podcasts/nice-white-parents-serial-2.html?action=click\&module=audio-series-bar\&region=header\&pgtype=Article}{\includegraphics{https://static01.graylady3jvrrxbe.onion/images/2020/07/30/podcasts/30nwp-art/nice-white-parents-album-art-thumbLarge.jpg}}

July 30, 2020~~•~ 53:37Episode Two: `I Still Believe in It'

\href{https://www.nytimes3xbfgragh.onion/2020/07/30/podcasts/nice-white-parents-serial.html?action=click\&module=audio-series-bar\&region=header\&pgtype=Article}{\includegraphics{https://static01.graylady3jvrrxbe.onion/images/2020/07/30/podcasts/30nwp-art/nice-white-parents-album-art-thumbLarge.jpg}}

July 30, 2020~~•~ 1:02:23Episode One: The Book of Statuses

\href{https://www.nytimes3xbfgragh.onion/2020/07/23/podcasts/nice-white-parents-serial.html?action=click\&module=audio-series-bar\&region=header\&pgtype=Article}{\includegraphics{https://static01.graylady3jvrrxbe.onion/images/2020/07/21/podcasts/nice-white-parents-album-art/nice-white-parents-album-art-thumbLarge.jpg}}

July 23, 2020~~•~ 2:49Introducing: Nice White Parents

\href{https://www.nytimes3xbfgragh.onion/column/nice-white-parents}{See
All Episodes ofNice White Parents}

Next

Aug. 13, 2020

\begin{itemize}
\item
\item
\item
\item
\item
\end{itemize}

``Nice White Parents'' is a new podcast from Serial Productions, a New
York Times Company, about the 60-year relationship between white parents
and the public school down the block.

\textbf{Listen to the show on your mobile device:}
\textbf{\href{https://podcasts.apple.com/us/podcast/nice-white-parents/id1524080195}{Via
Apple Podcasts}} \textbf{\textbar{}}
\textbf{\href{https://open.spotify.com/show/7oBSLCZFCgpdCaBjIG8mLV?si=YcEPLD3xT2ejXmpQz-tRpw}{Via
Spotify}} \textbf{\textbar{}}
\textbf{\href{https://podcasts.google.com/feed/aHR0cHM6Ly9yc3MuYXJ0MTkuY29tL25pY2Utd2hpdGUtcGFyZW50cw}{Via
Google}}

Public education won't be fair until school systems limit the power of
white parents. But is that even possible? Chana finds two schools that
are trying to do just that, and both are actually inside the I.S. 293
building. One is downstairs in the basement, where a charter school
called Success Academy Cobble Hill opened about seven years ago. The
other is upstairs at the newly-renamed Boerum Hill School for
International Studies.

The original score for ``Nice White Parents'' was written and performed
by The Bad Plus, a jazz group. The band consists of the bassist Reid
Anderson, the pianist Orrin Evans and the drummer Dave King.

Additional music from Matt McGinley, who has contributed music to Serial
Productions podcasts including ``S-Town'' and Season 3 of ``Serial.''
Matt is also the drummer and a co-founder of the band Gym Class Heroes.

Special thanks to Sam Dolnick, Julie Whitaker, Seth Lind, Julia Simon
and Lauren Jackson.

Advertisement

\protect\hyperlink{after-bottom}{Continue reading the main story}

\hypertarget{site-index}{%
\subsection{Site Index}\label{site-index}}

\hypertarget{site-information-navigation}{%
\subsection{Site Information
Navigation}\label{site-information-navigation}}

\begin{itemize}
\tightlist
\item
  \href{https://help.nytimes3xbfgragh.onion/hc/en-us/articles/115014792127-Copyright-notice}{©~2020~The
  New York Times Company}
\end{itemize}

\begin{itemize}
\tightlist
\item
  \href{https://www.nytco.com/}{NYTCo}
\item
  \href{https://help.nytimes3xbfgragh.onion/hc/en-us/articles/115015385887-Contact-Us}{Contact
  Us}
\item
  \href{https://www.nytco.com/careers/}{Work with us}
\item
  \href{https://nytmediakit.com/}{Advertise}
\item
  \href{http://www.tbrandstudio.com/}{T Brand Studio}
\item
  \href{https://www.nytimes3xbfgragh.onion/privacy/cookie-policy\#how-do-i-manage-trackers}{Your
  Ad Choices}
\item
  \href{https://www.nytimes3xbfgragh.onion/privacy}{Privacy}
\item
  \href{https://help.nytimes3xbfgragh.onion/hc/en-us/articles/115014893428-Terms-of-service}{Terms
  of Service}
\item
  \href{https://help.nytimes3xbfgragh.onion/hc/en-us/articles/115014893968-Terms-of-sale}{Terms
  of Sale}
\item
  \href{https://spiderbites.nytimes3xbfgragh.onion}{Site Map}
\item
  \href{https://help.nytimes3xbfgragh.onion/hc/en-us}{Help}
\item
  \href{https://www.nytimes3xbfgragh.onion/subscription?campaignId=37WXW}{Subscriptions}
\end{itemize}
