Sections

SEARCH

\protect\hyperlink{site-content}{Skip to
content}\protect\hyperlink{site-index}{Skip to site index}

\href{https://www.nytimes3xbfgragh.onion/section/science}{Science}

\href{https://myaccount.nytimes3xbfgragh.onion/auth/login?response_type=cookie\&client_id=vi}{}

\href{https://www.nytimes3xbfgragh.onion/section/todayspaper}{Today's
Paper}

\href{/section/science}{Science}\textbar{}`Terror Crocodile' the Size of
a Bus Fed on Dinosaurs, Study Says

\url{https://nyti.ms/3alRt49}

\begin{itemize}
\item
\item
\item
\item
\item
\end{itemize}

Advertisement

\protect\hyperlink{after-top}{Continue reading the main story}

Supported by

\protect\hyperlink{after-sponsor}{Continue reading the main story}

\hypertarget{terror-crocodile-the-size-of-a-bus-fed-on-dinosaurs-study-says}{%
\section{`Terror Crocodile' the Size of a Bus Fed on Dinosaurs, Study
Says}\label{terror-crocodile-the-size-of-a-bus-fed-on-dinosaurs-study-says}}

The monster animal, more closely related to American alligators than
modern crocodiles, had teeth the size of bananas and a strange enlarged
snout.

\includegraphics{https://static01.graylady3jvrrxbe.onion/images/2020/08/11/multimedia/11xp-gators/11xp-gators-articleLarge.jpg?quality=75\&auto=webp\&disable=upscale}

\href{https://www.nytimes3xbfgragh.onion/by/johnny-diaz}{\includegraphics{https://static01.graylady3jvrrxbe.onion/images/2019/11/05/reader-center/author-johnny-diaz/author-johnny-diaz-thumbLarge.png}}

By \href{https://www.nytimes3xbfgragh.onion/by/johnny-diaz}{Johnny Diaz}

\begin{itemize}
\item
  Aug. 13, 2020
\item
  \begin{itemize}
  \item
  \item
  \item
  \item
  \item
  \end{itemize}
\end{itemize}

They had teeth the size of bananas, were as long as buses and limos, and
preyed on dinosaurs that fed near their waterways.

Those are among some of the findings presented in a
\href{https://www.tandfonline.com/doi/full/10.1080/02724634.2020.1767638}{new
study}announced this week about an enormous ancient animal called the
``terror crocodile,'' or Deinosuchus.

The
\href{https://www.tandfonline.com/doi/full/10.1080/02724634.2020.1767638}{research},
published in the Journal of Vertebrate Paleontology, found that the
Deinosuchus, a lineage of the giant crocodilians from North America,
grew up to 33 feet long and ``was the largest carnivore in its
ecosystem,'' in the late Cretaceous period about 75 to 82 million years
ago.

Adam Cossette, a vertebrate paleobiologist who led the study, said in an
email Tuesday that while it was difficult to determine their average
size because there were so few known specimens*, ``*the specimens that
we do have are all HUGE.''

Dr. Cossette, of the New York Institute of Technology College of
Osteopathic Medicine of Arkansas State University, said that large
specimens were 30 to 35 feet long and weighed about 8,000 pounds. For
comparison, he said that a large American alligator today is about 12 to
13 feet long, weighs about 700 to 800 pounds and has teeth about three
inches long at the tip of its snout.

He also said that the ancient reptiles had a large enough head and
strong enough jaws to prey on dinosaurs that lived among them.

*``*Deinosuchus was a giant that must have terrorized dinosaurs that
came to the water's edge to drink,'' he said in a statement. ``Until
now, the complete animal was unknown. These new specimens we've examined
reveal a bizarre, monstrous predator.''

In addition to killing dinosaurs, the animals, because of their size,
probably preyed on just about anything that wandered their way.
Researchers found multiple bite marks on turtle shells and dinosaur
bones.

They also found that at least three species of the Deinosuchus roamed
what is now the United States and Mexico. Two species lived in the West,
from Montana to northern Mexico, and another species lived along the
Atlantic coastal plain, from New Jersey to Mississippi.

The study said that despite the name of the genus, which means ``terror
crocodile,''
\href{https://www.nationalgeographic.com/news/2010/3/100323-giant-croc-crocodile-dinosaurs-deinosuchus-feces-poop/\#close}{}the
creatures were more closely related to alligators. But because of an
``enormous skull,'' they did not look like either crocodiles or
alligators.

The snout was long and broad ``but inflated at the front around the nose
in a way not seen in any other crocodylian, living or extinct,''
according to the researchers, using an alternate spelling of
crocodilian.

The reason for its enlarged nose is unknown, the researchers said. They
also do not know why the animal had two large holes at the tip of its
snout, in front of the nose.

``These holes are unique to Deinosuchus,'' Dr. Cossette said. ``Further
research down the line will hopefully help us unpick this mystery.''

His colleague on the study, Christopher Brochu, a vertebrate
paleontologist at the University of Iowa, said that the fossils showed
how crocodilians were ``not `living fossils' that haven't changed since
the age of dinosaurs.''

``They've evolved just as dynamically as any other group,'' he said in
the statement.

``The earliest ancestors of the American alligator, such as Deinosuchus,
were bizarre and unlike anything that we see in the modern crocodylian
species of today,'' Dr. Cossette said in the email. ``The evolutionary
history of Crocodylia is much more fascinating than meets the eye.''

\href{https://www.nytimes3xbfgragh.onion/2016/02/28/nyregion/how-mark-norell-a-paleontologist-spends-his-sundays.html}{Mark
A. Norell,} the curator and chair of the American Museum of Natural
History's division of paleontology, said the study had many new
findings, especially about the strange inflation at the end of the
skull, the animal's heaviness and size, and the shape of its skull.

And Dr. Norell said there was much more to learn, because the animals'
fossils remained rare, undercollected and understudied. ``Usually their
collection and study is an afterthought,'' he said, ``as most work done
on these formations and deposits is dominated by dinosaurs.''

Advertisement

\protect\hyperlink{after-bottom}{Continue reading the main story}

\hypertarget{site-index}{%
\subsection{Site Index}\label{site-index}}

\hypertarget{site-information-navigation}{%
\subsection{Site Information
Navigation}\label{site-information-navigation}}

\begin{itemize}
\tightlist
\item
  \href{https://help.nytimes3xbfgragh.onion/hc/en-us/articles/115014792127-Copyright-notice}{©~2020~The
  New York Times Company}
\end{itemize}

\begin{itemize}
\tightlist
\item
  \href{https://www.nytco.com/}{NYTCo}
\item
  \href{https://help.nytimes3xbfgragh.onion/hc/en-us/articles/115015385887-Contact-Us}{Contact
  Us}
\item
  \href{https://www.nytco.com/careers/}{Work with us}
\item
  \href{https://nytmediakit.com/}{Advertise}
\item
  \href{http://www.tbrandstudio.com/}{T Brand Studio}
\item
  \href{https://www.nytimes3xbfgragh.onion/privacy/cookie-policy\#how-do-i-manage-trackers}{Your
  Ad Choices}
\item
  \href{https://www.nytimes3xbfgragh.onion/privacy}{Privacy}
\item
  \href{https://help.nytimes3xbfgragh.onion/hc/en-us/articles/115014893428-Terms-of-service}{Terms
  of Service}
\item
  \href{https://help.nytimes3xbfgragh.onion/hc/en-us/articles/115014893968-Terms-of-sale}{Terms
  of Sale}
\item
  \href{https://spiderbites.nytimes3xbfgragh.onion}{Site Map}
\item
  \href{https://help.nytimes3xbfgragh.onion/hc/en-us}{Help}
\item
  \href{https://www.nytimes3xbfgragh.onion/subscription?campaignId=37WXW}{Subscriptions}
\end{itemize}
