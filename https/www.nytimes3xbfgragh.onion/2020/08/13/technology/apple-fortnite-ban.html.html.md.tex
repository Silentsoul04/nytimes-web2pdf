Sections

SEARCH

\protect\hyperlink{site-content}{Skip to
content}\protect\hyperlink{site-index}{Skip to site index}

\href{https://www.nytimes3xbfgragh.onion/section/technology}{Technology}

\href{https://myaccount.nytimes3xbfgragh.onion/auth/login?response_type=cookie\&client_id=vi}{}

\href{https://www.nytimes3xbfgragh.onion/section/todayspaper}{Today's
Paper}

\href{/section/technology}{Technology}\textbar{}Fortnite Creator Sues
Apple and Google After Ban From App Stores

\url{https://nyti.ms/2Y0Hn3P}

\begin{itemize}
\item
\item
\item
\item
\item
\end{itemize}

Advertisement

\protect\hyperlink{after-top}{Continue reading the main story}

Supported by

\protect\hyperlink{after-sponsor}{Continue reading the main story}

\hypertarget{fortnite-creator-sues-apple-and-google-after-ban-from-app-stores}{%
\section{Fortnite Creator Sues Apple and Google After Ban From App
Stores}\label{fortnite-creator-sues-apple-and-google-after-ban-from-app-stores}}

Epic Games, the maker of the popular game, provoked the tech giants by
violating their policies, then released a social media campaign and
lawsuits to confront them.

\includegraphics{https://static01.graylady3jvrrxbe.onion/images/2020/08/14/business/13JPfortnite-print/13fortnite2-articleLarge.jpg?quality=75\&auto=webp\&disable=upscale}

By \href{https://www.nytimes3xbfgragh.onion/by/jack-nicas}{Jack Nicas},
\href{https://www.nytimes3xbfgragh.onion/by/kellen-browning}{Kellen
Browning} and
\href{https://www.nytimes3xbfgragh.onion/by/erin-griffith}{Erin
Griffith}

\begin{itemize}
\item
  Aug. 13, 2020
\item
  \begin{itemize}
  \item
  \item
  \item
  \item
  \item
  \end{itemize}
\end{itemize}

Apple's and Google's spats with app developers over their cut of
revenues exploded into a high-stakes clash on Thursday when the tech
giants kicked the wildly popular game Fortnite out of their app stores
and the game's maker hit back with lawsuits.

The fight began on Thursday morning with a clear provocation. Epic
Games, the maker of Fortnite, started encouraging Fortnite's mobile-app
users to pay it directly, rather than through Apple or Google. The
companies require that they handle all such app payments, so they can
collect a 30 percent commission, a policy that has been at the center of
\href{https://www.nytimes3xbfgragh.onion/2020/06/16/business/apple-app-store-european-union-antitrust.html}{antitrust
complaints} against the companies.

Hours later, Apple responded, removing the Fortnite app from its App
Store.

``Epic enabled a feature in its app which was not reviewed or approved
by Apple, and they did so with the express intent of violating the App
Store guidelines,'' Apple said in a statement. ``We will make every
effort to work with Epic to resolve these violations so they can return
Fortnite to the App Store.''

Within an hour, Epic opened a multifront war against Apple that appeared
months in the making.

First, it sued Apple in federal court, accusing the company of violating
antitrust laws by forcing developers to use its payment systems.

``Apple's removal of Fortnite is yet another example of Apple flexing
its enormous power in order to impose unreasonable restraints and
unlawfully maintain its 100\% monopoly over the'' market for in-app
payments on iPhones, Epic said in its 62-page lawsuit.

Then Epic rolled out a sophisticated public-relations campaign that
depicted Apple, one of the industry's most image-conscious companies, as
the stodgy old guard trying to stifle the upstart. To do so, it used
Apple's own imagery against it, mimicking Apple's iconic ``1984'' ad
from its own fight against IBM 36 years ago. This time, Fortnite
characters were defying Apple's totalitarian regime. Within hours,
\#FreeFortnite was the top trend on Twitter.

Later on Thursday, Google also removed the Fortnite app from its
official Android app store, the Google Play Store, saying the app
violated Google's policies. Epic replied with a similar lawsuit.

Apple's confrontation with Epic has much higher stakes than Google's
because Fortnite remains available for Android devices. Google's Android
software allows people to download apps outside Google's app store,
unlike Apple's approach with iPhones, and Epic had
\href{https://www.theverge.com/2020/4/21/21229943/epic-games-fortnite-google-play-store-available-third-party-software}{added
Fortnite to the Play store} only in April.

In Epic, Apple has met arguably its toughest adversary in years. The
game maker has calculated exactly how to hit Apple where it hurts: by
making iPhones less attractive and Apple less cool.

Epic, a North Carolina company that is valued at roughly \$17 billion
and is partly owned by the Chinese internet giant Tencent, now appears
poised to sacrifice millions of dollars in revenue in a fight that will
keep Fortnite off iPhones. That immediately makes Apple's flagship
devices far less attractive to millions of people across the world ---
just ahead of Apple's most prominent iPhone introduction in years.

Apple, on the verge of a \$2 trillion valuation as its stock has soared
in recent weeks, now faces a battle with one of its most lucrative
partners over a crucial issue for antitrust regulators investigating the
power of Big Tech.

How Apple polices the App Store has drawn intense scrutiny over the past
year. App developers have complained that Apple is taking an unfair cut
of their business while, in many cases, also competing with their apps
with its own offerings.

European regulators, Justice Department officials and state attorneys
general are all investigating Apple's control over the App Store, and
House lawmakers interrogated Apple's chief executive, Tim Cook, on the
issue in a hearing last month. Google has faced more scrutiny for other
issues, in large part because of its more liberal app policy on Android.

For Apple, the world's most valuable company, there are few easy
options. Apple has largely staked its future on its services business,
which has become its second-largest source of revenue after sales of the
iPhone, at \$51.7 billion over the past year. But that business is
mostly built on its cut of other apps' sales, so enforcing its 30
percent commission is crucial to keeping its business growing.

As a result, backing down to Epic would set a dangerous precedent for
Apple, while standing up to the gaming company would prolong a fight
that risks shrinking its iPhone sales and damaging its carefully crafted
image.

In practical terms, kicking Fortnite out of the App Store means that new
users will not be able to download the app, but it will continue to work
on iPhones that already have the app installed. Yet Epic now cannot
update the Fortnite app, meaning it will eventually become obsolete as
Apple updates the iPhone software.

For Android users, there will be much less of an impact; they can still
download Fortnite from Epic's website. As a result, hordes of Fortnite
fans could now favor Google's devices over Apple's.

Suing Apple, in particular, serves two goals for Epic: winning in legal
court and winning in the court of public opinion, said Rebecca Haw
Allensworth, a professor of antitrust at Vanderbilt Law School. Epic is
more likely to succeed in the latter, she said. ``There is growing
business pressure against Apple,'' she said, noting an antitrust case
would be more complicated and difficult to win.

Fortnite has become an enormous enterprise, and
\href{https://twitter.com/FortniteGame/status/1258079550321446912}{announced
in May} that it had more than 350 million registered players. The game
generated \$1.8 billion in revenue last year, according to
\href{https://www.superdataresearch.com/2019-year-in-review}{analysis
firm SuperData}.

Since March 2018, Fortnite's app has been downloaded more than 133
million times on iPhones and iPads and brought in roughly \$1.2 billion,
according to Sensor Tower, an app analytics firm. Apple has taken \$360
million of that revenue as part of its commission, Sensor Tower said.

On Android devices, people have been much more likely to download
Fortnite from outside Google's Play Store. Since April, Fortnite's app
has been downloaded 11 million times on the Play Store and generated
about \$10 million in sales, according to Sensor Tower. Google took
about \$3 million of that.

``Epic could likely have worked out a privileged deal with Apple --- as
other big tech companies have,'' said Matthew Ball, the managing partner
at Epyllion Industries, which operates a venture capital fund.
``Instead, it is fighting for the marketplace.''

Apple has had a
\href{https://www.nytimes3xbfgragh.onion/2020/08/07/technology/facebook-apple-gaming-app-store.html}{series
of recent spats} with app makers. The music service Spotify has
complained to regulators in Europe and the United States. Blix, which
makes an email app that competes with Apple's service, also sued Apple
on antitrust grounds last year. And last week, Microsoft ended a pilot
of its mobile gaming app and Facebook watered down its gaming app on
iPhones because of Apple's rules.

Apple has said that all app developers are subject to the same rules,
and that its commission is fair. Apple has argued that it spends
billions of dollars on the App Store and iPhone technology, creating
business opportunities for companies like Epic.

``Epic has had apps on the App Store for a decade, and have benefited
from the App Store ecosystem, including its tools, testing, and
distribution that Apple provides to all developers,'' Apple said in a
statement on Thursday. ``The fact that their business interests now lead
them to push for a special arrangement does not change the fact that
these guidelines create a level playing field for all developers and
make the store safe for all users.''

In past disputes that led to bad publicity, Apple
\href{https://www.nytimes3xbfgragh.onion/2019/06/03/technology/apple-parental-control-apps.html}{tweaked
its rules} to let
\href{https://www.theverge.com/2020/6/22/21298552/apple-hey-email-app-approval-rules-basecamp-launch}{complaining
app developers back into the App Store}. But Epic's fights appears to be
about more than just its app.

In 2018, Epic released its own app store, and began charging developers
12 percent. Tim Sweeney, the chief executive and founder of Epic, said
in an interview last month that the Epic Games Store had processed more
than \$1 billion in transactions. Even with the lower fee, he said, Epic
still makes a profit of 5 to 7 percent.

Mr. Sweeney said that he felt obligated to ``make this industry a better
and fairer place.''

``It's critical to the future of humanity,'' he said. ``Otherwise you
have these corporations who control all commerce and all speech.''

Advertisement

\protect\hyperlink{after-bottom}{Continue reading the main story}

\hypertarget{site-index}{%
\subsection{Site Index}\label{site-index}}

\hypertarget{site-information-navigation}{%
\subsection{Site Information
Navigation}\label{site-information-navigation}}

\begin{itemize}
\tightlist
\item
  \href{https://help.nytimes3xbfgragh.onion/hc/en-us/articles/115014792127-Copyright-notice}{©~2020~The
  New York Times Company}
\end{itemize}

\begin{itemize}
\tightlist
\item
  \href{https://www.nytco.com/}{NYTCo}
\item
  \href{https://help.nytimes3xbfgragh.onion/hc/en-us/articles/115015385887-Contact-Us}{Contact
  Us}
\item
  \href{https://www.nytco.com/careers/}{Work with us}
\item
  \href{https://nytmediakit.com/}{Advertise}
\item
  \href{http://www.tbrandstudio.com/}{T Brand Studio}
\item
  \href{https://www.nytimes3xbfgragh.onion/privacy/cookie-policy\#how-do-i-manage-trackers}{Your
  Ad Choices}
\item
  \href{https://www.nytimes3xbfgragh.onion/privacy}{Privacy}
\item
  \href{https://help.nytimes3xbfgragh.onion/hc/en-us/articles/115014893428-Terms-of-service}{Terms
  of Service}
\item
  \href{https://help.nytimes3xbfgragh.onion/hc/en-us/articles/115014893968-Terms-of-sale}{Terms
  of Sale}
\item
  \href{https://spiderbites.nytimes3xbfgragh.onion}{Site Map}
\item
  \href{https://help.nytimes3xbfgragh.onion/hc/en-us}{Help}
\item
  \href{https://www.nytimes3xbfgragh.onion/subscription?campaignId=37WXW}{Subscriptions}
\end{itemize}
