Sections

SEARCH

\protect\hyperlink{site-content}{Skip to
content}\protect\hyperlink{site-index}{Skip to site index}

\href{https://www.nytimes3xbfgragh.onion/section/arts}{Arts}

\href{https://myaccount.nytimes3xbfgragh.onion/auth/login?response_type=cookie\&client_id=vi}{}

\href{https://www.nytimes3xbfgragh.onion/section/todayspaper}{Today's
Paper}

\href{/section/arts}{Arts}\textbar{}In Prison, Learning Magic by Mail

\url{https://nyti.ms/3anMGiN}

\begin{itemize}
\item
\item
\item
\item
\item
\item
\end{itemize}

Advertisement

\protect\hyperlink{after-top}{Continue reading the main story}

Supported by

\protect\hyperlink{after-sponsor}{Continue reading the main story}

Surfacing

\hypertarget{in-prison-learning-magic-by-mail}{%
\section{In Prison, Learning Magic by
Mail}\label{in-prison-learning-magic-by-mail}}

A deck of cards cut from milk cartons. A wand made with medical tape.
How a group of inmates learned ``the magic of magic.''

By Annalisa Quinn

Photographs and Video by Vincent Tullo

\includegraphics{https://static01.graylady3jvrrxbe.onion/images/2020/08/07/arts/07surfacing-magic-cards4/07surfacing-magic-cards4-articleLarge.jpg?quality=75\&auto=webp\&disable=upscale}

When the magician Kris Nevling was released from prison two years ago,
he had been incarcerated for most of his life.

What kept him sane, he said, was practicing magic, mostly card tricks.
He was physically small and struggled with addiction, but magic made him
feel valued and powerful. ``In prison everyone has a nickname,'' he
said. ``And mine was always Magic.''

Nevling, 43, ultimately served 23 years in Pennsylvania on charges
including drug possession and theft.

He started learning magic at 16. On a weekly trip from the juvenile
detention center to the library, he found magic books and started
writing letters to every magician he could find listed in the Yellow
Pages, asking for tips.

One of the magicians Nevling contacted was Joshua Jay. He wasn't alone.

\includegraphics{https://static01.graylady3jvrrxbe.onion/images/2020/08/07/arts/magic-cup-s0/magic-cup-s0-threeByTwoMediumAt2X.jpg}

When Jay began writing a column for the now-defunct Magic magazine in
2001, he did not expect his tricks to end up in the hands of prison
inmates around the world. Then the letters started to arrive. There was
a man serving life without parole for murder in California. An inmate in
Georgia, another in Australia. Their crimes varied. He wrote back to
them all.

It made sense, Jay came to realize, that people would turn to magic in
moments of despair and isolation: ``If there is anything missing from a
maximum-security prison, it's wonder.''

Over time, some of the inmates formed a community, exchanging letters
with step-by-step diagrams of sleight-of-hand tricks or tips for making
props with limited materials.

One of them is David Garza, who found Jay's monthly magazine column in
2008. Write in with questions, it said at the bottom, so Garza did.

Image

A set list and props used by some of the magicians.Credit...Vincent
Tullo for The New York Times

At the time Garza, 52, was in an Ohio state prison, serving part of what
would ultimately be a 14-year sentence on charges including criminal
sexual conduct.

``I put myself in the situation by my choices, and it's the
consequences,'' he said in a telephone interview. ``Something needed to
change, because I couldn't believe I got to the point where I was
actually so selfish, that I didn't care about how I impacted someone
else's life.''

Many of a magician's most basic tools --- ribbons, knives, coins --- are
not available in prison. So Garza sent Jay a list of permitted items,
including toothbrushes, cigarettes, pencils, playing cards and plastic
bottles. Jay designed tricks that used these limited materials, and
Garza began making his own props, like balls and poker chips, from glue
and toilet paper.

\includegraphics{https://static01.graylady3jvrrxbe.onion/images/2020/08/07/arts/magic-chip-s0/magic-chip-s0-threeByTwoMediumAt2X.jpg}

Image

A thumb tip Garza made from toilet paper, glue and handmade
dye.Credit...Vincent Tullo for The New York Times

He would dip toilet paper in water, shape it and let it dry completely
--- this took about a week --- before coating it in layers of glue, each
of which also had to dry before the next layer could be applied. He said
the props were often confiscated or destroyed by guards during routine
searches, and he would have to start again.

To Garza, magic was a way of ordering time. After the first week in
prison, he said, a question presents itself: ``What do you do? And you
look at it like, `I've got two years, I've got 10 years, I've got the
rest of my life in this place. What am I going to do with my time?'''
Some people pursue a G.E.D. He got into magic.

Image

In a letter, Garza explains his idea for a Christmas
show.Credit...Vincent Tullo for The New York Times

Before his 2012 arrest, Robert J. Williams, 30, was best known for
eating a light bulb in under 34 seconds. If magic tricks rely on making
the impossible look real, his stunts involved doing real things that
look impossible. Williams has eaten a watch (``it's time consuming,'' he
joked), swallowed swords, breathed fire and pierced parts of his body
(jaw, hand or bicep) with long spokes.

The day after his 22nd birthday, Williams said he was arrested after his
11th attempted bank robbery. He was incarcerated, first at Rikers Island
and then the Moriah Shock Incarceration Correctional Facility, a
minimum-security prison in upstate New York, before being released in
2014.

Image

In one of Garza's letters, he asks about a technique called
palming.Credit...Vincent Tullo for The New York Times

Without access to props, Williams's magic became more psychological, he
said, and more creative: He started focusing on acts like mind-reading
and hypnosis. Sometimes he put on an hourlong show with a single prop,
like a pencil: ``A pencil can stab something, a pencil can write, a
pencil can vanish,'' he said in a telephone interview.

Since at least World War I, hospitals have used magic for its documented
therapeutic effects. Programs within the criminal justice system are
rarer: Starting in 2007, the Hocus Pocus Project sent magicians into
juvenile detention centers and hospitals in New York before running out
of funds 10 years later.

Image

For the trick ``cups and balls,'' Rafiel Torre used cups from the prison
cafeteria and fashioned balls from kitchen sponges. The wand is made
from a wooden dowel and medical tape.Credit...Vincent Tullo for The New
York Times

It was organized by the Conjuring Arts Research Center, whose director,
William Kalush, said the program faced skepticism. He said people would
ask, ``What do you teach them, how to escape from handcuffs?'' But he
added, ``anybody who's done anything creatively understands the joy of
having something that you've invented for yourself.''

Rafiel Torre, 55, said that in 2003, he was convicted of murder and
sentenced to life in prison without the possibility of parole, and he
now has exhausted his appeals. With no access to the internet, he has
gleaned most of his magic knowledge from books, letters from Jay and
other magicians, and whatever magic he could catch on TV.

Torre said he sculpts his own props out of sponges. For string, he pulls
thread out of his underwear. During stretches in solitary confinement,
where playing cards are forbidden, he found that he could fashion four
makeshift cards from his daily milk carton.

\includegraphics{https://static01.graylady3jvrrxbe.onion/images/2020/08/07/arts/magic-cards-s0/magic-cards-s0-threeByTwoMediumAt2X.jpg}

``Everybody wants to feel like they matter. And in prison, most of us
don't,'' said Torre, now incarcerated at the Substance Abuse Treatment
Facility and State Prison in Corcoran, Calif. With magic, he said, ``I
can practice, and make my mind go someplace else. I get taken away for a
minute.''

Magicians sometimes talk about the moment right after they pull off a
trick, before the rational mind sets in, and the viewer is transported,
briefly, out of reality. ``You make them for a few seconds believe
you're doing the impossible. And for me, for a magician, you see that
look on their face --- '' Torre paused. ``I can't equate it to anything
else.''

After Garza was released from prison, he was astounded by how much
easier it was to get audiences to feel that sense of awe. He didn't have
to overcome the prison's atmosphere of fear, powerlessness, and
suspicion. ``They were already open to it. And that blew my mind,'' he
said.

Image

Garza folded up a card case so it would fit through the neck of the
bottle. He then reassembled it and inserted 52 playing cards, one by
one. You can see the full deck of cards from the corner of the box. How
did he position a sharpened pencil through the box? It's a
mystery.Credit...Vincent Tullo for The New York Times

Garza now works as a professional magician in Cleveland, though his
normal gigs in bars and restaurants have evaporated because of Covid-19.

In some ways, Garza said, his work on the outside feels less urgent than
his performances in prison. ``We're deemed a population of people that
are so terrible, so low-worth that we are not allowed to be out in the
world.''

But being able to overcome that feeling even for a moment, he said,
``That's the magic of magic.''

\includegraphics{https://static01.graylady3jvrrxbe.onion/images/2020/08/07/arts/magic-jumbo-s0/magic-jumbo-s0-threeByTwoLargeAt2X.jpg}

\begin{center}\rule{0.5\linewidth}{\linethickness}\end{center}

\href{https://www.nytimes3xbfgragh.onion/series/surfacing}{Surfacing} is
a biweekly column that explores the intersection of art and life,
produced by Alicia DeSantis, Jolie Ruben and Josephine Sedgwick.

Advertisement

\protect\hyperlink{after-bottom}{Continue reading the main story}

\hypertarget{site-index}{%
\subsection{Site Index}\label{site-index}}

\hypertarget{site-information-navigation}{%
\subsection{Site Information
Navigation}\label{site-information-navigation}}

\begin{itemize}
\tightlist
\item
  \href{https://help.nytimes3xbfgragh.onion/hc/en-us/articles/115014792127-Copyright-notice}{©~2020~The
  New York Times Company}
\end{itemize}

\begin{itemize}
\tightlist
\item
  \href{https://www.nytco.com/}{NYTCo}
\item
  \href{https://help.nytimes3xbfgragh.onion/hc/en-us/articles/115015385887-Contact-Us}{Contact
  Us}
\item
  \href{https://www.nytco.com/careers/}{Work with us}
\item
  \href{https://nytmediakit.com/}{Advertise}
\item
  \href{http://www.tbrandstudio.com/}{T Brand Studio}
\item
  \href{https://www.nytimes3xbfgragh.onion/privacy/cookie-policy\#how-do-i-manage-trackers}{Your
  Ad Choices}
\item
  \href{https://www.nytimes3xbfgragh.onion/privacy}{Privacy}
\item
  \href{https://help.nytimes3xbfgragh.onion/hc/en-us/articles/115014893428-Terms-of-service}{Terms
  of Service}
\item
  \href{https://help.nytimes3xbfgragh.onion/hc/en-us/articles/115014893968-Terms-of-sale}{Terms
  of Sale}
\item
  \href{https://spiderbites.nytimes3xbfgragh.onion}{Site Map}
\item
  \href{https://help.nytimes3xbfgragh.onion/hc/en-us}{Help}
\item
  \href{https://www.nytimes3xbfgragh.onion/subscription?campaignId=37WXW}{Subscriptions}
\end{itemize}
