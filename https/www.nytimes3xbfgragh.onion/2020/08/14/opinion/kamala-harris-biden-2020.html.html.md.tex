Sections

SEARCH

\protect\hyperlink{site-content}{Skip to
content}\protect\hyperlink{site-index}{Skip to site index}

\href{https://myaccount.nytimes3xbfgragh.onion/auth/login?response_type=cookie\&client_id=vi}{}

\href{https://www.nytimes3xbfgragh.onion/section/todayspaper}{Today's
Paper}

\href{/section/opinion}{Opinion}\textbar{}The Undertold, Undersold Story
of Kamala Harris

\url{https://nyti.ms/30VTOjp}

\begin{itemize}
\item
\item
\item
\item
\item
\item
\end{itemize}

Advertisement

\protect\hyperlink{after-top}{Continue reading the main story}

\href{/section/opinion}{Opinion}

Supported by

\protect\hyperlink{after-sponsor}{Continue reading the main story}

\hypertarget{the-undertold-undersold-story-of-kamala-harris}{%
\section{The Undertold, Undersold Story of Kamala
Harris}\label{the-undertold-undersold-story-of-kamala-harris}}

As a prosecutor she can make you tremble. But as a trailblazer can she
make you cry?

\href{https://www.nytimes3xbfgragh.onion/by/frank-bruni}{\includegraphics{https://static01.graylady3jvrrxbe.onion/images/2018/04/03/opinion/frank-bruni/frank-bruni-thumbLarge.png}}

By \href{https://www.nytimes3xbfgragh.onion/by/frank-bruni}{Frank Bruni}

Opinion Columnist

\begin{itemize}
\item
  Aug. 14, 2020
\item
  \begin{itemize}
  \item
  \item
  \item
  \item
  \item
  \item
  \end{itemize}
\end{itemize}

\includegraphics{https://static01.graylady3jvrrxbe.onion/images/2020/08/16/opinion/sunday/16bruni/16bruni-articleLarge.jpg?quality=75\&auto=webp\&disable=upscale}

When I saw on Wednesday morning that
\href{https://www.nytimes3xbfgragh.onion/2020/08/14/technology/kamala-harris-disinformation.html}{Kamala
Harris} had released a short video marking and celebrating her selection
as Joe Biden's running mate, I clicked --- eagerly and instantly. I
wanted to continue riding my wave of excitement about all the firsts:
first woman of color on a major party's presidential ticket, first Black
woman specifically, first Asian-American.

By the time I finished the video, that wave had crashed.

OK, that's an overstatement. But as I listened to her flat, desultory
recitation of her biography and philosophy, I did feel a sense of
frustration, and it was familiar. I'd wrestled with the same letdown
during the Democratic primary, when the experience of Harris didn't live
up to the idea of Harris. She often skipped or skimmed over facets of
her background that she would have benefited from dwelling on. She
frequently zoomed past the poetry to the prose, more a steely lawyer
rattling off lists than a soulful leader serving up inspiration.

Harris the prosecutor can find the holes in your argument and make you
tremble. But can Harris the history-making vice-presidential candidate
find the cracks in your heart and make you cry?

That's certainly not a requirement --- most politicians not named Barack
Obama fail to do that --- and I'm not complaining per se. As
\href{https://www.nytimes3xbfgragh.onion/2020/08/11/opinion/kamala-harris-biden-running-mate.html}{I
wrote when the news broke}, her presence on the Democratic ticket makes
total sense in terms of the experience that she possesses, the values
that she represents and the contrast that she helps Biden draw between
his politics of inclusion and Donald Trump's politics of division.

I'm articulating a wish, one that's tied to my belief that a decent
future for this country hinges on an end to Trump's presidency and my
concern that Biden and Harris use every arrow in their quivers to defeat
him. I'm venting a worry that Harris doesn't fully use one of her
arrows. She did poorly in the Democratic primary because, yes,
\href{https://www.nytimes3xbfgragh.onion/2019/11/29/us/politics/kamala-harris-2020.html}{her
campaign was a mess}. But she also did poorly because she never
discovered the right, stirring way to tell and sell her story.

I want her to discover it on Wednesday night, when she speaks at the
Democratic National Convention (or however we're describing its virtual
facsimile). I want her to hold on to it between then and Nov. 3, because
I want to call her the vice president of the United States soon after
that.

Although Trump would cringe at the following thought and never
understand it, Harris reflects this country's ideals and its reality
much better than he does. ``Her story's America's story,'' Biden said
when he and she first appeared together as running mates on Wednesday
afternoon. He's right, and I want her to embrace that and flesh it out
at every turn.

She's the biracial daughter of immigrants: Jamaican father, Indian
mother, both beckoned to this country of newcomers and transplants.
Growing up in California's Bay Area,
\href{https://www.nytimes3xbfgragh.onion/2019/06/30/us/politics/kamala-harris-berkeley-busing.html}{she
was bused} to an elementary school in a richer, whiter neighborhood than
her own, so she knows the fact of segregation and the dream of
integration from the road she traveled.

She understands how families, despite their best intentions, fray. Her
parents divorced, and when her mother found teaching and research work
in Montreal, she moved for her middle and high school years to that
largely white, French-speaking city. For college she went to Howard, a
historically Black university in Washington, D.C., that allowed her to
appraise America --- its past, present and future --- from a different
vantage point. From there she forged her own path, with her own rules.
She didn't marry until she was 49. Her husband is white and Jewish and
she's
\href{https://www.elle.com/culture/career-politics/a27422434/kamala-harris-stepmom-mothers-day/}{a
stepmother to his two children}.

What a rich mix of influences: as multiracial, multiethnic and
multicultural as the country in which her parents wisely invested their
hopes. What a portrait of life as it's lived, with all sorts of swerves.

In her public remarks she makes references to some of this, but they're
usually just that --- references. After she
\href{https://www.nytimes3xbfgragh.onion/2019/07/31/us/politics/kamala-harris-biden-busing.html}{mentioned
school busing} (``that little girl was me'') to attack Biden in a
primary debate for his opposition to it, I went back and looked at
\href{https://www.ktvu.com/news/transcript-kamala-harris-kicks-off-presidential-campaign-in-oakland}{the
big speech} that she'd given to kick off her presidential campaign.
Busing was nowhere to be found. In fact her speech didn't have all that
much biographical detail, period, at least if you edited out the
professional stuff. It was strikingly impersonal.

I recently read much of her memoir, ``The Truths We Hold: An American
Journey,'' published shortly before that speech, and came away with the
same impression. She gives you less of her history than you expect, not
more. She steers away from emotion, not toward it.

Maybe that's what a woman aiming for top jobs in a man's world has to
do. Maybe that's even more incumbent on a woman of color. A scintilla
too angry and you're unhinged. A soupçon too misty and you're
unraveling.

It's worth noting that when Representative Alexandria Ocasio-Cortez
stood in the House last month to
\href{https://www.nytimes3xbfgragh.onion/2020/07/23/us/alexandria-ocasio-cortez-sexism-congress.html}{call
out the misogynistic remarks} of a male colleague, she twice stressed
that she was speaking from the perspective of principle, not of upset.
``I want to be clear that Representative Yoho's comments were not deeply
hurtful or piercing to me,'' she said, adding that she was made of
tougher stuff than that. She later reiterated that she ``was not deeply
hurt or offended.'' To be at all emotional, she had to establish that
she was unemotional.

A male lawmaker wouldn't have felt that need. John Boehner, the House
speaker from 2011 to 2015, certainly didn't: He was
\href{https://www.usatoday.com/story/news/nation-now/2015/09/25/john-boehner-crying-emotional-speaker-house/72799814/}{famous
for weeping} at the drop of an amendment. If Nancy Pelosi behaved as
soggily, she'd be savaged.

But in \href{https://www.workman.com/products/the-firsts}{the book ``The
Firsts,''} about the women who entered Congress in record-breaking
numbers after the 2018 midterms, Jennifer Steinhauer of The Times notes
that several of those newcomers did permit themselves to tear up at
appropriate times --- for example, when discussing an issue that had
greatly affected them or their loved ones --- and it came across as
authentic, not weak. Maybe the world is changing for the better. Or
maybe there's more allowance for women in House seats than for women in
even higher posts.

The Kamala Harris I've met and chatted with informally, in person, is
warmer and more winning than the version I've watched on the stump. More
transparent. More accessible.

And I saw flashes of her on Wednesday afternoon, during
\href{https://www.youtube.com/watch?v=ymyY7jez0rM}{that appearance with
Biden}, when her huge smiles suggested that she couldn't contain her
exhilaration. She shouldn't try. Her story is genuinely exhilarating.

Storytelling is everything. Trump won the presidency with a story about
America that appealed to many Americans. It mixed imaginary villains
with real ones, lies with truths. But he told it expansively. He told it
effectively.

Obama's rise was rooted in his own story, in the eloquence with which he
spoke of Kenya and Kansas and how their commingling inside him was the
American dream. I can't count the number of times, on his path to the
White House, that he put a lump in my throat.

Harris's story is the rival of Obama's. She just has to give it a
comparably mythic shimmer. I know ``that little girl'' was her. But I
want to know more \emph{about} her, and I want to hear her voice.

\emph{I invite you to sign up for my free}
\href{https://www.nytimes3xbfgragh.onion/newsletters/frank-bruni}{\emph{weekly
email newsletter}}\emph{. You can follow me on Twitter
(}\href{https://twitter.com/FrankBruni}{\emph{@FrankBruni}}\emph{).}

\emph{Listen to}
\href{https://www.nytimes3xbfgragh.onion/column/the-argument}{\emph{``The
Argument'' podcast}} \emph{every Thursday morning, with Ross Douthat,
Michelle Goldberg and me.}

Advertisement

\protect\hyperlink{after-bottom}{Continue reading the main story}

\hypertarget{site-index}{%
\subsection{Site Index}\label{site-index}}

\hypertarget{site-information-navigation}{%
\subsection{Site Information
Navigation}\label{site-information-navigation}}

\begin{itemize}
\tightlist
\item
  \href{https://help.nytimes3xbfgragh.onion/hc/en-us/articles/115014792127-Copyright-notice}{©~2020~The
  New York Times Company}
\end{itemize}

\begin{itemize}
\tightlist
\item
  \href{https://www.nytco.com/}{NYTCo}
\item
  \href{https://help.nytimes3xbfgragh.onion/hc/en-us/articles/115015385887-Contact-Us}{Contact
  Us}
\item
  \href{https://www.nytco.com/careers/}{Work with us}
\item
  \href{https://nytmediakit.com/}{Advertise}
\item
  \href{http://www.tbrandstudio.com/}{T Brand Studio}
\item
  \href{https://www.nytimes3xbfgragh.onion/privacy/cookie-policy\#how-do-i-manage-trackers}{Your
  Ad Choices}
\item
  \href{https://www.nytimes3xbfgragh.onion/privacy}{Privacy}
\item
  \href{https://help.nytimes3xbfgragh.onion/hc/en-us/articles/115014893428-Terms-of-service}{Terms
  of Service}
\item
  \href{https://help.nytimes3xbfgragh.onion/hc/en-us/articles/115014893968-Terms-of-sale}{Terms
  of Sale}
\item
  \href{https://spiderbites.nytimes3xbfgragh.onion}{Site Map}
\item
  \href{https://help.nytimes3xbfgragh.onion/hc/en-us}{Help}
\item
  \href{https://www.nytimes3xbfgragh.onion/subscription?campaignId=37WXW}{Subscriptions}
\end{itemize}
