Sections

SEARCH

\protect\hyperlink{site-content}{Skip to
content}\protect\hyperlink{site-index}{Skip to site index}

\href{https://www.nytimes3xbfgragh.onion/section/science}{Science}

\href{https://myaccount.nytimes3xbfgragh.onion/auth/login?response_type=cookie\&client_id=vi}{}

\href{https://www.nytimes3xbfgragh.onion/section/todayspaper}{Today's
Paper}

\href{/section/science}{Science}\textbar{}How 14 Elephant Seals Assisted
an Antarctic Ice Study

\url{https://nyti.ms/30Wc8ZO}

\begin{itemize}
\item
\item
\item
\item
\item
\item
\end{itemize}

Advertisement

\protect\hyperlink{after-top}{Continue reading the main story}

Supported by

\protect\hyperlink{after-sponsor}{Continue reading the main story}

trilobites

\hypertarget{how-14-elephant-seals-assisted-an-antarctic-ice-study}{%
\section{How 14 Elephant Seals Assisted an Antarctic Ice
Study}\label{how-14-elephant-seals-assisted-an-antarctic-ice-study}}

Mapping currents in the Southern Ocean is vital to monitoring climate
change, but hard to conduct. So scientists turned to seals for help.

\includegraphics{https://static01.graylady3jvrrxbe.onion/images/2020/08/18/science/17SCI-SEAL/merlin_175626696_715c8f3e-7e66-45fa-b816-30667207d6ab-articleLarge.jpg?quality=75\&auto=webp\&disable=upscale}

By Oliver Whang

\begin{itemize}
\item
  Aug. 14, 2020
\item
  \begin{itemize}
  \item
  \item
  \item
  \item
  \item
  \item
  \end{itemize}
\end{itemize}

At the bottom of the planet is the Southern Ocean, its waters cold and
roiling and sheathed with ice many months of the year.

The edge of the ice cover, which melts during summer and forms again in
winter, is called the marginal ice zone, and it is incredibly difficult
to study. Large icebreaking ships, which have traditionally been used
for research in the region, cannot consistently observe small-scale
ocean activity. And sea gliders --- small, relatively cheap instruments
that sink in the water and bob back up periodically --- don't work under
the ice. ``It's a blind spot of knowledge in our climate system,'' said
Sebastiaan Swart, an oceanographer at the University of Gothenburg in
Sweden.

What is known about the marginal ice zone is that it is an important
storage system for carbon and heat emitted by humans. The global ocean
as a whole stores
\href{https://www.ipcc.ch/site/assets/uploads/2018/02/WG1AR5_Chapter03_FINAL.pdf}{more}
than 90 percent of Earth's excess heat, and the Southern Ocean is the
\href{https://journals.ametsoc.org/jcli/article/31/12/4727/94167/Southern-Ocean-Heat-Uptake-Redistribution-and}{portal}
through which much of this heat is transferred from the atmosphere. This
makes ignorance of the region particularly worrisome.

But Dr. Swart and Louise Biddle, a researcher also at Gothenburg, found
a way around this methodological roadblock in a
\href{https://agupubs.onlinelibrary.wiley.com/doi/full/10.1029/2019JC015587}{paper}
published in May. To do so, they turned to unique organic instruments
that can gather consistent information from under the ice: southern
elephant seals.

Seals in the Southern Ocean have been monitored for decades. Small
sensors and trackers that are attached to their bodies and the tops of
their heads, like tiny hats, transmit information from dives --- depth,
lateral distance, water temperature, salinity --- that gets filed into
open-access \href{http://www.meop.net/database/}{databases}. A typical
southern elephant seal is a masterful diver, and spends around 90
percent of its time underwater foraging for fish and squid, only
surfacing for a couple minutes between expeditions to catch its breath
before sinking back down to the inky depths.

Because of the frequency of these dives, seal data, like sea glider
data, can reveal small eddies and flows in the water. These water fluxes
result from many of the same forces, including winds and heat gradients,
that create large currents like the Gulf Stream, but are far smaller and
called submesoscale flows. Some are only the length of a football field
and last no more than a day.

As tiny as they are, submesoscale flows have a direct effect on what Dr.
Swart calls the ``window between the atmosphere and the whole ocean.''

This window is known as the mixed layer, a sliver of water on the
surface whose depth and stratification determines how much heat and
carbon are absorbed by the ocean; the deeper and more well-mixed the
layer, the wider the window opens and the easier it is for the ocean to
absorb heat and carbon from the atmosphere. Submesoscale flows change
this depth and stratification, and thus the aperture of the window.

Without the technology to peer under the ice cover, no one knew what
kind of submesoscale flows were occurring in the marginal ice zone.
Scientists guessed that the ice would dampen the strength of the eddies,
``but we didn't even have the observations to show if they were even
there,'' said Dr. Biddle.

Then the two researchers realized ``that the seals had been going under
the sea ice for years and years and years,'' Dr. Swart said. ``And
because they do that, they were collecting the right kind of
observations for us to look at the upper ocean under sea ice.'' The
open-access seal data sets could potentially illustrate what kind of
submesoscale flows occur under the ice, and whether they occur at all.

So the two turned to southern elephant seals, which, they found, were
challenging collaborators. Many of the dives, and the corresponding
data, were clustered outside the zone of study. ``You can't tell them
where to go,'' Dr. Biddle said, laughing. ``That's the biggest issue.
They follow the food.''

But there was enough information to provide a first glimpse of the tiny
currents swirling under the Southern Ocean's ice cover. And what Dr.
Biddle and Dr. Swart found, surprisingly, was that submesoscale flows
are nearly as active under the ice as they are in the open ocean, and
that they are strongest in the midwinter, when the ice is thickest.

In short, the seals showed that water in the Southern Ocean moves a lot
more under the ice, and particularly under thick ice, than many
scientists had anticipated. Perhaps this has to do with the variable
concentration of what Dr. Biddle called ``pancake ice,'' which creates
heat variations in the mixed layer. Perhaps it has to do with certain
wind and weather patterns. Either way, it is an important finding.

``If these submesoscales are to change in the future, they actually will
really change how much heat and carbon is stored in the atmosphere or in
the ocean,'' Dr. Swart said. ``And so they're really, really important,
cumulatively, to the habitable planet.''

\textbf{\emph{{[}}\href{http://on.fb.me/1paTQ1h}{\emph{Like the Science
Times page on Facebook.}}} ****** \emph{\textbar{} Sign up for the}
\textbf{\href{http://nyti.ms/1MbHaRU}{\emph{Science Times newsletter.}}}

Advertisement

\protect\hyperlink{after-bottom}{Continue reading the main story}

\hypertarget{site-index}{%
\subsection{Site Index}\label{site-index}}

\hypertarget{site-information-navigation}{%
\subsection{Site Information
Navigation}\label{site-information-navigation}}

\begin{itemize}
\tightlist
\item
  \href{https://help.nytimes3xbfgragh.onion/hc/en-us/articles/115014792127-Copyright-notice}{©~2020~The
  New York Times Company}
\end{itemize}

\begin{itemize}
\tightlist
\item
  \href{https://www.nytco.com/}{NYTCo}
\item
  \href{https://help.nytimes3xbfgragh.onion/hc/en-us/articles/115015385887-Contact-Us}{Contact
  Us}
\item
  \href{https://www.nytco.com/careers/}{Work with us}
\item
  \href{https://nytmediakit.com/}{Advertise}
\item
  \href{http://www.tbrandstudio.com/}{T Brand Studio}
\item
  \href{https://www.nytimes3xbfgragh.onion/privacy/cookie-policy\#how-do-i-manage-trackers}{Your
  Ad Choices}
\item
  \href{https://www.nytimes3xbfgragh.onion/privacy}{Privacy}
\item
  \href{https://help.nytimes3xbfgragh.onion/hc/en-us/articles/115014893428-Terms-of-service}{Terms
  of Service}
\item
  \href{https://help.nytimes3xbfgragh.onion/hc/en-us/articles/115014893968-Terms-of-sale}{Terms
  of Sale}
\item
  \href{https://spiderbites.nytimes3xbfgragh.onion}{Site Map}
\item
  \href{https://help.nytimes3xbfgragh.onion/hc/en-us}{Help}
\item
  \href{https://www.nytimes3xbfgragh.onion/subscription?campaignId=37WXW}{Subscriptions}
\end{itemize}
