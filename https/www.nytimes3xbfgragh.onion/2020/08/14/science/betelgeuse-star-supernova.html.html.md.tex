Sections

SEARCH

\protect\hyperlink{site-content}{Skip to
content}\protect\hyperlink{site-index}{Skip to site index}

\href{https://www.nytimes3xbfgragh.onion/section/science}{Science}

\href{https://myaccount.nytimes3xbfgragh.onion/auth/login?response_type=cookie\&client_id=vi}{}

\href{https://www.nytimes3xbfgragh.onion/section/todayspaper}{Today's
Paper}

\href{/section/science}{Science}\textbar{}This Star Looked Like It Would
Explode. Maybe It Just Sneezed

\url{https://nyti.ms/3fRPPIM}

\begin{itemize}
\item
\item
\item
\item
\item
\end{itemize}

Advertisement

\protect\hyperlink{after-top}{Continue reading the main story}

Supported by

\protect\hyperlink{after-sponsor}{Continue reading the main story}

Out There

\hypertarget{this-star-looked-like-it-would-explode-maybe-it-just-sneezed}{%
\section{This Star Looked Like It Would Explode. Maybe It Just
Sneezed}\label{this-star-looked-like-it-would-explode-maybe-it-just-sneezed}}

The mysterious dimming of the red supergiant Betelgeuse is the result of
a stellar exhalation, astronomers say.

\includegraphics{https://static01.graylady3jvrrxbe.onion/images/2020/08/14/science/14SCI-BETELGEUSE1/14SCI-BETELGEUSE1-articleLarge.jpg?quality=75\&auto=webp\&disable=upscale}

\href{https://www.nytimes3xbfgragh.onion/by/dennis-overbye}{\includegraphics{https://static01.graylady3jvrrxbe.onion/images/2018/07/30/multimedia/author-dennis-overbye/author-dennis-overbye-thumbLarge.png}}

By \href{https://www.nytimes3xbfgragh.onion/by/dennis-overbye}{Dennis
Overbye}

\begin{itemize}
\item
  Aug. 14, 2020
\item
  \begin{itemize}
  \item
  \item
  \item
  \item
  \item
  \end{itemize}
\end{itemize}

Apparently a star can sneeze.

That is
\href{https://www.nytimes3xbfgragh.onion/interactive/2020/01/09/science/betelgeuse-supernova-fading.html}{what
happened to Betelgeuse}, the red supergiant star that marks the armpit
of Orion the Hunter, according to Andrea Dupree, associate director of
the Harvard-Smithsonian Center for Astrophysics. She and an
international team of astronomers described that conclusion
\href{https://iopscience.iop.org/article/10.3847/1538-4357/aba516}{in a
paper} published this week in Astrophysical Journal.

Betelgeuse, one of the brightest and most prominent stars in the winter
sky, began
\href{https://www.nytimes3xbfgragh.onion/2020/01/09/science/astronomy-supernova-betelgeuse.html}{dramatically
and mysteriously dimming} in the fall of 2019, dwindling to less than
half its normal brightness. By February 2020, it was the faintest that
it had been since measurements began more than 150 years ago, according
to Dr. Dupree.

``The dimming was obvious to everyone when looking at the constellation
Orion,'' she said. ``It was very weird. Betelgeuse was almost missing,''

Some astronomers and excitable members of the public wondered if the
star was
\href{https://www.nytimes3xbfgragh.onion/interactive/2020/01/09/science/betelgeuse-supernova-fading.html}{about
to explode} as a supernova. Such aging stars are notoriously cranky and
moody, sputtering out bursts of gas and dust as their cores evolve and
change.

Something like that was happening to Betelgeuse last year, Dr. Dupree's
team now reports. Observations in ultraviolet light with the
\href{https://www.nytimes3xbfgragh.onion/2020/04/24/science/hubble-telescope-30th-birthday.html}{Hubble
Space Telescope} revealed gobs of dense hot gas shooting out through the
upper parts of atmosphere of Betelgeuse at speeds of 200,000 miles an
hour, ``almost like a sneeze,'' Dr. Dupree said.

At the same time a robot telescope called Stella ---
\href{https://www.aip.de/en/research/facilities/stella/}{the STELLA
Robotic Observatory}, in Tenerife, Spain --- recorded the surface of the
star pulsating outward, helping to propel the hot gas.

The surface of Betelgeuse, like that of the sun, is covered with big
blobs of rising and falling gas, called convection cells, that transmit
energy from the interior. ``We suspect that there was a confluence of a
big convective cell on the surface and also the outward radial velocity
that acted together to eject this material,'' Dr. Dupree said in an
email.

Radial velocity measurements showed that the surface, or photosphere,
was moving out during 2019. From May to August, Betelgeuse was near the
sun and out of view, she said: ``And in September we saw this bright hot
dense material moving out from the southern part of the star.''

Once the gas had gone a few million miles from the star, it
\href{https://www.nytimes3xbfgragh.onion/2020/02/14/science/betelgeuse-pictures-supernova.html}{cooled
into a dust cloud} that obscured the southern part of Betelgeuse.

\includegraphics{https://static01.graylady3jvrrxbe.onion/images/2020/08/14/science/14SCI-BETELGEUSE2/14SCI-BETELGEUSE2-articleLarge.jpg?quality=75\&auto=webp\&disable=upscale}

In an email, Edward Guinan of Villanova University, who has been
tracking Betelgeuse, called the new Hubble data ``fantastic,'' and said
Dr. Dupree's theory was ``a good working hypothesis.'' He added: ``But I
don't entirely agree that the `Mystery' is now solved.''

He noted that alternative explanations could explain the dimming: giant
sunspots, perhaps, or gigantic rising convection cells tens of millions
of miles across, radiating away their heat and energy and then cooling,
turning over and sinking again.

Adding to the mystery is that Betelgeuse, after regaining its normal
luminosity this May, has started to dim again. Betelgeuse has long been
known to vary in brightness --- although not so extremely as this year
--- in accordance with a 420-day cycle of pulsation in its size, so this
new fading is occurring early, for reasons unclear.

That the star will eventually blow up is certain. Betelgeuse, sometimes
pronounced ``beetle-juice,'' and also known as Alpha Orionis, is at
least 10 times and maybe 20 times as massive as the sun. If it were
placed in our solar system, its fiery gases would engulf everything out
to Jupiter's orbit.

The star is a so-called red supergiant in the last violent stages of its
evolution. It has already spent millions of years burning primordial
hydrogen and transforming it into the next lightest element, helium.
That helium is burning into more massive elements. Once the core of the
star becomes solid iron, sometime within the next 100,000 years, the
star will collapse and then rebound in a supernova explosion, probably
leaving behind a dense nugget called a neutron star.

Whatever Betelgeuse is going to do, it might have already done; we are
just waiting for the news. The star is some 725 light-years away, so the
light visible from Earth today, whether rising or falling, left the star
around the year 1300.

``No one knows how a star behaves in the weeks before it explodes,'' Dr.
Dupree said. ``And there were some ominous predictions that Betelgeuse
was ready to become a supernova. Chances are, however, that it will not
explode during our lifetime. But who knows?''

Dr. Guinan said: ``I am really looking forward to seeing what the star
will do this season. It will be fun to see.''

\href{https://www.nytimes3xbfgragh.onion/interactive/2020/science/2020-astronomy-space-calendar.html}{}

\includegraphics{https://static01.graylady3jvrrxbe.onion/images/2019/12/04/science/04SUN1/04SUN1-articleLarge.png}

\hypertarget{sync-your-calendar-with-the-solar-system}{%
\subsection{Sync your calendar with the solar
system}\label{sync-your-calendar-with-the-solar-system}}

Never miss an eclipse, a meteor shower, a rocket launch or any other
astronomical and space event that's out of this world.

Advertisement

\protect\hyperlink{after-bottom}{Continue reading the main story}

\hypertarget{site-index}{%
\subsection{Site Index}\label{site-index}}

\hypertarget{site-information-navigation}{%
\subsection{Site Information
Navigation}\label{site-information-navigation}}

\begin{itemize}
\tightlist
\item
  \href{https://help.nytimes3xbfgragh.onion/hc/en-us/articles/115014792127-Copyright-notice}{©~2020~The
  New York Times Company}
\end{itemize}

\begin{itemize}
\tightlist
\item
  \href{https://www.nytco.com/}{NYTCo}
\item
  \href{https://help.nytimes3xbfgragh.onion/hc/en-us/articles/115015385887-Contact-Us}{Contact
  Us}
\item
  \href{https://www.nytco.com/careers/}{Work with us}
\item
  \href{https://nytmediakit.com/}{Advertise}
\item
  \href{http://www.tbrandstudio.com/}{T Brand Studio}
\item
  \href{https://www.nytimes3xbfgragh.onion/privacy/cookie-policy\#how-do-i-manage-trackers}{Your
  Ad Choices}
\item
  \href{https://www.nytimes3xbfgragh.onion/privacy}{Privacy}
\item
  \href{https://help.nytimes3xbfgragh.onion/hc/en-us/articles/115014893428-Terms-of-service}{Terms
  of Service}
\item
  \href{https://help.nytimes3xbfgragh.onion/hc/en-us/articles/115014893968-Terms-of-sale}{Terms
  of Sale}
\item
  \href{https://spiderbites.nytimes3xbfgragh.onion}{Site Map}
\item
  \href{https://help.nytimes3xbfgragh.onion/hc/en-us}{Help}
\item
  \href{https://www.nytimes3xbfgragh.onion/subscription?campaignId=37WXW}{Subscriptions}
\end{itemize}
