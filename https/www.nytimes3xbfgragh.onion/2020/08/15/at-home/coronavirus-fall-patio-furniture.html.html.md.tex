Sections

SEARCH

\protect\hyperlink{site-content}{Skip to
content}\protect\hyperlink{site-index}{Skip to site index}

\href{https://www.nytimes3xbfgragh.onion/spotlight/at-home}{At Home}

\href{https://myaccount.nytimes3xbfgragh.onion/auth/login?response_type=cookie\&client_id=vi}{}

\href{https://www.nytimes3xbfgragh.onion/section/todayspaper}{Today's
Paper}

\href{/spotlight/at-home}{At Home}\textbar{}Don't Wait to Deck Out Your
Back Porch

\url{https://nyti.ms/2E7WO2Y}

\begin{itemize}
\item
\item
\item
\item
\item
\end{itemize}

\href{https://www.nytimes3xbfgragh.onion/spotlight/at-home?action=click\&pgtype=Article\&state=default\&region=TOP_BANNER\&context=at_home_menu}{At
Home}

\begin{itemize}
\tightlist
\item
  \href{https://www.nytimes3xbfgragh.onion/2020/08/14/dining/lobster-salad-recipe.html?action=click\&pgtype=Article\&state=default\&region=TOP_BANNER\&context=at_home_menu}{Make:
  Lobster Salad}
\item
  \href{https://www.nytimes3xbfgragh.onion/2020/08/15/at-home/coronavirus-at-home-quick-exercises.html?action=click\&pgtype=Article\&state=default\&region=TOP_BANNER\&context=at_home_menu}{Sneak
  In: Exercise}
\item
  \href{https://www.nytimes3xbfgragh.onion/interactive/2020/at-home/even-more-reporters-editors-diaries-lists-recommendations.html?action=click\&pgtype=Article\&state=default\&region=TOP_BANNER\&context=at_home_menu}{See:
  Reporters' Obsessions}
\item
  \href{https://www.nytimes3xbfgragh.onion/2020/08/15/at-home/coronavirus-fall-patio-furniture.html?action=click\&pgtype=Article\&state=default\&region=TOP_BANNER\&context=at_home_menu}{Deck
  Out: Your Porch}
\end{itemize}

Advertisement

\protect\hyperlink{after-top}{Continue reading the main story}

Supported by

\protect\hyperlink{after-sponsor}{Continue reading the main story}

\hypertarget{dont-wait-to-deck-out-your-back-porch}{%
\section{Don't Wait to Deck Out Your Back
Porch}\label{dont-wait-to-deck-out-your-back-porch}}

The pandemic has many people thinking about how they'll spend time
outside as the weather cools, meaning favorite items may soon be in
short supply.

\includegraphics{https://static01.graylady3jvrrxbe.onion/images/2020/08/15/multimedia/15ah-fallgear-WEB01/merlin_175677621_4725d4b0-08d8-4b9c-ab4e-7eed9ef60f49-articleLarge.jpg?quality=75\&auto=webp\&disable=upscale}

By A.C. Shilton

\begin{itemize}
\item
  Published Aug. 15, 2020Updated Aug. 18, 2020
\item
  \begin{itemize}
  \item
  \item
  \item
  \item
  \item
  \end{itemize}
\end{itemize}

Toilet paper notwithstanding, the best thing Todd Osterman did back in
April was start thinking about how to furnish his patio. But really,
``my wife was the one who was thinking way ahead,'' the Wisconsin
teacher said. ``She was the one who thought this might be a long-term
thing.'' If they were going to be home all summer, they might as well
enjoy their patio --- which meant getting a heater for those
early-spring nights.

That decision paid off. ``I was at my local Home Depot last week, and
they were completely sold out,'' Mr. Osterman said. ``This is the middle
of the summer --- not the time you would want to buy a patio heater.''

If the inflatable backyard pool was the hero product of summer 2020,
patio firepits and heaters are poised to be --- pun absolutely intended
--- the hottest items for fall.

Now is the moment to take stock of what you're going to need for your
outdoor patio. Sure, researching heaters in the deep sweat of August
feels odd. But it's clear we may all spend a lot of time at home this
fall --- and want to maximize our outdoor area.

To begin, think a bit about how you'll use your space. When Shai DeLuca,
a Toronto-based designer and TV personality, works with clients with
larger patios, he divides up the square footage, much the way a house
with an open floor plan is divided into various living arrangements.
Split your patio into zones, creating intimate, small spaces that offer
a refuge when you've had enough togetherness.

Of course, how you design and outfit your patio may come down to what
you can still get. At the online retailer Wayfair, searches for patio
heaters over the past five weeks were up more than 70 percent from this
time last year. Dan Crepps, the general manager of Interior Design for
Wayfair Professional, said that the site was already seeing increased
sales on heaters, fire pits, layered textiles (think pillows and
blankets) and outdoor lighting.

John Merris, the chief executive of Solo Stove, which makes smokeless
backyard fire pits, said his company had finally restocked its inventory
after an unusually lucrative summer. In a typical year, Solo Stove moves
a lot of product in early spring, but things taper in July and August.
That lull never happened.

Fire pits have been big for a while. What is new, however, is how people
are using their outdoor living spaces. Joe Raboine, who has spent 25
years in the patio industry, most recently with the paver company
Belgard, says that previously clients wanted patio spaces that would
work for entertaining. Now families are focused on building spaces that
bring them comfort and improve everyday quality of life.

If you're looking for comfort this fall, start by looking for warmth,
Mr. DeLucasaid. Fire pits are an obvious choice, but that's hardly your
only option. In fact, many condo and city dwellers may find that
homeowners associations or local municipalities do not allow
wood-burning products, though some condos may allow small gas fire
tables. While not as warm, Mr. Raboine said, they can serve as a
stunning centerpiece.

Patio heaters take considerably less tending than fire pits, and can
really crank out heat. Most heaters sold for residential use should warm
a 7- to 10-foot radius. For a small patio, one should suffice, but if
you're warming a large space, you'll want one every 10 feet.

When fire pits or heaters are off limits, turn to warm fabrics to make
your porch inviting. ``It's very much about the finishes you choose,''
said Mr. DeLuca, adding that no one wants to sit on wrought iron on a
chilly evening. Look for pillows, cushions and throws created for
outdoor use, which will dry quickly after rain and hold up under
ultraviolet rays. If you love something, but it's not made to live
outside, consider adding an outdoor table or ottoman that doubles as a
watertight storage container. You can tuck these items away between
uses.

Good lighting can also cozy up a space, said Mr. Raboine, adding that
outdoor lighting is something few people think about until it's midnight
black at 5 p.m. Mr. DeLuca advises his clients to think of lighting in
layers --- just as you would inside. Consider a floor or table lamp, a
few candles, and an overhead light, to give a range of types and levels
of light.

Beyond creating family memories and expanding our quarantine worlds just
a bit, there's also a hard-to-deny appeal to products that provide
warmth, light and comfort as days get shorter, while the news stays
unfailingly grim. ``There's this physical comfort that people want to
experience,'' said Liz Wells, marketing director for Rumpl, which makes
blankets geared toward outdoor use.

Like Solo Stove, Rumpl is worried about having enough inventory for the
fall rush --- especially because it's followed by the holiday season.
And there are other problems beyond inventory. Mr. Merris said his
company had struggled with finding space on ocean freighters, while
Rumpl has had issues moving products quickly through distribution
centers, because of increased safety measures for workers.

Bottom line: If you want to accessorize your patio with a heater, or
twinkling string lights, or weatherproof curtains or whatever else
brings you that sense of comfort, don't wait.

Advertisement

\protect\hyperlink{after-bottom}{Continue reading the main story}

\hypertarget{site-index}{%
\subsection{Site Index}\label{site-index}}

\hypertarget{site-information-navigation}{%
\subsection{Site Information
Navigation}\label{site-information-navigation}}

\begin{itemize}
\tightlist
\item
  \href{https://help.nytimes3xbfgragh.onion/hc/en-us/articles/115014792127-Copyright-notice}{©~2020~The
  New York Times Company}
\end{itemize}

\begin{itemize}
\tightlist
\item
  \href{https://www.nytco.com/}{NYTCo}
\item
  \href{https://help.nytimes3xbfgragh.onion/hc/en-us/articles/115015385887-Contact-Us}{Contact
  Us}
\item
  \href{https://www.nytco.com/careers/}{Work with us}
\item
  \href{https://nytmediakit.com/}{Advertise}
\item
  \href{http://www.tbrandstudio.com/}{T Brand Studio}
\item
  \href{https://www.nytimes3xbfgragh.onion/privacy/cookie-policy\#how-do-i-manage-trackers}{Your
  Ad Choices}
\item
  \href{https://www.nytimes3xbfgragh.onion/privacy}{Privacy}
\item
  \href{https://help.nytimes3xbfgragh.onion/hc/en-us/articles/115014893428-Terms-of-service}{Terms
  of Service}
\item
  \href{https://help.nytimes3xbfgragh.onion/hc/en-us/articles/115014893968-Terms-of-sale}{Terms
  of Sale}
\item
  \href{https://spiderbites.nytimes3xbfgragh.onion}{Site Map}
\item
  \href{https://help.nytimes3xbfgragh.onion/hc/en-us}{Help}
\item
  \href{https://www.nytimes3xbfgragh.onion/subscription?campaignId=37WXW}{Subscriptions}
\end{itemize}
