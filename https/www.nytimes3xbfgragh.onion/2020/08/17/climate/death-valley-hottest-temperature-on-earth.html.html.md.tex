Sections

SEARCH

\protect\hyperlink{site-content}{Skip to
content}\protect\hyperlink{site-index}{Skip to site index}

\href{https://www.nytimes3xbfgragh.onion/section/climate}{Climate}

\href{https://myaccount.nytimes3xbfgragh.onion/auth/login?response_type=cookie\&client_id=vi}{}

\href{https://www.nytimes3xbfgragh.onion/section/todayspaper}{Today's
Paper}

\href{/section/climate}{Climate}\textbar{}Death Valley Just Recorded the
Hottest Temperature on Earth

\url{https://nyti.ms/3aAXmdO}

\begin{itemize}
\item
\item
\item
\item
\item
\end{itemize}

\hypertarget{climate-and-environment}{%
\subsubsection{\texorpdfstring{\href{https://www.nytimes3xbfgragh.onion/section/climate?name=styln-climate\&region=TOP_BANNER\&variant=undefined\&block=storyline_menu_recirc\&action=click\&pgtype=Article\&impression_id=c3e73840-e384-11ea-8a8e-3f7ffed08450}{Climate
and
Environment}}{Climate and Environment}}\label{climate-and-environment}}

\begin{itemize}
\tightlist
\item
  \href{https://www.nytimes3xbfgragh.onion/2020/08/17/climate/alaska-oil-drilling-anwr.html?name=styln-climate\&region=TOP_BANNER\&variant=undefined\&block=storyline_menu_recirc\&action=click\&pgtype=Article\&impression_id=c3e75f50-e384-11ea-8a8e-3f7ffed08450}{Arctic
  Refuge}
\item
  \href{https://www.nytimes3xbfgragh.onion/interactive/2020/climate/trump-environment-rollbacks.html?name=styln-climate\&region=TOP_BANNER\&variant=undefined\&block=storyline_menu_recirc\&action=click\&pgtype=Article\&impression_id=c3e75f51-e384-11ea-8a8e-3f7ffed08450}{Trump's
  Changes}
\item
  \href{https://www.nytimes3xbfgragh.onion/interactive/2020/04/19/climate/climate-crash-course-1.html?name=styln-climate\&region=TOP_BANNER\&variant=undefined\&block=storyline_menu_recirc\&action=click\&pgtype=Article\&impression_id=c3e75f52-e384-11ea-8a8e-3f7ffed08450}{Climate
  101}
\item
  \href{https://www.nytimes3xbfgragh.onion/interactive/2018/08/30/climate/how-much-hotter-is-your-hometown.html?name=styln-climate\&region=TOP_BANNER\&variant=undefined\&block=storyline_menu_recirc\&action=click\&pgtype=Article\&impression_id=c3e75f53-e384-11ea-8a8e-3f7ffed08450}{Is
  Your Hometown Hotter?}
\end{itemize}

Advertisement

\protect\hyperlink{after-top}{Continue reading the main story}

Supported by

\protect\hyperlink{after-sponsor}{Continue reading the main story}

\hypertarget{death-valley-just-recorded-the-hottest-temperature-on-earth}{%
\section{Death Valley Just Recorded the Hottest Temperature on
Earth}\label{death-valley-just-recorded-the-hottest-temperature-on-earth}}

Scientists still have to validate the reading of 130 degrees Fahrenheit
on Sunday, the equivalent of 54 degrees Celsius.

\includegraphics{https://static01.graylady3jvrrxbe.onion/images/2020/08/17/multimedia/17xp-deathvalley/merlin_175821324_43bd1446-d7e5-414d-ae3b-c09aca06ee91-articleLarge.jpg?quality=75\&auto=webp\&disable=upscale}

\href{https://www.nytimes3xbfgragh.onion/by/concepcion-de-leon}{\includegraphics{https://static01.graylady3jvrrxbe.onion/images/2018/07/16/multimedia/author-concepcion-de-leon/author-concepcion-de-leon-thumbLarge.png}}\href{https://www.nytimes3xbfgragh.onion/by/john-schwartz}{\includegraphics{https://static01.graylady3jvrrxbe.onion/images/2018/02/16/multimedia/author-john-schwartz/author-john-schwartz-thumbLarge.jpg}}

By
\href{https://www.nytimes3xbfgragh.onion/by/concepcion-de-leon}{Concepción
de León} and
\href{https://www.nytimes3xbfgragh.onion/by/john-schwartz}{John
Schwartz}

\begin{itemize}
\item
  Aug. 17, 2020
\item
  \begin{itemize}
  \item
  \item
  \item
  \item
  \item
  \end{itemize}
\end{itemize}

In the popular imagination, Death Valley in Southern California is the
hottest place on earth. At 3:41 p.m. on Sunday, it lived up to that
reputation when the temperature at the aptly named Furnace Creek reached
130 degrees Fahrenheit, according to the NOAA Weather Prediction center.

If that reading --- the equivalent of 54 degrees Celsius --- is verified
by climate scientists, a process that could take months, it would be the
highest temperature ever reliably recorded on earth.

Death Valley is no stranger to heat. Sitting 282 feet below sea level in
the Mojave Desert in southeastern California near the Nevada border, it
is the lowest, driest and hottest location in the United States. It is
sparsely populated, with just 576 residents, according to the most
recent census.

Brandi Stewart, the spokeswoman for Death Valley National Park, said
that the valley is so hot because of the configuration of its
lower-than-sea-level basin and surrounding mountains. The superheated
air gets trapped in a pocket and just circulates. ``It's like stepping
into a convection oven every day in July and August,'' she said.

So how does 130 degrees, which she walked out into on Sunday, feel? ``It
doesn't feel that different from 125 degrees,'' she said. ``The feeling
of that heat on my face, it can almost take your breath away.''

She added that ``People say, `Oh, but it's a dry heat!' I want to do a
little bit of an eye roll there,'' she said. ``Humidity has its
downsides too, but dry heat is also not fun.''

She grew up in western Pennsylvania and her last posting with the park
service was Mount Rainier National Park, one of the snowiest places on
earth. ``I'm ready for cooler temperatures,'' she said.

\href{https://www.nytimes3xbfgragh.onion/section/climate?action=click\&pgtype=Article\&state=default\&region=MAIN_CONTENT_1\&context=storylines_keepup}{}

\hypertarget{climate-and-environment-}{%
\subsubsection{Climate and Environment
›}\label{climate-and-environment-}}

\hypertarget{keep-up-on-the-latest-climate-news}{%
\paragraph{Keep Up on the Latest Climate
News}\label{keep-up-on-the-latest-climate-news}}

Updated Aug. 18, 2020

Here's what you need to know this week:

\begin{itemize}
\item
  \begin{itemize}
  \tightlist
  \item
    Five automakers
    \href{https://www.nytimes3xbfgragh.onion/2020/08/17/climate/california-automakers-pollution.html?action=click\&pgtype=Article\&state=default\&region=MAIN_CONTENT_1\&context=storylines_keepup}{sealed
    a binding agreement} with California to follow the state's stricter
    tailpipe emissions rules.
  \item
    The Trump
    administration\href{https://www.nytimes3xbfgragh.onion/2020/08/13/climate/trump-methane.html?action=click\&pgtype=Article\&state=default\&region=MAIN_CONTENT_1\&context=storylines_keepup}{eliminated
    a major methane rule}, even as leaks are worsening, in a decision
    that researchers warned ignored science.
  \item
    Climate change leaders said
    \href{https://www.nytimes3xbfgragh.onion/2020/08/12/climate/kamala-harris-environmental-justice.html?action=click\&pgtype=Article\&state=default\&region=MAIN_CONTENT_1\&context=storylines_keepup}{the
    vice-presidential choice of Kamala Harris} signaled that Democrats
    will have a focus on environmental justice.
  \end{itemize}
\end{itemize}

The heat rises through the afternoon, generally reaching the peak from 4
p.m. to 5:30 p.m. The high on Monday was 127.

Confirming a record temperature like this is not as simple as looking at
a thermometer. There are caveats.

Higher temperatures have been reported than the one recorded on Sunday,
but many climate scientists have questioned the reliability of these
readings.

For example, Death Valley claims the record for the hottest temperature
ever recorded in 1913, at 134 degrees. But a 2016 analysis by the
extreme weather expert Christopher Burt found that the reading did not
align with other observations made in the region, concluding that it was
``not possible from a meteorological perspective.''

Setting aside that 107-year-old claim, and some other unverified
readings over the years, the previous record for highest temperature was
also observed in Death Valley on June 30, 2013, at 129 degrees. The same
temperature was also recorded in Kuwait and Pakistan several years
later.

And that is also important to understand: There may be hotter places
than Death Valley, such as parts of the Sahara, but they are too remote
for reliable monitoring, said Daniel Swain, a climate scientist at the
University of California, Los Angeles and the National Center for
Atmospheric Research.

Measuring temperatures reliably is tricky. The thermometers should be
shielded from the sun and elevated above ground, according to standards
set by the World Meteorological Organization. The Death Valley
instrument, called a thermistor, was shielded and sends readings to a
satellite hourly.

Record temperatures are validated by the Climate Extremes Committee, a
collaboration of weather experts from the National Oceanic and
Atmospheric Administration and other organizations, according to Daniel
Berc, a meteorologist with NOAA.

If the Death Valley temperature is validated, then Dr. Swain said it
should be thought of as ``the hottest reliably measured temperature in
recorded history on Earth,'' at least for now.

As the greenhouse gases that humans generate continue heating the
planet, more records are expected, and not just in Death Valley.

``I don't think any of this is really surprising,'' said Jeremy Pal, an
environmental engineering professor at Loyola Marymount University in
Los Angeles. ``As climate continues to warm, we'd expect more of these
events and more of these record-breaking temperatures.''

The broiling temperatures in Death Valley are part of ``a laundry list
of atmospheric phenomena that have unfolded in very unusual or extreme
ways,'' Dr. Swain said, adding that they will only get worse in the
coming decades.

\includegraphics{https://static01.graylady3jvrrxbe.onion/images/2020/08/17/multimedia/17xp-deathvalley2/merlin_175821321_61b1f901-8f6c-4e7a-b1e0-e06bafe0b64d-articleLarge.jpg?quality=75\&auto=webp\&disable=upscale}

California is experiencing a record-breaking heat wave, with unusual
humidity, which has included a rare set of violent lightning storms that
have, in turn, sparked wildfires.

The possible record in Death Valley, he said, ``is part of that,'' and
today's forecasts suggested the temperature could go even higher, to 133
degrees.

``The Earth is getting warmer, and Death Valley is already a hot
place,'' said Dr. Swain, noting that he visited the depopulated desert
area when the temperature was about 115 degrees.

As the planet continues to warm, he said, a temperature of 130 degrees
in a remote place is ``a number we may eventually see in places that
people actually live.''

Dr. J. Marshall Shepherd, an atmospheric scientist at the University of
Georgia and a former president of the American Meteorological Society,
said, ``People notice the changes in extremes because they affect
everything from our health to the productivity of the very food that we
eat.''

Advertisement

\protect\hyperlink{after-bottom}{Continue reading the main story}

\hypertarget{site-index}{%
\subsection{Site Index}\label{site-index}}

\hypertarget{site-information-navigation}{%
\subsection{Site Information
Navigation}\label{site-information-navigation}}

\begin{itemize}
\tightlist
\item
  \href{https://help.nytimes3xbfgragh.onion/hc/en-us/articles/115014792127-Copyright-notice}{©~2020~The
  New York Times Company}
\end{itemize}

\begin{itemize}
\tightlist
\item
  \href{https://www.nytco.com/}{NYTCo}
\item
  \href{https://help.nytimes3xbfgragh.onion/hc/en-us/articles/115015385887-Contact-Us}{Contact
  Us}
\item
  \href{https://www.nytco.com/careers/}{Work with us}
\item
  \href{https://nytmediakit.com/}{Advertise}
\item
  \href{http://www.tbrandstudio.com/}{T Brand Studio}
\item
  \href{https://www.nytimes3xbfgragh.onion/privacy/cookie-policy\#how-do-i-manage-trackers}{Your
  Ad Choices}
\item
  \href{https://www.nytimes3xbfgragh.onion/privacy}{Privacy}
\item
  \href{https://help.nytimes3xbfgragh.onion/hc/en-us/articles/115014893428-Terms-of-service}{Terms
  of Service}
\item
  \href{https://help.nytimes3xbfgragh.onion/hc/en-us/articles/115014893968-Terms-of-sale}{Terms
  of Sale}
\item
  \href{https://spiderbites.nytimes3xbfgragh.onion}{Site Map}
\item
  \href{https://help.nytimes3xbfgragh.onion/hc/en-us}{Help}
\item
  \href{https://www.nytimes3xbfgragh.onion/subscription?campaignId=37WXW}{Subscriptions}
\end{itemize}
