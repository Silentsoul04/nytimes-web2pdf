Sections

SEARCH

\protect\hyperlink{site-content}{Skip to
content}\protect\hyperlink{site-index}{Skip to site index}

\href{https://www.nytimes3xbfgragh.onion/section/food}{Food}

\href{https://myaccount.nytimes3xbfgragh.onion/auth/login?response_type=cookie\&client_id=vi}{}

\href{https://www.nytimes3xbfgragh.onion/section/todayspaper}{Today's
Paper}

\href{/section/food}{Food}\textbar{}One Mask Rule Most New Yorkers
Ignore

\url{https://nyti.ms/3iHCC7g}

\begin{itemize}
\item
\item
\item
\item
\item
\item
\end{itemize}

\href{https://www.nytimes3xbfgragh.onion/spotlight/at-home?action=click\&pgtype=Article\&state=default\&region=TOP_BANNER\&context=at_home_menu}{At
Home}

\begin{itemize}
\tightlist
\item
  \href{https://www.nytimes3xbfgragh.onion/2020/08/14/dining/lobster-salad-recipe.html?action=click\&pgtype=Article\&state=default\&region=TOP_BANNER\&context=at_home_menu}{Make:
  Lobster Salad}
\item
  \href{https://www.nytimes3xbfgragh.onion/2020/08/15/at-home/coronavirus-at-home-quick-exercises.html?action=click\&pgtype=Article\&state=default\&region=TOP_BANNER\&context=at_home_menu}{Sneak
  In: Exercise}
\item
  \href{https://www.nytimes3xbfgragh.onion/interactive/2020/at-home/even-more-reporters-editors-diaries-lists-recommendations.html?action=click\&pgtype=Article\&state=default\&region=TOP_BANNER\&context=at_home_menu}{See:
  Reporters' Obsessions}
\item
  \href{https://www.nytimes3xbfgragh.onion/2020/08/15/at-home/coronavirus-fall-patio-furniture.html?action=click\&pgtype=Article\&state=default\&region=TOP_BANNER\&context=at_home_menu}{Deck
  Out: Your Porch}
\end{itemize}

Advertisement

\protect\hyperlink{after-top}{Continue reading the main story}

Supported by

\protect\hyperlink{after-sponsor}{Continue reading the main story}

Critic's Notebook

\hypertarget{one-mask-rule-most-new-yorkers-ignore}{%
\section{One Mask Rule Most New Yorkers
Ignore}\label{one-mask-rule-most-new-yorkers-ignore}}

Restaurant diners are supposed to wear facial coverings except when
eating or drinking. Hardly anyone does.

\includegraphics{https://static01.graylady3jvrrxbe.onion/images/2020/08/19/dining/17mask1/17mask1-articleLarge-v2.jpg?quality=75\&auto=webp\&disable=upscale}

By \href{https://www.nytimes3xbfgragh.onion/by/pete-wells}{Pete Wells}

\begin{itemize}
\item
  Aug. 17, 2020
\item
  \begin{itemize}
  \item
  \item
  \item
  \item
  \item
  \item
  \end{itemize}
\end{itemize}

In the early weeks of the pandemic, when we had to make our own masks or
go without, a friend sewed pleats into a soothing blue-and-white
rectangle of cotton with ties at the corners and gave them away. I still
wear mine, even though many New Yorkers have moved on to the style with
elastics that loop behind the ears.

Whenever I mask up, I have to make two shoelace knots behind my head,
and I am proud to say that after almost five months, I no longer need to
say, ``the bunny rabbit comes out of the hole and goes around the tree''
each time I leave the house. But I still fumble with my mask like a kid
who is about to come in last in a game at the world's most boring
birthday party.

So while the bartender of a Brooklyn restaurant stood next to my
sidewalk table last week, waiting for me to tell him what I wanted to
drink, I struggled with the laces at the back of my scalp. A minute
stretched into two or three. I don't get faster at tying knots behind my
head when I'm being watched.

``I thought the rule was `Sit Down, Masks Down; Stand Up, Masks Up,'''
he said. ``Isn't that what we're doing?''

Actually, I don't know what we're doing. As far as I can tell, nobody
else is quite sure, either. Walk past almost any open business in the
city and you'll see signs informing you that you need to wear a face
covering to come inside. Almost everybody complies.

But outdoor restaurant dining is where New York's more-or-less general
agreement about masks unravels into confusion.

\href{https://www.governor.ny.gov/sites/governor.ny.gov/files/atoms/files/OutdoorTakeoutDeliveryFoodServicesMasterGuidance.pdf}{The
state Department of Health} says that restaurants ``should encourage,
but not require customers to wear face coverings when not eating and/or
drinking.'' The website of the city's Department of Health and Mental
Hygiene
\href{https://www1.nyc.gov/assets/doh/downloads/pdf/imm/covid-19-dining-safely.pdf}{has
posted similar guidance} for diners. According to the department, ``face
coverings are not required when diners are seated, but wearing a face
covering as much as possible once at the table is the best way to reduce
the risk of Covid-19 transmission for restaurant workers and dining
companions.''

My outdoor restaurant meals since late June suggest that this advice has
reached very few diners. I've seen a smattering of people who keep their
faces covered until the first drink or plate of food arrives, then cover
up again when they've had enough. Should the meal be served in actual
courses, a rarity these days, these people bring the masks out again
between the appetizer and the main course.

\includegraphics{https://static01.graylady3jvrrxbe.onion/images/2020/08/19/dining/17mask2/merlin_175184076_e3fb7d0a-c995-499d-b381-1650516d1e58-articleLarge.jpg?quality=75\&auto=webp\&disable=upscale}

Far more New Yorkers choose to set their faces free the minute they sit
down, if not before, stowing their masks in their pockets, around their
necks like surgeons or in that chin-strap position that makes them look
as if they've just been to see a medieval dentist. They stay that way,
chatting and gossiping and sucking on metal straws and putting away
lunch or dinner until the check has been paid. Usually, but not always,
they slip their masks on as they leave.

The appeal of this approach is clear. Masking and unmasking repeatedly
can be awkward, particularly when you've got a fork in one hand and a
knife in the other. Besides, conversing in a mask is a bit like swimming
in a jumpsuit. It can be done, but it takes more effort. If you have
reason to believe everybody at your table is healthy, the temptation to
talk the way you used to do, employing the full range of lower facial
contortions from the closelipped smirk of an inside joke to the
slack-jawed gape of astonishment, can be very strong.

Gov. Andrew M. Cuomo has chided restaurants that flagrantly disregard
social distancing and masking rules. A state task force has cited
hundreds of establishments for violations, and suspended the liquor
licenses of some, frequently when unmasked customers were standing in
tight clusters.

Much less has been made of the official guidance on wearing masks while
seated. It's rare to find a restaurant that enforces, or even mentions,
the advice, despite the preponderance of signs instructing diners how to
pay through Venmo or bring up a menu by scanning a QR code. The downtown
restaurant
\href{https://www.nytimes3xbfgragh.onion/2018/07/10/dining/frenchette-review-restaurant-tribeca.html}{Frenchette}
is unusual for having
\href{https://www.frenchettenyc.com/reservations}{a note on its
website}asking diners to wear masks ``when any staff is table-side.'' I
went the other night. The only people in masks were the ones who worked
there.

Another restaurant downtown,
\href{https://www.nytimes3xbfgragh.onion/2017/06/06/dining/king-restaurant-review-soho.html}{King},
asks customers in person to wear masks while talking to servers. ``We
get the occasional rude response,'' Annie Shi, one of the owners,
\href{https://blog.resy.com/2020/07/hospitality-save-restaurants-uncertainty-annie-shi-king/}{wrote
recently on Resy}, ``but most guests appreciate that we are taking care
of our staff, and as result, of them.''

Other restaurant owners may not be aware of the health department's
advice. Or they may have heard about mask-hating thugs who've threatened
mask-wearing workers in other parts of the country. Most likely, though,
restaurateurs are simply afraid to do anything that might keep customers
away. In a summer when a thunderstorm can wipe out a night's revenue,
every table counts.

The writer and editor Corby Kummer, whose
\href{https://www.aspeninstitute.org/programs/food-and-society-program/}{Food
and Society Program of the Aspen Institute} collaborated with the
\href{https://www.jamesbeard.org/}{James Beard Foundation} to prepare
detailed Covid-19 safety protocols for restaurants, is now working on
what he calls a ``code of conduct'' for diners. The rules, which could
be made a condition of placing a reservation, would be simple and few:
Whether sitting indoors or out, don't crowd the host stand or the
restrooms, wear a mask when away from the table and comply with polite
requests from the staff. Even these modest requests can make some owners
nervous.

If every restaurant simply repeated the city's advice, posting it on
signs and menus, the message might stick. And it might help make up for
the lapses in physical distancing that are almost inevitable in
restaurant dining. Just because all the tables are spaced six feet apart
does not mean that all the seats will be. Some groups of people tend to
sprawl out and take more space. While eating out, I've found myself
closer to other diners than I would have liked.

I've also been guilty of getting too close to pedestrians, just by
sitting at an outdoor table. Restaurants are supposed to leave a clear,
eight-foot-wide corridor on the sidewalk. In practice, though, I've
found myself less than two feet away from the faces of people out who
are out walking the dog, strolling the toddler or simply circling the
block trying to remember what it felt like to come home from work.

That's one reason I wear a mask when my mouth isn't full. The main one,
though, is that when my face isn't covered and I have to talk to a
bartender, busser, server or manager in a mask, I feel like a heel.
Maybe those people came back to work because they want to help me enjoy
a night away from home. Or maybe they were afraid that if they said no
to working this week, they might not be asked again next week.

But however those employees ended up on the sidewalk, helping me get
over the lingering effects of house arrest, they're taking a risk on my
behalf. Putting on a mask is a simple way to show some respect for them,
even if it means reaching awkwardly behind my ears.

\emph{Follow} \href{https://twitter.com/nytfood}{\emph{NYT Food on
Twitter}} \emph{and}
\href{https://www.instagram.com/nytcooking/}{\emph{NYT Cooking on
Instagram}}\emph{,}
\href{https://www.facebookcorewwwi.onion/nytcooking/}{\emph{Facebook}}\emph{,}
\href{https://www.youtube.com/nytcooking}{\emph{YouTube}} \emph{and}
\href{https://www.pinterest.com/nytcooking/}{\emph{Pinterest}}\emph{.}
\href{https://www.nytimes3xbfgragh.onion/newsletters/cooking}{\emph{Get
regular updates from NYT Cooking, with recipe suggestions, cooking tips
and shopping advice}}\emph{.}

Advertisement

\protect\hyperlink{after-bottom}{Continue reading the main story}

\hypertarget{site-index}{%
\subsection{Site Index}\label{site-index}}

\hypertarget{site-information-navigation}{%
\subsection{Site Information
Navigation}\label{site-information-navigation}}

\begin{itemize}
\tightlist
\item
  \href{https://help.nytimes3xbfgragh.onion/hc/en-us/articles/115014792127-Copyright-notice}{©~2020~The
  New York Times Company}
\end{itemize}

\begin{itemize}
\tightlist
\item
  \href{https://www.nytco.com/}{NYTCo}
\item
  \href{https://help.nytimes3xbfgragh.onion/hc/en-us/articles/115015385887-Contact-Us}{Contact
  Us}
\item
  \href{https://www.nytco.com/careers/}{Work with us}
\item
  \href{https://nytmediakit.com/}{Advertise}
\item
  \href{http://www.tbrandstudio.com/}{T Brand Studio}
\item
  \href{https://www.nytimes3xbfgragh.onion/privacy/cookie-policy\#how-do-i-manage-trackers}{Your
  Ad Choices}
\item
  \href{https://www.nytimes3xbfgragh.onion/privacy}{Privacy}
\item
  \href{https://help.nytimes3xbfgragh.onion/hc/en-us/articles/115014893428-Terms-of-service}{Terms
  of Service}
\item
  \href{https://help.nytimes3xbfgragh.onion/hc/en-us/articles/115014893968-Terms-of-sale}{Terms
  of Sale}
\item
  \href{https://spiderbites.nytimes3xbfgragh.onion}{Site Map}
\item
  \href{https://help.nytimes3xbfgragh.onion/hc/en-us}{Help}
\item
  \href{https://www.nytimes3xbfgragh.onion/subscription?campaignId=37WXW}{Subscriptions}
\end{itemize}
