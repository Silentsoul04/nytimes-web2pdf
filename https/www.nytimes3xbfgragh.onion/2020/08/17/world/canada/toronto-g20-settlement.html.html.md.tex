Sections

SEARCH

\protect\hyperlink{site-content}{Skip to
content}\protect\hyperlink{site-index}{Skip to site index}

\href{https://www.nytimes3xbfgragh.onion/section/world/canada}{Canada}

\href{https://myaccount.nytimes3xbfgragh.onion/auth/login?response_type=cookie\&client_id=vi}{}

\href{https://www.nytimes3xbfgragh.onion/section/todayspaper}{Today's
Paper}

\href{/section/world/canada}{Canada}\textbar{}Toronto Police to Pay
\$12.5 Million to Protesters Arrested During 2010 G-20 Summit

\url{https://nyti.ms/2Q56QEZ}

\begin{itemize}
\item
\item
\item
\item
\item
\end{itemize}

Advertisement

\protect\hyperlink{after-top}{Continue reading the main story}

Supported by

\protect\hyperlink{after-sponsor}{Continue reading the main story}

\hypertarget{toronto-police-to-pay-125-million-to-protesters-arrested-during-2010-g-20-summit}{%
\section{Toronto Police to Pay \$12.5 Million to Protesters Arrested
During 2010 G-20
Summit}\label{toronto-police-to-pay-125-million-to-protesters-arrested-during-2010-g-20-summit}}

Brutal police tactics during the global meeting shocked many Canadians,
led to 1,100 arrests and were found unlawful.

\includegraphics{https://static01.graylady3jvrrxbe.onion/images/2020/08/17/world/17canada/17canada-articleLarge.jpg?quality=75\&auto=webp\&disable=upscale}

\href{https://www.nytimes3xbfgragh.onion/by/ian-austen}{\includegraphics{https://static01.graylady3jvrrxbe.onion/images/2019/07/18/reader-center/author-ian-austen/author-ian-austen-thumbLarge.png}}

By \href{https://www.nytimes3xbfgragh.onion/by/ian-austen}{Ian Austen}

\begin{itemize}
\item
  Aug. 17, 2020
\item
  \begin{itemize}
  \item
  \item
  \item
  \item
  \item
  \end{itemize}
\end{itemize}

As world leaders gathered in Toronto a decade ago they were at first met
by largely peaceful demonstrators pressing for action on climate change,
and Indigenous and gay rights. But a small group of protesters smashed
windows, looted shops and burned police cars in the main shopping
district​.

At first the police did nothing, but then about 20,000 officers from a
variety of forces began making
\href{https://www.nytimes3xbfgragh.onion/2010/06/28/world/americas/28security.html}{arrests},
often violently. Most of those detained, a
\href{https://www.nytimes3xbfgragh.onion/2012/05/17/world/americas/canadian-police-violated-laws-in-g-20-sweep-inquiry-finds.html}{later,
independent inquiry} found, were peaceful or even just bystanders.
Others were taken from homes without warrants, the report said.

The inquiry determined that the police had acted outside of the law.

That was in 2010, at the meeting of the Group of 20 in Toronto. On
Monday, the city's police force reached a multimillion dollar settlement
in a class action brought by the demonstrators.

``Canada had never seen anything like what happened at the G-20 summit,
and hopefully it never will again,'' Murray Klippenstein, one of the two
lawyers who led the class action, representing about 1,100 people, said
in a statement.

The Toronto Police Services Board said in a statement that it was
``pleased that through the efforts of all involved that a mutually
agreeable resolution has been reached.''

\href{http://www.g20classaction.ca/wp-content/uploads/2020/08/G20-Settlement-Agreement-July-3-2020.pdf}{Under
the settlement,} the police force will pay up to 16.5 million Canadian
dollars, \$12.5 million, to the claimants. Individual payments will vary
from 5,000 to 16,000 Canadian dollars.

Violent clashes between police and protesters had certainly occurred in
Canada before then. But the scale of the police action, and the harsh
measures used shocked many Canadians. Security costs for the meeting, as
well as for a smaller
\href{https://www.nytimes3xbfgragh.onion/2010/06/25/world/americas/25canada.html}{Group
of 8 meeting, held around the same time,}totaled about \$1 billion,
making it one of the most costly police actions in Canadian history.

Some of the techniques used by the police
\href{https://www.nytimes3xbfgragh.onion/2020/06/05/nyregion/police-kettling-protests-nyc.html}{have
reappeared this summer in the United States}. Among them are encircling
large crowds to prevent people from peacefully dispersing before
charging in to make arrests, a practice known as ``kettling.''

In 2012, Bill Blair, then Toronto's police chief,
\href{https://www.cbc.ca/toronto/news/blair-release-051912.html}{acknowledged}
``that there were things that were not done well,'' and promised to hold
officers who engaged in ``misconduct'' accountable. It was widely noted
at the time, however, that he did not apologize, except for an incident
involving one protester.

Mr. Blair is now Prime Minister Justin Trudeau's public safety minister.

Five years ago, a Toronto officer was convicted of assault with a
weapon; the evidence included a video that showed several officers
repeatedly punching, kicking and striking with a baton a demonstrator in
front of Ontario's legislature. None of the other officers in the video
were charged.

As part of the settlement, which will go to a judge for approval in
October, the police will issue a statement about its activities in 2010
and a plan for policing future protests. It will also expunge any
records on individuals it arrested during the summit.

Advertisement

\protect\hyperlink{after-bottom}{Continue reading the main story}

\hypertarget{site-index}{%
\subsection{Site Index}\label{site-index}}

\hypertarget{site-information-navigation}{%
\subsection{Site Information
Navigation}\label{site-information-navigation}}

\begin{itemize}
\tightlist
\item
  \href{https://help.nytimes3xbfgragh.onion/hc/en-us/articles/115014792127-Copyright-notice}{©~2020~The
  New York Times Company}
\end{itemize}

\begin{itemize}
\tightlist
\item
  \href{https://www.nytco.com/}{NYTCo}
\item
  \href{https://help.nytimes3xbfgragh.onion/hc/en-us/articles/115015385887-Contact-Us}{Contact
  Us}
\item
  \href{https://www.nytco.com/careers/}{Work with us}
\item
  \href{https://nytmediakit.com/}{Advertise}
\item
  \href{http://www.tbrandstudio.com/}{T Brand Studio}
\item
  \href{https://www.nytimes3xbfgragh.onion/privacy/cookie-policy\#how-do-i-manage-trackers}{Your
  Ad Choices}
\item
  \href{https://www.nytimes3xbfgragh.onion/privacy}{Privacy}
\item
  \href{https://help.nytimes3xbfgragh.onion/hc/en-us/articles/115014893428-Terms-of-service}{Terms
  of Service}
\item
  \href{https://help.nytimes3xbfgragh.onion/hc/en-us/articles/115014893968-Terms-of-sale}{Terms
  of Sale}
\item
  \href{https://spiderbites.nytimes3xbfgragh.onion}{Site Map}
\item
  \href{https://help.nytimes3xbfgragh.onion/hc/en-us}{Help}
\item
  \href{https://www.nytimes3xbfgragh.onion/subscription?campaignId=37WXW}{Subscriptions}
\end{itemize}
