Sections

SEARCH

\protect\hyperlink{site-content}{Skip to
content}\protect\hyperlink{site-index}{Skip to site index}

\href{https://www.nytimes3xbfgragh.onion/section/nyregion}{New York}

\href{https://myaccount.nytimes3xbfgragh.onion/auth/login?response_type=cookie\&client_id=vi}{}

\href{https://www.nytimes3xbfgragh.onion/section/todayspaper}{Today's
Paper}

\href{/section/nyregion}{New York}\textbar{}1.5 Million Antibody Tests
Show What Parts of N.Y.C. Were Hit Hardest

\url{https://nyti.ms/3hbuawJ}

\begin{itemize}
\item
\item
\item
\item
\item
\end{itemize}

\hypertarget{the-coronavirus-outbreak}{%
\subsubsection{\texorpdfstring{\href{https://www.nytimes3xbfgragh.onion/news-event/coronavirus?name=styln-coronavirus-national\&region=TOP_BANNER\&variant=undefined\&block=storyline_menu_recirc\&action=click\&pgtype=Article\&impression_id=0ac2cc50-e387-11ea-9f84-5f58ac2a67a1}{The
Coronavirus
Outbreak}}{The Coronavirus Outbreak}}\label{the-coronavirus-outbreak}}

\begin{itemize}
\tightlist
\item
  live\href{https://www.nytimes3xbfgragh.onion/2020/08/20/world/coronavirus-covid.html?name=styln-coronavirus-national\&region=TOP_BANNER\&variant=undefined\&block=storyline_menu_recirc\&action=click\&pgtype=Article\&impression_id=0ac2cc51-e387-11ea-9f84-5f58ac2a67a1}{Latest
  Updates}
\item
  \href{https://www.nytimes3xbfgragh.onion/interactive/2020/us/coronavirus-us-cases.html?name=styln-coronavirus-national\&region=TOP_BANNER\&variant=undefined\&block=storyline_menu_recirc\&action=click\&pgtype=Article\&impression_id=0ac2cc52-e387-11ea-9f84-5f58ac2a67a1}{Maps
  and Cases}
\item
  \href{https://www.nytimes3xbfgragh.onion/interactive/2020/science/coronavirus-vaccine-tracker.html?name=styln-coronavirus-national\&region=TOP_BANNER\&variant=undefined\&block=storyline_menu_recirc\&action=click\&pgtype=Article\&impression_id=0ac2cc53-e387-11ea-9f84-5f58ac2a67a1}{Vaccine
  Tracker}
\item
  \href{https://www.nytimes3xbfgragh.onion/2020/08/19/us/colleges-closing-covid.html?name=styln-coronavirus-national\&region=TOP_BANNER\&variant=undefined\&block=storyline_menu_recirc\&action=click\&pgtype=Article\&impression_id=0ac2cc54-e387-11ea-9f84-5f58ac2a67a1}{Colleges
  Closing}
\item
  \href{https://www.nytimes3xbfgragh.onion/live/2020/08/20/business/stock-market-today-coronavirus?name=styln-coronavirus-national\&region=TOP_BANNER\&variant=undefined\&block=storyline_menu_recirc\&action=click\&pgtype=Article\&impression_id=0ac2cc55-e387-11ea-9f84-5f58ac2a67a1}{Economy}
\end{itemize}

Advertisement

\protect\hyperlink{after-top}{Continue reading the main story}

Supported by

\protect\hyperlink{after-sponsor}{Continue reading the main story}

\hypertarget{15-million-antibody-tests-show-what-parts-of-nyc-were-hit-hardest}{%
\section{1.5 Million Antibody Tests Show What Parts of N.Y.C. Were Hit
Hardest}\label{15-million-antibody-tests-show-what-parts-of-nyc-were-hit-hardest}}

The data from the city is on a far larger scale than previously released
information, and includes all antibody test results reported to the
city's Department of Health.

\includegraphics{https://static01.graylady3jvrrxbe.onion/images/2020/08/18/nyregion/18nyvirus-antibody1/merlin_171942465_723f8c77-8593-449c-8f39-160e1fdcf35e-articleLarge.jpg?quality=75\&auto=webp\&disable=upscale}

\href{https://www.nytimes3xbfgragh.onion/by/joseph-goldstein}{\includegraphics{https://static01.graylady3jvrrxbe.onion/images/2018/07/16/multimedia/author-joseph-goldstein/author-joseph-goldstein-thumbLarge.png}}

By \href{https://www.nytimes3xbfgragh.onion/by/joseph-goldstein}{Joseph
Goldstein}

\begin{itemize}
\item
  Published Aug. 19, 2020Updated Aug. 20, 2020
\item
  \begin{itemize}
  \item
  \item
  \item
  \item
  \item
  \end{itemize}
\end{itemize}

New York City on Tuesday released more than
\href{https://www1.nyc.gov/site/doh/covid/covid-19-data-testing.page}{1.46
million coronavirus antibody test results}, the largest number to date,
providing more evidence of how the virus penetrated deeply into some
lower-income communities while passing more lightly across affluent
parts of the city.

In one ZIP code in Queens, more than 50 percent of people who had gotten
tested were found to have antibodies, a strikingly high rate. But no ZIP
code south of 96th Street in Manhattan had a positive rate of more than
20 percent.

Across the city, more than 27 percent of those tested had positive
\href{https://www.nytimes3xbfgragh.onion/2020/08/20/nyregion/nyc-coronavirus-antibody-testing.html}{antibody
results}. The borough with the highest rate was the Bronx, at 33
percent. Manhattan had the lowest rate, at 19 percent.

The data is likely to renew discussion about whether some neighborhoods
or communities in New York City may be nearing herd immunity --- the
point at which enough people have immunity that the virus is no longer
able to spread widely within a community.

Percent of people tested with Covid-19

antibodies in New York City, by ZIP code

15

25

35

45\%

BRONX

MANHATTAN

10023

Upper West Side

12.6\%

11369

East Elmhurst

45.7\%

QUEENS

11368

Corona

51.6\%

BROOKLYN

STATEN ISLAND

11219

Borough Park

46.8\%

Percent of people tested with Covid-19 antibodies in New York City, by
ZIP code

BRONX

15

25

35

45\%

QUEENS

MANHATTAN

11368

Corona

51.6\%

BROOKLYN

STATEN ISLAND

Source: N.Y.C. Department of Health

By Matthew Bloch

Until now, public data for antibody rates in New York City has been
limited. CityMD, which plays a key role in the city's testing program,
\href{https://www.nytimes3xbfgragh.onion/2020/07/09/nyregion/nyc-coronavirus-antibodies.html}{had
shared some data} from its network of urgent care clinics.

Gov. Andrew M. Cuomo's office had released some details from a survey
--- which involved testing some 28,419 people across the state --- that
suggested that roughly 21.6 percent of New York City residents had
antibodies.

But the antibody testing data released on Tuesday is on a far larger
scale, encompassing more than 15 percent of New York City residents. It
included all antibody test results reported to the New York City
Department of Health.

``This gives us a sense at a deeper level of the magnitude of the
penetration of the infection into the population,'' said Wafaa El-Sadr,
an epidemiology professor at Columbia University.

Here's what it means:

\hypertarget{some-hard-hit-areas-may-fare-better-during-a-second-wave}{%
\subsection{Some hard-hit areas may fare better during a second
wave}\label{some-hard-hit-areas-may-fare-better-during-a-second-wave}}

Much remains unknown about the
\href{https://www.nytimes3xbfgragh.onion/2020/08/16/health/coronavirus-immunity-antibodies.html}{degree
of protection against Covid-19 that antibodies may offer}, or how long
that protection may last. But the neighborhoods with more residents who
were infected at the height of New York's outbreak in March and April
may
\href{https://www.nytimes3xbfgragh.onion/2020/07/09/nyregion/nyc-coronavirus-antibodies.html}{be
less likely to be among the hardest hit during a second wave.}

\hypertarget{latest-updates-the-coronavirus-outbreak}{%
\section{\texorpdfstring{\href{https://www.nytimes3xbfgragh.onion/2020/08/20/world/coronavirus-covid.html?action=click\&pgtype=Article\&state=default\&region=MAIN_CONTENT_1\&context=storylines_live_updates}{Latest
Updates: The Coronavirus
Outbreak}}{Latest Updates: The Coronavirus Outbreak}}\label{latest-updates-the-coronavirus-outbreak}}

Updated 2020-08-21T07:46:15.883Z

\begin{itemize}
\tightlist
\item
  \href{https://www.nytimes3xbfgragh.onion/2020/08/20/world/coronavirus-covid.html?action=click\&pgtype=Article\&state=default\&region=MAIN_CONTENT_1\&context=storylines_live_updates\#link-68774d88}{Shutdowns,
  warnings and scoldings follow alarming incidents on college campuses.}
\item
  \href{https://www.nytimes3xbfgragh.onion/2020/08/20/world/coronavirus-covid.html?action=click\&pgtype=Article\&state=default\&region=MAIN_CONTENT_1\&context=storylines_live_updates\#link-26b58724}{Biden
  knocks Trump's pandemic response, and outlines a national strategy.}
\item
  \href{https://www.nytimes3xbfgragh.onion/2020/08/20/world/coronavirus-covid.html?action=click\&pgtype=Article\&state=default\&region=MAIN_CONTENT_1\&context=storylines_live_updates\#link-4e542da3}{U.S.
  health agencies announce moves to confront the flu season and
  plummeting child vaccination rates.}
\end{itemize}

\href{https://www.nytimes3xbfgragh.onion/2020/08/20/world/coronavirus-covid.html?action=click\&pgtype=Article\&state=default\&region=MAIN_CONTENT_1\&context=storylines_live_updates}{See
more updates}

More live coverage:
\href{https://www.nytimes3xbfgragh.onion/live/2020/08/20/business/stock-market-today-coronavirus?action=click\&pgtype=Article\&state=default\&region=MAIN_CONTENT_1\&context=storylines_live_updates}{Markets}

On the other hand, neighborhoods in which few residents have been
infected may find themselves more vulnerable in the event of a
resurgence.

Some
\href{https://www.nytimes3xbfgragh.onion/2020/08/17/health/coronavirus-herd-immunity.html}{researchers
have expressed hope that herd immunity} for the new coronavirus may only
require about half of the people in a given community to have immunity
--- while others have suggested a higher threshold, like 70 percent.

Of course, neighborhoods are not sealed off from one another, and even
under the most optimistic predictions, most neighborhoods --- and
millions of New Yorkers --- remain vulnerable to infection in a second
wave.

Still, the relatively high prevalence of antibodies may partly explain
why New York has
\href{https://www.nytimes3xbfgragh.onion/2020/08/17/nyregion/coronavirus-second-wave-nyc.html}{not
seen a significant uptick in cases} over the past several months, even
as the city has begun to reopen and some New Yorkers have begun to relax
their social distancing.

In interviews, epidemiologists have generally expressed surprise and
relief at the relatively low rate of new infections in New York City,
chalking it up to a variety of measures: widespread use of face
coverings, adherence to social distancing recommendations and the
continued ban on indoor dining and bars.

``If antibody tests are accurately identifying those who have been
infected and if these antibodies are protective, are some communities in
N.Y.C. close to achieving herd immunity, like Corona in Queens, for
example,'' Professor El-Sadr wrote in an email. ``That is an intriguing
thought.''

The results released on Tuesday showed that 51.6 percent of people who
got tested in one ZIP code in Corona had tested positive for antibodies.

The New York City health commissioner, Dr. Dave Chokshi, warned against
drawing too many conclusions from antibody rates since there are still
many outstanding questions about immunity.

``Remaining vigilant about precautions like hand washing, distancing,
face coverings and staying home if ill is essential to keeping
transmission low,'' he said.

\hypertarget{neighborhoods-with-large-households-were-vulnerable}{%
\subsection{Neighborhoods with large households were
vulnerable}\label{neighborhoods-with-large-households-were-vulnerable}}

The hardest hit ZIP code in the city --- 11368 --- was the one in
Corona, a predominantly Hispanic neighborhood with many construction
workers and restaurant employees. Many had to work throughout the
pandemic, raising their risk of infection.

About 23 percent of residents in the ZIP code have gotten an antibody
test.

That part of Queens has an especially high
rate\href{https://www.nytimes3xbfgragh.onion/2016/03/01/nyregion/overcrowding-worsens-in-new-york-as-working-families-double-up.html}{of
household crowding}, which may also partly explain the high positive
antibody rate. Experts have said
\href{https://www.who.int/docs/default-source/coronaviruse/who-china-joint-mission-on-covid-19-final-report.pdf}{transmission}
within households is a leading driver of the disease's spread.

\href{https://www.nytimes3xbfgragh.onion/news-event/coronavirus?action=click\&pgtype=Article\&state=default\&region=MAIN_CONTENT_3\&context=storylines_faq}{}

\hypertarget{the-coronavirus-outbreak-}{%
\subsubsection{The Coronavirus Outbreak
›}\label{the-coronavirus-outbreak-}}

\hypertarget{frequently-asked-questions}{%
\paragraph{Frequently Asked
Questions}\label{frequently-asked-questions}}

Updated August 17, 2020

\begin{itemize}
\item ~
  \hypertarget{why-does-standing-six-feet-away-from-others-help}{%
  \paragraph{Why does standing six feet away from others
  help?}\label{why-does-standing-six-feet-away-from-others-help}}

  \begin{itemize}
  \tightlist
  \item
    The coronavirus spreads primarily through droplets from your mouth
    and nose, especially when you cough or sneeze. The C.D.C., one of
    the organizations using that measure,
    \href{https://www.nytimes3xbfgragh.onion/2020/04/14/health/coronavirus-six-feet.html?action=click\&pgtype=Article\&state=default\&region=MAIN_CONTENT_3\&context=storylines_faq}{bases
    its recommendation of six feet} on the idea that most large droplets
    that people expel when they cough or sneeze will fall to the ground
    within six feet. But six feet has never been a magic number that
    guarantees complete protection. Sneezes, for instance, can launch
    droplets a lot farther than six feet,
    \href{https://jamanetwork.com/journals/jama/fullarticle/2763852}{according
    to a recent study}. It's a rule of thumb: You should be safest
    standing six feet apart outside, especially when it's windy. But
    keep a mask on at all times, even when you think you're far enough
    apart.
  \end{itemize}
\item ~
  \hypertarget{i-have-antibodies-am-i-now-immune}{%
  \paragraph{I have antibodies. Am I now
  immune?}\label{i-have-antibodies-am-i-now-immune}}

  \begin{itemize}
  \tightlist
  \item
    As of right
    now,\href{https://www.nytimes3xbfgragh.onion/2020/07/22/health/covid-antibodies-herd-immunity.html?action=click\&pgtype=Article\&state=default\&region=MAIN_CONTENT_3\&context=storylines_faq}{that
    seems likely, for at least several months.} There have been
    frightening accounts of people suffering what seems to be a second
    bout of Covid-19. But experts say these patients may have a
    drawn-out course of infection, with the virus taking a slow toll
    weeks to months after initial exposure. People infected with the
    coronavirus typically
    \href{https://www.nature.com/articles/s41586-020-2456-9}{produce}
    immune molecules called antibodies, which are
    \href{https://www.nytimes3xbfgragh.onion/2020/05/07/health/coronavirus-antibody-prevalence.html?action=click\&pgtype=Article\&state=default\&region=MAIN_CONTENT_3\&context=storylines_faq}{protective
    proteins made in response to an
    infection}\href{https://www.nytimes3xbfgragh.onion/2020/05/07/health/coronavirus-antibody-prevalence.html?action=click\&pgtype=Article\&state=default\&region=MAIN_CONTENT_3\&context=storylines_faq}{.
    These antibodies may} last in the body
    \href{https://www.nature.com/articles/s41591-020-0965-6}{only two to
    three months}, which may seem worrisome, but that's perfectly normal
    after an acute infection subsides, said Dr. Michael Mina, an
    immunologist at Harvard University. It may be possible to get the
    coronavirus again, but it's highly unlikely that it would be
    possible in a short window of time from initial infection or make
    people sicker the second time.
  \end{itemize}
\item ~
  \hypertarget{im-a-small-business-owner-can-i-get-relief}{%
  \paragraph{I'm a small-business owner. Can I get
  relief?}\label{im-a-small-business-owner-can-i-get-relief}}

  \begin{itemize}
  \tightlist
  \item
    The
    \href{https://www.nytimes3xbfgragh.onion/article/small-business-loans-stimulus-grants-freelancers-coronavirus.html?action=click\&pgtype=Article\&state=default\&region=MAIN_CONTENT_3\&context=storylines_faq}{stimulus
    bills enacted in March} offer help for the millions of American
    small businesses. Those eligible for aid are businesses and
    nonprofit organizations with fewer than 500 workers, including sole
    proprietorships, independent contractors and freelancers. Some
    larger companies in some industries are also eligible. The help
    being offered, which is being managed by the Small Business
    Administration, includes the Paycheck Protection Program and the
    Economic Injury Disaster Loan program. But lots of folks have
    \href{https://www.nytimes3xbfgragh.onion/interactive/2020/05/07/business/small-business-loans-coronavirus.html?action=click\&pgtype=Article\&state=default\&region=MAIN_CONTENT_3\&context=storylines_faq}{not
    yet seen payouts.} Even those who have received help are confused:
    The rules are draconian, and some are stuck sitting on
    \href{https://www.nytimes3xbfgragh.onion/2020/05/02/business/economy/loans-coronavirus-small-business.html?action=click\&pgtype=Article\&state=default\&region=MAIN_CONTENT_3\&context=storylines_faq}{money
    they don't know how to use.} Many small-business owners are getting
    less than they expected or
    \href{https://www.nytimes3xbfgragh.onion/2020/06/10/business/Small-business-loans-ppp.html?action=click\&pgtype=Article\&state=default\&region=MAIN_CONTENT_3\&context=storylines_faq}{not
    hearing anything at all.}
  \end{itemize}
\item ~
  \hypertarget{what-are-my-rights-if-i-am-worried-about-going-back-to-work}{%
  \paragraph{What are my rights if I am worried about going back to
  work?}\label{what-are-my-rights-if-i-am-worried-about-going-back-to-work}}

  \begin{itemize}
  \tightlist
  \item
    Employers have to provide
    \href{https://www.osha.gov/SLTC/covid-19/standards.html}{a safe
    workplace} with policies that protect everyone equally.
    \href{https://www.nytimes3xbfgragh.onion/article/coronavirus-money-unemployment.html?action=click\&pgtype=Article\&state=default\&region=MAIN_CONTENT_3\&context=storylines_faq}{And
    if one of your co-workers tests positive for the coronavirus, the
    C.D.C.} has said that
    \href{https://www.cdc.gov/coronavirus/2019-ncov/community/guidance-business-response.html}{employers
    should tell their employees} -\/- without giving you the sick
    employee's name -\/- that they may have been exposed to the virus.
  \end{itemize}
\item ~
  \hypertarget{what-is-school-going-to-look-like-in-september}{%
  \paragraph{What is school going to look like in
  September?}\label{what-is-school-going-to-look-like-in-september}}

  \begin{itemize}
  \tightlist
  \item
    It is unlikely that many schools will return to a normal schedule
    this fall, requiring the grind of
    \href{https://www.nytimes3xbfgragh.onion/2020/06/05/us/coronavirus-education-lost-learning.html?action=click\&pgtype=Article\&state=default\&region=MAIN_CONTENT_3\&context=storylines_faq}{online
    learning},
    \href{https://www.nytimes3xbfgragh.onion/2020/05/29/us/coronavirus-child-care-centers.html?action=click\&pgtype=Article\&state=default\&region=MAIN_CONTENT_3\&context=storylines_faq}{makeshift
    child care} and
    \href{https://www.nytimes3xbfgragh.onion/2020/06/03/business/economy/coronavirus-working-women.html?action=click\&pgtype=Article\&state=default\&region=MAIN_CONTENT_3\&context=storylines_faq}{stunted
    workdays} to continue. California's two largest public school
    districts --- Los Angeles and San Diego --- said on July 13, that
    \href{https://www.nytimes3xbfgragh.onion/2020/07/13/us/lausd-san-diego-school-reopening.html?action=click\&pgtype=Article\&state=default\&region=MAIN_CONTENT_3\&context=storylines_faq}{instruction
    will be remote-only in the fall}, citing concerns that surging
    coronavirus infections in their areas pose too dire a risk for
    students and teachers. Together, the two districts enroll some
    825,000 students. They are the largest in the country so far to
    abandon plans for even a partial physical return to classrooms when
    they reopen in August. For other districts, the solution won't be an
    all-or-nothing approach.
    \href{https://bioethics.jhu.edu/research-and-outreach/projects/eschool-initiative/school-policy-tracker/}{Many
    systems}, including the nation's largest, New York City, are
    devising
    \href{https://www.nytimes3xbfgragh.onion/2020/06/26/us/coronavirus-schools-reopen-fall.html?action=click\&pgtype=Article\&state=default\&region=MAIN_CONTENT_3\&context=storylines_faq}{hybrid
    plans} that involve spending some days in classrooms and other days
    online. There's no national policy on this yet, so check with your
    municipal school system regularly to see what is happening in your
    community.
  \end{itemize}
\end{itemize}

The ZIP code with the second highest rate of positive antibody tests was
in the Borough Park neighborhood in South Brooklyn. There, more than
46.8 percent of antibody tests were positive.

Borough Park is home to many Hasidic Jews, a particularly hard-hit
demographic. Large households are common, and the tight-knit, communal
way of life was likely also a factor. Many Hasidic Jews fell sick in
early March, right after Purim --- a holiday of joyous celebration, full
of parties and mingling.

The lowest rate --- 12.4 percent --- came from a tiny ZIP code in
Queens, consisting of just a couple of blocks of waterfront in Long
Island City. The next lowest rates were in Manhattan, where in several
ZIP codes fewer than 13 percent of people tested had antibodies. The
lowest Manhattan rates --- 12.6 percent --- occurred in one ZIP code on
the Upper East Side and another on the Upper West Side.

In Brooklyn, the ZIP code with the lowest rate of positive antibody
tests --- 13.2 percent --- includes much of Park Slope, a wealthy and
predominantly white neighborhood.

\hypertarget{children-were-especially-likely-to-have-antibodies}{%
\subsection{Children were especially likely to have
antibodies.}\label{children-were-especially-likely-to-have-antibodies}}

Among various age groups, those 17 and under were the most likely to
have antibodies, with 32.6 percent of their tests coming back positive.

But the data may not add much value to
\href{https://www.nytimes3xbfgragh.onion/2020/08/18/nyregion/schools-reopen-nyc.html}{the
debate about school reopenings}, since children were also by far the
least likely age group to be tested ---~about 6,500 per 100,000 people
tested, compared to almost 27,500 per 100,000 people tested in the 45-64
age group.

The age results also were not separated by neighborhoods.

People aged 45 to 64 were more likely to have a positive antibody rate
(28 percent) than adults in other age cohorts.

A higher percentage of men (28 percent) tested positive for antibodies
than women (25 percent).

In ZIP codes classified as having high poverty rates, more than 30
percent of those tested had antibodies. In those with lower poverty
rates, about 18.5 percent of people tested had antibodies.

The data did not include a breakdown by race or ethnicity. Other data
has shown that Black and Hispanic New Yorkers have significantly higher
rates --- when adjusted for age --- of being hospitalized or dying of
Covid-19 than white people.

\hypertarget{the-data-still-has-limitations}{%
\subsection{The data still has
limitations}\label{the-data-still-has-limitations}}

The accuracy of antibody testing varies widely. Moreover, most antibody
tests were done for people seeking them out, which means that those who
got tested are a self-selecting group and not a random sample.

It also appears that New Yorkers in some neighborhoods with lower
infection rates were more likely to seek antibody testing.

In some wealthy and largely white ZIP codes in Manhattan, some 30
percent of people may have gotten an antibody test. But in Corona, for
instance, where the positive rate was highest, less than a quarter of
people got tested.

Advertisement

\protect\hyperlink{after-bottom}{Continue reading the main story}

\hypertarget{site-index}{%
\subsection{Site Index}\label{site-index}}

\hypertarget{site-information-navigation}{%
\subsection{Site Information
Navigation}\label{site-information-navigation}}

\begin{itemize}
\tightlist
\item
  \href{https://help.nytimes3xbfgragh.onion/hc/en-us/articles/115014792127-Copyright-notice}{©~2020~The
  New York Times Company}
\end{itemize}

\begin{itemize}
\tightlist
\item
  \href{https://www.nytco.com/}{NYTCo}
\item
  \href{https://help.nytimes3xbfgragh.onion/hc/en-us/articles/115015385887-Contact-Us}{Contact
  Us}
\item
  \href{https://www.nytco.com/careers/}{Work with us}
\item
  \href{https://nytmediakit.com/}{Advertise}
\item
  \href{http://www.tbrandstudio.com/}{T Brand Studio}
\item
  \href{https://www.nytimes3xbfgragh.onion/privacy/cookie-policy\#how-do-i-manage-trackers}{Your
  Ad Choices}
\item
  \href{https://www.nytimes3xbfgragh.onion/privacy}{Privacy}
\item
  \href{https://help.nytimes3xbfgragh.onion/hc/en-us/articles/115014893428-Terms-of-service}{Terms
  of Service}
\item
  \href{https://help.nytimes3xbfgragh.onion/hc/en-us/articles/115014893968-Terms-of-sale}{Terms
  of Sale}
\item
  \href{https://spiderbites.nytimes3xbfgragh.onion}{Site Map}
\item
  \href{https://help.nytimes3xbfgragh.onion/hc/en-us}{Help}
\item
  \href{https://www.nytimes3xbfgragh.onion/subscription?campaignId=37WXW}{Subscriptions}
\end{itemize}
