Sections

SEARCH

\protect\hyperlink{site-content}{Skip to
content}\protect\hyperlink{site-index}{Skip to site index}

\href{https://www.nytimes3xbfgragh.onion/section/business/economy}{Economy}

\href{https://myaccount.nytimes3xbfgragh.onion/auth/login?response_type=cookie\&client_id=vi}{}

\href{https://www.nytimes3xbfgragh.onion/section/todayspaper}{Today's
Paper}

\href{/section/business/economy}{Economy}\textbar{}Unemployment Claims
Rise as Rollout of \$300 Benefit Lags

\url{https://nyti.ms/2Yj920b}

\begin{itemize}
\item
\item
\item
\item
\item
\end{itemize}

\hypertarget{the-coronavirus-outbreak}{%
\subsubsection{\texorpdfstring{\href{https://www.nytimes3xbfgragh.onion/news-event/coronavirus?name=styln-coronavirus-markets\&region=TOP_BANNER\&variant=undefined\&block=storyline_menu_recirc\&action=click\&pgtype=Article\&impression_id=86da1990-e384-11ea-8b08-6ddcf202df60}{The
Coronavirus
Outbreak}}{The Coronavirus Outbreak}}\label{the-coronavirus-outbreak}}

\begin{itemize}
\tightlist
\item
  live\href{https://www.nytimes3xbfgragh.onion/2020/08/20/world/coronavirus-covid.html?name=styln-coronavirus-markets\&region=TOP_BANNER\&variant=undefined\&block=storyline_menu_recirc\&action=click\&pgtype=Article\&impression_id=86da1991-e384-11ea-8b08-6ddcf202df60}{Latest
  Updates}
\item
  \href{https://www.nytimes3xbfgragh.onion/interactive/2020/us/coronavirus-us-cases.html?name=styln-coronavirus-markets\&region=TOP_BANNER\&variant=undefined\&block=storyline_menu_recirc\&action=click\&pgtype=Article\&impression_id=86da1992-e384-11ea-8b08-6ddcf202df60}{Maps
  and Cases}
\item
  \href{https://www.nytimes3xbfgragh.onion/interactive/2020/science/coronavirus-vaccine-tracker.html?name=styln-coronavirus-markets\&region=TOP_BANNER\&variant=undefined\&block=storyline_menu_recirc\&action=click\&pgtype=Article\&impression_id=86da1993-e384-11ea-8b08-6ddcf202df60}{Vaccine
  Tracker}
\item
  \href{https://www.nytimes3xbfgragh.onion/2020/08/19/us/colleges-closing-covid.html?name=styln-coronavirus-markets\&region=TOP_BANNER\&variant=undefined\&block=storyline_menu_recirc\&action=click\&pgtype=Article\&impression_id=86da1994-e384-11ea-8b08-6ddcf202df60}{Colleges
  Closing}
\item
  \href{https://www.nytimes3xbfgragh.onion/live/2020/08/20/business/stock-market-today-coronavirus?name=styln-coronavirus-markets\&region=TOP_BANNER\&variant=undefined\&block=storyline_menu_recirc\&action=click\&pgtype=Article\&impression_id=86da1995-e384-11ea-8b08-6ddcf202df60}{Economy}
\end{itemize}

Advertisement

\protect\hyperlink{after-top}{Continue reading the main story}

Supported by

\protect\hyperlink{after-sponsor}{Continue reading the main story}

\hypertarget{unemployment-claims-rise-as-rollout-of-300-benefit-lags}{%
\section{Unemployment Claims Rise as Rollout of \$300 Benefit
Lags}\label{unemployment-claims-rise-as-rollout-of-300-benefit-lags}}

With the labor market showing new fragility, most states have yet to
seek funds under President Trump's stopgap plan to supplement weekly
jobless pay.

\includegraphics{https://static01.graylady3jvrrxbe.onion/images/2020/08/21/business/20Jpvirus-jobless-print/merlin_175662330_af94c2ca-5bd1-4cfb-879d-166cc38fa370-articleLarge.jpg?quality=75\&auto=webp\&disable=upscale}

By \href{https://www.nytimes3xbfgragh.onion/by/nelson-d-schwartz}{Nelson
D. Schwartz} and
\href{https://www.nytimes3xbfgragh.onion/by/tiffany-hsu}{Tiffany Hsu}

\begin{itemize}
\item
  Aug. 20, 2020
\item
  \begin{itemize}
  \item
  \item
  \item
  \item
  \item
  \end{itemize}
\end{itemize}

The job market shows signs of softening, even as a move by President
Trump to replace lost unemployment benefits is struggling to get off the
ground.

The Labor Department reported Thursday that new state unemployment
claims jumped to 1.1 million last week, a sign that some employers
continue to lay off workers in the face of the coronavirus pandemic
while others remain reluctant to hire.

``It definitely suggests that momentum in the recovery is slowing,''
said Scott Anderson, chief economist at Bank of the West. ``The labor
market is in the I.C.U., and it needs a shot of adrenaline in the form
of federal aid.''

There are no signs that kind of boost is imminent, however. Nearly 30
million people are drawing unemployment pay in some form, but a \$600
weekly supplement to state benefits --- credited with keeping millions
afloat --- expired at the end of July. Democrats and Republicans have
been at an impasse on a new round of aid, and no action is expected
before September.

\href{https://www.whitehouse.gov/presidential-actions/memorandum-authorizing-needs-assistance-program-major-disaster-declarations-related-coronavirus-disease-2019/}{President
Trump bypassed Capitol Hill this month} to provide a \$300 weekly
supplement, drawn from federal disaster funds, to those receiving
unemployment pay. But by Thursday, fewer than a quarter of the states
had been approved for the program, and only Arizona had put it into
action.

Florida, New York and Texas have held off on applying as they seek
guidance on the program's rules and mull the technological needs for
processing payments. Even states that intend to take part, like
Pennsylvania, have raised doubts about whether it is workable.

``The president's convoluted, temporary, half-baked concept has left
many states, including Pennsylvania, with more questions than a clear
path forward,'' said Penny Ickes, a spokeswoman for the Department of
Labor and Industry in the state's Democratic administration.

Mr. Trump's executive action caps spending on the program at \$44
billion, a figure that officials from the Federal Emergency Management
Agency and the Labor Department said Thursday should be enough to last
four to five weeks. The funds are intended to be retroactive to Aug. 1,
so recipients might be paid only through early September.

The previous \$600 weekly benefit, in place for four months, contributed
\$70 billion a month to the economy, or nearly 5 percent of total
household income.

``That's a pretty substantial chunk of gross domestic product,'' said
Gus Faucher, chief economist at the PNC Financial Services Group. ``And
the households that get it are in a precarious position and pretty much
spend all of it. I'm concerned the expiration of benefits will weigh on
the economy in the second half of the year.''

\hypertarget{latest-updates-the-coronavirus-outbreak-and-the-economy}{%
\section{\texorpdfstring{\href{https://www.nytimes3xbfgragh.onion/live/2020/08/20/business/stock-market-today-coronavirus?action=click\&pgtype=Article\&state=default\&region=MAIN_CONTENT_1\&context=storylines_live_updates}{Latest
Updates: The Coronavirus Outbreak and the
Economy}}{Latest Updates: The Coronavirus Outbreak and the Economy}}\label{latest-updates-the-coronavirus-outbreak-and-the-economy}}

\href{https://www.nytimes3xbfgragh.onion/live/2020/08/20/business/stock-market-today-coronavirus?action=click\&pgtype=Article\&state=default\&region=MAIN_CONTENT_1\&context=storylines_live_updates\#the-producer-of-unhinged-makes-a-big-bet-on-audiences-returning-to-theaters}{10h
ago}

\href{https://www.nytimes3xbfgragh.onion/live/2020/08/20/business/stock-market-today-coronavirus?action=click\&pgtype=Article\&state=default\&region=MAIN_CONTENT_1\&context=storylines_live_updates\#the-producer-of-unhinged-makes-a-big-bet-on-audiences-returning-to-theaters}{The
producer of `Unhinged' makes a big bet on audiences returning to
theaters.}

\href{https://www.nytimes3xbfgragh.onion/live/2020/08/20/business/stock-market-today-coronavirus?action=click\&pgtype=Article\&state=default\&region=MAIN_CONTENT_1\&context=storylines_live_updates\#american-airlines-to-stop-flights-to-15-cities-after-government-aid-ends}{18h
ago}

\href{https://www.nytimes3xbfgragh.onion/live/2020/08/20/business/stock-market-today-coronavirus?action=click\&pgtype=Article\&state=default\&region=MAIN_CONTENT_1\&context=storylines_live_updates\#american-airlines-to-stop-flights-to-15-cities-after-government-aid-ends}{American
Airlines to stop flights to 15 cities after government aid ends.}

\href{https://www.nytimes3xbfgragh.onion/live/2020/08/20/business/stock-market-today-coronavirus?action=click\&pgtype=Article\&state=default\&region=MAIN_CONTENT_1\&context=storylines_live_updates\#without-school-plays-and-assemblies-a-technicians-livelihood-withers}{19h
ago}

\href{https://www.nytimes3xbfgragh.onion/live/2020/08/20/business/stock-market-today-coronavirus?action=click\&pgtype=Article\&state=default\&region=MAIN_CONTENT_1\&context=storylines_live_updates\#without-school-plays-and-assemblies-a-technicians-livelihood-withers}{Without
school plays and assemblies, a technician's livelihood withers.}

\href{https://www.nytimes3xbfgragh.onion/live/2020/08/20/business/stock-market-today-coronavirus?action=click\&pgtype=Article\&state=default\&region=MAIN_CONTENT_1\&context=storylines_live_updates}{See
more updates}

More live coverage:
\href{https://www.nytimes3xbfgragh.onion/2020/08/20/world/coronavirus-covid.html?action=click\&pgtype=Article\&state=default\&region=MAIN_CONTENT_1\&context=storylines_live_updates}{Global}

Millions of unemployment recipients are already feeling the loss.

\includegraphics{https://static01.graylady3jvrrxbe.onion/images/2020/08/20/business/20markets-brf-jobless2/merlin_175177389_c1c36014-36f8-4279-b4a9-10bbe40e59e9-articleLarge.jpg?quality=75\&auto=webp\&disable=upscale}

``That extra \$600 is what's been keeping us alive,'' said David Leske,
a lighting and sound technician in Ridgway, Pa. Without it, he and his
wife have been forced to dip into their savings account. ``It's scary,''
he said.

This should be a time of keen anticipation for Mr. Leske. He works in
local schools to make plays, assemblies and other shows come to life.

But a few weeks before the school year is to begin, the pandemic is
still preventing large indoor gatherings. In some cases, schools are
sticking to online instruction.

``Our local district has no intention of doing school plays,'' Mr. Leske
said. ``The high school auditorium is now a storage area.''

Mr. Leske, 52, said that work began to dry up in March and that the
Pandemic Unemployment Assistance program --- an emergency federal
program for freelancers and others not eligible for state benefits ---
had been crucial in keeping him afloat, especially with the \$600 weekly
federal supplement.

He expects to be out of work through September 2021 as schools hold off
on plays and assemblies. But Pandemic Unemployment Assistance expires at
the end of this year.

While longer-term federal relief is in unresolved, FEMA
\href{https://www.fema.gov/press-release/20200819/fema-announces-lost-wages-grant-maryland}{has
approved} Arizona, Colorado, Idaho, Iowa, Louisiana, Maryland, Missouri,
Montana, New Mexico, Oklahoma and Utah for access to three weeks of
funds for the \$300 supplement. Officials from FEMA and the Labor
Department said on a conference call with reporters on Thursday that
FEMA had approved \$2.4 billion in grants so far and that an additional
eight states had applied for funds.

Arizona was the first state to make the so-called lost wages payments,
sending \$96 million to 320,000 people on Monday and Tuesday. But the
timeline for payments ``will be all over the map,'' potentially taking
several weeks, said John Pallasch, the assistant secretary for
employment and training at the Labor Department.

The challenges include reprogramming antiquated state computer systems
to handle the new benefit --- a factor that caused weeks of delays with
the \$600 supplement --- and dealing with an additional federal agency,
FEMA.

``We have to build a whole new subset system with new rules and new
reporting requirements with a department that we're not really familiar
with,'' said Bill McCamley, the secretary of the New Mexico Department
of Workforce Solutions. ``We want to dot all of our i's and cross all
our t's.''

In a call with reporters on Wednesday, Gov. Andrew M. Cuomo of New York
expressed concern about the legality of Mr. Trump's executive action and
said that ``if the states need to reinvent their unemployment insurance
administration program, it will be weeks or months before anyone gets a
check.''

``I'd rather do business with the old-time bookie on the street corner
than do business with FEMA,'' Mr. Cuomo added.

Mr. Trump's resort to federal disaster funds for the supplement followed
the breakdown on a congressional aid package that would appropriate new
funds. Democrats want to reinstitute the \$600 weekly supplement;
Republicans have called for a lesser amount, saying anything more would
dissuade the unemployed from seeking work.

Initial weekly unemployment claims,

both regular and those under the Pandemic Unemployment Assistance
program

6 million

1.1 million regular claims last week after falling below 1 million the
week before

5

4

3

2

1

0

Feb.

March

April

May

June

July

Aug.

Initial weekly unemployment claims, both regular and those under the
Pandemic Unemployment Assistance program

6 million

5

1.1 million regular claims last week after falling below 1 million the
week before

4

3

2

1

0

Feb.

March

April

May

June

July

Aug.

Pandemic Unemployment Assistance extends eligibility to some workers who
would not otherwise be able to apply for unemployment benefits, such as
part-time and self-employed workers. Regular claims are seasonally
adjusted but P.U.A. claims are not.

Source: Labor Department

By Ella Koeze

As the stalemate continues, the latest jobless claims numbers cast a
further pall. The rise in new state filings last week, from 971,000,
followed two weeks of declines that had brought applications for
unemployment insurance to under one million for the first time since the
pandemic struck.

There were 543,000 new claims last week for Pandemic Unemployment
Assistance. That number, unlike the figures for state claims, is not
seasonally adjusted.

Despite the discouraging report on jobless claims, Mr. Faucher of PNC
Financial pointed to pockets of strength.

``We see continued improvement, with housing starts increasing, consumer
spending increasing and industrial production increasing,'' he said.
``But the pace of improvement is slowing.''

As the pandemic continues to buffet the economy, some workers have been
able to find new positions, but not without considerable personal
sacrifice.

After spending up to six hours a day submitting more than 600
applications since being furloughed this spring and then laid off in
late July, Sonia Vance, 42, finally landed a new job.

Image

It was ``heartbreaking and very emotional'' to be out of work, Sonia
Vance said. She has moved to Maryland from Tennessee to start a new
job.Credit...Timothy Nwachukwu for The New York Times

In a few weeks, she starts as an eyewear consultant in California, Md.,
earning \$16 an hour. The position pays far less than the dream job she
had before --- a \$48,000-a-year human resources role at a staffing
company --- but it comes with health insurance.

The cushion is comforting, because Ms. Vance must now go to work each
day in an office, despite health issues that she fears could complicate
a recovery if she catches the coronavirus.

Reflecting the experience of millions whose careers evaporated in the
pandemic, Ms. Vance said the past few months had been ``heartbreaking
and very emotional.''

This week, she moved from Maryville, Tenn., and will stay temporarily
with a friend. She is finishing up bankruptcy paperwork and expects to
lose her mobile home.

``You do feel relief that you have a job, but there's also a sense of
shame and embarrassment,'' Ms. Vance said. ``You're out there doing
everything you can to be a good member of society and to take care of
your own, but it just takes a few months to wipe out all of your hard
work.''

Advertisement

\protect\hyperlink{after-bottom}{Continue reading the main story}

\hypertarget{site-index}{%
\subsection{Site Index}\label{site-index}}

\hypertarget{site-information-navigation}{%
\subsection{Site Information
Navigation}\label{site-information-navigation}}

\begin{itemize}
\tightlist
\item
  \href{https://help.nytimes3xbfgragh.onion/hc/en-us/articles/115014792127-Copyright-notice}{©~2020~The
  New York Times Company}
\end{itemize}

\begin{itemize}
\tightlist
\item
  \href{https://www.nytco.com/}{NYTCo}
\item
  \href{https://help.nytimes3xbfgragh.onion/hc/en-us/articles/115015385887-Contact-Us}{Contact
  Us}
\item
  \href{https://www.nytco.com/careers/}{Work with us}
\item
  \href{https://nytmediakit.com/}{Advertise}
\item
  \href{http://www.tbrandstudio.com/}{T Brand Studio}
\item
  \href{https://www.nytimes3xbfgragh.onion/privacy/cookie-policy\#how-do-i-manage-trackers}{Your
  Ad Choices}
\item
  \href{https://www.nytimes3xbfgragh.onion/privacy}{Privacy}
\item
  \href{https://help.nytimes3xbfgragh.onion/hc/en-us/articles/115014893428-Terms-of-service}{Terms
  of Service}
\item
  \href{https://help.nytimes3xbfgragh.onion/hc/en-us/articles/115014893968-Terms-of-sale}{Terms
  of Sale}
\item
  \href{https://spiderbites.nytimes3xbfgragh.onion}{Site Map}
\item
  \href{https://help.nytimes3xbfgragh.onion/hc/en-us}{Help}
\item
  \href{https://www.nytimes3xbfgragh.onion/subscription?campaignId=37WXW}{Subscriptions}
\end{itemize}
