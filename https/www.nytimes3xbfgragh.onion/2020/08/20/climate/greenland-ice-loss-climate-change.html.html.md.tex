Sections

SEARCH

\protect\hyperlink{site-content}{Skip to
content}\protect\hyperlink{site-index}{Skip to site index}

\href{https://www.nytimes3xbfgragh.onion/section/climate}{Climate}

\href{https://myaccount.nytimes3xbfgragh.onion/auth/login?response_type=cookie\&client_id=vi}{}

\href{https://www.nytimes3xbfgragh.onion/section/todayspaper}{Today's
Paper}

\href{/section/climate}{Climate}\textbar{}Loss of Greenland Ice Sheet
Reached a Record Last Year

\url{https://nyti.ms/2FIGmar}

\begin{itemize}
\item
\item
\item
\item
\item
\end{itemize}

\hypertarget{climate-and-environment}{%
\subsubsection{\texorpdfstring{\href{https://www.nytimes3xbfgragh.onion/section/climate?name=styln-climate\&region=TOP_BANNER\&variant=undefined\&block=storyline_menu_recirc\&action=click\&pgtype=Article\&impression_id=a753f690-e3a8-11ea-a20a-a7f4fdc1a30f}{Climate
and
Environment}}{Climate and Environment}}\label{climate-and-environment}}

\begin{itemize}
\tightlist
\item
  \href{https://www.nytimes3xbfgragh.onion/2020/08/17/climate/alaska-oil-drilling-anwr.html?name=styln-climate\&region=TOP_BANNER\&variant=undefined\&block=storyline_menu_recirc\&action=click\&pgtype=Article\&impression_id=a753f691-e3a8-11ea-a20a-a7f4fdc1a30f}{Arctic
  Refuge}
\item
  \href{https://www.nytimes3xbfgragh.onion/interactive/2020/climate/trump-environment-rollbacks.html?name=styln-climate\&region=TOP_BANNER\&variant=undefined\&block=storyline_menu_recirc\&action=click\&pgtype=Article\&impression_id=a753f692-e3a8-11ea-a20a-a7f4fdc1a30f}{Trump's
  Changes}
\item
  \href{https://www.nytimes3xbfgragh.onion/interactive/2020/04/19/climate/climate-crash-course-1.html?name=styln-climate\&region=TOP_BANNER\&variant=undefined\&block=storyline_menu_recirc\&action=click\&pgtype=Article\&impression_id=a7541da0-e3a8-11ea-a20a-a7f4fdc1a30f}{Climate
  101}
\item
  \href{https://www.nytimes3xbfgragh.onion/interactive/2018/08/30/climate/how-much-hotter-is-your-hometown.html?name=styln-climate\&region=TOP_BANNER\&variant=undefined\&block=storyline_menu_recirc\&action=click\&pgtype=Article\&impression_id=a7541da1-e3a8-11ea-a20a-a7f4fdc1a30f}{Is
  Your Hometown Hotter?}
\end{itemize}

Advertisement

\protect\hyperlink{after-top}{Continue reading the main story}

Supported by

\protect\hyperlink{after-sponsor}{Continue reading the main story}

\hypertarget{loss-of-greenland-ice-sheet-reached-a-record-last-year}{%
\section{Loss of Greenland Ice Sheet Reached a Record Last
Year}\label{loss-of-greenland-ice-sheet-reached-a-record-last-year}}

The ice loss in 2019 was more than twice the annual average since 2003,
scientists said.

\includegraphics{https://static01.graylady3jvrrxbe.onion/images/2020/08/20/climate/20CLI-GREENLAND2/merlin_175818159_1541e7d8-f7df-4e7a-b368-69110772f8a1-articleLarge.jpg?quality=75\&auto=webp\&disable=upscale}

By \href{https://www.nytimes3xbfgragh.onion/by/henry-fountain}{Henry
Fountain}

\begin{itemize}
\item
  Aug. 20, 2020
\item
  \begin{itemize}
  \item
  \item
  \item
  \item
  \item
  \end{itemize}
\end{itemize}

Greenland lost a record amount of ice in 2019, researchers reported
Thursday. Nearly half of it was lost in July, when the region roasted
from an unusual heat wave.

The net ice loss of more than 530 billion metric tons was more than
twice the annual average since 2003, the scientists said. In July,
\href{https://www.nytimes3xbfgragh.onion/2019/08/02/climate/european-heatwave-climate-change.html}{when
warm air from Europe moved north}, leading to temperatures that were
well above normal and causing widespread surface melting of the ice
sheet, the loss was roughly equal to the average loss in a full year.

Ingo Sasgen, a geoscientist at the Alfred Wegener Institute in
Bremerhaven, Germany, and the lead author of a paper
\href{https://www.nature.com/articles/s43247-020-0010-1}{describing the
findings} in the journal Communications and Environment, said with the
warmth last summer, he and his colleagues suspected that 2019 would be a
bad one for the ice sheet.

They analyzed data from a pair of satellites that precisely measure the
gravitational pull, and thus the mass, of the area they are orbiting
over.

``It took us some time to analyze it and quantify it robustly, but it
turned out to be another record melt year,'' Dr. Sasgen said. In the
previous record year, 2012, the net loss was about 460 billion metric
tons.

Yara Mohajerani, a postdoctoral researcher at the University of
Washington who was not involved in the study, said it was ``part of a
series of studies that have shown the same thing,'' including work that
he did that
\href{https://agupubs.onlinelibrary.wiley.com/doi/abs/10.1029/2020GL087291}{reported
record ice loss in summer 2019}.

\href{https://www.nytimes3xbfgragh.onion/section/climate?action=click\&pgtype=Article\&state=default\&region=MAIN_CONTENT_1\&context=storylines_keepup}{}

\hypertarget{climate-and-environment-}{%
\subsubsection{Climate and Environment
›}\label{climate-and-environment-}}

\hypertarget{keep-up-on-the-latest-climate-news}{%
\paragraph{Keep Up on the Latest Climate
News}\label{keep-up-on-the-latest-climate-news}}

Updated Aug. 18, 2020

Here's what you need to know this week:

\begin{itemize}
\item
  \begin{itemize}
  \tightlist
  \item
    Five automakers
    \href{https://www.nytimes3xbfgragh.onion/2020/08/17/climate/california-automakers-pollution.html?action=click\&pgtype=Article\&state=default\&region=MAIN_CONTENT_1\&context=storylines_keepup}{sealed
    a binding agreement} with California to follow the state's stricter
    tailpipe emissions rules.
  \item
    The Trump
    administration\href{https://www.nytimes3xbfgragh.onion/2020/08/13/climate/trump-methane.html?action=click\&pgtype=Article\&state=default\&region=MAIN_CONTENT_1\&context=storylines_keepup}{eliminated
    a major methane rule}, even as leaks are worsening, in a decision
    that researchers warned ignored science.
  \item
    Climate change leaders said
    \href{https://www.nytimes3xbfgragh.onion/2020/08/12/climate/kamala-harris-environmental-justice.html?action=click\&pgtype=Article\&state=default\&region=MAIN_CONTENT_1\&context=storylines_keepup}{the
    vice-presidential choice of Kamala Harris} signaled that Democrats
    will have a focus on environmental justice.
  \end{itemize}
\end{itemize}

Greenland's ice sheet is nearly two miles thick in places, and if all
the ice were to melt, sea levels would rise about 24 feet, or about 7.5
meters.

That would take centuries. But since the 1990s, as the Arctic has warmed
faster than any other part of the planet, ice loss from Greenland and
its contribution to sea level rise have accelerated. At the current rate
of loss, Greenland's ice accounts for about one-quarter inch per decade
of the global total increase of about
\href{https://climate.nasa.gov/vital-signs/sea-level/}{one and a quarter
inches per decade}.

But ice loss can vary from year to year. In their paper, Dr. Sasgen and
his colleagues found that net loss in 2017 and 2018 was about half the
annual average since 2003.

So far in 2020, he said, net ice loss appears to be a little below
average.

Both 2017 and 2018 had colder-than-usual summers, Dr. Sasgen said, when
cold air flowed from the north along the west coast of Greenland,
reducing ice loss. But in 2019 that circulation pattern was reversed,
with warm air coming from the south.

\includegraphics{https://static01.graylady3jvrrxbe.onion/images/2020/08/20/climate/20CLI-GREENLAND1/20CLI-GREENLAND1-articleLarge.jpg?quality=75\&auto=webp\&disable=upscale}

Similar reversals have happened before. ``It was really fascinating that
it jumped again from very cold historic conditions to a record melt
year,'' he said.

The shift to north-flowing air occurs when a region of high-pressure
air, a result of changes in the jet stream, lingers over Greenland.
Referred to as a ``block,'' these zones of stationary air have become
more frequent in the Arctic, and while there is debate as to why, many
scientists are increasingly seeing a link to global warming that is made
worse by sea-ice loss in the Arctic Ocean.

``It's very clear that the last 10, 15, 20 years have produced more
stationary wave patterns and more blocking situations over Greenland,''
Dr. Sasgen said. ``There's a very likely chance it's connected to
sea-ice loss. But it's really hard to prove.''

In Greenland, ice loss results from runoff of surface meltwater and from
discharge of ice from glaciers that serve as outlets for the ice sheet,
connecting it to the ocean. Accumulation results from snowfall that,
compressed over years, eventually becomes ice. When runoff and discharge
exceed accumulation, the result is net loss.

\href{https://www.nature.com/articles/s43247-020-0001-2\#Sec9}{A paper
published last week} in the same journal showed that ice discharge from
outlet glaciers, which includes both calving of icebergs and underwater
melting, had increased by about 14 percent since the 1980s.

Most of the increase was from 2000 to 2005, and discharge rates have
remained relatively consistent at this higher level since then, said the
study's lead author, Michalea King, who recently earned her doctorate
from Ohio State University and will soon be a researcher at the
University of Washington.

The increase in ice discharge, coupled with the trend toward increasing
meltwater runoff over the past several decades, make it increasingly
unlikely that Greenland will have years with a net ice gain, Dr. King
said.

``It's kind of a double whammy,'' she said. ``Only one of every 100
years would we expect to have mass gain.''

The new paper, which Dr. King contributed to, illustrates that point,
she said. While the abnormal cold summers of 2017 and 2018 led to more
ice accumulation and less surface melt runoff, ``even with all of that
they're still mass-loss years,'' she said, largely because of the higher
glacier discharge rate.

``Mass loss is not going away anytime soon,'' Dr. King said. ``But of
course we have control over the rate" by taking steps to mitigate
climate change.

``It's not a throw-your-hands-up kind of situation,'' she said.

Advertisement

\protect\hyperlink{after-bottom}{Continue reading the main story}

\hypertarget{site-index}{%
\subsection{Site Index}\label{site-index}}

\hypertarget{site-information-navigation}{%
\subsection{Site Information
Navigation}\label{site-information-navigation}}

\begin{itemize}
\tightlist
\item
  \href{https://help.nytimes3xbfgragh.onion/hc/en-us/articles/115014792127-Copyright-notice}{©~2020~The
  New York Times Company}
\end{itemize}

\begin{itemize}
\tightlist
\item
  \href{https://www.nytco.com/}{NYTCo}
\item
  \href{https://help.nytimes3xbfgragh.onion/hc/en-us/articles/115015385887-Contact-Us}{Contact
  Us}
\item
  \href{https://www.nytco.com/careers/}{Work with us}
\item
  \href{https://nytmediakit.com/}{Advertise}
\item
  \href{http://www.tbrandstudio.com/}{T Brand Studio}
\item
  \href{https://www.nytimes3xbfgragh.onion/privacy/cookie-policy\#how-do-i-manage-trackers}{Your
  Ad Choices}
\item
  \href{https://www.nytimes3xbfgragh.onion/privacy}{Privacy}
\item
  \href{https://help.nytimes3xbfgragh.onion/hc/en-us/articles/115014893428-Terms-of-service}{Terms
  of Service}
\item
  \href{https://help.nytimes3xbfgragh.onion/hc/en-us/articles/115014893968-Terms-of-sale}{Terms
  of Sale}
\item
  \href{https://spiderbites.nytimes3xbfgragh.onion}{Site Map}
\item
  \href{https://help.nytimes3xbfgragh.onion/hc/en-us}{Help}
\item
  \href{https://www.nytimes3xbfgragh.onion/subscription?campaignId=37WXW}{Subscriptions}
\end{itemize}
