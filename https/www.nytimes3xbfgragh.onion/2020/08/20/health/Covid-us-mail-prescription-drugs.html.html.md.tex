Sections

SEARCH

\protect\hyperlink{site-content}{Skip to
content}\protect\hyperlink{site-index}{Skip to site index}

\href{https://www.nytimes3xbfgragh.onion/section/health}{Health}

\href{https://myaccount.nytimes3xbfgragh.onion/auth/login?response_type=cookie\&client_id=vi}{}

\href{https://www.nytimes3xbfgragh.onion/section/todayspaper}{Today's
Paper}

\href{/section/health}{Health}\textbar{}U.S. Mail Delays Slow Delivery
of Medicines

\url{https://nyti.ms/2FF89bD}

\begin{itemize}
\item
\item
\item
\item
\item
\end{itemize}

Advertisement

\protect\hyperlink{after-top}{Continue reading the main story}

Supported by

\protect\hyperlink{after-sponsor}{Continue reading the main story}

\hypertarget{us-mail-delays-slow-delivery-of-medicines}{%
\section{U.S. Mail Delays Slow Delivery of
Medicines}\label{us-mail-delays-slow-delivery-of-medicines}}

In addition to concerns over mail-in voting, the cost-cutting at the
Postal Service affects the millions of people who get their
prescriptions by mail.

\includegraphics{https://static01.graylady3jvrrxbe.onion/images/2020/08/20/science/20VIRUS-MAILDRUGS1/merlin_175862724_0042e56f-ec68-4c66-9b70-4597dedf5d21-articleLarge.jpg?quality=75\&auto=webp\&disable=upscale}

\href{https://www.nytimes3xbfgragh.onion/by/reed-abelson}{\includegraphics{https://static01.graylady3jvrrxbe.onion/images/2018/07/16/multimedia/author-reed-abelson/author-reed-abelson-thumbLarge.png}}

By \href{https://www.nytimes3xbfgragh.onion/by/reed-abelson}{Reed
Abelson}

\begin{itemize}
\item
  Aug. 20, 2020
\item
  \begin{itemize}
  \item
  \item
  \item
  \item
  \item
  \end{itemize}
\end{itemize}

To many people who rely on the mail to deliver their prescriptions, the
\href{https://www.nytimes3xbfgragh.onion/2020/08/18/us/politics/postal-service-suspends-changes.html?searchResultPosition=6}{latest
political skirmishing} over the U.S. Postal Service doesn't really
matter. They've been dealing with delays for weeks, and while some are
not urgent, others are more worrisome.

Dr. Toula Milios Guilfoyle, a retired physician in Jefferson, N.H., is
among those who say their mail-order prescriptions have become
dangerously late. Dr. Guilfoyle, who is 62 and disabled, needs
antibiotics for a chronic infection. While her prescriptions from the
Dartmouth-Hitchcock Medical Center usually arrive a few days after they
are sent, last month she had to wait two weeks to receive them, and
subsequent prescriptions have also been delayed.

Without the antibiotics she needed, her infection spread. ``I got worse
and worse,'' she said.

She initially thought the first delay was ``a fluke,'' but her
prescriptions have been late two more times. ``Everything has slowed
down,'' Dr. Guilfoyle said. She worries that the reports of the removal
of mailboxes and postal equipment that have already taken place will
continue to make the delivery of mail sluggish.

\href{https://www.nytimes3xbfgragh.onion/2020/08/15/us/post-office-vote-by-mail.html?action=click\&module=RelatedLinks\&pgtype=Article}{Most
of the criticism over the cost-cutting actions} taken by Postmaster
General Louis DeJoy, a major donor to President Trump, has focused on
whether the measures could jeopardize mail-in voting for the upcoming
election. But there is increasing recognition of the effect the cutbacks
would have on consumers who receive their medicines via the mail.

Nearly one in five Americans said they received medications through the
mail last week, according to
\href{https://www.ipsos.com/en-us/news-polls/axios-ipsos-coronavirus-index}{an
Axios-Ipsos poll} released Aug. 18. Of those, a quarter said they
experienced some delay or lack of delivery.

Although only 5 percent of the nation's retail prescriptions were
delivered to consumers by mail last year, the Postal Service handled
perhaps half of the volume, some 100 million prescriptions,
\href{https://twitter.com/DrugChannels/status/1296146532157272064?s=20}{by
one estimate.}

But use of mail order for prescriptions rose by 20 percent when the U.S.
outbreaks spread in March, compared with the previous year, as people
stockpiled medications during lockdown.

In addition, many of the drugs shipped are critical medicines for people
with chronic conditions like diabetes or high blood pressure.

``It's just unacceptable for the Senate to be in recess while
prescriptions continue to sit in sorting facilities,'' said Senator
Jeanne Shaheen, a New Hampshire Democrat and one of the lawmakers Dr.
Guilfoyle contacted. ``More than 3,000 constituents have contacted my
office recently about Postal Service delays, and they can't afford to
have the Trump administration playing politics with their mail and their
medications.''

Doctors and pharmacists also expressed their apprehension about patients
not getting their prescriptions in a timely manner, especially when many
are advised to stay at home during the coronavirus pandemic.

``Any disruption in the U.S. mail is of concern,'' said Dr. Jacqueline
Fincher, the president of the American College of Physicians, which
represents internists. ``Patients are being put at risk for no good
reason, it would seem,'' she said.

Missed doses could cause adverse health issues.

``If they go without for several days, the concern is always `Are you
going to have a bad outcome?''' said Dr. Fincher, who warned patients
might need to go to the hospital if their conditions significantly
worsened. ``This is not the time you want to be in the hospital for one
of your chronic conditions that is out of whack.''

\includegraphics{https://static01.graylady3jvrrxbe.onion/images/2020/08/20/science/20VIRUS-MAILDRUGS2/merlin_175343967_379f66b8-39b0-4f9c-86f3-e3648e0b130d-articleLarge.jpg?quality=75\&auto=webp\&disable=upscale}

Even though Mr. DeJoy announced earlier this week that he would suspend
some of the changes, Senator Gary Peters, a Michigan Democrat, plans to
hold an
\href{https://www.peters.senate.gov/newsroom/press-releases/peters-secures-senate-oversight-hearing-with-postmaster-general-dejoy}{oversight
hearing} on Friday to question the postmaster general.

``It's clear that the policies directed by Postmaster General DeJoy have
hurt people who use the Postal Service as a lifeline for everyday needs,
including Michigan seniors and veterans and people in rural communities
who rely on live-saving prescriptions they receive in the mail,'' Mr.
Peters said.

House Democrats
\href{https://www.nytimes3xbfgragh.onion/2020/08/16/us/politics/coronavirus-postal-service-stimulus-bill.html}{are
also scheduling a vote} on Saturday on legislation that would revoke
changes Mr. DeJoy has already made, and include \$25 billion in funding
for the agency.

Veterans have been particularly affected by mail delays because the
Department of Veterans Affairs relies on the Postal Service for
delivery. Jan Stowe, 74, of Traverse City, Mich., said that in July, for
the first time, she did not get her prescription before she ran out of
the Valium she takes for chronic muscle spasms. She suffered acute pain
for four days, she said. ``This medicine is to me lifesaving because it
keeps me upright and mobile,'' Ms. Stowe said.

Exactly how many people are being affected is unclear. The major
pharmacy benefit managers, including CVS Health, Express Scripts and
OptumRx, will only say they are closely watching the situation.

Some pharmacies say they are seeing some delays, like AllianceRX
Walgreens Prime, which is operated by Walgreens, the drugstore chain,
and Prime Therapeutics, a pharmacy benefit manager. Delays of three or
more days have been reported, although the partnership said their
customers were not suffering from any disruption in treatment.

``There has now been a noticeable difference,'' said Dr. Jessica
Nouhavandi, a pharmacist and a founder of an online pharmacy, Honeybee
Health, which says it has 50,000 customers. While there are competing
services like UPS or FedEx, the Postal Service ``is by far the most
cost-effective option for our patients,'' she said, and other shipping
companies depend on the U.S. mail service for the last leg of a
delivery.

Honeybee's rate of lost packages rose to about 5 percent in July from an
average of less than 1 percent in February, Dr. Nouhavandi said.
Prescription packages shipped via first-class mail took an average of 11
days, compared with just four or five before.

``Once it leaves our facility, it really is out of our control,'' she
said. The pharmacy is considering recommending that patients pay more
for two-day delivery to ensure they get their medications on time.

While the number of patients relying on their mail ``is not trivial,'' a
vast majority of patients are likely to be unaffected because they get
their prescriptions through retail pharmacies, said Adam Fein, the chief
executive of the Drug Channels Institute, which analyzes the
distribution of medicines. He calculates that of the 5 percent of
prescriptions in 2019 that were delivered by mail, 55 percent used the
U.S. mail because the type of medicine did not need special handling.

Although some patients turned to mail order because of the pandemic and
the concern about leaving their home to go to a pharmacy, Mr. Fein said
these delays were not likely to have a widespread effect on patients
because most health plans allow people to refill a prescription early or
go to a pharmacy if they run out. ``There's really very little
likelihood that people would not get access to prescriptions,'' he said.

The Medicare Rights Center, a consumer group, suggests people who switch
should still try to reduce their risk of exposure to the virus by
choosing a retail pharmacy that offers home delivery or curbside pickup.
It also recommends people check their health plan to make sure they have
the option of a retail pharmacy and to see how much their medications
will cost if they make the change.

And not everyone has easy access to a pharmacy, said Dr. Scott J. Knoer,
a pharmacist who is the chief executive of the American Pharmacists
Association. Many urban and rural areas have ``pharmacy deserts,'' he
said, where residents do not live near a drugstore.

All of this makes patients and their families anxious, said Laura
Hatcher, the director of communications for the Little Lobbyists, a
group representing children with complex medical needs. ``We're just
starting to see people have real issues with the mail,'' she said.

Many of the medications these children need may not be available at the
local drugstore, because they are a controlled substance or must be
individually compounded. If there are delays at the Postal Service
``then it will be an enormous problem,'' Ms. Hatcher said.

Advertisement

\protect\hyperlink{after-bottom}{Continue reading the main story}

\hypertarget{site-index}{%
\subsection{Site Index}\label{site-index}}

\hypertarget{site-information-navigation}{%
\subsection{Site Information
Navigation}\label{site-information-navigation}}

\begin{itemize}
\tightlist
\item
  \href{https://help.nytimes3xbfgragh.onion/hc/en-us/articles/115014792127-Copyright-notice}{©~2020~The
  New York Times Company}
\end{itemize}

\begin{itemize}
\tightlist
\item
  \href{https://www.nytco.com/}{NYTCo}
\item
  \href{https://help.nytimes3xbfgragh.onion/hc/en-us/articles/115015385887-Contact-Us}{Contact
  Us}
\item
  \href{https://www.nytco.com/careers/}{Work with us}
\item
  \href{https://nytmediakit.com/}{Advertise}
\item
  \href{http://www.tbrandstudio.com/}{T Brand Studio}
\item
  \href{https://www.nytimes3xbfgragh.onion/privacy/cookie-policy\#how-do-i-manage-trackers}{Your
  Ad Choices}
\item
  \href{https://www.nytimes3xbfgragh.onion/privacy}{Privacy}
\item
  \href{https://help.nytimes3xbfgragh.onion/hc/en-us/articles/115014893428-Terms-of-service}{Terms
  of Service}
\item
  \href{https://help.nytimes3xbfgragh.onion/hc/en-us/articles/115014893968-Terms-of-sale}{Terms
  of Sale}
\item
  \href{https://spiderbites.nytimes3xbfgragh.onion}{Site Map}
\item
  \href{https://help.nytimes3xbfgragh.onion/hc/en-us}{Help}
\item
  \href{https://www.nytimes3xbfgragh.onion/subscription?campaignId=37WXW}{Subscriptions}
\end{itemize}
