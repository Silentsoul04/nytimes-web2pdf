Sections

SEARCH

\protect\hyperlink{site-content}{Skip to
content}\protect\hyperlink{site-index}{Skip to site index}

\href{https://www.nytimes3xbfgragh.onion/section/nyregion}{New York}

\href{https://myaccount.nytimes3xbfgragh.onion/auth/login?response_type=cookie\&client_id=vi}{}

\href{https://www.nytimes3xbfgragh.onion/section/todayspaper}{Today's
Paper}

\href{/section/nyregion}{New York}\textbar{}323,911 Accusations of
N.Y.P.D. Misconduct Are Released Online

\url{https://nyti.ms/3gcA9jr}

\begin{itemize}
\item
\item
\item
\item
\item
\item
\end{itemize}

\hypertarget{race-and-america}{%
\subsubsection{\texorpdfstring{\href{https://www.nytimes3xbfgragh.onion/news-event/george-floyd-protests-minneapolis-new-york-los-angeles?name=styln-george-floyd\&region=TOP_BANNER\&variant=undefined\&block=storyline_menu_recirc\&action=click\&pgtype=Article\&impression_id=b3286630-e38b-11ea-b524-5975769cbb40}{Race
and America}}{Race and America}}\label{race-and-america}}

\begin{itemize}
\tightlist
\item
  \href{https://www.nytimes3xbfgragh.onion/interactive/2020/07/03/us/george-floyd-protests-crowd-size.html?name=styln-george-floyd\&region=TOP_BANNER\&variant=undefined\&block=storyline_menu_recirc\&action=click\&pgtype=Article\&impression_id=b3286631-e38b-11ea-b524-5975769cbb40}{Black
  Lives Matter Movement}
\item
  \href{https://www.nytimes3xbfgragh.onion/interactive/2020/06/28/us/i-cant-breathe-police-arrest.html?name=styln-george-floyd\&region=TOP_BANNER\&variant=undefined\&block=storyline_menu_recirc\&action=click\&pgtype=Article\&impression_id=b3286632-e38b-11ea-b524-5975769cbb40}{History
  of `I Can't Breathe'}
\item
  \href{https://www.nytimes3xbfgragh.onion/interactive/2020/06/10/upshot/black-lives-matter-attitudes.html?name=styln-george-floyd\&region=TOP_BANNER\&variant=undefined\&block=storyline_menu_recirc\&action=click\&pgtype=Article\&impression_id=b3286633-e38b-11ea-b524-5975769cbb40}{How
  Public Opinion Shifted}
\item
  \href{https://www.nytimes3xbfgragh.onion/interactive/2020/07/16/us/black-lives-matter-protests-louisville-breonna-taylor.html?name=styln-george-floyd\&region=TOP_BANNER\&variant=undefined\&block=storyline_menu_recirc\&action=click\&pgtype=Article\&impression_id=b3286634-e38b-11ea-b524-5975769cbb40}{45
  Days in Louisville}
\end{itemize}

Advertisement

\protect\hyperlink{after-top}{Continue reading the main story}

Supported by

\protect\hyperlink{after-sponsor}{Continue reading the main story}

\hypertarget{323911-accusations-of-nypd-misconduct-are-released-online}{%
\section{323,911 Accusations of N.Y.P.D. Misconduct Are Released
Online}\label{323911-accusations-of-nypd-misconduct-are-released-online}}

The records had been sealed for decades, but last month, New York
repealed a law keeping them secret after national protests against
police brutality.

\includegraphics{https://static01.graylady3jvrrxbe.onion/images/2020/07/30/nyregion/00nypdrecords-HFO/merlin_173317527_0c1a0c62-cc65-4ba0-98e8-b6fe599309aa-articleLarge.jpg?quality=75\&auto=webp\&disable=upscale}

\href{https://www.nytimes3xbfgragh.onion/by/ashley-southall}{\includegraphics{https://static01.graylady3jvrrxbe.onion/images/2018/02/20/multimedia/author-ashley-southall/author-ashley-southall-thumbLarge.jpg}}

By \href{https://www.nytimes3xbfgragh.onion/by/ashley-southall}{Ashley
Southall}

\begin{itemize}
\item
  Aug. 20, 2020
\item
  \begin{itemize}
  \item
  \item
  \item
  \item
  \item
  \item
  \end{itemize}
\end{itemize}

Over 323,000 accusations of misconduct against current and former New
York City police officers were
\href{https://www.nyclu.org/en/campaigns/nypd-misconduct-database}{published
online} on Thursday, a major milestone in a long and contentious
political battle to open records of police discipline to public
scrutiny.

The records include all civilian complaints filed since 1985 with the
city's independent police watchdog agency, the Civilian Complaint Review
Board, and closed after an investigation.

Some 81,550 officers --- from the rank-and-file to the current
commissioner --- were named in the complaints. Together they offer the
public the broadest look to date at how officers are investigated and
punished for a range of offenses, from using profanity and slurs to
beating or choking people during arrests.

The complaints were published in an online database by the New York
Civil Liberties Union, which obtained the records from the review board
after state lawmakers
\href{https://www.nytimes3xbfgragh.onion/2020/06/12/nyregion/50a-repeal-police-floyd.html}{repealed
a law} that had kept them secret.

The civil liberties union noted that less than 3 percent of the 323,911
complaints resulted in a penalty for officers, 12 of whom had been
terminated. Christopher Dunn, the organization's legal director, said in
a statement that the records showed that the Police Department, whose
commissioner makes the final decision on disciplinary matters, ``is
unwilling to police itself.''

``The release of this database is an important step towards greater
transparency and accountability,'' Mr. Dunn added, ``and is just the
beginning of unraveling the monopoly the N.Y.P.D. holds on public
information and officer discipline.''

Al Baker, a police spokesman, said the department had refined the
discipline system and implemented changes recommended by a panel of
prosecutors and judges. ``All of this advances the NYPD's priority to
make its internal disciplinary system as fair, effective, and
transparent as it can,'' he said.

Fred Davie, the chairman of the review board, said in a statement that
the agency released the records in response to demands from the public
for greater police accountability, which have been underscored by the
nationwide protests following the death of George Floyd in police
custody in Minneapolis.

``All New Yorkers have a right to transparency'' under the state law
granting access to public records, said Mr. Davie, who promised that his
agency ``will hold paramount the people's right to know how their
communities are policed.''

But that might become even more difficult for the review board, which is
preparing to possibly lay off staff because of budget cuts brought on by
the coronavirus pandemic. The Independent Budget Office recently said
the review board, which has a \$20 million budget and 200-person staff,
was
\href{https://www.nydailynews.com/new-york/nyc-crime/ny-nypd-watchdog-agency-budget-cut-layoffs-20200812-3w2hbgkgnfhybkjawmuucrc5qm-story.html}{too
small to oversee the police} department, which has a \$6 billion budget
and 36,000 officers.

The records released Thursday include all allegations of excessive
force, abuse of authority, discourtesy and offensive language
investigated by the review board though mid-July, as well as the board's
findings and any discipline that the police commissioner imposed. They
do not include complaints under investigation by the review board or
those investigated by the Police Department itself.

The complaints were shrouded in secrecy until June, when, as protests
against police brutality spread across the country, the State
Legislature in New York repealed a 44-year-old law that had been used to
prevent their release to the public.

After a legal challenge from labor unions representing police officers,
firefighters and corrections officers whose records were shielded by the
law, a federal appeals court on Thursday ruled that the data could be
released while the case continued in a lower court.

The unions vowed to continue fighting against what Hank Sheinkopf, a
spokesman for the union coalition, said was ``the improper dumping of
thousands of documents containing unproven, career-damaging,
unsubstantiated allegations that put our members and their families at
risk.''

The publication of the records, policing experts said, chips away at a
legal wall of confidentiality built up by police unions, which for
decades have used their political clout to block efforts to publicly
release complaints about officers and the punishment they receive.

Samuel Walker, a professor at the University of Nebraska Omaha who is a
leading expert on police accountability, said the data would allow
academic researchers and policymakers to identify patterns and problems.

``That provides the fodder for policy changes, and that is terribly
important,'' he said.

The topic of how and whether to disclose police disciplinary records has
been contentious for decades. States like Delaware have laws keeping the
records secret, while others like Florida and Arizona permit the release
of some or all records, according to the Police Executive Research
Forum, a policy and research nonprofit.

``Departments have come to recognize that's part of what transparency
looks like,'' Chuck Wexler, the executive director, said.

The records became a lightning-rod issue when Mayor Bill de Blasio's
administration fought against disclosing those of Daniel Pantaleo, a
police officer who put Eric Garner in a banned chokehold on Staten
Island in 2014 that proved fatal.

Officer Pantaleo was fired last year, but the law the city relied on to
shield his records did not change until this past June, after the death
of Mr. Floyd.

The records released Thursday add to a patchwork of data about police
misconduct in New York City created by media and legal organizations.

ProPublica published
\href{https://projects.propublica.org/nypd-ccrb/}{data on nearly 4,000
officers} who had at least one allegation of misconduct substantiated
after an investigation, and BuzzFeed in 2018 published leaked records
\href{https://www.buzzfeednews.com/article/kendalltaggart/nypd-police-misconduct-database}{of
1,800 cases}. The Legal Aid Society also maintains a
\href{https://www.capstat.nyc/}{database of federal lawsuits} against
officers.

Most people who have unpleasant encounters with the police do not file
formal complaints. For many who have filed formal complaints that the
review board substantiated, the publication of the records offers an
opportunity to finally learn what punishment, if any, those officers
received.

And for those who are facing criminal charges based on officers'
testimony, or who have filed lawsuits against the city accusing police
of misconduct and abuse, the records make it easier to learn more about
an officer's history.

Richard Emery, a former chairman of the review board, said the data
provided a treasure trove for researchers and policymakers.

But the release of the records also puts pressure on the Police
Department to change how unsubstantiated claims affect officers'
careers, he said. Even if not proven true, misconduct complaints can
have a negative effect on officers' opportunities for transfers,
promotions and more desirable assignments.

``It's a very, very complicated, messy problem that has a lot of strings
to it,'' Mr. Emery said. ``The release of these statistics and
information can be something that's useful and it can support reform,
but it also can be something that undermines reform and puts the police
in the position where they cannot and will not do their job.''

The police unions have argued in court that no information about
misconduct cases should be released in which the police commissioner has
not ended up handing down punishment. But the repeal of the secrecy law,
known as 50-a, ``made clear that disciplinary records can now be made
available to the public,'' James E. Johnson, the city's corporation
counsel said in a statement Thursday.

The unions' **** argument would have prevented the release of the vast
majority of the review board's data. Only in 8,699 of the more than
300,000 cases did officers receive any disciplinary action, according to
the civil liberties union. **** The penalties can range from a letter in
their personnel file to suspension or firing, according to the review
board.

Reasons for this vary, but the majority of civilian complaints end
because investigators did not have enough evidence to determine what
actually happened or found evidence to disprove the allegations, or the
person filing the complaint withdrew from the investigation.

The Police Department often
\href{https://www.propublica.org/article/the-nypd-is-withholding-evidence-from-investigations-into-police-abuse}{withholds
some evidence}, including body camera footage. The review board has said
cases in which there is video are more likely to be substantiated.

Kate Levine, a criminal law professor at Yeshiva University, said the
data should not be used to ``name and shame'' officers, but to
understand how policing is in need of fundamental change.

"Let's get rid of the worst people first,'' she said, ``but also let's
take those dollars and reinvest them in the many different ways that we
can better the lives of marginalized communities.''

Advertisement

\protect\hyperlink{after-bottom}{Continue reading the main story}

\hypertarget{site-index}{%
\subsection{Site Index}\label{site-index}}

\hypertarget{site-information-navigation}{%
\subsection{Site Information
Navigation}\label{site-information-navigation}}

\begin{itemize}
\tightlist
\item
  \href{https://help.nytimes3xbfgragh.onion/hc/en-us/articles/115014792127-Copyright-notice}{©~2020~The
  New York Times Company}
\end{itemize}

\begin{itemize}
\tightlist
\item
  \href{https://www.nytco.com/}{NYTCo}
\item
  \href{https://help.nytimes3xbfgragh.onion/hc/en-us/articles/115015385887-Contact-Us}{Contact
  Us}
\item
  \href{https://www.nytco.com/careers/}{Work with us}
\item
  \href{https://nytmediakit.com/}{Advertise}
\item
  \href{http://www.tbrandstudio.com/}{T Brand Studio}
\item
  \href{https://www.nytimes3xbfgragh.onion/privacy/cookie-policy\#how-do-i-manage-trackers}{Your
  Ad Choices}
\item
  \href{https://www.nytimes3xbfgragh.onion/privacy}{Privacy}
\item
  \href{https://help.nytimes3xbfgragh.onion/hc/en-us/articles/115014893428-Terms-of-service}{Terms
  of Service}
\item
  \href{https://help.nytimes3xbfgragh.onion/hc/en-us/articles/115014893968-Terms-of-sale}{Terms
  of Sale}
\item
  \href{https://spiderbites.nytimes3xbfgragh.onion}{Site Map}
\item
  \href{https://help.nytimes3xbfgragh.onion/hc/en-us}{Help}
\item
  \href{https://www.nytimes3xbfgragh.onion/subscription?campaignId=37WXW}{Subscriptions}
\end{itemize}
