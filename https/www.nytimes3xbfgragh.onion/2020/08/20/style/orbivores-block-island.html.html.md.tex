\href{/section/style}{Style}\textbar{}The Treasure Hunters of Block
Island

\url{https://nyti.ms/34kEezN}

\begin{itemize}
\item
\item
\item
\item
\item
\item
\end{itemize}

\includegraphics{https://static01.graylady3jvrrxbe.onion/images/2020/08/20/fashion/20ORBIVORES2/merlin_175838955_a558ddd9-0970-4b5d-a64f-0dc2f8b46a41-articleLarge.jpg?quality=75\&auto=webp\&disable=upscale}

Sections

\protect\hyperlink{site-content}{Skip to
content}\protect\hyperlink{site-index}{Skip to site index}

\hypertarget{the-treasure-hunters-of-block-island}{%
\section{The Treasure Hunters of Block
Island}\label{the-treasure-hunters-of-block-island}}

In this summer of sorrow, a game searching for glass orbs has taken on
new and magical meaning.

Crystal ball: Eben Horton, a glass artist, hides his creations for
visitors to find and collect.Credit...Jillian Freyer for The New York
Times

Supported by

\protect\hyperlink{after-sponsor}{Continue reading the main story}

By Hillary Richard

\begin{itemize}
\item
  Published Aug. 20, 2020Updated Aug. 21, 2020, 7:27 a.m. ET
\item
  \begin{itemize}
  \item
  \item
  \item
  \item
  \item
  \item
  \end{itemize}
\end{itemize}

BLOCK ISLAND, R.I. --- In a forest here, Katie Hall and members of her
extended family were dangling her mother's bra into a hollowed-out tree
trunk with the laser focus of open-heart surgeons. When that failed, Ms.
Hall, her mother, brother, aunt and cousin untied their shoes, using the
laces to form a makeshift net in the hopes of picking up something just
out of arm's reach. Two hours of MacGyver-ing and one downpour later,
success arrived, with the help of a spaghetti spoon taped to a branch.

On another part of the island, Dawn Holmes walked back up a dirt path
after completing her mission. A group of women approached, still reeling
from getting lost at Rodman's Hollow without cellphone signal. Ms.
Holmes lingered casually, adjusting her now empty backpack. When the
women discovered what she'd hidden, screams of glee pierced the air.

Hundreds of beautiful handblown glass fishing floats are scattered
across the wilderness of Block Island each year, attracting an
underground society of fanatics who will stop at nothing to find them.

These are the orbivores. Eben Horton is their Pied Piper.

\includegraphics{https://static01.graylady3jvrrxbe.onion/images/2020/08/20/fashion/20ORBIVORES1/merlin_175855686_2df9e801-fa50-4e09-9ffc-315dde234c42-articleLarge.jpg?quality=75\&auto=webp\&disable=upscale}

``I love the term `orbivore.' They call them orbs. Technically they're
floats, but sure, they're orbs,'' Mr. Horton said, with a laugh. ``It's
like a tribe. I'm absolutely amazed and humbled and proud but, like,
what have I created? Am I a cult leader? I had no idea that this would
turn into this.''

Mr. Horton, a 46-year-old glass artist, started the Glass Float Project
in 2012 as a whimsical scavenger hunt, just for fun. He spent his
childhood sailing the 12 miles from mainland Rhode Island to Block
Island, a 7,000-acre dot in the Atlantic. For roughly a century, foreign
fishermen attached glass floats to their nets. The floats inevitably
detached and traveled across the sea, washing up in fated spots after
unimaginable journeys. Mr. Horton, an avid beachcomber, loves mulling
the history of these pieces.

The first year, Mr. Horton handmade 150 glass floats at his studio in
Wakefield, R.I. The event began that June, gathering unanticipated
momentum with locals and tourists. Block Island swells from under 1,000
year-round residents to roughly 20,000 visitors per day at the height of
summer. People come for the beaches and restaurants, but most never
explore the deceptively wild and desolate natural areas. The scavenger
hunt became a gateway to a different world.

Now in its eighth year, the Glass Float Project has worked its way into
the daily flow of Mr. Horton's gallery. While waiting for the furnace to
get hot each morning, he, his wife, Jennifer Nauck, and their two
assistants fill in time gaps by making floats.

The four-person crew makes 550 highly coveted floats each year. They
etch the year and a number onto each with a Dremel bit. Most are clear
glass. The annual No. 1 typically has gilding.

A certain amount (corresponding with the current year) are colored.
Found floats are registered on the island's website to keep track of
what's still at large. Orbivores are held to an honor system: only one
float per person per year, to keep things fair.

\hypertarget{hunting-the-rona}{%
\subsection{Hunting the Rona}\label{hunting-the-rona}}

For 2020, Mr. Horton felt inspired to create something beautiful and
lighthearted in a year that has been anything but. This year's No. 1 is
a colorful coronavirus particle, which he calls ``the Rona.'' A typical
glass float takes 10 minutes to make, but the Rona took Mr. Horton
several hours.

The orbivores went wild, asking whether he would make more available for
purchase. As a compromise, Mr. Horton created a second Rona to be
raffled off, with proceeds helping to fund future floats.

The Glass Float Project (which costs an estimated \$20,000 per year) is
funded largely by donations with help from the Block Island Tourism
Council. Mr. Horton has been offered plenty of opportunities to sponsor
the project, but he steadfastly refuses to commercialize it in any way.
It is an art adventure for the people, and he intends to keep his labor
of love humble.

Image

An orb in the midst of creation.Credit...Jillian Freyer for The New York
Times

Image

Orbs hidden like Easter eggs.Credit...Jillian Freyer for The New York
Times

While orb hunting is a year-round activity, Mr. Horton starts each
summer ``season'' by packing a box of 100 sturdy glass floats onto his
sailboat and heading to Block Island, like some kind of seafaring Santa
Claus. He is easily recognizable and highly accessible, through both his
glass studio and an online group he started. The orbivores openly adore
him, which has some drawbacks when it comes to covert operations like
hiding.

``These orbivores know what he drives, what he sails, they constantly
see him,'' said Jessica Willi, the director of the tourism council. ``If
he's coming to Block Island, they're coming to Block Island. Poor guy
can't go anywhere.''

Ms. Willi is one of several top-secret friends Mr. Horton trusts to help
hide orbs. Over the years, she's had to make some life adjustments. She
uses a different car when she hides, so nobody recognizes her. She tries
not to post photos of her leisure walks, because her more suspicious
friends have rushed in after her.

``Everyone on the trails thinks that if you're not hunting, you're
hiding,'' she said. ``The orbivores have their special walking sticks
they poke in the bushes, because there's poison ivy. They have their
water bottles. They pack lunches. They're prepared.''

To dodge them, Ms. Willi has occasionally ventured out in the dead of
night with only a headlamp. Darkness didn't stop Katie Nelson, 53, a
retired sales and marketing director from Montana who once hunted under
a full moon with her flashlight in the hopes that the light would
reflect off any glass. (It rained, and she returned empty-handed.)

Hiding spots are completely up to the hider, outside of some logistical
rules that developed over time. There is no hiding on private property,
near the school or in any environmentally sensitive areas (like the
dunes, beach grass or bluffs). Everything else is up to chance. Even Mr.
Horton has no idea where most of the orbs are. He prefers it that way,
to keep himself honest when orbivores attempt to pry him for
information.

``Sometimes I'll put a smaller one in a fisherman's glove on the beach,
or in a lobster pot,'' Mr. Horton said.

Image

Like some kind of seafaring Santa Claus: Mr. Horton.Credit...Jillian
Freyer for The New York Times

``Twice, I've dropped some off the ferry so they'd wash up on Block
Island. One was found later on Long Island,'' he said. ``At that point,
am I littering? Is that cool or is it bad? I don't know.''

Mr. Horton likes hiding spots where the glass orbs are easy to see but
hard to retrieve. He frequently hides some in an island cannon, because
they roll all the way to the back.

Tricia Serio, 50, a professor and dean at the University of
Massachusetts Amherst, who lives in Leverett, Mass., was driving back
from an ice cream date with her husband and two sons when her boys
suddenly insisted that she stop the car near a cannon. One hour (and
lots of screaming) later, the mosquito-bitten foursome was still
standing around in the dark debating how to get a float out.

``I went to our rental house and brought back a ladle and some duct
tape, which did the trick,'' Ms. Serio said. ``Everyone was ecstatic.
Later that night that we remembered that they're called floats and that
we should have poured water into the cannon.''

After coming home orb-less on his first island visit in 2017, Bill
Holbrook spent hours devising a ``divide and conquer'' strategy. He
compiled five years' worth of data from the town website, built queries
to generate statistically relevant hiding spots and created a heat map
of the island.

In 2018, he covered 22 miles in 48 hours, with no luck. In 2019, after
another unsuccessful day left him physically and emotionally exhausted,
his girlfriend, Lisa, suggested they head back via a stretch of sand
near Payne's Dock. There, Lisa casually found an orb hidden in an old
tire.

``I now believe it's more about karma than data,'' said Mr. Holbrook,
52, a marketing director from South Weymouth, Mass. ``Was this little
stretch of sand on my heat map? No way. Was this little treasure meant
to be found by Lisa? Absolutely.''

This is a common theme of orbivore stories: The floats appear when you
least expect them to, or when you've lost your will, or when you need a
little wink from the universe.

Last winter, Isaac Ariel, 67, a retired I.T. professional and Block
Island resident, casually found a No. 61 float on the beach, two days
before the 61st birthday of his orbivore wife, Susan.

``Finding a float is a bit like finding love,'' said Ms. Holmes, a
retired educator from Newburyport, Mass., who honeymooned on the island.
``You have to be open to it and not try too hard, but then when the
moment is right it finds you.''

Ms. Holmes, who is also a hider, found her first float unexpectedly
after days of intensely scouring the trails. She spotted a snake in the
grass one afternoon. After a brief moment of panic, she realized it was
wrapped around a float, like it was some kind of glass apple in the
Garden of Eden.

Mr. Horton likes to say that the point to all of this is there is no
point. It's about the journey, being outside, enjoying the island.
Orbivores may be obsessive, but they take that to heart. Some go so far
as to restage floats they've found for their loved ones, like
grandparents who can no longer walk far, or small children with high
hopes.

In 2018, Nicole Dorfman's boyfriend at the time, Mark, hatched a plan to
propose via orb. He ordered an inscribed float through Mr. Horton's
studio, then sneaked it into a spot alongside some canoes at Fresh Pond.

Image

Hands on: Mr. Horton and his wife ready an orb ready for
hiding.Credit...Jillian Freyer for The New York Times

``I found it and obviously said yes, but I knew it didn't truly count
since it wasn't a real orb,'' said Ms. Dorfman, 27, a third-grade
teacher from Schaumburg, Ill., who has been searching since 2012. Last
July, after an anniversary dinner celebration, she and her family
stopped by Island Cemetery on a whim. She found an orb between a rock
wall and a tree.

The orb hunting Facebook group (started by Mr. Horton) is a rarity on
social media: thousands of strangers with a common obsession and no
off-topic rants. It is a hyper-focused, passionate community that
somehow --- even in 2020 --- remains supportive and positive.

People post photo updates of their seeking missions. Fellow orbivores
celebrate their successes and lament their failures. Occasionally, Mr.
Horton will pop in with a clue, which he believes ruins the fun, but the
group disagrees so he acquiesces.

Leah Melius, a 38-year-old physical therapist from Essex, Conn., has
been orb hunting with her family since 2012. One day in 2017, her phone
dinged with a new hint on the Facebook page, where Mr. Horton had posted
a photo with a jetty she recognized. They piled into two cars and sped
across the island from Mohegan Bluffs to Charleston Beach.

``We felt like we were on `The Amazing Race.' We couldn't believe our
eyes when it was still there!'' said Ms. Melius, whose mother now
displays the float proudly in her living room. ``Everyone was cheering
and high five-ing and crying tears of joy.''

Josie Lawrence, 15, has been orb hunting since she was 9. She was
unsuccessful until 2018, when after an online hint and some quick
sleuthing, she and her mother raced over to the beach near the North
Light lighthouse. With no orb in sight, they watched a seal play
offshore instead. On the way back to the car, Josie noticed something
glinting inside a log.

``At that moment I started crying,'' she said. ``Though it took a lot of
time and hard work, all the summers of long walks and hikes have been
completely worth it.'' Mr. Horton and his hiders enjoy keeping up with
the orbivores online, where they get to put a face to a found float. He
thinks a lot about how this project took on an effortless life of its
own and why it resonates so deeply with so many.

``I think people want answers for everything in life now,'' he said.
``The thing with the glass floats is there are no answers. You don't
find them, they find you.''

``It's all just a magical thing. People \emph{want} answers but people
\emph{need} some magic.''

Advertisement

\protect\hyperlink{after-bottom}{Continue reading the main story}

\hypertarget{site-index}{%
\subsection{Site Index}\label{site-index}}

\hypertarget{site-information-navigation}{%
\subsection{Site Information
Navigation}\label{site-information-navigation}}

\begin{itemize}
\tightlist
\item
  \href{https://help.nytimes3xbfgragh.onion/hc/en-us/articles/115014792127-Copyright-notice}{©~2020~The
  New York Times Company}
\end{itemize}

\begin{itemize}
\tightlist
\item
  \href{https://www.nytco.com/}{NYTCo}
\item
  \href{https://help.nytimes3xbfgragh.onion/hc/en-us/articles/115015385887-Contact-Us}{Contact
  Us}
\item
  \href{https://www.nytco.com/careers/}{Work with us}
\item
  \href{https://nytmediakit.com/}{Advertise}
\item
  \href{http://www.tbrandstudio.com/}{T Brand Studio}
\item
  \href{https://www.nytimes3xbfgragh.onion/privacy/cookie-policy\#how-do-i-manage-trackers}{Your
  Ad Choices}
\item
  \href{https://www.nytimes3xbfgragh.onion/privacy}{Privacy}
\item
  \href{https://help.nytimes3xbfgragh.onion/hc/en-us/articles/115014893428-Terms-of-service}{Terms
  of Service}
\item
  \href{https://help.nytimes3xbfgragh.onion/hc/en-us/articles/115014893968-Terms-of-sale}{Terms
  of Sale}
\item
  \href{https://spiderbites.nytimes3xbfgragh.onion}{Site Map}
\item
  \href{https://help.nytimes3xbfgragh.onion/hc/en-us}{Help}
\item
  \href{https://www.nytimes3xbfgragh.onion/subscription?campaignId=37WXW}{Subscriptions}
\end{itemize}
