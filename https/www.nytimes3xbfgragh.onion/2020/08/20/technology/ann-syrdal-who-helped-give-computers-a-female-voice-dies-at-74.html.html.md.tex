Sections

SEARCH

\protect\hyperlink{site-content}{Skip to
content}\protect\hyperlink{site-index}{Skip to site index}

\href{https://www.nytimes3xbfgragh.onion/section/technology}{Technology}

\href{https://myaccount.nytimes3xbfgragh.onion/auth/login?response_type=cookie\&client_id=vi}{}

\href{https://www.nytimes3xbfgragh.onion/section/todayspaper}{Today's
Paper}

\href{/section/technology}{Technology}\textbar{}Ann Syrdal, Who Helped
Give Computers a Female Voice, Dies at 74

\url{https://nyti.ms/2YjQnkK}

\begin{itemize}
\item
\item
\item
\item
\item
\end{itemize}

Advertisement

\protect\hyperlink{after-top}{Continue reading the main story}

Supported by

\protect\hyperlink{after-sponsor}{Continue reading the main story}

\hypertarget{ann-syrdal-who-helped-give-computers-a-female-voice-dies-at-74}{%
\section{Ann Syrdal, Who Helped Give Computers a Female Voice, Dies at
74}\label{ann-syrdal-who-helped-give-computers-a-female-voice-dies-at-74}}

As a researcher at AT\&T, she helped lay the groundwork for modern
digital assistants like Siri and Alexa.

\includegraphics{https://static01.graylady3jvrrxbe.onion/images/2020/08/21/obituaries/21Syrdal1/merlin_175874970_46d7a90b-dc98-4d11-8627-2a780167c62c-articleLarge.jpg?quality=75\&auto=webp\&disable=upscale}

\href{https://www.nytimes3xbfgragh.onion/by/cade-metz}{\includegraphics{https://static01.graylady3jvrrxbe.onion/images/2018/11/26/multimedia/author-cade-metz/author-cade-metz-thumbLarge.png}}

By \href{https://www.nytimes3xbfgragh.onion/by/cade-metz}{Cade Metz}

\begin{itemize}
\item
  Aug. 20, 2020
\item
  \begin{itemize}
  \item
  \item
  \item
  \item
  \item
  \end{itemize}
\end{itemize}

Ann Syrdal, a psychologist and computer science researcher who helped
develop synthetic voices that sounded like women, laying the groundwork
for such modern digital assistants as Apple's Siri and Amazon's Alexa,
died on July 24 at her home in San Jose, Calif. She was 74.

Her daughter Kristen Lasky said the cause was cancer.

As a researcher at AT\&T, Dr. Syrdal was part of a small community of
scientists who began developing synthetic speech systems in the
mid-1980s.

It was not an entirely new phenomenon; AT\&T had unveiled one of the
first synthetic voices, developed at its
\href{https://www.bell-labs.com/about/history-bell-labs/}{Bell Labs}, at
the 1939 World's Fair in New York City. But more than 40 years later,
despite increasingly powerful computers, speech synthesis was still
relatively primitive.

``It just sounded robotic,'' said Tom Gruber, who worked on synthetic
speech systems in the early '80s and went on to create the digital
assistant that became Siri when Apple acquired it in 2010.

By 1990, companies like AT\&T had started to deploy these new systems,
allowing the hearing-impaired, for example, to generate synthetic speech
for phone calls. The voices, though, typically sounded male.

That year, at the Bell Labs research center in Naperville, Ill., Dr.
Syrdal developed a voice that
\href{https://webcache.googleusercontent.com/search?q=cache:oduhdQYYUVsJ:https://www.chicagotribune.com/news/ct-xpm-1990-12-09-9004120037-story.html+\&cd=1\&hl=en\&ct=clnk\&gl=us}{sounded
female} --- a much harder result to achieve, in part because so much of
the previous engineering work had been done for male voices.

\hypertarget{att-natural-voices}{%
\subsubsection{AT\&T Natural Voices}\label{att-natural-voices}}

A sample of a voice named Julia that was part of Ann Syrdal's pioneering
work in the 1990s to develop female synthetic voices.

A decade later, she was part of a team at another AT\&T lab, in Florham
Park, N.J., that developed
\href{https://www.nytimes3xbfgragh.onion/2001/08/09/technology/the-desktop-that-does-elvis.html}{a
system called Natural Voices}. It became a standard-bearer for speech
synthesis, featuring what Dr. Syrdal and others
\href{https://www.engineergirl.org/2939/Ann-Syrdal}{called} ``the first
truly high quality female synthetic voice.''

In 2008, she was named a fellow of the Acoustical Society of America in
recognition of her contributions to the rise of female speech synthesis,
which is now a part of everyday life, thanks to Siri and Alexa.

``She was driven --- and I mean driven --- to optimize the quality of
female voices,'' said Juergen Schroeter, who ran the Natural Voices
project.

Ann Kristen Syrdal was born on Dec. 13, 1945, in Minneapolis. Her
parents, Richard and Marjorie (Paulson) Syrdal, had met while working at
Minneapolis-Honeywell (now Honeywell), a heating company that grew into
a technology giant in the years before World War II.

Her father, a physicist and engineer who developed vacuum tubes and
other electrical technologies, died when Ann was 2. She was raised by
her mother, a sales clerk at a Minneapolis department store.

When she enrolled at the University of Minnesota, Dr. Syrdal had not
considered a science career. But when a psychology professor asked for
her help with a lab experiment involving rats, she fell in love with lab
work --- even after realizing that she was severely allergic to rats.

She went on to earn both bachelor's and Ph.D. degrees in psychology
before being hired as a researcher by the Callier Center for
Communication Disorders at the University of Texas at Dallas. In the
early 1980s, after receiving a five-year grant from the National
Institutes of Health, she began exploring the mechanics of human speech
at the \href{https://www.kth.se/en}{KTH Royal Institute of Technology}
in Stockholm and at the Massachusetts Institute of Technology.

\includegraphics{https://static01.graylady3jvrrxbe.onion/images/2020/08/20/obituaries/Syrdal2/merlin_175874967_db43109c-a6cd-4e39-a86c-ab3699984a00-articleLarge.jpg?quality=75\&auto=webp\&disable=upscale}

When she moved to Bell Labs, female voice synthesis was not a major area
of research anywhere.

``They all thought a female voice was just a higher frequency version of
the male voice, but that never works,'' said H.S. Gopal, a speech
researcher who worked alongside Dr. Syrdal during those years. ``The
male engineers just didn't take female speech as seriously.''

At first, she improved on earlier efforts to build female voices, but in
the late 1990s she joined a project that would help change the nature of
speech synthesis. Rather than generating sounds from scratch, she and
her colleagues developed ways of piecing together snippets of recorded
human speech to form new words and new sentences on the fly. Dr. Syrdal
oversaw the recordings.

The first recordings were made with six women, and when AT\&T's Natural
Voices system topped an international competition for speech
synthesizers in 1998 --- an inflection point for this technology --- it
used a female voice.

Dr. Syrdal's marriages to Scot O'Malley, Robert Lasky and Stephen Marcus
ended in divorce. In addition to her daughter Kristen Lasky, she is
survived by her partner of 23 years, Alistair Conkie, who worked
alongside her at AT\&T; a son, Sean O'Malley; another daughter, Barbara
Evelyn Lasky; and eight grandchildren.

When Siri was integrated into Apple's iPhone in 2011, both female and
male voices were offered. ``We did it because we wanted gender equality
--- and because it was possible,'' Mr. Gruber said ``People respond
differently to different voices.''

In many countries, including the United States and Japan, female voices
became the standard.

Advertisement

\protect\hyperlink{after-bottom}{Continue reading the main story}

\hypertarget{site-index}{%
\subsection{Site Index}\label{site-index}}

\hypertarget{site-information-navigation}{%
\subsection{Site Information
Navigation}\label{site-information-navigation}}

\begin{itemize}
\tightlist
\item
  \href{https://help.nytimes3xbfgragh.onion/hc/en-us/articles/115014792127-Copyright-notice}{©~2020~The
  New York Times Company}
\end{itemize}

\begin{itemize}
\tightlist
\item
  \href{https://www.nytco.com/}{NYTCo}
\item
  \href{https://help.nytimes3xbfgragh.onion/hc/en-us/articles/115015385887-Contact-Us}{Contact
  Us}
\item
  \href{https://www.nytco.com/careers/}{Work with us}
\item
  \href{https://nytmediakit.com/}{Advertise}
\item
  \href{http://www.tbrandstudio.com/}{T Brand Studio}
\item
  \href{https://www.nytimes3xbfgragh.onion/privacy/cookie-policy\#how-do-i-manage-trackers}{Your
  Ad Choices}
\item
  \href{https://www.nytimes3xbfgragh.onion/privacy}{Privacy}
\item
  \href{https://help.nytimes3xbfgragh.onion/hc/en-us/articles/115014893428-Terms-of-service}{Terms
  of Service}
\item
  \href{https://help.nytimes3xbfgragh.onion/hc/en-us/articles/115014893968-Terms-of-sale}{Terms
  of Sale}
\item
  \href{https://spiderbites.nytimes3xbfgragh.onion}{Site Map}
\item
  \href{https://help.nytimes3xbfgragh.onion/hc/en-us}{Help}
\item
  \href{https://www.nytimes3xbfgragh.onion/subscription?campaignId=37WXW}{Subscriptions}
\end{itemize}
