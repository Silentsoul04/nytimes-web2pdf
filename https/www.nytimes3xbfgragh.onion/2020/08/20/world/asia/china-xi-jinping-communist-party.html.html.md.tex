Sections

SEARCH

\protect\hyperlink{site-content}{Skip to
content}\protect\hyperlink{site-index}{Skip to site index}

\href{https://www.nytimes3xbfgragh.onion/section/world/asia}{Asia
Pacific}

\href{https://myaccount.nytimes3xbfgragh.onion/auth/login?response_type=cookie\&client_id=vi}{}

\href{https://www.nytimes3xbfgragh.onion/section/todayspaper}{Today's
Paper}

\href{/section/world/asia}{Asia Pacific}\textbar{}`Drive the Blade In':
Xi Shakes Up China's Law-and-Order Forces

\url{https://nyti.ms/3iU4AwH}

\begin{itemize}
\item
\item
\item
\item
\item
\end{itemize}

Advertisement

\protect\hyperlink{after-top}{Continue reading the main story}

Supported by

\protect\hyperlink{after-sponsor}{Continue reading the main story}

\hypertarget{drive-the-blade-in-xi-shakes-up-chinas-law-and-order-forces}{%
\section{`Drive the Blade In': Xi Shakes Up China's Law-and-Order
Forces}\label{drive-the-blade-in-xi-shakes-up-chinas-law-and-order-forces}}

Citing Mao, the Communist Party has started a campaign to ensure that
police officers, judges and state security agents remain loyal and
disciplined in challenging times.

\includegraphics{https://static01.graylady3jvrrxbe.onion/images/2020/08/20/world/20china-purge-1/merlin_172913739_b6caaf9d-7cc0-49ab-be8a-f2941d6442b0-articleLarge.jpg?quality=75\&auto=webp\&disable=upscale}

\href{https://www.nytimes3xbfgragh.onion/by/chris-buckley}{\includegraphics{https://static01.graylady3jvrrxbe.onion/images/2018/10/08/multimedia/author-chris-buckley/author-chris-buckley-thumbLarge.png}}

By \href{https://www.nytimes3xbfgragh.onion/by/chris-buckley}{Chris
Buckley}

\begin{itemize}
\item
  Aug. 20, 2020
\item
  \begin{itemize}
  \item
  \item
  \item
  \item
  \item
  \end{itemize}
\end{itemize}

\href{https://cn.nytimes3xbfgragh.onion/china/20200821/china-xi-jinping-communist-party/}{阅读简体中文版}\href{https://cn.nytimes3xbfgragh.onion/china/20200821/china-xi-jinping-communist-party/zh-hant/}{閱讀繁體中文版}

Across China, police officers, judges, prosecutors and feared state
security agents have been studying Mao's methods for political purges,
absorbing them as guidance for a new Communist Party drive against
graft, abuses and disloyalty in their ranks.

The campaign is shaping up as a sharp tool for the Communist Party
leader, Xi Jinping, to bolster domestic discipline as he prepares for a
leadership shake-up in two years, as well as
\href{https://www.nytimes3xbfgragh.onion/2020/07/14/world/asia/cold-war-china-us.html}{continuing
strife} with the United States and other countries.

Officials in China's law-and-order apparatus have been ordered to
``\href{http://www.chinalaw.gov.cn/news/content/2020-07/30/zfyw_3253482.html}{drive
the blade in}'' and ``scrape poison off the bone,'' setting aside
personal loyalties to expose wayward colleagues. The model for this
``education and rectification'' program, leaders have told them, should
be Mao Zedong's
\href{https://www.ide.go.jp/library/English/Publish/Periodicals/De/pdf/71_01_06.pdf}{drive
of the 1940s}, which cemented his dominance over the party from a base
in the city of Yan'an.

``Root out the harmful members of the herd,'' Chen Yixin, a chief
enforcer of the campaign,
\href{http://news.ifeng.com/c/7xwhnCdD5zm}{said at a kickoff meeting}
last month. ``Root out `two-faced people' who are disloyal and dishonest
to the party.''

Such mobilization sessions have proliferated across China --- in courts,
police headquarters, prison administrations and the secretive
\href{http://www.legaldaily.com.cn/index_article/content/2020-07/19/content_8251365.htm}{Ministry
of State Security,} which controls the country's main civilian
surveillance and spy forces.

They and other law-and-order agencies come under the Central Political
and Legal Affairs Commission, a bastion of party power, along with the
military. Mr. Xi
\href{https://chinadigitaltimes.net/2015/02/xis-sharp-words-ominous-legal-reform/}{calls
command} of the security system the party's ``knife handle,'' a menacing
term taken from Mao.

The genuflections to Mao, who remains revered by the Communist Party,
reflect Mr. Xi's desire to use the campaign to help fireproof his and
the party's power against possible turbulence.

``The Yan'an rectification was about obeying Mao in everything, and
that's the biggest signal from learning from Yan'an this time,'' Deng
Yuwen, a former Chinese editor for a Communist Party newspaper, said in
an interview from the United States, where he now lives. ``The core goal
of cleaning up the political and legal system is also to obey Xi in
everything.''

Soon after Mr. Xi came to office eight years ago, he unleashed a wave of
anticorruption cases that have felled hundreds of senior officials. The
former chief of the domestic security apparatus, Zhou Yongkang, was
\href{https://www.nytimes3xbfgragh.onion/2015/06/12/world/asia/zhou-yongkang-former-security-chief-in-china-gets-life-sentence-for-corruption.html}{sentenced
to life in prison} on corruption charges in 2015.

\includegraphics{https://static01.graylady3jvrrxbe.onion/images/2020/08/20/world/20china-purge-2/merlin_128043563_e943e89a-eb4f-4688-9fbd-16d97c9ecfd2-articleLarge.jpg?quality=75\&auto=webp\&disable=upscale}

Despite those efforts, experts and recent Chinese studies said the party
leadership has still struggled to manage its hydra-headed bureaucracy of
police forces, security agencies, courts, prosecutors and prisons. Early
last year, the party
\href{http://www.gov.cn/zhengce/2019-01/18/content_5359135.htm}{issued
new rules} to tighten top-down control of the system. Studies by Chinese
researchers
\href{http://www.cssn.cn/fx/202007/t20200721_5158007.shtml}{have said}
that fragmentation and rivalry between agencies remain problems.

Months of protests in Hong Kong last year, and the pandemic crisis this
year, seem to have reinforced Mr. Xi's push for iron authority right
down to local police stations.

``Resolutely put absolute loyalty, absolute purity and absolute
dependability into action,'' the minister of public security, Zhao
Kezhi,
\href{http://www.gov.cn/xinwen/2020-08/09/content_5533406.htm}{said this
month} while inspecting enforcement of the campaign in northeast China.

China's leaders appear most worried about lower- and midlevel police
officers and legal officials, said Qin Qianhong, a professor of law at
Wuhan University in central China. A separate campaign since 2018 to
break alliances between crime gangs and officials reinforced senior
officials' worries that their local forces remained compromised by
corruption, he said.

``Although China's investigations of official criminality and corruption
have taken down a bunch of people, the main political-legal structure
has not been replaced for the most part,'' Professor Qin said. Invoking
Mao's Yan'an purge did not mean that officials were applying its harsh
methods, he said.

``It's to show that this rectification must be taken seriously,'' he
said. ``But Yan'an was about establishing a core leader and nurturing
loyalty, and that must be followed.''

Image

``Resolutely put absolute loyalty, absolute purity and absolute
dependability into action,'' the minister of public security, Zhao
Kezhi, above in 2018, said this month while inspecting enforcement of
the campaign in northeast China.Credit...Pool photo by Jason Lee

The campaign is scheduled to last until early 2022, the cusp of a
Communist Party congress that will install a new cohort of central
officials and, most likely, extend Mr. Xi's time in power. Publicity
about the campaign has described local officials studying Mr. Xi's
writings and speeches
\href{https://www.thepaper.cn/newsDetail_forward_8685469}{in
indoctrination classes} deep into the night.

Teams of investigators have already plucked out cadres accused of
corruption and other abuses. **** In the first week of the campaign, 21
officials from the public security or legal systems came under
investigation, officials
\href{https://mp.weixin.qq.com/s/0vI6lilmapIoCGDbzGANDw}{announced}.

Investigators disclosed this week that the chief of the Shanghai Public
Security Bureau, Gong Dao'an, had been placed
\href{http://www.ccdi.gov.cn/toutiao/202008/t20200818_223984.html}{under
investigation} on allegations that were not publicly specified, making
him the most prominent police official toppled since the campaign began.

Other officials who have recently fallen include a
\href{http://www.ccdi.gov.cn/yaowen/202006/t20200622_220588.html}{former
head of prisons} in Inner Mongolia, a region of northern China; the
chief of
\href{http://fanfu.people.com.cn/n1/2020/0712/c64371-31779959.html}{public
security of Jiangmen}, a city in southern China; and a former longtime
\href{https://baijiahao.baidu.com/s?id=1672115211523911540\&wfr=spider\&for=pc}{state
security official} in the eastern province of Jiangsu. The specific
allegations against them were not made public.

Earlier this year, Sun Lijun, a vice minister of public security, was
\href{http://www.xinhuanet.com/english/2020-04/19/c_138990402.htm}{put
under investigation}. Unconfirmed rumors that retired central security
leaders may be investigated have spread among political insiders in
Beijing and spilled onto the internet.

``It suggests a continued push on Xi Jinping's part to remake China's
coercive apparatus into a force that is entirely politically responsive
to his direction,'' said Sheena Chestnut Greitens, an
\href{https://lbj.utexas.edu/greitens-sheena-chestnut}{associate
professor} at the University of Texas at Austin who studies Chinese
policing and has written a forthcoming paper about the drive to clean up
China's law-and-order bureaucracy.

Mr. Xi wants ``to push his authority downward throughout the lower
levels of the political-legal system'' before the party congress in
2022, she said.

The campaign has also confirmed the rise of Mr. Chen, a 60-year-old
official who over the past couple of years has handled a succession of
politically tricky tasks. He has also led the drive against local crime
protection rackets, and seized hold of efforts to stifle the coronavirus
epidemic in Wuhan
\href{https://www.scmp.com/news/china/politics/article/3050087/beijing-pins-hopes-guy-emperors-sword-restore-order-coronavirus}{in
February}, when the city where the outbreak began appeared overwhelmed.

``These high-profile roles have certainly given him a lot of name
recognition and the opportunity to build a base of following,''
\href{https://ufind.univie.ac.at/en/person.html?id=60055}{Ling Li}, an
expert on Chinese politics and law at the University of Vienna, said in
emailed answers to questions. ``It looks like he is prepared for bigger
roles.''

Image

Mao memorabilia at a market in Beijing last year. Mr. Xi~calls
command~of the security system the Communist Party's ``knife handle,'' a
menacing term taken from Mao.Credit...Noel Celis/Agence France-Presse
--- Getty Images

Some analysts have seen this campaign as an effort by Mr. Xi to drive
out factional opponents. But the breadth of the actions indicate that
Mr. Xi wants to shake up the entire hierarchy, said
\href{https://www.christopherjcarothers.com/}{Christopher J. Carothers},
a postdoctoral fellow at the University of Pennsylvania who studies
anticorruption policies in China.

``Xi's vision of a highly controlled society demands a strong apparatus
to enforce that control; corruption is a threat to that,'' he said by
email. ``Even if there hasn't been any new spike in disloyalty or abuses
in these institutions, the Communist Party leadership may still not be
satisfied that they are effectively handling a growing and quickly
changing mission.''

Mr. Xi has also telegraphed lately that he is bracing his government for
a difficult few years.

China has emerged from the coronavirus crisis, and its economy is
recovering. But Mr. Xi and other senior officials meeting in Beijing
late last month
\href{http://cpc.people.com.cn/n1/2020/0731/c64094-31804564.html}{warned}
that China's ``international environment grows ever more complex, and
instability and uncertainty have clearly increased.'' They cited Mao's
notion of waging ``protracted war'' to drive home that warning.

Earlier this year, the party
\href{http://www.xinhuanet.com/politics/leaders/2020-04/21/c_1125887251.htm}{established
another} policing policy committee --- called the Secure China
Development group --- to step up efforts against unrest and crime.

``The next five years are a crucial window of time for China,'' said Mr.
Deng, the former editor, citing rising rivalry with the United States
and efforts to push the Chinese economy into a new phase of growth. ``In
Xi Jinping's view, the rectification campaign in the political-legal
system is to ensure that no problems can turn into a severe domestic
crisis.''

Amber Wang contributed research.

Advertisement

\protect\hyperlink{after-bottom}{Continue reading the main story}

\hypertarget{site-index}{%
\subsection{Site Index}\label{site-index}}

\hypertarget{site-information-navigation}{%
\subsection{Site Information
Navigation}\label{site-information-navigation}}

\begin{itemize}
\tightlist
\item
  \href{https://help.nytimes3xbfgragh.onion/hc/en-us/articles/115014792127-Copyright-notice}{©~2020~The
  New York Times Company}
\end{itemize}

\begin{itemize}
\tightlist
\item
  \href{https://www.nytco.com/}{NYTCo}
\item
  \href{https://help.nytimes3xbfgragh.onion/hc/en-us/articles/115015385887-Contact-Us}{Contact
  Us}
\item
  \href{https://www.nytco.com/careers/}{Work with us}
\item
  \href{https://nytmediakit.com/}{Advertise}
\item
  \href{http://www.tbrandstudio.com/}{T Brand Studio}
\item
  \href{https://www.nytimes3xbfgragh.onion/privacy/cookie-policy\#how-do-i-manage-trackers}{Your
  Ad Choices}
\item
  \href{https://www.nytimes3xbfgragh.onion/privacy}{Privacy}
\item
  \href{https://help.nytimes3xbfgragh.onion/hc/en-us/articles/115014893428-Terms-of-service}{Terms
  of Service}
\item
  \href{https://help.nytimes3xbfgragh.onion/hc/en-us/articles/115014893968-Terms-of-sale}{Terms
  of Sale}
\item
  \href{https://spiderbites.nytimes3xbfgragh.onion}{Site Map}
\item
  \href{https://help.nytimes3xbfgragh.onion/hc/en-us}{Help}
\item
  \href{https://www.nytimes3xbfgragh.onion/subscription?campaignId=37WXW}{Subscriptions}
\end{itemize}
