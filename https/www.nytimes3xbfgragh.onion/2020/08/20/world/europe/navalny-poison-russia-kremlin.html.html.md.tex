Sections

SEARCH

\protect\hyperlink{site-content}{Skip to
content}\protect\hyperlink{site-index}{Skip to site index}

\href{https://www.nytimes3xbfgragh.onion/section/world/europe}{Europe}

\href{https://myaccount.nytimes3xbfgragh.onion/auth/login?response_type=cookie\&client_id=vi}{}

\href{https://www.nytimes3xbfgragh.onion/section/todayspaper}{Today's
Paper}

\href{/section/world/europe}{Europe}\textbar{}Don't Drink the Tea:
Poison Is a Favored Weapon in Russia

\url{https://nyti.ms/3gcL6lf}

\begin{itemize}
\item
\item
\item
\item
\item
\end{itemize}

Advertisement

\protect\hyperlink{after-top}{Continue reading the main story}

Supported by

\protect\hyperlink{after-sponsor}{Continue reading the main story}

\hypertarget{dont-drink-the-tea-poison-is-a-favored-weapon-in-russia}{%
\section{Don't Drink the Tea: Poison Is a Favored Weapon in
Russia}\label{dont-drink-the-tea-poison-is-a-favored-weapon-in-russia}}

Poison has been a preferred tool of the Russian security service for
more than a century, and critics of the Kremlin say it remains in the
arsenal today.

\includegraphics{https://static01.graylady3jvrrxbe.onion/images/2020/08/20/world/20navalny-poison/20navalny-poison-articleLarge.jpg?quality=75\&auto=webp\&disable=upscale}

\href{https://www.nytimes3xbfgragh.onion/by/andrew-e-kramer}{\includegraphics{https://static01.graylady3jvrrxbe.onion/images/2018/10/15/multimedia/author-andrew-e-kramer/author-andrew-e-kramer-thumbLarge.png}}

By \href{https://www.nytimes3xbfgragh.onion/by/andrew-e-kramer}{Andrew
E. Kramer}

\begin{itemize}
\item
  Published Aug. 20, 2020Updated Aug. 21, 2020, 12:23 a.m. ET
\item
  \begin{itemize}
  \item
  \item
  \item
  \item
  \item
  \end{itemize}
\end{itemize}

\href{https://cn.nytimes3xbfgragh.onion/world/20200821/navalny-poison-russia-kremlin/}{阅读简体中文版}\href{https://cn.nytimes3xbfgragh.onion/world/20200821/navalny-poison-russia-kremlin/zh-hant/}{閱讀繁體中文版}

MOSCOW --- Without the slightest premonition of any issues with his
health, Pyotr Verzilov, a Russian opposition activist, suddenly fell
violently ill two years ago and slipped into a coma, a common problem
for opponents of the Kremlin.

Mr. Verzilov, who is known for staging antigovernment performance art,
displayed the same, mysterious symptoms
\href{https://www.nytimes3xbfgragh.onion/2020/08/20/world/europe/navalny-poison-russia.html}{that
struck Aleksei Navalny}, Russia's most prominent opposition politician,
on Thursday as he was on a flight to Moscow.

``I was in exactly the same condition,'' Mr. Verzilov said in an
interview Thursday with Rain TV, an independent Russian television
station, of his monthlong illness in 2018.

Mr. Verzilov was kept alive on a ventilator and later flown to Germany
for treatment. Though doctors found no trace of poison, he said that he
is convinced it was the cause, and that the Kremlin was responsible.

Image

Pyotr Verzilov being transported to a hospital in Berlin in 2018. He
claims he was poisoned by the Russian government.~Credit...Reuters

Poison, though redolent of medieval intrigue, has been a favored tool of
Russian intelligence agencies for more than a century. And critics of
the Kremlin and independent analysts say the weapon remains in use
today. While other countries, including the United States and Israel,
have targeted killing programs, they are strictly limited to
counterterrorism efforts. Russia, by contrast, has been accused of
targeting a wide variety of opponents both at home and abroad.

The Soviet Union operated a secret laboratory to research tasteless and
untraceable poisons that were tested on condemned gulag prisoners,
security service defectors have said.

After a series of assassinations and attempted assassinations of
dissidents, journalists, defectors and opposition leaders in Russia and
abroad over the past two decades, researchers have concluded the
post-Soviet government has
\href{https://www.nytimes3xbfgragh.onion/2016/08/21/world/europe/moscow-kremlin-silence-critics-poison.html}{turned
to its poison arsenal} as a preferred weapon.

Substances that have been identified or suspected in poisonings blamed
on the Russian government include radioactive polonium-210; heavy
metals; gelsemium, a rare Himalayan plant toxin; and Novichuk, a
military nerve agent lethal to the touch.

Lacing a meal or a cup of tea --- the last substance Mr. Navalny is said
to have consumed at an airport cafe before falling ill --- with poison
is simple and requires no special training, Gennadi V. Gudkov, a former
opposition member of Parliament and onetime colonel in the K.G.B., said
in a telephone interview on Thursday.

``It is easy, and easy to cover your tracks,'' he said. ``Any person can
use poison.'' Poisons can be intended either to kill or to incapacitate
a person with a long and unpleasant illness, he said.

Ukraine's former pro-Western president, Viktor A. Yushchenko, for
example, was left with his face disfigured after a poisoning with the
industrial pollutant dioxin --- most likely concealed in a meal of
boiled crayfish. Mr. Yushchenko attributed the poisoning to Russian
agents.

\includegraphics{https://static01.graylady3jvrrxbe.onion/images/2020/08/20/world/20navalny-poison2/20navalny-poison2-articleLarge.jpg?quality=75\&auto=webp\&disable=upscale}

The Kremlin has for years regarded Mr. Navalny as an enemy because of
his investigations into graft by officials. He has been harassed and
jailed numerous times, but only for short periods.

Mr. Navalny was rushed to a Siberian hospital after his flight made an
emergency landing on Thursday because of his sudden illness. Doctors
have not announced a cause.

The state-owned news agency Tass quoted an unidentified law enforcement
source as saying that the authorities were not yet considering the
possibility of a deliberate poisoning. Mr. Navalny's personal doctor,
Yaroslav Ashikhmin, said he has not seen Mr. Navalny since the illness
began, so he could not say whether poison was the cause. But, he added,
``it looks like it.''

If Mr. Navalny were poisoned before or during his flight, it would not
be the first time an opposition figure was targeted while sitting in the
controlled environment of a commercial airplane in Russia.

In 2015, the opposition activist Vladimir Kara-Murza fell into a
weeklong coma in Moscow. He later said he believed he had ingested a
poison during the in-flight service on an Aeroflot plane.

His symptoms included swelling in his brain and kidney failure. His
wife, Yevgenia, recalled that his arms and legs took on a blue hue, an
alarming, almost cartoonish reaction to poison.

Mr. Kara-Murza said he was poisoned and survived a second time in 2017
while traveling in Russia to show a documentary about another Russian
politician, Boris Y. Nemtsov, who in 2015 was shot and killed on a
bridge in Moscow.

Image

 Vladimir Kara-Murza said he has survived two poisonings in 2015 and
2017.Credit...Al Drago/The New York Times

In 2004, the opposition journalist Anna Politkovskaya was also poisoned
on a domestic flight operated by another airline, Karat. She drank a
poisoned cup of tea, she said. She survived but two years later was shot
and killed in her apartment elevator.

Toxins lethal to the touch have also come to light. The
\href{https://www.nytimes3xbfgragh.onion/2009/02/01/world/europe/01torture.html}{Arab-born
terrorist Ibn al-Khattab died in 2002 in his mountain hide-out in
Chechnya} after opening a letter laced with a nerve agent.

Some toxins may have also slipped from government arsenals and into the
organized crimes wars in Russia in the early post-Soviet period. In
1995, for example, a Russian banker, Ivan K. Kivelidi, died after coming
in contact with a poison deadly to the touch. The cause of his death
might have remained a mystery had his secretary not also died of the
same symptoms, apparently because the poison had been spread on an
office telephone handset.

Advertisement

\protect\hyperlink{after-bottom}{Continue reading the main story}

\hypertarget{site-index}{%
\subsection{Site Index}\label{site-index}}

\hypertarget{site-information-navigation}{%
\subsection{Site Information
Navigation}\label{site-information-navigation}}

\begin{itemize}
\tightlist
\item
  \href{https://help.nytimes3xbfgragh.onion/hc/en-us/articles/115014792127-Copyright-notice}{©~2020~The
  New York Times Company}
\end{itemize}

\begin{itemize}
\tightlist
\item
  \href{https://www.nytco.com/}{NYTCo}
\item
  \href{https://help.nytimes3xbfgragh.onion/hc/en-us/articles/115015385887-Contact-Us}{Contact
  Us}
\item
  \href{https://www.nytco.com/careers/}{Work with us}
\item
  \href{https://nytmediakit.com/}{Advertise}
\item
  \href{http://www.tbrandstudio.com/}{T Brand Studio}
\item
  \href{https://www.nytimes3xbfgragh.onion/privacy/cookie-policy\#how-do-i-manage-trackers}{Your
  Ad Choices}
\item
  \href{https://www.nytimes3xbfgragh.onion/privacy}{Privacy}
\item
  \href{https://help.nytimes3xbfgragh.onion/hc/en-us/articles/115014893428-Terms-of-service}{Terms
  of Service}
\item
  \href{https://help.nytimes3xbfgragh.onion/hc/en-us/articles/115014893968-Terms-of-sale}{Terms
  of Sale}
\item
  \href{https://spiderbites.nytimes3xbfgragh.onion}{Site Map}
\item
  \href{https://help.nytimes3xbfgragh.onion/hc/en-us}{Help}
\item
  \href{https://www.nytimes3xbfgragh.onion/subscription?campaignId=37WXW}{Subscriptions}
\end{itemize}
