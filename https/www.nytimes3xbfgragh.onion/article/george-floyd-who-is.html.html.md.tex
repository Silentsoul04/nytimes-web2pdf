Sections

SEARCH

\protect\hyperlink{site-content}{Skip to
content}\protect\hyperlink{site-index}{Skip to site index}

\href{https://www.nytimes3xbfgragh.onion/section/us}{U.S.}

\href{https://myaccount.nytimes3xbfgragh.onion/auth/login?response_type=cookie\&client_id=vi}{}

\href{https://www.nytimes3xbfgragh.onion/section/todayspaper}{Today's
Paper}

\href{/section/us}{U.S.}\textbar{}George Floyd, From `I Want to Touch
the World' to `I Can't Breathe'

\url{https://nyti.ms/2AWztzD}

\begin{itemize}
\item
\item
\item
\item
\item
\item
\end{itemize}

\href{https://www.nytimes3xbfgragh.onion/news-event/george-floyd-protests-minneapolis-new-york-los-angeles?action=click\&pgtype=Article\&state=default\&region=TOP_BANNER\&context=storylines_menu}{Race
and America}

\begin{itemize}
\tightlist
\item
  \href{https://www.nytimes3xbfgragh.onion/2020/07/26/us/protests-portland-seattle-trump.html?action=click\&pgtype=Article\&state=default\&region=TOP_BANNER\&context=storylines_menu}{Protesters
  Return to Other Cities}
\item
  \href{https://www.nytimes3xbfgragh.onion/2020/07/24/us/portland-oregon-protests-white-race.html?action=click\&pgtype=Article\&state=default\&region=TOP_BANNER\&context=storylines_menu}{Portland
  at the Center}
\item
  \href{https://www.nytimes3xbfgragh.onion/2020/07/23/podcasts/the-daily/portland-protests.html?action=click\&pgtype=Article\&state=default\&region=TOP_BANNER\&context=storylines_menu}{Podcast:
  Showdown in Portland}
\item
  \href{https://www.nytimes3xbfgragh.onion/interactive/2020/07/16/us/black-lives-matter-protests-louisville-breonna-taylor.html?action=click\&pgtype=Article\&state=default\&region=TOP_BANNER\&context=storylines_menu}{45
  Days in Louisville}
\end{itemize}

Advertisement

\protect\hyperlink{after-top}{Continue reading the main story}

Supported by

\protect\hyperlink{after-sponsor}{Continue reading the main story}

\hypertarget{george-floyd-from-i-want-to-touch-the-world-to-i-cant-breathe}{%
\section{George Floyd, From `I Want to Touch the World' to `I Can't
Breathe'}\label{george-floyd-from-i-want-to-touch-the-world-to-i-cant-breathe}}

Mr. Floyd had big plans for life nearly 30 years ago. His death in
police custody is powering a movement against police brutality and
racial injustice.

\includegraphics{https://static01.graylady3jvrrxbe.onion/images/2020/06/09/us/00unrest-floydprofile01alt/merlin_173327667_67685113-1ffa-472b-8493-57383d2b8ffd-articleLarge.jpg?quality=75\&auto=webp\&disable=upscale}

By \href{https://www.nytimes3xbfgragh.onion/by/manny-fernandez}{Manny
Fernandez} and
\href{https://www.nytimes3xbfgragh.onion/by/audra-d-s-burch}{Audra D. S.
Burch}

\begin{itemize}
\item
  July 29, 2020
\item
  \begin{itemize}
  \item
  \item
  \item
  \item
  \item
  \item
  \end{itemize}
\end{itemize}

\href{https://www.nytimes3xbfgragh.onion/es/2020/06/09/espanol/mundo/George-Floyd-quien-es.html}{Leer
en español}

\emph{{[}Follow the live updates on}
\href{https://www.nytimes3xbfgragh.onion/2020/06/22/us/seattle-shooting-roosevelt-statue-nascar-noose.html}{\emph{Seattle,
Bubba Wallace, statues and the confederate flag}}\emph{.{]}}

HOUSTON --- It was the last day of 11th grade at Jack Yates High School
in Houston, nearly three decades ago. A group of close friends, on their
way home, were contemplating what senior year and beyond would bring.
They were black teenagers on the precipice of manhood. What, they asked
one another, did they want to do with their lives?

``George turned to me and said, `I want to touch the world,''' said
Jonathan Veal, 45, recalling the aspiration of one of the young men ---
a tall, gregarious star athlete named
\href{https://www.nytimes3xbfgragh.onion/2020/06/10/podcasts/the-daily/george-floyd-protests-funeral.html}{George
Floyd} whom he had met in the school cafeteria on the first day of sixth
grade. To their 17-year-old minds, touching the world maybe meant the
N.B.A. or the N.F.L.

``It was one of the first moments I remembered after learning what
happened to him,'' Mr. Veal said. ``He could not have imagined that this
is the tragic way people would know his name.''

The world now knows George Perry Floyd Jr. through his
\href{https://www.nytimes3xbfgragh.onion/2020/05/29/us/derek-chauvin-george-floyd-worked-together.html}{final
harrowing moments}, as he begged for air, his face wedged for
\href{https://www.nytimes3xbfgragh.onion/2020/05/31/us/george-floyd-investigation.html}{nearly
nine minutes} between a city street and a police officer's knee.

Mr. Floyd's gasping death, immortalized on a bystander's cellphone video
during the twilight hours of Memorial Day, has powered two weeks of
\href{https://www.nytimes3xbfgragh.onion/news-event/george-floyd-protests-minneapolis-new-york-los-angeles}{sprawling
protests across America} against police brutality. He has been
memorialized in
\href{https://www.nytimes3xbfgragh.onion/2020/07/29/us/george-floyd-memorial.html}{Minneapolis},
where he died; in North Carolina, where he was born; and in Houston,
where thousands stood in the unrelenting heat on Monday afternoon to
file past his gold coffin and bid him farewell
\href{https://www.nytimes3xbfgragh.onion/2020/06/08/us/george-floyd-viewing-funeral-houston-unrest.html}{in
the city where he spent most of his life}.

Many of those who attended the public viewing said they saw Mr. Floyd as
one of them --- a fellow Houstonian who could have been their father,
their brother or their son.

``This is something that touched really close,'' said Kina Ardoin, 43, a
nurse who stood in a line that stretched far from the church entrance.
``This could have been anybody in my family.''

\includegraphics{https://static01.graylady3jvrrxbe.onion/images/2017/01/29/podcasts/the-daily-album-art/the-daily-album-art-articleInline-v2.jpg?quality=75\&auto=webp\&disable=upscale}

\hypertarget{listen-to-the-daily-i-want-to-touch-the-world}{%
\subsubsection{Listen to `The Daily': `I Want To Touch the
World'}\label{listen-to-the-daily-i-want-to-touch-the-world}}

Today we remember George Perry Floyd Jr.

transcript

Back to The Daily

bars

0:00/33:27

-33:27

transcript

\hypertarget{listen-to-the-daily-i-want-to-touch-the-world-1}{%
\subsection{Listen to `The Daily': `I Want To Touch the
World'}\label{listen-to-the-daily-i-want-to-touch-the-world-1}}

\hypertarget{hosted-by-michael-barbaro-and-caitlin-dickerson-produced-by-clare-toeniskoetter-michael-simon-johnson-adizah-eghan-daniel-guillemette-asthaa-chaturvedi-bianca-giaever-and-stella-tan-and-edited-by-lisa-tobin-and-liz-o-baylen}{%
\subsubsection{Hosted by Michael Barbaro and Caitlin Dickerson, produced
by Clare Toeniskoetter, Michael Simon Johnson, Adizah Eghan, Daniel
Guillemette, Asthaa Chaturvedi, Bianca Giaever and Stella Tan, and
edited by Lisa Tobin and Liz O.
Baylen}\label{hosted-by-michael-barbaro-and-caitlin-dickerson-produced-by-clare-toeniskoetter-michael-simon-johnson-adizah-eghan-daniel-guillemette-asthaa-chaturvedi-bianca-giaever-and-stella-tan-and-edited-by-lisa-tobin-and-liz-o-baylen}}

\hypertarget{today-we-remember-george-perry-floyd-jr}{%
\paragraph{Today we remember George Perry Floyd
Jr.}\label{today-we-remember-george-perry-floyd-jr}}

\begin{itemize}
\tightlist
\item
  archived recording (jonathan veal)\\
  My name is Jonathan Veal. I have known George Floyd since the sixth
  grade at James D. Ryan Middle School in the community of Third Ward,
  which is located in Houston, Texas. The first day I saw him, I was in
  the cafeteria, and he came in. And I was just blown away by his
  height. He was 6'2``, and I was just in awe, just like wow, that's a
  tall guy. And he was just tall and skinny. This guy is in the sixth
  grade? And that was the beginning of our relationship. I remember it
  was the last day of school in our junior year, and there was this
  place just north of our school, maybe three blocks, that we called The
  Hill. And we would just kind of go there just to hang out. And for
  some reason, the conversation shifted to OK, we're about to graduate.
  It was like we're no longer going to be teenagers anymore. So I know I
  talked about just wanting to get married, and George talked about
  college. And all of a sudden he made this statement. He says, man, I
  want to be big. I want to touch the world.
\end{itemize}

{[}music{]}

Most of us had not seen the world outside of, you know, Third Ward or
the Houston community so it was just like, oh. Wow. OK.

\begin{itemize}
\item
  archived recording 1\\
  He was just a fun person to be around. There was never a dull moment.
  Never a dull moment.
\item
  archived recording 2\\
  Me and Big George used to go to school all the time. And he'd get out
  and listen to music and talk about, you know, about the music world,
  and how he want to do this and do that, and just be successful.
\item
  archived recording 3\\
  We were young, just kids. We trying to figure this thing out, you
  know? It's when you're in your 20s, your early 20s and you're trying
  to figure out --- you're trying to see what direction you're going to
  go in, just waking up and just trying to figure it out.
\item
  archived recording\\
  I met Floyd seven to 10 years ago while I was trying to plant a
  church, Resurrection Houston's Ministry, in the middle of Third Ward
  Houston, Texas, in the Cuney Homes Housing Project. And say I go to a
  neighborhood, I can knock on 50 doors. 50 people may come out. Floyd
  comes out the door, 100 people come out. Everybody knows him. He's
  connected. Man, just to see his impact was amazing, his road to
  redemption. And then how God used him in this season and in this
  moment.
\end{itemize}

{[}music{]}

\begin{itemize}
\item
  archived recording 1\\
  Soon as he come in the door, he asks you, are you good? You all right?
  Always. And he would say --- he always said things twice sometimes. He
  always called me Al-Al, and he called Teresa T-T. He just --- that's
  just him. Every time we cooked him a meal, gave him a plate, he'd come
  down rubbing his tummy and just go, ``thank you, thank you, thank
  you.'' You know what I mean? And he always said this for the whole
  time that Teresa and me and him lived here together. He always would
  tell us, ``I want y'all to know I appreciate you.'' He always would
  tell us that.
\item
  archived recording (jonathan veal)\\
  After I learned that this was my friend, just a flood of emotions came
  about. I didn't sleep the next couple of nights just thinking about
  what happened. And then that's when it became global. And then I was
  like, wow, it's literally happening. He's touching the world. He's
  touching the world. I was just like, wow.
\end{itemize}

caitlin dickerson

From The New York Times, I'm Caitlin Dickerson. This is ``The Daily.''
Today: George Floyd's funeral. My colleague Manny Fernandez was in
Houston. It's Wednesday, June 10.

{[}phone ringing{]}

manny fernandez

Hi, guys.

caitlin dickerson

Hey, Manny. It's Caitlin.

manny fernandez

How you doing?

caitlin dickerson

I'm OK. How are you and where are you?

manny fernandez

I am in the parking lot of the Fountain of Praise Church in Southwest
Houston, where George Floyd's funeral was just held.

caitlin dickerson

And what was today about?

manny fernandez

So today was about two different things --- and you saw this during the
service itself, and then I got this sense from talking to people
outside. On the one hand, there was a lot of people who wanted to talk
about George Floyd as a symbol of a movement, and George Floyd's death
not being in vain. And yet on the other hand, a lot of people were
trying to say, hold on, wait, let's talk about him as a man. And let's
kind of talk about the jokes he used to crack, and the pranks he used to
pull, and what he was like in the projects of Houston where he's from.
And so I think that there was that two-sided story that you kind of
heard today. Let's remember the man who's become this symbol, and let's
also just remember the man himself.

caitlin dickerson

And this is a familiar dynamic for you, right? I mean, you've covered
funerals for other people who've died at the hands of police, and you've
seen this dynamic before.

manny fernandez

Yeah, absolutely. It reminded me of 2014 with Michael Brown's funeral,
when people gather around, and they say, give us a little bit of space
in this social justice movement that's popping up around this person's
death. Give us a few hours in a day to talk about them and their flaws,
right? And to sort of talk about them as a full human before their life
becomes more myth than reality. And I think that the people here at the
funeral tried to sort of hold onto that space as long as they can before
the train has left the station.

caitlin dickerson

And you heard some of that today, but you've also been reporting for the
last few weeks on George Floyd, who he was. So what have you learned
about his life?

manny fernandez

I spent a lot of time at the place where he's from. And he's from a
place called the Bricks. And the Bricks are a nickname for the Cuney
Homes Public Housing Project in Houston. And he grew up in the Cuney
Homes in the `80s, in the `90s and the early 2000s. And it's a hard
world. But by all accounts, he's a pretty happy kid. George's mother was
sort of a matron of the Cuney Homes. She was raising her kids. She was
raising George. And at the same time, she started raising her own
grandchildren for a time, and she started raising some of the neighbor's
children. And she fed them, they spent the night at her apartment. And
that's who Miss Cissy was. That's who George Floyd's mother was, a
mother to a lot of Cuney Homes.

caitlin dickerson

So what happens once George moves into high school and then adulthood?

manny fernandez

So George Floyd goes to high school just down the street from Cuney
Homes. He goes to Jack Yates High School. He's a big kid. Eventually he
grows to 6'6``, and he kind of immediately becomes a star basketball
player and a star football player. He helps take the football team to
state shampionships in 1992, and he is so good that he gets a basketball
scholarship to go to college in South Florida. And he goes there, and he
plays a little bit of basketball. It doesn't work out. He transfers back
to Texas. He goes to the Kingsville campus of Texas A\&M University, and
he goes there for a couple years. Meanwhile, he's going back and forth
to Houston, back and forth to the Third Ward. And as he's doing that, he
meets this legendary producer named DJ Screw ---

\begin{itemize}
\item
  archived recording (dj screw)\\
  {[}MUSIC{]}:

  --- who eventually becomes sort of a legend in Houston rap circles.

  (SINGING) Hey. Hey!
\end{itemize}

manny fernandez

And there was a time in the early `90s when DJ Screw made a bunch of mix
tapes.

\begin{itemize}
\tightlist
\item
  archived recording (dj screw)\\
  (RAPPING) Welcome, y'all to the fabulous Carolina West. I own this
  {[}EXPLETIVE{]}.
\end{itemize}

manny fernandez

And DJ Screw is rapping on these tapes, but he also invites other
rappers to come in. And a lot of these rappers are just kids from the
neighborhood ---

\begin{itemize}
\tightlist
\item
  archived recording (george floyd)\\
  (RAPPING) Man, it's going down. Know what I'm saying?
\end{itemize}

manny fernandez

--- while George Floyd is one of those guys rapping on DJ Screw's
mixtape.

\begin{itemize}
\tightlist
\item
  archived recording (george floyd)\\
  (RAPPING) Know what I'm saying? Big Floyd representing
  {[}INAUDIBLE{]}.
\end{itemize}

manny fernandez

And he calls himself Big Floyd.

\begin{itemize}
\tightlist
\item
  archived recording (george floyd)\\
  (RAPPING) --- going down like a {[}EXPLETIVE{]}, know what I'm saying?
  Watch me crawl low on my {[}EXPLETIVE{]} spiders. Welcome to the
  ghetto. It's Third Ward, Texas. Boys shopping blades on they
  {[}EXPLETIVE{]} mixes. Boys in ---
\end{itemize}

manny fernandez

And then meanwhile, he's still in college. He's going to Texas A\&M
Kingsville. And it doesn't work out. He pulls out of Texas A\&M, he
never gets his degree and he goes back to Cuney Homes. And that's when
his life sort of takes another turn. And it's in 1997 that he gets his
first run-in with law enforcement. And so for about a decade of his
life, from the age of 23 in 1997, to when he was 34 in 2008, he had a
string of arrests in Houston. Some of the arrests were felonies. Some of
them were misdemeanors. He was arrested for drugs and for robbery, and a
few other charges. His most serious case comes in 2008. He's arrested
for his role in a home invasion robbery, according to court documents.
And so he pleaded guilty to aggravated robbery with a deadly weapon.
He's sentenced to five years in state prison. He only serves four years,
and he's released in 2013. And after he's released from prison, he
really starts to turn his life around. He becomes more religious. George
Floyd has a daughter who's born around that time after he gets out of
prison. And it turns out what we learned at the funeral is that he
actually had five children and two grandchildren. And he starts
reconnecting with his kids. He starts speaking out about and against gun
violence. And he becomes almost this unofficial community leader back in
the Cuney Homes, back in Third Ward, and he has a lot of respect out
there. And then eventually, he gets plugged into this program that will
eventually take him to Minneapolis.

We've been criticized for not writing about and publicizing more of the
details of his criminal history. I think some people have this world
view where if you're an ex-con, then you're an ex-con, and that's all
you'll ever be in your life. And the people in the Cuney Homes, a lot of
them have run-ins with law enforcement. But, you know, your life moves
on after that, and people change. And so I think it's sort of a balance
to try to write about the totality of somebody's life, the good and the
bad, and try to do that in a way that honors the memory of a person
whose reason for being in the news has to do with him being a victim of
a crime and not the perpetrator of one.

caitlin dickerson

So tell me about George Floyd's final years and his final chapter.

manny fernandez

He has a pretty quiet life in Minneapolis. He's living with roommates.
He's working as a security guard at a nightclub. He has a girlfriend.
He's still very religious, reading the Bible. And he has this sort of
quiet life. He called it his new chapter in Minneapolis. The people who
knew him here in Houston say they thought he was pretty happy out there.

{[}music{]}

caitlin dickerson

We'll be right back.

So that's George Floyd the person. And like you said, there's also
George Floyd the symbol and the beginning of a movement. So how did
those two ideas of him play out during his funeral today?

manny fernandez

Yeah.

\begin{itemize}
\tightlist
\item
  archived recording\\
  Amen. Amen.
\end{itemize}

manny fernandez

So the funeral is at a church in Houston called the Fountain of Praise.
And the media wasn't allowed inside. And so I spent most of the day
outside talking to people.

\begin{itemize}
\tightlist
\item
  archived recording\\
  Pastor Wright, we want to bring greetings to everyone who is within
  the sanctuary walls as well as those who are watching via stream or
  some platform today.
\end{itemize}

manny fernandez

But it was live streamed.

\begin{itemize}
\tightlist
\item
  archived recording\\
  {[}ORGAN PLAYING{]} In the tradition of the African-American church,
  this will be a home-going celebration. Come on. I want to say it
  again. This will be a home-going celebration of brother George Floyd
  tonight.
\end{itemize}

manny fernandez

And here you had a number of elected officials, including many of the
African-American political leaders in Houston and in Texas.

\begin{itemize}
\tightlist
\item
  archived recording (sylvester turner)\\
  Let me just speak, briefly say --- let me --- on behalf of the city of
  Houston ---
\end{itemize}

manny fernandez

Mayor Turner of Houston spoke.

\begin{itemize}
\tightlist
\item
  archived recording (sylvester turner)\\
  But as I speak right now, the city attorney is drafting an executive
  order.
\end{itemize}

manny fernandez

And said that ---

\begin{itemize}
\tightlist
\item
  archived recording (sylvester turner)\\
  We will ban chokeholds and strangleholds.
\end{itemize}

manny fernandez

--- he wants to ban chokeholds in the Houston Police Department.

\begin{itemize}
\tightlist
\item
  archived recording (al green)\\
  And I have a resolution that will be presented to the family.
\end{itemize}

manny fernandez

You had Congressman Al Green come up.

\begin{itemize}
\tightlist
\item
  archived recording (al green)\\
  This resolution is going to say to those who look through the vista of
  time that at this time, there lived one among us who was a child of
  God who was taken untimely. But we're going to make sure that those
  who have look through time, that they will know that he made a
  difference within his time, because he changed not only this country,
  not only the United States, he changed the world. George Floyd changed
  the world.
\end{itemize}

manny fernandez

And also ---

\begin{itemize}
\tightlist
\item
  archived recording (joe biden)\\
  Hello, everyone. On this day of prayer where we try to understand
  God's plan and our pain ---
\end{itemize}

manny fernandez

--- Joe Biden made a video message.

\begin{itemize}
\tightlist
\item
  archived recording (joe biden)\\
  Now is the time for racial justice. That's the answer we must give to
  our children when they ask, why? Because when there is justice for
  George Floyd, we will truly be on our way to racial justice in
  America.
\end{itemize}

manny fernandez

And they all sort of talked about and told the family that his death
would not be in vain.

\begin{itemize}
\item
  archived recording (joe biden)\\
  God bless you all. God bless you all. {[}APPLAUSE{]}
\item
  archived recording\\
  I want to ask the members of the family who are going to come up and
  speak at this time, if you would please make your way to the stage.
\end{itemize}

manny fernandez

And then after the first half of the funeral is sort of taken up by
politicians ---

\begin{itemize}
\tightlist
\item
  archived recording (kathleen mcgee)\\
  Welcome, everyone. I am George Floyd's aunt. And I just want to thank
  everybody, and I would like to thank the whole world, what it has done
  for my family today.
\end{itemize}

manny fernandez

--- the family sort of takes over.

\begin{itemize}
\item
  archived recording (kathleen mcgee)\\
  But I just want to make this statement. The world knows George Floyd.
  I know Perry Jr. He was a pesky little rascal. {[}LAUGHS{]}

  But we all loved him.
\end{itemize}

manny fernandez

And they sort of physically take over, and they're up there as a group.

\begin{itemize}
\item
  archived recording (terrence floyd)\\
  (CRYING) I just want to say that I'm going to miss my brother a whole
  lot. And --- {[}APPLAUSE{]}

  I love him. I just want to say to him, I love you. And I thank God for
  giving me my own personal Superman. God bless you all.
\end{itemize}

manny fernandez

And they start talking about their brother and their uncle.

\begin{itemize}
\tightlist
\item
  archived recording (brooke williams)\\
  Hello. My name is Brooke Williams, George Floyd's niece. And I can
  breathe. As long as I'm breathing, justice will be served for Perry.
  First off, I want to thank all of you for coming out to support George
  Perry Floyd. My uncle was a father, brother, uncle and a cousin to
  many. Spiritually grounded, an activist, he always moved people with
  his words.
\end{itemize}

manny fernandez

And it becomes very powerful to hear them talk in a very intimate way
about their relatives.

\begin{itemize}
\item
  archived recording (brooke williams)\\
  My most favorite memory when my uncle was when he paid me to scratch
  his head. After long days of work, we arrived at home. We even created
  a song about it called ``Scratch my head, scratch my head, yeah!''
  {[}LAUGHS{]}

  But after that, I knew he was a comedian. He always told me, baby
  girl, you're going to go so far with that beautiful smile and brains
  of yours.''=
\item
  archived recording (cyril white)\\
  Well then fast forward to 1998, I started a college exhibition tour
  team touring around the country going to play different colleges and
  exhibition games. And Big Floyd, that was my first power forward. I
  would be calling around, trying to get contracts with the different
  schools, and the coaches would ask me, who's your big man? And I would
  say, George Floyd. They'd say, oh, you got Big Floyd. OK, well your
  team must be pretty good. And so then we would go off and play.
\end{itemize}

manny fernandez

And it was those little moments and those little anecdotes that really,
I think, helped people get a sense of who George was.

\begin{itemize}
\tightlist
\item
  archived recording (philonese floyd)\\
  Everybody know who Big Floyd is now. Third Ward, Cuney Homes ---
\end{itemize}

manny fernandez

As the family spoke ---

\begin{itemize}
\tightlist
\item
  archived recording (brady bob)\\
  From the Cuney Home to Jack Yates High ---
\end{itemize}

manny fernandez

--- you really heard ---

\begin{itemize}
\tightlist
\item
  archived recording (cyril white)\\
  --- from Third Ward and the Cuney Homes to come and join me.
\end{itemize}

manny fernandez

--- this sort of Third Ward pride come up.

\begin{itemize}
\tightlist
\item
  archived recording\\
  --- in Third Ward Cuney Home, Texas.
\end{itemize}

manny fernandez

Very historic, black-elected officials live there. It's home to the only
black-owned banking institution in Texas. Beyonce is from the Third
Ward. It's just a place of a lot of black pride and a lot of black
history. At the same time, it's also a place of a lot of struggle and a
lot of poverty. And there's a real strong sense that George Floyd is
from this place that is a hard-fought and very proud place.

\begin{itemize}
\item
  archived recording\\
  {[}ORGAN PLAYING{]}

  At the direction of Senior Pastor, Pastor Remus Wright ---
\end{itemize}

manny fernandez

And then ---

\begin{itemize}
\tightlist
\item
  archived recording\\
  --- my privilege and my honor today ---
\end{itemize}

manny fernandez

--- you have the final eulogy ---

\begin{itemize}
\tightlist
\item
  archived recording\\
  --- a man who needs no introduction but deserves one.
\end{itemize}

manny fernandez

--- delivered by the Reverend Al Sharpton.

\begin{itemize}
\tightlist
\item
  archived recording\\
  Al Sharpton. {[}APPLAUSE{]}
\end{itemize}

manny fernandez

And he appears. He's standing there in a black and white preacher's
robe.

\begin{itemize}
\tightlist
\item
  archived recording (al sharpton)\\
  I hear people talk about what happened to George Floyd like there was
  something less than a crime. This was not just a tragedy, it was a
  crime.
\end{itemize}

manny fernandez

And to me, there was this one moment early on. He's standing up there
and then he puts his glasses on, and he starts reading from this list.

\begin{itemize}
\tightlist
\item
  archived recording (al sharpton)\\
  --- I give him recognition. I must also recognize several families are
  here.
\end{itemize}

manny fernandez

As if he's going to thank some of the different people. And he starts
talking about some of the people who are there, and he says ---

\begin{itemize}
\tightlist
\item
  archived recording (al sharpton)\\
  The mother of Trayvon Martin, will you stand?
\end{itemize}

manny fernandez

--- ``The mother of Trayvon Martin, will you stand?''

\begin{itemize}
\tightlist
\item
  archived recording (al sharpton)\\
  The mother ---
\end{itemize}

manny fernandez

``The mother of Eric Garner, will you stand?''

\begin{itemize}
\tightlist
\item
  archived recording (al sharpton)\\
  The mother of Eric Garner, will you stand?
\end{itemize}

manny fernandez

And he runs through this long list. It's like a roll call.

\begin{itemize}
\tightlist
\item
  archived recording (al sharpton)\\
  The sister of Botham Jean, will you stand?
\end{itemize}

manny fernandez

And people are cheering.

\begin{itemize}
\tightlist
\item
  archived recording (al sharpton)\\
  The family of Pamela Turner right here in Houston, will you stand?
\end{itemize}

manny fernandez

They are standing, the crowd is standing.

\begin{itemize}
\tightlist
\item
  archived recording (al sharpton)\\
  The father of Michael Brown from Ferguson, Missouri, will you stand?
\end{itemize}

caitlin dickerson

Wow. They're all there.

manny fernandez

Yeah.

\begin{itemize}
\tightlist
\item
  archived recording (al sharpton)\\
  The father of Ahmaud Arbery, will you stand?
\end{itemize}

manny fernandez

And to have all of them there at this funeral, they know the pain of
this more than anyone. And they have the right to be angrier than
everyone else. And yet, here they are grieving with George Floyd's
family. And you realize that George Floyd is part of this family of
victims that should not be a family.

\begin{itemize}
\tightlist
\item
  archived recording (al sharpton)\\
  All of these families came to stand with this family, because they
  know better than anyone else the pain they will suffer from the loss
  that they have gone through.
\end{itemize}

manny fernandez

So there was one moment when I think Sharpton pulled together these two
strands of the man and the symbol of George Floyd.

\begin{itemize}
\tightlist
\item
  archived recording (al sharpton)\\
  God always uses unlikely people to do his will.
\end{itemize}

manny fernandez

And that was a moment when Sharpton was alluding to George Floyd's
arrest history.

\begin{itemize}
\tightlist
\item
  archived recording (al sharpton)\\
  If George Floyd had been an Ivy League school graduate and one of
  these ones with a long title, we would have been accused of reacting
  to his prominence. If he'd been a multimillionaire, they would have
  said that we were reacting to his wealth. If he had been a famous
  athlete, as he was on the trajectory to be, we would have said we were
  reacting to his fame. But God took an ordinary brother ---
\end{itemize}

manny fernandez

And he was sort of talking about him as an ordinary ---

\begin{itemize}
\tightlist
\item
  archived recording (al sharpton)\\
  --- from the Third Ward ---
\end{itemize}

manny fernandez

--- imperfect person ---

\begin{itemize}
\tightlist
\item
  archived recording (al sharpton)\\
  --- from the housing projects ---
\end{itemize}

manny fernandez

--- from the Third Ward projects.

\begin{itemize}
\tightlist
\item
  archived recording (al sharpton)\\
  --- that nobody thought much about but those that knew him and loved
  him. He took the rejected stone.
\end{itemize}

manny fernandez

And it was a very powerful moment where he called George Floyd a
rejected stone, making a reference to scripture.

\begin{itemize}
\tightlist
\item
  archived recording (al sharpton)\\
  God took the rejected stone and made him the cornerstone of a movement
  that's going to change the whole wide world. {[}APPLAUSE{]}
\end{itemize}

manny fernandez

And how those officers may have thought that nobody cared about a guy
like that.

\begin{itemize}
\tightlist
\item
  archived recording (al sharpton)\\
  Oh, if you would have had any idea that all of us would react, you'd
  have took your knee off his neck.
\end{itemize}

manny fernandez

And obviously the world knows now that the world did care about somebody
like that, and how he died and how he was treated.

\begin{itemize}
\tightlist
\item
  archived recording (al sharpton)\\
  If you had any idea that preachers white and black was going to line
  up in a pandemic when we were told to stay inside, and we'd come out
  and march in the streets at the risk of our health, you'd have took
  your knee off his neck. Because you thought his neck didn't mean
  nothing. But God made his neck to connect his head to his body, and
  you had no right to put your knee on that neck.
\end{itemize}

manny fernandez

I think in the past, I think there has been this desire to only pay
attention to sort of perfect victims, only to give attention to cases in
which the person had this sort of holy life. And any brush with the law,
no matter how many years ago, somehow was thought to taint how people
viewed whatever police killing was in the news. And I think that shifted
a little bit. And I see the difference in George Floyd.

\begin{itemize}
\tightlist
\item
  archived recording (al sharpton)\\
  Your family's going to miss you, George. But your nation is going to
  remember your name.
\end{itemize}

manny fernandez

And Sharpton ended his remarks by touching on this idea that George
Floyd was imperfect, and he still deserves the movement that was
happening.

\begin{itemize}
\item
  archived recording (al sharpton)\\
  So we're going to lay you to your mama now. You called for mama. We're
  going to lay your body next to hers. But I know mama's already
  embraced you, George. You fought a good fight. You kept the faith. You
  finished your course. Go on and get your rest now. Go on and see mama
  now. We're going to fight on. We're going to fight on. We're going to
  fight on. We're going to fight on. {[}ORGAN MUSIC{]}
\item
  archived recording (george floyd)\\
  I'm going to speak to y'all real quick. I just want to say, man, that
  I got my shortcomings and my flaws, and I ain't better than nobody
  else. But man, the shootings that's going on man, I don't care what
  hood you're from man, where you're at, man, I love you and God loves
  you, man. Put them guns down, man. That ain't what it is. You know, we
  grow up this, man. And y'all hold y'all head up, man. You got parents
  out here selling plates, man, trying to bury their kids, man. Think
  about it, man. Love y'all.
\end{itemize}

{[}music{]}

caitlin dickerson

We'll be right back.

Here's what else you need to know today. On Tuesday morning, President
Trump endorsed a conspiracy theory that a 75-year-old man --- who police
were filmed pushing to the ground during a protest in Buffalo last week
--- had been using his cell phone to knock out law enforcement radios on
behalf of the Antifa movement. In a tweet, the president offered no
evidence of the theory but named a right wing news organization, One
America News Network, in his tweet.

\begin{itemize}
\item
  archived recording\\
  Did you have a reaction to the president's tweet early ---
\item
  archived recording (mark meadows)\\
  I learned a long time ago not to comment on tweets, and I'm not going
  to break that ---
\item
  archived recording\\
  But they are official statements.
\end{itemize}

caitlin dickerson

Later in the day, Republican lawmakers and administration officials,
including the White House chief of staff, Mark Meadows, dodged questions
from reporters. The man who was injured in the incident, Martin Gugino,
is still recovering in the hospital from a serious head injury.
Meanwhile, a police officer in New York City was arrested and charged
with assault on Tuesday after shoving a young woman to the ground,
giving her a concussion, another scene that was filmed on a cell phone.
And ---

\begin{itemize}
\tightlist
\item
  archived recording\\
  This is wrong! This is America! Please, God, help us! I mean it! This
  is a crisis in our world to make us not exercise our right to vote!
\end{itemize}

caitlin dickerson

Five states held their primary elections on Tuesday, including Georgia,
where a new voting system put into place in 2018 after claims of voter
suppression experienced catastrophic meltdowns. State-ordered voting
machines were said to be missing or malfunctioning, causing voters to
wait in line for hours at polling places across the state. Some gave up
and left before casting a vote. The problems were made worse by the
coronavirus pandemic, which left fewer poll workers available than usual
and added to wait times, because machines had to be disinfected.
Predominantly black areas of Georgia experienced some of the worst
obstacles to voting, raising concerns that the problems would further
disenfranchise black voters.

{[}music{]}

That's it for ``The Daily.'' I'm Caitlin Dickerson. See you tomorrow.

\includegraphics{https://static01.graylady3jvrrxbe.onion/images/2020/06/08/us/08FLOYD-PROFILE-07/08FLOYD-PROFILE-07-articleLarge.jpg?quality=75\&auto=webp\&disable=upscale}

Now a time stamp in the prolonged history of violence against black
people, Mr. Floyd's killing has inspired people of every race to
\href{https://www.nytimes3xbfgragh.onion/2020/06/06/us/george-floyd-memorial-protests.html}{march
in the streets} and kneel, chanting ``black lives matter'' in hundreds
of cities and small towns.

But Mr. Floyd, 46, was more than the nearly nine-minute graphic video of
his death. He was more than the 16 utterances, captured in the
recording, of some version of ``I can't breathe.''

He was an outsize man who dreamed equally big, unswayed by the setbacks
of his life.

Growing up in one of Houston's poorest neighborhoods, he enjoyed a star
turn as a basketball and football player, with three catches for 18
yards in a
\href{https://www.expressnews.com/texas-sports-nation/highschool/article/George-Floyd-death-Yates-High-School-football-15298818.php}{state
championship game} his junior year.

He was the first of his siblings to go to college, and he did so on an
athletic scholarship. But he returned to Texas after a couple of years,
and lost nearly a decade to arrests and incarcerations on mostly
drug-related offenses. By the time he left his hometown for good a few
years ago, moving 1,200 miles to Minneapolis for work, he was ready for
a fresh start.

When he traveled to Houston in 2018 for his mother's funeral --- they
died two years, one week apart --- he told his family that Minneapolis
had begun to feel like home. He had his mother's name tattooed on his
belly, a fact that was noted in his autopsy.

\hypertarget{life-in-the-bricks}{%
\subsection{Life in the Bricks}\label{life-in-the-bricks}}

Mr. Floyd was born in Fayetteville, N.C., to George Perry and Larcenia
Floyd. But he was really from a Houston neighborhood called the Bricks.

After his parents split up, his mother moved him and his siblings to
Texas, where he grew up in the red brick world of Cuney Homes, a
low-slung 564-unit public housing complex
\href{https://www.chron.com/news/houston-texas/houston/article/George-Floyd-police-brutality-minneapolis-dead-vid-15296192.php}{in
Houston's Third Ward} that was named for Norris Wright Cuney, one of the
most politically powerful black men in the state in the late 1800s.

Mr. Floyd's mother --- who was known as Cissy --- was among the leaders
of Cuney Homes and an active member of the resident council. She raised
her own children and, at times, some of her grandchildren and some of
her neighbors' children, too.

As a child, Mr. Floyd was known in the Bricks as Perry, his middle name.
As he grew, so, too, did his nicknames. He was Big Floyd, known as much
for his big personality as his sense of humor.

Mr. Floyd's height --- he was more than six feet tall in middle school
--- created a kind of mystique.

``You can just imagine this tall kid as a freshman in high school
walking the hallways. We were like, `Man, who is that guy?' He was a
jokester, always laughing and cracking jokes,'' said Herbert Mouton, 45,
who played on the Yates high school football team with Mr. Floyd. ``We
were talking the other day with classmates trying to think, `Had Floyd
even ever had a fight before?' And we couldn't recall it.''

Mr. Mouton said that after the loss of a big game, Mr. Floyd would let
the team sulk for a few minutes before telling a joke to lighten the
mood. ``He never wanted us to feel bad for too long,'' he said.

Image

Mr. Floyd in a classroom at Jack Yates High School in Houston. He was a
celebrated football and basketball athlete.

Mr. Floyd saw sports as the path out of the Bricks. And so he leaned
into his size and athletic prowess in a sports-obsessed state. As a
tight end, Mr. Floyd helped power his football team to the state
championship game in 1992.

In one exhilarating moment that was captured on video --- and circulated
after his death ---
\href{https://twitter.com/CourtneyABC13/status/1265603534852063233}{Mr.
Floyd soars above an opponent} in the end zone to catch a touchdown
pass.

After graduating from high school, Mr. Floyd left Texas on a basketball
scholarship to South Florida Community College (now South Florida State
College).

``I was looking for a power forward and he fit the bill. He was athletic
and I liked the way he handled the ball,'' said George Walker, who
recruited Mr. Floyd. ``He was a starter and scored 12 to 14 points and
seven to eight rebounds.''

Mr. Floyd transferred two years later, in 1995, to Texas A\&M
University's Kingsville campus, but he did not stay long. He returned
home to Houston --- and to the Third Ward --- without a degree.

Known locally as the Tré, the Third Ward, south of downtown, is among
the city's historic black neighborhoods, and it has been featured in the
music of one of the most famous people to grow up there, Beyoncé.

At times, life in the Bricks was unforgiving. Poverty, drugs, gangs and
violence scarred many Third Ward families. Several of Mr. Floyd's
classmates did not live past their 20s.

Soon after returning, Mr. Floyd started rapping. He appeared as Big
Floyd on mixtapes created by DJ Screw, a fixture in Houston's hip-hop
scene in the 1990s. His voice deep, his rhymes purposefully delivered at
a slow-motion clip, Mr. Floyd rapped about ``choppin' blades'' ---
driving cars with oversize rims --- and his Third Ward pride.

For about a decade starting in his early 20s, Mr. Floyd had a string of
arrests in Houston, according to court and police records. One of those
arrests, for a \$10 drug deal in 2004, cost him 10 months in a state
jail.

Four years later, Mr. Floyd pleaded guilty to aggravated robbery with a
deadly weapon and spent four years in prison. He was released in 2013
and returned home again --- this time to begin the long, hard work of
trying to turn his life around, using his missteps as a lesson for
others.

\href{https://www.nytimes3xbfgragh.onion/2020/06/11/sports/basketball/stephen-jackson-george-floyd-protests.html}{Stephen
Jackson}, a retired professional basketball player from Port Arthur,
Texas, met Mr. Floyd a year or two before Mr. Jackson joined the N.B.A.
They had sports in common, Mr. Jackson said, but they also looked alike
--- enough to call each other ``twin'' as a term of endearment.

``I tell people all the time, the only difference between me and George
Floyd, the only difference between me and my twin, the only difference
between me and Georgie, is the fact that I had more opportunities,'' he
said, later adding, ``If George would have had more opportunities, he
might have been a pro athlete in two sports.''

Image

Veronica DeBoest said Mr. Floyd's mother,~Larcenia Floyd, was one of the
leaders of the Cuney Homes housing complex.~Credit...Michael Starghill
Jr. for The New York Times

After prison, Mr. Floyd became even more committed to his church.
Inspired by a daughter, Gianna Floyd, born after he was released, Mr.
Floyd spent a lot of time at Resurrection Houston, a church that holds
many of its services on the basketball court in the middle of Cuney
Homes. He would set up chairs and drag out to the center of the court
the service's main attraction --- the baptism tub.

``We'd baptize people on the court and we've got this big old horse
trough. And he'd drag that thing by himself onto that court,'' said
Patrick Ngwolo, a lawyer and pastor of Resurrection Houston, who
described Mr. Floyd as a father figure for younger community residents.

Eventually, Mr. Floyd became involved in a Christian program with a
history of taking men to Minnesota from the Third Ward and providing
them with drug rehabilitation and job placement services.

``When you say, `I'm going to Minnesota,' everybody knows you're going
to this church-work program out of Minnesota,'' Mr. Ngwolo said, ``and
you're getting out of this environment.''

His move would be a fresh start, Mr. Ngwolo said, his story one of
redemption.

Image

In a baby book for Gianna Floyd, the daughter of George Floyd, is a
photo of the two of them together.Credit...Victor J. Blue for The New
York Times

\hypertarget{a-protector-of-people}{%
\subsection{A Protector of People}\label{a-protector-of-people}}

In Minnesota, Mr. Floyd lived in a red clapboard duplex with two
roommates on the eastern edge of St. Louis Park, a leafy, gentrifying
Minneapolis suburb.

Beginning sometime in 2017, he worked as a security guard at the
Salvation Army's Harbor Light Center, a downtown homeless shelter and
transitional housing facility. The staff members got to know Mr. Floyd
as someone with a steady temperament, whose instinct to protect
employees included walking them to their cars.

``It takes a special person to work in the shelter environment,'' said
Brian Molohon, executive director of development at the Salvation Army
Northern Division. ``Every day you are bombarded with heartache and
brokenness.''

Even as Mr. Floyd settled into his position, he looked for other jobs.
While working at the Salvation Army, he answered a job ad for a bouncer
at Conga Latin Bistro, a restaurant and dance club.

Jovanni Thunstrom, the owner, said Mr. Floyd quickly became part of the
work family. He came in early and left late. And though he tried, he
never quite mastered salsa dancing.

``Right away I liked his attitude,'' said Mr. Thunstrom, who was also
Mr. Floyd's landlord. ``He would shake your hand with both hands. He
would bend down to greet you.''

Mr. Floyd kept a Bible by his bed. Often, he read it aloud. And despite
his height, Mr. Floyd would fold himself in the hallway to frequently
pray with Theresa Scott, one of his roommates.

``He had this real cool way of talking. His voice reminded me of Ray
Charles. He'd talk fast and he was so soft-spoken,'' said Alvin Manago,
55, who met Mr. Floyd at a 2016 softball game. They bonded instantly and
became roommates. ``He had this low-pitched bass. You had to get used to
his accent to understand him. He'd say, `Right-on, right-on,
right-on.'''

Mr. Floyd spent the final weeks of his life recovering from the
coronavirus, which he learned he had in early April. After he was
better, he started spending more time with his girlfriend, and he had
not seen his roommates in a few weeks, Mr. Manago said.

Like millions of people, his roommates in the city that was to be his
fresh start watched the video that captured Mr. Floyd taking his last
breaths. They heard him call out for his late mother --- ``Mama! Mama!''

On Tuesday morning, 15 days after that anguished cry, Mr. Floyd will be
laid to rest beside her.

Image

Thousands of protesters gathered near the White House on Saturday to
protest the killing of Mr. Floyd.~Credit...Erin Schaff/The New York
Times

Manny Fernandez reported from Houston and Audra D. S. Burch from
Hollywood, Fla. Contributing reporting were Marc Stein from Dallas,
Erica L. Green from Washington, and Dionne Searcey and Matt Furber from
Minneapolis. Susan Beachy contributed research.

Advertisement

\protect\hyperlink{after-bottom}{Continue reading the main story}

\hypertarget{site-index}{%
\subsection{Site Index}\label{site-index}}

\hypertarget{site-information-navigation}{%
\subsection{Site Information
Navigation}\label{site-information-navigation}}

\begin{itemize}
\tightlist
\item
  \href{https://help.nytimes3xbfgragh.onion/hc/en-us/articles/115014792127-Copyright-notice}{©~2020~The
  New York Times Company}
\end{itemize}

\begin{itemize}
\tightlist
\item
  \href{https://www.nytco.com/}{NYTCo}
\item
  \href{https://help.nytimes3xbfgragh.onion/hc/en-us/articles/115015385887-Contact-Us}{Contact
  Us}
\item
  \href{https://www.nytco.com/careers/}{Work with us}
\item
  \href{https://nytmediakit.com/}{Advertise}
\item
  \href{http://www.tbrandstudio.com/}{T Brand Studio}
\item
  \href{https://www.nytimes3xbfgragh.onion/privacy/cookie-policy\#how-do-i-manage-trackers}{Your
  Ad Choices}
\item
  \href{https://www.nytimes3xbfgragh.onion/privacy}{Privacy}
\item
  \href{https://help.nytimes3xbfgragh.onion/hc/en-us/articles/115014893428-Terms-of-service}{Terms
  of Service}
\item
  \href{https://help.nytimes3xbfgragh.onion/hc/en-us/articles/115014893968-Terms-of-sale}{Terms
  of Sale}
\item
  \href{https://spiderbites.nytimes3xbfgragh.onion}{Site Map}
\item
  \href{https://help.nytimes3xbfgragh.onion/hc/en-us}{Help}
\item
  \href{https://www.nytimes3xbfgragh.onion/subscription?campaignId=37WXW}{Subscriptions}
\end{itemize}
