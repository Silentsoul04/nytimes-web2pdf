Sections

SEARCH

\protect\hyperlink{site-content}{Skip to
content}\protect\hyperlink{site-index}{Skip to site index}

\href{https://www.nytimes3xbfgragh.onion/section/health}{Health}

\href{https://myaccount.nytimes3xbfgragh.onion/auth/login?response_type=cookie\&client_id=vi}{}

\href{https://www.nytimes3xbfgragh.onion/section/todayspaper}{Today's
Paper}

\href{/section/health}{Health}\textbar{}How to Protect Yourself and
Prepare for the Coronavirus

\url{https://nyti.ms/2x0zErW}

\begin{itemize}
\item
\item
\item
\item
\item
\end{itemize}

\href{https://www.nytimes3xbfgragh.onion/news-event/coronavirus?action=click\&pgtype=Article\&state=default\&region=TOP_BANNER\&context=storylines_menu}{The
Coronavirus Outbreak}

\begin{itemize}
\tightlist
\item
  live\href{https://www.nytimes3xbfgragh.onion/2020/08/04/world/coronavirus-cases.html?action=click\&pgtype=Article\&state=default\&region=TOP_BANNER\&context=storylines_menu}{Latest
  Updates}
\item
  \href{https://www.nytimes3xbfgragh.onion/interactive/2020/us/coronavirus-us-cases.html?action=click\&pgtype=Article\&state=default\&region=TOP_BANNER\&context=storylines_menu}{Maps
  and Cases}
\item
  \href{https://www.nytimes3xbfgragh.onion/interactive/2020/science/coronavirus-vaccine-tracker.html?action=click\&pgtype=Article\&state=default\&region=TOP_BANNER\&context=storylines_menu}{Vaccine
  Tracker}
\item
  \href{https://www.nytimes3xbfgragh.onion/2020/08/02/us/covid-college-reopening.html?action=click\&pgtype=Article\&state=default\&region=TOP_BANNER\&context=storylines_menu}{College
  Reopening}
\item
  \href{https://www.nytimes3xbfgragh.onion/live/2020/08/04/business/stock-market-today-coronavirus?action=click\&pgtype=Article\&state=default\&region=TOP_BANNER\&context=storylines_menu}{Economy}
\end{itemize}

Advertisement

\protect\hyperlink{after-top}{Continue reading the main story}

Supported by

\protect\hyperlink{after-sponsor}{Continue reading the main story}

April 6, 2020, 6:17 p.m. ET

April 6, 2020, 6:17 p.m. ET

\hypertarget{how-to-protect-yourself-and-prepare-for-the-coronavirus}{%
\section{How to Protect Yourself and Prepare for the
Coronavirus}\label{how-to-protect-yourself-and-prepare-for-the-coronavirus}}

With a clear head and some simple tips, you can help reduce your risk,
prepare your family and do your part to protect others.

By \href{https://www.nytimes3xbfgragh.onion/by/amelia-nierenberg}{Amelia
Nierenberg} and
\href{https://www.nytimes3xbfgragh.onion/by/tim-herrera}{Tim Herrera}

\href{https://www.nytimes3xbfgragh.onion/es/article/el-coronavirus-proteger-preparar.html}{Leer
en español}

\hypertarget{heres-what-you-can-do}{%
\subsubsection{Here's what you can do:}\label{heres-what-you-can-do}}

\begin{itemize}
\tightlist
\item
  \protect\hyperlink{link-5b4d0175}{Stay home}
\item
  \protect\hyperlink{link-7e8a3c72}{When going outside, be extra
  cautious}
\item
  \protect\hyperlink{link-2963e7f4}{Consider wearing a mask in public}
\item
  \protect\hyperlink{link-364448a4}{Wash your hands. With soap. Then
  wash them again.}
\item
  \protect\hyperlink{link-4b7fc30f}{With children, keep calm, carry on
  and get the flu shot}
\item
  \protect\hyperlink{link-385f8aea}{Stock up on groceries, medicine and
  resources}
\end{itemize}

The
\href{https://www.nytimes3xbfgragh.onion/news-event/coronavirus}{coronavirus
continues to spread worldwide}, with over 1.2 million confirmed cases
and at least 72,000 dead. In the United States, there have been at least
350,000 cases and more than 10,500 deaths,
\href{https://www.nytimes3xbfgragh.onion/interactive/2020/world/coronavirus-maps.html?action=click\&module=RelatedLinks\&pgtype=Article}{according
to a New York Times database}.

The
\href{https://www.nytimes3xbfgragh.onion/2020/04/03/technology/coronavirus-masks-shortage.html}{coronavirus}
is
\href{https://www.nytimes3xbfgragh.onion/interactive/2020/world/asia/china-coronavirus-contain.html}{spreading
very quickly}. Older Americans, those with underlying health conditions
and
\href{https://www.nytimes3xbfgragh.onion/2020/03/10/us/coronavirus-homeless.html?smtyp=cur\&smid=tw-nytimes}{those
without a social safety net} are the most vulnerable to the infection
and to its
\href{https://www.nytimes3xbfgragh.onion/live/2020/coronavirus-usa-03-16}{societal
disruption}.

Though life as we know it is sharply off kilter, there are measures you
can take.

Most important: \emph{Do not panic}. With a clear head and
\href{https://www.nytimes3xbfgragh.onion/2020/03/10/us/politics/coronavirus-guidelines.html}{some
simple tips}, you can help reduce your risk, prepare your family and do
your part to protect others.

\hypertarget{stay-home}{%
\subsection{Stay home}\label{stay-home}}

\hypertarget{it-can-be-its-own-challenge-here-are-some-tips}{%
\subsubsection{It can be its own challenge. Here are some
tips.}\label{it-can-be-its-own-challenge-here-are-some-tips}}

For people fortunate enough to be able to stay home, being stuck inside
24 hours a day for weeks on end is unlike anything any of us has ever
experienced. It's a whole new set of stressors and unique experiences
--- on top of the very real
\href{https://www.nytimes3xbfgragh.onion/2020/03/25/business/coronavirus-families-cabin-fever.html}{cabin
fever} that can set in. But as difficult as sheltering in place can be,
remember that it's all about keeping you, your loved ones and your
community safe.

First, remember that it's OK to feel stressed and unproductive;
\href{https://www.nytimes3xbfgragh.onion/2020/03/23/smarter-living/coronavirus-coping-tips.html}{give
yourself permission to feel whatever it is you're feeling}. Because
we're spending so much time online, it can feel like you're falling
behind --- \emph{why haven't I finished that book and knitted that scarf
and cooked that feast yet?!} ---
\href{https://www.nytimes3xbfgragh.onion/2020/04/01/style/productivity-coronavirus.html}{``but
staying inside and attending to basic needs is plenty.''} And if you
have children,
\href{https://www.nytimes3xbfgragh.onion/2020/04/01/parenting/coronavirus-help-anxious-kid.html}{acknowledge
that these changes to daily life are difficult}.

Among those basic needs is organizing and cleaning your home, both
vastly different tasks than they used to be. To keep the home running
smoothly, consider these
\href{https://www.nytimes3xbfgragh.onion/2020/04/02/smarter-living/protect-your-home-against-the-onslaught.html}{tips
to keep your appliances functioning, the mess to a minimum and the
clutter at bay} and
\href{https://www.nytimes3xbfgragh.onion/2020/03/26/style/how-to-do-laundry-coronavirus.html}{changes
you could make in how you do laundry}.

As for cleaning your home, prioritize high-touch surfaces, including
door knobs, light switches, refrigerator and microwave doors, drawer
pulls, TV remotes, counters and table tops, toilet handles and faucet
handles.
\href{https://www.nytimes3xbfgragh.onion/2020/03/18/world/clean-home-coronavirus.html}{Here's
everything you need to know about cleaning your home}. But remember:
You're dealing with potentially harmful chemicals,
\href{https://www.nytimes3xbfgragh.onion/2020/04/02/smarter-living/coronavirus-clean-home-house-disinfect.html}{so
don't accidentally poison yourself while cleaning}.

Finally:
\href{https://www.nytimes3xbfgragh.onion/2020/04/01/well/move/steps-walking-longevity-health.html}{Don't
forget to keep moving}. It's good for your health, mind and soul.

\includegraphics{https://static01.graylady3jvrrxbe.onion/images/2020/05/01/multimedia/01prepare-for-coronavirus1/merlin_171019440_adf10b1a-c5ee-43cc-b330-ed71c6ff16ca-articleLarge.jpg?quality=75\&auto=webp\&disable=upscale}

\hypertarget{when-going-outside-be-extra-cautious}{%
\subsection{When going outside, be extra
cautious}\label{when-going-outside-be-extra-cautious}}

\hypertarget{you-can-do-your-part-to-help-your-community-and-the-world-do-not-get-close-to-other-people}{%
\subsubsection{You can do your part to help your community and the
world. Do not get close to other
people.}\label{you-can-do-your-part-to-help-your-community-and-the-world-do-not-get-close-to-other-people}}

This is called
``\href{https://www.nytimes3xbfgragh.onion/2020/03/16/smarter-living/coronavirus-social-distancing.html}{social
distancing}'' or ``physical distancing,'' and is basically a call to
stand far away from other people, even if you have no underlying health
conditions or coronavirus symptoms. Experts believe the coronavirus
\href{https://www.nytimes3xbfgragh.onion/2020/03/02/health/coronavirus-how-it-spreads.html}{travels
through droplets}, so limiting your exposure to other people is a good
way to protect yourself.

Avoid public transportation when possible, limit nonessential travel,
work from home and skip social gatherings. You can go outside, as long
as you avoid being in close contact with people.

\emph{{[}How to keep your distance:}
\href{https://www.nytimes3xbfgragh.onion/2020/03/19/well/live/coronavirus-quarantine-social-distancing.html}{\emph{A
guide to help you make the right decisions}}\emph{{]}}

\hypertarget{consider-wearing-a-mask-in-public}{%
\subsection{Consider wearing a mask in
public}\label{consider-wearing-a-mask-in-public}}

\hypertarget{the-cdc-advises-all-americans-to-wear-cloth-masks-in-public-president-trump-says-it-wont-be-mandatory}{%
\subsubsection{\texorpdfstring{\textbf{The C.D.C. advises all Americans
to wear cloth masks in public. President Trump says it won't be
mandatory.}}{The C.D.C. advises all Americans to wear cloth masks in public. President Trump says it won't be mandatory.}}\label{the-cdc-advises-all-americans-to-wear-cloth-masks-in-public-president-trump-says-it-wont-be-mandatory}}

\href{https://www.nytimes3xbfgragh.onion/2020/04/03/world/coronavirus-news-updates.html}{This
C.D.C. recommendation} is a shift in federal guidance, and reflects
\href{https://www.nytimes3xbfgragh.onion/2020/03/31/health/coronavirus-asymptomatic-transmission.html}{concerns
that the coronavirus is being spread by infected people who have no
symptoms}.

Until now, experts at the C.D.C. had been saying that ordinary people
didn't need to wear masks unless they were sick and coughing. Part of
the reason was to preserve medical-grade masks for health care workers
who desperately need them at a time when they are in continuously short
supply. (The New York Times and other news outlets had been reporting
the C.D.C.'s previous guidance.)

Top officials at the C.D.C. had been pushing for Mr. Trump to advise
everyone --- even people who appear to be healthy --- to wear a mask
when shopping at the grocery store or going out in other public places,
to avoid unwittingly spreading the virus. Public health officials have
stressed that N95 masks and surgical masks should be saved for
front-line doctors and nurses, who have been in dire need of protective
gear.

Mask wearing doesn't replace hand washing and social distancing.

\emph{{[}}\href{https://www.nytimes3xbfgragh.onion/article/coronavirus-N95-mask-DIY-face-mask-health.html}{\emph{Here
is our guidance}} \emph{on how to best protect yourself, including a
pattern to make}
\textbf{\href{https://www.nytimes3xbfgragh.onion/article/how-to-make-face-mask-coronavirus.html}{\emph{your
own cloth mask}}}\emph{.{]}}

Image

Experts have stressed steps like washing your hands and social
distancing to help stop the coronavirus's spread.Credit...Caitlin
Ochs/Reuters

\hypertarget{wash-your-hands-with-soap-then-wash-them-again}{%
\subsection{Wash your hands. With soap. Then wash them
again.}\label{wash-your-hands-with-soap-then-wash-them-again}}

\hypertarget{its-not-sexy-but-it-works}{%
\subsubsection{It's not sexy, but it
works.}\label{its-not-sexy-but-it-works}}

Wash your hands, wash your hands,
\href{https://www.nytimes3xbfgragh.onion/2020/03/13/world/how-to-wash-your-hands-coronavirus.html}{\emph{wash
your hands}}. That splash-under-water flick won't cut it anymore.

A refresher: Wet your hands and scrub them with soap, taking care to get
between your fingers and under your nails. Wash for at least 20 seconds
(or about the time it takes to sing ``Happy Birthday'' twice), and dry.
Make sure you get your thumbs, too. The C.D.C. also recommends you
\href{https://www.cdc.gov/coronavirus/2019-ncov/hcp/guidance-prevent-spread.html}{avoid
touching your eyes, nose and mouth with unwashed hands}
(\href{https://www.nytimes3xbfgragh.onion/2020/03/05/health/stop-touching-your-face-coronavirus.html}{tough
one, we know}).

Alcohol-based hand sanitizers, which should be rubbed in for about 20
seconds, can also work, but the gel must contain at least 60 percent
alcohol. (No, Tito's Handmade Vodka
\href{https://www.cnn.com/2020/03/05/us/titos-vodka-coronavirus-trnd/index.html}{doesn't
work}.)

Also,
\href{https://www.cdc.gov/coronavirus/2019-ncov/hcp/guidance-prevent-spread.html}{clean
``high-touch'' surfaces}, like
\href{https://www.nytimes3xbfgragh.onion/2020/03/12/smarter-living/clean-your-phone.html}{phones},
tablets and handles.
\href{https://support.apple.com/en-us/HT204172?mod=article_inline}{Apple
recommends} using 70 percent isopropyl alcohol, wiping gently. ``Don't
use bleach,'' the company said.

To disinfect any surface, the C.D.C. recommends wearing disposable
gloves and washing hands thoroughly immediately after removing the
gloves. Most household disinfectants registered by the Environmental
Protection Agency will work.

Try to stand away from other people, especially if they seem sick. Wave,
bow or give an elbow bump, rather than shaking hands.

\emph{{[}Watch our guide on}
\href{https://www.nytimes3xbfgragh.onion/2020/03/13/world/how-to-wash-your-hands-coronavirus.html}{\emph{how
to wash your hands}}\emph{.{]}}

\hypertarget{with-children-keep-calm-carry-on-and-get-the-flu-shot}{%
\subsection{With children, keep calm, carry on and get the flu
shot}\label{with-children-keep-calm-carry-on-and-get-the-flu-shot}}

\hypertarget{the-good-news-is-that-cases-in-children-have-been-very-rare}{%
\subsubsection{\texorpdfstring{\textbf{The good news is that cases in
children have been very
rare.}}{The good news is that cases in children have been very rare.}}\label{the-good-news-is-that-cases-in-children-have-been-very-rare}}

Right now, there's little reason for parents to worry about their
children, the experts say; coronavirus cases in children have been very
rare.

The flu vaccine is a must, as vaccinating children is good protection
for older people. And
\href{https://www.nytimes3xbfgragh.onion/2020/03/09/parenting/coronavirus-parents-need-to-know.html}{take
the same precautions} you would during a normal flu season: Encourage
frequent hand-washing, move away from people who appear sick, and get
the flu shot.

As in airplanes, it's always best to make sure your metaphorical oxygen
mask is on before helping others. When
\href{https://parenting.nytimes3xbfgragh.onion/childrens-health/coronavirus-kids-talk}{talking
to your children about an outbreak}, make sure that you first assess
their knowledge of the virus and that you process your own anxiety. It's
important that you don't dismiss their fears and that you speak to them
at an age-appropriate level.

Be sure to be in communication with
\href{https://parenting.nytimes3xbfgragh.onion/childrens-health/coronavirus-outbreak-schools}{your
child's school}, including about early dismissals or possible online
instruction.
\href{https://www.nytimes3xbfgragh.onion/interactive/2020/nyregion/school-closings-ny-nj.html}{Be
prepared for schools to close}; many districts and universities around
the world have already taken that step.

It's also good to communicate with your workplace about child-care
concerns that you have.

If your children are
\href{https://parenting.nytimes3xbfgragh.onion/preschooler/coronavirus-schools-lessons?module=latest-filters-feed\&action=click\&rank=3\&position=5}{stuck
at home}, get some games going, turn on a movie and
\href{https://parenting.nytimes3xbfgragh.onion/childrens-health/coronavirus-parents-need-to-know?module=latest-filters-feed\&action=click\&rank=5\&position=5}{try
to make it feel a little like a vacation}, at least for the first few
days.

\emph{{[}For more information about children and the pandemic, read}
\href{https://www.nytimes3xbfgragh.onion/2020/03/09/parenting/coronavirus-parents-need-to-know.html}{\emph{11
Questions Parents May Have About Coronavirus}}\emph{.{]}}

\hypertarget{stock-up-on-groceries-medicine-and-resources}{%
\subsection{Stock up on groceries, medicine and
resources}\label{stock-up-on-groceries-medicine-and-resources}}

\hypertarget{preparation-is-the-best-way-to-protect-your-family-and-loved-ones}{%
\subsubsection{\texorpdfstring{\textbf{Preparation is the best way to
protect your family and loved
ones.}}{Preparation is the best way to protect your family and loved ones.}}\label{preparation-is-the-best-way-to-protect-your-family-and-loved-ones}}

Stock up on a 30-day supply of groceries, household supplies and
prescriptions.

That doesn't mean you'll need to eat only beans and ramen. Here are tips
to
\href{https://www.nytimes3xbfgragh.onion/2020/03/06/dining/how-to-stock-a-pantry.html}{stock
a pantry} with shelf-stable and tasty foods. (Don't forget the
chocolate.) Once you've got the food you'll need, use
\href{https://www.nytimes3xbfgragh.onion/2020/04/06/realestate/virus-organize-pantry-tips.html}{this
guide to organize your pantry.} One quick rule of thumb: Put everyday
items at eye level for easy access.
\href{https://www.nytimes3xbfgragh.onion/2020/03/26/well/eat/coronavirus-shopping-food-groceries-infection.html}{Also,
be careful when you're buying those groceries}.

If you take prescription medications, or are low on any over-the-counter
essentials, go to the pharmacy sooner rather than later.

And, in no particular order, make sure you're set with soap, toiletries,
laundry detergent, toilet paper and, if you have small children,
diapers.

\hypertarget{when-it-comes-to-money-uncertainty-is-the-new-normal}{%
\subsection{When it comes to money, uncertainty is the new
normal}\label{when-it-comes-to-money-uncertainty-is-the-new-normal}}

\hypertarget{its-unclear-what-an-economic-recovery-will-look-like--or-when-it-will-come}{%
\subsubsection{It's unclear what an economic recovery will look like ---
or when it will
come.}\label{its-unclear-what-an-economic-recovery-will-look-like--or-when-it-will-come}}

The impact of the virus on the United States economy has been swift and
devastating. Nearly 10 million Americans have filed for unemployment
insurance in the past two weeks, and some estimates say the unemployment
rate is likely
\href{https://www.nytimes3xbfgragh.onion/2020/04/03/upshot/coronavirus-jobless-rate-great-depression.html}{higher
than at any point since the Great Depression}. As we struggle to fight
the virus itself, it's unclear what an economic recovery will look like
--- or when it will come.

If you're filing for unemployment, there is a lot to know, so
\href{https://www.nytimes3xbfgragh.onion/2020/03/17/your-money/unemployment-insurance-coronavirus.html}{read
this guide on unemployment insurance}. (You should also be prepared for
a potentially
\href{https://www.nytimes3xbfgragh.onion/2020/04/04/nyregion/coronavirus-ny-unemployment-benefits.html}{tough
journey through bureaucracy}.)

Don't forget to work on your emergency fund;
\href{https://www.nytimes3xbfgragh.onion/2020/03/20/your-money/coronavirus-emergency-fund.html}{here's
how to keep building it during a financial crisis}.

For Americans with a retirement account, it has been gut-wrenching to
watch double-digit percentages of it evaporate in a matter of weeks. Not
only have we seen the market's
\href{https://www.nytimes3xbfgragh.onion/2020/03/16/business/stock-market-drops-recap.html\#link-7c85d039}{largest
single-day drop} since Black Monday, in 1987, but
\href{https://www.nytimes3xbfgragh.onion/2020/03/20/business/coronavirus-trump-stock-market.html}{all
of the gains from the past few years have essentially been wiped out}.

But for long-term investors --- which is what most of us should be ---
the age-old advice still holds: Do nothing and just wait it out.

``The only two days that really matter in investing are the day you buy
and the day you sell. All the ups and downs in between are simply
noise,'' Mel Lindauer, co-author of ``The Bogleheads' Guide to
Investing,'' said in
\href{https://www.nytimes3xbfgragh.onion/2020/03/15/smarter-living/corona-stock-market-tips-dealing-with-financial-crash-crisis.html}{this
guide on how to keep calm during a market crash.}

For all of your other money questions, our Your Money team has put
together two handy guides:
\href{https://www.nytimes3xbfgragh.onion/article/coronavirus-money-advice.html}{This
personal finance Q\&A} covers topics including whether you should
rebalance your portfolio, when to buy more stocks, whether you should
refinance your mortgage and much more; and
\href{https://www.nytimes3xbfgragh.onion/article/coronavirus-stimulus-package-questions-answers.html}{this
Q\&A covers the stimulus package.}

\hypertarget{stay-informed}{%
\subsection{Stay informed}\label{stay-informed}}

\hypertarget{knowing-what-is-accurate-can-protect-you-and-your-family}{%
\subsubsection{Knowing what is accurate can protect you and your
family.}\label{knowing-what-is-accurate-can-protect-you-and-your-family}}

There's a lot of information flying around, and knowing what is going on
will go a long way toward protecting your family.

\emph{{[}Stay Informed:}
\href{https://www.nytimes3xbfgragh.onion/news-event/coronavirus?action=click\&module=Spotlight\&pgtype=Homepage}{\emph{The
New York Times is providing free coverage of the crisis}}\emph{.{]}}

Johns Hopkins has a \href{https://coronavirus.jhu.edu/}{comprehensive
web guide}, as does
\href{https://www.health.harvard.edu/diseases-and-conditions/coronavirus-resource-center}{Harvard
Medical School}. The C.D.C. has up-to-date information, and your local
health department is a great resource for questions.

\hypertarget{call-your-doctor-if-you-are-feeling-sick}{%
\subsection{Call your doctor if you are feeling
sick}\label{call-your-doctor-if-you-are-feeling-sick}}

If you develop a high fever, shortness of breath or another, more
serious symptom, call your doctor.

There's a good chance you won't be tested: Testing for coronavirus is
\href{https://www.nytimes3xbfgragh.onion/2020/03/02/health/coronavirus-testing-cdc.html}{still
inconsistent} ---
\href{https://www.nytimes3xbfgragh.onion/2020/03/06/health/testing-coronavirus.html}{there
are not enough kits}, and it's dangerous to go into a doctor's office
and risk infecting others. Also, check the
\href{https://www.cdc.gov/coronavirus/2019-ncov/if-you-are-sick/steps-when-sick.html?CDC_AA_refVal=https\%3A\%2F\%2Fwww.cdc.gov\%2Fcoronavirus\%2F2019-ncov\%2Fabout\%2Fsteps-when-sick.html}{Centers
for Disease Control and Prevention website} and your local health
department for advice about how and where to be tested.

\emph{Abby Goodnough, Apoorva Mandavilli, Margot Sanger-Katz} \emph{and
Knvul Sheikh contributed reporting.}

Advertisement

\protect\hyperlink{after-bottom}{Continue reading the main story}

\hypertarget{site-index}{%
\subsection{Site Index}\label{site-index}}

\hypertarget{site-information-navigation}{%
\subsection{Site Information
Navigation}\label{site-information-navigation}}

\begin{itemize}
\tightlist
\item
  \href{https://help.nytimes3xbfgragh.onion/hc/en-us/articles/115014792127-Copyright-notice}{©~2020~The
  New York Times Company}
\end{itemize}

\begin{itemize}
\tightlist
\item
  \href{https://www.nytco.com/}{NYTCo}
\item
  \href{https://help.nytimes3xbfgragh.onion/hc/en-us/articles/115015385887-Contact-Us}{Contact
  Us}
\item
  \href{https://www.nytco.com/careers/}{Work with us}
\item
  \href{https://nytmediakit.com/}{Advertise}
\item
  \href{http://www.tbrandstudio.com/}{T Brand Studio}
\item
  \href{https://www.nytimes3xbfgragh.onion/privacy/cookie-policy\#how-do-i-manage-trackers}{Your
  Ad Choices}
\item
  \href{https://www.nytimes3xbfgragh.onion/privacy}{Privacy}
\item
  \href{https://help.nytimes3xbfgragh.onion/hc/en-us/articles/115014893428-Terms-of-service}{Terms
  of Service}
\item
  \href{https://help.nytimes3xbfgragh.onion/hc/en-us/articles/115014893968-Terms-of-sale}{Terms
  of Sale}
\item
  \href{https://spiderbites.nytimes3xbfgragh.onion}{Site Map}
\item
  \href{https://help.nytimes3xbfgragh.onion/hc/en-us}{Help}
\item
  \href{https://www.nytimes3xbfgragh.onion/subscription?campaignId=37WXW}{Subscriptions}
\end{itemize}
