Sections

SEARCH

\protect\hyperlink{site-content}{Skip to
content}\protect\hyperlink{site-index}{Skip to site index}

\href{https://www.nytimes3xbfgragh.onion/section/business}{Business}

\href{https://myaccount.nytimes3xbfgragh.onion/auth/login?response_type=cookie\&client_id=vi}{}

\href{https://www.nytimes3xbfgragh.onion/section/todayspaper}{Today's
Paper}

\href{/section/business}{Business}\textbar{}F.A.Q. on Coronavirus Relief
for Small Businesses, Freelancers and More

\url{https://nyti.ms/2xbeeJ4}

\begin{itemize}
\item
\item
\item
\item
\item
\item
\end{itemize}

\href{https://www.nytimes3xbfgragh.onion/news-event/coronavirus?action=click\&pgtype=Article\&state=default\&region=TOP_BANNER\&context=storylines_menu}{The
Coronavirus Outbreak}

\begin{itemize}
\tightlist
\item
  live\href{https://www.nytimes3xbfgragh.onion/2020/08/04/world/coronavirus-cases.html?action=click\&pgtype=Article\&state=default\&region=TOP_BANNER\&context=storylines_menu}{Latest
  Updates}
\item
  \href{https://www.nytimes3xbfgragh.onion/interactive/2020/us/coronavirus-us-cases.html?action=click\&pgtype=Article\&state=default\&region=TOP_BANNER\&context=storylines_menu}{Maps
  and Cases}
\item
  \href{https://www.nytimes3xbfgragh.onion/interactive/2020/science/coronavirus-vaccine-tracker.html?action=click\&pgtype=Article\&state=default\&region=TOP_BANNER\&context=storylines_menu}{Vaccine
  Tracker}
\item
  \href{https://www.nytimes3xbfgragh.onion/2020/08/02/us/covid-college-reopening.html?action=click\&pgtype=Article\&state=default\&region=TOP_BANNER\&context=storylines_menu}{College
  Reopening}
\item
  \href{https://www.nytimes3xbfgragh.onion/live/2020/08/04/business/stock-market-today-coronavirus?action=click\&pgtype=Article\&state=default\&region=TOP_BANNER\&context=storylines_menu}{Economy}
\end{itemize}

Advertisement

\protect\hyperlink{after-top}{Continue reading the main story}

Supported by

\protect\hyperlink{after-sponsor}{Continue reading the main story}

\hypertarget{faq-on-coronavirus-relief-for-small-businesses-freelancers-and-more}{%
\section{F.A.Q. on Coronavirus Relief for Small Businesses, Freelancers
and
More}\label{faq-on-coronavirus-relief-for-small-businesses-freelancers-and-more}}

Many small companies and nonprofits are eligible for federal grants and
low-interest loans. But red tape abounds.

\includegraphics{https://static01.graylady3jvrrxbe.onion/images/2020/04/07/business/07smallbizfaq/07smallbizfaq-articleLarge.jpg?quality=75\&auto=webp\&disable=upscale}

\href{https://www.nytimes3xbfgragh.onion/by/stacy-cowley}{\includegraphics{https://static01.graylady3jvrrxbe.onion/images/2018/10/03/multimedia/author-stacy-cowley/author-stacy-cowley-thumbLarge.png}}

By \href{https://www.nytimes3xbfgragh.onion/by/stacy-cowley}{Stacy
Cowley}

\begin{itemize}
\item
  June 2, 2020
\item
  \begin{itemize}
  \item
  \item
  \item
  \item
  \item
  \item
  \end{itemize}
\end{itemize}

The
\href{https://www.nytimes3xbfgragh.onion/2020/06/02/business/economy/major-employers-coronavirus-relief.html}{federal
stimulus bills} enacted last month, including
\href{https://www.nytimes3xbfgragh.onion/2020/03/26/us/coronavirus-senate-stimulus-package.html}{a
bipartisan \$2 trillion economic relief plan}, offer help for the
millions of American small businesses affected by the coronavirus
pandemic.

The provisions include cash grants, low-interest loans and payments to
offset eight weeks of payroll costs for businesses that retain workers
or rehire those they have laid off. There are also enhancements to
unemployment insurance and paid leave that affect small businesses.

Here are the answers to common questions about these programs. We'll
update this article as we learn more about the details.

More information on help, including details on the stimulus checks that
many people will be receiving, can be found in our
\href{https://www.nytimes3xbfgragh.onion/article/coronavirus-stimulus-package-questions-answers.html}{F.A.Q.
for individuals about stimulus relief} and our
\href{https://www.nytimes3xbfgragh.onion/article/coronavirus-money-unemployment.html}{Hub
for Help}. If you have questions, or have applied for small business aid
and can tell us how the process went,
\href{mailto:stacy.cowley@NYTimes.com}{we'd love to hear from you}.

\hypertarget{the-basics}{%
\subsection{The Basics}\label{the-basics}}

\textbf{Who is eligible for relief?}

Businesses and nonprofit organizations with fewer than 500 workers are
eligible for aid, including sole proprietorships, independent
contractors and freelancers. Larger companies
\href{https://www.sba.gov/document/support--table-size-standards}{in
some industries} are also eligible.

\textbf{What help is being offered?}

There are two main federal aid programs, which are being managed by the
Small Business Administration. Business owners can get help from both at
the same time, but there are some restrictions.

The paycheck protection program is a forgivable loan intended to pay for
eight weeks of a business's payroll costs, so the company can retain
workers or hire back those it has already laid off.

The government has also expanded the existing economic injury
\href{https://www.nytimes3xbfgragh.onion/2020/08/03/business/small-business-loans-coronavirus.html}{disaster
loan program}, which offers low-interest loans to cover most business
expenses. A portion of those loans do not have to be paid back.

\textbf{Will the money for these programs run out?}

Probably yes, and possibly quickly. Both programs have limited funding
and are first come first served. The Trump administration has
\href{https://www.nytimes3xbfgragh.onion/2020/04/09/us/politics/congress-coronavirus-small-businesses.html}{asked
Congress for at least \$250 billion more} for the paycheck program.

See below for the specifics of each program.

\hypertarget{paycheck-protection-program}{%
\subsection{Paycheck Protection
Program}\label{paycheck-protection-program}}

\textbf{What does this do?}

The program offers loans of up to \$10 million to cover eight weeks of
payroll plus some additional expenses, like rent and utilities.

The loan can effectively turn into a grant. Most, and in some cases all,
of the loan will be forgiven if a company uses the money to retain
workers or hire back positions it had to cut. The S.B.A. has waived many
of its usual requirements for these loans and will not require
collateral for them.

\textbf{How does the loan forgiveness work?}

Both full and partial forgiveness is available.

Businesses can have their loans forgiven in full if they maintain their
full-time equivalent head count (based on a 40-hour workweek) and wages
for eight weeks after the loan is disbursed, the Treasury Department
said. The agency said that ``not more than 25 percent'' of the forgiven
amount may be used for nonpayroll costs, like rent.

Forgiveness will be ``reduced'' for companies that trim their head count
or cut workers' wages by more than 25 percent, the Treasury said. Also,
businesses can borrow and have forgiven only the first \$100,000 in
payroll for each employee.

If a business uses money from the economic injury disaster loan program
to cover payroll, it can try to refinance it through the paycheck
protection program and be eligible for forgiveness.

\hypertarget{latest-updates-economy}{%
\section{\texorpdfstring{\href{https://www.nytimes3xbfgragh.onion/live/2020/08/04/business/stock-market-today-coronavirus?action=click\&pgtype=Article\&state=default\&region=MAIN_CONTENT_1\&context=storylines_live_updates}{Latest
Updates:
Economy}}{Latest Updates: Economy}}\label{latest-updates-economy}}

\href{https://www.nytimes3xbfgragh.onion/live/2020/08/04/business/stock-market-today-coronavirus?action=click\&pgtype=Article\&state=default\&region=MAIN_CONTENT_1\&context=storylines_live_updates\#fox-corporations-plunging-profit-is-cushioned-by-fox-news}{15m
ago}

\href{https://www.nytimes3xbfgragh.onion/live/2020/08/04/business/stock-market-today-coronavirus?action=click\&pgtype=Article\&state=default\&region=MAIN_CONTENT_1\&context=storylines_live_updates\#fox-corporations-plunging-profit-is-cushioned-by-fox-news}{Fox
Corporation's plunging profit is cushioned by Fox News.}

\href{https://www.nytimes3xbfgragh.onion/live/2020/08/04/business/stock-market-today-coronavirus?action=click\&pgtype=Article\&state=default\&region=MAIN_CONTENT_1\&context=storylines_live_updates\#trading-in-kodak-shares-comes-under-scrutiny}{39m
ago}

\href{https://www.nytimes3xbfgragh.onion/live/2020/08/04/business/stock-market-today-coronavirus?action=click\&pgtype=Article\&state=default\&region=MAIN_CONTENT_1\&context=storylines_live_updates\#trading-in-kodak-shares-comes-under-scrutiny}{Trading
in Kodak shares comes under scrutiny.}

\href{https://www.nytimes3xbfgragh.onion/live/2020/08/04/business/stock-market-today-coronavirus?action=click\&pgtype=Article\&state=default\&region=MAIN_CONTENT_1\&context=storylines_live_updates\#disney-lost-4-7-billion-last-quarter-but-its-newest-business-was-a-big-hit}{1h
ago}

\href{https://www.nytimes3xbfgragh.onion/live/2020/08/04/business/stock-market-today-coronavirus?action=click\&pgtype=Article\&state=default\&region=MAIN_CONTENT_1\&context=storylines_live_updates\#disney-lost-4-7-billion-last-quarter-but-its-newest-business-was-a-big-hit}{Disney
lost \$4.7 billion last quarter, but its newest business was a big hit.}

\href{https://www.nytimes3xbfgragh.onion/live/2020/08/04/business/stock-market-today-coronavirus?action=click\&pgtype=Article\&state=default\&region=MAIN_CONTENT_1\&context=storylines_live_updates}{See
more updates}

More live coverage:
\href{https://www.nytimes3xbfgragh.onion/2020/08/04/world/coronavirus-cases.html?action=click\&pgtype=Article\&state=default\&region=MAIN_CONTENT_1\&context=storylines_live_updates}{Global}

\textbf{How much can I borrow?}

Companies can borrow up to two months of their average monthly payroll
costs for the past year, plus an additional 25 percent, up to \$10
million. ``Payroll costs'' include salary, wages, tips, commissions,
paid leave benefits, employer-paid health insurance premiums, and state
and local payroll taxes.

\textbf{What if I already laid off my workers?}

You can still have your loan forgiven if you hire them back before the
loan money hits your bank account.

The program generally uses Feb. 15, 2020, for calculating your
pre-pandemic payroll. (Seasonal businesses can use a different date.) As
long as workers laid off after that date are brought back (or new
workers are hired), the layoffs don't affect a borrower's eligibility
for full forgiveness.

Business that need time to bring back workers will have to move fast.
The eight-week clock begins on the date that the loan is disbursed to
the borrower, the
\href{https://home.treasury.gov/system/files/136/Paycheck-Protection-Program-Frequenty-Asked-Questions.pdf}{Treasury
Department said}. Lenders must make that payout no more than 10 calendar
days after the loan is approved, the agency said.

\textbf{I'm self-employed. How do I calculate my payroll cost?}

The
\href{https://www.congress.gov/bill/116th-congress/house-bill/748/text}{CARES
Act text} says that you can claim your ``wage, commission, income, net
earnings from self-employment or similar compensation,'' up to \$100,000
a year. You may need to work with your accountant or lender to confirm
what qualifies.

\textbf{I have a lot of part-time workers and independent contractors.
Can I include them in my payroll calculation?}

You can use the loan to cover payments to your part-time employees,
based on their pre-pandemic average hours and earnings.

But independent contractors and gig workers aren't covered in your
payroll calculation. They are eligible to apply for their own paycheck
protection program loans.

\textbf{How do I apply?}

You must apply through a bank or other lender, so start by contacting
one you already have a relationship with.
\href{https://www.nytimes3xbfgragh.onion/2020/04/03/business/sba-loans-coronavirus.html}{Many
banks are imposing restrictions} and choosing to work only with their
existing business customers.

The Treasury Department has said it wants financial technology companies
and lenders that have not traditionally participated in S.B.A. programs
to make paycheck program loans, but
\href{https://www.nytimes3xbfgragh.onion/2020/04/09/technology/online-lenders-stimulus-virus.html}{there's
significant red tape involved}. More of those lenders might join the
program in the coming weeks. The S.B.A. has
\href{https://www.sba.gov/paycheckprotection/find}{a search tool} to
help you find nearby lenders, but, again, they may not take new
customers.

Some lenders have said they will work with new customers. They include
\href{https://www.kabbage.com/}{Kabbage}, an online lender that teamed
up with a bank;
\href{https://radiusbank.com/business/sba-loans/paycheck-protection-program/}{Radius
Bank} (which is also funding loans through several partners, including
\href{https://www.northone.com/sba-loan-application}{NorthOne}, a
digital bank); and
\href{https://www.fountainheadcc.com/ppp/}{Fountainhead} and
\href{https://ppp.readycapital.com/}{Ready Capital}, two nonbank lenders
already approved to make S.B.A. loans. If you are a lender accepting new
customers, \href{mailto:stacy.cowley@NYTimes.com}{please send us a
note}.

\textbf{Why does my bank say it's not ready to take applications yet?}

There have been a lot of problems getting the program off the ground as
quickly as promised.

The Treasury Department said the program would start taking applications
on April 3, a week after the bill was signed into law. But the
department didn't give lenders necessary technical information until
just hours before the program was scheduled to start --- and lenders are
still waiting for some key guidance and documents, bankers said. Many
are still developing their application rules and systems.

Applications opened on April 10 for business owners who are sole
proprietors, such as freelancers and independent contractors.

\textbf{How much hardship do I have to have to apply?}

There are no specific requirements. You do not need to prove a sharp
drop in sales or a forced business closing, for example.

Applicants simply have to certify that ``current economic uncertainty
makes this loan request necessary'' to support their ongoing operations.

\textbf{What documents do I need?}

Each lender will set its own rules, but this
\href{https://home.treasury.gov/system/files/136/Paycheck-Protection-Program-Application-3-30-2020-v3.pdf}{sample
loan application} covers the basics. You'll need to document your
average monthly payroll for the past 12 months and provide records on
other expenses you're looking to cover, like rent and utilities.

Once it's time to ask for the loan to be forgiven, expect your lender to
ask for more documentation.

\textbf{What if my loan is only partly forgiven?}

You'll have two years to pay off the balance, at a 1 percent interest
rate. No payments are due for the first six months after you get the
loan.

\textbf{What's the application deadline?}

June 30.

That's the date Congress set for disbursement of the \$349 billion it
allocated for the program, but the money is likely to run out faster.
That means some applicants will be turned away unless Congress
authorizes more funding. On Tuesday, the Treasury Department said
\href{https://www.nytimes3xbfgragh.onion/2020/04/07/business/stock-market-today-coronavirus.html\#link-4be214b5}{it
would request at least another \$200 billion}.

\hypertarget{economic-injury-disaster-loans}{%
\subsection{Economic Injury Disaster
Loans}\label{economic-injury-disaster-loans}}

\textbf{What does this program do?}

This is a longstanding program offering low-interest loans of up to \$2
million for businesses that have suffered losses from some kind of
disaster. The loans are made directly by the S.B.A., and
\href{https://covid19relief.sba.gov/\#/}{you can apply on its website}.
You don't have to go through a bank.

Federal legislation in response to the pandemic committed more money and
relaxed some of the S.B.A.'s usual requirements for disaster loans. It
also added a provision essentially offering applicants small grants.

However,
\href{https://www.nytimes3xbfgragh.onion/2020/04/09/business/smallbusiness/small-business-disaster-loans-coronavirus.html}{demand
for the program has overwhelmed the S.B.A.}, and funding appears to be
running out. Many applicants said they have been told that initial
disbursements will be capped at \$15,000 per business. S.B.A.
representatives did not respond to repeated questions about whether caps
have been imposed.

\textbf{Does this program offer loan forgiveness?}

Partly. Businesses can request up to \$10,000 of a disaster loan as a
grant. It's described on the application as a ``loan advance,'' but
S.B.A. officials confirmed that it did not have to be repaid. Borrowers
will be on the hook for the rest.

The CARES Act made the \$10,000 grant available to any qualified
applicants, whether they are approved for a loan or not. But the S.B.A.
appears to be imposing an additional restriction: A spokeswoman said the
amount available to any individual business would be ``based on the
number of predisaster employees.''

The agency did not respond to questions about how the formula will work.
A
\href{https://www.schatz.senate.gov/coronavirus/small-businesses/sba-economic-injury-disaster-loan-and-emergency-grant}{message
posted on the website} of Senator Brian Schatz, Democrat of Hawaii, a
member of the Senate Banking Committee, said the S.B.A. had limited the
advance and was providing \$1,000 per employee for up to 10 employees.

\textbf{What do I need to know about repayment?}

The rest of the loan can be repaid on a term of up to 30 years. The
interest rate is 3.75 percent for small businesses and 2.75 percent for
nonprofits. No payment is due for the first year.

\textbf{Is there a catch?}

Yes, but it's not as big as before.

Usually, disaster loans require a ``personal guarantee'' of repayment,
meaning that the S.B.A. can seize your personal assets --- like your
house, if you own one --- if you default.

But in response to the pandemic, the S.B.A. is changing that.

It will not require a personal guarantee on loans of less than
\$200,000. Business assets, like machinery and equipment, can be used to
secure loans of up to \$500,000, an agency spokeswoman said.

Larger loans will require real estate --- whether it's your business's
or your own --- if you have it. If you don't, the agency said, it will
not turn borrowers away because they lack collateral.

\textbf{What documents do I need to apply?}

The \href{https://covid19relief.sba.gov/\#/}{online loan application} is
designed to be simple and can be completed in about 15 minutes. You'll
need to know your business's gross revenues and cost of goods sold for
the 12 months that ended Jan. 31, 2020.

When the agency processes your loan, it may request additional material,
including federal tax returns and a year-to-date profit-and-loss
statement.

\textbf{When will I get the money?}

The CARES Act ordered the S.B.A. to disburse up to \$10,000 --- the
portion that does not have to be repaid --- within three days of
receiving an application from an owner who self-certifies that he or she
is eligible for the aid.

That time frame is not being met. More than 400 applicants who contacted
The New York Times said they had been waiting a week or longer, some for
nearly a month.

In normal times, it typically takes around two weeks for the S.B.A. to
make a decision on a disaster loan application, and up to a week after
that for the loan check to be disbursed. It is currently running weeks
behind that schedule.

\textbf{How much can I borrow?}

For these loans, you don't request an amount. The S.B.A. will determine
how much you can borrow using a formula intended to approximate six
months of your operating expenses.

To calculate that, the agency will generally subtract the cost of goods
sold from revenue and loan you up to half of that sum.

But because of high demand, the S.B.A. has told many borrowers that, at
least initially,
\href{https://int.graylady3jvrrxbe.onion/data/documenthelper/6871-sba-note-about-15000-cap/optimized/full.pdf}{it
will lend them only up to two months of working capital}, capped at
\$15,000.

\textbf{What can I use the money for?}

The loans cannot be used to refinance previous loans; anything else is
fair game. If you use money from a disaster loan to pay your employees,
you can try to refinance through the paycheck protection program, which
allows for the loan to be forgiven.

\textbf{What's the application deadline?}

Portions of the program, like the \$10,000 grants, end on Dec. 30, 2020.

\hypertarget{other-questions}{%
\subsection{Other Questions}\label{other-questions}}

\textbf{I'm a freelancer or sole proprietor. Am I eligible for
unemployment if I'm no longer working?}

Yes. Self-employed people are newly eligible for benefits, including the
additional \$600 weekly benefit provided by the federal government as
part of the CARES Act.

Our
\href{https://www.nytimes3xbfgragh.onion/article/self-employed-workers-unemployment-coronavirus-stimulus-package.html}{guide
for self-employed workers} has more details.

\textbf{If my business is still operating, do I need to provide my
workers with paid family leave and paid sick leave?}

It depends on how big your company is.

A new law requires businesses with fewer than 500 workers to give
qualified workers two weeks of paid sick leave if they are ill,
quarantined or seeking diagnosis or preventive care for coronavirus, or
if they are caring for sick family members. It gives 12 weeks of paid
leave to people caring for children whose schools are closed or whose
child care provider is unavailable because of the coronavirus. There is
a daily maximum amount, and the government will reimburse employers.

Companies with fewer than 50 employees
\href{https://www.nytimes3xbfgragh.onion/2020/04/02/us/politics/coronavirus-paid-leave.html}{can
opt out of providing paid leave} for the care of children whose school
or day care is closed, but they must still give their employees paid
sick leave. Companies with more than 500 employees are not obligated to
give either kind of leave.

Employers
\href{https://www.irs.gov/newsroom/covid-19-related-tax-credits-for-required-paid-leave-provided-by-small-and-midsize-businesses-faqs}{can
claim refundable tax credits} to cover the cost of their employees'
leave. Our
\href{https://www.nytimes3xbfgragh.onion/2020/03/19/upshot/coronavirus-paid-leave-guide.html}{F.A.Q.
on the paid leave law has more information on the specifics}.

\textbf{I'm having trouble paying my bills. What can I do?}

If you already have a loan backed by the S.B.A., there's good news: The
agency is
\href{https://www.sba.gov/funding-programs/loans/coronavirus-relief-options/sba-debt-relief}{paying
in full for six months} the monthly payments (principal and interest) on
existing loans and any new ones made before Sept. 27, 2020. The relief
applies to borrowers with 7(a), 504 and microloans. Congress estimates
that will help 320,000 borrowers.

Borrowers with existing disaster loans are eligible for a payment
deferral through the end of the year, but interest will continue to
accrue.

Many banks, landlords and other creditors have said that they will work
with borrowers affected by the pandemic, but it is generally up to them
what aid, if any, they are willing to offer.

\textbf{I'm keeping some or all of my workers, and I don't want a loan.
Is anything else available?}

The
\href{https://www.irs.gov/newsroom/irs-employee-retention-credit-available-for-many-businesses-financially-impacted-by-covid-19}{employee
retention credit} is a refundable tax credit for employers of all sizes
who hang on to their workers through the pandemic. It offers a 50
percent credit on up to \$10,000 in wages per employee, meaning the
credit will be for up to \$5,000 per worker. It's available on wages
paid between March 12, 2020, and Jan. 1, 2021, and companies can
\href{https://www.irs.gov/forms-pubs/about-form-7200}{file with the
I.R.S. for an advance payment} on the credit.

Companies are eligible if the government ordered them to fully or
partially close, or if they have a drop of at least 50 percent in their
gross receipts. Eligible businesses with fewer than 100 workers can
claim the credit, even if their employees kept working through the
crisis.

However, the credit is \emph{not} available to companies that take out a
paycheck protection program loan. (You can still claim the credit if you
take out an economic injury disaster loan.)

Tara Siegel Bernard and Claire Cain Miller contributed reporting.

Advertisement

\protect\hyperlink{after-bottom}{Continue reading the main story}

\hypertarget{site-index}{%
\subsection{Site Index}\label{site-index}}

\hypertarget{site-information-navigation}{%
\subsection{Site Information
Navigation}\label{site-information-navigation}}

\begin{itemize}
\tightlist
\item
  \href{https://help.nytimes3xbfgragh.onion/hc/en-us/articles/115014792127-Copyright-notice}{©~2020~The
  New York Times Company}
\end{itemize}

\begin{itemize}
\tightlist
\item
  \href{https://www.nytco.com/}{NYTCo}
\item
  \href{https://help.nytimes3xbfgragh.onion/hc/en-us/articles/115015385887-Contact-Us}{Contact
  Us}
\item
  \href{https://www.nytco.com/careers/}{Work with us}
\item
  \href{https://nytmediakit.com/}{Advertise}
\item
  \href{http://www.tbrandstudio.com/}{T Brand Studio}
\item
  \href{https://www.nytimes3xbfgragh.onion/privacy/cookie-policy\#how-do-i-manage-trackers}{Your
  Ad Choices}
\item
  \href{https://www.nytimes3xbfgragh.onion/privacy}{Privacy}
\item
  \href{https://help.nytimes3xbfgragh.onion/hc/en-us/articles/115014893428-Terms-of-service}{Terms
  of Service}
\item
  \href{https://help.nytimes3xbfgragh.onion/hc/en-us/articles/115014893968-Terms-of-sale}{Terms
  of Sale}
\item
  \href{https://spiderbites.nytimes3xbfgragh.onion}{Site Map}
\item
  \href{https://help.nytimes3xbfgragh.onion/hc/en-us}{Help}
\item
  \href{https://www.nytimes3xbfgragh.onion/subscription?campaignId=37WXW}{Subscriptions}
\end{itemize}
