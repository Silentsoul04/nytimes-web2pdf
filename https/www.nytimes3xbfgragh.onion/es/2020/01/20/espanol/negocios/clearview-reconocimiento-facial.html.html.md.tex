\href{/es/section/negocios}{Negocios}\textbar{}La compañía misteriosa
que podría acabar con la privacidad que conocemos

\href{https://nyti.ms/2v6ckrH}{https://nyti.ms/2v6ckrH}

\begin{itemize}
\item
\item
\item
\item
\item
\end{itemize}

\includegraphics{https://static01.graylady3jvrrxbe.onion/images/2020/01/19/business/19clearview-promo-sub/19clearview-promo-sub-square640.png}

Credit...Adam Ferriss

Sections

\protect\hyperlink{site-content}{Skip to
content}\protect\hyperlink{site-index}{Skip to site index}

\hypertarget{la-compauxf1uxeda-misteriosa-que-podruxeda-acabar-con-la-privacidad-que-conocemos}{%
\section{La compañía misteriosa que podría acabar con la privacidad que
conocemos}\label{la-compauxf1uxeda-misteriosa-que-podruxeda-acabar-con-la-privacidad-que-conocemos}}

Una empresa emergente creó una herramienta que ha ayudado a policías y
otras autoridades a identificar los rostros de personas desconocidas
gracias a sus retratos en redes sociales.

Credit...Adam Ferriss

Supported by

\protect\hyperlink{after-sponsor}{Continue reading the main story}

Por \href{https://www.nytimes3xbfgragh.onion/by/kashmir-hill}{Kashmir
Hill}

\begin{itemize}
\item
  20 de enero de 2020
\item
  \begin{itemize}
  \item
  \item
  \item
  \item
  \item
  \end{itemize}
\end{itemize}

\href{https://www.nytimes3xbfgragh.onion/2020/01/18/technology/clearview-privacy-facial-recognition.html}{Read
in English}

Hasta hace poco, el éxito más grande de Hoan Ton-That era una aplicación
que permite que los usuarios le pongan el distintivo cabello amarillo de
Donald Trump a sus fotografías.

Después, Ton-That ---un empresario australiano del rubro tecnológico---
hizo algo trascendental: inventó una herramienta que podría acabar con
los días en los que caminabas por la calle de manera anónima. Y luego le
proporcionó esa creación a cientos de agencias que vigilan el orden
público, desde policías locales en Florida hasta el FBI y el
Departamento de Seguridad Nacional de Estados Unidos.

Clearview AI, su pequeña compañía, ideó una aplicación de reconocimiento
facial innovadora. Tomas la fotografía de una persona, la cargas y
puedes ver fotografías públicas de esa persona con enlaces que te llevan
a los sitios donde aparecieron esas fotos. El sistema ---que depende de
una base de datos de más de 3000 millones de imágenes que, según
Clearview, fueron extraídas de Facebook, YouTube, Venmo y millones de
sitios web más--- va mucho más allá de cualquier otra cosa que hayan
creado el gobierno estadounidense o los gigantes de Silicon Valley.

Algunos agentes de la policía federal y estatal de Estados Unidos
dijeron que, aunque solo tenían información limitada sobre la manera en
que funciona Clearview y de quienes la crearon, habían usado la
aplicación para ayudar a resolver casos de hurto en tiendas, robo de
identidad, fraude con tarjetas de crédito, asesinato y explotación
sexual infantil.

Hasta ahora, las tecnologías que identifican de inmediato a las personas
a partir de su rostro han sido tabú a consecuencia de la erosión radical
que provocan en la privacidad.

Sin embargo, al margen del escrutinio público, más de 600 agencias del
orden público han comenzado a usar Clearview el año pasado, de acuerdo
con la compañía, que rechazó proporcionar una lista. El código
informático en el que se basa su aplicación, el cual fue analizado por
The New York Times, incluye lenguaje de programación que permite
combinarlo con visores de realidad aumentada; los usuarios posiblemente
serían capaces de identificar a todas las personas que vieran. La
herramienta podría identificar a activistas en una protesta o a una
persona atractiva que vemos en el metro, revelaría no solo sus nombres,
sino también en dónde viven, qué hacen y con qué personas se relacionan.

Clearview también ha autorizado el uso de la aplicación a por lo menos
un puñado de empresas por motivos de seguridad.

``Las posibilidades de convertirla en un arma son infinitas'', dijo Eric
Goldman, codirector del High Tech Law Institute en la Universidad de
Santa Clara. ``Imaginemos a un agente de policía deshonesto que quiere
acosar a posibles parejas románticas o a un gobierno extranjero que la
usa para revelar secretos sobre personas que después podrá chantajear o
enviar a la cárcel''.

Clearview se ha rodeado de hermetismo para evitar el debate sobre su
tecnología, que difumina los límites establecidos de privacidad. Cuando
comencé a investigar a la compañía en noviembre, su sitio web era una
página austera que refería a una dirección falsa en Manhattan como su
oficina. El único empleado de la compañía que puede consultarse en
LinkedIn, un gerente de ventas llamado ``John Good'', resultó ser el
propio Ton-That, pero con un nombre falso. Durante un mes, las personas
afiliadas a la empresa no respondieron mis correos electrónicos ni mis
llamadas telefónicas.

Aunque la empresa me estaba evadiendo, también me estaba monitoreando.
Después de que así lo solicité, algunos policías cargaron mi fotografía
en la aplicación de Clearview. Poco después, recibieron llamadas de
representantes de la compañía que les preguntaron si estaban hablando
con los medios, una señal de que Clearview tiene la capacidad y, en este
caso, el apetito de monitorear a las personas que la policía está
buscando.

La tecnología de reconocimiento facial siempre ha sido controvertida. La
aplicación de Clearview conlleva riesgos adicionales porque las agencias
del orden público están cargando fotografías confidenciales a los
servidores de una empresa cuya capacidad de proteger sus datos no ha
sido puesta a prueba por organismos expertos independientes.

La compañía terminó por responder mis preguntas y señaló que su silencio
previo era típico en una empresa emergente que comienza bajo el radar.
Ton-That reconoció haber diseñado un prototipo para su uso con visores
de realidad aumentada, pero dijo que la empresa no tenía planeado
lanzarlo. Además, dijo que mi fotografía había sonado la alarma porque
la aplicación ``identifica posibles comportamientos anómalos de
búsqueda'' con el fin de evitar que los usuarios lleven a cabo lo que
describió como ``búsquedas inapropiadas''.

Además de Ton-That, Clearview fue fundada por Richard Schwartz ---quien
fue asesor de Rudolph W. Giuliani cuando fue alcalde de Nueva York--- y
respaldada financieramente por Peter Thiel, el inversor de capital de
riesgo detrás de Facebook y Palantir.

Otro primer inversionista es una pequeña firma llamada Kirenaga
Partners. David Slazo, su fundador, rechazó las preocupaciones acerca de
que Clearview está haciendo que se pueda buscar a las personas por su
rostro en internet y dijo que es una herramienta valiosa para resolver
delitos.

``He llegado a la conclusión de que, debido a que la información aumenta
constantemente, jamás habrá privacidad'', comentó Scalzo. ``Las leyes
deben determinar lo que es legal, pero no puedes prohibir la tecnología.
Aunque, claro, podría llevarnos a un futuro distópico o algo así, pero
no puedes prohibirla''.

\includegraphics{https://static01.graylady3jvrrxbe.onion/images/2020/01/19/business/20HillES-2/merlin_166989150_339af8f6-72dc-4344-8bea-138e3b8ba466-articleLarge.jpg?quality=75\&auto=webp\&disable=upscale}

\hypertarget{adictos-a-la-inteligencia-artificial}{%
\subsection{Adictos a la inteligencia
artificial}\label{adictos-a-la-inteligencia-artificial}}

Ton-That, de 31 años, creció muy lejos de Silicon Valley, en Australia,
su país de origen. En 2007, abandonó la universidad y se mudó a San
Francisco. El iPhone acababa de lanzarse y su objetivo era incursionar
de manera temprana en lo que, según esperaba, sería un vibrante mercado
de aplicaciones de redes sociales.

En 2015, creó \href{https://www.148apps.com/app/1040750174/}{Trump
Hair}, una aplicación que agregaba el peinado distintivo del presidente
de Estados Unidos a las personas que aparecían en una fotografía y un
programa de intercambio de imágenes. Ambos fracasaron.

Desanimado, Ton-That se mudó a Nueva York en 2016. Alto y delgado, con
el pelo largo y negro, consideró dedicarse al modelaje, dijo. Pero
después de una sesión de fotografías volvió a su intento de descubrir la
próxima gran novedad en el mundo de la tecnología. Comenzó a leer
artículos académicos sobre inteligencia artificial, reconocimiento de
imágenes y aprendizaje automático.

Schwartz y Ton-That se conocieron en 2016 en un evento de libros en el
Instituto Manhattan, un centro de pensamiento conservador. Schwartz,
ahora de 61 años, había reunido un número impresionante de contactos
cuando trabajó para Giuliani en la década de los noventa. Poco después,
ambos decidieron incursionar juntos en el negocio del reconocimiento
facial: Ton-That construiría la aplicación y Schwartz utilizaría sus
contactos para activar el interés comercial.

Los departamentos de policía han tenido acceso a las herramientas de
reconocimiento facial durante casi veinte años, pero históricamente se
han limitado a buscar imágenes proporcionadas por el gobierno, como
fotografías de prontuario y de licencias de conducir.

Ton-That quería ir más allá de eso. Comenzó en 2016 con el reclutamiento
de un par de ingenieros. Uno ayudó a diseñar un programa que puede
recolectar automáticamente imágenes del rostro de la gente en internet,
a través de sitios de empleo y redes sociales, por ejemplo. Los
representantes de esas compañías dijeron que sus políticas prohíben ese
tipo de recolección de datos y Twitter
\href{https://developer.twitter.com/en/developer-terms/more-on-restricted-use-cases}{anunció}
la prohibición explícita del uso de sus datos en tecnología de
reconocimiento facial.

Contrataron a otro ingeniero para perfeccionar un algoritmo de
reconocimiento facial que se creó a partir de artículos académicos. El
resultado: un sistema que usa lo que Ton-That describe como una ``red
neuronal de vanguardia'' para convertir todas las imágenes en fórmulas
matemáticas ---o vectores--- con base en la geometría facial, con datos
como la distancia que hay entre los ojos de una persona, por ejemplo.
Clearview creó un gran directorio que agrupaba todas las fotografías con
vectores similares en ``vecindarios''. Cuando un usuario carga la
fotografía de un rostro en el sistema de Clearview, este convierte el
rostro en un vector y después muestra todas las fotos extraídas y
almacenadas en el vecindario de ese vector, junto con los enlaces a los
sitios de donde provienen esas imágenes.

Image

Ton-That mostró los resultados de una búsqueda de una fotografía en la
que él sale.Credit...Amr Alfiky para The New York Times

Image

Clearview es una compañía diminuta incluso después de haber recaudado 7
millones de dólares de inversionistas, según
\href{https://pitchbook.com/profiles/company/232177-96}{Pitchbook}, un
sitio web que da seguimiento a las inversiones de las empresas
emergentes. La compañía rechazó confirmar la cantidad.

\hypertarget{un-uxe9xito-viral}{%
\subsection{Un éxito viral}\label{un-uxe9xito-viral}}

En febrero, la policía estatal de Indiana comenzó a experimentar con
Clearview. Resolvieron un caso a los veinte minutos de comenzar a usar
la aplicación. Dos hombres se habían peleado en un parque y el conflicto
terminó cuando uno le disparó al otro en el estómago. Un transeúnte
grabó el delito con su celular, así que la policía tenía una imagen del
rostro del atacante, la cual pudieron cargar en la aplicación de
Clearview.

De inmediato encontraron una coincidencia: el hombre aparecía en un
video que alguien había publicado en redes sociales y su nombre estaba
incluido en la información que acompaña al video. ``No tenía licencia de
conducir ni había sido arrestado de adulto, así que no estaba en las
bases de datos del gobierno'', dijo Chuck Cohen, capitán de la Policía
Estatal de Indiana en ese momento.

El hombre fue arrestado y acusado; Cohen dijo que quizá no lo habrían
identificado sin la capacidad de buscar su rostro en las redes sociales.
La Policía Estatal de Indiana se convirtió en el primer cliente
comercial de Clearview, de acuerdo con la compañía. (La policía rechazó
hacer comentarios más allá de señalar que habían probado la aplicación
de Clearview).

La técnica de ventas más eficaz de la empresa fue ofrecer periodos de
prueba de 30 días a los oficiales. Ton-That por fin tuvo su primer éxito
viral.

Las fuerzas de seguridad federales, incluyendo al FBI y al Departamento
de Seguridad Nacional de Estados Unidos, la están probando, al igual que
autoridades de la policía canadiense, de acuerdo con la compañía y con
funcionarios de gobierno.

\hypertarget{todos-estamos-arruinados}{%
\subsection{``Todos estamos
arruinados''}\label{todos-estamos-arruinados}}

Ton-That dijo que la herramienta no siempre funciona. La mayoría de las
fotografías en la base de datos de Clearview se toman al nivel de los
ojos y gran parte del material que la policía carga está a nivel aéreo,
proviene de cámaras de vigilancia montadas en techos o en lo alto de los
muros.

``Pusieron cámaras de vigilancia demasiado altas'', se lamentó Ton-That.
``El ángulo no es el mejor para un buen reconocimiento facial''.

A pesar de eso, señaló la compañía, su herramienta encuentra
coincidencias hasta en el 75 por ciento de los casos. Pero no está claro
con qué frecuencia la herramienta ofrece coincidencias falsas, porque no
ha sido probada por instancias expertas independientes, como el
Instituto Nacional de Estándares y Tecnología, una agencia federal
estadounidense que
\href{https://www.nist.gov/programs-projects/face-recognition-vendor-test-frvt-ongoing}{califica
el rendimiento} de los algoritmos de reconocimiento facial.

Image

Este gráfico de Clearview afirma que tiene una base de datos de 3000
millones de fotografías y la compara con las que~ tienen a su
disposición el FBI y las policías de Florida y Los
Ángeles.~Credit...Clearview

``No tenemos datos que sugieran que esta aplicación sea precisa'', dijo
Clare Garvie, investigadora del Centro de Privacidad y Tecnología de la
Universidad de Georgetown, quien ha estudiado el uso del reconocimiento
facial por parte del gobierno. ``Cuanto más grande sea la base de datos,
mayor es el riesgo de identificación errónea por el efecto
\emph{doppelgänger}. Están hablando de una base de datos masiva de
personas aleatorias que han encontrado en internet''.

Pero funcionarios policiales retirados y en activo han dicho que la
aplicación es efectiva. ``Para nosotros, la prueba fue si funcionó o
no'', dijo Cohen, el excapitán de la Policía Estatal de Indiana.

Una razón por la que Clearview está teniendo éxito es que su servicio es
único. Eso se debe a que Facebook y otras redes sociales prohíben que
las personas extraigan las imágenes de los usuarios. Al hacerlo,
Clearview está violando los términos de servicio de estas plataformas.

``Mucha gente lo está haciendo'', dijo Ton-That mientras se encogía de
hombros. ``Facebook lo sabe''.

Jay Nancarrow, un portavoz de Facebook, dijo que la compañía estaba
conversando la situación con Clearview y que tomarán ``las medidas
apropiadas si descubrimos que están violando nuestras reglas''.

Algunos agentes de las fuerzas de seguridad dijeron que no se dieron
cuenta de que las fotografías que cargaron se habían enviado y
almacenado en los servidores de Clearview. La compañía trata de
anticiparse a estas preocupaciones con un documento que incluye
\href{https://int.graylady3jvrrxbe.onion/data/documenthelper/6690-clearview-faq/c8b081a0bcca12e7903a/optimized/full.pdf\#page=1}{preguntas
frecuentes}, el cual proporciona a los posibles clientes y en el que
señala que sus empleados de atención a clientes no ven las fotografías
que carga la policía.

Image

Algunos materiales de mercadotecnia de Clearview, obtenidos a través de
una solicitud de registros públicos en Atlanta

Clearview contrató a Paul Clement, fiscal general de Estados Unidos
durante el gobierno de George W. Bush, para disipar preocupaciones sobre
la legalidad de la aplicación.

En un
\href{https://int.graylady3jvrrxbe.onion/data/documenthelper/6689-clearview-legal-memo/c8b081a0bcca12e7903a/optimized/full.pdf\#page=1}{memorando
de agosto} que Clearview proporcionó a sus posibles clientes, incluyendo
al
\href{https://www.muckrock.com/foi/atlanta-325/facial-recognition-atlanta-ga-76491/}{Departamento
de Policía de Atlanta} y la Oficina del Alguacil del Condado de Pinellas
en Florida, Clement dijo que las agencias del orden público ``no violan
la constitución o las leyes relevantes y vigentes de biométrica y
privacidad cuando usan Clearview como se debe''.

Clement, ahora socio de Kirkland \& Ellis, escribió que las autoridades
no tienen que decirles a los acusados que fueron identificados a través
de Clearview siempre y cuando no sea la única razón por la que
obtuvieron una orden para arrestarlos. Clement no respondió a varias
solicitudes para hacer comentarios.

El memorando pareció ser eficaz; la policía de Atlanta y la Oficina del
Alguacil del Condado de Pinellas comenzaron a usar Clearview poco
después.

Dado que la policía sube fotos de personas que intenta identificar,
Clearview posee una base de datos cada vez mayor de personas que han
atraído la atención de las autoridades. La compañía también tiene la
capacidad de manipular los resultados que ve la policía. Después de que
la compañía se dio cuenta de que yo le solicité a oficiales que subieran
mi foto a la aplicación, los sistemas de Clearview no mostraron
coincidencias de mi rostro por un buen tiempo. Cuando Ton-That fue
cuestionado por esto, se echó a reír y lo llamó un ``error de
\emph{software}''.

``Es espeluznante lo que están haciendo. Y creo que habrá muchas más de
estas empresas. No hay monopolio en las matemáticas'', dijo Al Gidari,
profesor de privacidad en la Facultad de Derecho de Stanford. ``En
ausencia de una ley federal de privacidad fuerte, todos estamos
arruinados''.

Woodrow Hartzog, profesor de Derecho e Informática en la Universidad
Northeastern en Boston, considera que Clearview es la prueba más
reciente de que
\href{https://www.nytimes3xbfgragh.onion/2019/10/17/opinion/facial-recognition-ban.html}{debe
prohibirse} el reconocimiento facial en Estados Unidos.

``Hemos dependido de los esfuerzos de la industria para autovigilarse y
no adoptar una tecnología tan riesgosa, pero ahora esos controles se
están desintegrando porque hay mucho dinero sobre la mesa'', dijo
Hartzog. ``No veo un futuro en el que aprovechemos los beneficios de la
tecnología de reconocimiento facial sin el abuso paralizante de la
vigilancia que implica. La única manera de frenarla es prohibirla''.

\hypertarget{el-lugar-en-donde-todos-saben-tu-nombre}{%
\subsection{El lugar en donde todos saben tu
nombre}\label{el-lugar-en-donde-todos-saben-tu-nombre}}

Durante una entrevista reciente en las oficinas de Clearview ---en un
espacio de WeWork en Chelsea en Manhattan---, Ton-That hizo una prueba
de la aplicación con una foto suya. Se tomó una selfi y la subió. La
aplicación sacó 23 fotos de él. En una de esas fotos aparece sin camisa,
con un cigarrillo en la boca y está cubierto en lo que parece sangre.

Ton-That luego tomó mi foto y la subió a la aplicación. El ``error de
\emph{software}'' se había solucionado y ahora mi foto arrojó muchos
resultados, algunos que tienen más de una década, incluidas fotos que no
había visto antes. Después volví a sacarme una foto en la que me tapé la
nariz y la parte inferior de mi cara y aún así la aplicación dio siete
coincidencias correctas.

Los oficiales de policía y los inversionistas de Clearview predicen que
la aplicación eventualmente estará disponible para el público.

Pero Ton-That dijo que estaba renuente a eso. ``Siempre hay una
comunidad de personas malas que querrán usarla de manera incorrecta'',
dijo.

Incluso si Clearview no pone a disposición del público general la
aplicación, alguna otra empresa podría hacerlo ahora que el tabú está
roto. Buscar a una persona por su rostro podría ser tan fácil como
buscar un nombre en Google. Los extraños podrían escuchar conversaciones
delicadas, tomar fotos de quienes participan y conocer secretos
personales. Alguien caminando por la calle sería identificable
inmediatamente y la dirección de su casa estaría a solo unos clics de
distancia. Sería el fin del anonimato público.

Cuando Ton-That fue cuestionado sobre las implicaciones de traer tal
poder al mundo, el empresario pareció desconcertado.

``Tengo que pensar en eso'', dijo. ``Creemos que este es el mejor uso de
la tecnología''.

Jennifer Valentino-DeVries, Gabriel J. X. Dance y Aaron Krolik
colaboraron en este reportaje. Kitty Bennett colaboró con la
investigación.

Kashmir Hill es una periodista de tecnología. Escribe sobre las maneras
inesperadas y a veces ominosas en las que la tecnología está cambiando
nuestras vidas, en especial cuando se trata de nuestra privacidad.
\href{https://twitter.com/kashhill?ref_src=twsrc\%5Egoogle\%7Ctwcamp\%5Eserp\%7Ctwgr\%5Eauthor}{@kashhill}

\begin{center}\rule{0.5\linewidth}{\linethickness}\end{center}

Advertisement

\protect\hyperlink{after-bottom}{Continue reading the main story}

\hypertarget{site-index}{%
\subsection{Site Index}\label{site-index}}

\hypertarget{site-information-navigation}{%
\subsection{Site Information
Navigation}\label{site-information-navigation}}

\begin{itemize}
\tightlist
\item
  \href{https://help.nytimes3xbfgragh.onion/hc/en-us/articles/115014792127-Copyright-notice}{©~2020~The
  New York Times Company}
\end{itemize}

\begin{itemize}
\tightlist
\item
  \href{https://www.nytco.com/}{NYTCo}
\item
  \href{https://help.nytimes3xbfgragh.onion/hc/en-us/articles/115015385887-Contact-Us}{Contact
  Us}
\item
  \href{https://www.nytco.com/careers/}{Work with us}
\item
  \href{https://nytmediakit.com/}{Advertise}
\item
  \href{http://www.tbrandstudio.com/}{T Brand Studio}
\item
  \href{https://www.nytimes3xbfgragh.onion/privacy/cookie-policy\#how-do-i-manage-trackers}{Your
  Ad Choices}
\item
  \href{https://www.nytimes3xbfgragh.onion/privacy}{Privacy}
\item
  \href{https://help.nytimes3xbfgragh.onion/hc/en-us/articles/115014893428-Terms-of-service}{Terms
  of Service}
\item
  \href{https://help.nytimes3xbfgragh.onion/hc/en-us/articles/115014893968-Terms-of-sale}{Terms
  of Sale}
\item
  \href{https://spiderbites.nytimes3xbfgragh.onion}{Site Map}
\item
  \href{https://help.nytimes3xbfgragh.onion/hc/en-us}{Help}
\item
  \href{https://www.nytimes3xbfgragh.onion/subscription?campaignId=37WXW}{Subscriptions}
\end{itemize}
