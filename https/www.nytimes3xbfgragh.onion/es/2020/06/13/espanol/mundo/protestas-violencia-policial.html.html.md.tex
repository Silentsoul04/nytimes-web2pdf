Sections

SEARCH

\protect\hyperlink{site-content}{Skip to
content}\protect\hyperlink{site-index}{Skip to site index}

\href{https://www.nytimes3xbfgragh.onion/es/section/mundo}{Mundo}

\href{https://myaccount.nytimes3xbfgragh.onion/auth/login?response_type=cookie\&client_id=vi}{}

\href{https://www.nytimes3xbfgragh.onion/section/todayspaper}{Today's
Paper}

\href{/es/section/mundo}{Mundo}\textbar{}¿Vigilar o proteger al pueblo?
Lecciones de otros países para la crisis en Estados Unidos

\url{https://nyti.ms/2B6eaM6}

\begin{itemize}
\item
\item
\item
\item
\item
\end{itemize}

\hypertarget{race-and-america}{%
\subsubsection{\texorpdfstring{\href{https://www.nytimes3xbfgragh.onion/news-event/george-floyd-protests-minneapolis-new-york-los-angeles?name=styln-george-floyd\&region=TOP_BANNER\&variant=undefined\&block=storyline_menu_recirc\&action=click\&pgtype=Article\&impression_id=81bbdb70-e3a6-11ea-b968-698657e275f6}{Race
and America}}{Race and America}}\label{race-and-america}}

\begin{itemize}
\tightlist
\item
  \href{https://www.nytimes3xbfgragh.onion/interactive/2020/07/03/us/george-floyd-protests-crowd-size.html?name=styln-george-floyd\&region=TOP_BANNER\&variant=undefined\&block=storyline_menu_recirc\&action=click\&pgtype=Article\&impression_id=81bc50a0-e3a6-11ea-b968-698657e275f6}{Black
  Lives Matter Movement}
\item
  \href{https://www.nytimes3xbfgragh.onion/interactive/2020/06/28/us/i-cant-breathe-police-arrest.html?name=styln-george-floyd\&region=TOP_BANNER\&variant=undefined\&block=storyline_menu_recirc\&action=click\&pgtype=Article\&impression_id=81bc50a1-e3a6-11ea-b968-698657e275f6}{History
  of `I Can't Breathe'}
\item
  \href{https://www.nytimes3xbfgragh.onion/interactive/2020/06/10/upshot/black-lives-matter-attitudes.html?name=styln-george-floyd\&region=TOP_BANNER\&variant=undefined\&block=storyline_menu_recirc\&action=click\&pgtype=Article\&impression_id=81bc77b0-e3a6-11ea-b968-698657e275f6}{How
  Public Opinion Shifted}
\item
  \href{https://www.nytimes3xbfgragh.onion/interactive/2020/07/16/us/black-lives-matter-protests-louisville-breonna-taylor.html?name=styln-george-floyd\&region=TOP_BANNER\&variant=undefined\&block=storyline_menu_recirc\&action=click\&pgtype=Article\&impression_id=81bc77b1-e3a6-11ea-b968-698657e275f6}{45
  Days in Louisville}
\end{itemize}

Advertisement

\protect\hyperlink{after-top}{Continue reading the main story}

Supported by

\protect\hyperlink{after-sponsor}{Continue reading the main story}

The Interpreter

\hypertarget{vigilar-o-proteger-al-pueblo-lecciones-de-otros-pauxedses-para-la-crisis-en-estados-unidos}{%
\section{¿Vigilar o proteger al pueblo? Lecciones de otros países para
la crisis en Estados
Unidos}\label{vigilar-o-proteger-al-pueblo-lecciones-de-otros-pauxedses-para-la-crisis-en-estados-unidos}}

Quienes proponen rehacer el modelo de policía que ha provocado
disturbios y protestas harían bien en mirar experiencias de Asia, África
y Europa.

\includegraphics{https://static01.graylady3jvrrxbe.onion/images/2020/06/10/world/12unrest-policing-ES-01/merlin_173013729_7325fc21-511e-4ab1-ae2a-60c31ba1430a-articleLarge.jpg?quality=75\&auto=webp\&disable=upscale}

\href{https://www.nytimes3xbfgragh.onion/by/amanda-taub}{\includegraphics{https://static01.graylady3jvrrxbe.onion/images/2019/07/16/reader-center/author-amanda-taub/author-amanda-taub-thumbLarge.png}}

Por \href{https://www.nytimes3xbfgragh.onion/by/amanda-taub}{Amanda
Taub}

\begin{itemize}
\item
  13 de junio de 2020
\item
  \begin{itemize}
  \item
  \item
  \item
  \item
  \item
  \end{itemize}
\end{itemize}

\href{https://www.nytimes3xbfgragh.onion/2020/06/11/world/police-brutality-protests.html}{Read
in English}

\href{https://www.nytimes3xbfgragh.onion/newsletters/el-times}{Regístrate
para recibir nuestro boletín} con lo mejor de The New York Times.

\begin{center}\rule{0.5\linewidth}{\linethickness}\end{center}

\emph{``Sin justicia no hay paz. ¡No a la policía racista!''}

Esa exigencia, coreada, ha retumbado en las calles de todo Estados
Unidos durante semanas de manifestaciones que han dejado algo bien
claro: la policía estadounidense enfrenta una crisis de legitimidad. Y
las consecuencias van mucho más allá de la misma labor policial.

Quienes están decididos a reconstruir las fuerzas del orden público para
reparar décadas de injusticia racial, harían bien en voltear a mirar las
experiencias de otros países que han luchado con ese mismo desafío. Lo
mismo aplica para quienes insisten en que no hay un problema que
solucionar.

De muchas maneras, este momento, arraigado en siglos de supremacía
blanca y alimentado por la salvaje polarización política de los últimos
años, no podría ser más estadounidense. Sin embargo, existen otros
precedentes, y casi todos son de países donde las prácticas sistemáticas
de brutalidad policial han sido utilizadas para mantener a una minoría
privilegiada en el poder.

La conclusión que ha quedado en evidencia en las calles del país es que
el patrullaje en Estados Unidos está coartando los derechos de muchos de
sus ciudadanos, y está convirtiendo en mentira la promesa constitucional
de la igualdad de protección ante la ley.

La intensidad de la violencia policial en Estados Unidos, el hecho de
que recae desproporcionadamente sobre la población negra y otras
minorías fuertemente vigiladas, y la continua impunidad de los agentes
policiales que cometen las transgresiones, la cual ha institucionalizado
el abuso, son un difícil desafío para la democracia estadounidense.

``Para estas comunidades, la policía es la manera en la que interpretan
la democracia estadounidense ``, afirmó Vesla Weaver, politóloga de la
Universidad Johns Hopkins, quien estudia la labor policial y la
legitimidad democrática en Estados Unidos. ``Establecen vínculos desde
sus experiencias con la policía hacia cuán robusta es su ciudadanía
democrática''.

\includegraphics{https://static01.graylady3jvrrxbe.onion/images/2020/06/10/world/12unrest-policing-ES-02/merlin_173383824_ae30179f-1671-433d-bcc5-27164bfcbf5d-articleLarge.jpg?quality=75\&auto=webp\&disable=upscale}

Y aunque las comunidades pobres son las más afectadas por la violencia
policial, los afroestadounidenses de todas las clases sociales siguen
viviendo bajo su sombra.

``Aquí en mi casa en Cambridge, sin policías dentro de mi hogar, sigo
pensando en eso'', afirmó Megan Ming Francis, profesora adjunta de la
Universidad de Washington y profesora asociada visitante en Harvard
actualmente.

Las crisis de legitimidad policial no son exclusivas de Estados Unidos.
También han ocurrido en lugares como Irlanda del Norte, Sudáfrica, Sri
Lanka y Birmania, entre otros. Y si bien algunas de sus experiencias
ofrecerían una orientación sobre cómo Estados Unidos pudiera empezar a
reparar los problemas subyacentes a los actuales disturbios, también
suponen una dura advertencia sobre la magnitud del problema que enfrenta
Estados Unidos.

``La policía puede perder legitimidad muy rápidamente'', afirmó
Christopher Rickard, investigador sobre labores policiales y política en
Irlanda del Norte. ``Es muy difícil recuperarla''.

\hypertarget{semejante-a-un-enclave-autoritario}{%
\subsubsection{`Semejante a un enclave
autoritario'}\label{semejante-a-un-enclave-autoritario}}

Hasta hace poco, el debate dominante había tendido a tratar los
asesinatos policiales como incidentes aislados de errores cometidos por
oficiales individuales o al mal comportamiento de ``unas cuantas
manzanas podridas'', en lugar de consecuencias predecibles de problemas
sistémicos (aquellos que prefieren esa frase de las manzanas parecen
haber olvidado que el resto del adagio es ``arruinan toda la cesta'').

En todo caso, la distinción podría ser falsa.

En sociedades divididas, la incapacidad de contener esas llamadas
``manzanas podridas'' dentro la policía y otras fuerzas de seguridad
``no es un problema de capacidad sino una decisión política'', afirmó
Kate Cronin-Furman, profesora de ciencia política en la University
College de Londres, quien ha estudiado abusos en Sri Lanka, Birmania y
la República Democrática del Congo.

``Lo que esto logra es decirle a las minorías marginadas que nunca
estarán a salvo, que no poseen toda la gama de los derechos ciudadanos,
y que su humanidad siempre será cuestionada'', dijo Cronin-Furman.

Image

Agentes de policía de Birmania en el municipio de Maungdaw, en
2019Credit...Adam Dean para The New York Times

Eso ofrece doble protección a la clase o grupo dominante: la violencia
de la policía conserva su posición en la jerarquía social, y al
alentarla a través de un permiso implícito en vez de a través de órdenes
explícitas desde arriba, aquellos en el poder mantienen una negación
plausible sobre su participación en el abuso.

Los estadounidenses no están acostumbrados a escuchar a su país
comparado con los lugares que Cronin-Furman estudia. Pero hay evidencia
sustancial de que a través de la historia estadounidense, la vigilancia
violenta y represiva ha enviado un mensaje similar a los estadounidenses
negros y a otros grupos minoritarios que viven en vecindarios pobres y
muy vigilados, y que continúa haciéndolo hoy.

En las décadas de 1860 y 1870, los estados y ciudades del sur diseñaron
medidas policiales para garantizar que las personas negras recién
liberadas permanecieran económicamente subyugadas y excluidas
políticamente de los derechos ciudadanos, dijo Francis, quien estudia la
violencia impuesta por el estado en contra de los estadounidenses
negros.

``Lo escribieron, así que es muy claro lo que estaban tratando de
hacer'', dijo. ``Esto fue visto como una manera de eliminar esos nuevos
derechos de ciudadanía''.

Muchas de esas leyes escritas finalmente cambiaron, particularmente
después del final de la era de Jim Crow en el sur. Pero la violencia
policial aún envía el mensaje a muchos estadounidenses negros de que no
tienen acceso total a los derechos y protecciones de la ciudadanía.

Image

Una manifestación por los derechos civiles en Birmingham, Alabama, en
1963Credit...Bill Hudson/Associated Press

\href{https://www.portalspolicingproject.com/?campaign_id=30\&emc=edit_int_20200605\&instance_id=19128\&nl=the-interpreter\&regi_id=12757726\&segment_id=30197\&te=1\&user_id=e5a68dc0fda40cd7cd986f21b7d9bdbf}{Portals
Policing Project}, un estudio realizado por un equipo de investigadores
de las universidades Johns Hopkins y Yale, transformó contenedores de
transporte en espacios temporales de reunión, y los ubicó en 12
vecindarios fuertemente vigilados por la policía en seis ciudades
estadounidenses. En cada uno se instalaron equipos de comunicación para
que las personas discutieran sus experiencias con aquellos en otros
``portales'' de contenedores, como si estuvieran compartiendo el mismo
espacio.

Cuando los investigadores analizaron los datos de miles de
conversaciones entre portales durante tres años, consiguieron un retrato
de la vida estadounidense que tenía sorprendentes similitudes con lo que
Cronin-Furman había observado en Asia y África.

``En realidad, los participantes de los portales estaban narrando algo
semejante a un enclave autoritario'', donde las prácticas policiales les
habían arrebatado las protecciones más valiosas de su ciudadanía, afirmó
Weaver, la profesora de Johns Hopkins, quien fue una de las principales
investigadoras del proyecto.

Los investigadores descubrieron que en una conversación tras otra las
personas indicaban lo que sabían debían ser sus derechos, pero luego
expresaban cómo sentían que se los habían arrebatado.

Tenían un derecho oficial a la privacidad, pero la policía podría
detenerlos y registrarlos en cualquier momento. Tenían el derecho
oficial a permanecer en silencio, pero temían que la policía los acosara
o castigara si no respondían a sus preguntas. Tenían un derecho oficial
de reunirse pacíficamente, pero en la práctica no podían caminar a un
parque en un grupo de tres o más personas sin que la policía los
esposara y detuviera bajo sospecha de actividad criminal.

``Lo que están experimentando es muy similar, en el sentido de los altos
niveles de abandono y negligencia estatal, junto con altos niveles de
vigilancia en el sur de Jim Crow'', dijo. ``Se parece mucho a los
regímenes autoritarios''.

\hypertarget{cuxf3mo-no-ser-policuxeda}{%
\subsubsection{`Cómo no ser policía'}\label{cuxf3mo-no-ser-policuxeda}}

A medida que la población toma conciencia de la magnitud de los
problemas con la policía estadounidense,
\href{https://www.nytimes3xbfgragh.onion/2020/06/08/us/unrest-defund-police.html}{hay
cada vez más llamados} a realizar un cambio radical. La ciudad de
Mineápolis votó la semana pasada para desmantelar por completo su
departamento policial y reemplazarlo con una nueva fuerza y un nuevo
enfoque de la seguridad pública. Organizaciones como Black Lives Matter
han exigido que se le retiren los fondos a los departamentos policiales
y que la mayoría de sus funciones sean reasignadas. Nueva York y Los
Ángeles ya se han comprometido a reducir millones de los presupuestos
policiales.

La experiencia de otros países sugiere que, incluso cuando existe
voluntad política para realizar una reforma policial profunda, los
enfoques que realmente plantean un cambio han sido difíciles de
implementar.

Cuando las primeras elecciones democráticas de Sudáfrica lograron
expulsar el régimen del apartheid en 1994, una de las promesas del nuevo
gobierno fue reformar la temida policía de la era del apartheid. Cuatro
años después, en Irlanda del Norte, el Acuerdo de Viernes Santo marcó el
final de cuatro décadas de violenta lucha sectaria. También prometió
revisiones y reformas policiales.

Para ganar legitimidad, la policía en Sudáfrica e Irlanda del Norte
necesitaban superar sus reputaciones como protectoras del dominio blanco
y protestante, respectivamente. El éxito o el fracaso de esos esfuerzos
giraron en torno a si la policía iba a continuar con sus viejas
prácticas de tratar a la población como una amenaza a ser controlada a
través de una fuerza abrumadora, o si podrían convertirse en un
organismo más receptivo a la necesidad de protección de los civiles,
ganando con el tiempo la confianza de la sociedad.

En Sudáfrica, ``la policía mantuvo las mismas prácticas de la era del
apartheid de tener poderosos grupos de paramilitares fuertemente
armados'', afirmó Jonny Steinberg, profesor de la Universidad de Oxford
que estudia prácticas policiales y política sudafricana.

Con el tiempo, los sudafricanos de la clase media y alta recurrieron a
la seguridad privada. Contrataron vigilantes, se mudaron a comunidades
cerradas y viajaron solo en sus propios autos. Los pobres, que no podían
costearse este tipo de medidas, quedaron vulnerables.

``Lo que Estados Unidos puede aprender es negativo'', dijo Steinberg.
``Es un caso paradigmático de cómo no ser policía en una población
urbana pobre''.

Image

Un guardia de seguridad privado en Ciudad del Cabo,
SudáfricaCredit...Roger Sedres/Gallo Images, vía Getty Images

Es una lección que muchos departamentos de policía estadounidenses
todavía no han asimilado. De hecho, las tácticas de estilo paramilitar,
por lo general con equipamiento militar obtenido a través de programas
de superávit gubernamentales, han adquirido mayor popularidad en Estados
Unidos en los últimos años.

``La policía estadounidense, en las últimas dos décadas, se ha
desplegado de manera amenazante'', dijo Tom Tyler, profesor de la
Facultad de Derecho de Yale que estudia la legitimidad de la policía y
el gobierno. Interna y externamente, han cultivado una imagen de
guerreros listos para usar la fuerza contra una población peligrosa, en
lugar de la de guardianes de sus comunidades.

El lenguaje de la guerra y la amenaza se exhibieron en una
\href{https://m.facebookcorewwwi.onion/story.php?story_fbid=3016295705125288\&id=111377202283834}{carta}
que Ed Mullins, líder del sindicato de policía más grande de la ciudad
de Nueva York, envió a los oficiales la semana pasada. ``Ganaremos esta
guerra en Nueva York'', escribió. ``Es el bien contra el mal y el bien
siempre gana''.

En Irlanda del Norte, en cambio, la policía sí se alejó de muchas de las
tácticas contrainsurgentes militarizadas que implementaron durante el
conflicto norirlandés. Pero incluso ahora, más de dos décadas después
del acuerdo de paz, a la policía a menudo le cuesta convencer a las
víctimas de delitos a que confíen en el sistema de justicia oficial en
vez de acudir a la justicia ruda que ofrecen los grupos paramilitares
sectarios, afirmó Rickard, el investigador.

Image

Graffiti en Irlanda del Norte el año pasado: ``PSNI/RUC cuiden sus
espaldas''. El Servicio de Policía de Irlanda del Norte (PSNI, por su
sigla en inglés) reemplazó al Royal Ulster Constabular
(RUC).Credit...Andrew Testa para The New York Times

``Ha sido tan difícil en Irlanda del Norte'', dijo. ``No me imagino cómo
se pueden siquiera empezar a resolver algunos de los problemas
estructurales en Estados Unidos''.

Abordar estos problemas profundos será inevitablemente difícil, dijo
Kanisha Bond, politóloga de SUNY Binghamton que estudia el movimiento
Black Lives Matter y otras movilizaciones negras para el cambio social.

``Esta es toda la fuerza de la historia ejercida, transmitida a través
del comportamiento contemporáneo'', dijo. ``Ves a los manifestantes,
protestantes, rebeldes yendo al corazón de lo que ven como una solución
sistémica. Y eso siempre va a ser incómodo''.

Advertisement

\protect\hyperlink{after-bottom}{Continue reading the main story}

\hypertarget{site-index}{%
\subsection{Site Index}\label{site-index}}

\hypertarget{site-information-navigation}{%
\subsection{Site Information
Navigation}\label{site-information-navigation}}

\begin{itemize}
\tightlist
\item
  \href{https://help.nytimes3xbfgragh.onion/hc/en-us/articles/115014792127-Copyright-notice}{©~2020~The
  New York Times Company}
\end{itemize}

\begin{itemize}
\tightlist
\item
  \href{https://www.nytco.com/}{NYTCo}
\item
  \href{https://help.nytimes3xbfgragh.onion/hc/en-us/articles/115015385887-Contact-Us}{Contact
  Us}
\item
  \href{https://www.nytco.com/careers/}{Work with us}
\item
  \href{https://nytmediakit.com/}{Advertise}
\item
  \href{http://www.tbrandstudio.com/}{T Brand Studio}
\item
  \href{https://www.nytimes3xbfgragh.onion/privacy/cookie-policy\#how-do-i-manage-trackers}{Your
  Ad Choices}
\item
  \href{https://www.nytimes3xbfgragh.onion/privacy}{Privacy}
\item
  \href{https://help.nytimes3xbfgragh.onion/hc/en-us/articles/115014893428-Terms-of-service}{Terms
  of Service}
\item
  \href{https://help.nytimes3xbfgragh.onion/hc/en-us/articles/115014893968-Terms-of-sale}{Terms
  of Sale}
\item
  \href{https://spiderbites.nytimes3xbfgragh.onion}{Site Map}
\item
  \href{https://help.nytimes3xbfgragh.onion/hc/en-us}{Help}
\item
  \href{https://www.nytimes3xbfgragh.onion/subscription?campaignId=37WXW}{Subscriptions}
\end{itemize}
