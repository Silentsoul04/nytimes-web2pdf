Sections

SEARCH

\protect\hyperlink{site-content}{Skip to
content}\protect\hyperlink{site-index}{Skip to site index}

\href{https://www.nytimes3xbfgragh.onion/es/section/mundo}{Mundo}

\href{https://myaccount.nytimes3xbfgragh.onion/auth/login?response_type=cookie\&client_id=vi}{}

\href{https://www.nytimes3xbfgragh.onion/section/todayspaper}{Today's
Paper}

\href{/es/section/mundo}{Mundo}\textbar{}Racismo en Japón: una discusión
postergada

\url{https://nyti.ms/3gwhjEv}

\begin{itemize}
\item
\item
\item
\item
\item
\end{itemize}

\hypertarget{race-and-america}{%
\subsubsection{\texorpdfstring{\href{https://www.nytimes3xbfgragh.onion/news-event/george-floyd-protests-minneapolis-new-york-los-angeles?name=styln-george-floyd\&region=TOP_BANNER\&variant=undefined\&block=storyline_menu_recirc\&action=click\&pgtype=Article\&impression_id=d8a7e2b0-e3ad-11ea-81b0-ed0ce085a510}{Race
and America}}{Race and America}}\label{race-and-america}}

\begin{itemize}
\tightlist
\item
  \href{https://www.nytimes3xbfgragh.onion/interactive/2020/07/03/us/george-floyd-protests-crowd-size.html?name=styln-george-floyd\&region=TOP_BANNER\&variant=undefined\&block=storyline_menu_recirc\&action=click\&pgtype=Article\&impression_id=d8a7e2b1-e3ad-11ea-81b0-ed0ce085a510}{Black
  Lives Matter Movement}
\item
  \href{https://www.nytimes3xbfgragh.onion/interactive/2020/06/28/us/i-cant-breathe-police-arrest.html?name=styln-george-floyd\&region=TOP_BANNER\&variant=undefined\&block=storyline_menu_recirc\&action=click\&pgtype=Article\&impression_id=d8a7e2b2-e3ad-11ea-81b0-ed0ce085a510}{History
  of `I Can't Breathe'}
\item
  \href{https://www.nytimes3xbfgragh.onion/interactive/2020/06/10/upshot/black-lives-matter-attitudes.html?name=styln-george-floyd\&region=TOP_BANNER\&variant=undefined\&block=storyline_menu_recirc\&action=click\&pgtype=Article\&impression_id=d8a7e2b3-e3ad-11ea-81b0-ed0ce085a510}{How
  Public Opinion Shifted}
\item
  \href{https://www.nytimes3xbfgragh.onion/interactive/2020/07/16/us/black-lives-matter-protests-louisville-breonna-taylor.html?name=styln-george-floyd\&region=TOP_BANNER\&variant=undefined\&block=storyline_menu_recirc\&action=click\&pgtype=Article\&impression_id=d8a7e2b4-e3ad-11ea-81b0-ed0ce085a510}{45
  Days in Louisville}
\end{itemize}

Advertisement

\protect\hyperlink{after-top}{Continue reading the main story}

Supported by

\protect\hyperlink{after-sponsor}{Continue reading the main story}

\hypertarget{racismo-en-japuxf3n-una-discusiuxf3n-postergada}{%
\section{Racismo en Japón: una discusión
postergada}\label{racismo-en-japuxf3n-una-discusiuxf3n-postergada}}

La creencia de que el racismo institucional es un problema lejano evita
que el país enfrente por completo la discriminación arraigada.

\includegraphics{https://static01.graylady3jvrrxbe.onion/images/2020/06/26/world/06Japon-racismo-ES/merlin_173944887_b41543d8-32d3-4068-84ff-6861fd579756-articleLarge.jpg?quality=75\&auto=webp\&disable=upscale}

Por \href{https://www.nytimes3xbfgragh.onion/by/motoko-rich}{Motoko
Rich} y Hikari Hida

\begin{itemize}
\item
  6 de julio de 2020
\item
  \begin{itemize}
  \item
  \item
  \item
  \item
  \item
  \end{itemize}
\end{itemize}

\href{https://www.nytimes3xbfgragh.onion/2020/07/01/world/asia/japan-racism-black-lives-matter.html}{Read
in English}

\href{https://www.nytimes3xbfgragh.onion/newsletters/el-times}{Regístrate
para recibir nuestro boletín} con lo mejor de The New York Times.

\begin{center}\rule{0.5\linewidth}{\linethickness}\end{center}

TOKIO --- A medida que las protestas se extendían por todo el mundo en
respuesta al asesinato de George Floyd por parte de la policía, Sierra
Todd, una estudiante universitaria afroestadounidense en Japón, organizó
una marcha el mes pasado en Tokio para mostrar solidaridad con los
manifestantes estadounidenses.

Dijo que esperaba que también llevase a los manifestantes japoneses a
pensar en el racismo en su propio país. ``Por supuesto, queremos hablar
sobre asuntos estadounidenses, y Black Lives Matter es algo
estadounidense'', dijo Todd, de 19 años, quien estudia en el campus en
Tokio de la Universidad de Temple. ``Pero también vivimos en Japón''.

Una reacción violenta siguió rápidamente. Los críticos en las redes
sociales acusaron a los participantes de ignorar los riesgos de propagar
el coronavirus. Una
\href{https://www.youtube.com/watch?v=ucdbcHod9iU}{entrevista con Todd
publicada en YouTube} suscitó comentarios de que: ``Eso es un problema
estadounidense'' y ``Por favor, haga esto en su propio país''.

Con imágenes de la lucha racial de Estados Unidos en las pantallas de
televisión, en Japón algunos han insistido en que el racismo
institucional es un problema lejano. Eso, dicen activistas y académicos,
evita que el público aproveche el momento para reconocer la arraigada
discriminación contra los grupos marginados en Japón.

Una facción vocal de conservadores japoneses respalda las nociones
racistas de pureza con base en la sangre. Y la población en gran medida
homogénea a menudo se ha resistido a reconocer la diferencia o a
participar en el tipo de introspección sobre el racismo y la desigualdad
que está
\href{https://www.nytimes3xbfgragh.onion/2020/06/22/us/racism-white-americans.html?action=click\&module=Top\%20Stories\&pgtype=Homepage}{ocurriendo
en Estados Unidos}.

``En esencia, los japoneses no tienen mucha experiencia de ver otras
razas'', dijo Yasumasa Fujinaga, profesor de estudios estadounidenses en
la Universidad de Mujeres de Japón. ``Así que no creen que el racismo
exista''.

\includegraphics{https://static01.graylady3jvrrxbe.onion/images/2020/06/26/world/06Japon-racismo-ES-02/merlin_173944614_e483f86a-8a0e-42d4-b7c6-6353d7b3908a-articleLarge.jpg?quality=75\&auto=webp\&disable=upscale}

Pero Japón tiene una larga historia de discriminación contra las
minorías, incluidos los descendientes de los
\href{https://www.nytimes3xbfgragh.onion/2019/02/25/world/asia/korea-japan-diaspora.html}{coreanos}
traídos a Japón para hacer trabajos forzados antes y durante la Segunda
Guerra Mundial; grupos indígenas como los ainu de la isla más
septentrional, Hokkaido; aquellos cuyo linaje se remonta a una
\href{https://www.nytimes3xbfgragh.onion/2009/01/16/world/asia/16outcasts.html}{clase
feudal de marginados conocidos como buraku}, e
\href{https://www.nytimes3xbfgragh.onion/2015/05/30/world/asia/biracial-beauty-queen-strives-for-change-in-mono-ethnic-japan.html}{individuos}
de raza mixta.

El maltrato a las personas de raza mixta a lo largo de sus años
escolares y más allá ha llamado especialmente la atención a medida que
un número creciente de celebridades birraciales se han pronunciado.

En un emotivo testimonio
\href{https://twitter.com/LOUISOKOYEMoM/status/1272536049013514242}{publicado
en Twitter} el mes pasado, Louis Okoye, un jugador de béisbol de origen
japonés y nigeriano, describió cómo a menudo había sido acosado cuando
era niño en Japón debido al color de su piel.

``Yo miraba desde el balcón de nuestra casa y pensaba que si saltaba y
nacía de nuevo, quizás podía volver como una persona japonesa normal'',
escribió en la publicación, que ha sido retuiteada 52.000 veces. La
mayoría de los comentarios fueron abrumadoramente de apoyo.

Aún así, la conversación está cambiando solo gradualmente en Japón, y la
resistencia puede ser férrea. Cuando Bako Nguasong y V. Athena Lisane,
profesoras de inglés en Fukuoka ---la ciudad más grande de la isla
sureña de Kyushu--- organizaron una marcha ahí el mes pasado, una
persona
\href{https://twitter.com/randomyoko?ref_src=twsrc\%5Egoogle\%7Ctwcamp\%5Eserp\%7Ctwgr\%5Eauthor}{escribió}
en Twitter que ``no sentiría ninguna piedad si se dice que estas
personas sean deportadas por los japoneses locales''.

Nguasong, de 36 años, dejó atrás una casa y una carrera en Washington y
se mudó a Japón hace dos años porque estaba cansada de temer por su
seguridad como mujer negra en Estados Unidos. ``Sabía que no iba a ser
diverso, pero también sabía que no iba a temer por mi vida'', dijo
Nguasong, quien anteriormente trabajó como coordinadora de salud mental
para exreclusos.

Encontró seguridad física en Japón, donde las tasas de criminalidad son
bajas y los asesinatos policiales, raros. Pero no pudo escapar del
racismo, incluso si en Japón toma una forma menos violenta.

Nguasong dice que ha notado miradas y susurros de los japoneses y que
los pasajeros han evitado sentarse junto a ella en los trenes. Dos
mujeres mayores, dijo, una vez manosearon sus senos en un elevador, con
aparente sorpresa por su figura.

``No es la misma naturaleza insidiosa'', dijo Nguasong. ``Pero el
racismo existe en Japón''.

Un recordatorio evidente de eso llegó el mes pasado cuando NHK, la
emisora pública, transmitió un segmento sobre las protestas de Black
Lives Matter en Estados Unidos.

Image

Una manifestación de Black Lives Matter en Osaka, el 7 de
junioCredit...Dai Kurokawa/EPA vía Shutterstock

Un clip mostraba a los afroestadounidenses como caricaturas
excesivamente musculosas, que tocaban música y saqueaban, y presentó las
protestas como el producto de la frustración con la disparidad económica
y el coronavirus, sin mencionar a la brutalidad policial. Después del
rechazo en Twitter, NHK se disculpó y retiró el video.

La insularidad del país ha generado no solo prejuicios abiertos e
inconscientes contra las personas en el extranjero, sino también una
desconfianza hacia los extranjeros que vienen a Japón.

A medida que el país abre lentamente sus puertas a los trabajadores
externos para ayudar a enfrentar una crisis demográfica, mejorar su
trato a los extranjeros puede ser crucial para el futuro de Japón. Pero
según una encuesta de 2017 del Ministerio de Justicia, el 30 por ciento
de los extranjeros en Japón dijeron que habían sido objeto de
discriminación, y muchos mencionaron problemas para conseguir trabajo o
vivienda.

Un candidato a la gubernatura de Tokio, Makoto Sakurai, hace campaña con
una plataforma que incluye el eslogan ``abolir la seguridad social para
extranjeros''.

Al mismo tiempo, sin embargo, algunos japoneses muestran una fascinación
por los extranjeros, incluida la cultura pop negra. Eso ha llevado a
acusaciones de apropiación cultural, y dejó a algunos
afroestadounidenses disgustados porque más personas en Japón no
reflexionen sobre su propio racismo.

``A los japoneses que les gusta la cultura negra les gusta todo lo que
es estereotípicamente negro, como los dientes de oro'', dijo Farah
Albritton, de 28 años, una profesora de inglés en Fukuoka quien es de
Brooklyn. ``O cambiarán su cabello para que tenga una textura afro o
para peinarlo con trenzas africanas''

Albritton ha experimentado incidentes de racismo en Japón, como cuando
un hombre en la calle le gritó ``eres tan asquerosa'', o cuando un
agente de casting para un trabajo de modelo le pidió que demostrara un
``apretón de manos negro'' en una audición.

Image

Naomi Osaka se ha convertido cada vez más en una voz crítica contra el
racismo.Credit...William West/Agence France-Presse --- Getty Images

Dijo que se ofendió porque sus conocidos japoneses que emulan a las
estrellas pop negras han mostrado poco o ningún apoyo público a Black
Lives Matter. ``Estás participando de nuestra cultura, y te hemos
aceptado en nuestra cultura'', dijo Albritton. ``Y ni siquiera puedes
publicar sobre nuestros amigos que están muriendo, la gente que te está
inspirando''.

En comparación con las marchas de Black Lives Matter en
\href{https://www.nytimes3xbfgragh.onion/es/2020/06/20/espanol/mundo/protestas-racismo-francia.html}{Francia}
o
\href{https://www.nytimes3xbfgragh.onion/2020/06/06/world/george-floyd-global-protests.html}{Gran
Bretaña}, que han atraído a decenas de miles de personas, las
manifestaciones en Japón han sido de tamaño modesto. La más grande, en
Tokio, atrajo a unas 3500 personas, y muchos participantes tenían raíces
extranjeras.

Algunos académicos temen que el público japonés solo vea el racismo en
el extranjero sin reflexionar sobre él más cerca de casa.

``Si solo dicen: `Oh, vaya, los negros en Estados Unidos enfrentan cosas
horribles y tenemos que ayudarlos', eso es casi como caridad'', dijo
Haeng-ja Chung, profesora de antropología en la Universidad de Okayama y
quien pertenece al grupo étnico coreano y nació en Japón. ``Antes de
acusar a otras sociedades, debemos parar y pensar: `¿Y nosotros?'''.

La campeona de tenis Naomi Osaka, hija de un padre
haitiano-estadounidense y una madre japonesa cuyo estatus de
superestrella ha inspirado una reevaluación de la identidad tradicional
japonesa, ha
\href{https://www.washingtonpost.com/world/asia_pacific/japanese-tennis-player-naomi-osaka-speaks-out-for-black-lives-matter-faces-backlash/2020/06/08/f8432ca0-a92f-11ea-a43b-be9f6494a87d_story.html}{criticado}
a aquellos en las redes sociales que afirman que no hay racismo en
Japón.

En una publicación en Twitter, recordó a sus seguidores de un incidente
cuando unos comediantes japoneses dijeron que ella necesitaba ``usar
cloro'' porque estaba ``demasiado quemada por el sol''.

Hiromi Okamura, de 57 años, quien asistió a la marcha de Black Lives
Matter en Tokio el mes pasado, dijo que le había ayudado a pensar en
cómo ``las acciones inconscientes son las que a menudo conducen a los
prejuicios''.

``Creo que hay potencialmente un racismo muy arraigado'' en Japón, dijo
Okamura. ``Lo importante es comprenderlo y comunicarlo cuidadosamente''.

Image

Ramazan Celik, un inmigrante turco que fue maltratado por la
policíaCredit...Noriko Hayashi para The New York Times

Algunos japoneses están trabajando para llamar más la atención sobre los
prejuicios contra los extranjeros. Después de que apareciera un video en
las redes sociales que mostraba a unos agentes de policía de Tokio
maltratando a un inmigrante turco, Ramazan Celik, durante un control de
tránsito, Kento Suzuki, voluntario en un centro de detención donde se
encuentran solicitantes de asilo y otros extranjeros, organizó dos
protestas.

``Desde hace mucho tiempo he pensado que Japón es una sociedad
racista'', dijo Suzuki, de 28 años. ``Hay muchas personas que se
involucran cada vez más en el movimiento para ayudar a los inmigrantes
en Japón''.

Aún así, a Suzuki le preocupan los casos en los que los solicitantes de
asilo y los inmigrantes han dicho que fueron abusados o descuidados
durante su detención.

``Siento esperanza y también temor'', dijo. ``Es una batalla constante
entre estas dos emociones''.

Image

Kento Suzuki organizó protestas contra el maltrato a los
inmigrantes.Credit...Noriko Hayashi para The New York Times

Motoko Rich es la jefa de la corresponsalía en Tokio. Ha cubierto un
amplio rango de temas para el Times, incluyendo bienes raíces (durante
un boom), economía (durante una crisis), libros y educación.
\href{https://twitter.com/motokorich}{@motokorich} \textbar{}
\href{https://www.facebookcorewwwi.onion/motoko.rich}{Facebook}

\begin{center}\rule{0.5\linewidth}{\linethickness}\end{center}

Advertisement

\protect\hyperlink{after-bottom}{Continue reading the main story}

\hypertarget{site-index}{%
\subsection{Site Index}\label{site-index}}

\hypertarget{site-information-navigation}{%
\subsection{Site Information
Navigation}\label{site-information-navigation}}

\begin{itemize}
\tightlist
\item
  \href{https://help.nytimes3xbfgragh.onion/hc/en-us/articles/115014792127-Copyright-notice}{©~2020~The
  New York Times Company}
\end{itemize}

\begin{itemize}
\tightlist
\item
  \href{https://www.nytco.com/}{NYTCo}
\item
  \href{https://help.nytimes3xbfgragh.onion/hc/en-us/articles/115015385887-Contact-Us}{Contact
  Us}
\item
  \href{https://www.nytco.com/careers/}{Work with us}
\item
  \href{https://nytmediakit.com/}{Advertise}
\item
  \href{http://www.tbrandstudio.com/}{T Brand Studio}
\item
  \href{https://www.nytimes3xbfgragh.onion/privacy/cookie-policy\#how-do-i-manage-trackers}{Your
  Ad Choices}
\item
  \href{https://www.nytimes3xbfgragh.onion/privacy}{Privacy}
\item
  \href{https://help.nytimes3xbfgragh.onion/hc/en-us/articles/115014893428-Terms-of-service}{Terms
  of Service}
\item
  \href{https://help.nytimes3xbfgragh.onion/hc/en-us/articles/115014893968-Terms-of-sale}{Terms
  of Sale}
\item
  \href{https://spiderbites.nytimes3xbfgragh.onion}{Site Map}
\item
  \href{https://help.nytimes3xbfgragh.onion/hc/en-us}{Help}
\item
  \href{https://www.nytimes3xbfgragh.onion/subscription?campaignId=37WXW}{Subscriptions}
\end{itemize}
