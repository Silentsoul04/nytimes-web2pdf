\href{/es/section/estilos-de-vida}{Estilos de Vida}\textbar{}En la
economía del coronavirus solo es posible tener una cosa: hijos o empleo.
¿Por qué nadie habla sobre esto?

\url{https://nyti.ms/38HR0Zp}

\begin{itemize}
\item
\item
\item
\item
\item
\end{itemize}

\href{https://www.nytimes3xbfgragh.onion/es/spotlight/coronavirus?action=click\&pgtype=Article\&state=default\&region=TOP_BANNER\&context=storylines_menu}{El
brote de coronavirus}

\begin{itemize}
\tightlist
\item
  \href{https://www.nytimes3xbfgragh.onion/es/interactive/2020/espanol/mundo/coronavirus-en-estados-unidos.html?action=click\&pgtype=Article\&state=default\&region=TOP_BANNER\&context=storylines_menu}{Mapa
  y casos en EE. UU.}
\item
  \href{https://www.nytimes3xbfgragh.onion/es/2020/07/23/espanol/america-latina/bolivia-cloro-coronavirus-ivermectina.html?action=click\&pgtype=Article\&state=default\&region=TOP_BANNER\&context=storylines_menu}{Dióxido
  de cloro, ivermectina y más: ¿funcionan?}
\item
  \href{https://www.nytimes3xbfgragh.onion/es/interactive/2020/science/coronavirus-tratamientos-curas.html?action=click\&pgtype=Article\&state=default\&region=TOP_BANNER\&context=storylines_menu}{Fármacos
  y tratamientos}
\item
  \href{https://www.nytimes3xbfgragh.onion/es/2020/07/28/espanol/ciencia-y-tecnologia/anticuerpos-coronavirus-inmunidad.html?action=click\&pgtype=Article\&state=default\&region=TOP_BANNER\&context=storylines_menu}{Anticuerpos
  e inmunidad}
\item
  \href{https://www.nytimes3xbfgragh.onion/es/2020/04/29/espanol/estilos-de-vida/oximetro-para-que-sirve.html?action=click\&pgtype=Article\&state=default\&region=TOP_BANNER\&context=storylines_menu}{Oxímetros}
\end{itemize}

\includegraphics{https://static01.graylady3jvrrxbe.onion/images/2020/07/05/business/02Schoolparent-illo/02Schoolparent-illo-articleLarge.jpg?quality=75\&auto=webp\&disable=upscale}

Sections

\protect\hyperlink{site-content}{Skip to
content}\protect\hyperlink{site-index}{Skip to site index}

\hypertarget{en-la-economuxeda-del-coronavirus-solo-es-posible-tener-una-cosa-hijos-o-empleo-por-quuxe9-nadie-habla-sobre-esto}{%
\section{En la economía del coronavirus solo es posible tener una cosa:
hijos o empleo. ¿Por qué nadie habla sobre
esto?}\label{en-la-economuxeda-del-coronavirus-solo-es-posible-tener-una-cosa-hijos-o-empleo-por-quuxe9-nadie-habla-sobre-esto}}

Nuestras dificultades no son una preocupación emocional. No estamos
agotados, estamos siendo aplastados por una economía que ha declarado de
manera incomprensible que los padres que trabajan no son esenciales.

Credit...Taylor Callery

Supported by

\protect\hyperlink{after-sponsor}{Continue reading the main story}

Por Deb Perelman

\begin{itemize}
\item
  9 de julio de 2020
\item
  \begin{itemize}
  \item
  \item
  \item
  \item
  \item
  \end{itemize}
\end{itemize}

\href{https://www.nytimes3xbfgragh.onion/2020/07/02/business/covid-economy-parents-kids-career-homeschooling.html}{Read
in English}

\href{https://www.nytimes3xbfgragh.onion/newsletters/el-times}{Regístrate
para recibir nuestro boletín} con lo mejor de The New York Times.

\begin{center}\rule{0.5\linewidth}{\linethickness}\end{center}

La semana pasada recibí un correo electrónico del director de la escuela
de mis hijos en el que nos comentaba sobre algunos de los primeros
detalles para reabrir las escuelas de la ciudad de Nueva York dentro de
unos meses. El mensaje explicaba que el Departamento de Educación de la
ciudad, de acuerdo con los lineamientos federales, requerirá que cada
estudiante tenga un espacio de seis metros cuadrados en el aula. No
todos podrán estar en el edificio al mismo tiempo. El resultado es que
mis hijos podrán
\href{https://www.nytimes3xbfgragh.onion/2020/06/26/us/coronavirus-schools-reopen-fall.html}{asistir
físicamente a la escuela} una de cada tres semanas.

Al mismo tiempo, al parecer, muchos adultos ---al menos los que tienen
la suerte de haber conservado su empleo--- estarán de regreso en la
oficina cuando la economía se reactive. Lo que me desconcierta es que
estos dos planes avanzan a buen ritmo sin considerar a los padres que
trabajan, quienes quedarán atrapados en este mecanismo cuando entre en
marcha.

Permítanme hablar con claridad de lo que no se dice; en la economía de
la COVID-19 solo puedes tener una cosa: hijos \emph{o} empleo.

¿Por qué nadie habla sobre esto? ¿Por qué no escuchamos un alarido
primitivo tan ensordecedor que no se pueda implementar ninguna política
de trabajo sin escuchar a la gente afectada por ella? ¿Por qué yo, una
bloguera de recetas mejor conocida por éxitos como ``Masa para pay suave
con mucha mantequilla'' y el pastel ``Se me antoja un pastel de
chocolate'', es quien alerta sobre esto? Creo que es porque cuando
estamos
\href{https://www.nytimes3xbfgragh.onion/2020/04/27/upshot/coronavirus-exposes-workplace-truths.html?action=click\&module=RelatedLinks\&pgtype=Article}{escolarizando
en la casa todo el día} sin llevar a cabo el trabajo para el que nos
contrataron sino hasta la madrugada, y lo hacemos durante 106 días sin
parar (no es que lleve la cuenta), podríamos estar demasiado exhaustos
como para canalizar nuestro coraje de manera eficaz.

He refunfuñado sobre esto durante meses: en mensajes de texto para
grupos, en grupos secretos de Facebook para madres, cuando me topo en la
calle y converso con la madre de algún amigo de mis hijos. Todas nos
preguntamos por qué no estamos haciendo más alboroto. El consenso es que
todos concuerdan en que esto es una catástrofe, pero que estamos
\href{https://www.nytimes3xbfgragh.onion/es/2020/05/01/espanol/escuela-casa-coronavirus.html}{demasiado
agotados para alzar la voz} más allá de un gemido, mucho menos para
gritarlo por un megáfono. Todo mundo
\href{https://www.nytimes3xbfgragh.onion/2020/06/29/opinion/coronavirus-school-reopening.html}{confiesa
estar desgastado, abatido} y sintiéndose como si estuviera volviéndose
loco, y sabe en su interior que esto es insostenible.

Debería ser obvio, pero una condición previa no negociable para
``retornar a la normalidad ``es que las familias también necesitan
regresar a la normalidad. Pero tan pronto como lo expresas, rápidamente
la conversación se ve empañada con el tipo de argumentos tangenciales e
irrelevantes que no se aceptarían en ningún grupo de debate escolar.

``Pero ni siquiera sabemos si es
\href{https://www.nytimes3xbfgragh.onion/2020/06/12/upshot/epidemiologists-decisions-children-school-coronavirus.html}{seguro
volver a mandar a los niños a la escuela}'' es un argumento totalmente
correcto, pero no es el tema fundamental. La otra cara más triste de la
moneda ---la amiga que me dijo que si la escuela de sus hijos vuelve a
abrir, ellos van a volver a ir aunque no sea seguro porque ella no puede
darse el lujo de dejar de trabajar--- se aproxima más.

``¿Por qué quieren que se enfermen las maestras?'' tampoco es mi plan,
pero es difícil imaginar que un sistema en donde cada niño pasará dos de
cada tres semanas en manos de varios cuidadores solo para volver a
coincidir en un aula, lo cual aumenta al infinito la posible cantidad de
interacciones que pueden transmitir el virus, protegería más a un
maestro que un grupo fijo de alumnos todas las semanas cuyas
interacciones externas están restringidas al mínimo.

``No deberías tener hijos si no puedes cuidarlos'' es algo que diría un
trol y casi da risa, pero ha surgido con tanta frecuencia que cabe la
pregunta de si esperan que les demos clases a nuestros hijos en las
noches. O tal vez deberíamos haber pagado alguna guardería para todas
las edades que se quedara vacía mientras los niños estuvieran en la
escuela, pero que sirviera en el caso de que la escuela que permanece
abierta gracias a tus impuestos y que requiere, por ley, que tu hijo
asista cerrase de manera repentina durante todo el año.

``¿Por qué no disfrutas el tiempo de calidad adicional con tu hijo?''
deja al descubierto lo que en realidad subyace a la superficie: una idea
retrógrada de que quizás algunos padres
(\href{https://www.nytimes3xbfgragh.onion/2020/06/03/business/economy/coronavirus-working-women.html}{se
refieren en realidad a la mamá}) no deberían trabajar, que hacerlo es
muy malo para los niños, que es egoísta pretender beneficios financieros
(o solvencia, como te dirán las madres y padres que trabajan). Es un
sentir tan vinculado a nuestra psique cultural que hacer la insinuación
racional de que no tendríamos que abandonar una profesión o un medio de
subsistencia si las oficinas vuelven a abrir antes que las escuelas, las
guarderías o los campamentos se consideran una oportunidad para volver a
debatir al respecto.

No lo es, y también estás fuera del grupo de debate.

He sabido de algunos padres que tienen la suerte de que algún abuelo
pueda salir al rescate o tienen el dinero suficiente para pagar una
niñera de tiempo completo o un tutor particular para su hijo cuando las
escuelas están cerradas. Todo eso suena envidiable, pero sería absurdo
dejar que la política la determinen personas que tienen algún respaldo.
Si tienes el privilegio de renunciar al trabajo y deseas hacerlo,
perfecto. Pero no lo esgrimas como un arma para azuzar a los demás
porque muchas más personas se están viendo obligadas a dejar de trabajar
este año y nunca se recuperarán ni financiera ni profesionalmente.

Me molestan los artículos que consideran las dificultades de los padres
que trabajan durante este año como un tema sentimental. No estamos
desgastados porque la vida sea difícil este año. Lo estamos porque nos
arrolla una economía que ha declarado de manera incomprensible que los
padres que trabajan no son esenciales.

\includegraphics{https://static01.graylady3jvrrxbe.onion/images/2020/07/02/business/08working-parents-ES-2/merlin_171532341_6bf69907-be52-4e33-902b-e0104088b403-articleLarge.jpg?quality=75\&auto=webp\&disable=upscale}

\hypertarget{maestros-de-medio-tiempo-padres-de-tiempo-completo}{%
\subsection{Maestros de medio tiempo, padres de tiempo
completo}\label{maestros-de-medio-tiempo-padres-de-tiempo-completo}}

Para ponerlo en contexto, permítanme contarles cómo han sido los últimos
meses para mi familia. Las primeras semanas del
\href{https://www.nytimes3xbfgragh.onion/2020/05/01/nyregion/coronavirus-new-york-update.html}{cierre
de escuelas y las empresas} fueron estresantes al máximo. Yo trabajo por
mi cuenta y ya trabajaba de tiempo completo desde mi casa, así que esa
parte no necesitó transición alguna. Pero tenía que usar esta
flexibilidad para asegurarme de que mi esposo, quien normalmente habría
estado en la oficina, no se perdiera de ninguna reunión, llamada o
correo electrónico, al mismo tiempo que gestionaba el
\href{https://www.nytimes3xbfgragh.onion/2020/06/13/health/school-learning-online-education.html}{programa
curricular de aprendizaje a distancia} de nuestros dos hijos, uno en
preescolar y uno en quinto grado. Lo compensaba al trabajar hasta cerca
de las dos de la madrugada todas las noches.

Tres semanas después, se evaporó nuestro estrés marital de mantener un
equilibrio con el trabajo cuando mi esposo fue puesto en licencia sin
goce de sueldo. Él se encargó de la escolaridad en casa e hizo casi todo
lo demás mientras yo me convertí en la única proveedora e intentaba
trabajar tanto como podía en todo momento. Hace unas semanas, terminaron
por despedirlo.

A pesar de nuestra presión económica, hemos pagado a la niñera que nos
ayudaba a llevar a los niños a sus clases mientras trabajábamos, aunque
desde marzo no ha trabajado con nosotros. Incluso si le pidiéramos su
ayuda en la escolaridad en casa este otoño, ¿quién lo haría para sus
hijos en edad escolar? ¿Cuándo podrá mi esposo buscar trabajo? ¿Cómo
puede regresar a trabajar si no hay nadie que cuide a los niños?

Y yo hablo desde una posición privilegiada. Hasta hace poco, éramos una
familia con dos ingresos y con ahorros, pagábamos más del mínimo de las
horas que necesitábamos cada día para el cuidado de los niños solo para
tener cubiertos los imprevistos y vivíamos en una de las ciudades más
caras del mundo. Tenemos computadoras, tabletas, wifi y no lo pensamos
dos veces antes de hacer compras de pánico de lápices, papel, marcadores
y cualquier cosa que pensáramos pudieran necesitar los niños.

Pero mi familia, como una unidad social y económica, no puede funcionar
para siempre en el marco que las autoridades contemplan para dentro de
unos meses. Hay tantas formas en las que la situación que nos ha tocado
---en la cual las empresas planean volver a abrir sin debatir sobre las
repercusiones para las familias con hijos en edad escolar--- es todavía
más insostenible para otras personas.

En las mejores circunstancias, de
\href{https://www.nytimes3xbfgragh.onion/2020/06/05/us/coronavirus-education-lost-learning.html}{cualquier
manera será sumamente considerable el impacto para los niños}. Los
estudiantes perderán la mayor parte del año de escolaridad, ya que los
padres ---los nuevos maestros sin capacitación--- no pueden
supervisarlos de ningún modo satisfactorio mientras ellos mismos tienen
reuniones por Zoom con la oficina. En el mejor de los casos, los niños
estarán malhumorados e inquietos porque no tienen suficiente actividad
física puesto que ahora están atados a los espacios de trabajo de sus
padres todo el día y corren por la sala en lugar de al aire libre. Sin
interacción social con otros niños, demandan constantemente la atención
de sus padres de mala manera, lo que tensa aún más el ambiente en casa.
Y estos son escenarios ideales.

¿Pero qué pasa con los niños que no pueden aprender de forma remota?
¿Qué pasa con los niños que necesitan servicios vinculados a las
escuelas? ¿O aquellos que tienen un mayor riesgo de complicaciones si
contraen el virus y no podrían regresar ni una semana de las tres?

Mientras los planes de aprendizaje para los niños con necesidades
especiales no pueden ser seguidos adecuadamente este año, se eliminaron
rápidamente las ganancias académicas para muchos estudiantes. El
aprendizaje remoto ya ha ampliado las brechas de logros raciales y
socioeconómicos debido a las disparidades en el acceso a los tutores
tecnológicos. A medida que la economía de la COVID aplasta a los padres,
también lo hace con los niños que necesitan más apoyo. No es de extrañar
que la
\href{https://www.nytimes3xbfgragh.onion/2020/06/30/us/coronavirus-schools-reopening-guidelines-aap.html}{Academia
Estadounidense de Pediatría haya publicado una declaración} que insta a
que los estudiantes estén físicamente presentes en la escuela tanto como
sea posible este otoño.

Las pérdidas a largo plazo para los adultos profesionales también serán
incalculables y afectarán de manera desproporcionada a las madres. Las
madres que trabajan sienten que las expulsan de la fuerza laboral o que
las obligan a tomar empleos de medio tiempo debido a que sus
responsabilidades en casa han aumentado diez veces.

Incluso quienes encontraron una solución a corto plazo porque podían
darse el lujo de poner en pausa sus proyectos y profesiones esta
primavera para gestionar los efectos de la pandemia ---basados en la
suposición de que dentro de unos meses habría un retorno a la escuela y
a las guarderías--- tal vez ahora no tengan más opción que dejar de
trabajar. Una amiga acaba de postularse para un empleo y me dice que no
puede ni siquiera imaginar cómo podrá hacerlo si sus hijos no regresan
de lleno a la escuela. Existe la idea de que la gente puede abandonar su
profesión y retomarla donde se quedó, aunque sabemos que las mujeres
\href{https://www.nytimes3xbfgragh.onion/2014/09/07/upshot/a-child-helps-your-career-if-youre-a-man.html}{que
dejan el trabajo} para cuidar a los niños a menudo tienen problemas para
regresar al mercado laboral.

Y para que no crean que todos están contra los maestros, no puedo pensar
en un grupo para el cual esta situación sea más injusta. ¿De verdad se
espera que los maestros enseñen en las aulas de tiempo completo, pero al
mismo tiempo den clases a distancia? Incluso en épocas en las que no hay
pandemias, los maestros te dirían que en las noches y los fines de
semana ya trabajan tiempo extra para planear y calificar sin que se lo
paguen. ¿De dónde, exactamente, salen las horas extra? Para los maestros
que además tienen hijos en edad escolar, la situación no solo es
insostenible, es imposible.

\hypertarget{los-ricos-ganan-de-nuevo}{%
\subsection{Los ricos ganan. De nuevo}\label{los-ricos-ganan-de-nuevo}}

Sin duda, la reapertura de las escuelas es una tarea colosal. No hay
soluciones fáciles para encontrar suficiente espacio para que los
estudiantes se distancien socialmente, y asegurar que los maestros y el
personal estén protegidos, añadir más lavamanos y personal de limpieza e
implementar controles de temperatura, pruebas y rastreo de contactos.

Pero después de casi cuatro meses desde que comenzaron los
confinamientos ---cuatro meses de trabajar a toda hora, a niveles de
estrés notables, mientras nuestros hijos se han quedado sin citas para
jugar y parques infantiles y todos los otros estímulos que los ayudan a
crecer--- la mayoría de los padres se sorprendió al descubrir que los
gobiernos estatales no tenían ninguna solución creativa o plausible.

Para los padres que simplemente no pueden resolverlo, nuestra respuesta
nacional parece más una novela distópica en la que solo los ricos pueden
limitar su exposición y sobrevivir ilesos a la pandemia. Permitir que
los lugares de trabajos se vuelvan a abrir mientras las escuelas, los
campamentos y las guarderías permanecen cerradas le dice a una
generación de padres que trabajan que está bien si pierden sus empleos,
seguros y medios de vida en el proceso. Es indignante, y me temo que si
no hacemos el mayor ruido posible sobre esto, seremos borrados de la
economía.

Deb Perelman es una escritora de Nueva York y creadora del blog de
comida \href{http://smittenkitchen.com/}{smittenkitchen.com}.

\begin{center}\rule{0.5\linewidth}{\linethickness}\end{center}

Advertisement

\protect\hyperlink{after-bottom}{Continue reading the main story}

\hypertarget{site-index}{%
\subsection{Site Index}\label{site-index}}

\hypertarget{site-information-navigation}{%
\subsection{Site Information
Navigation}\label{site-information-navigation}}

\begin{itemize}
\tightlist
\item
  \href{https://help.nytimes3xbfgragh.onion/hc/en-us/articles/115014792127-Copyright-notice}{©~2020~The
  New York Times Company}
\end{itemize}

\begin{itemize}
\tightlist
\item
  \href{https://www.nytco.com/}{NYTCo}
\item
  \href{https://help.nytimes3xbfgragh.onion/hc/en-us/articles/115015385887-Contact-Us}{Contact
  Us}
\item
  \href{https://www.nytco.com/careers/}{Work with us}
\item
  \href{https://nytmediakit.com/}{Advertise}
\item
  \href{http://www.tbrandstudio.com/}{T Brand Studio}
\item
  \href{https://www.nytimes3xbfgragh.onion/privacy/cookie-policy\#how-do-i-manage-trackers}{Your
  Ad Choices}
\item
  \href{https://www.nytimes3xbfgragh.onion/privacy}{Privacy}
\item
  \href{https://help.nytimes3xbfgragh.onion/hc/en-us/articles/115014893428-Terms-of-service}{Terms
  of Service}
\item
  \href{https://help.nytimes3xbfgragh.onion/hc/en-us/articles/115014893968-Terms-of-sale}{Terms
  of Sale}
\item
  \href{https://spiderbites.nytimes3xbfgragh.onion}{Site Map}
\item
  \href{https://help.nytimes3xbfgragh.onion/hc/en-us}{Help}
\item
  \href{https://www.nytimes3xbfgragh.onion/subscription?campaignId=37WXW}{Subscriptions}
\end{itemize}
