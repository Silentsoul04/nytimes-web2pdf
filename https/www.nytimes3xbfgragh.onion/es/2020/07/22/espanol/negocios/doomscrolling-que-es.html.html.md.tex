Sections

SEARCH

\protect\hyperlink{site-content}{Skip to
content}\protect\hyperlink{site-index}{Skip to site index}

\href{https://www.nytimes3xbfgragh.onion/es/section/negocios}{Negocios}

\href{https://myaccount.nytimes3xbfgragh.onion/auth/login?response_type=cookie\&client_id=vi}{}

\href{https://www.nytimes3xbfgragh.onion/section/todayspaper}{Today's
Paper}

\href{/es/section/negocios}{Negocios}\textbar{}Estás viendo contenido
apocalíptico de nuevo. Te decimos cómo dejar de hacerlo

\href{https://nyti.ms/30zszd0}{https://nyti.ms/30zszd0}

\begin{itemize}
\item
\item
\item
\item
\item
\item
\end{itemize}

\href{https://www.nytimes3xbfgragh.onion/es/spotlight/coronavirus?action=click\&pgtype=Article\&state=default\&region=TOP_BANNER\&context=storylines_menu}{El
brote de coronavirus}

\begin{itemize}
\tightlist
\item
  \href{https://www.nytimes3xbfgragh.onion/es/interactive/2020/espanol/mundo/coronavirus-en-estados-unidos.html?action=click\&pgtype=Article\&state=default\&region=TOP_BANNER\&context=storylines_menu}{Mapa
  y casos en EE. UU.}
\item
  \href{https://www.nytimes3xbfgragh.onion/es/2020/07/23/espanol/america-latina/bolivia-cloro-coronavirus-ivermectina.html?action=click\&pgtype=Article\&state=default\&region=TOP_BANNER\&context=storylines_menu}{Dióxido
  de cloro, ivermectina y más: ¿funcionan?}
\item
  \href{https://www.nytimes3xbfgragh.onion/es/interactive/2020/science/coronavirus-tratamientos-curas.html?action=click\&pgtype=Article\&state=default\&region=TOP_BANNER\&context=storylines_menu}{Fármacos
  y tratamientos}
\item
  \href{https://www.nytimes3xbfgragh.onion/es/2020/07/28/espanol/ciencia-y-tecnologia/anticuerpos-coronavirus-inmunidad.html?action=click\&pgtype=Article\&state=default\&region=TOP_BANNER\&context=storylines_menu}{Anticuerpos
  e inmunidad}
\item
  \href{https://www.nytimes3xbfgragh.onion/es/2020/04/29/espanol/estilos-de-vida/oximetro-para-que-sirve.html?action=click\&pgtype=Article\&state=default\&region=TOP_BANNER\&context=storylines_menu}{Oxímetros}
\end{itemize}

Advertisement

\protect\hyperlink{after-top}{Continue reading the main story}

Supported by

\protect\hyperlink{after-sponsor}{Continue reading the main story}

Tecnología

\hypertarget{estuxe1s-viendo-contenido-apocaluxedptico-de-nuevo-te-decimos-cuxf3mo-dejar-de-hacerlo}{%
\section{Estás viendo contenido apocalíptico de nuevo. Te decimos cómo
dejar de
hacerlo}\label{estuxe1s-viendo-contenido-apocaluxedptico-de-nuevo-te-decimos-cuxf3mo-dejar-de-hacerlo}}

En una pandemia que nos obliga a quedarnos en casa, consumir de manera
continua noticias catastrofistas parece inevitable. Unos expertos en
salud ofrecen ayuda para romper la adicción.

\includegraphics{https://static01.graylady3jvrrxbe.onion/images/2020/07/16/business/16Techfix-illo/15Techfix-illo-articleLarge.gif?quality=75\&auto=webp\&disable=upscale}

\href{https://www.nytimes3xbfgragh.onion/by/brian-x-chen}{\includegraphics{https://static01.graylady3jvrrxbe.onion/images/2018/02/16/multimedia/author-brian-x-chen/author-brian-x-chen-thumbLarge.jpg}}

Por \href{https://www.nytimes3xbfgragh.onion/by/brian-x-chen}{Brian X.
Chen}

\begin{itemize}
\item
  Publicado 22 de julio de 2020Actualizado 24 de julio de 2020
\item
  \begin{itemize}
  \item
  \item
  \item
  \item
  \item
  \item
  \end{itemize}
\end{itemize}

\href{https://www.nytimes3xbfgragh.onion/2020/07/15/technology/personaltech/youre-doomscrolling-again-heres-how-to-snap-out-of-it.html}{Read
in English}

\href{https://www.nytimes3xbfgragh.onion/newsletters/el-times}{Regístrate
para recibir nuestro boletín} con lo mejor de The New York Times.

\begin{center}\rule{0.5\linewidth}{\linethickness}\end{center}

La alarma de tu celular suena a las seis de la mañana. Abres algunos
sitios de noticias y Facebook. Ves una mala noticia tras otra. Los casos
de coronavirus aumentan, al igual que las muertes. Los niños no pueden
regresar a la escuela. Tu restaurante y tu peluquería favoritos siguen
cerrados. La gente está perdiendo su empleo.

Todo es horrible. El mundo como lo recordamos ya no existe. De pronto ya
son las nueve de la mañana. No has salido de tu hoyo de desolación ni
siquiera para bañarte. Repites ese ejercicio de masoquismo durante el
almuerzo y una vez más mientras te preparas para dormir.

La experiencia de hundirse en las arenas movedizas de las emociones
mientras consumes sin parar noticias apocalípticas es tan común que
ahora hay una palabra en inglés para decirlo:
``\href{https://www.merriam-webster.com/words-at-play/doomsurfing-doomscrolling-words-were-watching}{\emph{doomscrolling}}''
(viene de la palabra ``doom'', que significa ``perdición'', y
``scroll'', que es el desplazamiento en vertical que se hace en las
pantallas). Las órdenes de confinamiento, que han agravado este
comportamiento, nos dejan pocas alternativas más para pasar el tiempo
que fijar la mirada en nuestras pantallas; según algunos cálculos,
nuestro
\href{https://www.axios.com/kids-screen-time-coronavirus-562073f6-0638-47f2-8ea3-4f8781d6b31b.html}{tiempo
frente a las pantallas ha aumentado al menos un 50 por ciento}.

No somos los únicos, pues muchos pasan por lo mismo. Sin embargo, ver
las redes hasta la perdición, combinado con la adicción a las pantallas,
podría afectar de manera importante nuestro bienestar mental y físico,
de acuerdo con expertos en materia de salud. Esta actividad puede hacer
que nos sintamos enojados, ansiosos, deprimidos, poco productivos y
menos conectados con nuestros seres queridos y con nosotros mismos.

``Es el camino de menos resistencia para seguir consumiendo de manera
pasiva a través de las redes sociales'', dijo Vivek Murthy, un
exdirector general de Sanidad de Estados Unidos que ha escrito de manera
exhaustiva acerca del impacto de la soledad en la salud personal.
``Debes salir de ese ciclo. No solo se trata de desconectarte, sino
también de lidiar con el impacto que eso tiene en tu mente, el cual a
menudo puede durar horas''.

No temas: aún no estamos condenados y hay enfoques para modificar
nuestro comportamiento. Podemos establecer una estructura en nuestra
vida, para empezar, y continuar con la práctica de técnicas de
meditación Esto es lo que dicen los expertos en salud y bienestar.

\hypertarget{crea-un-plan-para-controlar-tu-tiempo}{%
\subsection{Crea un plan para controlar tu
tiempo}\label{crea-un-plan-para-controlar-tu-tiempo}}

Por naturaleza, las personas consumen información, y las noticias son
como dulces digitales que se sirven las 24 horas del día. Para
resistirse a este consumo continuo de información, podemos crear un plan
para controlar la cantidad de contenido que consumimos, de la misma
manera en que las personas pueden crear un régimen para perder peso,
dijo Adam Gazzaley, neurocientífico y coautor del libro
\href{https://mitpress.mit.edu/books/distracted-mind}{\emph{The
Distracted Mind: Ancient Brains in a High-Tech World}}.

El primer paso es reconocer la carga que deslizar el dedo para ver más
noticias implica en nuestra salud, dijo Gazzaley. ``Debes darte cuenta
de que no quieres vivir tu vida en un ciclo infinito de consumo de
noticias'', agregó. ``Eso dejará de ser útil y te afectará, y ser una
persona informada será una ganancia decreciente''.

El segundo paso es crear un plan realista al que puedas apegarte y
repetirlo hasta que se convierta en una costumbre.

Crear un calendario es un enfoque efectivo. Comienza por apartar tiempo
en tu calendario para todo tipo de cosas, desde actividades mundanas,
como dar un paseo en el exterior, hasta asuntos de negocios, como las
reuniones por videoconferencia.

Aparta ciertos momentos del día para leer noticias, si debes hacerlo. Si
te resulta útil, programa una alarma que suene tras diez minutos para
que dejes de ver contenido. Otro truco es ponerte una liga en la mano
mientras lees las noticias y, cuando creas que estás siendo víctima de
un círculo vicioso, date un ligazo en la muñeca, dijo Murthy.

También es importante reconsiderar las pausas. Antes de la pandemia,
solíamos visitar Facebook durante una de nuestras pausas típicas para el
almuerzo. Sin ningún lugar al cual ir para almorzar debido a las órdenes
de confinamiento, explorar internet se ha convertido en la pausa de
trabajo por defecto, una trampa evidente que podría llevar a estar
viendo noticias hasta la perdición.

En vez de quedarte pegado a una pantalla, da un paseo por la cuadra,
sube a la bicicleta fija o prepara tu bocadillo favorito. Y, sí, aparta
tiempo en el calendario incluso para tus pausas, dijo Gazzaley.

\hypertarget{practica-meditaciuxf3n}{%
\subsection{Practica meditación}\label{practica-meditaciuxf3n}}

Los ejercicios de conciencia pueden ayudarnos a romper el ciclo de
consumo continuo de información o evitar que lleguemos a un lugar más
oscuro.

Sharon Salzberg, profesora de meditación y autora del libro
\href{https://www.sharonsalzberg.com/realchange/}{\emph{Real Change:
Mindfulness to Heal Ourselves and the World}}, recomendó este ejercicio
para sentirse una conexión con los demás en una época en que no podemos
ver a muchas personas:

\begin{itemize}
\item
  Inhala y exhala algunas veces, y piensa en las personas que te han
  ayudado en el pasado. Podrían ser tus amigos, colegas e incluso los
  empleados del restaurante que empacan tu comida para llevar.
\item
  Mientras imaginas a esas personas, mándales deseos positivos. Por
  ejemplo: ``Que seas feliz. Que tengas paz. Que estés seguro. Que
  tengas salud''.
\end{itemize}

``De esa manera, regalas buenas vibras'', dijo Salzberg. ``Es una manera
distinta de relacionarse y no sentirse aislado''.

\hypertarget{conuxe9ctate-con-los-demuxe1s}{%
\subsection{Conéctate con los
demás}\label{conuxe9ctate-con-los-demuxe1s}}

El libro de Murthy,
\href{http://www.harperwave.com/book/9780062913296/Together-Vivek-H.-Murthy-MD/}{\emph{Together:
The Healing Power of Human Connection in a Sometimes}}, enfatiza la
importancia de pasar 15 minutos al día en contacto con las personas que
más te importan. Eso puede ayudarnos a sentirnos menos solos y
resistirnos a ver las redes hasta la perdición.

Pero, ¿cómo podemos conectarnos con las personas cuando no podemos
verlas fácilmente? Al inicio de la pandemia, muchos de nosotros
recurrimos a las aplicaciones de videoconferencia para conectarnos
virtualmente con amigos, colegas y seres queridos. Ahora, más de cuatro
meses tras el inicio de la pandemia, muchos experimentan
``\href{https://www.nytimes3xbfgragh.onion/2020/05/20/smarter-living/coronavirus-zoom-facetime-fatigue.html}{fatiga
de Zoom}''.

Murthy dijo que él también se estaba cansado del dolor de cuello causado
por las videollamadas constantes y había comenzado a tomar muchas
llamadas personales y de trabajo en el celular al dar un paseo, lo cual
le da energía y lo ayuda a mantenerse concentrado.

Murthy también recomendó que las personas traten de formar un
\emph{moai}, que en japonés significa grupo de apoyo social. Podría ser
un pequeño grupo de amigos que se reúnan de manera constante ---por
teléfono, por videollamada o en persona a una distancia segura--- y
acepten cuidarse. Él y dos amigos formaron un \emph{moai}, y, una vez al
mes, pasan dos horas poniéndose al corriente mientras conversan de
manera sincera acerca de asuntos personales relacionados con la salud,
las parejas y las finanzas.

Puede ser difícil modificar tu comportamiento de manera independiente.
Podrías decirle a tu \emph{moai} que quieres dejar de estar leyendo las
redes incansablemente, y ellos podrían ayudarte a cumplir. Murthy dijo
que la conversación con sus amigos en el formato \emph{moai} estaba
cerca y que planeaba hablar de tener una relación más sana con las redes
sociales, porque en ocasiones él también cae en el círculo vicioso.

``La idea de apartar tiempo para las personas que te importan, ya sean
quince minutos o más, es muy importante en un mundo donde han
desaparecido los límites entre el día y la noche, entre la semana y el
fin de semana'', dijo.

Brian X. Chen es columnista de tecnología. Reseña productos y escribe
\href{https://www.nytimes3xbfgragh.onion/column/tech-fix}{Tech Fix}, una
columna sobre cómo resolver problemas relacionados con la tecnología.
Antes de unirse al Times en 2011, reportó sobre Apple y la industria
inalámbrica para Wired. \href{https://twitter.com/bxchen}{@bxchen}

\begin{center}\rule{0.5\linewidth}{\linethickness}\end{center}

Advertisement

\protect\hyperlink{after-bottom}{Continue reading the main story}

\hypertarget{site-index}{%
\subsection{Site Index}\label{site-index}}

\hypertarget{site-information-navigation}{%
\subsection{Site Information
Navigation}\label{site-information-navigation}}

\begin{itemize}
\tightlist
\item
  \href{https://help.nytimes3xbfgragh.onion/hc/en-us/articles/115014792127-Copyright-notice}{©~2020~The
  New York Times Company}
\end{itemize}

\begin{itemize}
\tightlist
\item
  \href{https://www.nytco.com/}{NYTCo}
\item
  \href{https://help.nytimes3xbfgragh.onion/hc/en-us/articles/115015385887-Contact-Us}{Contact
  Us}
\item
  \href{https://www.nytco.com/careers/}{Work with us}
\item
  \href{https://nytmediakit.com/}{Advertise}
\item
  \href{http://www.tbrandstudio.com/}{T Brand Studio}
\item
  \href{https://www.nytimes3xbfgragh.onion/privacy/cookie-policy\#how-do-i-manage-trackers}{Your
  Ad Choices}
\item
  \href{https://www.nytimes3xbfgragh.onion/privacy}{Privacy}
\item
  \href{https://help.nytimes3xbfgragh.onion/hc/en-us/articles/115014893428-Terms-of-service}{Terms
  of Service}
\item
  \href{https://help.nytimes3xbfgragh.onion/hc/en-us/articles/115014893968-Terms-of-sale}{Terms
  of Sale}
\item
  \href{https://spiderbites.nytimes3xbfgragh.onion}{Site Map}
\item
  \href{https://help.nytimes3xbfgragh.onion/hc/en-us}{Help}
\item
  \href{https://www.nytimes3xbfgragh.onion/subscription?campaignId=37WXW}{Subscriptions}
\end{itemize}
