Sections

SEARCH

\protect\hyperlink{site-content}{Skip to
content}\protect\hyperlink{site-index}{Skip to site index}

\href{https://www.nytimes3xbfgragh.onion/es/section/ciencia-y-tecnologia}{Ciencia
y Tecnología}

\href{https://myaccount.nytimes3xbfgragh.onion/auth/login?response_type=cookie\&client_id=vi}{}

\href{https://www.nytimes3xbfgragh.onion/section/todayspaper}{Today's
Paper}

\href{/es/section/ciencia-y-tecnologia}{Ciencia y
Tecnología}\textbar{}¿Podemos volver a infectarnos del coronavirus? Los
expertos creen que es muy poco probable

\url{https://nyti.ms/2ZYHoqA}

\begin{itemize}
\item
\item
\item
\item
\item
\end{itemize}

\href{https://www.nytimes3xbfgragh.onion/es/spotlight/coronavirus?action=click\&pgtype=Article\&state=default\&region=TOP_BANNER\&context=storylines_menu}{El
brote de coronavirus}

\begin{itemize}
\tightlist
\item
  \href{https://www.nytimes3xbfgragh.onion/es/interactive/2020/espanol/mundo/coronavirus-en-estados-unidos.html?action=click\&pgtype=Article\&state=default\&region=TOP_BANNER\&context=storylines_menu}{Mapa
  y casos en EE. UU.}
\item
  \href{https://www.nytimes3xbfgragh.onion/es/2020/07/23/espanol/america-latina/bolivia-cloro-coronavirus-ivermectina.html?action=click\&pgtype=Article\&state=default\&region=TOP_BANNER\&context=storylines_menu}{Dióxido
  de cloro, ivermectina y más: ¿funcionan?}
\item
  \href{https://www.nytimes3xbfgragh.onion/es/interactive/2020/science/coronavirus-tratamientos-curas.html?action=click\&pgtype=Article\&state=default\&region=TOP_BANNER\&context=storylines_menu}{Fármacos
  y tratamientos}
\item
  \href{https://www.nytimes3xbfgragh.onion/es/2020/07/28/espanol/ciencia-y-tecnologia/anticuerpos-coronavirus-inmunidad.html?action=click\&pgtype=Article\&state=default\&region=TOP_BANNER\&context=storylines_menu}{Anticuerpos
  e inmunidad}
\item
  \href{https://www.nytimes3xbfgragh.onion/es/2020/04/29/espanol/estilos-de-vida/oximetro-para-que-sirve.html?action=click\&pgtype=Article\&state=default\&region=TOP_BANNER\&context=storylines_menu}{Oxímetros}
\end{itemize}

Advertisement

\protect\hyperlink{after-top}{Continue reading the main story}

Supported by

\protect\hyperlink{after-sponsor}{Continue reading the main story}

\hypertarget{podemos-volver-a-infectarnos-del-coronavirus-los-expertos-creen-que-es-muy-poco-probable}{%
\section{¿Podemos volver a infectarnos del coronavirus? Los expertos
creen que es muy poco
probable}\label{podemos-volver-a-infectarnos-del-coronavirus-los-expertos-creen-que-es-muy-poco-probable}}

Los reportes de reinfección pueden ser casos de enfermedad prolongada.
Una disminución en los anticuerpos es normal después de algunas semanas,
y las personas están protegidas del coronavirus de otras maneras.

\includegraphics{https://static01.graylady3jvrrxbe.onion/images/2020/07/20/science/24virus-reinfection-ES-1/merlin_174703059_d8cefca6-857f-481c-aa7f-6802d23fc7c0-articleLarge.jpg?quality=75\&auto=webp\&disable=upscale}

Por
\href{https://www.nytimes3xbfgragh.onion/by/apoorva-mandavilli}{Apoorva
Mandavilli}

\begin{itemize}
\item
  24 de julio de 2020
\item
  \begin{itemize}
  \item
  \item
  \item
  \item
  \item
  \end{itemize}
\end{itemize}

\href{https://www.nytimes3xbfgragh.onion/2020/07/22/health/covid-antibodies-herd-immunity.html}{Read
in English}

\href{https://www.nytimes3xbfgragh.onion/newsletters/el-times}{Regístrate
para recibir nuestro boletín} con lo mejor de The New York Times.

\begin{center}\rule{0.5\linewidth}{\linethickness}\end{center}

Las anécdotas son preocupantes. Parecía que una mujer de Los Ángeles
\href{https://www.foxla.com/news/southern-california-woman-tests-positive-for-covid-19-for-second-time-after-initial-recovery}{se
estaba recuperando} de la COVID-19, pero, unas semanas después, empeoró
y volvió a dar positivo. Un médico de Nueva Jersey
\href{https://dailyvoice.com/new-jersey/monmouth/news/central-jersey-doctor-reports-patients-reinfected-with-coronavirus/790555/}{afirmó}
que muchos pacientes se curaron de un embate para luego volver a
infectarse con el coronavirus. Y otro médico señaló que, para algunas
personas, un segundo asalto de la enfermedad era una realidad y que este
era mucho más grave.

Estos relatos se conectan con los miedos más profundos que tienen las
personas de estar destinadas a sucumbir a la COVID una y otra vez, de
que se sentirán cada vez más enfermas, y de que nunca saldrán de esta
pandemia de pesadilla. Además, estas historias alimentan el temor de que
no podremos alcanzar la inmunidad comunitaria, el destino final en el
que el virus ya no puede encontrar el número suficiente de víctimas como
para ser una amenaza mortal.

Sin embargo, las anécdotas solo son eso: relatos que no ofrecen ninguna
prueba de reinfecciones, según casi una docena de expertos que estudian
los virus. ``No he sabido de ningún caso en el que esto se haya
demostrado clara y rotundamente'', señaló Marc Lipsitch, epidemiólogo de
la Escuela de Salud Pública T. H. Chang de la Universidad de Harvard.

Otros expertos fueron todavía más alentadores. A pesar de que, en
definitiva, se sabe muy poco acerca del coronavirus a tan solo siete
meses de que comenzó la pandemia, este nuevo virus se está comportando
como la mayoría de los demás virus, sostuvieron, y confirmaron la idea
de que con una vacuna se puede lograr la inmunidad comunitaria.

Podría ser posible que el coronavirus ataque a la misma persona dos
veces, pero es muy poco probable que lo haga en un lapso tan breve o que
la gente se ponga más grave la segunda vez, señalan. Lo que es más
factible es que, en algunas personas, el proceso de la infección sea
prolongado, que los estragos del virus se produzcan con lentitud semanas
o meses después de la exposición inicial.

Las personas que están infectadas con el coronavirus normalmente
\href{https://www.nature.com/articles/s41586-020-2456-9}{generan}
moléculas inmunitarias llamadas anticuerpos. En fechas recientes, varios
equipos han informado que los niveles de estos anticuerpos disminuyen en
\href{https://www.medrxiv.org/content/10.1101/2020.07.09.20149633v1?\%253fcollection=}{dos}
o \href{https://www.nature.com/articles/s41591-020-0965-6}{tres meses},
lo que ha provocado cierta inquietud. Pero la disminución de anticuerpos
es totalmente normal luego de que desaparece una infección aguda,
comentó Michael Mina, inmunólogo de la Universidad de Harvard.

Muchos médicos ``parecen confundidos y dicen: `Qué virus tan
increíblemente extraño que no genera una inmunidad estable', pero están
totalmente equivocados'', señaló Mina. ''Su comportamiento es como de
manual''.

Los anticuerpos no son los únicos que nos protegen contra los patógenos.
El coronavirus también desencadena una
\href{https://www.biorxiv.org/content/10.1101/2020.06.29.174888v1}{firme}
\href{https://www.medrxiv.org/content/10.1101/2020.04.11.20062349v2?\%253fcollection=}{defensa}
de células
\href{https://www.medrxiv.org/content/10.1101/2020.05.13.20100636v1?\%253fcollection=}{inmunitarias}
que \href{https://pubmed.ncbi.nlm.nih.gov/32473127/}{pueden matar al
virus} y activar refuerzos de manera rápida para las batallas futuras.
No se sabe mucho acerca de cuánto tiempo duran estos linfocitos T de
memoria (los que reconocen a otros coronavirus pueden durar toda la
vida), pero pueden reforzar las defensas contra el nuevo coronavirus.

``Si estos se conservan y, sobre todo, si se mantienen dentro del pulmón
y de las vías respiratorias, entonces creo que pueden evitar muy bien
que la infección se propague'', dijo Akiko Iwasaki, inmunóloga de la
Universidad de Yale.

Megan Kent, una patóloga del lenguaje de 37 años que vive en las afueras
de Boston, dio positivo para el virus el 30 de marzo, luego de que se
enfermara su novio. Recuerda que no tenía sentido del gusto ni del
olfato, pero por lo demás, se sentía bien. Tras una cuarentena de 14
días, regresó a trabajar al Hospital Melrose Wakefield y también fue
voluntaria en un asilo de ancianos.

El 8 de mayo, Kent se sintió mal de pronto. ``Me sentía como si me
hubiera atropellado un tráiler'', contó. Durmió durante todo el fin de
semana y el lunes fue al hospital, convencida de que tenía
mononucleosis. Al día siguiente, dio positivo para el
coronavirus\ldots{} otra vez. Estuvo sintiéndose mal durante casi un mes
y sabe que desde entonces tiene anticuerpos.

``Esta vez fue 100 veces peor'', dijo. ``¿Me volví a infectar?''.

Según los expertos, existen otras explicaciones más viables para lo que
vivió Kent. ``No estoy diciendo que no pueda ocurrir. Pero por lo que he
visto hasta ahora, sería un fenómeno poco común'', señaló Peter Hotez,
decano de la Escuela Nacional de Medicina Tropical de la Facultad de
Medicina Baylor.

Es posible, por ejemplo, que Kent no se haya recuperado por completo.
Quizás el virus se haya ocultado en algunas zonas del cuerpo ---como se
sabe que lo hace el virus del ébola--- y luego haya vuelto a aparecer.
Kent no se realizó pruebas intermedias entre las dos que resultaron
positivas, pero incluso si lo hubiera hecho, si las pruebas son
defectuosas y los niveles del virus son bajos, esto puede producir
falsos negativos.

Debido a estos escenarios más probables, Mina dedicó unas palabras a los
médicos que desataron el pánico con los informes de reinfecciones.
``Esto está muy mal, la gente se ha vuelto loca'', señaló. ``Solo es un
señuelo sensacionalista para atraer la atención''.

En las primeras semanas de la pandemia, algunas personas en China, Japón
y Corea del Sur dieron positivo dos veces, lo que
\href{https://www.nytimes3xbfgragh.onion/2020/02/29/health/coronavirus-reinfection.html}{desencadenó
temores parecidos}.

Los Centros para el Control y la Prevención de Enfermedades de Corea del
Sur
\href{https://www.cdc.go.kr/board/board.es?mid=a30402000000\&bid=0030}{investigaron
285 de esos casos} y descubrieron que varios de los que salieron
positivo por segunda vez se dieron dos meses después de la primera vez,
y uno de los casos se vio 82 días más tarde. Casi la mitad de las
personas tenía síntomas en la segunda prueba. Pero los investigadores no
pudieron cultivar virus vivos de ninguna de las muestras, y las personas
infectadas no habían contagiado el virus a otras personas.

``Fue una prueba virológica y epidemiológica bastante sólida de que no
se estaba presentando una reinfección, al menos en esas personas'',
afirmó Angela Rasmussen, viróloga de la Universidad de Columbia en Nueva
York.

La mayoría de la gente que está expuesta al coronavirus
\href{https://www.nytimes3xbfgragh.onion/2020/05/07/health/coronavirus-antibody-prevalence.html}{genera
anticuerpos} que pueden destruir al virus; cuanto más graves son los
síntomas, más intensa es la respuesta. (Solo algunas personas no
producen anticuerpos, pero eso sucede con cualquier virus). La inquietud
acerca de la reinfección ha sido resultado de algunos estudios recientes
que sugieren que estos niveles de anticuerpos se desploman.

\includegraphics{https://static01.graylady3jvrrxbe.onion/images/2020/07/20/science/24virus-reinfection-ES-2/merlin_172461039_33c5b1a6-f9c1-414b-b9ff-9c601d9c45b7-articleLarge.jpg?quality=75\&auto=webp\&disable=upscale}

En un estudio publicado en junio, por ejemplo, se descubrió que, en el
40 por ciento de las personas asintomáticas, los anticuerpos para una
parte del virus
\href{https://www.nytimes3xbfgragh.onion/2020/06/18/health/coronavirus-antibodies.html}{disminuían
a niveles indetectables} al cabo de tres meses. La semana pasada, un
estudio que aún no se ha publicado en ninguna revista evaluada por
expertos demostró que los anticuerpos neutralizantes ---el potente
subtipo de anticuerpos que pueden evitar que el virus infecte las
células---
\href{https://www.medrxiv.org/content/10.1101/2020.07.09.20148429v1}{disminuían
de manera considerable} en un mes.

``En realidad es muy deprimente'', señaló Michael Malim, virólogo del
King's College de Londres. ``Es una gran disminución''.

No obstante, otro estudio sugiere que los niveles de anticuerpos
disminuyen y que luego se estabilizan. En
\href{https://www.medrxiv.org/content/10.1101/2020.07.14.20151126v1}{un
estudio con casi 20.000 personas}, publicado el 17 de julio en la página
de internet MedRxiv, la gran mayoría generaron muchos anticuerpos, y la
mitad de quienes tenían niveles bajos siguieron teniendo anticuerpos que
podían destruir al virus.

``Desde un punto de vista biológico, en realidad nada de esto
sorprende'', señaló Florian Krammer, el inmunólogo que dirigió el
estudio en la Escuela Icahn de Medicina de Monte Sinaí.

Mina concordó con esto. ``Se trata de una dinámica conocida de cómo los
anticuerpos se desarrollan después de una infección: suben muchísimo y
luego vuelven a descender'', señaló.

Luego abundó en el tema: las primeras células que producen anticuerpos
durante una infección se llaman células plasmáticas, las cuales crecen
de manera exponencial hasta formar una agrupación de millones. Pero el
cuerpo no puede mantener esos niveles. Cuando la infección disminuye,
una pequeña parte de las células entra a la médula ósea y se instala
para generar una memoria de inmunidad a largo plazo, la cual puede
producir anticuerpos cuando vuelvan a necesitarse. El resto de las
células plasmáticas se debilitan y mueren.

En los niños, cada exposición posterior a un virus ---o a una vacuna---
aumenta la inmunidad hasta que, en la edad adulta, la respuesta de
anticuerpos es constante y fuerte.

Mina dijo que lo inusual en la pandemia actual es ver cómo se desarrolla
esta dinámica en los adultos, porque rara vez experimentan un virus por
primera vez.

Incluso después de que la primera oleada de inmunidad se desvanezca, es
probable que haya alguna protección residual. Y aunque los anticuerpos
han recibido toda la atención porque son más fáciles de estudiar y
detectar, las células de memoria T y B también son poderosas guerreras
inmunes en la lucha contra cualquier patógeno.

Un estudio publicado el 15 de julio, por ejemplo, analizó tres grupos
diferentes. En uno, cada una de las 36 personas expuestas al nuevo virus
tenía \href{https://www.nature.com/articles/s41586-020-2550-z}{células
T, que reconocen} una proteína que se ve similar en todos los
coronavirus. En otro estudio, 23 personas infectadas con el virus del
\href{https://www.cdc.gov/sars/about/fs-sars-sp.html}{SRAS} en 2003
también tenían esas células T, al igual que 37 personas en el tercer
grupo que nunca estuvieron expuestas a ninguno de los patógenos.

``Parece existir un nivel de inmunidad preexistente contra el SARS-CoV2
en la población general'', dijo Antonio Bertoletti, virólogo de la
Escuela de Medicina Duke NUS en Singapur.

La inmunidad puede haber sido estimulada por la
\href{https://immunology.sciencemag.org/content/5/48/eabd2071}{exposición
previa} a coronavirus que causan resfriados comunes. Es posible que
estas células T no eviten la infección, pero atenuarían la enfermedad y
podrían explicar por qué algunas personas con la COVID-19 tienen
síntomas leves o nulos. ``Creo que la inmunidad celular y de anticuerpos
será igualmente importante'', dijo Bertoletti.

Es posible que los ensayos de vacunas que siguen de cerca a los
voluntarios proporcionen más información acerca de las características
de la inmunidad al nuevo coronavirus y del nivel que se necesita para
evitar una reinfección. Las investigaciones en
\href{https://science.sciencemag.org/content/early/2020/07/01/science.abc5343}{monos}
nos dan esperanzas: en un estudio de
\href{https://science.sciencemag.org/content/early/2020/05/19/science.abc4776}{nueve
monos rhesus}, por ejemplo, la exposición al virus produjo una inmunidad
que fue lo
\href{https://www.nytimes3xbfgragh.onion/2020/05/20/health/coronavirus-vaccine-harvard.html}{suficientemente
fuerte para evitar} una segunda infección.

Los investigadores están dando un seguimiento de los monos infectados
para determinar cuánto tiempo dura esta protección. ``Por su carácter,
los estudios de duración implican cierto tiempo'', señaló Dan Barouch,
el virólogo que dirigió el estudio en el Centro Médico Beth Israel
Deaconess de Boston.

Barouch y otros expertos están en contra de los temores de que quizás
nunca se alcance la inmunidad comunitaria.

``Con vacunas que no son tan perfectas siempre alcanzamos la inmunidad
comunitaria'', afirmó Saad Omer, director del Instituto de Yale para la
Salud Global. ``De hecho es muy poco común contar con vacunas cien por
ciento eficaces''.

Se considera que una vacuna que solo protege a la mitad de la población
que la recibe es moderadamente eficaz, y que una que defiende a más del
80 por ciento es muy eficaz. Incluso una vacuna que solo inhibe los
niveles del virus frenaría su propagación hacia otras personas.

Los expertos señalaron que se habían presentado reinfecciones con otros
patógenos, incluyendo la influenza, pero subrayaron que se trató de
casos excepcionales, y que era probable que sucediera lo mismo con el
nuevo coronavirus.

``Yo diría que, aunque no es muy probable, es posible que sí se
presenten reinfecciones, pero no pensaría que sean comunes'', dijo
Rasmussen. ``No obstante, incluso las pocas incidencias pueden parecer
demasiado frecuentes cuando una gran cantidad de personas han resultado
infectadas''.

Apoorva Mandavilli es reportera del Times y se enfoca en ciencia y salud
global. En 2019 ganó el premio Victor Cohn a la Excelencia en Reportaje
sobre Ciencias Médicas.
\href{https://twitter.com/apoorva_nyc}{@apoorva\_nyc}

Advertisement

\protect\hyperlink{after-bottom}{Continue reading the main story}

\hypertarget{site-index}{%
\subsection{Site Index}\label{site-index}}

\hypertarget{site-information-navigation}{%
\subsection{Site Information
Navigation}\label{site-information-navigation}}

\begin{itemize}
\tightlist
\item
  \href{https://help.nytimes3xbfgragh.onion/hc/en-us/articles/115014792127-Copyright-notice}{©~2020~The
  New York Times Company}
\end{itemize}

\begin{itemize}
\tightlist
\item
  \href{https://www.nytco.com/}{NYTCo}
\item
  \href{https://help.nytimes3xbfgragh.onion/hc/en-us/articles/115015385887-Contact-Us}{Contact
  Us}
\item
  \href{https://www.nytco.com/careers/}{Work with us}
\item
  \href{https://nytmediakit.com/}{Advertise}
\item
  \href{http://www.tbrandstudio.com/}{T Brand Studio}
\item
  \href{https://www.nytimes3xbfgragh.onion/privacy/cookie-policy\#how-do-i-manage-trackers}{Your
  Ad Choices}
\item
  \href{https://www.nytimes3xbfgragh.onion/privacy}{Privacy}
\item
  \href{https://help.nytimes3xbfgragh.onion/hc/en-us/articles/115014893428-Terms-of-service}{Terms
  of Service}
\item
  \href{https://help.nytimes3xbfgragh.onion/hc/en-us/articles/115014893968-Terms-of-sale}{Terms
  of Sale}
\item
  \href{https://spiderbites.nytimes3xbfgragh.onion}{Site Map}
\item
  \href{https://help.nytimes3xbfgragh.onion/hc/en-us}{Help}
\item
  \href{https://www.nytimes3xbfgragh.onion/subscription?campaignId=37WXW}{Subscriptions}
\end{itemize}
