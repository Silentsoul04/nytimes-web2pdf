\href{/es/section/mundo}{Mundo}\textbar{}Él tiene 83, ella 84 y modelan
la ropa que la gente olvida en su lavandería

\url{https://nyti.ms/39DDHK6}

\begin{itemize}
\item
\item
\item
\item
\item
\item
\end{itemize}

\includegraphics{https://static01.graylady3jvrrxbe.onion/images/2020/07/24/world/27Taiwan-Laundry-ES/merlin_174888354_cae974a1-0311-4912-827f-e631000138ca-articleLarge.jpg?quality=75\&auto=webp\&disable=upscale}

Sections

\protect\hyperlink{site-content}{Skip to
content}\protect\hyperlink{site-index}{Skip to site index}

Asia

\hypertarget{uxe9l-tiene-83-ella-84-y-modelan-la-ropa-que-la-gente-olvida-en-su-lavanderuxeda}{%
\section{Él tiene 83, ella 84 y modelan la ropa que la gente olvida en
su
lavandería}\label{uxe9l-tiene-83-ella-84-y-modelan-la-ropa-que-la-gente-olvida-en-su-lavanderuxeda}}

Los dueños de un centro de lavado en Taiwán se han convertido en
estrellas de Instagram por posar con prendas que los clientes abandonan.

Chang Wan-ji, a la derecha, y Hsu Sho-er son dueños de una lavandería en
Taichung, Taiwán y un éxito en Instagram.Credit...Reef Chang

Supported by

\protect\hyperlink{after-sponsor}{Continue reading the main story}

Por Chris Horton

\begin{itemize}
\item
  Publicado 28 de julio de 2020Actualizado 31 de julio de 2020
\item
  \begin{itemize}
  \item
  \item
  \item
  \item
  \item
  \item
  \end{itemize}
\end{itemize}

\href{https://www.nytimes3xbfgragh.onion/2020/07/24/world/asia/taiwan-octogenarian-couple-instagram-laundry.html}{Read
in English}

\href{https://www.nytimes3xbfgragh.onion/newsletters/el-times}{Regístrate
para recibir nuestro boletín} con lo mejor de The New York Times.

\begin{center}\rule{0.5\linewidth}{\linethickness}\end{center}

TAICHUNG, Taiwán --- En la lavandería Wansho en la zona central de
Taiwán, la mayoría de la ropa sucia que la gente deja ahí para planchar,
lavar o limpiar en seco termina de regreso en las manos de sus
propietarios, más limpia de lo que estaba cuando llegó.

Sin embargo, las prendas olvidadas pueden encontrarse en Instagram.

Las blusas, las faldas y los pantalones adornan el cuerpo de los
propietarios octogenarios de la lavandería, Chang Wan-ji y Hsu Sho-er,
quienes se han vuelto famosos en todo el mundo por modelar atuendos
estilizados a partir de cientos de prendas que sus clientes distraídos
olvidaron en su negocio.

El más impactado por la fama reciente de la pareja es su nieto y
estilista improvisado de 31 años, Reef Chang. ``Quedé muy sorprendido'',
dijo el joven Chang hace poco. ``No tenía idea de que tantos extranjeros
se interesarían en mis abuelos''.

A él se le ocurrió la idea de la cuenta en Instagram, dijo. El ritmo del
negocio había bajado durante la pandemia del coronavirus y a sus abuelos
les preocupaba salir, aunque
\href{https://www.nytimes3xbfgragh.onion/interactive/2020/04/09/world/asia/coronavirus-hong-kong-singapore-taiwan.html?searchResultPosition=26}{en
Taiwán} se tomaron medidas muy eficaces para combatir el virus. Con casi
24 millones de habitantes, Taiwán ha reportado solo
\href{https://www.cdc.gov.tw/En}{458 casos, 55 transmisiones locales y
siete muertes}.

``No tenían nada que hacer'', comentó. ``Vi lo aburridos que estaban y
quise iluminar sus vidas''.

\includegraphics{https://static01.graylady3jvrrxbe.onion/images/2020/07/24/world/27Taiwan-Laundry-ES-01/merlin_174867414_81c51ab8-54fe-432c-9dae-1624711482e7-articleLarge.jpg?quality=75\&auto=webp\&disable=upscale}

Son modelos natos frente a la cámara. Hsu, de 84 años, despide la
altanería de una supermodelo pero conserva un aire lúdico. Chang, de 83
años, es la pareja ideal, pues complementa el estilo de su esposa con
una disposición relajada enmarcada por sus cejas pobladas.

``Sus cejas son algo muy especial'', dijo Hsu sonriendo durante una
entrevista en la parte trasera de su lavandería, al lado de un pequeño
altar dedicado a Tudigong, el dios de la Tierra, un elemento común de
los hogares taiwaneses tradicionales.

La ropa que modelan es ecléctica, moderna y divertida. Ambos presumen
zapatos deportivos con agujetas a juego, así como gorras y sombreros
alegres. A veces usa gafas oscuras de colores. En una fotografía a ella
se le ve relajada mientras se recarga en una lavadora gigante, con los
brazos cruzados, mientras él abre la puerta de manera casual, sonriendo.
Posan en un lugar que conocen bien: su local, que proporciona el fondo
hacendoso de la ropa de los clientes, apilada y enrollada en paquetes de
plástico o colgada de estantes.

La actitud jovial de la pareja atrae a un gran número de seguidores
---136.000 hasta el momento--- a pesar de tener tan solo 19
publicaciones en su cuenta,
\href{https://www.instagram.com/wantshowasyoung/?hl=en}{@wantshowasyoung},
desde su creación el 27 de junio.

``Mi nieto es muy creativo'', dijo Hsu. ``Su creatividad nos ha hecho
felices, y también a otras personas''.

La cuenta ha atraído a fanáticos de todo Taiwán y del mundo entero, y
muchos vieron las fotos como un bálsamo durante un año oscurecido por
las preocupaciones sobre una pandemia global, la ruina económica, el
cambio climático y la
\href{https://www.nytimes3xbfgragh.onion/2020/07/01/world/asia/taiwan-china-hong-kong.html?searchResultPosition=1}{tensión
geopolítica}.

``Mirar las fotos de Wan-ji y Sho-er mejora mi estado de ánimo'',
escribió un usuario de Instagram llamado tibbar1 el jueves 23 en
respuesta a una foto que celebraba que la cuenta superaba los 100.000
seguidores. ``Sus fotos realmente tienen una vibra encantadora que no
cualquiera puede lograr''.

Image

Chang y Hsu reciben mensajes de fanáticos de todo el mundo.Credit...Reef
Chang

Image

Él dice que desea que los clientes vuelvan a recoger su ropa y paguen
sus facturas.Credit...Reef Chang

La pareja quizá sea famosa en internet hoy, pero su relación de 61 años
tuvo un comienzo más tradicional. Su historia es paralela a la del
Taiwán moderno, pues comenzó durante la era de la represión cuando
estaba sometido a la ley marcial y se desarrolló conforme Taiwán poco a
poco se volvió menos aislacionista y más seguro.

Chang, entonces de 21 años, conoció a Hsu a finales de la década de
1950, cuando su hermana mayor y su tía se le acercaron en Houli, el
lugar de origen de la pareja, un distrito semirrural en el norte de la
ciudad de Taichung, con el objetivo de llegar a un acuerdo matrimonial.
Cuando lo llevaron a casa para conocer a Hsu, no se quedó mucho tiempo,
lo cual la consternó.

``Quería que se sentara junto a mí, pero no lo hizo'', dijo. En ese
entonces las cosas eran más conservadoras. ``Él era muy tímido'',
agregó.

Pero no se sintió desanimado en absoluto. ``La primera vez que la vi,
quedé fascinado'', comentó Chang. ``Poco después, comenzamos a hablar de
matrimonio''.

La pareja se casó en 1959. Se volvieron padres de dos hijos y dos hijas,
y terminaron por ser abuelos de seis nietos. Trabajaban juntos en el
negocio que él había estado administrando desde que tenía 14 años, donde
lavaban y limpiaban en seco la ropa de los vecinos en Houli. Se hicieron
de una cuantiosa clientela, y algunos aún llevan su ropa ahí a pesar de
haberse mudado desde hace tiempo al centro de Taichung.

Ahora, la lavandería Wansho, cuyo nombre proviene de los segundos
caracteres de los nombres de los propietarios, está abierta a diario de
las ocho de la mañana a las nueve de la noche, aunque a veces cierran
temprano si está lloviendo, contó Chang. Él y su esposa son los únicos
empleados.

Image

Cientos de prendas han sido olvidadas.Credit...An Rong Xu para The New
York Times

Image

La preparación de un atuendo con ropas abandonadas.Credit...An Rong Xu
para The New York Times

En la década de 1980, los dos comenzaron a viajar al extranjero después
de 38 años de ley marcial en Taiwán, y visitaron Estados Unidos, Japón,
Europa y Australia. Ahora, esos viajes ayudan a conectarlos con muchos
de los mensajes que llegan de todos los rincones del mundo a través de
sus fotos en Instagram, dijo el joven Chang.

``Les leo algunos de los mensajes que recibimos y les digo de dónde
vienen, y ellos dicen: `Ah, ¡he estado allí!''', dijo.

Chang dijo que esperaba que la experiencia de ambos inspire a otros
residentes mayores de Taiwán y otros lugares a mantenerse activos.

``Es mejor que sentarse a mirar televisión o tomar una siesta'', dijo.
``Puedo estar envejeciendo, pero no me siento viejo''.

Reef Chang dijo que las últimas semanas han sido un momento especial
para sus abuelos: los clientes se quedan y conversan un poco más, lo que
ha hecho a la pareja más feliz. También los alegran los mensajes
amistosos enviados desde todo el mundo. ``Últimamente, cuando comemos
juntos'', dijo, ``puedo notar que están eufóricos''.

Image

Chang actúa como el estilista improvisado de sus abuelos. Dice que ellos
están eufóricos con la respuesta.Credit...An Rong Xu para The New York
Times

La fama en internet es célebremente pasajera, y los propietarios de la
lavandería Wansho no tienen deseos de ganar dinero con esta actividad
secundaria. Aunque estarían felices si los cientos de personas que han
olvidado recoger su ropa regresaran a pagar la cuenta, dijo Chang
Wan-ji.

``Sería agradable hablar con ellos'', dijo, levantando una ceja. ``Y que
nos paguen''.

El 23 de julio por la mañana, por primera vez en casi siete décadas,
algo inusual ocurrió en la lavandería Wansho. Un cliente que había
dejado su ropa hacía más de un año y vio a la pareja en las noticias
locales finalmente regresó por sus prendas y a pagar la cuenta.

\begin{center}\rule{0.5\linewidth}{\linethickness}\end{center}

Advertisement

\protect\hyperlink{after-bottom}{Continue reading the main story}

\hypertarget{site-index}{%
\subsection{Site Index}\label{site-index}}

\hypertarget{site-information-navigation}{%
\subsection{Site Information
Navigation}\label{site-information-navigation}}

\begin{itemize}
\tightlist
\item
  \href{https://help.nytimes3xbfgragh.onion/hc/en-us/articles/115014792127-Copyright-notice}{©~2020~The
  New York Times Company}
\end{itemize}

\begin{itemize}
\tightlist
\item
  \href{https://www.nytco.com/}{NYTCo}
\item
  \href{https://help.nytimes3xbfgragh.onion/hc/en-us/articles/115015385887-Contact-Us}{Contact
  Us}
\item
  \href{https://www.nytco.com/careers/}{Work with us}
\item
  \href{https://nytmediakit.com/}{Advertise}
\item
  \href{http://www.tbrandstudio.com/}{T Brand Studio}
\item
  \href{https://www.nytimes3xbfgragh.onion/privacy/cookie-policy\#how-do-i-manage-trackers}{Your
  Ad Choices}
\item
  \href{https://www.nytimes3xbfgragh.onion/privacy}{Privacy}
\item
  \href{https://help.nytimes3xbfgragh.onion/hc/en-us/articles/115014893428-Terms-of-service}{Terms
  of Service}
\item
  \href{https://help.nytimes3xbfgragh.onion/hc/en-us/articles/115014893968-Terms-of-sale}{Terms
  of Sale}
\item
  \href{https://spiderbites.nytimes3xbfgragh.onion}{Site Map}
\item
  \href{https://help.nytimes3xbfgragh.onion/hc/en-us}{Help}
\item
  \href{https://www.nytimes3xbfgragh.onion/subscription?campaignId=37WXW}{Subscriptions}
\end{itemize}
