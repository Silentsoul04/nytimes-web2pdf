Sections

SEARCH

\protect\hyperlink{site-content}{Skip to
content}\protect\hyperlink{site-index}{Skip to site index}

\href{https://www.nytimes3xbfgragh.onion/es/section/mundo}{Mundo}

\href{https://myaccount.nytimes3xbfgragh.onion/auth/login?response_type=cookie\&client_id=vi}{}

\href{https://www.nytimes3xbfgragh.onion/section/todayspaper}{Today's
Paper}

\href{/es/section/mundo}{Mundo}\textbar{}Angustia, vino blanco y gel
antibacterial

\url{https://nyti.ms/305AvDC}

\begin{itemize}
\item
\item
\item
\item
\item
\item
\end{itemize}

\href{https://www.nytimes3xbfgragh.onion/es/spotlight/coronavirus?action=click\&pgtype=Article\&state=default\&region=TOP_BANNER\&context=storylines_menu}{El
brote de coronavirus}

\begin{itemize}
\tightlist
\item
  \href{https://www.nytimes3xbfgragh.onion/es/interactive/2020/espanol/mundo/coronavirus-en-estados-unidos.html?action=click\&pgtype=Article\&state=default\&region=TOP_BANNER\&context=storylines_menu}{Mapa
  y casos en EE. UU.}
\item
  \href{https://www.nytimes3xbfgragh.onion/es/2020/07/23/espanol/america-latina/bolivia-cloro-coronavirus-ivermectina.html?action=click\&pgtype=Article\&state=default\&region=TOP_BANNER\&context=storylines_menu}{Dióxido
  de cloro, ivermectina y más: ¿funcionan?}
\item
  \href{https://www.nytimes3xbfgragh.onion/es/interactive/2020/science/coronavirus-tratamientos-curas.html?action=click\&pgtype=Article\&state=default\&region=TOP_BANNER\&context=storylines_menu}{Fármacos
  y tratamientos}
\item
  \href{https://www.nytimes3xbfgragh.onion/es/2020/07/28/espanol/ciencia-y-tecnologia/anticuerpos-coronavirus-inmunidad.html?action=click\&pgtype=Article\&state=default\&region=TOP_BANNER\&context=storylines_menu}{Anticuerpos
  e inmunidad}
\item
  \href{https://www.nytimes3xbfgragh.onion/es/2020/04/29/espanol/estilos-de-vida/oximetro-para-que-sirve.html?action=click\&pgtype=Article\&state=default\&region=TOP_BANNER\&context=storylines_menu}{Oxímetros}
\end{itemize}

Advertisement

\protect\hyperlink{after-top}{Continue reading the main story}

Supported by

\protect\hyperlink{after-sponsor}{Continue reading the main story}

Europa

\hypertarget{angustia-vino-blanco-y-gel-antibacterial}{%
\section{Angustia, vino blanco y gel
antibacterial}\label{angustia-vino-blanco-y-gel-antibacterial}}

La pandemia del coronavirus y los aranceles impuestos por el gobierno de
Donald Trump perjudicaron al mercado del vino francés. Ahora el destino
de la cosecha es convertirse en desinfectante para manos.

\includegraphics{https://static01.graylady3jvrrxbe.onion/images/2020/07/24/world/28Francia-vino-ES/merlin_174871056_ae254e73-15d3-440c-997b-65cde45a173f-articleLarge.jpg?quality=75\&auto=webp\&disable=upscale}

\href{https://www.nytimes3xbfgragh.onion/by/adam-nossiter}{\includegraphics{https://static01.graylady3jvrrxbe.onion/images/2018/10/15/multimedia/author-adam-nossiter/author-adam-nossiter-thumbLarge.png}}

Por \href{https://www.nytimes3xbfgragh.onion/by/adam-nossiter}{Adam
Nossiter}

\begin{itemize}
\item
  Publicado 28 de julio de 2020Actualizado 31 de julio de 2020
\item
  \begin{itemize}
  \item
  \item
  \item
  \item
  \item
  \item
  \end{itemize}
\end{itemize}

\href{https://www.nytimes3xbfgragh.onion/2020/07/27/world/europe/france-alsace-wine-coronavirus.html}{Read
in English}

\href{https://www.nytimes3xbfgragh.onion/newsletters/el-times}{Regístrate
para recibir nuestro boletín} con lo mejor de The New York Times.

\begin{center}\rule{0.5\linewidth}{\linethickness}\end{center}

HUNAWIHR, Francia --- El camión cisterna se estacionó y era hora de
dejarlo ir. La decisión de enviar el vino a la destilería se había
tomado semanas atrás. Aún dolía. Poco después, el vino sería gel
antibacterial.

``Tenemos que cargarlo ya'', dijo Jérôme Mader, un vitivinicultor de 38
años, mientras murmuraba para sí mismo. ``Muy bien. Ya no voy a pensar
en eso'', dijo en voz baja. ``Se acabó''.

Cabizbajo, arrastró las mangueras a través de su cobertizo, las fijó a
las válvulas del camión con la ayuda del conductor, se dirigió a su
bodega fría y abrió las bombas. El vino ---buen vino blanco de Alsacia,
vino bebible--- pasó por las mangueras y entró al contenedor del camión.
Era demasiado insoportable pensar en su destino.

Por todo el campo vinícola color esmeralda de Alsacia, ahora cubierto de
viñedos con tonos verde oscuro ---y en otras regiones vinícolas
francesas también--- miles de productores de vino, famosos y
desconocidos, enfrentan momentos similares de angustia.

La crisis económica provocada por el coronavirus, combinada con el
impuesto del 25 por ciento que Trump impuso a los vinos franceses en la
guerra comercial con Europa, ha colapsado el mercado vinícola.

Mader, cuyos rieslings y Gewürztraminers de primera calidad se envían a
restaurantes y tiendas elegantes en ambos costados del Atlántico, ha
perdido la mitad de sus ventas desde diciembre.

``La COVID es una catástrofe para nosotros'', dijo.

\includegraphics{https://static01.graylady3jvrrxbe.onion/images/2020/07/24/world/28Francia-vino-ES-01/merlin_174871080_804e4be1-60d0-478c-8585-aa8d55bcafae-articleLarge.jpg?quality=75\&auto=webp\&disable=upscale}

Así que algunos de los sutiles y suculentos vinos blancos por los que la
región es famosa, cultivados en las laderas rocosas y soleadas de
Alsacia, terminarán siendo gel antibacterial.

Como otros productores, Mader no tiene espacio en su bodega para
almacenar el vino que no se ha vendido. ``No podemos acumular lo que no
hemos vendido'', dijo.

La cosecha precoz de 2020, bendecida por la luz abundante del sol, está
a menos de un mes. Los lagares deben vaciarse para la nueva producción.
Para obtener una modesta compensación, la destilería es la única opción.

El conductor de la destilería había estado haciendo rondas en las
bodegas de los enólogos toda la mañana. ``Algunos de ellos lo toman de
muy mala manera, porque este vino tiene valor comercial'', dijo de
manera mordaz Lucas Neret, el conductor.

``Estamos produciendo más de lo que podemos vender'', dijo Thibaut
Specht, un enólogo de Mittelwihr, cerca de ahí. ``No tenemos
alternativa''.

Image

Viñedos cerca de Reichsfeld, FranciaCredit...Dmitry Kostyukov para The
New York Times

El negocio familiar de Marion Borès, Domaine Borès, en Reichsfeld,
enviará el 30 por ciento de su producción: 19.000 litros. ``Es como si
te despidieras de alguien que quieres mucho'', comentó.

``Este no es exactamente el destino que teníamos en mente cuando hicimos
este vino'', agregó la bodeguera de 27 años.

El vino viejo termina en el enorme silo de acero de la destilería Romann
cerca de ahí, donde lo hervirán para transformarlo en alcohol.

Tan solo en Alsacia, más de seis millones de litros de vino terminarán
así. Mader enviará el 15 por ciento de su producción, el vino que llama
\emph{Edelzwicker}, o ``mezcla noble'' en el dialecto alsaciano.
Generalmente vendido al mayoreo, ``sigue siendo muy bueno'', señaló
Mader.

En la destilería, el olor del vino hervido, como la esencia de una rica
salsa de res al vino tinto, se percibía intensamente arriba del
establecimiento una mañana cálida esta semana.

Image

Erwin Brouard en la destilería Romann en Sigolsheim,
Francia.Credit...Dmitry Kostyukov para The New York Times

``Destilamos continuamente'', dijo Erwin Brouard, el director de la
compañía. ``Es muy triste para los vitivinicultores. Sus reservas son
demasiado grandes. Tienen que hacer espacio. Y la cosecha ha comenzado
antes de tiempo este año''.

El gobierno francés, ansioso por proteger su precioso legado
vitivinícola, está dando subsidios para la operación, y compensó a
algunos de los 5000 vitivinicultores que hasta ahora se han inscrito con
una fracción del valor del vino, menos de un dólar por litro, en lo que
el gobierno llama ``destilación de crisis''.

``Mi bodega está a reventar'', dijo Guillaume Klauss, propietario de una
bodega cercana. ``Si no me deshago de él, no como. Evidentemente, esto
me ha destrozado. Son tres años de trabajo, y ni siquiera nos lo pagan
como se debe''.

Alsacia ha tenido que recurrir a la destilación de crisis por primera
vez en su historia, aunque no es una estrategia desconocida en otras
regiones vitivinícolas. La última vez que ocurrió eso fue en 2009,
después del colapso financiero.

Image

El camión cisterna se dirige a Hunawihr.Credit...Dmitry Kostyukov para
The New York Times

``Una gran mayoría se ha visto golpeada por esta crisis'', dijo Francis
Backert, dirigente de la Asociación de Productores de Vino
Independientes de Alsacia. ``Estas personas están sufriendo de verdad''.

``Todos los puntos de venta están bloqueados'', agregó. ``La exportación
está bloqueada. Trump, la COVID. Hay muy pocos negocios fuera de
Francia. El mercado estadounidense está bloqueado''.

Los comerciantes de vino al mayoreo enfrentan pérdidas del 70 por
ciento, dijo.

Sin embargo, las pérdidas monetarias son una cosa. También está el golpe
psicológico.

``Mira, estas personas tienen mucha cautela y vergüenza'', dijo Backert.
``Simplemente no quieren hablar al respecto. Obviamente la situación les
está rompiendo el corazón''.

Algunos productores de vino de la región se negaron a ser entrevistados
sobre el tema.

La relación con sus viñedos, y lo que se produce a partir de ellos, es
tanto personal como financiera. Muchos viven en casas modestas y se
dedican a un negocio familiar que, a menudo, se remonta a siglos. La
fecha tallada sobre la bodega original de Borès es 1723.

En las laderas de esquisto y arenisca bañadas por el sol sobre
Reichsfeld, Borès patrullaba las vides en las que ha trabajado desde los
diez años, sacaba hojas secas y arrancaba uvas marchitas. Su toque era
ligero.

Image

``Es como si te despidieras de alguien que quieres mucho'', dijo Marion
Borès, una bodegueraCredit...Dmitry Kostyukov para The New York Times

``Estas son las vides de las que nos preocupamos durante todo el año'',
dijo su madre, Marie-Claire. ``Lo hacemos todo a mano. Y ahora esto.
Terrible''.

Al subir la empinada ladera, Marion dijo: ``jugábamos en estos
viñedos''. Y agregó que ella también participa en la cosecha.

``El esquisto es mágico'', dijo. Es lo que hace que el vino sea
dinámico. Hay momentos en los que estás muy contento de estar solo en
estas viñas''.

En su carrera, Mader ha ganado premios y enfrentado el problema opuesto
del que tiene actualmente: no tener suficiente vino para satisfacer la
demanda.

``Hace unos años habría sido impensable que un camión-cisterna tendría
que pasar aquí algún día'', dijo, mientras su voz se desvanecía.

Durante días aplazó la toma de una decisión acerca de la destilería.

``Vacilé'', dijo. ``Pensé que lo superaríamos. Esperé hasta el último
día para decidir. Siempre creo que el siguiente día será mejor''.

No obstante, la decisión no podría posponerse; el gobierno presionaba
con su fecha límite de inscripción.

Después, para consolarse a él y a sus colegas, dijo, ``llamé a un amigo,
y bebimos un par de botellas''.

``Si el vino es bueno, siempre habrá esperanza'', agregó.

Desde hace poco los pedidos han aumentado un poco. Además, ``las uvas
este año son verdaderamente magníficas'', dijo.

Image

Para los productores de vino en Alsacia, esta crisis les ha roto el
corazón. El paisaje en Colmar, en Alsacia.Credit...Dmitry Kostyukov para
The New York Times

Adam Nossiter es el jefe del buró en París. Anteriormente, fue
corresponsal en París, jefe del buró en África Occidental, y dirigió el
equipo que ganó el Premio Pulitzer en la categoría de Reportería
Internacional en 2015, por la cobertura de la epidemia de ébola.

\begin{center}\rule{0.5\linewidth}{\linethickness}\end{center}

Advertisement

\protect\hyperlink{after-bottom}{Continue reading the main story}

\hypertarget{site-index}{%
\subsection{Site Index}\label{site-index}}

\hypertarget{site-information-navigation}{%
\subsection{Site Information
Navigation}\label{site-information-navigation}}

\begin{itemize}
\tightlist
\item
  \href{https://help.nytimes3xbfgragh.onion/hc/en-us/articles/115014792127-Copyright-notice}{©~2020~The
  New York Times Company}
\end{itemize}

\begin{itemize}
\tightlist
\item
  \href{https://www.nytco.com/}{NYTCo}
\item
  \href{https://help.nytimes3xbfgragh.onion/hc/en-us/articles/115015385887-Contact-Us}{Contact
  Us}
\item
  \href{https://www.nytco.com/careers/}{Work with us}
\item
  \href{https://nytmediakit.com/}{Advertise}
\item
  \href{http://www.tbrandstudio.com/}{T Brand Studio}
\item
  \href{https://www.nytimes3xbfgragh.onion/privacy/cookie-policy\#how-do-i-manage-trackers}{Your
  Ad Choices}
\item
  \href{https://www.nytimes3xbfgragh.onion/privacy}{Privacy}
\item
  \href{https://help.nytimes3xbfgragh.onion/hc/en-us/articles/115014893428-Terms-of-service}{Terms
  of Service}
\item
  \href{https://help.nytimes3xbfgragh.onion/hc/en-us/articles/115014893968-Terms-of-sale}{Terms
  of Sale}
\item
  \href{https://spiderbites.nytimes3xbfgragh.onion}{Site Map}
\item
  \href{https://help.nytimes3xbfgragh.onion/hc/en-us}{Help}
\item
  \href{https://www.nytimes3xbfgragh.onion/subscription?campaignId=37WXW}{Subscriptions}
\end{itemize}
