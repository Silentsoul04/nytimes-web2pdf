Sections

SEARCH

\protect\hyperlink{site-content}{Skip to
content}\protect\hyperlink{site-index}{Skip to site index}

\href{https://www.nytimes3xbfgragh.onion/es/section/opinion}{Opinión}

\href{https://myaccount.nytimes3xbfgragh.onion/auth/login?response_type=cookie\&client_id=vi}{}

\href{https://www.nytimes3xbfgragh.onion/section/todayspaper}{Today's
Paper}

\href{/es/section/opinion}{Opinión}\textbar{}¿Por qué no explota
Argentina?

\url{https://nyti.ms/3hDU4ZC}

\begin{itemize}
\item
\item
\item
\item
\item
\end{itemize}

\href{https://www.nytimes3xbfgragh.onion/es/spotlight/coronavirus?action=click\&pgtype=Article\&state=default\&region=TOP_BANNER\&context=storylines_menu}{El
brote de coronavirus}

\begin{itemize}
\tightlist
\item
  \href{https://www.nytimes3xbfgragh.onion/es/interactive/2020/espanol/mundo/coronavirus-en-estados-unidos.html?action=click\&pgtype=Article\&state=default\&region=TOP_BANNER\&context=storylines_menu}{Mapa
  y casos en EE. UU.}
\item
  \href{https://www.nytimes3xbfgragh.onion/es/2020/07/23/espanol/america-latina/bolivia-cloro-coronavirus-ivermectina.html?action=click\&pgtype=Article\&state=default\&region=TOP_BANNER\&context=storylines_menu}{Dióxido
  de cloro, ivermectina y más: ¿funcionan?}
\item
  \href{https://www.nytimes3xbfgragh.onion/es/interactive/2020/science/coronavirus-tratamientos-curas.html?action=click\&pgtype=Article\&state=default\&region=TOP_BANNER\&context=storylines_menu}{Fármacos
  y tratamientos}
\item
  \href{https://www.nytimes3xbfgragh.onion/es/2020/07/28/espanol/ciencia-y-tecnologia/anticuerpos-coronavirus-inmunidad.html?action=click\&pgtype=Article\&state=default\&region=TOP_BANNER\&context=storylines_menu}{Anticuerpos
  e inmunidad}
\item
  \href{https://www.nytimes3xbfgragh.onion/es/2020/04/29/espanol/estilos-de-vida/oximetro-para-que-sirve.html?action=click\&pgtype=Article\&state=default\&region=TOP_BANNER\&context=storylines_menu}{Oxímetros}
\end{itemize}

Advertisement

\protect\hyperlink{after-top}{Continue reading the main story}

\href{/es/section/opinion}{Opinión}

Supported by

\protect\hyperlink{after-sponsor}{Continue reading the main story}

Comentario

\hypertarget{por-quuxe9-no-explota-argentina}{%
\section{¿Por qué no explota
Argentina?}\label{por-quuxe9-no-explota-argentina}}

Lecciones de 2001, una fuerte política social asistencialista y una
grieta política potente han alejado de momento otro estallido social,
pero solo una nueva política cooperativa podrá terminar de disipar ese
fantasma.

\includegraphics{https://static01.graylady3jvrrxbe.onion/images/2020/07/28/multimedia/28garcia-timerman-ES-3/merlin_174681399_76c8b4e4-076e-44d4-9479-f301c2198739-articleLarge.jpg?quality=75\&auto=webp\&disable=upscale}

Por Marcelo J. García y Jordana Timerman

Son periodistas argentinos.

\begin{itemize}
\item
  28 de julio de 2020
\item
  \begin{itemize}
  \item
  \item
  \item
  \item
  \item
  \end{itemize}
\end{itemize}

\href{https://www.nytimes3xbfgragh.onion/newsletters/el-times}{Regístrate
para recibir nuestro boletín} con lo mejor de The New York Times.

\begin{center}\rule{0.5\linewidth}{\linethickness}\end{center}

BUENOS AIRES --- Cualquier argentino puede recordar qué estaba haciendo
en diciembre de 2001, cuando el país estalló en una crisis de
proporciones épicas que incluyó la renuncia de dos presidentes en diez
días, un default masivo de la deuda, estado de sitio, represión y
muertes en la Plaza de Mayo y un vacío de poder simbolizado por un grito
en las calles: ``Que se vayan todos''. Desde entonces, ``diciembre de
2001'' es sinónimo de debacle y los argentinos esperamos su regreso. Es
un fantasma que nos aterroriza.

En el actual momento de crisis magnificado por la pandemia, todas las
condiciones parecen dadas para que vuelva el pasado. Este 2020 empezó
con dos años de recesión a cuestas, sumados a varios previos de
estancamiento. La pobreza alcanzó
\href{https://www.clarin.com/economia/2019-termino-35-5-pobreza-8-indigencia_0__VnW6AFXG.html}{35,5
por ciento} en el segundo semestre de 2019 y se estima que ya ascendió a
\href{https://www.ambito.com/economia/pobreza/la-subio-7-puntos-la-pandemia-y-ya-se-ubica-el-45-segun-la-uca-n5114991}{45
por ciento} durante la pandemia. Nos enfrentamos a una montaña de deuda
que parece impagable (al igual que en 2001) y la inflación anual no baja
de 40 por ciento. Con el impacto de la pandemia, cuando termine el año
se calcula que la economía del país se habrá contraído un
\href{https://www.reuters.com/article/us-argentina-economy-poll/argentina-economy-to-shrink-12-in-2020-as-pandemic-batters-output-central-bank-poll-idUSKBN2442OY}{12
por ciento}. En 2002, por los efectos de 2001, había caído
\href{https://www.lanacion.com.ar/economia/el-pbi-cayo-109-por-ciento-durante-2002-nid482035/}{10,9
por ciento}.

``¿Cuándo llega 2001?'', nos preguntamos cada tanto.
``\href{https://www.batimes.com.ar/news/argentina/mario-ishii-by-late-august-well-be-where-we-were-in-2001.phtml}{Para
fines de agosto {[}de 2020{]} vamos a estar en 2001''}, dijo
recientemente un intendente de la zona metropolitana de Buenos Aires,
sin notar la paradoja temporal subyacente. Pero Argentina se resiste a
estallar, por razones económicas y políticas cuyo origen están en parte
en las experiencias de aquella época.

El miedo del retorno se mezcla con el deseo en algunos casos. Muchos
creen que, aún con toda su tragedia a cuestas, 2001 fue un momento de
destrucción creativa para la sociedad argentina, un grito de hartazgo
que reinició a un sistema político bipartidarista que había fracasado en
resolver los problemas económicos fundamentales del país durante las
casi dos décadas que siguieron a la dictadura militar que terminó en
1983.

Pero nada garantiza hoy que reiniciar el juego político de manera tan
disruptiva, en todo sentido, pueda tener un buen resultado. El
\href{https://www.nytimes3xbfgragh.onion/es/2019/02/01/espanol/opinion/crisis-partidos-politicos.html}{surgimiento
de ``outsiders'' políticos} en otros países de la región en estos años
es un indicio poco auspicioso. Al contrario, el mayor desafío político a
dos décadas de esa crisis consiste en evadir una nueva debacle. Y esto
requiere que intentos incipientes de transformar nuestra política
sectaria hacia formas de mayor cooperación sean exitosos.

Algunos consensos ya existen de hecho. En lo económico, desde la salida
de aquella crisis el país generó una red de sustento estatal a los
sectores menos favorecidos, a través de una serie de programas que
institucionalizaron desde el Estado una transferencia masiva de
recursos.

En 2005, el gobierno de Néstor Kirchner abrió una moratoria para que
\href{https://www.batimes.com.ar/news/argentina/mario-ishii-by-late-august-well-be-where-we-were-in-2001.phtml}{más
de dos millones de personas} que no habían hecho aportes suficientes
pudiesen jubilarse. En 2009, la entonces presidenta Cristina Fernández
de Kirchner creó la Asignación Universal por Hijo (AUH), un ingreso
básico destinado a padres desempleados o
\href{https://www.baenegocios.com/economia/Con-el-fin-del-IFE-aparecera-el-Ingreso-Universal-Quienes-lo-cobraran-20200714-0017.html}{que
trabajen en la economía informal}, que beneficia actualmente a casi
cuatro millones de niños. Lejos de avanzar contra esos derechos, el
gobierno del conservador Mauricio Macri
\href{https://www.lanacion.com.ar/politica/el-gobierno-amplia-la-asignacion-por-hijo-y-anuncia-baja-del-iva-nid1889924}{amplió
ese beneficio} en 2016 a familias cuyos padres trabajan por cuenta
propia (llamados monotributistas). Este año, en medio de la pandemia, el
presidente Alberto Fernández
\href{https://www.pagina12.com.ar/277812-el-ife-llego-a-los-sectores-mas-sumergidos-e-invisibilizados}{lanzó
un Ingreso Familiar de Emergencia (IFE)} para trabajadores informales
que alcanzó en pocas semanas a casi nueve millones de personas. La
continuidad, en general, es más pragmática que ideológica: ningún
político desea ser víctima del próximo ``2001''.

Sin embargo, ese apoyo amplio a las políticas de asistencia social no es
explícito. Al contrario, la ``grieta'', como se le conoce a la profunda
polarización que divide a la política argentina, nace precisamente de un
\href{https://www.france24.com/es/20191023-kirchneristas-antikirchneristas-fin-grieta-argentina}{conflicto
en 2008} en torno a la distribución de la renta del sector agropecuario,
el más dinámico y competitivo de la economía nacional. Desde entonces,
la dirigencia actúa separada en polos que parecen irreconciliables: la
coalición peronista ahora gobernante dice representar solidaridad y
distribución; la oposición centroderechista que llevó a Macri al poder
se agrupa alrededor de la libertad y el mérito. Tanto Macri como
Fernández prometieron ``unir a los argentinos'', pero en la práctica los
sectores más duros de sus coaliciones alimentan las divisiones, y les
sacan provecho.

Los argentinos, en tanto, encontramos en la disputa abierta una forma de
canalizar las frustraciones cada vez más acuciantes de una economía que
no logra encausarse. Paradójicamente, la ``grieta'' parece protegernos
de otro ``2001'': ya no queremos que se vayan todos, sino solo los
otros.

\includegraphics{https://static01.graylady3jvrrxbe.onion/images/2020/07/28/multimedia/28garcia-timerman-ES-1/merlin_174411642_a25cae0e-90a5-4ccc-80b4-139fd0440c60-articleLarge.jpg?quality=75\&auto=webp\&disable=upscale}

El horizonte de un modelo con mayor cooperación parece cercano y lejano
a la vez. La pandemia brindó un ejemplo de acuerdo entre Fernández, el
jefe de gobierno de la Ciudad de Buenos Aires, el opositor Horacio
Rodríguez Larreta, y el gobernador de la provincia de Buenos Aires, Axel
Kicillof, de la coalición de Fernández. El trabajo conjunto entre los
tres líderes, que gestionan el mayor foco de contagio del país,
consiguió un inusitado apoyo social y logró trascender las grietas
ideológicas, a fuerza de seguir los consejos de un
\href{https://www.perfil.com/noticias/politica/coronavirus-medicos-quien-es-quien-en-el-comite-de-expertos-que-asesora-a-alberto-fernandez.phtml}{comité
de expertos científicos}. La sociedad argentina parece dispuesta y
receptiva a escuchar y adoptar soluciones racionales y consensuadas a
problemas complejos.

La vocación dialoguista de Fernández es un comienzo importante, que
deberá concretarse en avances reales y encontrar contrapartes
constructivas en el resto del espectro político. Los éxitos iniciales de
la política pandémica deberán trasladarse al resto de la gestión y
sostenerse a lo largo del tiempo.

La agenda posible es amplia. Para empezar, la dirigencia tiene que
acordar cómo va a sostener fiscalmente a la red de sustento social que
construyó en estos años. Para eso, cada bando tiene que poder
interactuar con contrapartes que, más allá de las diferencias, también
quieren que al país le vaya bien.

La condición para que se pueda alcanzar ese acuerdo mínimo es que se
abandonen los intentos recurrentes de perseguir judicialmente al
adversario sin que ello signifique renunciar a la búsqueda de justicia.
Un proyecto de
\href{https://www.nytimes3xbfgragh.onion/es/2020/06/30/espanol/opinion/espionaje-argentina.html}{reforma
judicial} que se va a discutir en las próximas semanas puede ser una
buena oportunidad de dar un primer paso.

A casi 20 años de 2001, nuestros políticos deben buscar respuestas,
coincidencias y soluciones para problemas acumulados. No hacerlo podría
hacer que la historia se repita y que, otra vez, todos se tengan que ir.

Marcelo J. García es analista argentino. Jordana Timerman es editora del
Latin America Daily Briefing.

Advertisement

\protect\hyperlink{after-bottom}{Continue reading the main story}

\hypertarget{site-index}{%
\subsection{Site Index}\label{site-index}}

\hypertarget{site-information-navigation}{%
\subsection{Site Information
Navigation}\label{site-information-navigation}}

\begin{itemize}
\tightlist
\item
  \href{https://help.nytimes3xbfgragh.onion/hc/en-us/articles/115014792127-Copyright-notice}{©~2020~The
  New York Times Company}
\end{itemize}

\begin{itemize}
\tightlist
\item
  \href{https://www.nytco.com/}{NYTCo}
\item
  \href{https://help.nytimes3xbfgragh.onion/hc/en-us/articles/115015385887-Contact-Us}{Contact
  Us}
\item
  \href{https://www.nytco.com/careers/}{Work with us}
\item
  \href{https://nytmediakit.com/}{Advertise}
\item
  \href{http://www.tbrandstudio.com/}{T Brand Studio}
\item
  \href{https://www.nytimes3xbfgragh.onion/privacy/cookie-policy\#how-do-i-manage-trackers}{Your
  Ad Choices}
\item
  \href{https://www.nytimes3xbfgragh.onion/privacy}{Privacy}
\item
  \href{https://help.nytimes3xbfgragh.onion/hc/en-us/articles/115014893428-Terms-of-service}{Terms
  of Service}
\item
  \href{https://help.nytimes3xbfgragh.onion/hc/en-us/articles/115014893968-Terms-of-sale}{Terms
  of Sale}
\item
  \href{https://spiderbites.nytimes3xbfgragh.onion}{Site Map}
\item
  \href{https://help.nytimes3xbfgragh.onion/hc/en-us}{Help}
\item
  \href{https://www.nytimes3xbfgragh.onion/subscription?campaignId=37WXW}{Subscriptions}
\end{itemize}
