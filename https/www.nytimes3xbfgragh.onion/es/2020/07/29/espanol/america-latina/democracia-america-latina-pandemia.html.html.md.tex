Sections

SEARCH

\protect\hyperlink{site-content}{Skip to
content}\protect\hyperlink{site-index}{Skip to site index}

\href{https://www.nytimes3xbfgragh.onion/es/section/america-latina}{América
Latina}

\href{https://myaccount.nytimes3xbfgragh.onion/auth/login?response_type=cookie\&client_id=vi}{}

\href{https://www.nytimes3xbfgragh.onion/section/todayspaper}{Today's
Paper}

\href{/es/section/america-latina}{América Latina}\textbar{}América
Latina se enfrenta a un retroceso de la democracia durante la pandemia

\url{https://nyti.ms/30WaGFv}

\begin{itemize}
\item
\item
\item
\item
\item
\item
\end{itemize}

\href{https://www.nytimes3xbfgragh.onion/es/spotlight/coronavirus?action=click\&pgtype=Article\&state=default\&region=TOP_BANNER\&context=storylines_menu}{El
brote de coronavirus}

\begin{itemize}
\tightlist
\item
  \href{https://www.nytimes3xbfgragh.onion/es/interactive/2020/espanol/mundo/coronavirus-en-estados-unidos.html?action=click\&pgtype=Article\&state=default\&region=TOP_BANNER\&context=storylines_menu}{Mapa
  y casos en EE. UU.}
\item
  \href{https://www.nytimes3xbfgragh.onion/es/2020/07/23/espanol/america-latina/bolivia-cloro-coronavirus-ivermectina.html?action=click\&pgtype=Article\&state=default\&region=TOP_BANNER\&context=storylines_menu}{Dióxido
  de cloro, ivermectina y más: ¿funcionan?}
\item
  \href{https://www.nytimes3xbfgragh.onion/es/interactive/2020/science/coronavirus-tratamientos-curas.html?action=click\&pgtype=Article\&state=default\&region=TOP_BANNER\&context=storylines_menu}{Fármacos
  y tratamientos}
\item
  \href{https://www.nytimes3xbfgragh.onion/es/2020/07/28/espanol/ciencia-y-tecnologia/anticuerpos-coronavirus-inmunidad.html?action=click\&pgtype=Article\&state=default\&region=TOP_BANNER\&context=storylines_menu}{Anticuerpos
  e inmunidad}
\item
  \href{https://www.nytimes3xbfgragh.onion/es/2020/04/29/espanol/estilos-de-vida/oximetro-para-que-sirve.html?action=click\&pgtype=Article\&state=default\&region=TOP_BANNER\&context=storylines_menu}{Oxímetros}
\end{itemize}

Advertisement

\protect\hyperlink{after-top}{Continue reading the main story}

Supported by

\protect\hyperlink{after-sponsor}{Continue reading the main story}

\hypertarget{amuxe9rica-latina-se-enfrenta-a-un-retroceso-de-la-democracia-durante-la-pandemia}{%
\section{América Latina se enfrenta a un retroceso de la democracia
durante la
pandemia}\label{amuxe9rica-latina-se-enfrenta-a-un-retroceso-de-la-democracia-durante-la-pandemia}}

El coronavirus está perjudicando los sistemas de salud y las economías
de la región. También amenaza sus frágiles libertades políticas.

\includegraphics{https://static01.graylady3jvrrxbe.onion/images/2020/07/16/world/29LatAm-Democracy-Es-00/00latam-top-articleLarge.jpg?quality=75\&auto=webp\&disable=upscale}

Por Anatoly Kurmanaev

\begin{itemize}
\item
  29 de julio de 2020
\item
  \begin{itemize}
  \item
  \item
  \item
  \item
  \item
  \item
  \end{itemize}
\end{itemize}

\href{https://www.nytimes3xbfgragh.onion/2020/07/29/world/americas/latin-america-democracy-pandemic.html}{Read
in English}

\href{https://www.nytimes3xbfgragh.onion/newsletters/el-times}{Regístrate
para recibir nuestro boletín} con lo mejor de The New York Times.

\begin{center}\rule{0.5\linewidth}{\linethickness}\end{center}

CARACAS--- Elecciones pospuestas. Tribunales marginados. Persecución a
los opositores.

A medida que la pandemia de coronavirus arrasa América Latina y el
Caribe, al cobrar la vida de más de 180.000 personas y destruir el medio
de sustento de decenas de millones en la región, también socava las
normas democráticas que ya se encontraban bajo presión.

Desde el centro-derecha hasta la extrema izquierda, los líderes han
usado la crisis como excusa para extender sus mandatos, debilitar la
vigilancia a las acciones gubernamentales y acallar a los críticos,
medidas que bajo otras circunstancias serían descritas como autoritarias
y antidemocráticas pero que ahora se presentan como acciones salvadoras,
necesarias para detener la propagación de la enfermedad.

El debilitamiento gradual de las normas democráticas durante una crisis
económica y la catástrofe de salud pública podrían dejar a América
Latina condenada a un crecimiento más lento y un aumento de la
corrupción y los abusos contra los derechos humanos, advirtieron los
expertos. Esto es particularmente cierto en los lugares donde los
derechos políticos y la rendición de cuentas ya estaban en un declive
precipitado.

``No es un asunto de izquierda o de derecha, es una disminución de la
democracia en general en toda la región'', dijo Alessandra Pinna,
investigadora para América Latina en Freedom House, una organización con
sede en Washington que monitorea las libertades políticas globales.

Actualmente hay cinco países en América Latina y el Caribe con historias
democráticas recientes ---Venezuela, Nicaragua, Guyana, Bolivia y
Haití--- donde los gobiernos no fueron electos en elecciones libres y
justas o han extendido su mandato. Es la mayor cantidad desde finales de
los años ochenta, cuando la Guerra Fría se encaminaba a su final y
varios países que se encontraban inmersos en guerras civiles o bajo
dictaduras militares hicieron transiciones hacia la paz y la democracia.

\includegraphics{https://static01.graylady3jvrrxbe.onion/images/2020/07/16/world/29LatAm-Democracy-Es-01/merlin_164809362_1399db40-307c-4b40-a421-c071e0f7a183-articleLarge.jpg?quality=75\&auto=webp\&disable=upscale}

La mayoría de estos líderes ya manipulaban las reglas democráticas para
permanecer en el poder antes de la pandemia, pero aprovecharon las
condiciones de emergencia creadas por la propagación del virus para
fortalecer su posición.

El presidente de Venezuela, Nicolás Maduro, detuvo o allanó las casas de
decenas de periodistas, activistas sociales y líderes de la oposición
que han cuestionado las dudosas cifras del gobierno sobre el
coronavirus.

En Nicaragua, el presidente Daniel Ortega liberó a miles de presos
debido a la amenaza que representa el virus, pero
\href{https://www.barrons.com/news/nicaragua-excludes-political-prisoners-from-mass-release-01586430304}{ha
mantenido tras las rejas} a los presos políticos, mientras que en
Guyana, un bloqueo impidió las protestas contra el intento del gobierno
de mantenerse en el poder a pesar de haber perdido una elección.

En Bolivia, el gobierno interino ha usado la pandemia para posponer las
elecciones, ha recurrido a la ayuda de emergencia para apuntalar su
campaña electoral y ha amenazado con prohibir que el principal candidato
de oposición postule.

Y en las islas de San Cristóbal y Nieves, el gobierno impuso en junio
una cuarentena estricta a sus 50.000 habitantes, durante la campaña para
las elecciones generales, lo que obstaculizó los esfuerzos de la
oposición para llegar a los votantes y al mismo tiempo evitó que los
observadores electorales internacionales viajaran al país.

Fue la primera vez en la historia reciente que a la Organización de
Estados Americanos, un grupo regional que promueve la democracia, se le
retiró la invitación de observar las elecciones.

Image

El gobierno de Añez ha pospuesto las elecciones en Bolivia debido a la
pandemia.Credit...Federico Rios para The New York Times

Aunque la pérdida de la confianza pública en América Latina no es
reciente, la erosión de las normas democráticas durante la pandemia
llegó en un momento en que el crecimiento económico y el progreso social
de la región ya se estaban desmoronando, lo que ha dejado mucha
incertidumbre en torno a la capacidad de los líderes democráticos para
resolver problemas arraigados, como la desigualdad, el crimen y la
corrupción.

En 2018, solo uno de cada cuatro latinoamericanos decía estar satisfecho
con la democracia, el número más bajo desde que Latinobarómetro, una
encuestadora regional, comenzó a hacer esa pregunta hace 25 años.

El descontento con el sistema político llevó en años recientes a una ola
de victorias populistas entre las que se cuentan las de Jair Bolsonaro,
quien se encuentra en la extrema derecha, y Andrés Manuel López Obrador,
de México, posicionado hacia la izquierda. También condujo a protestas
callejeras masivas en varios países latinoamericanos el año pasado.

En este tiempo de convulsión política, la pandemia ha sumido a la región
en la recesión más profunda de su historia, lo que ha exacerbado las
debilidades en los sistemas de bienestar y salud y puesto en evidencia
los muchos modos en que los líderes son incapaces de satisfacer las
demandas de la población.

``Todas las cosas que ya los latinoamericanos estaban pidiendo ---más
igualdad, mejores servicios--- han empeorado dramáticamente con la
pandemia'' dijo Cynthia Arnson, directora de programa para Latinoamérica
del Centro Wilson, un centro de análisis en Washington. ``El sufrimiento
económico es dramático y pone una tensión adicional en instituciones de
por sí débiles''.

Image

Venta de mascarillas en las afueras de Ciudad de Guatemala. La pandemia
ha arrasado la región, dejando más de 180.000 muertos.Credit...Daniele
Volpe para The New York Times

Los sistemas de salud de la región, en dificultades, también se han
visto perjudicados. Latinoamérica se ha convertido en una zona crítica
del virus a nivel global y Brasil, México y Perú se encuentran entre los
10 países con más cantidad de fallecimientos en el mundo. De acuerdo con
las Naciones Unidas, se espera que unos 16 millones de latinoamericanos
caerán a la extrema pobreza este año, un revés a todos los avances de la
región en este siglo.

Además de estos desafíos, la democracia en América Latina también ha
perdido el apoyo de Estados Unidos, que después del fin de la Guerra
Fría había desempeñado un papel importante en la promoción de la
democracia y financió programas de buen gobierno y denunció abusos
autoritarios.

Con el presidente Donald Trump, Estados Unidos se ha enfocado
principalmente en una política exterior regional destinada a oponerse a
los autócratas izquierdistas de Venezuela y Cuba, y en reducir la
inmigración al condicionar la ayuda a los países centroamericanos,
algunos de los más pobres de la región, a la cooperación en materia
migratoria.

El gobierno de Trump también se abstuvo de emitir declaraciones cuando
Nayib Bukele, el
\href{https://www.nytimes3xbfgragh.onion/es/2020/05/06/espanol/america-latina/bukele-el-salvador-virus.html}{presidente
de El Salvador,} ignoró los fallos de la Corte Suprema y usó al ejército
para implementar medidas enérgicas en contra de los infractores de la
cuarentena establecida durante la pandemia.

El apoyo estadounidense a las iniciativas a favor de la democracia en
América Latina se redujo a casi la mitad el año pasado, a 326 millones
de dólares, según cifras preliminares compiladas por la Agencia de los
Estados Unidos para el Desarrollo Internacional (USAID por su sigla en
inglés).

``En los últimos años, no solo hemos abandonado nuestro papel como
fuerza democratizadora en América Latina y en el mundo sino que hemos
promovido fuerzas negativas'', dijo Orlando Pérez, politólogo de la
Universidad del Norte de Texas. ``Nuestra política ahora es: `Están
solos, Estados Unidos es primero'''.

Image

Los partidarios del presidente David Granger de Guyana celebraron
después de la elección en marzo. Granger perdió pero se negó a
renunciar.Credit...Adriana Loureiro Fernandez para The New York Times

En los pocos bastiones democráticos en América Latina, como Uruguay y
Costa Rica, los líderes respondieron a la pandemia con eficiencia y
transparencia, lo que aumentó la confianza de la gente en el gobierno.
En República Dominicana y Surinam, los presidentes en funciones
recientemente se retiraron del poder después de perder las elecciones,
que se celebraron a pesar de la pandemia.

En muchos casos, los jueces y otros funcionarios públicos han resistido
a los ataques contra instituciones democráticas durante la pandemia,
dijo Javier Corrales, profesor de estudios latinoamericanos en el
Amherst College de Massachusetts. ``Los defensores de la democracia
liberal en América Latina no han sido derrotados'', dijo Corrales. ``Los
aspirantes a autócratas no tienen el camino libre''.

Sin embargo, en la mayoría de las naciones latinoamericanas, el
coronavirus aceleró el declive democrático ya existente al dejar en
evidencia la debilidad y la corrupción de los gobiernos ante la
catástrofe.

``Al enfrentarse a una amenaza existencial, los países que aún no tenían
sistemas democráticos profundos recurren a tácticas que ayudan a los
líderes a consolidar su poder'`, dijo John Polga-Hacimovich, politólogo
de la Academia Naval Estadounidense en Maryland.

Las tensiones políticas que aquejan a la región durante la pandemia
podrían ser solo el comienzo de una ola más prolongada de disturbios y
autoritarismo, dijo Thomas Carothers, miembro de la Fundación Carnegie
para la Paz Internacional. ``Va a arrastrar a la región a un peor
desempeño económico'', dijo. ``También significa un peor trato a los
seres humanos, su dignidad y sus derechos''.

\begin{center}\rule{0.5\linewidth}{\linethickness}\end{center}

Advertisement

\protect\hyperlink{after-bottom}{Continue reading the main story}

\hypertarget{site-index}{%
\subsection{Site Index}\label{site-index}}

\hypertarget{site-information-navigation}{%
\subsection{Site Information
Navigation}\label{site-information-navigation}}

\begin{itemize}
\tightlist
\item
  \href{https://help.nytimes3xbfgragh.onion/hc/en-us/articles/115014792127-Copyright-notice}{©~2020~The
  New York Times Company}
\end{itemize}

\begin{itemize}
\tightlist
\item
  \href{https://www.nytco.com/}{NYTCo}
\item
  \href{https://help.nytimes3xbfgragh.onion/hc/en-us/articles/115015385887-Contact-Us}{Contact
  Us}
\item
  \href{https://www.nytco.com/careers/}{Work with us}
\item
  \href{https://nytmediakit.com/}{Advertise}
\item
  \href{http://www.tbrandstudio.com/}{T Brand Studio}
\item
  \href{https://www.nytimes3xbfgragh.onion/privacy/cookie-policy\#how-do-i-manage-trackers}{Your
  Ad Choices}
\item
  \href{https://www.nytimes3xbfgragh.onion/privacy}{Privacy}
\item
  \href{https://help.nytimes3xbfgragh.onion/hc/en-us/articles/115014893428-Terms-of-service}{Terms
  of Service}
\item
  \href{https://help.nytimes3xbfgragh.onion/hc/en-us/articles/115014893968-Terms-of-sale}{Terms
  of Sale}
\item
  \href{https://spiderbites.nytimes3xbfgragh.onion}{Site Map}
\item
  \href{https://help.nytimes3xbfgragh.onion/hc/en-us}{Help}
\item
  \href{https://www.nytimes3xbfgragh.onion/subscription?campaignId=37WXW}{Subscriptions}
\end{itemize}
