Sections

SEARCH

\protect\hyperlink{site-content}{Skip to
content}\protect\hyperlink{site-index}{Skip to site index}

\href{https://www.nytimes3xbfgragh.onion/es/section/ciencia-y-tecnologia}{Ciencia
y Tecnología}

\href{https://myaccount.nytimes3xbfgragh.onion/auth/login?response_type=cookie\&client_id=vi}{}

\href{https://www.nytimes3xbfgragh.onion/section/todayspaper}{Today's
Paper}

\href{/es/section/ciencia-y-tecnologia}{Ciencia y
Tecnología}\textbar{}Las mascarillas pueden reducir la dosis viral,
afirman algunos expertos

\url{https://nyti.ms/31cDJVx}

\begin{itemize}
\item
\item
\item
\item
\item
\end{itemize}

\href{https://www.nytimes3xbfgragh.onion/es/spotlight/coronavirus?action=click\&pgtype=Article\&state=default\&region=TOP_BANNER\&context=storylines_menu}{El
brote de coronavirus}

\begin{itemize}
\tightlist
\item
  \href{https://www.nytimes3xbfgragh.onion/es/interactive/2020/espanol/mundo/coronavirus-en-estados-unidos.html?action=click\&pgtype=Article\&state=default\&region=TOP_BANNER\&context=storylines_menu}{Mapa
  y casos en EE. UU.}
\item
  \href{https://www.nytimes3xbfgragh.onion/es/2020/07/23/espanol/america-latina/bolivia-cloro-coronavirus-ivermectina.html?action=click\&pgtype=Article\&state=default\&region=TOP_BANNER\&context=storylines_menu}{Dióxido
  de cloro, ivermectina y más: ¿funcionan?}
\item
  \href{https://www.nytimes3xbfgragh.onion/es/interactive/2020/science/coronavirus-tratamientos-curas.html?action=click\&pgtype=Article\&state=default\&region=TOP_BANNER\&context=storylines_menu}{Fármacos
  y tratamientos}
\item
  \href{https://www.nytimes3xbfgragh.onion/es/2020/07/28/espanol/ciencia-y-tecnologia/anticuerpos-coronavirus-inmunidad.html?action=click\&pgtype=Article\&state=default\&region=TOP_BANNER\&context=storylines_menu}{Anticuerpos
  e inmunidad}
\item
  \href{https://www.nytimes3xbfgragh.onion/es/2020/04/29/espanol/estilos-de-vida/oximetro-para-que-sirve.html?action=click\&pgtype=Article\&state=default\&region=TOP_BANNER\&context=storylines_menu}{Oxímetros}
\end{itemize}

Advertisement

\protect\hyperlink{after-top}{Continue reading the main story}

Supported by

\protect\hyperlink{after-sponsor}{Continue reading the main story}

\hypertarget{las-mascarillas-pueden-reducir-la-dosis-viral-afirman-algunos-expertos}{%
\section{Las mascarillas pueden reducir la dosis viral, afirman algunos
expertos}\label{las-mascarillas-pueden-reducir-la-dosis-viral-afirman-algunos-expertos}}

Según la evidencia, las personas que usan protección facial captarán
menos partículas de coronavirus, lo que hará que la enfermedad sea menos
grave.

\includegraphics{https://static01.graylady3jvrrxbe.onion/images/2020/07/27/science/29masks-es/merlin_174293325_b383aca7-5ae6-4923-a029-03dd665d16ac-articleLarge.jpg?quality=75\&auto=webp\&disable=upscale}

Por
\href{https://www.nytimes3xbfgragh.onion/by/katherine-j--wu}{Katherine
J. Wu}

\begin{itemize}
\item
  29 de julio de 2020
\item
  \begin{itemize}
  \item
  \item
  \item
  \item
  \item
  \end{itemize}
\end{itemize}

\href{https://www.nytimes3xbfgragh.onion/2020/07/27/health/coronavirus-mask-protection.html}{Read
in English}

\href{https://www.nytimes3xbfgragh.onion/newsletters/el-times}{Regístrate
para recibir nuestro boletín} con lo mejor de The New York Times.

\begin{center}\rule{0.5\linewidth}{\linethickness}\end{center}

Desde hace tiempo los investigadores saben que las mascarillas
\href{https://journals.plos.org/plospathogens/article?id=10.1371/journal.ppat.1003205}{pueden
evitar que las personas propaguen los gérmenes} de sus vías
respiratorias a otros, eso ha impulsado gran parte de la conversación en
torno a estos accesorios que se han vuelto cruciales durante la pandemia
del coronavirus.

Pero ahora, a medida que los casos continúan
\href{https://www.nytimes3xbfgragh.onion/es/interactive/2020/espanol/mundo/coronavirus-en-estados-unidos.html}{aumentando
en todo Estados Unidos}, los expertos han apuntado a una serie de
pruebas que sugieren que los cubrebocas
\href{https://pubmed.ncbi.nlm.nih.gov/23498357/}{también protegen a las
personas que los usan}, al disminuir la gravedad de los síntomas o, en
algunos casos, al evitar por completo la infección.

Diferentes tipos de mascarillas ``bloquean el virus en grados
diferentes, pero todas bloquean la entrada del virus'', dijo Monica
Gandhi, especialista en enfermedades infecciosas de la Universidad de
California en San Francisco. Si las partículas de virus traspasan estas
barreras, dijo, la enfermedad podría ser más leve.

Gandhi y sus colegas formulan este argumento en un
\href{https://ucsf.app.box.com/s/blvolkp5z0mydzd82rjks4wyleagt036}{nuevo
artículo} que se publicará en el Journal of General Internal Medicine. A
partir de experimentos con animales, y la observación de varios eventos
sucedidos durante la pandemia, sostienen que las personas que usan
cubrebocas absorberán menos partículas de coronavirus, lo que facilitará
que sus sistemas inmunitarios puedan combatir intrusos.

Tsion Firew, médica de emergencias de la Universidad de Columbia que no
participó en el proyecto, advirtió que los vínculos entre el uso de
mascarillas y un grado más leve de la enfermedad aún no han sido
demostrados como una relación de causa y efecto. Sin embargo, el nuevo
documento ``reitera lo que decimos sobre los cubrebocas'', dijo. ``No es
solo un acto desinteresado''.

Hay nociones sobre la importancia de la dosis viral en el desarrollo de
las enfermedades en la literatura médica desde
\href{https://academic.oup.com/aje/article-abstract/27/3/493/99616}{al
menos la década de 1930}, cuando dos investigadores notaron formalmente
que los ratones expuestos a grandes cantidades de gérmenes tenían más
probabilidades de morir. Más recientemente, los científicos han llegado
a \href{https://pubmed.ncbi.nlm.nih.gov/25416753/}{rociaron diferentes
cantidades de un virus de la gripe} dentro de las narices de voluntarios
humanos. Descubrieron que cuanto más virus había en el dispositivo nasal
utilizado, más probable era que los participantes se infectaran y
experimentasen síntomas.

Este tipo de experimento no se puede hacer éticamente con el nuevo
coronavirus, dado lo peligroso que es. Pero a principios de este año, un
equipo de investigadores en China probó algo similar en los hámsters:
alojaron en jaulas adyacentes animales sanos e infectados con
coronavirus, algunos de los cuales estaban separados por tampones hechos
de mascarillas quirúrgicas. Muchos de los hámsters sanos detrás de las
particiones nunca se infectaron. Y los desafortunados animales que sí lo
hicieron,
\href{https://academic.oup.com/cid/article/doi/10.1093/cid/ciaa644/5848814}{se
enfermaron menos} que sus vecinos ``sin mascarilla''.

También se han recabado algunos datos indirectos de las personas. Los
investigadores han estimado tentativamente que alrededor del 40 por
ciento de las infecciones por coronavirus
\href{https://www.cdc.gov/coronavirus/2019-ncov/hcp/planning-scenarios.html}{no
producen ningún síntoma}. Pero cuando algunas personas
\href{https://www.oregonlive.com/coronavirus/2020/06/big-coronavirus-outbreak-at-newport-seafood-plants-is-contained-health-authorities-say.html}{usan
mascarillas}, la proporción de casos asintomáticos
\href{https://apnews.com/4b9d38f206db9ce5267a5898ac24f238}{parece
dispararse}: en un brote en una planta procesadora de mariscos en Oregon
superaron el 90 por ciento. El uso de protección facial no hace que las
personas sean inmunes al contagio, pero estas tendencias de casos
asintomáticos podrían sugerir que los cubrebocas logran que la
enfermedad sea más leve, lo que podría reducir las hospitalizaciones y
los fallecimientos.

Gandhi afirma que los datos de los cruceros, que reúnen a grupos
numerosos de personas en lugares cerrados, resultan particularmente
convincentes. La especialista asevera que
\href{https://www.ncbi.nlm.nih.gov/pmc/articles/PMC7078829/}{más del 80
por ciento} de los contagiados
\href{https://www.nytimes3xbfgragh.onion/es/2020/03/10/espanol/mundo/coronavirus-crucero.html}{en
el Diamond Princess,} que en febrero se declaró en cuarentena en Japón,
antes de que el uso de mascarillas se convirtiera en una práctica común,
presentó síntomas. Pero en otro barco que salió de Argentina en marzo, y
en el que a todos los pasajeros se les dieron mascarillas quirúrgicas
después de que a alguien le dio fiebre, el nivel de casos sintomáticos
fue \href{https://thorax.bmj.com/content/75/8/693}{inferior al 20 por
ciento}.

Algunos expertos independientes dicen que el documento es una buena
actualización, debido a la idea generalizada de que usar cubrebocas es
un acto mayormente altruista.

``Una gran deficiencia en los mensajes sobre el uso de cubrebocas ha
sido decir que solo protege a los demás'', dijo Charles Haas, ingeniero
ambiental y experto en evaluación de riesgos en la Universidad de
Drexel. ``Desde el principio, eso nunca tuvo sentido desde el punto de
vista científico''.

También en otros entornos, desde
\href{https://jamanetwork.com/journals/jama/fullarticle/2768533}{hospitales}
hasta
\href{https://www.nytimes3xbfgragh.onion/2020/07/14/health/coronavirus-hair-salon-masks.html}{salones
de belleza}, las protecciones faciales podrían haber reducido las tasas
de contagio general, y tal vez evitado brotes desastrosos. Además,
países como Japón, Taiwán y Corea del Sur, donde los brotes detonaron
rápidamente el uso generalizado de cubrebocas, lograron controlar la
cantidad de hospitalizaciones y muertes relacionadas con el coronavirus
desde el principio.

Incluso en Estados Unidos, la tendencia lenta pero ascendente en el uso
de mascarillas ha coincidido con lo que parece ser una
\href{https://www.nytimes3xbfgragh.onion/2020/07/03/health/coronavirus-mortality-testing.html}{tasa
de letalidad más modesta}, en comparación con
\href{https://www.nytimes3xbfgragh.onion/es/interactive/2020/espanol/mundo/coronavirus-en-estados-unidos.html}{el
aumento que ocurrió} después de que el virus llegó a América del Norte.
Es probable que estas tendencias también hayan sido alteradas por el
aumento de pruebas, el descenso en la edad promedio de las personas que
contraen el virus y las mejoras en los tratamientos. De cualquier forma,
no perdemos nada con usar cubrebocas, dijo Gandhi.

Aunque aún no se ha comprobado que cubrirse el rostro puede minimizar la
gravedad de la enfermedad, ``tiene todo el sentido'', afirmó Linsey
Marr, experta en transmisión de virus en Virginia Tech. ``Es otro
argumento a favor de usar mascarillas''.

Marr y otros investigadores aún están determinando exactamente cuántos
virus entrantes o salientes bloquean los diferentes tipos de cubrebocas.
Pero según una
\href{https://journals.plos.org/plosone/article?id=10.1371/journal.pone.0002618}{gran
cantidad}de
\href{https://www.nature.com/articles/s41591-020-0843-2}{pruebas hechas
en el pasado} y
\href{https://www.thelancet.com/journals/lancet/article/PIIS0140-6736(20)31142-9/fulltext\#\%20}{observaciones
recientes}, la cantidad que se filtra probablemente sea alta, quizás el
50 por ciento o más en el caso de los aerosoles más grandes que se
esparcen en ambas direcciones, dijo Marr. Ciertos accesorios de
protección, como los respiradores N95, funcionan mejor que otros, pero
incluso los paños más holgados pueden eliminar algunas partículas
virales.

Sin embargo, algunos expertos no están listos para aceptar todas las
ideas sobre la protección bidireccional.

Lo que se describe en el documento de Gandhi ``sigue siendo solo una
teoría y requiere más investigación'', dijo Nancy Leung, epidemióloga de
la Universidad de Hong Kong. Aunque hay evidencia contundente de que las
mascarillas reducen la propagación del virus dentro de una población, es
mucho más difícil precisar cómo las protecciones faciales influyen en
los síntomas, dijo Leung, en parte ``debido a la dificultad para
realizar esos estudios''.

Gandhi reconoció estas limitaciones. Sin embargo, sin un fin previsible
a la pandemia actual, afirmó que la necesidad de cubrebocas solo crece,
especialmente a medida que los investigadores continúan documentando la
capacidad que tiene el virus para propagarse en silencio. Incluso las
personas que no tienen síntomas pueden esparcir el virus en su entorno
cuando estornudan, tosen, cantan, hablan o incluso respiran. Y aquellos
que se enferman pueden ser
\href{https://www.nature.com/articles/s41591-020-0869-5}{más
contagiosos} en los días previos a la aparición de los primeros síntomas
de la enfermedad.

Para frenar esta pandemia las personas deben actuar como si estuvieran
infectadas, ``incluso si te sientes bien'', dijo Gandhi.

Las mascarillas por sí solas no son un sustituto de otras medidas de
salud pública como el distanciamiento físico y la buena higiene. Pero a
diferencia de los largos confinamientos que mantienen a las personas
separadas, proteger nuestros rostros es más fácil y más sostenible, dijo
Gandhi.

Protegerte a tí mismo y a los otros de esta enfermedad mortal, agregó,
``es tan simple como cubrir los dos agujeros en la cara que arrojan el
virus''.

Katherine J. Wu es reportera del Times, donde cubre ciencia y salud.
Tiene un doctorado en microbiología e inmunobiología de la Universidad
de Harvard. \href{https://twitter.com/KatherineJWu}{@KatherineJWu}

\begin{center}\rule{0.5\linewidth}{\linethickness}\end{center}

Advertisement

\protect\hyperlink{after-bottom}{Continue reading the main story}

\hypertarget{site-index}{%
\subsection{Site Index}\label{site-index}}

\hypertarget{site-information-navigation}{%
\subsection{Site Information
Navigation}\label{site-information-navigation}}

\begin{itemize}
\tightlist
\item
  \href{https://help.nytimes3xbfgragh.onion/hc/en-us/articles/115014792127-Copyright-notice}{©~2020~The
  New York Times Company}
\end{itemize}

\begin{itemize}
\tightlist
\item
  \href{https://www.nytco.com/}{NYTCo}
\item
  \href{https://help.nytimes3xbfgragh.onion/hc/en-us/articles/115015385887-Contact-Us}{Contact
  Us}
\item
  \href{https://www.nytco.com/careers/}{Work with us}
\item
  \href{https://nytmediakit.com/}{Advertise}
\item
  \href{http://www.tbrandstudio.com/}{T Brand Studio}
\item
  \href{https://www.nytimes3xbfgragh.onion/privacy/cookie-policy\#how-do-i-manage-trackers}{Your
  Ad Choices}
\item
  \href{https://www.nytimes3xbfgragh.onion/privacy}{Privacy}
\item
  \href{https://help.nytimes3xbfgragh.onion/hc/en-us/articles/115014893428-Terms-of-service}{Terms
  of Service}
\item
  \href{https://help.nytimes3xbfgragh.onion/hc/en-us/articles/115014893968-Terms-of-sale}{Terms
  of Sale}
\item
  \href{https://spiderbites.nytimes3xbfgragh.onion}{Site Map}
\item
  \href{https://help.nytimes3xbfgragh.onion/hc/en-us}{Help}
\item
  \href{https://www.nytimes3xbfgragh.onion/subscription?campaignId=37WXW}{Subscriptions}
\end{itemize}
