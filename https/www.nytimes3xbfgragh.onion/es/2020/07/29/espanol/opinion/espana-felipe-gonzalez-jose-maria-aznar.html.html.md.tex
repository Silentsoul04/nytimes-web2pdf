Sections

SEARCH

\protect\hyperlink{site-content}{Skip to
content}\protect\hyperlink{site-index}{Skip to site index}

\href{https://www.nytimes3xbfgragh.onion/es/section/opinion}{Opinión}

\href{https://myaccount.nytimes3xbfgragh.onion/auth/login?response_type=cookie\&client_id=vi}{}

\href{https://www.nytimes3xbfgragh.onion/section/todayspaper}{Today's
Paper}

\href{/es/section/opinion}{Opinión}\textbar{}Manual de comportamiento
para expresidentes insoportables

\url{https://nyti.ms/3faVNEm}

\begin{itemize}
\item
\item
\item
\item
\item
\end{itemize}

Advertisement

\protect\hyperlink{after-top}{Continue reading the main story}

\href{/es/section/opinion}{Opinión}

Supported by

\protect\hyperlink{after-sponsor}{Continue reading the main story}

Comentario

\hypertarget{manual-de-comportamiento-para-expresidentes-insoportables}{%
\section{Manual de comportamiento para expresidentes
insoportables}\label{manual-de-comportamiento-para-expresidentes-insoportables}}

La incapacidad de algunos exmandatarios de España de aceptar su
jubilación viene en parte de una falta de cultura democrática. Algunos
países de Latinoamérica tienen el mismo problema.

\includegraphics{https://static01.graylady3jvrrxbe.onion/images/2020/07/29/multimedia/29Jimenez-ES/29Jimenez-ES-articleLarge.jpg?quality=75\&auto=webp\&disable=upscale}

Por David Jiménez

Es colaborador regular de The New York Times.

\begin{itemize}
\item
  29 de julio de 2020
\item
  \begin{itemize}
  \item
  \item
  \item
  \item
  \item
  \end{itemize}
\end{itemize}

\href{https://www.nytimes3xbfgragh.onion/newsletters/el-times}{Regístrate
para recibir nuestro boletín} con lo mejor de The New York Times.

\begin{center}\rule{0.5\linewidth}{\linethickness}\end{center}

MADRID --- Celos hacia sus sucesores, intromisión en la política
nacional y dificultad crónica para despegarse de lo que consideran suyo.
A los expresidentes les cuesta tanto dejar el poder que en España
suavizamos su partida con un sueldo vitalicio, despacho con asistente,
coche oficial y servicio de seguridad. Y, sin embargo, siguen tan
desubicados como cuando Felipe González, del Partido Socialista Obrero
Español (PSOE), los
\href{https://www.lavozdegalicia.es/noticia/espana/2016/06/03/dos-jarrones-chinos-estorban-partidos-suman-votos/0003_201606G3P18993.htm}{comparó}
con jarrones chinos: ``Se supone que tienen valor y nadie se atreve a
tirarlos a la basura, pero en realidad estorban en todas partes''.

El más veterano de los expresidentes españoles y el hombre que lo
sucedió, José María Aznar, del Partido Popular (PP), son dos arquetipos
de exlíderes incomprendidos. Aunque opuestos ideológicamente, y antiguos
adversarios acérrimos, comparten su apego por la intriga política, una
irresistible tendencia a aleccionar al país y la vanidad de creer que su
tiempo no pasará nunca. El legítimo derecho de los expresidentes a
opinar, cuando va acompañado de un intento de seguir influyendo en el
rumbo del país, limita la renovación de personas e ideas, alarga
personalismos prescindibles y resta dinamismo a las democracias que los
padecen.

Los dos principales jarrones chinos de la política española podrían
haber escogido el camino contrario de la utilidad. El manual del buen
expresidente, que nadie ha escrito, les aconsejaría que se transformaran
en estadistas independientes, representaran a su país con dignidad en el
exterior y aprovecharan los privilegios de una jubilación pagada por el
contribuyente para priorizar los intereses nacionales sobre los
personales. En su lugar han vivido una \emph{jubilación} llena de
incoherencias morales: a la vez que seguían medrando en la política
nacional, y especialmente en la de sus partidos, cobraban en
\href{https://www.abc.es/espana/20150729/abci-trabajos-presidentes-espana-zapatero-201507281742.html}{consejos
de administración} de empresas que sus gobiernos supervisaron en su día.

Pero quizá nada les une más que su aversión a la autocrítica sobre unas
presidencias que impulsaron el desarrollo de España, pero también
dejaron el pesado lastre de un modelo económico sostenido en el
\href{https://www.nytimes3xbfgragh.onion/es/2019/08/05/espanol/opinion/turismo-espana.html}{ladrillo
y un turismo de bajo coste}, una partitocracia ineficaz que ha
parasitado las instituciones y un cultura de la corrupción que terminó
con algunos de sus
\href{https://www.eldiario.es/politica/sumarios-ministros-aznar-llegaron-corrupcion_1_2108010.html}{principales
colaboradores} en prisión.

Todo sería más perdonable, y lo bueno más memorable, si González y Aznar
no emplearan tantas energías en hacer oposición a los presidentes que
los sucedieron, incluidos los de su propio partido. González impulsó a
sus candidatos durante años y hoy es el más efectivo opositor del
presidente del Gobierno español, Pedro Sánchez, de su propio partido, a
cuyo gobierno
\href{https://www.elindependiente.com/politica/2020/06/11/felipe-gonzalez-compara-el-gobierno-con-el-camarote-de-los-hermanos-marx/}{compara}
con el ``camarote de los hermanos Marx''. Aznar escogió para sucederlo a
Mariano Rajoy, para después hacerle la vida imposible con constantes
críticas a su gestión. Hoy tiene a su delfín, Pablo Casado, como jefe de
la oposición y ejerce una influencia decisiva en la estrategia del PP a
través de la presidencia de la fundación FAES, un centro de pensamiento
vinculado al partido.

La incapacidad de los expresidentes para aceptar su jubilación con más
deportividad viene en parte de su ego y en parte de una falta de cultura
democrática en España que también existe, y a veces se acentúa, en
algunos países latinoamericanos. El empeño de políticos como
\href{https://www.nytimes3xbfgragh.onion/es/2018/06/07/espanol/opinion/opinion-colombo-alvaro-uribe-colombia-duque.html}{Álvaro
Uribe en Colombia} o Rafael Correa en Ecuador por aferrarse al poder, en
persona o a través de intermediarios, polariza sus sociedades y
obstaculiza el legítimo intento de sus sucesores de seguir su propio
camino. Mientras Correa sabotea a Lenín Moreno desde Bélgica ---una
tarea que el presidente ecuatoriano hace sencilla con
\href{https://www.nytimes3xbfgragh.onion/es/2020/07/08/espanol/opinion/ecuador-lenin-moreno.html}{su
gestión}---, el senador Uribe se ha convertido en una presencia tóxica y
permanente en la política colombiana.

Los liderazgos personalistas iberoamericanos a menudo evolucionan hacia
movimientos políticos destinados a eternizarse, se llamen uribismo o
correísmo. Son seguidos y detestados con el fervor de un equipo de
fútbol y bloquean la renovación política necesaria para impulsar
reformas. ¿Por qué no una retirada honrosa, la experiencia puesta al
servicio de opiniones constructivas y más tiempo dedicado a la
escritura, como
\href{https://www.semana.com/semana-tv/vicky-en-semana/articulo/piedad-cordoba-explico-por-que-uribe-y-petro-se-deben-retirar--colombia-hoy/687160}{sugiere}
para Uribe la exsenadora colombiana Piedad Córdoba?

El contraste es evidente con democracias más asentadas, donde la
alternancia del poder se asume con naturalidad. Si apenas sabemos nada
de los estrellados primeros ministros David Cameron o Theresa May en el
Reino Unido es porque se han esforzado por hacerse invisibles. Incluso
un líder que generó tanta división en Estados Unidos como George W. Bush
mantiene una pulcra posición neutral sobre la política estadounidense,
siguiendo el buen ejemplo de su padre al despedirse de la presidencia
dejando a Bill Clinton
\href{https://www.lavanguardia.com/economia/20200725/482489048270/seguridad-social-cita-previa-covid.html}{una
nota} que decía: ``No dejes que las críticas te desanimen o te aparten
del curso. {[}\ldots{}{]} Tu éxito ahora es el éxito de nuestro país''.

Los verdaderos estadistas entienden que, más allá de ideologías o
partidos políticos, gobernaron para toda la ciudadanía. Y que, de igual
modo, su papel institucional como expresidentes debe ponerse al servicio
del país, no de intereses partidistas. La esperanza de que ese perfil se
imponga en España ha recaído en los dos últimos políticos que
abandonaron el Palacio de la Moncloa, la residencia oficial.

José Luis Rodríguez Zapatero, del PSOE, y Mariano Rajoy, del PP, apenas
intervienen en asuntos nacionales y, si conspiran contra sus sucesores,
lo disimulan mejor. El primero de ellos se ha centrado en una poco
exitosa carrera como mediador internacional, un trabajo en principio
adecuado para alguien con sus contactos y experiencia diplomática. Pero
Zapatero ha malogrado su credibilidad en Venezuela, donde se ha
convertido en un comodín internacional del régimen de Nicolás Maduro.

Mariano Rajoy, expulsado del poder tras una
\href{https://www.nytimes3xbfgragh.onion/es/2018/05/24/espanol/partido-popular-corrupcion-rajoy-gurtel.html}{sentencia}
contra su partido por corrupción en 2018, sorprendió a todos cuando
\href{https://www.elconfidencial.com/espana/2018-10-22/mariano-rajoy-plaza-registro-mercantil-madrid_1633814/}{decidió
solicitar su plaza} como registrador de la propiedad. Desde entonces no
muestra apego a su vida anterior y mantiene una discreción que se rompió
brevemente al descubrirse que
\href{https://www.elconfidencial.com/espana/2020-04-29/policia-mariano-rajoy-ruptura-confinamiento_2573336/}{se
saltó el confinamiento} durante los peores momentos de la pandemia de
coronavirus. Es aún pronto para saber si el desliz es parte del
convencimiento de los expresidentes de que sus privilegios trascienden a
sus mandatos o una prueba de que el político gallego busca con
determinación volver a ser un ciudadano más: la policía
\href{https://www.laprovincia.es/espana/2020/06/20/alarma-cierra-1-millones-multas/1293308.html}{multó}
a otros 1,2 millones de españoles por el mismo motivo.

El papel de los expresidentes, ahora regulado en sus beneficios, debería
concretarse con una reforma legislativa que incluya un marco de
incompatibilidades mientras sigan cobrando del Estado y más concreción
sobre las ocupaciones con las que justifiquen esos sueldos. Pero al
final del día, es el carácter de cada uno lo que determina si se
convierten en un valor para las sociedades que lideraron o en molestos
jarrones chinos. La lección para los votantes es que debemos escoger a
líderes con generosidad democrática, suficiente humildad para alejarlos
de la tentación de hacer sombra a sus sucesores, comprometidos con la
renovación natural de la política y que contribuyan a hacer la
democracia más dinámica, haciéndose a un lado cuando los ciudadanos les
enseñan la puerta de salida.

David Jiménez es escritor y periodista. Su libro más reciente es
\emph{El director}.
\href{https://twitter.com/DavidJimenezTW}{@DavidJimenezTW}

Advertisement

\protect\hyperlink{after-bottom}{Continue reading the main story}

\hypertarget{site-index}{%
\subsection{Site Index}\label{site-index}}

\hypertarget{site-information-navigation}{%
\subsection{Site Information
Navigation}\label{site-information-navigation}}

\begin{itemize}
\tightlist
\item
  \href{https://help.nytimes3xbfgragh.onion/hc/en-us/articles/115014792127-Copyright-notice}{©~2020~The
  New York Times Company}
\end{itemize}

\begin{itemize}
\tightlist
\item
  \href{https://www.nytco.com/}{NYTCo}
\item
  \href{https://help.nytimes3xbfgragh.onion/hc/en-us/articles/115015385887-Contact-Us}{Contact
  Us}
\item
  \href{https://www.nytco.com/careers/}{Work with us}
\item
  \href{https://nytmediakit.com/}{Advertise}
\item
  \href{http://www.tbrandstudio.com/}{T Brand Studio}
\item
  \href{https://www.nytimes3xbfgragh.onion/privacy/cookie-policy\#how-do-i-manage-trackers}{Your
  Ad Choices}
\item
  \href{https://www.nytimes3xbfgragh.onion/privacy}{Privacy}
\item
  \href{https://help.nytimes3xbfgragh.onion/hc/en-us/articles/115014893428-Terms-of-service}{Terms
  of Service}
\item
  \href{https://help.nytimes3xbfgragh.onion/hc/en-us/articles/115014893968-Terms-of-sale}{Terms
  of Sale}
\item
  \href{https://spiderbites.nytimes3xbfgragh.onion}{Site Map}
\item
  \href{https://help.nytimes3xbfgragh.onion/hc/en-us}{Help}
\item
  \href{https://www.nytimes3xbfgragh.onion/subscription?campaignId=37WXW}{Subscriptions}
\end{itemize}
