Sections

SEARCH

\protect\hyperlink{site-content}{Skip to
content}\protect\hyperlink{site-index}{Skip to site index}

\href{https://www.nytimes3xbfgragh.onion/es/section/estados-unidos}{Estados
Unidos}

\href{https://myaccount.nytimes3xbfgragh.onion/auth/login?response_type=cookie\&client_id=vi}{}

\href{https://www.nytimes3xbfgragh.onion/section/todayspaper}{Today's
Paper}

\href{/es/section/estados-unidos}{Estados Unidos}\textbar{}Por qué Trump
no puede postergar las elecciones de 2020

\url{https://nyti.ms/30er8Sr}

\begin{itemize}
\item
\item
\item
\item
\item
\item
\end{itemize}

\begin{itemize}
\item
  \href{https://www.nytimes3xbfgragh.onion/interactive/2020/08/04/us/elections/results-arizona-kansas-michigan-missouri-primaries.html?action=click\&pgtype=Article\&state=default\&region=TOP_BANNER\&context=storylines_menu}{Latest
  Results}
\item
  \href{https://www.nytimes3xbfgragh.onion/article/biden-vice-president-2020.html?action=click\&pgtype=Article\&state=default\&region=TOP_BANNER\&context=storylines_menu}{Biden's
  V.P. Search}
\item
  \href{https://www.nytimes3xbfgragh.onion/interactive/2020/07/24/us/politics/trump-biden-campaign-donors.html?action=click\&pgtype=Article\&state=default\&region=TOP_BANNER\&context=storylines_menu}{Map
  of Donations}
\item
  \href{https://www.nytimes3xbfgragh.onion/interactive/2020/us/elections/delegate-count-primary-results.html?action=click\&pgtype=Article\&state=default\&region=TOP_BANNER\&context=storylines_menu}{Delegate
  Count}
\item
  \href{https://www.nytimes3xbfgragh.onion/interactive/2019/us/politics/2020-presidential-candidates.html?action=click\&pgtype=Article\&state=default\&region=TOP_BANNER\&context=storylines_menu}{The
  Candidates}
\item
  \href{https://www.nytimes3xbfgragh.onion/newsletters/politics?action=click\&pgtype=Article\&state=default\&region=TOP_BANNER\&context=storylines_menu}{Politics
  Newsletter}
\end{itemize}

Advertisement

\protect\hyperlink{after-top}{Continue reading the main story}

Supported by

\protect\hyperlink{after-sponsor}{Continue reading the main story}

Elecciones 2020

\hypertarget{por-quuxe9-trump-no-puede-postergar-las-elecciones-de-2020}{%
\section{Por qué Trump no puede postergar las elecciones de
2020}\label{por-quuxe9-trump-no-puede-postergar-las-elecciones-de-2020}}

Respondemos algunas preguntas clave sobre la realización de elecciones
durante una crisis. Y no, el presidente de Estados Unidos no puede
cancelar una elección por su cuenta.

\includegraphics{https://static01.graylady3jvrrxbe.onion/images/2020/07/30/us/politics/30Trump-elecciones-ES-1/30election-explainer-articleLarge.jpg?quality=75\&auto=webp\&disable=upscale}

\href{https://www.nytimes3xbfgragh.onion/by/alexander-burns}{\includegraphics{https://static01.graylady3jvrrxbe.onion/images/2018/09/25/multimedia/author-alexander-burns/author-alexander-burns-thumbLarge-v2.png}}

Por
\href{https://www.nytimes3xbfgragh.onion/by/alexander-burns}{Alexander
Burns}

\begin{itemize}
\item
  30 de julio de 2020
\item
  \begin{itemize}
  \item
  \item
  \item
  \item
  \item
  \item
  \end{itemize}
\end{itemize}

\href{https://www.nytimes3xbfgragh.onion/2020/07/30/us/politics/trump-postpone-election.html}{Read
in English}

\href{https://www.nytimes3xbfgragh.onion/newsletters/el-times}{Regístrate
para recibir nuestro boletín} con lo mejor de The New York Times.

\begin{center}\rule{0.5\linewidth}{\linethickness}\end{center}

El presidente
\href{https://www.nytimes3xbfgragh.onion/es/interactive/2020/espanol/estados-unidos/donald-trump-elecciones.html}{Donald
Trump}, a quien le ha ido mal en las encuestas en la carrera por la Casa
Blanca,
\href{https://www.nytimes3xbfgragh.onion/2020/07/30/us/elections/biden-vs-trump.html}{sugirió
el jueves} que las elecciones generales del 3 de noviembre podrían
postergarse ``hasta que la gente pueda votar de manera adecuada, segura
y sin percances''. Incluso para él, insinuar la idea de posponer las
elecciones fue una violación extraordinaria del decoro presidencial.

Pero el presidente de Estados Unidos no tiene la autoridad para cambiar
la fecha de una elección federal. Y el otro alegato que Trump hizo el
jueves ---de que una votación generalizada por correo haría que las
elecciones fueran ``inexactas y fraudulentas''--- es falsa.

Aquí están las respuestas a algunas preguntas clave sobre la celebración
de elecciones en una crisis.

\hypertarget{puede-el-presidente-cancelar-o-posponer-una-elecciuxf3n-con-una-orden-ejecutiva}{%
\subsection{¿Puede el presidente cancelar o posponer una elección con
una orden
ejecutiva?}\label{puede-el-presidente-cancelar-o-posponer-una-elecciuxf3n-con-una-orden-ejecutiva}}

No.

\hypertarget{por-quuxe9-no}{%
\subsection{¿Por qué no?}\label{por-quuxe9-no}}

\href{https://constitution.congress.gov/browse/essay/artII-S1-C4-1/ALDE_00000230/}{El
artículo II} de la Constitución de Estados Unidos
\href{https://crsreports.congress.gov/product/pdf/R/R46413}{faculta al
Congreso} a elegir el momento de las elecciones generales.
\href{https://www.loc.gov/law/help/statutes-at-large/28th-congress/session-2/c28s2ch1.pdf}{Una
ley federal de 1845} estableció la fecha como el primer martes después
del primer lunes de noviembre.

Se necesitaría un cambio en la ley federal para cambiar la fecha. Eso
significaría una legislación promulgada por el Congreso, firmada por el
presidente y sujeta a impugnación en los tribunales.

\textbf{{[}¿Hay algo que quieras saber sobre las elecciones
presidenciales en Estados Unidos?}
\textbf{\href{https://www.nytimes3xbfgragh.onion/es/2020/07/30/espanol/estados-unidos/trump-retrasar-elecciones.html\#commentsContainer}{Participa
en nuestra sección de Comentarios}{]}}

\hypertarget{quuxe9-posibilidades-hay-de-que-las-elecciones-de-noviembre-se-retrasen}{%
\subsection{¿Qué posibilidades hay de que las elecciones de noviembre se
retrasen?}\label{quuxe9-posibilidades-hay-de-que-las-elecciones-de-noviembre-se-retrasen}}

¿Mencionamos que la Cámara de Representantes, controlada por el Partido
Demócrata, el Senado, controlado por el Republicano, y el presidente
Trump tendrían que aprobar esa legislación?

Llamarlo improbable sería quedarse corto.

Incluso si todo eso sucediera, no habría mucha flexibilidad para elegir
una fecha de elección alternativa: la Constitución establece que el
nuevo Congreso debe jurar el 3 de enero, y que el mandato del nuevo
presidente debe comenzar el 20 de enero. Esas fechas no se pueden
cambiar simplemente con la aprobación de legislación normal.

Marc Elias, demócrata y destacado abogado de elecciones, rechazó el
jueves la idea de que Trump cambiara las elecciones por su cuenta.

\hypertarget{pero-acaso-muchos-estados-no-pospusieron-sus-elecciones-primarias-este-auxf1o}{%
\subsection{¿Pero acaso muchos estados no pospusieron sus elecciones
primarias este
año?}\label{pero-acaso-muchos-estados-no-pospusieron-sus-elecciones-primarias-este-auxf1o}}

Sí: en respuesta a la pandemia del coronavirus,
\href{https://www.nytimes3xbfgragh.onion/article/2020-campaign-primary-calendar-coronavirus.html}{16
estados y dos territorios} retrasaron sus primarias presidenciales o
extendieron los plazos para votar por correo.

Los estados tienen una amplia autonomía para definir el momento y los
procedimientos para las elecciones primarias. El proceso exacto para
establecer fechas para las primarias varía de estado a estado.

Por ejemplo, en Louisiana, la ley estatal permite al gobernador
reprogramar una elección debido a una emergencia, siempre y cuando el
secretario de Estado local haya certificado que existe una emergencia.
En marzo, el gobernador John Bel Edwards y el secretario de Estado de
Louisiana, R. Kyle Ardoin, hicieron exactamente eso. (De hecho, más
tarde pospusieron las elecciones primarias por segunda vez, lo que le
ganó más tiempo al estado para prepararse y celebrar su votación en
medio de la pandemia).

\hypertarget{han-considerado-los-funcionarios-federales-cambiar-de-fecha-una-elecciuxf3n-general-en-el-pasado}{%
\subsection{¿Han considerado los funcionarios federales cambiar de fecha
una elección general en el
pasado?}\label{han-considerado-los-funcionarios-federales-cambiar-de-fecha-una-elecciuxf3n-general-en-el-pasado}}

Se informó en 2004 que algunos funcionarios del gobierno de George W.
Bush habían discutido la posibilidad de posponer una elección federal en
caso de un ataque terrorista. Pero esa idea fracasó rápidamente, y
Condoleezza Rice, la entonces asesora de seguridad nacional,
\href{https://www.nytimes3xbfgragh.onion/2004/07/13/trail/trail/white-house-tries-to-calm-hubbub-over-vote-delay.html}{dijo}
que Estados Unidos había celebrado elecciones ``cuando estábamos en
guerra, incluso cuando estábamos en guerra civil. Y deberíamos tener las
elecciones a tiempo''.

\hypertarget{quuxe9-pasa-con-los-procedimientos-para-votar-en-las-elecciones-de-noviembre}{%
\subsection{¿Qué pasa con los procedimientos para votar en las
elecciones de
noviembre?}\label{quuxe9-pasa-con-los-procedimientos-para-votar-en-las-elecciones-de-noviembre}}

Si bien la fecha de la elección presidencial está establecida por ley
federal, los procedimientos para votar generalmente se controlan a nivel
estatal.

Es por eso que Estados Unidos tiene un mosaico tan complicado de
regulaciones de votación, con algunos estados que permiten el sufragio
anticipado y a distancia; algunos permiten votar por correo o si el
votante se registra ese mismo día; otros requieren ciertos tipos de
identificación para los votantes; y muchos estados hacen pocas o ninguna
de estas cosas.

Durante la pandemia, varios estados han tratado de facilitar a los
votantes el uso de las boletas por correo, ayudándoles a evitar acudir a
los lugares de votación el día de las elecciones.
\href{https://www.nytimes3xbfgragh.onion/2020/05/20/us/politics/trump-mail-in-voting-absentee-ballots.html}{En
Michigan}, por ejemplo, la secretaria de Estado, Jocelyn Benson, envió
por correo las solicitudes de boletas para sufragar a distancia a todos
los 7,7 millones de votantes registrados para las elecciones primarias
de agosto y las elecciones generales de noviembre.

Incluso antes de este año, cinco estados ---Colorado, Hawái, Oregón,
Utah y Washington--- han llevado a cabo sus elecciones casi
completamente por correo.

Otros estados han tenido problemas en gestionar una avalancha de boletas
para votar a distancia. En Nueva York, los votantes solicitaron cientos
de miles más de boletas para sufragar a distancia que en una elección
típica y
\href{https://www.nytimes3xbfgragh.onion/2020/07/17/nyregion/election-absentee-ballots-primary.html}{los
funcionarios siguen contando los votos} más de un mes después del día de
las primarias. Una elección clave en el Décimosegundo Distrito del
Congreso aún no se ha resuelto.

Eso puede ofrecer un panorama previo de lo
\href{https://www.nytimes3xbfgragh.onion/2020/06/24/us/politics/november-2020-election-day-results.html}{que
podría suceder la noche de las elecciones de noviembre}: a menos que un
candidato gane de forma arrolladora, puede que no haya un ganador claro
e inmediato en la carrera presidencial. Pero eso no significa que la
elección sea fraudulenta, solo que puede tomar más tiempo determinar al
vencedor.

\hypertarget{es-correcta-la-afirmaciuxf3n-de-trump-de-que-votar-por-correo-conduce-al-fraude-electoral}{%
\subsection{¿Es correcta la afirmación de Trump de que votar por correo
conduce al fraude
electoral?}\label{es-correcta-la-afirmaciuxf3n-de-trump-de-que-votar-por-correo-conduce-al-fraude-electoral}}

\href{https://www.nytimes3xbfgragh.onion/article/mail-in-voting-explained.html}{No}.

Numerosos estudios han demostrado que
\href{https://www.nytimes3xbfgragh.onion/article/mail-in-voting-explained.html}{todas
las formas de fraude electoral son muy raras} en Estados Unidos. Un
panel que Trump estableció para investigar la corrupción electoral se
disolvió en 2018 después de que
\href{https://www.nytimes3xbfgragh.onion/2018/01/03/us/politics/trump-voter-fraud-commission.html}{no
encontró evidencia real} de fraude.

Los expertos han dicho que votar por correo es menos seguro que votar en
persona, pero aún así es extremadamente raro ver casos de fraude
electoral.

En Washington, uno de los estados que vota casi completamente por
correo, un estudio realizado por el secretario de Estado, quien es
republicano, encontró que 142 boletas, de más de 3,1 millones emitidas,
eran casos potenciales de votación inadecuada en las elecciones de 2018
y fueron remitidas a los alguaciles y fiscales del condado para acciones
legales. Esto representa aproximadamente el 0,004 por ciento del
electorado.

Uno de los casos más prominentes de fraude se produjo en el Noveno
Distrito del Congreso de Carolina del Norte, donde un agente político
\href{https://www.nytimes3xbfgragh.onion/2019/07/30/us/mccrae-dowless-indictment.html}{fue
acusado} de recolectar y enviar de forma fraudulenta boletas para
sufragar a distancia, en un intento de manipular los resultados de las
elecciones a favor del candidato republicano. Pero es probable que se
detecten esquemas tan ambiciosos como este, dicen los expertos; el
distrito celebró un segundo intento de elección.

Y el propio Trump votó por correo en las últimas elecciones.

Reid J. Epstein y Linda Qiu colaboraron con reportería.

Alexander Burns es un corresponsal de política nacional que cubre las
elecciones y el poder político en todo Estados Unidos, incluida la
campaña de 2016 de Donald Trump. Antes de unirse al Times en 2015,
cubrió las elecciones de 2012 para Politico.
\href{https://twitter.com/alexburnsNYT}{@alexburnsNYT}

\hypertarget{our-2020-election-guide}{%
\section{Our 2020 Election Guide}\label{our-2020-election-guide}}

Updated Aug. 4, 2020

\begin{itemize}
\item
  \begin{center}\rule{0.5\linewidth}{\linethickness}\end{center}

  \hypertarget{the-latest}{%
  \subsection{The Latest}\label{the-latest}}

  \begin{itemize}
  \tightlist
  \item
    Kris Kobach, a polarizing figure in Kansas politics,
    \href{https://www.nytimes3xbfgragh.onion/2020/08/04/us/politics/kobach-tlaib.html?action=click\&pgtype=Article\&state=default\&region=BELOW_MAIN_CONTENT\&context=storylines_guide}{lost
    the Senate primary there}, relieving G.O.P. officials who feared he
    could jeopardize a safe seat.
  \end{itemize}
\item
  \begin{center}\rule{0.5\linewidth}{\linethickness}\end{center}

  \hypertarget{bidens-vp-search}{%
  \subsection{Biden's V.P. Search}\label{bidens-vp-search}}

  \begin{itemize}
  \tightlist
  \item
    \href{https://www.nytimes3xbfgragh.onion/article/biden-vice-president-2020.html?action=click\&pgtype=Article\&state=default\&region=BELOW_MAIN_CONTENT\&context=storylines_guide}{Here
    are 13 women} who have been under consideration to be Joe Biden's
    running mate, and why each might be chosen --- and might not be.
  \end{itemize}
\item
  \begin{center}\rule{0.5\linewidth}{\linethickness}\end{center}

  \hypertarget{keep-up-with-our-coverage}{%
  \subsection{Keep Up With Our
  Coverage}\label{keep-up-with-our-coverage}}

  \begin{itemize}
  \tightlist
  \item
    Get an
    \href{https://www.nytimes3xbfgragh.onion/newsletters/politics?action=click\&pgtype=Article\&state=default\&region=BELOW_MAIN_CONTENT\&context=storylines_guide}{email}
    recapping the day's news
  \end{itemize}

  \begin{itemize}
  \tightlist
  \item
    Download our mobile app on
    \href{https://apps.apple.com/us/app/nytimes/id284862083?ls=1\&mat_click_id=5c79ae7455014fd1bd66b5610c05b8f2-20191112-16948\&referrer=mat_click_id\%3D5c79ae7455014fd1bd66b5610c05b8f2-20191112-16948\%26link_click_id\%3D722930677036718082}{iOS}
    and
    \href{http://a.localytics.com/android?id=com.nytimes.android\&referrer=utm_source\%3Dother_nyt_mobile_web\%26utm_medium\%3DWeb\%2520page\%26utm_term\%3DGeneral\%2520Mobile\%2520Page\%26utm_campaign\%3DNYT\%2520Mobile\%2520General\%2520Page}{Android}
    and turn on Breaking News and Politics alerts
  \end{itemize}
\end{itemize}

Advertisement

\protect\hyperlink{after-bottom}{Continue reading the main story}

\hypertarget{site-index}{%
\subsection{Site Index}\label{site-index}}

\hypertarget{site-information-navigation}{%
\subsection{Site Information
Navigation}\label{site-information-navigation}}

\begin{itemize}
\tightlist
\item
  \href{https://help.nytimes3xbfgragh.onion/hc/en-us/articles/115014792127-Copyright-notice}{©~2020~The
  New York Times Company}
\end{itemize}

\begin{itemize}
\tightlist
\item
  \href{https://www.nytco.com/}{NYTCo}
\item
  \href{https://help.nytimes3xbfgragh.onion/hc/en-us/articles/115015385887-Contact-Us}{Contact
  Us}
\item
  \href{https://www.nytco.com/careers/}{Work with us}
\item
  \href{https://nytmediakit.com/}{Advertise}
\item
  \href{http://www.tbrandstudio.com/}{T Brand Studio}
\item
  \href{https://www.nytimes3xbfgragh.onion/privacy/cookie-policy\#how-do-i-manage-trackers}{Your
  Ad Choices}
\item
  \href{https://www.nytimes3xbfgragh.onion/privacy}{Privacy}
\item
  \href{https://help.nytimes3xbfgragh.onion/hc/en-us/articles/115014893428-Terms-of-service}{Terms
  of Service}
\item
  \href{https://help.nytimes3xbfgragh.onion/hc/en-us/articles/115014893968-Terms-of-sale}{Terms
  of Sale}
\item
  \href{https://spiderbites.nytimes3xbfgragh.onion}{Site Map}
\item
  \href{https://help.nytimes3xbfgragh.onion/hc/en-us}{Help}
\item
  \href{https://www.nytimes3xbfgragh.onion/subscription?campaignId=37WXW}{Subscriptions}
\end{itemize}
