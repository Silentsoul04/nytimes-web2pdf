Sections

SEARCH

\protect\hyperlink{site-content}{Skip to
content}\protect\hyperlink{site-index}{Skip to site index}

\href{https://www.nytimes3xbfgragh.onion/es/section/opinion}{Opinión}

\href{https://myaccount.nytimes3xbfgragh.onion/auth/login?response_type=cookie\&client_id=vi}{}

\href{https://www.nytimes3xbfgragh.onion/section/todayspaper}{Today's
Paper}

\href{/es/section/opinion}{Opinión}\textbar{}Sí, el coronavirus está en
el aire

\url{https://nyti.ms/3gxJi7r}

\begin{itemize}
\item
\item
\item
\item
\item
\end{itemize}

\href{https://www.nytimes3xbfgragh.onion/es/spotlight/coronavirus?action=click\&pgtype=Article\&state=default\&region=TOP_BANNER\&context=storylines_menu}{El
brote de coronavirus}

\begin{itemize}
\tightlist
\item
  \href{https://www.nytimes3xbfgragh.onion/es/interactive/2020/espanol/america-latina/coronavirus-en-mexico.html?action=click\&pgtype=Article\&state=default\&region=TOP_BANNER\&context=storylines_menu}{Mapa
  y casos en México}
\item
  \href{https://www.nytimes3xbfgragh.onion/es/2020/07/31/espanol/ciencia-y-tecnologia/ninos-contagio-coronavirus.html?action=click\&pgtype=Article\&state=default\&region=TOP_BANNER\&context=storylines_menu}{Los
  niños y el virus}
\item
  \href{https://www.nytimes3xbfgragh.onion/es/interactive/2020/science/coronavirus-tratamientos-curas.html?action=click\&pgtype=Article\&state=default\&region=TOP_BANNER\&context=storylines_menu}{Fármacos
  y tratamientos}
\item
  \href{https://www.nytimes3xbfgragh.onion/es/2020/07/06/espanol/ciencia-y-tecnologia/coronavirus-transmision-aire.html?action=click\&pgtype=Article\&state=default\&region=TOP_BANNER\&context=storylines_menu}{Cómo
  se transmite el coronavirus}
\item
  \href{https://www.nytimes3xbfgragh.onion/es/2020/07/14/espanol/estilos-de-vida/botiquin-medicina-coronavirus.html?action=click\&pgtype=Article\&state=default\&region=TOP_BANNER\&context=storylines_menu}{Prepara
  tu botiquín}
\end{itemize}

Advertisement

\protect\hyperlink{after-top}{Continue reading the main story}

\href{/es/section/opinion}{Opinión}

Supported by

\protect\hyperlink{after-sponsor}{Continue reading the main story}

Comentario

\hypertarget{suxed-el-coronavirus-estuxe1-en-el-aire}{%
\section{Sí, el coronavirus está en el
aire}\label{suxed-el-coronavirus-estuxe1-en-el-aire}}

La transmisión por aerosoles es importante, y quizá sea mucho más
relevante de lo que hemos podido comprobar hasta ahora.

\includegraphics{https://static01.graylady3jvrrxbe.onion/images/2020/07/30/opinion/01Marr-ES-1/30Marr-articleLarge.jpg?quality=75\&auto=webp\&disable=upscale}

Por Linsey C. Marr

Es profesora de ingeniería.

\begin{itemize}
\item
  1 de agosto de 2020
\item
  \begin{itemize}
  \item
  \item
  \item
  \item
  \item
  \end{itemize}
\end{itemize}

\href{https://www.nytimes3xbfgragh.onion/2020/07/30/opinion/coronavirus-aerosols.html}{Read
in English}

\href{https://www.nytimes3xbfgragh.onion/newsletters/el-times}{Regístrate
para recibir nuestro boletín} con lo mejor de The New York Times.

\begin{center}\rule{0.5\linewidth}{\linethickness}\end{center}

Por fin. La Organización Mundial de la Salud (OMS) ha reconocido
formalmente que el SARS-CoV-2, el virus que causa la COVID-19,
\href{https://www.nytimes3xbfgragh.onion/2020/07/09/health/virus-aerosols-who.html}{se
transmite por el aire} y que puede transportarse
\href{https://www.nature.com/articles/d41586-020-02058-1}{en partículas
minúsculas de aerosol}.

Cuando tosemos y estornudamos, hablamos o tan solo respiramos,
\href{https://www.sciencedirect.com/science/article/pii/S0021850211001200}{expulsamos
al aire de manera natural} gotículas (pequeñas partículas de fluido) y
aerosoles (partículas más pequeñas de fluido). Sin embargo, hasta
principios de este mes, la OMS ---al igual que los Centros para el
Control y la Prevención de Enfermedades de Estados Unidos o la agencia
Public Health England---
\href{https://www.who.int/news-room/commentaries/detail/modes-of-transmission-of-virus-causing-covid-19-implications-for-ipc-precaution-recommendations}{había}
\href{https://www.who.int/news-room/commentaries/detail/modes-of-transmission-of-virus-causing-covid-19-implications-for-ipc-precaution-recommendations}{advertido
principalmente} sobre la transmisión del nuevo coronavirus mediante el
contacto directo y las gotículas liberadas a una corta distancia.

La organización solo había advertido sobre los aerosoles en
circunstancias extraordinarias, como después de la intubación y otros
\href{https://www.who.int/publications/i/item/WHO-2019-nCoV-IPC-2020.4}{procedimientos
médicos} relacionados con pacientes infectados en hospitales.

Después de
\href{https://www.nature.com/articles/d41586-020-00974-w\#ref-CR5}{varios
meses de insistencia por parte de los científicos}, el 9 de julio, la
OMS cambió su postura, pasó de la negación a
\href{https://www.who.int/news-room/commentaries/detail/transmission-of-sars-cov-2-implications-for-infection-prevention-precautions}{una
aceptación parcial y reticente}: ``Se requieren más estudios para
determinar si es posible detectar el SARS-CoV-2 viable en muestras de
aire tomadas de ambientes donde no se realicen procedimientos que
generen microgotas de aerosol y cómo influyen los aerosoles en la
transmisión''.

Soy una ingeniera civil y ambiental que estudia cómo los virus y las
bacterias se propagan por el aire, también soy
\href{https://www.nytimes3xbfgragh.onion/es/2020/07/06/espanol/ciencia-y-tecnologia/coronavirus-transmision-aire.html}{una
de los 239 científicos} que firmaron
\href{https://academic.oup.com/cid/article/doi/10.1093/cid/ciaa939/5867798}{una
carta abierta} a finales de junio para presionar a la OMS a tomar más en
serio el riesgo de la transmisión aérea.

Un mes después, considero que la transmisión del SARS-CoV-2 por medio de
aerosoles es mucho más importante de lo que se ha reconocido
oficialmente hasta la fecha.

En un \href{https://www.nature.com/articles/s41598-020-69286-3}{estudio
arbitrado} publicado en la revista Nature el 29 de julio, investigadores
del Centro Médico de la Universidad de Nebraska hallaron que las
microgotas de aerosol tomadas de habitaciones de hospital de pacientes
con la COVID-19 contenían el coronavirus.

Esto confirma los resultados de
\href{https://www.medrxiv.org/content/10.1101/2020.05.31.20115154v1}{un
estudio} (no arbitrado) de finales de mayo en el que se descubrió que
los pacientes con la COVID-19 liberaban el SARS-CoV-2 al simplemente
exhalar, sin toser o siquiera hablar. Los autores de ese estudio dijeron
que este hallazgo implicaba que la transmisión aérea ``influye de manera
significativa'' en la propagación del virus.

Aceptar estas conclusiones no cambiaría en mucho las recomendaciones
actuales en cuanto a las prácticas más adecuadas. La mejor protección
contra el SARS-CoV-2, ya sea que esté presente primordialmente en
gotículas o en aerosoles, en esencia es la misma: mantener la distancia
y usar cubrebocas.

Los hallazgos recientes son más bien un recordatorio importante de que
también debemos estar atentos a abrir las ventanas y mejorar la
circulación del aire en interiores. Además, contribuyen a la evidencia
de que la calidad de los cubrebocas y la manera en que se ajustan al
rostro también son factores importantes.

Una ``gotícula'', según la definición de la OMS, es
\href{https://www.who.int/news-room/commentaries/detail/modes-of-transmission-of-virus-causing-covid-19-implications-for-ipc-precaution-recommendations}{una
partícula de más de 5 micrómetros} que no viaja a una distancia mayor a
un metro.

En realidad, no hay un límite claro ni significativo ---ya sean de 5
micrómetros o de cualquier tamaño--- entre las gotículas y los
aerosoles: todas son gotitas minúsculas de líquido y su tamaño varía en
un rango que va de lo muy pequeño a lo microscópico.

(Estoy colaborando con historiadores médicos para identificar los
fundamentos científicos de la definición que proporciona la OMS y, hasta
el momento, no hemos encontrado una explicación razonable).

Es cierto que las gotículas tienden a volar por el aire como balas de
cañón miniatura y caen al suelo con mucha velocidad, mientras que los
aerosoles pueden quedarse suspendidos durante muchas horas.

Sin embargo, la física también plantea que una gotícula de 5 micrómetros
tarda aproximadamente media hora en caer al suelo desde la boca de un
adulto de estatura promedio y, en ese tiempo, la gotícula puede viajar
muchos metros en una corriente de aire. Las gotículas que se expulsan
con la tos o los estornudos también
\href{https://academic.oup.com/jid/advance-article/doi/10.1093/infdis/jiaa189/5820886}{viajan
distancias mucho mayores a un metro}.

Esta es otra idea equivocada: en la medida (limitada) en la que se había
reconocido la importancia de los aerosoles hasta ahora, solían
mencionarse como algo que estaba suspendido en el aire y que se iba con
el viento: una amenaza lejana.

No obstante, antes de poder alejarse, los aerosoles deben viajar por el
aire que está cerca: esto quiere decir que también son un peligro a
corta distancia. E incluso lo son más porque, al igual que el humo de un
cigarrillo, los aerosoles se concentran más cerca de la persona
infectada (o el fumador) y se diluyen en el aire a medida que se alejan.

\href{https://www.sciencedirect.com/science/article/abs/pii/S0360132320302183?via\%3Dihub}{Un
estudio arbitrado} realizado por científicos de la Universidad de Hong
Kong y la Universidad de Zhejiang, en Hangzhou, China, publicado en
junio en la revista Building and Environment llegó a la conclusión de
que ``cuanto más pequeñas son las gotículas exhaladas, más importante es
la ruta aérea de corta distancia''.

¿Qué significa todo esto exactamente, en la práctica?

¿Puedes entrar a una habitación vacía y contraer el virus si una persona
infectada, que ya se fue, estuvo ahí antes que tú? Tal vez, pero solo es
probable si la habitación es pequeña y está mal ventilada.

¿El virus puede flotar en los edificios hacia arriba y hacia abajo por
los ductos de ventilación o las tuberías? Quizá, aunque eso no se ha
confirmado.

La investigación sugiere que lo más probable es que los aerosoles sean
relevantes en contextos sumamente mundanos.

Veamos
\href{https://www.nytimes3xbfgragh.onion/2020/04/20/health/airflow-coronavirus-restaurants.html}{el
caso de un restaurante en Cantón}, en el sur de China, a principios de
este año, en el que un comensal portador del SARS-CoV-2 en una mesa
contagió a un total de nueve personas sentadas en su mesa y en dos mesas
más.

Yuguo Li, profesor de Ingeniería en la Universidad de Hong Kong, y sus
colegas\href{https://www.medrxiv.org/content/10.1101/2020.04.16.20067728v1}{analizaron
las grabaciones de vigilancia} del restaurante y en una prepublicación
(no arbitrada) que dieron a conocer en abril no encontraron pruebas de
un contacto cercano entre los comensales.

La transmisión, en este caso, no se puede atribuir a las gotículas, al
menos no la que se dio entre las personas que estaban en mesas distintas
a la de la persona infectada: las gotículas habrían caído al suelo antes
de llegar a las otras mesas.

Sin embargo, las tres mesas estaban en una sección mal ventilada del
restaurante y una unidad de aire acondicionado hacía circular el aire
entre ellas. También cabe mencionar que no se contagió ningún miembro
del personal del restaurante y ninguno de los otros comensales,
incluidos los que estaban sentados en dos mesas justo afuera de la
corriente de aire acondicionado.

En un caso similar, se piensa que una sola persona contagió a 52 de las
otras 60 personas presentes en
\href{https://www.nytimes3xbfgragh.onion/2020/05/12/health/coronavirus-choir.html}{un
ensayo de coro} en el condado de Skagit, Washington, en marzo.

Analicé ese evento, con la ayuda de varios colegas de distintas
universidades y en
\href{https://www.medrxiv.org/content/10.1101/2020.06.15.20132027v2}{una
prepublicación (no arbitrada) divulgada en junio} llegamos a la
conclusión de que los aerosoles probablemente fueron la vía principal de
transmisión.

Los presentes habían usado desinfectante para manos y habían evitado
darse abrazos y apretones de mano, lo cual limitó el potencial de
contagio por medio del contacto directo o las gotículas. Por otro lado,
la ventilación de la habitación era deficiente, el ensayo duró mucho
tiempo (2,5 horas) y es bien sabido que cantar produce microgotas de
aerosol y
\href{https://www.atsjournals.org/doi/abs/10.1164/arrd.1968.98.2.297}{facilita
la propagación de enfermedades como la tuberculosis}.

¿Qué hay sobre el brote en el crucero de Diamond Princess que salió de
Japón a principios de este año? Unos 712 de los 3711 pasajeros se
contagiaron.

El profesor Li y otros también
\href{https://www.medrxiv.org/content/10.1101/2020.04.09.20059113v1}{investigaron
ese caso} y en una prepublicación (no arbitrada) de abril concluyeron
que no había habido transmisión entre las habitaciones luego de que la
gente se puso en cuarentena: el sistema de aire acondicionado del barco
no propagó el virus a distancias más largas.

Al parecer, la causa más probable de transmisión, según el estudio, fue
el contacto cercano con personas infectadas o con objetos contaminados
antes de que los pasajeros y los tripulantes se aislaran. (Los
investigadores no definieron con precisión a qué se referían con
``contacto'' y tampoco esclarecieron si este incluía gotículas o
aerosoles de corto alcance).

Sin embargo,
\href{https://www.medrxiv.org/content/10.1101/2020.07.13.20153049v1}{otra
prepublicación reciente} (no arbitrada) acerca del caso del Diamond
Princess concluyó que ``es muy probable que la inhalación de aerosoles
haya sido el factor predominante de la transmisión de la COVID-19''
entre los pasajeros de la embarcación.

Podría parecer lógico, o tener sentido intuitivo, deducir que las
gotículas más grandes contienen más virus que las microgotas más
pequeñas de aerosol, pero no es así.

\href{https://www.thelancet.com/journals/lanres/article/PIIS2213-2600(20)30323-4/fulltext}{Un
artículo publicado esta semana} en The Lancet Respiratory Medicine que
analizó los aerosoles producidos por la tos y las exhalaciones de
pacientes con distintas infecciones respiratorias reveló ``un predominio
de patógenos en partículas pequeñas'' (de menos de 5 micrómetros). ``No
hay evidencia'', concluyó el estudio, ``de que algunos patógenos solo se
transporten en gotículas grandes''.

Una
\href{https://www.medrxiv.org/content/10.1101/2020.07.13.20041632v2}{prepublicación
reciente} (no arbitrada) realizada por investigadores del Centro Médico
de la Universidad de Nebraska descubrió que las muestras de virus
tomadas de aerosoles emitidos por pacientes con COVID-19 eran
contagiosas.

Algunos científicos
\href{https://jamanetwork.com/journals/jama/fullarticle/2768396}{han
argumentado} que el mero hecho de que los aerosoles puedan contener el
SARS-CoV-2 no comprueba por sí mismo que estos puedan provocar una
infección y que si el SARS-CoV-2 se propagara principalmente por medio
de aerosoles, habría más evidencia de transmisiones a larga distancia.

Concuerdo con la idea de que la transmisión a larga distancia por medio
de aerosoles no es significativa, pero considero que, en conjunto,
muchas de las pruebas recabadas hasta la fecha sugieren que la
transmisión \emph{a corta distancia} por medio de aerosoles es
significativa, quizá muy significativa, y sin duda más significativa que
el rocío directo de gotículas.

Las implicaciones prácticas son simples:

\begin{itemize}
\item
  \textbf{El distanciamiento social sí es muy importante}. Nos mantiene
  alejados de las zonas con mayor concentración de expulsiones
  respiratorias. Así que es importante mantener al menos 2 metros de
  distancia de otras personas, aunque entre más lejos estés, más seguro
  estarás.
\item
  \textbf{Usa cubrebocas.} Los cubrebocas ayudan a bloquear los
  aerosoles que libera el portador.
  \href{https://ucsf.app.box.com/s/blvolkp5z0mydzd82rjks4wyleagt036}{La
  evidencia científica} también plantea que
  \href{https://www.nytimes3xbfgragh.onion/2020/07/27/health/coronavirus-mask-protection.html?campaign_id=154\&emc=edit_cb_20200727\&instance_id=20696\&nl=coronavirus-briefing\&regi_id=65413713\&segment_id=34503\&te=1\&user_id=bd32fbf008e5183a7928ed61c60669f7}{los
  cubrebocas evitan que el portador inhale los aerosoles} que le rodean.
\end{itemize}

Cuando hablamos de cubrebocas, el tamaño \emph{sí} importa.

El estándar de oro es el respirador N95 o el KN95, que, si se ajusta
adecuadamente al rostro, puede filtrar las microgotas de aerosol y evita
que el portador las inhale al menos en un 95 por ciento.

La eficacia de las mascarillas quirúrgicas contra los aerosoles varía
mucho.

\href{https://pubmed.ncbi.nlm.nih.gov/23498357/}{Un estudio de 2013}
reveló que las mascarillas quirúrgicas reducían la exposición a los
virus gripales entre un 10 y un 98 por ciento (dependiendo del diseño de
la mascarilla).

Un artículo reciente descubrió que las mascarillas quirúrgicas pueden
impedir por completo que los coronavirus estacionales
\href{https://www.nature.com/articles/s41591-020-0843-2}{se expulsen al
aire}.

Según tengo entendido, aún no se ha realizado ningún estudio similar
para el SARS-CoV-2, pero estos hallazgos también podrían aplicarse a
este virus, ya que es parecido a los coronavirus estacionales, en
términos de tamaño y estructura.

Mi laboratorio ha estado haciendo pruebas con cubrebocas de tela en un
maniquí que succiona aire por la boca a un ritmo realista. Descubrimos
que incluso un pañuelo amarrado, sin apretar, sobre boca y nariz
bloqueaba el paso de la mitad o más de las microgotas de aerosol de más
de 2 micrómetros.

También descubrimos que para bloquear los aerosoles muy pequeños ---de
menos de un micrómetro--- es más efectivo usar una tela más suave (que
es más fácil de ajustar al rostro) que una tela más rígida (la cual
ofrece un mejor filtro, pero no suele ajustarse bien al rostro y deja
huecos).

\begin{itemize}
\item
  \textbf{Evita las multitudes.} Cuantas más personas haya a tu
  alrededor, más probable será que alguna esté infectada. Evita las
  multitudes sobre todo en lugares cerrados, donde pueden acumularse los
  aerosoles.
\item
  \textbf{La ventilación es relevante.} Abre ventanas y puertas. Ajusta
  el regulador de tu sistema de aire acondicionado y calefacción. Mejora
  los filtros de esos sistemas. Añade purificadores de aire portátiles o
  instala tecnología de lámparas germicidas de luz ultravioleta para
  eliminar o matar partículas de virus en el aire.
\end{itemize}

No está clara la medida en que este coronavirus se transmite a través de
los aerosoles en comparación con las gotículas o el contacto con
superficies contaminadas. Incluso
\href{https://journals.plos.org/plospathogens/article?id=10.1371/journal.ppat.1008704}{en
el caso de la influenza}, que se ha estudiado durante décadas, tampoco
sabemos la respuesta a esa pregunta todavía.

Sin embargo, esto es lo que sí sabemos hasta el momento: los aerosoles
son relevantes en la transmisión de la COVID-19 y quizá sean más
relevantes de lo que hemos podido comprobar hasta ahora.

Linsey C. Marr es la profesora de la cátedra Charles P. Lunsford de
Ingeniería Civil y Ambiental en el Instituto Politécnico y Universidad
Estatal de Virginia.
\href{https://twitter.com/linseymarr?lang=en}{@linseymarr}

Advertisement

\protect\hyperlink{after-bottom}{Continue reading the main story}

\hypertarget{site-index}{%
\subsection{Site Index}\label{site-index}}

\hypertarget{site-information-navigation}{%
\subsection{Site Information
Navigation}\label{site-information-navigation}}

\begin{itemize}
\tightlist
\item
  \href{https://help.nytimes3xbfgragh.onion/hc/en-us/articles/115014792127-Copyright-notice}{©~2020~The
  New York Times Company}
\end{itemize}

\begin{itemize}
\tightlist
\item
  \href{https://www.nytco.com/}{NYTCo}
\item
  \href{https://help.nytimes3xbfgragh.onion/hc/en-us/articles/115015385887-Contact-Us}{Contact
  Us}
\item
  \href{https://www.nytco.com/careers/}{Work with us}
\item
  \href{https://nytmediakit.com/}{Advertise}
\item
  \href{http://www.tbrandstudio.com/}{T Brand Studio}
\item
  \href{https://www.nytimes3xbfgragh.onion/privacy/cookie-policy\#how-do-i-manage-trackers}{Your
  Ad Choices}
\item
  \href{https://www.nytimes3xbfgragh.onion/privacy}{Privacy}
\item
  \href{https://help.nytimes3xbfgragh.onion/hc/en-us/articles/115014893428-Terms-of-service}{Terms
  of Service}
\item
  \href{https://help.nytimes3xbfgragh.onion/hc/en-us/articles/115014893968-Terms-of-sale}{Terms
  of Sale}
\item
  \href{https://spiderbites.nytimes3xbfgragh.onion}{Site Map}
\item
  \href{https://help.nytimes3xbfgragh.onion/hc/en-us}{Help}
\item
  \href{https://www.nytimes3xbfgragh.onion/subscription?campaignId=37WXW}{Subscriptions}
\end{itemize}
