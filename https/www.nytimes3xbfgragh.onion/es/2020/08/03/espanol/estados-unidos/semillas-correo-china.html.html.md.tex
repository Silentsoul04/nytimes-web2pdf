Sections

SEARCH

\protect\hyperlink{site-content}{Skip to
content}\protect\hyperlink{site-index}{Skip to site index}

\href{https://www.nytimes3xbfgragh.onion/es/section/estados-unidos}{Estados
Unidos}

\href{https://myaccount.nytimes3xbfgragh.onion/auth/login?response_type=cookie\&client_id=vi}{}

\href{https://www.nytimes3xbfgragh.onion/section/todayspaper}{Today's
Paper}

\href{/es/section/estados-unidos}{Estados Unidos}\textbar{}Estados
Unidos identifica algunas de las misteriosas semillas enviadas desde
China

\url{https://nyti.ms/2XoQ0Vt}

\begin{itemize}
\item
\item
\item
\item
\item
\end{itemize}

Advertisement

\protect\hyperlink{after-top}{Continue reading the main story}

Supported by

\protect\hyperlink{after-sponsor}{Continue reading the main story}

\hypertarget{estados-unidos-identifica-algunas-de-las-misteriosas-semillas-enviadas-desde-china}{%
\section{Estados Unidos identifica algunas de las misteriosas semillas
enviadas desde
China}\label{estados-unidos-identifica-algunas-de-las-misteriosas-semillas-enviadas-desde-china}}

Las 14 variedades reconocidas incluyen plantas comunes como hibisco,
campanilla morada y lavanda. Aún así, los expertos advirtieron a los
destinatarios que no las sembraran.

\includegraphics{https://static01.graylady3jvrrxbe.onion/images/2020/08/03/multimedia/03Semillas-ES/merlin_175213314_648d3f4c-1f6e-4788-9e08-818f0f116ff1-articleLarge.jpg?quality=75\&auto=webp\&disable=upscale}

Por \href{https://www.nytimes3xbfgragh.onion/by/allyson-waller}{Allyson
Waller}

\begin{itemize}
\item
  3 de agosto de 2020
\item
  \begin{itemize}
  \item
  \item
  \item
  \item
  \item
  \end{itemize}
\end{itemize}

\href{https://www.nytimes3xbfgragh.onion/2020/08/02/us/Seed-packets-China-USA.html}{Read
in English}

\href{https://www.nytimes3xbfgragh.onion/newsletters/el-times}{Regístrate
para recibir nuestro boletín} con lo mejor de The New York Times.

\begin{center}\rule{0.5\linewidth}{\linethickness}\end{center}

Una agencia federal estadounidense dijo que había identificado 14 tipos
de plantas a partir de los
\href{https://www.nytimes3xbfgragh.onion/2020/07/26/us/seeds-from-china-mail.html}{paquetes
de semillas no solicitadas} que parecen haber sido enviadas por correo
desde China, una ``mezcla de especies ornamentales, frutas y vegetales,
hierbas y malezas''.

Entre las especies de plantas que los botánicos han identificado hasta
ahora hay repollo, hibisco, lavanda, menta, campanilla morada, mostaza,
rosa, romero y salvia,
\href{https://www.aphis.usda.gov/publications/plant_health/faq-unsolicited-seeds.pdf}{según
el Servicio de Inspección Sanitaria de Animales y Plantas.}

``Esto es solo un subconjunto de las muestras que hemos recolectado
hasta ahora'',
\href{https://www.usda.gov/media/radio/daily-newsline/2020-07-29/actuality-unsolicited-seeds-are-several-plant-species}{dijo
la semana pasada} Osama El-Lissy, administrador adjunto para la
protección de plantas y cuarentena del servicio.

El mes pasado, varios estados del país informaron que algunos de sus
residentes recibieron paquetes de semillas que no habían pedido.

Desde entonces, los 50 estados han emitido advertencias sobre los
paquetes no solicitados y el servicio de inspección dijo que había
recibido paquetes desde, por lo menos, 22 estados.

Doyle Crenshaw, de Booneville, Arkansas, dijo que había plantado algunas
de las semillas no solicitadas que recibió.

``Le dije a mi esposa: `No se parecen a ninguna semilla de flor que haya
visto en mi vida''', relató el domingo.

Crenshaw dijo que había pedido en Amazon semillas de zinnia azul, pero
que cuando el paquete llegó, hace un par de meses, contenía las semillas
de zinnia azul y también paquetes de semillas que no había ordenado.

Dijo que la etiqueta del paquete decía ``aretes de tachuelas'' y
``China''.

``Es una planta en verdad bonita'', dijo, al describir lo que creció de
las semillas no solicitadas. ``Parece una planta de calabaza gigante''

Un representante de Amazon no pudo ser contactado de inmediato el
domingo.

Crenshaw dijo que llamó al Departamento de Agricultura de Arkansas y que
los funcionarios irían a su casa esta semana para desenterrar la planta
que creció a partir de las semillas. También planea que recojan otro
paquete no solicitado que recibió ---pero que no ha abierto--- cuya
etiqueta dice que contiene cuentas.

Dijo que después de recibir estos paquetes, ahora él y su esposa
ordenarán sus semillas en la localidad.

La
\href{https://www.aphis.usda.gov/aphis/newsroom/stakeholder-info/sa_by_date/sa-2020/sa-07/seeds-china}{agencia
federal de inspección dijo} que la evidencia indica que los paquetes son
parte de una ``estafa de cepillado'' en la cual los vendedores envían
artículos no solicitados con la esperanza de aumentar las ventas.

Aunque es bajo el riesgo de una consecuencia nefasta, como la
introducción de una especie exótica en Estados Unidos o alguna forma de
guerra biológica, los destinatarios no deben plantar las semillas, dijo
Art Gover, investigador de ciencia de las plantas en la Universidad de
Penn State.

Gover dijo que estas semillas pueden ser problemáticas porque pueden
introducir malezas y enfermedades.

Lisa Delissio, profesora de biología en la Universidad Estatal de Salem,
en Massachusetts, advirtió que si alguna de las semillas no
identificadas resultara ser una especie invasora, podría desplazar a las
plantas nativas y competir por los recursos y dañar el medio ambiente,
la agricultura o la salud humana.

Bernd Blossey, profesor del departamento de recursos naturales de la
Universidad de Cornell en Ithaca, Nueva York, dijo que recibió algunas
llamadas de preocupados receptores de los paquetes de semillas.

``Obviamente, plantar romero o tomillo en tu jardín no es algo que ponga
en peligro nuestro medio ambiente'', dijo. ``Pero puede haber cosas ahí
que aún no han sido identificadas. Cada vez que recibas algo
desconocido, mi sugerencia es que ni siquiera lo tires a la basura, que
lo quemes''.

Los jardineros han sido responsables de la introducción de especies
invasoras en el pasado y de nutrirlas con su buena mano para las
plantas, incluido el arbusto de las mariposas, la Fallopia japonica y
algunas hierbas ornamentales, dijo Blossey.

``¿Quién sabe quién está detrás o qué está detrás de esto?'', se
preguntó. ``Creo que puede haber más en esta historia''.

Marie Fazio y Christina Morales colaboraron con reportería.

Allyson Waller es parte de la clase 2020-2021 del New York Times
Fellowship y es reportera de temas generales en la sección Express.
\href{https://twitter.com/allyson_renee7}{@allyson\_renee7}

\begin{center}\rule{0.5\linewidth}{\linethickness}\end{center}

Advertisement

\protect\hyperlink{after-bottom}{Continue reading the main story}

\hypertarget{site-index}{%
\subsection{Site Index}\label{site-index}}

\hypertarget{site-information-navigation}{%
\subsection{Site Information
Navigation}\label{site-information-navigation}}

\begin{itemize}
\tightlist
\item
  \href{https://help.nytimes3xbfgragh.onion/hc/en-us/articles/115014792127-Copyright-notice}{©~2020~The
  New York Times Company}
\end{itemize}

\begin{itemize}
\tightlist
\item
  \href{https://www.nytco.com/}{NYTCo}
\item
  \href{https://help.nytimes3xbfgragh.onion/hc/en-us/articles/115015385887-Contact-Us}{Contact
  Us}
\item
  \href{https://www.nytco.com/careers/}{Work with us}
\item
  \href{https://nytmediakit.com/}{Advertise}
\item
  \href{http://www.tbrandstudio.com/}{T Brand Studio}
\item
  \href{https://www.nytimes3xbfgragh.onion/privacy/cookie-policy\#how-do-i-manage-trackers}{Your
  Ad Choices}
\item
  \href{https://www.nytimes3xbfgragh.onion/privacy}{Privacy}
\item
  \href{https://help.nytimes3xbfgragh.onion/hc/en-us/articles/115014893428-Terms-of-service}{Terms
  of Service}
\item
  \href{https://help.nytimes3xbfgragh.onion/hc/en-us/articles/115014893968-Terms-of-sale}{Terms
  of Sale}
\item
  \href{https://spiderbites.nytimes3xbfgragh.onion}{Site Map}
\item
  \href{https://help.nytimes3xbfgragh.onion/hc/en-us}{Help}
\item
  \href{https://www.nytimes3xbfgragh.onion/subscription?campaignId=37WXW}{Subscriptions}
\end{itemize}
