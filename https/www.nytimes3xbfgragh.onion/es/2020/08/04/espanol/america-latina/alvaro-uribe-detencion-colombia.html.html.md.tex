Sections

SEARCH

\protect\hyperlink{site-content}{Skip to
content}\protect\hyperlink{site-index}{Skip to site index}

\href{https://www.nytimes3xbfgragh.onion/es/section/america-latina}{América
Latina}

\href{https://myaccount.nytimes3xbfgragh.onion/auth/login?response_type=cookie\&client_id=vi}{}

\href{https://www.nytimes3xbfgragh.onion/section/todayspaper}{Today's
Paper}

\href{/es/section/america-latina}{América Latina}\textbar{}Álvaro Uribe
enfrenta una posible detención

\url{https://nyti.ms/39TMSGn}

\begin{itemize}
\item
\item
\item
\item
\item
\end{itemize}

Advertisement

\protect\hyperlink{after-top}{Continue reading the main story}

Supported by

\protect\hyperlink{after-sponsor}{Continue reading the main story}

\hypertarget{uxe1lvaro-uribe-enfrenta-una-posible-detenciuxf3n}{%
\section{Álvaro Uribe enfrenta una posible
detención}\label{uxe1lvaro-uribe-enfrenta-una-posible-detenciuxf3n}}

La decisión de detener al expresidente de Colombia, que aún no ha sido
confirmada por la Corte Suprema, podría ser un punto de inflexión en una
nación acostumbrada a ver que políticos poderosos evaden la justicia a
pesar de años de investigaciones.

\includegraphics{https://static01.graylady3jvrrxbe.onion/images/2020/08/04/world/04uribe-ES/merlin_138461055_e8f62ffd-95e1-4b71-8a94-dda7b8b50083-articleLarge.jpg?quality=75\&auto=webp\&disable=upscale}

Por \href{https://www.nytimes3xbfgragh.onion/by/julie-turkewitz}{Julie
Turkewitz}

\begin{itemize}
\item
  4 de agosto de 2020
\item
  \begin{itemize}
  \item
  \item
  \item
  \item
  \item
  \end{itemize}
\end{itemize}

\href{https://www.nytimes3xbfgragh.onion/2020/08/04/world/americas/colombia-president-uribe-charged.html}{Read
in English}

\textbf{{[}Esta es una noticia en desarrollo. Para leer la versión más
reciente,}
\textbf{\href{https://www.nytimes3xbfgragh.onion/2020/08/04/world/americas/colombia-president-uribe-charged.html}{consulta
el artículo en inglés{]}}}

\begin{center}\rule{0.5\linewidth}{\linethickness}\end{center}

BOGOTÁ, Colombia --- **** El presidente de Colombia, Iván Duque, atacó a
su propio sistema judicial el martes por perseguir a su mentor, el
expresidente Álvaro Uribe, quien enfrenta una posible detención en medio
de una investigación de fraude y soborno.

Duque hizo sus declaraciones después de que Uribe se lamentara de ``la
privación de mi libertad'' en Twitter, lo que alimentó la especulación
generalizada en Colombia de que el expresidente estaba a punto de ser
detenido por la Corte Suprema. Pero para el martes por la noche, la
Corte Suprema aún no había emitido oficialmente una orden de detención,
lo que aumentó la confusión.

Uribe es una encumbrada figura en la política colombiana y su
aprehensión alteraría el panorama político en un país donde los
políticos más poderosos rara vez han sido llamados a responder por sus
acciones en tribunales a pesar de acumular años de pesquisas en su
contra.

\href{https://www.nytimes3xbfgragh.onion/newsletters/el-times}{{[}Inscríbete
para recibir nuestro boletín} en español con lo mejor de The New York
Times{]}.

Uribe sería el primer presidente colombiano en la historia moderna en
ser detenido.

Mientras que otras naciones en América Latina han abordado agresivamente
la corrupción en los últimos años, a veces con juicios a presidentes,
Colombia no ha hecho lo mismo.

Poco después de que Uribe lamentó en Twitter su inminente aprehensión,
el presidente Duque
\href{https://twitter.com/IvanDuque/status/1290755832330813442}{denunció}
que a su mentor no se le permita esperar la resolución de su caso en
libertad, algo que se les ha permitido hacer a algunos criminales y
guerrilleros.

``Duele como colombiano'', dijo Duque, que ``a un servidor público
ejemplar, que ha ocupado la más alta dignidad del Estado, no se le
permita defenderse en libertad, con la presunción de inocencia''.

Considerado ampliamente como el político colombiano más poderoso de las
últimas décadas, Uribe ha sido objeto de investigación durante años,
pero esto es lo más cerca que ha estado de enfrentar a un panel de
jueces.

Su capacidad para evitar el enjuiciamiento había llevado a muchos
colombianos a llamarlo el ``presidente de teflón''.

``La privación de mi libertad me causa profunda tristeza'',
\href{https://twitter.com/AlvaroUribeVel/status/1290712262504779784}{escribió}
Uribe en Twitter el martes, anticipando su detención, ``por mi señora,
por mi familia y por los colombianos que todavía creen que algo bueno he
hecho por la patria''.

Uribe fue presidente de 2002 a 2010, y continúa ejerciendo un poder
descomunal desde su curul de senador. El actual presidente, Iván Duque,
era poco conocido antes de que Uribe lo respaldara y ganó con la promesa
de restaurar el legado del expresidente.

Su posición en Colombia hizo que la detención de Uribe ``realmente sea
algo significativo para nuestro país'', lo que indicaría un posible
cambio para obligar a los políticos previamente intocables a responder
por presuntos delitos, dijo Francisco Bernate, profesor de derecho en la
Universidad del Rosario en Bogotá, la capital.

Los fiscales aún no han presentado cargos formales contra Uribe, pero el
sistema de justicia colombiano permite que los jueces lo encarcelen a la
espera de una acusación si creen que hay riesgo de fuga o que podría
alterar las pruebas. Podría estar en la cárcel hasta durante un año
mientras avanza la investigación.

El caso se deriva de una investigación que los magistrados en la Corte
Suprema comenzaron en 2018. Los jueces están examinando si Uribe intentó
influir en el testimonio de un presunto paramilitar, Juan Guillermo
Monsalve, al presionarlo a retractarse de declaraciones en las que
vinculaba a Uribe a la creación de grupos paramilitares.

Uno de los abogados de Uribe, Diego Cadena, también está bajo
investigación en el caso.

Uribe ha negado la conexión con grupos paramilitares y, en cambio, dijo
que ha luchado contra ellos. También ha negado haberle pedido a alguien
que obstruyera la justicia.

Si se lo encuentra culpable, Uribe podría enfrentar aproximadamente de
seis a ocho años en prisión, dijo Bernate, aunque es probable que en su
lugar pase ese tiempo en arresto domiciliario.

La detención amenaza con polarizar aún más la política colombiana y
aumentar el conflicto entre los aliados de Uribe y sus oponentes sobre
el legado del expresidente.

La decisión también podría impactar al presidente Duque, cuya
popularidad disminuyó durante su primer año en el cargo, hasta que
remontó por su manejo de la pandemia. Sus partidarios de la derecha
podrían volverse en su contra por no hacer más para mantener a su mentor
en libertad, mientras que los críticos de izquierda pueden usar la
detención de Uribe para manchar a Duque e involucrarlo con criminales.

Duque defendió a su mentor el martes, al decir que el expresidente
encarnaba la ``honorabilidad''. Al hablar en una estación nacional de
radio, Duque dijo que la idea de que Uribe estaría alineado con grupos
paramilitares era un ``absurdo''.

El caso es una de las varias investigaciones en la Corte Suprema sobre
la conducta de Uribe a lo largo de los años.

La investigación de fraude y soborno se produjo después de que Uribe
acusó a un oponente político, el senador Iván Cepeda, de manipular
testigos en su contra, lo que provocó una investigación sobre Cepeda.
Esa indagación se cerró en 2018, y la corte decidió, en cambio, proceder
con la investigación de Uribe.

En una entrevista, Cepeda dijo que había ``convincente, abundante''
evidencia contra Uribe.

``Yo creo que es un cambio muy importante para consolidar la
democracia'', dijo Cepeda. ``Colombia ha sido un país que tiene unos
aspectos, un comportamiento monárquico, en la cual hay ciertas figuras
políticas que son intocables. Bueno, aquí no puede haber nadie que está
por encima de la Constitución y la ley y la justicia''.

Jenny Carolina González colaboró con reportería desde Bogotá.

Julie Turkewitz es jefa del buró de los Andes, que cubre Colombia,
Venezuela, Bolivia, Ecuador, Perú, Surinam y Guyana. Antes de mudarse a
Sudamérica, fue corresponsal de temas nacionales y cubrió el oeste de
Estados Unidos.
\href{https://twitter.com/julieturkewitz}{@julieturkewitz}

\begin{center}\rule{0.5\linewidth}{\linethickness}\end{center}

Advertisement

\protect\hyperlink{after-bottom}{Continue reading the main story}

\hypertarget{site-index}{%
\subsection{Site Index}\label{site-index}}

\hypertarget{site-information-navigation}{%
\subsection{Site Information
Navigation}\label{site-information-navigation}}

\begin{itemize}
\tightlist
\item
  \href{https://help.nytimes3xbfgragh.onion/hc/en-us/articles/115014792127-Copyright-notice}{©~2020~The
  New York Times Company}
\end{itemize}

\begin{itemize}
\tightlist
\item
  \href{https://www.nytco.com/}{NYTCo}
\item
  \href{https://help.nytimes3xbfgragh.onion/hc/en-us/articles/115015385887-Contact-Us}{Contact
  Us}
\item
  \href{https://www.nytco.com/careers/}{Work with us}
\item
  \href{https://nytmediakit.com/}{Advertise}
\item
  \href{http://www.tbrandstudio.com/}{T Brand Studio}
\item
  \href{https://www.nytimes3xbfgragh.onion/privacy/cookie-policy\#how-do-i-manage-trackers}{Your
  Ad Choices}
\item
  \href{https://www.nytimes3xbfgragh.onion/privacy}{Privacy}
\item
  \href{https://help.nytimes3xbfgragh.onion/hc/en-us/articles/115014893428-Terms-of-service}{Terms
  of Service}
\item
  \href{https://help.nytimes3xbfgragh.onion/hc/en-us/articles/115014893968-Terms-of-sale}{Terms
  of Sale}
\item
  \href{https://spiderbites.nytimes3xbfgragh.onion}{Site Map}
\item
  \href{https://help.nytimes3xbfgragh.onion/hc/en-us}{Help}
\item
  \href{https://www.nytimes3xbfgragh.onion/subscription?campaignId=37WXW}{Subscriptions}
\end{itemize}
