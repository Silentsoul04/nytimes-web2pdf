Sections

SEARCH

\protect\hyperlink{site-content}{Skip to
content}\protect\hyperlink{site-index}{Skip to site index}

\href{https://www.nytimes3xbfgragh.onion/es/section/mundo}{Mundo}

\href{https://myaccount.nytimes3xbfgragh.onion/auth/login?response_type=cookie\&client_id=vi}{}

\href{https://www.nytimes3xbfgragh.onion/section/todayspaper}{Today's
Paper}

\href{/es/section/mundo}{Mundo}\textbar{}Cómo logró Italia contener la
calamidad del coronavirus

\url{https://nyti.ms/3fv17Th}

\begin{itemize}
\item
\item
\item
\item
\item
\end{itemize}

\href{https://www.nytimes3xbfgragh.onion/es/spotlight/coronavirus?action=click\&pgtype=Article\&state=default\&region=TOP_BANNER\&context=storylines_menu}{El
brote de coronavirus}

\begin{itemize}
\tightlist
\item
  \href{https://www.nytimes3xbfgragh.onion/es/interactive/2020/espanol/america-latina/coronavirus-en-mexico.html?action=click\&pgtype=Article\&state=default\&region=TOP_BANNER\&context=storylines_menu}{Mapa
  y casos en México}
\item
  \href{https://www.nytimes3xbfgragh.onion/es/2020/07/31/espanol/ciencia-y-tecnologia/ninos-contagio-coronavirus.html?action=click\&pgtype=Article\&state=default\&region=TOP_BANNER\&context=storylines_menu}{Los
  niños y el virus}
\item
  \href{https://www.nytimes3xbfgragh.onion/es/interactive/2020/science/coronavirus-tratamientos-curas.html?action=click\&pgtype=Article\&state=default\&region=TOP_BANNER\&context=storylines_menu}{Fármacos
  y tratamientos}
\item
  \href{https://www.nytimes3xbfgragh.onion/es/2020/07/06/espanol/ciencia-y-tecnologia/coronavirus-transmision-aire.html?action=click\&pgtype=Article\&state=default\&region=TOP_BANNER\&context=storylines_menu}{Cómo
  se transmite el coronavirus}
\item
  \href{https://www.nytimes3xbfgragh.onion/es/2020/07/14/espanol/estilos-de-vida/botiquin-medicina-coronavirus.html?action=click\&pgtype=Article\&state=default\&region=TOP_BANNER\&context=storylines_menu}{Prepara
  tu botiquín}
\end{itemize}

Advertisement

\protect\hyperlink{after-top}{Continue reading the main story}

Supported by

\protect\hyperlink{after-sponsor}{Continue reading the main story}

Europa

\hypertarget{cuxf3mo-logruxf3-italia-contener-la-calamidad-del-coronavirus}{%
\section{Cómo logró Italia contener la calamidad del
coronavirus}\label{cuxf3mo-logruxf3-italia-contener-la-calamidad-del-coronavirus}}

Después de un comienzo tambaleante, el país pasó de ser paria global a
adoptar un modelo ---aunque imperfecto--- de contención viral que ofrece
lecciones para sus vecinos y para Estados Unidos.

\includegraphics{https://static01.graylady3jvrrxbe.onion/images/2020/07/30/world/04italy-ES-00/merlin_173911632_5410458a-d14a-451c-b210-c7578df59244-articleLarge.jpg?quality=75\&auto=webp\&disable=upscale}

\href{https://www.nytimes3xbfgragh.onion/by/jason-horowitz}{\includegraphics{https://static01.graylady3jvrrxbe.onion/images/2018/10/10/multimedia/author-jason-horowitz/author-jason-horowitz-thumbLarge.png}}

Por \href{https://www.nytimes3xbfgragh.onion/by/jason-horowitz}{Jason
Horowitz}

\begin{itemize}
\item
  5 de agosto de 2020 a las 03:00 ET
\item
  \begin{itemize}
  \item
  \item
  \item
  \item
  \item
  \end{itemize}
\end{itemize}

\href{https://www.nytimes3xbfgragh.onion/2020/07/31/world/europe/italy-coronavirus-reopening.html}{Read
in
English}\href{https://www.nytimes3xbfgragh.onion/2020/07/31/world/europe/italy-coronavirus-reopening.html}{Read
in English}

\href{https://www.nytimes3xbfgragh.onion/newsletters/el-times}{Regístrate
para recibir nuestro boletín} con lo mejor de The New York Times.

\begin{center}\rule{0.5\linewidth}{\linethickness}\end{center}

ROMA --- Cuando el coronavirus se desató en Occidente, Italia
\href{https://www.nytimes3xbfgragh.onion/interactive/2020/03/27/world/europe/coronavirus-italy-bergamo.html}{era
el epicentro dantesco}, un lugar que debía evitarse a toda costa y, para
Estados Unidos y gran parte de Europa, sinónimo de una infección
descontrolada.

``Miren lo que está pasando en Italia'', dijo el presidente
estadounidense Donald Trump, a unos periodistas el 17 de marzo. ``No
queremos estar en una situación como esa''.
\href{https://www.nytimes3xbfgragh.onion/es/interactive/2020/espanol/estados-unidos/joe-biden-elecciones.html}{Joe
Biden}, el postulante demócrata, se refirió a los hospitales saturados
de Italia como prueba de su oposición a ``Medicare para todos'' en un
debate presidencial.
\href{https://edition.cnn.com/politics/live-news/2020-democratic-debate-live-updates/h_501d1e381370480bd021916a86029534}{``Ahora
no le está funcionando a Italia'', dijo}.

Unos meses después,
\href{https://www.nytimes3xbfgragh.onion/es/interactive/2020/espanol/mundo/coronavirus-en-estados-unidos.html}{Estados
Unidos ha sufrido decenas de miles de muertes} más que cualquier otro
país en el mundo. Las naciones europeas que en algún momento
contemplaron a Italia con desdén ahora se enfrentan a nuevos brotes.
Algunas están imponiendo restricciones nuevas y sopesando si deberían
decretar otro confinamiento.

El 31 de julio, el primer ministro del Reino Unido, Boris Johnson,
anunció que habría un retraso en el relajamiento de restricciones que se
había planeado, pues la tasa de infección de ese país ha aumentado.
Incluso Alemania, un país elogiado por su respuesta eficiente y
rigurosidad al rastrear contactos, advirtió a su población que un
comportamiento negligente está provocando un repunte en el número de
casos.

¿Y qué pasa con Italia? En sus hospitales casi no hay pacientes de
COVID-19. Las muertes diarias atribuidas al virus en Lombardía, la
región septentrional que más padeció la pandemia, son alrededor de cero.
El número de casos diarios ha descendido drásticamente y es ``uno de los
más bajos de Europa y el mundo'', dijo Giovanni Rezza, director del
Departamento de Enfermedades Infecciosas en el Instituto Nacional de
Salud de dicho país. ``Hemos sido muy prudentes'', afirmó.

Y afortunados. Hoy, a pesar de un ligerísimo aumento en el número de
casos la semana pasada, los italianos tienen el optimismo modesto de que
han controlado el virus, a pesar de que los principales expertos de
salud en el país advierten que
\href{https://www.nytimes3xbfgragh.onion/es/2020/07/21/espanol/mundo/errores-europa-coronavirus.html}{la
complacencia sigue siendo el combustible de la pandemia}. Están
conscientes de que el panorama podría cambiar en cualquier momento.

\includegraphics{https://static01.graylady3jvrrxbe.onion/images/2020/07/30/world/04italy-ES-01/merlin_173670762_9f840673-acff-40dc-8c08-694c8c07ece9-articleLarge.jpg?quality=75\&auto=webp\&disable=upscale}

La manera en que Italia ha pasado de ser un
\href{https://www.nytimes3xbfgragh.onion/es/2020/03/22/espanol/coronavirus-lecciones-italia.html}{paria
global} a un modelo, si bien imperfecto, de la contención de un virus,
es materia de estudio para el resto del mundo, incluyendo Estados
Unidos, donde el coronavirus, que nunca ha estado controlado, ahora
causa estragos en todo el país.

Tras un inicio dificultoso, Italia ha consolidado, o al menos
conservado, los frutos de un confinamiento estricto a nivel nacional,
los cuales obtuvo gracias a una mezcla de vigilancia y competencia
médica adquirida con gran pesar.

Comités científicos y técnicos han guiado al gobierno. Los médicos
locales, hospitales y autoridades de salud cada día reúnen más de 20
indicadores del virus y los envían a las autoridades regionales, quienes
a su vez los mandan al Instituto Nacional de Salud.

El resultado es una radiografía semanal de la salud del país, en la que
se basan las decisiones para implementar políticas. Una situación muy
lejana del estado de pánico cercano al colapso que asoló a Italia en
marzo.

La semana pasada, el Parlamento votó para extender los poderes de
emergencia del gobierno hasta el 15 de octubre, después de que el primer
ministro Giuseppe Conte argumentó que la nación no podía bajar la
guardia ``porque el virus sigue circulando''.

Dichos poderes permiten que el gobierno mantenga las restricciones y
responda a la brevedad (incluso con confinamientos) ante nuevos brotes.
El gobierno de Italia ya impuso restricciones a los viajeros procedentes
de aproximadamente 15 países, pues ahora el principal temor del gobierno
es la importación del virus.

``Hay muchas situaciones en Francia, España, los Balcanes, lo que
significa que el virus no está en absoluto acabado'', dijo Ranieri
Guerra, subdirector general de iniciativas estratégicas de la
Organización Mundial de la Salud y médico italiano. ``Podría volver en
cualquier momento''.

Image

El primer ministro de Italia, Giuseppe Conte, dijo al Parlamento el
martes que la nación no podía bajar la guardia ``porque el virus todavía
está circulando''.Credit...Fabio Frustaci/EPA, vía Shutterstock

No hay duda de que las privaciones del encierro fueron costosas para la
economía. Durante tres meses se ordenó el cierre de negocios y
restaurantes, el desplazamiento estuvo muy restringido (incluso entre
regiones, pueblos y calles) y el turismo se detuvo. Se espera que Italia
pierda alrededor del 10 por ciento de su producto interno bruto este
año.

Pero, en un cierto punto, cuando el virus amenazaba con propagarse de
manera descontrolada, las autoridades italianas decidieron anteponer la
vida de las personas a la economía. ``La salud de los italianos está
primero y así será siempre'', dijo Conte en ese momento.

Las autoridades italianas ahora esperan que lo peor del virus les haya
tocado en una sola dosis enorme (con aquel doloroso confinamiento) y que
el país ahora esté a salvo para retomar su vida normal, aunque con
limitaciones. Sostienen que la única manera de reactivar la economía es
seguir reprimiendo el virus cada vez que aparezca, incluso ahora.

Esta estrategia de cerrar completamente recibió críticas de que la
precaución excesiva del gobierno estaba paralizando la economía. Pero a
la larga quizá resulte más provechoso que intentar reabrir la economía
mientras el virus sigue devorando todo, como está sucediendo en países
como
\href{https://www.nytimes3xbfgragh.onion/2020/03/13/us/coronavirus-deaths-estimate.html}{Estados
Unidos},
\href{https://www.nytimes3xbfgragh.onion/article/brazil-coronavirus-cases.html}{Brasil}
y
\href{https://www.nytimes3xbfgragh.onion/es/2020/06/05/espanol/america-latina/amlo-mexico-muertes-coronavirus.html}{México}.

Como en otras partes del mundo, eso no significa que las exhortaciones a
la vigilancia continua hayan sido inmunes a la burla, la resistencia y
la exasperación. En eso, Italia no difiere de los demás.

Los cubrebocas a menudo brillan por su ausencia o la gente se los quita
en los trenes o autobuses, donde son obligatorios. Los jóvenes salen
y\href{https://www.nytimes3xbfgragh.onion/2020/05/29/world/europe/italy-young-people-coronavirus.html}{hacen
las cosas que hacen los jóvenes}, y de esa manera se arriesgan a
propagar el virus a los sectores más susceptibles de la población. Los
adultos comenzaron a
\href{https://www.nytimes3xbfgragh.onion/2020/05/27/world/europe/italy-beaches-coronavirus-reopening.html}{reunirse
en la playa} y en las parrilladas de cumpleaños. Todavía no hay un plan
claro para el regreso a la escuela en septiembre.

Además hay un floreciente y políticamente motivado contingente
antimascarillas liderado por el nacionalista Matteo Salvini, quien el 27
de julio declaró que remplazar los apretones de manos y abrazos por los
choques de codos era ``el fin de la especie humana''.

En sus mítines, Salvini, líder del partido populista Liga, aún da
apretones de mano y usa el cubrebocas en la barbilla. En julio, durante
una conferencia de prensa, acusó al gobierno italiano de ``importar''
inmigrantes infectados para crear nuevos focos de infección y extender
el estado de emergencia.

Image

Matteo Salvini, el político nacionalista de extrema derecha, ha dicho
que reemplazar los apretones de mano y los abrazos con toques de codos
era ``el fin de la especie humana''.~Credit...Remo Casilli/Reuters

Esta semana, Salvini se unió a otros escépticos de los cubrebocas
---apodados ``negacionistas'' por los críticos--- para protestar en la
biblioteca del Senado, junto con invitados especiales como el cantante
italiano Andrea Bocelli, quien dijo que no creía que la pandemia fuera
tan grave porque ``conozco a mucha gente y no conozco a nadie que haya
terminado en terapia intensiva''.

Sin embargo, los principales expertos en salud del país dijeron que la
falta de casos graves indica una disminución del volumen de infecciones,
ya que solo un pequeño porcentaje de los infectados se enferman
gravemente. Y hasta ahora, los inconformes de Italia no han sido tan
numerosos o poderosos como para socavar lo que ha sido una trayectoria
de éxito, ganada con grandes dificultades para combatir el virus tras un
comienzo calamitoso.

El aislamiento inicial de Italia por parte de sus vecinos europeos al
principio de la crisis, cuando casi no llegaban mascarillas y
ventiladores desde el otro lado de las fronteras, pudo haber ayudado,
dijo Guerra, el experto de la OMS.

``Inicialmente hubo competencia; no colaboración'', señaló Guerra. ``Y
todo el mundo reconoció que Italia se quedó sola en ese momento''. Como
resultado, dijo, ``lo que tuvieron que hacer en ese momento porque nos
dejaron solos resultó ser más eficaz que en otros países''.

Italia primero
\href{https://www.nytimes3xbfgragh.onion/2020/02/23/world/europe/italy-coronavirus.html}{puso
en cuarentena las ciudades}, luego
\href{https://www.nytimes3xbfgragh.onion/2020/03/07/world/europe/coronavirus-italy.html}{la
región de Lombardía} en el norte y después
\href{https://www.nytimes3xbfgragh.onion/2020/03/09/world/europe/italy-lockdown-coronavirus.html}{a
toda la península} y sus islas, a pesar de que en gran parte del centro
y sur de Italia el virus estaba prácticamente ausente. Esto no solo
impidió que los trabajadores del norte industrial regresaran a sus
hogares en el sur, que es mucho más vulnerable, sino que también fomentó
y forzó una respuesta nacional unificada.

Image

La plaza Duomo en Milán durante el confinamiento de Italia a principios
de abrilCredit...Alessandro Grassani para The New York Times

Durante el confinamiento, el desplazamiento estaba estrictamente
limitado entre regiones y pueblos e incluso manzanas de la ciudad, y la
gente tenía que rellenar formularios de ``autocertificación'' para
demostrar que necesitaba salir por motivos de trabajo, salud u ``otras
necesidades''. Algunas autoridades regionales hacían cumplir los
reglamentos de usar mascarilla y mantener el distanciamiento social con
multas considerables. Generalmente, aunque de mala gana,
\href{https://www.nytimes3xbfgragh.onion/2020/03/10/world/europe/italy-coronavirus-movement-restrictions.html}{las
reglas eran respetadas}.

A medida que se difundían las
\href{https://www.nytimes3xbfgragh.onion/2020/03/16/world/europe/italy-coronavirus-funerals.html}{escenas
desgarradoras de sufrimiento humano}, las calles vacías y el
\href{https://www.nytimes3xbfgragh.onion/2020/03/04/world/europe/coronavirus-italy-elderly.html}{elevado
número de víctimas de una generación de ancianos}italianos del norte, la
tasa de transmisión del virus disminuyó rápidamente y la curva se
aplanó, a diferencia de lo que ocurría en otros países europeos,
\href{https://www.nytimes3xbfgragh.onion/2020/07/07/business/sweden-economy-coronavirus.html}{como
Suecia}, que optó por una alternativa al encerramiento.

El hecho de que el brote inicial se localizara en hospitales abrumados
creó una enorme tensión, pero también permitió a los médicos y
enfermeras acelerar el rastreo de contactos.

Luego el país reabrió, poco a poco, ampliando las libertades a
intervalos de dos semanas a fin de adaptarse al periodo de incubación
del virus.

El confinamiento tuvo a la larga el efecto secundario de disminuir el
volumen de virus que circulaba en la sociedad, reduciendo así la
probabilidad de entrar en contacto con alguien que lo tuviera. Al final
del encierro, la circulación del virus se había reducido drásticamente,
y en algunas regiones centrales y meridionales casi no había cadenas de
transmisión.

``Siempre es una cuestión de probabilidad con estos patógenos'', señaló
Guerra y añadió que los nuevos sistemas de alarma temprana, como la
inspección de las aguas residuales para detectar rastros de virus,
habían reducido aún más la probabilidad de infección.

Image

Italia reabrió gradualmente, expandiendo las libertades en intervalos de
dos semanas. No se permitió viajar entre las regiones del país hasta
principios de junio.Credit...Claudio Furlan/LaPresse, vía Associated
Press

Algunos médicos italianos dicen que creen que el virus ahora se comporta
de manera diferente en Italia. Matteo Bassetti, un médico especialista
en enfermedades infecciosas en la ciudad noroccidental de Génova, dijo
que durante el apogeo de la crisis, su hospital fue inundado con 500
casos de COVID-19 a la vez. Ahora, dijo, su unidad de cuidados
intensivos, con 50 camas, no tiene pacientes con coronavirus, y la
unidad de COVID-19 de 60 camas, construida especialmente para la crisis,
está vacía.

Dijo que pensaba que el virus se había debilitado, una opinión no
comprobada, reconoció, que sin embargo ha encontrado una audiencia
entusiasta en Salvini y otros políticos que se oponen a extender el
estado de emergencia.

La mayoría de los expertos en salud dijeron que el virus aún se cernía
sobre el país, y mientras el gobierno considera un nuevo decreto para
reabrir clubes nocturnos, festivales y viajes en cruceros, muchos de
ellos han implorado al país que no baje la guardia.

``Aunque la situación sea mejor que en otros países, debemos seguir
siendo muy prudentes'', dijo Rezza, del Instituto Nacional de Salud, y
agregó que consideraba que era mejor plantearse la cuestión de qué había
hecho bien Italia ``al final de la epidemia''.

``No podemos descartar que tengamos brotes en Italia en los días
siguientes'', dijo. ``Tal vez sea solo cuestión de tiempo''.

Emma Bubola colaboró con este reportaje desde Milán.

Jason Horowitz es el jefe del buró en Roma; cubre Italia, Grecia y otros
sitios del sur de Europa. Cubrió la campaña presidencial de 2016 en
Estados Unidos, el gobierno de Obama y el congreso estadounidense con un
énfasis en perfiles políticos y especiales.
\href{https://twitter.com/jasondhorowitz}{@jasondhorowitz}

\begin{center}\rule{0.5\linewidth}{\linethickness}\end{center}

Advertisement

\protect\hyperlink{after-bottom}{Continue reading the main story}

\hypertarget{site-index}{%
\subsection{Site Index}\label{site-index}}

\hypertarget{site-information-navigation}{%
\subsection{Site Information
Navigation}\label{site-information-navigation}}

\begin{itemize}
\tightlist
\item
  \href{https://help.nytimes3xbfgragh.onion/hc/en-us/articles/115014792127-Copyright-notice}{©~2020~The
  New York Times Company}
\end{itemize}

\begin{itemize}
\tightlist
\item
  \href{https://www.nytco.com/}{NYTCo}
\item
  \href{https://help.nytimes3xbfgragh.onion/hc/en-us/articles/115015385887-Contact-Us}{Contact
  Us}
\item
  \href{https://www.nytco.com/careers/}{Work with us}
\item
  \href{https://nytmediakit.com/}{Advertise}
\item
  \href{http://www.tbrandstudio.com/}{T Brand Studio}
\item
  \href{https://www.nytimes3xbfgragh.onion/privacy/cookie-policy\#how-do-i-manage-trackers}{Your
  Ad Choices}
\item
  \href{https://www.nytimes3xbfgragh.onion/privacy}{Privacy}
\item
  \href{https://help.nytimes3xbfgragh.onion/hc/en-us/articles/115014893428-Terms-of-service}{Terms
  of Service}
\item
  \href{https://help.nytimes3xbfgragh.onion/hc/en-us/articles/115014893968-Terms-of-sale}{Terms
  of Sale}
\item
  \href{https://spiderbites.nytimes3xbfgragh.onion}{Site Map}
\item
  \href{https://help.nytimes3xbfgragh.onion/hc/en-us}{Help}
\item
  \href{https://www.nytimes3xbfgragh.onion/subscription?campaignId=37WXW}{Subscriptions}
\end{itemize}
