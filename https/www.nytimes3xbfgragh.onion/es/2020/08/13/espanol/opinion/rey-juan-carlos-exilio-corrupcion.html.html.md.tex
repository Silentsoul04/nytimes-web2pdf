Sections

SEARCH

\protect\hyperlink{site-content}{Skip to
content}\protect\hyperlink{site-index}{Skip to site index}

\href{https://www.nytimes3xbfgragh.onion/es/section/opinion}{Opinión}

\href{https://myaccount.nytimes3xbfgragh.onion/auth/login?response_type=cookie\&client_id=vi}{}

\href{https://www.nytimes3xbfgragh.onion/section/todayspaper}{Today's
Paper}

\href{/es/section/opinion}{Opinión}\textbar{}El rey iba desnudo y España
miró a otro lado

\url{https://nyti.ms/31RCET4}

\begin{itemize}
\item
\item
\item
\item
\item
\item
\end{itemize}

Advertisement

\protect\hyperlink{after-top}{Continue reading the main story}

\href{/es/section/opinion}{Opinión}

Supported by

\protect\hyperlink{after-sponsor}{Continue reading the main story}

Comentario

\hypertarget{el-rey-iba-desnudo-y-espauxf1a-miruxf3-a-otro-lado}{%
\section{El rey iba desnudo y España miró a otro
lado}\label{el-rey-iba-desnudo-y-espauxf1a-miruxf3-a-otro-lado}}

Una cultura de pleitesía desfasada, pero todavía vigente, permitió a
Juan Carlos I convertirse en el millonario lobista de las dictaduras
árabes y ocultar su fortuna durante décadas. Solo una reforma profunda
puede rescatar a la monarquía de su peor crisis.

\includegraphics{https://static01.graylady3jvrrxbe.onion/images/2020/08/13/opinion/13jimenez-ES/13jimenez-articleLarge.jpg?quality=75\&auto=webp\&disable=upscale}

Por David Jiménez

Es periodista y colaborador regular de The New York Times.

\begin{itemize}
\item
  13 de agosto de 2020
\item
  \begin{itemize}
  \item
  \item
  \item
  \item
  \item
  \item
  \end{itemize}
\end{itemize}

\href{https://www.nytimes3xbfgragh.onion/2020/08/13/opinion/king-juan-carlos-exile.html}{Read
in English}

MADRID --- Uno de los primeros encargos que recibí como reportero, en
1996, fue entrevistar a una supuesta amante del rey de España, Juan
Carlos de Borbón y Borbón. Mis editores en El Mundo investigaban si la
actriz de cine y televisión Barbara Rey estaba exigiendo al Estado
dinero a cambio de no revelar su relación con el rey. No conseguí la
entrevista, la amiga del monarca guardó silencio tras denunciar
presiones y las dos grandes debilidades de nuestro rey ---mujeres y
dinero--- continuaron siendo el secreto peor guardado del país otras dos
décadas.

Admitámoslo: los españoles siempre supimos que el rey iba desnudo, pero
decidimos mirar a otro lado.

Una desfasada cultura de pleitesía, la opacidad que rodea a la monarquía
española y una Constitución que excluye a nuestros reyes de cualquier
responsabilidad penal enviaron al monarca el mensaje de que estaba por
encima de la ley. Sus privilegios, como la inmunidad judicial, diseñados
para dar estabilidad a la institución, fueron aprovechados para amasar
una fortuna cuyo principal origen fueron las millonarias donaciones de
dictadores árabes. Las cantidades eran lo suficientemente importantes
como para que en 2012, en mitad de la gran recesión y con
\href{https://www.macrotrends.net/countries/ESP/spain/unemployment-rate}{una
cuarta parte de los españoles sin empleo}, el rey emérito transfiriera
\href{https://www.publico.es/politica/juan-carlos-i-reclamo-65-millones-corinna-larsen-despues-abdicar.html}{65
millones de euros} a otra de sus amantes, la empresaria alemana Corinna
Larsen.

La revelación de ese ``regalo'' real, que Larsen atribuye al
\href{https://english.elpais.com/spanish_news/2020-07-06/recipient-of-65m-from-spains-emeritus-king-claimed-money-was-a-gift-and-not-a-bid-to-hide-the-funds.html}{``amor
y la gratitud''} y los investigadores a un intento de ocultar dinero
ilícito, es solo la punta del iceberg de un escándalo que ha forzado el
\href{https://www.nytimes3xbfgragh.onion/es/2020/08/05/espanol/opinion/juan-carlos-exilio-espana.html}{exilio
del monarca}. Los españoles desconocemos el paradero de Juan Carlos I
desde que la semana pasada se anunció su salida del país. La estrategia
de alejarlo de los focos, adoptada tras una negociación secreta entre la
Casa Real y el gobierno, demuestra que no hemos aprendido nada.

Juan Carlos, quien
\href{https://www.nytimes3xbfgragh.onion/2014/06/12/world/europe/spanish-lawmakers-clear-way-for-kings-abdication.html}{abdicó
al trono a favor de su hijo Felipe VI} en 2014, debería haber
permanecido en el país que
\href{https://www.nytimes3xbfgragh.onion/slideshow/2014/06/19/blogs/20140619KING/s/20140619KING-slide-DNFH.html}{reinó
durante casi cuatro décadas} hasta que se aclaren las causas por las que
está siendo investigado en Suiza y España, incluida la entrega de
\href{https://www.eldiario.es/politica/millones-juan-carlos-cronologia-retrata_1_1027438.html}{100
millones de dólares} por parte de Arabia Saudí en 2008. El botín real
bajo sospecha, acumulado durante décadas, incluye coches Ferrari, un
yate, viajes de lujo, tierras en Marruecos o un piso londinense valorado
en más 62 millones de euros,
\href{https://www.publico.es/politica/juan-carlos-i-recibio-atico-londres-abdicar-comprado-oman-luego-revendio.html}{obsequio
del sultán de Omán}. Solo alguien que crea ciegamente en los cuentos de
hadas puede pensar que tanta generosidad no tuvo un precio.

El Tribunal Supremo español investiga si el presunto pago de los 100
millones de dólares de los saudíes fueron
\href{https://www.bbc.com/news/world-europe-52977739}{una comisión
pagada a Juan Carlos I} por conseguir que empresas españolas
construyeran el tren de alta velocidad entre Medina y La Meca por un
valor de 6700 millones de euros. Ahora sabemos que durante años el jefe
del Estado mantuvo una doble vida como lobista y que sus benefactores
obtuvieron a cambio una influencia decisiva en España. ¿Cuánta
influencia? El interés de las autoridades por mirar bajo de esa alfombra
es mínimo.

El parlamento ha
\href{https://www.lavanguardia.com/politica/20200616/481814522302/mesa-ongreso-rechaza-investigar-juan-carlos-i.html}{bloqueado}
la creación de una comisión de investigación que podría haber servido
para desvelar las implicaciones geopolíticas del comportamiento del rey
emérito. Los ciudadanos se pierden así la oportunidad de que se pregunte
a los cuatro últimos presidentes del gobierno español qué sabían de los
negocios del rey y cómo estos influyeron en la política exterior
española. El conocido empresario, Javier de la Rosa,
\href{https://www.elespanol.com/opinion/carta-del-director/20200322/matar-padre-hora-negra/476652332_20.html}{reveló
ya en 1995} al entonces director de El Mundo, Pedro J. Ramírez, que
Kuwait pagó 100 millones de dólares como recompensa por el apoyo del rey
para que el gobierno español apoyara la coalición contra Sadam Hussein
en la primera Guerra del Golfo.

España ha sido durante décadas uno de los principales valedores de las
dictaduras árabes, que han encontrado en nuestra monarquía una manera de
legitimarse internacionalmente. En noviembre de 2018, en mitad de la
indignación por el descuartizamiento del periodista
\href{https://www.nytimes3xbfgragh.onion/es/2019/06/21/espanol/opinion/jamal-khashoggi-asesinato.html}{Jamal
Khashoggi}, el gobierno saudí distribuyó la fotografía de un amigable
saludo entre Juan Carlos I y el hombre que algunos acusan de ordenar el
asesinato, el príncipe heredero Mohamed bin Salmán.

Tampoco la represión de manifestantes que pedían democracia en Baréin
impidió los constantes viajes del rey emérito a otra de las ``monarquías
hermanas'' que engordaron sus cuentas bancarias. Uno de los gestores del
patrimonio de Juan Carlos I reveló a la fiscalía suiza que el exjefe del
Estado español volvió de un viaje a Manama con
\href{https://english.elpais.com/spanish_news/2020-05-02/spains-former-king-received-alleged-17m-donation-from-ruler-of-bahrain-swiss-probe-shows.html}{un
maletín con 1,9 millones de dólares}.

\includegraphics{https://static01.graylady3jvrrxbe.onion/images/2020/08/13/opinion/13jimenez-ES-2/13jimenez2-articleLarge.jpg?quality=75\&auto=webp\&disable=upscale}

A la espera de las decisiones que tomen los jueces en Suiza y España,
nadie puede dudar de la inmoralidad del comportamiento del que durante
décadas fuera el hombre más admirado de España. Pero no importa el
cúmulo de evidencias o el avance de las investigaciones: el mismo poder
establecido que extendió un manto de impunidad sobre el rey, incluida
una clase política, un empresariado y una prensa cortesanas, ha acudido
a su rescate. Lo que debería ser una cuestión de decencia y rendición de
cuentas ha degenerado en un polarizado debate entre favorables y
contrarios a la monarquía.

Los defensores de Juan Carlos I aseguran que, más allá de sus faltas,
\href{https://www.euronews.com/2020/08/04/spain-s-ex-king-juan-carlos-from-hero-of-democracy-to-tainted-exile}{su
legado como padre de la democracia española} es imborrable. Consideran
la protección de la institución clave en un momento de gran fractura
política y tensiones territoriales, incluido el desafío independentista
del gobierno de Cataluña. El argumento es legítimo, pero pierde su
sentido cuando se adorna de teorías conspiranoicas sobre un ataque
coordinado de los enemigos del país para tumbar la monarquía. Nadie ha
hecho más por sabotearla que el propio rey emérito.

Las monarquías europeas son reliquias del pasado cuyo papel se ha
reducido a labores de representación diplomática, simbolismo patriótico
y, por qué no, entretenimiento para las masas. La vida disoluta de sus
miembros ha sido tradicionalmente aceptada, dentro de unos límites. Pero
cuando los escándalos implican a una red de abusos de menores, como
ocurre estos días con la conexión del
\href{https://www.nytimes3xbfgragh.onion/2020/07/05/us/politics/prince-andrew-jeffrey-epstein.html}{príncipe
Andrés de Inglaterra} con Jeffrey Epstein, o sospechas de corrupción
como en el caso de Juan Carlos I, ese pacto no escrito se rompe y la
pregunta resurge. ¿Necesitamos la monarquía?

Una institución como la española no puede ser salvada buscando al exrey
a un plácido retiro, blindándolo de las consecuencias de sus actos y
manteniendo la opacidad de siempre, mientras se envía el mensaje
equivocado a su hijo y actual monarca, Felipe VI, de que recibirá el
mismo trato independientemente de sus actos.

Lo que se necesita es un debate abierto sobre el modelo de Estado,
reformas profundas que adapten la monarquía a los tiempos, empezando por
el fin de la impunidad judicial, y la instauración de una cultura de
transparencia. La idea de que en pleno siglo XXI los reyes pueden
mostrarse desnudos, como en el clásico cuento de Hans Christian
Andersen, y esperar que sus súbditos simplemente miren a otro lado, solo
puede terminar en un final infeliz.

David Jiménez
(\href{https://twitter.com/DavidJimenezTW}{@DavidJimenezTW}) es escritor
y periodista. Su libro más reciente es \emph{El director}.

Advertisement

\protect\hyperlink{after-bottom}{Continue reading the main story}

\hypertarget{site-index}{%
\subsection{Site Index}\label{site-index}}

\hypertarget{site-information-navigation}{%
\subsection{Site Information
Navigation}\label{site-information-navigation}}

\begin{itemize}
\tightlist
\item
  \href{https://help.nytimes3xbfgragh.onion/hc/en-us/articles/115014792127-Copyright-notice}{©~2020~The
  New York Times Company}
\end{itemize}

\begin{itemize}
\tightlist
\item
  \href{https://www.nytco.com/}{NYTCo}
\item
  \href{https://help.nytimes3xbfgragh.onion/hc/en-us/articles/115015385887-Contact-Us}{Contact
  Us}
\item
  \href{https://www.nytco.com/careers/}{Work with us}
\item
  \href{https://nytmediakit.com/}{Advertise}
\item
  \href{http://www.tbrandstudio.com/}{T Brand Studio}
\item
  \href{https://www.nytimes3xbfgragh.onion/privacy/cookie-policy\#how-do-i-manage-trackers}{Your
  Ad Choices}
\item
  \href{https://www.nytimes3xbfgragh.onion/privacy}{Privacy}
\item
  \href{https://help.nytimes3xbfgragh.onion/hc/en-us/articles/115014893428-Terms-of-service}{Terms
  of Service}
\item
  \href{https://help.nytimes3xbfgragh.onion/hc/en-us/articles/115014893968-Terms-of-sale}{Terms
  of Sale}
\item
  \href{https://spiderbites.nytimes3xbfgragh.onion}{Site Map}
\item
  \href{https://help.nytimes3xbfgragh.onion/hc/en-us}{Help}
\item
  \href{https://www.nytimes3xbfgragh.onion/subscription?campaignId=37WXW}{Subscriptions}
\end{itemize}
