Sections

SEARCH

\protect\hyperlink{site-content}{Skip to
content}\protect\hyperlink{site-index}{Skip to site index}

\href{https://www.nytimes3xbfgragh.onion/es/section/opinion}{Opinión}

\href{https://myaccount.nytimes3xbfgragh.onion/auth/login?response_type=cookie\&client_id=vi}{}

\href{https://www.nytimes3xbfgragh.onion/section/todayspaper}{Today's
Paper}

\href{/es/section/opinion}{Opinión}\textbar{}Jubilar a un caudillo

\url{https://nyti.ms/3iO73st}

\begin{itemize}
\item
\item
\item
\item
\item
\end{itemize}

Advertisement

\protect\hyperlink{after-top}{Continue reading the main story}

\href{/es/section/opinion}{Opinión}

Supported by

\protect\hyperlink{after-sponsor}{Continue reading the main story}

Comentario

\hypertarget{jubilar-a-un-caudillo}{%
\section{Jubilar a un caudillo}\label{jubilar-a-un-caudillo}}

Álvaro Uribe, expresidente de Colombia, redefinió y casi monopolizó el
poder desde 2002. Su detención es una oportunidad única para que ese
país se desprenda del personalismo atávico.

\includegraphics{https://static01.graylady3jvrrxbe.onion/images/2020/08/17/multimedia/17Alvarado-ES-1/merlin_162401571_170239b0-0906-4446-8e51-56db59818ef9-articleLarge.jpg?quality=75\&auto=webp\&disable=upscale}

Por Sinar Alvarado

Es periodista.

\begin{itemize}
\item
  17 de agosto de 2020
\item
  \begin{itemize}
  \item
  \item
  \item
  \item
  \item
  \end{itemize}
\end{itemize}

\href{https://www.nytimes3xbfgragh.onion/newsletters/el-times}{Regístrate
para recibir nuestro boletín} con lo mejor de The New York Times.

\begin{center}\rule{0.5\linewidth}{\linethickness}\end{center}

BOGOTÁ --- Álvaro Uribe nos acostumbró al escepticismo. Su inmunidad
duradera nos llevó a pensar que la ley jamás lo alcanzaría.
\href{https://www.semana.com/nacion/articulo/procesos-judiciales-en-contra-de-alvaro-uribe-velez/691746}{Decenas
de investigaciones} sobre corrupción, espionaje y masacres paramilitares
lo persiguen sin éxito desde hace décadas. Y solo un caso menor, por
fraude procesal y soborno a un testigo, consiguió que la Corte Suprema
de Justicia de Colombia ordenara su detención para evitar que obstruya
el proceso. Una caída sorpresiva que trae nuevas tensiones para el país.

En este proceso cargado de interpretaciones políticas, Colombia necesita
lo improbable: que los magistrados actúen sin presiones. Para que haya
justicia, por supuesto. Pero también, y más importante, para confirmar
que nuestra democracia y sus instituciones están por encima del caudillo
que ha dominado la política nacional durante las dos últimas décadas.
Esta es la mejor forma de demostrar que todos somos iguales ante la ley.

Uribe redefinió y casi monopolizó el poder en este país desde que ganó
la presidencia por primera vez en 2002. Cuatro años después, entre visos
de ilegalidad, antiguos aliados dicen que compró votos en el Congreso
para cambiar la constitución y aspirar a un segundo mandato, que terminó
en 2010. Entonces buscó un tercero, pero
\href{https://www.eltiempo.com/archivo/documento/CMS-7304227}{lo detuvo}
la Corte Constitucional. El líder recurrió a una estrategia que ha sido
común en Latinoamérica, y prolongó su influencia endosando votos a dos
candidatos-pupilos ---Juan Manuel Santos en 2010 e Iván Duque en 2018---
que logró convertir en presidentes.

Es el peligro de los caudillos, que someten sin mayor resistencia
nuestras democracias inmaduras. Con hechos se convencen de su destino
manifiesto. Se creen omnímodos, infalibles, eternos. Y el ecosistema
republicano a su alrededor termina por creerse la farsa. Al final los
países quedan atrapados en ese círculo, donde un jefe ---el único---
convence a muchos de su carácter imprescindible.

Alérgico al retiro, Álvaro Uribe ha cazado sucesivas peleas como
expresidente y después como el senador más votado de la historia
reciente colombiana. Durante los últimos años se ha opuesto de manera
enconada al acuerdo de paz que su sucesor, Santos, firmó con las Fuerzas
Armadas Revolucionarias de Colombia (Farc). En su partido, el Centro
Democrático, recurrieron a
\href{https://www.elcolombiano.com/colombia/acuerdos-de-gobierno-y-farc/entrevista-a-juan-carlos-velez-sobre-la-estrategia-de-la-campana-del-no-en-el-plebiscito-CE5116400}{campañas
sucias} para convencer al electorado del supuesto peligro que corría el
país frente a un armisticio negociado con la insurgencia. Y bajo el
gobierno de Iván Duque, heredero de Uribe, diversas voces advierten que
este acuerdo valioso corre el riesgo de ser abortado. El uribismo, de
nuevo en el poder, es una piedra en el zapato de una paz que aún cojea.

Después de la detención, los defensores de Uribe han propagado un falso
dilema. Dicen que es inaceptable ver al expresidente en cautiverio
mientras los antiguos líderes de las Farc siguen en libertad. Es
llamativo que planteen una equivalencia entre el antiguo jefe de Estado
y un grupo irregular que secuestró y asesinó hasta poner a la sociedad
colombiana en jaque. Pero además la indignación es infundada: los
excombatientes están sometidos a la Jurisdicción Especial para la Paz,
un mecanismo de justicia transicional que investiga a los guerrilleros
desmovilizados y puede juzgarlos cuando concluyan sus deliberaciones.

Las reacciones confirman el enorme peso que aún ejerce Uribe en nuestra
vida política. Sus detractores tocaron cacerolas en varias ciudades y
acumularon mensajes de júbilo en las redes sociales. Mientras sus
adeptos, con indignación, organizaban largas caravanas de apoyo.

Cuando se conoció la orden de arresto domiciliario, el presidente Duque
defendió rápidamente a su mentor.
\href{https://twitter.com/IvanDuque/status/1290755832330813442}{Apeló
enseguida} a su ``inocencia y honorabilidad'', en una movida que
\href{https://www.france24.com/es/20200807-colombia-duque-extralimitado-detenci\%C3\%B3n-domiciliaria-uribe}{algunos
juristas} consideran inaceptable por la influencia que puede generar
sobre el proceso. Duque ignoró las pruebas que tiene la Corte, una
institución independiente del poder Ejecutivo, y elogió la virtud sin
grietas del líder caído. Como un feligrés. Como el creyente de un culto
que se aferra al dogma sin dudas. Otros apóstoles fueron más lejos y
propusieron una asamblea constituyente para reformar el poder judicial.
Uribe acusa a la Corte de estar politizada y dice que está en campaña
para reformarla vía referendo.

Pero la voluntad de los colombianos no respaldaría esta aventura.
\href{https://www.elespectador.com/noticias/politica/el-64-de-los-colombianos-esta-de-acuerdo-con-la-detencion-de-uribe-datexco/}{Una
encuesta} reveló que la mayoría apoya la detención de Uribe, un antiguo
ídolo popular ahora débil. En las últimas elecciones regionales su
partido derechista salió
\href{https://www.elespectador.com/noticias/politica/uribe-el-gran-derrotado-en-las-elecciones-regionales-2019/}{derrotado}.
Y en su lugar ganaron espacio opciones de centro que emergen como
alternativa a la polarización.

Porque este país está cambiando. Ya no existe un panorama binario que
nos obligue a escoger entre el hombre fuerte y las Farc, su enemigo
histórico. Uribe dejó de ser el patrón a caballo ** que protege al
pueblo de la amenaza guerrillera. Y el grupo subversivo por fin entregó
las armas que lo hacían temible. El combate armado, por medio de la
política, dio paso al debate de ideas. Ahora Colombia, a través de la
justicia y la ley, intenta consolidar un nuevo escenario, más
civilizado, donde el garrote del uribismo resulta anacrónico.

Y ese es el tablero donde debe moverse Iván Duque. El presidente que se
promovió durante su campaña como un boleto a la modernidad necesita
desprenderse de ese legado atávico, y ofrecerle al país una nueva forma
de gobernar, apegada a la legalidad y respetuosa de la independencia de
poderes. Las condiciones básicas de toda democracia moderna.

El mayor desafío, sin embargo, lo tienen las instituciones que dan forma
al Estado junto al Poder Judicial. Inocente o culpable, Uribe es un
político que va de salida. Su posible juicio, como algunos piensan, no
busca determinar cuán bueno o malo fue como presidente. Se trata de
juzgar su conducta en este caso, y determinar si estuvo o no por fuera
de la ley. Más allá del veredicto seguirán pendientes varios asuntos
mayores: el desempleo
\href{https://www.dane.gov.co/index.php/estadisticas-por-tema/mercado-laboral/empleo-y-desempleo}{más
alto} en años, el asesinato de
\href{https://www.elespectador.com/colombia2020/pais/farc-denuncia-que-200-excombatientes-han-sido-asesinados/}{excombatientes}
y
\href{https://www.elespectador.com/colombia2020/pais/despues-del-acuerdo-de-paz-el-52-de-los-lideres-sociales-fueron-asesinados-en-este-gobierno/}{líderes
sociales}, las altas cifras de
\href{https://elpais.com/internacional/2020-07-17/el-impacto-economico-de-la-pandemia-frena-el-optimismo-por-la-reduccion-de-la-pobreza-en-colombia.html}{pobreza
tras la pandemia}, entre muchas otras. Porque la agenda política de
Colombia, por fortuna, cada vez depende menos de su destino individual.

El menguado protagonismo de Álvaro Uribe es una oportunidad única. Por
primera vez el país podrá decidir sus prioridades sin él como una
variable decisiva. Y su ejemplo podría servir además para conjurar la
aparición de cualquier otro caudillo potencial. Las naciones son
---deben ser--- construcciones colectivas, y no pueden manejarse más
como fincas desmesuradas cuyo porvenir lo traza un capataz intocable.

Sinar Alvarado es periodista y escribe sobre Colombia para medios
internacionales.

Advertisement

\protect\hyperlink{after-bottom}{Continue reading the main story}

\hypertarget{site-index}{%
\subsection{Site Index}\label{site-index}}

\hypertarget{site-information-navigation}{%
\subsection{Site Information
Navigation}\label{site-information-navigation}}

\begin{itemize}
\tightlist
\item
  \href{https://help.nytimes3xbfgragh.onion/hc/en-us/articles/115014792127-Copyright-notice}{©~2020~The
  New York Times Company}
\end{itemize}

\begin{itemize}
\tightlist
\item
  \href{https://www.nytco.com/}{NYTCo}
\item
  \href{https://help.nytimes3xbfgragh.onion/hc/en-us/articles/115015385887-Contact-Us}{Contact
  Us}
\item
  \href{https://www.nytco.com/careers/}{Work with us}
\item
  \href{https://nytmediakit.com/}{Advertise}
\item
  \href{http://www.tbrandstudio.com/}{T Brand Studio}
\item
  \href{https://www.nytimes3xbfgragh.onion/privacy/cookie-policy\#how-do-i-manage-trackers}{Your
  Ad Choices}
\item
  \href{https://www.nytimes3xbfgragh.onion/privacy}{Privacy}
\item
  \href{https://help.nytimes3xbfgragh.onion/hc/en-us/articles/115014893428-Terms-of-service}{Terms
  of Service}
\item
  \href{https://help.nytimes3xbfgragh.onion/hc/en-us/articles/115014893968-Terms-of-sale}{Terms
  of Sale}
\item
  \href{https://spiderbites.nytimes3xbfgragh.onion}{Site Map}
\item
  \href{https://help.nytimes3xbfgragh.onion/hc/en-us}{Help}
\item
  \href{https://www.nytimes3xbfgragh.onion/subscription?campaignId=37WXW}{Subscriptions}
\end{itemize}
