Sections

SEARCH

\protect\hyperlink{site-content}{Skip to
content}\protect\hyperlink{site-index}{Skip to site index}

\href{https://www.nytimes3xbfgragh.onion/es/section/america-latina}{América
Latina}

\href{https://myaccount.nytimes3xbfgragh.onion/auth/login?response_type=cookie\&client_id=vi}{}

\href{https://www.nytimes3xbfgragh.onion/section/todayspaper}{Today's
Paper}

\href{/es/section/america-latina}{América Latina}\textbar{}Una escuela
temporal para los niños en busca de asilo

\url{https://nyti.ms/3gaM6pK}

\begin{itemize}
\item
\item
\item
\item
\item
\item
\end{itemize}

\hypertarget{schools-reopening}{%
\subsubsection{\texorpdfstring{\href{https://www.nytimes3xbfgragh.onion/spotlight/schools-reopening?name=styln-coronavirus-schools-reopening\&region=TOP_BANNER\&variant=undefined\&block=storyline_menu_recirc\&action=click\&pgtype=Article\&impression_id=6c72b100-e386-11ea-ba56-c354277706d3}{Schools
Reopening}}{Schools Reopening}}\label{schools-reopening}}

\begin{itemize}
\tightlist
\item
  \href{https://www.nytimes3xbfgragh.onion/2020/08/19/us/colleges-closing-covid.html?name=styln-coronavirus-schools-reopening\&region=TOP_BANNER\&variant=undefined\&block=storyline_menu_recirc\&action=click\&pgtype=Article\&impression_id=6c72b101-e386-11ea-ba56-c354277706d3}{Colleges
  Closing}
\item
  \href{https://www.nytimes3xbfgragh.onion/2020/08/20/us/schools-reopening-nurses-covid.html?name=styln-coronavirus-schools-reopening\&region=TOP_BANNER\&variant=undefined\&block=storyline_menu_recirc\&action=click\&pgtype=Article\&impression_id=6c72b102-e386-11ea-ba56-c354277706d3}{Missing
  School Nurses}
\item
  \href{https://www.nytimes3xbfgragh.onion/2020/08/18/parenting/homeschool-families.html?name=styln-coronavirus-schools-reopening\&region=TOP_BANNER\&variant=undefined\&block=storyline_menu_recirc\&action=click\&pgtype=Article\&impression_id=6c72b103-e386-11ea-ba56-c354277706d3}{Home-Schooling
  Families}
\item
  \href{https://www.nytimes3xbfgragh.onion/2020/08/05/parenting/parents-distance-learning.html?name=styln-coronavirus-schools-reopening\&region=TOP_BANNER\&variant=undefined\&block=storyline_menu_recirc\&action=click\&pgtype=Article\&impression_id=6c72d810-e386-11ea-ba56-c354277706d3}{Prepare
  for Distance Learning}
\end{itemize}

Advertisement

\protect\hyperlink{after-top}{Continue reading the main story}

Supported by

\protect\hyperlink{after-sponsor}{Continue reading the main story}

\hypertarget{una-escuela-temporal-para-los-niuxf1os-en-busca-de-asilo}{%
\section{Una escuela temporal para los niños en busca de
asilo}\label{una-escuela-temporal-para-los-niuxf1os-en-busca-de-asilo}}

Los esfuerzos por educar a los niños en la frontera entre México y
Estados Unidos se han visto frustrados por la pandemia. Unos voluntarios
están llenando ese vacío.

\includegraphics{https://static01.graylady3jvrrxbe.onion/images/2020/08/18/multimedia/18borderschools-ES-00/merlin_175763541_5bd3b445-6b29-48c7-94e0-58947723d57a-articleLarge.jpg?quality=75\&auto=webp\&disable=upscale}

Por Myriam Vidal Valero y Rodrigo Pérez Ortega

\begin{itemize}
\item
  18 de agosto de 2020
\item
  \begin{itemize}
  \item
  \item
  \item
  \item
  \item
  \item
  \end{itemize}
\end{itemize}

\href{https://www.nytimes3xbfgragh.onion/2020/08/18/parenting/migrant-children-school-border.html}{Read
in English}

\href{https://www.nytimes3xbfgragh.onion/newsletters/el-times}{Regístrate
para recibir nuestro boletín} con lo mejor de The New York Times.

\begin{center}\rule{0.5\linewidth}{\linethickness}\end{center}

Ana Morales Becerra, una madre soltera de Michoacán, México, describe su
antiguo hogar como un lugar tranquilo en medio de una guerra de
cárteles. Con tantos narcos en su barrio, en la ciudad de Uruapan,
estaba segura de que nadie se atrevería a entrar a robarle. Pero aún
así, siempre se sintió incómoda porque su rutina diaria ---trabajar en
dos empleos y cuidar a sus hijos--- estaba marcada por el paso de
camionetas llenas de gente armada.

Sin embargo, la gota que derramó el vaso fue cuando las camionetas de
los narcos comenzaron a seguir a sus hijos. ``¡Hasta aquí!'', recuerda
haber dicho. ``Me voy''. Huyendo de la violencia de los cárteles, el
abuso sexual y las amenazas de muerte, dejó su hogar para buscar una
nueva oportunidad en Estados Unidos. En octubre pasado llegó a Tijuana,
México, con sus cuatro hijos, muy poco dinero y sin un lugar donde
quedarse.

Pero buscar asilo, algo que Morales Becerra pensó que sería un proceso
relativamente rápido, resultó ser un pantano administrativo que la
dejaría varada a ella y a su familia durante meses mientras esperaban a
que un juez decidiera su destino. ``No sabía que había que llevar todo
este proceso'', dijo. Con sus vidas en suspenso y sin acceso a trabajos
formales ni a la escuela, han vivido en el albergue Embajadores de
Jesús, a solo cinco kilómetros al sur de la frontera entre Estados
Unidos y México, durante casi un año.

Como la familia Morales Becerra, miles de familias de Centroamérica y
México han llegado a la frontera sur de Estados Unidos en los últimos
años para escapar de la violencia. Los Protocolos de Protección a
Migrantes de la Casa Blanca, también conocido como el programa ``Quédate
en México'', ha
\href{https://www.nytimes3xbfgragh.onion/interactive/2019/08/18/us/mexico-immigration-asylum.html}{obligado
a los migrantes a esperar en México durante meses}, sin garantías de
asilo.

Durante este tiempo, los niños tienen poco o ningún acceso a la
educación formal. ``Jesús, mi hijo más grande, estaba preocupado'', dijo
Morales Becerra. ``Me decía: `Ya perdí un año, mamá, no quiero perder
otro'''.

En la última década, Estados Unidos
\href{https://www.unhcr.org/globaltrends2019/}{registró aproximadamente
1,7 millones} de solicitudes de asilo, según la Agencia de las Naciones
Unidas para los Refugiados. El gobierno de Donald Trump
\href{https://www.migrationpolicy.org/article/refugees-and-asylees-united-states-2018}{redujo
la cantidad de refugiados} que Estados Unidos acepta anualmente de
110.000 en 2017 a 30.000 en 2019, menos del diez por ciento de las
solicitudes presentadas ese año. Entre los que buscan asilo, ``los niños
son mucho más vulnerables'', dijo Germán Casas, un psiquiatra infantil
que vive en Colombia y es el presidente para América Latina de Médicos
Sin Fronteras.

El trauma que algunos experimentan en el camino ---separación familiar,
violencia física, secuestro, abuso sexual y trata de personas--- es
perjudicial para su desarrollo y salud mental, dijo Casas. Muchos niños
migrantes tienen dificultades para regular sus comportamientos y
emociones, manejar el estrés y desarrollar empatía,
\href{https://onlinelibrary.wiley.com/doi/book/10.1002/9780470669280}{según
investigaciones}.

Con poca ayuda del gobierno de Tijuana, voluntarios de ambos lados de la
frontera han intervenido para ofrecer clases a algunos niños. Pero justo
cuando uno de estos proyectos ganaba fuerza, llegó la pandemia de la
COVID-19.

\includegraphics{https://static01.graylady3jvrrxbe.onion/images/2020/08/18/multimedia/18borderschools-ES-01/merlin_175817601_1c53c837-ccbf-44ea-916f-a4b825c09adb-articleLarge.jpg?quality=75\&auto=webp\&disable=upscale}

\hypertarget{llega-un-aula-con-los-colores-del-arcouxedris}{%
\subsection{Llega un aula con los colores del
arcoíris}\label{llega-un-aula-con-los-colores-del-arcouxedris}}

Andrea Rincón Cortés, de 21 años, siente una profunda conexión con los
migrantes. Su padre intentó cruzar la frontera en 1992, pero terminó por
asentarse en Tijuana, donde ella nació y se crio. A medida que crecía,
vio que cruzar era una cuestión de supervivencia para la mayoría de los
migrantes. Cuando era adolescente, comenzó a visitar albergues y a
coordinar donaciones. ``Sentí este acercamiento con ellos porque me veo
reflejada ahí'', dijo.

En julio de 2019, mientras hacía malabares con los cursos universitarios
y su trabajo en una organización sin fines de lucro llamada Border
Angels, Rincón Cortés descubrió el
\href{https://www.schoolboxproject.org/us-mexican-border}{School Box
Project}, una organización internacional que lleva actividades
educativas a niños refugiados en Grecia, Bangladés y Siria. Rápidamente
propuso llevar también estas actividades a los niños migrantes en la
frontera mexicana.

Durante los siguientes meses, ella y otros cuatro voluntarios de ambos
lados de la frontera, a bordo de un autobús escolar con los colores del
arcoíris convertido en un aula móvil, visitaron tres albergues de
Tijuana para dar lecciones de dos horas a los niños. ``Nos enfocamos al
principio en hacer actividades de arteterapia para identificar qué
necesidades educativas y emocionales había'', dijo Rincón Cortés.

Image

En 2019, Andrea Rincón Cortés, de 21 años, se convirtió en voluntaria
para dar clases a niños en los albergues de la frontera mexicana. Cuando
la pandemia obligó a parar su programa de aula móvil fundó su propia
organización sin fines de lucro para darle clases virtuales a los niños
migrantes.Credit...Guillermo Arias para The New York Times

Después de crecer en lugares peligrosos y experimentar traumas durante
su trayecto a la frontera, los niños migrantes desarrollan a menudo
inseguridades permanentes y tienen problemas para relacionarse con el
mundo, explicó Casas. También tienen un
\href{https://oxfordre.com/publichealth/view/10.1093/acrefore/9780190632366.001.0001/acrefore-9780190632366-e-12}{mayor
riesgo de desarrollar trastornos de salud mental} como el trastorno de
estrés postraumático.

La educación y el sentido de la rutina adquieren un significado más
profundo para ellos, según Casas, que ha tratado a niños refugiados
durante más de 20 años.
\href{https://onlinelibrary.wiley.com/doi/10.1002/9780470669280.ch12}{Dijo
que disminuye su ansiedad} al proporcionar un entorno seguro donde
pueden concentrarse en el conocimiento útil, en lugar de la atmósfera
angustiante que los rodea.

En una fría mañana de diciembre del año pasado, abordamos el
autobús-salón de clases en el puerto de entrada El Chaparral y viajamos
con dos voluntarios que darían clases aquel día. La gente miraba el
autobús escolar pintado de arcoíris en medio de una corriente de autos
monótonos en las calles de Tijuana. Tan pronto llegamos al albergue, una
decena de niños salió corriendo para saludarnos, abrazándonos las
piernas y saltando sonrientes. Luego se sentaron a pintar, unos con los
dedos, al azar, otros representaban su viaje por el desierto.

Zaida Guillén, la directora del albergue Embajadores de Jesús, dijo que
las clases cambiaron la conducta de los niños y les permitieron
florecer. ``Los niños se empezaron a integrar, empezaron a tener más
respeto, trabajaban en equipo''.

La escuela móvil pareció distraerlos de sus terribles experiencias,
señaló Dulce García, abogada de inmigración en San Diego y directora de
Border Angels. ``Por lo menos tienen ese momento donde son niños, donde
tienen que hacer tarea o pueden hablar de la situación con un experto''.

Image

Niños migrantes juegan en la iglesia y albergue Embajadores de Jesús, a
unos 5 kilómetros al sur de la frontera entre México y Estados Unidos,
en Tijuana.Credit...Guillermo Arias para The New York Times

\hypertarget{un-trabajo-que-ya-es-difuxedcil-se-vuelve-casi-imposible}{%
\subsection{Un trabajo que ya es difícil se vuelve casi
imposible}\label{un-trabajo-que-ya-es-difuxedcil-se-vuelve-casi-imposible}}

Después de ocho meses, el proyecto del autobús escolar funcionaba sin
problemas. Los niños estaban acostumbrados al horario, confiaban en los
voluntarios (que también enseñaban matemáticas e inglés) y los
extrañaban cuando no podían llegar. ``Ellos ya te ven como parte de sus
vidas'', dijo Rincón Cortés.

Pero luego, en marzo, ambos países cerraron sus fronteras y los
gobiernos emitieron órdenes de quedarse en casa debido a la pandemia. La
directora de School Box Project le dijo a Rincón Cortés que no podían
seguir impartiendo clases de manera segura y que terminaban sus
programas en todo el mundo. Rincón Cortés llevó a los niños al cine como
una excursión de despedida, y luego ella y los voluntarios se quedaron
sin el autobús para continuar con su enseñanza.

Los hijos de Morales Becerra, junto con los otros 75 niños en los tres
albergues, quedaron repentinamente a la deriva, en confinamiento,
mientras sus padres se enteraban que sus citas en la corte para
solicitar asilo
\href{https://www.nytimes3xbfgragh.onion/2020/03/17/us/politics/trump-coronavirus-mexican-border.html}{se
retrasarían} debido al coronavirus. O peor aún: que podrían verse
obligados a
\href{https://www.nytimes3xbfgragh.onion/es/2020/03/19/espanol/america-latina/solicitantes-de-asilo-estados-unidos.html}{volver
a la violencia de la que huían}.

A medida que los donativos y la ayuda disminuyeron durante los
siguientes dos meses, los niños del albergue Embajadores de Jesús
estaban desesperados, estresados y aburridos sin sus lecciones. ``Todas
las ayudas dejaron de venir. Los doctores, las donaciones, el
psicólogo\ldots{} Todo'', dijo Morales Becerra.

Su hijo mayor, Jesús, de 12 años, tenía una copia de \emph{Harry Potter
y la piedra filosofal}, que cuenta las aventuras de un joven mago.
``Como no tenía nada que hacer, lo terminaba y lo volvía a leer. Lo
terminaba y lo volvía a leer'', dijo.

Image

Los cursos virtuales, coordinados por la organización sin fines de lucro
International Activist Youth, ofrecen sentido y estructura a la vida de
los niños migrantes, muchos de los cuales no han estado en un salón de
clase desde que abandonaron sus hogares.Credit...Guillermo Arias para
The New York Times

\hypertarget{los-profesores-se-ponen-creativos}{%
\subsection{Los profesores se ponen
creativos}\label{los-profesores-se-ponen-creativos}}

Cuando vivía en Michoacán, Morales Becerra había sido una madre soltera
que tenía dos trabajos. Ahora, en el albergue durante el confinamiento,
languidecía en depresión. ``Yo no he estado acostumbrada a no hacer
nada, siempre tengo que estar activa'', dijo. ``Yo ya estaba
desesperada''.

\href{https://www.nytimes3xbfgragh.onion/spotlight/schools-reopening?action=click\&pgtype=Article\&state=default\&region=MAIN_CONTENT_3\&context=storylines_keepup}{}

\hypertarget{schools-reopening-}{%
\subsubsection{Schools Reopening ›}\label{schools-reopening-}}

\hypertarget{back-to-school}{%
\paragraph{Back to School}\label{back-to-school}}

Updated Aug. 20, 2020

The latest on how schools are reopening amid the pandemic.

\begin{itemize}
\item
  \begin{itemize}
  \tightlist
  \item
    Much more is
    \href{https://www.nytimes3xbfgragh.onion/2020/08/20/us/schools-reopening-nurses-covid.html?action=click\&pgtype=Article\&state=default\&region=MAIN_CONTENT_3\&context=storylines_keepup}{expected
    of America's school nurses} during the pandemic, but many schools
    don't have one.
  \item
    A vast majority of parents have resigned themselves to
    \href{https://www.nytimes3xbfgragh.onion/2020/08/19/us/colleges-closing-covid.html?action=click\&pgtype=Article\&state=default\&region=MAIN_CONTENT_3\&context=storylines_keepup}{going
    it alone in the pandemic school year}, according to a new survey for
    The New York Times.
  \item
    Alabama is betting that a
    \href{https://www.nytimes3xbfgragh.onion/2020/08/19/business/alabama-uab-coronavirus-tests.html?action=click\&pgtype=Article\&state=default\&region=MAIN_CONTENT_3\&context=storylines_keepup}{robust
    student testing and technology program} will be enough to hinder
    outbreaks on college campuses.
  \item
    We want to hear from teachers making difficult choices. How are you
    thinking about the start of the school year?
    \href{https://www.nytimes3xbfgragh.onion/2020/08/19/us/teachers-school-reopenings.html?action=click\&pgtype=Article\&state=default\&region=MAIN_CONTENT_3\&context=storylines_keepup}{Tell
    us here}.
  \end{itemize}
\end{itemize}

Cuando se dio cuenta de que su hijo menor, Axel, de cinco años, no
recordaba la mayor parte de lo que había aprendido en la guardería el
año anterior, le preguntó a Guillén si podían iniciar clases informales
para los más pequeños, y pronto se encontró enseñando matemáticas y
lectura a los habitantes más jóvenes del albergue.

Mientras tanto, al otro lado de la ciudad, Rincón Cortés elaboraba su
propio plan para seguir con la enseñanza. Para ella, era algo más que
ofrecer clases a los niños. Quería hacerles sentir que alguien los
estaba cuidando, dijo. ``Que ellos importan''.

Fundó su propia organización sin fines de lucro, llamada
\href{https://www.internationalactivistyouth.com/}{International
Activist Youth}, y reclutó a otros estudiantes universitarios para
ayudar a enseñar. Pero era obvio que el aprendizaje a distancia era la
única forma segura de llegar a los niños. Recurrir a métodos en línea
significaba que tenían que establecer en el albergue un servicio de
internet confiable y computadoras. Una donación de 500 dólares los ayudó
a impulsar el nuevo proyecto.

Para julio, tenían ya una conexión a internet en el albergue, y trajeron
proyectores, bocinas, sillas y otros materiales donados para las clases.
Rincón Cortés también tuvo que entrenar a los maestros voluntarios a
interactuar con los niños migrantes. Los pequeños detalles, como
aprenderse el nombre de un niño o reconocer activamente su trabajo, les
da a los pequeños un sentido de confianza en sí mismos y dignidad.

A mediados de julio, empezaron a enseñar. Rincón Cortés y su equipo de
14 voluntarios ahora brindan más lecciones de las que podían con el
autobús escolar. Matemáticas, inglés, lectura y arte en línea ocupan la
mayor parte de los días de los niños. ``Ya mis hijos me advirtieron que
no los voy a ver en todo el día por tanta actividad que hay'', dijo
Morales Becerra, riendo.

Aunque su hijo Jesús extraña la interacción en persona con sus maestros,
le gusta tener más lecciones. También hay
\href{https://www.nytimes3xbfgragh.onion/2020/05/20/nyregion/coronavirus-students-schools.html}{otros
aspectos positivos}. ``Me siento mejor porque si estuvieran aquí los
maestros me daría más vergüenza'', dijo Jesús, quien siempre ha sido
tímido. Ahora que las lecciones son en línea, participa más.

Su hermano menor, Axel, también está ocupado con las clases. ``Apenas
estoy aprendiendo a leer'', dijo. ``Puedo leer: `Mamá me ama'''.

Ambos niños siguen soñando con su futuro. Mientras Jesús quiere
convertirse en biólogo marino o arquitecto, Axel se debate entre ser
policía, soldado o pizzero.

Image

Dos estudiantes ríen durante una clase virtual en la iglesia y albergue
de Tijuana en agosto.Credit...Guillermo Arias para The New York Times

Las clases virtuales también incluyen lecciones sobre
los\href{https://www.unicef.org/es/convencion-derechos-nino/convencion-version-ninos}{derechos
internacionales básicos del niño}, como el derecho a tener un hogar
seguro, a estar protegido contra la violencia o a recibir una educación.
``Vemos un derecho por sesión'', dijo Rincón Cortés. Esto ayuda a
preparar tanto a los niños como a los padres para reconocer los abusos y
la violencia. El nuevo programa también ayudará a las familias a ponerse
en contacto con consejeros y organizaciones para obtener asesoramiento
legal o psicológico.

Aunque México y Estados Unidos han comenzado a abrirse después del
confinamiento, Rincón Cortés planea continuar con las clases virtuales.
Morales Becerra dijo que ella y muchos otros padres comienzan a
encontrar estabilidad y un sentido de esperanza, aunque su objetivo aún
es cruzar la frontera después de que se reanuden las citas en las
cortes.

``Tengo muchos planes'', dijo. ``Quiero estudiar y solo tengo la
secundaria y espero me ayude para sacar a mis chiquillos adelante''.

\begin{center}\rule{0.5\linewidth}{\linethickness}\end{center}

Myriam Vidal Valero es una periodista mexicana que cubre salud y
ciencia. Es miembro de la Red Mexicana de Periodistas de Ciencia.

Rodrigo Pérez Ortega es un periodista radicado en Washington D. C. que
cubre salud y ciencia.

\emph{La reportería para esta historia fue financiada por la Beca
Rosalynn Carter para Periodismo en Salud Mental.}

Advertisement

\protect\hyperlink{after-bottom}{Continue reading the main story}

\hypertarget{site-index}{%
\subsection{Site Index}\label{site-index}}

\hypertarget{site-information-navigation}{%
\subsection{Site Information
Navigation}\label{site-information-navigation}}

\begin{itemize}
\tightlist
\item
  \href{https://help.nytimes3xbfgragh.onion/hc/en-us/articles/115014792127-Copyright-notice}{©~2020~The
  New York Times Company}
\end{itemize}

\begin{itemize}
\tightlist
\item
  \href{https://www.nytco.com/}{NYTCo}
\item
  \href{https://help.nytimes3xbfgragh.onion/hc/en-us/articles/115015385887-Contact-Us}{Contact
  Us}
\item
  \href{https://www.nytco.com/careers/}{Work with us}
\item
  \href{https://nytmediakit.com/}{Advertise}
\item
  \href{http://www.tbrandstudio.com/}{T Brand Studio}
\item
  \href{https://www.nytimes3xbfgragh.onion/privacy/cookie-policy\#how-do-i-manage-trackers}{Your
  Ad Choices}
\item
  \href{https://www.nytimes3xbfgragh.onion/privacy}{Privacy}
\item
  \href{https://help.nytimes3xbfgragh.onion/hc/en-us/articles/115014893428-Terms-of-service}{Terms
  of Service}
\item
  \href{https://help.nytimes3xbfgragh.onion/hc/en-us/articles/115014893968-Terms-of-sale}{Terms
  of Sale}
\item
  \href{https://spiderbites.nytimes3xbfgragh.onion}{Site Map}
\item
  \href{https://help.nytimes3xbfgragh.onion/hc/en-us}{Help}
\item
  \href{https://www.nytimes3xbfgragh.onion/subscription?campaignId=37WXW}{Subscriptions}
\end{itemize}
