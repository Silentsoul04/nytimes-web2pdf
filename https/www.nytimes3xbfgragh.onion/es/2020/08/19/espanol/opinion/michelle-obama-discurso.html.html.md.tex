Sections

SEARCH

\protect\hyperlink{site-content}{Skip to
content}\protect\hyperlink{site-index}{Skip to site index}

\href{https://www.nytimes3xbfgragh.onion/es/section/opinion}{Opinión}

\href{https://myaccount.nytimes3xbfgragh.onion/auth/login?response_type=cookie\&client_id=vi}{}

\href{https://www.nytimes3xbfgragh.onion/section/todayspaper}{Today's
Paper}

\href{/es/section/opinion}{Opinión}\textbar{}Michelle Obama nos mostró
cuál es nuestra mejor esperanza

\url{https://nyti.ms/31bcnjJ}

\begin{itemize}
\item
\item
\item
\item
\item
\end{itemize}

\hypertarget{el-brote-de-coronavirus}{%
\subsubsection{\texorpdfstring{\href{https://www.nytimes3xbfgragh.onion/es/spotlight/coronavirus?name=styln-coronavirus-es\&region=TOP_BANNER\&variant=undefined\&block=storyline_menu_recirc\&action=click\&pgtype=Article\&impression_id=12b121f0-e387-11ea-8198-d359754c6cb2}{El
brote de
coronavirus}}{El brote de coronavirus}}\label{el-brote-de-coronavirus}}

\begin{itemize}
\tightlist
\item
  \href{https://www.nytimes3xbfgragh.onion/es/interactive/2020/espanol/america-latina/coronavirus-en-mexico.html?name=styln-coronavirus-es\&region=TOP_BANNER\&variant=undefined\&block=storyline_menu_recirc\&action=click\&pgtype=Article\&impression_id=12b14900-e387-11ea-8198-d359754c6cb2}{Mapa
  y casos en México}
\item
  \href{https://www.nytimes3xbfgragh.onion/es/interactive/2020/08/06/espanol/ciencia-y-tecnologia/tengo-covid-19-sintomas.html?name=styln-coronavirus-es\&region=TOP_BANNER\&variant=undefined\&block=storyline_menu_recirc\&action=click\&pgtype=Article\&impression_id=12b14901-e387-11ea-8198-d359754c6cb2}{Identifica
  los síntomas}
\item
  \href{https://www.nytimes3xbfgragh.onion/es/interactive/2020/science/coronavirus-tratamientos-curas.html?name=styln-coronavirus-es\&region=TOP_BANNER\&variant=undefined\&block=storyline_menu_recirc\&action=click\&pgtype=Article\&impression_id=12b14902-e387-11ea-8198-d359754c6cb2}{Fármacos
  y tratamientos}
\item
  \href{https://www.nytimes3xbfgragh.onion/es/2020/04/29/espanol/estilos-de-vida/oximetro-para-que-sirve.html?name=styln-coronavirus-es\&region=TOP_BANNER\&variant=undefined\&block=storyline_menu_recirc\&action=click\&pgtype=Article\&impression_id=12b14903-e387-11ea-8198-d359754c6cb2}{¿Necesitas
  un oxímetro?}
\item
  \href{https://www.nytimes3xbfgragh.onion/es/2020/07/02/espanol/ciencia-y-tecnologia/sobrevivientes-coronavirus-recuperacion.html?name=styln-coronavirus-es\&region=TOP_BANNER\&variant=undefined\&block=storyline_menu_recirc\&action=click\&pgtype=Article\&impression_id=12b14904-e387-11ea-8198-d359754c6cb2}{Las
  secuelas del virus}
\end{itemize}

Advertisement

\protect\hyperlink{after-top}{Continue reading the main story}

\href{/es/section/opinion}{Opinión}

Supported by

\protect\hyperlink{after-sponsor}{Continue reading the main story}

Comentario

\hypertarget{michelle-obama-nos-mostruxf3-cuuxe1l-es-nuestra-mejor-esperanza}{%
\section{Michelle Obama nos mostró cuál es nuestra mejor
esperanza}\label{michelle-obama-nos-mostruxf3-cuuxe1l-es-nuestra-mejor-esperanza}}

En Estados Unidos puedes ``odiar la política'', pero no puedes quedarte
fuera de estas elecciones.

\includegraphics{https://static01.graylady3jvrrxbe.onion/images/2020/08/18/opinion/19Bruni-ES/18bruni1-articleLarge.jpg?quality=75\&auto=webp\&disable=upscale}

\href{https://www.nytimes3xbfgragh.onion/by/frank-bruni}{\includegraphics{https://static01.graylady3jvrrxbe.onion/images/2018/04/03/opinion/frank-bruni/frank-bruni-thumbLarge.png}}

Por \href{https://www.nytimes3xbfgragh.onion/by/frank-bruni}{Frank
Bruni}

Es columnista de The New York Times.

\begin{itemize}
\item
  19 de agosto de 2020
\item
  \begin{itemize}
  \item
  \item
  \item
  \item
  \item
  \end{itemize}
\end{itemize}

\href{https://www.nytimes3xbfgragh.onion/2020/08/18/opinion/michelle-obama-dnc-election-2020.html}{Read
in English}

\href{https://www.nytimes3xbfgragh.onion/newsletters/el-times}{Regístrate
para recibir nuestro boletín} con lo mejor de The New York Times.

\begin{center}\rule{0.5\linewidth}{\linethickness}\end{center}

Supongo que debería cuestionar si el Partido Demócrata realmente puede
expandirse lo suficiente en ambas direcciones como para incluir tanto a
un demócrata socialista como Bernie Sanders como a un republicano
arrepentido como John Kasich, compañeros fundamentalmente extraños que
fueron oradores y funcionaron como polos ideológicos en la noche
inaugural de la
\href{https://www.nytimes3xbfgragh.onion/es/2020/08/18/espanol/estados-unidos/horario-cnd-convencion-democrata.html}{Convención
Nacional Demócrata}, o como sea que llamemos a ese jubileo político, más
sucinto y deformado por la pandemia que no sucede en Milwaukee.

Tal vez debería describir esa deformación, analizar los ajustes
correspondientes y encontrar errores aquí, allá y en todas partes. Eso
es lo que hacemos los expertos. Objetamos. Disentimos. En ese sentido,
somos insoportables.

Pero este no es el momento para hacer lo que solíamos hacer siempre. No
es un momento normal. No me refiero al coronavirus en particular o a la
histórica elección de una mujer negra como candidata a la
vicepresidencia del Partido Demócrata ni a una faceta o una dinámica
específica de los días que tenemos por delante. Me refiero a lo que está
en juego en los próximos días. Es inconmensurable.

En ningún otro momento de mis 55 años de vida, el éxito de los
demócratas ha sido más importante para el bienestar, la cordura ---el
futuro--- de Estados Unidos que en este, porque la alternativa nunca
había sido un presidente republicano tan profundamente amoral,
fundamentalmente corrupto y llanamente incompetente como el que busca
gobernar cuatro años más.

Donald Trump ha dejado claro que, si es el único modo de ``ganar'', está
dispuesto a robarle esta elección a Joe Biden. De hecho, ya inició el
atraco. Está listo para destruir por completo la fe en nuestras
instituciones y el orgullo que nos provoca el sistema democrático, y
construir un trono entre los escombros. Y tiene un número enorme de
cómplices, entre ellos ---hasta la fecha--- la mayoría de los
republicanos en el Congreso, que lo animan a continuar o se muerden
tanto la lengua que les escurre sangre de la boca.

En ese contexto, lo que vi el lunes en la noche no fue un evento que
pueda analizar o calificar. Se trató de algo a lo que hay que acercarnos
a toda prisa y atesorar: un bufé para los hambrientos. Fue la salvación.
No tengo ningún interés en discernir si el puré de papas estaba tan
esponjoso como debería estar o si los espárragos estaban demasiado
cocidos.

En cambio, quiero señalar que no hay nadie en el círculo cercano de
Trump con la seriedad y la gracia de Michelle Obama, porque alguien como
ella no duraría ni un nanosegundo allí. Trump consideraría que su
ejemplo es demasiado amenazante, el estándar que impone lo
empequeñecería demasiado. Y para ella, el ecosistema ético de Trump le
parecería inhabitable: la fría y oscura superficie de la Luna sin un
traje espacial.

Quiero disfrutar de cada palabra que ella pronunció el lunes, cuando
destiló de manera tan bella lo que está mal con Trump: ``Él simplemente
no puede ser quien necesitamos que sea'', dijo; y de esa forma
inquietante definió lo que se siente vivir en el Estados Unidos de
Trump.

``Los niños en este país están viendo qué es lo que sucede cuando
dejamos de exigir empatía entre nosotros'', dijo. ``Están viendo a su
alrededor y se preguntan si todo este tiempo les hemos estado mintiendo
sobre lo que somos en realidad y sobre lo que en verdad valoramos''.

``Lo que está ocurriendo no está bien'', añadió. ``Esto no es quienes
queremos ser''. Fue un discurso excelente ---pronunciado de manera
espléndida---, y lo fue porque en buena medida Michelle Obama reconoció
y acentuó el hecho de que muchos estadounidenses la ven como una persona
menos partidista y más pragmática que cualquier orador tradicional de
las convenciones.

``Ya saben que odio la política'', dijo sin pedir perdón por esto.
``Pero también saben que esta nación me importa. Saben lo mucho que me
importan todos nuestros niños''. Desde esta perspectiva, vino su
petición: ``Tenemos que votar por Joe Biden como si nuestras vidas
dependieran de ello''.

Ese fue el mensaje de muchos de los oradores de la convención. Sus
palabras variaron, pero no la urgencia. Cuando los políticos usan tonos
épicos y graves, por lo general dan ganas de lavar con cloro tanta
grandilocuencia. Pero el lunes en la noche solo asentí (y, como Obama,
también derramé alguna lágrima).

Antes del inicio de la convención se habló mucho sobre el deseo
apremiante y el esfuerzo constante de los demócratas por proyectar
unidad. Para mis ojos y oídos, llevaron las cosas a un nivel más alto,
no porque tuvieran un talento especial para la diplomacia o un don para
la coreografía televisiva, sino porque clara, genuina y apasionadamente
comparten la convicción de que si Biden falla, Estados Unidos cae; al
menos el Estados Unidos al que todavía estamos tratando de aferrarnos,
el Estados Unidos sobre el que cantamos y al que nos referimos cuando
ponemos las manos sobre nuestros corazones y miramos hacia la bandera.

``Durante el gobierno de este presidente, lo impensable se ha convertido
en lo normal'', dijo Sanders durante su discurso. ``El futuro de nuestra
democracia está en riesgo. El futuro de nuestra economía está en riesgo.
El futuro de nuestro planeta está en riesgo''.

Sanders, de manera deliberada y específica, exhortó a sus seguidores
frustrados a alinearse y votar por Biden. ``Debemos estar juntos'',
dijo. ``El precio del fracaso, amigos míos, es demasiado grande para
imaginarlo siquiera''. No sonaba así en 2016. En ese entonces los daños
que causaría Trump en la presidencia eran apenas predicciones, eran
teóricos. Ahora los daños se ven en las cifras de desempleo y en las
morgues.

Esta convención no está exenta de peleas ideológicas, egos lastimados y
otros dramas fuera de la pantalla. Pero cualquiera que se concentre en
eso perderá de vista el panorama general. Cualquiera que esté
concentrado en eso no será capaz de reconocer cuántos de los elogios que
le hicieron a Biden el lunes nunca se le podrían dirigir a Trump.

Me refiero a cumplidos simples, como cuando Kasich halagó a Biden
refiriéndose a él como ``un hombre de fe, un unificador''. O como cuando
el congresista James Clyburn se refirió a él como ``tan buen hombre como
buen líder''. Ni siquiera los mentirosos experimentados que apoyan a
Trump intentan salirse con la suya llamándolo ``bueno''. Son réprobos,
no comediantes.

Biden no reúne todo lo que me gustaría que tuviera un presidente, pero
tiene bondad, y para un país que necesita recuperar su decencia,
refundar su dignidad y enmendar su camino, ese no es mal lugar para
empezar.

La convención de los demócratas continuará por dos noches más, pero no
necesito ver un segundo más para saber cómo está la cosa.

Frank Bruni ha trabajado en el Times desde 1995 y, antes de convertirse
en columnista en 2011, ha ocupado diversos cargos, incluido el de
reportero de la Casa Blanca, jefe de la corresponsalía de Roma y crítico
jefe de restaurantes. \href{https://twitter.com/FrankBruni}{@FrankBruni}
\textbar{}
\href{https://www.facebookcorewwwi.onion/frankbruninyt}{Facebook}

Advertisement

\protect\hyperlink{after-bottom}{Continue reading the main story}

\hypertarget{site-index}{%
\subsection{Site Index}\label{site-index}}

\hypertarget{site-information-navigation}{%
\subsection{Site Information
Navigation}\label{site-information-navigation}}

\begin{itemize}
\tightlist
\item
  \href{https://help.nytimes3xbfgragh.onion/hc/en-us/articles/115014792127-Copyright-notice}{©~2020~The
  New York Times Company}
\end{itemize}

\begin{itemize}
\tightlist
\item
  \href{https://www.nytco.com/}{NYTCo}
\item
  \href{https://help.nytimes3xbfgragh.onion/hc/en-us/articles/115015385887-Contact-Us}{Contact
  Us}
\item
  \href{https://www.nytco.com/careers/}{Work with us}
\item
  \href{https://nytmediakit.com/}{Advertise}
\item
  \href{http://www.tbrandstudio.com/}{T Brand Studio}
\item
  \href{https://www.nytimes3xbfgragh.onion/privacy/cookie-policy\#how-do-i-manage-trackers}{Your
  Ad Choices}
\item
  \href{https://www.nytimes3xbfgragh.onion/privacy}{Privacy}
\item
  \href{https://help.nytimes3xbfgragh.onion/hc/en-us/articles/115014893428-Terms-of-service}{Terms
  of Service}
\item
  \href{https://help.nytimes3xbfgragh.onion/hc/en-us/articles/115014893968-Terms-of-sale}{Terms
  of Sale}
\item
  \href{https://spiderbites.nytimes3xbfgragh.onion}{Site Map}
\item
  \href{https://help.nytimes3xbfgragh.onion/hc/en-us}{Help}
\item
  \href{https://www.nytimes3xbfgragh.onion/subscription?campaignId=37WXW}{Subscriptions}
\end{itemize}
