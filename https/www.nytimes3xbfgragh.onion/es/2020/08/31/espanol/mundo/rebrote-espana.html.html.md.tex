Sections

SEARCH

\protect\hyperlink{site-content}{Skip to
content}\protect\hyperlink{site-index}{Skip to site index}

\href{https://www.nytimes3xbfgragh.onion/es/section/mundo}{Mundo}

\href{https://myaccount.nytimes3xbfgragh.onion/auth/login?response_type=cookie\&client_id=vi}{}

\href{https://www.nytimes3xbfgragh.onion/section/todayspaper}{Today's
Paper}

\href{/es/section/mundo}{Mundo}\textbar{}`Allá vamos otra vez': España
vive una segunda ola de coronavirus

\url{https://nyti.ms/3lz8Bss}

\begin{itemize}
\item
\item
\item
\item
\item
\item
\end{itemize}

\hypertarget{el-brote-de-coronavirus}{%
\subsubsection{\texorpdfstring{\href{https://www.nytimes3xbfgragh.onion/es/spotlight/coronavirus?name=styln-coronavirus-es\&region=TOP_BANNER\&block=storyline_menu_recirc\&action=click\&pgtype=Article\&impression_id=bfd38190-efb5-11ea-9d10-87f4af94003e\&variant=undefined}{El
brote de
coronavirus}}{El brote de coronavirus}}\label{el-brote-de-coronavirus}}

\begin{itemize}
\tightlist
\item
  \href{https://www.nytimes3xbfgragh.onion/es/interactive/2020/08/06/espanol/ciencia-y-tecnologia/tengo-covid-19-sintomas.html?name=styln-coronavirus-es\&region=TOP_BANNER\&block=storyline_menu_recirc\&action=click\&pgtype=Article\&impression_id=bfd38191-efb5-11ea-9d10-87f4af94003e\&variant=undefined}{Síntomas}
\item
  \href{https://www.nytimes3xbfgragh.onion/es/2020/09/02/espanol/ciencia-y-tecnologia/vacunas-experimentales-coronavirus.html?name=styln-coronavirus-es\&region=TOP_BANNER\&block=storyline_menu_recirc\&action=click\&pgtype=Article\&impression_id=bfd38192-efb5-11ea-9d10-87f4af94003e\&variant=undefined}{Vacunas
  experimentales}
\item
  \href{https://www.nytimes3xbfgragh.onion/es/2020/08/31/espanol/mundo/rebrote-espana.html?name=styln-coronavirus-es\&region=TOP_BANNER\&block=storyline_menu_recirc\&action=click\&pgtype=Article\&impression_id=bfd3a8a0-efb5-11ea-9d10-87f4af94003e\&variant=undefined}{Rebrote
  en España}
\item
  \href{https://www.nytimes3xbfgragh.onion/es/2020/09/02/espanol/negocios/desalojos-trump.html?name=styln-coronavirus-es\&region=TOP_BANNER\&block=storyline_menu_recirc\&action=click\&pgtype=Article\&impression_id=bfd3a8a1-efb5-11ea-9d10-87f4af94003e\&variant=undefined}{Moratoria
  a los desalojos}
\item
  \href{https://www.nytimes3xbfgragh.onion/es/2020/08/26/espanol/ciencia-y-tecnologia/coronavirus-afecta-hombres.html?name=styln-coronavirus-es\&region=TOP_BANNER\&block=storyline_menu_recirc\&action=click\&pgtype=Article\&impression_id=bfd3a8a2-efb5-11ea-9d10-87f4af94003e\&variant=undefined}{El
  impacto en los hombres}
\end{itemize}

Advertisement

\protect\hyperlink{after-top}{Continue reading the main story}

Supported by

\protect\hyperlink{after-sponsor}{Continue reading the main story}

Europa

\hypertarget{alluxe1-vamos-otra-vez-espauxf1a-vive-una-segunda-ola-de-coronavirus}{%
\section{`Allá vamos otra vez': España vive una segunda ola de
coronavirus}\label{alluxe1-vamos-otra-vez-espauxf1a-vive-una-segunda-ola-de-coronavirus}}

El coronavirus se extiende mucho más rápido en ese país que en cualquier
otro de Europa. Después de una relativa pausa durante el verano, los
expertos temen que sea la señal de un rebrote en todo el continente.

\includegraphics{https://static01.graylady3jvrrxbe.onion/images/2020/08/31/world/31Spain-ES-00/31virus-spain1-articleLarge.jpg?quality=75\&auto=webp\&disable=upscale}

Por
\href{https://www.nytimes3xbfgragh.onion/by/patrick-kingsley}{Patrick
Kingsley} y José Bautista

\begin{itemize}
\item
  31 de agosto de 2020
\item
  \begin{itemize}
  \item
  \item
  \item
  \item
  \item
  \item
  \end{itemize}
\end{itemize}

\href{https://www.nytimes3xbfgragh.onion/2020/08/31/world/europe/coronavirus-covid-spain-second-wave.html}{Read
in English}

\href{https://www.nytimes3xbfgragh.onion/newsletters/el-times}{Regístrate
para recibir nuestro boletín} con lo mejor de The New York Times.

\begin{center}\rule{0.5\linewidth}{\linethickness}\end{center}

MÁLAGA, España --- El domingo a mediodía había 31 pacientes dentro del
principal centro de tratamiento de coronavirus en Málaga, la ciudad con
la tasa de infección de más rápido crecimiento del sur de España. A las
12:15 p.m., el paciente 32 llegó en una ambulancia. Media hora después
llegó el número 33.

El cubo de basura de la puerta se desbordó con cubrebocas y guantes
quirúrgicos azules. Afuera, los parientes rondaban en silencio, uno de
ellos lloraba, otra sentía una punzada de \emph{déjà-vu}.

``Mi cuñado tuvo el virus en primavera'', dijo Julia Bautista, una
administradora de oficina jubilada de 58 años que el domingo esperaba
noticias de su padre, de 91 años.

``Allá vamos otra vez'', añadió.

Si en febrero Italia fue el heraldo de la primera ola de la pandemia de
coronavirus en Europa, España es el presagio de la segunda.

Francia también crece, al igual que algunas partes de Europa del Este, y
los casos también aumentan en Alemania, Grecia, Italia y Bélgica. Pero
en la última semana, España ha registrado con diferencia el mayor número
de nuevos casos en el continente,
\href{https://www.nytimes3xbfgragh.onion/interactive/2020/world/europe/spain-coronavirus-cases.html}{más
de 53.000}. Con 114 nuevas infecciones por cada 100.000 personas en ese
tiempo, el virus se propaga más rápidamente en España que en Estados
Unidos, más del doble de rápido que en Francia, alrededor de ocho veces
la tasa en Italia y Gran Bretaña, y diez veces la tasa en Alemania.

España ya era uno de los países más afectados de Europa, y ahora tiene
alrededor de 440.000 casos y más de 29.000 muertes. Pero después de
\href{https://www.nytimes3xbfgragh.onion/2020/04/07/world/europe/spain-coronavirus.html?searchResultPosition=119}{uno
de los cierres más estrictos del mundo}, que frenó la propagación del
virus, disfrutó de una de las reaperturas más rápidas. El regreso de la
vida nocturna y las actividades de grupo ---mucho más veloz que en la
mayoría de sus vecinos europeos--- ha contribuido al resurgimiento de la
epidemia.

\includegraphics{https://static01.graylady3jvrrxbe.onion/images/2020/08/31/world/31Spain-ES-01/merlin_176406663_8bc9e92e-d9ac-42bc-999d-84b3bd334d23-articleLarge.jpg?quality=75\&auto=webp\&disable=upscale}

Ahora, mientras otros europeos reflexionan sobre cómo reiniciar sus
economías mientras siguen protegiendo la vida humana, los españoles se
han convertido en los primeros en prever cómo podría ocurrir una segunda
ola, cuán fuerte podría golpear y cómo podría ser contenida.

``Tal vez España sea el canario en la mina de carbón'', dijo Antoni
Trilla, un epidemiólogo del grupo de investigación del Instituto de
Salud Global de Barcelona. ``Muchos países pueden seguirnos, pero
esperemos que no a la misma velocidad o con el mismo número de casos que
estamos enfrentando''.

Los médicos y los políticos no están tan aterrorizados por la segunda
ola de España como lo estaban por la primera. La tasa de mortalidad es
aproximadamente la mitad de la tasa en el punto álgido de la crisis: del
12 por ciento del pico de mayo, ha caído al 6,6 por ciento.

La edad promedio de los enfermos ha bajado de 60 a 37 años. Los casos
asintomáticos representan más del 50 por ciento de los resultados
positivos, lo que se debe en parte a que las pruebas se han
cuadruplicado. Y las instituciones sanitarias se sienten mucho más
preparadas ahora.

``Ahora sí tenemos experiencia'', dijo María del Mar Vázquez, directora
médica del hospital de Málaga donde el padre de Bautista es tratado.

``Ahora tenemos mucho más equipamiento disponible, hemos puesto en
marcha protocolos, estamos más preparados'', dijo Vázquez. ``Los
hospitales se llenarán, pero estaremos preparados''.

Sin embargo, parte del hospital sigue siendo una obra en construcción,
los contratistas aún no han terminado la renovación del ala que se ocupa
de los pacientes con coronavirus. Nadie esperaba la segunda oleada por
lo menos hasta dentro de un mes más.

Y los epidemiólogos no están seguros de por qué llegó tan pronto.

Las explicaciones incluyen el aumento de las reuniones familiares
numerosas; el regreso del turismo en ciudades como Málaga; la decisión
de devolver la responsabilidad de la lucha contra el virus a las
autoridades locales al final del confinamiento nacional, y la falta de
viviendas adecuadas y atención sanitaria para los migrantes.

Image

Recién casados en Málaga el viernesCredit...Samuel Aranda para The New
York Times

También se ha culpado al resurgimiento de la vida nocturna, que se
reinstauró antes y con menos restricciones que en muchas otras partes de
Europa.

``Tenemos este factor cultural relacionado con nuestra rica vida
social'', dijo Ildefonso Hernández Aguado, exdirector general de salud
pública del gobierno español. ``La gente es cercana. Les gusta
reunirse''.

Durante varias semanas, en lugares como Málaga se permitió que los
clubes nocturnos y discotecas abrieran hasta las 5:00 a.m., mientras los
políticos regionales intentaban revivir una economía dependiente de los
turistas y los amantes de la vida nocturna. A los juerguistas solo se
les permitía bailar alrededor de una mesa con amigos, en lugar de
mezclarse con desconocidos, pero las reglas no siempre se cumplían.

En un notorio incidente a principios de agosto,
\href{https://www.rtve.es/noticias/20200803/coronavirus-torremolinos-cierra-discoteca-dj-escupio-publico/2036701.shtml}{un
DJ fue capturado en vídeo mientras escupía al público} en una concurrida
pista de baile de un club de playa en las afueras de Málaga.

El lugar fue clausurado rápidamente y dos semanas después se ordenó que
todos los clubes nocturnos cerraran, y ahora los bares deben cerrar a la
1:00 a.m. Pero los críticos temen que las restricciones sean todavía
demasiado laxas.

Mientras las camas seguían llenándose en los hospitales de Málaga este
fin de semana, los residentes siguen abarrotando los bares a lo largo de
la playa hasta bien pasada la medianoche. En algunos bares, las mesas
estaban muy cercanas, mucho más de lo que permiten las normas actuales
de dos metros.

Image

Limpieza de las calles del centro de MálagaCredit...Samuel Aranda para
The New York Times

A la hora de cerrar, los bebedores se desparramaron por las playas y los
pontones, la mayoría sin cubrebocas. Allí se congregaron en grupos de
más de 20, lo normal durante cualquier otro verano español, pero mucho
más grande que las reuniones de diez o menos ahora permitidas por la
ley.

Algunos eran adolescentes que decían haberse recuperado recientemente de
una forma leve del virus, y que por lo tanto se consideran inmunes.
Otros sentían que las restricciones de la pandemia eran una exageración.

``No creo que el coronavirus sea de verdad'', dijo Victor Bermúdez, el
dependiente de una tienda de 23 años en una reunión matutina en un
pontón que se adentra en el Mediterráneo. ``Bueno, sí, es real, pero no
es tan serio como lo pintan. Es todo un plan para matar a los pobres y
reforzar a los ricos''.

Durante el confinamiento, el gobierno central estableció una clara
agenda desde Madrid. Pero al levantar el estado de emergencia a finales
de junio, se devolvieron ciertos poderes a cada uno de los 17 gobiernos
regionales de España, lo que llevó a un enfoque desarticulado y confuso.

Cuando las regiones intentaron imponer restricciones a la vida local,
algunas de sus decisiones fueron anuladas por los jueces locales, que
argumentaron que solo el congreso central tenía el poder de imponer
tales medidas.

``Seguimos sin instrumentos jurídicos que nos den garantía para tomar
decisiones sin que un juez nos las pueda levantar a las 24 horas'', dijo
Juan Manuel Moreno, presidente de la junta regional de Andalucía, la
región en la que se encuentra Málaga.

Image

Toman la temperatura afuera de un cine.Credit...Samuel Aranda para The
New York Times

El debate también se ha convertido en el último ejemplo de un amargo
conflicto sobre la Constitución de España que se viene gestando desde
hace más de cuatro décadas. Para los federalistas y los separatistas
catalanes, por ejemplo, la debacle revela cómo el poder nunca fue
debidamente delegado tras la muerte del dictador Francisco Franco en
1975. Para los nacionalistas españoles, en cambio, muestra cómo el
proceso de descentralización ya ha ido demasiado lejos.

``Se está librando una especie de guerra por mostrar qué tipo de sistema
político es mejor'', dijo Nacho Calle, editor de
\href{https://maldita.es/coronavirus/}{Maldita}, un destacado servicio
de verificación de hechos. El enfoque descentralizado ha dado lugar a un
régimen poco sistemático de seguimiento y localización de las posibles
víctimas de coronavirus. En algunas regiones se emplean varios miles de
rastreadores para localizar a las personas que podrían haber estado en
contacto con personas infectadas, mientras que en otras solo se
contratan unas pocas decenas, lo que reduce el ritmo al que se dice a
los posibles pacientes que entren en cuarentena.

E incluso en regiones con un gran número de rastreadores, como
Andalucía, los trabajadores sanitarios sobre el terreno informan de que
el proceso es todavía demasiado lento y con poco personal en ciertos
lugares.

Francisca Morente, enfermera en una clínica al oeste de Málaga, fue una
de los cientos de enfermeros locales que fueron habilitadas
temporalmente este verano para trabajar como rastreadores debido a la
escasez de personal en la unidad oficial de rastreo de su distrito.

Pero incluso ahora, Morente es una de solo cinco rastreadores que
trabajan en su clínica, no lo suficiente para hacer los cientos de
llamadas diarias que requiere un servicio de rastreo adecuado. E incluso
una vez que logra rastrear a los pacientes con coronavirus, esos
pacientes todavía tienen que esperar una semana hasta que se procesen
sus pruebas, debido a los cuellos de botella en los laboratorios
locales.

``Necesitamos más rastreadores y más recursos'', dijo. ``Necesitamos una
unidad de rastreadores especializados en cada centro sanitario, en vez
de este sistema temporal que tenemos ahora''.

Image

La policía local le pide a las personas afuera de un bar que se pongan
los cubrebocas.Credit...Samuel Aranda para The New York Times

Según algunos expertos, la falta de apoyo institucional a los migrantes
indocumentados también ha contribuido a la segunda ola. Algunos brotes
recientes comenzaron entre agricultores extranjeros que vivían en
alojamientos comunales atestados.

Al estar impedidos de buscar beneficios de desempleo y carecer de
contratos de trabajo formales, los migrantes indocumentados no pueden
fácilmente tomar tiempo libre del trabajo si están enfermos. Tampoco
pueden pagar por el tipo de hogares que les permitirían aislarse
fácilmente.

``Si tengo que ponerme en cuarentena, entonces no podré trabajar'', dijo
María Perea, una trabajadora de limpieza colombiana de 50 años a la
espera de los resultados de una prueba de coronavirus. ``Y si no puedo
trabajar, no tengo dinero''.

Pero en general, los médicos dicen que España está en una posición mucho
más fuerte para luchar contra el virus de lo que estaba en marzo.

La coordinación nacional ha mejorado. El gobierno central acordó la
semana pasada desplegar 2000 soldados como rastreadores de contacto. La
velocidad de las pruebas se ha acelerado; en Málaga el hospital más
grande puede procesar las pruebas en una sola mañana, gracias a la
reciente compra de una serie de robots. Al otro lado de la carretera, un
hospital improvisado construido con prisa en abril está vacío, listo
para un aumento de casos.

``No es como la primera ola'', dijo Carmen Cerezo, de 38 años, asistente
en un tren que esperaba fuera del hospital de Málaga mientras en el
interior su padre era examinado por coronavirus.

``Esta vez estamos más tranquilos'', dijo.

Image

Un club nocturno cerrado en Málaga. Durante varias semanas después de
que España levantara el cierre, se permitió a las discotecas y clubes
nocturnos permanecer abiertos hasta las 5 a.m.Credit...Samuel Aranda
para The New York Times

Patrick Kingsley es un corresponsal internacional que se enfoca en
proyectos de reportaje de largo aliento. Ha reportado desde más de 40
países, escrito dos libros y antes cubrió migración y Medio Oriente para
The Guardian.
\href{https://twitter.com/PatrickKingsley}{@PatrickKingsley}

\begin{center}\rule{0.5\linewidth}{\linethickness}\end{center}

Advertisement

\protect\hyperlink{after-bottom}{Continue reading the main story}

\hypertarget{site-index}{%
\subsection{Site Index}\label{site-index}}

\hypertarget{site-information-navigation}{%
\subsection{Site Information
Navigation}\label{site-information-navigation}}

\begin{itemize}
\tightlist
\item
  \href{https://help.nytimes3xbfgragh.onion/hc/en-us/articles/115014792127-Copyright-notice}{©~2020~The
  New York Times Company}
\end{itemize}

\begin{itemize}
\tightlist
\item
  \href{https://www.nytco.com/}{NYTCo}
\item
  \href{https://help.nytimes3xbfgragh.onion/hc/en-us/articles/115015385887-Contact-Us}{Contact
  Us}
\item
  \href{https://www.nytco.com/careers/}{Work with us}
\item
  \href{https://nytmediakit.com/}{Advertise}
\item
  \href{http://www.tbrandstudio.com/}{T Brand Studio}
\item
  \href{https://www.nytimes3xbfgragh.onion/privacy/cookie-policy\#how-do-i-manage-trackers}{Your
  Ad Choices}
\item
  \href{https://www.nytimes3xbfgragh.onion/privacy}{Privacy}
\item
  \href{https://help.nytimes3xbfgragh.onion/hc/en-us/articles/115014893428-Terms-of-service}{Terms
  of Service}
\item
  \href{https://help.nytimes3xbfgragh.onion/hc/en-us/articles/115014893968-Terms-of-sale}{Terms
  of Sale}
\item
  \href{https://spiderbites.nytimes3xbfgragh.onion}{Site Map}
\item
  \href{https://help.nytimes3xbfgragh.onion/hc/en-us}{Help}
\item
  \href{https://www.nytimes3xbfgragh.onion/subscription?campaignId=37WXW}{Subscriptions}
\end{itemize}
