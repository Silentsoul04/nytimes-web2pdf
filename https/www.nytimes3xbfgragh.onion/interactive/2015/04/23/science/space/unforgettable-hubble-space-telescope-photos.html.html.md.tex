 **NYTimes.com no longer supports Internet Explorer 9 or earlier. Please
upgrade your browser.
\href{http://www.nytimes3xbfgragh.onion/content/help/site/ie9-support.html}{LEARN
MORE »}

**Sections

**Home

**Search

\hypertarget{the-new-york-times}{%
\subsection{\texorpdfstring{\href{http://www.nytimes3xbfgragh.onion/}{The
New York Times}}{The New York Times}}\label{the-new-york-times}}

 \href{https://www.nytimes3xbfgragh.onion/section/science/space}{Space
\& Cosmos} \textbar{}Unforgettable Hubble Space Telescope Photos

Log In

**0

**Settings

**Close search

\hypertarget{site-search-navigation}{%
\subsection{Site Search Navigation}\label{site-search-navigation}}

Search NYTimes.com

**Clear this text input

Go

\url{https://nyti.ms/1zQHnOS}

\begin{enumerate}
\def\labelenumi{\arabic{enumi}.}
\item
  Loading...
\end{enumerate}

See next articles

See previous articles

\hypertarget{site-navigation}{%
\subsection{Site Navigation}\label{site-navigation}}

\hypertarget{site-mobile-navigation}{%
\subsection{Site Mobile Navigation}\label{site-mobile-navigation}}

Advertisement

\hypertarget{-space--cosmos-}{%
\subsection{\texorpdfstring{
\href{https://www.nytimes3xbfgragh.onion/section/science/space}{Space \&
Cosmos} }{ Space \& Cosmos }}\label{-space--cosmos-}}

\hypertarget{unforgettable-hubble-space-telescope-photos}{%
\section{Unforgettable Hubble Space Telescope
Photos}\label{unforgettable-hubble-space-telescope-photos}}

APRIL 23, 2015

On the eve of the 25th anniversary of the launching of the Hubble Space
Telescope, NASA released the new image below. We asked astronomers and
others involved in the telescope's groundbreaking and occasionally rocky
story to tell us about their favorite photos taken over the years.
\href{https://www.nytimes3xbfgragh.onion/video/science/100000003647066/hubble-reflects-the-cosmos.html}{Related
Article}

\includegraphics{https://static01.graylady3jvrrxbe.onion/packages/flash/multimedia/ICONS/transparent.png}

``This is a really new birthplace of stars,'' Jennifer Wiseman, senior
project scientist for the Hubble Space Telescope, said of the new image,
which shows a cluster of 3,000 stars known as Westerlund 2 in the
constellation Carina. ``The cluster is only about two million years old,
which in stellar terms is very young. And it contains some of the
galaxy's hottest, brightest and most massive stars that we know of.''

\begin{enumerate}
\def\labelenumi{\arabic{enumi}.}
\item
  \includegraphics{https://static01.graylady3jvrrxbe.onion/packages/flash/multimedia/ICONS/transparent.png}

  Tadpole Galaxy, 2002

  John Grunsfeld, astronaut and associate administrator for science,
  NASA

  \emph{Mr. Grunsfeld is best known as the Hubble repairman for
  journeying to space to make repairs on Hubble three different times.}

  This is a really interesting spiral galaxy that has only recently had
  an encounter with a dwarf galaxy. It perturbed the spiral structure,
  but you can see it's still an intact spiral and going out a hundred
  kilo parsecs is this huge tail. And in that tail are brand new
  star-forming regions, incredible activity caused by shocks traveling
  through this tadpole tail.

  Finally, with Hubble, we were able to see the universe with the same
  kind of resolution that we see with our eyeballs. In the background,
  virtually every pixel that is lit up, is another galaxy.

  You could pick any small region of the image --- not the tadpole. You
  could pick any other small region and do great science.

  This is why I risked my life, to be able to allow the Hubble to
  provide this level of wonderful science. This is the image I have
  hanging up in my home.
\item
  \includegraphics{https://static01.graylady3jvrrxbe.onion/packages/flash/multimedia/ICONS/transparent.png}

  Proof of Blurriness, 1990

  Sandra Faber, professor of astronomy and astrophysics, University of
  California, Santa Cruz

  \emph{Soon after the Hubble Space Telescope was deployed in 1990,
  astronomers and engineers discovered that images taken by Hubble's
  camera were not as clear as they should have been. The}
  \emph{telescope's mirror had been ground to the wrong curvature,
  making it near-sighted.}

  Here is the actual figure that Jon Holtzman, then at Lowell
  Observatory, and I produced for NASA that clinched the diagnosis of
  spherical aberration. We got the project to run the telescope through
  focus and take images as a test. The top series is our set of model
  through-focus images. The bottom series is the set of real
  through-focus images.
\item
  \includegraphics{https://static01.graylady3jvrrxbe.onion/packages/flash/multimedia/ICONS/transparent.png}

  Carina Nebula, 2007

  Pam Jeffries, graphic designer at the Space Telescope Science
  Institute, which operates Hubble

  This large panoramic image of the Carina Nebula is my favorite Hubble
  image because of the sweeping visual movement combined with compelling
  and interesting tiny details. Hubble has taken many close-up shots,
  and each detail is its own masterpiece. I loved it when I first
  started working with Hubble images in 2010, and I still love its
  breathtaking iconic appeal. It is always humbling to think about the
  birth of other stars and how they mirror our own. The time, distance
  and forces involved always put your stress and self-importance in
  perspective.
\item
  \includegraphics{https://static01.graylady3jvrrxbe.onion/packages/flash/multimedia/ICONS/transparent.png}

  Baby Planets? 1994

  Sara Seager, professor of planetary science and physics, Massachusetts
  Institute of Technology

  This striking 1994 Hubble Space Telescope image shows the rich
  tapestry of a small region of the Orion Nebula, a stellar nursery
  1,500 light years away. Newborn stars are seen with surrounding disks
  of raw planetary materials --
  \href{http://www.nytimes3xbfgragh.onion/1997/08/12/science/big-gas-disk-may-form-a-new-planet-system.htm}{protoplanetary}
  disks or"propylids" -- that will likely evolve into planets. This
  image is my favorite because even more astonishing than the beauty is
  the implication -- one that came well before exoplanets orbiting
  nearby stars was established -- that protoplanetary disks around
  newborn stars are abundant and that planet-forming processes (and
  hence planets) should be common in the Milky Way Galaxy.
\item
  \includegraphics{https://static01.graylady3jvrrxbe.onion/packages/flash/multimedia/ICONS/transparent.png}

  The Hubble Deep Field, 1996

  Robert Williams, former director, Space Telescope Science Institute

  The prevailing opinion before Hubble telescope's launch was that a
  very long exposure with the telescope was not likely to reveal distant
  new objects. Any would be too faint to be detected. One of the unique
  features of astronomy is that one can directly look back into the past
  because of the finite speed of light, and there is no better way to
  piece together the changing nature of the universe than to detect the
  most distant objects. They represent the ancestors of all that we see
  around us. A deep field with the Hubble simply had to be tried.

  When we announced to the science community that we would attempt to
  take a long series of exposures for a "deep field," a number of our
  colleagues were very troubled by our plans.
  \href{http://www.nytimes3xbfgragh.onion/1997/04/02/nyregion/lyman-spitzer-jr-dies-at-82-inspired-hubble-telescope.html}{Lyman
  Spitzer}, who along with
  \href{http://www.nytimes3xbfgragh.onion/2005/08/19/nyregion/19bahcall.html?pagewanted=all\&_r=0}{John
  Bahcall} was one of the essential advocates to bring about Hubble, was
  serious, but muted in expressing his concerns.

  On several occasions he asked me at council meetings, ``Are you sure
  you want to do this?'' His colleague John Bahcall was much more vocal
  in working to prevent what he believed to be a much too risky venture,
  coming so soon after Hubble's embarrassing spherical aberration had
  been fixed by the historic NASA Shuttle
  \href{http://www.nytimes3xbfgragh.onion/1999/03/11/us/nasa-shuttle-mission-is-moved-up-to-repair-hubble-telescope.html\%20}{servicing
  mission}. His concern, certainly understandable, was that if a large
  segment of time on the telescope produced little or no useful results,
  which would no doubt become public, the fallout could tarnish the
  mission of the telescope beyond repair.

  Scientific progress requires risk, so we executed the observations
  over a two-week period in December 1995. The resulting Hubble Deep
  Field image yielded a wondrous display of galaxies, many of them very
  small, faint, and distant. The image is really a core sample of the
  universe.

  The Hubble Deep Field has truly opened up the entire universe of
  galaxies to study and interpretation by simple imaging, turning the
  pretty faces of those galaxies into true talking heads that have
  helped us understand how structure formed and evolved over time in the
  universe.

  This image was created in 1995, but released in 1996.
\item
  \includegraphics{https://static01.graylady3jvrrxbe.onion/packages/flash/multimedia/ICONS/transparent.png}

  The Antennae, 2005

  Brad Whitmore, astronomer, Space Telescope Science Institute

  The Antennae are the most famous example of a colliding pair of
  galaxies, a phenomenon that results in some of the favorite Hubble
  images due to their great diversity and graceful appearance.

  For astronomers, they are also a wonderful laboratory for studying the
  formation of stars, since the collision ignites a starburst that
  lights up the galaxy like a display of fireworks. A surprise was that
  most of the star formation is in the form of star clusters --- some of
  which survive to form the equivalent of the ancient globular clusters
  in our own galaxy --- but most of which dissolve, their stars
  spreading out to form the stars that make up the galaxy as a whole.
  The contrast between the very bright blue stars formed in the
  starburst, the red light from hydrogen gas emission and the intricate
  dark dust structures result in spectacular images as the galaxies go
  through their gravitational dance and eventually merge to form an
  elliptical galaxy.
\item
  \includegraphics{https://static01.graylady3jvrrxbe.onion/packages/flash/multimedia/ICONS/transparent.png}

  A Black Hole in Every Galaxy, 2003

  Meg Urry, professor of astronomy, Yale University, and president of
  the American Astronomical Society

  My very favorite is the very deep image from the
  \href{http://www.stsci.edu/science/goods/)\%20}{GOODS survey,} which I
  designed while still at the Space Telescope Science Institute (along
  with Mark Dickinson and Mauro Giavalisco). It's quiet and perhaps not
  as dramatic as many other Hubble Space Telescope images but it tells
  us about the history of the universe, including the growth of
  supermassive black holes --- one in every galaxy, like a chicken in
  every pot --- and the assembly of galaxies. So much information from
  such a little picture. (GOODS was an acronym we came up with: Great
  Observatories Origins Deep Survey. ``Great Observatories because the
  full multiwavelength survey used Hubble, Spitzer and Chandra, i.e.,
  what NASA referred to as the Great Observatories; ``Deep Survey''
  because it was the deepest multiwavelength survey ever done at that
  time, about a quarter the size of the full moon; and ``Origins''
  because without another letter the name was pure hubris.)

  Happy birthday, Hubble!
\item
  \includegraphics{https://static01.graylady3jvrrxbe.onion/packages/flash/multimedia/ICONS/transparent.png}

  ``The Eye of God,'' 2003

  Barbara A. Mikulski, senator from Maryland and senior Democrat on the
  appropriations subcommittee that funds NASA

  I keep a huge print of ``The Eye of God'' hanging in my office. It's
  the planetary nebulae nearest to Earth, extending 2.5 light-years
  across, making it larger than our entire solar system.

  The scientists and staff of the Space Telescope Science Institute in
  Baltimore autographed it and gave it to me after we saved the Hubble
  in 1993 during Servicing Mission 1. Every time I stand in front of it
  I'm reminded, not just of the insight and beauty that Hubble brought
  home, but also of the people -- the engineers, scientists,
  technicians, support staff, cafeteria workers and custodians -- who
  have all done so much to advance our understanding of the cosmos. So
  now not only do I have God looking down on me as I work, I have the
  angels of the Space Telescope Science Institute watching over me as
  well.
\item
  \includegraphics{https://static01.graylady3jvrrxbe.onion/packages/flash/multimedia/ICONS/transparent.png}

  The Hubble Ultra Deep Field, 2004

  Steven Beckwith, professor of astronomy at University of California,
  Berkeley and former director of the Space Telescope Science Institute

  When we first looked at the final Ultra Deep Field on the big computer
  screen at the institute, the entire image was filled with galaxies:
  blue, yellow and red, a menagerie of different shapes and sizes. We
  were looking back to the time when the universe was very young. These
  early galaxies were not at all like the ones we see today; they were
  little train wrecks, clumps of stars and clusters of stars beginning
  to assemble the structures that would eventually become the beautiful
  spiral and elliptical galaxies we see today. It was a magical moment
  for all of us, one of those times that you never forget.

  \emph{After the loss of the space shuttle Columbia in 2003, Sean
  O'Keefe, then NASA's administrator, canceled the last planned space
  shuttle mission for another round of repairs and upgrades for Hubble,
  calling it too risky. Dr. Beckwith, then director of the Space
  Telescope Science Institute, sparred publicly and loudly with Mr.
  O'Keefe. Mr. O'Keefe stepped down as NASA administrator in 2004; his
  successor, Michael Griffin, reinstated the repair mission.}
\item
  \includegraphics{https://static01.graylady3jvrrxbe.onion/packages/flash/multimedia/ICONS/transparent.png}

  Tarantula Nebula, 2012

  John Troeltzsch, senior program manager, Ball Aerospace

  \emph{Ball Aerospace built scientific instruments for the Hubble,
  including the "corrective lens" that fixed its vision in 1993.}

  This image is my favorite because I used Hubble in 1990 to image this
  region for an engineering test shortly after launch. The image was
  severely degraded by Hubble's focus problem. The modern version of
  Hubble took a spectacular image of the region in 2012, which shows
  both the beauty of a stellar nursery and the immense power of some of
  the most massive stars in our universe. Astronomy has come so far in
  the past 25 years thanks to Hubble and the people who built and
  operate it.
\item
  \includegraphics{https://static01.graylady3jvrrxbe.onion/packages/flash/multimedia/ICONS/transparent.png}

  Pluto and Its Moons, 2006

  Max Mutchler, research and instrument scientist, Space Telescope
  Science Institute

  My favorite, personally most exciting and meaningful, is a no-brainer:
  the discovery of Pluto's small moons.

  Late at night on June 15, 2005, I was working in my office on Hubble
  observations of Pluto for Hal Weaver of the Johns Hopkins Applied
  Physics Laboratory, who intentionally over-exposed a series of images
  to try and detect faint moons. Hal is the project scientist for NASA's
  New Horizons mission to Pluto. I thought I was merely cleaning up the
  raw images, so I hadn't really begun to search them yet, when a moment
  of pure scientific discovery snuck up on me: Pluto has two small moons
  (to go with the large moon Charon discovered in 1978)!

  The actual discovery images were nowhere near as clear and obvious as
  this later confirmation image (which has also been heavily processed),
  but it was still surprisingly easy for Hubble to clearly detect not
  just one but \emph{two} moons which had been missed in previous
  searches. It actually seemed too easy, and we gave ourselves many
  months to confirm the discovery and convince ourselves they weren't
  some camera artifact (but I was 90 percent sure they were real that
  night).

  By 2012, the New Horizons team used Hubble to discover two more small
  Pluto moons, Styx and Kerberos. To me, this image represents the
  symbiosis of two great NASA missions working in concert to discover
  and explore new worlds. Hubble set the stage, and now New Horizons
  will fly through the Pluto system on July 14, and turn these small
  dots into real places for the first time.
\item
  \includegraphics{https://static01.graylady3jvrrxbe.onion/packages/flash/multimedia/ICONS/transparent.png}

  Just One Favorite?

  Neil deGrasse Tyson, director, Hayden Planetarium, American Museum of
  Natural History

  Can't do it. Like picking your favorite child. What's certain,
  however, is that nobody ever required a caption to appreciate them.
  The images convey their own majesty and splendor without it
\end{enumerate}

\subsection{}

\begin{itemize}
\item
  \href{https://www.nytimes3xbfgragh.onion/video/science/100000003647066/hubble-reflects-the-cosmos.html}{}

  \includegraphics{https://static01.graylady3jvrrxbe.onion/images/2015/04/23/multimedia/out-there-hubble25/out-there-hubble25-mediumThreeByTwo225.jpg}

  \hypertarget{hubble-reflects-the-cosmos}{%
  \subsection{Hubble Reflects the
  Cosmos}\label{hubble-reflects-the-cosmos}}

  Dec. 12, 2017
\item
  \href{https://www.nytimes3xbfgragh.onion/video/science/100000003552687/out-there-einsteins-telescope.html}{}

  \includegraphics{https://static01.graylady3jvrrxbe.onion/images/2015/03/04/multimedia/out-there-einstein/out-there-einstein-mediumThreeByTwo225.jpg}

  \hypertarget{out-there--einsteins-telescope}{%
  \subsection{Out There \textbar{} Einstein's
  Telescope}\label{out-there--einsteins-telescope}}

  May 30, 2019
\item
  \href{https://www.nytimes3xbfgragh.onion/2015/04/24/science/25-years-later-hubble-sees-beyond-troubled-start.html}{}

  \includegraphics{https://static01.graylady3jvrrxbe.onion/images/2015/04/23/multimedia/out-there-hubble25/out-there-hubble25-mediumThreeByTwo225.jpg}

  \hypertarget{25-years-later-hubble-sees-beyond-troubled-start}{%
  \subsection{25 Years Later, Hubble Sees Beyond Troubled
  Start}\label{25-years-later-hubble-sees-beyond-troubled-start}}

  Dec. 21, 2017
\end{itemize}

Advertisement

\hypertarget{site-information-navigation}{%
\subsection{Site Information
Navigation}\label{site-information-navigation}}

\begin{itemize}
\tightlist
\item
  \href{https://help.nytimes3xbfgragh.onion/hc/en-us/articles/115014792127-Copyright-notice}{©
  2020 The New York Times Company}
\item
  \href{https://www.nytimes3xbfgragh.onion}{Home}
\item
  \href{https://www.nytimes3xbfgragh.onion/search/}{Search}
\item
  Accessibility concerns? Email us at
  \href{mailto:accessibility@NYTimes.com}{\nolinkurl{accessibility@NYTimes.com}}.
  We would love to hear from you.
\item
  \href{https://help.nytimes3xbfgragh.onion/hc/en-us/articles/115015385887-Contact-Us}{Contact
  Us}
\item
  \href{https://www.nytco.com/careers/}{Work with us}
\item
  \href{https://nytmediakit.com/}{Advertise}
\item
  \href{https://help.nytimes3xbfgragh.onion/hc/en-us/articles/115014892108-Privacy-policy\#pp}{Your
  Ad Choices}
\item
  \href{https://help.nytimes3xbfgragh.onion/hc/en-us/articles/115014892108-Privacy-policy}{Privacy}
\item
  \href{https://help.nytimes3xbfgragh.onion/hc/en-us/articles/115014893428-Terms-of-service}{Terms
  of Service}
\item
  \href{https://help.nytimes3xbfgragh.onion/hc/en-us/articles/115014893968-Terms-of-sale}{Terms
  of Sale}
\end{itemize}

\hypertarget{site-information-navigation-1}{%
\subsection{Site Information
Navigation}\label{site-information-navigation-1}}

\begin{itemize}
\tightlist
\item
  \href{https://spiderbites.nytimes3xbfgragh.onion}{Site Map}
\item
  \href{https://help.nytimes3xbfgragh.onion/hc/en-us}{Help}
\item
  \href{https://help.nytimes3xbfgragh.onion/hc/en-us/articles/115015385887-Contact-Us?redir=myacc}{Site
  Feedback}
\item
  \href{https://www.nytimes3xbfgragh.onion/subscription?campaignId=37WXW}{Subscriptions}
\end{itemize}
