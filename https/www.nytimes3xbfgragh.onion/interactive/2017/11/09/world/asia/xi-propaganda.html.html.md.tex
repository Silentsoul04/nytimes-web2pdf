 **NYTimes.com no longer supports Internet Explorer 9 or earlier. Please
upgrade your browser.
\href{http://www.nytimes3xbfgragh.onion/content/help/site/ie9-support.html}{LEARN
MORE »}

**Sections

**Home

**Search

\hypertarget{the-new-york-times}{%
\subsection{\texorpdfstring{\href{http://www.nytimes3xbfgragh.onion/}{The
New York Times}}{The New York Times}}\label{the-new-york-times}}

 \href{/section/world/asia}{Asia Pacific} \textbar{}Why Xi Jinping's
(Airbrushed) Face Is Plastered All Over China

Log In

**0

**Settings

**Close search

\hypertarget{site-search-navigation}{%
\subsection{Site Search Navigation}\label{site-search-navigation}}

Search NYTimes.com

**Clear this text input

Go

\url{https://nyti.ms/2jdTpFB}

\begin{enumerate}
\def\labelenumi{\arabic{enumi}.}
\item
  Loading...
\end{enumerate}

See next articles

See previous articles

\hypertarget{site-navigation}{%
\subsection{Site Navigation}\label{site-navigation}}

\hypertarget{site-mobile-navigation}{%
\subsection{Site Mobile Navigation}\label{site-mobile-navigation}}

Advertisement

\hypertarget{-asia-pacific-}{%
\subsection{\texorpdfstring{ \href{/section/world/asia}{Asia Pacific}
}{ Asia Pacific }}\label{-asia-pacific-}}

\hypertarget{why-xi-jinpings-airbrushed-face-is-plastered-all-over-china}{%
\section{Why Xi Jinping's (Airbrushed) Face Is Plastered All Over
China}\label{why-xi-jinpings-airbrushed-face-is-plastered-all-over-china}}

Mao Zedong, c. 1950

Xi Jinping, 2017

Mao Zedong, c. 1950

Xi Jinping, 2017

President Xi Jinping is China's most powerful leader in decades. Not
since the days of Mao Zedong has one figure so dominated Chinese life.
Mr. Xi, who welcomed President Trump to China on Wednesday, cannot yet
match Mao's grandeur. But he has inspired a devout following that some
critics describe as the early stages of a personality cult.

Here's how Mr. Xi has used the tried-and-true strategies of autocrats to
present himself as a transformative figure.

\hypertarget{putting-himself-on-a-pedestal}{%
\subsection{Putting Himself on a
Pedestal}\label{putting-himself-on-a-pedestal}}

Perhaps the most telling sign of Mr. Xi's dominance came last month,
when he was awarded a second five-year term as China's leader.

The layouts of Communist Party newspapers are carefully designed to
signal the relative power of top officials after leadership reshuffles
every five years. For decades, the front page of People's Daily embodied
a ``collective leadership'' model as the party sought to spread power
more evenly after Mao's death in 1976.

1977

1982

1987

1992

1997

2002

2007

2012

1977

1982

1987

1992

1997

2002

2007

2012

1977

1982

1987

1992

1997

2002

2007

2012

But Mr. Xi was awarded a different layout. His beaming face evoked the
days of one-man rule and unmistakably placed him on a pedestal with Mao.

2017

The faces of the other six members of China's most powerful body are
barely visible.

2017

The faces of the other six members of China's most powerful body are
barely visible.

2017

The faces of the other six members of China's most powerful body are
barely visible.

During his decades in power, Mao exerted virtually unchecked authority
over the government. Even as Mao's decisions led to violence across
China during the Cultural Revolution, splitting families and engulfing
the country in chaos, the media depicted him as a generous leader
motivated only by his love of country.

\includegraphics{https://static01.graylady3jvrrxbe.onion/packages/flash/multimedia/ICONS/transparent.png}

This poster of Mao, circa 1968, calls for unity against ``class
enemies.'' \href{https://chineseposters.net/}{Chinese Posters
Foundation}

Mr. Xi is far from cultivating Mao's sort of following. And historians
said creating and maintaining the image of a cult figure is trickier
than it was in the days of Mao, when the novelty of loudspeakers and
television were able to reach a more receptive and captive audience.

``Nowadays, there's a certain cynicism,'' said Daniel Leese, a professor
of Chinese history at the University of Freiburg. ``At the time, it was
brand new.''

But Mr. Xi has consolidated power at a remarkable pace for a man who was
virtually unknown outside China when he rose to power in 2012.

By elevating himself to the status of Mao, Mr. Xi is sending a message
that he is not to be challenged, and that now is the time for China to
unite behind a singular force to push forward an ambitious agenda. He
has so far
\href{https://www.nytimes3xbfgragh.onion/2017/10/24/world/asia/xi-jinping-china.html}{avoided
designating a successor}, prompting speculation that he will seek to
extend his power beyond the end of his term in 2023.

\hypertarget{the-making-of-an-icon}{%
\subsection{The Making of an Icon}\label{the-making-of-an-icon}}

In the front-page version of Mr. Xi's portrait, the color and saturation
were adjusted, hiding gray hairs and skin imperfections and giving the
photo the feel of a painting, said Hany Farid, a professor at Dartmouth
College and an expert on photo forensics.

The processed image evokes the same visual style as portraits of Mao,
said Jan Plamper, professor of history at Goldsmiths, University of
London. ``It'd be like depicting Trump more like George Washington.''

At the height of the Mao era, people displayed his image in their homes
and wore it as a badge on their clothing. The ubiquity of the image in
the most private spaces of Chinese life helped to give him a deified
status, said Pang Laikwan, a professor of cultural studies at the
Chinese University of Hong Kong.

``Loving Mao was in the air and part of people's everyday life,'' Ms.
Pang said. ``There was no alternative.''

Mr. Xi's slogans are splashed across front pages, and his speeches
dominate the evening news. His voice booms from giant television screens
in busy plazas and his image hangs inside homes, restaurants and taxi
cabs -- often alongside Mao's.

\includegraphics{https://static01.graylady3jvrrxbe.onion/packages/flash/multimedia/ICONS/transparent.png}

Adam Dean for The New York Times

\includegraphics{https://static01.graylady3jvrrxbe.onion/packages/flash/multimedia/ICONS/transparent.png}

Aly Song/Reuters

Mr. Xi, the son of revolutionaries, has tapped into Communist Party lore
in a way that hearkens back to the era of Mao. By becoming a symbol of
the Communist cause, Mr. Xi has sought to bring the party and its values
back to the center of everyday life at a time when many Chinese are
focused on material wealth.

\hypertarget{a-warm-paternal-figure}{%
\subsection{A Warm, Paternal Figure}\label{a-warm-paternal-figure}}

Mr. Xi's predecessors were criticized as wooden and aloof. Former
President Hu Jintao, for example, appeared rigid while visiting the
countryside in the eastern province of Anhui in 2008.

Sina

By contrast, the current president has cultivated a warm, paternalistic
image. During a 2016 visit to the southern province of Jiangxi, Mr. Xi
came across as personable and down to earth.

CGTN

Mr. Xi is
\href{https://www.nytimes3xbfgragh.onion/2015/03/08/world/move-over-mao-beloved-papa-xi-awes-china.html}{known
popularly} as ``Xi Dada,'' or ``Uncle Xi.'' He has made the sorts of
casual visits that were rare for modern Chinese leaders -- for example,
\href{https://sinosphere.blogs.nytimes3xbfgragh.onion/2013/12/30/divining-chinas-direction-by-what-xi-ate/}{visiting
a steamed bun shop} for lunch in 2013, where he paid his own bill and
bussed his own tray.

The avuncular nickname echoes other autocratic leaders. Joseph Stalin,
the
``\href{http://press-files.anu.edu.au/downloads/press/n2129/html/ch03.xhtml?referer=2129\&page=9}{father
of the peoples},'' often surrounded himself with children, conveying
absolute authority and benevolence. Ho Chi Minh, known as ``Uncle Ho,''
did the same in Vietnam.

Joseph Stalin depicted in 1947, and Ho Chi Minh posing with children in
1954. Russian State Library (Joseph Stalin), Getty Images (Ho Chi Minh)

Mr. Xi's fatherly persona helps maintain unity behind the Communist
Party by crafting a narrative that everyone can understand. ``Images
have to be legible by the entire society,'' said Mr. Plamper.

The state-run media has experimented with other strategies for playing
up Mr. Xi's lighter side as well, sometimes portraying him as an
animated character. Here is his avatar in a People's Daily feature
promoting China's dream of developing a leading soccer team by
vanquishing corruption in the sports industry. (The ball is labeled
``fake.'')

\includegraphics{https://static01.graylady3jvrrxbe.onion/packages/flash/multimedia/ICONS/transparent.png}

Mr. Xi has deployed his
\href{https://www.nytimes3xbfgragh.onion/2017/10/08/world/asia/xi-jinping-china-propaganda-village.html}{personal
story} to inspire zeal and adulation. While recent Chinese leaders
published tired volumes of speeches, Mr. Xi has released a book about
his experiences as a young man
\href{https://www.nytimes3xbfgragh.onion/2017/10/08/world/asia/xi-jinping-china-propaganda-village.html?_r=0}{sent
to the countryside} under Mao. The book, based on interviews with
farmers and people who served alongside Mr. Xi, recounts his days
shoveling manure and sleeping in flea-infested caves.

\includegraphics{https://static01.graylady3jvrrxbe.onion/packages/flash/multimedia/ICONS/transparent.png}

A book about Mr. Xi, left, alongside a book about his predecessor.

Mr. Xi's writings are mostly aimed at a domestic audience, helping to
create a heroic aura and to place him in the same revered ranks of Mao
and Deng Xiaoping. His books are often featured in lavish displays with
stars and ribbons, a break with the more restrained exhibits afforded
many of his predecessors.

Increasingly, party officials see Mr. Xi's writings as a tool to
introduce Mr. Xi on the global stage, and his books have been translated
into dozens of languages.

As the United States enters a period of retreat, many Chinese see Mr. Xi
as their best chance at getting ahead globally and weathering storms of
the economy and corruption at home. And the party is making sure that
everyone in China and abroad knows it.

\hypertarget{more-on-nytimescom}{%
\subsection{More on NYTimes.com}\label{more-on-nytimescom}}

Advertisement

\hypertarget{site-information-navigation}{%
\subsection{Site Information
Navigation}\label{site-information-navigation}}

\begin{itemize}
\tightlist
\item
  \href{https://help.nytimes3xbfgragh.onion/hc/en-us/articles/115014792127-Copyright-notice}{©
  2020 The New York Times Company}
\item
  \href{https://www.nytimes3xbfgragh.onion}{Home}
\item
  \href{https://www.nytimes3xbfgragh.onion/search/}{Search}
\item
  Accessibility concerns? Email us at
  \href{mailto:accessibility@NYTimes.com}{\nolinkurl{accessibility@NYTimes.com}}.
  We would love to hear from you.
\item
  \href{https://help.nytimes3xbfgragh.onion/hc/en-us/articles/115015385887-Contact-Us}{Contact
  Us}
\item
  \href{https://www.nytco.com/careers/}{Work with us}
\item
  \href{https://nytmediakit.com/}{Advertise}
\item
  \href{https://help.nytimes3xbfgragh.onion/hc/en-us/articles/115014892108-Privacy-policy\#pp}{Your
  Ad Choices}
\item
  \href{https://help.nytimes3xbfgragh.onion/hc/en-us/articles/115014892108-Privacy-policy}{Privacy}
\item
  \href{https://help.nytimes3xbfgragh.onion/hc/en-us/articles/115014893428-Terms-of-service}{Terms
  of Service}
\item
  \href{https://help.nytimes3xbfgragh.onion/hc/en-us/articles/115014893968-Terms-of-sale}{Terms
  of Sale}
\end{itemize}

\hypertarget{site-information-navigation-1}{%
\subsection{Site Information
Navigation}\label{site-information-navigation-1}}

\begin{itemize}
\tightlist
\item
  \href{https://spiderbites.nytimes3xbfgragh.onion}{Site Map}
\item
  \href{https://help.nytimes3xbfgragh.onion/hc/en-us}{Help}
\item
  \href{https://help.nytimes3xbfgragh.onion/hc/en-us/articles/115015385887-Contact-Us?redir=myacc}{Site
  Feedback}
\item
  \href{https://www.nytimes3xbfgragh.onion/subscription?campaignId=37WXW}{Subscriptions}
\end{itemize}
