**NYTimes.com no longer supports Internet Explorer 9 or earlier. Please
upgrade your browser.
\href{http://www.nytimes3xbfgragh.onion/content/help/site/ie9-support.html}{LEARN
MORE »}

**Sections

**Home

**Search

\hypertarget{the-new-york-times}{%
\subsection{\texorpdfstring{\href{http://www.nytimes3xbfgragh.onion/}{The
New York Times}}{The New York Times}}\label{the-new-york-times}}

\href{https://www.nytimes3xbfgragh.onion/section/technology}{Technology}
\textbar{}How Facebook Lets Brands and Politicians Target You

Log In

**0

**Settings

**Close search

\hypertarget{site-search-navigation}{%
\subsection{Site Search Navigation}\label{site-search-navigation}}

Search NYTimes.com

**Clear this text input

Go

\href{https://nyti.ms/2GSEc3z}{https://nyti.ms/2GSEc3z}

\begin{enumerate}
\def\labelenumi{\arabic{enumi}.}
\item
  Loading...
\end{enumerate}

See next articles

See previous articles

\hypertarget{site-navigation}{%
\subsection{Site Navigation}\label{site-navigation}}

\hypertarget{site-mobile-navigation}{%
\subsection{Site Mobile Navigation}\label{site-mobile-navigation}}

Advertisement

\hypertarget{-technology-}{%
\subsection{\texorpdfstring{
\href{https://www.nytimes3xbfgragh.onion/section/technology}{Technology}
}{ Technology }}\label{-technology-}}

\hypertarget{how-facebook-lets-brands-and-politicians-target-you}{%
\section{How Facebook Lets Brands and Politicians Target
You}\label{how-facebook-lets-brands-and-politicians-target-you}}

One in five dollars spent on online advertising in the United States
currently goes to Facebook. The power of the company's ad platform comes
from the ability it gives politicians, brands, real estate agents,
nonprofits and others to precisely
\href{https://www.nytimes3xbfgragh.onion/2018/04/11/technology/facebook-privacy-hearings.html}{target
people on its social networks}.

Facebook has expanded its ad apparatus over time by continuously
accumulating details about its users and turning those details into
targetable data points. As it has grown, however, the company has drawn
the scrutiny of privacy advocates, regulators and lawmakers around the
world.

\hypertarget{the-early-days}{%
\subsection{The Early Days}\label{the-early-days}}

When Facebook introduced its ad platform in 2007, advertisers could
target people using \textbf{information they had volunteered} on the
platform.

This is what a typical target audience might have looked like with those
options:

Anyone who \textbf{lives in Philadelphia}, \textbf{studies philosophy in
college} and \textbf{is 18 to 22}.

In 2009, Facebook added several features, including the ability for
users to click a ``like'' button on posts in their newsfeed, which
refined the list of interests that advertisers could target. The company
also introduced ways for advertisers to target friends of those who had
interacted with their brands, and to target ads to people by \textbf{age
or birthday}.

Anyone who lives in Philadelphia, studies philosophy in college and
\textbf{is 21}.

\hypertarget{introducing-third-party-data}{%
\subsection{Introducing Third-Party
Data}\label{introducing-third-party-data}}

Three years later, Facebook introduced Custom Audiences, a feature that
allowed companies to \textbf{upload their own lists of people to
target}. A retailer, for example, could upload its customer list and
target ads at those who had recently bought a specific kind of T-shirt.

Anyone who lives in Philadelphia, studies philosophy in college, is 21
and \textbf{has bought a blue T-shirt in the past year}.

Of course, companies were not limited to using lists of their own
customers. They could also upload lists of consumers bought from
third-party marketing firms known as data brokers.

Data brokers
\href{https://www.nytimes3xbfgragh.onion/2012/06/17/technology/acxiom-the-quiet-giant-of-consumer-database-marketing.html}{collect
vast troves of information} from public records, retailers who sell
customer information and other sources. The brokers combine that data
into consumer profiles, and then resell lists filtered to target certain
demographics.

By this time, targeting ads to people based on their interests had
become a common online-advertising practice, and marketers and
researchers were beginning to tap into big data analysis to push that
practice further.

In 2013, a researcher working for the consulting firm Cambridge
Analytica released a personality quiz that people could take on
Facebook, with the results indicating how open, conscientiousness,
extroverted, agreeable or neurotic they were. The researcher compared
the quiz results with what those who took it had liked on Facebook,
ultimately determining which interests corresponded with which
personality traits.

Using that method, which had been
\href{https://www.nytimes3xbfgragh.onion/2018/03/20/technology/facebook-cambridge-behavior-model.html}{developed}
by researchers at Cambridge University, targeting certain interests was
now tantamount to targeting \textbf{personality traits}.

Anyone who lives in Philadelphia, studies philosophy in college, is 21,
has bought a blue T-shirt in the past year and \textbf{is neurotic}.

Around 2014, Facebook's ad program began to evolve dramatically,
allowing advertisers to pinpoint people even more precisely than before.
By then, more than 1.2 billion people on average used the social network
every day.

One new feature, Partner Categories, brought hundreds of
\textbf{targeting options from data brokers} into Facebook's ad
platform. Brands could now target people based on demographics like
salary; number of open credit lines; car make and model; and whether a
user fit into a category such as ``trendy moms.''

Anyone who lives in Philadelphia, studies philosophy in college, is 21,
has bought a blue T-shirt in the past year, is neurotic, \textbf{makes
less than \$28,000 a year} and \textbf{is likely to buy a minivan in the
next six months}.

In 2014, the company also introduced what it called Lookalike Audiences,
a feature that allowed organizations that uploaded customer lists to
also target people who had profiles similar to those customers.

Anyone who lives in Philadelphia, studies philosophy in college, is 21,
has bought a blue T-shirt in the past year, is neurotic, makes less than
\$28,000 a year and is likely to buy a minivan in the next six months.
\textbf{Plus, anyone on Facebook who is similar to them}.

\hypertarget{tracking-you-around-the-web-and-guessing-your-ethnicity}{%
\subsection{Tracking You Around the Web and Guessing Your
Ethnicity}\label{tracking-you-around-the-web-and-guessing-your-ethnicity}}

In mid-2014, Facebook incorporated users' \textbf{online browsing
history} into its ad-targeting platform. The company had been collecting
browsing data for years, from any web page that included a Facebook like
button or had let people log in through their Facebook accounts. The
company used the browsing data --- from, say, a user's visits to several
websites while she shopped for a new tent --- to refine the specific
interests that advertisers could target and to give them more confidence
about what they were targeting.

Anyone who lives in Philadelphia, studies philosophy in college, is 21,
has bought a blue T-shirt in the past year, is neurotic, makes less than
\$28,000 a year, is likely to buy a minivan in the next six months and
\textbf{is interested in camping}. Plus, anyone on Facebook who is
similar to them.

Around this time, Facebook also added a targeting option called
\textbf{Ethnic Affinity}. The company does not ask users to identify
their race, but it assigns ethnicities based on activities such as the
content and pages that users like. A user could be categorized as
African-American, for example, because his Facebook activity ``aligns
with African-American multicultural affinity.'' The other ethnicities
that can be assigned are Hispanic and Asian-American.

Anyone who lives in Philadelphia, studies philosophy in college, is 21,
has bought a blue T-shirt in the past year, is neurotic, makes less than
\$28,000 a year, is likely to buy a minivan in the next six months, is
interested in camping and \textbf{whose interests align with those of
African-Americans}. Plus, anyone on Facebook who is similar to them.

The recent
\href{https://www.nytimes3xbfgragh.onion/2018/03/17/us/politics/cambridge-analytica-trump-campaign.html}{revelation}
that Cambridge Analytica collected information of up to 87 million
Facebook users has renewed concerns about privacy on the platform, and
has prompted the company to pledge to improve its privacy tools and
transparency and to re-think at least one aspect of its ad practices.
The company has already said it would drop targeting options based on
third-party data from its advertising platform.

The changes coincide with the introduction of a new law in the European
Union, the
\href{https://www.nytimes3xbfgragh.onion/2018/04/08/technology/a-tough-task-for-facebook-european-type-privacy-for-all.html}{General
Data Protection Regulation}. The law, which is to take effect next
month, requires technology companies to limit their collection of user
data to what they need to perform services, and to obtain customers'
consent for how their data will be used and with whom it will be shared.

Mark Zuckerberg, Facebook's chief executive, has said the company would
offer all of its users the same privacy required under the European law,
regardless of where they live.

Source: eMarketer (online advertising spending).

\hypertarget{more-on-nytimescom}{%
\subsection{More on NYTimes.com}\label{more-on-nytimescom}}

Advertisement

\hypertarget{site-information-navigation}{%
\subsection{Site Information
Navigation}\label{site-information-navigation}}

\begin{itemize}
\tightlist
\item
  \href{https://help.nytimes3xbfgragh.onion/hc/en-us/articles/115014792127-Copyright-notice}{©
  2020 The New York Times Company}
\item
  \href{https://www.nytimes3xbfgragh.onion}{Home}
\item
  \href{https://www.nytimes3xbfgragh.onion/search/}{Search}
\item
  Accessibility concerns? Email us at
  \href{mailto:accessibility@NYTimes.com}{\nolinkurl{accessibility@NYTimes.com}}.
  We would love to hear from you.
\item
  \href{https://help.nytimes3xbfgragh.onion/hc/en-us/articles/115015385887-Contact-Us}{Contact
  Us}
\item
  \href{https://www.nytco.com/careers/}{Work with us}
\item
  \href{https://nytmediakit.com/}{Advertise}
\item
  \href{https://help.nytimes3xbfgragh.onion/hc/en-us/articles/115014892108-Privacy-policy\#pp}{Your
  Ad Choices}
\item
  \href{https://help.nytimes3xbfgragh.onion/hc/en-us/articles/115014892108-Privacy-policy}{Privacy}
\item
  \href{https://help.nytimes3xbfgragh.onion/hc/en-us/articles/115014893428-Terms-of-service}{Terms
  of Service}
\item
  \href{https://help.nytimes3xbfgragh.onion/hc/en-us/articles/115014893968-Terms-of-sale}{Terms
  of Sale}
\end{itemize}

\hypertarget{site-information-navigation-1}{%
\subsection{Site Information
Navigation}\label{site-information-navigation-1}}

\begin{itemize}
\tightlist
\item
  \href{https://spiderbites.nytimes3xbfgragh.onion}{Site Map}
\item
  \href{https://help.nytimes3xbfgragh.onion/hc/en-us}{Help}
\item
  \href{https://help.nytimes3xbfgragh.onion/hc/en-us/articles/115015385887-Contact-Us?redir=myacc}{Site
  Feedback}
\item
  \href{https://www.nytimes3xbfgragh.onion/subscription?campaignId=37WXW}{Subscriptions}
\end{itemize}
