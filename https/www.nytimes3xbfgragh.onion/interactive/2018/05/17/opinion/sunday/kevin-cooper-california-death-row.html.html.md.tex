 **NYTimes.com no longer supports Internet Explorer 9 or earlier. Please
upgrade your browser.
\href{http://www.nytimes3xbfgragh.onion/content/help/site/ie9-support.html}{LEARN
MORE »}

**Sections

**Home

**Search

\hypertarget{the-new-york-times}{%
\subsection{\texorpdfstring{\href{http://www.nytimes3xbfgragh.onion/}{The
New York Times}}{The New York Times}}\label{the-new-york-times}}

\hypertarget{-opinion-}{%
\subsubsection{\texorpdfstring{ \href{/section/opinion}{Opinion}
}{ Opinion }}\label{-opinion-}}

 \href{https://www.nytimes3xbfgragh.onion/section/opinion/sunday}{Sunday
Review} \textbar{}One Test Could Exonerate Him. Why Won't California Do
It?

**Close search

\hypertarget{site-search-navigation}{%
\subsection{Site Search Navigation}\label{site-search-navigation}}

Search NYTimes.com

**Clear this text input

Go

\url{https://nyti.ms/2k5u6CS}

\hypertarget{site-navigation}{%
\subsection{Site Navigation}\label{site-navigation}}

\hypertarget{site-mobile-navigation}{%
\subsection{Site Mobile Navigation}\label{site-mobile-navigation}}

\hypertarget{one-test-could-exonerate-him-why-wont-california-do-it}{%
\section{One Test Could Exonerate Him. Why Won't California Do
It?}\label{one-test-could-exonerate-him-why-wont-california-do-it}}

 Opinion One Test Could Exonerate Him. Why Won't California Do It?

By Nicholas Kristof

In 1983, four people were murdered in a home in Chino Hills, Calif.
\includegraphics{https://static01.graylady3jvrrxbe.onion/newsgraphics/2018/04/18/opinion-kristof-deathrow/1832807d31f2b2fd84f755f0d92d38f9ec869ede/photos/top_artop2.jpg}

The sole survivor of the attack said three white intruders had committed
the murders.

Then a woman told the police that her boyfriend, a white convicted
murderer, was probably involved, and she gave deputies his bloody
coveralls.

\includegraphics{https://static01.graylady3jvrrxbe.onion/newsgraphics/2018/04/18/opinion-kristof-deathrow/1832807d31f2b2fd84f755f0d92d38f9ec869ede/photos/topart1_mobile.jpg}

Declaration by Thomas R. Parker, retired special agent of the F.B.I.

So here's what sheriff's deputies did:

They threw away the bloody coveralls and arrested a young black man
named Kevin Cooper. He is now awaiting execution.

\includegraphics{https://static01.graylady3jvrrxbe.onion/newsgraphics/2018/04/18/opinion-kristof-deathrow/1832807d31f2b2fd84f755f0d92d38f9ec869ede/cover/kevincooper_desktop.png}
\includegraphics{https://static01.graylady3jvrrxbe.onion/newsgraphics/2018/04/18/opinion-kristof-deathrow/1832807d31f2b2fd84f755f0d92d38f9ec869ede/cover/kevincooper_mobile.png}

 Opinion Was Kevin Cooper\\
Framed for Murder? By Nicholas Kristof With Jessia Ma and Stuart A.
Thompson

The first sign that something was wrong was a continuous busy signal on
the home phone of Doug and Peggy Ryen.

Bill Hughes, who lived nearby, wasn't initially concerned. His
11-year-old son, Chris, had slept over at the Ryens' and he thought
maybe they had all gone out for breakfast. But finally at noon Hughes
drove over to pick Chris up and, when no one answered the Ryens' door,
he peered through the sliding glass doors --- and his brain couldn't
process all the red. ``This is paint, makeup,'' he thought wildly.

Then reality sank in, and he kicked the kitchen door in. Blood from the
five victims was splattered everywhere. Hughes rushed to his son, but
the body was cold. Doug and Peggy Ryen, both nude, had also been stabbed
to death, and the bloody corpse of their 10-year-old daughter, Jessica,
was in a doorway. But Josh Ryen, 8 years old, was moving feebly on the
floor even though his throat had been slashed and his skull fractured.

Josh

Ryen

Wounded,

8 years old

Doug ryen

Murdered,

41 years old

Peggy Ryen

Murdered,

41 years old

Jessica

Ryen

Murdered,

10 years old

Chris

Hughes

Murdered,

11 years old

Source: Inland Valley Daily Bulletin

From left to right:

Jessica Ryen, murdered, 10 years old. Josh Ryen, wounded, 8 years old.
Doug Ryen, murdered, 41 years old. Peggy Ryen, murdered, 41 years old.
Christopher Hughes, murdered, 11 years old.

Source: Inland Valley Daily Bulletin

Soon sheriff's deputies were swarming all over the Ryen house in
affluent, suburban Chino Hills, east of Los Angeles, that day in June
1983. Several signs, including Josh's personal account, pointed to three
white attackers, and blond or brown hairs were found in the victims'
hands, as if torn off in a struggle.

Sheriff's deputies were also contacted by the woman whose boyfriend was
a convicted murderer, recently released from prison, whom she suspected
of involvement in the Ryen killings. She not only gave deputies his
bloody coveralls but also told them that his hatchet was missing from
his tool rack and resembled one of the weapons reportedly used in the
attacks.

But instead of testing the coveralls for the Ryens' blood, the deputies
threw them away--and pursued Cooper. After a racially charged trial, he
was convicted of murdering the Ryens and Chris Hughes and is now on
death row at San Quentin Prison.

Gov. Jerry Brown is refusing to allow advanced DNA testing that might
finally resolve the question of who committed the murders, even though
Cooper's defense would pay for it. Brown refuses to allow even advanced
testing of the blond or brown hairs
\includegraphics{https://static01.graylady3jvrrxbe.onion/newsgraphics/2018/04/18/opinion-kristof-deathrow/1832807d31f2b2fd84f755f0d92d38f9ec869ede/design/camera.svg}
that were found in the victims' hands.

Locks of hair

Advanced DNA testing could identify the source of blond or brown hairs
recovered from Jessica Ryen's clutched hand.

This is the story of a broken justice system. It appears that an
innocent man was framed by sheriff's deputies and is on death row in
part because of dishonest cops, sensational media coverage and flawed
political leaders --- including Democrats like Brown and Kamala Harris,
the state attorney general before becoming a U.S. senator, who refused
to allow newly available DNA testing for a black man convicted of
hacking to death a beautiful white family and young neighbor. This was a
failure at every level, and it should prompt reflection not just about
one man on death row but also about profound inequities in our entire
system of justice.

I'm using strong language, I know. But I went to San Quentin to
interview Cooper, reviewed trial transcripts and other documents, spoke
to innumerable people on and off the record, and in 34 years at The New
York Times, I've never come across a case in America as outrageous as
Kevin Cooper's. So hear me out.

Smarter people than me have come to the same conclusion. ``This guy is
innocent,'' said Thomas R. Parker, a 30-year law enforcement veteran who
was deputy head of the F.B.I.'s office in Los Angeles. ``The evidence
was planted, he was framed, the cops lied on the stand.''

Parker said the case involved ``abject racism,'' and he has volunteered
his time investigating the case for the last seven years because he is
horrified that a man he believes was framed is nearing execution.

Or listen to Judge William A. Fletcher of the U.S. Court of Appeals for
the Ninth Circuit. ``He is on death row because the San Bernardino
Sheriff's Department framed him,'' Fletcher declared in a
\href{http://www.nyulawreview.org/sites/default/files/pdf/NYULawReview-89-3-Fletcher_0.pdf}{searing
2013 lecture}.

This appears to be a replay of a tragedy we've seen before: The police
are under great pressure to solve a sensational crime, they are sure
they have the culprit, and when evidence is lacking they plant it and
give false testimony. This is called ``testilying,'' and it's more
common than we'd like to think. In New York City alone,
\href{https://www.nytimes3xbfgragh.onion/2018/03/18/nyregion/testilying-police-perjury-new-york.html}{The
New York Times found} ``an entrenched perjury problem,'' with more than
25 instances of probable testilying just since 2015.

How did we get here?

Initially, the authorities searched for three white men, which fit the
evidence from the crime.

The coroner initially determined that there had been several assailants
using a hatchet, an ice pick and either one or two knives to stab and
slash the murder victims some 140 times.

\includegraphics{https://static01.graylady3jvrrxbe.onion/newsgraphics/2018/04/18/opinion-kristof-deathrow/1832807d31f2b2fd84f755f0d92d38f9ec869ede/coroners-reports-Artboard_7.jpg}

DOUG RYEN

PEGGY RYEN

37 inflicted

wounds

33 inflicted

wounds

JESSICA RYEN

CHRIS HUGHES

46 inflicted

wounds

25 inflicted

wounds

\includegraphics{https://static01.graylady3jvrrxbe.onion/newsgraphics/2018/04/18/opinion-kristof-deathrow/1832807d31f2b2fd84f755f0d92d38f9ec869ede/coroners-reports-Artboard_13.jpg}

DOUG RYEN

PEGGY RYEN

33 inflicted

wounds

37 inflicted

wounds

JESSICA RYEN

CHRIS HUGHES

46 inflicted

wounds

25 inflicted

wounds

\includegraphics{}

Doug Ryen 37 inflicted wounds Peggy Ryen 33 inflicted wounds Jessica
Ryen 46 inflicted wounds Chris Hughes 25 inflicted wounds

 The following image depicts the crime scene in graphic detail.

Peggy Ryen was strong and athletic and kept a loaded pistol in the
bedside drawer, directly beside where Doug's body was found. Doug was a
6-foot-2 former Marine and military policeman. His loaded rifle was in
the closet behind his body.

Loaded pistol

Loaded rifle

Doug Ryen (Behind bed)

Peggy Ryen

Jessica Ryen

Chris Hughes

\includegraphics{https://static01.graylady3jvrrxbe.onion/newsgraphics/2018/04/18/opinion-kristof-deathrow/1832807d31f2b2fd84f755f0d92d38f9ec869ede/killing_scene.jpg}

Compilation of crime scene photos from documents provided by Cooper's
lawyers.

It seemed that there must have been several attackers to use multiple
weapons, overpower the Ryens and keep them from getting to their guns
--- not a lone 155-pound individual like Cooper.

The evidence from the Ryen home and testimony from Josh pointed to three
white perpetrators.

Over the course of the investigation, several witnesses claimed they had
seen three white men in a car resembling the station wagon stolen from
the Ryens' home the night of the murders.

Several other witnesses said they saw three disorderly white patrons at
the Canyon Corral Bar late on the evening of the killings wearing bloody
clothing, in a car that may have been the Ryens'.

The men were surprised when the blood was pointed out --- and afterward
two bloody shirts were found nearby.

But none of this ultimately mattered to the sheriff's office after
deputies made what they thought was a breakthrough discovery: A
25-year-old black man convicted of burglary had escaped from a
minimum-security Chino prison by walking through a fence.

And he had holed up for two days in an empty house \ldots{}

... just 125 yards from the Ryens' home.

Source: Google (satellite imagery)

\includegraphics{https://static01.graylady3jvrrxbe.onion/packages/flash/multimedia/ICONS/transparent.png}

Kevin Cooper\\

That was Cooper, and deputies who examined his file and mug shot saw a
black man with a huge Afro who fit their narrative of an incorrigible
criminal. He had a long arrest record dating back to when he was 7 years
old.

The sheriff's deputies were sure they had their man: an escaped felon,
one who they thought looked suitably evil. The authorities pivoted and
focused on Cooper, ignoring other threads.

Still, the authorities had a problem: Although they were sure Cooper was
the killer, they couldn't find fingerprints, hairs or other evidence
implicating him.

So evidence began to turn up in mysterious ways.

\includegraphics{https://static01.graylady3jvrrxbe.onion/newsgraphics/2018/04/18/opinion-kristof-deathrow/1832807d31f2b2fd84f755f0d92d38f9ec869ede/lease-house/room_empty.jpg}
\includegraphics{https://static01.graylady3jvrrxbe.onion/newsgraphics/2018/04/18/opinion-kristof-deathrow/1832807d31f2b2fd84f755f0d92d38f9ec869ede/lease-house/room_empty.jpg}

\includegraphics{https://static01.graylady3jvrrxbe.onion/newsgraphics/2018/04/18/opinion-kristof-deathrow/1832807d31f2b2fd84f755f0d92d38f9ec869ede/lease-house/room_empty.jpg}
\includegraphics{https://static01.graylady3jvrrxbe.onion/newsgraphics/2018/04/18/opinion-kristof-deathrow/1832807d31f2b2fd84f755f0d92d38f9ec869ede/lease-house/room_buttonsheath.jpg}

\includegraphics{https://static01.graylady3jvrrxbe.onion/newsgraphics/2018/04/18/opinion-kristof-deathrow/1832807d31f2b2fd84f755f0d92d38f9ec869ede/lease-house/room_empty.jpg}
\includegraphics{https://static01.graylady3jvrrxbe.onion/newsgraphics/2018/04/18/opinion-kristof-deathrow/1832807d31f2b2fd84f755f0d92d38f9ec869ede/lease-house/room_cooper.jpg}

Two deputies searched Cooper's hide-out on June 6, a day after the
Ryens' bodies had been found. The police didn't find anything suspicious
in the house.

The next day, deputies searched the hide-out again, now having
identified Cooper as their primary suspect. A bloody green button from a
prison uniform materialized on the floor along with a hatchet sheath.
(Eventually, it turned out that Cooper had been wearing a brown jacket,
not green.)

Deputies claimed they had not entered that room before. But later, one
deputy's fingerprints were found inside the closet, indicating he had
been in the room --- and had lied about it.

 Or consider the Ryens' station wagon.

It was found in Long Beach, 30 miles away, and inconveniently had blood
on the driver's seat, the front passenger seat and the back seat ---
suggesting at least three killers. A bloody hatchet was also found near
the Ryen house, probably thrown out of the station wagon window --- on
the passenger side.

 Areas with blood found in Ryens' station wagon
\includegraphics{https://static01.graylady3jvrrxbe.onion/newsgraphics/2018/04/18/opinion-kristof-deathrow/1832807d31f2b2fd84f755f0d92d38f9ec869ede/photos/fixedseats.jpg}

A thorough search of the station wagon found no evidence that Cooper had
used the car. That problem was remedied when a second search of the
vehicle turned up some of Cooper's cigarette butts; sheriff's deputies
had found such cigarette butts in the empty house where he had stayed,
but the butts had vanished.

Another challenge for the prosecution was motive. After escaping from
the prison, Cooper was desperate for money, yet some cash had been left
on the counter
\includegraphics{https://static01.graylady3jvrrxbe.onion/newsgraphics/2018/04/18/opinion-kristof-deathrow/1832807d31f2b2fd84f755f0d92d38f9ec869ede/design/camera.svg}
in the Ryens' house.

Cash left behind

Cooper was desperate for money, yet cash was left in plain view on the
counter inside the Ryen home.

The prosecution suggested that Cooper wanted to steal the station wagon.
But the Ryens kept the keys in the car; there was no need to enter the
house.

Nevertheless, four days after the discovery of the murders, the sheriff
announced the crime had been solved: Cooper was being sought for murder.

\includegraphics{https://static01.graylady3jvrrxbe.onion/packages/flash/multimedia/ICONS/transparent.png}

San Bernardino County Sheriff Floyd Tidwell identifies the man sought
for murder. (Doug Pizac/Associated Press)

While the police were desperately trying to connect Cooper to the crime,
another man who should have been a prime suspect was not being
investigated.

That's a remarkable element of this case: Not only has the evidence
against Cooper largely been discredited, but evidence has accumulated
against another individual, who happens also to be a convicted murderer.
Fletcher, the federal judge, wrote a long section in a judicial opinion
implicating this man, whom I'll identify only by his first name, Lee.

\includegraphics{https://static01.graylady3jvrrxbe.onion/packages/flash/multimedia/ICONS/transparent.png}

Lee in the 1980s (eyes obscured to protect anonymity) Photo illustration

It was his girlfriend, Diana Roper, who had alerted deputies after the
murders made the news to the reasons she believed that he had
participated in the Ryen murders.

Roper and her sister said that Lee came home late on the night of the
killings in a station wagon like that of the Ryens, wearing
blood-drenched coveralls, and that his hatchet was missing from his tool
rack and resembled one of the murder weapons described by authorities.
She said that on the day of the killing she had laid out for Lee a
medium-size tan Fruit of the Loom T-shirt with a pocket; she remembered
because she had just bought it for Lee at Kmart. It was exactly like a
Fruit of the Loom T-shirt found by the bar with blood on it; testing
showed it was the Ryens' blood.

Roper said in an affidavit
\includegraphics{https://static01.graylady3jvrrxbe.onion/newsgraphics/2018/04/18/opinion-kristof-deathrow/1832807d31f2b2fd84f755f0d92d38f9ec869ede/design/expand.svg}:
``Lee was wearing long sleeve coveralls \ldots{} splattered with blood.
\ldots{} He did not have the beige T-shirt. Lee took the coveralls off
and left them on the floor of the closet. \ldots{} A few days after,
\ldots{} Lee had changed his appearance by cutting most of his hair off
and trimmed his sideburns and his `Fu Manchu' moustache.''

Diana Roper describes Lee's bloody overalls

In an affidavit, Roper describes how Lee returned home on the night of
the murders wearing bloody overalls.

Roper gave deputies the bloody coveralls. But instead of testing them to
see if the blood was from the Ryens, the sheriff's office threw them
out.

Coveralls destroyed

A private investigator for Cooper's defense, Ron Forbush, interviewed
Deputy Frederick Eckley about the bloody coveralls, which he collected
from Diana Roper and later destroyed.

+

\includegraphics{https://static01.graylady3jvrrxbe.onion/packages/flash/multimedia/ICONS/transparent.png}

A private investigator for Cooper's defense, Ron Forbush, interviewed
Deputy Frederick Eckley about the bloody coveralls, which he collected
from Diana Roper and later destroyed.

Roper said she cannot be sure that Lee's missing hatchet is the same as
the one used in the murders, but she added that ``the curvature of the
handle is the same'' and it had a similar ``American Indian pattern to
it.'' Her sister, Karee Kellison, who was with Roper, confirmed much of
her story.

Then there was the peculiar matter of the recovery of the Ryens' station
wagon.

The station wagon didn't turn up until a week after the killings ---
near the home of Lee's stepmother in Long Beach.

Ryen

home

Car

found

CALIFORNIA

North

Pacific

Ocean

Lee's

stepmother's

home

But Cooper had made his way south to Tijuana, Mexico. He arrived there
the evening after the killings.

Kevin Cooper

spotted in Tijuana

MEXICO

20 miles

Source: Planet (satellite imagery)

The station wagon didn't turn up until a week after the killings ---
near the home of Lee's stepmother in Long Beach.

Ryen home

Car found

Lee's

stepmother's

home

CALIFORNIA

But Cooper had made his way south to Tijuana, Mexico. He arrived there
the evening after the killings.

Kevin Cooper

spotted in Tijuana

North Pacific Ocean

MEXICO

20 miles

Source: Planet (satellite imagery)

The station wagon didn't turn up until a week after the killings ---
near the home of Lee's stepmother in Long Beach.

Ryen

home

Car

found

CALIFORNIA

North

Pacific

Ocean

Lee's

stepmother's

home

But Cooper had made his way south to Tijuana, Mexico. He arrived there
the evening after the killings.

Kevin Cooper

spotted in Tijuana

MEXICO

Source: Planet (satellite imagery)

20 miles

Source: Planet (satellite imagery)

The station wagon didn't turn up until a week after the killings ---
near the home of Lee's stepmother in Long Beach.

Ryen

home

Car

found

North

Pacific

Ocean

Lee's

stepmother's

home

CALIFORNIA

But Cooper had made his way south to Tijuana, Mexico. He arrived there
the evening after the killings.

Kevin Cooper

spotted in Tijuana

MEXICO

20 miles

Source: Planet (satellite imagery)

The sheriff's office claimed that Cooper took the Ryens' station wagon,
but aside from the witnesses who reported seeing several white men
driving it on the night of the murders, a new witness has emerged who
saw the car the next day.

The woman, who is scared of being identified as a witness for now but
says she will testify under oath if necessary, said three white people
in the Ryens' car were driving crazily and almost crashed into her
vehicle.

Three white individuals spotted

A witness says she saw three white individuals driving in a station
wagon with a license plate matching the Ryens' the day after the
murders.

Three white individuals spotted

A witness says she saw three white individuals driving in a station
wagon with a license plate matching the Ryens' the day after the
murders.

+

Her grandmother, who was with her that day, wrote down the license plate
number. Hours later, after the murder was discovered, the authorities
broadcast a description of the missing car with its license plate
number.

``I ran out to the car and got the slip of paper on which my grandmother
had written the license number,'' the woman wrote in a formal
declaration. ``It was exactly the same.'' She said that she wrote to the
police with her information, but the authorities did not follow up or
share it with the defense.

Shown an old photo of Lee, this woman said that it resembled the driver
but that she couldn't be sure it was the same man.

If there's no apparent motive for Cooper, there are only hints of one
for Lee. His previous murder, of a 17-year-old girl, was at the behest
of a gang leader, Clarence Ray Allen, who raised the same kind of
Arabian horses as the Ryens. There's some --- very squishy, unconfirmed
--- evidence that Allen may have previously threatened to kill Peggy
Ryen, that they had a quarrel over a horse sale gone sour, and that she
was terrified of him.

All this said, let's be clear that if there's one lesson from the Cooper
case, it's that we should be very wary of assuming guilt on the basis of
fragmentary evidence. I tracked down Lee, now 68, and he strongly denied
any involvement in the case. However, he did not want to discuss it and
asked not to be contacted again.

One point in Lee's favor: He has avoided serious tangles with the law in
the decades since the Ryen killings.

With all these uninvestigated threads, it's worth considering the
motives of the San Bernardino sheriff's office, which handled the
investigation.

Sheriff Floyd Tidwell had recently been appointed to his position and
was facing election that year, adding to the pressure to solve the most
brutal crime in the county's memory.

It's clear that the sheriff's office wasn't a stickler for rules.
Tidwell was later convicted for stealing more than 500 guns from county
evidence rooms. A lab technician who ``found'' shoe print evidence
against Cooper was later fired for stealing heroin from the evidence
room.

**** The sheriff's office also bungled the forensics, so that 70 people
\includegraphics{https://static01.graylady3jvrrxbe.onion/newsgraphics/2018/04/18/opinion-kristof-deathrow/1832807d31f2b2fd84f755f0d92d38f9ec869ede/design/expand.svg}
trampled through the crime scene.

Crowded murder scene

A document by the sheriff's office shows more than 70 people present at
the Ryen crime scene.

Then, a day after the bodies were discovered, the district attorney
closed the on-scene investigation for fear, he said, of gathering so
much evidence that defense experts could spin complicated theories.

Concerns with the San Bernardino sheriff's deputies have continued since
then.

Almost exactly 10 years after the Ryen murders, there was another
terrible murder in San Bernardino County, and a man named William
Richards was convicted in part based on evidence ``discovered'' by the
same sheriff's office lab technician who earlier had ``found'' evidence
against Cooper. Later, it turned out that this evidence was planted, and
Richards was eventually exonerated. (The sheriff's office declined to
comment for this article.)

The only witness to the murders themselves, of course, was Josh Ryen,
who endured a physical and emotional trauma that is unimaginable. By the
time of the trial, he had no clear memory of what happened or of seeing
an intruder.

Yet his first memories were clearer. I tracked down Don Gamundoy, who at
the time of the murders was a social worker at the hospital to which
Josh was rushed. ``He was awake and alert,'' Gamundoy recalled.

Josh could hear but couldn't speak because of the wound to his throat,
so Gamundoy wrote the alphabet, the numbers and the words ``yes'' and
``no'' on a piece of paper and asked Josh to point to the letters to
spell his name, phone number and birth date. Josh did so correctly,
showing that the method worked.

Then Gamundoy asked Josh if the people who did this were black.

``He pointed to `no,''' Gamundoy told me. Communicating in the same way,
Josh said that the attackers were white, and that there were three or
four of them.

Josh's account

During testimony, social worker Don Gamundoy recalled his conversation
with Josh, where he identified multiple white attackers.

+

\includegraphics{https://static01.graylady3jvrrxbe.onion/packages/flash/multimedia/ICONS/transparent.png}

During testimony, social worker Don Gamundoy recalled his conversation
with Josh.

This was a chaotic scene unfolding as doctors were struggling to treat
the boy, but Gamundoy said he had asked each question twice to make sure
the answers were not a mistake. Sheriff's deputies were present and
observing, he said, and in interviews with deputies later, Josh referred
to the attackers as ``they,'' saying that ``they'' had chased him.

With a good defense, Cooper might have prevailed. But his county public
defender was overwhelmed and made a series of practical legal mistakes.

``Kevin got convicted because they framed him and because he didn't have
a half-decent defense,'' said Norman C. Hile, his current lawyer. Hile,
now retired as a partner in the international law firm Orrick,
Herrington \& Sutcliffe, has volunteered on the case for the last 14
years because he fiercely believes in Cooper's innocence.

This is a familiar pattern: Inmates have third-rate defenders at trial,
but after they are sentenced to death they get the help of brilliant
free counsel; by then it is often too late to undo the damage.

Cooper's trial unfolded amid the ugliest racism. At a hearing, a crowd
displayed signs reading ``Hang the Nigger.'' One man displayed a noose
around a stuffed gorilla.

Newspapers carried inaccurate reports, apparently based on prosecution
leaks, that tied Cooper to the murder scene and suggested falsely that
he was gay (seizing upon 1980s homophobia as well as racism).

The San Bernardino Sun Clip

An article published Jun 29 1983, reporting on a bulletin released by
Floyd Tidwell which describes Cooper as ``reportedly homosexual''.

+

The San Bernardino Sun, June 29, 1983

The San Bernardino Sun, June 29, 1983

The San Bernardino Sun, June 29, 1983

Still, the trial outcome was close. The jury took a week to convict
Cooper, and one juror told reporters that there would have been no
conviction ``if there had been one less piece of evidence.''

Cooper's DNA found on bloody T-shirt

A DNA test conducted in 2002 found Cooper's DNA on the recovered bloody
T-shirt, alongside Doug Ryen's blood.

Cooper was scheduled to be executed at 12:01 a.m. on Feb. 10, 2004. On
Feb. 9, he was offered a last meal (he turned it down), and led on the
``dead man walking'' path to a holding area beside the execution cell.
He was strip-searched, given new clothes to die in, and guards searched
his arms for veins that could be used to administer lethal injections. A
pastor visited to pray with him.

Yet on what was supposed to be his last day, the Court of Appeals for
the Ninth Circuit granted a stay of execution, and a few hours before
the end, the warden halted the machinery of death.

Cooper was now permitted to conduct a new test on the tan T-shirt, and
this time the labs found something extraordinary. Yes, that may have
been Cooper's blood on it --- but the blood had a chemical preservative
called EDTA in it. That suggested that the blood came not from Cooper
directly but from a test tube of his blood. Sure enough, the sheriff's
deputies had taken a sample of Cooper's blood and had kept it in a test
tube with EDTA.

Now the lab checked a swatch of blood from that test tube. More wonders!
The test tube miraculously contained the blood of two or more people
\includegraphics{https://static01.graylady3jvrrxbe.onion/newsgraphics/2018/04/18/opinion-kristof-deathrow/1832807d31f2b2fd84f755f0d92d38f9ec869ede/design/expand.svg}.

Test reveals Cooper blood sample was contaminated

A test conducted on a blood sample in 2004 showed it contained a mixture
of ``two or more'' DNA types.

This indicated that the sheriff's office may have used the test tube of
Cooper's blood to frame him, and then topped off the test tube with
someone else's blood.

\includegraphics{https://static01.graylady3jvrrxbe.onion/newsgraphics/2018/04/18/opinion-kristof-deathrow/1832807d31f2b2fd84f755f0d92d38f9ec869ede/photos/fletcher.gif}
New York University School of Law lecture, 2013

``How could there be blood from two people? Well, I ask you to remember
the teenager's trick. Drink liquor from mom and dad's bottle, and then
you put some water back in to bring it back up to the line. How do we
have blood from two people? Well how do you bring it back up to the line
after you've taken blood from it?''

--- William A. Fletcher

A United States Ninth Circuit Court of Appeals Judge, speaking about
Kevin Cooper's case

Cooper's case began to get traction. The Ninth Circuit Court of Appeals
en banc refused to hear an appeal by Cooper, but Fletcher wrote a
remarkable 100-page dissent, concluding, ``The State of California may
be about to execute an innocent man.'' Four judges joined in this
extraordinary judicial opinion.

Likewise, the
\href{https://kevincooperorg.files.wordpress.com/2017/01/iachr-report-no-26-15-dated-7-21-15.pdf}{Inter-American
Commission on Human Rights found}in 2015 that there had been profound
flaws in the case and called for a review. The deans of four law schools
and the president of the American Bar Association expressed concerns. At
the end of his term in office, Gov. Arnold Schwarzenegger urged a
``thorough and careful review'' of the case.

Five of the original jurors signed declarations expressing concerns
about the case and calling for new DNA testing or for clemency. An
award-winning book, ``Scapegoat,'' concluded that Cooper had been
framed. In February 2016, Hile and the Orrick law firm submitted to
Governor Brown
\href{https://kevincooperorg.files.wordpress.com/2017/02/kevin-cooper_s-petition-for-executive-clemency.pdf}{a
235-page clemency petition}, pleading for advanced DNA testing of
evidence from the case.

Cooper's lawyers ask above all for new ``touch DNA'' testing --- capable
of detecting microscopic residues --- of the tan T-shirt, the hatchet
and the blond or brown hairs found in the victims' hands. This might
determine who wore the tan T-shirt or handled the hatchet, and whom the
hairs came from. Was it Kevin Cooper? Or was it Lee?

As state attorney general, Kamala Harris refused to allow this advanced
DNA testing and showed no interest in the case (on Friday, after the
online publication of this column, Senator Harris called me to say "I
feel awful about this" and put out
\href{https://www.facebookcorewwwi.onion/KamalaHarris/posts/10156769408692923}{a
statement} saying: "As a firm believer in DNA testing, I hope the
governor and the state will allow for such testing in the case of Kevin
Cooper."). As for Brown, he has not responded in the two years since the
petition was filed, and he refused to be interviewed. His spokesman,
Gareth Lacy, told me that the petition ``remains under review.'' Brown
leaves office in January, and I think he is running out the clock.

One reason Brown may be hesitant to weigh in: For four years before
becoming governor, he was attorney general, and during that time he
suggested that no one on death row was innocent. I hope that this won't
keep him from allowing advanced DNA testing.

California voters in 2016 approved a ballot measure to hasten
executions. So, depending on how litigation unfolds, Cooper could again
be led to the execution chamber sometime in the next year or so --- and
even if he delays execution, he feels he is wasting away.

\includegraphics{https://static01.graylady3jvrrxbe.onion/packages/flash/multimedia/ICONS/transparent.png}

Kevin Cooper at San Quentin

``Look at how white my hair is,'' Cooper told me, bending over to show
how his hair is graying. ``I don't have as much time left. Every day is
one I won't get back.''

I was speaking to him in San Quentin Prison, in a cage where inmates are
allowed to meet outsiders. Cooper lives on death row in San Quentin, in
a 4.5-foot-by-11-foot cell.

Cooper told me about his abusive and troubled childhood in Pennsylvania,
where he was adopted as a baby. When prosecutors said that Cooper had
tangled with the law since the age of 7, they were right, but he says
that the reason is that he was running away from home to escape
beatings. His childhood involved shoplifting, marijuana smoking,
juvenile detention and negligible education; he never graduated from
high school.

These days in prison, Cooper has remedied his lack of education with a
G.E.D. diploma and comes across as smart, passionate and articulate. But
he's not optimistic that the governor or courts will block his
execution.

``I don't have any confidence,'' he told me. ``I don't believe in the
system.'' He also spends his time writing a memoir, which now stands at
more than 300 pages. ``That's my motivating factor to get out of here,
to tell my story and tell the truth about this rotten-ass system,'' he
said.

I asked Cooper whom he blamed. The sheriff? The jury? ``I blame myself
first and foremost, for walking out of Chino prison, for letting those
people get their hands on me,'' he said. ``I regret that every day of my
life.''

Time and again, Cooper came back to a larger point: The criminal justice
system is unfair to poor people and members of minorities.

``I'm frameable, because I'm an uneducated black man in America,'' he
said. ``Sometimes it's race, and sometimes it's class.''

``The only people here on death row to my understanding are the poor,''
he added. ``Even the white people on death row, they're poor. If they're
white, racism goes away and classism jumps in and takes its place.''

Although Cooper's defenders note that before the murders he had never
been convicted of a violent offense, or even charged with one, it's a
bit more complicated: He has been accused of rape without being charged.

I'm particularly troubled by one episode. Cooper admits forcing a
17-year-old girl into a vehicle in 1982. She says that he also hit her,
threatened to kill her and raped her, and she went afterward to a
hospital to seek treatment; he flatly denies hitting or raping her. Hile
says that if the evidence had been strong, Cooper would have been
charged with rape. For my part, I can't think why the girl would have
lied, and although it's impossible to know after 36 years what happened,
it bothers me.

It's obvious to you by now that this is not a usual column --- I'm not
sure The Times has ever published a column of this length --- so why am
I exploring the case with such passion? I became interested primarily
because Fletcher and other respected federal appeals judges had said he
was framed. That just doesn't happen.

Death Row Exonerations

By State Since 1973

Each shape represents

one person released

from death row

Calif.

Ill.

Fla.

Tex.

La.

Okla.

Ariz.

N.C.

Ohio

Ala.

Ga.

Penn.

Miss.

Mass.

N.M.

Mo.

Tenn.

Del.

Ark.

Ind.

S.C.

Ida.

Ky.

Md.

Neb.

Sources: Death Penalty Information Center, news reports

I'm also haunted by something else. In 2000, I proposed reporting a
lengthy piece about doubts about the conviction of Cameron Willingham,
who was then on death row in Texas for the arson murder of his three
children. An editor talked me out of it, and I never did write about
Willingham, who was executed in 2004. Since then,
\href{https://www.washingtonpost.com/politics/letter-from-witness-casts-further-doubt-on-2004-texas-execution/2015/03/09/d9ebdab8-c451-11e4-ad5c-3b8ce89f1b89_story.html?utm_term=.45bc6814a119}{growing
evidence}has
\href{https://theintercept.com/2017/05/02/texas-prosecutor-in-junk-science-execution-case-stands-trial-for-misconduct/}{emerged}
that he was innocent, and perhaps it's partly to atone for my earlier
failure that I've taken up Cooper's case.

If Cooper is innocent, he would have plenty of company. The Death
Penalty Information Center says that since 1973,
\href{https://deathpenaltyinfo.org/documents/FactSheet.pdf}{at least 162
people} sentenced to death have been exonerated.
\href{http://www.pnas.org/content/111/20/7230}{One peer-reviewed study}
estimated that at least 4.1 percent of those sentenced to death in the
United States are innocent; that would mean that on California's death
row alone, where 746 people await execution, about 30 have been
wrongfully convicted.

Moreover, there's abundant evidence that executions in America are
linked to race:
\href{https://deathpenaltyinfo.org/documents/WashRaceStudy2014.pdf}{One
study in Washington State} found that jurors were three times as likely
to hand down a death sentence for a black defendant as for a white
defendant in a similar case.

Decades after Cooper's trial, many of the people involved have died or
didn't want to talk to me. Some who were willing to talk insist that the
trial was fair and Cooper was properly convicted.

William Baird, the sheriff's office lab expert who in 1983 found
suspicious shoe print evidence supposedly linking Cooper to the crime
scene, told me that the evidence was real. He acknowledged having stolen
heroin from the evidence room but said that had nothing to do with the
evidence against Cooper.

I also spoke to Bill Hughes, who discovered the bodies of the Ryens and
of his son, Chris. He is certain that Cooper is responsible: ``There is
no doubt in my mind that he did that.'' His wife, Mary Ann, is equally
passionate: She spoke of her family's suffering as the case drags on
without closure, of her certainty that Cooper is simply trying to
distract from overwhelming evidence against him, of her frustration at
calls for further testing when there has already been forensic testing
for 35 years.

I told Bill and Mary Ann Hughes that my heart breaks for them. And of
course, I can't be sure that Kevin Cooper is innocent. One lesson to
absorb from the criminal justice system's past mistakes is that we need
some humility about our own ability to ferret out truth.

That's why the governor should allow advanced DNA testing, especially of
the hairs and of the T-shirt and hatchet, and why Dianne Feinstein,
Gavin Newsom and other California politicians should back the call.

 Hatchet Hatchet found near the Ryen home

 Hair Locks of hair found in victim Jessica Ryen's hands

 Tan t-shirt T-shirt recovered with victim's blood

Source: Court documents

I know readers will ask me what they can do, and I don't have a good
answer beyond \href{https://govapps.gov.ca.gov/gov39mail/}{contacting
Brown's office} or signing a
\href{https://act.amnestyusa.org/ea-action/action?ea.client.id=1839\&ea.campaign.id=40574}{petition}
calling for clemency. Another takeaway is to regard our criminal justice
system, especially in its interactions with the poor or racial
minorities, with greater skepticism.

Maybe in the grand scheme of things, the fate of one man on death row
doesn't seem so important; innumerable people die tragically every day.
Yet we aspire to be a nation where we are all equal before the law, and
if we execute a man in so flawed a case without even bothering to test
the evidence rigorously, then a piece of our justice system dies along
with Kevin Cooper.

Governor Brown, if you're reading this, I understand that you may
believe that Cooper is guilty. But other smart people, including federal
judges and law school deans, believe him innocent. So how can you
possibly execute him without even allowing advanced DNA testing, at the
defense's expense, to resolve the doubt? What's your argument for
refusing to allow testing?

The former Supreme Court Justice Sandra Day O'Connor once wrote that
``the execution of a legally and factually innocent person would be a
constitutionally intolerable event.'' She's right: It is not just
Cooper's life that is at stake, but also the legitimacy of our system of
laws. This is a test of Governor Brown, of our justice system, of our
politicians, and of us.

``This is bigger than me,'' Cooper told me in our prison meeting. ``This
is bigger than any one person.''

\hypertarget{nicholas-kristof--may-10th-2018}{%
\section{Nicholas Kristof / May 10th
2018}\label{nicholas-kristof--may-10th-2018}}

An interview with San Bernardino County Inmate, Kevin Cooper, edited for
clarity

\includegraphics{https://static01.graylady3jvrrxbe.onion/newsgraphics/2018/04/18/opinion-kristof-deathrow/1832807d31f2b2fd84f755f0d92d38f9ec869ede/photos/cooper_copcar.jpg}

 Additional contributions by: Tony Cenicola, Sahil Chinoy, Mika Grondahl

 Opening illustration by: Cristiana Couceiro

 Opening photograph from: Associated Press

\hypertarget{more-on-nytimescom}{%
\subsection{More on NYTimes.com}\label{more-on-nytimescom}}

Advertisement

\hypertarget{site-information-navigation}{%
\subsection{Site Information
Navigation}\label{site-information-navigation}}

\begin{itemize}
\tightlist
\item
  \href{https://help.nytimes3xbfgragh.onion/hc/en-us/articles/115014792127-Copyright-notice}{©
  2020 The New York Times Company}
\item
  \href{https://www.nytimes3xbfgragh.onion}{Home}
\item
  \href{https://www.nytimes3xbfgragh.onion/search/}{Search}
\item
  Accessibility concerns? Email us at
  \href{mailto:accessibility@NYTimes.com}{\nolinkurl{accessibility@NYTimes.com}}.
  We would love to hear from you.
\item
  \href{https://help.nytimes3xbfgragh.onion/hc/en-us/articles/115015385887-Contact-Us}{Contact
  Us}
\item
  \href{https://www.nytco.com/careers/}{Work with us}
\item
  \href{https://nytmediakit.com/}{Advertise}
\item
  \href{https://help.nytimes3xbfgragh.onion/hc/en-us/articles/115014892108-Privacy-policy\#pp}{Your
  Ad Choices}
\item
  \href{https://help.nytimes3xbfgragh.onion/hc/en-us/articles/115014892108-Privacy-policy}{Privacy}
\item
  \href{https://help.nytimes3xbfgragh.onion/hc/en-us/articles/115014893428-Terms-of-service}{Terms
  of Service}
\item
  \href{https://help.nytimes3xbfgragh.onion/hc/en-us/articles/115014893968-Terms-of-sale}{Terms
  of Sale}
\end{itemize}

\hypertarget{site-information-navigation-1}{%
\subsection{Site Information
Navigation}\label{site-information-navigation-1}}

\begin{itemize}
\tightlist
\item
  \href{https://spiderbites.nytimes3xbfgragh.onion}{Site Map}
\item
  \href{https://help.nytimes3xbfgragh.onion/hc/en-us}{Help}
\item
  \href{https://help.nytimes3xbfgragh.onion/hc/en-us/articles/115015385887-Contact-Us?redir=myacc}{Site
  Feedback}
\item
  \href{https://www.nytimes3xbfgragh.onion/subscription?campaignId=37WXW}{Subscriptions}
\end{itemize}
