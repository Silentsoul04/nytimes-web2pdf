 **NYTimes.com no longer supports Internet Explorer 9 or earlier. Please
upgrade your browser.
\href{http://www.nytimes3xbfgragh.onion/content/help/site/ie9-support.html}{LEARN
MORE »}

**Sections

**Home

**Search

\hypertarget{the-new-york-times}{%
\subsection{\texorpdfstring{\href{http://www.nytimes3xbfgragh.onion/}{The
New York Times}}{The New York Times}}\label{the-new-york-times}}

 \href{/section/climate}{Climate} \textbar{}How Does Your State Make
Electricity?

**Close search

\hypertarget{site-search-navigation}{%
\subsection{Site Search Navigation}\label{site-search-navigation}}

Search NYTimes.com

**Clear this text input

Go

\url{https://nyti.ms/2RcZTUu}

\hypertarget{site-navigation}{%
\subsection{Site Navigation}\label{site-navigation}}

\hypertarget{site-mobile-navigation}{%
\subsection{Site Mobile Navigation}\label{site-mobile-navigation}}

\hypertarget{how-does-your-state-make-electricity}{%
\section{How Does Your State Make
Electricity?}\label{how-does-your-state-make-electricity}}

America isn't making electricity the way it did two decades ago: Natural
gas has edged out coal as the country's leading generation source
\ldots{}

\ldots{} and renewables like wind and solar have made small yet speedy
gains. But, each state has its own story.

In Nevada, natural gas surpassed coal as the top source of electricity
generation in 2005, earlier than in many other states. Coal's role in
the state's power mix has continued to decline since then.

In Iowa, wind power has taken off over the past decade. It now makes up
nearly 40 percent of the electricity produced in the state.

But in West Virginia, coal still fuels nearly all electricity
generation.

Overall, fossil fuels still dominate electricity generation in the
United States. But the shift from coal to natural gas has helped to
lower carbon dioxide emissions and other pollution. Last year, coal was
the main source of electricity generation for 18 states, down from 32
states in 2001.

Top Source of Electricity Generation In Every State

Coal

Natural gas

Nuclear

Hydroelectric

Petroleum

2001

2017

Coal

Natural gas

Nuclear

Hydroelectric

Petroleum

2001

2017

Coal

Natural gas

Nuclear

Hydroelectric

Petroleum

2001

2017

Coal

Natural gas

Nuclear

Hydroelectric

Petroleum

2001

2017

But experts warn that a shift to natural gas alone won't be enough to
curb emissions and avoid dangerous global warming.

``Switching from coal to gas is a fine thing to do in the short run, but
it's not a solution in the longer run,'' said Severin Borenstein,
Director of the Energy Institute at the University of California,
Berkeley's Haas School of Business. ``Gas still produces a lot of
greenhouse gases. We can't stay on gas and solve this problem.
Ultimately we're going to have to go to much lower or zero-carbon
sources.''

We charted every state's electricity generation mix between 2001 and
2017 using data from the United States Energy Information
Administration. Scroll down or skip to your state: **

In 2001, coal fueled more than half of the electricity produced in
Alabama, but several of the state's aging coal plants have closed since
then or transitioned to burning cheaper natural gas. By 2017, natural
gas was the top electricity source in the state, followed by nuclear.
Coal came in third place, providing just under a quarter of the state's
power generation.

Alabama generates more electricity than it consumes, and typically sends
about one-third of its output to nearby states.

Natural gas has been Alaska's top source of electricity generation since
2001, but hydroelectric power has increased its share during that time.
The state aims to get 50 percent of its electricity from renewable
sources by 2025, but
\href{http://www.legis.state.ak.us/basis/get_bill_text.asp?hsid=HB0306A\&session=26}{that
goal} is voluntary and has no legal weight.

Alaska has its own electric grid, which means that ``whatever
electricity is created there is what they're consuming,'' said Glenn
McGrath, a power systems analyst at the Energy Information
Administration. ``It's about as isolated as you can get.''

Many of Alaska's rural communities are not connected to the main grid at
all and use diesel generators for power.

Coal was Arizona's top source of electricity generation until 2016, when
natural gas produced more power. Last year, natural gas, nuclear and
coal each provided a little less than a third of the electricity
produced in the state.

But coal power is expected to decline further. The state's Navajo
Generating Station, the largest coal-fired power plant in the West, is
slated to close in 2019, largely because of competition from cheaper
natural gas.

Arizona supplies electricity throughout the Southwest. The state has
abundant solar potential and
\href{https://www.energy.gov/savings/renewable-energy-standard-0}{will
require} utilities to get 15 percent of their electricity from renewable
sources by 2025. In November,
\href{https://www.nytimes3xbfgragh.onion/2018/11/07/climate/climate-change-midterm-elections.html}{voters
rejected a ballot initiative} that would have raised that target to a
more ambitious 50 percent by 2035.

Coal was the top source of electricity produced in Arkansas every year
between 2001 and 2017, but its generation share slowly decreased during
that time. Natural gas, meanwhile, grew to provide more than a quarter
of the electricity produced in the state last year, up from just 6
percent in 2001.

Arkansas generates more electricity than it consumes and exports power
to nearby states.

Natural gas has been California's top electricity source since 2001. But
half of the power produced in the state last year came from renewable
sources, including solar, wind, geothermal, and hydroelectricity.

Hydroelectric power, which dwindled between 2014 and 2015 because of
drought, rose again last year to provide the largest share of renewable
generation in the state. Solar power has grown quickly over the past
five years, largely because of state policies like an aggressive
renewable energy standard. This year, California commited to get all of
its electricity from
\href{https://www.nytimes3xbfgragh.onion/2018/08/28/business/energy-environment/california-clean-energy.html}{zero-carbon}\href{https://www.nytimes3xbfgragh.onion/2018/08/28/business/energy-environment/california-clean-energy.html}{sources
by 2045}.

Last year, about a fourth of the electricity consumed in the state,
including some generated by coal power, came from outside of its
borders. (Imports are not pictured in the graphic above.) But California
plans to stop buying electricity from coal-burning plants in Utah and
other states.

The vast majority of the electricity generated in Colorado comes from
fossil fuel sources: about half from coal, and a quarter from natural
gas. But wind power has been on the rise over the past decade. Last
year, wind was the third-largest source of electricity produced in
Colorado, accounting for nearly a fifth of the state's generation.

Colorado has set a requirement that 30 percent of the electricity sold
by utilities come from renewable sources by 2020.

Nuclear power and natural gas supplied the vast majority of electricity
generated in Connecticut between 2001 and 2017. Natural gas power has
been on the rise during that time, accounting for nearly half of the
state's electricity generation last year, up from just 13 percent nearly
two decades earlier. Coal-fired generation has almost entirely
disappeared in the state and Connecticut's last remaining coal plant,
Bridgeport Harbor, is
\href{https://www.courant.com/community/bridgeport/hc-last-ct-coal-plant-20160211-story.html}{scheduled
to close in 2021}.

Five percent of the electricity generated in Connecticut came from
renewable sources in 2017. This year, the state expanded its renewable
energy standard to
\href{https://www.utilitydive.com/news/connecticut-lawmakers-pass-sweeping-energy-bill/523264/}{require}
that utilities get 40 percent of the electricity they sell to consumers
from renewable sources by 2030.

Natural gas displaced coal as the primary source of electricity produced
in Delaware in 2010, and coal's generation share has declined
dramatically since then. Coal provided 70 percent of the power produced
in Delaware in 2008, its peak year, but slightly less than 5 percent by
2017. Natural gas more than quadrupled its generation share during the
same period.

Thanks in part to this shift, carbon dioxide emissions from the state's
electricity sector have fallen over the past decade. Delaware
\href{http://www.dnrec.delaware.gov/energy/information/otherinfo/Pages/Renewable.aspx}{will
require} that utilities get 25 percent of their electricity from
renewable sources by 2025.

Power produced in the state supplies ``between two-thirds and
three-fourths of the electricity sold to Delaware customers,'' according
to the E.I.A. The rest comes from neighboring states through the
regional grid. (Imports are not shown in the chart above.)

In 2001, more than a third of the electricity produced in Florida came
from burning coal, but natural gas surpassed coal as the state's top
generation source two years later and continued to expand its share of
the state's power mix. By 2017, natural gas made up two thirds of
Florida's electricity generation, more than double the national average.

Florida is the second-largest producer of electricity nationwide, after
Texas, but still relies on imports from neighboring states to meet
consumer demand.

Despite its nickname, the Sunshine State generates very little power
through solar energy and has no renewable energy requirements.

Coal provided the majority of Georgia's power generation through the
2000s but declined as natural gas power increased. In recent years,
coal's generation share has dropped sharply as several aging coal-fired
plants have been retired.

Utilities in the state are in the process of building
\href{https://www.nytimes3xbfgragh.onion/2017/08/31/business/georgia-vogtle-nuclear-reactors.html}{two
new nuclear reactors}, the only new nuclear projects under construction
in the country.

About a tenth of Georgia's power generation came from renewable sources
last year, mostly biomass and hydroelectricity. But solar power is
growing quickly in the state. Georgia doesn't impose any statewide
renewable energy requirements, but the city of Atlanta is developing a
plan to get \href{http://100atl.com/}{all of its electricity from
renewable sources by 2035}.

Hawaii has relied heavily on imported petroleum to make electricity for
the past two decades. But the state has a
\href{http://energy.hawaii.gov/renewable-energy}{bold plan} to generate
all of its power from local renewable sources by 2045.

Last year, renewables accounted for a fourth of the power produced in
Hawaii, up from less than a tenth in 2001. Solar generation, mostly from
small-scale rooftop panels, has grown rapidly in the state over the past
five years.

Hydroelectric power has long dominated Idaho's generation mix. But, in
recent years, its share has fallen, partly because of drought. The state
still produces the majority of its electricity from renewable sources,
with wind powering 15 percent of in-state generation last year, up from
less than 2 percent a decade ago. Solar power, while still a small
share, increased sharply between 2016 and 2017.

Idaho relies heavily on out of state imports to meet electricity demand.
While coal makes up only a fraction of in-state generation, in the end
``about one-third of the electricity consumed in Idaho is from
coal-fired power plants located in other states,''
\href{https://www.eia.gov/state/analysis.php?sid=ID\#51}{according to
the E.I.A.} (Import data is not shown in the chart above.)

Nuclear power is Illinois' top source of electric generation. It has
provided more than half of the power produced in the state for nearly
two decades. Coal is an important source of power for the state, too --
even surpassing nuclear as the top generation source twice over the past
decade, in 2004 and again in 2008 -- but its share has declined in
recent years as old power plants have been retired or converted to burn
natural gas. Both natural gas and wind power have increased over the
past decade.

Illinois produces ``considerably more'' electricity than it uses
in-state, \href{https://www.eia.gov/state/analysis.php?sid=IL}{according
to the E.I.A.} It sends the surplus to Mid-Atlantic and Midwestern
states through regional grids.

Coal has generated most of the electricity made in Indiana for nearly
two decades, but, in recent years, natural gas and wind power have made
inroads. Natural gas accounted for 2 percent of the state's electricity
generation in 2001 but grew to provide nearly 20 percent in 2017.

The Indiana Legislature established a voluntary clean energy standard in
2011 that encourages electric utilities to get an increasing amount of
power from renewable and other alternative energy sources. However, no
Indiana utilities participated in the program last year, according to
the E.I.A.

Wind power has exploded in Iowa over the past decade. Wind provided just
1 percent of the electricity produced in the state in 2001 but climbed
to nearly 40 percent by 2017. Iowa still produces nearly half its
electricity from coal, but coal's generation share has declined since
2010.

In absolute terms, the state, one of the
\href{https://windexchange.energy.gov/maps-data/319}{windiest in the
country}, was the third-largest producer of wind power last year, after
Texas and Oklahoma. Iowa produces more power than it consumes, sending
the surplus to nearby states.

Iowa in 1983 became the first state to pass legislation requiring
utilities to get some amount of electricity from renewable resources,
but the state has not updated its standards.

Like many \href{https://windexchange.energy.gov/maps-data/319}{Great
Plains states}, Kansas has seen significant growth in wind power over
the past decade. The share of electricity generated from wind has
increased fivefold since 2010.

In 2009, the Kansas Legislature passed a renewable energy standard
requiring utilities to get an increasing amount of electricity from
wind, solar and other renewable sources -- up to 20 percent by 2020. But
Gov. Sam Brownback and state legislators softened the measure in 2015,
making the goal voluntary, after conservative groups with ties to the
industrial conglomerate Koch Industries
\href{https://www.kansascity.com/news/politics-government/article20186187.html}{lobbied
against the stricter standard}.

Coal still powers the vast majority of the electricity produced in
Kentucky, a longtime coal mining state. Last year, coal was the source
of nearly 80 percent of state generation, but for most of the past two
decades that number hovered closer to 90 percent.

Since 2014, a number of Kentucky's older coal plants have been shut down
or converted to burn natural gas, which provided 13 percent of the
state's electricity generation in 2017.

Natural gas provides the bulk of electricity generation in Louisiana,
one of the top-five producers of natural gas in the country. Last year,
gas accounted for 60 percent of electricity made in the state, up from
46 percent in 2001. During that time, coal-fired generation declined,
dropping from its position as the second-biggest source of power in the
state to third place.

Louisiana also gets some electricity from neighboring states. (Imports
are not in the chart above.)

Maine ``leads New England in wind power generation,'' according to the
E.I.A. Last year, wind supplied one-fifth of the electricity produced in
the state. Hydroelectric and biomass power, which comes from burning
wood and other organic material, were the next-biggest sources of
generation.

Since 2000, the state has required that electricity providers get 30
percent of the power they sell to customers from existing renewable
resources. In 2017, utilities were expected to get 10 percent from new
renewable sources. The state has separate goals for wind-energy
development.

The total amount of electricity created in Maine has declined since
2010, especially from natural gas power, and the state has increasingly
relied on energy imports from Canada. (Imports are not included in the
chart above.)

Coal power has been on the decline in Maryland for a decade and has
provided less than half of the electricity produced in the state since
2012. During that time, the share of electricity generated by nuclear
power and natural gas has increased.

Solar power generation, while still small, has grown quickly over the
past several years. Since 2004, the state has required that an
increasing amount of the electricity sold by utilities come from
renewable sources,
\href{https://www.psc.state.md.us/electricity/renewable-energy/}{with a
target
of}\href{https://www.psc.state.md.us/electricity/renewable-energy/}{25
percent by 2020}.

Maryland consumes more electricity than it generates and imports nearly
half of its power from other Mid-Atlantic States through the regional
grid. (Imports are not included in the chart above.)

Natural gas has more than doubled its share of electricity generation in
Massachusetts over the past two decades. Coal and oil generation fell
sharply during that same period, and the state's last large coal-fired
power plant
\href{https://www.wbur.org/bostonomix/2017/05/31/brayton-power-plant-somerset}{shut
down last year}. The amount of power created from solar energy has
increased sharply in the state since 2013.

This year, the state toughened its mandate for utilities to sell
electricity from renewable sources, raising the requirement to
\href{https://blog.ucsusa.org/john-rogers/massachusetts-2018-clean-energy-bill}{3}\href{https://blog.ucsusa.org/john-rogers/massachusetts-2018-clean-energy-bill}{5
percent of total sales by 2030}. The new legislation also encourages
offshore wind development.

Massachusetts consumes more electricity than it produces in-state and
gets the remainder from nearby states through the regional grid.
(Imports are not shown in the chart above).

Coal remained the top source of electricity produced in Michigan last
year, but its generation share declined from a little over 60 percent in
2001 to just under 40 percent in 2017. During the same period, natural
gas nearly doubled its generation share. Wind, Michigan's main renewable
energy source, provided nearly 5 percent of the electricity produced in
the state last year.

In 2008, Michigan required utilities and other electricity providers to
get at least 10 percent of the power they sell to customer from
renewable sources by 2015. That goal was met and subsequently expanded
to 15 percent by 2021.

Coal has been the top source of electricity generated in Minnesota for
the past two decades. But coal's generation share declined between 2001
and 2017 as wind and natural gas generation grew.

The state requires utilities to gradually sell an increasing amount of
electricity from renewable sources, with a requirement of 25 percent of
total sales by 2025.

Natural gas powered more than three-quarters of the electricity
generated in Mississippi last year. Coal, once the state's top source of
electricity, has declined over the past decade, outcompeted by cheaper
natural gas. Coal provided 36 percent of the electricity produced
in-state in 2001, but just 8 percent in 2017.

Missouri's electricity generation mix hasn't changed much in nearly two
decades. Coal provided the vast majority of power generated in the state
between 2001 and 2017, declining only slightly during that time as older
coal-fired plants went offline or switched to burning natural gas.

Missouri
\href{http://programs.dsireusa.org/system/program/detail/2622}{will
require} utilities to get at least 15 percent of the electricity they
sell from renewable sources by 2021, including a small amount from solar
power.

Coal has been the top source of electricity produced in Montana for
nearly two decades but its generation share declined from 70 percent in
2001 to just under 50 percent last year. Hydropower, the state's
second-largest source of electricity, increased its share during that
time to nearly 40 percent, and wind power grew to 8 percent of in-state
generation.

Montanans only use about half of the electricity produced in the state,
\href{https://www.eia.gov/state/analysis.php?sid=MT}{according to the
E.I.A.} The state sends the rest to its Western neighbors.

Coal has been the top source of electricity produced in Nebraska for
nearly two decades, but its generation share declined slightly between
2001 and 2017. Nuclear power provided 25 percent of the state's
electricity generation on average during that time, but its share varied
from year to year.

Wind has been increasing its share of total generation over the past
decade, accounting for 15 percent of the electricity produced in the
state last year. Nebraska has the potential for substantially more wind
power, according to the E.I.A.

Natural gas edged out coal as Nevada's top electricity generation source
in 2005. The state's largest coal plant, the Mohave Generating Station,
went offline at the end of that year, further shrinking coal's role in
the state's power mix. More Nevada coal generators have shuttered since
then because of
\href{https://www.omaha.com/money/buffett/nv-energy-s-move-away-from-coal-mirrors-other-utilities/article_ee1390c5-def4-54e3-af46-9f15add993b1.html}{competition
from cheap natural gas} and
\href{https://lasvegassun.com/news/2013/jun/04/nv-energy-bill-wins-passage-signaling-shift-coal/}{state
laws} that require renewable energy development.

Last year, natural gas provided nearly 70 percent of electricity
produced in the state, followed by solar power, which supplied 12
percent of the state's generation. Until recently, Nevada required that
25 percent of the electricity sold by utilities in the state come from
renewable sources by 2025. In November, Nevadans
\href{https://grist.org/article/why-nevada-upped-its-renewable-energy-standards-and-arizona-didnt/}{voted
to increase that requirement} to 50 percent by 2030.

The bulk of electricity generated in New Hampshire comes from the
Seabrook nuclear power plant, the largest reactor in New England.
Natural gas has provided about a fifth of the power produced in the
state since the early 2000s, when two new generating stations began
operating. The share of New Hampshire's electricity generated from coal
has dwindled over the past two decades, from 25 percent in 2001 to less
than 2 percent in 2017.

The state is requiring utilities to get 25 percent of the electricity
they sell to customers from renewable resources by 2025. The top two
sources of renewable energy in the state are biomass, or energy that
comes from burning wood and other organic matter, and hydroelectric
power.

New Hampshire produces more power than is consumed in-state and sends
about half to neighboring states through New England's regional electric
grid. (Exports are not included in the chart above.)

Nuclear power was the top source of electricity generated in New Jersey
until recently, when it was edged out by natural gas. Last year, natural
gas accounted for nearly half of the state's power generation, and
nuclear power supplied 45 percent. Solar energy contributed 4 percent of
the state's electricity.

This year, New Jersey
\href{https://www.nytimes3xbfgragh.onion/2018/04/12/nyregion/new-jersey-renewable-energy.html}{increased
its renewable energy standard} to require that 21 percent of the
electricity sold in the state come from renewable sources by 2021, with
that requirement increasing to 35 percent by 2025 and to 50 percent by
2030. In an effort to further reduce carbon emissions, the state also
passed legislation to prop up its nuclear plants, which currently
provide the largest portion of zero-emissions energy.

The state gets some of the power it consumes through the Mid-Atlantic
regional grid. (Imports are not included in the chart above.)

Coal has been New Mexico's primary source of electricity generation for
nearly two decades. But coal-fired power declined since 2004 ``in
response to tougher air quality regulations, cheaper natural gas, and
California's decision in 2014 to stop purchasing electricity generated
from coal'' in neighboring states, according to the E.I.A.

Natural gas, wind and solar accounted for a little less than half of the
electricity produced in New Mexico last year, up from just 15 percent
two decades earlier. The state will require utilities to get 20 percent
of the electricity they sell from renewable energy by 2020. New Mexico
is also looking to increase generation from zero-carbon sources because
it sends a significant amount power to California, a state with some of
the strictest renewable energy policies in the country.

Natural gas and nuclear power have supplied the majority of electricity
generated in New York for nearly two decades and their share has
expanded as coal use in the state has declined. For the past decade, New
York has also produced about a fifth of its electricity from hydropower,
the state's largest source of renewable energy.

The state will require utilities to get
\href{https://www.nytimes3xbfgragh.onion/2015/11/23/nyregion/gov-cuomo-to-order-large-increase-in-renewable-energy-in-new-york-by-2030.html?module=inline}{50
percent of the
power}\href{https://www.nytimes3xbfgragh.onion/2015/11/23/nyregion/gov-cuomo-to-order-large-increase-in-renewable-energy-in-new-york-by-2030.html?module=inline}{they
sell to
consumers}\href{https://www.nytimes3xbfgragh.onion/2015/11/23/nyregion/gov-cuomo-to-order-large-increase-in-renewable-energy-in-new-york-by-2030.html?module=inline}{from
renewable sources by 2030}, an ambitious goal, and aims to substantially
reduce greenhouse gas emissions. Wind and solar energy make up a small
but growing portion of New York's electricity generation, together
providing just over 4 percent of the state's electricity last year.

New York tends to consume more energy than it creates and imports some
electricity from neighboring states and Canada. (Electricity imports are
not included in the chart above.)

Coal provided the majority of North Carolina's electricity generation
between 2001 and 2011. But
\href{https://www.eia.gov/state/analysis.php?sid=NC}{nearly}\href{https://www.eia.gov/state/analysis.php?sid=NC}{30}\href{https://www.eia.gov/state/analysis.php?sid=NC}{of
the state's coal-burning units shut down} over the following six years
and, by 2017, coal generation had dropped below nuclear and natural gas
power. Natural gas generation increased after the national fracking boom
of the late 2000s and became the second-largest source of electricity
generation in the state in 2016.

North Carolina is currently the only Southern state with significant
solar generation. The state's unique implementation of a decades-old
federal mandate, the Public Utility Regulatory Policies Act of 1978, has
\href{https://www.eia.gov/todayinenergy/detail.php?id=27632}{encouraged
the growth of utility-scale solar}. North Carolina has also set a
requirement that utilities get 12.5 percent of the electricity they sell
to consumers from renewable energy resources by 2021.

As in many Great Plains states, wind energy has taken off in North
Dakota over the past decade. Last year, wind powered more than a quarter
of the electricity produced in the state, up from less than 2 percent a
decade earlier.

In 2007, the North Dakota Legislature set a voluntary goal for
utilities: to get 10 percent of the electricity sold to consumers from
renewable or recycled energy by 2015. That goal was met and even
surpassed,
\href{https://bismarcktribune.com/news/state-and-regional/low-cost-of-electricity-in-nd-discouraging-solar-energy-investments/article_ff5184db-d4fd-51c1-b91b-6dbc0e5d3d0c.html}{according
to utility analysts}.

North Dakota produces more electricity than is consumed in the state and
about half is sent to its neighbors. (Exports are not charted above.)

Coal has been the top source of electricity produced in Ohio for nearly
two decades but its generation share has been decreasing since 2011 as
several of the state's coal-fired power plants have closed down. Over
that same period, natural gas has increased its share in Ohio's electric
generation mix.

Wind is currently the state's top source of renewable energy, though it
provided only about 1 percent of the electricity generated in Ohio last
year. The state wants to expand that, however. It
\href{http://programs.dsireusa.org/system/program/detail/2934}{will
require} utilities to get at least 12.5 percent of the electricity they
sell to consumers from renewable sources by the end of 2026.

The bulk of Oklahoma's power generation for much of the past two decades
has come from natural gas and coal, with the two often competing to be
the state's top source of electricity. But in 2016, wind surpassed coal
as the second-largest source of electricity produced in the state.

Last year, the state was second only to Texas in total electricity
generation from wind.

In 2010, Oklahoma
\href{http://programs.dsireusa.org/system/program/detail/4178}{requ}\href{http://programs.dsireusa.org/system/program/detail/4178}{ired}
that 15 percent of its generation capacity comes from renewable sources
by 2015. It also designated natural gas as its preferred choice for new
fossil fuel projects. The state had
\href{http://digitalprairie.ok.gov/cdm/ref/collection/stgovpub/id/216170}{exceeded
the renewable target by 2012}.

Most of the electricity produced in Oregon in any given year comes from
hydropower but the share produced from water fluctuates with
precipitation. Power from natural gas typically increases during drought
years, and decreases in years with ample hydroelectricity.

Over the past decade, wind power has grown to become the third-largest
source of electricity generated in the state. In an effort to encourage
more non-hydroelectric renewable energy, Oregon will require its largest
utilities to get 50 percent of the electricity they sell from new
renewable energy sources by 2040. The program covers projects introduced
or upgraded since 1995, a cutoff that would exclude older hydropower.

Coal powered the bulk of electricity produced in Pennsylvania through
2014, when it fell below nuclear for the first time. Coal's generation
share in the state decreased after the late-2000s fracking boom as aging
coal power plants closed because of competition from cheaper natural
gas.

Last year, nuclear power was the top source of electricity generated in
Pennsylvania. But natural gas is putting economic pressure on the
state's nuclear generators, too, with one reactor scheduled to shut down
in 2019. Pro-nuclear groups,
\href{https://www.npr.org/sections/thetwo-way/2017/05/30/530708793/three-mile-island-nuclear-power-plant-to-shut-down-in-2019}{saying
the loss of this emissions-free electricity is bad news} for climate
change, have sought state subsidies for nuclear energy.

Pennsylvania will require that 18 percent of the electricity that
utilities sell to consumers come from renewable and alternative energy
by 2021, with at least 0.5 percent coming from solar power. Last year,
renewable energy made up about 5 percent of in-state generation.

Pennsylvania is the country's third-biggest generator of electricity,
behind Texas and Florida. The state is a big supplier of energy to the
Mid-Atlantic region.

Natural gas dominates electricity generation in Rhode Island, but wind
and solar energy, while still small, have grown quickly in recent years.

Rhode Island
\href{http://programs.dsireusa.org/system/program/detail/1095}{will
require} electricity providers to get nearly two-fifths of the power
they sell to consumers from renewable sources by 2035. The state
consumes more electricity than it generates and gets the rest from
neighboring states. (Imports are not included in the chart above.)

A majority of the electricity generated in South Carolina comes from
nuclear power, with coal and natural gas taking second and third place,
respectively. Coal's generation share has declined over the past decade
as power from natural gas has increased.

South Carolina produces more power than it consumes and sends the
surplus to neighboring states.

Hydropower has supplied the majority of the electricity created in South
Dakota for most of the past two decades, but coal generation surpassed
hydroelectricity during three years: 2001, 2004 and 2008. Since then,
coal's share of the state generation mix has declined, while the share
coming from wind power has increased.

Last year, wind was the second-largest source of electricity produced in
South Dakota, accounting for nearly a third of generation in the state.

South Dakota exports power to states across the Central and Western
United States.

Coal supplied most of the electricity produced in Tennessee between 2001
and 2016, but its generation share started to decline about a decade ago
as natural gas power gained share. Last year, coal-powered generation
dipped below nuclear for the first time in nearly two decades.

Tennessee consumes more power than it produces and makes up the
shortfall with electricity from nearby states. (Imports are not included
in the chart above.)

Texas produces more electricity than any other state, and natural gas
has been its top generation source since 2001, with coal in second
place. But coal's generation share has declined as wind power has
increased. In 2014, wind overtook nuclear power as the third-largest
source of electricity produced in the state. Texas produces more power
\href{https://www.nytimes3xbfgragh.onion/2019/04/03/climate/fact-check-trump-windmills.html}{from
wind} in total than any other state, with Oklahoma and Iowa in second
and third place.

Texas
\href{http://programs.dsireusa.org/system/program/detail/182}{adopted a
renewable energy requirement} in 1999 , requiring the state to install
10,000 megawatts of renewable energy capacity by 2025. It has already
reached that goal.

The majority of electricity produced in Utah comes from coal, but coal's
share has declined over the last several years as natural gas has
increased.

The state produces more energy than it consumes and sends the surplus to
nearby states like California. At least one Utah power plant is
switching from burning coal
\href{http://articles.latimes.com/2013/apr/23/local/la-me-ln-council-coal-energy-20130423}{to
natural gas} to comply with California's stricter environmental
regulations.

Solar power grew to become the largest renewable generation source in
the state in 2016 and expanded its share again last year. Utah has set a
goal for utilities to get 20 percent of the electricity they sell from
renewable sources by 2025.

Most of the electricity generated in Vermont came from nuclear power
until 2014, when the state's only nuclear plant, the Vermont Yankee
station, closed down. Since then, nearly all of the electricity produced
in the state has come from renewable sources, including hydropower,
biomass, wind and solar. But, Vermont's absolute generation capacity has
substantially declined.

Vermont imports most of its electricity from nearby states and Canada.
Last year, the state's own generation ``provided only about two-fifths
of the electricity consumed in Vermont,'' according to E.I.A.

Vermont's ambitious renewable energy goal requires that 75 percent of
electricity sold in the state come from renewable sources by 2032,
including 10 percent from small, in-state sources.

Coal was the top source of electricity produced in Virginia between 2001
and 2008, when its share began to decline. Natural gas power increased
in the state following the national fracking boom of the late 2000s and
took over as the state's primary generation source in 2015. Nuclear
generation has provided a little more than a third of Virginia's
electricity, on average, over the past two decades.

Virginia consumes more electricity that it generates, so it gets
additional power from nearby states through the Mid-Atlantic regional
grid. The state has established a voluntary goal for utilities to get 15
percent of the electricity they sell from renewable sources by 2025.

Hydropower has supplied a majority of the electricity created in
Washington every year since 2001, but its share of state generation has
fluctuated with precipitation. Coal, natural gas, nuclear and wind power
have alternated as the second-largest source of electricity produced in
the state for most of the past two decades.

Washington produces more electricity than it consumes and exports power
to Canada and other Western states. The state will require its larger
utilities to get 15 percent of their electricity sales from new
renewable sources by 2020.

Coal dominates West Virginia's power generation mix, supplying more than
90 percent of the electricity produced in the state every year for
nearly two decades. Hydropower provided a small portion of in-state
generation between 2001 and 2017. Wind and natural gas have increased
their generation share in recent years, but each of those sources only
accounted for about 2 percent of electricity created in the state last
year.

After years of lobbying by conservative groups, West Virginia became the
first state to
\href{https://www.wvgazettemail.com/news/politics/w-va-house-votes-to-repeal-energy-law/article_be7ce0b8-13cd-5bdb-91da-7e679de9c4d5.html}{repeal}\href{https://www.wvgazettemail.com/news/politics/w-va-house-votes-to-repeal-energy-law/article_be7ce0b8-13cd-5bdb-91da-7e679de9c4d5.html}{its
renewable energy standard} in 2015. The law would have required
utilities to get 25 percent of their electricity from alternative and
renewable energy sources by 2025. Opponents of the standard said it was
hurting coal jobs and raising electricity rates, while supporters said
it would help to diversify the state's electric sector at a time when
the national coal market was in decline.

West Virginia generates more electricity than it consumes and supplies
about half of its power to other Mid-Atlantic States through the shared
regional grid. (Exports are not pictured in the chart above.)

Most of the electricity produced in Wisconsin comes from coal, but
natural gas generation has increased over the past three years. Wind
power established a foothold in the state a decade ago and has slowly
expanded its share of electricity generation, too.

Wisconsin required its utilities to get 10 percent of the electricity
sold in the state from renewable sources by the end of 2015. That goal
was surpassed two years ahead of schedule.

The vast majority of electricity generated in Wyoming comes from coal,
but wind power has made inroads during the past decade. Last year, wind
provided nearly a tenth of the electricity produced in the state.

Because of its small population, Wyoming produces much more power than
it consumes and sends about 60 percent to nearby states.

Additional development by Josh Williams.

\hypertarget{more-on-nytimescom}{%
\subsection{More on NYTimes.com}\label{more-on-nytimescom}}

Advertisement

\hypertarget{site-information-navigation}{%
\subsection{Site Information
Navigation}\label{site-information-navigation}}

\begin{itemize}
\tightlist
\item
  \href{https://help.nytimes3xbfgragh.onion/hc/en-us/articles/115014792127-Copyright-notice}{©
  2020 The New York Times Company}
\item
  \href{https://www.nytimes3xbfgragh.onion}{Home}
\item
  \href{https://www.nytimes3xbfgragh.onion/search/}{Search}
\item
  Accessibility concerns? Email us at
  \href{mailto:accessibility@NYTimes.com}{\nolinkurl{accessibility@NYTimes.com}}.
  We would love to hear from you.
\item
  \href{https://help.nytimes3xbfgragh.onion/hc/en-us/articles/115015385887-Contact-Us}{Contact
  Us}
\item
  \href{https://www.nytco.com/careers/}{Work with us}
\item
  \href{https://nytmediakit.com/}{Advertise}
\item
  \href{https://help.nytimes3xbfgragh.onion/hc/en-us/articles/115014892108-Privacy-policy\#pp}{Your
  Ad Choices}
\item
  \href{https://help.nytimes3xbfgragh.onion/hc/en-us/articles/115014892108-Privacy-policy}{Privacy}
\item
  \href{https://help.nytimes3xbfgragh.onion/hc/en-us/articles/115014893428-Terms-of-service}{Terms
  of Service}
\item
  \href{https://help.nytimes3xbfgragh.onion/hc/en-us/articles/115014893968-Terms-of-sale}{Terms
  of Sale}
\end{itemize}

\hypertarget{site-information-navigation-1}{%
\subsection{Site Information
Navigation}\label{site-information-navigation-1}}

\begin{itemize}
\tightlist
\item
  \href{https://spiderbites.nytimes3xbfgragh.onion}{Site Map}
\item
  \href{https://help.nytimes3xbfgragh.onion/hc/en-us}{Help}
\item
  \href{https://help.nytimes3xbfgragh.onion/hc/en-us/articles/115015385887-Contact-Us?redir=myacc}{Site
  Feedback}
\item
  \href{https://www.nytimes3xbfgragh.onion/subscription?campaignId=37WXW}{Subscriptions}
\end{itemize}
