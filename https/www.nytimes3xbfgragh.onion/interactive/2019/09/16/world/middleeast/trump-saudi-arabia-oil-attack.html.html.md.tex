Sections

SEARCH

\protect\hyperlink{site-content}{Skip to
content}\protect\hyperlink{site-index}{Skip to site index}

\href{https://www.nytimes3xbfgragh.onion/section/world/middleeast}{Middle
East}

\href{https://myaccount.nytimes3xbfgragh.onion/auth/login?response_type=cookie\&client_id=vi}{}

\href{https://www.nytimes3xbfgragh.onion/section/todayspaper}{Today's
Paper}

\href{/section/world/middleeast}{Middle East}\textbar{}Who Was Behind
the Saudi Oil Attack? What the Evidence Shows

\url{https://nyti.ms/2O3mBfO}

\begin{itemize}
\item
\item
\item
\item
\item
\item
\end{itemize}

Advertisement

\protect\hyperlink{after-top}{Continue reading the main story}

\hypertarget{comments}{%
\subsection{\texorpdfstring{\protect\hyperlink{commentsContainer}{Comments}}{Comments}}\label{comments}}

\href{}{Who Was Behind the Saudi Oil Attack? What the Evidence
Shows}\href{}{Skip to Comments}

The comments section is closed. To submit a letter to the editor for
publication, write to
\href{mailto:letters@NYTimes.com}{\nolinkurl{letters@NYTimes.com}}.

\hypertarget{who-was-behind-the-saudi-oil-attack-what-the-evidence-shows}{%
\section{Who Was Behind the Saudi Oil Attack? What the Evidence
Shows}\label{who-was-behind-the-saudi-oil-attack-what-the-evidence-shows}}

By
\href{https://www.nytimes3xbfgragh.onion/by/david-d-kirkpatrick}{David
D. Kirkpatrick},
\href{https://www.nytimes3xbfgragh.onion/by/christoph-koettl}{Christoph
Koettl},
\href{https://www.nytimes3xbfgragh.onion/by/allison-mccann}{Allison
McCann}, \href{https://www.nytimes3xbfgragh.onion/by/eric-schmitt}{Eric
Schmitt},
\href{https://www.nytimes3xbfgragh.onion/by/anjali-singhvi}{Anjali
Singhvi} and Gus WezerekSept. 16, 2019

\begin{itemize}
\item
\item
\item
\item
\item
  \emph{166}
\end{itemize}

Mediterranean Sea

IRAN

IRAQ

Oil and gas facilities

damaged during

attacks on Saturday

EGYPT

Khurais

Abqaiq

Gulf of Oman

SAUDI

ARABIA

Red

Sea

OMAN

SUDAN

Arabian Sea

YEMEN

IRAN

IRAQ

Oil and gas facilities

damaged during

attacks on Saturday

Khurais

EGYPT

Abqaiq

SAUDI

ARABIA

SUDAN

OMAN

YEMEN

Arabian Sea

IRAN

IRAQ

Oil and gas facilities

damaged during

attacks on Saturday

Khurais

EGYPT

Abqaiq

SAUDI

ARABIA

SUDAN

OMAN

YEMEN

Arabian Sea

The New York Times

The Trump administration said Iran was most likely behind the attacks on
Saudi Arabian oil facilities on Saturday, citing intelligence
assessments and satellite photographs showing what officials said was
\href{https://www.nytimes3xbfgragh.onion/2019/09/15/world/middleeast/iran-us-saudi-arabia-attack.html?action=click\&module=Top\%20Stories\&pgtype=Homepage}{evidence
of Iranian involvement}.

Trump administration officials said that
\href{https://www.nytimes3xbfgragh.onion/2019/09/14/world/middleeast/saudi-arabia-refineries-drone-attack.html?module=inline}{the
attacks} might have involved a combination of drones and cruise missiles
and that they did not originate from Yemen, where the Iranian-backed
Houthi militia claimed responsibility. Iran has denied any involvement.

The publicly available evidence is consistent with a few aspects of the
White House claims. But American officials have offered no evidence
beyond the satellite photos, which analysts said were insufficient to
prove where the attack came from, which weapons were used and who fired
them.

We analyzed the satellite photos that the Trump administration released,
comparing them with independent sources when possible, to determine what
they show and what they leave unclear:

The sophistication of the attack far exceeds that shown in previous
attacks by the Houthis, raising the likelihood of direct Iranian
involvement.

The satellite photos alone are not enough to support American claims
that the strikes appeared to have come from the direction of Iran or
Iraq.

There's not enough evidence to show what kinds of weapons were used, but
the precision of the strikes is consistent with a guided missile.

\hypertarget{the-attacks-complexity-exceeds-past-houthi-capabilities}{%
\subsection{The attack's complexity exceeds past Houthi
capabilities}\label{the-attacks-complexity-exceeds-past-houthi-capabilities}}

Military analysts who have studied the Yemen war say that the range,
scale and complexity of the attack on the Saudi oil installations far
exceed any capabilities that the Houthis have previously demonstrated.

American officials released satellite photographs showing what officials
said were at least 17 points of impact at two Saudi energy facilities,
though not all of the weapons necessarily hit their targets.

At one facility, Abqaiq, satellite imagery showed damage to storage
tanks and a processing train.

\includegraphics{https://static01.graylady3jvrrxbe.onion/newsgraphics/2019/09/16/saudi-oil-attacks/assets/images/Persian_Gulf_Tensions-2000.jpg}

Source: U.S. Government/DigitalGlobe, via Associated Press

In 2015, Saudi Arabia launched a campaign to try to roll back a takeover
by the Houthis, a Yemeni faction backed by Iran. The war has killed
thousands of civilians and put more than
\href{https://www.nytimes3xbfgragh.onion/interactive/2018/10/26/world/middleeast/saudi-arabia-war-yemen.html}{12
million people at risk of starvation}, but has failed to dislodge the
Houthis from control of the capital and much of the country.

Although the Houthis have often used drones to try to attack Saudi
Arabia, they have generally relied on the Samad 3, an inexpensive,
small, slow and clumsy drone that is unlikely to be able to penetrate
Saudi air defenses and reach targets with the accuracy and coordination
seen over the weekend.

More recently, they have used a more advanced drone, the Quds 1. It
could be described as a small cruise missile or a large drone with a
payload approaching that of a cruise missile.

Analysts say this is what the Houthis appear to have used to
\href{https://www.nytimes3xbfgragh.onion/2019/06/12/world/middleeast/saudi-airport-attack.html}{hit
the Abha International Airport} in southern Saudi Arabia a few months
ago. But the Quds 1 lacks the range to get from northern Yemen to the
oil installations in Saudi Arabia.

The range, scale and precision of the latest attack --- including the
successful penetration of Saudi air defenses and the avoidance of
obstacles like power lines and communication towers --- far exceeds
anything the Houthis have ever done.

Still, some security specialists say that the Houthis have greatly
improved their drone and cruise missiles, with what American officials
say is important help from Iran.

Iran's Islamic Revolutionary Guards Corps has been training its militia
proxies in the region, from Lebanon to Yemen, in more sophisticated
warfare using drones. After Houthi missiles targeting Saudi Arabia were
intercepted, Iran moved to train Houthis in drone technology, taking
groups to Iran to master assembling, managing and repairing drones.

Independent satellite imagery from Planet Labs showed the same damage as
the imagery released by the Trump administration. Storage tanks were hit
in a consistent pattern and from the same direction, another indication
of the attack's precision.

N

Likely damage

Oil storage

tanks

Processing

train

Storage tanks

Floating-top oil

storage tanks

Likely damage

Processing

train

Storage tanks

Floating-top oil

storage tanks

N

Possible damage

Oil storage

tanks

Processing

train

Storage tanks

Floating-top oil

storage tanks

Likely damage

Processing

train

Storage tanks

Floating-top oil

storage tanks

By The New York Times ·Sources: Center for Strategic and International
Studies; Imagery from Planet Labs

\hypertarget{where-the-strikes-originated-remains-unclear}{%
\subsection{Where the strikes originated remains
unclear}\label{where-the-strikes-originated-remains-unclear}}

Administration officials have not publicly said where they believe the
attack originated from. They did say that the satellite imagery was
consistent with strikes from the north or northwest, which would point
to an attack coming from the direction of Iran or Iraq, rather than from
Yemen.

But the satellite photographs released on Sunday were not as clear-cut
as officials suggested, with some appearing to show damage on the
western side of the facilities, not from the direction of Iran or Iraq.

To Iran

North

To Yemen

To Iraq

Approx. orientation

of impact points

To Iran

North

To Yemen

To Iraq

Approx. orientation

of impact points

To Iran

North

To Yemen

To Iraq

Approx. orientation

of impact points

By The New York Times ·Source: U.S. Government/DigitalGlobe, via
Associated Press

Even if the impact points indicated that the portions of the facilities
that were damaged faced Iran or Iraq, security analysts say, that does
not prove where they were launched from. Cruise missiles can be
programmed to change course, hitting a wall opposite the direction from
which they were fired.

``The origin of the attacks has to be established,'' said Behnam Ben
Taleblu, a senior fellow at the Foundation for Defense of Democracies.
``If they came from Iran, that's a game changer.''

\hypertarget{the-precision-of-the-strikes-is-consistent-with-guided-missiles}{%
\subsection{The precision of the strikes is consistent with guided
missiles}\label{the-precision-of-the-strikes-is-consistent-with-guided-missiles}}

Administration officials
\href{https://www.nytimes3xbfgragh.onion/2019/09/15/world/middleeast/iran-us-saudi-arabia-attack.html}{said
on Sunday}, without offering evidence, that the attacks came from a
combination of drones and guided cruise missiles.

The satellite photos do not offer enough information to determine what
kind of weapons were used. But Adam Simmons, a geospatial analyst at
Midgard Raven, said the precision and consistency of the damage to the
storage tanks was consistent with some type of guided munition, such as
a missile.

Analysts also pointed to photos of what appears to be missile wreckage
posted on Saudi social media that could provide further clues about how
the attack was carried out. The location and date of the photos could
not be confirmed, but the images seem to be new.

\includegraphics{https://static01.graylady3jvrrxbe.onion/newsgraphics/2019/09/16/saudi-oil-attacks/assets/images/wreckage1-2000.jpg}

The wreckage appears consistent with a Quds 1, said Fabian Hinz, a
researcher at the James Martin Center for Nonproliferation Studies,
though he cautioned that further confirmation was needed.

If the photos of wreckage are connected to the attacks, it would make it
less likely that the attack originated in Yemen, because the range of
the Quds 1 missile may not be enough to reach the site, Mr. Hinz wrote
in a
\href{https://www.armscontrolwonk.com/archive/1208062/meet-the-quds-1/}{blog
post}.

Administration officials said on Sunday they would move to make more
evidence available in the coming days. Forensic analyses of the
recovered weapons could answer questions about what they were, who
manufactured them and who launched them.

Read 166 Comments

\begin{itemize}
\item
\item
\item
\item
\end{itemize}

Advertisement

\protect\hyperlink{after-bottom}{Continue reading the main story}

\hypertarget{site-index}{%
\subsection{Site Index}\label{site-index}}

\hypertarget{site-information-navigation}{%
\subsection{Site Information
Navigation}\label{site-information-navigation}}

\begin{itemize}
\tightlist
\item
  \href{https://help.nytimes3xbfgragh.onion/hc/en-us/articles/115014792127-Copyright-notice}{©~2020~The
  New York Times Company}
\end{itemize}

\begin{itemize}
\tightlist
\item
  \href{https://www.nytco.com/}{NYTCo}
\item
  \href{https://help.nytimes3xbfgragh.onion/hc/en-us/articles/115015385887-Contact-Us}{Contact
  Us}
\item
  \href{https://www.nytco.com/careers/}{Work with us}
\item
  \href{https://nytmediakit.com/}{Advertise}
\item
  \href{http://www.tbrandstudio.com/}{T Brand Studio}
\item
  \href{https://www.nytimes3xbfgragh.onion/privacy/cookie-policy\#how-do-i-manage-trackers}{Your
  Ad Choices}
\item
  \href{https://www.nytimes3xbfgragh.onion/privacy}{Privacy}
\item
  \href{https://help.nytimes3xbfgragh.onion/hc/en-us/articles/115014893428-Terms-of-service}{Terms
  of Service}
\item
  \href{https://help.nytimes3xbfgragh.onion/hc/en-us/articles/115014893968-Terms-of-sale}{Terms
  of Sale}
\item
  \href{https://spiderbites.nytimes3xbfgragh.onion}{Site Map}
\item
  \href{https://help.nytimes3xbfgragh.onion/hc/en-us}{Help}
\item
  \href{https://www.nytimes3xbfgragh.onion/subscription?campaignId=37WXW}{Subscriptions}
\end{itemize}
