\hypertarget{what-does-it-mean-when-an-artist-retreats-from-public-life}{%
\section{What Does It Mean When an Artist Retreats From Public
Life?}\label{what-does-it-mean-when-an-artist-retreats-from-public-life}}

April 13, 2020

\begin{itemize}
\item
\item
\item
\item
\end{itemize}

A small and highly influential group has chosen to disappear from
society in favor of letting their work speak for itself.

\href{https://www.nytimes3xbfgragh.onion/interactive/2020/04/13/t-magazine/culture-issue-2020.html}{We
Are Family}

\hypertarget{chapter-3-legends-pioneers-and-survivors}{%
\subparagraph{Chapter 3: Legends Pioneers and
Survivors}\label{chapter-3-legends-pioneers-and-survivors}}

\hypertarget{previous}{%
\subparagraph{Previous}\label{previous}}

\hypertarget{next}{%
\subparagraph{Next}\label{next}}

\hypertarget{what-does-it-mean-when-an-artist-retreats-from-public-life-1}{%
\section{What Does It Mean When an Artist Retreats From Public
Life?}\label{what-does-it-mean-when-an-artist-retreats-from-public-life-1}}

\hypertarget{the-shadows}{%
\subsection{The Shadows}\label{the-shadows}}

These days, artists of all kinds are expected to be available for public
consumption. But a small and highly influential group of them has chosen
to disappear from society in favor of letting their work speak for
itself. What does it mean to be inaccessible in an age of oversharing?

By \href{https://www.nytimes3xbfgragh.onion/by/megan-o-grady}{Megan
O'Grady}

April 13, 2020

SHARE

FOR THOSE OF us old enough to remember an era when we didn't account for
our existence on social media, when we could attend a dinner party
without being tagged like a shot deer on someone's Instagram story, when
privacy was respected and deeper meanings had room to quietly take root
and bloom, it is no surprise to see artists flinching from the din of
publicity. How can we really look and listen when we are so busy being
seen and heard?

Art, as
\href{https://www.nytimes3xbfgragh.onion/topic/person/susan-sontag}{Susan
Sontag} wrote in a 1967 essay,
``\href{http://www.susansontag.com/SusanSontag/books/stylesOfRadicalWillExerpt.shtml}{The
Aesthetics of Silence},'' has acquired a spiritual quality in secular
culture, becoming a place to reckon with and question the human project
and, perhaps, even transcend it. To create, in other words, isn't only
about self-expression; it is also a realm of mystification, satisfying
our ``craving for the cloud of unknowing beyond knowledge and for the
silence beyond speech,'' as she puts it. Silence is an essential part of
the creative process, opening a space for contemplation. ``So far as he
is serious, the artist is continually tempted to sever the dialogue he
has with an audience,'' she goes on. To withdraw from the public is
``the artist's ultimate otherworldly gesture: by silence, he frees
himself from servile bondage to the world, which appears as patron,
client, consumer, antagonist, arbiter and distorter of his work.''

The T List \textbar{}

Sign up here

All art legislates between private and public spheres; it has also often
been a way of hiding in plain sight, a place for coded identities, for
the obliqueness of lyricism. As
\href{https://www.nytimes3xbfgragh.onion/topic/person/marcel-proust}{Marcel
Proust} claimed, ``That which enables us to see through the bodies of
poets and lets us look into their souls is not their eyes, nor the
events of their lives, but their books, precisely where their souls,
with an instinctive desire, would like to be immortalized.'' And so it
fits that Sontag --- both an outspoken critic and a novelist --- would
appreciate these tensions: the artist's need for abstraction and
ambiguity, the critic's desire to elucidate. That she wrote this before
the art market exploded, before artists were deified and cast as saviors
of a broken world, before we looked to them not only for beauty,
inspiration and affirmation but also for a form of self-critique, surely
had a lot to do with her own fraught relationship with celebrity.
Sontag, one of the last public intellectuals until her death in 2004,
knew firsthand the cost of attention: what a distraction it could be;
the risk of self-censorship to buff one's own image. (Even posthumously,
her biographer, Benjamin Moser, took her to task in his 2019 book,
``\href{https://www.harpercollins.com/9780062896391/sontag/}{Sontag: Her
Life and Work},'' for not speaking out about her own sexuality during
the AIDS crisis.) Today, we expect artists to perform a public role, to
assent to interviews and magazine profiles in which they explain and
justify their work, to attend openings in enviable clothes, to hold
forth on feminism and racism and social injustice and the latest
catastrophes, political and environmental.

YET THERE HAS always existed a small but powerful shadow world of
creators who have managed to outfox public expectation to varying
degrees, evading de rigueur press and book tours while making their
impact resonantly felt. Some have pulled this off with pseudonyms, among
them Banksy and
\href{https://www.nytimes3xbfgragh.onion/2018/10/31/magazine/elena-ferrante-hbo-my-brilliant-friend.html}{Elena
Ferrante}, who has written copiously on the liberation she found in
detaching her public face from her work. Others have employed alter egos
to convey their message, like
\href{https://www.nytimes3xbfgragh.onion/topic/person/david-bowie}{David
Bowie}, who adopted the persona of Ziggy Stardust, an alien rock star
who comes to Earth with a message of hope only to be destroyed by his
fans and his own excesses. It is one of music's great commentaries on
fame, created at a time when Bowie himself was self-destructing in
celebrity's glare. As a critic whose work hinges on the notion that
there's a great deal of value to be learned from the particular
contexts, personal and otherwise, in which art is made, I find myself
shuttling between two impulses: the desire to get closer to those
difficult truths and to understand the very real costs to exposure. I
want to protect inspiration's riverbank, those Romantic ``thoughts of
more deep seclusion,'' as Wordsworth put it, while also making space for
the kind of powerful storytelling possible in art, stories that, so
often these days, seek to fill a historical void.

Rare, in fact, is the artist who has succeeded in entirely separating
personal identity from work, like Martin Margiela, one of the most
influential designers of all time despite the fact that few fashion
insiders know what he looks like; his name has become a metonym for
avant-garde cool. In fine arts, withdrawal from public life is often
interpreted as an extension of a larger artistic project, as when the
conceptual artist Lee Lozano pulled a Duchamp and retired with ``Dropout
Piece'' around 1970, refusing contact with longtime friends and
collaborators and essentially drawing a frame around her own absence,
writing in her notebook that it was the ``hardest work I have ever
done,'' because it ``involves destruction of (or at least complete
understanding of) powerful emotional habits.'' Cady Noland, still among
the highest-selling living female artists, stopped showing her work
around 2000 and even began to
\href{https://www.nytimes3xbfgragh.onion/2019/03/11/t-magazine/artists-destroy-past-work.html}{disavow
some of it}: In 2011, she renounced a damaged 1990 silk screen,
``Cowboys Milking''; in 2014, it was her 1990 sculpture ``Log Cabin
Facade,'' which had been extensively restored without her consent or
consultation. Like that of the interventionist artist Laurie Parsons,
who left her art career in 1994 to become a social worker, these women's
departures feel not so much like the ``ultimate otherworldly gesture''
but rather a deliberate form of resistance to a patriarchal and
market-oriented art world. But few artists have been as successful at
this kind of recusal as a form of protest as David Hammons, among the
most respected
\href{https://www.nytimes3xbfgragh.onion/2019/07/15/t-magazine/most-important-contemporary-art.html}{contemporary
artists} despite the fact that he rarely submits to interviews (he
likens them to police interrogations) or attends his own openings. This,
too, has largely been viewed as a commentary on the art world's smug ---
and still largely white --- self-regard, but the artist, who is black,
has said that he is simply too private to talk about where his work
comes from, that doing so would feel like a bodily violation.

Hammons is famous enough to let his work largely speak for itself, not
unlike the author Thomas Pynchon, whose
\href{https://www.nytimes3xbfgragh.onion/2017/11/14/t-magazine/artistic-recluse-jd-salinger-thomas-pynchon.html}{reclusiveness}
hasn't diminished his eminence or influence on American letters
(arguably, the opposite is true). And yet withdrawing from public life
entirely is never without risk: I suspect for any Pynchon or Greta Garbo
or Hammons there's someone like
\href{https://www.moma.org/artists/670}{Lee Bontecou}, who was one of
the most exciting names in 1960s art before she left New York in the
early 1970s and faded from view. The first woman represented by the
powerful gallerist Leo Castelli, Bontecou was known for her strikingly
original, imposing wall reliefs made of steel and canvas; often, they
featured the motif of a black hole. At the time, it almost seemed as if
she had disappeared into one of her own works when in fact she'd only
moved out of the city with her husband and daughter (and continued to
teach at Brooklyn College for the next two decades). She never stopped
making art, as a 2003 show at the University of California, Los
Angeles's Hammer Museum revealed: Over the course of three decades of
relative isolation, her work had evolved into more delicate, elaborate
sculptures that evoked celestial bodies, solar systems and star charts.
It's hard to imagine that without this period of seclusion it would have
looked quite the same. Her story is a reminder of just how arbitrary ---
and how irrelevant --- public accolades can be to creation itself. As
the artist once told Ann Philbin, the director of the Hammer, ``I've
never left the art world. I'm in the real art world.''

How much do we need to really know about the artists we admire? I
thought of this recently when reading a Pitchfork profile of Dan Bejar
of the music act Destroyer, who claims that some of his best shows have
resulted when he turns his back to the audience and sings toward his
bandmates. ``As a member of the audience for all the shows I've ever
seen, I just wanted to be flummoxed. That's all I ever ask from art.
Just stagger me, stop me in my tracks. We don't need to go through
something together,''
\href{https://pitchfork.com/features/profile/destroyer-dan-bejar-have-we-met-interview/}{he
said}. But surely we do go through something together --- or at least,
that's the spell cast by the song or novel or film we ``love'': The very
language we use to talk about art is suggestive of romance. It's
difficult for me --- hearing the voice, reading the words --- not to
feel a connection to the person behind any creative work that succeeds
in truly flummoxing me. Like love, this experience of art is rare and
real and wonderful and ultimately unpin-downable; like love, it is
privately felt and personal in origin yet publicly affirmed by our
culture. And so we seek to know more, to maybe even find ourselves in
the artist's story and become part of its mystery. It's worth the risk,
we think, of actually solving it.

Megan O'Grady is a writer at large for T Magazine.

\href{https://www.nytimes3xbfgragh.onion/2017/11/14/t-magazine/artistic-recluse-jd-salinger-thomas-pynchon.html}{}

\hypertarget{is-the-age-of-the-artistic-recluse-overnov-14-2017}{%
\paragraph{Is the Age of the Artistic Recluse Over?Nov. 14,
2017}\label{is-the-age-of-the-artistic-recluse-overnov-14-2017}}

\includegraphics{https://static01.graylady3jvrrxbe.onion/images/2017/11/14/t-magazine/14tmag-reclusive-slide-NWY5/14tmag-reclusive-slide-NWY5-mediumThreeByTwo210-v2.jpg}
\href{https://www.nytimes3xbfgragh.onion/2018/02/09/t-magazine/art/steve-cannon-david-hammons.html}{}

\hypertarget{a-blind-publisher-poet--and-link-to-the-lower-east-sides-cultural-historyfeb-9-2018}{%
\paragraph{A Blind Publisher, Poet --- and Link to the Lower East Side's
Cultural HistoryFeb. 9,
2018}\label{a-blind-publisher-poet--and-link-to-the-lower-east-sides-cultural-historyfeb-9-2018}}

\includegraphics{https://static01.graylady3jvrrxbe.onion/images/2018/02/09/t-magazine/cannon-slide-OOG0/cannon-slide-OOG0-mediumThreeByTwo210.jpg}
\href{https://www.nytimes3xbfgragh.onion/interactive/2017/02/06/t-magazine/jenny-meirens-margiela-interview.html}{}

\hypertarget{the-woman-behind-martin-margielafeb-6-2017}{%
\paragraph{The Woman Behind Martin MargielaFeb. 6,
2017}\label{the-woman-behind-martin-margielafeb-6-2017}}

\includegraphics{https://static01.graylady3jvrrxbe.onion/images/2017/02/06/t-magazine/06tmag-jennypromo/06tmag-jennypromo-mediumThreeByTwo210.jpg}

\hypertarget{we-are-family-1}{%
\subsubsection{We Are Family}\label{we-are-family-1}}

\hypertarget{chapter-1-heirs-and-alumni}{%
\paragraph{Chapter 1: Heirs and
Alumni}\label{chapter-1-heirs-and-alumni}}

\href{/interactive/2020/04/13/t-magazine/black-art-galleries.html}{}

\hypertarget{the-artists}{%
\subparagraph{The Artists}\label{the-artists}}

\href{/interactive/2020/04/13/t-magazine/italian-fashion-design-houses.html}{}

\hypertarget{the-dynasties}{%
\subparagraph{The Dynasties}\label{the-dynasties}}

\href{/interactive/2020/04/13/t-magazine/gordon-parks.html}{}

\hypertarget{the-directors}{%
\subparagraph{The Directors}\label{the-directors}}

\href{/interactive/2020/04/13/t-magazine/enrique-olvera-chef.html}{}

\hypertarget{the-disciples}{%
\subparagraph{The Disciples}\label{the-disciples}}

\href{/interactive/2020/04/13/t-magazine/royal-academy-antwerp.html}{}

\hypertarget{the-graduates}{%
\subparagraph{The Graduates}\label{the-graduates}}

\hypertarget{chapter-2-reunions-and-reconsiderations}{%
\paragraph{Chapter 2: Reunions and
Reconsiderations}\label{chapter-2-reunions-and-reconsiderations}}

\href{/interactive/2020/04/13/t-magazine/ninth-street-greenwich-village-neighbors.html}{}

\hypertarget{the-neighbors}{%
\subparagraph{The Neighbors}\label{the-neighbors}}

\href{/interactive/2020/04/13/t-magazine/omen-restaurant-nyc.html}{}

\hypertarget{the-regulars}{%
\subparagraph{The Regulars}\label{the-regulars}}

\href{/interactive/2020/04/13/t-magazine/hair-musical-broadway.html}{}

\hypertarget{hair-1967}{%
\subparagraph{Hair (1967)}\label{hair-1967}}

\href{/interactive/2020/04/13/t-magazine/sweeney-todd-revival.html}{}

\hypertarget{sweeney-todd-2005-revival}{%
\subparagraph{Sweeney Todd (2005
Revival)}\label{sweeney-todd-2005-revival}}

\href{/interactive/2020/04/13/t-magazine/daughters-of-the-dust.html}{}

\hypertarget{daughters-of-the-dust-1991}{%
\subparagraph{Daughters of the Dust
(1991)}\label{daughters-of-the-dust-1991}}

\hypertarget{chapter-3-legends-pioneers-and-survivors-1}{%
\paragraph{Chapter 3: Legends Pioneers and
Survivors}\label{chapter-3-legends-pioneers-and-survivors-1}}

\href{/interactive/2020/04/13/t-magazine/butch-stud-lesbian.html}{}

\hypertarget{the-renegades}{%
\subparagraph{The Renegades}\label{the-renegades}}

\href{/interactive/2020/04/13/t-magazine/act-up-aids.html}{}

\hypertarget{the-activists}{%
\subparagraph{The Activists}\label{the-activists}}

\href{/interactive/2020/04/13/t-magazine/artist-recluse.html}{}

\hypertarget{the-shadows-1}{%
\subparagraph{The Shadows}\label{the-shadows-1}}

\href{/interactive/2020/04/13/t-magazine/black-actresses-bassett-berry-blige-henson-whitfield-elise.html}{}

\hypertarget{the-veterans}{%
\subparagraph{The Veterans}\label{the-veterans}}

\hypertarget{chapter-4-the-new-guard}{%
\paragraph{Chapter 4: The New Guard}\label{chapter-4-the-new-guard}}

\href{/interactive/2020/04/13/t-magazine/asian-american-fashion-designers.html}{}

\hypertarget{the-designers}{%
\subparagraph{The Designers}\label{the-designers}}

\href{13tmag-beauties.html}{}

\hypertarget{the-beauties}{%
\subparagraph{The Beauties}\label{the-beauties}}

\href{/interactive/2020/04/13/t-magazine/nyc-downtown-nightlife-party-scene.html}{}

\hypertarget{the-scenemakers}{%
\subparagraph{The Scenemakers}\label{the-scenemakers}}

\href{/interactive/2020/04/13/t-magazine/maria-cornejo-olivier-rousteing-telfar-clemens-alessandro-michele.html\#olivier-rousteing-and-co}{}

\hypertarget{olivier-rousteing-and-co}{%
\subparagraph{Olivier Rousteing and
Co.}\label{olivier-rousteing-and-co}}

\href{/interactive/2020/04/13/t-magazine/maria-cornejo-olivier-rousteing-telfar-clemens-alessandro-michele.html\#maria-cornejo-and-co}{}

\hypertarget{maria-cornejo-and-co}{%
\subparagraph{Maria Cornejo and Co.}\label{maria-cornejo-and-co}}

\href{/interactive/2020/04/13/t-magazine/maria-cornejo-olivier-rousteing-telfar-clemens-alessandro-michele.html\#telfar-clemens-and-co}{}

\hypertarget{telfar-clemens-and-co}{%
\subparagraph{Telfar Clemens and Co.}\label{telfar-clemens-and-co}}

\href{/interactive/2020/04/13/t-magazine/maria-cornejo-olivier-rousteing-telfar-clemens-alessandro-michele.html\#alessandro-michele-and-co}{}

\hypertarget{alessandro-michele-and-co}{%
\subparagraph{Alessandro Michele and
Co.}\label{alessandro-michele-and-co}}

\href{/interactive/2020/04/13/t-magazine/foreign-correspondents.html}{}

\hypertarget{the-journalists}{%
\subparagraph{The Journalists}\label{the-journalists}}

\begin{itemize}
\item
\item
\item
\item
\end{itemize}

Advertisement

\protect\hyperlink{after-bottom}{Continue reading the main story}

\hypertarget{site-index}{%
\subsection{Site Index}\label{site-index}}

\hypertarget{site-information-navigation}{%
\subsection{Site Information
Navigation}\label{site-information-navigation}}

\begin{itemize}
\tightlist
\item
  \href{https://help.nytimes3xbfgragh.onion/hc/en-us/articles/115014792127-Copyright-notice}{©~2020~The
  New York Times Company}
\end{itemize}

\begin{itemize}
\tightlist
\item
  \href{https://www.nytco.com/}{NYTCo}
\item
  \href{https://help.nytimes3xbfgragh.onion/hc/en-us/articles/115015385887-Contact-Us}{Contact
  Us}
\item
  \href{https://www.nytco.com/careers/}{Work with us}
\item
  \href{https://nytmediakit.com/}{Advertise}
\item
  \href{http://www.tbrandstudio.com/}{T Brand Studio}
\item
  \href{https://www.nytimes3xbfgragh.onion/privacy/cookie-policy\#how-do-i-manage-trackers}{Your
  Ad Choices}
\item
  \href{https://www.nytimes3xbfgragh.onion/privacy}{Privacy}
\item
  \href{https://help.nytimes3xbfgragh.onion/hc/en-us/articles/115014893428-Terms-of-service}{Terms
  of Service}
\item
  \href{https://help.nytimes3xbfgragh.onion/hc/en-us/articles/115014893968-Terms-of-sale}{Terms
  of Sale}
\item
  \href{https://spiderbites.nytimes3xbfgragh.onion}{Site Map}
\item
  \href{https://help.nytimes3xbfgragh.onion/hc/en-us}{Help}
\item
  \href{https://www.nytimes3xbfgragh.onion/subscription?campaignId=37WXW}{Subscriptions}
\end{itemize}
