Sections

SEARCH

\protect\hyperlink{site-content}{Skip to
content}\protect\hyperlink{site-index}{Skip to site index}

\hypertarget{comments}{%
\subsection{\texorpdfstring{\protect\hyperlink{commentsContainer}{Comments}}{Comments}}\label{comments}}

\href{}{Can a Vaccine for Covid-19 Be Developed in Record
Time?}\href{}{Skip to Comments}

The comments section is closed. To submit a letter to the editor for
publication, write to
\href{mailto:letters@NYTimes.com}{\nolinkurl{letters@NYTimes.com}}.

\hypertarget{can-a-vaccine-for-covid-19-be-developed-in-record-time}{%
\section{Can a Vaccine for Covid-19 Be Developed in Record
Time?}\label{can-a-vaccine-for-covid-19-be-developed-in-record-time}}

June 9, 2020

\begin{itemize}
\item
\item
\item
\item
\item
  \emph{147}
\end{itemize}

A discussion moderated by Siddhartha Mukherjee.

\hypertarget{can-a-vaccine-for-covid-19-be-developed-in-record-time-1}{%
\section{Can a Vaccine for Covid-19 Be Developed in Record
Time?}\label{can-a-vaccine-for-covid-19-be-developed-in-record-time-1}}

A discussion moderated by Siddhartha Mukherjee.

June 9, 2020

SHARE

In the history of medicine, rarely has a vaccine been developed in less
than five years. Among the fastest to be
\href{https://www.nytimes3xbfgragh.onion/2013/05/07/health/maurice-hilleman-mmr-vaccines-forgotten-hero.html}{developed
was the current mumps vaccine}, which was isolated from the throat
washings of a child named Jeryl Lynn in 1963. Over the next months, the
virus was systematically ``weakened'' in the lab by her father, a
biomedical scientist named Maurice Hilleman. Such a weakened or
attenuated virus stimulates an immune response but does not cause the
disease; the immune response protects against future infections with the
actual virus. Human trials were carried out over the next two years, and
the vaccine was licensed by Merck in December 1967.

Antiviral drugs, too, have generally taken decades to develop; effective
combinations of them take even longer. The first cases of AIDS were
described in the early 1980s; it took more than a decade to develop and
validate the highly effective triple drug cocktails that are now the
mainstay of therapy. We are still continuing to develop new classes of
medicines against H.I.V., and notably, there is no vaccine for that
disease. And yet the oft-cited target for creating a vaccine against
SARS-CoV-2, the virus that causes Covid-19, is 12 months, 18 at the
outside.

Pulling that off is arguably the most important scientific undertaking
in generations. The Times assembled (virtually, of course) a round table
to help us understand the maddening complexity of the challenge and the
extraordinary collaboration it has already inspired. The group included
a virologist; a vaccine scientist; an immunologist and oncologist; a
biotech scientist and inventor; and a former head of the Food and Drug
Administration.

\hypertarget{the-panelists}{%
\paragraph{The Panelists}\label{the-panelists}}

\textbf{Siddhartha Mukherjee} is an associate professor of medicine at
Columbia University and a cancer physician and researcher. He is the
author of ``The Emperor of All Maladies: A Biography of Cancer,'' which
was the winner of the 2011 Pulitzer Prize in general nonfiction, and
``The Gene: An Intimate History.'' He was recently appointed to Gov.
Andrew Cuomo's blue-ribbon commission to reimagine New York.

\textbf{Dan Barouch} is the director of the Center for Virology and
Vaccine Research at Beth Israel Deaconess Medical Center in Boston and a
professor of medicine at Harvard Medical School.

\textbf{Margaret (Peggy) Hamburg} is the foreign secretary of the
National Academy of Medicine. She was commissioner of the Food and Drug
Administration from 2009 to 2015.

\textbf{Susan R. Weiss} is a professor and vice-chairwoman of the
Department of Microbiology at the University of Pennsylvania and a
co-director of the Penn Center for Research on Coronaviruses and Other
Emerging Pathogens.

\textbf{George Yancopoulos} is co-founder, president and chief
scientific officer of Regeneron.

\hypertarget{what-it-takes-to-find-a-vaccine}{%
\paragraph{What It Takes to Find a
Vaccine}\label{what-it-takes-to-find-a-vaccine}}

\textbf{George Yancopoulos}: Most people don't realize that successfully
inventing and developing any new drug or vaccine is quantifiably among
the hardest things that human beings try to do. This is reflected in the
numbers. Although there are thousands of major medical institutions and
thousands of biotech and biopharma companies that collectively involve
millions of researchers and hundreds of billions of dollars invested per
year --- and all are working on new vaccines and medicines --- the vast
majority of efforts fail, with the F.D.A. only approving 20 to 50 new
medicines a year. And each of the rare success stories usually occurs
over many years, often a decade or two.

\textbf{Peggy Hamburg}: So with Covid-19, we are moving at record speed,
in terms of the history of vaccine development.

\textbf{Siddhartha Mukherjee}: Can you put a number down for how quickly
we can get an effective vaccine developed? Is the 12-to-18-month time
frame we've been hearing realistic?

\textbf{Dan Barouch}: The hope is that it will be within a year, but
that is not in any way guaranteed. That projection will be refined as
time goes on --- and a year assumes that everything goes smoothly from
this point forward. That's never been done before. And safety cannot be
compromised.

\textbf{Hamburg}: Realistically, the 12 to 18 months that most people
have been saying would be a pretty good marker but still optimistic.

\textbf{Susan R. Weiss}: I'd agree.

\textbf{Mukherjee}: To think about whether there's any way to make this
process go faster, let's start by talking about how the search for a
vaccine usually happens. Dan, what is the general principle of what a
vaccine is and how it works?

\textbf{Barouch}: The goal of a vaccine is to raise an immune response
against a virus or a bacterium. Later, when a vaccinated person is
exposed to the actual virus or bacterium, the immune system will then
block or rapidly control the pathogen so that the person doesn't get
sick. The immune cells that make antibodies are called B cells. Once
they've been triggered by a vaccine to raise an immune response, some of
these B cells can last for years and are always standing ready to make
antibodies against the pathogen when it is encountered, thereby
protecting against the disease for a prolonged period of time.

\textbf{Hamburg}: Under normal conditions, drug-and-vaccine development
begins with ``preclinical'' work --- basic science --- to identify the
nature of the disease in question.

\textbf{Weiss}: In virology labs like mine, we try to identify the viral
proteins that a vaccine might target, usually the protein that
recognizes and attaches to the host-cell receptor. All coronaviruses
have a so-called spike protein, which is what gives the virus its
corona-like morphology, the ``crownlike shape,'' as can be visualized in
an electron microscope. To invade a cell, the spike protein attaches to
a receptor --- another protein, usually --- on the cell's outer
membrane. This eventually results in the genetic material of the virus,
in this case, an RNA protein complex, being internalized in the cell.
And once that happens, replication can begin and a person can get sick.
If you can identify the viral protein that interacts with the cellular
receptor, then you can try to create a vaccine. This spike protein
represents a particularly attractive candidate for a vaccine, because it
is a protein that most prominently sticks outside of the surface of the
virus, and so it's the part of the virus that is most visible to the
immune system.

\textbf{Mukherjee}: So what are the different approaches that you can
take to finding vaccines?

\textbf{Barouch}: A tried-and-true vaccine approach is a whole
inactivated virus vaccine --- that's when you grow up the actual virus
in the laboratory, for example in cells or in eggs, and then
``inactivate'' it with chemicals or another method to make it unable to
infect cells but still able to elicit an immune response. A company in
China, Sinovac Biotech, currently has an inactivated SARS-CoV-2 vaccine
in clinical trials. The pros are that there's a long clinical history of
multiple vaccines that have been successful in that regard, such as the
inactivated polio vaccine and the inactivated flu vaccine. The cons are
that there are always some safety considerations around proving that the
virus has been fully inactivated. If the virus is not fully inactivated,
the danger is that it might actually cause the disease.

Because of those issues, many groups are working on approaches that are
called gene-based vaccines. Gene-based vaccines, such as DNA vaccines
and RNA vaccines, do not consist of the entire virus particle. Rather,
these vaccines use just a small fraction --- sometimes even just one
gene --- from the virus.

That still leaves the question of how to get that gene into human cells.
A vector-based vaccine uses a delivery vehicle --- one example would be
a recombinant adenovirus, a ``harmless virus'' carrier, like a
common-cold virus --- to deliver the protein into a person's cells. For
instance, you can take the spike-protein DNA from SARS-CoV-2 and
``stitch'' it into the DNA of the harmless cold virus using
genetic-engineering techniques. The virus delivers the spike-protein DNA
into cells, but it cannot replicate in cells, so it's a safe delivery
system. Still other approaches use purified proteins, such as the spike
protein itself, as vaccines.

\textbf{Yancopoulos}: This gene-based approach was used in the case of
Ebola. Scientists figured out that the protein the virus used to invade
human cells is one that's called the GP protein. They were able to make
a very successful vaccine. They actually stitched in the GP protein,
using genetic-engineering techniques, into a benign virus. When this
virus infected cells, it made the GP protein, and the body recognized
that protein as ``foreign'' and made antibodies against it.

\textbf{Mukherjee}: There is also the idea of ditching the ``harmless''
virus altogether and just using a snippet of a viral gene, by itself, as
an inoculum. As soon as the sequence of the SARS-CoV-2 genome became
available in January, it became possible to design such a viral-gene
snippet. Through reasons that we still don't fully understand, the
body's own cells take up that viral gene, even without a carrier, and
produce viral protein from it. Why and how cells take up this so-called
naked DNA or RNA is still being worked out, but the viral protein is
recognized as ``foreign'' because it has never been seen by the body,
and the host raises antibodies against it. And because it is not carried
by any virus, this kind of vaccine can be easier to manufacture
initially, although scaling up the manufacture may be tough. This is the
approach that Moderna is taking to try and make a vaccine. In Moderna's
case, it involves using RNA as the inoculum.

But what is the track record so far of a real human vaccine for a real
human disease using these genetic techniques?

\textbf{Barouch}: Currently there are no approved DNA vaccines or RNA
vaccines. Some of them have been tested in small, early-phase clinical
trials, for safety and ability to induce an immune response. However,
they have not previously been tested in large-scale efficacy trials or
mass produced or approved for clinical use.

\textbf{Mukherjee}: Yes, so we have to be very careful with these
vaccines. The data discussed by Moderna in May would suggest that their
vaccine can elicit antibodies in humans. It did so in eight patients.
But whether that is protective against SARS-CoV-2, and how long the
protection lasts, is an open question. More so, because elderly people
need particular protection, and we need to understand how much of this
vaccine, or ones like it, are eliciting long-term immunity in the
elderly, where the immune system might be already somewhat attenuated in
its response.

Once you have a vaccine that you want to test, then you begin animal
studies. What animals are coronavirus vaccines and drugs being tested
on? And how do scientists know which ones to use?

\textbf{Weiss}: An ideal animal model is one that reproduces the human
disease as closely as possible --- in which, for example, clinical signs
resembling symptoms in humans are observed, virus replication is
observed in similar organs, immune response mirrors that in humans and
so on. In addition, an animal model is often used to demonstrate whether
a virus can be transmitted from an infected to an uninfected animal.
Scientists use animal models to understand how the virus causes disease.
They are also useful to determine whether a vaccine will be successful
to prevent infection or a drug will be able to reduce or eliminate the
disease. With Covid-19, there's currently a hamster model that looks
like it works pretty well to mimic the disease and also some promising
research with mice, ferrets and also nonhuman primates. None of these
models are perfect, but each one of them tells us something about the
immune response and may be useful for testing of vaccine efficacy as
well as for antiviral therapies.

\textbf{Mukherjee}: Dan, you have been working with the monkey model and
a genetically engineered vaccine. Can you tell us more about it?

\textbf{Barouch}: We're collaborating with Johnson \& Johnson in the
development of a SARS-CoV-2 vaccine. This vaccine involves a recombinant
adenovirus vector --- a common-cold virus that's been altered to make it
harmless --- to shuttle the spike protein into cells. That is an
efficient way of inducing potent immune responses to a pathogen and
elicits durable immune responses as well.

There are two key scientific questions related to vaccine development
that are crucial. First, is there evidence that natural immunity induced
by infection protects against a subsequent encounter with the virus? And
then, of course, there is the question of which vaccines to test and how
effective any of these will be. We have just published some preliminary
answers to these questions in the animal model in monkeys.

In the first experiment, we infected nine monkeys with SARS-CoV-2 in the
nose and in the lungs. All the animals developed a viral pneumonia,
which was similar to human disease, except the monkey disease was mild
and all animals recovered. Thirty-five days later, we re-exposed the
animals a second time to the virus --- and found that all the animals
were protected. There was no virus or very low levels of the virus in
the lungs. This is an important question because, historically, it has
proved much easier to develop vaccines when there is natural protective
immunity against the virus. For H.I.V., for instance, there is no
natural protective immunity, and that's part of the reason that H.I.V.
vaccines have been so hard to develop.

In the second study, we developed a series of prototype vaccines. These
were ``naked DNA'' vaccines --- ones without any delivery vehicles.
These are not the vaccines that we are planning to take into the clinic,
but rather these prototype DNA vaccines can teach us a lot about the
immune responses needed to protect against this virus. We used six
different versions of the spike-protein gene --- some encoding the full
protein, some with smaller pieces. Twenty-five monkeys received these
vaccines and 10 received just saline as a control. Each monkey received
two vaccine shots. Six weeks after the first shot, they were exposed to
the virus, and we found that vaccinated animals were protected from
SARS-CoV-2. Eight of the 25 animals had no virus detected after
exposure, and the remaining animals showed low levels of the virus. Most
important, the level of antibodies induced by the vaccine correlated
with the level of protection, and this biomarker may therefore be useful
for monitoring vaccine studies moving forward. The full-length spike
protein appeared to work the best.

The implications of these two studies are that both natural immunity and
vaccine-induced immunity can exist in primates and that the amount of
antibody may serve as a useful marker for vaccine effectiveness. But of
course, these are animal studies, and we will need to study these
questions in humans.

\textbf{Mukherjee}: So what happens next in terms of making a real
vaccine out of these data?

\textbf{Barouch}: DNA encoding the full-length spike protein has been
stitched into the common-cold virus vector as a more efficient way of
transporting the spike-protein DNA into cells. That's the basis of the
vaccine we are developing with J.\& J.

\textbf{Mukherjee}: In addition to the J.\& J. and Moderna vaccines,
there are several other programs. One is the Oxford program. What do we
know about that one?

\textbf{Barouch}: The Oxford vaccine is based on a chimpanzee
common-cold virus, and it also encodes the spike protein. Early data
shows that in monkeys, the vaccine was able to reduce the amount of
virus in the lung but not in the nose, following exposure to SARS-CoV-2.
This vaccine has already started early human testing, but we are still
waiting for definitive data. Several vaccines are also in clinical
trials in China, including Sinovac's inactivated-virus vaccine and
another vaccine based on a human common-cold virus.

\hypertarget{getting-from-a-discovery-to-a-cure}{%
\paragraph{Getting From a Discovery to a
Cure}\label{getting-from-a-discovery-to-a-cure}}

\textbf{Mukherjee}: Once you know the vaccine you want to test, what
happens next?

\textbf{Barouch}: Vaccine development for a new pathogen traditionally
takes many years or even decades. The process includes small-scale
manufacturing; Phase 1, Phase 2 and Phase 3 clinical trials; and then
regulatory approval and large-scale manufacturing. For SARS-CoV-2, the
goal is to compress these timelines considerably without compromising
safety, which is absolutely critical for any vaccine that will be given
to large numbers of individuals.

\textbf{Hamburg}: And there are a lot of hurdles that come up along the
way. Sometimes developers have a good idea but can't translate it into a
viable vaccine. Or you wind up with unexpected side effects in early
human studies. Worse, you can find safety or efficacy problems further
along, or you find that you can't reliably scale up manufacturing or you
run into problems with regulatory authorities.

\textbf{Mukherjee}: Walk us through what happens with human testing.

\textbf{Yancopoulos}: The first human trials are called Phase 1 trials
and consist of small ``safety trials'' exploring increasing doses of the
drug, to prove you can get into an effective level of the drug without
obvious harm. This usually takes from a few months to a year or two. If
you are satisfied that the Phase 1 trial has established that an
effective level of the drug does not cause obvious harm (albeit in the
small number of patients in Phase 1), you then proceed in the next stage
of trials, Phase 2, that further establish safety in larger patient
numbers while also demonstrating that your drug has some beneficial
effect.

For example, Phase 2 might show that a drug lowers ``bad cholesterol.''
But it doesn't necessarily mean that it prevents heart attacks. Only
when you do very large, well-controlled Phase 3 studies can you prove
that lowering the ``bad cholesterol'' also can prevent heart attacks and
save lives. For example, for the heart-disease drug Praluent that my
company developed, we could show that it effectively lowered ``bad
cholesterol'' in Phase 2 trials of a couple hundred patients, but the
Phase 3 trial that was required to show it prevented heart attacks and
improved survival involved about 20,000 patients and about five years.

\textbf{Mukherjee}: In what you just described, the Phase 1 alone can be
up to two years. How could we accelerate the whole process?

\textbf{Hamburg}: Well, we can't abandon the rigor of the science. And
we certainly can't abandon the ethics of how we do studies either. But
what we can do is, frankly, ask developers to take more risks
themselves. Vaccine development can be costly and success uncertain. As
compared to a drug that someone may take every day, the return on
investment versus risk of failed development is pretty high for
vaccines. Because vaccines are often viewed as a public good, protecting
both people and communities, there can be considerable pressure on
companies to restrict price on vaccines, so a company rarely has a
``blockbuster'' vaccine the way that a cancer treatment, ulcer drug or
cholesterol-lowering drug can be. Also, there are liability issues
because you are giving a vaccine to a healthy person to protect them
from disease rather than treating an existing problem. So the trade-off
of development risk and benefit often does not favor vaccines. In order
to manage those risks, trials of different vaccine candidates tend to be
done one step at a time.

\textbf{Barouch}: For Covid-19, developers are talking about performing
as many steps in parallel as possible, as opposed to sequentially. For
example, multiple vaccine manufacturers are willing to take enormous
financial risks --- planning for large-scale manufacturing up front,
even before knowing whether the vaccine works or not.

There might be a half dozen vaccines that will get to the Phase 3 stage.
How do we select which ones go forward? Do we prioritize vaccines that
are similar to ones that have been tested previously in humans and that
have shown safety and potency? Or do we turn to vaccines that can be
mass produced quickly and safely? Ultimately, the prioritization is a
complicated process that involves many decisions. The F.D.A. has to be
involved, as well as governments and regulatory agencies and
stakeholders around the world. There are questions about safety,
efficacy, manufacturability and scalability that must be tackled.

\textbf{Hamburg}: Obviously we're looking for ones that work in the
early trials. But we don't just need a vaccine that works; we need one
that can be reliably scaled up to manufacture in very large volumes.
Ideally, it would be one that doesn't require multiple doses to be
effective, certainly not beyond, say, a two-dose regimen. And ideally it
wouldn't require refrigerated storage, so it can be made more available
in resource-poor settings. So, there are characteristics of a vaccine in
addition to safety and efficacy that are going to matter.

\textbf{Mukherjee}: Are there other ways to speed up the process?
Typically in Phase 3 trials, you'd go out into the field and give, say,
15,000 people the vaccine you're testing and 15,000 people a placebo.
And then you wait and see how many of those with the vaccine came down
with the disease versus the numbers who were given the placebo. But of
course, that takes a long time. You'd need to wait months or years to
watch this natural experiment. One ethically fraught possibility that
some experts have floated is the use of so-called ``challenge'' trials,
in which young, healthy people are given a vaccine and then deliberately
exposed to the virus. This would only happen once the safety of the
vaccine was established and there was some hint that there is an immune
response. But what are the ethical concerns about accelerating a vaccine
process?

\textbf{Barouch}: For certain pathogens, it has been considered ethical
to perform human-challenge studies but typically only for pathogens for
which there is a highly effective treatment. For example, for malaria
there is a very widely and effectively used human-challenge model in
which vaccines or other interventions can be tested. Human volunteers
can be inoculated with malaria and then, if they develop the disease,
can be treated rapidly so that they don't actually get sick.

\textbf{Mukherjee}: Why can't we do this for Covid-19?

\textbf{Barouch}: The dilemma for Covid-19 is that there currently is no
curative therapy. So, if a volunteer in a potential human-challenge
study gets severely ill, there may not be a way to cure that person. In
fact, far from it: The drugs in our armamentarium are not perfect, and
so we would have no guarantee that we could ``rescue'' a person who got
severely ill.

\textbf{Weiss}: I'm not an ethicist, but my gut feeling is that
challenge trials are too dangerous. Young people can get quite sick and
die from Covid-19.

\textbf{Mukherjee}: If we could develop a drug or an antibody that would
be able to mitigate the disease, we would still need to think about the
ethical concerns of a human challenge. There's also the question of who
``volunteers'' for such a challenge. There's been a whole history ---
extremely fraught --- where minorities were used as experimental
subjects without their understanding or consent. How do we ensure that
the volunteers understand the consent? How do we ensure that they are
not given perverse incentives? A young person might believe that if they
get vaccinated and challenged, then they have an ``immunity passport''
against the disease. But what if, in fact, they fall sick and we have no
effective therapy. The question of a challenge experiment therefore
requires both deep ethical thinking --- who, what, how many --- and
scientific thinking: Is there a strategy to ``rescue'' a patient if the
challenge resulted in a real disease.

\textbf{Hamburg}: There's also the question of how much a challenge
trial on young healthy patients will even tell us what we need to know.
If we do a study using these lowest-risk patients, will that give us
adequate information about the value of the vaccine in the elderly
populations, who in many ways are the most important target for this
vaccine?

\textbf{Barouch}: Absolutely. The problem for human-challenge studies,
beyond the ethical questions, is that a controlled human-challenge
experiment doesn't necessarily translate to showing how a vaccine would
actually perform in the real world. Participants in any such challenge
studies would likely be young, healthy individuals at lowest risk, and
so the data generated may not be applicable to elderly and vulnerable
populations that need to be protected with a vaccine. There may also be
different doses and different viral variants.

\textbf{Mukherjee}: Peggy, is there something we should be doing while
we're waiting for all these trials?

\textbf{Hamburg}: Well, I think we definitely need to be thinking about
the scale-up and manufacturing issues, as we said already. Another issue
that we need to be thinking about is working with the communities where
these large-scale Phase 3 studies will be done. Some will be done in the
U.S., but others will be done in other places around the world ---
lower-resourced places that may not have the kind of clinical-research
infrastructure that we have here, whether it's having enough trained
researchers or the sophisticated health care services they need, like
lab and diagnostic tools and basic things like refrigeration and cold
storage.

\textbf{Mukherjee}: So in speeding up the vaccine-development process,
we have three things going for us: We have cleaner and likely safer
technologies to create vaccines; we know the viral proteins that are
likely to raise a good immune response; and we know how to measure that
immune response with much greater accuracy in humans that have been
given a test dose of the vaccine. All of these we hope will accelerate
the Phase 1 safety trials --- some of which have already started,
between March and May --- so that they can be done in four to six
months. After that, we're still looking at a roughly 12-month period to
test the vaccine in real human populations, so it seems we're pushing
toward the 18-month marker. Dan, what's your sense of the time that it
will take to actually deploy the vaccine across the population of the
world once we have it?

\textbf{Barouch}: There are two timelines that matter. One is the
infrastructure and timeline needed to manufacture massive numbers of
doses of the vaccine, and a separate, potentially different timeline to
actually deploy the vaccine.

\textbf{Hamburg}: On the manufacturing front, you've probably heard
about Bill Gates's decision to begin to invest in a range of different
types of manufacturing capabilities, to capture the different categories
of vaccines, not knowing which of the different types of vaccine
candidates are actually going to make it over the finish line.

\textbf{Mukherjee}: Right, so that once the ``winner'' is identified,
that winner can go forward without having to wait for capacity.

\textbf{Barouch}: The reason to build out this capacity in advance is
that different vaccines are made very differently. For example, the
manufacturing process for an RNA vaccine is entirely different than for
an adenovirus vector-based vaccine. For rapid deployment of a vaccine
after clinical efficacy is shown, large-scale manufacturing of multiple
vaccine candidates has to begin before there is demonstration of vaccine
efficacy.

\textbf{Hamburg}: But even if this initiative moves forward, I think
there may be a misunderstanding in the public at large about the
challenges of scale-up and manufacturing. Once a vaccine is approved, it
is not going be available the next day for whoever wants it.

\textbf{Mukherjee}: Tell us about that. That's important.

\textbf{Hamburg}: Manufacturing has to be done in a high-quality and
consistent way. There are materials that are needed that can be in
limited supply, like the vials and the stoppers that you need for
packaging. And then there are chains for distribution, and sometimes
vaccines have to be kept frozen at very low temperatures. So you have to
have all of those important systems for manufacturing, packaging and
delivery and distribution up and running and the supply chains flowing
in order to actually get what might now be an approved vaccine actually
into the bodies of the individuals who need it.

\textbf{Mukherjee}: And then this returns to the distribution and
inoculation of the vaccine, and epidemiological studies that follow it.

\textbf{Hamburg}: Yes, I think we need to create systems for assuring
fair and equitable and public-health-driven distribution of the vaccine
as well. One concern that many have is that there's going to be a huge
nationalistic push for countries to try to get hold of as much vaccine
as they can for use within their own borders, yet ultimately the safety
of any country or community depends on addressing and protecting against
this virus all over the world.

\hypertarget{what-will-we-have-in-the-meantime}{%
\paragraph{What Will We Have in the
Meantime?}\label{what-will-we-have-in-the-meantime}}

\textbf{Yancopoulos}: As we've been saying, with all the challenges
regarding developing, testing, manufacturing and distributing a safe and
effective vaccine --- no matter how much effort so many scientists and
companies put on the problem --- it could still take years or even
longer. This is why it's so important to have additional efforts ongoing
in parallel to try to fight back against this pandemic. If we don't have
a safe and effective vaccine for one to two years, or even longer, we
need to develop other treatments as a bridge to a vaccine --- to allow
society to have a path toward reopening and functioning, while we await
a vaccine.

\textbf{Mukherjee}: So let's work backward as we work frantically toward
the vaccine. What can we do now that will help? How can we move from
where we are --- isolate, quarantine, mask, distance --- toward a
therapy that will bridge us to the vaccine?

\textbf{Yancopoulos}: The world has gotten interested in the drug
remdesivir, which inhibits the process of RNA replication and has been
shown to be active in a lot of viruses that use these mechanisms to
replicate themselves.

\textbf{Mukherjee}: An initial study from China, published in the
Lancet, showed a rather disappointing effect from remdesivir. There was
a hint of clinical improvement in treated patients versus controls, but
it was not statistically significant. But there were several problems
with that trial. That trial enrolled patients who had the onset of
symptoms from one day to 12 days, so the spectrum of disease severity
was very broad. And although the study had a placebo control, it did not
have a lot of patients: 236 in total, 158 in the treatment and 78 in
placebo. In late May there was a study from the National Institutes of
Health published in The New England Journal of Medicine that showed
remdesivir might have an effect --- albeit, again, a modest one --- on
hospitalized patients. The number of days that patients spent in
hospital was reduced, and the study again hinted that there was a
reduction in mortality from 11.9 percent in the placebo group to 7.1
percent in the treated group, though this was not shown to be
statistically significant. But again, this was a study that involved a
very broad range of patients --- some with moderate lung damage and some
on ventilators.

\textbf{Hamburg}: Until there's a vaccine, I don't think there's going
to be one magic bullet for treating this thing, and we're certainly not
going to find that magic-bullet drug treatment in a repurposed drug
pulled off the shelf.

\textbf{Yancopoulos}: History has told us that. Repurposed drugs are
usually not panaceas.

\textbf{Hamburg}: And meanwhile, every day we are learning more about
this virus, its life cycle and the complexity of how it causes disease.
We initially thought of Covid-19 as a lung disease, and then realized
that many of the people who became seriously ill had their disease
course worsened by a hyperactive immune response. Now we realize that
many other vital organs can be seriously compromised, including the
kidneys, the gut and the brain, and that something about this virus is
triggering a very dangerous hypercoagulability syndrome, where the blood
starts clotting in dangerous ways. And there's an apparent association
of this novel coronavirus with a very serious hyperimmune syndrome in
children, the so-called Kawasaki-like syndrome.

I think we need to draw on our best scientific understanding and the
work of virologists like Susan to identify where are the targets for
intervention, for what will likely be a combination therapy that
addresses different points in the life cycle of the virus and the human
immune response.

\textbf{Mukherjee}: What about using antibodies to tide us through this
period? As we've discussed, the entire purpose of vaccines is to induce
the body to make its own ``protective antibodies'' that bind and kill
the virus. George, you have pioneered ways of making these same type of
``anti-viral antibodies'' outside the body, manufacturing them and
purifying them and then giving them back to individuals --- so these
people now have antibodies against the virus. It's like they have
already been vaccinated: They now have antibodies but without needing to
go through the actual vaccine step.

\textbf{Yancopoulos}: Right. Earlier I mentioned Ebola. Over the past 10
to 20 years, we have developed a series of technologies that are
designed to make antibodies against many disease targets, including
viruses. These technologies were used by our scientists to develop a
cocktail of three antibodies to bind and block the GP protein of Ebola
--- the GP protein is the Ebola equivalent of the spike protein --- our
so-called REGN-EB3 cocktail, and this treatment was very effective in
patients already infected with Ebola, as shown by the World Health
Organization in a clinical trial. And now we have used these same
technologies to rapidly make an antibody cocktail --- REGN-COV2 --- that
binds and blocks the spike protein of Covid-19.

\textbf{Mukherjee}: Explain what that means.

\textbf{Yancopoulos}: You can almost think of it as a temporary vaccine.
Instead of waiting for a vaccine that will make the body make its own
antibodies against the virus, we can make exactly those kinds of
antibodies and inject them into people.

\textbf{Mukherjee}: And how long would that take?

\textbf{Yancopoulos}: For Ebola, we went from starting the project to
being in clinical trials in just nine months. With Covid-19, we've cut
that in almost half: We have already made thousands and thousands of
these antibodies and started to grow them up and tested them for
blocking the virus. And we plan to start human trials in June. We will
conduct three types of trials. First, prophylaxis, in which we give
REGN-COV2 to patients not yet infected but at high-risk and hopefully
show we can prevent infection --- much like a vaccine would hope to do
but not inducing the ``permanent immunity'' that a vaccine can confer.
Then, we will give REGN-COV2 to patients recently infected, who are
asymptomatic and/or who don't have severe disease, and see if we can
rapidly ``cure'' them and eliminate the virus and prevent them from
progressing to the severe-and-critical stage that would require
hospitalization and ventilation. Then, finally, we would give the
REGN-COV2 to severe-and-critical-stage patients, who are in the
hospital, many on ventilators, with poor prognosis, and hopefully show
we can rescue them, get them off ventilators and save their lives.

\textbf{Mukherjee}: And how long would the antibodies last in terms of
protection?

\textbf{Yancopoulos}: We hope each injection should last at least a
month, if not several months. Beyond this antibody cocktail, there are
quite a few drugs that are being repurposed to see if they have
potential in Covid-19. One particularly promising story that came out of
China was that blocking the inflammation that seems to be causing the
lung problems in Covid-19 --- in particular by blocking an inflammatory
factor called interleukin-6, or IL-6, that is an important driver of the
inflammation in rheumatoid arthritis --- might help patients with lung
problems due to Covid-19. This promise was based on a small,
uncontrolled but positive experience in China. We are now doing large
Phase 3 trials to definitively test whether our IL-6 blocking drug ---
which as I said is already approved for treating the inflammation in
rheumatoid arthritis --- may help with the lung inflammation in Covid-19
patients who are critically ill. But if you look at either repurposed
drugs like remdesivir or even the IL-6 blocking approach, those are not
the sort of drugs that I think would make Dan happy, because they are
just incremental.

\textbf{Hamburg}: Or me.

\textbf{Yancopoulos}: Or anybody. But they could provide a benefit.
Every life that's saved or every disease course that's shortened is
important.

\textbf{Mukherjee}: But the incremental effects that you are describing
may be because the trials are still being run on patients with moderate
to severe disease and in particular on hospitalized patients. In
virtually every infectious disease, the use of antibacterials or
antivirals or even antibodies against a virus early in the course of
disease is better. In terms of remdesivir, it's possible that the drug
is much more likely to be efficacious when used early than late, and in
fact, the published trial from the N.I.H. has a hint of that. As we just
said, thus far, the trials have generally involved a broad spectrum of
patients --- hospitalized patients and some of the sickest --- and the
benefits have been modest. But experts such as Francisco Marty, an
infectious-disease doctor at Brigham and Women's Hospital, have argued
that this was precisely the wrong population to use the drug. By the
time you have lung inflammation and tissue injury, killing the virus is
not enough. It's too late; the body has turned on itself, and an
antiviral drug cannot tackle the inflammation. And so a second fleet of
trials is being designed to evaluate whether the drug might be more
successful if given as early as possible --- for instance, as soon as
you have detected the virus and the oxygen level has begun to drop,
particularly in high-risk patients. It would invert the paradigm: Rather
than quarantine and sit at home, you would get the drug sooner rather
than later. Infectious-disease doctors, such as Marty, have early
evidence that this strategy works: Patients given early remdesivir
recover and do not progress to the fulminant lung disease. A group of
us, including Marty, are in conversations with the Gates Foundation and
others to launch trials of such a strategy.

And then there's the question of combining drugs: Perhaps an antiviral
drug would work even better if used in combination with antibodies. But
all of this is logistically complex. All these drugs have to be given
intravenously, so you have to go to an infusion center to get them. But
I cannot emphasize enough the urgency of doing these early-treatment
trials. It would be really a shame to give up a valuable drug in our
very limited armamentarium because we couldn't study the right patients.
And the drug itself is in short supply, so every time it's used on a
patient that it would not benefit, we are losing ground. We need
political support, financial support and support from Gilead, the
manufacturer, to get this urgent early treatment trial done as soon as
possible.

Susan, what is your sense of combining two targets? Maybe a replication
inhibitor along with an antibody. And I know you've been involved with
other laboratories, testing new drugs. What is your sense of the
development of these drugs?

\textbf{Weiss}: That might turn out to be useful, but right now drugs
are still being evaluated. Quite a few people are testing all kinds of
F.D.A.-approved drugs and unapproved drugs against the virus. There are
a lot of potential drug targets in the viral-replication cycle --- that
is, the enzymes that are needed to replicate the virus as well as the
cellular factors needed for the virus to enter the cell. And I guess I
feel like a lot of compounds may inhibit replication of the virus,
especially in combination and aimed at multiple targets. But I don't
know how many of them will actually become drugs. In addition,
replication inhibitors may not be enough to stop the virus after the
early stages of infection, when the individual may be asymptomatic. We
may need a combination of an antiviral drug to be effective early in the
disease and an anti-inflammatory drug for the later ``cytokine storm''
phase of the disease.

\textbf{Mukherjee}: Yes, just because a compound inhibits the virus in a
petri dish, doesn't mean that it can immediately become an antiviral
drug for human use. The compound might be toxic to humans. It might be
degraded into an inactive substance by the body. Its dose might be so
high that it's impossible to administer. But I do think while we're
waiting for the antibodies and the vaccines, it seems reasonable to
proceed with testing thousands of drugs against the virus --- called
``drug screening'' --- so that if something does come up, we might find
a drug to combine with remdesivir or with antibodies, making an
anti-viral cocktail.

From what I've been able to see, there is unusual urgency and
cooperation among scientists in this effort.

\textbf{Hamburg}: It's remarkable, in terms of the collaboration across
disciplines and research institutions and sectors and borders. There's
been more openness and sharing than I've seen in past crises like Ebola
or Zika or H1N1. Regulatory authorities around the world are coming
together in ways that are very, very important to reduce barriers and to
make sure that they're bringing the best possible science to bear on
decision making, trying to identify what are the critical questions that
have to be asked and answered, what kind of study designs and
preclinical work is going to be necessary, so that you don't have
companies facing different regulatory authorities with different
standards and requests and approaches, so that the hard questions can be
more effectively addressed through bringing together the best minds,
wherever they are.

\textbf{Weiss}: I've never seen this before, either. Our C.D.C. permit
to receive the virus, which is classified as a biosafety Level 3 agent,
was approved in less than two days. We received at least two
material-transfer agreements, which have to be signed by a number of
institutional officials and sometimes lawyers, in a matter of hours.
Both of those processes have taken much longer, sometimes weeks, in the
past. This is just one sign of administrators and scientists
collaborating with each other and acting extra efficiently to facilitate
the science.

\textbf{Barouch}: I'll just echo that. From a research perspective, I
have never seen such collaborative spirit, such open sharing of
materials, data, protocols, thoughts and ideas among academic groups,
industry groups, government groups and the clinicians on the front
lines.

\textbf{Yancopoulos}: I've seen unprecedented collaboration from all
forces. I can get on the phone and call my counterpart, Mikael Dolsten,
at Pfizer, and his first question is, ``Well, what can we do to help?''
Whether it's scientists in academia, whether it's people at biotech and
pharma companies, whether it's the doctors and health care workers who
are at the epicenter at hospitals like Mount Sinai or Columbia in New
York City, whether it's the F.D.A. --- we are all coming together, and
things are happening at unprecedented rates because we realize that we
have a common enemy.

\hypertarget{developing-a-covid-19-vaccinewhat-if-working-from-home-goes-on--foreverthe-pandemic-and-architectureinformation-can-be-the-best-medicine}{%
\paragraph{\texorpdfstring{\href{https://www.nytimes3xbfgragh.onion/interactive/2020/06/09/magazine/covid-vaccine.html}{Developing
a Covid-19
Vaccine}\href{https://www.nytimes3xbfgragh.onion/interactive/2020/06/09/magazine/remote-work-covid.html}{What
If Working From Home Goes on \ldots{}
Forever}\href{https://www.nytimes3xbfgragh.onion/interactive/2020/06/09/magazine/architecture-covid.html}{The
Pandemic and
Architecture}\href{https://www.nytimes3xbfgragh.onion/interactive/2020/06/10/magazine/covid-data.html}{Information
Can Be the Best
Medicine}}{Developing a Covid-19 VaccineWhat If Working From Home Goes on \ldots{} ForeverThe Pandemic and ArchitectureInformation Can Be the Best Medicine}}\label{developing-a-covid-19-vaccinewhat-if-working-from-home-goes-on--foreverthe-pandemic-and-architectureinformation-can-be-the-best-medicine}}

\begin{center}\rule{0.5\linewidth}{\linethickness}\end{center}

This discussion has been edited and condensed for clarity.

Maria Toutoudaki/Getty Images (bottle); Jan Olofsson/EyeEm, via Getty
Images (needle); Ilbusca/Getty Images (human figure); Guido Mieth/Getty
Images (model); Hannah A. Bullock and Azaibi Tamin/C.D.C. (Covid-19).

The Tech \& Design Issue

\begin{itemize}
\tightlist
\item
  Developing a Covid-19 Vaccine
\item
  \href{https://www.nytimes3xbfgragh.onion/interactive/2020/06/09/magazine/remote-work-covid.html}{What
  If Working From Home Goes on \ldots{} Forever}
\item
  \href{https://www.nytimes3xbfgragh.onion/interactive/2020/06/09/magazine/architecture-covid.html}{The
  Pandemic and Architecture}
\item
  \href{https://www.nytimes3xbfgragh.onion/interactive/2020/06/10/magazine/covid-data.html}{Information
  Can Be the Best Medicine}
\end{itemize}

\protect\hyperlink{}{} \protect\hyperlink{}{}

\includegraphics{https://static01.graylady3jvrrxbe.onion/newsgraphics/2020/06/04/2020-magtechdesign/a4fdae938d97ed501d80d0405d6760d098697e64/caret.svg}

Read 147 Comments

\begin{itemize}
\item
\item
\item
\item
\end{itemize}

Advertisement

\protect\hyperlink{after-bottom}{Continue reading the main story}

\hypertarget{site-index}{%
\subsection{Site Index}\label{site-index}}

\hypertarget{site-information-navigation}{%
\subsection{Site Information
Navigation}\label{site-information-navigation}}

\begin{itemize}
\tightlist
\item
  \href{https://help.nytimes3xbfgragh.onion/hc/en-us/articles/115014792127-Copyright-notice}{©~2020~The
  New York Times Company}
\end{itemize}

\begin{itemize}
\tightlist
\item
  \href{https://www.nytco.com/}{NYTCo}
\item
  \href{https://help.nytimes3xbfgragh.onion/hc/en-us/articles/115015385887-Contact-Us}{Contact
  Us}
\item
  \href{https://www.nytco.com/careers/}{Work with us}
\item
  \href{https://nytmediakit.com/}{Advertise}
\item
  \href{http://www.tbrandstudio.com/}{T Brand Studio}
\item
  \href{https://www.nytimes3xbfgragh.onion/privacy/cookie-policy\#how-do-i-manage-trackers}{Your
  Ad Choices}
\item
  \href{https://www.nytimes3xbfgragh.onion/privacy}{Privacy}
\item
  \href{https://help.nytimes3xbfgragh.onion/hc/en-us/articles/115014893428-Terms-of-service}{Terms
  of Service}
\item
  \href{https://help.nytimes3xbfgragh.onion/hc/en-us/articles/115014893968-Terms-of-sale}{Terms
  of Sale}
\item
  \href{https://spiderbites.nytimes3xbfgragh.onion}{Site Map}
\item
  \href{https://help.nytimes3xbfgragh.onion/hc/en-us}{Help}
\item
  \href{https://www.nytimes3xbfgragh.onion/subscription?campaignId=37WXW}{Subscriptions}
\end{itemize}
