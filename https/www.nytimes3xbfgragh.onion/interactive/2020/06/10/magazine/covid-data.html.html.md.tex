Sections

SEARCH

\protect\hyperlink{site-content}{Skip to
content}\protect\hyperlink{site-index}{Skip to site index}

\hypertarget{comments}{%
\subsection{\texorpdfstring{\protect\hyperlink{commentsContainer}{Comments}}{Comments}}\label{comments}}

\href{}{How Data Became One of the Most Powerful Tools to Fight an
Epidemic}\href{}{Skip to Comments}

The comments section is closed. To submit a letter to the editor for
publication, write to
\href{mailto:letters@NYTimes.com}{\nolinkurl{letters@NYTimes.com}}.

\hypertarget{how-data-became-one-of-the-most-powerful-tools-to-fight-an-epidemic}{%
\section{How Data Became One of the Most Powerful Tools to Fight an
Epidemic}\label{how-data-became-one-of-the-most-powerful-tools-to-fight-an-epidemic}}

By Steven JohnsonJune 11, 2020

\begin{itemize}
\item
\item
\item
\item
\item
  \emph{71}
\end{itemize}

As public-health experts have known since the 19th century, information
can be the best medicine. What new data streams could help quell future
outbreaks?

\hypertarget{how-data-became-one-of-the-most-powerful-tools-to-fight-an-epidemic-1}{%
\section{How Data Became One of the Most Powerful Tools to Fight an
Epidemic}\label{how-data-became-one-of-the-most-powerful-tools-to-fight-an-epidemic-1}}

As public-health experts have known since the 19th century, information
can be the best medicine. What new data streams could help quell future
outbreaks?

By Steven Johnson

June 10, 2020

SHARE

The River Lea originates in the suburbs north of London, winding its way
southward until it reaches the city's East End, where it empties into
the Thames near Greenwich and the Isle of Dogs. In the early 1700s, the
river was connected to a network of canals that supported the growing
dockyards and industrial plants in the area. By the next century, the
Lea had become one of the most polluted waterways in all of Britain,
deployed to flush out what used to be called the city's ``stink
industries.''

In June 1866, a laborer named Hedges was living with his wife on the
edge of the Lea, in a neighborhood called Bromley-by-Bow. Almost nothing
is known today about Hedges and his wife other than the sad facts of
their demise: On June 27 of that year, both of them died of cholera.

The deaths were not in themselves notable. Cholera had haunted London
since its arrival in 1832, with waves of epidemics that could kill
thousands in a matter of weeks. While the disease was on the decline in
recent years, a handful of cholera deaths had been reported in the
preceding weeks, and it was not unheard-of for two people sharing a home
to die of the disease on the same day.

But the deaths of Mr. and Mrs. Hedges turned out to be the start of a
much bigger outbreak. Within a few weeks, the working-class
neighborhoods surrounding the Lea were suffering one of the worst
cholera epidemics in London's history. The newspapers delivered the same
sort of morbid accounting that has obsessed us all in the age of the
SARS-CoV-2 coronavirus: the terrifying upward trajectory of runaway
growth. Twenty cholera deaths were reported in the East End the week
ending July 14. The following week's tally was 308. By August, the
weekly death toll had reached almost a thousand. London had not
experienced a major outbreak of cholera for 12 years. But by the second
week of August, the evidence was unmistakable: The city was under siege.

Then, as now, the first line of defense was data. Londoners were able to
track the march of cholera across the East End in close to real time,
thanks primarily to the work of one man: a doctor and statistician named
William Farr. For most of the Victorian era, Farr oversaw the collection
of public-health statistics in England and Wales. You could say without
exaggeration that the news environment that surrounds us now is one that
William Farr invented: a world where the latest numbers tracking the
spread of a virus --- how many intubations today? What's the growth rate
in hospitalizations? --- have become the single most important data
stream available, rendering the old metrics of stock tickers or
political polls mere afterthoughts.

In 1866, Farr had become a convert to a theory of cholera first proposed
by the London doctor John Snow more than a decade before --- the idea,
which turned out to be true, that the disease was being transmitted in
drinking water. And so as the deaths began to mount in the East End,
Farr immediately began investigating the water sources in the
neighborhood.

By the mid-1860s, a significant portion of working-class communities
were receiving their water through private companies that ran the pipes
to specific addresses, much as cable companies do today. Farr decided to
sort the population that had died in the recent outbreak not by
residence but by the company that supplied their drinking water. The
data he assembled revealed a clear pattern: An overwhelming number of
people who became ill drank out of East London Waterworks Company pipes.

The company claimed that their water had been effectively filtered at
their new covered reservoirs. But investigators soon tracked down the
source of contamination: The water in one East London company reservoir
had not been properly isolated from the nearby River Lea. Looking
through the mortality reports from earlier in the summer, the
investigators discovered the deaths of Mr. and Mrs. Hedges, who lived
near the reservoir. An examination of their residence revealed that
their toilet was expelling waste directly into the river, thereby
introducing cholera bacteria into the water supply and triggering the
outbreak. It was a brilliant piece of detective work, carried out with
remarkable speed and efficiency. And it turned out to be a momentous
one: 1866 marked the last significant cholera outbreak in the history of
London.

Farr was among the first to think systematically about how data on
outbreaks, their distribution in space and over time, could be used to
curb them as they unfolded --- and to minimize future ones. The field he
helped invent has come to be called epidemiology, but in its infancy it
was known by another name: vital statistics. (``Vital'' as in
\emph{vita}, Latin for life.) The innovations in this field do not look
like our traditional model of medical breakthroughs: They are not
packaged in the form of miracle drugs or new imaging technologies. At
their core, they are simply new ways of counting, new ways of discerning
patterns.

At this stage of the coronavirus pandemic, we find ourselves in a
situation not all that different from the Victorians, despite the vast
gulf in scientific, technological and medical expertise that separates
us from them. We lack vaccines to protect the uninfected; no drug has
yet emerged to cure Covid-19, the disease caused by the virus. Our main
protection right now is the one that Farr began building almost two
centuries ago: the collection and analysis of data. Data lets us see
where the disease is spreading and where health care systems are likely
to be overrun. It allows us to calculate infection rates and map hot
spots down to the level of ZIP codes.

Eventually, medicine will protect us from SARS-CoV-2, but for the time
being, vital statistics are the best defense we have. In the spirit of
William Farr, multiple new experiments in data gathering and analysis
have sprung up during the pandemic, experiments that might save
thousands of lives before the crisis is over. And they may well prevent
future pandemics from developing in the first place.

\textbf{Born in 1807} into a rural family of little means, William Farr
was a precocious learner who attracted the support of a wealthy patron
and mentors as a teenager, apprenticing with a local physician before
studying medicine in Paris and at University College London. By his
mid-20s, Farr had established a medical practice in London. But his true
passion was for vital statistics: He was an early member of the London
Statistical Society and came to believe that understanding macropatterns
in mortality could become a lifesaving tool as effective as any
traditional medical intervention. In fact, given the sorry state of
medicine in the 1830s, data was by far the more powerful instrument. The
use of data to understand patterns of life and death had been almost
exclusively a commercial interest during the 18th century, a science
developed largely for the mercenary aims of insurance companies. But
Farr and a handful of his peers saw the potential of vital statistics as
a tool for reform, a means of diagnosing the ills of society and shining
light on its inequalities.

After publishing a few papers in The Lancet analyzing medical data, Farr
was hired in 1837 as a ``compiler of abstracts'' at the General Register
Office, a new government body tasked with tracking births and deaths in
England and Wales. At Farr's encouragement, the G.R.O. began recording a
much wider range of data in its mortality reports, including cause of
death, occupation and age.

At the G.R.O., where he worked for nearly his entire career, Farr was
responsible for taking raw data and making it meaningful: discovering
interesting trends in the numbers, comparing health outcomes for
different subgroups in the population, inventing new forms of
visualization. His statistical inquiries would at times take him to some
disturbing positions. He spent years developing a bizarre theory about
the connection between topographic elevation and disease, which led to
some xenophobic ideas about the inferiority of lowland peoples. But the
enduring legacy of Farr's vital statistics turned out to be an
egalitarian one: exposing inequalities of health outcomes, using
scientific thinking to dispel the longstanding assumption, prevalent
among the ruling class, of a causal link between disease and moral
turpitude in low-income communities.

Counting the dead itself was not a new technique: London parish clerks
had been publishing weekly ``bills of mortality'' since the Elizabethan
era. But Farr devised new ways to make that information useful.
Collecting and publishing data was not merely a matter of reporting the
facts but instead a more subtle, exploratory art: testing and
challenging hypotheses, building explanatory models. As Farr wrote in an
essay published the year he joined the G.R.O., ``Facts, however
numerous, do not constitute a science. Like innumerable grains of sand
on the seashore, single facts appear isolated, useless, shapeless; it is
only when compared, when arranged in their natural relations, when
crystallized by the intellect, that they constitute the eternal truths
of science.''

The first question that Farr used statistics to answer is relevant to
our present crisis as well: To what extent did urban density contribute
to the death rate? Perhaps because of his own life journey --- growing
up in the agricultural region Shropshire, now living in the largest city
on the planet --- Farr decided to devote one of his first studies to the
differences in health outcomes between the country and the city.

Farr was a pioneer not just in collecting data but also in devising
ingenious new ways of representing it. One way of measuring the health
of a society is what was called in Farr's time a ``life table'':
breaking down the death rate in a given population by age. (Life tables
are what allowed us to see that Covid-19's lethality has been
disproportionately concentrated among the elderly, unlike the flu
pandemic of 1918, which killed an unusual number of young adults.) In
one early report, Farr experimented with an ingenious way of
representing those different outcomes, drawing upon data collected from
three separate communities: metropolitan London, industrial Liverpool
and rural Surrey. It was, in effect, a tale of two cities --- and one
countryside. Viewed as a triptych, the illustrations conveyed a clear
message: Density was destiny.

In Surrey, the increase of mortality after birth was a gentle slope
upward, like a dune rising above a waterline. The spike in the cities,
by comparison, looked more like the cliffs of Dover. That steep ascent
condensed thousands of individual tragedies into one vivid and
scandalous image: In Liverpool, more than half of all children born were
dead before their 15th birthday.

Despite those grim numbers, Farr remained hopeful that the health crisis
emerging in the industrial cities could be ameliorated. ``Is the
excessive mortality of cities inevitable?'' Farr wrote in the 1840
annual report of the G.R.O. ``The first writers who established
satisfactorily the high mortality of cities took a gloomy and perhaps
fanatical view of the question. Cities were declared vortices of vice,
misery, disease and death; they were proclaimed `the graves of
mankind.''' And yet, he continued, ``there is reason to believe that the
aggregation of mankind in towns is not inevitably disastrous.''

In that same report, Farr turned his attention to another puzzling
pattern in the data he had collected: what he called the laws of action
of epidemics, now known to epidemiologists as Farr's Law. Analyzing a
smallpox outbreak in Liverpool, Farr divided the mortality counts into
10 separate periods. ``The mortality increased up to the fourth
registered period; the deaths in the first were 2,513, in the second
3,289, in the third 4,242; and it will be perceived at a glance that
these numbers increased very nearly at the rate of 30 percent.'' But the
rate of increase, he observed, ``only rises to 6 percent in the next,
where it remains stationary, like a projectile at the summit of the
curve which it is destined to describe.'' Farr's Law was the first
attempt to describe the rise and fall of contagious diseases
mathematically. All the models that have shaped so much private angst
and public scrutiny --- the Imperial College London models that steered
Prime Minister Boris Johnson away from the initial strategy of herd
immunity, the University of Washington Covid-19 projections that have
heavily influenced the Trump White House --- all these forecasts are
descendants of the laws of action that Farr originally sketched out in
1840. When we talk about flattening the curve, the curve in question was
first drawn by William Farr.

\textbf{Victorian scientists} would have immediately recognized many of
the core categories of data assembled by epidemiologists working on
Covid-19: infections, deaths, locations and so on. Today's vital
statisticians obviously have access to a wider pool of information ---
antibody-test results, comorbidities of victims, even different genetic
strains of the virus --- than Farr was able to assemble. And they have
software that allows them to build models that project the
epidemiological curve that Farr first identified.

But the coronavirus pandemic has also revealed some crucial holes in the
way we collect data during an emerging outbreak. As unlikely as it might
sound, given the existence of organizations like the C.D.C. or the
W.H.O., in the early days of the coronavirus's spread, no single data
repository existed where information about all the known cases could be
accessed and analyzed by public-health officials and researchers.
``There really has never been a successful effort to share comprehensive
open data sources during any of the modern epidemics,'' says Samuel V.
Scarpino, who runs the Emergent Epidemics Lab at Northeastern
University. ``The vast majority of public-health data during epidemics
are still largely organized on pen, paper, Excel and PDFs.''

Scarpino was one of a handful of volunteers, including the Oxford
research fellow Moritz Kraemer and a Ph.D. student at Tsinghua
University in Beijing named Bo Xu, who formed an ad hoc organization in
late January to create a 21st-century equivalent of Farr's mortality
reports: a single open-source archive of every recorded Covid-19 case
anywhere in the world. By early February, the Open Covid-19 Data Working
Group had assembled detailed records for 10,000 cases. Today an informal
network of hundreds of volunteers has assembled records for more than a
million cases in 142 countries around the world. It may well be the
single most accurate portrait of the virus's spread through the human
population in existence.

Of course, the greatest value in that kind of data set lies in the clues
it can give us about the future path of the disease and how that path
can potentially be interrupted. But again, the work of building those
models has entirely taken the form of impromptu efforts organized at a
handful of academic institutions around the world. The Johns Hopkins
University epidemiologist Caitlin Rivers argues that the coronavirus
pandemic has made it clear that one crucial innovation we need is a new
kind of institution, what Rivers called a ``center for epidemic
forecasting.'' Rivers draws an analogy to institutions like the National
Weather Service. ``There were a few big storms at the turn of the
century with terrible loss of life and also enormous economic
consequence, so there was interest at the time in figuring how to
predict the weather,'' Rivers explains. With meaningful investment,
Rivers believes, ``we can get to the place where we are with the Weather
Service, where we have reliable forecasts that inform our everyday lives
as the public, and also help decision makers to understand how best to
respond to these outbreaks.''

Forecasts are only as good as the underlying data that support them, and
in the case of disease outbreaks, most of the data collection --- even
in comprehensive archives like the one assembled by the Open Covid-19
Data group --- suffers from a crucial liability: The information is
captured too late. Numbers like hospitalizations or deaths are vital
statistics to be sure, but they are tracking the end stages in the path
of a disease. In the case of Covid-19, by the time the average person
makes it to the hospital, around 10 days have passed since their initial
contact with the virus. ``Public-health reporting is usually very
late,'' says the epidemiologist Larry Brilliant, who helped eradicate
smallpox in the 1970s. ``It's just shortly before the peak of an
outbreak, historically, because as people get more alarmed, they go to
their doctor, and their doctor goes to the public-health official and
they report it.''

With a disease like Covid-19, where presymptomatic and asymptomatic
carriers are capable of spreading the virus, the lag in reporting can
make the difference between a runaway outbreak and effective
containment. A typical case of Covid-19 that ends in a death follows
this timeline, which can stretch to 30 days or more:

\begin{quote}
Infection -\textgreater{} Incubation -\textgreater{} Presymptomatic
spread -\textgreater{} Symptoms and spread -\textgreater{} Doctor's
visit -\textgreater{} Hospitalization -\textgreater{} Intensive care
-\textgreater{} Death
\end{quote}

In the standard regime, even in the best-case scenario, data collection
doesn't begin until Day 10, during the doctor's visit. Covid-19 has
prompted an inspiring scramble of experiments designed to move the data
gathering earlier on the timeline. Some of them involve what is called
``sentinel surveillance'' --- widespread, early-stage testing in
critical populations that may be at risk. ``There's testing for the
individual who needs to understand do they have this disease, do they
need to isolate or seek care,'' says Lorna Thorpe, director of the
epidemiology division at New York University's medical school. ``But to
manage the outbreak, you need to know where it is, you need to be ahead
of it.'' Much like the outbreak of 1866, Covid-19 has hit hardest in
low-income communities, which generally have reduced access to the
health care system, where most data is collected. ``Oftentimes the
communities that need our attention during the outbreak, that are most
likely to be hit early, may also be the ones that we understand the
least about,'' Scarpino says.

In part because of the limited supply of tests, the first few months of
data about Covid-19 were almost entirely oriented toward people
experiencing severe symptoms, who would show up at hospitals. But a
sentinel-surveillance program could have targeted communities --- like
nursing homes or low-income neighborhoods --- that had not yet
experienced symptomatic infections, potentially detecting those
outbreaks before they became unstoppable. Thorpe points to the success
of the Seattle Flu Study, an initiative that began in 2019, which set up
testing kiosks, analyzed samples from hospitals and distributed home
nasal swabs to a broad section of the city's population, asking them to
send in samples if they developed symptoms of respiratory infection.
Tellingly, the program was the first to detect community transmission of
SARS-CoV-2 in the United States.

The Seattle Flu Study was a variation on another emerging technique that
has already played an important role in the fight against Covid-19:
``syndromic surveillance.'' The idea is simple: Supplement the official
data from patients entering the health care system with data tracking
the appearances of disease symptoms before they get to a doctor or a
hospital. One influential early project that drew on this approach was a
program called Google Flu Trends, introduced in 2008 as a collaboration
between Google and the C.D.C. The service didn't track symptoms directly
but instead analyzed patterns in Google search queries associated with
influenza: ``My child has a fever,'' say, or ``aches and pains.'' By
mapping those queries geographically, the service aimed to identify
influenza hot spots days or weeks before they showed up on the radar of
the C.D.C. Then, in 2011, an epidemiologist at Boston Children's
Hospital named John Brownstein helped create a website called Flu Near
You that relied on user-submitted data tracking fever and other flu
symptoms directly through a small but statistically representative group
of volunteers. In the early days of the SARS-CoV-2 outbreak, Brownstein
spun off a new version called Covid Near You. ``Most people with Covid
have mild illness and are unlikely to interact with any health system,''
Brownstein says. ``Data from self-reported symptoms can help fill in
gaps, especially in light of limited testing.'' A visitor to the site
answers a few simple questions: What's your ZIP code? How are you
feeling? If you're not feeling well, what are your symptoms? The data
collected allows the service to map emerging hot spots before they show
up in the clinics or in the official county health reports, effectively
shifting the data-collection timeline five days to the left. In late
March, when much of the focus was on the explosion of cases in New York
City, Covid Near You was already picking up a surge in Covid-19 symptoms
in less densely settled areas. ``Despite the urban hot spots,''
Brownstein says, they saw signs of outbreaks in rural communities,
``especially in locations where people may have second homes.''

New technology has also made syndromic surveillance more feasible. The
San Francisco-based start-up Kinsa has been selling an
internet-connected thermometer since 2014. According to Inder Singh,
Kinsa's chief executive and founder, who formerly oversaw the Clinton
Foundation's work on infectious disease, the original vision was for the
company to detect these early patterns of illness without forcing people
to change their usual routines. ``The idea was: Let's take an existing
behavior, the only thing that people do in the home when illness
strikes,'' Singh explains. ``They grab the thermometer.'' From the
consumer's point of view, the interaction with Kinsa's thermometer is
straightforward enough, but behind the scenes the device sends
anonymous, geolocated information about the results to Kinsa's servers.
That new data stream enables the company to maintain what it calls
health weather maps for the entire country, with real-time data on
atypical fevers reported down to the level of individual counties.

Starting on March 4, 2020, Kinsa's charts began tracking a statistically
meaningful increase in the number of fevers in New York, 19 days before
the city went into a full lockdown. (The first case in the city was
reported on March 1.) By March 10, the number of people registering an
elevated temperature in Brooklyn was 50 percent higher than normal,
suggesting that the virus was already rampant throughout the five
boroughs, even though the official case load was still less than 200.

One limitation of our current data has to do with geography rather than
time. As Marc Gourevitch, chair of the department of population health
at N.Y.U.'s medical school, observes, most of our tools for mapping
outbreaks aren't granular enough. ``In many cities and urban
neighborhoods,'' Gourevitch says, ``there can be great variation within
a couple of blocks, or a fraction of a mile, in terms of the conditions
that really drive health. So if you want to look at variations in health
and risk and outcomes, you need to take a granular view of the geography
that you're talking about if you want to be able to think about
strategies of protecting at these small scales. It's really the scale at
which health is fundamentally determined: whether it's by crowding,
access to good schools, to air quality --- all kinds of drivers that
vary on a small scale.'' By default, our health care data is generally
organized geographically by county. But in a city like New York, where a
single county contains millions of people, that scale is all wrong for
tracking a fast-moving virus.

In many cases, that wide-angle view has been established deliberately as
a privacy protection. A few years ago, Gourevitch helped organize an
online resource called City Health Dashboard, which presents community
life-expectancy averages by census tract, showcasing the broad
inequalities in health outcomes in communities living just a few blocks
from one another. But even that resource was controversial. ``It took
years and pressure to get state authorities and the C.D.C. to contribute
to the estimates of life expectancy at the census-tract level,''
Gourevitch says. ``That was a multiyear effort because of legitimate
concerns about the privacy issue.''

One potential solution that Gourevitch sees is a kind of geographic
blurring for outbreak data. In John Snow's famous map of the 1854
cholera outbreak --- the one that ultimately led to the understanding
that the disease was caused by contaminated water --- he documented
deaths at the level of individual street addresses, revealing a cluster
of deaths around a widely used drinking well. But in the middle of an
outbreak like Covid-19, you don't need to be zoomed in that far to get a
meaningful sense of where the outbreak is spreading. Instead of a
pushpin on the map denoting an infection at a specific address,
Gourevitch suggests deliberately making the targeting less precise:
perhaps a city block, not a specific address. That level of granularity
would be tight enough to detect the spread of the outbreak through
microcommunities in the city, but not so tight that individual
identities can be discerned in public data.

\textbf{While all these} forms of disease surveillance offer
improvements on the basic model that Farr and Snow helped invent in the
middle of the 19th century, they share one key characteristic: They are
based on data assembled from human beings, as they pass either through
the health care system or through some self-reporting mechanism.
Shifting the timeline even further to the left may require new sources
of data that are not anchored in individual cases.

In the early 1990s, a Dutch microbiologist named Gertjan Medema was
conducting experiments with triathletes racing in the Rhine delta.
Medema and his colleagues were interested in the health impact of
open-river swimming, and so as part of their experiment they collected
river water, which they subsequently analyzed for the presence of a
whole host of pathogens: bacteria, fecal pathogens, enteric viruses and
other dangerous microbes. In those days, testing a sample for the
presence of these organisms took weeks. While Medema and his team were
still waiting for their results, the news broke of an unusual outbreak
of polio in Streefkerk, a town six miles downriver from the testing
site. Medema analyzed the river water they collected three weeks before
and discovered that poliovirus was clearly detectable in the samples.
The river held clues pointing to an emerging outbreak weeks before the
health authorities did. ``It was a lucky coincidence --- if `luck' is
the right word to use,'' Medema says now.

Back in 1992, those clues were effectively useless from a public-health
perspective because they took too long to decipher. But today's tools
allow scientists like Medema to detect a virus based on its precise
genetic sequence in a matter of hours. Because many dangerous pathogens
are expelled in human waste, sewage samples are the most direct way of
surveying viral or bacterial activity in a given community --- short of
testing people directly. ``When I saw SARS-CoV-2 hit in China,'' Medema
says, ``I was looking for reports of fecal shedding of the virus.''
Before long, evidence began circulating that some sufferers of Covid-19
experienced diarrhea. ``And that's when I said, it may be that this
virus will come to our country, so we'd better prepare sewage
surveillance. Not because we think it's an important risk for
transmission, but because you could use sewage to monitor the
circulation of the virus in the population.''

On Feb. 6, Medema and his colleagues gathered sewage samples from six
points in the Netherlands, including a waste-treatment plant near
Amsterdam's Schiphol Airport, on the premise that the virus could
potentially first arrive via air travel. The results came back negative.
But a month later, when the outbreak was still in its earliest stages in
the Netherlands, they returned to the same locations to collect samples.
This time, they found evidence of the virus in several of the locations.
``If we compare our prior sewage reporting with the number of reported
cases,'' Medema says, ``it looks likely we can pick up the signal of the
virus if we are at about one in 100,000 people reported infected.'' (A
preliminary study of a sewage-treatment plant in New Haven, Conn., this
spring showed that presence of the virus in wastewater peaked seven days
before reported Covid-19 cases.) In Farr's era, sewage was a primary
cause of epidemics. But in the 21st century, sewage might well offer us
important data to contain their spread.

Not all pathogens are expelled in human excrement, which means that
Medema's approach has some limitations as a defense against future
outbreaks. But sewage surveillance has a critical advantage over
syndromic surveillance with a virus like SARS-CoV-2, which has an
unusually high concentration of carriers who show no symptoms
whatsoever. ``The difficulty for this kind of virus is that containment
doesn't work because there is a lot of silent transmission,'' Medema
says. ``But we can use sewage surveillance with these sorts of viruses
--- to pick them up and understand the virus circulation better. There's
a projection that we may see waves and waves of this virus. Maybe sewage
surveillance can be an early warning to see if there's another wave
coming.''

The most radical technique for shifting the data-collection timeline to
the left --- but the one that might offer the most significant
protection against future epidemics --- involves cutting people out of
the equation altogether. The underlying data that allowed William Farr
to draw the first epidemic curve back in 1840 was, understandably,
limited to patterns of life and death in the human population. Syndromic
or sewage surveillance allows us to pick up signals earlier in the cycle
by detecting symptoms or fecal shedding before people make contact with
the health system. But for many of the most terrifying diseases that
have emerged in the past few decades, the initial human cases showed up
in the middle of a much longer timeline. ``Covid, SARS, MERS, swine flu,
bird flu, Ebola, H.I.V., Zika all were at one point animal diseases,''
Larry Brilliant says. ``Instead of syndromic surveillance, going two
steps to the left is surveillance of animal diseases. You move it where
it belongs into the realm of the zoonotic diseases, about 50 of which
have jumped species from animals to humans in the last three decades.''

The promise of applying Farr's vital statistics to the realm of animal
diseases is a simple one: You can stop an emerging zoonotic disease
before it makes the jump from animal to human. Animal surveillance could
ward off the potential pandemic that experts have historically worried
about the most: an influenza outbreak along the lines of the 1918 avian
flu. ``When you have 20 of your chickens die off, and your whole
livelihood depends upon them,'' Brilliant says, ``if you have a hotline
as they do in Cambodia, you can call the government and say `I have 20
dead chickens,' and they'll come and bring you 30 live ones and clean up
your place. That's a phenomenal bi-direction system that cleans up the
virus for you, puts you back into business, and the epidemic is aborted.
Being able to survey bats, pigs, birds --- that's going way beyond
syndromic surveillance. That's what we're going to have to do in the age
of pandemics.''

Public-health data began with that most elemental form of accounting:
how many people died on this day in this place. The insights that arose
from the collection of that information helped turn cities from the
``graves of mankind'' into communities that today enjoy some of the
longest life expectancies anywhere on the planet. But during an
epidemic, from the perspective of vital statistics, a human death tells
the story of an infection that happened in the past. A hundred dead
chickens, on the other hand, could tell the story of a future infection
--- and maybe even stop it from emerging at all.

\hypertarget{developing-a-covid-19-vaccinewhat-if-working-from-home-goes-on--foreverthe-pandemic-and-architectureinformation-can-be-the-best-medicine}{%
\paragraph{\texorpdfstring{\href{https://www.nytimes3xbfgragh.onion/interactive/2020/06/09/magazine/covid-vaccine.html}{Developing
a Covid-19
Vaccine}\href{https://www.nytimes3xbfgragh.onion/interactive/2020/06/09/magazine/remote-work-covid.html}{What
If Working From Home Goes on \ldots{}
Forever}\href{https://www.nytimes3xbfgragh.onion/interactive/2020/06/09/magazine/architecture-covid.html}{The
Pandemic and
Architecture}\href{https://www.nytimes3xbfgragh.onion/interactive/2020/06/10/magazine/covid-data.html}{Information
Can Be the Best
Medicine}}{Developing a Covid-19 VaccineWhat If Working From Home Goes on \ldots{} ForeverThe Pandemic and ArchitectureInformation Can Be the Best Medicine}}\label{developing-a-covid-19-vaccinewhat-if-working-from-home-goes-on--foreverthe-pandemic-and-architectureinformation-can-be-the-best-medicine}}

\begin{center}\rule{0.5\linewidth}{\linethickness}\end{center}

Steven Johnson is the author of twelve books, including his account of
the 1854 cholera epidemic, ``The Ghost Map,'' and most recently, ``Enemy
of All Mankind: A True Story of Piracy, Power, and History's First
Global Manhunt.''

The Tech \& Design Issue

\begin{itemize}
\tightlist
\item
  Developing a Covid-19 Vaccine
\item
  \href{https://www.nytimes3xbfgragh.onion/interactive/2020/06/09/magazine/remote-work-covid.html}{What
  If Working From Home Goes on \ldots{} Forever}
\item
  \href{https://www.nytimes3xbfgragh.onion/interactive/2020/06/09/magazine/architecture-covid.html}{The
  Pandemic and Architecture}
\item
  \href{https://www.nytimes3xbfgragh.onion/interactive/2020/06/10/magazine/covid-data.html}{Information
  Can Be the Best Medicine}
\end{itemize}

\protect\hyperlink{}{} \protect\hyperlink{}{}

\includegraphics{https://static01.graylady3jvrrxbe.onion/newsgraphics/2020/06/04/2020-magtechdesign/a4fdae938d97ed501d80d0405d6760d098697e64/caret.svg}

Read 71 Comments

\begin{itemize}
\item
\item
\item
\item
\end{itemize}

Advertisement

\protect\hyperlink{after-bottom}{Continue reading the main story}

\hypertarget{site-index}{%
\subsection{Site Index}\label{site-index}}

\hypertarget{site-information-navigation}{%
\subsection{Site Information
Navigation}\label{site-information-navigation}}

\begin{itemize}
\tightlist
\item
  \href{https://help.nytimes3xbfgragh.onion/hc/en-us/articles/115014792127-Copyright-notice}{©~2020~The
  New York Times Company}
\end{itemize}

\begin{itemize}
\tightlist
\item
  \href{https://www.nytco.com/}{NYTCo}
\item
  \href{https://help.nytimes3xbfgragh.onion/hc/en-us/articles/115015385887-Contact-Us}{Contact
  Us}
\item
  \href{https://www.nytco.com/careers/}{Work with us}
\item
  \href{https://nytmediakit.com/}{Advertise}
\item
  \href{http://www.tbrandstudio.com/}{T Brand Studio}
\item
  \href{https://www.nytimes3xbfgragh.onion/privacy/cookie-policy\#how-do-i-manage-trackers}{Your
  Ad Choices}
\item
  \href{https://www.nytimes3xbfgragh.onion/privacy}{Privacy}
\item
  \href{https://help.nytimes3xbfgragh.onion/hc/en-us/articles/115014893428-Terms-of-service}{Terms
  of Service}
\item
  \href{https://help.nytimes3xbfgragh.onion/hc/en-us/articles/115014893968-Terms-of-sale}{Terms
  of Sale}
\item
  \href{https://spiderbites.nytimes3xbfgragh.onion}{Site Map}
\item
  \href{https://help.nytimes3xbfgragh.onion/hc/en-us}{Help}
\item
  \href{https://www.nytimes3xbfgragh.onion/subscription?campaignId=37WXW}{Subscriptions}
\end{itemize}
