Sections

SEARCH

\protect\hyperlink{site-content}{Skip to
content}\protect\hyperlink{site-index}{Skip to site index}

\hypertarget{comments}{%
\subsection{\texorpdfstring{\protect\hyperlink{commentsContainer}{Comments}}{Comments}}\label{comments}}

\href{}{From the Magazine: `It Is Time for Reparations'}\href{}{Skip to
Comments}

The comments section is closed. To submit a letter to the editor for
publication, write to
\href{mailto:letters@NYTimes.com}{\nolinkurl{letters@NYTimes.com}}.

\hypertarget{from-the-magazine-it-is-time-for-reparations}{%
\section{From the Magazine: `It Is Time for
Reparations'}\label{from-the-magazine-it-is-time-for-reparations}}

By
\href{https://www.nytimes3xbfgragh.onion/by/nikole-hannah-jones}{Nikole
Hannah-Jones}June 24, 2020

\begin{itemize}
\item
\item
\item
\item
\item
  \emph{1158}
\end{itemize}

Nikole Hannah-Jones explains the history of economic injustice and
argues that if black lives are to truly matter in America, the nation
must finally pay its debts.

\hypertarget{from-the-magazine-it-is-time-for-reparations-1}{%
\section{From the Magazine: `It Is Time for
Reparations'}\label{from-the-magazine-it-is-time-for-reparations-1}}

Chester Higgins

Atlanta, 1973.

Devin Allen

Washington, 2015.

If true justice and equality are ever to be achieved in the United
States, the country must finally take seriously what it owes black
Americans.

By
\href{https://www.nytimes3xbfgragh.onion/by/nikole-hannah-jones}{Nikole
Hannah-Jones}

June 30, 2020

SHARE

Lettering by Bobby C. Martin Jr.

\textbf{It feels different this time.}

Black Americans protesting the violation of their rights are a defining
tradition of this country. In the last century, there have been hundreds
of uprisings in black communities in response to white violence. Some
have produced substantive change. After the assassination of the Rev.
Dr. Martin Luther King Jr. in 1968, uprisings in more than 100 cities
broke the final congressional deadlock over whether it should be illegal
to deny people housing simply because they descended from people who had
been enslaved. The Fair Housing Act, which prohibits housing
discrimination on the basis of race, gender and religion, among other
categories, seemed destined to die in Congress as
\href{https://www.propublica.org/article/living-apart-how-the-government-betrayed-a-landmark-civil-rights-law}{white
Southerners were joined by many of their Northern counterparts who knew
housing segregation was central to how Jim Crow was accomplished in the
North.} But just seven days after King's death, President Lyndon B.
Johnson signed the act into law from the smoldering capital, which was
still under protection from the National Guard.

Most of the time these uprisings have produced hand-wringing and
consternation but few necessary structural changes. After black
uprisings swept the nation in the mid-1960s, Johnson created
\href{https://digitalcommons.ilr.cornell.edu/cgi/viewcontent.cgi?article=1047\&context=hrpubs}{the
Kerner Commission} to examine their causes, and the report it issued in
1968 recommended a national effort to dismantle segregation and
structural racism across American institutions. It was shelved by the
president, like so many similar reports, and instead white Americans
voted in a ``law and order'' president, Richard Nixon. The following
decades brought increased police militarization, law-enforcement
spending and mass incarceration of black Americans.

\hypertarget{listen-to-this-article}{%
\subparagraph{Listen to This Article}\label{listen-to-this-article}}

Audio Recording by Audm

To hear more audio stories from publishers like The New York Times,
download
\href{https://www.audm.com/?utm_source=nytmag\&utm_medium=embed\&utm_campaign=owed_black_americans}{Audm
for iPhone or Android}.

The changes we're seeing today in some ways seem shockingly swift, and
in other ways rage-inducingly slow. After years of black-led activism,
protest and organizing, the weeks of protests since George Floyd's
killing have moved lawmakers to ban chokeholds by police officers,
consider stripping law enforcement of the qualified immunity that has
made it almost impossible to hold responsible officers who kill, and
discuss moving significant parts of ballooning police budgets into
funding for social services. Black Lives Matter, the group founded in
2013 by three black women, Patrisse Khan-Cullors, Alicia Garza and Opal
Tometi, after the acquittal of Trayvon Martin's killer, saw its support
among American voters
\href{https://www.nytimes3xbfgragh.onion/interactive/2020/06/10/upshot/black-lives-matter-attitudes.html}{rise
almost as much in the two weeks after Floyd's killing} than in the last
two years.
\href{https://civiqs.com/results/black_lives_matter?uncertainty=true\&annotations=true\&zoomIn=true}{According
to polling by Civiqs,} more than 50 percent of registered voters now say
they support the movement.

The cascading effect of these protests has been something to behold. The
commissioner of the N.F.L., which blackballed Colin Kaepernick for
daring to respectfully protest police brutality, announced that the
N.F.L. had, in fact, been wrong and that black lives actually do matter.
(Kaepernick, on the other hand, still has no job.) HBO Max announced
that it would temporarily pull from its roster the Lost Cause propaganda
film ``Gone With the Wind'' --- which in classically American fashion
holds the spot as the
\href{https://www.nytimes3xbfgragh.onion/2020/06/14/movies/gone-with-the-wind-battle.html}{highest-grossing
feature film of all time.} NASCAR came to the sudden realization that
its decades-long permissiveness toward fans' waving the battle flag of a
traitorous would-be nation that fought to preserve the right to traffic
black people was, in fact, contrary to its
\href{https://www.nascar.com/news-media/2020/06/10/nascar-statement-on-confederate-flag/}{``commitment
to providing a welcoming and inclusive environment for all fans, our
competitors and our industry.''} Bubba Wallace, the only full-time black
driver at the sport's top level, who had called on NASCAR to make the
move, drove victory laps in an all-black stock car emblazoned with the
words ``\#BLACKLIVESMATTER.''

Dr. Ernest C. Withers/Withers Family Trust

Memphis, 1968.

Anthony Geathers

Harlem, 2020.

Multiracial groups of Americans have defaced or snatched down monuments
to enslavers and bigots from Virginia to Philadelphia to Minneapolis and
New Mexico, leading local and state politicians to locate the moral
courage to realize that they indeed did have the power to purge from
public spaces icons to white supremacy. Even the University of Alabama,
the place where Gov. George Wallace quite literally stood in the
schoolhouse door to try to block the court-ordered admission of two
black students, a place whose Grecian-columned campus and
\href{https://www.al.com/news/birmingham/2018/08/alabama_sororities_desegregate.html}{still
largely segregated sororities} pose the living embodiment of Dixie, is
\href{https://www.montgomeryadvertiser.com/story/news/2020/06/09/university-of-alabama-remove-plaques-honor-confederates-study-renaming-buildings/5325754002/}{removing
three plaques honoring Confederate soldiers} and will study whether to
rename buildings holding the monikers of enslavers and white
supremacists after a student-led campaign garnered more than 17,000
signatures.

Unlike so many times in the past, in which black people mostly marched
and protested alone to demand recognition of their full humanity and
citizenship, a multiracial and multigenerational protest army has taken
to the streets over the last month. They've spread across all 50 states
in places big and small, including historically all-white towns like
Vidor, Texas, where as recently as 1993 a federal judge had to order its
public housing integrated. Shortly after,
\href{https://www.texasmonthly.com/news/black-lives-matter-vidor/}{white
supremacists ran out of town the handful of black people who had moved
in.} That Vidor, Texas, which remains 91 percent white and 0.5 percent
black, held a Black Lives Matter rally in early June. In countries as
disparate as England, Brazil, Kenya and Turkey, crowds pumped fists and
carried signs with George Floyd's name.

And this month,
\href{https://www.monmouth.edu/polling-institute/reports/monmouthpoll_US_060220/}{a
Monmouth University poll} showed that 76 percent of Americans, and 71
percent of white Americans, believe that racial and ethnic
discrimination is a ``big problem'' in the United States. Just a few
years ago, little more than half of white Americans believed that. The
numbers in the Monmouth poll were so high that it left some political
scientists questioning the poll's quality.

``This number is crazy,'' Hakeem Jefferson, a Stanford University
political scientist, told me. ``When I saw it, I thought, `This is a
polling error.' So I did what good social scientists do. I opened the
methodological report, worried that they had done a weird sampling. But
this is high-quality data.''

It is hard in the midst of something momentous to pinpoint exactly what
has caused it. What we're seeing is most likely a result of unrelenting
organizing by the Black Lives Matter movement. It's the pandemic, which
virtually overnight left staggering numbers of Americans without enough
money to buy food, pay rent and sustain their businesses. For many white
Americans who may have once, consciously or unconsciously, looked down
upon this nation's heavily black and brown low-wage service workers,
Covid-19 made them realize that it was the delivery driver and grocery
clerk and meatpacker who made it possible for them to remain safely
sequestered in their homes --- and these workers were dying for it.
Black Americans, in particular, have borne
\href{https://www.apmresearchlab.org/covid/deaths-by-race}{a
disproportionate number of deaths from both Covid-19} and law
enforcement, and many nonblack protesters have reasoned that black
people should not have to risk their lives alone in taking to the
streets demanding that the state not execute its citizens without
consequence. And as they did, white Americans both in the streets and
through the screens of their phones and televisions got a taste of the
wanton police violence that black Americans regularly face. They saw the
police beating up white women, pushing down an elderly white man and
throwing tear gas and shooting rubber bullets at demonstrators
exercising their democratic right to peacefully protest.

With so many Americans working from home or not working at all, they
have had the time to show up to protests every day. These protests not
only give Americans who are not black a moral reason to leave their
homes after weeks of social isolation; they also allow protesters to
vent anger at the incompetence of the man in the White House, himself a
product of this nation's inability to escape its death pact with white
supremacy, who they sense is imperiling this terribly flawed but
miraculous country.

\begin{center}\rule{0.5\linewidth}{\linethickness}\end{center}

\textbf{It has been more than 150 years} since the white planter class
last
\href{https://lawenforcementmuseum.org/2019/07/10/slave-patrols-an-early-form-of-american-policing/}{called
up the slave patrols} and deputized every white citizen to stop,
question and subdue any black person who came across their paths in
order to control and surveil a population who refused to submit to their
enslavement. It has been 150 years since white Americans could enforce
slave laws that said white people acting in the interest of the planter
class would not be punished for killing a black person, even for the
most minor alleged offense. Those laws morphed into the black codes,
passed by white Southern politicians at the end of the Civil War to
criminalize behaviors like not having a job. Those black codes were
struck down, then altered and over the course of decades eventually
transmuted into stop-and-frisk, broken windows and, of course, qualified
immunity. The names of the mechanisms of social control have changed,
but the presumption that white patrollers have the legal right to kill
black people deemed to have committed minor infractions or to have
breached the social order has remained.

In a country erected on the explicitly codified conviction that black
lives mattered less, graveyards across this land hold the bodies of
black Americans, men, women and \emph{children}, legally killed by the
institutional descendants of those slave patrols for alleged
transgressions like walking from the store with Skittles, playing with a
toy gun in the park, sleeping in their homes and selling untaxed
cigarettes. We collectively know only a small number of their names:
Michael Brown, Tamir Rice, Trayvon Martin, Kendra James, Breonna Taylor,
Rekia Boyd, Eric Garner, Aiyana Stanley-Jones and Tanisha Anderson are
just a few.

And because of what is happening now, George Floyd's name will forever
stand out since enough Americans have decided that his death mattered.

MPI/Getty Images

New York, mid-1930s.

Devin Allen/Collection of the Smithsonian National Museum of African
American History and Culture

Baltimore, 2015.

What has spawned this extraordinary reckoning, the fire \emph{this}
time, was our collective witness of what must be described without
hyperbole as a modern-day lynching. In his 1933 book, ``The Tragedy of
Lynching,'' the sociologist Arthur F. Raper estimated that, based on his
study of 100 lynchings, white police officers participated in at least
half of all lynchings and that in 90 percent of others law-enforcement
officers ``either condone or wink at the mob action.'' The nonchalant
look on Officer Derek Chauvin's face --- as, hand in pocket, for 8
minutes 46 seconds, he pressed his knee against the neck of a facedown
black man begging for his life --- reminds me of every callous white
face captured in the \href{https://www.withoutsanctuary.org/}{grisly
photos taken in the 1900s} to mark the gleeful spectacle of the public
killings of black men and women.

It devastates black people that all the other black deaths before George
Floyd did not get us here. It devastates black people to recall all the
excuses that have come before. That big black boy, Michael Brown, must
have charged the weapon-carrying officer. Eric Garner should have
stopped struggling. Breonna Taylor's boyfriend had a weapon in her home
and shouldn't have shot at the people who, without a knock or an
announcement, burst through her door. We're not sure \emph{what} Ahmaud
Arbery was doing in that predominantly white neighborhood. Rayshard
Brooks, who in the midst of nationwide protests against police violence
was shot in the back twice by a police officer, just shouldn't have
resisted.

It should devastate us all that in 2020 it took a cellphone video
broadcast across the globe of a black man dying from the oldest and most
terrifying tool in the white-supremacist arsenal to make a vast majority
of white Americans decide that, well, this might be enough.

\begin{center}\rule{0.5\linewidth}{\linethickness}\end{center}

\textbf{We, now, have finally arrived} at the point of this essay.
Because when it comes to truly explaining racial injustice in this
country, the table should never be set quickly: There is too much to
know, and yet we aggressively choose not to know it.

No one can predict whether this uprising will lead to lasting change.
History does not bode well. But there does seem to be a widespread
acceptance of the most obvious action we could take toward equality in a
nation built on the espoused ideals of inalienable, universal rights:
pass reforms and laws that ensure that black people cannot be killed by
armed agents of the state without consequence.

But on its own, this cannot bring justice to America. If we are truly at
the precipice of a transformative moment, the most tragic of outcomes
would be that the demand be too timid and the resolution too small. If
we are indeed serious about creating a more just society, we must go
much further than that. We must get to the root of it.

Read More From the Magazine and Nikole Hannah-Jones

\href{https://www.nytimes3xbfgragh.onion/interactive/2019/08/14/magazine/1619-america-slavery.html}{}

\hypertarget{the-1619-projectaug-14-2019}{%
\paragraph{The 1619 ProjectAug. 14,
2019}\label{the-1619-projectaug-14-2019}}

\includegraphics{https://static01.graylady3jvrrxbe.onion/images/2019/08/18/magazine/18mag-1619landing-hp1/18mag-1619landing-hp1-mediumThreeByTwo210.jpg}
\href{https://www.nytimes3xbfgragh.onion/interactive/2019/08/14/magazine/black-history-american-democracy.html}{}

\hypertarget{america-wasnt-a-democracy-until-black-americans-made-it-oneaug-14-2019}{%
\paragraph{America Wasn't a Democracy, Until Black Americans Made It
OneAug. 14,
2019}\label{america-wasnt-a-democracy-until-black-americans-made-it-oneaug-14-2019}}

\includegraphics{https://static01.graylady3jvrrxbe.onion/images/2019/08/14/magazine/hpdemocracy/hpdemocracy-mediumThreeByTwo210.jpg}

Fifty years since the bloody and brutally repressed protests and freedom
struggles of black Americans brought about the end of legal
discrimination in this country, so much of what makes black lives hard,
what takes black lives earlier, what causes black Americans to be
vulnerable to the type of surveillance and policing that killed Breonna
Taylor and George Floyd, what steals opportunities, is the lack of
wealth that has been a defining feature of black life since the end of
slavery.

Wealth, not income, is the means to security in America. Wealth ---
assets and investments minus debt --- is what enables you to buy homes
in safer neighborhoods with better amenities and better-funded schools.
It is what enables you to send your children to college without saddling
them with tens of thousands of dollars of debt and what provides you
money to put a down payment on a house. It is what prevents family
emergencies or unexpected job losses from turning into catastrophes that
leave you homeless and destitute. It is what ensures what every parent
wants --- that your children will have fewer struggles than you did.
Wealth is security and peace of mind. It's not incidental that wealthier
people are healthier and live longer. Wealth is, as a recent Yale study
states, ``the most consequential index of economic well-being'' for most
Americans. But wealth is not something people create solely by
themselves; it is accumulated across generations.

While unchecked discrimination still plays a significant role in
shunting opportunities for black Americans, it is white Americans'
centuries-long economic head start that most effectively maintains
racial caste today. As soon as laws began to ban racial discrimination
against black Americans, white Americans created so-called race-neutral
means of maintaining political and economic power. For example, soon
after the 15th Amendment granted black men the right to vote, white
politicians in many states, understanding that recently freed black
Americans were impoverished, implemented poll taxes. In other words,
white Americans have long known that in a country where black people
have been kept disproportionately poor and prevented from building
wealth, rules and policies involving money can be nearly as effective
for maintaining the color line as legal segregation. You do not have to
have laws forcing segregated housing and schools if white Americans,
using their generational wealth and higher incomes, can simply buy their
way into expensive enclaves with exclusive public schools that are out
of the price range of most black Americans.

It has worked with impressive efficiency. Today black Americans remain
the most segregated group of people in America and are five times as
likely to live in high-poverty neighborhoods as white Americans. Not
even high earnings inoculate black people against racialized
disadvantage. Black families earning \$75,000 or more a year live in
poorer neighborhoods than white Americans earning less than \$40,000 a
year,
\href{https://s4.ad.brown.edu/projects/diversity/Data/Report/report0727.pdf}{research
by John Logan,} a Brown University sociologist, shows. According to
another study, by the Stanford sociologist Sean Reardon and his
colleagues, the average black family earning \$100,000 a year lives in a
neighborhood with an average annual income of \$54,000. Black Americans
with high incomes are still black: They face discrimination across
American life. But it is because their families have not been able to
build wealth that they are often unable to come up with a down payment
to buy in more affluent neighborhoods, while white Americans with lower
incomes often use familial wealth to do so.

The difference between the lived experience of black Americans and white
Americans when it comes to wealth --- along the entire spectrum of
income from the poorest to the richest --- can be described as nothing
other than a chasm. According to research published this year by
scholars at Duke University and Northwestern University that doesn't
even take into account the yet-unknown financial wreckage of Covid-19,
the average black family with children holds
\href{https://journals.sagepub.com/doi/full/10.1177/2378023120916616}{just
one cent of wealth for every dollar} that the average white family with
children holds.

As President Johnson, architect of the Great Society,
\href{https://www.youtube.com/watch?v=vcfAuodA2x8}{explained in a 1965
speech} titled ``To Fulfill These Rights'': ``Negro poverty is not white
poverty. \ldots{} These differences are not racial differences. They are
solely and simply the consequence of ancient brutality, past injustice
and present prejudice. They are anguishing to observe. For the Negro
they are a constant reminder of oppression. For the white they are a
constant reminder of guilt. But they must be faced, and they must be
dealt with, and they must be overcome; if we are ever to reach the time
when the only difference between Negroes and whites is the color of
their skin.''

\begin{center}\rule{0.5\linewidth}{\linethickness}\end{center}

\textbf{We sometimes forget, and perhaps} it is an intentional
forgetting, that the racism we are fighting today was originally
conjured to justify working unfree black people, often until death, to
generate extravagant riches for European colonial powers, the white
planter class and all the ancillary white people from Midwestern farmers
to bankers to sailors to textile workers, who earned their living and
built their wealth from free black labor and the products that labor
produced. The prosperity of this country is inextricably linked with the
forced labor of the ancestors of 40 million black Americans for whom
these marches are now occurring, just as it is linked to the stolen land
of the country's indigenous people. Though our high school history books
seldom make this plain: Slavery and the 100-year period of racial
apartheid and racial terrorism known as Jim Crow were, above all else,
systems of economic exploitation. To borrow from Ta-Nehisi Coates's
phrasing, racism is the child of economic profiteering, not the father.

Numerous legal efforts to strip black people of their humanity existed
to justify the extraction of profit. Beginning in the 1660s, white
officials ensured that all children born to enslaved women would also be
enslaved and belong not to their mothers but to the white men who owned
their mothers.
\href{https://www.swarthmore.edu/SocSci/bdorsey1/41docs/24-sla.html}{They
passed laws dictating that the child's status would follow that of the
mother} not the father, upending European norms and guaranteeing that
the children of enslaved women who were sexually assaulted by white men
would be born enslaved and not free. It meant that profit for white
people could be made from black women's wombs. Laws determining that
enslaved people, just like animals, had no recognized kinship ties
ensured that human beings could be bought and sold at will to pay debts,
buy more acres or
\href{https://www.nytimes3xbfgragh.onion/2016/04/17/us/georgetown-university-search-for-slave-descendants.html}{save
storied universities} like Georgetown from closing. Laws barred enslaved
people from making wills or owning property, distinguishing black people
in America from every other group on these shores and assuring that
everything of value black people managed to accrue would add to the
wealth of those who enslaved them. At the time of the Civil War, the
value of the enslaved human
\href{https://www.nytimes3xbfgragh.onion/interactive/2019/08/14/magazine/slavery-capitalism.html}{beings
held as property added up to more than all of this nations' railroads
and factories combined.} And yet, enslaved people saw not a dime of this
wealth. They owned nothing and were owed nothing from all that had been
built from their toil.

Slavery's demise provided this nation the chance for redemption. Out of
the ashes of sectarian strife, we could have birthed a new country, one
that recognized the humanity and natural rights of those who helped
forge this country, one that attempted to atone and provide redress for
the unspeakable atrocities committed against black people in the name of
profit. We could have finally, 100 years after the Revolution, embraced
its founding ideals.

And, oh so briefly, during the period known as Reconstruction, we moved
toward that goal. The historian Eric Foner refers to these 12 years
after the Civil War as this nation's second founding, because it is here
that America began to redeem the grave sin of slavery. Congress passed
\href{https://constitutioncenter.org/learn/educational-resources/historical-documents/the-reconstruction-amendments}{amendments
abolishing human bondage, enshrining equal protection before the law in
the Constitution} and guaranteeing black men the right to vote. This
nation witnessed its first period of biracial governance as the formerly
enslaved were elected to public offices at all levels of government. For
a fleeting moment, a few white men listened to the pleas of black people
who had fought for the Union and helped deliver its victory. Land in
this country has always meant wealth and, more important, independence.
Millions of black people, liberated with not a cent to their name,
desperately wanted property so they could work, support themselves and
be left alone. Black people implored federal officials to take the land
confiscated from enslavers who had taken up arms against their own
country and grant it to those who worked it for generations. They were
asking to, as the historian Robin D.G. Kelley puts it, ``inherit the
earth they had turned into wealth for idle white people.''

Bettmann Archive/Getty Images

Birmingham, Ala., 1963.

Scott Olson/Getty Images

Ferguson, Mo., 2014.

In January 1865, Gen. William Tecumseh Sherman issued Special Field
Order 15, providing for the distribution of hundreds of thousands of
acres of former Confederate land issued in 40-acre tracts to newly freed
people along coastal South Carolina and Georgia. But just four months
later, in April, Lincoln was assassinated. Andrew Johnson, the racist,
pro-Southern vice president who took over, immediately reneged upon this
promise of 40 acres, overturning Sherman's order. Most white Americans
felt that black Americans should be grateful for their freedom, that the
bloody Civil War had absolved any debt. The government confiscated the
land from the few formerly enslaved families who had started to eke out
a life away from the white whip and gave it back to the traitors. And
with that, the only real effort this nation ever made to compensate
black Americans for 250 years of chattel slavery ended.

Freed people, during and after slavery, tried again and again to compel
the government to provide restitution for slavery, to provide at the
very least a pension for those who spent their entire lives working for
no pay. They filed lawsuits. They organized to lobby politicians. And
every effort failed. To this day, the only Americans who have ever
received government restitution for slavery were white enslavers in
Washington, D.C., who were compensated for their loss of human property.

The way we are taught this in school, Lincoln ``freed the slaves,'' and
then the nearly four million people who the day before had been treated
as property suddenly enjoyed the privileges of being Americans like
everyone else. We are not prodded to contemplate what it means to
achieve freedom without a home to live in, without food to eat, a bed to
sleep on, clothes for your children or money to buy any of it.
Narratives collected of
\href{https://www.loc.gov/collections/slave-narratives-from-the-federal-writers-project-1936-to-1938/about-this-collection/}{formerly
enslaved people during the Federal Writers' Project} of the 1930s reveal
the horrors of massive starvation, of ``liberated'' black people seeking
shelter in burned-out buildings and scrounging for food in decaying
fields before eventually succumbing to the heartbreak of returning to
bend over in the fields of their former enslavers, as sharecroppers,
just so they would not die. ``With the advent of emancipation,'' writes
the historian Keri Leigh Merritt, ``blacks became the only race in the
U.S. ever to start out, as an entire people, with close to zero
capital.''

In 1881, Frederick Douglass, surveying the utter privation in which the
federal government left the formerly enslaved, wrote: ``When the Hebrews
were emancipated, they were told to take spoil from the Egyptians. When
the serfs of Russia were emancipated, they were given three acres of
ground upon which they could live and make a living. But not so when our
slaves were emancipated. They were sent away empty-handed, without
money, without friends and without a foot of land on which they could
live and make a living. Old and young, sick and well, were turned loose
to the naked sky, naked to their enemies.''

\begin{center}\rule{0.5\linewidth}{\linethickness}\end{center}

\textbf{Just after the federal government} decided that black people
were undeserving of restitution, it began bestowing millions of acres in
the West to white Americans under the Homestead Act, while also enticing
white foreigners to immigrate with the offer of free land. From 1868 to
1934, the federal government gave away 246 million acres in 160-acre
tracts, nearly 10 percent of all the land in the nation, to more than
1.5 million white families, native-born and foreign. As Merritt points
out, some 46 million American adults today, nearly 20 percent of all
American adults, descend from those homesteaders. ``If that many white
Americans can trace their legacy of wealth and property ownership to a
single entitlement program,'' Merritt writes, ``then the perpetuation of
black poverty must also be linked to national policy.''

The federal government turned its back on its financial obligations to
four million newly liberated people, and then it left them without
protection as well, as white rule was reinstated across the South
starting in the 1880s. Federal troops pulled out of the South, and white
Southerners overthrew biracial governance using violence, coups and
election fraud.

The campaigns of white terror that marked the period after
Reconstruction, known as Redemption, once again guaranteed an
exploitable, dependent labor force for the white South. Most black
Southerners had no desire to work on the same forced-labor camps where
they had just been enslaved. But white Southerners passed state laws
that made it a crime if they didn't sign labor contracts with white
landowners or changed employers without permission or sold cotton after
sunset, and then as punishment for these ``crimes,'' black people were
forcibly leased out to companies and individuals. Through sharecropping
and convict leasing, black people were compelled back into quasi
slavery. This arrangement ensured that once-devasted towns like
Greenwood, Miss., were again able to call themselves the cotton capitals
of the world, and companies like United States Steel secured a steady
supply of unfree black laborers who could be worked to death, in what
Douglass A. Blackmon, in his Pulitzer Prize-winning book, calls
``slavery by another name.''

Yet black Americans persisted, and despite the odds, some managed to
acquire land, start businesses and build schools for their children. But
it was the most prosperous black people and communities that elicited
the most vicious response. Lynchings, massacres and generalized racial
terrorism were regularly deployed against black people who had bought
land, opened schools, built thriving communities, tried to organize
sharecroppers' unions or opened their own businesses, depriving white
owners of economic monopolies and the opportunity to cheat black buyers.

At least \href{https://lynchinginamerica.eji.org/report/}{6,500 black
people were lynched} from the end of the Civil War to 1950, an average
of nearly two a week for nine decades. Nearly five black people, on
average, have been
\href{https://www.washingtonpost.com/graphics/investigations/police-shootings-database/?itid=lk_inline_manual_5}{killed
a week by law enforcement since 2015.}

The scale of the destruction during the 1900s is incalculable. Black
farms were stolen, shops burned to the ground. Entire prosperous black
neighborhoods and communities were razed by white mobs from
\href{https://www.ajc.com/news/national/the-rosewood-massacre-how-lie-destroyed-black-town/wTcKjELkGskePsWiwutQuO/}{Florida}
to \href{https://www.ncpedia.org/wilmington-race-riot}{North Carolina}
to
\href{https://www.georgiaencyclopedia.org/articles/history-archaeology/atlanta-race-riot-1906}{Atlanta}
to
\href{https://www.nytimes3xbfgragh.onion/2019/09/30/opinion/elaine-massacre-1919-arkansas.html}{Arkansas.}
One of the most infamous of these, and yet still widely unknown among
white Americans,
\href{https://www.tulsahistory.org/exhibit/1921-tulsa-race-massacre/}{occurred
in Tulsa, Okla.,} when gangs of white men, armed with guns supplied by
public officials, destroyed a black district so successful that it was
known as Black Wall Street. They burned more than 1,200 homes and
businesses, including a department store, a library and a hospital, and
killed hundreds
\href{https://www.nytimes3xbfgragh.onion/2020/06/19/opinion/tulsa-race-riot-massacre-graves.html}{who
it is believed were buried in mass graves.} In 2001, a commission on the
massacre recommended that the state pay financial restitution for the
victims, but the State Legislature refused. And this is the place that
in the midst of weeks of protests crying out for black lives to matter,
Donald Trump, nearly 100 years later, chose to restart his campaign
rallies.

Leonard Freed/Magnum Photos

Washington, 1963.

Malike Sidibe for The New York Times

Brooklyn, 2020.

Even black Americans who did not experience theft and violence were
continuously deprived of the ability to build wealth. They were
\href{https://www.lib.umd.edu/unions/social/african-americans-rights}{denied
entry into labor unions} and union jobs that ensured middle-class wages.
North and South, racist hiring laws and policies forced them into
service jobs, even when they earned college degrees. They were legally
relegated into segregated, substandard neighborhoods and segregated,
substandard schools that made it impossible to compete economically even
had they not faced rampant discrimination in the job market. In the
South, for most of the period after the Civil War until the 1960s,
nearly all the black people who wanted to earn professional degrees ---
law, medical and master's degrees --- had to leave the region to do so
even as white immigrants attended state colleges in the former
Confederacy that black American tax dollars helped pay for.

As part of the New Deal programs, the federal government created
redlining maps, marking neighborhoods where black people lived in red
ink to denote that they were uninsurable. As a result, 98 percent of the
loans the Federal Housing Administration insured from 1934 to 1962 went
to white Americans, locking nearly all black Americans out of the
government program credited with building the modern (white) middle
class.

``At the very moment a wide array of public policies was providing most
white Americans with valuable tools to advance their social welfare ---
ensure their old age, get good jobs, acquire economic security, build
assets and gain middle-class status --- most black Americans were left
behind or left out,'' the historian Ira Katznelson writes in his book,
``When Affirmative Action Was White.'' ``The federal government \ldots{}
functioned as a commanding instrument of white privilege.''

In other words, while black Americans were being systematically,
generationally deprived of the ability to build wealth, while also being
robbed of the little they had managed to gain, white Americans were not
only free to earn money and accumulate wealth with exclusive access to
the best jobs, best schools, best credit terms, but they were also
getting substantial government help in doing so.

\begin{center}\rule{0.5\linewidth}{\linethickness}\end{center}

\textbf{The civil rights movement} ostensibly ended white advantage by
law. And in the gauzy way white Americans tend to view history,
particularly the history of racial inequality, the end of legal
discrimination, after 350 years, is all that was required to vanquish
this dark history and its effects. Changing the laws, too many Americans
have believed, marked the end of the obligation. But civil rights laws
passed in the 1960s merely guaranteed black people rights they should
have always had. They dictated that from that day forward, the
government would no longer sanction legal racial discrimination. But
these laws did not correct the harm nor restore what was lost.

Brown v. Board of Education did not end segregated and unequal schools;
it just ended segregation in the law. It took court orders and, at
times, federal troops to see any real integration. Nevertheless, more
than six decades after the nation's highest court proclaimed school
segregation unconstitutional, black children remain as segregated from
white kids as they were in the early 1970s. There has never been a point
in American history where even half the black children in this country
have attended a majority-white school.

Making school segregation illegal did nothing to repay black families
for the theft of their educations or make up for generations of black
Americans, many of them still living, who could never go to college
because white officials believed that only white students needed a high
school education and so refused to operate high schools for black
children. As late as the 1930s, most communities in the South, where the
vast majority of black Americans lived, failed to provide a single
public high school for black children, according to ``The Education of
Blacks in the South, 1860-1935,'' by the historian James D. Anderson.
Heavily black Richmond County in Georgia, for instance, did not provide
a four-year black high school from 1897 to 1945.

The Fair Housing Act prohibited discrimination in housing, but it did
not reset real estate values so that homes in redlined black
neighborhoods whose prices were artificially deflated would be valued
the same as identical homes in white neighborhoods, which had been
artificially inflated. It did not provide restitution for generations of
black homeowners forced into predatory loans because they had been
locked out of the prime credit market. It did not repay every black
soldier who returned from World War II to find that he could not use his
G.I. Bill to buy a home for his family in any of the new whites-only
suburbs subsidized by the same government he fought for. It did not
break up the still-entrenched housing segregation that took decades of
government and private policy to create. Lay those redlining maps over
almost any city in America with a significant black population, and you
will see that the government-sanctioned segregation patterns remain
stubbornly intact and that those same communities bore the brunt of the
predatory lending and foreclosure crisis of the late 2000s that stole
years of black homeownership and wealth gains.

Making employment discrimination illegal did not come with a check for
black Americans to compensate for all the high-paying jobs they were
legally barred from, for the promotions they never got solely because of
their race, for the income and opportunities lost to the centuries of
discrimination. Nor did these laws end ongoing discrimination any more
than speed limits without enforcement stop people from driving too fast.
These laws opened up opportunities for limited numbers of black
Americans while largely leaving centuries of meticulously orchestrated
inequities soundly in place, but now with the sheen of colorblind
magnanimity.

\begin{center}\rule{0.5\linewidth}{\linethickness}\end{center}

\textbf{The inclination to bandage over} and move on is a definitive
American feature when it comes to anti-black racism and its social and
material effects. A joint
\href{https://journals.sagepub.com/doi/full/10.1177/1745691619863049}{2019
study by faculty members at Yale University's School of Management,
Department of Psychology and Institute for Social and Policy Studies}
describes this phenomenon this way: ``A firm belief in our nation's
commitment to racial egalitarianism is part of the collective
consciousness of the United States of America. \ldots{} We have a strong
and persistent belief that our national disgrace of racial oppression
has been overcome, albeit through struggle, and that racial equality has
largely been achieved.'' The authors point out how white Americans love
to play up moments of racial progress like the Emancipation
Proclamation, Brown v. Board of Education and the election of Barack
Obama, while playing down or ignoring lynching, racial apartheid or the
\href{https://www.theguardian.com/us-news/2020/may/10/move-1985-bombing-reconciliation-philadelphia}{1985
bombing of a black neighborhood in Philadelphia.} ``When it comes to
race relations in the United States \ldots{} most Americans hold an
unyielding belief in a specific, optimistic narrative regarding racial
progress that is robust to counterexamples: that society has come a very
long way already and is moving rapidly, perhaps naturally toward full
racial equality.''

This remarkable imperviousness to facts when it comes to white advantage
and architected black disadvantage is what emboldens some white
Americans to quote the passage from Martin Luther King's 1963 ``I Have A
Dream'' speech about being judged by the content of your character and
not by the color of your skin. It's often used as a cudgel against calls
for race-specific remedies for black Americans --- while ignoring the
part of that same speech where King says black people have marched on
the capital to cash ``a check which has come back marked `insufficient
funds.'''

King has been evoked continuously during this season of protests,
sometimes to defend those who looted and torched buildings, sometimes to
condemn them. But in this time of foment, there has been an astounding
silence around his most radical demands. The seldom-quoted King is the
one who said that the true battle for equality, the actualization of
justice, required economic repair.

After watching Northern cities explode even as his movement's efforts to
pass the 1964 Civil Rights Act and the 1965 Voting Rights Act came to
fruition, King gave a speech in
\href{https://www.theatlantic.com/magazine/archive/2018/02/martin-luther-king-hungry-club-forum/552533/}{1967
in Atlanta before the Hungry Club Forum,} a secret gathering of white
politicians and civil rights leaders.

King said: ``For well now 12 years, the struggle was basically a
struggle to end legal segregation. In a sense it was a struggle for
decency. It was a struggle to get rid of all of the humiliation and the
syndrome of depravation surrounding the system of legal segregation. And
I need not remind you that those were glorious days. \ldots{} It is now
a struggle for genuine equality on all levels, and this will be a much
more difficult struggle. You see, the gains in the first period, or the
first era of struggle, were obtained from the power structure at bargain
rates; it didn't cost the nation anything to integrate lunch counters.
It didn't cost the nation anything to integrate hotels and motels. It
didn't cost the nation a penny to guarantee the right to vote. Now we
are in a period where it will cost the nation billions of dollars to get
rid of poverty, to get rid of slums, to make quality integrated
education a reality. This is where we are now. Now we're going to lose
some friends in this period. The allies who were with us in Selma will
not all stay with us during this period. We've got to understand what is
happening. Now they often call this the white backlash. \ldots{} It's
just a new name for an old phenomenon. The fact is that there has never
been any single, solid, determined commitment on the part of the vast
majority of white Americans to genuine equality for Negroes.''

A year later, in March 1968, just a month before his assassination, in a
speech to striking, impoverished
\href{http://www.nowcrj.org/wp-content/uploads/2015/01/King-Speech-Excerpts-1968-03-18-FINAL.pdf}{black
sanitation workers in Memphis,}King said: ``Now our struggle is for
genuine equality, which means economic equality. For we know that it
isn't enough to integrate lunch counters. What does it profit a man to
be able to eat at an integrated lunch counter if he doesn't have enough
money to buy a hamburger?''

\begin{center}\rule{0.5\linewidth}{\linethickness}\end{center}

\textbf{As we focus on police violence,} we cannot ignore an even
starker indication of our societal failures: Racial income disparities
today look no different than they did the decade before King's March on
Washington. In 1950,
\href{https://www.minneapolisfed.org/institute/working-papers-institute/iwp9.pdf}{according
to a forthcoming study} by the economists Moritz Schularick, Moritz Kuhn
and Ulrike Steins in The Journal of Political Economy, black median
household income was about half that of white Americans, and today it
remains so. More critical, the racial wealth gap is about the same as it
was in the 1950s as well. The typical black household today is poorer
than 80 percent of white households. ``No progress has been made over
the past 70 years in reducing income and wealth inequalities between
black and white households,'' according to the study.

And yet most Americans are in an almost pathological denial about the
depth of black financial struggle. That 2019 Yale University study,
called
\href{https://journals.sagepub.com/doi/full/10.1177/1745691619863049}{``The
Misperception of Racial Economic Inequality,''} found that Americans
believe that black households hold \$90 in wealth for every \$100 held
by white households. The actual amount is \$10.

About 97 percent of study participants overestimated black-white wealth
equality, and most assumed that highly educated, high-income black
households were the most likely to achieve economic parity with white
counterparts. That is also wrong. The magnitude of the wealth gap only
widens as black people earn more income.

``These data suggest that Americans are largely unaware of the striking
persistence of racial economic inequality in the United States,'' the
study's authors write. Americans, they write, tend to explain away or
justify persistent racial inequality by ignoring the ``tailwinds that
have contributed to their economic success while justifying inequalities
of wealth and poverty by invoking the role of individuals' traits and
skills as explanations for these disparities.'' They use the exceptional
examples of very successful black people to prove that systemic racism
does not hold black Americans back and point to the large numbers of
impoverished black people as evidence that black people are largely
responsible for their own struggles.

Jack Thornell/Associated Press

Memphis, 1968.

Jabin Botsford/Getty Images

New York, 2020.

In 2018, Duke University's Samuel DuBois Cook Center on Social Equity
and the Insight Center for Community Economic Development published
\href{https://socialequity.duke.edu/wp-content/uploads/2019/10/what-we-get-wrong.pdf}{a
report called ``What We Get Wrong About Closing the Racial Wealth Gap''}
that examined the common misperceptions about the causes of the racial
wealth gap and presented data and social-science research that refutes
them all.

The study shows that the racial wealth gap is not about poverty. Poor
white families earning less than \$27,000 a year hold nearly the same
amount of wealth as black families earning between \$48,000 and \$76,000
annually. It's not because of black spending habits. Black Americans
have lower incomes over all but save at a slightly higher rate than
white Americans with similar incomes. It's not that black people need to
value education more. Black parents, when controlling for household type
and socioeconomic status, actually offer more financial support for
their children's higher education than white parents do, according to
the study. And some studies have shown that black youths, when compared
with white youths whose parents have similar incomes and education
levels, are actually more likely to go to college and earn additional
credentials.

But probably most astounding to many Americans is that college simply
does not pay off for black Americans the way it does for other groups.
Black college graduates are about as likely to be unemployed as white
Americans with a high school diploma, and black Americans with a college
education hold less wealth than white Americans who have not even
completed high school. Further, because black families hold almost no
wealth to begin with, black students are the most likely to borrow money
to pay for college and then to borrow more. That debt, in turn, means
that black students cannot start saving immediately upon graduation like
their less-debt-burdened peers.

It's not a lack of homeownership. While it's true that black Americans
have the lowest homeownership rates in the nation, simply owning a home
is not the same asset that it is for white Americans. Black Americans
get higher mortgage rates even with equal credit worthiness, and homes
in black neighborhoods do not appreciate at the same rate as those in
white areas, because housing prices are still driven by the racial
makeup of communities. As the Duke University economist William Darity
Jr., the study's lead author, points out, the ability to purchase a home
in the first place is seldom a result of just the hard work and
frugality of the buyer. ``It's actually parental and grandparental
wealth that facilitates the acquisition of a home.''

It's not because a majority of black families are led by a single
mother. White single women with children hold the same amount of wealth
as single black women with no children, and the typical white single
parent has twice the wealth of the typical two-parent black family.

To summarize, none of the actions we are told black people must take if
they want to ``lift themselves'' out of poverty and gain financial
stability --- not marrying, not getting educated, not saving more, not
owning a home --- can mitigate 400 years of racialized plundering.
Wealth begets wealth, and white Americans have had centuries of
government assistance to accumulate wealth, while the government has for
the vast history of this country worked against black Americans doing
the same.

``The cause of the gap must be found in the structural characteristics
of the American economy, heavily infused at every point with both an
inheritance of racism and the ongoing authority of white supremacy,''
the authors of the Duke study write. ``There are no actions that black
Americans can take unilaterally that will have much of an effect on
reducing the wealth gap. For the gap to be closed, America must undergo
a vast social transformation produced by the adoption of bold national
policies.''

Underwood Archives/Getty Images

New York, 1917.

David Dee Delgado/Getty Images

New York, 2020.

\begin{center}\rule{0.5\linewidth}{\linethickness}\end{center}

\textbf{At the center of those policies} must be reparations. **** ``The
process of creating the racial wealth chasm begins with the failure to
provide the formerly enslaved with the 40 acres they were promised,''
Darity told me. ``So the restitution has never been given, and it's 155
years overdue.''

Darity has been studying and advocating reparations for 30 years, and
this spring he and his partner, A. Kirsten Mullen, published the book
``From Here to Equality: Reparations for Black Americans in the 21st
Century.'' Both history and road map, the book answers the questions
about who should receive reparations and how a program would work. I
will not spend much time on that here, except to make these few points.
Reparations are not about punishing white Americans, and white Americans
are not the ones who would pay for them. It does not matter if your
ancestors engaged in slavery or if you just immigrated here two weeks
ago. Reparations are a societal obligation in a nation where our
Constitution sanctioned slavery, Congress passed laws protecting it and
our federal government initiated, condoned and practiced legal racial
segregation and discrimination against black Americans until half a
century ago. And so it is the federal government that pays.

Reparations would go to any person who has documentation that he or she
identified as a black person for at least 10 years before the beginning
of any reparations process and can trace at least one ancestor back to
American slavery. Reparations should include a commitment to vigorously
enforcing existing civil rights prohibitions against housing,
educational and employment discrimination, as well as targeted
investments in government-constructed segregated black communities and
the segregated schools that serve a disproportionate number of black
children. But critically, reparations must include individual cash
payments to descendants of the enslaved in order to close the wealth
gap.

The technical details, frankly, are the easier part. The real obstacle,
the obstacle that we have never overcome, is garnering the political
will --- convincing enough Americans that the centuries-long forced
economic disadvantage of black Americans should be remedied, that
restitution is owed to people who have never had an equal chance to take
advantage of the bounty they played such a significant part in creating.

This country can be remarkably generous. Each year Congress allocates
money --- this year \$5 million ---
\href{https://www.govinfo.gov/content/pkg/CREC-2019-12-17/pdf/CREC-2019-12-17-house-bk3.pdf\#page=19}{to
help support Holocaust survivors living in America.} In backing the
funding measure, Representative Richard E. Neal, a Democrat from
Massachusetts,
\href{https://neal.house.gov/media-center/press-releases/statement-congressman-richard-neal-holocaust-remembrance-day}{said
in 2018} that this country has a ``responsibility to support the
surviving men and women of the Holocaust and their families.'' And he is
right. It is the moral thing to do. And yet Congress has refused for
three decades to
\href{https://www.congress.gov/bill/116th-congress/house-bill/40}{pass
H.R. 40,} a bill to simply study the issue of reparations. Its drafter,
Representative John Conyers Jr., a Michigan Democrat and descendant of
enslaved Americans, died in 2019 --- during
\href{https://www.nytimes3xbfgragh.onion/interactive/2019/08/14/magazine/1619-america-slavery.html?mtrref=www.google.com\&assetType=REGIWALL}{the
400th anniversary of the arrival of the first Africans enslaved in
Virginia} --- without the bill ever making it out of committee.

There are living victims of racial apartheid and terrorism born in
\emph{this} country, including civil rights activists who lost their
homes and jobs fighting to make this country a democracy, who have never
received any sort of restitution for what they endured. Soon, like their
enslaved ancestors, they will all be dead, too, and then we'll hear the
worn excuse that this country owes no reparations because none of the
victims are still alive. Darity and Mullen call this the ``delay until
death'' tactic. Procrastination, they say, does not erase what is owed.

The coronavirus pandemic has dispatched the familiar lament that even if
it is the right thing to do, this nation simply cannot afford to make
restitution to the 40 million descendants of American slavery. It took
Congress just a matter of weeks to pass a \$2.2 trillion stimulus bill
to help families and businesses struggling from the Covid-19 shutdowns.
When, then, will this nation pass a stimulus package to finally respond
to the singularity of black suffering?

Colossal societal ruptures have been the only things potent enough to
birth transformative racial change in this country, and perhaps a viral
pandemic colliding with our nation's 400-year racial one has forced that
type of rupture today. Maybe it had to be this way; this deep and
collective suffering was necessary for white Americans to feel enough of
the pain that black Americans have always known to tilt the scale.

With Covid-19, black Americans face a financial catastrophe unlike any
in nearly a century. Black Americans had already lost the largest share
of their wealth of all racial groups as a result of the last recession
and have struggled the most to recover. They are the only racial group
whose household median income is less than it was in 2000. Today already
more than half of black adults are out of work. Black businesses are
withering. Their owners were almost completely shut out of the federal
paycheck-protection program --- just 12 percent of black and Latino
business owners who applied for the small-business loans received the
full amounts they requested,
\href{http://publications.unidosus.org/bitstream/handle/123456789/2051/UnidosUS-Color-Of-Change-Federal-Simulus-Survey-Findings.pdf}{according
to a Global Strategy Group survey last month.} Nearly half the
respondents said they would most likely shutter permanently within six
months. Black children are expected to
\href{https://www.mckinsey.com/industries/public-sector/our-insights/covid-19-and-student-learning-in-the-united-states-the-hurt-could-last-a-lifetime}{lose
10 months' worth of academic gains} because of school closures, more
than any other group, and yet
\href{https://edbuild.org/content/23-billion}{they attend the schools
with the least resources already,} schools that will have even fewer
resources as states slash spending to make up for budget shortfalls.
\href{https://www.washingtonpost.com/business/2020/06/13/black-wealth-matters-homeownership-is-key/?arc404=true}{One
in five black homeowners and one in four renters} have missed at least
one home payment since the shutdowns began --- the highest of all racial
groups.

The pandemic, Keeanga-Yamahtta Taylor, a scholar of social movements and
racial inequality at Princeton University, told me, ``has pulled what is
hidden and buried on the bottom to the surface so that it can't actually
be ignored. It is a radicalizing factor because conditions that have
been so dire, now combined with revolts in the street, might lead one to
believe that not only is the society unraveling, but it might cause you
to question what foundation it was built upon in the first place.''

Race-neutral policies simply will not address the depth of disadvantage
faced by people this country once believed were chattel. Financial
restitution cannot end racism, of course, but it can certainly mitigate
racism's most devastating effects. If we do nothing, black Americans may
never recover from this pandemic, and they will certainly never know the
equality the nation has promised.

So we are left with a choice. Will this moment only feel different? Or
will it actually \emph{be} different?

If black lives are to truly matter in America, this nation must move
beyond slogans and symbolism. Citizens don't inherit just the glory of
their nation, but its wrongs too. A truly great country does not ignore
or excuse its sins. It confronts them and then works to make them right.
If we are to be redeemed, if we are to live up to the magnificent ideals
upon which we were founded, we must do what is just.

It is time for this country to pay its debt. It is time for reparations.

\begin{center}\rule{0.5\linewidth}{\linethickness}\end{center}

Nikole Hannah-Jones is a staff writer for the magazine. In 2020, she won
the Pulitzer Prize for commentary for her essay about
\href{https://www.nytimes3xbfgragh.onion/interactive/2019/08/14/magazine/black-history-american-democracy.html}{black
Americans and democracy}. She is the creator of
\href{https://www.nytimes3xbfgragh.onion/interactive/2019/08/14/magazine/1619-america-slavery.html}{The
1619 Project}, which won the National Magazine Award for public interest
and a George Polk special award this year. She is also a 2017 MacArthur
fellow.

Read More From the Magazine and Nikole Hannah-Jones

\href{https://www.nytimes3xbfgragh.onion/interactive/2019/08/14/magazine/1619-america-slavery.html}{}

\hypertarget{the-1619-projectaug-14-2019-1}{%
\paragraph{The 1619 ProjectAug. 14,
2019}\label{the-1619-projectaug-14-2019-1}}

\includegraphics{https://static01.graylady3jvrrxbe.onion/images/2019/08/18/magazine/18mag-1619landing-hp1/18mag-1619landing-hp1-mediumThreeByTwo210.jpg}
\href{https://www.nytimes3xbfgragh.onion/interactive/2019/08/14/magazine/black-history-american-democracy.html}{}

\hypertarget{america-wasnt-a-democracy-until-black-americans-made-it-oneaug-14-2019-1}{%
\paragraph{America Wasn't a Democracy, Until Black Americans Made It
OneAug. 14,
2019}\label{america-wasnt-a-democracy-until-black-americans-made-it-oneaug-14-2019-1}}

\includegraphics{https://static01.graylady3jvrrxbe.onion/images/2019/08/14/magazine/hpdemocracy/hpdemocracy-mediumThreeByTwo210.jpg}

Read 1158 Comments

\begin{itemize}
\item
\item
\item
\item
\end{itemize}

Advertisement

\protect\hyperlink{after-bottom}{Continue reading the main story}

\hypertarget{site-index}{%
\subsection{Site Index}\label{site-index}}

\hypertarget{site-information-navigation}{%
\subsection{Site Information
Navigation}\label{site-information-navigation}}

\begin{itemize}
\tightlist
\item
  \href{https://help.nytimes3xbfgragh.onion/hc/en-us/articles/115014792127-Copyright-notice}{©~2020~The
  New York Times Company}
\end{itemize}

\begin{itemize}
\tightlist
\item
  \href{https://www.nytco.com/}{NYTCo}
\item
  \href{https://help.nytimes3xbfgragh.onion/hc/en-us/articles/115015385887-Contact-Us}{Contact
  Us}
\item
  \href{https://www.nytco.com/careers/}{Work with us}
\item
  \href{https://nytmediakit.com/}{Advertise}
\item
  \href{http://www.tbrandstudio.com/}{T Brand Studio}
\item
  \href{https://www.nytimes3xbfgragh.onion/privacy/cookie-policy\#how-do-i-manage-trackers}{Your
  Ad Choices}
\item
  \href{https://www.nytimes3xbfgragh.onion/privacy}{Privacy}
\item
  \href{https://help.nytimes3xbfgragh.onion/hc/en-us/articles/115014893428-Terms-of-service}{Terms
  of Service}
\item
  \href{https://help.nytimes3xbfgragh.onion/hc/en-us/articles/115014893968-Terms-of-sale}{Terms
  of Sale}
\item
  \href{https://spiderbites.nytimes3xbfgragh.onion}{Site Map}
\item
  \href{https://help.nytimes3xbfgragh.onion/hc/en-us}{Help}
\item
  \href{https://www.nytimes3xbfgragh.onion/subscription?campaignId=37WXW}{Subscriptions}
\end{itemize}
