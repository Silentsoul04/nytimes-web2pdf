Sections

SEARCH

\protect\hyperlink{site-content}{Skip to
content}\protect\hyperlink{site-index}{Skip to site index}

\href{https://www.nytimes3xbfgragh.onion/section/us}{U.S.}

\href{https://myaccount.nytimes3xbfgragh.onion/auth/login?response_type=cookie\&client_id=vi}{}

\href{https://www.nytimes3xbfgragh.onion/section/todayspaper}{Today's
Paper}

\href{/section/us}{U.S.}\textbar{}Black Lives Matter May Be the Largest
Movement in U.S. History

\url{https://nyti.ms/2ZqRyOU}

\begin{itemize}
\item
\item
\item
\item
\item
\item
\end{itemize}

\href{https://www.nytimes3xbfgragh.onion/news-event/george-floyd-protests-minneapolis-new-york-los-angeles?action=click\&pgtype=Article\&state=default\&region=TOP_BANNER\&context=storylines_menu}{Race
and America}

\begin{itemize}
\tightlist
\item
  \href{https://www.nytimes3xbfgragh.onion/2020/07/26/us/protests-portland-seattle-trump.html?action=click\&pgtype=Article\&state=default\&region=TOP_BANNER\&context=storylines_menu}{Protesters
  Return to Other Cities}
\item
  \href{https://www.nytimes3xbfgragh.onion/2020/07/24/us/portland-oregon-protests-white-race.html?action=click\&pgtype=Article\&state=default\&region=TOP_BANNER\&context=storylines_menu}{Portland
  at the Center}
\item
  \href{https://www.nytimes3xbfgragh.onion/2020/07/23/podcasts/the-daily/portland-protests.html?action=click\&pgtype=Article\&state=default\&region=TOP_BANNER\&context=storylines_menu}{Podcast:
  Showdown in Portland}
\item
  \href{https://www.nytimes3xbfgragh.onion/interactive/2020/07/16/us/black-lives-matter-protests-louisville-breonna-taylor.html?action=click\&pgtype=Article\&state=default\&region=TOP_BANNER\&context=storylines_menu}{45
  Days in Louisville}
\end{itemize}

Advertisement

\protect\hyperlink{after-top}{Continue reading the main story}

\hypertarget{comments}{%
\subsection{\texorpdfstring{\protect\hyperlink{commentsContainer}{Comments}}{Comments}}\label{comments}}

\href{}{Black Lives Matter May Be the Largest Movement in U.S.
History}\href{}{Skip to Comments}

The comments section is closed. To submit a letter to the editor for
publication, write to
\href{mailto:letters@NYTimes.com}{\nolinkurl{letters@NYTimes.com}}.

\hypertarget{black-lives-matter-may-be-the-largest-movement-in-us-history}{%
\section{Black Lives Matter May Be the Largest Movement in U.S.
History}\label{black-lives-matter-may-be-the-largest-movement-in-us-history}}

By \href{https://www.nytimes3xbfgragh.onion/by/larry-buchanan}{Larry
Buchanan},
\href{https://www.nytimes3xbfgragh.onion/by/quoctrung-bui}{Quoctrung
Bui} and
\href{https://www.nytimes3xbfgragh.onion/by/jugal-k-patel}{Jugal K.
Patel}July 3, 2020

\begin{itemize}
\item
\item
\item
\item
\item
  \emph{60}
\end{itemize}

\hypertarget{black-lives-matter-protests-on-june-6}{%
\subsubsection{Black Lives Matter protests on June
6}\label{black-lives-matter-protests-on-june-6}}

Sources: Crowd Counting Consortium, Edwin Chow and New York Times
analysis \textbar{} Note: The Times partnered with Edwin Chow, an
associate professor at Texas State University, to count the protesters
based on available aerial images from June 6 and added those estimates
to data from the Crowd Counting Consortium. Counting efforts are still
ongoing, so the map is not comprehensive and totals shown are an average
of high and low estimates.

The recent Black Lives Matter protests peaked on June 6, when half a
million people turned out in nearly 550 places across the United States.
That was a single day in more than a month of protests that still
continue to today.

Four recent polls --- including one released this week by
\href{https://www.civisanalytics.com/}{Civis Analytics}, a data science
firm that works with businesses and Democratic campaigns --- suggest
that about 15 million to 26 million people in the United States have
participated in demonstrations over the death of George Floyd and others
in recent weeks.

These figures would make the recent protests the largest movement in the
country's history, according to interviews with scholars and
crowd-counting experts.

\hypertarget{number-of-people-in-us-who-said-they-protested-according-to-polls}{%
\subsubsection{Number of people in U.S. who said they protested,
according to
polls}\label{number-of-people-in-us-who-said-they-protested-according-to-polls}}

\begin{longtable}[]{@{}llll@{}}
\toprule
Poll & Pct. who protested & Implied population & Polling
period\tabularnewline
\midrule
\endhead
\href{https://www.kff.org/disparities-policy/report/kff-health-tracking-poll-june-2020/}{Kaiser
Family Foundation} (n = 1296) & 10\% & 26 million & June
8-14\tabularnewline
\href{https://www.civisanalytics.com/blog/data-science/coronavirus-pulse-survey-research/\#BLM}{Civis
Analytics} (4446) & 9\% & 23 million & June 12-22\tabularnewline
\href{http://www.apnorc.org/PDFs/AP-NORC\%20June\%202020/topline_release3.pdf}{N.O.R.C.}
(1310) & 7\% & 18 million & June 11-15\tabularnewline
\href{https://www.pewsocialtrends.org/2020/06/12/amid-protests-majorities-across-racial-and-ethnic-groups-express-support-for-the-black-lives-matter-movement/}{Pew}
(9654) & 6\% & 15 million & June 4-10\tabularnewline
\bottomrule
\end{longtable}

Note: Surveys are of the adult population in the United States

``I've never seen self-reports of protest participation that high for a
specific issue over such a short period,'' said Neal Caren, associate
professor at the University of North Carolina at Chapel Hill, who
studies social movements in the United States.

While it's possible that more people said they protested than actually
did, even if only half told the truth, the surveys suggest more than
seven million people participated in recent demonstrations.

The
\href{https://www.nytimes3xbfgragh.onion/2017/01/21/us/womens-march.html}{Women's
March of 2017} had a turnout of about three million to five million
people on a single day, but that was a highly organized event.
Collectively, the recent Black Lives Matter protests --- more organic in
nature --- appear to have far surpassed those numbers, according to
polls.

``Really, it's hard to overstate the scale of this movement,'' said Deva
Woodly, an associate professor of politics at the New School.

Professor Woodly said that the civil rights marches in the 1960s were
considerably smaller in number. ``If we added up all those protests
during that period, we're talking about hundreds of thousands of people,
but not millions,'' she said.

Even protests to unseat government leadership or for independence
typically succeed when they involve 3.5 percent of the population at
their peak, according to a review of
\href{https://carrcenter.hks.harvard.edu/files/cchr/files/CCDP_005.pdf}{international
protests} by Erica Chenoweth, a professor at Harvard Kennedy School who
co-directs the Crowd **** Counting Consortium, which collects data on
crowd sizes of political protests.

\hypertarget{why-this-movement-is-different}{%
\subsection{Why this movement is
different}\label{why-this-movement-is-different}}

Precise turnout at protests is difficult to count and has led to some
famous
\href{https://www.nytimes3xbfgragh.onion/1995/10/21/us/federal-parks-chief-calls-million-man-count-low.html}{disputes}.
An amalgam of estimates from organizers, the police and local news
reports often make up the official total.

But tallies by teams of crowd counters are revealing numbers of
extraordinary scale. On June 6, for example, at least 50,000 people
turned out in Philadelphia, 20,000 in Chicago's Union Park and up to
10,000 on the Golden Gate Bridge, according to estimates by Edwin Chow,
an associate professor at Texas State University, and researchers at the
Crowd Counting Consortium.

\hypertarget{philadelphia-on-june-6-2020-when-50000-to-80000-people-protested}{%
\subsubsection{Philadelphia on June 6, 2020, when 50,000 to 80,000
people
protested.}\label{philadelphia-on-june-6-2020-when-50000-to-80000-people-protested}}

Source: EarthCam

Across the United States, there have been more than 4,700
demonstrations, or an average of 140 per day, since the
\href{https://www.nytimes3xbfgragh.onion/article/george-floyd-protests-timeline.html}{first
protests began in Minneapolis} on May 26, according to a Times analysis.
Turnout has ranged from dozens to tens of thousands in about 2,500
\href{https://www.nytimes3xbfgragh.onion/interactive/2020/06/13/us/george-floyd-protests-cities-photos.html}{small
towns and large cities}.

500 protests

Protests against racism and

police violence per day

400

June 6

Juneteenth

300

200

100

0

May 31

Jun 7

Jun 14

Jun 21

Jun 28

500 protests

Protests against racism and

police violence per day

June 6

400

Juneteenth

300

200

100

0

May 31

Jun 7

Jun 14

Jun 21

Jun 28

Protests against racism and

police violence per day

500 protests

400

June 6

Juneteenth

300

200

100

May 31

Jun 7

Jun 14

Jun 21

Jun 28

Source: Crowd Counting Consortium

``The geographic spread of protest is a really important characteristic
and helps signal the depth and breadth of a movement's support,'' said
Kenneth Andrews, a sociology professor at the University of North
Carolina at Chapel Hill.

One of the reasons there have been protests in so many places in the
United States is the backing of organizations like Black Lives Matter.
While the group isn't necessarily directing each protest, it provides
materials, guidance and a framework for new activists, Professor Woodly
said. Those activists are taking to social media to quickly share
protest details to a wide audience.

Black Lives Matter has been around
\href{https://www.nytimes3xbfgragh.onion/2020/06/05/sunday-review/black-lives-matter-protests-floyd.html}{since
2013}, but there's been a
\href{https://www.nytimes3xbfgragh.onion/interactive/2020/06/10/upshot/black-lives-matter-attitudes.html}{big
shift in public opinion} about the movement as well as broader support
for recent protests. A deluge of public support from organizations like
the N.F.L. and NASCAR for Black Lives Matter may have also encouraged
supporters who typically would sit on the sidelines to get involved.

The protests may also be benefitting from a country that is more
conditioned to protesting. The adversarial stance that the Trump
administration has taken on issues like guns, climate change and
immigration has led to more protests than under any other presidency
since the Cold War.

According to a poll
\href{https://www.washingtonpost.com/news/national/wp/2018/04/06/feature/in-reaction-to-trump-millions-of-americans-are-joining-protests-and-getting-political/}{from
The Washington Post and the Kaiser Family Foundation}, one in five
Americans said that they had participated in a protest since the start
of the Trump administration, and 19 percent said they were new to
protesting.

\hypertarget{who-is-protesting}{%
\subsection{Who is protesting}\label{who-is-protesting}}

More than 40 percent of counties in the United States --- at least 1,360
--- have had a protest. Unlike with past Black Lives Matter protests,
nearly 95 percent of counties that had a protest recently are majority
white, and nearly three-quarters of the counties are more than 75
percent white.

Percentage of population that is white

in counties that had protests

Wash.

0

50

75

90

100\%

Me.

Mont.

N.D.

Vt.

Ore.

Minn.

N.H.

Idaho

Mass.

S.D.

N.Y.

Wis.

Mich.

R.I.

Wyo.

Conn.

Iowa

Pa.

N.J.

Neb.

Nev.

Ohio

Md.

Del.

Ill.

Ind.

Utah

W.Va.

Colo.

Va.

Calif.

Kan.

Mo.

Ky.

N.C.

Tenn.

Okla.

Ariz.

N.M.

S.C.

ARK.

Ga.

MISS.

ALA.

TEX.

LA.

Alaska

FLA.

HAWAII

Percentage of population that is white

in counties that had protests

Wash.

0

50

75

90

100\%

Me.

Mont.

N.D.

Vt.

Ore.

Minn.

N.H.

Idaho

Mass.

S.D.

N.Y.

Wis.

Mich.

R.I.

Wyo.

Conn.

Iowa

Pa.

N.J.

Neb.

Nev.

Ohio

Md.

Del.

Ill.

Ind.

Utah

W.Va.

Colo.

Va.

Calif.

Kan.

Mo.

Ky.

N.C.

Tenn.

Okla.

Ariz.

N.M.

S.C.

ARK.

Ga.

MISS.

ALA.

TEX.

LA.

Alaska

FLA.

HAWAII

Percentage of population that is white

in counties that had protests

0

50

75

90

100\%

Wash.

Me.

Mont.

N.D.

Vt.

Ore.

Minn.

N.H.

Idaho

Mass.

S.D.

N.Y.

Wis.

Mich.

R.I.

Wyo.

Conn.

Iowa

Pa.

N.J.

Neb.

Nev.

Ohio

Md.

Del.

Ill.

Ind.

Utah

W.Va.

Colo.

Va.

Calif.

Kan.

Mo.

Ky.

N.C.

Tenn.

Okla.

Ariz.

N.M.

S.C.

ARK.

Ga.

MISS.

ALA.

TEX.

LA.

Alaska

FLA.

HAWAII

Percentage of population that is white

in counties that had protests

0

50

75

90

100\%

The New York Times·Source: 2018 Census via Social Explorer; Crowd
Counting Consortium protests database; New York Times protests database

``Without gainsaying the reality and significance of generalized white
support for the movement in the early 1960s, the number of whites who
were active in a sustained way in the struggle were comparatively few,
and certainly nothing like the percentages we have seen taking part in
recent weeks,'' said Douglas McAdam, an emeritus professor at Stanford
University who studies social movements.

According to the Civis Analytics poll, the movement appears to have
attracted protesters who are younger and wealthier. The age group with
the largest share of protesters was people under 35 and the income group
with the largest share of protesters was those earning more than
\$150,000.

Half of those who said they protested said that this was their first
time getting involved with a form of activism or demonstration. A
majority said that they watched a video of police violence toward
protesters or the Black community within the last year. And of those
people, half said that it made them more supportive of the Black Lives
Matter movement.

The protests are colliding with another watershed moment: the country's
most devastating pandemic in modern history.

``With being home and not being able to do as much, that might be
amplifying something that is already sort of critical, something that's
already a powerful catalyst, and that is the video,'' said Daniel Q.
Gillion, a professor at the University of Pennsylvania who has written
several books on protests and politics.

``If you aren't moved by the George Floyd video, you have nothing in
you,'' he said. ``And that catalyst can now be amplified by the fact
that individuals probably have more time to engage in protest
activity.''

Besides the spike in demonstrations on
\href{https://www.nytimes3xbfgragh.onion/interactive/2020/06/18/style/juneteenth-celebration.html}{Juneteeth},
the number of protests has fallen considerably over the last two weeks
according to the Crowd Counting Consortium.

But the amount of change that the protests have been able to produce in
such a short period of time is significant. In Minneapolis, the City
Council pledged to
\href{https://www.nytimes3xbfgragh.onion/2020/06/07/us/minneapolis-police-abolish.html}{dismantle}
its police department. In New York, lawmakers
\href{https://www.nytimes3xbfgragh.onion/2020/06/12/nyregion/50a-repeal-police-floyd.html}{repealed}
a law that kept police disciplinary records secret. Cities and
\href{https://www.desmoinesregister.com/story/news/politics/2020/06/12/police-misconduct-chokehold-law-governor-kim-reynolds-sign-black-lives-matter-george-floyd/5347514002/}{states}
across the country passed new laws banning chokeholds. Mississippi
lawmakers
\href{https://www.nytimes3xbfgragh.onion/2020/06/28/us/mississippi-flag-confederacy.html?action=click\&pgtype=Article\&state=default\&module=styln_george_floyd_protests_keepup\&variant=1_show\&region=body\&context=keep_up}{voted
to retire their state flag}, which prominently includes a Confederate
battle emblem.

``It looks, for all the world, like these protests are achieving what
very few do: setting in motion a period of significant, sustained, and
widespread social, political change,'' Professor McAdam said. ``We
appear to be experiencing a social change tipping point --- that is as
rare in society as it is potentially consequential.''

Bedel Saget and Anjali Singhvi contributed reporting.

Read 60 Comments

\begin{itemize}
\item
\item
\item
\item
\end{itemize}

Advertisement

\protect\hyperlink{after-bottom}{Continue reading the main story}

\hypertarget{site-index}{%
\subsection{Site Index}\label{site-index}}

\hypertarget{site-information-navigation}{%
\subsection{Site Information
Navigation}\label{site-information-navigation}}

\begin{itemize}
\tightlist
\item
  \href{https://help.nytimes3xbfgragh.onion/hc/en-us/articles/115014792127-Copyright-notice}{©~2020~The
  New York Times Company}
\end{itemize}

\begin{itemize}
\tightlist
\item
  \href{https://www.nytco.com/}{NYTCo}
\item
  \href{https://help.nytimes3xbfgragh.onion/hc/en-us/articles/115015385887-Contact-Us}{Contact
  Us}
\item
  \href{https://www.nytco.com/careers/}{Work with us}
\item
  \href{https://nytmediakit.com/}{Advertise}
\item
  \href{http://www.tbrandstudio.com/}{T Brand Studio}
\item
  \href{https://www.nytimes3xbfgragh.onion/privacy/cookie-policy\#how-do-i-manage-trackers}{Your
  Ad Choices}
\item
  \href{https://www.nytimes3xbfgragh.onion/privacy}{Privacy}
\item
  \href{https://help.nytimes3xbfgragh.onion/hc/en-us/articles/115014893428-Terms-of-service}{Terms
  of Service}
\item
  \href{https://help.nytimes3xbfgragh.onion/hc/en-us/articles/115014893968-Terms-of-sale}{Terms
  of Sale}
\item
  \href{https://spiderbites.nytimes3xbfgragh.onion}{Site Map}
\item
  \href{https://help.nytimes3xbfgragh.onion/hc/en-us}{Help}
\item
  \href{https://www.nytimes3xbfgragh.onion/subscription?campaignId=37WXW}{Subscriptions}
\end{itemize}
