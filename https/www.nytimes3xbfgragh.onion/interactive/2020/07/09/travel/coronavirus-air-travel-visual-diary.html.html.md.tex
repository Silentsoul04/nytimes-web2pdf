\hypertarget{comments}{%
\subsection{\texorpdfstring{\protect\hyperlink{commentsContainer}{Comments}}{Comments}}\label{comments}}

\href{}{How the Coronavirus Has Changed Air Travel: A Visual Diary of a
Flight}\href{}{Skip to Comments}

The comments section is closed. To submit a letter to the editor for
publication, write to
\href{mailto:letters@NYTimes.com}{\nolinkurl{letters@NYTimes.com}}.

Asset 9

\hypertarget{how-the-coronavirus-has-changed-air-travel-a-visual-diary-of-a-flight}{%
\section{How the Coronavirus Has Changed Air Travel: A Visual Diary of a
Flight}\label{how-the-coronavirus-has-changed-air-travel-a-visual-diary-of-a-flight}}

\hypertarget{by-moris-moreno-july-9-2020}{%
\subparagraph{By Moris Moreno July 9,
2020}\label{by-moris-moreno-july-9-2020}}

In late June, I flew with my family from Seattle, where I work as an
architectural photographer, to Boston. Here's a step-by-step account of
my journey.

This account is anecdotal, and is not necessarily representative of what
other air travelers have experienced. Still, it may be helpful to see
how certain security and passenger protocols have changed --- including
the use of masks, protective panels and social-distancing measures ---
amid the coronavirus pandemic.

Much of what I experienced felt both ordinary and slightly surreal.

At my point of departure, Seattle Tacoma International Airport, the
check-in kiosks were continually cleaned.

Passengers --- most of whom wore masks or face shields --- were guided
by social distancing decals placed on the floor at the luggage drop-off
line.

Floor decals were ubiquitous.

The T.S.A. screening line was short. Signs and stickers reminded us to
practice social distancing.

On the day of my flight, June 29th, the T.S.A. screened 625,235
passengers, down from 2,455,536 passengers on the same day last year.

Signs were tailored to their Pacific Northwest audience.

At the T.S.A. checkpoints, officers wore masks and gloves and were
separated from passengers by a plexiglass panel.

After showing their boarding passes and handing over their IDs,
passengers were asked to lower their masks to help confirm their
identity.

At some checkpoints, the T.S.A. is using a new authentication system to
verify travelers' identification and flight status. (Among other things,
the new system is better at detecting fraudulent IDs.)

Passenger and luggage screening procedures were completed while
maintaining proper social distancing measures.

If carry-on luggage required a second inspection, T.S.A. officers
inspected the bags by hand.

Advertisement

Skip Advertisement

Past the security line, few stores and restaurants were open.

After first class and priority boarding, passengers boarded the plane
from back to front.

The plane was about half full. Middle seats were left empty, though
couples and families were free to sit directly beside each other.

Passengers were required to wear masks throughout the flight. (From what
I saw, everyone was compliant.)

I flew with Alaska Airlines, and my onboard experience was, of course,
specific to my flight. In recent weeks, travelers on other carriers have
encountered full or nearly full aircraft.

In-flight service was reduced, and many passengers brought their own
food.

After the five-hour flight, passengers were anxious to deplane ...

... and crowded each other to gather their bags.

At my destination, Boston Logan International Airport, the social
distancing reminders continued.

Some travelers seemed to be vigilant about (properly) wearing their
masks or face shields, but I spotted others who weren't wearing them.

Floor decals indicated where passengers should stand while waiting for
their luggage ...

... but certain areas lacked decals, and some travelers crowded
together.

The shuttle to the car-rental area was nearly empty ...

... as was the car-rental area itself.

Overall, the flight felt fairly safe, in large part because of the extra
space between passengers. Still, an experience that had once been
routine now felt like an ordeal, and I was relieved to be on my way.

Moris Moreno is an architectural photographer who lives in Seattle.

Produced by Stephen Hiltner

More Stories Like This
