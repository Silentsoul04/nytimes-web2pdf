Sections

SEARCH

\protect\hyperlink{site-content}{Skip to
content}\protect\hyperlink{site-index}{Skip to site index}

\hypertarget{battle-in-the-himalayas}{%
\section{Battle in the Himalayas}\label{battle-in-the-himalayas}}

By \href{https://www.nytimes3xbfgragh.onion/by/jin-wu}{Jin Wu} and
\href{https://www.nytimes3xbfgragh.onion/by/steven-lee-myers}{Steven Lee
Myers}July 18, 2020

\begin{itemize}
\item
\item
\item
\item
\end{itemize}

China and India are locked in a tense, deadly struggle for advantage on
their disputed mountain border.

China and India have stumbled once again into
\href{https://www.nytimes3xbfgragh.onion/2020/06/16/world/asia/indian-china-border-clash.html}{a
bloody clash} over some of the most inhospitable terrain on Earth.

A deadly brawl last month killed 20 Indian border troops and an unknown
number of Chinese soldiers, punctuating a decades-old border dispute
that has become one of the world's most intractable geopolitical
conflicts. It has inflamed tensions at a time when the world is consumed
by the coronavirus pandemic, and it has scuttled
\href{https://www.nytimes3xbfgragh.onion/2019/10/11/world/asia/narendra-modi-xi-jinping-india-china.html}{recent
efforts} by the two Asian powers to set aside their historical
differences.

In the weeks since, the two sides have tried to walk back from the
brink, with military commanders and senior diplomats negotiating quietly
to disengage. By late last week, satellite photographs indicated that
Chinese troops had pulled out of one disputed area where a brawl sparked
the latest tensions.

Even so, the broader dispute between the world's two most populous
nations, both armed with nuclear weapons, remains unresolved and
dangerous. It involves a region called Ladakh, a sparsely populated
area, high in the Himalayas, with close historical and cultural ties to
Tibet. It was divided in the years after India gained independence from
Britain in 1947 and the Communist Party established the People's
Republic of China two years later.

Xinjiang

Disputed

borders

china

Line of Actual Control

(approximate)

Highway 219

connecting Xinjiang

and Tibet

Daulat Beg Oldi

Gilgit-Baltistan

Controlled by

Pakistan

Aksai Chin

Controlled by China,

claimed by India

The all-weather DSDBO Road connects India's remote military camp to the
center of Ladakh.

Galwan Valley

Line of Control between India and Pakistan

Tibet

Leh

china

Pangong Lake

Ladakh

Area controlled by India

CHINA

Area of detail

INDIA

India

Bay of

Bengal

Arabian

Sea

Highway 219

connecting Xinjiang

and Tibet

Disputed

borders

china

Line of Actual Control

(approximate)

Daulat Beg Oldi

The all-weather DSDBO Road connects India's remote military camp to the
center of Ladakh.

Aksai Chin

Controlled by China,

claimed by India

Galwan Valley

Tibet

Leh

china

Pangong Lake

Ladakh

Area controlled by India

CHINA

Area of detail

INDIA

India

During its invasion of Tibet in 1950, Mao Zedong's China seized the
northern part of Ladakh, called Aksai Chin, and has held it ever since
--- in no small part because a crucial road connecting Tibet with
another restive province, Xinjiang, runs through it. In 1962, the two
countries went to war over the same terrain, but despite an overwhelming
Chinese victory, the de facto frontier --- known as the Line of Actual
Control --- remained roughly the same.

The clashes this spring and summer stemmed from India's recent efforts
to build up the road network on its side of the frontier, catching up
--- belatedly, critics say --- to China's buildup on its side. Last
year, India completed an all-weather road connecting Leh, the capital of
Ladakh, to its northernmost outpost at Daulat Beg Oldi. In the last two
decades, India has constructed nearly 5,000 kilometers of roads,
allowing it to move military forces more easily along the mountainous
border region.

China appeared alarmed by that and by India's
\href{https://www.nytimes3xbfgragh.onion/2019/08/05/world/asia/india-pakistan-kashmir-jammu.html}{decision
last year} to impose direct national rule over the Ladakh region.

"China is very sensitive to Indian activity in the western sector,''
said M. Taylor Fravel, director of the Security Studies Program at the
Massachusetts Institute of Technology, ``and it goes back to the reasons
why it decided to fight in 1962 --- to defend that road that connected
Xinjiang to Tibet.''

CHINA

Galwan Valley

Leads to Daulat Beg Oldi,

India's military base in Ladakh

Line of Actual Control

(Approximate)

More than 1,800 meters

above the valley

(Elev: 6,000+ meters)

INDIA

Shyok

River

Galwan Valley

Line of Actual Control

(Approximate)

CHINA

Leads to Daulat Beg Oldi,

India's military base in Ladakh

More than 1,800 meters

above the valley

(Elev: 6,000+ meters)

INDIA

Shyok

River

CHINA

Galwan Valley

Line of Actual Control

(Approximate)

Leads to Daulat Beg Oldi,

India's military base in Ladakh

INDIA

Shyok

River

CHINA

Location of the deadly clash

on June 15

Galwan Valley

Shyok River

INDIA

CHINA

Location of the deadly clash

on June 15

Galwan Valley

Shyok River

INDIA

CHINA

Location of the deadly

clash on June 15

Galwan Valley

INDIA

Shyok River

Shyok River

Galwan River

Line of Actual Control

(approximate)

Indian military

installments

Where China claims

its sovereignty

Location of the

clash on June 15

Line of Actual Control

(approximate)

Shyok

River

Where China claims

its sovereignty

Location of the

clash on June 15

Galwan River

Indian military

installments

Line of Actual Control

(approximate)

Shyok River

Where China claims

its sovereignty

Galwan River

Location of the

clash on June 15

Indian military

installments

\begin{itemize}
\item
\item
\item
\item
\end{itemize}

This disputed land near Galwan Valley has some of the most treacherous
terrain on Earth. While no border has ever officially been negotiated
along the forbidding stretch of land high in the Himalayas that divides
the two nations, the truce established a 2,100-mile-long Line of Actual
Control.

On the night of June 15, Indian and Chinese soldiers clashed at an area
near a sharp bend in the Galwan River, where Chinese forces had set up
tents. It was the first deadly clash on the border since 1975 and the
deadliest since 1967.

Indian officials have claimed that China was moving farther down the
Galwan River than it had in the past. By occupying the valley, the
Chinese could easily monitor Indian vehicles passing through on the main
road.

A spokesman for China's Ministry of National Defense, Senior Col. Wu
Qian, said last month that China has sovereignty over the entire valley
to the point where the Galwan and Shyok rivers meet. He blamed Indian
troops for crossing into Chinese territory. ``The responsibility lies
entirely with India,'' he said.

Galwan Valley is not the only hotspot along the frontier. By late April
and early May, Indian troops began to observe a buildup of Chinese
forces at two other spots along the Line of Actual Control: Pangong Lake
and Hot Springs.

While no clashes occurred in Hot Springs, the Chinese brought up
significant weaponry. About three kilometers away from the Line of
Actual Control, companies of tanks and batteries of towed artillery
appeared in existing Chinese positions north and east of Gogra.

Tanks

Artillery batteries

Tanks

Artillery batteries

Sources: Satellite image taken by Maxar Technologies on May 22, 2020;
Henry Boyd and Meia Nouwens, International Institute for Strategic
Studies.

The tensions this year first boiled over on the northern shore of
Pangong Lake, a glacial lake split by the de facto border.

In early May, troops from both countries brawled in disputed territory
there. There were a number of injuries, some serious, though no deaths.
That fight put both sides on edge, contributing to the deadly clash in
the Galwan Valley a little more than a month later. Years ago, the two
countries agreed that their troops should not shoot at each other during
border standoffs. But China seems to be testing the limits. In the June
fighting, Indian commanders said that Chinese troops used iron clubs
bristling with spikes.

LADAKH

Area controlled by India

1

2

AKSAI CHIN

Controlled by China

Claimed by India

3

Pangong Lake

4

7

Sirijap

8

5

6

India claims territories

up to Finger 8.

Line of Actual Control

(Approximate)

Chushul

Sirijap

AKSAI CHIN

Controlled by China

Claimed by India

8

7

6

India claims territories

up to Finger 8.

5

4

LADAKH

Area controlled by India

3

Line of Actual Control

(Approximate)

2

1

Pangong Lake

AKSAI CHIN

Controlled by China

Claimed by India

Sirijap

India claims territories

up to Finger 8.

8

7

LADAKH

Area controlled by India

6

5

Line of Actual Control

(Approximate)

4

3

2

1

Pangong Lake

Chinese posts

Indian posts

Pangong Lake

Chinese posts

Line of Actual Control

(Approximate)

Chinese posts

Indian posts

Chinese posts

Pangong Lake

Chinese posts

Indian posts

Chinese posts

Pangong Lake

Indian military

settlements

Helicopter pads

Many more tents were seen on

satellite images captured on July 10,

compared to one month before.

Bridge drainage

Many more tents were seen on

satellite images captured on July 10,

compared to one month before.

Indian military

settlements

Bridge drainage

Helicopter pads

Many more tents were seen on

satellite images captured on July 10,

compared to one month before.

Indian military

settlements

Bridge drainage

Helicopter pads

LADAKH

Controlled by India

AKSAI CHIN

Controlled by China

Claimed by India

Trucks can be seen coming

from other camps

Construction activities

by the Chinese forces

Line of Actual Control

(Approximate)

Line of Actual Control

(Approximate)

Construction activities

by the Chinese forces

AKSAI CHIN

Controlled by China

Claimed by India

LADAKH

Controlled by India

Line of Actual Control

(Approximate)

Construction activities

by the Chinese forces

AKSAI CHIN

Controlled by China

Claimed by India

LADAKH

Controlled by India

LADAKH

Area controlled by India

AKSAI CHIN

Controlled by China

Claimed by India

No more clear

Chinese constructions

Line of Actual Control

(Approximate)

Line of Actual Control

(Approximate)

No more clear

Chinese constructions

AKSAI CHIN

Controlled by China

Claimed by India

LADAKH

Controlled by India

Line of Actual Control

(Approximate)

No more clear

Chinese constructions

AKSAI CHIN

Controlled by China

Claimed by India

LADAKH

Controlled by India

Chinese military

settlements

Tents spread out in this area

Multiple roads have

been constructed

A map of China has been inscribed on

the disputed banks of Pangong Lake.

Interceptor craft

A map of China has been inscribed on

the disputed banks of Pangong Lake.

Interceptor craft

Tents spread out in this area

Multiple roads have

been constructed

Chinese military

settlements

A map of China has been inscribed on

the disputed banks of Pangong Lake.

Tents spread out in this area

Multiple roads have

been constructed

Chinese military

settlements

June 26, 2020

July 10, 2020

\begin{itemize}
\item
\item
\item
\item
\item
\item
\end{itemize}

This is Pangong Lake, where the slopes of the mountains jut into the
lake from eight directions, referred to as the `fingers.' India and
China have different interpretations of where exactly the Line of Actual
Control passses.

While military personnel patrol most of the areas by foot, there are
several military settlements built along the bank. The first standoff
this year occurred here on May 5.

India's most advanced post in this region was located at Finger 3, which
is well connected by a road from deeper within its territory.

Satellite images taken on June 26 showed construction activity by the
Chinese in this region.

But in images captured on July 10, the Chinese positions have thinned
out, after a troop withdrawal.

Despite the partial withdrawal, Chinese forces continue to dominate the
spurs in this region.

China's actions in the Himalayas have mirrored similar efforts to assert
or reinforce its territorial claims, especially in the South China Sea.
Chinese warships have this year menaced fishing and research vessels
from Vietnam, Malaysia and Indonesia. In recent weeks, China is reported
to have expanded its territorial claims
\href{https://thediplomat.com/2020/07/whats-behind-chinas-expansion-of-its-territorial-dispute-with-bhutan/}{in
Bhutan}, which has a close defense relationship with India.

Some analysts have argued that China is acting while the world is
distracted by the coronavirus pandemic; others say China needs to
distract its own population with nationalist propaganda about defending
Chinese sovereignty. In any case, the tensions are unlikely to diminish.

Additional reporting by Jeffrey Gettleman. Additional work by Josh
Williams and Anjali Singhvi.

Satellite images are from CNES/Airbus, Maxar Technologies via Google
Earth Studio, and Planet Labs.

\begin{itemize}
\item
\item
\item
\item
\end{itemize}

Advertisement

\protect\hyperlink{after-bottom}{Continue reading the main story}

\hypertarget{site-index}{%
\subsection{Site Index}\label{site-index}}

\hypertarget{site-information-navigation}{%
\subsection{Site Information
Navigation}\label{site-information-navigation}}

\begin{itemize}
\tightlist
\item
  \href{https://help.nytimes3xbfgragh.onion/hc/en-us/articles/115014792127-Copyright-notice}{©~2020~The
  New York Times Company}
\end{itemize}

\begin{itemize}
\tightlist
\item
  \href{https://www.nytco.com/}{NYTCo}
\item
  \href{https://help.nytimes3xbfgragh.onion/hc/en-us/articles/115015385887-Contact-Us}{Contact
  Us}
\item
  \href{https://www.nytco.com/careers/}{Work with us}
\item
  \href{https://nytmediakit.com/}{Advertise}
\item
  \href{http://www.tbrandstudio.com/}{T Brand Studio}
\item
  \href{https://www.nytimes3xbfgragh.onion/privacy/cookie-policy\#how-do-i-manage-trackers}{Your
  Ad Choices}
\item
  \href{https://www.nytimes3xbfgragh.onion/privacy}{Privacy}
\item
  \href{https://help.nytimes3xbfgragh.onion/hc/en-us/articles/115014893428-Terms-of-service}{Terms
  of Service}
\item
  \href{https://help.nytimes3xbfgragh.onion/hc/en-us/articles/115014893968-Terms-of-sale}{Terms
  of Sale}
\item
  \href{https://spiderbites.nytimes3xbfgragh.onion}{Site Map}
\item
  \href{https://help.nytimes3xbfgragh.onion/hc/en-us}{Help}
\item
  \href{https://www.nytimes3xbfgragh.onion/subscription?campaignId=37WXW}{Subscriptions}
\end{itemize}
