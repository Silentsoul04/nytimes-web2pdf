Sections

SEARCH

\protect\hyperlink{site-content}{Skip to
content}\protect\hyperlink{site-index}{Skip to site index}

\href{https://www.nytimes3xbfgragh.onion/section/us}{U.S.}

\href{https://myaccount.nytimes3xbfgragh.onion/auth/login?response_type=cookie\&client_id=vi}{}

\href{https://www.nytimes3xbfgragh.onion/section/todayspaper}{Today's
Paper}

\href{/section/us}{U.S.}\textbar{}More Than 6,600 Coronavirus Cases Have
Been Linked to U.S. Colleges

\url{https://nyti.ms/3f9KEU0}

\begin{itemize}
\item
\item
\item
\item
\item
\item
\end{itemize}

\hypertarget{schools-reopening}{%
\subsubsection{\texorpdfstring{\href{https://www.nytimes3xbfgragh.onion/spotlight/schools-reopening?name=styln-coronavirus-schools-reopening\&region=TOP_BANNER\&variant=undefined\&block=storyline_menu_recirc\&action=click\&pgtype=Interactive\&impression_id=36459b80-e3a2-11ea-974d-d37311179bca}{Schools
Reopening}}{Schools Reopening}}\label{schools-reopening}}

\begin{itemize}
\tightlist
\item
  \href{https://www.nytimes3xbfgragh.onion/2020/08/19/us/colleges-closing-covid.html?name=styln-coronavirus-schools-reopening\&region=TOP_BANNER\&variant=undefined\&block=storyline_menu_recirc\&action=click\&pgtype=Interactive\&impression_id=36459b81-e3a2-11ea-974d-d37311179bca}{Colleges
  Closing}
\item
  \href{https://www.nytimes3xbfgragh.onion/2020/08/20/us/schools-reopening-nurses-covid.html?name=styln-coronavirus-schools-reopening\&region=TOP_BANNER\&variant=undefined\&block=storyline_menu_recirc\&action=click\&pgtype=Interactive\&impression_id=36459b82-e3a2-11ea-974d-d37311179bca}{Missing
  School Nurses}
\item
  \href{https://www.nytimes3xbfgragh.onion/2020/08/18/parenting/homeschool-families.html?name=styln-coronavirus-schools-reopening\&region=TOP_BANNER\&variant=undefined\&block=storyline_menu_recirc\&action=click\&pgtype=Interactive\&impression_id=3645c290-e3a2-11ea-974d-d37311179bca}{Home-Schooling
  Families}
\item
  \href{https://www.nytimes3xbfgragh.onion/2020/08/05/parenting/parents-distance-learning.html?name=styln-coronavirus-schools-reopening\&region=TOP_BANNER\&variant=undefined\&block=storyline_menu_recirc\&action=click\&pgtype=Interactive\&impression_id=3645c291-e3a2-11ea-974d-d37311179bca}{Prepare
  for Distance Learning}
\end{itemize}

Advertisement

\protect\hyperlink{after-top}{Continue reading the main story}

\hypertarget{comments}{%
\subsection{\texorpdfstring{\protect\hyperlink{commentsContainer}{Comments}}{Comments}}\label{comments}}

\href{}{More Than 6,600 Coronavirus Cases Have Been Linked to U.S.
Colleges}\href{}{Skip to Comments}

The comments section is closed. To submit a letter to the editor for
publication, write to
\href{mailto:letters@NYTimes.com}{\nolinkurl{letters@NYTimes.com}}.

\hypertarget{more-than-6600-coronavirus-cases-have-been-linked-to-us-colleges}{%
\section{More Than 6,600 Coronavirus Cases Have Been Linked to U.S.
Colleges}\label{more-than-6600-coronavirus-cases-have-been-linked-to-us-colleges}}

By \href{https://www.nytimes3xbfgragh.onion/by/weiyi-cai}{Weiyi Cai},
\href{https://www.nytimes3xbfgragh.onion/by/danielle-ivory}{Danielle
Ivory}, \href{https://www.nytimes3xbfgragh.onion/by/mitch-smith}{Mitch
Smith}, Alex Lemonides and Lauryn HigginsJuly 29, 2020

\begin{itemize}
\item
\item
\item
\item
\item
  \emph{407}
\end{itemize}

As college students and professors decide whether to head back to class,
and as universities weigh how and whether to reopen, the coronavirus is
already on campus.

A New York Times survey of every public four-year college in the
country, as well as every private institution that competes in Division
I sports or is a member of an
\href{https://www.aau.edu/sites/default/files/AAU-Files/Who-We-Are/AAU-Member-List.pdf}{elite
group of research universities}, revealed at least 6,600 cases tied to
about 270 colleges over the course of the pandemic. And the new academic
year has not even begun at most schools.

\hypertarget{confirmed-coronavirus-cases-on-college-campuses}{%
\subsubsection{Confirmed coronavirus cases on college
campuses}\label{confirmed-coronavirus-cases-on-college-campuses}}

More than 50 cases

11-50 cases

4-10 cases

Fewer than 4 cases

Note: Data as of July 28.

Outbreaks have emerged on Greek Row this summer at the University of
Washington, where at least 136 residents were infected, and at
Harris-Stowe State University in St. Louis, where administrators were
re-evaluating their plans for fall after eight administrative workers
tested positive.

The virus has turned up in a science building at Western Carolina, on
the football team at Clemson and among employees at the University of
Denver.

At Appalachian State in North Carolina, at least 41 construction workers
have tested positive while working on campus buildings. The Times has
identified at least 14 coronavirus-related deaths at colleges.

\hypertarget{search-for-a-school}{%
\subsubsection{Search for a school}\label{search-for-a-school}}

The list includes public, four-year universities in the United States,
as well as private colleges that compete in Division I sports or are
members of an elite group of \href{https://www.aau.edu/}{research
universities}. Only schools that reported cases are shown.

Collapse list

\hypertarget{school}{%
\subsection{School}\label{school}}

Cases

Location

Weekly local cases per capita

Fewer

More

Show all

*All reported cases were in the athletic department.\\
Note: The charts show the cases per 100,000 residents reported each week
in the county where each school is located. The location of a
university's main campus is listed unless otherwise specified. In
several instances, colleges noted that some cases were tied to branch
campuses or satellite locations. Universities with no case total listed
either did not respond to inquiries, declined to provide information or
said they had no known infections.

There is no standardized reporting method for coronavirus cases and
deaths at colleges, **** and the information is not being publicly
tracked at a national level. Of nearly 1,000 institutions contacted by
The Times, some had already posted case information online, some
provided full or partial numbers and others refused to answer basic
questions, citing privacy concerns. Hundreds of colleges did not respond
at all.

Still, the Times survey represents the most comprehensive look at the
toll the virus has already taken on the country's colleges and
universities.

Coronavirus infections on campuses might go unnoticed if not for
reporting by academic institutions themselves because they do not always
show up in official state or countywide tallies, which generally exclude
people who have permanent addresses elsewhere, as students often do.

The Times survey included four-year public schools in the United States,
some of which are subject to public records laws, that are members of
the Association of American Universities or that compete at the highest
level of college sports. It has not yet expanded to include hundreds of
other institutions, including most private schools and community
colleges, where students, faculty and staff are struggling with the same
difficult decisions.

Among the colleges that provided information, many offered no details
about who contracted the virus, when they became ill or whether a case
was connected to a larger outbreak. It is possible that some of the
cases were identified months ago, in the early days of the outbreak in
the United States before in-person learning was cut short, and that
others involved students and employees who had not been on campus
recently.

\hypertarget{return-to-campus}{%
\subsection{Return to Campus}\label{return-to-campus}}

This data, which is almost certainly an undercount, shows the risks
colleges face as they prepare for a school year in the midst of a
pandemic. But because universities vary widely in size, and because some
refused to provide information, comparing case totals from campus to
campus may not provide a full picture of the relative risk.

What is clear is that despite months of planning for a safe return to
class, and despite drastic changes to campus life, the virus is already
spreading widely at universities.

Some institutions, like the California State University system, have
moved most fall classes online. Others, including those in the Patriot
League and Ivy League, have decided to not hold fall sports. But many
institutions still plan to welcome freshmen to campus in the coming
days, to hold in-person classes and to host sporting events.

\hypertarget{plans-for-fall-instruction}{%
\subsubsection{Plans for fall
instruction}\label{plans-for-fall-instruction}}

\hypertarget{the-chart-shows-how-schools-with-reported-coronavirus-cases-plan-to-offer-instruction-for-the-fall-semester-according-to-a-database-from-davidson-college-hover-or-tap-the-circles-to-see-the-schools}{%
\paragraph{\texorpdfstring{The chart shows how schools with reported
coronavirus cases plan to offer instruction for the fall semester,
according to a database from
\href{https://collegecrisis.shinyapps.io/dashboard/}{Davidson College}.
Hover or tap the circles to see the
schools.}{The chart shows how schools with reported coronavirus cases plan to offer instruction for the fall semester, according to a database from Davidson College. Hover or tap the circles to see the schools.}}\label{the-chart-shows-how-schools-with-reported-coronavirus-cases-plan-to-offer-instruction-for-the-fall-semester-according-to-a-database-from-davidson-college-hover-or-tap-the-circles-to-see-the-schools}}

More than 50 cases

11-50 cases

4-10 cases

Fewer than 4 cases

\hypertarget{primarily-or-fully-online}{%
\section{Primarily or fully online}\label{primarily-or-fully-online}}

\hypertarget{hybrid}{%
\section{Hybrid}\label{hybrid}}

\hypertarget{primarily-or-fully-in-person}{%
\section{Primarily or fully
in-person}\label{primarily-or-fully-in-person}}

\hypertarget{waiting-to-decide-or-no-information}{%
\section{Waiting to decide or no
information}\label{waiting-to-decide-or-no-information}}

Source: College Crisis Initiative at Davidson College. Reopening data as
of July 24.

At the University of Texas at Austin, where more than 440 students and
employees have tested positive since the spring, in-person classes will
be capped at 40 percent of capacity and final exams will be taken
online.

At Peru State College in Nebraska, where there have been no known cases,
classes are expected to resume on schedule, but with stepped-up cleaning
procedures and a recommendation for dorm residents to wear masks in
common areas.

The University of Georgia has announced plans for in-person classes
despite rising deaths from the virus in the state. The university has
recorded at least 390 infections involving students, faculty and staff.

O'Bryan Moore, a senior at the school, said he was worried about the
safety of his classmates and teachers. He said he was skeptical that
students would widely follow guidelines to wear masks once they return
in August.

``There is no way I can see this ending without outbreaks on campus,''
said Mr. Moore, who is studying to become a park ranger.

Mr. Moore said that online classes have not been as effective as
in-person classes, but that he still hoped the university would change
its plans for students to return to campus.

``I think we should remain online for this semester, even if it'll hurt
my education,'' he said. ``Because it's the right thing to do.''

Case numbers may be larger at some universities with tens of thousands
of students, including Central Florida and the University of Texas at
Austin, and at others where many university employees work in hospitals
where coronavirus patients
\href{https://www.utsouthwestern.edu/covid-19/}{have been treated},
including at the University of Texas Southwestern Medical Center.

Though hundreds of universities responded to The Times's request for
data --- including a mix of public and private colleges, both small and
large, in states across the country --- others declined to cooperate.
Some said they were not tracking such cases. Others invoked privacy
concerns, even though The Times asked for aggregate case totals, not a
list of individuals who were infected. Others did not respond at all.

A spokesman at Arizona State, for example, said they ``chose months ago
to not release data/names/results'' on coronavirus cases. A spokesman
for Montana State University said the school ``does not provide health
information on its students, faculty or staff, even on general
subgroups.'' The United States Naval Academy cited ``operational
security'' concerns. A spokeswoman for Washburn University in Kansas
said she believed giving such information would violate privacy laws.
And while the University of Missouri's athletic department confirmed 10
cases, a spokesman at the flagship campus would not provide information
about other students and employees.

As students have started trickling back onto campuses in recent weeks,
the early returns have been troubling. After 10 students tested positive
this month at West Virginia University, officials pledged to deep-clean
the places on campus where they had been. At Kansas State University,
off-season football workouts were paused last month after an outbreak on
the team.

\hypertarget{athletic-departments-at-high-risk}{%
\subsection{Athletic Departments at High
Risk}\label{athletic-departments-at-high-risk}}

Many of the first arrivals on campus have been athletes hoping to
compete this fall. A separate Times survey of the 130 universities that
compete at the highest level of Division I football revealed more than
630 cases on 68 campuses among athletes, coaches and other employees.

\hypertarget{coronavirus-cases-in-division-i-athletic-departments}{%
\subsubsection{Coronavirus Cases in Division I Athletic
Departments}\label{coronavirus-cases-in-division-i-athletic-departments}}

As universities make plans for the fall semester --- online, in person,
or a mix of the two --- administrators have had to weigh shifting public
health guidance and financial and academic concerns, as well as the
difficult reality that some students and faculty members are likely to
test positive no matter how classes are held.

``There is simply no way to completely eliminate risk, whether we are
in-person or online,'' Martha E. Pollack, the president of Cornell,
\href{https://covid.cornell.edu/updates/20200630-reactivate-campus.cfm}{wrote
in a letter} explaining the decision to bring students back to campus.

\hypertarget{are-there-coronavirus-cases-on-your-campus}{%
\subsection{Are There Coronavirus Cases on Your
Campus?}\label{are-there-coronavirus-cases-on-your-campus}}

The college case data is current as of July 28. It is based on reports
from colleges and government sources and may lag. Colleges and
government agencies report this data differently, so exercise caution
when comparing institutions. Some colleges declined to provide data or
did not respond to inquiries. At some institutions, cases may be spread
across multiple campuses.

Sources: Case data from a
\href{https://www.nytimes3xbfgragh.onion/interactive/2020/us/coronavirus-us-cases.html}{New
York Times database} of state and local reports; school logos from
\href{https://clearbit.com}{Clearbit} and ESPN.

Reporting was contributed by Jordan Allen, Yuriria Avila, Elisha Brown,
Alyssa Burr, Sarah Cahalan, Matt Craig, Yves De Jesus, Brandon Dupré,
Timmy Facciola, Bianca Fortis, Grace Gorenflo, Barbara Harvey, Shawn
Hubler, Jacob LaGesse, Alex Lim, Alex Leeds Matthews, Jaylynn
Moffat-Mowatt, Ashlyn O'Hara, Cierra S. Queen, Natasha Rodriguez, Alison
Saldanha, Emily Schwing, Sarena Snider, Brandon Thorp and Billy Witz.

Read 407 Comments

\begin{itemize}
\item
\item
\item
\item
\end{itemize}

Advertisement

\protect\hyperlink{after-bottom}{Continue reading the main story}

\hypertarget{site-index}{%
\subsection{Site Index}\label{site-index}}

\hypertarget{site-information-navigation}{%
\subsection{Site Information
Navigation}\label{site-information-navigation}}

\begin{itemize}
\tightlist
\item
  \href{https://help.nytimes3xbfgragh.onion/hc/en-us/articles/115014792127-Copyright-notice}{©~2020~The
  New York Times Company}
\end{itemize}

\begin{itemize}
\tightlist
\item
  \href{https://www.nytco.com/}{NYTCo}
\item
  \href{https://help.nytimes3xbfgragh.onion/hc/en-us/articles/115015385887-Contact-Us}{Contact
  Us}
\item
  \href{https://www.nytco.com/careers/}{Work with us}
\item
  \href{https://nytmediakit.com/}{Advertise}
\item
  \href{http://www.tbrandstudio.com/}{T Brand Studio}
\item
  \href{https://www.nytimes3xbfgragh.onion/privacy/cookie-policy\#how-do-i-manage-trackers}{Your
  Ad Choices}
\item
  \href{https://www.nytimes3xbfgragh.onion/privacy}{Privacy}
\item
  \href{https://help.nytimes3xbfgragh.onion/hc/en-us/articles/115014893428-Terms-of-service}{Terms
  of Service}
\item
  \href{https://help.nytimes3xbfgragh.onion/hc/en-us/articles/115014893968-Terms-of-sale}{Terms
  of Sale}
\item
  \href{https://spiderbites.nytimes3xbfgragh.onion}{Site Map}
\item
  \href{https://help.nytimes3xbfgragh.onion/hc/en-us}{Help}
\item
  \href{https://www.nytimes3xbfgragh.onion/subscription?campaignId=37WXW}{Subscriptions}
\end{itemize}
