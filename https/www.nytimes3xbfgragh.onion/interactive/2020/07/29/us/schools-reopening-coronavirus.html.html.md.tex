Sections

SEARCH

\protect\hyperlink{site-content}{Skip to
content}\protect\hyperlink{site-index}{Skip to site index}

\hypertarget{comments}{%
\subsection{\texorpdfstring{\protect\hyperlink{commentsContainer}{Comments}}{Comments}}\label{comments}}

\href{}{What Back to School Might Look Like in the Age of
Covid-19}\href{}{Skip to Comments}

The comments section is closed. To submit a letter to the editor for
publication, write to
\href{mailto:letters@NYTimes.com}{\nolinkurl{letters@NYTimes.com}}.

\hypertarget{what-back-to-school-might-look-like-in-the-age-of-covid-19}{%
\section{What Back to School Might Look Like in the Age of
Covid-19}\label{what-back-to-school-might-look-like-in-the-age-of-covid-19}}

By \href{https://www.nytimes3xbfgragh.onion/by/dana-goldstein}{Dana
Goldstein},
\href{https://www.nytimes3xbfgragh.onion/by/yuliya-parshina-kottas}{Yuliya
Parshina-Kottas} and
\href{https://www.nytimes3xbfgragh.onion/by/aliza-aufrichtig}{Aliza
Aufrichtig}July 30, 2020

\begin{itemize}
\item
\item
\item
\item
\item
  \emph{169}
\end{itemize}

An illustrated guide to how schools will try to control the coronavirus
when students return to their classrooms, this fall or in the future.

\includegraphics{https://static01.graylady3jvrrxbe.onion/newsgraphics/2020/06/29/virus-back-to-school/f4a8a920272371cae148b89fa0b2ad3e54fdd623/top-d.png}

\hypertarget{what-back-to-school-might-look-like-in-the-age-of-covid-19-1}{%
\section{What Back to School Might Look Like in the Age of
Covid-19}\label{what-back-to-school-might-look-like-in-the-age-of-covid-19-1}}

\hypertarget{an-illustrated-guide-to-how-schools-will-try-to-control-the-coronavirus-when-students-return-to-their-classrooms-this-fall-or-in-the-future}{%
\subsection{An illustrated guide to how schools will try to control the
coronavirus when students return to their classrooms, this fall or in
the
future.}\label{an-illustrated-guide-to-how-schools-will-try-to-control-the-coronavirus-when-students-return-to-their-classrooms-this-fall-or-in-the-future}}

Text by \href{https://www.nytimes3xbfgragh.onion/by/dana-goldstein}{Dana
Goldstein} Illustrations by
\href{https://www.nytimes3xbfgragh.onion/by/yuliya-parshina-kottas}{Yuliya
Parshina-Kottas}\\
Produced by
\href{https://www.nytimes3xbfgragh.onion/by/aliza-aufrichtig}{Aliza
Aufrichtig} July 29, 2020

\includegraphics{https://static01.graylady3jvrrxbe.onion/newsgraphics/2020/06/29/virus-back-to-school/f4a8a920272371cae148b89fa0b2ad3e54fdd623/top-hs-student.png}

A typical American school day requires proximity: High school lab
partners leaning over a vial. Kindergarten students sharing finger
paints. Middle schoolers passing snacks around a cafeteria table.

This year, nothing about school will be typical. Many of the nation's
largest districts plan to
\href{https://www.nytimes3xbfgragh.onion/2020/07/14/us/coronavirus-schools-fall.html}{start
the academic year online}, and it is unclear when students and teachers
will be back in classrooms. Others plan hybrid models, while some are
determined to go five days a week.

When school buildings do reopen, whether this fall or next year, buses,
hallways, cafeterias and classrooms will need to look very different as
long as the coronavirus remains a threat. Even teaching, which has
evolved in recent decades to emphasize fewer lectures and more
collaborative lessons, must change.

``This is the biggest adaptive challenge in my career, and in the
history of public education,'' said Cindy Marten, superintendent of the
San Diego public schools.

Education decisions are largely made at the local level, and leaders are
relying on a host of conflicting federal, state and public health
guidelines. There is still
\href{https://www.nytimes3xbfgragh.onion/2020/07/11/health/coronavirus-schools-reopen.html}{considerable
uncertainty and debate} over how easily
\href{https://www.nytimes3xbfgragh.onion/2020/07/18/health/coronavirus-children-schools.html}{children
of different ages} contract and spread the virus, and whether some of
the recommended safety guidelines would help or are even necessary.

As a result, schools are adopting a wide range of approaches for the
pandemic era. But those recommendations largely agree on at least some
adaptations, and they all come down to eliminating one factor:
proximity.

\hypertarget{riding-the-bus}{%
\subsection{Riding the Bus}\label{riding-the-bus}}

For about half of American students, the school day typically begins
with a bus trip.

For many districts, getting children to school will be one of the most
difficult logistical challenges during the pandemic. Parents will be
asked to consider whether they can arrange other forms of
transportation, like dropping their children off or arranging car pools.

\includegraphics{https://static01.graylady3jvrrxbe.onion/newsgraphics/2020/06/29/virus-back-to-school/f4a8a920272371cae148b89fa0b2ad3e54fdd623/bus_d1.png}

Families should not cluster at the bus stop, as they might have in the
past. And parents will be told: Do not send children to school if they
have a fever, cough or other symptoms.

\includegraphics{https://static01.graylady3jvrrxbe.onion/newsgraphics/2020/06/29/virus-back-to-school/f4a8a920272371cae148b89fa0b2ad3e54fdd623/bus_d2-1.png}

In non-pandemic times, a typical bus might carry 54 children.

\includegraphics{https://static01.graylady3jvrrxbe.onion/newsgraphics/2020/06/29/virus-back-to-school/f4a8a920272371cae148b89fa0b2ad3e54fdd623/bus_d2-3.png}

Enforce strict social distancing guidelines of six feet, and you're down
to eight.

\includegraphics{https://static01.graylady3jvrrxbe.onion/newsgraphics/2020/06/29/virus-back-to-school/f4a8a920272371cae148b89fa0b2ad3e54fdd623/bus_d2-2.png}

Some state guidelines sketch an alternative scenario in which masked
students sit in a zigzag pattern to allow more on board.

\includegraphics{https://static01.graylady3jvrrxbe.onion/newsgraphics/2020/06/29/virus-back-to-school/f4a8a920272371cae148b89fa0b2ad3e54fdd623/bus_d3.png}

Options are expensive. Schools in Marietta, Ga., plan to spend \$640,000
to hire 55 monitors to check students' symptoms before they board.
Dundee, Mich., expects to spend over \$300,000 to add routes. In Odessa,
Texas, there are plans for buses to run on continuous routes, like city
transit, with students arriving and leaving school at staggered times.

\hypertarget{entering-the-building}{%
\subsection{Entering the Building}\label{entering-the-building}}

When students arrive at school, most will be checked to see if they are
running a temperature or showing other symptoms. If adults are dropping
off children, they will likely remain behind a barrier.

Public health experts agree that a key step in keeping the coronavirus
out of schools will be limiting the number of visitors inside.

\includegraphics{https://static01.graylady3jvrrxbe.onion/packages/flash/multimedia/ICONS/transparent.png}

\includegraphics{https://static01.graylady3jvrrxbe.onion/newsgraphics/2020/06/29/virus-back-to-school/f4a8a920272371cae148b89fa0b2ad3e54fdd623/arrival.png}

Temperature checks run the risk of missing asymptomatic or atypical
coronavirus cases, raising false alarms about ordinary illnesses and
taking up valuable time that students could spend learning.
Nevertheless, most districts plan them.

About 60 percent of American schools
\href{https://higherlogicdownload.s3.amazonaws.com/NASN/3870c72d-fff9-4ed7-833f-215de278d256/UploadedImages/PDFs/Advocacy/2017_Workforce_Study_Infographic_School_Nurses_in_the_Nation.pdf}{did
not have full-time nurses} on site in 2018, but many are hoping for
additional federal stimulus money to rectify that amid the pandemic.

Students who fail the symptom check should be isolated while they await
a caretaker to pick them up, guidelines say. Doing so may require
real-estate-strapped schools to designate both safe indoor and outdoor
locations to hold ill and potentially contagious children.

\hypertarget{in-elementary-school-classrooms}{%
\subsection{In Elementary School
Classrooms}\label{in-elementary-school-classrooms}}

Young children may be the hardest to keep apart, given their frenetic
energy, need for hands-on play and affectionate nature. And most
guidelines acknowledge that it is not realistic to expect them to wear
masks all day.

\includegraphics{https://static01.graylady3jvrrxbe.onion/newsgraphics/2020/06/29/virus-back-to-school/f4a8a920272371cae148b89fa0b2ad3e54fdd623/kindergarten_d1.png}

Many schools will try to keep students in pods by limiting class sizes
to about 12 students and by reducing interaction between classrooms.
That way, they can avoid shutting down entirely if a single pod has a
positive case.

\includegraphics{https://static01.graylady3jvrrxbe.onion/newsgraphics/2020/06/29/virus-back-to-school/f4a8a920272371cae148b89fa0b2ad3e54fdd623/kindergarten_d2.png}

Some schools will use X's to indicate where students should sit for
story time.

Some guidelines suggest clear face shields as an alternative to masks
for teachers. Seeing an adult's mouth move helps children understand the
connections between spoken sounds and the written word --- a
\href{https://www.nytimes3xbfgragh.onion/2020/02/15/us/reading-phonics.html}{key
concept} in early reading.

\includegraphics{https://static01.graylady3jvrrxbe.onion/newsgraphics/2020/06/29/virus-back-to-school/f4a8a920272371cae148b89fa0b2ad3e54fdd623/kindergarten_d3.png}

Two students may sit at tables usually used by four or six, with
individual boxes of materials that are typically shared, like art
supplies --- an expense that schools, teachers or families will have to
bear.

Many schools plan to repurpose large spaces, like gyms and cafeterias,
for socially distanced academic work. Students will eat in their
classrooms, either bringing food from home or receiving a boxed lunch.
No buffet lines.

\includegraphics{https://static01.graylady3jvrrxbe.onion/newsgraphics/2020/06/29/virus-back-to-school/f4a8a920272371cae148b89fa0b2ad3e54fdd623/kindergarten_d4.png}

Districts are investing heavily in cleaning and hygiene supplies, such
as hand sanitizer and portable air filters. Adults will disinfect
surfaces several times a day. Federal guidelines recommend that soft
toys that cannot be easily cleaned, like stuffed animals, stay off
limits.

Teachers, who are likely
\href{https://www.nytimes3xbfgragh.onion/2020/07/15/health/coronavirus-schools-reopening.html}{at
greater risk from the virus} than most young students, typically come
into contact with many people in the course of their daily work:
children, parents, other educators. To help reduce risk, staff planning
meetings and parent-teacher conferences can be held remotely.

But many educators, like those who work with children with special
needs, are stationed inside classrooms with other teachers, where they
must attend to students in a hands-on way. Text messaging or in-ear
communication within the classroom may help, with masks to provide
protection.

Teachers will be encouraged to keep classroom windows open to
\href{https://www.nytimes3xbfgragh.onion/2020/07/06/health/coronavirus-airborne-aerosols.html}{promote
air circulation}. Some districts are upgrading heating and cooling
systems to install filtration features, a much more expensive fix.

Moving instruction outdoors when possible would be one way to reduce the
risk of
\href{https://www.nytimes3xbfgragh.onion/2020/07/06/health/coronavirus-airborne-aerosols.html}{airborne
transmission of the virus}. In Marietta, some elementary school students
will bring their own folding lawn chairs to class. Athletics and singing
are activities that, if they occur at all, should be done in the open
air, experts say.

\hypertarget{in-middle-and-high-schools}{%
\subsection{In Middle and High
Schools}\label{in-middle-and-high-schools}}

Older students typically move between classrooms during the day for
different subjects. Instead, health guidelines call for them to remain
in self-contained pods to the greatest extent possible. Schools will
have to figure out another way to deliver an individualized curriculum.

\includegraphics{https://static01.graylady3jvrrxbe.onion/newsgraphics/2020/06/29/virus-back-to-school/f4a8a920272371cae148b89fa0b2ad3e54fdd623/secondary_d1.png}

Teenagers may be
\href{https://www.nytimes3xbfgragh.onion/2020/07/18/health/coronavirus-children-schools.html}{more
at risk from the coronavirus} than younger children are, recent research
suggests, so physical distancing will be more important with this age
group. Some districts are spending hundreds of thousands of dollars on
plexiglass desk dividers for classrooms in which students cannot stay
six feet apart.

\includegraphics{https://static01.graylady3jvrrxbe.onion/newsgraphics/2020/06/29/virus-back-to-school/f4a8a920272371cae148b89fa0b2ad3e54fdd623/secondary_d2.png}

Some students could be remote learning even while in class. An algebra
lesson could be taught at the front of the room, while those who have
moved onto pre-calculus use laptops to participate in an online lesson
in the back.

\includegraphics{https://static01.graylady3jvrrxbe.onion/newsgraphics/2020/06/29/virus-back-to-school/f4a8a920272371cae148b89fa0b2ad3e54fdd623/secondary_d3.png}

Schools are not planning to follow a traditional bell schedule. Instead,
individual pods of students will travel through unidirectional hallways
at specific times, including, in some cases, for prescheduled bathroom
breaks.

Teachers who have adopted a more project-based style say they do not
want to spend the whole year delivering lectures. To get around that,
group work could be done online from home, with teachers focused on
introducing concepts and answering questions while in the classroom.

In some countries that have reopened schools using similar guidelines,
distancing measures have been
\href{https://www.washingtonpost.com/world/europe/schools-reopening-coronavirus/2020/07/10/865fb3e6-c122-11ea-8908-68a2b9eae9e0_story.html}{relaxed
within months} as infection numbers have remained low. The United States
is different from much of the world because some schools are trying to
reopen while infections are still high in their communities.

Dr. Ronald E. Dahl, an expert on adolescent health and development at
the University of California, Berkeley, suggested that a key factor in
making the reconfigured school day work would be for students to feel
invested. To accomplish that, teachers could engage them in group
discussions about the science of the virus and the importance of
physical distancing, and brainstorm ways of enforcing new social norms
among peers.

``It will be very challenging,'' Dr. Dahl acknowledged, given the
natural desire of children and teenagers to interact with one another,
jostling, teasing, flirting and pushing boundaries. But young people
also have a strong sense of right and wrong, he said, and are motivated
to help others, which could inspire them to embrace rules that keep
their friends and teachers healthy.

If the new practices ``honor their desire to be respected and admired,''
Dr. Dahl said, ``young people can shift their behavior quickly.''

Read 169 Comments

\begin{itemize}
\item
\item
\item
\item
\end{itemize}

Advertisement

\protect\hyperlink{after-bottom}{Continue reading the main story}

\hypertarget{site-index}{%
\subsection{Site Index}\label{site-index}}

\hypertarget{site-information-navigation}{%
\subsection{Site Information
Navigation}\label{site-information-navigation}}

\begin{itemize}
\tightlist
\item
  \href{https://help.nytimes3xbfgragh.onion/hc/en-us/articles/115014792127-Copyright-notice}{©~2020~The
  New York Times Company}
\end{itemize}

\begin{itemize}
\tightlist
\item
  \href{https://www.nytco.com/}{NYTCo}
\item
  \href{https://help.nytimes3xbfgragh.onion/hc/en-us/articles/115015385887-Contact-Us}{Contact
  Us}
\item
  \href{https://www.nytco.com/careers/}{Work with us}
\item
  \href{https://nytmediakit.com/}{Advertise}
\item
  \href{http://www.tbrandstudio.com/}{T Brand Studio}
\item
  \href{https://www.nytimes3xbfgragh.onion/privacy/cookie-policy\#how-do-i-manage-trackers}{Your
  Ad Choices}
\item
  \href{https://www.nytimes3xbfgragh.onion/privacy}{Privacy}
\item
  \href{https://help.nytimes3xbfgragh.onion/hc/en-us/articles/115014893428-Terms-of-service}{Terms
  of Service}
\item
  \href{https://help.nytimes3xbfgragh.onion/hc/en-us/articles/115014893968-Terms-of-sale}{Terms
  of Sale}
\item
  \href{https://spiderbites.nytimes3xbfgragh.onion}{Site Map}
\item
  \href{https://help.nytimes3xbfgragh.onion/hc/en-us}{Help}
\item
  \href{https://www.nytimes3xbfgragh.onion/subscription?campaignId=37WXW}{Subscriptions}
\end{itemize}
