Sections

SEARCH

\protect\hyperlink{site-content}{Skip to
content}\protect\hyperlink{site-index}{Skip to site index}

\hypertarget{comments}{%
\subsection{\texorpdfstring{\protect\hyperlink{commentsContainer}{Comments}}{Comments}}\label{comments}}

\href{}{Hope, Despair, Control: The 1950s China My Father Saw, Echoed
Today}\href{}{Skip to Comments}

The comments section is closed. To submit a letter to the editor for
publication, write to
\href{mailto:letters@NYTimes.com}{\nolinkurl{letters@NYTimes.com}}.

\hypertarget{hope-despair-control-the-1950s-china-my-father-saw-echoed-today}{%
\section{Hope, Despair, Control: The 1950s China My Father Saw, Echoed
Today}\label{hope-despair-control-the-1950s-china-my-father-saw-echoed-today}}

By
\href{https://www.nytimes3xbfgragh.onion/by/alexandra-stevenson}{Alexandra
Stevenson}July 31, 2020

\href{https://cn.nytimes3xbfgragh.onion/china/20200731/china-1950s-echoed-today/}{阅读简体中文版}\href{https://cn.nytimes3xbfgragh.onion/china/20200731/china-1950s-echoed-today/zh-hant/}{閱讀繁體中文版}

\begin{itemize}
\item
\item
\item
\item
\item
  \emph{34}
\end{itemize}

William Stevenson was one of the first foreign correspondents to visit
the People's Republic of China. Decades later, despite its
transformation, I recognize the same country.

\includegraphics{https://static01.graylady3jvrrxbe.onion/packages/flash/multimedia/ICONS/transparent.png}

\includegraphics{https://static01.graylady3jvrrxbe.onion/newsgraphics/2020/07/21/china-then-now/assets/images/as_200714_id_card_02-2000.jpg}

\includegraphics{https://static01.graylady3jvrrxbe.onion/packages/flash/multimedia/ICONS/transparent.png}

\includegraphics{https://static01.graylady3jvrrxbe.onion/newsgraphics/2020/07/21/china-then-now/assets/images/as_200714_07-2000.jpg}

\includegraphics{https://static01.graylady3jvrrxbe.onion/packages/flash/multimedia/ICONS/transparent.png}

\includegraphics{https://static01.graylady3jvrrxbe.onion/newsgraphics/2020/07/21/china-then-now/e1bab7f7496eba9a648a21c465620dc6714581a6/AS_200714_Notebook_2-no-bg.png}

\includegraphics{https://static01.graylady3jvrrxbe.onion/packages/flash/multimedia/ICONS/transparent.png}

\includegraphics{https://static01.graylady3jvrrxbe.onion/newsgraphics/2020/07/21/china-then-now/assets/images/as_200714_document_1-2000.jpg}

\includegraphics{https://static01.graylady3jvrrxbe.onion/packages/flash/multimedia/ICONS/transparent.png}

\includegraphics{https://static01.graylady3jvrrxbe.onion/newsgraphics/2020/07/21/china-then-now/assets/images/as_200714_06-2000.jpg}

SHENZHEN-HONG KONG BORDER --- The bridge was only 20 yards long, but it
was the longest journey of my father's life. Holding a flimsy piece of
paper with a Swiss watermark and Chinese characters, he crossed the
bridge from the British colony of Hong Kong into Mao's China, one of the
first foreign correspondents to report on a country largely unknown to
the rest of the world in 1954. The paper was his golden ticket.

Some six decades later, I found myself staring out at the same
footbridge from the other side.

In mainland China on my own coveted journalism visa, I peeked out
through the metal bars separating me from Hong Kong, now a
semiautonomous territory of China. The closest my father had previously
come to China was approaching this bridge to meet missionaries who, he
wrote, stumbled
``\href{https://archive.nytimes3xbfgragh.onion/www.nytimes3xbfgragh.onion/library/world/asia/china-index-timeline.html}{out
of the Chinese Revolution} with tragic tales fully confirmed by their
emaciated bodies and haggard eyes.''

As the bamboo gate swung closed behind him, my father put one foot down
on Chinese soil and looked up to see a simple mud village at the
precipice of a new era. Decades later, I looked back to see a different
view altogether: a towering skyline of glass and metal with one of the
world's tallest buildings in a city going through its own dramatic
transformation.

It was almost impossible to get to China from the West at the start of
Mao's rule. The country had
\href{https://archive.nytimes3xbfgragh.onion/www.nytimes3xbfgragh.onion/library/world/asia/100149mao-communism.html}{declared
itself the People's Republic of China} five years earlier, and it was
the early days of the Cold War that divided Communist countries from
Western democracies.

\includegraphics{https://static01.graylady3jvrrxbe.onion/packages/flash/multimedia/ICONS/transparent.png}

\includegraphics{https://static01.graylady3jvrrxbe.onion/newsgraphics/2020/07/21/china-then-now/assets/images/as_200714_01-2000.jpg}

My father at a newsstand in Hong Kong, where he was stationed for much
of the 1950s.Richard Harrington, via Stephen Bulger Gallery

My father had carved out an unusual beat, reporting for The Toronto Star
and The Star Weekly from one newly Communist country to another,
chronicling the path of each. On his travels he searched for a Chinese
diplomatic office where he could get a visa to visit.

If he could find a friendly Chinese official in Moscow or another
capital in Eastern Europe, he might have a chance to talk that person
into giving him a visa. Yet in his early travels behind the Iron
Curtain, China remained elusive. He persisted, propelled by an urgency
to understand this huge nation.

Eventually, during a trip to Poland, his determination paid off.

In July 1954, he traveled to Bern, Switzerland, where he was told to
pick up his visa.

My father left behind written notes and newspaper clippings, stacks of
passports with visas, photos and transcripts from his first and
subsequent trips to China. They have allowed me to imagine conversations
that we might have had in the six years since he died. Conversations
about how the country he saw back then --- brimming with hope and
enthusiasm yet also tightly controlled --- is in some ways the same
today.

\includegraphics{https://static01.graylady3jvrrxbe.onion/packages/flash/multimedia/ICONS/transparent.png}

\includegraphics{https://static01.graylady3jvrrxbe.onion/newsgraphics/2020/07/21/china-then-now/assets/images/scanned_434-uk-passport-cover-1200dpi-2000.jpg}

\includegraphics{https://static01.graylady3jvrrxbe.onion/packages/flash/multimedia/ICONS/transparent.png}

\includegraphics{https://static01.graylady3jvrrxbe.onion/newsgraphics/2020/07/21/china-then-now/assets/images/scanned_437-uk-passport-suisse-lebanon-poland-visas-1200dpi-2000.jpg}

\includegraphics{https://static01.graylady3jvrrxbe.onion/packages/flash/multimedia/ICONS/transparent.png}

\includegraphics{https://static01.graylady3jvrrxbe.onion/newsgraphics/2020/07/21/china-then-now/assets/images/scanned_436-uk-passport-valid-countries-1200dpi-2000.jpg}

\includegraphics{https://static01.graylady3jvrrxbe.onion/packages/flash/multimedia/ICONS/transparent.png}

\includegraphics{https://static01.graylady3jvrrxbe.onion/newsgraphics/2020/07/21/china-then-now/assets/images/scanned_435-uk-passport-picture-page-1200dpi-2000.jpg}

One of my father's canceled passports, with immigration stamps for his
trips to Poland and Switzerland.

His first trip to China spanned two months and thousands of miles. He
met
\href{https://archive.nytimes3xbfgragh.onion/www.nytimes3xbfgragh.onion/library/world/asia/061357mao-inthenews.html}{Mao
Zedong} (whom he tapped on the shoulder from behind his camera,
mistaking the chairman for a ``humble courtier'' blocking his shot) and
Zhou Enlai, the premier and foreign minister at the time. But he also
talked with factory workers, actors, newspaper editors and shop owners.

He described being filled with hope for the human spirit he witnessed.
But he also felt despair because a government-provided handler was never
too far away, ready to silence anyone who veered too far from the
Communist Party line.

China defied any broad-brush statement. ``And yet,'' he wrote in one
notebook, ``under the current leadership, the way in which the
government silences alternative points of view makes it hard not to.''

\includegraphics{https://static01.graylady3jvrrxbe.onion/packages/flash/multimedia/ICONS/transparent.png}

\includegraphics{https://static01.graylady3jvrrxbe.onion/newsgraphics/2020/07/21/china-then-now/assets/images/img_0286-2000.jpg}

A picture of Shenzhen's skyline taken from one of its many high
rises.Alexandra Stevenson/The New York Times

A version of this exists today. I have a long list of names of people
who wouldn't talk to me because I work for The New York Times, portrayed
in Chinese state media as the source of ``smears and lies.'' Sources
I've interviewed privately are later threatened by the local police,
while stridently nationalist rhetoric dominates the state media.

Several months after I returned to Hong Kong, the Chinese government in
March\href{https://www.nytimes3xbfgragh.onion/2020/03/18/world/asia/china-expels-journalists.html}{}\href{https://www.nytimes3xbfgragh.onion/2020/03/18/world/asia/china-expels-journalists.html}{expelled
my American colleagues} as part of a diplomatic dispute with the United
States. In the past
month,\href{https://www.nytimes3xbfgragh.onion/2020/06/29/world/asia/china-hong-kong-security-law-rules.html?action=click\&module=RelatedLinks\&pgtype=Article}{}\href{https://www.nytimes3xbfgragh.onion/2020/06/29/world/asia/china-hong-kong-security-law-rules.html?action=click\&module=RelatedLinks\&pgtype=Article}{Beijing
has tightened its grip} over Hong Kong with a new national security law,
threatening\href{https://www.nytimes3xbfgragh.onion/2020/07/01/world/asia/hong-kong-security-law-china.html}{}\href{https://www.nytimes3xbfgragh.onion/2020/07/01/world/asia/hong-kong-security-law-china.html}{free
speech and other civil liberties} in the city.

During his trip, my father traveled from Shenzhen, Guangzhou and
Chongqing in the south, to cities farther north like Shenyang, Shanghai,
Wuhan and Beijing. Some of the datelines in his dispatches were
different from today --- Canton, Hankow, Mukden, Peiping --- yet much of
his observations still ring true.

In Beijing, he found more than just a city but also a way of life that
defied the strictures of Communism. ``No rubber stamp yet dictates the
passions and peculiarities of its people,'' he wrote.

\includegraphics{https://static01.graylady3jvrrxbe.onion/packages/flash/multimedia/ICONS/transparent.png}

\includegraphics{https://static01.graylady3jvrrxbe.onion/newsgraphics/2020/07/21/china-then-now/assets/images/as_200714_notebook_2-2000.jpg}

\hypertarget{you-are-filled-with-indignation-in-one-moment-and-moved-to-admiration-in-the-next}{%
\subsection{``You are filled with indignation in one moment and moved to
admiration in the
next.''}\label{you-are-filled-with-indignation-in-one-moment-and-moved-to-admiration-in-the-next}}

Excerpt from a 1950s notebook

\includegraphics{https://static01.graylady3jvrrxbe.onion/packages/flash/multimedia/ICONS/transparent.png}

\includegraphics{https://static01.graylady3jvrrxbe.onion/newsgraphics/2020/07/21/china-then-now/assets/images/14-beijing-laobaixing-2-2000.jpg}

A selection from one of my father's reporting notebooks, as well as a
photo he took of a man in Beijing.

It is the same today. In the summer heat, men
\href{https://www.nytimes3xbfgragh.onion/2016/09/02/world/what-in-the-world/china-summer-beijing-bikini.html}{roll
up their shirts to expose their bellies}, even though the government
calls the act ``uncivilized'' and has tried to crack down. The
unsuspecting bicycle rider is never too far from crashing into a manic
delivery man zipping down narrow bike paths on the wrong side of the
road. Smokers stub out their cigarettes on the No Smoking signs
plastered everywhere.

In one of his notebooks, my father noted a seriousness to the people he
met and interviewed. But, he added, it was hard to resist a smile, ``and
everyone seems to smile; surely not all by government order?''

The people my father met shared their aspirations, both personal and
professional.

\includegraphics{https://static01.graylady3jvrrxbe.onion/packages/flash/multimedia/ICONS/transparent.png}

\includegraphics{https://static01.graylady3jvrrxbe.onion/newsgraphics/2020/07/21/china-then-now/assets/images/as_200714_12-2000.jpg}

\includegraphics{https://static01.graylady3jvrrxbe.onion/packages/flash/multimedia/ICONS/transparent.png}

\includegraphics{https://static01.graylady3jvrrxbe.onion/newsgraphics/2020/07/21/china-then-now/assets/images/as_200714_02-2000.jpg}

\includegraphics{https://static01.graylady3jvrrxbe.onion/packages/flash/multimedia/ICONS/transparent.png}

\includegraphics{https://static01.graylady3jvrrxbe.onion/newsgraphics/2020/07/21/china-then-now/assets/images/as_200714_09-2000.jpg}

A selection of images from my father's reporting trips, including one of
a Chinese peasant in the Tibetan foothills and another of families of
Chinese fisherman who, my father wrote, "crowd into Hong Kong to escape
harsh attempts to make them join People's Communes.''·William Stevenson

One young factory worker told him she had no time to think about getting
married. Knitting, cooking and doing domestic chores were a waste of
time, she said. And anyway, once she did get around to having a baby she
would keep working.

``After 14 months a baby has to look out for itself,'' she told him. So
she would leave the baby at the factory nursery, taking the child home
only once the workweek was over.

I have interviewed women who felt that the Communist Party today had
failed them when
it\href{https://www.nytimes3xbfgragh.onion/2019/07/16/world/asia/china-women-discrimination.html}{}\href{https://www.nytimes3xbfgragh.onion/2019/07/16/world/asia/china-women-discrimination.html}{comes
to the family}, leaving them with no support.

Mao told them they were equal to men in work and life. Yet policymakers
have intervened again and again to dictate how women should govern their
bodies. First, they could
have\href{https://www.nytimes3xbfgragh.onion/2015/11/06/magazine/the-long-shadow-of-chinas-one-child-policy.html}{}\href{https://www.nytimes3xbfgragh.onion/2015/11/06/magazine/the-long-shadow-of-chinas-one-child-policy.html}{only
one child}. Now, they are being told they
should\href{https://www.nytimes3xbfgragh.onion/2018/08/11/world/asia/china-one-child-policy-birthrate.html}{}\href{https://www.nytimes3xbfgragh.onion/2018/08/11/world/asia/china-one-child-policy-birthrate.html}{have
two children} if they want to be patriotic.

For many women, motherhood is a losing proposition. They need to keep
their jobs
but\href{https://www.nytimes3xbfgragh.onion/2019/11/01/business/china-mothers-discrimination-working-.html}{}\href{https://www.nytimes3xbfgragh.onion/2019/11/01/business/china-mothers-discrimination-working-.html}{risk
getting demoted or fired} when they get pregnant.

\includegraphics{https://static01.graylady3jvrrxbe.onion/packages/flash/multimedia/ICONS/transparent.png}

\includegraphics{https://static01.graylady3jvrrxbe.onion/newsgraphics/2020/07/21/china-then-now/assets/images/gm_chinamaternity_036-2000.jpg}

Li Xiaoping in her apartment in the Shandong Province last year.Giulia
Marchi for The New York Times

``Should a woman just go back to fulfilling her traditional role as a
wife and be shut out of society after giving birth?'' Li Xiaoping asked
me. The 33-year-old said she was fired for being pregnant. After she
left, the electronics company she worked for sent her a bill equivalent
to five years of salary for the hassle.

During his first trip, my father was pushed around by unfriendly
officials.

While visiting the Great Wall, he left his guide to chase two men over
the other side of the wall with his camera. Two People's Liberation Army
soldiers were launched into action, he wrote, ``before you could say
`Chiang Kai-shek,''' referring to the Chinese Nationalist leader, who
had fled to Taiwan after his defeat by the Communists in 1949.

He waved cheerily, and they retreated. It was over, he thought, until
his guide told him that he had taken unauthorized photographs and that
the military was waiting for him in Beijing where he would be forced to
give up his camera. But the developed film was eventually returned,
``with thanks by a grinning official who agreed the only military secret
it recorded was this breathtaking and ageless barrier --- the Great Wall
of China.''

\includegraphics{https://static01.graylady3jvrrxbe.onion/packages/flash/multimedia/ICONS/transparent.png}

\includegraphics{https://static01.graylady3jvrrxbe.onion/newsgraphics/2020/07/21/china-then-now/assets/images/16-greatwall-2000.jpg}

A group of men my father captured crossing a passageway.William
Stevenson

Today officials
frequently\href{https://www.nytimes3xbfgragh.onion/2019/02/27/technology/personaltech/digital-footprint-surveillance.html}{}\href{https://www.nytimes3xbfgragh.onion/2019/02/27/technology/personaltech/digital-footprint-surveillance.html}{demand
journalists delete photos} from their smartphones. Last summer, my
colleague and I found ourselves in a small town in the heart of China's
coal country looking
for\href{https://www.nytimes3xbfgragh.onion/2019/11/10/business/china-debt-hospitals.html}{}\href{https://www.nytimes3xbfgragh.onion/2019/11/10/business/china-debt-hospitals.html}{empty
stadiums and half-built government vanity projects}. As we were
preparing to leave, we were suddenly circled by more than a dozen police
officers and government officials.

\includegraphics{https://static01.graylady3jvrrxbe.onion/packages/flash/multimedia/ICONS/transparent.png}

\includegraphics{https://static01.graylady3jvrrxbe.onion/newsgraphics/2020/07/21/china-then-now/assets/images/img_0807-2000.jpg}

The female police officer as she approached me and my
colleague.Alexandra Stevenson/The New York Times

They scanned our IDs. They questioned our motives. They threatened our
driver. They pleaded with us to write a positive story. They asked to
see our phones, to delete our photos. We got a Beijing official on
speakerphone to tell the police we were allowed to be there, to no
avail.

\href{https://www.nytimes3xbfgragh.onion/2019/11/10/business/china-reporter-police.html}{The
charade went on} for two hours before another female cop inexplicably
walked up to us, shook my colleague's hand and said, ``You're welcome
here, thanks for your cooperation.''

These interactions are not new. I experienced similar acts of
intimidation when I was working in China a decade ago. But there is an
undercurrent now that feels different, one that I recognize in some of
my father's writing.

He struggled to reconcile what he saw with what he believed to be true.
The ``sinister regime where jails and punishment cells awaited the
unfaithful'' was mostly invisible on his first trip. Yet, he later
wondered, what had happened to those acquaintances who disappeared and
then later reappeared with confessions in hand?

\includegraphics{https://static01.graylady3jvrrxbe.onion/packages/flash/multimedia/ICONS/transparent.png}

\includegraphics{https://static01.graylady3jvrrxbe.onion/newsgraphics/2020/07/21/china-then-now/assets/images/as_200714_document_1-2000.jpg}

\includegraphics{https://static01.graylady3jvrrxbe.onion/packages/flash/multimedia/ICONS/transparent.png}

\includegraphics{https://static01.graylady3jvrrxbe.onion/newsgraphics/2020/07/21/china-then-now/assets/images/as_200714_document_2-2000.jpg}

\hypertarget{you-start-such-a-ride-with-mixed-feelings-you-are-apprehensive-or-maybe-elated-you-feel-intrepid-or-inadequate-to-the-challenges-ahead-but-whatever-you-feel-you-certainly-suffer-a-sense-of-foolishness}{%
\subsection{``You start such a ride with mixed feelings. You are
apprehensive, or maybe elated. You feel intrepid or inadequate to the
challenges ahead. But whatever you feel, you certainly suffer a sense of
foolishness.''}\label{you-start-such-a-ride-with-mixed-feelings-you-are-apprehensive-or-maybe-elated-you-feel-intrepid-or-inadequate-to-the-challenges-ahead-but-whatever-you-feel-you-certainly-suffer-a-sense-of-foolishness}}

A draft from one of my father's stories describing the start of his
China trip

\includegraphics{https://static01.graylady3jvrrxbe.onion/packages/flash/multimedia/ICONS/transparent.png}

\includegraphics{https://static01.graylady3jvrrxbe.onion/newsgraphics/2020/07/21/china-then-now/assets/images/as_200714_look_02_page51-2000.jpg}

\includegraphics{https://static01.graylady3jvrrxbe.onion/packages/flash/multimedia/ICONS/transparent.png}

\includegraphics{https://static01.graylady3jvrrxbe.onion/newsgraphics/2020/07/21/china-then-now/assets/images/as_200714_look_03_page52-2000.jpg}

\includegraphics{https://static01.graylady3jvrrxbe.onion/packages/flash/multimedia/ICONS/transparent.png}

\includegraphics{https://static01.graylady3jvrrxbe.onion/newsgraphics/2020/07/21/china-then-now/assets/images/as_200714_look_04_page54-2000.jpg}

A collection of my father's work, including a transcript for a Canadian
television broadcast and an article for Look Magazine published on March
8, 1955.

The government's heavy handedness would inevitably emerge. In Shanghai,
he visited a theater, elated because for the first time in weeks there
appeared to be no political subtext to the visit. But when he sneaked
backstage he bumped into a big blackboard.

\includegraphics{https://static01.graylady3jvrrxbe.onion/packages/flash/multimedia/ICONS/transparent.png}

\includegraphics{https://static01.graylady3jvrrxbe.onion/newsgraphics/2020/07/21/china-then-now/assets/images/ws-boston-globe-15-nov-1954-2000.jpg}

An excerpt from an article about the actor that ran in The Boston Globe.

On it was an essay written by one of the actors, he was told. ``It is
called: `Who are my friends and who are my enemies?''' It turned out, in
fact, to be a confession written by someone who had complained, ``this
government gives me a pain.''

As my six-month assignment in China came to an end, the country was
preparing to celebrate 70 years of Chinese Communist Party rule. Every
corner of the country was whipped up into celebratory fervor. Huge
billboards of a smiling Xi Jinping with proclamations about China lined
the highways. When my husband and I traveled through the mountains on a
rickety bus in the southwest, we started a new game to pass the time:
Spot President Xi.

The day before the parade I found myself sharing a cab to the airport in
Shenzhen with Walter Liu, a 37-year-old Beijing native who now lives in
California. Mr. Liu and his high school had participated in the 50th
anniversary parade in 1999 when he was 17. He and his classmates were
given pink and yellow blocks of paper to hold in a formation on
Tiananmen Square. From above the sign read ``50.''

It was the culmination of two months of rehearsals, first at his high
school and then later during midnight rehearsals on Tiananmen Square.

What Mr. Liu remembered most vividly was the excitement of being able to
see his girlfriend during those midnight sessions. ``It is rare that you
could see your girlfriend at night,'' he said, smiling as he recalled
it. ``We could just look at each other from the crowd and wink wink. We
couldn't even talk.''

On the day of the parade, his parents squinted, trying to find him on
their television. ``I don't think they could see me because I was so
tiny,'' said Mr. Liu, laughing. ``I was one color pixel on TV.''

\includegraphics{https://static01.graylady3jvrrxbe.onion/packages/flash/multimedia/ICONS/transparent.png}

\includegraphics{https://static01.graylady3jvrrxbe.onion/newsgraphics/2020/07/21/china-then-now/assets/images/as_200714_id_card_01-2000.jpg}

My ticket to the anniversary parade.

On the day of
the\href{https://www.nytimes3xbfgragh.onion/2019/09/28/world/asia/china-national-day-70th-anniversary.html}{}\href{https://www.nytimes3xbfgragh.onion/2019/09/28/world/asia/china-national-day-70th-anniversary.html}{70th
anniversary parade} I, too, was a pixel. I had managed to persuade the
government to give me a highly prized ticket to watch the parade from
the stands, just as my father had done at the end of his first China
tour.

It was an unusually hot day and the air was heavy with smog. Everyone
had an identifier. Blue uniformed sanitation workers. Green soldiers.
Dark blue naval officers. Blue-and-white track-suited volunteers. A
thousand government workers from one Beijing district with white shirts
and a red bird logo. I felt out of place, even though I was given a
bright red flag to wave.

My father had stood in the same place for the fifth anniversary parade.
He noted similar columns of troops, guns and tanks, with soldiers
marching in unison and such ``terrifying rhythm'' that it was as though
they were ``pouring straight off the production line of some human
factory.''

From the stands, my father focused his binoculars on Mao, who stood
beneath 10 huge lanterns waving and laughing. His gold-colored helmet
had tipped to one side and his hands were hidden behind a thick cloak.

\includegraphics{https://static01.graylady3jvrrxbe.onion/packages/flash/multimedia/ICONS/transparent.png}

\includegraphics{https://static01.graylady3jvrrxbe.onion/newsgraphics/2020/07/21/china-then-now/assets/images/img_1835-2000.jpg}

The crowd from the bleachers, with Mr. Xi projected in the
background.Alexandra Stevenson/The New York Times

I did not need binoculars to find Xi Jinping. He was projected, standing
stiff, on huge screens at every angle. Just as Mao had done long before
him, he came rolling out onto Chang'an Avenue in a special retro-styled
black car to greet and inspect the troops.

The two-hour parade ended with towering portraits of the Communist
Party's top leaders over the decades since 1949. As they rolled out on
huge floats, loud cheers erupted from the bleachers. Mao's portrait came
first. The biggest cheer was reserved for the last portrait, of Xi.

There is much discussion today among intellectuals in China about how
the state
looks\href{https://www.nytimes3xbfgragh.onion/interactive/2017/11/09/world/asia/xi-propaganda.html}{}\href{https://www.nytimes3xbfgragh.onion/interactive/2017/11/09/world/asia/xi-propaganda.html}{much
more like it did under Mao} than at any other time since the country
opened itself up to the world four decades ago.

I wish I could ask my father about that. But I have a pretty good idea
what he would say.

Alexandra Stevenson is a business correspondent based in Hong Kong,
covering Chinese corporate giants, the changing landscape for
multinational companies and China's growing economic and financial
influence in Asia.

Top images of William Stevenson: Richard Harrington, via Stephen Bulger
Gallery.

Alain Delaquérière contributed research from New York. Design and
production by Gabriel Gianordoli and Renee Melides. Additional
production by Adriana Ramic.

Read 34 Comments

\begin{itemize}
\item
\item
\item
\item
\end{itemize}

Advertisement

\protect\hyperlink{after-bottom}{Continue reading the main story}

\hypertarget{site-index}{%
\subsection{Site Index}\label{site-index}}

\hypertarget{site-information-navigation}{%
\subsection{Site Information
Navigation}\label{site-information-navigation}}

\begin{itemize}
\tightlist
\item
  \href{https://help.nytimes3xbfgragh.onion/hc/en-us/articles/115014792127-Copyright-notice}{©~2020~The
  New York Times Company}
\end{itemize}

\begin{itemize}
\tightlist
\item
  \href{https://www.nytco.com/}{NYTCo}
\item
  \href{https://help.nytimes3xbfgragh.onion/hc/en-us/articles/115015385887-Contact-Us}{Contact
  Us}
\item
  \href{https://www.nytco.com/careers/}{Work with us}
\item
  \href{https://nytmediakit.com/}{Advertise}
\item
  \href{http://www.tbrandstudio.com/}{T Brand Studio}
\item
  \href{https://www.nytimes3xbfgragh.onion/privacy/cookie-policy\#how-do-i-manage-trackers}{Your
  Ad Choices}
\item
  \href{https://www.nytimes3xbfgragh.onion/privacy}{Privacy}
\item
  \href{https://help.nytimes3xbfgragh.onion/hc/en-us/articles/115014893428-Terms-of-service}{Terms
  of Service}
\item
  \href{https://help.nytimes3xbfgragh.onion/hc/en-us/articles/115014893968-Terms-of-sale}{Terms
  of Sale}
\item
  \href{https://spiderbites.nytimes3xbfgragh.onion}{Site Map}
\item
  \href{https://help.nytimes3xbfgragh.onion/hc/en-us}{Help}
\item
  \href{https://www.nytimes3xbfgragh.onion/subscription?campaignId=37WXW}{Subscriptions}
\end{itemize}
