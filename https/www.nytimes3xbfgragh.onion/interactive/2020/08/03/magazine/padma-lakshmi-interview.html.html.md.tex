Sections

SEARCH

\protect\hyperlink{site-content}{Skip to
content}\protect\hyperlink{site-index}{Skip to site index}

\hypertarget{comments}{%
\subsection{\texorpdfstring{\protect\hyperlink{commentsContainer}{Comments}}{Comments}}\label{comments}}

\href{}{Padma Lakshmi Wants Us to Eat More Adventurously}\href{}{Skip to
Comments}

The comments section is closed. To submit a letter to the editor for
publication, write to
\href{mailto:letters@NYTimes.com}{\nolinkurl{letters@NYTimes.com}}.

Talk

\hypertarget{padma-lakshmi-wants-us-to-eat-more-adventurously}{%
\section{Padma Lakshmi Wants Us to Eat More
Adventurously}\label{padma-lakshmi-wants-us-to-eat-more-adventurously}}

By David MarcheseAug. 3, 2020

\begin{itemize}
\item
\item
\item
\item
\item
  \emph{98}
\end{itemize}

``There's such a laziness about reaching for the thing that is most
familiar.''

\includegraphics{https://static01.graylady3jvrrxbe.onion/newsgraphics/2020/07/09/talk/a28b9c874788a9e2b60f2798087577317201a315/close.svg}

\textbf{Talk} Aug. 3, 2020

\hypertarget{padma-lakshmi-wants-us-to-eat-more-adventurously-1}{%
\section{Padma Lakshmi Wants Us to Eat More
Adventurously}\label{padma-lakshmi-wants-us-to-eat-more-adventurously-1}}

By David Marchese

SHARE

If you're only familiar with Padma Lakshmi through her work as a host
and judge on Bravo's long-running cooking competition ``Top Chef,'' then
the 49-year-old's new show might seem like a detour. Part food
travelogue, part exploration of the benefits and blind spots of
multiculturalism, Hulu's ``Taste the Nation'' finds Lakshmi cracking
crab shells with South Carolina's Gullah Geechee community, partaking of
brats and beer at Oktoberfest in Milwaukee and comparing flour and corn
tortillas along the border in El Paso. It's a long way from the glamour
of ``Top Chef,'' but as Lakshmi tells it, the show is the culmination of
her aim to ``demystify foods that are part of our culture but get
othered by the greater American culture.'' Pursuing that aim has been
the hidden throughline connecting her three cookbooks, her pre-``Top
Chef'' TV appearances on the Food Network and even her well-regarded
2016 memoirs, ``Love, Loss, and What We Ate.'' It is, she says,
``something I've been thinking about for a long time.''

\textbf{An idea that's implicit in ``Taste the Nation'' is that the more
we know about the cultural history of our food, the more that leads to
cultural openness. What makes you believe that this idea is more than
just a platitude?} Listen, I'm under no illusions. I'm not one of these
kumbaya people. But I think the willingness to break bread with someone
shows a crack of openness. I believe in that quote,
\href{http://nytimes3xbfgragh.onion\#tooltip-1}{``Tell me what you eat,
and I will tell you who you are.''} Through food, you can tell a lot
about not only a person or a family but also a community. You can trace
history through foods. You can trace colonization. Food can be a great
instrument, and that is how I try to use it.

\textbf{The new show is really about diversity. ``Top Chef'' hasn't
necessarily had the greatest}
\textbf{\href{http://nytimes3xbfgragh.onion\#tooltip-2}{track record in
that area.}} \textbf{Could the show be doing more?} Everybody should be
doing more. I think that we have gotten better. I think we have a long
way to go. As a producer, I have power now that I didn't have when I
started on ``Top Chef.'' I think we've done well in the last few years,
but there has to be a revolution from the ground up. What I mean by that
is: Why don't we teach African-American cuisine in our cooking schools
in this country? Why does it always have to be French-centric? Why isn't
it a requirement in culinary school to understand the Native foods of
North America? And these chefs who have power now, usually white male
chefs, they're often mentoring people whom it's easy for them to mentor.
I would love to see those chefs go into urban environments and high
schools or colleges and search for people to mentor who aren't
necessarily already in their universe, so that when people want to come
on ``Top Chef'' they're trained properly and can compete on equal
footing.

\textbf{There has been a ton of discussion and controversy lately about
race and cultural appropriation in food media. Did you have much sense
of the dynamics going on at}
\textbf{\href{http://nytimes3xbfgragh.onion\#tooltip-3}{a place like Bon
Appétit?}} \textbf{Or in food media more generally?} I didn't know to
what degree they went on at Bon Appétit. I certainly didn't know about
the pay discrepancy. I don't know Adam Rapoport socially beyond
food-world things. That picture of him and his wife dressed up is the
least of the issue, in my opinion. I think Adam Rapoport is a symptom of
something much bigger and more insidious, which is that there is
unconscious racism and subconscious racism and bias and favoritism
because we are attracted to people like us. Look at the people who get
things greenlit. For the most part, they're white. That's what it feels
like. When I walk around New York City or El Paso or Las Vegas, I see a
whole bunch of different kinds of people. There's such a laziness ---
it's not often malicious --- about reaching for the thing that is most
familiar. But it's not only ethical to be more inclusive; it's good for
business.

\textbf{You've said elsewhere recently that over the years you've had
trouble getting attention and coverage from certain outlets and
publications. Can you tell me more about that?} Listen, I pitched
``Taste the Nation'' to several networks. I flew to Los Angeles on my
own dime two or three times, and everybody said no. When my agent told
me that Hulu called and said they'd love to talk, I said: ``I'm not
flying to L.A. again. I'm done.'' I hated coming home after being away
from my kid, and she's saying, ``Mommy, did you sell it?'' and I have to
look at this 9-year-old and say, ``No, I didn't.'' One entity --- I
won't name names, but he's no longer at the network --- even wrote me a
long email about \emph{why} he said no. I guess he was trying to be
respectful, but I don't need a 900-word email about how my show idea is
derivative. Especially when there's nothing that I can see on TV like
it. I've heard an Italian expression, \emph{``È come essere
schiaffeggiato nel buio,''} which means ``It's like being slapped in the
dark.'' You don't know where it's coming from, and you don't know why
it's happening to you. I have experienced this in a million ways. You
have to remember, I've been on prime-time television for 14 years. I
have a show that airs in countries all over the world. I was well known
before ``Top Chef.'' My show has been nominated for an Emmy
\href{http://nytimes3xbfgragh.onion\#tooltip-4}{every single year that
I've been doing it.} And yet all these networks that claim they want
diversity --- and here was ``Taste the Nation,'' a show about the
diversity of our country, and they said no. I started to think, Maybe
I'm the only one interested in this stuff. It's the same thing when I
see other, white women being published constantly, and their books
selling, and I know that their recipe is a watered-down version of an
Indian recipe or a Moroccan recipe.

\textbf{Is that a reference to}
\textbf{\href{http://nytimes3xbfgragh.onion\#tooltip-5}{Alison Roman's
stew?}} **** I'm not going to comment on anybody specific, because I
don't think that's productive.

\textbf{Without commenting on individuals, what did}
\textbf{\href{http://nytimes3xbfgragh.onion\#tooltip-6}{the blow-up}}
\textbf{with her and Chrissy Teigen and Marie Kondo signify to you?} I
think she, like all of us sometimes, suffered from a bad case of
foot-in-mouth disease. It's unfortunate. I think all three of those
women probably want the story to go away. That's all I will say about
it.

\textbf{Has your thinking about cultural appropriation and food changed?
There was a profile of you in New York magazine last year, and}
\textbf{\href{http://nytimes3xbfgragh.onion\#tooltip-7}{in it you said
--- I'm paraphrasing ---}} \textbf{that if cultural appropriation gets
more people open to more flavors, then you're OK with it. Do you still
feel that way?} Look, I'm not saying that you can't use turmeric on a
menu or in a cookbook unless you do a doctoral dissertation on ayurvedic
medicine. I'm just saying that a couple of sentences at the top of a
recipe would place it in context. I love the commingling of cultures. My
cookbooks are not all Indian, because I don't eat like that. I don't
experience life like that, and I don't think most Americans do, either.
So I'm not saying that Indian food should only be cooked by Indians. But
it would be great if a recipe that went viral were placed in the context
of its own history. It's not taking anything away from creativity to do
that. It is acknowledging that these things didn't come out of a vacuum.

\textbf{Aside from that, what might a more culturally equitable food
world look like to you?} I would like to see the food section of papers
like The New York Times not be so white. I would like to see Condé Nast
have more editors who are not white. That's a real, concrete ask that
I'm making. You have to make sure you're hiring writers who have a
different perspective than the rest of your staff, because that's good
for your newspaper or magazine. I would like them to consider balancing
whom they interview, even bending over backward a little bit, to even
out our presence.

\textbf{``Top Chef'' excepted, the}
\textbf{\href{http://nytimes3xbfgragh.onion\#tooltip-8}{other
food-related shows}} \textbf{you've done have been weighted toward
non-European food. Does that suggest biases about what television
executives are comfortable with you doing? Would it give them pause if
you pitched a show about French cuisine?} I don't think so, because I
have 14 years on ``Top Chef.'' But I am a brown woman working in a
white, male Hollywood. It is very hard for us to get a show to begin
with, never mind the subject matter. But it's a good question. If you're
talking about my situation, I would never pitch a show like what you
described. I already have a successful show. I'm very thankful for it.
\href{http://nytimes3xbfgragh.onion\#tooltip-9}{It has provided my
daughter} and me with a great lifestyle. If I'm going to take time out
of my life, it's got to be something that I feel is worthwhile. And
``Taste the Nation'' is what I feel is most worthwhile. A lot of
immigrants, we live in this weird in-between land; there is a lot of
code-switching that goes on when you walk into your family home and then
when you go to school. We have to navigate that. So on ``Taste the
Nation'' I want to show a Thai grandmother making her dish so that
\href{http://nytimes3xbfgragh.onion\#tooltip-10}{the Thai immigrant
version of me} who's in elementary school now can see her and say: ``Oh,
OK. My grandma is not that weird, because this other grandma was on
Hulu.'' I know that sounds like a little thing, but it's not.

\textbf{What you're talking about is a kind of acceptance, which
connects to something you wrote about in your memoirs: You had a hard
time during your modeling career reconciling your intellectual interests
with the work you were doing, and that struggle turned into low
self-esteem or even self-loathing. Was it hard to manage those feelings
in a productive way?} I didn't start modeling until I was 21, which
helped psychologically, but I had to disassociate what I did for a
living from my sense of self. I was able to do that because
\href{http://nytimes3xbfgragh.onion\#tooltip-11}{I would write.} That
was my outlet. I also had to remind myself constantly that modeling
wasn't personal, that it had to do with the color of your skin or that
they just wanted a blond girl or a flat-chested girl. It takes time to
develop who you are as a person, and I spent a lot of years trying to be
as girl-next-door as I could, as salable, commercial, whatever the job
market told me I needed to be in order to succeed. And in the end, when
I finally got success, it was because I just did whatever the hell I
wanted.

\textbf{Did you have to make certain market concessions in order to get
your first cookbook published? It's hard to imagine your using a title
like ``Easy Exotic'' and using similarly sultry photos in a cookbook
today.} Of course. I wouldn't have gotten that contract if I wasn't a
model. It was because I was a really good cook who also happened to be a
model. It was also not my lifelong dream to be a lingerie model, but
guess what? That is how I paid off my college loans before any of my
classmates. We all do what we have to do to get by. I love the pictures
in ``Easy Exotic,'' but that's the thing, we put people in these boxes:
I have to be a pretty model who doesn't eat or I have to be an
intellectual person who's not wearing certain clothes or I have to be a
cookbook author and be very Martha Stewart. Well, I'm not. There are
different sides to me, and I think today people are accepting of
dimensionality in a person. I'm a complicated person, like most human
beings.

\textbf{This last question doesn't have to do with food: You've had a
lot of}
\textbf{\href{http://nytimes3xbfgragh.onion\#tooltip-12}{traumatic
events}} \textbf{in your life, and it seems as if it would be easy for
somebody who's had those experiences to end up cynical or pessimistic.
You're not. How did you avoid that?} Yes, a lot of {[}expletive{]} has
gone down. I do have a bit of ``the sky is falling''; people who are
close to me would say, ``She's always worst-case scenario.'' But I
remember something my grandfather said to me. He said, ``Whenever you go
to sleep, I want you to feel like you did something good today.'' You
have control over what you accomplish. You don't always have control
over what happens to you, but you have control over how you react. In
spite of everything that happened to me, look where I am today.

\begin{center}\rule{0.5\linewidth}{\linethickness}\end{center}

\emph{This interview has been edited and condensed for clarity from two
conversations.}

Read 98 Comments

\begin{itemize}
\item
\item
\item
\item
\end{itemize}

Advertisement

\protect\hyperlink{after-bottom}{Continue reading the main story}

\hypertarget{site-index}{%
\subsection{Site Index}\label{site-index}}

\hypertarget{site-information-navigation}{%
\subsection{Site Information
Navigation}\label{site-information-navigation}}

\begin{itemize}
\tightlist
\item
  \href{https://help.nytimes3xbfgragh.onion/hc/en-us/articles/115014792127-Copyright-notice}{©~2020~The
  New York Times Company}
\end{itemize}

\begin{itemize}
\tightlist
\item
  \href{https://www.nytco.com/}{NYTCo}
\item
  \href{https://help.nytimes3xbfgragh.onion/hc/en-us/articles/115015385887-Contact-Us}{Contact
  Us}
\item
  \href{https://www.nytco.com/careers/}{Work with us}
\item
  \href{https://nytmediakit.com/}{Advertise}
\item
  \href{http://www.tbrandstudio.com/}{T Brand Studio}
\item
  \href{https://www.nytimes3xbfgragh.onion/privacy/cookie-policy\#how-do-i-manage-trackers}{Your
  Ad Choices}
\item
  \href{https://www.nytimes3xbfgragh.onion/privacy}{Privacy}
\item
  \href{https://help.nytimes3xbfgragh.onion/hc/en-us/articles/115014893428-Terms-of-service}{Terms
  of Service}
\item
  \href{https://help.nytimes3xbfgragh.onion/hc/en-us/articles/115014893968-Terms-of-sale}{Terms
  of Sale}
\item
  \href{https://spiderbites.nytimes3xbfgragh.onion}{Site Map}
\item
  \href{https://help.nytimes3xbfgragh.onion/hc/en-us}{Help}
\item
  \href{https://www.nytimes3xbfgragh.onion/subscription?campaignId=37WXW}{Subscriptions}
\end{itemize}
