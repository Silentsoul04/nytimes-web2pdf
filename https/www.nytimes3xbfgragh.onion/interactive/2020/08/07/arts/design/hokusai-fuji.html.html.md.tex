Sections

SEARCH

\protect\hyperlink{site-content}{Skip to
content}\protect\hyperlink{site-index}{Skip to site index}

\hypertarget{a-picture-of-change-for-a-world-in-constant-motion}{%
\section{A Picture of Change for a World in Constant
Motion}\label{a-picture-of-change-for-a-world-in-constant-motion}}

By \href{https://www.nytimes3xbfgragh.onion/by/jason-farago}{Jason
Farago}Aug. 7, 2020

\begin{itemize}
\item
\item
\item
\item
\end{itemize}

\includegraphics{https://static01.graylady3jvrrxbe.onion/packages/flash/multimedia/ICONS/transparent.png}

\includegraphics{https://static01.graylady3jvrrxbe.onion/newsgraphics/2020/07/13/zoomy-hokusai/assets/images/hokusai-trimmed-4000.jpg}

Early spring. A heavy sky. Chilly, but not bitter. We're near Suruga
Bay, on the south coast of Honshu; maybe you can taste the salt in the
air.

The year is 1830 or so: the waning days of the Tokugawa shogunate. And
from the northwest, a wind is blowing with the force of a steamroller.

It's not his most famous work, but this is my favorite woodblock print
by Katsushika Hokusai: ``Ejiri in Suruga Province.'' It's the 10th image
in his renowned cycle ``Thirty-Six Views of Mount Fuji.''

I love it most for how it captures an instant, with an exactitude that
feels almost photographic. Here. Now. A country road, two trees,
daytime: hold onto your hats.

And for something else: the story it tells about how images circulate in
a cosmopolitan world.

Today Hokusai stands --- for Western audiences, and in Asia too --- at
the pinnacle of ``Japanese art.'' But if you told the grandees of
19th-century Edo that Hokusai would become the most famous artist in the
country's history, they'd never believe you.

Woodblock prints like his --- called ** Ukiyo-e, or ``pictures of the
floating world,'' and turned out by the thousands in private printing
houses --- were considered vulgar, commercial images.

How does a single artist --- of mass-market pictures, no less --- come
to embody a national culture?

Let's start with the road. A serpentine passage cuts through an ordinary
little marsh, on a highway that connects Kyoto to Edo (now Tokyo). No
graceful landscape, this. We're somewhere commonplace, undistinguished.

Most of the marsh grasses are billowing gently. But the gusting wind has
bent the trees, and it's blowing the young leaves right off their
branches.

More than in the landscape, you see the wind's strength in the
travelers' bodies. This figure has stepped off the road, and is gripping
his hat with both hands. His body is crumpled. The gusting wind has bent
him, literally, out of shape.

These two walkers are also leaning into the wind, though they appear to
be pressing on with their journey. Crouched down, one foot in front of
the other.

This drab country road is a crossroads of classes. In the foreground,
walking in the other direction, is this gentleman, able to hang onto his
hat with just one hand.

We know he's at least moderately wealthy, because he's not traveling
alone. He's walking with a porter, who is having worse luck.

On his head is a circular cushion . . .

from which his woven hat has blown right off.

He's the only traveler with a clearly visible face: male, middle age?
And yet we have hardly more insight into his life than his fellow
commuters'. This print is not a character study.

Look closely at its most compelling figure: a woman whose striped blue
kimono has blown into her face. A sheaf of papers she's been carrying
has been tossed into the air. They could be a poetry manuscript, or
property deeds. Some scholars identify the fugitive sheets as tissues
the woman would be carrying under her kimono.

It hardly matters who she is. It hardly matters what the papers say.
They're going everywhere, each one a little vacuum of white in the reedy
green.

Sudden accidents, trivial messes, the thousand natural shocks of travel.

It all means nothing to the trees, the marsh, and the mountain: Mount
Fuji, impassive in the back.

\includegraphics{https://static01.graylady3jvrrxbe.onion/packages/flash/multimedia/ICONS/transparent.png}

\includegraphics{https://static01.graylady3jvrrxbe.onion/newsgraphics/2020/07/13/zoomy-hokusai/assets/images/dp141020-4000.png}

The august mountain appears in each of the ``Thirty-Six Views,'' but
only rarely does Hokusai put it front and center. Fuji can be a
snow-capped cone, a tourist attraction for visitors to a temple . . .

\includegraphics{https://static01.graylady3jvrrxbe.onion/packages/flash/multimedia/ICONS/transparent.png}

\includegraphics{https://static01.graylady3jvrrxbe.onion/newsgraphics/2020/07/13/zoomy-hokusai/assets/images/dp141041_-1--4000.png}

or a sharp outcropping, sitting in the backdrop of a construction site .
. .

\includegraphics{https://static01.graylady3jvrrxbe.onion/packages/flash/multimedia/ICONS/transparent.png}

\includegraphics{https://static01.graylady3jvrrxbe.onion/newsgraphics/2020/07/13/zoomy-hokusai/assets/images/dp132437-jpg_original_300dpi-4000.png}

or an obscure little knoll, swallowed up by a clawing tsunami.

Fuji is so modest, in this most famous of the ``Thirty-Six Views,'' that
some viewers confuse it for foam on the waves. And that, too, is a clue
to Hokusai's transcultural meaning.

The little Fuji, beneath the waves, sits at the composition's vanishing
point: a hallmark of Renaissance image-making. Hokusai was among the
first Japanese artists to employ Western perspective, though he used it
playfully.

During Hokusai's lifetime, Japanese were barred from leaving the
country, on pain of death. But the country was not totally closed. Some
foreign goods could come in, via Nagasaki --- such as the rich Prussian
blue ink used here.

\includegraphics{https://static01.graylady3jvrrxbe.onion/packages/flash/multimedia/ICONS/transparent.png}

\includegraphics{https://static01.graylady3jvrrxbe.onion/newsgraphics/2020/07/13/zoomy-hokusai/assets/images/hokusai-trimmed-4000.jpg}

And some foreign techniques, too.

Do you see, here, how the traveler in the back is so much smaller than
the woman who's lost her papers? And how sharply the landscape slopes
up? Hokusai would have picked up this perspectival technique from Dutch
prints circulating in Edo . . .

even as elsewhere, in the same image, Hokusai employs a perspectival
technique common in Asian painting, with similarly sized figures
positioned along diagonal sightlines. That, too, was imported knowledge,
absorbed from Chinese examples into earlier Japanese painting.

Even in a ``closed'' Japan, Hokusai was weaving together a lavish array
of cosmopolitan influences. But only after his death, and after Japan
opened up, would his own papers float beyond the archipelago.

\includegraphics{https://static01.graylady3jvrrxbe.onion/packages/flash/multimedia/ICONS/transparent.png}

\includegraphics{https://static01.graylady3jvrrxbe.onion/newsgraphics/2020/07/13/zoomy-hokusai/assets/images/japanese_satsuma_pavillion_at_the_french_expo_1867-4000.jpg}

In 1867, the World's Fair took place in Paris. Japan participated for
the first time, and displayed coats of armor, swords, statues --- and
Ukiyo-e.

\includegraphics{https://static01.graylady3jvrrxbe.onion/packages/flash/multimedia/ICONS/transparent.png}

\includegraphics{https://static01.graylady3jvrrxbe.onion/newsgraphics/2020/07/13/zoomy-hokusai/assets/images/dp141069-4000.png}

The French went wild. A critic at the fair singled out Hokusai as ``the
freest and most sincere of the Japanese masters.''

What these young moderns loved were the prints: their flat spaces, their
simplified lines, their quotidian subject matter.

Hokusai's example would soon influence the work of Paris's modern
artists.

\includegraphics{https://static01.graylady3jvrrxbe.onion/packages/flash/multimedia/ICONS/transparent.png}

\includegraphics{https://static01.graylady3jvrrxbe.onion/newsgraphics/2020/07/13/zoomy-hokusai/assets/images/20001128_090333_cassatt-4-4000.jpg}

Mary Cassatt, for instance. She learned from Hokusai and other Japanese
printmakers to create spaces of blocky color, with hard transitions from
tone to tone.

\includegraphics{https://static01.graylady3jvrrxbe.onion/packages/flash/multimedia/ICONS/transparent.png}

\includegraphics{https://static01.graylady3jvrrxbe.onion/newsgraphics/2020/07/13/zoomy-hokusai/assets/images/dt840-4000.jpg}

Or her friend Edgar Degas, whose flat and asymmetrical spaces channel
the Japanese model into the opera house and the ballet studio.

\includegraphics{https://static01.graylady3jvrrxbe.onion/packages/flash/multimedia/ICONS/transparent.png}

\includegraphics{https://static01.graylady3jvrrxbe.onion/newsgraphics/2020/07/13/zoomy-hokusai/assets/images/vuillard-4000.png}

Or, later, Édouard Vuillard, who drew on Japanese examples for this
flat, asymmetrical scene of bourgeois families in a public garden.

\includegraphics{https://static01.graylady3jvrrxbe.onion/packages/flash/multimedia/ICONS/transparent.png}

\includegraphics{https://static01.graylady3jvrrxbe.onion/newsgraphics/2020/07/13/zoomy-hokusai/assets/images/vangogh-4000.jpg}

These Parisians understood the prints they were looking at only in part.
They made foolish, patronizing generalizations.

(Van Gogh, who painted these branches: ``Isn't it a true religion that
these simple Japanese teach us, who live in nature as though they
themselves were flowers?'')

Like most fantasies, ``Japonisme'' said more about the fantasizer than
the fantasized. These Parisians, defeated in war and rocketing through
industrialization, saw themselves in landscapes that were both ageless
and adrift. And Hokusai, who'd already metabolized Western technique
into his images of Japan, was the perfect vessel for their dreaming.

\includegraphics{https://static01.graylady3jvrrxbe.onion/packages/flash/multimedia/ICONS/transparent.png}

\includegraphics{https://static01.graylady3jvrrxbe.onion/newsgraphics/2020/07/13/zoomy-hokusai/assets/images/canvas_copy-4000.jpg}

But a new Japan was dawning. With the Meiji Restoration of 1868, the
archipelago charged into the industrial age.. Kobayashi Kiyochika, a
later Ukiyo-e ** artist, used the mass form of printmaking to depict a
different story: ** Japan as a modern imperial power.

\includegraphics{https://static01.graylady3jvrrxbe.onion/packages/flash/multimedia/ICONS/transparent.png}

\includegraphics{https://static01.graylady3jvrrxbe.onion/newsgraphics/2020/07/13/zoomy-hokusai/assets/images/canvas_2_copy-4000.jpg}

He and other Meiji printmakers would update Hokusai's mixing of
perspectives, his careful use of light and shadow. And they gave the
market what it wanted: images of an empire triumphant in war.

\includegraphics{https://static01.graylady3jvrrxbe.onion/packages/flash/multimedia/ICONS/transparent.png}

\includegraphics{https://static01.graylady3jvrrxbe.onion/newsgraphics/2020/07/13/zoomy-hokusai/assets/images/hokusai-trimmed-4000.jpg}

In the century to come, Japanese museums and corporations would become
major buyers of French painting.

And in that light they would also elevate Hokusai into the highest
exemplar of Japanese high art. A whole museum in Tokyo is devoted to his
work alone.

\includegraphics{https://static01.graylady3jvrrxbe.onion/packages/flash/multimedia/ICONS/transparent.png}

\includegraphics{https://static01.graylady3jvrrxbe.onion/newsgraphics/2020/07/13/zoomy-hokusai/assets/images/jeff_wall_a_sudden_gust_of_wind_-after_hokusai-_1993-4000.jpg}

By the turn of the millennium, Hokusai was no longer an exotic import to
Western image-making, but a global master. In 1993, the photographer
Jeff Wall completed ``A Sudden Gust of Wind (After Hokusai),'' shot at a
cranberry farm in British Columbia.

It's one of this great Canadian photographer's most complex images, and
channels Hokusai in a totally different way to the Japonistes of modern
Paris.

He has turned Hokusai's porter into a businessman, who's lost his trilby
in the breeze.

The woman in the kimono here wears a pantsuit, and her papers fly out of
a red folder, over the dike and the river.

And where Fuji once loomed, we see the treeline of suburban Vancouver.

The photograph does not capture a single moment. It's a composition of
more than one hundred exposures, joined as cunningly as the woodblocks
in Hokusai's printer's studio.

Here we have a Canadian photographer nourished by French modern art,
which was nourished by Japanese printmaking, which was nourished by
Dutch book illustration and Chinese painting. The wind has blown from
the printing house to the darkroom, across the Pacific and the Atlantic.

\includegraphics{https://static01.graylady3jvrrxbe.onion/packages/flash/multimedia/ICONS/transparent.png}

\includegraphics{https://static01.graylady3jvrrxbe.onion/newsgraphics/2020/07/13/zoomy-hokusai/assets/images/hokusai-trimmed-4000.jpg}

Hokusai already knew, in 1830, how quickly and thoroughly an image's
meaning can change. It's already there in the picture of Ejiri.

The mountain rises against a largely empty sky. Against all the road's
activity --- the wind, the dashed marsh grass, the color --- Fuji is
just a single, calligraphic stroke: swoop up, slash right, swoop down.

An ordinary artist would picture Fuji as a vision of beauty, a symbol of
permanence. But Hokusai, the sharpest of ironists, does the opposite.

In this ``view'' of Fuji, the wind is so strong and sudden that it has
washed the mountain away. It's as if Fuji, when these storm-struck
travelers can't see it, recedes from our view as well.

Here, in a crummy little marsh under Fuji, Hokusai gave us a vision of
culture in constant motion.

Because art's meaning lies not only in what it looks like, but in how it
circulates. And if you can't fully control circulation, you can't fully
control meaning either. Least of all today, when digital images blow
every which way.

You hold on to what you can in this explosion of images. But the
mountain fades in the distance, and the papers end up in the air.

Images: ``Ejiri in Suruga Province'' via The Metropolitan Museum of Art;
``Sazai Hall at the Temple of the Five Hundred Arhats,'' via The
Metropolitan Museum of Art; ``Tatekawa in Honjō'' via The Metropolitan
Museum of Art; ``Under the Wave off Kanagawa,'' via The Metropolitan
Museum of Art; The Japanese Pavilion at the 1867 Paris World Exposition;
``Yoshida on the Tōkaidō,'' via The Metropolitan Museum of Art; ``Woman
Bathing,'' via The Metropolitan Museum of Art; ``Dancers Practicing at
the Barre,'' via The Metropolitan Museum of Art; ``Public Gardens'';
``Almond Blossom,'' Van Gogh Museum, Amsterdam; ``In the Battle of the
Yellow Sea a Sailor onboard Our Japanese Warship Matsushima, on the
Verge of Dying, Asked Whether or Not the Enemy Ship Had Been
Destroyed,'' via The Smithsonian; ``Illustration of the Arrival of the
Emperor at Shinbashi Station Following a Victory,'' via The Metropolitan
Museum of Art; ``A Sudden Gust of Wind (After Hokusai),'' via Tate
Museum.

Produced by Alicia DeSantis, Gabriel Gianordoli and Josephine Sedgwick.

\begin{itemize}
\item
\item
\item
\item
\end{itemize}

Advertisement

\protect\hyperlink{after-bottom}{Continue reading the main story}

\hypertarget{site-index}{%
\subsection{Site Index}\label{site-index}}

\hypertarget{site-information-navigation}{%
\subsection{Site Information
Navigation}\label{site-information-navigation}}

\begin{itemize}
\tightlist
\item
  \href{https://help.nytimes3xbfgragh.onion/hc/en-us/articles/115014792127-Copyright-notice}{©~2020~The
  New York Times Company}
\end{itemize}

\begin{itemize}
\tightlist
\item
  \href{https://www.nytco.com/}{NYTCo}
\item
  \href{https://help.nytimes3xbfgragh.onion/hc/en-us/articles/115015385887-Contact-Us}{Contact
  Us}
\item
  \href{https://www.nytco.com/careers/}{Work with us}
\item
  \href{https://nytmediakit.com/}{Advertise}
\item
  \href{http://www.tbrandstudio.com/}{T Brand Studio}
\item
  \href{https://www.nytimes3xbfgragh.onion/privacy/cookie-policy\#how-do-i-manage-trackers}{Your
  Ad Choices}
\item
  \href{https://www.nytimes3xbfgragh.onion/privacy}{Privacy}
\item
  \href{https://help.nytimes3xbfgragh.onion/hc/en-us/articles/115014893428-Terms-of-service}{Terms
  of Service}
\item
  \href{https://help.nytimes3xbfgragh.onion/hc/en-us/articles/115014893968-Terms-of-sale}{Terms
  of Sale}
\item
  \href{https://spiderbites.nytimes3xbfgragh.onion}{Site Map}
\item
  \href{https://help.nytimes3xbfgragh.onion/hc/en-us}{Help}
\item
  \href{https://www.nytimes3xbfgragh.onion/subscription?campaignId=37WXW}{Subscriptions}
\end{itemize}
