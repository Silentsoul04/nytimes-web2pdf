Sections

SEARCH

\protect\hyperlink{site-content}{Skip to
content}\protect\hyperlink{site-index}{Skip to site index}

\href{https://www.nytimes3xbfgragh.onion/section/opinion}{Opinion}

\href{https://myaccount.nytimes3xbfgragh.onion/auth/login?response_type=cookie\&client_id=vi}{}

\href{https://www.nytimes3xbfgragh.onion/section/todayspaper}{Today's
Paper}

\href{/section/opinion}{Opinion}\textbar{}Democratic Convention: Best
and Worst Moments of Night 2

\begin{itemize}
\item
\item
\item
\item
\item
\item
\end{itemize}

Advertisement

\protect\hyperlink{after-top}{Continue reading the main story}

\hypertarget{comments}{%
\subsection{\texorpdfstring{\protect\hyperlink{commentsContainer}{Comments}}{Comments}}\label{comments}}

\href{}{Democratic Convention: Best and Worst Moments of Night
2}\href{}{Skip to Comments}

The comments section is closed. To submit a letter to the editor for
publication, write to
\href{mailto:letters@NYTimes.com}{\nolinkurl{letters@NYTimes.com}}.

\href{/section/opinion}{Opinion}

\hypertarget{democratic-convention-best-and-worst-moments-of-night-2}{%
\section{Democratic Convention: Best and Worst Moments of Night
2}\label{democratic-convention-best-and-worst-moments-of-night-2}}

By The New York Times OpinionAug. 19, 2020

\begin{itemize}
\item
\item
\item
\item
\item
  \emph{1357}
\end{itemize}

Welcome to Opinion's commentary for the second night of the Democratic
National Convention. In this special feature, Times Opinion writers rank
the evening on a scale of 1 to 10: 1 means the night was a disaster for
Democrats; 10 means it could lead to a big polling bump for
Biden-Harris. Here's what our columnists and contributors thought of the
event, which highlighted the roll call, Alexandria Ocasio-Cortez, Bill
Clinton and Jill Biden.

See rankings from the first night of the Democratic National Convention
\href{https://www.nytimes3xbfgragh.onion/interactive/2020/08/18/opinion/democratic-convention-best-worst.html}{here}.

\hypertarget{best-moment}{%
\subsubsection{Best moment}\label{best-moment}}

\includegraphics{https://static01.graylady3jvrrxbe.onion/images/2020/08/19/opinion/19scorecard-biden/merlin_175883646_b58587ef-f599-4b9e-9973-8e345882b527-articleLarge.jpg}

Joe Biden accepted the Democratic nomination via livestream from
Delaware, flanked by family. Brian Snyder/Pool via AP

\textbf{Wajahat Ali}~~ Democrats are making the case that they represent
a majority, with a broad coalition of diverse communities, including
life-long Republicans. Well, they showed it.

\textbf{Jamelle Bouie}~~ The roll call! I am a sucker for earnest pride
in one's home and community, and I found it genuinely moving to see
Americans of all colors and backgrounds speak to that pride and to their
faith in this country. It is good stuff! Also, it should be a reminder
that the United States owes its territories either independence or full
voting rights and representation in Congress.

\textbf{Frank Bruni}~~ Joe Biden saying, with a Roman candle of a smile,
``thank you, thank you, thank you,'' when the nomination was finally and
formally his. Gratitude, along with humility, is foreign to Donald
Trump.

\textbf{Gail Collins}~~ Have to admit the virtual roll call was better
than expected. Really thought I'd miss all those delegates howling
commercials for their state from the convention floor. But actually
seeing them on their home turf was nice.

\textbf{Michelle Cottle}~~ This is how nominating roll calls should be
conducted. Rather than focusing on spun-up delegates in daffy hats
jammed into a convention hall, this vote looked outward at America, with
on-site shots from every state and territory, starting from the Edmund
Pettus bridge in Alabama.

\textbf{Michelle Goldberg}~~ In many ways, a virtual convention is a
pale imitation of a real one, but the tour-of-America roll call vote,
with its moving diversity, homespun production values and slightly
uncanny masked tableaus, was a huge improvement over the usual
procedure.

\textbf{Nicole Hemmer}~~ The roll call. Over the past several months,
many Americans have barely left their homes, much less their states, so
that tour of the country felt a little like traveling. And it brought
some kitschy fun to a convention woefully short on funny hats.

\textbf{Liz Mair}~~ It was, by far, when Rhode Island used its roll-call
vote to feature a man holding a platter of calamari --- a prime-time
earned media ad for ``the calamari comeback state.'' The video roll call
was genuinely fun and gave a good glimpse of the breadth and depth of
American culture.

\textbf{Daniel McCarthy}~~ The Biden family video humanized him well
after his rather stiff acceptance of the nomination, and Jill Biden's
follow-up in the classroom was potent.

\textbf{Melanye Price}~~ The delegate roll call. There is no way the
Republican Party can match the Democrats when it comes to reflecting the
diversity of America. Effective and uplifting!

\textbf{Mimi Swartz}~~ Jill Biden. Gee, it would be nice to have a real
first lady again.

\textbf{Héctor Tobar}~~ The world's biggest Zoom conference call, i.e.,
the around-the-U.S.A. delegate vote. Great landscapes, and a wonderfully
diverse sampling of young Democratic leaders, activists and citizens. It
offered the viewer a real ``proud to be an American'' moment. Even the
Fox News pundits liked it.

\textbf{Peter Wehner}~~ Not any of the speeches, which were average at
best, but two D.N.C. videos --- one about Jill Biden and her
relationship with Joe; the other on the ``unlikely friendship'' between
Biden and John McCain. Honorable mention to the virtual roll call, which
was better and more interesting than any in the past.

\hypertarget{worst-moment}{%
\subsubsection{Worst moment}\label{worst-moment}}

\includegraphics{https://static01.graylady3jvrrxbe.onion/images/2020/08/19/opinion/19scorecard-worst2/merlin_175881108_00f2601d-803c-4512-946f-234c5585d548-articleLarge.jpg}

Senator Chuck Schumer of New York spoke with the Statue of Liberty in
the background. Democratic National Convention, via Associated Press

\textbf{Wajahat Ali}~~ Tom Perez needs to stop trying so hard. As the
party chairman, just come out and say it straight. Bill Clinton
delivered, as he always does, but Democrats have to navigate their
future without him and his scandals. The rising star Alexandria
Ocasio-Cortez needed more than 96 seconds.

\textbf{Jamelle Bouie}~~ I am finding it hard to identify a worst moment
that isn't just an ideological gripe on my part. (And on that point, the
foreign policy segment wasn't for me!) I suppose I would have loved to
hear more from Representative Ocasio-Cortez, who is an extremely
talented political communicator.

\textbf{Frank Bruni}~~ The relay-race keynote speech. This gimmick meant
the remarks had no shape, pacing or heft, and the swiftly changing faces
and backdrops instilled motion sickness: Political bromides met ``The
Perfect Storm.'' At one point 17 keynoters said, in unison, ``That's a
big effing deal!'' I effing cringed.

\textbf{Gail Collins}~~ Sticking to a five-minute speech must have been
hell for Bill Clinton, and he sounded sorta flat. And John Kerry --- oh,
wow, forgot what it was like to stare blankly at a screen when John
Kerry was making an important address.

\textbf{Michelle Cottle}~~ After the roll call, the cameras lingered a
bit too long on Joe and Jill standing around grinning awkwardly as
people in masks threw streamers at them. The nominee looked happy ---
but also as though he wasn't sure what to do. Wave? Dance? Hug Jill?
Let's keep it crisp, people!

\textbf{Michelle Goldberg}~~ Post \#MeToo, there was no reason to have
Bill Clinton speak.

\textbf{Nicole Hemmer}~~ Chuck Schumer kept gesturing meaningfully
toward the Statue of Liberty during his speech, seemingly unaware that
it was just a distant green smudge. You could spin that as symbolism ---
the erosion of liberty and all that --- but it was just bad camera work
for an otherwise forgettable speech.

\textbf{Liz Mair}~~ Both parties desperately need a new inventory of
celebratory music. Teeing up tunes that remind everyone of autumn 1980
isn't a great way of projecting youth, vigor, stamina and a
forward-facing outlook.

\textbf{Daniel McCarthy}~~ Colin Powell is an impressive man who put his
prestige behind a needless and disastrous war in Iraq, which Biden
backed, too. Powell and the videos before and after him were a reminder
that Biden is the candidate of the war party.

\textbf{Melanye Price}~~ The old guys in the middle. John Kerry, Colin
Powell, Chuck Hagel and even Bill Clinton harkened back to a version of
politics that is on life support. In a high-tech, diverse, increasingly
progressive political moment, they seemed too stoic and out of place.

\textbf{Mimi Swartz}~~ It pains me to say this, but the scene of Joe
Biden in the library after winning the nomination was underwhelming
after the heroic buildup. Maybe anyone would long for the lift of a live
crowd at that moment, but Biden looked like someone still waiting for
his cue.

\textbf{Héctor Tobar}~~ Chuck Schumer. A speech completely devoid of any
original ideas, delivered with a stiff posture and wooden tone, with the
Statue of Liberty in the background. Ugh. It was the one moment of the
night that most resembled a ``Saturday Night Live'' sketch.

\textbf{Peter Wehner}~~ Alexandria Ocasio-Cortez's searing indictment of
America, which fit in a lot in 96 seconds: racial injustice,
colonization, misogyny, homophobia, the violence and xenophobia of our
past, and the ``unsustainable brutality'' of our economy. Democrats
should be glad she wasn't granted more time.

\hypertarget{what-else-mattered}{%
\subsubsection{What else mattered}\label{what-else-mattered}}

\includegraphics{https://static01.graylady3jvrrxbe.onion/images/2020/08/19/opinion/19scorecard-best/merlin_175883925_0b9f085a-f50c-4b93-bc58-3595eae3bb1e-articleLarge.jpg}

The broadcast included livestreams from across the country. Democratic
National Convention, via Associated Press

\textbf{Wajahat Ali}~~ It's easy to forget Republicans remain committed
to dismantling Obamacare. The Democrats put a human face on health care.
Americans with disabilities shared their stories with Biden. He listened
and cared. He showed us what's at stake if Trump gets another term: the
health of a nation.

\textbf{Jamelle Bouie}~~ The showcase of young leaders. The next
generation of Democratic Party leaders are here and working their way
through the ranks and up the ladder. Seeing them was also a reminder of
how America's national leadership is in deep need of generational
turnover.

\textbf{Frank Bruni}~~ John Kerry's devastating review of Trump on the
world stage (``when this president goes overseas, it isn't a goodwill
mission, it's a blooper reel''), followed by testimonials from Marie
Yovanovitch and Colin Powell, was a crucial reminder that foreign policy
matters and that Trump's stinks.

\textbf{Gail Collins}~~ It would be pretty hard to come out of the night
not liking Joe Biden at least a little bit. Or if you already did, a
little bit more.

\textbf{Michelle Cottle}~~ The group ``keynote address'' delivered by 17
Democratic up-and-comers --- county commissioners, mayors, state
legislators, etc. --- helped focus attention on the party's future
leaders, not just its past ones. It was a risk, and a tech challenge,
but it had great energy and flow. Having Stacey Abrams as the closer was
an especially nice touch.

\textbf{Michelle Goldberg}~~ The people who put this thing together
struck just the right tone, mixing grief-stricken sobriety with earnest
hope. But while I understand why Biden's team wants Republican
validators, his campaign didn't need to give more time to John McCain
than to A.O.C. and Stacey Abrams combined.

\textbf{Nicole Hemmer}~~ Just eight years ago, Bill Clinton stole the
convention with his lengthy prime-time speech. This year, his role
shrank to five unexceptional minutes, making clear that the Party of
Clinton has been fully eclipsed by the Party of Obama.

\textbf{Liz Mair}~~ Biden as listener and empathizer in chief works very
well. Biden moderating panels with ordinary Americans on issues makes
for some compelling and watchable content. But tonight's session on
health care --- when he talked about his personal experiences and
thoughts as Beau Biden lay dying of cancer --- showcased what's most
appealing about him: his personality, as opposed to a set of policies or
good soundbites, and the message that this is about you, not him.

\textbf{Daniel McCarthy}~~ Trump is the overwhelmingly dominant theme of
the convention so far --- forward-looking hope and change have given way
to anti-Trumpism and nostalgia.

\textbf{Melanye Price}~~ It was crystal clear that the people who know
Biden really like him. It's less clear whether he can maintain the crisp
and coherent narratives that have been so powerful here, as the campaign
continues.

\textbf{Mimi Swartz}~~ The roll call worked in that hokey American way
--- the landscapes, the regional accents, the hopes and enthusiasms, and
the kids holding up signs of what looked like Joe Biden's sunglasses.
Even Mayor Pete, on what looked like the movie set for ``Delegates in
Black.''

\textbf{Héctor Tobar}~~ The storytelling discipline of the producers of
this nightly infomercial is admirable. They're hammering away
relentlessly at two storylines: Joe Biden as the defender of common
Americans, with a compassion born of personal suffering; and Trump as
the nation's callous divider in chief.

\textbf{Peter Wehner}~~ Day 2 of the Democratic convention lacked the
energy and galvanizing moments of the first day. But what came through
to me is how effectively the convention is at humanizing Joe Biden.
Personal tragedy and loss are central to his story, and so, too, is
empathy, decency and healing. That doesn't guarantee he'll be a
successful president, but those qualities mean something, especially in
the age of Trump.

\emph{The Times is committed to publishing}
\href{https://www.nytimes3xbfgragh.onion/2019/01/31/opinion/letters/letters-to-editor-new-york-times-women.html}{a
diversity of letters} \emph{to the editor. We'd like to hear what you
think about this or any of our articles. Here are some}
\href{https://help.nytimes3xbfgragh.onion/hc/en-us/articles/115014925288-How-to-submit-a-letter-to-the-editor}{tips}.
And here's our email:
\href{mailto:letters@NYTimes.com}{\nolinkurl{letters@NYTimes.com}}.

Follow The New York Times Opinion section on
\href{https://www.facebookcorewwwi.onion/nytopinion}{Facebook}\emph{,}
\href{http://twitter.com/NYTOpinion}{Twitter (@NYTopinion)} \emph{and}
\href{https://www.instagram.com/nytopinion/}{Instagram}\emph{.}

\textbf{About the authors}

Jamelle Bouie, Frank Bruni, Gail Collins and Michelle Goldberg are Times
columnists.

Wajahat Ali
(\href{https://twitter.com/WajahatAli?ref_src=twsrc\%5Egoogle\%7Ctwcamp\%5Eserp\%7Ctwgr\%5Eauthor}{@WajahatAli})
is a playwright, lawyer and contributing opinion writer.

Michelle Cottle (\href{https://twitter.com/mcottle}{@mcottle}) is a
member of the Times editorial board.

Nicole Hemmer
(\href{https://twitter.com/pastpunditry?ref_src=twsrc\%5Egoogle\%7Ctwcamp\%5Eserp\%7Ctwgr\%5Eauthor}{@pastpunditry})
is an associate research scholar at Columbia University and the author
of ``Messengers of the Right: Conservative Media and the Transformation
of American Politics.''

Liz Mair (\href{https://twitter.com/LizMair}{@LizMair}), a strategist
for campaigns by Scott Walker, Roy Blunt, Rand Paul, Carly Fiorina and
Rick Perry, is the founder and president of Mair Strategies.

Daniel McCarthy
(\href{https://twitter.com/ToryAnarchist}{@ToryAnarchist}) is the editor
of \href{https://home.isi.org/modern-age}{Modern Age: A Conservative
Quarterly}.

Melanye Price (\href{https://twitter.com/ProfMTP}{@ProfMTP}), a
professor of political science at Prairie View A\&M University in Texas,
is the author, most recently, of
``\href{https://nyupress.org/9781479819256/the-race-whisperer/}{The Race
Whisperer: Barack Obama and the Political Uses of Race}.''

Mimi Swartz (\href{https://twitter.com/mimiswartz}{@mimiswartz}), an
executive editor at Texas Monthly, is a contributing opinion writer.

Héctor Tobar
(\href{https://twitter.com/TobarWriter?ref_src=twsrc\%5Egoogle\%7Ctwcamp\%5Eserp\%7Ctwgr\%5Eauthor}{@TobarWriter}),
an associate professor at the University of California, Irvine, is the
author of ``Deep Down Dark: The Untold Stories of 33 Men Buried in a
Chilean Mine, and the Miracle That Set Them Free'' and a contributing
opinion writer.

Peter Wehner
(\href{https://twitter.com/Peter_Wehner?ref_src=twsrc\%5Egoogle\%7Ctwcamp\%5Eserp\%7Ctwgr\%5Eauthor}{@Peter\_Wehner}),
a senior fellow at the Ethics and Public Policy Center, served in the
previous three Republican administrations, is a contributing opinion
writer and also the author of
``\href{https://www.harpercollins.com/9780062820792/the-death-of-politics/}{The
Death of Politics}: How to Heal Our Frayed Republic After Trump.''

Read 1357 Comments

\begin{itemize}
\item
\item
\item
\item
\end{itemize}

Advertisement

\protect\hyperlink{after-bottom}{Continue reading the main story}

\hypertarget{site-index}{%
\subsection{Site Index}\label{site-index}}

\hypertarget{site-information-navigation}{%
\subsection{Site Information
Navigation}\label{site-information-navigation}}

\begin{itemize}
\tightlist
\item
  \href{https://help.nytimes3xbfgragh.onion/hc/en-us/articles/115014792127-Copyright-notice}{©~2020~The
  New York Times Company}
\end{itemize}

\begin{itemize}
\tightlist
\item
  \href{https://www.nytco.com/}{NYTCo}
\item
  \href{https://help.nytimes3xbfgragh.onion/hc/en-us/articles/115015385887-Contact-Us}{Contact
  Us}
\item
  \href{https://www.nytco.com/careers/}{Work with us}
\item
  \href{https://nytmediakit.com/}{Advertise}
\item
  \href{http://www.tbrandstudio.com/}{T Brand Studio}
\item
  \href{https://www.nytimes3xbfgragh.onion/privacy/cookie-policy\#how-do-i-manage-trackers}{Your
  Ad Choices}
\item
  \href{https://www.nytimes3xbfgragh.onion/privacy}{Privacy}
\item
  \href{https://help.nytimes3xbfgragh.onion/hc/en-us/articles/115014893428-Terms-of-service}{Terms
  of Service}
\item
  \href{https://help.nytimes3xbfgragh.onion/hc/en-us/articles/115014893968-Terms-of-sale}{Terms
  of Sale}
\item
  \href{https://spiderbites.nytimes3xbfgragh.onion}{Site Map}
\item
  \href{https://help.nytimes3xbfgragh.onion/hc/en-us}{Help}
\item
  \href{https://www.nytimes3xbfgragh.onion/subscription?campaignId=37WXW}{Subscriptions}
\end{itemize}
