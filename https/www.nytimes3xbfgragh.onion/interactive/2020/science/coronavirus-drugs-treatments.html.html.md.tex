Sections

SEARCH

\protect\hyperlink{site-content}{Skip to
content}\protect\hyperlink{site-index}{Skip to site index}

\href{https://www.nytimes3xbfgragh.onion/section/science}{Science}

\href{https://myaccount.nytimes3xbfgragh.onion/auth/login?response_type=cookie\&client_id=vi}{}

\href{https://www.nytimes3xbfgragh.onion/section/todayspaper}{Today's
Paper}

\href{/section/science}{Science}\textbar{}Coronavirus Drug and Treatment
Tracker

\url{https://nyti.ms/3h684eG}

\begin{itemize}
\item
\item
\item
\item
\item
\end{itemize}

\hypertarget{the-coronavirus-outbreak}{%
\subsubsection{\texorpdfstring{\href{https://www.nytimes3xbfgragh.onion/news-event/coronavirus?name=styln-coronavirus-national\&region=TOP_BANNER\&variant=undefined\&block=storyline_menu_recirc\&action=click\&pgtype=Interactive\&impression_id=15f07e60-e39b-11ea-89f5-73e141fcc61b}{The
Coronavirus
Outbreak}}{The Coronavirus Outbreak}}\label{the-coronavirus-outbreak}}

\begin{itemize}
\tightlist
\item
  live\href{https://www.nytimes3xbfgragh.onion/2020/08/21/world/covid-19-coronavirus.html?name=styln-coronavirus-national\&region=TOP_BANNER\&variant=undefined\&block=storyline_menu_recirc\&action=click\&pgtype=Interactive\&impression_id=15f07e61-e39b-11ea-89f5-73e141fcc61b}{Latest
  Updates}
\item
  \href{https://www.nytimes3xbfgragh.onion/interactive/2020/us/coronavirus-us-cases.html?name=styln-coronavirus-national\&region=TOP_BANNER\&variant=undefined\&block=storyline_menu_recirc\&action=click\&pgtype=Interactive\&impression_id=15f07e62-e39b-11ea-89f5-73e141fcc61b}{Maps
  and Cases}
\item
  \href{https://www.nytimes3xbfgragh.onion/interactive/2020/science/coronavirus-vaccine-tracker.html?name=styln-coronavirus-national\&region=TOP_BANNER\&variant=undefined\&block=storyline_menu_recirc\&action=click\&pgtype=Interactive\&impression_id=15f07e63-e39b-11ea-89f5-73e141fcc61b}{Vaccine
  Tracker}
\item
  \href{https://www.nytimes3xbfgragh.onion/2020/08/19/us/colleges-closing-covid.html?name=styln-coronavirus-national\&region=TOP_BANNER\&variant=undefined\&block=storyline_menu_recirc\&action=click\&pgtype=Interactive\&impression_id=15f0a570-e39b-11ea-89f5-73e141fcc61b}{Colleges
  Closing}
\item
  \href{https://www.nytimes3xbfgragh.onion/live/2020/08/20/business/stock-market-today-coronavirus?name=styln-coronavirus-national\&region=TOP_BANNER\&variant=undefined\&block=storyline_menu_recirc\&action=click\&pgtype=Interactive\&impression_id=15f0a571-e39b-11ea-89f5-73e141fcc61b}{Economy}
\end{itemize}

\hypertarget{coronavirus-drug-and-treatment-tracker}{%
\section{Coronavirus Drug and Treatment
Tracker}\label{coronavirus-drug-and-treatment-tracker}}

By \href{https://www.nytimes3xbfgragh.onion/by/jonathan-corum}{Jonathan
Corum},
\href{https://www.nytimes3xbfgragh.onion/by/katherine-j--wu}{Katherine
J. Wu} and \href{https://www.nytimes3xbfgragh.onion/by/carl-zimmer}{Carl
Zimmer}Updated August 10, 2020

\href{https://www.nytimes3xbfgragh.onion/es/interactive/2020/science/coronavirus-tratamientos-curas.html}{Leer
en español}

\begin{itemize}
\item
\item
\item
\item
\end{itemize}

\href{https://www.nytimes3xbfgragh.onion/interactive/2020/world/coronavirus-maps.html}{World}~

COUNTRIES

\textbar{}
\href{https://www.nytimes3xbfgragh.onion/interactive/2020/us/coronavirus-us-cases.html}{U.S.A.}~

STATES

~
\href{https://www.nytimes3xbfgragh.onion/interactive/2020/us/coronavirus-testing.html}{Testing}

\href{https://www.nytimes3xbfgragh.onion/interactive/2020/world/americas/brazil-coronavirus-cases.html}{Brazil}\href{https://www.nytimes3xbfgragh.onion/interactive/2020/world/canada/canada-coronavirus-cases.html}{Canada}\href{https://www.nytimes3xbfgragh.onion/interactive/2020/world/europe/france-coronavirus-cases.html}{France}\href{https://www.nytimes3xbfgragh.onion/interactive/2020/world/europe/germany-coronavirus-cases.html}{Germany}\href{https://www.nytimes3xbfgragh.onion/interactive/2020/world/asia/india-coronavirus-cases.html}{India}\href{https://www.nytimes3xbfgragh.onion/interactive/2020/world/europe/italy-coronavirus-cases.html}{Italy}\href{https://www.nytimes3xbfgragh.onion/interactive/2020/world/americas/mexico-coronavirus-cases.html}{Mexico}\href{https://www.nytimes3xbfgragh.onion/interactive/2020/world/europe/spain-coronavirus-cases.html}{Spain}\href{https://www.nytimes3xbfgragh.onion/interactive/2020/world/europe/united-kingdom-coronavirus-cases.html}{U.K.}

\href{https://www.nytimes3xbfgragh.onion/interactive/2020/us/alabama-coronavirus-cases.html}{Alabama}\href{https://www.nytimes3xbfgragh.onion/interactive/2020/us/alaska-coronavirus-cases.html}{Alaska}\href{https://www.nytimes3xbfgragh.onion/interactive/2020/us/arizona-coronavirus-cases.html}{Arizona}\href{https://www.nytimes3xbfgragh.onion/interactive/2020/us/arkansas-coronavirus-cases.html}{Arkansas}\href{https://www.nytimes3xbfgragh.onion/interactive/2020/us/california-coronavirus-cases.html}{California}\href{https://www.nytimes3xbfgragh.onion/interactive/2020/us/colorado-coronavirus-cases.html}{Colorado}\href{https://www.nytimes3xbfgragh.onion/interactive/2020/us/connecticut-coronavirus-cases.html}{Connecticut}\href{https://www.nytimes3xbfgragh.onion/interactive/2020/us/delaware-coronavirus-cases.html}{Delaware}\href{https://www.nytimes3xbfgragh.onion/interactive/2020/us/florida-coronavirus-cases.html}{Florida}\href{https://www.nytimes3xbfgragh.onion/interactive/2020/us/georgia-coronavirus-cases.html}{Georgia}\href{https://www.nytimes3xbfgragh.onion/interactive/2020/us/hawaii-coronavirus-cases.html}{Hawaii}\href{https://www.nytimes3xbfgragh.onion/interactive/2020/us/idaho-coronavirus-cases.html}{Idaho}\href{https://www.nytimes3xbfgragh.onion/interactive/2020/us/illinois-coronavirus-cases.html}{Illinois}\href{https://www.nytimes3xbfgragh.onion/interactive/2020/us/indiana-coronavirus-cases.html}{Indiana}\href{https://www.nytimes3xbfgragh.onion/interactive/2020/us/iowa-coronavirus-cases.html}{Iowa}\href{https://www.nytimes3xbfgragh.onion/interactive/2020/us/kansas-coronavirus-cases.html}{Kansas}\href{https://www.nytimes3xbfgragh.onion/interactive/2020/us/kentucky-coronavirus-cases.html}{Kentucky}\href{https://www.nytimes3xbfgragh.onion/interactive/2020/us/louisiana-coronavirus-cases.html}{Louisiana}\href{https://www.nytimes3xbfgragh.onion/interactive/2020/us/maine-coronavirus-cases.html}{Maine}\href{https://www.nytimes3xbfgragh.onion/interactive/2020/us/maryland-coronavirus-cases.html}{Maryland}\href{https://www.nytimes3xbfgragh.onion/interactive/2020/us/massachusetts-coronavirus-cases.html}{Massachusetts}\href{https://www.nytimes3xbfgragh.onion/interactive/2020/us/michigan-coronavirus-cases.html}{Michigan}\href{https://www.nytimes3xbfgragh.onion/interactive/2020/us/minnesota-coronavirus-cases.html}{Minnesota}\href{https://www.nytimes3xbfgragh.onion/interactive/2020/us/mississippi-coronavirus-cases.html}{Mississippi}\href{https://www.nytimes3xbfgragh.onion/interactive/2020/us/missouri-coronavirus-cases.html}{Missouri}\href{https://www.nytimes3xbfgragh.onion/interactive/2020/us/montana-coronavirus-cases.html}{Montana}\href{https://www.nytimes3xbfgragh.onion/interactive/2020/us/nebraska-coronavirus-cases.html}{Nebraska}\href{https://www.nytimes3xbfgragh.onion/interactive/2020/us/nevada-coronavirus-cases.html}{Nevada}\href{https://www.nytimes3xbfgragh.onion/interactive/2020/us/new-hampshire-coronavirus-cases.html}{New
Hampshire}\href{https://www.nytimes3xbfgragh.onion/interactive/2020/us/new-jersey-coronavirus-cases.html}{New
Jersey}\href{https://www.nytimes3xbfgragh.onion/interactive/2020/us/new-mexico-coronavirus-cases.html}{New
Mexico}\href{https://www.nytimes3xbfgragh.onion/interactive/2020/us/new-york-coronavirus-cases.html}{New
York}\href{https://www.nytimes3xbfgragh.onion/interactive/2020/us/north-carolina-coronavirus-cases.html}{North
Carolina}\href{https://www.nytimes3xbfgragh.onion/interactive/2020/us/north-dakota-coronavirus-cases.html}{North
Dakota}\href{https://www.nytimes3xbfgragh.onion/interactive/2020/us/ohio-coronavirus-cases.html}{Ohio}\href{https://www.nytimes3xbfgragh.onion/interactive/2020/us/oklahoma-coronavirus-cases.html}{Oklahoma}\href{https://www.nytimes3xbfgragh.onion/interactive/2020/us/oregon-coronavirus-cases.html}{Oregon}\href{https://www.nytimes3xbfgragh.onion/interactive/2020/us/pennsylvania-coronavirus-cases.html}{Pennsylvania}\href{https://www.nytimes3xbfgragh.onion/interactive/2020/us/puerto-rico-coronavirus-cases.html}{Puerto
Rico}\href{https://www.nytimes3xbfgragh.onion/interactive/2020/us/rhode-island-coronavirus-cases.html}{Rhode
Island}\href{https://www.nytimes3xbfgragh.onion/interactive/2020/us/south-carolina-coronavirus-cases.html}{South
Carolina}\href{https://www.nytimes3xbfgragh.onion/interactive/2020/us/south-dakota-coronavirus-cases.html}{South
Dakota}\href{https://www.nytimes3xbfgragh.onion/interactive/2020/us/tennessee-coronavirus-cases.html}{Tennessee}\href{https://www.nytimes3xbfgragh.onion/interactive/2020/us/texas-coronavirus-cases.html}{Texas}\href{https://www.nytimes3xbfgragh.onion/interactive/2020/us/utah-coronavirus-cases.html}{Utah}\href{https://www.nytimes3xbfgragh.onion/interactive/2020/us/vermont-coronavirus-cases.html}{Vermont}\href{https://www.nytimes3xbfgragh.onion/interactive/2020/us/virginia-coronavirus-cases.html}{Virginia}\href{https://www.nytimes3xbfgragh.onion/interactive/2020/us/washington-coronavirus-cases.html}{Washington}\href{https://www.nytimes3xbfgragh.onion/interactive/2020/us/washington-dc-coronavirus-cases.html}{Washington,
D.C.}\href{https://www.nytimes3xbfgragh.onion/interactive/2020/us/west-virginia-coronavirus-cases.html}{West
Virginia}\href{https://www.nytimes3xbfgragh.onion/interactive/2020/us/wisconsin-coronavirus-cases.html}{Wisconsin}\href{https://www.nytimes3xbfgragh.onion/interactive/2020/us/wyoming-coronavirus-cases.html}{Wyoming}

~

The Covid-19 pandemic is one of the greatest challenges modern medicine
has ever faced. Doctors and scientists are scrambling to find treatments
and drugs that can save the lives of infected people and perhaps even
prevent them from getting sick in the first place.

Below is an updated list of \textbf{20 of the most-talked-about
treatments for the coronavirus}. While some are accumulating evidence
that they're effective, most are still at early stages of research. We
also included a warning about a few that are just bunk.

We are following 20 coronavirus treatments for effectiveness and safety:

2

2

11

2

3

Widely

used

Promising

evidence

Tentative or

mixed evidence

Not

promising

Pseudoscience

or fraud

We are following 20 coronavirus treatments

for effectiveness and safety:

2

11

2

Promising

evidence

Tentative or

mixed evidence

2

3

Widely

used

Not

promising

Pseudoscience

or fraud

We are following 20 coronavirus treatments

for effectiveness and safety:

2

11

2

Promising

evidence

Mixed

evidence

2

3

Widely

used

Not

promising

Pseudoscience

or fraud

There is no cure yet for Covid-19. And even the most promising
treatments to date only help certain groups of patients and await
validation from further trials. The F.D.A. has not fully licensed any
treatment specifically for the coronavirus. Although it has granted
\href{https://www.fda.gov/emergency-preparedness-and-response/mcm-legal-regulatory-and-policy-framework/emergency-use-authorization}{emergency
use authorization} to some treatments, their effectiveness against
Covid-19 has yet to be demonstrated in large-scale, randomized clinical
trials.

This list provides a snapshot of the latest research on the coronavirus,
but does not constitute medical endorsements. Always consult your doctor
about treatments for Covid-19.

\textbf{New additions and recent updates}:

•~ Added \textbf{\protect\hyperlink{ivermectin}{ivermectin}}, a drug
typically used against parasitic worms that is being increasingly
prescribed in Latin America. Aug. 10

•~ Updated descriptions for several treatments. Aug. 10

We will update and expand the list as new evidence emerges. For details
on evaluating treatments, see the
\href{https://www.covid19treatmentguidelines.nih.gov/}{N.I.H. Covid-19
Treatment Guidelines}. For the current status of vaccine development,
see our
\href{https://www.nytimes3xbfgragh.onion/interactive/2020/science/coronavirus-vaccine-tracker.html}{Coronavirus
Vaccine Tracker}.

\hypertarget{what-the-labels-mean}{%
\subsection{What the Labels~Mean}\label{what-the-labels-mean}}

WIDELY USED: These treatments have been used widely by doctors and
nurses to treat patients hospitalized for diseases that affect the
respiratory system, including Covid-19.

PROMISING EVIDENCE: Early evidence from studies on patients suggests
effectiveness, but more research is needed. This category includes
treatments that have shown improvements in morbidity, mortality and
recovery in at least one randomized controlled trial, in which some
people get a treatment and others get a placebo.

TENTATIVE OR MIXED EVIDENCE: Some treatments show promising results in
cells or animals, which need to be confirmed in people. Others have
yielded encouraging results in retrospective studies in humans, which
look at existing datasets rather than starting a new trial. Some
treatments have produced different results in different experiments,
raising the need for larger, more rigorously designed studies to clear
up the confusion.

NOT PROMISING: **** Early evidence suggests that these treatments do not
work.

PSEUDOSCIENCE OR FRAUD: These are not treatments that researchers have
ever considered using for Covid-19. Experts have warned against trying
them, because they do not help against the disease and can instead be
dangerous. Some people have even been arrested for their false promises
of a Covid-19 cure.

EVIDENCE IN CELLS, ANIMALS or HUMANS: These labels indicate where the
evidence for a treatment comes from. Researchers often start out with
experiments on cells and then move onto animals. Many of those animal
experiments often fail; if they don't, researchers may consider moving
on to research on humans, such as retrospective studies or randomized
clinical trials. In some cases, scientists are testing out treatments
that were developed for other diseases, allowing them to move directly
to human trials for Covid-19.

Filter the list of treatments:

All treatments

Widely used

Promising

Tentative or mixed

Not promising

Pseudoscience

\hypertarget{blocking-the-virus}{%
\subsection{Blocking the~Virus}\label{blocking-the-virus}}

\emph{Antivirals can stop viruses such as H.I.V. and hepatitis C from
hijacking our cells. Scientists are searching for antivirals that work
against the new coronavirus.}

PROMISING EVIDENCE EVIDENCE IN CELLS, ANIMALS AND HUMANS EMERGENCY USE
AUTHORIZATION\\
Remdesivir\\
\href{https://www.nytimes3xbfgragh.onion/2020/05/23/health/coronavirus-remdesivir.html}{Remdesivir},
made by Gilead Sciences, was the first drug to get emergency
authorization from the F.D.A. for use on Covid-19. It stops viruses from
replicating by inserting itself into new viral genes. Remdesivir was
originally tested as an antiviral against Ebola and Hepatitis C, only to
deliver lackluster results. But preliminary data from trials that began
this spring suggested the drug can reduce the recovery time of people
hospitalized with Covid-19 from
\href{https://www.nejm.org/doi/full/10.1056/NEJMoa2007764}{15 to 11
days}. (The study defined recovery as ``either discharge from the
hospital or hospitalization for infection-control purposes only.'')
These early results did not show any effect on mortality, though
retrospective data released in July hints that the drug might
\href{https://www.gilead.com/news-and-press/press-room/press-releases/2020/7/gilead-presents-additional-data-on-investigational-antiviral-remdesivir-for-the-treatment-of-covid-19}{reduce
death rates} among those who are very ill.

TENTATIVE OR MIXED EVIDENCE EVIDENCE IN CELLS, ANIMALS AND HUMANS\\
Favipiravir\\
Originally designed to beat back influenza, favipiravir blocks a virus's
ability to copy its genetic material. A
\href{https://www.sciencedirect.com/science/article/pii/S2095809920300631?via\%3Dihub}{small
study} in March indicated the drug might help purge the coronavirus from
the airway, but results from larger, well-designed clinical trials are
still pending.

TENTATIVE OR MIXED EVIDENCE EVIDENCE IN CELLS, ANIMALS AND HUMANS\\
MK-4482\\
Another antiviral originally designed to fight the flu, MK-4482
(previously known as EIDD-2801) has had promising
\href{https://stm.sciencemag.org/content/12/541/eabb5883}{results}
against the new coronavirus in studies in cells and on animals. Merck,
which has been running clinical trials on the drug this summer, has
\href{https://www.cnbc.com/2020/07/31/merck-aims-to-start-large-pivotal-studies-on-coronavirus-treatment-in-september.html}{announced}
it will launch a large Phase III trial in September.\\
Updated Aug. 6

TENTATIVE OR MIXED EVIDENCE EVIDENCE IN CELLS\\
Recombinant ACE-2\\
To enter cells, the coronavirus must first
\href{https://www.nytimes3xbfgragh.onion/interactive/2020/03/11/science/how-coronavirus-hijacks-your-cells.html}{unlock
them} --- a feat it accomplishes by latching onto a human protein called
ACE-2. Scientists have created artificial ACE-2 proteins which might be
able to act as decoys, luring the coronavirus away from vulnerable
cells. Recombinant ACE-2 proteins have shown promising
\href{https://doi.org/10.1016/j.cell.2020.04.004}{results} in
experiments on cells, but not yet in animals or people.

TENTATIVE OR MIXED EVIDENCE EVIDENCE IN CELLS AND HUMANS\\
Ivermectin\\
For decades, ivermectin has served as a potent drug to treat parasitic
worms. Doctors use it against river blindness and other diseases, while
veterinarians give dogs a different formulation to cure heartworm.
Studies on cells have suggested ivermectin might also kill viruses. But
scientists have yet to find evidence in animal studies or human trials
that it can treat viral diseases. As a result, Ivermectin is not
approved to use as an antiviral.

In April, Australian researchers
\href{https://www.sciencedirect.com/science/article/pii/S0166354220302011}{reported}
that the drug blocked coronaviruses in cell cultures, but they used a
dosage that was so high it might have dangerous side effects in people.
The FDA immediately issued a
\href{https://www.fda.gov/animal-veterinary/product-safety-information/fda-letter-stakeholders-do-not-use-ivermectin-intended-animals-treatment-covid-19-humans}{warning}
against taking pet medications to treat or prevent Covid-19. ``These
animal drugs can cause serious harm in people,'' the agency warned.

Since then a number of clinical trials have been launched to see if a
safe dose of ivermectin can fight Covid-19. In Singapore, for example,
the National University Hospital is running a 5,000-person
\href{https://clinicaltrials.gov/ct2/show/NCT04446104}{trial} to see if
it can prevent people from getting infected. As of now, there's no firm
evidence that it works. Nevertheless ivermectin is being prescribed
\href{https://www.nytimes3xbfgragh.onion/2020/07/23/world/americas/chlorine-coronavirus-bolivia-latin-america.html?searchResultPosition=1}{increasingly
often} in Latin America,
\href{http://www.ajtmh.org/content/journals/10.4269/ajtmh.20-0271}{much
to the distress} of disease experts.\\
Updated Aug. 10

NOT PROMISING EVIDENCE IN CELLS AND HUMANS\\
Lopinavir and ritonavir\\
Twenty years ago, the F.D.A. approved this combination of drugs to treat
H.I.V. Recently, researchers tried them out on the new coronavirus and
found that they stopped the virus from replicating. But clinical trials
in patients proved disappointing. In early July, the World Health
Organization
\href{https://www.who.int/news-room/detail/04-07-2020-who-discontinues-hydroxychloroquine-and-lopinavir-ritonavir-treatment-arms-for-covid-19}{suspended}
trials on patients hospitalized for Covid-19. They didn't rule out
studies to see if the drugs could help patients not sick enough to be
hospitalized, or to prevent people exposed to the new coronavirus from
falling ill. The drug could also still have a role to play in certain
\href{https://www.thelancet.com/journals/lancet/article/PIIS0140-6736(20)31042-4/fulltext}{combination
treatments}.

NOT PROMISING EVIDENCE IN CELLS, ANIMALS AND HUMANS\\
Hydroxychloroquine and chloroquine\\
German chemists synthesized chloroquine in the 1930s as a drug against
malaria. A less toxic version, called hydroxychloroquine, was
\href{https://www.nature.com/articles/s41421-020-0156-0\#:~:text=Hydroxychloroquine\%20(HCQ)\%20sulfate\%2C\%20a,than\%20CQ\%20in\%20animals4.}{invented
in 1946}, and later was approved for other diseases such as lupus and
rheumatoid arthritis. At the start of the Covid-19 pandemic, researchers
discovered that both drugs could stop the coronavirus from replicating
in cells.

Since then, they've had a tumultuous ride. A few small studies on
patients offered some hope that hydroxychloroquine could treat Covid-19.
The World Health Organization launched a randomized clinical trial in
March to see if it was indeed safe and effective for Covid-19, as did
Novartis and a number of universities. Meanwhile, President Trump
repeatedly
\href{https://www.nytimes3xbfgragh.onion/2020/04/06/us/politics/coronavirus-trump-malaria-drug.html}{promoted
hydroxychloroquine} at press conferences, touting it as a ``game
changer,'' and even
\href{https://www.nytimes3xbfgragh.onion/2020/05/18/us/politics/trump-hydroxychloroquine-covid-coronavirus.html}{took
it himself}. The F.D.A. temporarily granted hydroxychloroquine emergency
authorization for use in Covid-19 patients --- which a whistleblower
later
\href{https://www.buzzfeednews.com/article/zahrahirji/fda-eua-hydroxychloroquine-chloroquine}{claimed}
was the result of political pressure. In the wake of the drug's newfound
publicity,
\href{https://www.nytimes3xbfgragh.onion/2020/04/25/us/coronavirus-trump-chloroquine-hydroxychloroquine.html}{demand
spiked}, resulting in
\href{https://ard.bmj.com/content/early/2020/07/01/annrheumdis-2020-218164}{shortages}
for people who rely on hydroxychloroquine as a treatment for other
diseases.

But more detailed studies proved disappointing. A
\href{https://www.nature.com/articles/s41586-020-2558-4}{study} on
monkeys found that hydroxychloroquine didn't prevent the animals from
getting infected and didn't clear the virus once they got sick.
Randomized clinical trials found that hydroxychloroquine
\href{https://www.recoverytrial.net/news/statement-from-the-chief-investigators-of-the-randomised-evaluation-of-covid-19-therapy-recovery-trial-on-hydroxychloroquine-5-june-2020-no-clinical-benefit-from-use-of-hydroxychloroquine-in-hospitalised-patients-with-covid-19}{didn't
help people with Covid-19 get better} or
\href{https://www.nytimes3xbfgragh.onion/2020/06/03/health/hydroxychloroquine-coronavirus-trump.html}{prevent
healthy people from contracting the coronavirus}. Another
\href{https://www.acpjournals.org/doi/10.7326/M20-4207}{randomized
clinical trial} found that giving hydroxychloroquine to people right
after being diagnosed with Covid-19 didn't reduce the severity of their
disease. (One large-scale study that concluded the drug was harmful as
well was later
\href{https://www.nytimes3xbfgragh.onion/2020/06/04/health/coronavirus-hydroxychloroquine.html?searchResultPosition=1}{retracted}.)
The
\href{https://www.who.int/news-room/detail/04-07-2020-who-discontinues-hydroxychloroquine-and-lopinavir-ritonavir-treatment-arms-for-covid-19}{World
Health Organization}, the National Institutes of Health and Novartis
have since halted trials investigating hydroxychloroquine as a treatment
for Covid-19, and the F.D.A.
\href{https://www.nytimes3xbfgragh.onion/2020/06/15/health/fda-hydroxychloroquine-malaria.html}{revoked
its emergency approval}. The F.D.A. now
\href{https://www.fda.gov/drugs/drug-safety-and-availability/fda-cautions-against-use-hydroxychloroquine-or-chloroquine-covid-19-outside-hospital-setting-or}{warns}
that the drug can cause a host of serious side effects to the heart and
other organs when used to treat Covid-19.

In July, researchers at Henry Ford hospital in Detroit published a
\href{https://www.statnews.com/2020/07/08/a-flawed-covid-19-study-gets-the-white-houses-attention-and-the-fda-may-pay-the-price/}{study}
finding that hydroxychloroquine was associated with a reduction in
mortality in Covid-19 patients. President Trump
\href{https://twitter.com/realDonaldTrump/status/1280328830218051584}{praised
the study on Twitter}, but experts raised
\href{https://www.sciencedirect.com/science/article/pii/S1201971220305300?dgcid=rss_sd_all}{doubts}
about it. The study was not a randomized controlled trial, in which some
people got a placebo instead of hydroxychloroquine. The study's results
might not be due to the drug killing the virus. Instead, doctors may
have given the drug to people who were less sick, and thus more likely
to recover anyway.

Despite negative results, a number of hydroxychloroquine trials have
continued, although most are small, testing a few dozen or a few hundred
patients. A recent analysis by STAT and Applied XL found more than 180
\href{https://www.statnews.com/2020/07/06/data-show-panic-and-disorganization-dominate-the-study-of-covid-19-drugs/}{ongoing
clinical trials} testing hydroxychloroquine or chloroquine, for treating
or preventing Covid-19. Although it's clear the drugs are no panacea,
it's theoretically possible they could provide some benefit in
combination with other treatments, or when given in early stages of the
disease. Only well-designed trials can determine if that's the case.\\
Updated Aug. 10

\hypertarget{mimicking-the-immune-system}{%
\subsection{Mimicking the
Immune~System}\label{mimicking-the-immune-system}}

\emph{Most people who get Covid-19 successfully fight off the virus with
a strong immune response. Drugs might help people who can't mount an
adequate defense.}

TENTATIVE OR MIXED EVIDENCE EVIDENCE IN CELLS AND HUMANS\\
Convalescent plasma\\
A century ago, doctors filtered plasma from the blood of recovered flu
patients. So-called convalescent plasma, rich with antibodies, helped
people sick with flu fight their illness. Now researchers are trying out
this
\href{https://www.nytimes3xbfgragh.onion/2020/04/24/smarter-living/coronavirus-convalescent-plasma-antibodies.html?searchResultPosition=1}{strategy}
on Covid-19. In May, the F.D.A. designated convalescent plasma an
``\href{https://www.fda.gov/vaccines-blood-biologics/investigational-new-drug-ind-or-device-exemption-ide-process-cber/recommendations-investigational-covid-19-convalescent-plasma}{investigational
product.}'' That means that despite not yet being shown as safe and
effective, plasma can be used in clinical trials and given to some
patients who are seriously ill with Covid-19. Tens of thousands of
patients in the U.S. have received plasma through a program launched by
the Mayo Clinic and the federal government.

The Trump administration has praised convalescent plasma, despite the
lack of evidence yet that it works. The first wave of trials have been
\href{https://www.nytimes3xbfgragh.onion/2020/05/22/health/coronarvirus-convalescent-serum.html?searchResultPosition=2}{small}
and the results have been mixed. Large randomized clinical trials are
underway, but they've
\href{https://www.nytimes3xbfgragh.onion/2020/08/04/health/trump-plasma.html}{struggled
to enroll enough participants}, some of whom worry they will receive a
placebo instead of the treatment itself.

Experts say that it's vital to complete these trials to determine if
convalescent plasma is safe and effective. If these trials are
successful, it could serve as an important stopgap measure until more
potent therapies become widely available.\\
Updated Aug. 10

TENTATIVE OR MIXED EVIDENCE EVIDENCE IN CELLS, ANIMALS AND HUMANS\\
Monoclonal antibodies\\
Convalescent plasma from people who recover from Covid-19 contains a mix
of different antibodies. Some of the molecules can attack the
coronavirus, but many are directed at other pathogens. Researchers have
sifted through this slurry to find the most potent antibodies against
Covid-19. They have then manufactured synthetic copies of these
molecules, known as monoclonal antibodies. Researchers have begun
investigating them as a treatment for Covid-19, either individually or
in cocktails.

Monoclonal antibodies were first developed as a therapy in the 1970s,
and since then the F.D.A. has approved them for
\href{https://jbiomedsci.biomedcentral.com/articles/10.1186/s12929-019-0592-z}{79
diseases}, ranging from cancer to AIDS. Since the start of the pandemic,
researchers have found dozens of monoclonal antibodies that show promise
against Covid-19 in
\href{https://www.nytimes3xbfgragh.onion/reuters/2020/08/03/us/03reuters-health-coronavirus-regeneron.html}{preclinical
studies} on cells and animals.
\href{https://www.nytimes3xbfgragh.onion/2020/07/09/health/regeneron-monoclonal-antibodies.html}{Companies
like Eli Lilly and Regeneron recently began clinical trials} studying
monoclonal antibodies. Several other firms, as well as teams at
universities, are slated to enter the race soon as well.\\
Updated Aug. 10

TENTATIVE OR MIXED EVIDENCE EVIDENCE IN CELLS, ANIMALS AND HUMANS\\
Interferons\\
Interferons are molecules our cells naturally produce in response to
viruses. They have profound effects on the immune system, rousing it to
attack the invaders, while also reining it in to avoid damaging the
body's own tissues. Injecting synthetic interferons is now a standard
treatment for a number of immune disorders.
\href{https://www.nationalmssociety.org/Treating-MS/Medications/Rebif}{Rebif},
for example, is prescribed for multiple sclerosis.

As part of its strategy to attack our bodies, the coronavirus appears to
\href{https://www.nytimes3xbfgragh.onion/2020/08/04/health/coronavirus-immune-system.html}{tamp
down interferon}. That finding has encouraged researchers to see whether
a boost of interferon might help people weather Covid-19, particularly
early in infection. Early studies, including experiments in
\href{https://academic.oup.com/jid/article/doi/10.1093/infdis/jiaa350/5860074}{cells}
and \href{https://pubmed.ncbi.nlm.nih.gov/32511406/}{mice}, have yielded
encouraging results that have led to clinical trials.

An open-label study in China suggested that the molecules could
\href{https://www.medrxiv.org/content/10.1101/2020.04.11.20061473v2}{help
prevent healthy people from getting infected}. On July 20, the British
pharmaceutical company Synairgen
\href{https://www.nytimes3xbfgragh.onion/2020/07/20/world/covid-19-treatment-synairgen-interferon-beta.html}{announced}
that an inhaled form of interferon called SNG001 lowered the risk of
severe Covid-19 in infected patients in a small clinical trial. The full
data have not yet been released to the public, or published in a
scientific journal. On August 6, the National Institute of Allergy and
Infectious Diseases
\href{https://www.nih.gov/news-events/news-releases/nih-clinical-trial-testing-remdesivir-plus-interferon-beta-1a-covid-19-treatment-begins}{launched}
a Phase III trial on a combination of Rebif and the antiviral
remdesivir, with results expected by fall 2020.\\
Updated Aug. 10

\hypertarget{putting-out-friendly-fire}{%
\subsection{Putting Out Friendly~Fire}\label{putting-out-friendly-fire}}

\emph{The most severe symptoms of Covid-19 are the result of the immune
system's overreaction to the virus. Scientists are testing drugs that
can rein in its attack.}

PROMISING EVIDENCE EVIDENCE IN HUMANS\\
Dexamethasone\\
This cheap and widely available steroid blunts many types of immune
responses. Doctors have long used it to treat allergies, asthma and
inflammation. In June, it became the first drug shown to
\href{https://www.nytimes3xbfgragh.onion/2020/06/16/world/europe/dexamethasone-coronavirus-covid.html?searchResultPosition=5}{reduce
Covid-19 deaths}. That
\href{https://www.nejm.org/doi/full/10.1056/NEJMoa2021436?query=featured_home}{study}
of more than 6,000 people, which in July was published in the New
England Journal of Medicine, found that dexamethasone reduced deaths by
one-third in patients on ventilators, and by one-fifth in patients on
oxygen. It may be
\href{https://www.nytimes3xbfgragh.onion/2020/06/24/health/coronavirus-dexamethasone.html?searchResultPosition=2}{less
likely to help} --- and may even harm --- patients who are at an earlier
stage of Covid-19 infections, however. In its Covid-19 treatment
guidelines, the National Institutes of Health
\href{https://www.covid19treatmentguidelines.nih.gov/dexamethasone/}{recommends}
only using dexamethasone in patients with COVID-19 who are on a
ventilator or are receiving supplemental oxygen.

TENTATIVE OR MIXED EVIDENCE EVIDENCE IN HUMANS\\
Cytokine Inhibitors\\
The body produces signaling molecules called cytokines to fight off
diseases. But manufactured in excess, cytokines can trigger the immune
system to wildly overreact to infections, in a process sometimes called
a cytokine storm. Researchers have created a number of drugs to halt
cytokine storms, and they have proven effective against arthritis and
other inflammatory disorders. Some turn off the supply of molecules that
launch the production of the cytokines themselves. Others block the
receptors on immune cells to which cytokines would normally bind. A few
block the cellular messages they send. Depending on how the drugs are
formulated, they can block one cytokine at a time, or muffle signals
from several at once.

Against the coronavirus, several of these drugs have
\href{https://www.medrxiv.org/content/10.1101/2020.06.01.20119149v2}{offered
modest help} in some trials, but faltered in others. Drug companies
Regeneron and Roche drug both
\href{https://newsroom.regeneron.com/news-releases/news-release-details/regeneron-and-sanofi-provide-update-kevzarar-sarilumab-phase-3}{recently}
\href{https://www.roche.com/investors/updates/inv-update-2020-07-29.htm}{announced}
that two drugs called sarilumab and tocilizumab, which both target the
cytokine IL-6, did not appear to benefit patients in Phase 3 clinical
trials. Many other trials remain underway, several of which combine
cytokine inhibitors with other treatments.\\
Updated Aug. 10

TENTATIVE OR MIXED EVIDENCE EVIDENCE IN HUMANS EMERGENCY USE
AUTHORIZATION\\
Blood filtration systems\\
The F.D.A. has
\href{https://www.nytimes3xbfgragh.onion/2020/08/10/health/stephen-hahn-fda.html}{granted
emergency use authorization} to several devices that
\href{https://www.nytimes3xbfgragh.onion/2020/06/11/health/coronavirus-cytokine-storm.html}{filter
cytokines} from the blood in an attempt to cool
\href{https://www.nytimes3xbfgragh.onion/2020/06/11/health/coronavirus-cytokine-storm.html}{cytokine
storms}. One machine, called Cytosorb, can reportedly purify a patient's
entire blood supply about 70 times in a 24-hour period. A small study in
March suggested that Cytosorb had helped dozens of severely ill Covid-19
patients in Europe and China, but it was not a randomized clinical trial
that could conclusively demonstrate it was effective. A number of
studies on blood filtration systems are underway, but experts caution
that these devices carry some risks. For example, such filters could
\href{https://www.fda.gov/media/136866/download}{remove beneficial
components} of blood as well, such as vitamins or medications.\\
Updated Aug. 10

TENTATIVE OR MIXED EVIDENCE EVIDENCE IN HUMANS\\
Stem cells\\
Certain kinds of stem cells can secrete anti-inflammatory molecules.
Over the years, researchers have tried to use them as a
\href{https://celltrials.org/news/role-msc-treat-coronavirus-pneumonia-and-ards-part-1-is-emperor-wearing-clothes}{treatment
for cytokine storms}, and now dozens of clinical
\href{https://clinicaltrials.gov/ct2/results?term=stem+cell\&cond=COVID-19\&age_v=\&gndr=\&type=\&rslt=\&phase=0\&phase=1\&phase=2\&phase=3\&Search=Apply}{trials}
are under way to see if they can help patients with Covid-19. But these
stem cell treatments haven't worked well in the past, and it's not clear
yet if they'll work against the coronavirus.

\hypertarget{other-treatments}{%
\subsection{Other Treatments}\label{other-treatments}}

\emph{Doctors and nurses often administer other supportive treatments to
help patients with Covid-19.}

WIDELY USED\\
Prone positioning\\
The simple act of flipping Covid-19 patients onto their bellies
\href{https://www.nytimes3xbfgragh.onion/2020/05/13/health/coronavirus-proning-lungs.html}{opens
up the lungs}. The maneuver has become commonplace in hospitals around
the world since the start of the pandemic. It might help some
individuals avoid the need for ventilators entirely. The treatment's
benefits continue to be tested in a range of clinical trials.

WIDELY USED EMERGENCY USE AUTHORIZATION\\
Ventilators and other respiratory support devices\\
Devices that help people breathe are an essential tool in the fight
against deadly respiratory illnesses. Some patients do well if they get
an extra supply of oxygen through the nose or via a mask connected to an
oxygen machine. Patients in severe respiratory distress may need to have
a
\href{https://www.nytimes3xbfgragh.onion/interactive/2020/05/08/health/coronavirus-covid-lungs-ventilators.html}{ventilator
breathe for them} until their lungs heal. Doctors are divided about how
long to treat patients with noninvasive oxygen before deciding whether
or not they need a ventilator. Not all Covid-19 patients who go on
ventilators survive, but the devices are thought to be
\href{https://www.nytimes3xbfgragh.onion/2020/04/26/health/coronavirus-patient-ventilator.html}{lifesaving
in many cases}.

TENTATIVE OR MIXED EVIDENCE EVIDENCE IN HUMANS\\
Anticoagulants\\
The coronavirus can invade cells in the lining of blood vessels, leading
to tiny clots that can cause strokes and other serious harm.
Anticoagulants are commonly used for other conditions, such as heart
disease, to slow the formation of clots, and doctors sometimes use them
on patients with Covid-19 who have clots. Many clinical trials teasing
out this relationship are now underway. Some of these trials are looking
at whether giving anticoagulants before any sign of clotting is
beneficial.

\hypertarget{pseudoscience-and-fraud}{%
\subsection{Pseudoscience and~Fraud}\label{pseudoscience-and-fraud}}

\emph{False claims about Covid-19 cures abound. The F.D.A. maintains a}
\href{https://www.fda.gov/consumers/health-fraud-scams/fraudulent-coronavirus-disease-2019-covid-19-products}{list}
\emph{of more than 80 fraudulent Covid-19 products, and the W.H.O.}
\href{https://www.who.int/emergencies/diseases/novel-coronavirus-2019/advice-for-public/myth-busters}{debunks}
\emph{many myths about the disease.}

WARNING: DO NOT DO THIS\\
Drinking or injecting bleach and disinfectants\\
In April, President Trump
\href{https://www.nytimes3xbfgragh.onion/2020/04/24/health/sunlight-coronavirus-trump.html}{suggested}
that disinfectants such as alcohol or bleach might be effective against
the coronavirus if directly injected into the body. His comments were
immediately
\href{https://www.nytimes3xbfgragh.onion/2020/04/24/us/politics/trump-inject-disinfectant-bleach-coronavirus.html}{refuted}
by health professionals and researchers around the world --- as well as
the
\href{https://www.prweek.com/article/1681380/lysol-clorox-respond-trump-comment-injecting-disinfectant}{makers
of Lysol and Clorox}. Ingesting disinfectant would not only be
ineffective against the virus, but also hazardous --- possibly even
deadly. In July, Federal prosecutors
\href{https://www.wtsp.com/article/news/regional/florida/miracle-mineral-solution-genesis-ii-church-of-health-and-healing/67-b33b7f2e-2b0c-4853-8434-90732359d730}{charged}
four Florida men with marketing bleach as a cure for COVID-19.

WARNING: NO EVIDENCE\\
UV light\\
President Trump also
\href{https://www.nytimes3xbfgragh.onion/2020/04/24/health/sunlight-coronavirus-trump.html}{speculated}
about hitting the body with ``ultraviolet or just very powerful light.''
Researchers have used UV light to sterilize surfaces, including killing
viruses, in carefully managed laboratories. But UV light would not be
able to purge the virus from within a sick persons' body. This kind of
radiation can also damage the skin. Most skin cancers are a result of
exposure to the UV rays naturally present in sunlight.

WARNING: NO EVIDENCE\\
Silver\\
The F.D.A. has threatened legal action against a host of people claiming
silver-based products are safe and effective against Covid-19 ---
including televangelist
\href{https://www.fda.gov/inspections-compliance-enforcement-and-criminal-investigations/warning-letters/jim-bakker-show-604820-03062020}{Jim
Bakker} and InfoWars host
\href{https://www.fda.gov/inspections-compliance-enforcement-and-criminal-investigations/warning-letters/free-speech-systems-llc-dba-infowarscom-605802-04092020}{Alex
Jones}. Several metals do have
\href{https://www.nytimes3xbfgragh.onion/article/copper-coronavirus-masks.html}{natural
antimicrobial properties}. But products made from them have not been
shown to prevent or treat the coronavirus.

\hypertarget{tracking-the-coronavirus}{%
\subsection{Tracking the Coronavirus}\label{tracking-the-coronavirus}}

\hypertarget{united-states}{%
\subsubsection{United States}\label{united-states}}

\href{https://www.nytimes3xbfgragh.onion/interactive/2020/us/coronavirus-us-cases.html}{}

\includegraphics{https://static01.graylady3jvrrxbe.onion/newsgraphics/2020/03/16/coronavirus-maps/4f1984be305c252f9ee31badc15a159ecb4475a3/images/orphan_usa-threeByTwoSmallAt2X.png}

\hypertarget{latest-maps-and-data}{%
\paragraph{Latest Maps and Data}\label{latest-maps-and-data}}

Cases and deaths for every county

\href{https://www.nytimes3xbfgragh.onion/interactive/2020/05/05/us/coronavirus-death-toll-us.html}{}

\includegraphics{https://static01.graylady3jvrrxbe.onion/newsgraphics/2020/03/16/coronavirus-maps/4f1984be305c252f9ee31badc15a159ecb4475a3/images/footer-thumbs/deaths-us.jpg}

\hypertarget{deaths-above-normal}{%
\paragraph{Deaths Above Normal}\label{deaths-above-normal}}

The true toll of coronavirus in the U.S.

\href{https://www.nytimes3xbfgragh.onion/interactive/2020/04/23/upshot/five-ways-to-monitor-coronavirus-outbreak-us.html}{}

\includegraphics{https://static01.graylady3jvrrxbe.onion/newsgraphics/2020/03/16/coronavirus-maps/4f1984be305c252f9ee31badc15a159ecb4475a3/images/footer-thumbs/metros.png}

\hypertarget{cities-and-metro-areas}{%
\paragraph{Cities and Metro Areas}\label{cities-and-metro-areas}}

Where it is getting better and worse

\href{https://www.nytimes3xbfgragh.onion/interactive/2020/us/coronavirus-testing.html}{}

\includegraphics{https://static01.graylady3jvrrxbe.onion/newsgraphics/2020/03/16/coronavirus-maps/4f1984be305c252f9ee31badc15a159ecb4475a3/images/footer-thumbs/testing.png}

\hypertarget{testing}{%
\paragraph{Testing}\label{testing}}

Is your state doing enough?

\href{https://www.nytimes3xbfgragh.onion/interactive/2020/us/coronavirus-nursing-homes.html}{}

\includegraphics{https://static01.graylady3jvrrxbe.onion/newsgraphics/2020/03/16/coronavirus-maps/4f1984be305c252f9ee31badc15a159ecb4475a3/images/footer-thumbs/nursing-homes.png}

\hypertarget{nursing-homes}{%
\paragraph{Nursing Homes}\label{nursing-homes}}

The hardest-hit states and facilities

\href{https://www.nytimes3xbfgragh.onion/interactive/2020/us/states-reopen-map-coronavirus.html}{}

\includegraphics{https://static01.graylady3jvrrxbe.onion/images/2020/04/24/us/states-reopen-map-coronavirus-promo-1587778728210/states-reopen-map-coronavirus-promo-1587778728210-threeByTwoSmallAt2X-v96.png}

\hypertarget{reopening}{%
\paragraph{Reopening}\label{reopening}}

Which states are open and closed

\hypertarget{world}{%
\subsubsection{World}\label{world}}

\href{https://www.nytimes3xbfgragh.onion/interactive/2020/world/coronavirus-maps.html}{}

\includegraphics{https://static01.graylady3jvrrxbe.onion/newsgraphics/2020/03/16/coronavirus-maps/4f1984be305c252f9ee31badc15a159ecb4475a3/images/orphan_world-threeByTwoSmallAt2X.png}

\hypertarget{latest-maps-and-data-1}{%
\paragraph{Latest Maps and Data}\label{latest-maps-and-data-1}}

Cases and deaths for every country

\href{https://www.nytimes3xbfgragh.onion/interactive/2020/04/21/world/coronavirus-missing-deaths.html}{}

\includegraphics{https://static01.graylady3jvrrxbe.onion/newsgraphics/2020/03/16/coronavirus-maps/4f1984be305c252f9ee31badc15a159ecb4475a3/images/footer-thumbs/deaths-world.jpg}

\hypertarget{deaths-above-normal-1}{%
\paragraph{Deaths Above Normal}\label{deaths-above-normal-1}}

The true toll of coronavirus around the world

\hypertarget{health}{%
\subsubsection{Health}\label{health}}

\href{https://www.nytimes3xbfgragh.onion/interactive/2020/science/coronavirus-vaccine-tracker.html}{}

\includegraphics{https://static01.graylady3jvrrxbe.onion/newsgraphics/2020/03/16/coronavirus-maps/4f1984be305c252f9ee31badc15a159ecb4475a3/images/footer-thumbs/vaccines.png}

\hypertarget{vaccines}{%
\paragraph{Vaccines}\label{vaccines}}

Track their development

\href{https://www.nytimes3xbfgragh.onion/interactive/2020/science/coronavirus-drugs-treatments.html}{}

\includegraphics{https://static01.graylady3jvrrxbe.onion/newsgraphics/2020/03/16/coronavirus-maps/4f1984be305c252f9ee31badc15a159ecb4475a3/images/footer-thumbs/treatments.png}

\hypertarget{treatments}{%
\paragraph{Treatments}\label{treatments}}

Rated by effectiveness and safety

\hypertarget{countries}{%
\subsubsection{Countries}\label{countries}}

\begin{itemize}
\tightlist
\item
  \href{https://www.nytimes3xbfgragh.onion/interactive/2020/world/americas/brazil-coronavirus-cases.html}{Brazil}
\item
  \href{https://www.nytimes3xbfgragh.onion/interactive/2020/world/canada/canada-coronavirus-cases.html}{Canada}
\item
  \href{https://www.nytimes3xbfgragh.onion/interactive/2020/world/europe/france-coronavirus-cases.html}{France}
\item
  \href{https://www.nytimes3xbfgragh.onion/interactive/2020/world/europe/germany-coronavirus-cases.html}{Germany}
\item
  \href{https://www.nytimes3xbfgragh.onion/interactive/2020/world/asia/india-coronavirus-cases.html}{India}
\item
  \href{https://www.nytimes3xbfgragh.onion/interactive/2020/world/europe/italy-coronavirus-cases.html}{Italy}
\item
  \href{https://www.nytimes3xbfgragh.onion/interactive/2020/world/americas/mexico-coronavirus-cases.html}{Mexico}
\item
  \href{https://www.nytimes3xbfgragh.onion/interactive/2020/world/europe/spain-coronavirus-cases.html}{Spain}
\item
  \href{https://www.nytimes3xbfgragh.onion/interactive/2020/world/europe/united-kingdom-coronavirus-cases.html}{U.K.}
\item
  \href{https://www.nytimes3xbfgragh.onion/interactive/2020/us/coronavirus-us-cases.html}{United
  States}
\end{itemize}

\hypertarget{states-territories-and-cities}{%
\subsubsection{States, Territories and
Cities}\label{states-territories-and-cities}}

\begin{itemize}
\tightlist
\item
  \href{https://www.nytimes3xbfgragh.onion/interactive/2020/us/alabama-coronavirus-cases.html}{Alabama}
\item
  \href{https://www.nytimes3xbfgragh.onion/interactive/2020/us/alaska-coronavirus-cases.html}{Alaska}
\item
  \href{https://www.nytimes3xbfgragh.onion/interactive/2020/us/arizona-coronavirus-cases.html}{Arizona}
\item
  \href{https://www.nytimes3xbfgragh.onion/interactive/2020/us/arkansas-coronavirus-cases.html}{Arkansas}
\item
  \href{https://www.nytimes3xbfgragh.onion/interactive/2020/us/california-coronavirus-cases.html}{California}
\item
  \href{https://www.nytimes3xbfgragh.onion/interactive/2020/us/colorado-coronavirus-cases.html}{Colorado}
\item
  \href{https://www.nytimes3xbfgragh.onion/interactive/2020/us/connecticut-coronavirus-cases.html}{Connecticut}
\item
  \href{https://www.nytimes3xbfgragh.onion/interactive/2020/us/delaware-coronavirus-cases.html}{Delaware}
\item
  \href{https://www.nytimes3xbfgragh.onion/interactive/2020/us/florida-coronavirus-cases.html}{Florida}
\item
  \href{https://www.nytimes3xbfgragh.onion/interactive/2020/us/georgia-coronavirus-cases.html}{Georgia}
\item
  \href{https://www.nytimes3xbfgragh.onion/interactive/2020/us/hawaii-coronavirus-cases.html}{Hawaii}
\item
  \href{https://www.nytimes3xbfgragh.onion/interactive/2020/us/idaho-coronavirus-cases.html}{Idaho}
\item
  \href{https://www.nytimes3xbfgragh.onion/interactive/2020/us/illinois-coronavirus-cases.html}{Illinois}
\item
  \href{https://www.nytimes3xbfgragh.onion/interactive/2020/us/indiana-coronavirus-cases.html}{Indiana}
\item
  \href{https://www.nytimes3xbfgragh.onion/interactive/2020/us/iowa-coronavirus-cases.html}{Iowa}
\item
  \href{https://www.nytimes3xbfgragh.onion/interactive/2020/us/kansas-coronavirus-cases.html}{Kansas}
\item
  \href{https://www.nytimes3xbfgragh.onion/interactive/2020/us/kentucky-coronavirus-cases.html}{Kentucky}
\item
  \href{https://www.nytimes3xbfgragh.onion/interactive/2020/us/louisiana-coronavirus-cases.html}{Louisiana}
\item
  \href{https://www.nytimes3xbfgragh.onion/interactive/2020/us/maine-coronavirus-cases.html}{Maine}
\item
  \href{https://www.nytimes3xbfgragh.onion/interactive/2020/us/maryland-coronavirus-cases.html}{Maryland}
\item
  \href{https://www.nytimes3xbfgragh.onion/interactive/2020/us/massachusetts-coronavirus-cases.html}{Massachusetts}
\item
  \href{https://www.nytimes3xbfgragh.onion/interactive/2020/us/michigan-coronavirus-cases.html}{Michigan}
\item
  \href{https://www.nytimes3xbfgragh.onion/interactive/2020/us/minnesota-coronavirus-cases.html}{Minnesota}
\item
  \href{https://www.nytimes3xbfgragh.onion/interactive/2020/us/mississippi-coronavirus-cases.html}{Mississippi}
\item
  \href{https://www.nytimes3xbfgragh.onion/interactive/2020/us/missouri-coronavirus-cases.html}{Missouri}
\item
  \href{https://www.nytimes3xbfgragh.onion/interactive/2020/us/montana-coronavirus-cases.html}{Montana}
\item
  \href{https://www.nytimes3xbfgragh.onion/interactive/2020/us/nebraska-coronavirus-cases.html}{Nebraska}
\item
  \href{https://www.nytimes3xbfgragh.onion/interactive/2020/us/nevada-coronavirus-cases.html}{Nevada}
\item
  \href{https://www.nytimes3xbfgragh.onion/interactive/2020/us/new-hampshire-coronavirus-cases.html}{New
  Hampshire}
\item
  \href{https://www.nytimes3xbfgragh.onion/interactive/2020/us/new-jersey-coronavirus-cases.html}{New
  Jersey}
\item
  \href{https://www.nytimes3xbfgragh.onion/interactive/2020/us/new-mexico-coronavirus-cases.html}{New
  Mexico}
\item
  \href{https://www.nytimes3xbfgragh.onion/interactive/2020/us/new-york-coronavirus-cases.html}{New
  York}
\item
  \href{https://www.nytimes3xbfgragh.onion/interactive/2020/nyregion/new-york-city-coronavirus-cases.html}{New
  York City}
\item
  \href{https://www.nytimes3xbfgragh.onion/interactive/2020/us/north-carolina-coronavirus-cases.html}{North
  Carolina}
\item
  \href{https://www.nytimes3xbfgragh.onion/interactive/2020/us/north-dakota-coronavirus-cases.html}{North
  Dakota}
\item
  \href{https://www.nytimes3xbfgragh.onion/interactive/2020/us/ohio-coronavirus-cases.html}{Ohio}
\item
  \href{https://www.nytimes3xbfgragh.onion/interactive/2020/us/oklahoma-coronavirus-cases.html}{Oklahoma}
\item
  \href{https://www.nytimes3xbfgragh.onion/interactive/2020/us/oregon-coronavirus-cases.html}{Oregon}
\item
  \href{https://www.nytimes3xbfgragh.onion/interactive/2020/us/pennsylvania-coronavirus-cases.html}{Pennsylvania}
\item
  \href{https://www.nytimes3xbfgragh.onion/interactive/2020/us/puerto-rico-coronavirus-cases.html}{Puerto
  Rico}
\item
  \href{https://www.nytimes3xbfgragh.onion/interactive/2020/us/rhode-island-coronavirus-cases.html}{Rhode
  Island}
\item
  \href{https://www.nytimes3xbfgragh.onion/interactive/2020/us/south-carolina-coronavirus-cases.html}{South
  Carolina}
\item
  \href{https://www.nytimes3xbfgragh.onion/interactive/2020/us/south-dakota-coronavirus-cases.html}{South
  Dakota}
\item
  \href{https://www.nytimes3xbfgragh.onion/interactive/2020/us/tennessee-coronavirus-cases.html}{Tennessee}
\item
  \href{https://www.nytimes3xbfgragh.onion/interactive/2020/us/texas-coronavirus-cases.html}{Texas}
\item
  \href{https://www.nytimes3xbfgragh.onion/interactive/2020/us/utah-coronavirus-cases.html}{Utah}
\item
  \href{https://www.nytimes3xbfgragh.onion/interactive/2020/us/vermont-coronavirus-cases.html}{Vermont}
\item
  \href{https://www.nytimes3xbfgragh.onion/interactive/2020/us/virginia-coronavirus-cases.html}{Virginia}
\item
  \href{https://www.nytimes3xbfgragh.onion/interactive/2020/us/washington-coronavirus-cases.html}{Washington}
\item
  \href{https://www.nytimes3xbfgragh.onion/interactive/2020/us/washington-dc-coronavirus-cases.html}{Washington,
  D.C.}
\item
  \href{https://www.nytimes3xbfgragh.onion/interactive/2020/us/west-virginia-coronavirus-cases.html}{West
  Virginia}
\item
  \href{https://www.nytimes3xbfgragh.onion/interactive/2020/us/wisconsin-coronavirus-cases.html}{Wisconsin}
\item
  \href{https://www.nytimes3xbfgragh.onion/interactive/2020/us/wyoming-coronavirus-cases.html}{Wyoming}
\end{itemize}

Note: After additional discussions with experts we have adjusted several
labels on the tracker. The ``Strong evidence'' label has been removed
until further research identifies treatments that consistently benefit
groups of patients infected by the coronavirus. In its place,
``Promising evidence'' will be used for drugs such as remdesivir and
dexamethasone that have shown promise in at least one randomized
controlled trial, and ``Widely used'' for treatments such as proning and
ventilators that are often used with severely ill patients, including
those with Covid-19. And we may reintroduce the ``Ineffective'' label
when ongoing clinical trials repeatedly end with disappointing results.

Sources: National Library of Medicine; National Institutes of Health;
William Amarquaye, University of South Florida; Paul Bieniasz,
Rockefeller University; Jeremy Faust, Brigham \& Women's Hospital; Matt
Frieman, University of Maryland School of Medicine; Noah Haber, Stanford
University; Swapnil Hiremath, University of Ottawa; Akiko Iwasaki, Yale
University; Paul Knoepfler, University of California, Davis; Elena
Massarotti, Brigham and Women's Hospital; John Moore and Douglas Nixon,
Weill Cornell Medical College; Erica Ollman Saphire, La Jolla Institute
for Immunology; Regina Rabinovich, Harvard T.H. Chan School of Public
Health; Ilan Schwartz, University of Alberta; Phyllis Tien, University
of California, San Francisco.

\begin{itemize}
\item
\item
\item
\item
\end{itemize}

Advertisement

\protect\hyperlink{after-bottom}{Continue reading the main story}

\hypertarget{site-index}{%
\subsection{Site Index}\label{site-index}}

\hypertarget{site-information-navigation}{%
\subsection{Site Information
Navigation}\label{site-information-navigation}}

\begin{itemize}
\tightlist
\item
  \href{https://help.nytimes3xbfgragh.onion/hc/en-us/articles/115014792127-Copyright-notice}{©~2020~The
  New York Times Company}
\end{itemize}

\begin{itemize}
\tightlist
\item
  \href{https://www.nytco.com/}{NYTCo}
\item
  \href{https://help.nytimes3xbfgragh.onion/hc/en-us/articles/115015385887-Contact-Us}{Contact
  Us}
\item
  \href{https://www.nytco.com/careers/}{Work with us}
\item
  \href{https://nytmediakit.com/}{Advertise}
\item
  \href{http://www.tbrandstudio.com/}{T Brand Studio}
\item
  \href{https://www.nytimes3xbfgragh.onion/privacy/cookie-policy\#how-do-i-manage-trackers}{Your
  Ad Choices}
\item
  \href{https://www.nytimes3xbfgragh.onion/privacy}{Privacy}
\item
  \href{https://help.nytimes3xbfgragh.onion/hc/en-us/articles/115014893428-Terms-of-service}{Terms
  of Service}
\item
  \href{https://help.nytimes3xbfgragh.onion/hc/en-us/articles/115014893968-Terms-of-sale}{Terms
  of Sale}
\item
  \href{https://spiderbites.nytimes3xbfgragh.onion}{Site Map}
\item
  \href{https://help.nytimes3xbfgragh.onion/hc/en-us}{Help}
\item
  \href{https://www.nytimes3xbfgragh.onion/subscription?campaignId=37WXW}{Subscriptions}
\end{itemize}
