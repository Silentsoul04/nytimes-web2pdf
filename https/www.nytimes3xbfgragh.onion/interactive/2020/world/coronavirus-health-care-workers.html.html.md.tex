Sections

SEARCH

\protect\hyperlink{site-content}{Skip to
content}\protect\hyperlink{site-index}{Skip to site index}

\hypertarget{comments}{%
\subsection{\texorpdfstring{\protect\hyperlink{commentsContainer}{Comments}}{Comments}}\label{comments}}

\href{}{In Harm's Way: Fighting the Summer Surge}\href{}{Skip to
Comments}

The comments section is closed. To submit a letter to the editor for
publication, write to
\href{mailto:letters@NYTimes.com}{\nolinkurl{letters@NYTimes.com}}.

\hypertarget{in-harms-way-fighting-the-summer-surge}{%
\section{In Harm's Way: Fighting the Summer
Surge}\label{in-harms-way-fighting-the-summer-surge}}

By The New York TimesMay 5, 2020

\begin{itemize}
\item
\item
\item
\item
\item
  \emph{+}
\end{itemize}

Meet health care workers in the South and Southwest battling outbreaks
they hoped would never happen.

\hypertarget{in-harms-way}{%
\section{In Harm's Way}\label{in-harms-way}}

Health care workers around the world are risking their lives --- and
those of their families --- to fight the coronavirus pandemic. Since
April, The Times
\href{https://www.nytimes3xbfgragh.onion/article/doctors-treating-coronavirus.html}{has
collected} their reflections.

In Harm's Way

View List

\hypertarget{fighting-the-summer-surge}{%
\subsubsection{Fighting the Summer
Surge}\label{fighting-the-summer-surge}}

In May, many parts of the United States with low coronavirus infection
rates began to reopen. It was a gamble that often resulted in a flood of
cases, especially in the South and parts of the Southwest, where health
care workers are now battling outbreaks they hoped would never reach
them.

\hypertarget{updated-august-1-2020}{%
\subparagraph{Updated August 1, 2020}\label{updated-august-1-2020}}

\protect\hyperlink{item-michael-kennedy}{}

Doctor
\includegraphics{https://static01.graylady3jvrrxbe.onion/packages/flash/multimedia/ICONS/transparent.png}

\begin{itemize}
\tightlist
\item
  Michael Kennedy
\item
  Savannah, Ga.
\item
  July 27
\end{itemize}

\protect\hyperlink{item-brittany-schilling}{}

Nurse
\includegraphics{https://static01.graylady3jvrrxbe.onion/packages/flash/multimedia/ICONS/transparent.png}

\begin{itemize}
\tightlist
\item
  Brittany Schilling
\item
  Phoenix
\item
  July 9
\end{itemize}

\protect\hyperlink{item-pooja-pundhir}{}

Doctor
\includegraphics{https://static01.graylady3jvrrxbe.onion/packages/flash/multimedia/ICONS/transparent.png}

Our hospitals are full. We're running out of respiratory equipment.

\begin{itemize}
\tightlist
\item
  Pooja Pundhir
\item
  Houston
\item
  July 7
\end{itemize}

\protect\hyperlink{item-kimberly-brown}{}

Doctor
\includegraphics{https://static01.graylady3jvrrxbe.onion/packages/flash/multimedia/ICONS/transparent.png}

My days off have been filled with paralyzing anxiety.

\begin{itemize}
\tightlist
\item
  Kimberly Brown
\item
  Memphis
\item
  July 13
\end{itemize}

\protect\hyperlink{item-nicole-battaglioli}{}

Doctor
\includegraphics{https://static01.graylady3jvrrxbe.onion/packages/flash/multimedia/ICONS/transparent.png}

\begin{itemize}
\tightlist
\item
  Nicole Battaglioli
\item
  Atlanta
\item
  July 7
\end{itemize}

\protect\hyperlink{item-aliiah-v-jourdain}{}

Doctor
\includegraphics{https://static01.graylady3jvrrxbe.onion/packages/flash/multimedia/ICONS/transparent.png}

\begin{itemize}
\tightlist
\item
  Aliiah V. Jourdain
\item
  Moncure, N.C.
\item
  July 10
\end{itemize}

\protect\hyperlink{item-john-james-jr}{}

Nurse
\includegraphics{https://static01.graylady3jvrrxbe.onion/packages/flash/multimedia/ICONS/transparent.png}

It seems a loved one has to contract the virus for people to really
understand.

\begin{itemize}
\tightlist
\item
  John James Jr.
\item
  Memphis
\item
  July 13
\end{itemize}

\protect\hyperlink{item-kimberly-darpoh}{}

Nurse
\includegraphics{https://static01.graylady3jvrrxbe.onion/packages/flash/multimedia/ICONS/transparent.png}

\begin{itemize}
\tightlist
\item
  Kimberly Darpoh
\item
  Atlanta
\item
  July 13
\end{itemize}

\protect\hyperlink{item-melhim-bou-alwan}{}

Doctor
\includegraphics{https://static01.graylady3jvrrxbe.onion/packages/flash/multimedia/ICONS/transparent.png}

On the 11th day, I cried.

\begin{itemize}
\tightlist
\item
  Melhim Bou Alwan
\item
  Atlanta
\item
  June 21
\end{itemize}

\protect\hyperlink{item-catrina-rugar}{}

Nurse
\includegraphics{https://static01.graylady3jvrrxbe.onion/packages/flash/multimedia/ICONS/transparent.png}

How can I have given myself to New York and Texas and not do it here?

\begin{itemize}
\tightlist
\item
  Catrina Rugar
\item
  Crystal River, Fla.
\item
  July 31
\end{itemize}

\protect\hyperlink{item-jay-quinn}{}

Nurse
\includegraphics{https://static01.graylady3jvrrxbe.onion/packages/flash/multimedia/ICONS/transparent.png}

\begin{itemize}
\tightlist
\item
  Jay Quinn
\item
  Clover, S.C.
\item
  July 13
\end{itemize}

\protect\hyperlink{item-damien-shields}{}

Doctor
\includegraphics{https://static01.graylady3jvrrxbe.onion/packages/flash/multimedia/ICONS/transparent.png}

\begin{itemize}
\tightlist
\item
  Damien Shields
\item
  Houston
\item
  July 7
\end{itemize}

\protect\hyperlink{item-felino-taruc}{}

Nurse
\includegraphics{https://static01.graylady3jvrrxbe.onion/packages/flash/multimedia/ICONS/transparent.png}

\begin{itemize}
\tightlist
\item
  Felino Taruc
\item
  El Paso
\item
  July 13
\end{itemize}

\protect\hyperlink{item-lillie-lodge}{}

Nurse
\includegraphics{https://static01.graylady3jvrrxbe.onion/packages/flash/multimedia/ICONS/transparent.png}

How long can we keep this up? When will we reach the breaking point?

\begin{itemize}
\tightlist
\item
  Lillie Lodge
\item
  Raleigh, N.C.
\item
  July 7
\end{itemize}

\protect\hyperlink{item-crystal-monjardin}{}

Nurse
\includegraphics{https://static01.graylady3jvrrxbe.onion/packages/flash/multimedia/ICONS/transparent.png}

\begin{itemize}
\tightlist
\item
  Crystal Monjardin
\item
  Yuma, Ariz.
\item
  July 14
\end{itemize}

\protect\hyperlink{item-lydia-lopez}{}

Nurse
\includegraphics{https://static01.graylady3jvrrxbe.onion/packages/flash/multimedia/ICONS/transparent.png}

\begin{itemize}
\tightlist
\item
  Lydia Lopez
\item
  Phoenix
\item
  July 16
\end{itemize}

\protect\hyperlink{item-ana-laura-gonzalez-de-la-paz}{}

Nurse
\includegraphics{https://static01.graylady3jvrrxbe.onion/packages/flash/multimedia/ICONS/transparent.png}

Aside from being a nurse, I am also a DACA recipient.

\begin{itemize}
\tightlist
\item
  Ana Laura Gonzalez De La Paz
\item
  Austin, Texas
\item
  July 14
\end{itemize}

\protect\hyperlink{item-tim-ellsberry}{}

Nurse
\includegraphics{https://static01.graylady3jvrrxbe.onion/packages/flash/multimedia/ICONS/transparent.png}

\begin{itemize}
\tightlist
\item
  Tim Ellsberry
\item
  Atlanta
\item
  July 16
\end{itemize}

\protect\hyperlink{item-anna-maria-ruiz}{}

Nurse
\includegraphics{https://static01.graylady3jvrrxbe.onion/packages/flash/multimedia/ICONS/transparent.png}

I had it. I've seen it. I look it dead in the face every day.

\begin{itemize}
\tightlist
\item
  Anna Maria Ruiz
\item
  Austin, Texas
\item
  July 10
\end{itemize}

\protect\hyperlink{item-tamar-sternfeld}{}

Nurse
\includegraphics{https://static01.graylady3jvrrxbe.onion/packages/flash/multimedia/ICONS/transparent.png}

\begin{itemize}
\tightlist
\item
  Tamar Sternfeld
\item
  Charleston, S.C.
\item
  July 15
\end{itemize}

\protect\hyperlink{item-zola-nlandu}{}

Doctor
\includegraphics{https://static01.graylady3jvrrxbe.onion/packages/flash/multimedia/ICONS/transparent.png}

Every day, my wife says a word of prayer before I head off to work.

\begin{itemize}
\tightlist
\item
  Zola Nlandu
\item
  Tampa, Fla.
\item
  July 10
\end{itemize}

\protect\hyperlink{item-jesse-hart}{}

Doctor
\includegraphics{https://static01.graylady3jvrrxbe.onion/packages/flash/multimedia/ICONS/transparent.png}

\begin{itemize}
\tightlist
\item
  Jesse Hart
\item
  Chapel Hill, N.C.
\item
  July 15
\end{itemize}

\protect\hyperlink{item-erin-malone}{}

Nurse
\includegraphics{https://static01.graylady3jvrrxbe.onion/packages/flash/multimedia/ICONS/transparent.png}

\begin{itemize}
\tightlist
\item
  Erin Malone
\item
  Miami
\item
  July 22
\end{itemize}

\protect\hyperlink{item-courtney-ehioghae}{}

Nurse
\includegraphics{https://static01.graylady3jvrrxbe.onion/packages/flash/multimedia/ICONS/transparent.png}

I've had a few arguments with people who believe in conspiracy theories.

\begin{itemize}
\tightlist
\item
  Courtney Ehioghae
\item
  Dallas
\item
  July 15
\end{itemize}

\protect\hyperlink{item-darshan-gandhi}{}

Doctor
\includegraphics{https://static01.graylady3jvrrxbe.onion/packages/flash/multimedia/ICONS/transparent.png}

\begin{itemize}
\tightlist
\item
  Darshan Gandhi
\item
  Dallas
\item
  July 7
\end{itemize}

\protect\hyperlink{item-jennifer-wurster}{}

Nurse Practitioner
\includegraphics{https://static01.graylady3jvrrxbe.onion/packages/flash/multimedia/ICONS/transparent.png}

\begin{itemize}
\tightlist
\item
  Jennifer Wurster
\item
  Tucson, Ariz.
\item
  July 9
\end{itemize}

\protect\hyperlink{item-anna-fong}{}

Doctor
\includegraphics{https://static01.graylady3jvrrxbe.onion/packages/flash/multimedia/ICONS/transparent.png}

I lied the good lie. Without hope, what is the purpose of life?

\begin{itemize}
\tightlist
\item
  Anna Fong
\item
  Memphis
\item
  July 5
\end{itemize}

\protect\hyperlink{item-tamika-bush}{}

Doctor
\includegraphics{https://static01.graylady3jvrrxbe.onion/packages/flash/multimedia/ICONS/transparent.png}

I am emotionally exhausted and physically drained.

\begin{itemize}
\tightlist
\item
  Tamika Bush
\item
  Houston
\item
  July 13
\end{itemize}

\protect\hyperlink{item-jennifer-gillen}{}

Nurse
\includegraphics{https://static01.graylady3jvrrxbe.onion/packages/flash/multimedia/ICONS/transparent.png}

\begin{itemize}
\tightlist
\item
  Jennifer Gillen
\item
  Richmond, Texas
\item
  July 11
\end{itemize}

\protect\hyperlink{item-andriana-love}{}

Doctor
\includegraphics{https://static01.graylady3jvrrxbe.onion/packages/flash/multimedia/ICONS/transparent.png}

\begin{itemize}
\tightlist
\item
  Andriana Love
\item
  Murrells Inlet, S.C.
\item
  July 21
\end{itemize}

\protect\hyperlink{item-absar-mirza}{}

Doctor
\includegraphics{https://static01.graylady3jvrrxbe.onion/packages/flash/multimedia/ICONS/transparent.png}

The day we released him, I was ecstatic.

\begin{itemize}
\tightlist
\item
  Absar Mirza
\item
  Alpharetta, Ga.
\item
  June 22
\end{itemize}

\protect\hyperlink{item-sarah-berger}{}

Nurse
\includegraphics{https://static01.graylady3jvrrxbe.onion/packages/flash/multimedia/ICONS/transparent.png}

\begin{itemize}
\tightlist
\item
  Sarah Berger
\item
  McAllen, Texas
\item
  July 27
\end{itemize}

\protect\hyperlink{item-jairo-hernando-barrantes-perez}{}

Doctor
\includegraphics{https://static01.graylady3jvrrxbe.onion/packages/flash/multimedia/ICONS/transparent.png}

\begin{itemize}
\tightlist
\item
  Jairo Hernando Barrantes Perez
\item
  Houston
\item
  July 21
\end{itemize}

\protect\hyperlink{item-camila-poblete}{}

Nurse
\includegraphics{https://static01.graylady3jvrrxbe.onion/packages/flash/multimedia/ICONS/transparent.png}

Things are horribly worse.

\begin{itemize}
\tightlist
\item
  Camila Poblete
\item
  Miami
\item
  July 22
\end{itemize}

\protect\hyperlink{item-erik-adler}{}

Doctor
\includegraphics{https://static01.graylady3jvrrxbe.onion/packages/flash/multimedia/ICONS/transparent.png}

\begin{itemize}
\tightlist
\item
  Erik Adler
\item
  Golden, Colo.
\item
  July 29
\end{itemize}

\protect\hyperlink{item-sophina-calderon}{}

Doctor
\includegraphics{https://static01.graylady3jvrrxbe.onion/packages/flash/multimedia/ICONS/transparent.png}

\begin{itemize}
\tightlist
\item
  Sophina Calderon
\item
  Tuba City, Ariz.
\item
  July 30
\end{itemize}

\protect\hyperlink{item-caitlin-ortiz}{}

Nurse
\includegraphics{https://static01.graylady3jvrrxbe.onion/packages/flash/multimedia/ICONS/transparent.png}

People have thrown their masks on the floor at my feet. They refuse to
wear the mask correctly.

\begin{itemize}
\tightlist
\item
  Caitlin Ortiz
\item
  Austin, Texas
\item
  May 6
\end{itemize}

\protect\hyperlink{item-mary-catherine-keckeisen}{}

Nurse
\includegraphics{https://static01.graylady3jvrrxbe.onion/packages/flash/multimedia/ICONS/transparent.png}

This statistic is a human who at some point had a first kiss, learned to
drive, loved a child.

\begin{itemize}
\tightlist
\item
  Mary Catherine Keckeisen
\item
  Dallas
\item
  July 8
\end{itemize}

\hypertarget{around-the-world}{%
\subsubsection{Around the World}\label{around-the-world}}

Nurses, doctors and other health care workers reflect on fighting the
coronavirus.

\hypertarget{updated-august-1-2020-1}{%
\subparagraph{Updated August 1, 2020}\label{updated-august-1-2020-1}}

\protect\hyperlink{item-crescenzo-sala}{}

Doctor
\includegraphics{https://static01.graylady3jvrrxbe.onion/packages/flash/multimedia/ICONS/transparent.png}

People here have started living as if nothing has happened, but in my
hospital we are ready.

\begin{itemize}
\tightlist
\item
  Crescenzo Sala
\item
  Naples, Italy
\item
  July 30
\end{itemize}

\protect\hyperlink{item-hina-ghory}{}

Doctor
\includegraphics{https://static01.graylady3jvrrxbe.onion/packages/flash/multimedia/ICONS/transparent.png}

\begin{itemize}
\tightlist
\item
  Hina Ghory
\item
  Princeton Junction, N.J.
\item
  May 1
\end{itemize}

\protect\hyperlink{item-estell-williams}{}

Doctor
\includegraphics{https://static01.graylady3jvrrxbe.onion/packages/flash/multimedia/ICONS/transparent.png}

George Floyd survived illness with Covid-19 just to die in the hands of
police.

\begin{itemize}
\tightlist
\item
  Estell Williams
\item
  Seattle
\item
  June 14
\end{itemize}

\protect\hyperlink{item-pablo-trujillo}{}

Nurse
\includegraphics{https://static01.graylady3jvrrxbe.onion/packages/flash/multimedia/ICONS/transparent.png}

I was fully aware of what the virus was doing to my body and how likely
it was that I was going to die.

\begin{itemize}
\tightlist
\item
  Pablo Trujillo
\item
  Mexico City
\item
  June 9
\end{itemize}

\protect\hyperlink{item-everett-moss-ii}{}

Nurse
\includegraphics{https://static01.graylady3jvrrxbe.onion/packages/flash/multimedia/ICONS/transparent.png}

\begin{itemize}
\tightlist
\item
  Everett Moss II
\item
  Atlanta
\item
  May 5
\end{itemize}

\protect\hyperlink{item-louai-razzouk}{}

Doctor
\includegraphics{https://static01.graylady3jvrrxbe.onion/packages/flash/multimedia/ICONS/transparent.png}

\begin{itemize}
\tightlist
\item
  Louai Razzouk
\item
  Manhattan, N.Y.
\item
  July 8
\end{itemize}

\protect\hyperlink{item-vanda-ortega-witoto}{}

Nursing Technician
\includegraphics{https://static01.graylady3jvrrxbe.onion/packages/flash/multimedia/ICONS/transparent.png}

In our Indigenous communities, health conditions are precarious. We had
no money to buy medicines.

\begin{itemize}
\tightlist
\item
  Vanda Ortega Witoto
\item
  Parque das Tribos, Manaus, Brazil
\item
  June 29
\end{itemize}

\protect\hyperlink{item-annalisa-malara}{}

Doctor
\includegraphics{https://static01.graylady3jvrrxbe.onion/packages/flash/multimedia/ICONS/transparent.png}

\begin{itemize}
\tightlist
\item
  Annalisa Malara
\item
  Codogno, Italy
\item
  May 5
\end{itemize}

\protect\hyperlink{item-kim-hyeon-hee}{}

Nurse
\includegraphics{https://static01.graylady3jvrrxbe.onion/packages/flash/multimedia/ICONS/transparent.png}

\begin{itemize}
\tightlist
\item
  Kim Hyeon-hee
\item
  Daegu, South Korea
\item
  April 28
\end{itemize}

\protect\hyperlink{item-arezoo-habibi}{}

Doctor
\includegraphics{https://static01.graylady3jvrrxbe.onion/packages/flash/multimedia/ICONS/transparent.png}

\begin{itemize}
\tightlist
\item
  Arezoo Habibi
\item
  Herat City, Afghanistan
\item
  April 25
\end{itemize}

\protect\hyperlink{item-thomas-lo}{}

Doctor
\includegraphics{https://static01.graylady3jvrrxbe.onion/packages/flash/multimedia/ICONS/transparent.png}

I'm right by the patient's airway, often inches away, as I place the
breathing tube.

\begin{itemize}
\tightlist
\item
  Thomas Lo
\item
  Queens, N.Y.
\item
  April 24
\end{itemize}

\protect\hyperlink{item-taylor-olson}{}

Nurse
\includegraphics{https://static01.graylady3jvrrxbe.onion/packages/flash/multimedia/ICONS/transparent.png}

So many people have dropped everything and come to our state to help
out. I just felt like I could return the favor.

\begin{itemize}
\tightlist
\item
  Taylor Olson
\item
  Naples, Fla.
\item
  April 30
\end{itemize}

\protect\hyperlink{item-brooke-spence}{}

Nurse
\includegraphics{https://static01.graylady3jvrrxbe.onion/packages/flash/multimedia/ICONS/transparent.png}

\begin{itemize}
\tightlist
\item
  Brooke Spence
\item
  Las Vegas
\item
  April 10
\end{itemize}

\protect\hyperlink{item-jesus-valverde}{}

Doctor
\includegraphics{https://static01.graylady3jvrrxbe.onion/packages/flash/multimedia/ICONS/transparent.png}

We're exhausted, but we draw strength to go on from wherever we can.

\begin{itemize}
\tightlist
\item
  Jesús Valverde
\item
  Lima, Peru
\item
  June 24
\end{itemize}

\protect\hyperlink{item-kirk-tapia-bobst}{}

Nurse
\includegraphics{https://static01.graylady3jvrrxbe.onion/packages/flash/multimedia/ICONS/transparent.png}

\begin{itemize}
\tightlist
\item
  Kirk Tapia-Bobst
\item
  Chicago
\item
  July 10
\end{itemize}

\protect\hyperlink{item-micaela-sager}{}

Physician Assistant
\includegraphics{https://static01.graylady3jvrrxbe.onion/packages/flash/multimedia/ICONS/transparent.png}

\begin{itemize}
\tightlist
\item
  Micaela Sager
\item
  Point Pleasant, N.J.
\item
  June 6
\end{itemize}

\protect\hyperlink{item-kristina-woo}{}

Clinical Care Technician and E.M.T.
\includegraphics{https://static01.graylady3jvrrxbe.onion/packages/flash/multimedia/ICONS/transparent.png}

In times of chaos someone needs to remain calm and show up.

\begin{itemize}
\tightlist
\item
  Kristina Woo
\item
  Watchung, N.J.
\item
  April 9
\end{itemize}

\protect\hyperlink{item-maria-notaro}{}

Doctor
\includegraphics{https://static01.graylady3jvrrxbe.onion/packages/flash/multimedia/ICONS/transparent.png}

The pandemic has been like war.

\begin{itemize}
\tightlist
\item
  Maria Notaro
\item
  Naples, Italy
\item
  June 15
\end{itemize}

\protect\hyperlink{item-kp-mendoza}{}

Nurse
\includegraphics{https://static01.graylady3jvrrxbe.onion/packages/flash/multimedia/ICONS/transparent.png}

\begin{itemize}
\tightlist
\item
  KP Mendoza
\item
  Manhattan, N.Y.
\item
  June 9
\end{itemize}

\protect\hyperlink{item-ryan-lee}{}

Doctor
\includegraphics{https://static01.graylady3jvrrxbe.onion/packages/flash/multimedia/ICONS/transparent.png}

\begin{itemize}
\tightlist
\item
  Ryan Lee
\item
  Canton, Mass.
\item
  April 20
\end{itemize}

\protect\hyperlink{item-federica-brena}{}

Doctor
\includegraphics{https://static01.graylady3jvrrxbe.onion/packages/flash/multimedia/ICONS/transparent.png}

\begin{itemize}
\tightlist
\item
  Federica Brena
\item
  Bergamo, Italy
\item
  May 5
\end{itemize}

\protect\hyperlink{item-josline-azki}{}

Nurse
\includegraphics{https://static01.graylady3jvrrxbe.onion/packages/flash/multimedia/ICONS/transparent.png}

\begin{itemize}
\tightlist
\item
  Josline Azki
\item
  Baalbek, Lebanon
\item
  June 3
\end{itemize}

\protect\hyperlink{item-charline-kass}{}

Doctor
\includegraphics{https://static01.graylady3jvrrxbe.onion/packages/flash/multimedia/ICONS/transparent.png}

When I'm on my own at home, the sadness starts to creep over me. It can
be overwhelming.

\begin{itemize}
\tightlist
\item
  Charline Kass
\item
  Antofagasta, Chile
\item
  June 9
\end{itemize}

\protect\hyperlink{item-adam-j-milam}{}

Doctor
\includegraphics{https://static01.graylady3jvrrxbe.onion/packages/flash/multimedia/ICONS/transparent.png}

We cannot sit idly by while Black people die at alarming rates.

\begin{itemize}
\tightlist
\item
  Adam J. Milam
\item
  Los Angeles
\item
  June 11
\end{itemize}

\protect\hyperlink{item-kious-kelly}{}

Nurse
\includegraphics{https://static01.graylady3jvrrxbe.onion/packages/flash/multimedia/ICONS/transparent.png}

\begin{itemize}
\tightlist
\item
  Kious Kelly
\item
  New York City
\item
  Died on March 24
\end{itemize}

\protect\hyperlink{item-valeria-alfano}{}

Nurse
\includegraphics{https://static01.graylady3jvrrxbe.onion/packages/flash/multimedia/ICONS/transparent.png}

\begin{itemize}
\tightlist
\item
  Valeria Alfano
\item
  Naples, Italy
\item
  July 8
\end{itemize}

\protect\hyperlink{item-michaela-casey}{}

Nurse
\includegraphics{https://static01.graylady3jvrrxbe.onion/packages/flash/multimedia/ICONS/transparent.png}

I held his hand, smoothed his hair and told him his family loves him and
wishes they could be with him so much.

\begin{itemize}
\tightlist
\item
  Michaela Casey
\item
  Baltimore
\item
  April 13
\end{itemize}

\protect\hyperlink{item-ousseni-w-tiemtore}{}

Doctor
\includegraphics{https://static01.graylady3jvrrxbe.onion/packages/flash/multimedia/ICONS/transparent.png}

\begin{itemize}
\tightlist
\item
  Ousseni W. Tiemtore
\item
  Ouagadougou, Burkina Faso
\item
  June 3
\end{itemize}

\protect\hyperlink{item-alexandra-moran}{}

Doctor
\includegraphics{https://static01.graylady3jvrrxbe.onion/packages/flash/multimedia/ICONS/transparent.png}

\begin{itemize}
\tightlist
\item
  Alexandra Moran
\item
  Miami
\item
  July 14
\end{itemize}

\protect\hyperlink{item-oscar-caicho-caicedo}{}

Nurse
\includegraphics{https://static01.graylady3jvrrxbe.onion/packages/flash/multimedia/ICONS/transparent.png}

\begin{itemize}
\tightlist
\item
  Óscar Caicho Caicedo
\item
  Guayaquil, Ecuador
\item
  June 24
\end{itemize}

\protect\hyperlink{item-chanel-fischetti}{}

Doctor
\includegraphics{https://static01.graylady3jvrrxbe.onion/packages/flash/multimedia/ICONS/transparent.png}

It was hard not to come home and fall asleep with a little anxiety,
thinking, ``Today could've been the day I got sick.''

\begin{itemize}
\tightlist
\item
  Chanel Fischetti
\item
  Boston
\item
  June 9
\end{itemize}

\protect\hyperlink{item-edward-rippe}{}

Doctor
\includegraphics{https://static01.graylady3jvrrxbe.onion/packages/flash/multimedia/ICONS/transparent.png}

\begin{itemize}
\tightlist
\item
  Edward Rippe
\item
  Manhattan, N.Y.
\item
  April 18
\end{itemize}

\protect\hyperlink{item-nathan-a-villada}{}

Paramedic
\includegraphics{https://static01.graylady3jvrrxbe.onion/packages/flash/multimedia/ICONS/transparent.png}

The killing of Breonna Taylor was a pivotal moment for me. She was an
E.M.T.

\begin{itemize}
\tightlist
\item
  Nathan A. Villada
\item
  New York City
\item
  June 15
\end{itemize}

\protect\hyperlink{item-mustafa-alam}{}

Doctor
\includegraphics{https://static01.graylady3jvrrxbe.onion/packages/flash/multimedia/ICONS/transparent.png}

\begin{itemize}
\tightlist
\item
  Mustafa Alam
\item
  Brooklyn, N.Y.
\item
  April 8
\end{itemize}

\protect\hyperlink{item-martynas-gedminas}{}

Doctor
\includegraphics{https://static01.graylady3jvrrxbe.onion/packages/flash/multimedia/ICONS/transparent.png}

\begin{itemize}
\tightlist
\item
  Martynas Gedminas
\item
  Šiauliai, Lithuania
\item
  July 8
\end{itemize}

\protect\hyperlink{item-agnese-marcelli}{}

Doctor
\includegraphics{https://static01.graylady3jvrrxbe.onion/packages/flash/multimedia/ICONS/transparent.png}

There still remains the fear of an invisible enemy that could come back
and take away our serenity.

\begin{itemize}
\tightlist
\item
  Agnese Marcelli
\item
  Rome
\item
  July 31
\end{itemize}

\protect\hyperlink{item-whitney-douglass}{}

Nurse
\includegraphics{https://static01.graylady3jvrrxbe.onion/packages/flash/multimedia/ICONS/transparent.png}

\begin{itemize}
\tightlist
\item
  Whitney Douglass
\item
  Denver
\item
  April 28
\end{itemize}

\protect\hyperlink{item-daetney-ewing}{}

Certified Nursing Assistant
\includegraphics{https://static01.graylady3jvrrxbe.onion/packages/flash/multimedia/ICONS/transparent.png}

We are all exhausted. But that isn't keeping us from doing our due
diligence.

\begin{itemize}
\tightlist
\item
  Daetney Ewing
\item
  Minneapolis
\item
  June 11
\end{itemize}

\protect\hyperlink{item-bryan-adrian-priego-parra}{}

Doctor
\includegraphics{https://static01.graylady3jvrrxbe.onion/packages/flash/multimedia/ICONS/transparent.png}

\begin{itemize}
\tightlist
\item
  Bryan Adrian Priego Parra
\item
  Boca del Río, Mexico
\item
  June 25
\end{itemize}

\protect\hyperlink{item-ninoska-flores-chavez}{}

Nurse
\includegraphics{https://static01.graylady3jvrrxbe.onion/packages/flash/multimedia/ICONS/transparent.png}

\begin{itemize}
\tightlist
\item
  Ninoska Flores Chávez 
\item
  Guayaquil, Ecuador
\item
  June 24
\end{itemize}

\protect\hyperlink{item-colleen-hill}{}

Physician Assistant
\includegraphics{https://static01.graylady3jvrrxbe.onion/packages/flash/multimedia/ICONS/transparent.png}

\begin{itemize}
\tightlist
\item
  Colleen Hill
\item
  Atlanta
\item
  April 16
\end{itemize}

\protect\hyperlink{item-daniel-akinyemi}{}

Nurse
\includegraphics{https://static01.graylady3jvrrxbe.onion/packages/flash/multimedia/ICONS/transparent.png}

We said a little prayer together over the phone and I asked, ``Does she
have a favorite song?''

\begin{itemize}
\tightlist
\item
  Daniel Akinyemi
\item
  Montclair, N.J.
\item
  April 23
\end{itemize}

\protect\hyperlink{item-maite-silva-martins-gadelha}{}

Doctor
\includegraphics{https://static01.graylady3jvrrxbe.onion/packages/flash/multimedia/ICONS/transparent.png}

In the first cities we reached, we met a population totally without
assistance and without information about the pandemic.

\begin{itemize}
\tightlist
\item
  Maitê Silva Martins Gadelha
\item
  Belém do Pará, Brazil
\item
  June 29
\end{itemize}

\protect\hyperlink{item-joyce-lamb}{}

Nurse
\includegraphics{https://static01.graylady3jvrrxbe.onion/packages/flash/multimedia/ICONS/transparent.png}

\begin{itemize}
\tightlist
\item
  Joyce Lamb
\item
  Queens, N.Y.
\item
  May 3
\end{itemize}

\protect\hyperlink{item-sanjai-sinha}{}

Doctor
\includegraphics{https://static01.graylady3jvrrxbe.onion/packages/flash/multimedia/ICONS/transparent.png}

\begin{itemize}
\tightlist
\item
  Sanjai Sinha
\item
  Pelham, N.Y.
\item
  April 25
\end{itemize}

\protect\hyperlink{item-kenji-fujiwara}{}

Doctor
\includegraphics{https://static01.graylady3jvrrxbe.onion/packages/flash/multimedia/ICONS/transparent.png}

\begin{itemize}
\tightlist
\item
  Kenji Fujiwara
\item
  Fukuoka, Japan
\item
  April 28
\end{itemize}

\protect\hyperlink{item-romolo-villani}{}

Doctor
\includegraphics{https://static01.graylady3jvrrxbe.onion/packages/flash/multimedia/ICONS/transparent.png}

We cannot yet let our guard down in the fight against the coronavirus.

\begin{itemize}
\tightlist
\item
  Romolo Villani
\item
  Salerno, Italy
\item
  May 3
\end{itemize}

\protect\hyperlink{item-shane-smith}{}

Nurse
\includegraphics{https://static01.graylady3jvrrxbe.onion/packages/flash/multimedia/ICONS/transparent.png}

\begin{itemize}
\tightlist
\item
  Shane Smith
\item
  Queens Village, N.Y.
\item
  June 26
\end{itemize}

\protect\hyperlink{item-josefina-opazo}{}

Doctor
\includegraphics{https://static01.graylady3jvrrxbe.onion/packages/flash/multimedia/ICONS/transparent.png}

\begin{itemize}
\tightlist
\item
  Josefina Opazo
\item
  Santiago, Chile
\item
  June 9
\end{itemize}

\protect\hyperlink{item-carmen-presti}{}

Nurse Practitioner
\includegraphics{https://static01.graylady3jvrrxbe.onion/packages/flash/multimedia/ICONS/transparent.png}

\begin{itemize}
\tightlist
\item
  Carmen Presti
\item
  Hollywood, Fla.
\item
  April 9
\end{itemize}

\protect\hyperlink{item-phillip-scotti}{}

Physician Assistant
\includegraphics{https://static01.graylady3jvrrxbe.onion/packages/flash/multimedia/ICONS/transparent.png}

\begin{itemize}
\tightlist
\item
  Phillip Scotti
\item
  Brooklyn, N.Y.
\item
  April 9
\end{itemize}

\protect\hyperlink{item-marion-levigne}{}

Nurse
\includegraphics{https://static01.graylady3jvrrxbe.onion/packages/flash/multimedia/ICONS/transparent.png}

We may have lost our bearings, but not the meaning of the job we chose
to do.

\begin{itemize}
\tightlist
\item
  Marion Levigne
\item
  Lyon, France
\item
  April 19
\end{itemize}

\protect\hyperlink{item-linda-wang}{}

Doctor
\includegraphics{https://static01.graylady3jvrrxbe.onion/packages/flash/multimedia/ICONS/transparent.png}

It has been emotionally draining to navigate this crisis.

\begin{itemize}
\tightlist
\item
  Linda Wang
\item
  Manhattan, N.Y.
\item
  April 20
\end{itemize}

\protect\hyperlink{item-gila-zarbiv}{}

Midwife
\includegraphics{https://static01.graylady3jvrrxbe.onion/packages/flash/multimedia/ICONS/transparent.png}

\begin{itemize}
\tightlist
\item
  Gila Zarbiv
\item
  Jerusalem
\item
  July 7
\end{itemize}

\protect\hyperlink{item-dolapo-olugbile}{}

Technician
\includegraphics{https://static01.graylady3jvrrxbe.onion/packages/flash/multimedia/ICONS/transparent.png}

\begin{itemize}
\tightlist
\item
  Dolapo Olugbile
\item
  Laurel, Md.
\item
  June 11
\end{itemize}

\protect\hyperlink{item-stephanie-benjamin}{}

Doctor
\includegraphics{https://static01.graylady3jvrrxbe.onion/packages/flash/multimedia/ICONS/transparent.png}

\begin{itemize}
\tightlist
\item
  Stephanie Benjamin
\item
  San Diego
\item
  April 9
\end{itemize}

\protect\hyperlink{item-marie-campbell}{}

Nurse
\includegraphics{https://static01.graylady3jvrrxbe.onion/packages/flash/multimedia/ICONS/transparent.png}

\begin{itemize}
\tightlist
\item
  Marie Campbell
\item
  Philadelphia
\item
  June 10
\end{itemize}

\protect\hyperlink{item-ariel-philip-flores}{}

Chaplain
\includegraphics{https://static01.graylady3jvrrxbe.onion/packages/flash/multimedia/ICONS/transparent.png}

\begin{itemize}
\tightlist
\item
  Ariel-Philip Flores
\item
  Los Angeles
\item
  July 4
\end{itemize}

\protect\hyperlink{item-anggy-barrera}{}

Doctor
\includegraphics{https://static01.graylady3jvrrxbe.onion/packages/flash/multimedia/ICONS/transparent.png}

\begin{itemize}
\tightlist
\item
  Anggy Barrera
\item
  Caracas, Venezuela
\item
  June 30
\end{itemize}

\protect\hyperlink{item-sneh-sonaiya}{}

Doctor
\includegraphics{https://static01.graylady3jvrrxbe.onion/packages/flash/multimedia/ICONS/transparent.png}

\begin{itemize}
\tightlist
\item
  Sneh Sonaiya
\item
  Gujarat, India
\item
  June 9
\end{itemize}

\protect\hyperlink{item-katty-renard}{}

Nurse
\includegraphics{https://static01.graylady3jvrrxbe.onion/packages/flash/multimedia/ICONS/transparent.png}

\begin{itemize}
\tightlist
\item
  Katty Renard
\item
  Brussels
\item
  July 11
\end{itemize}

\protect\hyperlink{item-manya-gupta}{}

Doctor
\includegraphics{https://static01.graylady3jvrrxbe.onion/packages/flash/multimedia/ICONS/transparent.png}

\begin{itemize}
\tightlist
\item
  Manya Gupta
\item
  Chicago
\item
  May 5
\end{itemize}

\protect\hyperlink{item-tiago-valim}{}

Doctor
\includegraphics{https://static01.graylady3jvrrxbe.onion/packages/flash/multimedia/ICONS/transparent.png}

\begin{itemize}
\tightlist
\item
  Tiago Valim
\item
  São Paulo, Brazil
\item
  June 3
\end{itemize}

\protect\hyperlink{item-amanda-leone}{}

Physical Therapist
\includegraphics{https://static01.graylady3jvrrxbe.onion/packages/flash/multimedia/ICONS/transparent.png}

\begin{itemize}
\tightlist
\item
  Amanda Leone
\item
  Waltham, Mass.
\item
  June 9
\end{itemize}

\protect\hyperlink{item-lawrence-asprec}{}

Doctor
\includegraphics{https://static01.graylady3jvrrxbe.onion/packages/flash/multimedia/ICONS/transparent.png}

\begin{itemize}
\tightlist
\item
  Lawrence Asprec
\item
  Manhattan, N.Y.
\item
  April 11
\end{itemize}

\protect\hyperlink{item-sarita-nori}{}

Doctor
\includegraphics{https://static01.graylady3jvrrxbe.onion/packages/flash/multimedia/ICONS/transparent.png}

\begin{itemize}
\tightlist
\item
  Sarita Nori
\item
  Somerville, Mass.
\item
  April 21
\end{itemize}

\protect\hyperlink{item-alejandra-ramirez}{}

Doctor
\includegraphics{https://static01.graylady3jvrrxbe.onion/packages/flash/multimedia/ICONS/transparent.png}

\begin{itemize}
\tightlist
\item
  Alejandra Ramírez
\item
  Tijuana, Mexico
\item
  June 30
\end{itemize}

\protect\hyperlink{item-geneva-tatem}{}

Doctor
\includegraphics{https://static01.graylady3jvrrxbe.onion/packages/flash/multimedia/ICONS/transparent.png}

Being the only Black pulmonary and critical care physician at my
hospital has made my experience very different.

\begin{itemize}
\tightlist
\item
  Geneva Tatem
\item
  Detroit
\item
  April 14
\end{itemize}

\protect\hyperlink{item-natasha-raziuddin}{}

Physician Assistant
\includegraphics{https://static01.graylady3jvrrxbe.onion/packages/flash/multimedia/ICONS/transparent.png}

\begin{itemize}
\tightlist
\item
  Natasha Raziuddin
\item
  Jessup, Md.
\item
  May 6
\end{itemize}

\protect\hyperlink{item-danielle-s-wilcock}{}

Nurse
\includegraphics{https://static01.graylady3jvrrxbe.onion/packages/flash/multimedia/ICONS/transparent.png}

\begin{itemize}
\tightlist
\item
  Danielle S. Wilcock
\item
  Ballston Spa, N.Y.
\item
  July 7
\end{itemize}

\protect\hyperlink{item-anil-magge}{}

Doctor
\includegraphics{https://static01.graylady3jvrrxbe.onion/packages/flash/multimedia/ICONS/transparent.png}

\begin{itemize}
\tightlist
\item
  Anil Magge
\item
  West Hartford, Conn.
\item
  May 2
\end{itemize}

\protect\hyperlink{item-hadia-kohi}{}

Midwife
\includegraphics{https://static01.graylady3jvrrxbe.onion/packages/flash/multimedia/ICONS/transparent.png}

\begin{itemize}
\tightlist
\item
  Hadia Kohi
\item
  Feroz Koh, Afghanistan
\item
  April 25
\end{itemize}

\protect\hyperlink{item-ferrukh-faruqui}{}

Doctor
\includegraphics{https://static01.graylady3jvrrxbe.onion/packages/flash/multimedia/ICONS/transparent.png}

\begin{itemize}
\tightlist
\item
  Ferrukh Faruqui
\item
  Ottawa
\item
  April 10
\end{itemize}

\protect\hyperlink{item-shane-woods}{}

Clinical Technician
\includegraphics{https://static01.graylady3jvrrxbe.onion/packages/flash/multimedia/ICONS/transparent.png}

\begin{itemize}
\tightlist
\item
  Shane Woods
\item
  Arlington, Va.
\item
  May 5
\end{itemize}

\protect\hyperlink{item-viquar-mundozie}{}

Doctor
\includegraphics{https://static01.graylady3jvrrxbe.onion/packages/flash/multimedia/ICONS/transparent.png}

\begin{itemize}
\tightlist
\item
  Viquar Mundozie
\item
  Lake in the Hills, Ill.
\item
  June 8
\end{itemize}

\protect\hyperlink{item-zaf-qasim}{}

Doctor
\includegraphics{https://static01.graylady3jvrrxbe.onion/packages/flash/multimedia/ICONS/transparent.png}

\begin{itemize}
\tightlist
\item
  Zaf Qasim
\item
  Philadelphia
\item
  April 18
\end{itemize}

\protect\hyperlink{item-patricio-acosta}{}

Doctor
\includegraphics{https://static01.graylady3jvrrxbe.onion/packages/flash/multimedia/ICONS/transparent.png}

\begin{itemize}
\tightlist
\item
  Patricio Acosta
\item
  Buenos Aires
\item
  June 30
\end{itemize}

\protect\hyperlink{item-lee-kojanis}{}

Doctor
\includegraphics{https://static01.graylady3jvrrxbe.onion/packages/flash/multimedia/ICONS/transparent.png}

\begin{itemize}
\tightlist
\item
  Lee Kojanis
\item
  Manhattan, N.Y.
\item
  July 7
\end{itemize}

\protect\hyperlink{item-matheus-lopes}{}

Doctor
\includegraphics{https://static01.graylady3jvrrxbe.onion/packages/flash/multimedia/ICONS/transparent.png}

\begin{itemize}
\tightlist
\item
  Matheus Lopes
\item
  Recife, Brazil
\item
  May 3
\end{itemize}

\protect\hyperlink{item-gino-picano}{}

Physician Assistant
\includegraphics{https://static01.graylady3jvrrxbe.onion/packages/flash/multimedia/ICONS/transparent.png}

\begin{itemize}
\tightlist
\item
  Gino Picano
\item
  Mount Kisco, N.Y.
\item
  April 8
\end{itemize}

\protect\hyperlink{item-heather-geiger}{}

Nurse
\includegraphics{https://static01.graylady3jvrrxbe.onion/packages/flash/multimedia/ICONS/transparent.png}

My anxiety has skyrocketed. My mind is constantly running.

\begin{itemize}
\tightlist
\item
  Heather Geiger
\item
  Boston
\item
  April 22
\end{itemize}

\protect\hyperlink{item-rishi-chopra}{}

Doctor
\includegraphics{https://static01.graylady3jvrrxbe.onion/packages/flash/multimedia/ICONS/transparent.png}

This hit our family incredibly hard and we will never be the same.

\begin{itemize}
\tightlist
\item
  Rishi Chopra
\item
  Brooklyn, N.Y.
\item
  June 10
\end{itemize}

\protect\hyperlink{item-claudine-aguilera}{}

Doctor
\includegraphics{https://static01.graylady3jvrrxbe.onion/packages/flash/multimedia/ICONS/transparent.png}

\begin{itemize}
\tightlist
\item
  Claudine Aguilera
\item
  St. Augustine, Fla.
\item
  April 9
\end{itemize}

\protect\hyperlink{item-autumn-hicks}{}

Radiation Therapist
\includegraphics{https://static01.graylady3jvrrxbe.onion/packages/flash/multimedia/ICONS/transparent.png}

\begin{itemize}
\tightlist
\item
  Autumn Hicks
\item
  Moncks Corner, S.C.
\item
  May 20
\end{itemize}

\protect\hyperlink{item-joe-berger}{}

Nurse
\includegraphics{https://static01.graylady3jvrrxbe.onion/packages/flash/multimedia/ICONS/transparent.png}

\begin{itemize}
\tightlist
\item
  Joe Berger
\item
  Duluth, Minn.
\item
  July 24
\end{itemize}

\protect\hyperlink{item-laura-oakes}{}

Nurse
\includegraphics{https://static01.graylady3jvrrxbe.onion/packages/flash/multimedia/ICONS/transparent.png}

\begin{itemize}
\tightlist
\item
  Laura Oakes
\item
  New Orleans
\item
  July 28
\end{itemize}

\protect\hyperlink{item-enrique-bolona-gilbert}{}

Doctor
\includegraphics{https://static01.graylady3jvrrxbe.onion/packages/flash/multimedia/ICONS/transparent.png}

\begin{itemize}
\tightlist
\item
  Enrique Boloña Gilbert
\item
  Guayaquil, Ecuador
\item
  April 30
\end{itemize}

\protect\hyperlink{item-justin-sanborn}{}

Physician Assistant
\includegraphics{https://static01.graylady3jvrrxbe.onion/packages/flash/multimedia/ICONS/transparent.png}

\begin{itemize}
\tightlist
\item
  Justin Sanborn
\item
  League City, Texas
\item
  April 9
\end{itemize}

\protect\hyperlink{item-italo-m-brown}{}

Doctor
\includegraphics{https://static01.graylady3jvrrxbe.onion/packages/flash/multimedia/ICONS/transparent.png}

\begin{itemize}
\tightlist
\item
  Italo M. Brown
\item
  Palo Alto, Calif.
\item
  June 11
\end{itemize}

\protect\hyperlink{item-alyne-freiberg}{}

Doctor
\includegraphics{https://static01.graylady3jvrrxbe.onion/packages/flash/multimedia/ICONS/transparent.png}

The pandemic is the current problem, but its repercussions will last.

\begin{itemize}
\tightlist
\item
  Alyne Freiberg
\item
  São Paulo, Brazil
\item
  June 8
\end{itemize}

\protect\hyperlink{item-dawn-arroyo}{}

Nurse
\includegraphics{https://static01.graylady3jvrrxbe.onion/packages/flash/multimedia/ICONS/transparent.png}

\begin{itemize}
\tightlist
\item
  Dawn Arroyo
\item
  Visalia, Calif.
\item
  June 12
\end{itemize}

\protect\hyperlink{item-vanessa-hernandez}{}

Nurse
\includegraphics{https://static01.graylady3jvrrxbe.onion/packages/flash/multimedia/ICONS/transparent.png}

\begin{itemize}
\tightlist
\item
  Vanessa Hernandez
\item
  Miami
\item
  April 9
\end{itemize}

\protect\hyperlink{item-melissa-todice}{}

Nurse
\includegraphics{https://static01.graylady3jvrrxbe.onion/packages/flash/multimedia/ICONS/transparent.png}

It kills me when I hear, ``Well, that's what they signed up for.'' I
never took a pandemic nursing class.

\begin{itemize}
\tightlist
\item
  Melissa Todice
\item
  Shelton, Conn.
\item
  April 12
\end{itemize}

\protect\hyperlink{item-nuria-poveda}{}

E.M.T.
\includegraphics{https://static01.graylady3jvrrxbe.onion/packages/flash/multimedia/ICONS/transparent.png}

I don't know what happened to her. The next day, I was afraid to ask.

\begin{itemize}
\tightlist
\item
  Núria Poveda
\item
  Barcelona, Spain
\item
  April 28
\end{itemize}

\protect\hyperlink{item-lysette-masendu-droh}{}

Doctor
\includegraphics{https://static01.graylady3jvrrxbe.onion/packages/flash/multimedia/ICONS/transparent.png}

\begin{itemize}
\tightlist
\item
  Lysette Masendu Droh
\item
  Abidjan, Ivory Coast
\item
  April 30
\end{itemize}

\protect\hyperlink{item-farhan-bashir}{}

Doctor
\includegraphics{https://static01.graylady3jvrrxbe.onion/packages/flash/multimedia/ICONS/transparent.png}

\begin{itemize}
\tightlist
\item
  Farhan Bashir
\item
  Abbottabad, Pakistan
\item
  July 12
\end{itemize}

\protect\hyperlink{item-kim-king-smith}{}

EKG Technician
\includegraphics{https://static01.graylady3jvrrxbe.onion/packages/flash/multimedia/ICONS/transparent.png}

\begin{itemize}
\tightlist
\item
  Kim King-Smith
\item
  Piscataway, N.J.
\item
  Died on March 31
\end{itemize}

\protect\hyperlink{item-charles-huschle}{}

Chaplain
\includegraphics{https://static01.graylady3jvrrxbe.onion/packages/flash/multimedia/ICONS/transparent.png}

\begin{itemize}
\tightlist
\item
  Charles Huschle
\item
  Tyngsborough, Mass.
\item
  April 9
\end{itemize}

\protect\hyperlink{item-siya-dayal}{}

Doctor
\includegraphics{https://static01.graylady3jvrrxbe.onion/packages/flash/multimedia/ICONS/transparent.png}

\begin{itemize}
\tightlist
\item
  Siya Dayal
\item
  London
\item
  April 28
\end{itemize}

\protect\hyperlink{item-richelle-sipiora}{}

Physical Therapist
\includegraphics{https://static01.graylady3jvrrxbe.onion/packages/flash/multimedia/ICONS/transparent.png}

\begin{itemize}
\tightlist
\item
  Richelle Sipiora
\item
  Brunswick, Maine
\item
  July 7
\end{itemize}

\protect\hyperlink{item-jacqueline-stapleton}{}

Nurse
\includegraphics{https://static01.graylady3jvrrxbe.onion/packages/flash/multimedia/ICONS/transparent.png}

\begin{itemize}
\tightlist
\item
  Jacqueline Stapleton
\item
  Houston
\item
  April 8
\end{itemize}

\protect\hyperlink{item-laura-bishop}{}

Doctor
\includegraphics{https://static01.graylady3jvrrxbe.onion/packages/flash/multimedia/ICONS/transparent.png}

\begin{itemize}
\tightlist
\item
  Laura Bishop
\item
  New Albany, Ind.
\item
  June 11
\end{itemize}

\protect\hyperlink{item-neo-liu}{}

Nurse
\includegraphics{https://static01.graylady3jvrrxbe.onion/packages/flash/multimedia/ICONS/transparent.png}

\begin{itemize}
\tightlist
\item
  Neo Liu
\item
  Wuhan, China
\item
  April 28
\end{itemize}

\protect\hyperlink{item-claudio-del-monte}{}

Chaplain
\includegraphics{https://static01.graylady3jvrrxbe.onion/packages/flash/multimedia/ICONS/transparent.png}

It's a medical situation, but it is also about the soul.

\begin{itemize}
\tightlist
\item
  Claudio Del Monte
\item
  Bergamo, Italy
\item
  April 7
\end{itemize}

\protect\hyperlink{item-hannah-hillebrand}{}

Nurse
\includegraphics{https://static01.graylady3jvrrxbe.onion/packages/flash/multimedia/ICONS/transparent.png}

\begin{itemize}
\tightlist
\item
  Hannah Hillebrand
\item
  Portland, Ore.
\item
  April 21
\end{itemize}

\protect\hyperlink{item-rebecca-mahn}{}

Doctor
\includegraphics{https://static01.graylady3jvrrxbe.onion/packages/flash/multimedia/ICONS/transparent.png}

\begin{itemize}
\tightlist
\item
  Rebecca Mahn
\item
  Manhattan, N.Y.
\item
  April 23
\end{itemize}

\protect\hyperlink{item-stephen-berns}{}

Doctor
\includegraphics{https://static01.graylady3jvrrxbe.onion/packages/flash/multimedia/ICONS/transparent.png}

\begin{itemize}
\tightlist
\item
  Stephen Berns
\item
  Burlington, Vt.
\item
  April 20
\end{itemize}

\protect\hyperlink{item-angelica-carolina-olivares-cuellar}{}

Nurse
\includegraphics{https://static01.graylady3jvrrxbe.onion/packages/flash/multimedia/ICONS/transparent.png}

\begin{itemize}
\tightlist
\item
  Angélica Carolina Olivares Cuellar
\item
  Reynosa, Mexico
\item
  June 30
\end{itemize}

\protect\hyperlink{item-becki-rwubusisi}{}

Nurse Practitioner
\includegraphics{https://static01.graylady3jvrrxbe.onion/packages/flash/multimedia/ICONS/transparent.png}

\begin{itemize}
\tightlist
\item
  Becki Rwubusisi
\item
  Denver
\item
  April 10
\end{itemize}

\protect\hyperlink{item-paul-m-shaniuk}{}

Doctor
\includegraphics{https://static01.graylady3jvrrxbe.onion/packages/flash/multimedia/ICONS/transparent.png}

\begin{itemize}
\tightlist
\item
  Paul M. Shaniuk
\item
  Lyndhurst, Ohio
\item
  April 9
\end{itemize}

\protect\hyperlink{item-joshua-schwarzbaum}{}

Doctor
\includegraphics{https://static01.graylady3jvrrxbe.onion/packages/flash/multimedia/ICONS/transparent.png}

\begin{itemize}
\tightlist
\item
  Joshua Schwarzbaum
\item
  Jersey City, N.J.
\item
  April 20
\end{itemize}

\protect\hyperlink{item-ana-ruth-santana}{}

Nursing Technician
\includegraphics{https://static01.graylady3jvrrxbe.onion/packages/flash/multimedia/ICONS/transparent.png}

I would never have imagined that I would be hospitalized in the same
place I work.

\begin{itemize}
\tightlist
\item
  Ana Ruth Santana
\item
  Belém do Pará, Brazil
\item
  June 29
\end{itemize}

\protect\hyperlink{item-timothy-wong}{}

Doctor
\includegraphics{https://static01.graylady3jvrrxbe.onion/packages/flash/multimedia/ICONS/transparent.png}

\begin{itemize}
\tightlist
\item
  Timothy Wong
\item
  Singapore
\item
  June 12
\end{itemize}

\protect\hyperlink{item-sahrish-ilyas}{}

Doctor
\includegraphics{https://static01.graylady3jvrrxbe.onion/packages/flash/multimedia/ICONS/transparent.png}

\begin{itemize}
\tightlist
\item
  Sahrish Ilyas
\item
  Detroit
\item
  April 8
\end{itemize}

\protect\hyperlink{item-jennie-sablayan}{}

Nurse
\includegraphics{https://static01.graylady3jvrrxbe.onion/packages/flash/multimedia/ICONS/transparent.png}

\begin{itemize}
\tightlist
\item
  Jennie Sablayan
\item
  London
\item
  Died on May 5
\end{itemize}

\protect\hyperlink{item-courtney-bell}{}

E.M.T.
\includegraphics{https://static01.graylady3jvrrxbe.onion/packages/flash/multimedia/ICONS/transparent.png}

\begin{itemize}
\tightlist
\item
  Courtney Bell
\item
  Tampa, Fla.
\item
  June 9
\end{itemize}

\protect\hyperlink{item-david-xavier-menendez-espinoza}{}

Doctor
\includegraphics{https://static01.graylady3jvrrxbe.onion/packages/flash/multimedia/ICONS/transparent.png}

\begin{itemize}
\tightlist
\item
  David Xavier Menéndez Espinoza
\item
  Portoviejo, Ecuador
\item
  July 1
\end{itemize}

\protect\hyperlink{item-gabriel-herrera}{}

Doctor
\includegraphics{https://static01.graylady3jvrrxbe.onion/packages/flash/multimedia/ICONS/transparent.png}

This pandemic has made me especially aware of the importance of the most
human aspects of being a doctor.

\begin{itemize}
\tightlist
\item
  Gabriel Herrera
\item
  Bogotá, Colombia
\item
  June 24
\end{itemize}

\protect\hyperlink{item-pria-anand}{}

Doctor
\includegraphics{https://static01.graylady3jvrrxbe.onion/packages/flash/multimedia/ICONS/transparent.png}

\begin{itemize}
\tightlist
\item
  Pria Anand
\item
  Boston
\item
  April 9
\end{itemize}

\protect\hyperlink{item-jo-murphy}{}

Chaplain
\includegraphics{https://static01.graylady3jvrrxbe.onion/packages/flash/multimedia/ICONS/transparent.png}

\begin{itemize}
\tightlist
\item
  Jo Murphy
\item
  Boston
\item
  April 23
\end{itemize}

\protect\hyperlink{item-eyal-kedar}{}

Doctor
\includegraphics{https://static01.graylady3jvrrxbe.onion/packages/flash/multimedia/ICONS/transparent.png}

\begin{itemize}
\tightlist
\item
  Eyal Kedar
\item
  Potsdam, N.Y.
\item
  April 9
\end{itemize}

\protect\hyperlink{item-valerie-vaughn}{}

Doctor
\includegraphics{https://static01.graylady3jvrrxbe.onion/packages/flash/multimedia/ICONS/transparent.png}

\begin{itemize}
\tightlist
\item
  Valerie Vaughn
\item
  Ann Arbor, Mich.
\item
  April 17
\end{itemize}

\protect\hyperlink{item-meg-mueller}{}

Doctor
\includegraphics{https://static01.graylady3jvrrxbe.onion/packages/flash/multimedia/ICONS/transparent.png}

\begin{itemize}
\tightlist
\item
  Meg Mueller
\item
  South Kingstown, R.I.
\item
  July 11
\end{itemize}

\protect\hyperlink{item-susana-de-anda}{}

Nurse
\includegraphics{https://static01.graylady3jvrrxbe.onion/packages/flash/multimedia/ICONS/transparent.png}

They didn't follow any of the social distancing measures. In a blink of
an eye, our E.R. became saturated.

\begin{itemize}
\tightlist
\item
  Susana de Anda
\item
  Mexico City
\item
  June 9
\end{itemize}

\protect\hyperlink{item-amanda-ramalho}{}

Nurse
\includegraphics{https://static01.graylady3jvrrxbe.onion/packages/flash/multimedia/ICONS/transparent.png}

I am afraid! I never thought I'd be living in this warlike situation.

\begin{itemize}
\tightlist
\item
  Amanda Ramalho
\item
  Pelotas, Brazil
\item
  April 13
\end{itemize}

\protect\hyperlink{item-abdul-ahad-amiri}{}

Doctor
\includegraphics{https://static01.graylady3jvrrxbe.onion/packages/flash/multimedia/ICONS/transparent.png}

\begin{itemize}
\tightlist
\item
  Abdul Ahad Amiri
\item
  Kunduz City, Afghanistan
\item
  April 25
\end{itemize}

\protect\hyperlink{item-nabeela-arbee-kalidas}{}

Doctor
\includegraphics{https://static01.graylady3jvrrxbe.onion/packages/flash/multimedia/ICONS/transparent.png}

I am living with guilt and sadness that I couldn't help in time.

\begin{itemize}
\tightlist
\item
  Nabeela Arbee-Kalidas
\item
  Johannesburg
\item
  April 9
\end{itemize}

\protect\hyperlink{item-h-joyce-morano}{}

Doctor
\includegraphics{https://static01.graylady3jvrrxbe.onion/packages/flash/multimedia/ICONS/transparent.png}

\begin{itemize}
\tightlist
\item
  H. Joyce Morano
\item
  Dallas, Pa.
\item
  July 7
\end{itemize}

\protect\hyperlink{item-torie-jones}{}

Nurse
\includegraphics{https://static01.graylady3jvrrxbe.onion/packages/flash/multimedia/ICONS/transparent.png}

\begin{itemize}
\tightlist
\item
  Torie Jones
\item
  Jersey City, N.J.
\item
  April 9
\end{itemize}

\protect\hyperlink{item-frank-gabrin}{}

Doctor
\includegraphics{https://static01.graylady3jvrrxbe.onion/packages/flash/multimedia/ICONS/transparent.png}

\begin{itemize}
\tightlist
\item
  Frank Gabrin
\item
  Manhattan, N.Y.
\item
  Died on March 31
\end{itemize}

\protect\hyperlink{item-michael-slater}{}

Doctor
\includegraphics{https://static01.graylady3jvrrxbe.onion/packages/flash/multimedia/ICONS/transparent.png}

\begin{itemize}
\tightlist
\item
  Michael Slater
\item
  Evanston, Ill.
\item
  April 9
\end{itemize}

\protect\hyperlink{item-leonardo-bianchi}{}

Doctor
\includegraphics{https://static01.graylady3jvrrxbe.onion/packages/flash/multimedia/ICONS/transparent.png}

\begin{itemize}
\tightlist
\item
  Leonardo Bianchi
\item
  São Paulo, Brazil
\item
  June 25
\end{itemize}

\protect\hyperlink{item-eva-gelernt}{}

Nursing Technician
\includegraphics{https://static01.graylady3jvrrxbe.onion/packages/flash/multimedia/ICONS/transparent.png}

\begin{itemize}
\tightlist
\item
  Eva Gelernt
\item
  Manhattan, N.Y.
\item
  May 19
\end{itemize}

\protect\hyperlink{item-lois-olney}{}

Nurse
\includegraphics{https://static01.graylady3jvrrxbe.onion/packages/flash/multimedia/ICONS/transparent.png}

\begin{itemize}
\tightlist
\item
  Lois Olney
\item
  Lancaster, Pa.
\item
  April 9
\end{itemize}

\protect\hyperlink{item-keiji-oi}{}

Doctor
\includegraphics{https://static01.graylady3jvrrxbe.onion/packages/flash/multimedia/ICONS/transparent.png}

\begin{itemize}
\tightlist
\item
  Keiji Oi
\item
  Tokyo
\item
  May 4
\end{itemize}

\protect\hyperlink{item-maria-alvares}{}

Nurse
\includegraphics{https://static01.graylady3jvrrxbe.onion/packages/flash/multimedia/ICONS/transparent.png}

\begin{itemize}
\tightlist
\item
  María Alvares
\item
  León, Mexico
\item
  July 1
\end{itemize}

\protect\hyperlink{item-aparna-parikh}{}

Doctor
\includegraphics{https://static01.graylady3jvrrxbe.onion/packages/flash/multimedia/ICONS/transparent.png}

\begin{itemize}
\tightlist
\item
  Aparna Parikh
\item
  Boston
\item
  May 8
\end{itemize}

\protect\hyperlink{item-rohail-asrar}{}

Doctor
\includegraphics{https://static01.graylady3jvrrxbe.onion/packages/flash/multimedia/ICONS/transparent.png}

\begin{itemize}
\tightlist
\item
  Rohail Asrar
\item
  North Brunswick, N.J.
\item
  April 9
\end{itemize}

\protect\hyperlink{item-patricia-lafontant}{}

Nurse Practitioner
\includegraphics{https://static01.graylady3jvrrxbe.onion/packages/flash/multimedia/ICONS/transparent.png}

I will never forget how sick he looked. That's when the virus became
very real to me.

\begin{itemize}
\tightlist
\item
  Patricia Lafontant
\item
  Washington, D.C.
\item
  April 10
\end{itemize}

\protect\hyperlink{item-erica-wendling}{}

Respiratory Therapist
\includegraphics{https://static01.graylady3jvrrxbe.onion/packages/flash/multimedia/ICONS/transparent.png}

\begin{itemize}
\tightlist
\item
  Erica Wendling
\item
  Minneapolis
\item
  April 9
\end{itemize}

\protect\hyperlink{item-sridevi-rajeeve}{}

Doctor
\includegraphics{https://static01.graylady3jvrrxbe.onion/packages/flash/multimedia/ICONS/transparent.png}

\begin{itemize}
\tightlist
\item
  Sridevi Rajeeve
\item
  Manhattan, N.Y.
\item
  July 10
\end{itemize}

\protect\hyperlink{item-zhang-wendan}{}

Nurse
\includegraphics{https://static01.graylady3jvrrxbe.onion/packages/flash/multimedia/ICONS/transparent.png}

\begin{itemize}
\tightlist
\item
  Zhang Wendan
\item
  Huanggang, China
\item
  May 2
\end{itemize}

\protect\hyperlink{item-min-chul-kim}{}

Doctor
\includegraphics{https://static01.graylady3jvrrxbe.onion/packages/flash/multimedia/ICONS/transparent.png}

\begin{itemize}
\tightlist
\item
  Min Chul Kim
\item
  Seoul, South Korea
\item
  April 27
\end{itemize}

\protect\hyperlink{item-nancy-dines}{}

Nurse
\includegraphics{https://static01.graylady3jvrrxbe.onion/packages/flash/multimedia/ICONS/transparent.png}

\begin{itemize}
\tightlist
\item
  Nancy Dines
\item
  Kingston, Pa.
\item
  April 25
\end{itemize}

\protect\hyperlink{item-chetna-singh}{}

Doctor
\includegraphics{https://static01.graylady3jvrrxbe.onion/packages/flash/multimedia/ICONS/transparent.png}

\begin{itemize}
\tightlist
\item
  Chetna Singh
\item
  Upper Freehold, N.J.
\item
  April 10
\end{itemize}

\protect\hyperlink{item-willie-grady}{}

Nurse
\includegraphics{https://static01.graylady3jvrrxbe.onion/packages/flash/multimedia/ICONS/transparent.png}

My daughter --- she's such a sweetheart --- she cried for like 30
minutes. But I knew I had to be here.

\begin{itemize}
\tightlist
\item
  Willie Grady
\item
  Atlanta
\item
  April 29
\end{itemize}

\protect\hyperlink{item-agustina-huilca}{}

Doctor
\includegraphics{https://static01.graylady3jvrrxbe.onion/packages/flash/multimedia/ICONS/transparent.png}

\begin{itemize}
\tightlist
\item
  Agustina Huilca
\item
  Loreto, Peru
\item
  June 10
\end{itemize}

\protect\hyperlink{item-ashley-luanne-kay-chermak}{}

Nurse
\includegraphics{https://static01.graylady3jvrrxbe.onion/packages/flash/multimedia/ICONS/transparent.png}

\begin{itemize}
\tightlist
\item
  Ashley Luanne Kay Chermak
\item
  Fridley, Minn.
\item
  April 10
\end{itemize}

\protect\hyperlink{item-patricia-tiu}{}

Nurse
\includegraphics{https://static01.graylady3jvrrxbe.onion/packages/flash/multimedia/ICONS/transparent.png}

\begin{itemize}
\tightlist
\item
  Patricia Tiu
\item
  Queens, N.Y.
\item
  June 6
\end{itemize}

\protect\hyperlink{item-elisa-dannemiller}{}

Doctor
\includegraphics{https://static01.graylady3jvrrxbe.onion/packages/flash/multimedia/ICONS/transparent.png}

\begin{itemize}
\tightlist
\item
  Elisa Dannemiller
\item
  Denver
\item
  July 7
\end{itemize}

\protect\hyperlink{item-petronella-benjamin}{}

Nurse
\includegraphics{https://static01.graylady3jvrrxbe.onion/packages/flash/multimedia/ICONS/transparent.png}

\begin{itemize}
\tightlist
\item
  Petronella Benjamin
\item
  Cape Town, South Africa
\item
  Died on April 29
\end{itemize}

\protect\hyperlink{item-samantha-irvine}{}

Nurse
\includegraphics{https://static01.graylady3jvrrxbe.onion/packages/flash/multimedia/ICONS/transparent.png}

\begin{itemize}
\tightlist
\item
  Samantha Irvine
\item
  Staten Island, N.Y.
\item
  April 10
\end{itemize}

\protect\hyperlink{item-liu-taotao}{}

Doctor
\includegraphics{https://static01.graylady3jvrrxbe.onion/packages/flash/multimedia/ICONS/transparent.png}

There were cases in which the whole family got sick and died. It feels
like a war.

\begin{itemize}
\tightlist
\item
  Liu Taotao
\item
  Beijing
\item
  April 28
\end{itemize}

\protect\hyperlink{item-laura-janneck}{}

Doctor
\includegraphics{https://static01.graylady3jvrrxbe.onion/packages/flash/multimedia/ICONS/transparent.png}

\begin{itemize}
\tightlist
\item
  Laura Janneck
\item
  Boston
\item
  July 7
\end{itemize}

\protect\hyperlink{item-yaciel-almira}{}

Nurse
\includegraphics{https://static01.graylady3jvrrxbe.onion/packages/flash/multimedia/ICONS/transparent.png}

\begin{itemize}
\tightlist
\item
  Yaciel Almira
\item
  Miami Beach, Fla.
\item
  May 15
\end{itemize}

\protect\hyperlink{item-marcelle-pignanelli}{}

Doctor
\includegraphics{https://static01.graylady3jvrrxbe.onion/packages/flash/multimedia/ICONS/transparent.png}

\begin{itemize}
\tightlist
\item
  Marcelle Pignanelli
\item
  Queens, N.Y.
\item
  April 10
\end{itemize}

\protect\hyperlink{item-moussa-alzouma-mahamadou}{}

Nurse
\includegraphics{https://static01.graylady3jvrrxbe.onion/packages/flash/multimedia/ICONS/transparent.png}

\begin{itemize}
\tightlist
\item
  Moussa Alzouma Mahamadou
\item
  Madarounfa, Niger
\item
  April 30
\end{itemize}

\protect\hyperlink{item-amanda-dasaro}{}

E.M.T.
\includegraphics{https://static01.graylady3jvrrxbe.onion/packages/flash/multimedia/ICONS/transparent.png}

Did I want to quit and stay home? Yes.

\begin{itemize}
\tightlist
\item
  Amanda Dasaro
\item
  Staten Island, N.Y.
\item
  April 28
\end{itemize}

\protect\hyperlink{item-beth-oller}{}

Doctor
\includegraphics{https://static01.graylady3jvrrxbe.onion/packages/flash/multimedia/ICONS/transparent.png}

Looking at every patient as a potential carrier changes you.

\begin{itemize}
\tightlist
\item
  Beth Oller
\item
  Stockton, Kan.
\item
  April 9
\end{itemize}

\protect\hyperlink{item-karina-hernandez-flores}{}

Nurse
\includegraphics{https://static01.graylady3jvrrxbe.onion/packages/flash/multimedia/ICONS/transparent.png}

\begin{itemize}
\tightlist
\item
  Karina Hernández Flores
\item
  Hermosillo, Mexico
\item
  July 1
\end{itemize}

\protect\hyperlink{item-james-house}{}

Nurse
\includegraphics{https://static01.graylady3jvrrxbe.onion/packages/flash/multimedia/ICONS/transparent.png}

\begin{itemize}
\tightlist
\item
  James House
\item
  Warren, Mich.
\item
  Died on March 31
\end{itemize}

\protect\hyperlink{item-jill-cohen}{}

Nurse
\includegraphics{https://static01.graylady3jvrrxbe.onion/packages/flash/multimedia/ICONS/transparent.png}

\begin{itemize}
\tightlist
\item
  Jill Cohen
\item
  Truckee, Calif.
\item
  April 8
\end{itemize}

\protect\hyperlink{item-imran-nazir}{}

Doctor
\includegraphics{https://static01.graylady3jvrrxbe.onion/packages/flash/multimedia/ICONS/transparent.png}

\begin{itemize}
\tightlist
\item
  Imran Nazir
\item
  Pulwama, Kashmir
\item
  April 24
\end{itemize}

\protect\hyperlink{item-ashley-crumpler}{}

Nurse
\includegraphics{https://static01.graylady3jvrrxbe.onion/packages/flash/multimedia/ICONS/transparent.png}

I'll never take each breath for granted again.

\begin{itemize}
\tightlist
\item
  Ashley Crumpler
\item
  Atlanta
\item
  May 1
\end{itemize}

\protect\hyperlink{item-sylvia-fallon}{}

Nurse Practitioner
\includegraphics{https://static01.graylady3jvrrxbe.onion/packages/flash/multimedia/ICONS/transparent.png}

\begin{itemize}
\tightlist
\item
  Sylvia Fallon
\item
  Brooklyn, N.Y.
\item
  April 10
\end{itemize}

\protect\hyperlink{item-shivyon-mitchell}{}

Nurse
\includegraphics{https://static01.graylady3jvrrxbe.onion/packages/flash/multimedia/ICONS/transparent.png}

\begin{itemize}
\tightlist
\item
  Shivyon Mitchell
\item
  Jackson, Wyo.
\item
  April 11
\end{itemize}

\protect\hyperlink{item-colette-badjo}{}

Doctor
\includegraphics{https://static01.graylady3jvrrxbe.onion/packages/flash/multimedia/ICONS/transparent.png}

\begin{itemize}
\tightlist
\item
  Colette Badjo
\item
  Montreal
\item
  April 30
\end{itemize}

\protect\hyperlink{item-ari-ciment}{}

Doctor
\includegraphics{https://static01.graylady3jvrrxbe.onion/packages/flash/multimedia/ICONS/transparent.png}

The patients don't know one another yet, but they're all interwoven.

\begin{itemize}
\tightlist
\item
  Ari Ciment
\item
  Miami Beach
\item
  April 12
\end{itemize}

\protect\hyperlink{item-jay-w-lee}{}

Doctor
\includegraphics{https://static01.graylady3jvrrxbe.onion/packages/flash/multimedia/ICONS/transparent.png}

\begin{itemize}
\tightlist
\item
  Jay W. Lee
\item
  Huntington Beach, Calif.
\item
  April 12
\end{itemize}

\protect\hyperlink{item-astrid-vazquez}{}

Doctor
\includegraphics{https://static01.graylady3jvrrxbe.onion/packages/flash/multimedia/ICONS/transparent.png}

\begin{itemize}
\tightlist
\item
  Astrid Vázquez
\item
  Puebla, Mexico
\item
  July 2
\end{itemize}

\protect\hyperlink{item-myeong-hae-kyung}{}

Nurse
\includegraphics{https://static01.graylady3jvrrxbe.onion/packages/flash/multimedia/ICONS/transparent.png}

\begin{itemize}
\tightlist
\item
  Myeong Hae-kyung
\item
  Daegu, South Korea
\item
  April 27
\end{itemize}

\protect\hyperlink{item-isabel-munoz}{}

Doctor
\includegraphics{https://static01.graylady3jvrrxbe.onion/packages/flash/multimedia/ICONS/transparent.png}

\begin{itemize}
\tightlist
\item
  Isabel Muñoz
\item
  Salamanca, Spain
\item
  Died on March 24
\end{itemize}

\protect\hyperlink{item-jennifer-tszeng}{}

Nurse
\includegraphics{https://static01.graylady3jvrrxbe.onion/packages/flash/multimedia/ICONS/transparent.png}

\begin{itemize}
\tightlist
\item
  Jennifer Tszeng
\item
  Manhattan, N.Y.
\item
  April 15
\end{itemize}

\protect\hyperlink{item-sophia-corbitt}{}

Nurse
\includegraphics{https://static01.graylady3jvrrxbe.onion/packages/flash/multimedia/ICONS/transparent.png}

\begin{itemize}
\tightlist
\item
  Sophia Corbitt
\item
  Maryland
\item
  April 8
\end{itemize}

\protect\hyperlink{item-richard-wang}{}

Doctor
\includegraphics{https://static01.graylady3jvrrxbe.onion/packages/flash/multimedia/ICONS/transparent.png}

\begin{itemize}
\tightlist
\item
  Richard Wang
\item
  San Francisco
\item
  April 15
\end{itemize}

\protect\hyperlink{item-sophia-walker-henry}{}

Nurse
\includegraphics{https://static01.graylady3jvrrxbe.onion/packages/flash/multimedia/ICONS/transparent.png}

\begin{itemize}
\tightlist
\item
  Sophia Walker Henry
\item
  Pikesville, Md.
\item
  June 9
\end{itemize}

\protect\hyperlink{item-anna-duarte-velasco}{}

Nurse
\includegraphics{https://static01.graylady3jvrrxbe.onion/packages/flash/multimedia/ICONS/transparent.png}

The look of fear of dying in many people's eyes will never be erased
from my memory. I feel rage and helplessness.

\begin{itemize}
\tightlist
\item
  Anna Duarte Velasco
\item
  Barcelona, Spain
\item
  April 28
\end{itemize}

\protect\hyperlink{item-li-wenliang}{}

Doctor
\includegraphics{https://static01.graylady3jvrrxbe.onion/packages/flash/multimedia/ICONS/transparent.png}

\begin{itemize}
\tightlist
\item
  Li Wenliang
\item
  Wuhan, China
\item
  Died on Feb. 7
\end{itemize}

\protect\hyperlink{item-john-rose}{}

Doctor
\includegraphics{https://static01.graylady3jvrrxbe.onion/packages/flash/multimedia/ICONS/transparent.png}

\begin{itemize}
\tightlist
\item
  John Rose
\item
  Davis, Calif.
\item
  May 5
\end{itemize}

\protect\hyperlink{item-jaques-sztajnbok}{}

Doctor
\includegraphics{https://static01.graylady3jvrrxbe.onion/packages/flash/multimedia/ICONS/transparent.png}

\begin{itemize}
\tightlist
\item
  Jaques Sztajnbok
\item
  São Paulo, Brazil
\item
  April 30
\end{itemize}

\protect\hyperlink{item-sarah-comrie}{}

Physician Assistant
\includegraphics{https://static01.graylady3jvrrxbe.onion/packages/flash/multimedia/ICONS/transparent.png}

\begin{itemize}
\tightlist
\item
  Sarah Comrie
\item
  Brooklyn, N.Y.
\item
  April 15
\end{itemize}

\protect\hyperlink{item-agustin-borjon}{}

Doctor
\includegraphics{https://static01.graylady3jvrrxbe.onion/packages/flash/multimedia/ICONS/transparent.png}

\begin{itemize}
\tightlist
\item
  Agustin Borjon
\item
  East Orange, N.J.
\item
  April 15
\end{itemize}

\protect\hyperlink{item-jennifer-vaisman}{}

Nurse
\includegraphics{https://static01.graylady3jvrrxbe.onion/packages/flash/multimedia/ICONS/transparent.png}

\begin{itemize}
\tightlist
\item
  Jennifer Vaisman
\item
  San Jose, Calif.
\item
  April 15
\end{itemize}

\protect\hyperlink{item-shawnaree-lee}{}

Doctor
\includegraphics{https://static01.graylady3jvrrxbe.onion/packages/flash/multimedia/ICONS/transparent.png}

It is crucially important that we all look at our history honestly.

\begin{itemize}
\tightlist
\item
  Shawnaree Lee
\item
  Oklahoma City
\item
  June 9
\end{itemize}

\protect\hyperlink{item-lauren-barlog}{}

Doctor
\includegraphics{https://static01.graylady3jvrrxbe.onion/packages/flash/multimedia/ICONS/transparent.png}

\begin{itemize}
\tightlist
\item
  Lauren Barlog
\item
  Brooklyn, N.Y.
\item
  April 16
\end{itemize}

\protect\hyperlink{item-eric-m-thomas}{}

Paramedic and Physician Assistant
\includegraphics{https://static01.graylady3jvrrxbe.onion/packages/flash/multimedia/ICONS/transparent.png}

\begin{itemize}
\tightlist
\item
  Eric M. Thomas
\item
  Rochester, N.Y.
\item
  April 16
\end{itemize}

\protect\hyperlink{item-k-pradeep-kumar}{}

Doctor
\includegraphics{https://static01.graylady3jvrrxbe.onion/packages/flash/multimedia/ICONS/transparent.png}

\begin{itemize}
\tightlist
\item
  K. Pradeep Kumar
\item
  Chennai, India
\item
  April 28
\end{itemize}

\protect\hyperlink{item-ansel-oommen}{}

Medical Lab Scientist
\includegraphics{https://static01.graylady3jvrrxbe.onion/packages/flash/multimedia/ICONS/transparent.png}

\begin{itemize}
\tightlist
\item
  Ansel Oommen
\item
  Manhattan, N.Y.
\item
  April 16
\end{itemize}

\protect\hyperlink{item-letizia-rossi}{}

Doctor
\includegraphics{https://static01.graylady3jvrrxbe.onion/packages/flash/multimedia/ICONS/transparent.png}

Life is slowly getting back to normal. However, you live with the
anxiety that we will start to see a new wave of contagion

\begin{itemize}
\tightlist
\item
   Letizia Rossi
\item
  Rome, Italy
\item
  June 12
\end{itemize}

\protect\hyperlink{item-alison-macleod}{}

Nurse Practitioner
\includegraphics{https://static01.graylady3jvrrxbe.onion/packages/flash/multimedia/ICONS/transparent.png}

\begin{itemize}
\tightlist
\item
  Alison MacLeod
\item
  Charleston, S.C.
\item
  April 16
\end{itemize}

\protect\hyperlink{item-viel-catig}{}

Nurse
\includegraphics{https://static01.graylady3jvrrxbe.onion/packages/flash/multimedia/ICONS/transparent.png}

\begin{itemize}
\tightlist
\item
  Viel Catig
\item
  Los Angeles
\item
  July 1
\end{itemize}

\protect\hyperlink{item-dalifer-freites-nunez}{}

Doctor
\includegraphics{https://static01.graylady3jvrrxbe.onion/packages/flash/multimedia/ICONS/transparent.png}

\begin{itemize}
\tightlist
\item
  Dalifer Freites Nuñez
\item
  Madrid
\item
  April 8
\end{itemize}

\protect\hyperlink{item-satoshi-toyama}{}

Doctor
\includegraphics{https://static01.graylady3jvrrxbe.onion/packages/flash/multimedia/ICONS/transparent.png}

\begin{itemize}
\tightlist
\item
  Satoshi Toyama
\item
  Tokyo
\item
  May 4
\end{itemize}

\protect\hyperlink{item-tom-corso}{}

Physician Assistant
\includegraphics{https://static01.graylady3jvrrxbe.onion/packages/flash/multimedia/ICONS/transparent.png}

\begin{itemize}
\tightlist
\item
  Tom Corso
\item
  Sayville, N.Y.
\item
  April 16
\end{itemize}

\protect\hyperlink{item-becky-williams}{}

Paramedic
\includegraphics{https://static01.graylady3jvrrxbe.onion/packages/flash/multimedia/ICONS/transparent.png}

I feel like I am working in the Twilight Zone now.

\begin{itemize}
\tightlist
\item
  Becky Williams
\item
  San Bernardino, Calif.
\item
  April 21
\end{itemize}

\protect\hyperlink{item-kayla-sudduth}{}

Technician
\includegraphics{https://static01.graylady3jvrrxbe.onion/packages/flash/multimedia/ICONS/transparent.png}

\begin{itemize}
\tightlist
\item
  Kayla Sudduth
\item
  New Bedford, Mass.
\item
  April 16
\end{itemize}

\protect\hyperlink{item-karla-chamorro}{}

Doctor
\includegraphics{https://static01.graylady3jvrrxbe.onion/packages/flash/multimedia/ICONS/transparent.png}

\begin{itemize}
\tightlist
\item
  Karla Chamorro
\item
  San José, Costa Rica
\item
  April 8
\end{itemize}

\protect\hyperlink{item-hesham-hassaballa}{}

Doctor
\includegraphics{https://static01.graylady3jvrrxbe.onion/packages/flash/multimedia/ICONS/transparent.png}

\begin{itemize}
\tightlist
\item
  Hesham Hassaballa
\item
  Hinsdale, Ill.
\item
  April 16
\end{itemize}

\protect\hyperlink{item-doug-entenman}{}

Nurse
\includegraphics{https://static01.graylady3jvrrxbe.onion/packages/flash/multimedia/ICONS/transparent.png}

\begin{itemize}
\tightlist
\item
  Doug Entenman
\item
  Dover, Del.
\item
  Died on May 10
\end{itemize}

\protect\hyperlink{item-chinazo-cunningham}{}

Doctor
\includegraphics{https://static01.graylady3jvrrxbe.onion/packages/flash/multimedia/ICONS/transparent.png}

I'm cautiously hopeful. I'm also tired, frankly, because this is not
new.

\begin{itemize}
\tightlist
\item
  Chinazo Cunningham
\item
  Nyack, N.Y.
\item
  June 15
\end{itemize}

\protect\hyperlink{item-leigh-barrow}{}

Doctor
\includegraphics{https://static01.graylady3jvrrxbe.onion/packages/flash/multimedia/ICONS/transparent.png}

What's frustrating now is living in a city that's wide open and pretends
Covid-19 never existed.

\begin{itemize}
\tightlist
\item
  Leigh Barrow
\item
  Tulsa, Okla.
\item
  June 10
\end{itemize}

\protect\hyperlink{item-maria-walls}{}

Nurse Practitioner
\includegraphics{https://static01.graylady3jvrrxbe.onion/packages/flash/multimedia/ICONS/transparent.png}

\begin{itemize}
\tightlist
\item
  Maria Walls
\item
  St. Louis
\item
  April 16
\end{itemize}

\protect\hyperlink{item-juju-linssen}{}

Paramedic
\includegraphics{https://static01.graylady3jvrrxbe.onion/packages/flash/multimedia/ICONS/transparent.png}

\begin{itemize}
\tightlist
\item
  JuJu Linssen
\item
  Stewartville, Minn.
\item
  April 16
\end{itemize}

\protect\hyperlink{item-rachel-zang}{}

Doctor
\includegraphics{https://static01.graylady3jvrrxbe.onion/packages/flash/multimedia/ICONS/transparent.png}

\begin{itemize}
\tightlist
\item
  Rachel Zang
\item
  Philadelphia
\item
  May 5
\end{itemize}

\protect\hyperlink{item-eleanore-kim}{}

Doctor
\includegraphics{https://static01.graylady3jvrrxbe.onion/packages/flash/multimedia/ICONS/transparent.png}

\begin{itemize}
\tightlist
\item
  Eleanore Kim
\item
  Rye, N.Y.
\item
  April 16
\end{itemize}

\protect\hyperlink{item-awah-kenneth-achua}{}

Doctor
\includegraphics{https://static01.graylady3jvrrxbe.onion/packages/flash/multimedia/ICONS/transparent.png}

\begin{itemize}
\tightlist
\item
  Awah Kenneth Achua
\item
  Buea, Cameroon
\item
  April 30
\end{itemize}

\protect\hyperlink{item-maria-saavedra-karlsson}{}

Doctor
\includegraphics{https://static01.graylady3jvrrxbe.onion/packages/flash/multimedia/ICONS/transparent.png}

\begin{itemize}
\tightlist
\item
  Maria Saavedra Karlsson
\item
  Hermosa Beach, Calif.
\item
  April 16
\end{itemize}

\protect\hyperlink{item-luke-hashiguchi}{}

Doctor
\includegraphics{https://static01.graylady3jvrrxbe.onion/packages/flash/multimedia/ICONS/transparent.png}

\begin{itemize}
\tightlist
\item
  Luke Hashiguchi
\item
  Akron, Ohio
\item
  April 17
\end{itemize}

\protect\hyperlink{item-christopher-reverte}{}

Doctor
\includegraphics{https://static01.graylady3jvrrxbe.onion/packages/flash/multimedia/ICONS/transparent.png}

\begin{itemize}
\tightlist
\item
  Christopher Reverte
\item
  Manhattan, N.Y.
\item
  April 17
\end{itemize}

\protect\hyperlink{item-giovanna-cezarino}{}

Doctor
\includegraphics{https://static01.graylady3jvrrxbe.onion/packages/flash/multimedia/ICONS/transparent.png}

\begin{itemize}
\tightlist
\item
  Giovanna Cezarino
\item
  São Paulo, Brazil
\item
  April 30
\end{itemize}

\protect\hyperlink{item-rajat-thawani}{}

Doctor
\includegraphics{https://static01.graylady3jvrrxbe.onion/packages/flash/multimedia/ICONS/transparent.png}

\begin{itemize}
\tightlist
\item
  Rajat Thawani
\item
  Brooklyn, N.Y.
\item
  April 17
\end{itemize}

\protect\hyperlink{item-ilaria-sommonte}{}

Nurse
\includegraphics{https://static01.graylady3jvrrxbe.onion/packages/flash/multimedia/ICONS/transparent.png}

I thought I was a weak person. Now I am discovering that I have power
and courage above all my expectations.

\begin{itemize}
\tightlist
\item
  Ilaria Sommonte
\item
  Naples, Italy
\item
  April 24
\end{itemize}

\protect\hyperlink{item-hillary-duenas}{}

Doctor
\includegraphics{https://static01.graylady3jvrrxbe.onion/packages/flash/multimedia/ICONS/transparent.png}

\begin{itemize}
\tightlist
\item
  Hillary Duenas
\item
  New York City
\item
  June 13
\end{itemize}

\protect\hyperlink{item-chan-pich}{}

Nurse
\includegraphics{https://static01.graylady3jvrrxbe.onion/packages/flash/multimedia/ICONS/transparent.png}

\begin{itemize}
\tightlist
\item
  Chan Pich
\item
  Greensburg, Pa.
\item
  June 9
\end{itemize}

\protect\hyperlink{item-patrick-dougherty}{}

Physician Assistant
\includegraphics{https://static01.graylady3jvrrxbe.onion/packages/flash/multimedia/ICONS/transparent.png}

\begin{itemize}
\tightlist
\item
  Patrick Dougherty
\item
  Mattawan, Mich.
\item
  April 17
\end{itemize}

\protect\hyperlink{item-zagidat-amayeva}{}

Doctor
\includegraphics{https://static01.graylady3jvrrxbe.onion/packages/flash/multimedia/ICONS/transparent.png}

\begin{itemize}
\tightlist
\item
  Zagidat Amayeva
\item
  Kizilyurt, Russia
\item
  Died on May 11
\end{itemize}

\protect\hyperlink{item-sheridan-brown}{}

Nurse Practitioner
\includegraphics{https://static01.graylady3jvrrxbe.onion/packages/flash/multimedia/ICONS/transparent.png}

\begin{itemize}
\tightlist
\item
  Sheridan Brown
\item
  Manhattan, N.Y.
\item
  April 17
\end{itemize}

\protect\hyperlink{item-jose-manuel-siurana}{}

Doctor
\includegraphics{https://static01.graylady3jvrrxbe.onion/packages/flash/multimedia/ICONS/transparent.png}

The first day was a shock. Seeing the numbers on the TV is not the same
as when you see the faces of the patients.

\begin{itemize}
\tightlist
\item
  José Manuel Siurana
\item
  Barcelona, Spain
\item
  April 18
\end{itemize}

\protect\hyperlink{item-chas-carlson}{}

Paramedic
\includegraphics{https://static01.graylady3jvrrxbe.onion/packages/flash/multimedia/ICONS/transparent.png}

\begin{itemize}
\tightlist
\item
  Chas Carlson
\item
  Philadelphia
\item
  April 18
\end{itemize}

\protect\hyperlink{item-madhavi-parekh}{}

Doctor
\includegraphics{https://static01.graylady3jvrrxbe.onion/packages/flash/multimedia/ICONS/transparent.png}

\begin{itemize}
\tightlist
\item
  Madhavi Parekh
\item
  Manhattan, N.Y.
\item
  April 18
\end{itemize}

\protect\hyperlink{item-nicanor-baltazar}{}

Nurse
\includegraphics{https://static01.graylady3jvrrxbe.onion/packages/flash/multimedia/ICONS/transparent.png}

\begin{itemize}
\tightlist
\item
  Nicanor Baltazar
\item
  Queens, N.Y.
\item
  Died on March 31
\end{itemize}

\protect\hyperlink{item-rita-macdonald}{}

Nurse
\includegraphics{https://static01.graylady3jvrrxbe.onion/packages/flash/multimedia/ICONS/transparent.png}

\begin{itemize}
\tightlist
\item
  Rita MacDonald
\item
  Royal Oak, Mich.
\item
  April 18
\end{itemize}

\protect\hyperlink{item-jose-satue}{}

Doctor
\includegraphics{https://static01.graylady3jvrrxbe.onion/packages/flash/multimedia/ICONS/transparent.png}

It's difficult to see friends and family and not be able to hug them,
because a kiss can be an act of risk rather than one of love.

\begin{itemize}
\tightlist
\item
  José Satué
\item
  Madrid
\item
  June 9
\end{itemize}

\protect\hyperlink{item-anna-condino}{}

Doctor
\includegraphics{https://static01.graylady3jvrrxbe.onion/packages/flash/multimedia/ICONS/transparent.png}

\begin{itemize}
\tightlist
\item
  Anna Condino
\item
  Seattle
\item
  April 16
\end{itemize}

\protect\hyperlink{item-diane-cusick}{}

Nurse
\includegraphics{https://static01.graylady3jvrrxbe.onion/packages/flash/multimedia/ICONS/transparent.png}

\begin{itemize}
\tightlist
\item
  Diane Cusick
\item
  Keyport, N.J.
\item
  April 18
\end{itemize}

\protect\hyperlink{item-jonathan-ilgen}{}

Doctor
\includegraphics{https://static01.graylady3jvrrxbe.onion/packages/flash/multimedia/ICONS/transparent.png}

\begin{itemize}
\tightlist
\item
  Jonathan Ilgen
\item
  Seattle
\item
  June 16
\end{itemize}

\protect\hyperlink{item-hussein-al-abbasi}{}

Doctor
\includegraphics{https://static01.graylady3jvrrxbe.onion/packages/flash/multimedia/ICONS/transparent.png}

\begin{itemize}
\tightlist
\item
  Hussein Al Abbasi
\item
  Wyong, Australia
\item
  April 24
\end{itemize}

\protect\hyperlink{item-taylor-laufer}{}

Nursing Aide
\includegraphics{https://static01.graylady3jvrrxbe.onion/packages/flash/multimedia/ICONS/transparent.png}

\begin{itemize}
\tightlist
\item
  Taylor Laufer
\item
  New York, N.Y.
\item
  April 18
\end{itemize}

\protect\hyperlink{item-elizabeth-pitre}{}

Nurse
\includegraphics{https://static01.graylady3jvrrxbe.onion/packages/flash/multimedia/ICONS/transparent.png}

\begin{itemize}
\tightlist
\item
  Elizabeth Pitre
\item
  Chicago
\item
  June 8
\end{itemize}

\protect\hyperlink{item-sewon-lee}{}

Nurse
\includegraphics{https://static01.graylady3jvrrxbe.onion/packages/flash/multimedia/ICONS/transparent.png}

One day, a patient noticed my ethnicity and said, ``Thank you for the
coronavirus.''

\begin{itemize}
\tightlist
\item
  Sewon Lee
\item
  Salina, Kan.
\item
  April 17
\end{itemize}

\protect\hyperlink{item-lauren-charlton}{}

Nurse
\includegraphics{https://static01.graylady3jvrrxbe.onion/packages/flash/multimedia/ICONS/transparent.png}

\begin{itemize}
\tightlist
\item
  Lauren Charlton
\item
  Summit, N.J.
\item
  April 8
\end{itemize}

\protect\hyperlink{item-dawn-zhao}{}

Doctor
\includegraphics{https://static01.graylady3jvrrxbe.onion/packages/flash/multimedia/ICONS/transparent.png}

\begin{itemize}
\tightlist
\item
  Dawn Zhao
\item
  New York, N.Y.
\item
  April 18
\end{itemize}

\protect\hyperlink{item-gabriele-somma}{}

Nurse
\includegraphics{https://static01.graylady3jvrrxbe.onion/packages/flash/multimedia/ICONS/transparent.png}

The patient's glance, with a silent eloquence, begged us not to send him
``downstairs.''

\begin{itemize}
\tightlist
\item
  Gabriele Somma
\item
  Naples, Italy
\item
  April 30
\end{itemize}

\protect\hyperlink{item-valerie-romo}{}

Nurse
\includegraphics{https://static01.graylady3jvrrxbe.onion/packages/flash/multimedia/ICONS/transparent.png}

\begin{itemize}
\tightlist
\item
  Valerie Romo
\item
  Palos Hills, Ill.
\item
  June 8
\end{itemize}

\protect\hyperlink{item-atif-sohail}{}

Doctor
\includegraphics{https://static01.graylady3jvrrxbe.onion/packages/flash/multimedia/ICONS/transparent.png}

\begin{itemize}
\tightlist
\item
  Atif Sohail
\item
  Arlington, Texas
\item
  April 18
\end{itemize}

\protect\hyperlink{item-nadina-abdurahmanovic}{}

Nurse
\includegraphics{https://static01.graylady3jvrrxbe.onion/packages/flash/multimedia/ICONS/transparent.png}

\begin{itemize}
\tightlist
\item
  Nadina Abdurahmanovic
\item
  Macomb, Mich.
\item
  April 19
\end{itemize}

\protect\hyperlink{item-rohan-singh}{}

Nurse
\includegraphics{https://static01.graylady3jvrrxbe.onion/packages/flash/multimedia/ICONS/transparent.png}

New York is still part of me, so I felt like I needed to help.

\begin{itemize}
\tightlist
\item
  Rohan Singh
\item
  Atlanta
\item
  April 29
\end{itemize}

\protect\hyperlink{item-annette-osher}{}

Doctor
\includegraphics{https://static01.graylady3jvrrxbe.onion/packages/flash/multimedia/ICONS/transparent.png}

\begin{itemize}
\tightlist
\item
  Annette Osher
\item
  Manhattan, N.Y.
\item
  April 22
\end{itemize}

\protect\hyperlink{item-hideo-yamanouchi}{}

Doctor
\includegraphics{https://static01.graylady3jvrrxbe.onion/packages/flash/multimedia/ICONS/transparent.png}

\begin{itemize}
\tightlist
\item
  Hideo Yamanouchi
\item
  Tokyo
\item
  May 4
\end{itemize}

\protect\hyperlink{item-manuel-penton-iii}{}

Doctor
\includegraphics{https://static01.graylady3jvrrxbe.onion/packages/flash/multimedia/ICONS/transparent.png}

\begin{itemize}
\tightlist
\item
  Manuel Penton III
\item
  Brooklyn, N.Y.
\item
  April 17
\end{itemize}

\protect\hyperlink{item-brian-lima}{}

Doctor
\includegraphics{https://static01.graylady3jvrrxbe.onion/packages/flash/multimedia/ICONS/transparent.png}

\begin{itemize}
\tightlist
\item
  Brian Lima
\item
  Roslyn, N.Y.
\item
  April 19
\end{itemize}

\protect\hyperlink{item-tawana-coates}{}

Doctor
\includegraphics{https://static01.graylady3jvrrxbe.onion/packages/flash/multimedia/ICONS/transparent.png}

I felt it was really important for us physicians to take a stand.

\begin{itemize}
\tightlist
\item
  Tawana Coates
\item
  Louisville, Ky.
\item
  June 14
\end{itemize}

\protect\hyperlink{item-gabriela-m-maradiaga-panayotti}{}

Doctor
\includegraphics{https://static01.graylady3jvrrxbe.onion/packages/flash/multimedia/ICONS/transparent.png}

\begin{itemize}
\tightlist
\item
  Gabriela M. Maradiaga Panayotti
\item
  Durham, N.C.
\item
  April 19
\end{itemize}

\protect\hyperlink{item-judit-amigo}{}

Doctor
\includegraphics{https://static01.graylady3jvrrxbe.onion/packages/flash/multimedia/ICONS/transparent.png}

\begin{itemize}
\tightlist
\item
  Judit Amigó
\item
  Barcelona, Spain
\item
  April 19
\end{itemize}

\protect\hyperlink{item-kimi-chan}{}

Doctor
\includegraphics{https://static01.graylady3jvrrxbe.onion/packages/flash/multimedia/ICONS/transparent.png}

\begin{itemize}
\tightlist
\item
  Kimi Chan
\item
  Manhattan, N.Y.
\item
  April 20
\end{itemize}

\protect\hyperlink{item-manish-garg}{}

Doctor
\includegraphics{https://static01.graylady3jvrrxbe.onion/packages/flash/multimedia/ICONS/transparent.png}

The way I view mental health has changed.

\begin{itemize}
\tightlist
\item
  Manish Garg
\item
  Manhattan, N.Y.
\item
  April 13
\end{itemize}

\protect\hyperlink{item-dania-sannino}{}

Doctor
\includegraphics{https://static01.graylady3jvrrxbe.onion/packages/flash/multimedia/ICONS/transparent.png}

As doctors, we still feel that it is not over at all.

\begin{itemize}
\tightlist
\item
  Dania Sannino
\item
  Naples, Italy
\item
  June 14
\end{itemize}

\protect\hyperlink{item-mitchell-elliott}{}

Doctor
\includegraphics{https://static01.graylady3jvrrxbe.onion/packages/flash/multimedia/ICONS/transparent.png}

\begin{itemize}
\tightlist
\item
  Mitchell Elliott
\item
  Louisville, Ky.
\item
  June 9
\end{itemize}

\protect\hyperlink{item-steven-miller}{}

Doctor
\includegraphics{https://static01.graylady3jvrrxbe.onion/packages/flash/multimedia/ICONS/transparent.png}

\begin{itemize}
\tightlist
\item
  Steven Miller
\item
  Brooklyn, N.Y.
\item
  April 20
\end{itemize}

\protect\hyperlink{item-jeffrey-oppenheim}{}

Doctor
\includegraphics{https://static01.graylady3jvrrxbe.onion/packages/flash/multimedia/ICONS/transparent.png}

\begin{itemize}
\tightlist
\item
  Jeffrey Oppenheim
\item
  Piermont, N.Y.
\item
  April 21
\end{itemize}

\protect\hyperlink{item-margit-anderegg}{}

Nurse
\includegraphics{https://static01.graylady3jvrrxbe.onion/packages/flash/multimedia/ICONS/transparent.png}

\begin{itemize}
\tightlist
\item
  Margit Anderegg
\item
  Manhattan, N.Y.
\item
  April 8
\end{itemize}

\protect\hyperlink{item-naomie-jean}{}

Doctor
\includegraphics{https://static01.graylady3jvrrxbe.onion/packages/flash/multimedia/ICONS/transparent.png}

\begin{itemize}
\tightlist
\item
  Naomie Jean
\item
  Mount Vernon, N.Y.
\item
  June 15
\end{itemize}

\protect\hyperlink{item-keleke-traore}{}

Nurse
\includegraphics{https://static01.graylady3jvrrxbe.onion/packages/flash/multimedia/ICONS/transparent.png}

\begin{itemize}
\tightlist
\item
  Keleke Traore
\item
  Koutiala, Mali
\item
  April 30
\end{itemize}

\protect\hyperlink{item-lynn-brown}{}

Nurse
\includegraphics{https://static01.graylady3jvrrxbe.onion/packages/flash/multimedia/ICONS/transparent.png}

I couldn't keep a six-foot distance. I took her in my arms.

\begin{itemize}
\tightlist
\item
  Lynn Brown
\item
  Snohomish, Wash.
\item
  April 9
\end{itemize}

\protect\hyperlink{item-gretel-honis}{}

Doctor
\includegraphics{https://static01.graylady3jvrrxbe.onion/packages/flash/multimedia/ICONS/transparent.png}

\begin{itemize}
\tightlist
\item
  Gretel Honis
\item
  Portland, Ore.
\item
  April 21
\end{itemize}

\protect\hyperlink{item-ilana-horowitz}{}

Social Worker
\includegraphics{https://static01.graylady3jvrrxbe.onion/packages/flash/multimedia/ICONS/transparent.png}

\begin{itemize}
\tightlist
\item
  Ilana Horowitz
\item
  Queens, N.Y.
\item
  April 21
\end{itemize}

\protect\hyperlink{item-christelle-nancy-diane-mike}{}

Doctor
\includegraphics{https://static01.graylady3jvrrxbe.onion/packages/flash/multimedia/ICONS/transparent.png}

\begin{itemize}
\tightlist
\item
  Christelle Nancy Diane Mike
\item
  Yaounde, Cameroon
\item
  April 30
\end{itemize}

\protect\hyperlink{item-tara-ardolino}{}

Nurse
\includegraphics{https://static01.graylady3jvrrxbe.onion/packages/flash/multimedia/ICONS/transparent.png}

\begin{itemize}
\tightlist
\item
  Tara Ardolino
\item
  Cranford, N.J.
\item
  April 22
\end{itemize}

\protect\hyperlink{item-eric-tyra}{}

Nurse
\includegraphics{https://static01.graylady3jvrrxbe.onion/packages/flash/multimedia/ICONS/transparent.png}

\begin{itemize}
\tightlist
\item
  Eric Tyra
\item
  Oakland, Calif.
\item
  April 22
\end{itemize}

\protect\hyperlink{item-yolanda-mozdzen}{}

Medical Assistant
\includegraphics{https://static01.graylady3jvrrxbe.onion/packages/flash/multimedia/ICONS/transparent.png}

\begin{itemize}
\tightlist
\item
  Yolanda Mozdzen
\item
  Staten Island, N.Y.
\item
  April 25
\end{itemize}

\protect\hyperlink{item-shawn-nishi}{}

Doctor
\includegraphics{https://static01.graylady3jvrrxbe.onion/packages/flash/multimedia/ICONS/transparent.png}

\begin{itemize}
\tightlist
\item
  Shawn Nishi
\item
  League City, Texas
\item
  April 17
\end{itemize}

\protect\hyperlink{item-sonia-compton}{}

Doctor
\includegraphics{https://static01.graylady3jvrrxbe.onion/packages/flash/multimedia/ICONS/transparent.png}

\begin{itemize}
\tightlist
\item
   Sonia Compton
\item
  Louisville, Ky.
\item
  June 11
\end{itemize}

\protect\hyperlink{item-alex-arreguin}{}

Respiratory Therapist
\includegraphics{https://static01.graylady3jvrrxbe.onion/packages/flash/multimedia/ICONS/transparent.png}

\begin{itemize}
\tightlist
\item
  Alex Arreguin
\item
  San Diego, Calif.
\item
  April 22
\end{itemize}

\protect\hyperlink{item-luis-eduardo-cunto-icaza}{}

Doctor
\includegraphics{https://static01.graylady3jvrrxbe.onion/packages/flash/multimedia/ICONS/transparent.png}

\begin{itemize}
\tightlist
\item
  Luis Eduardo Cuntó Icaza
\item
  Guayaquil, Ecuador
\item
  Died on March 28
\end{itemize}

\protect\hyperlink{item-maman-brah-ibrahim}{}

Infection Control
\includegraphics{https://static01.graylady3jvrrxbe.onion/packages/flash/multimedia/ICONS/transparent.png}

\begin{itemize}
\tightlist
\item
  Maman Brah Ibrahim
\item
  Maradi, Niger
\item
  April 30
\end{itemize}

\protect\hyperlink{item-maya-bunik}{}

Doctor
\includegraphics{https://static01.graylady3jvrrxbe.onion/packages/flash/multimedia/ICONS/transparent.png}

\begin{itemize}
\tightlist
\item
  Maya Bunik
\item
  Denver
\item
  June 9
\end{itemize}

\protect\hyperlink{item-francis-x-riedo}{}

Doctor
\includegraphics{https://static01.graylady3jvrrxbe.onion/packages/flash/multimedia/ICONS/transparent.png}

It dawns on you that your entire academic and professional life has
prepared you for this moment.

\begin{itemize}
\tightlist
\item
  Francis X. Riedo
\item
  Kirkland, Wash.
\item
  May 1
\end{itemize}

\protect\hyperlink{item-vanessa-gomez}{}

Nurse
\includegraphics{https://static01.graylady3jvrrxbe.onion/packages/flash/multimedia/ICONS/transparent.png}

The one thing I have learned is that our perseverance will prevail.

\begin{itemize}
\tightlist
\item
  Vanessa Gomez
\item
  Miami
\item
  April 29
\end{itemize}

\protect\hyperlink{item-jennifer-pierre-paul}{}

Nurse
\includegraphics{https://static01.graylady3jvrrxbe.onion/packages/flash/multimedia/ICONS/transparent.png}

\begin{itemize}
\tightlist
\item
  Jennifer Pierre Paul
\item
  Boston
\item
  April 8
\end{itemize}

\protect\hyperlink{item-kie-yamamoto}{}

Doctor
\includegraphics{https://static01.graylady3jvrrxbe.onion/packages/flash/multimedia/ICONS/transparent.png}

\begin{itemize}
\tightlist
\item
  Kie Yamamoto
\item
  Tokyo
\item
  May 4
\end{itemize}

\protect\hyperlink{item-denise-perry}{}

Nurse
\includegraphics{https://static01.graylady3jvrrxbe.onion/packages/flash/multimedia/ICONS/transparent.png}

\begin{itemize}
\tightlist
\item
  Denise Perry
\item
  Charlton, Mass.
\item
  April 9
\end{itemize}

\protect\hyperlink{item-maya-alexandri}{}

E.M.T.
\includegraphics{https://static01.graylady3jvrrxbe.onion/packages/flash/multimedia/ICONS/transparent.png}

\begin{itemize}
\tightlist
\item
  Maya Alexandri
\item
  Queens, N.Y.
\item
  May 1
\end{itemize}

\hypertarget{fighting-the-summer-surge-1}{%
\subsubsection{Fighting the Summer
Surge}\label{fighting-the-summer-surge-1}}

In May, many parts of the United States with low coronavirus infection
rates began to reopen. It was a gamble that often resulted in a flood of
cases, especially in the South and parts of the Southwest, where health
care workers are now battling outbreaks they hoped would never reach
them.

\hypertarget{updated-august-1-2020-2}{%
\subparagraph{Updated August 1, 2020}\label{updated-august-1-2020-2}}

\includegraphics{https://static01.graylady3jvrrxbe.onion/packages/flash/multimedia/ICONS/transparent.png}

\begin{itemize}
\tightlist
\item
  Michael Kennedy, General Surgeon
\item
  Savannah, Ga.
\item
  July 27
\end{itemize}

When the virus came, I volunteered in Brooklyn as soon as I could. I
went from being a surgeon to learning about pulmonary critical care.

The handful of shifts that I contributed to were truly humbling.

I moved to Savannah to start a job down here in the midst of a
resurgence. I am back on the general surgery service, so I am not as
involved with the critical care though we do get called for the Covid-19
patients that need tracheostomies.

A lot of people here have complained about masks and have the attitude
that this is more of a common cold. But others are frustrated that
officials down here aren't being more aggressive in protecting the
public.

Admissions and I.C.U. upgrades are increasing daily.

\includegraphics{https://static01.graylady3jvrrxbe.onion/packages/flash/multimedia/ICONS/transparent.png}

\begin{itemize}
\tightlist
\item
  Brittany Schilling, I.C.U. Nurse
\item
  Phoenix
\item
  July 9
\end{itemize}

Things have really ramped up in Phoenix. We're seeing large spikes in
cases and hospitalizations. We are tight on I.C.U. beds and need more
I.C.U. nurses to care for these patients. We are stretching these
ratios, picking up extra shifts and doing all we can to make it work.

We run from one thing to the next all day and go to bed to do it all
again the next day.

We've become each other's family, since many of us haven't seen our
actual families in months.

We see these patients, young and old, come in begging us to save their
lives. We do the best we can, but this terrible virus takes too many.

Our state opened too quickly with little planning in place. We saw large
groups of people together in high-risk activities.

We have seen patients who thought it was a hoax and quickly changed
their tune. I hope they'll help spread the truth.

\includegraphics{https://static01.graylady3jvrrxbe.onion/packages/flash/multimedia/ICONS/transparent.png}

Michael Starghill Jr. for The New York Times

\begin{itemize}
\tightlist
\item
  Pooja Pundhir, Hospitalist
\item
  Houston
\item
  July 7
\end{itemize}

Our hospitals are full. We're running out of respiratory equipment. Our
nurses and technicians are succumbing to the virus, and this is
exacerbating the scarcity of health care personnel when they're needed
the most.

Many of our patients are poor, Hispanic individuals, anywhere from 20 to
50 years old, who need I.C.U. support. Most are construction workers,
handymen, janitors and restaurant servers.

On a recent night, four of the patients with Covid-19 that required
hospitalization were in their 30s and 40s. They came in gasping for air
--- two men, two women; three Hispanic and one African-American.

Three were ``essential workers'' --- working at a fast-food chain, a
garbage disposal and construction. The other worked for a haughty boss
who wouldn't shut down a nonessential business. All worked part-time for
minimal wages.

Despite the gravity of their illnesses, all were anxious about how many
days of missed work and pay this illness would translate into.

The tears in their eyes reflected not only the fear of their lungs
drowning in the cytokine storm but also the struggle to keep fighting
the whirlpool of social inequity.

\includegraphics{https://static01.graylady3jvrrxbe.onion/packages/flash/multimedia/ICONS/transparent.png}

Andrea Morales for The New York Times

\begin{itemize}
\tightlist
\item
  Kimberly Brown, E.R. Doctor
\item
  Memphis
\item
  July 13
\end{itemize}

She was fighting back tears. I looked frantically for a box of tissues.
Gauze was the best that I had. I ripped open the package, apologized for
the rough gauze and lightly dabbed at her tears.

``My husband and son have to get tested,'' she said. ``They've been
taking care of me.''

I promised her that I'd call her husband. Her phone started ringing. I
asked her if she wanted to answer it. She said ``no,'' and just cried
harder.

``Do you want me to pray?'' She could only nod her head and cry. ``Dear
Heavenly Father,'' I began.

She was scared, and I was scared for her. There was nothing left that I
could do. It would be up to God to heal her.

Because she was a Black female with asthma and high blood pressure, the
odds were stacked against her. She was already on a mask for oxygen.

Her Covid-19 test returned as positive. She died a few days later.

Covid-19 has shaken me to my core. I feel like an intern again. My days
off have been filled with paralyzing anxiety.

I watched my colleagues in Detroit, New York and New Orleans wading
through seas of sick patients. I worried that Memphis would also be
overrun. I was scared that my majority Black city would see hundreds of
casualties, that I too would run an I.C.U. and the patients would look
just like me.

Our numbers are climbing right now. When I worked the other night, there
were no I.C.U. beds and we started holding patients in the E.R.

\includegraphics{https://static01.graylady3jvrrxbe.onion/packages/flash/multimedia/ICONS/transparent.png}

\begin{itemize}
\tightlist
\item
  Nicole Battaglioli, E.R. Doctor
\item
  Atlanta
\item
  July 7
\end{itemize}

That day's handoff felt different. The number of patients who were
positive for Covid-19 or were admitted to the hospital for respiratory
illnesses was far greater than numbers that I had seen in weeks. All of
our isolation rooms were full and patients were in the hallways
coughing.

My heart sank. On my way to work, I have seen crowds of Atlantans out in
public spaces, without masks and without concern for social distancing
practices. I have seen friends and acquaintances posting photos from
crowded beaches and large family gatherings on social media.

We started out in this pandemic as front-line heroes. Even though the
fanfare has gone, we will still be there at the other end of the social
gatherings and the disregard for social distancing.

While everyone is anxious to get back to their ``normal lives,'' there
is no end in sight for those of us in the emergency department. There is
no escape from the stress and mental fatigue that comes with witnessing
such levels of death and tragedy.

I remember telling a family that their sister had likely died due to
complications from coronavirus.

``We told her she should have come in sooner!'' her brother screamed.
``We wanted her to come in, but she just wanted to tough it out.''

In those moments, we feel failure weigh heavy on our shoulders. I
carried that grief home with me. I tried to distance myself from it as I
turned on the hot water in the shower.

I poured myself an oversize glass of wine, eager to attempt to numb and
push it down. This tactic never works, and I quickly felt burning in my
jaw and tightness in my chest. I gave in to the grief as the tears came
in big, ugly sobs.

\includegraphics{https://static01.graylady3jvrrxbe.onion/packages/flash/multimedia/ICONS/transparent.png}

\begin{itemize}
\tightlist
\item
  Aliiah V. Jourdain, E.R. Doctor
\item
  Moncure, N.C.
\item
  July 10
\end{itemize}

I work primarily with patients who served in the military. At this
point, Covid-19 has killed more Americans than World War I.

A large portion of my patients are people of color with comorbidities
that make them especially vulnerable. Cases are quickly rising. Forty
percent of the rapid Covid tests I ordered in my last shift were
positive, which is the most I've had in a single day so far.

The other day I was reviewing the medical record of a patient and
noticed the last in-person provider note was written just a few months
ago by a colleague who recently succumbed to Covid-19.

Understandably, patients and health care workers are pretty stressed out
right now. We wear masks, social distance and sanitize surfaces, but
beyond that, I'm trying to become intentional about taking a step back.

I try to make my patients laugh. They say, ``A merry heart does good
like a medicine.''

I encourage my patients to quit smoking and eat more plant-based foods.
A patient once replied to a mention of mangoes, saying, ``That's a
candy, right?'' That sparked a short jovial conversation which made him
smile behind his mask.

Sometimes I'll explicitly say, ``Your life matters. All I can do is
encourage you to make good decisions for that life.''

\includegraphics{https://static01.graylady3jvrrxbe.onion/packages/flash/multimedia/ICONS/transparent.png}

Andrea Morales for The New York Times

\begin{itemize}
\tightlist
\item
  John James Jr., I.C.U. Nurse
\item
  Memphis
\item
  July 13
\end{itemize}

In the beginning, we were slammed with patients and not enough resources
for patient care, just like New York. Our unit was overflowing and the
idea of using another department as an overflow area was discussed.
Lately, it seems that we are heading that way once again.

Regardless of what someone's belief is about the pandemic, it is here
and very real. Smaller cities or towns are not exempt.

Sadly, it seems a loved one has to contract the virus for people to
really understand how real and deadly this pandemic is.

I was caring for a patient recently who came in with a fever and
difficulty breathing. Once he received routine breathing treatments and
steroid medications, things were much better.

I came back to work the next day and the patient was on the ventilator
with his oxygen levels low and dropping. The next couple of days his
oxygen saturation was low with mucous constantly plugging his airway.

Sadly, he expired.

\includegraphics{https://static01.graylady3jvrrxbe.onion/packages/flash/multimedia/ICONS/transparent.png}

\begin{itemize}
\tightlist
\item
  Kimberly Darpoh, E.R. Nurse
\item
  Atlanta
\item
  July 13
\end{itemize}

It's been exhausting. The days have been longer and the volume of
patients has increased.

Being a new mom again, I make sure that I remain safe and that I don't
bring this virus home to my family. I'm always wearing my P.P.E. When I
clock out, I change out of my scrubs as an extra way to remain safe.

Through it all, I have still been able to find ways to smile and connect
with my patients. I was taking care of a Covid-positive patient, and he
thanked me over and over again for the service I was doing.

It feels good to know that you're appreciated during these tough times.
A simple thank you goes such a long way.

\includegraphics{https://static01.graylady3jvrrxbe.onion/packages/flash/multimedia/ICONS/transparent.png}

Kevin D. Liles for The New York Times

\begin{itemize}
\tightlist
\item
  Melhim Bou Alwan, Hospitalist
\item
  Atlanta
\item
  June 21
\end{itemize}

When I shook her hand and introduced myself, my patient nearly cried. No
one had wanted to touch her.

She was one of our first Covid-19 cases. She was also a schoolteacher.

For the first 10 days, I volunteered to be the sole hospital medicine
physician caring for our Covid patients because we had so few and it
made sense to limit the points of contact.

In those early days, I cared for two barbers, two pastors, an elementary
school teacher, a high school teacher --- a cross-section of our
community. It really brought home that this disease affects anyone and
everyone.

As health professionals, we're trained to live a not entirely normal
life. We expect to spend stretches of time away from our families. But I
never thought I'd be in a situation like this.

Those first 10 days when I was caring for our Covid patients, I was on
autopilot. I didn't have time to think about the magnitude of the
situation.

On the 11th day, I cried. I realized I might not get to go home to
Lebanon for a long time, to see my family.

Ensuring our front-line staff had an opportunity to express their
turbulent emotions became a main focus of mine. I have tried to make
sure I was there for my colleagues and teammates as they attempted to
make sense of what was happening. Oftentimes we couldn't. We would sit
in silence, vent or just cry together.

\includegraphics{https://static01.graylady3jvrrxbe.onion/packages/flash/multimedia/ICONS/transparent.png}

Hilary Swift for The New York Times

\begin{itemize}
\tightlist
\item
  Catrina Rugar, E.R. Nurse
\item
  Crystal River, Fla.
\item
  July 31
\end{itemize}

In the spring, I felt I needed to go where they were really struggling.
New York was heavy on my heart. My husband was scared, but we prayed
about it and prayed about it. We have six kids. Easter Sunday, something
just happened. My husband was like, ``If you feel led to go, we're going
to support you on this.''

I've always felt a calling to take care of people. But this was on such
a bigger level. It was a chance to be a part of history, and I wanted to
set a good example for my kids of how you behave in times of need. You
don't run away. You offer your help. I also felt like this would be my
opportunity to make a bit of money to go to nurse practitioner school.
That's ultimately what I want to do.

I was in New York for 33 days, and worked 26 days straight. I was home
for three weeks, then I went to Texas and served 22 days in a Covid
I.C.U. I was at a hospital about 5.5 miles from the border.

It was the worst thing I've experienced in nursing. It hit hard and it
hit fast and they weren't ready. The first week, in my section of the
Covid I.C.U., we lost every patient but one. The critical patients I was
working on were 55 and younger, some even in their 30s. I'm 34. Many
were first-generation Americans or immigrants who were uninsured and had
mismanaged health problems. To say the least, it was absolutely
heartbreaking.

I flew home to Florida on Monday. I needed a mental, physical,
spiritual, emotional vacation from this. But then I got a call for right
here in Florida. I went to my husband and said my heart was torn. How
can I have given myself to New York and Texas and not do it here?

As of yesterday, I'm working as a FEMA crisis nurse at a hospital in
Clearwater. In the next couple weeks, I start school to be a nurse
practitioner. Then I'll only work three days a week.

The silver lining in all this is that my kids and my husband have become
my heroes. I thought that we would sit around a table years from now and
the kids would look at me like the hero. But I'm realizing that they are
the heroes.

\includegraphics{https://static01.graylady3jvrrxbe.onion/packages/flash/multimedia/ICONS/transparent.png}

\begin{itemize}
\tightlist
\item
  Jay Quinn, I.C.U. Nurse
\item
  Clover, S.C.
\item
  July 13
\end{itemize}

I've seen what we all call ``Covid land'' from varying perspectives. I
spent nearly three months in New York at Elmhurst. Now, I'm here in
South Texas.

We have been overwhelmed here in McAllen. Patients will sit in the
emergency department holding for days before they are able to come to
the I.C.U. because we are so slammed.

If a patient codes, we will resuscitate them where they are with the
resources we have.

It's such a quick turnaround for patients that are in the I.C.U., to
passing away and to putting a new patient in their place.

By my 18th day, there had been more than 20 ``code blues'' and deaths.
The morgue has been overrun with patients who have passed.

I've been in more codes in a day than some people have in their careers.
Simultaneous codes happening is often the norm. I've walked off the unit
to debrief, ripped my mask off and let the tears stream down my face.

In Queens, I held phones while families FaceTimed loved ones who they
couldn't be with as they drew their last breath. I've read the lips of
my patient with a tracheostomy saying, ``I walked into the hospital, now
I can't feel my legs.''

These are things I'll never forget.

I've spent time during my nursing career in the Air Force. I've even
taken mass casualty training. But there is nothing that can prepare you
for this.

I fall asleep at night sometimes still hearing alarms.

\includegraphics{https://static01.graylady3jvrrxbe.onion/packages/flash/multimedia/ICONS/transparent.png}

\begin{itemize}
\tightlist
\item
  Damien Shields, E.R. Doctor
\item
  Houston
\item
  July 7
\end{itemize}

Our Covid-19 section of the E.R. was almost empty due to the shutdown.
Of course, as anticipated, all of that changed drastically and it has
come back stronger than ever.

Patient after patient, all shift long: fever, cough, weakness, shortness
of breath, hypoxia and multifocal pneumonia on chest imaging.

I find it quite disheartening. The sound of that horrible dry cough is
forever burned into my memory.

At times, I still wonder how we will ever get through this whole mess.

People are so resilient and powerful. I believe that if we could just
unify our country as one, with focus from the top down on getting
through this pandemic, we would be so much more successful.

But what has happened is the complete opposite. Division has separated
our country and allowed this virus to cause so much more damage.

When I have a moment to step back, it is unfathomable.

\includegraphics{https://static01.graylady3jvrrxbe.onion/packages/flash/multimedia/ICONS/transparent.png}

\begin{itemize}
\tightlist
\item
  Felino Taruc, Flight Nurse
\item
  El Paso
\item
  July 13
\end{itemize}

I work in the Four Corners area and in the Navajo Nation. I transport
critically ill patients with Covid-19 by helicopter or airplane from
rural areas to bigger cities with properly equipped hospitals to care
for them.

In these communities, most households are multigenerational, with
multiple families under one roof. My team transports one patient at a
time, but we have transported entire families by going back and forth.

Seeing whole families being torn apart from the virus is heartbreaking
and can take a toll on everyone mentally and physically.

I had an elderly man with low oxygen levels in his blood due to
Covid-19. He was able to speak but too weak to get up and move. He was
going to the same hospital where his wife was. She also had Covid-19,
was critically ill and had been transferred by another flight team.

On the morning of his transfer, the elderly man was also informed that
his son, who was in his mid-20s, had died from Covid-19 complications.
His son had three kids who were also being cared for by the elderly man
and his wife, but now they had no one to look after them.

I have had several co-workers fall ill from the virus as well. Being in
a small aircraft for multiple hours with a patient increases your
exposure. Our crews were overwhelmed with the volume of patients and
transfers. The intensity is increased by having proper P.P.E. on top of
a full flight suit, with 100-degree heat at 6,000 feet altitude.

Sometimes it feels like it will slow down but then a new surge will hit,
and we are overwhelmed again.

\includegraphics{https://static01.graylady3jvrrxbe.onion/packages/flash/multimedia/ICONS/transparent.png}

Veasey Conway for The New York Times

\begin{itemize}
\tightlist
\item
  Lillie Lodge, Cardiovascular Nurse
\item
  Raleigh, N.C.
\item
  July 7
\end{itemize}

I feel a little guilty that I took time off, that I went to see my
parents. But the reality is, I'm single. I live alone. I have been
isolated in ways I didn't know were possible since this all started. And
we really needed to see each other.

When I left work on that last day before my vacation, the Covid I.C.U.
was completely full and they were putting Covid patients elsewhere in
the hospital. I.C.U. nurses were staffing those beds.

I had a futile hope that when I came back from vacation today, the
numbers would be better, but it's still the same: completely full.

I didn't know the Covid I.C.U. nurses until I started volunteering with
them. They are courageous people. They did not ask to become the
designated I.C.U. It was assigned to them and they have risen to the
challenge. But the last time I walked over to say hello, they all looked
like they had been run over by a truck. They are exhausted.

In early May, it was easier for me to be optimistic, even in the midst
of all the devastation, because there was a hope that this would pass
quickly. Normal felt as if it were around the corner.

Work feels heavier now. That's the only way I can describe it. It feels
like you are walking through Jell-O.

How long can we keep this up? When will we reach the breaking point?
Because it's coming.

\includegraphics{https://static01.graylady3jvrrxbe.onion/packages/flash/multimedia/ICONS/transparent.png}

\begin{itemize}
\tightlist
\item
  Crystal Monjardin, Medical-Surgical Nurse
\item
  Yuma, Ariz.
\item
  July 14
\end{itemize}

I served in Operation Iraqi Freedom and now I serve my community during
this global war against Covid-19.

I'm a nurse from a small community where many didn't believe Covid-19
was real. Many still don't. Many have continued gathering in groups
without wearing masks. They believe that they are untouchable by such a
horror until that horror strikes them or someone they love.

We have had such a significant surge in cases that many call Arizona
``the new New York.''

When there is patient after patient and no more staff or anywhere else
to send them, we absorb them. Many also need more intense nursing care,
making our already difficult patient load more difficult.

We try to keep a positive mind-set. Many colleagues have decided to step
away from direct patient care, either to find a new job or to take a
step back from nursing, either for health reasons, stress, safety
concerns or a storm of all three. That is heartbreaking to see.

The weight of caring for people with Covid-19, who can quickly take a
turn for the worst, leaves one feeling anxious and defeated at times.
Some days I leave work in tears because it's just too much physically,
emotionally and mentally.

As nurses we give all of ourselves, comforting through the sadness,
reassuring through the anxiety, educating and, most of all, being
present for our patients to let them know they are not alone. We are
there fighting alongside them.

\includegraphics{https://static01.graylady3jvrrxbe.onion/packages/flash/multimedia/ICONS/transparent.png}

\begin{itemize}
\tightlist
\item
  Lydia Lopez, Pulmonary Care Nurse
\item
  Phoenix
\item
  July 16
\end{itemize}

I am the only Hispanic nurse on the unit that speaks Spanish. A patient
was so red and she could not breathe. She was terrified that no one
spoke her language.

Because of the patient-provider ratios and the language barrier, this
patient was neglected, fighting on her own to stay afloat until I rushed
into the room. She was started on a high-flow machine to help her
breathe. She was not afraid anymore and she did not feel alone.

The next day, the nurse taking care of her told me that the I.C.U.
needed that high-flow machine for their own patient.

How do you tell your patient that we have to take away the machine that
is helping her without knowing if we will get another one? How is
another life worth more?

She grabbed onto her nasal tube and said, ``No, No, No! Yo lo necesito
tambien!'' \emph{I need it too!}

I still check up on her. She is so wonderful and feels thankful to live,
but I think about that experience every day.

Most people on my floor will say I have grown since March. I have been a
shoulder to cry on and a person that listens. I have pushed myself
beyond my comfort zone to be there for my co-workers, to give hope and
positivity.

\includegraphics{https://static01.graylady3jvrrxbe.onion/packages/flash/multimedia/ICONS/transparent.png}

Tamir Kalifa for The New York Times

\begin{itemize}
\tightlist
\item
  Ana Laura Gonzalez De La Paz, I.C.U. Nurse
\item
  Austin, Texas
\item
  July 14
\end{itemize}

We don't feel like heroes. We feel tired. We feel anxious. We feel the
weight of our patients and families upon us.

Aside from being a nurse, I am also a DACA recipient.

I was taking care of two Spanish-speaking patients who were both on
breathing machines and unable to talk. The nurse before me said the
patients were anxious all day. I was expecting to have a pretty rough
night, but as I stepped into the room and spoke to them in Spanish, they
nodded calmly. I realized that they were just scared and needed to hear
a familiar language.

I called one of their families to give an update and my patient's
daughter answered. She sounded young. I could hear her mother telling
her what to ask in Spanish. The daughter began translating everything
and in that moment my heart was so warm.

I remember having to translate everything for my parents. I hated it,
and my sister and I would play rock, paper, scissors to see who would
have to do it. As an adult, though, it holds such a special place in my
heart knowing how much I would help them.

``I speak Spanish,'' I quickly said. ``I can talk to your mom if she has
any questions.''

Toward the end of the call, I could hear the mom in the background say,
``Dile gracias, muchas gracias.'' Her voice was breaking.

I was at my parents' house when I learned about the
\href{https://www.nytimes3xbfgragh.onion/2020/06/18/us/trump-daca-supreme-court.html}{Supreme
Court's decision} about DACA. I had been constantly refreshing my
screen. I'm about to start a doctorate program and in a way, the
decision would determine what my future would look like.

Once I saw the verdict, I jumped for joy and hugged my parents. I
thanked them for all of their sacrifices and enjoyed the moment.

\includegraphics{https://static01.graylady3jvrrxbe.onion/packages/flash/multimedia/ICONS/transparent.png}

\begin{itemize}
\tightlist
\item
  Tim Ellsberry, E.R. Nurse
\item
  Atlanta
\item
  July 16
\end{itemize}

Code Blue! His heart stopped! A man with Covid-19, who had just
presented to the E.R., was now in cardiac arrest. I jumped to action,
forgetting to don my own personal protective equipment.

Chest compressions. Check heart activity. Check pulses. Epi push.
Deliver shock. After a sweltering 45 minutes, we were finally able to
stabilize the patient.

I walked away with my head up feeling a sense of pride. I imagine that
this is what it would feel like if I were a superhero.

Then I hear, ``Oh, that guy was 80 years old. I am surprised he didn't
just kick the bucket,'' as if an 80-year-old man deserved less effort
and care for his life to be saved.

The patient was still fighting for his life 24 hours later. It was a
poor prognosis.

And that comment continued to bother me greatly. I wondered why.

Was it because the patient was a Black man and the comment came from my
white colleague?

Was it because I have an 83-year-old grandmother who is full of life and
love yet is most susceptible to this pandemic?

Was it because I am currently watching my grandmother mourn her sister
who lost her battle with Covid-19 recently?

It was all of it.

\includegraphics{https://static01.graylady3jvrrxbe.onion/packages/flash/multimedia/ICONS/transparent.png}

Tamir Kalifa for The New York Times

\begin{itemize}
\tightlist
\item
  Anna Maria Ruiz, E.R. Nurse
\item
  Austin, Texas
\item
  July 10
\end{itemize}

I graduated nursing school just last year. I was out of training maybe
three months before Covid-19 was declared a global pandemic. It's still
shaping me as a nurse.

I had a Covid patient who was younger than me and barely made it out of
the I.C.U. alive. That really hit me hard. He was in the I.C.U. for
months. Our entire staff followed his slow progression to health.
Thankfully, he made it out but seeing the young patients --- that's
what's intense.

Too many people think it's ``fake.'' Well, it's not. I had it. I've seen
it. I look it dead in the face every day. Every patient who has to leave
in a body bag shows us that this could cost us our lives if we don't
take it seriously.

The problem is not everyone sees it firsthand. If they saw how we're
running out of beds, supplies and medications, they'd think twice about
taking a chance. I worry about the at-risk members of my family. If they
happen to get sick, will the hospital they're at be equipped to care for
them adequately?

I feel unappreciated and unseen by both my government and my community.

Living in the South and being a Black nurse just adds on another layer
of stress and debilitation during this entire situation. I feel safer in
my scrubs than I do in my everyday street clothes.

But now with the racial tension, I feel that even as a nurse, I am seen
as ``one of them'' or a ``rioter,'' regardless of my attire.

\includegraphics{https://static01.graylady3jvrrxbe.onion/packages/flash/multimedia/ICONS/transparent.png}

\begin{itemize}
\tightlist
\item
  Tamar Sternfeld, Digestive Disease Nurse
\item
  Charleston, S.C.
\item
  July 15
\end{itemize}

Our state was completely locked down a few months ago. Now, people can
eat in restaurants again and pretty much function like ``normal.''

At the same time, our numbers of Covid-19 patients have grown
exponentially. We have run out of beds in our I.C.U.s and we have had to
open up additional units to treat our Covid patients, even placing them
in our children's hospital.

In May, the whole world seemed to be supporting us and sending lots of
love and good wishes. Now, the sacrifices have gotten greater. The
challenges we face are more difficult. Our patient loads are larger,
often meeting or exceeding the maximum number of patients we are each
allowed because staffing is so low.

While Covid-19 numbers are declining in many places, ours are
skyrocketing. If there was ever a time that we needed to feel supported,
it's now.

\includegraphics{https://static01.graylady3jvrrxbe.onion/packages/flash/multimedia/ICONS/transparent.png}

Eve Edelheit for The New York Times

\begin{itemize}
\tightlist
\item
  Zola Nlandu, Internal Medicine Doctor and Infectious Disease Fellow
\item
  Tampa, Fla.
\item
  July 10
\end{itemize}

Lately, we have been seeing a spike in Covid-19 cases here in Florida.
It has been a very physically and mentally taxing last couple of weeks,
to say the least.

When I am tired and feeling a bit at a loss in the midst of it all, I
remember a patient with a considerable degree of shortness of breath
telling me, ``Doc, just do whatever it takes to save my life.''

What struck me most were not the words uttered but the look of despair
written all over his face. That image will forever stay with me.

At the end of the day, I will never see myself as a hero. I am just
trying to do my job as best as I can for those in need in these trying
and uncertain times.

I could not have ever imagined that pursuing my passion would have led
me into the midst of a pandemic.

Every day, my wife says a word of prayer before I head off to work. She
always ends it by saying, ``And may no weapon formed against him
prosper.'' This helps calm my fear of contracting the virus or, even
worse, unknowingly transmitting it to my family.

\includegraphics{https://static01.graylady3jvrrxbe.onion/packages/flash/multimedia/ICONS/transparent.png}

\begin{itemize}
\tightlist
\item
  Jesse Hart, Pediatrician
\item
  Chapel Hill, N.C.
\item
  July 15
\end{itemize}

The other day, a child approached me in my clinic and held out her arms
to give me a hug. As she embraced me, I had to gently and carefully push
her back, redirecting her back to her mom. It felt horrible, foreign,
wrong.

I have had conversations with colleagues about writing letters to our
own children, getting wills in order and promising to not let each other
die alone in the I.C.U. if we contract Covid-19. This may seem dark and
extreme, but these are the intense emotions that have waxed and waned
over the past several months.

Covid entered my home. My husband, who has been working full-time from
home while taking care of our toddler and has left our house
approximately two times since this pandemic started, became infected
recently.

He had mild illness, but his diagnosis terrified me. I sobbed the entire
first day we knew. I had seen what can happen when Covid takes over,
even in young and healthy people.

Our son and I tested negative. Thankfully, my husband's illness remained
mild as he quarantined in our guest bedroom for 11 days with no physical
contact with us. However, even with mild illness, it is clear that his
recovery back to ``normal'' will be one that is long and slow.

\includegraphics{https://static01.graylady3jvrrxbe.onion/packages/flash/multimedia/ICONS/transparent.png}

\begin{itemize}
\tightlist
\item
  Erin Malone, I.C.U. Nurse
\item
  Miami
\item
  July 22
\end{itemize}

There has been a surge in the number of patients that we are seeing here
in Miami. The nurses are overwhelmed, the doctors are overwhelmed, and
resources are limited. Life has changed significantly.

I was one of the first nurses at my hospital to care for a Covid
patient. It was the very beginning and, despite the uncertainties and
fear I felt then, that was a piece of cake compared to when the numbers
started to increase and patients began to get sicker and die.

I've seen fevers that Tylenol couldn't break. Patients are dying every
day, sometimes two or three in one day.

I dress up every time I enter a patient's room, whether to turn off a
beeping IV pump or for a ``code blue.'' Doing this along with wearing an
N95 mask for 12 hours at a time, which causes the skin on one's nose to
break down, has made things difficult.

I have a young son. He's has been asking to see his friends and
expresses the desire to go back to school. He wants to go on the
playground, and I have to tell him no.

I feel helpless and scared of the long-term effects this could have on
his emotional and cognitive growth.

\includegraphics{https://static01.graylady3jvrrxbe.onion/packages/flash/multimedia/ICONS/transparent.png}

Allison V. Smith for The New York Times

\begin{itemize}
\tightlist
\item
  Courtney Ehioghae, Critical Care Nurse
\item
  Dallas
\item
  July 15
\end{itemize}

Working in the Covid I.C.U., where patients are proned, sedated,
ventilated and paralyzed with no families allowed to visit, means nurses
are everything to the patients. I update families and set up Zoom calls.

I hear and see the sadness when their loved ones take the last breath. I
hear and see the happiness when a patient gets extubated and can finally
breathe on their own.

We have to wear P.P.E. for 12 hours or more and don't get to complain if
we are sweating too much during a code or if the supplies are not what
we were used to.

It is sad going through so much to see a successful patient outcome and
then see people refuse to practice social distancing or wear a mask.
I've had a few arguments with people who believe in conspiracy theories
or that Covid-19 isn't real. It is real to all of us fighting to keep
patients alive just about every day of the week.

I have felt some hope, too. When a patient I had worked with for a long
time was finally able to mouth words and say my name, ``Courtney,'' I
felt courageous and so proud of the team I work with.

\includegraphics{https://static01.graylady3jvrrxbe.onion/packages/flash/multimedia/ICONS/transparent.png}

\begin{itemize}
\tightlist
\item
  Darshan Gandhi, Oncologist
\item
  Dallas
\item
  July 7
\end{itemize}

We were not prepared for the ``second surge'' that has brought Texas to
its knees. Our hospital's I.C.U. is almost at capacity and we have
converted many of the regular floors into ``Covid wards'' to create a
barrier.

We are beginning to cancel all elective procedures once again. We had
done that during the ``first surge'' and were just returning to
normalcy.

Many of our nursing staff tested positive for Covid-19 and had to be
quarantined, increasing the mental and financial burden on our employees
and their families.

My wife is a physician as well. We have been challenged with working out
shifts to babysit, especially as our nanny fell sick with Covid-19 and
had to go on an extended leave.

I lost two of my beloved and longtime patients to Covid-19. It was
utterly heart-wrenching.

\includegraphics{https://static01.graylady3jvrrxbe.onion/packages/flash/multimedia/ICONS/transparent.png}

\begin{itemize}
\tightlist
\item
  Jennifer Wurster, Family Nurse Practitioner
\item
  Tucson, Ariz.
\item
  July 9
\end{itemize}

I remember bragging to my elderly parents that I felt Tucson had been
spared from the virus since we had relatively low numbers in the
beginning of May.

Then our Governor reopened the state too soon. Young people flocked to
the bars in Phoenix and went rafting in big groups on the Salt River.
Temperatures climbed above 100 degrees and people went indoors more.
Then our numbers spiked.

I think people are finally starting to understand the gravity of this
virus. Tensions are high. Everyone is angry. Angry if they can't get
tested. Angry if the results take too long.

Our urgent care became a testing site for symptomatic Covid patients as
well as for asymptomatic essential workers. I have been averaging about
50 patients in a 13-hour shift for the last few weeks.

We still have nonbelievers who try to come into our urgent care, but we
kick them out if they refuse to wear a mask.

It only takes one personal connection to really hit home. My 22-year-old
niece from Indiana was in the hospital in March with Covid-19. She had
an oral temperature of 106 degrees and was on oxygen. She was all alone.
My sister-in-law would FaceTime her for hours just to keep her company.
It was terrifying.

\includegraphics{https://static01.graylady3jvrrxbe.onion/packages/flash/multimedia/ICONS/transparent.png}

Andrea Morales for The New York Times

\begin{itemize}
\tightlist
\item
  Anna Fong, E.R. Physician
\item
  Memphis
\item
  July 5
\end{itemize}

The first time I placed a patient with Covid-19 into the I.C.U. on life
support, I pondered what would become of humanity. He lay in the
hospital bed, struggling for every breath, battling the good fight.

Was what I was witnessing in that moment a glimpse of more to come --- a
singular event of humanity dying en masse?

``Do not be afraid,'' I told him. ``We will not leave your side. We just
need to place a tube to help your lungs breathe better. We will see you
when you wake up.''

I lied the good lie. Without hope, what is the purpose of life?

Our twice-a-day prayer over the hospital intercom broadcasts hourly now.
For those of whatever faith and those who claim no faith, we all want to
believe. We all want to believe that we will pull through this, if not
individually then as a whole.

I broke my promise to my patient in the I.C.U. Should I not have
promised something that is beyond what I can deliver?

In that last moment of wakefulness, he believed in hope in the midst of
despair.

He passed peacefully in his sleep.

\includegraphics{https://static01.graylady3jvrrxbe.onion/packages/flash/multimedia/ICONS/transparent.png}

Brandon Thibodeaux for The New York Times

\begin{itemize}
\tightlist
\item
  Tamika Bush, Pediatric Emergency Physician
\item
  Houston
\item
  July 13
\end{itemize}

We are getting flooded with patients coming in that are very sick to the
point where we are having to divert them because we are running out of
beds and space.

It is so bad that we are holding patients in the E.R. for days because
we have no rooms available on the inpatient floors.

If we feel the patients are simply too critical, we try to transfer them
to another hospital. Unfortunately, many of the hospitals in Houston are
also experiencing similar volume surges.

I am emotionally exhausted and physically drained. I have been doing
this since the pandemic started.

I still fear bringing things home to my child and family. I limit my
contact with them but it's so difficult because I miss them dearly.

When I am able to see them, I social distance and wear a mask just to be
safe. Behind the mask, I hold back tears and wonder if it will ever end
or if this is just our new normal.

\includegraphics{https://static01.graylady3jvrrxbe.onion/packages/flash/multimedia/ICONS/transparent.png}

\begin{itemize}
\tightlist
\item
  Jennifer Gillen, Medical-Surgical Nurse
\item
  Richmond, Texas
\item
  July 11
\end{itemize}

I remember the moment we were told we would become a dedicated Covid
unit. After that meeting, I snapped a selfie to document that moment.
When I look back on that image, I can see the fear, the panic and the
anxiety in my eyes.

We were in the dark. We were learning as we went.

It seemed like every shift we were changing something. It was terrifying
because we just didn't have answers.

Numbers in our area have been steadily increasing.

I would be lying if I said most days didn't feel daunting and full of
anxiety, but it is my calling. I have known my entire life that I wanted
to be a nurse. I wanted to be there, helping someone else when they
needed it most.

This journey has made me a stronger nurse.

\includegraphics{https://static01.graylady3jvrrxbe.onion/packages/flash/multimedia/ICONS/transparent.png}

\begin{itemize}
\tightlist
\item
  Andriana Love, Family Medicine Intern
\item
  Murrells Inlet, S.C.
\item
  July 21
\end{itemize}

This was my first rotation ever as a doctor, so it was kind of
interesting to start practicing during the pandemic. I hadn't learned
anything about Covid-19. We didn't go back to school from March onward.

Everything is really high stakes right now. We are getting several
patients a day. I think we'll be at max capacity pretty soon. We're
surprised we are not there now with the volume that we have. Some of the
surrounding hospitals aren't accepting transfers because they are at max
capacity, and we are pretty close to it.

I've seen a lot of sick people. People are getting better, too, but
there are sick ones that don't get better. The first patient I ever had,
one of the seniors ended up taking him, and he didn't make it. He passed
away in a week.

Initially, a lot of people here thought that young people weren't
getting it. But we had a 30-year-old who was really sick in the I.C.U.
It's like if it's not happening to them and their immediate circle, then
it doesn't exist. I went to Target the other day and nobody had a mask
on. I saw a bunch of people at a bar because it was Bike Week here, and
I was like, You guys don't understand what's really happening. I know
people are still going to the beach. It's pretty frustrating.

People are coming in sick every day, but something as simple as wearing
a mask could probably save a life. People just don't seem to care.

\includegraphics{https://static01.graylady3jvrrxbe.onion/packages/flash/multimedia/ICONS/transparent.png}

Kevin D. Liles for The New York Times

\begin{itemize}
\tightlist
\item
  Absar Mirza, Intensivist
\item
  Alpharetta, Ga.
\item
  June 22
\end{itemize}

It was the earliest days of the pandemic, when it still felt weird that
the streets and parking lots were empty, but our hospitals were quite
full. He was my first Covid-19 patient, and we had no idea what to
expect.

For an I.C.U. doctor, I'm usually pretty optimistic, but if I'm being
completely honest, for just half a second, as I prepared to enter his
room that first time, I thought, ``Something bad is going to happen.''

For me, connecting with my patients is essential. I need to see them as
people. But there were all these barriers between us: the mask, the
curtain and the limitation on how many times I could go into his room.
Everything felt different.

His smile became my bright spot, though. Through all the scary unknowns,
this man never assumed the role of being sick.

He would tell us, ``I'm feeling much better today,'' even though science
was telling us that his condition hadn't changed.

Medicine is partly science, partly art. You usually have some sense of
where you are heading. But not with Covid-19. Nothing seems to be
guaranteed. It works against logic sometimes --- a patient can show
signs of recovery and rapidly decline hours later.

The day we released him, I was ecstatic. Scientifically, I can't
quantify the connection between mind, body and optimism. But I will tell
you this, I believe it's what kept him going.

I believe it's what keeps all of us going.

\includegraphics{https://static01.graylady3jvrrxbe.onion/packages/flash/multimedia/ICONS/transparent.png}

\begin{itemize}
\tightlist
\item
  Sarah Berger, Nurse
\item
  McAllen, Texas
\item
  July 27
\end{itemize}

I work in a medium-size hospital. Our morgue is built to hold just two
bodies. The other night, I helped transport three deceased patients to
our morgue, which already had five bodies awaiting funeral home
arrangements.

We're breaking our backs to keep our heads above water.

We now have seven Covid-19 units and a designated holding area in the
E.R. where patients might be waiting up to a week for a bed upstairs.
Our free-standing emergency rooms have set up tents and trucks equipped
with oxygen for patients awaiting beds.

There's an unspoken understanding that's basically, A patient gets a
room when someone else dies.

We have several rooms with two patients together and little to no
privacy. Many patients are awake and alert. If one patient codes, we
can't protect the roommate from seeing everything. It's common procedure
for us but must be a nightmare for them.

Yesterday, my unit ran seven codes over two shifts. All of those
patients died. Their ages ranged from 21 to 66.

Recently, one patient was so traumatized by witnessing an unsuccessful
resuscitation attempt that his body seemed to stop fighting. He passed a
few days later.

A husband and wife recently passed together. As sad as it sounds, we
were grateful they could be near each other. Everyone else dies away
from their loved ones.

One patient lost her father who was in a room across the hall from hers.
She wasn't able to see him.

It's been difficult for me to come to terms with the fact that I have
absolutely no control over who recovers and who dies. I worry about my
patients constantly, which takes its toll. There's a Kevin Devine song I
listen to every day now on the drive to work that ends with, ``Don't
kill yourself to raise the dead. You'll only end up joining them.'' This
is the mind-set I've adopted to help me not break down.

\emph{Sarah's brother,}
\emph{\href{https://www.nytimes3xbfgragh.onion/interactive/2020/world/coronavirus-health-care-workers.html\#item-joe-berger}{Joe
Berger}, worked on the front lines in Chicago. They come from a family
of medical workers.}

\includegraphics{https://static01.graylady3jvrrxbe.onion/packages/flash/multimedia/ICONS/transparent.png}

\begin{itemize}
\tightlist
\item
  Jairo Hernando Barrantes Perez, I.C.U. Doctor and Pulmonologist
\item
  Houston
\item
  July 21
\end{itemize}

In the beginning, we were seeing more elderly patients with
comorbidities. Now we have younger people. They haven't been social
distancing.

They seem to bounce back better. Part of that is age, part of it is that
we understand more.

Now we see which medications are effective or have too many side
effects. We have a better understanding of what doesn't work. For
example, in the beginning everyone got hooked into a ventilator, but now
we know not everyone has to be hooked into one right away.

Our colleagues from Italy warned us upfront about clots. We are being
proactive in providing blood thinners to prevent them.

We need to have more experience on the long-term disabilities. There are
patients who develop severe inflammation and their muscles are
deteriorating in a matter of days. Recuperating takes rehab and good
nutrition. If you don't do that early, there is a good possibility you
will not regain muscle. I am concerned that after you have inflammation
in your lungs, you will have a scar. The lungs don't regenerate. Our
hope is that medications will prevent long-term scarring, but we don't
have enough information. It's too early.

\includegraphics{https://static01.graylady3jvrrxbe.onion/packages/flash/multimedia/ICONS/transparent.png}

Saul Martinez for The New York Times

\begin{itemize}
\tightlist
\item
  Camila Poblete, I.C.U. Nurse
\item
  Miami
\item
  July 22
\end{itemize}

Things are horribly worse compared to the first surge back in March.

Covid-19 cases are through the roof. My Covid I.C.U. has reached
capacity at 20 beds. Hospital administrators opened up two overflow
units that each hold 10 beds.

We are running low on drugs and oxygen. Nurses are overworked and
underpaid. We are still not receiving any hazard pay for our efforts. We
are extremely short-staffed with very critical patients. We feel
unappreciated by our superiors.

I was one patient's biggest advocate. I gave it my all to provide him
the best care every day. There were days he seemed better, and then days
he took three steps back. I knew deep down in my heart he was not going
to make it. I felt defeated and even angry.

It had been over a month since his family had seen him. They deserved to
see and speak to him in his final moments. As I FaceTimed his family, I
watched them cry, and my heart shattered. I tried my hardest to hold it
together but failed. We cried together over the phone as they expressed
their love and said their goodbyes.

I went home that night and realized that I do not want to waste my life
chasing money or love. I prefer to count the blessings God has gifted
me. I am in good health and have the ability to care for those who
cannot care for themselves.

\includegraphics{https://static01.graylady3jvrrxbe.onion/packages/flash/multimedia/ICONS/transparent.png}

\begin{itemize}
\tightlist
\item
  Erik Adler, E.R. Doctor
\item
  Golden, Colo.
\item
  July 29
\end{itemize}

I've taken care of hundreds of Covid patients. Some have been super
sick. Some, not so much. Sadly, a lot of the older ones have died.

We have definitely seen an uptick in cases recently.

I cared for a 22-year-old female patient yesterday, who came in after
having Covid-type symptoms for two weeks. She admitted to cough, fevers,
runny nose, muscle aches --- typical stuff.

I asked what was different today that made her go to the E.R. She told
me that her boss made her get checked out. I asked her where she worked.
She told me a local fast-food restaurant.

This woman had been serving food to people for two weeks despite these
symptoms!

She got tested and was positive. She was advised to avoid work and stay
home, but there is no way to enforce or regulate that.

When people wonder why we can't keep this under control, this is the
reason.

\includegraphics{https://static01.graylady3jvrrxbe.onion/packages/flash/multimedia/ICONS/transparent.png}

\begin{itemize}
\tightlist
\item
  Sophina Calderon, Family Physician
\item
  Tuba City, Ariz.
\item
  July 30
\end{itemize}

Before Covid-19 arrived in my hometown on the Navajo Nation, I was
working as a family doctor, treating patients of all ages. I worked in
the hospital that I was born in and alongside doctors who took care of
me and my relatives over the years. My patients were happy to see a
Navajo doctor who looked like them and understood their culture and
language.

Then many across the Navajo Nation attended a church rally that became
the Covid-19 superspreader event. We were a hidden hot zone.

Many families here live in multigenerational households, have poor
access to health facilities, no transportation, and many homes lack
running water or electricity. We suspected this had much to do with our
high rates.

My duties were quickly shifted to assist in the triage stations and
emergency room.

I saw familiar names among the list of positive tests and saw familiar
faces in the E.R.

I heard devastating stories from families who had multiple relatives
fall ill and as many as three subsequent deaths.

I saw my own relatives' names on those lists. I saw their faces in the
hallways and the emergency room.

I am saddened to see patients fall prisoner to their own anxiety and
depression, to see many lose what little income they had to feed their
grandchildren and pay their bills.

But our early response has been effective. We have seen our curve
flatten, and we watch the ongoing rise in the rest of Arizona.

There's an urgent call to reach back to traditional teachings of balance
and harmony, a concept known as ``hozhó.'' We collectively believe in
our own resilience. The history of our survival is proof to us that our
people will still come out strong.

\includegraphics{https://static01.graylady3jvrrxbe.onion/packages/flash/multimedia/ICONS/transparent.png}

Tamir Kalifa for The New York Times

\begin{itemize}
\tightlist
\item
  Caitlin Ortiz, E.R. Nurse
\item
  Austin, Texas
\item
  May 6
\end{itemize}

My co-workers and I have been hit or grabbed by patients while swabbing
them for Covid. It hurts when the swab goes in, but we have to do it
right so we can get accurate results. I have been threatened by patients
after swabbing them.

People have thrown their masks on the floor at my feet. They refuse to
wear the mask correctly.

We are seeing people coming in for anything from toe pain to a mild
headache who have the Covid-19 indicators, such as low oxygen levels or
high fevers. Even when those indicators are not there, many people test
positive when they get swabbed prior to surgery or when they're admitted
to the hospital for other reasons.

The thing that sticks with me the most is seeing people without their
families. I've seen little old ladies sick in the hospital who can't
have their husbands of 60 years at their bedside to comfort them and
hold their hands. So I try to take some extra time to do that for them.

\includegraphics{https://static01.graylady3jvrrxbe.onion/packages/flash/multimedia/ICONS/transparent.png}

Allison V. Smith for The New York Times

\begin{itemize}
\tightlist
\item
  Mary Catherine Keckeisen, Critical Care Nurse
\item
  Dallas
\item
  July 8
\end{itemize}

In Texas, people were ordered to stay at home for several weeks in April
and May. The number of patients in our coronavirus I.C.U. seemed to
plateau. Then, things evolved.

We have opened up several floors of the hospital as dedicated
coronavirus units. They have exploded with four times the number of
Covid patients. Our hospital had to rent five continuous dialysis
machines used to filter the blood of the sickest I.C.U. patients. We're
still running low on high-flow oxygen machines.

It's all hands on deck. If I have time, I like to clean any visible
spots on the floor. While this is typically the job of an environmental
service colleague, they are normally even busier than we are.

We prioritize video chats and allow families to grieve as much or as
little as they need. They remind me that this statistic is a human who
at some point had a first kiss, learned to drive, loved a child. I won't
ever forget hearing the sobs of these loved ones through a computer
monitor.

Post-mortem care can take anywhere from one hour to four. Several shifts
ago, I spent more time caring for the dead than the living. That was
new.

Some members of a community group crocheted tiny squares of yarn to
remind patients that they are being prayed for. I've seen several nurses
place one of these crocheted prayers in the body bag before zipping it
shut.

\hypertarget{around-the-world-1}{%
\subsubsection{Around the World}\label{around-the-world-1}}

Nurses, doctors and other health care workers reflect on fighting the
coronavirus.

\hypertarget{updated-august-1-2020-3}{%
\subparagraph{Updated August 1, 2020}\label{updated-august-1-2020-3}}

\includegraphics{https://static01.graylady3jvrrxbe.onion/packages/flash/multimedia/ICONS/transparent.png}

Gianni Cipriano for The New York Times

\begin{itemize}
\tightlist
\item
  Crescenzo Sala, Anesthesiologist
\item
  Naples, Italy
\item
  July 30
\end{itemize}

After so many months spent fighting this ``invisible'' war, my
colleagues and I are tired. We are holding on because the fear of a new
wave of infections is always high. People here have started living as if
nothing has happened, but in my hospital we are ready.

My fear is having to isolate myself again from my family. I have two
children, ages 4 and 2, and the thought of having to move away from them
again for who knows how long makes me feel very bad. But my job is to
help people, and that's what I'll do.

\includegraphics{https://static01.graylady3jvrrxbe.onion/packages/flash/multimedia/ICONS/transparent.png}

\begin{itemize}
\tightlist
\item
  Hina Ghory, E.R. Doctor
\item
  Princeton Junction, N.J.
\item
  May 1
\end{itemize}

It takes time to prepare for any intubation but even longer for a
Covid-19 patient because of all the P.P.E.

As I was getting ready, the patient went into cardiac arrest.

The patient was in one of our emergency department's smallest rooms. The
only light in the room was a single dull ceiling light.

I was unbelievably hot and dripping in sweat. My face shield fogged up
immediately and I could barely see as I began to intubate the patient.

It didn't help that the patient was being rocked continuously as my
colleagues did chest compressions to keep him alive.

Nurses and technicians rambled in and out of the patient's room.

``Keep the door closed!'' I yelled. ``I want only essential members of
the team in the room because this is a very high-risk procedure!''

My colleagues couldn't hear what I was saying through my P.P.E. It was a
disaster.

I successfully intubated the patient and his heart started beating
again. Unfortunately, he died later that night while up in the I.C.U.

\includegraphics{https://static01.graylady3jvrrxbe.onion/packages/flash/multimedia/ICONS/transparent.png}

Chloe Collyer for The New York Times

\begin{itemize}
\tightlist
\item
  Estell Williams, Surgeon
\item
  Seattle
\item
  June 14
\end{itemize}

We say that racism is a disease, because it is the underlying ideology
that leads to the chronic illness that we see within our communities.

George Floyd survived illness with Covid-19 just to die in the hands of
police. So even in the midst of a pandemic, I as a physician am treating
the diseases faced by Black Americans with my activism. And I had 10,000
colleagues who agreed, 100 percent wearing masks, so we can continue to
protect ourselves and our patients.

Breonna Taylor hit home for me because she was a young Black woman who
was an E.M.T. It felt personal. And then to have that followed up with
what happened with George Floyd, I was paralyzed. I sat for a while
trying to understand what could be done. Like many people, I felt like I
can't just sit still and do nothing.

As a medical community, we have a very powerful voice and we carry a lot
of power and privilege within our profession to say that this needs to
be looked at.

We cannot separate the disparities around police violence and racism and
the impact on mortality among Black Americans in the same way that we
cannot separate disparities of chronic illness in the Black community
and underlying hypertension, diabetes and heart disease and its impact
on mortality of Black Americans with Covid-19.

\includegraphics{https://static01.graylady3jvrrxbe.onion/packages/flash/multimedia/ICONS/transparent.png}

Meghan Dhaliwal for The New York Times

\begin{itemize}
\tightlist
\item
  Pablo Trujillo, I.C.U. Nurse
\item
  Mexico City
\item
  June 9
\end{itemize}

I was in bed 213 of the intensive care area and he was right in front of
me, in bed 210. I would ask my colleagues about his progress.

``It's not looking good,'' they would say.

One afternoon, I noticed his bed was empty. I started crying. I wanted
to help him so badly, but I couldn't do anything. I was fighting for my
own life. Edgar was a nurse, just like me. We had worked together for
years. He was my friend. He was always smiling.

I spent 12 days in the I.C.U., and during this time, seven other health
workers died. Only three of us made it. I don't understand why so many
of us are just slipping away.

I had tried really hard not to get infected. I even bought a snorkeling
mask, which I adapted with a ventilator filter for protection. At first,
I thought it was exhaustion from working two shifts where I basically
went 36 hours without sleep.

Lying in my hospital bed, I kept thinking about all the patients I had
treated just days before. I remembered the looks they had given me: eyes
of sorrow, pleading, anguish. I thought to myself, ``I must have the
same look on my face right now.'' But my eyes carried a double fear. I
was fully aware of what the virus was doing to my body and how likely it
was that I was going to die. At one point, the pain, the vomit, the
horrible headache, I couldn't take it anymore. I wanted to throw in the
towel.

Miraculously, soon after, I started getting better.I am now back at
work. I am still afraid, but I wanted to go back and help out, just like
they helped me.

\includegraphics{https://static01.graylady3jvrrxbe.onion/packages/flash/multimedia/ICONS/transparent.png}

\begin{itemize}
\tightlist
\item
  Everett Moss II, Emergency and Critical Care Nurse
\item
  Atlanta
\item
  May 5
\end{itemize}

The hardest thing for me during these Covid times has been seeing
patients die without the presence of family.

Having to call or FaceTime a patient's family is tough. It is hard to
explain and I can only imagine how hard it is for them to try to
understand and get closure when they can't be there by the side of their
loved ones.

\includegraphics{https://static01.graylady3jvrrxbe.onion/packages/flash/multimedia/ICONS/transparent.png}

\begin{itemize}
\tightlist
\item
  Louai Razzouk, Cardiologist
\item
  Manhattan, N.Y.
\item
  July 8
\end{itemize}

I volunteered to work in an I.C.U. ward when our hospital was getting
inundated by Covid patients. The crisis has made me appreciate how
interconnected specialties are within the medical field, how much we
depend on our nursing staff, respiratory technicians and janitorial
staff to keep these extremely ill patients alive, and how lucky we are
to work in a well-staffed hospital.

Covid admissions have definitely slowed down. I am back to taking care
of patients with heart disease and heart attacks, but we are bracing
ourselves for a second wave in the fall.

On a personal level, I was planning on getting married this month but
due to Covid we have had to postpone it to next June. We're hoping that
a vaccine will be available by then!

\includegraphics{https://static01.graylady3jvrrxbe.onion/packages/flash/multimedia/ICONS/transparent.png}

Alessandro Falco for The New York Times

\begin{itemize}
\tightlist
\item
  Vanda Ortega Witoto, Nursing Technician
\item
  Parque das Tribos, Manaus, Brazil
\item
  June 29
\end{itemize}

At first, it was like we did not really have the virus, only light
symptoms: fever and cough, loss of smell. But then on April 17, a member
of the Tuyuka ethnic group called me, worried for a woman who had been
lying in the hammock for more than seven days with a fever, cough, chest
pain and difficulty breathing.

In that moment, I understood that the virus had arrived in our
community.

I had 10 patients with severe breathing difficulties. They were
transferred to the hospital. But many in our community have been afraid
to go there.

``They will kill me in the hospital,'' said one man who had severe
symptoms for 15 days. He was very afraid of what he saw on the news,
that the people who went to the hospital did not return. He preferred to
die at home. I persuaded him to go to the hospital. He spent 21 days
there and is better now.

In our Indigenous communities, health conditions are precarious. We had
no money to buy medicines, so I asked friends and professors for help. I
didn't have any kind of protective gear, so I used a cloth mask until
people saw my working conditions and gave me material to protect me.

Our families mostly live on crafts and informal jobs, domestic workers
or bricklayer assistants. Our first problem, a week after the lockdown
decree, was food. So I started a campaign on social networks to collect
food donations.

We started producing informative videos. We were advising to practice
hand hygiene, but they didn't have water. The health issue goes into the
background when the priority is to be able to drink. This hit me.

Many people want everything to go back to ``normal.'' I don't want to go
back to what is ``normal'' here for Indigenous lives.

\includegraphics{https://static01.graylady3jvrrxbe.onion/packages/flash/multimedia/ICONS/transparent.png}

\begin{itemize}
\tightlist
\item
  Annalisa Malara, I.C.U. Doctor
\item
  Codogno, Italy
\item
  May 5
\end{itemize}

On Feb. 20, I was called to evaluate a 37-year-old man, healthy and
athletic, with very serious pneumonia on both lungs. His blood analyses
and CT scan showed that he probably had viral pneumonia, and his
condition had deteriorated quickly. The coronavirus was still a very
remote idea. We were aware of it, but we all saw it as something foreign
from us. We gave the patient, later known as ``Patient 1,'' the
traditional pneumonia treatments, which he did not respond to. When we
intubated him, his wife came to me and told me about a dinner her
husband had attended, where one of the participants had recently
returned from China.

We checked a protocol we had received in January: patients were
considered ``at risk of coronavirus'' only if they had respiratory
problems coming from China, or if they had been in contact with a
positive patient. That was not our patient's case, but we decided to
swab him anyway. Thanks to that, we understood that the virus was in
Europe.

After his result came back positive --- that same night --- we
hospitalized another young man who was also positive for coronavirus.
The following morning, another one came. Once we knew there was one
positive patient, we saw the others differently. The diagnoses became
very easy.

In the following days, an unimaginable quantity of patients started
coming in and we were literally overwhelmed.

Dr. Malara diagnosed Italy's first case of the coronavirus outbreak.

\includegraphics{https://static01.graylady3jvrrxbe.onion/packages/flash/multimedia/ICONS/transparent.png}

\begin{itemize}
\tightlist
\item
  Kim Hyeon-hee, Nurse
\item
  Daegu, South Korea
\item
  April 28
\end{itemize}

I was looking after regular inpatients at my hospital when I was
dispatched to another hospital that was emptying to prepare for Covid-19
patients. Nothing was ready. We were discussing ways to prepare for
patients but by noon, two patients already came in.

Soon 50, 60 and up to 80 patients were arriving in a day. Unimaginable,
no? Ambulances carrying patients were surrounding the hospital building.

The condition of some patients was suddenly getting worse. Family
members weren't allowed, even when things got bad. We nurses cried, it
was breaking our hearts. I was unable to console the families as I
normally would.

We worked about 12 to 14 hours a day. It felt very, very hard. I felt a
sense of dread. I asked myself, ``Is this something I can handle?'' I
felt scared, but there was no running away. It was what I had to do. My
colleagues all felt the same. I feel a strong bond with them now. The
sense of camaraderie is like that of a war veteran.

I felt so bad for my family. At first, they were against me being
dispatched to the hospital, but later they supported me. My daughter
said, ``Mom, you are amazing. My friends think so, too.'' When I came
home late one night, I cried in front of my husband and mother and said,
``I am grateful and I am sorry.'' I wouldn't have been able to hold on
without their support.

\includegraphics{https://static01.graylady3jvrrxbe.onion/packages/flash/multimedia/ICONS/transparent.png}

\begin{itemize}
\tightlist
\item
  Arezoo Habibi, Pathologist
\item
  Herat City, Afghanistan
\item
  April 25
\end{itemize}

Last year, I graduated from Herat University's medical program. I love
this job, and when the virus spread in Herat I volunteered to work at a
Covid-19 treatment center. When I told my family, everyone was against
me. They said that I was risking their lives, but I insisted and took
the job. Now they support me. But other relatives don't want to talk to
me.

It is my responsibility to take care of patients so they stay strong to
fight the virus. I don't just take care of their physical health; they
need to be supported mentally, because people think the virus is a
punishment from God. We console them and tell them it is a global
pandemic. It doesn't necessarily kill everyone.

We lost a few patients, and those were the hardest moments of my life.
The patients were in front of my eyes and I couldn't do anything.

I don't want to lose anyone else. I want to work harder. I think I am
stronger and kinder than I was.

\includegraphics{https://static01.graylady3jvrrxbe.onion/packages/flash/multimedia/ICONS/transparent.png}

Heather Sten for The New York Times

\begin{itemize}
\tightlist
\item
  Thomas Lo, Anesthesiologist
\item
  Queens, N.Y.
\item
  April 24
\end{itemize}

On 9/11, I was working at Morgan Stanley on the 73rd floor of the World
Trade Center. It was actually my birthday. When the second plane hit my
building, I felt the entire building shift and sway. I thought we were
going to topple over. I thought, ``I'm not going to survive this.''

An event like that gives you a different perspective on life. A few
years later I realized that I was meant to be a doctor and make my life
count by helping others.

Now we have another war in New York and in the world with Covid. I'm an
anesthesiologist, so I'm intubating patients who are no longer able to
breathe on their own. I'm right by the patient's airway, often inches
away, as I place the breathing tube. With the respirator, I feel safe.

It's my duty. Every patient I intubate who is able to survive Covid is a
life that is saved.

\includegraphics{https://static01.graylady3jvrrxbe.onion/packages/flash/multimedia/ICONS/transparent.png}

Hilary Swift for The New York Times

\begin{itemize}
\tightlist
\item
  Taylor Olson, N.I.C.U. Nurse
\item
  Naples, Fla.
\item
  April 30
\end{itemize}

Right now I'm at a hospital in the Bronx, staying at a hotel in Times
Square. I actually had to quit to come here. My hospital wasn't allowing
leaves and it was just something that I really wanted to do. So I
weighed my options and decided that this was a once-in-a-lifetime
opportunity.

If it were my hometown or state, I would want people to come help. We've
had plenty of hurricanes in the past, and so many people have dropped
everything and come to our state to help out. I just felt like I could
return the favor.

I'm actually a N.I.C.U. nurse, so this is way different for me. I have
big people. I know how to run a ventilator on a premature baby, but to
run a ventilator on an adult, the settings are different and the drips
are different. But I feel like every day I am becoming a better nurse
because I'm learning something new every single day that I come to work.

There are four of us nurses in the unit, all from different states.
After the first 12 hours, we worked so well together. Was it
overwhelming? Absolutely. I've never felt this level of exhaustion
before. But it's so nice to have people that I can lean on and count on.
Even when we're not at work, we'll text and be like, ``Hey, how are you
feeling, Hey, do you need anything? Can I get you anything?'' It's a
really strong support system.

Originally I was here for three weeks. I chose to stay for another
eight. And then we'll see if I'm still needed. I want to be here as long
as I can. I want to heal and touch every patient that I can before I
have to go home to my puppy in Florida.

\includegraphics{https://static01.graylady3jvrrxbe.onion/packages/flash/multimedia/ICONS/transparent.png}

\begin{itemize}
\tightlist
\item
  Brooke Spence, Critical Care Nurse
\item
  Las Vegas
\item
  April 10
\end{itemize}

My patients are watching the news and seeing how many people are dying
around the world, knowing they are positive. They ask me if they are
going to die. I tell them that we are all trying our absolute best to
get them back home to their loved ones.

I chose a career in health care --- specifically, critical care ---
because I wanted to help others in their most vulnerable and dire time
of need. I put my faith in God and take precautions by properly donning,
removing and disinfecting my personal protective equipment. But in the
back of my mind, I can't help but wonder how much the risk of infection
increases from constant exposure.

\includegraphics{https://static01.graylady3jvrrxbe.onion/packages/flash/multimedia/ICONS/transparent.png}

Angela Ponce for The New York Times

\begin{itemize}
\tightlist
\item
  Jesús Valverde, I.C.U. Doctor
\item
  Lima, Peru
\item
  June 24
\end{itemize}

We started off so ill-equipped for this tragedy. We are roughly 700
intensive care physicians for a country of over 32 million. We're way
below international standards.

Every I.C.U. bed in Lima is taken. Dozens of patients are waiting to be
assigned to one. We have to choose who will be saved and who won't. It's
terrible.

I've almost doubled my work shifts. Most of the week, I live either at
the hospital or the clinic. I'm only at home for a couple of days at a
time.

We're exhausted, but we draw strength to go on from wherever we can.

When someone makes it out of the I.C.U., we feel immense joy. That
patient is a survivor.

\includegraphics{https://static01.graylady3jvrrxbe.onion/packages/flash/multimedia/ICONS/transparent.png}

\begin{itemize}
\tightlist
\item
  Kirk Tapia-Bobst, E.R. Nurse
\item
  Chicago
\item
  July 10
\end{itemize}

Chicago has done well in flattening the curve, but many health-care
professionals around the city are concerned that we are going to have an
even bigger second wave either over the summer or during the next flu
season.

The level of stress has been sustaining at elevated levels. My elderly
father has been hospitalized several times, although luckily not with
Covid. It has been distressing for me to help my mother manage his
condition over the phone, since it is too risky for me to come home.

My husband is back to work after over two months of being unemployed but
it adds increased stress because I try to be available to drive him to
work. We avoid as much public transport as possible --- a feat that is
not easy when I work night shifts and he works evenings.

But I am one of the lucky ones. My medical center has been well
prepared, with an emergency department leadership team that is
proactive. After three months, I am still Covid- and antibody-free.

\includegraphics{https://static01.graylady3jvrrxbe.onion/packages/flash/multimedia/ICONS/transparent.png}

\begin{itemize}
\tightlist
\item
  Micaela Sager, Physician Assistant
\item
  Point Pleasant, N.J.
\item
  June 6
\end{itemize}

In May, when I was in my third trimester of my second pregnancy, many
people questioned why I was still working. My answer was, ``It is my
job, and it has been an honor and a privilege to serve my community and
do my part in this pandemic.''

Every morning, I would wake and say a prayer, not only for all my
patients and colleagues but for the safety of myself and my unborn
child.

I stopped seeing the critical patients as I tried to avoid exposure.
However, I found that inevitably every patient is at risk for Covid-19.

I have never been more grateful for P.P.E. I took precautions with every
patient, including the case of ``simple back pain.''

I had to wear a mask during my own delivery, which was actually worse
than wearing one on a 12-hour day in the E.R. The labor and delivery
ward staff did their best to ease my worries and make my experience as
normal as possible, but there were constant temperature checks for my
husband and me, and we were not able to leave our room.

There were no visitors, and no newborn photo shoots, but at least I was
able to have my husband by my side.

Now, my newborn daughter's health is my No. 1 concern.

\includegraphics{https://static01.graylady3jvrrxbe.onion/packages/flash/multimedia/ICONS/transparent.png}

Bryan Anselm for The New York Times

\begin{itemize}
\tightlist
\item
  Kristina Woo, Clinical Care Technician and E.M.T.
\item
  Watchung, N.J.
\item
  April 9
\end{itemize}

Both of my jobs require lots of patient contact. I am also
immunocompromised with Type 1 diabetes.

The worst experience was taking care of a patient who suffered from
Parkinson's disease and was positive for the coronavirus. He had a high
fever and wasn't eating. I was trying to feed him but he just kept
begging to die.

Whether it's cleaning bathrooms covered in explosive diarrhea or
cleaning up the patients, I continue to show up and help.

I work per diem and I make \$15 an hour. I don't get benefits and I
don't get hazard pay. I risk my safety and my family's safety because in
times of chaos someone needs to remain calm and show up.

Since the outbreak began, my responsibilities have remained the same.
But the stakes are higher. The risk is greater.

When life gets scary, what's the best option? Quit and hope you're safe?
Or step up and help those who need it?

For me there is no choice.

\includegraphics{https://static01.graylady3jvrrxbe.onion/packages/flash/multimedia/ICONS/transparent.png}

Gianni Cipriano for The New York Times

\begin{itemize}
\tightlist
\item
  Maria Notaro, Anesthesiologist and I.C.U. Doctor
\item
  Naples, Italy
\item
  June 15
\end{itemize}

The pandemic has been like war.

If I make a mistake, there is a high risk of unknowingly infecting my
colleague, my friend, my family, my patient or a stranger.

Everything ends. Covid-19 will too, but I will still feel responsible
for them.

I learned that teamwork is the only way to build something good.
Building is always tiring, but it is the most beautiful challenge.

\includegraphics{https://static01.graylady3jvrrxbe.onion/packages/flash/multimedia/ICONS/transparent.png}

\begin{itemize}
\tightlist
\item
  KP Mendoza, I.C.U. Nurse
\item
  Manhattan, N.Y.
\item
  June 9
\end{itemize}

While someone like me --- a nurse --- should be against protests in
order to contain the virus, I also recognize the importance of being an
ally, of participating in this movement, now more than ever. It is a
privilege to be able to walk alongside protesters.

One of the most chilling experiences I had was on Breonna Taylor's
birthday. We marched from Washington Square Park and made a brief stop
at the steps across from Penn Station. There, we sang Breonna Taylor a
happy birthday.

I carried a sign that read, ``I did not treat Covid patients only to
then silently stand by while Black lives are murdered by police.''

I wanted people to know they could tap my skills as an I.C.U. nurse, and
also that I stood with them.

In my backpack, I carried a myriad of medical supplies: saline flushes,
xeroform, three different kinds of tape, goggles to protect against
pepper spray and other items to help with immediate first aid needs.

What has really mattered to me is refuting the idea that one has to
choose between pandemic protective measures or protesting police
brutality. I don't think those two issues are mutually exclusive.

While Covid-19 is a pandemic and a public health threat, so too is
police brutality.

\includegraphics{https://static01.graylady3jvrrxbe.onion/packages/flash/multimedia/ICONS/transparent.png}

\begin{itemize}
\tightlist
\item
  Ryan Lee, Dental Oncologist
\item
  Canton, Mass.
\item
  April 20
\end{itemize}

I'm a major in the Massachusetts Army National Guard, leading mobile
Covid-19 strike teams in nursing homes, mental health facilities,
veterans' homeless shelters and memory care centers all across the
state.

Yesterday, during a shift change, a medic soldier in a Tyvek suit,
drenched in sweat, came out and informed me that we had ``two K.I.A.s in
there, sir.''

``What?'' I thought. ``We killed someone in there?''

Of course, what he meant to say was ``D.O.A.,'' or two elderly residents
who had passed away before we even had the opportunity to test them for
the coronavirus.

No more than 48 hours had passed since we were asked to swab test these
72 residents.Upon arrival, we were left with only 70.

Nothing prepares you for this. Not only is this virus deadly, it is
silent, insidious and sometimes faster than our best efforts.

\includegraphics{https://static01.graylady3jvrrxbe.onion/packages/flash/multimedia/ICONS/transparent.png}

\begin{itemize}
\tightlist
\item
  Federica Brena, Oncologist
\item
  Bergamo, Italy
\item
  May 5
\end{itemize}

I was deployed to the Covid ward as soon as the number of cases spiked
in my city, Bergamo. I started doing 12-hour shifts, taking one patient
after the other, wide-eyed and begging for good news. I was scared I was
going to do something stupid, because let's be honest, an oncologist
doesn't know much about infectious diseases.

I was also scared I would catch the disease and bring it to my family.

I started having symptoms and diagnosed myself with coronavirus. Four
days later, my 1-year-old son came down with a fever and a cough. I
started panicking: How can I take him to the hospital? What do I do if
he struggles to breathe? I had all these questions but I had to stay
calm. A doctor must think rationally. A few days later, my husband
started showing symptoms.

I spent the nights trying to visualize the moment during my three weeks
in the Covid ward when I caught the disease. I felt guilty that I was
the one who brought this virus home, I was my family's infector. My
husband's conditions deteriorated and with my mind full of images I had
seen at the hospital, I feared the worst. Despite my weakness and
temperature, I cared for him at home, measuring his oxygen saturation
and temperature. Slowly, our conditions improved.

On April 27, I returned to work. But I am never going to be the same
doctor I was before. This situation changed us all in an irreversible
way.

\includegraphics{https://static01.graylady3jvrrxbe.onion/packages/flash/multimedia/ICONS/transparent.png}

\begin{itemize}
\tightlist
\item
  Josline Azki, P.I.C.U. Nurse
\item
  Baalbek, Lebanon
\item
  June 3
\end{itemize}

Since the beginning of the pandemic, we have taken all the necessary
measures to keep our activities running as usual, but I cannot deny that
it is challenging.

Medical staff are at higher risk of being infected. We are working
longer shifts and we are staying away from our families as much as
possible to protect our loved ones.

What scares me the most is that if a large increase in Covid-19 patients
happens at once, Lebanon's health care system is at risk of becoming
completely overwhelmed, and then not everyone would get to receive
proper hospitalization.

\includegraphics{https://static01.graylady3jvrrxbe.onion/packages/flash/multimedia/ICONS/transparent.png}

Glenn Arcos for The New York Times

\begin{itemize}
\tightlist
\item
  Charline Kass, I.C.U. Doctor
\item
  Antofagasta, Chile
\item
  June 9
\end{itemize}

It's easy to feel abandoned up here in the desert. I was transferred
here a year ago. All of my family lives elsewhere, and the solitude can
be difficult. Working in a hospital is all about teamwork and
camaraderie, but this has been isolating for the staff. We even have to
sit far apart in the cafeteria because if we get ill, there's nobody
else.

Spirits are low, and that's hard to see among such a great group of
people. I try to find ways to lift morale.

When I heard about the E.R. doctor in New York who killed herself, I was
shaken. Lots of people got in touch to ask if I was OK.

I am, but when I'm on my own at home, the sadness starts to creep over
me. It can be overwhelming. I can't even go out at night to take my mind
off the situation because there are curfews in place and everywhere is
closed.

The virus is the first thing I think about when I wake up and the last
thing on my mind when I get into bed, exhausted. This is day after day
after day.

I'm not the same person I was a few months ago.

\includegraphics{https://static01.graylady3jvrrxbe.onion/packages/flash/multimedia/ICONS/transparent.png}

Bethany Mollenkof for The New York Times

\begin{itemize}
\tightlist
\item
  Adam J. Milam, Anesthesiologist
\item
  Los Angeles
\item
  June 11
\end{itemize}

I witness death often, and I have for more than 15 years, but I still
have not watched the videos of the deaths of Ahmaud Arbery or George
Floyd. I cannot bear that pain and I feel like I don't have time to
fully process it in the midst of the pandemic. It has been tough for me,
as a physician, as a public health professional and as a Black man.

Both the pandemic and the extrajudicial killings of African-Americans
clearly show that Black lives are not valued.

Working at the hospital and volunteering at a homeless shelter, I have
seen how this pandemic has disproportionately affected communities of
color. Access to testing has been a huge issue in our community during
this pandemic. Minorities and poor people throughout the country were
being turned away for testing, and consequently these populations
presented with more severe symptoms. This has a snowball effect;
combined with the problems of access, mistrust of the health care system
by communities of color and implicit biases of health care providers, it
is a recipe for disaster.

I participated in one of the marches on Beverly Boulevard. It was an
amazing experience. There were people of all ages, races and
ethnicities. People were walking, biking and sitting in cars. Everyone
was peaceful and the signs provided messages that the recent events were
not acceptable.

We cannot sit idly by while Black people die at alarming rates.

\includegraphics{https://static01.graylady3jvrrxbe.onion/packages/flash/multimedia/ICONS/transparent.png}

Marya Sherron, via Associated Press

\begin{itemize}
\tightlist
\item
  Kious Kelly, Nurse
\item
  New York City
\item
  Died on March 24
\end{itemize}

``I'm okay. Don't tell Mom and Dad. They'll worry,'' Kious Kelly texted
his sister on March 18 after telling her he had tested positive for the
coronavirus.

Her replies went unanswered. He died six days later.

Mr. Kelly, who was 48, was an assistant nurse manager at a hospital in
Manhattan. He had moved to New York City more than 20 years earlier to
pursue a dream of becoming a dancer, but eventually attended nursing
school.

He was known for cheering up grumpy colleagues with chocolate and
offering compassionate support to patients and their families.

``He went above and beyond,'' the son of a patient once wrote of Mr.
Kelly to the hospital's president.

His death became a rallying cry for the need for protective gear for
health care workers.

``He ran toward the oncoming enemy determined to bring survivors back
with him,'' his sister, Marya Patrice Sherron, wrote on Facebook after
his death. ``That's who he was.''

\emph{The Times previously wrote}
\emph{\href{https://www.nytimes3xbfgragh.onion/2020/03/31/obituaries/kious-kelly-dead-coronavirus.html}{an
obituary}} \emph{for Mr. Kelly, as well as}
\emph{\href{https://www.nytimes3xbfgragh.onion/2020/03/26/nyregion/nurse-dies-coronavirus-mount-sinai.html}{an
article}} \emph{about how his co-workers blamed a lack of protective
gear for his death.}

\includegraphics{https://static01.graylady3jvrrxbe.onion/packages/flash/multimedia/ICONS/transparent.png}

\begin{itemize}
\tightlist
\item
  Valeria Alfano, I.C.U. Nurse
\item
  Naples, Italy
\item
  July 8
\end{itemize}

Luckily the worst is over here, but we must not let our guard down.
Until a few months ago, I worked in a Covid department. Now that I am no
longer in that department, my desire for human contact is getting
stronger. Unfortunately, you still have to be patient and keep the right
distance. Despite that, I promise every day to give love to my patients.
I hope that everything will return to normal soon and that we will
embrace each other stronger than before.

\includegraphics{https://static01.graylady3jvrrxbe.onion/packages/flash/multimedia/ICONS/transparent.png}

Erin Schaff/The New York Times

\begin{itemize}
\tightlist
\item
  Michaela Casey, Coronary Care Nurse
\item
  Baltimore
\item
  April 13
\end{itemize}

I spent three nights in a row with a 77-year-old man who was nonverbal
because of a previous traumatic brain injury. He was dying, but slowly,
from Covid-19. His family couldn't be with him. He was Jewish and
couldn't celebrate Passover.

I held his hand, smoothed his hair and told him his family loves him and
wishes they could be with him so much.

I told him that I'm here and I smiled. He smiled back. And I held back
my tears.

I have been floated all over the hospital to different biocontainment
units outside of my home department. I have oriented and taught
intermediate nurses on I.C.U. care for critical patients to ``upskill''
them in case a time comes when we don't have enough I.C.U. nurses.

Every day, when I cross the street toward the hospital and greet the
security guard in a mask, I'm reminded that we are all in this together.

\includegraphics{https://static01.graylady3jvrrxbe.onion/packages/flash/multimedia/ICONS/transparent.png}

\begin{itemize}
\tightlist
\item
  Ousseni W. Tiemtore, Doctor
\item
  Ouagadougou, Burkina Faso
\item
  June 3
\end{itemize}

I am coordinating a project run by Doctors Without Borders to tackle
Covid-19 in Burkina Faso. In the capital, Ouagadougou, we have
transformed a former Ebola treatment center into a 50-bed Covid-19
treatment center with intensive care services.

Doctors and nurses must have jumpsuits, face shields and everything they
need to avoid getting infected, for their own safety and to treat
patients without spreading the virus. However, we have struggled to find
and bring in that necessary material as global shortages hampered their
purchase and travel restrictions hobbled the supply chain.

I do my best to comfort and reassure people in my environment. For
example, recently a friend's relative tested positive after returning
from Niger. He was afraid of being hospitalized, so they contacted me. I
focused on calming him down, providing him with insights and
recommendations from a medical perspective.

The same happened with a nurse I know. People are afraid of the disease
--- even health care workers need support.

We are dealing with the unknown, and nobody was ready for it. Many are
concerned about the disease, especially with all they read on social
media. Others are more worried about their own survival amid Covid-19
restrictions. Staying at home is a luxury many cannot afford.

\includegraphics{https://static01.graylady3jvrrxbe.onion/packages/flash/multimedia/ICONS/transparent.png}

\begin{itemize}
\tightlist
\item
  Alexandra Moran, E.R. Doctor
\item
  Miami
\item
  July 14
\end{itemize}

I live in the U.S. but I was in Ecuador on vacation when the border shut
down. I have a license to practice here and decided to join the front
lines.

I was in the peak of the Covid-19 curve here, when nothing was enough
--- not I.C.U. beds, respirators, P.P.E., not even body bags.

Body storage was something out of an unthinkable horror movie. People
couldn't find their loved ones in a sea of decomposing bodies at the
morgue.

We couldn't use the restrooms because eventually that's where we had to
temporarily put the dead.

The ones that died at home were left on the streets because this country
ran out of resources.

Today is my last day. I'm finishing this last 24-hour shift.

I get to go back home to the U.S.A., however, I'm absolutely dreading
the possibility of having to live through this again in Miami as soon as
I land.

\includegraphics{https://static01.graylady3jvrrxbe.onion/packages/flash/multimedia/ICONS/transparent.png}

\begin{itemize}
\tightlist
\item
  Óscar Caicho Caicedo, I.C.U. Nurse
\item
  Guayaquil, Ecuador
\item
  June 24
\end{itemize}

I would return home after doing 24-hour shifts and it was difficult to
fall asleep despite how tired I was. I would spend the night thinking of
people crying for help.

I remember a woman in her late 60s who took my hand and begged, ``Help
me. I can't breathe.'' I don't know what happened to her. She probably
died. She was not the only one: Many patients reached their hands to me
and said similar things.

We would leave the patients to get their prescriptions and, when we came
back, they had passed away.

We didn't even have time to go to the bathroom. Some of us decided to
wear diapers. Who could imagine that wearing a soaked diaper would
become part of the job?

We were really scared. Most of us working at the I.C.U. got infected. I
started having shivers and back pain in late March. I was very hungry,
but could not smell or taste the food.

When I got better, I went straight back to work.

Many nurse friends died. I would see them working on a Monday and know
by Friday that they were gone. My grandfather's death from the disease
struck me hard, but it was even harder to see my colleagues agonizing on
a stretcher.

The feeling of loss has been overwhelming. I lost something every day:
opportunities, moments, family, friends, neighbors. I heard people
yelling and crying every night in my neighborhood. Bodies were left
outside houses and were not being picked up. Their relatives begged me
for help, as if I could do something.

At the hospital, they told us to get therapy. But almost all the
hospital's psychologists left. Only one stayed. She spoke with us, with
the families of the deceased, with the patients, helping with
respiration techniques to calm us all. We felt miserable for not being
able to help. Now I feel a little bit better.

I did everything I humanly could.

\includegraphics{https://static01.graylady3jvrrxbe.onion/packages/flash/multimedia/ICONS/transparent.png}

Heather Sten for The New York Times

\begin{itemize}
\tightlist
\item
  Chanel Fischetti, E.R. Doctor
\item
  Boston
\item
  June 9
\end{itemize}

I was living out of a suitcase in between different cities in
Connecticut. I'd drive between several hospital sites and stay in hotels
or with my partner on days off.

Most of the patients I saw at Greenwich Hospital had Covid. We were
about 15 minutes from New Rochelle.

People would ask us all the time, ``Aren't you afraid?''

Had you asked me before the pandemic, I would've said no. But after
seeing what I saw at Greenwich Hospital, it was hard not to come home
and fall asleep with a little anxiety, thinking, ``Today could've been
the day I got sick.''

And whenever I had to intubate or go into the room, there's a split
second when you think of the risks involved, but then just move forward
because it needs to be done.

I think a lot of us feel the same way I'd imagine soldiers do.

\includegraphics{https://static01.graylady3jvrrxbe.onion/packages/flash/multimedia/ICONS/transparent.png}

\begin{itemize}
\tightlist
\item
  Edward Rippe, Internist
\item
  Manhattan, N.Y.
\item
  April 18
\end{itemize}

It was rather abrupt how quickly things quieted down. By early May we
had fewer patients in the hospital and started having patients with
health problems other than Covid. Although a lot of Covid patients
didn't survive, we recently discharged some patients who recovered after
hospital stays of over 40 days.

I caught Covid at the end of March. I was out of work for a week but it
took about five weeks to feel back to normal.

April 12 was hopefully my most intense experience during this outbreak.
I'd never experienced so much death in a single day.

It was the beginning of morning rounds when overhead came, ``Code blue,
level eight, 100 block.'' This was our block, which had been converted
to an I.C.U. to accommodate the influx of patients.

We went down the hall and gowned up. Resuscitating Covid patients is
risky for the staff because of the aerosolized viral particles. I
started chest compressions. The risk of exposure made the stakes and the
tension high for everyone.

Crack, crack, went his ribs as the C.P.R. continued. It's common to
break ribs in the process.

``Another epi!'' the senior resident called out.

The anesthesiologist at the head of the bed intubated the patient with a
contraption placed over the patient's face to prevent exposure. However,
despite 20 minutes of C.P.R., we could not achieve a pulse.

I participated in six other codes that day. I don't remember who lived
and who died but I can still feel those ribs cracking under my hands.

\includegraphics{https://static01.graylady3jvrrxbe.onion/packages/flash/multimedia/ICONS/transparent.png}

Heather Sten for The New York Times

\begin{itemize}
\tightlist
\item
  Nathan A. Villada, Paramedic
\item
  New York City
\item
  June 15
\end{itemize}

I risked my life at the front lines of the pandemic so that others could
live. I'm simply doing the same now for Black lives in these
demonstrations.

The killing of Breonna Taylor was a pivotal moment for me. She was an
E.M.T. So this obviously hit close to home.

As a paramedic, being a street medic comes naturally. On the job, we
often have to work on the streets, so the environment isn't new. In
these protests, I'm mostly sticking to first aid, but while helping
others I have to expect to be attacked by specialized police weapons.

I work in emergency medicine in the Bronx, and that plays a big role in
why I value this movement. It's obvious that people of color were
disproportionately affected by the pandemic, and I fully intend to keep
calling attention to the systemic racism that allowed for it.

\includegraphics{https://static01.graylady3jvrrxbe.onion/packages/flash/multimedia/ICONS/transparent.png}

\begin{itemize}
\tightlist
\item
  Mustafa Alam, E.R. Doctor
\item
  Brooklyn, N.Y.
\item
  April 8
\end{itemize}

I've seen the heartbreak and hope this pandemic has brought to our city.
I have seen sadness with the loss of life and joyous celebration from a
complete recovery from this virus.

I want to bring hope and healing. I want to save lives.

\includegraphics{https://static01.graylady3jvrrxbe.onion/packages/flash/multimedia/ICONS/transparent.png}

\begin{itemize}
\tightlist
\item
  Martynas Gedminas, E.R. Doctor
\item
  Šiauliai, Lithuania
\item
  July 8
\end{itemize}

The most intense was the first day we opened our Covid-19 department. I
remember seeing the first bunch of patients whose conditions weren't
that severe. They just complained of cough and fevers. It wasn't
difficult to handle but it still got me frightened that Covid-19 was all
over the place. I also felt dirty, thinking that now all my protective
equipment is covered with virus, and I won't be able to save myself from
infection or avoid bringing it home.

The situation in Lithuania is really calm now. We're more than happy to
have gone through the first wave without much harm. We are also happy
that so much more is now known about the pathophysiology of the virus
itself. It will help us a lot in preparing treatment regimens.

The epidemic exposed the inherent weaknesses of our system. I started
valuing my team a lot more. The connection between us and our
experiences together are a major contributor to my work satisfaction.

\includegraphics{https://static01.graylady3jvrrxbe.onion/packages/flash/multimedia/ICONS/transparent.png}

Nadia Shira Cohen for The New York Times

\begin{itemize}
\tightlist
\item
  Agnese Marcelli, I.C.U. Anesthesiologist
\item
  Rome
\item
  July 31
\end{itemize}

In Rome, it arrived slowly.

We watched, distraught and impotent, the images of our colleagues in the
hospitals in the north.

Then, the city turned off. Everything was surreal. The streets were
empty. The kids, at home, asked me what would become of them. The
certainties diminished all at once.

Many of the patients that I have had were elderly, torn away from their
houses, from nursing homes, from loved ones.

I experienced their gaze, full of suffering but also conscious of
already having lived a full life. The elderly are already conscious of
everything. They are fragile on the outside but strong on the inside.

Different gazes came from the young affected by Covid. Their eyes showed
anguish and fear of not being able to breathe and of dying.

There still remains the fear of an invisible enemy that could come back
and take away our serenity, our assurance and separate us from our loved
ones.

\includegraphics{https://static01.graylady3jvrrxbe.onion/packages/flash/multimedia/ICONS/transparent.png}

\begin{itemize}
\tightlist
\item
  Whitney Douglass, I.C.U. Nurse
\item
  Denver
\item
  April 28
\end{itemize}

I sweated through all my P.P.E. as I hustled to prevent the outcome I
saw coming so quickly. His body was shutting down.

We called his wife to update her. They have four daughters together. She
had dropped him off the day before and went home to self-quarantine. She
had no idea that would be the last time she would see her husband alive.

As the day continued, I ran between his room and other very sick Covid
patients' rooms. Each time: P.P.E. on, P.P.E. off. It's this little
dance we do all day.

We play music loud in the nurse's station. In quiet moments, we watch
zoo cams of monkeys swinging around --- the best distraction. We talk
about what we do on our days off. Most of us are now spending days off
alone to avoid infecting our loved ones because our exposure is so high.
We talk about how we will never take for granted just walking into a
patient's room for something.

Every time I went into my patient's room, I talked to him. I told him
who I was, what time it was and what was going on. I rubbed his back,
put ChapStick on his very swollen lips and used a warm washcloth to wipe
the gunk off his face.

Toward the end of my shift, I called his wife. As gently as I could, I
told her we did not see him making it through the night.

I put her on speakerphone and went into his room to put the phone up to
his ear and told her she could start speaking. She broke down and
pleaded with him to fight, said that she loved him, that she could not
go on without him.

I sobbed. I tried to tell her how sorry I was that we could not save
him, that I tried really hard.

I drove home that night in silence. Tears ran down my face.

He passed away the next day.

\includegraphics{https://static01.graylady3jvrrxbe.onion/packages/flash/multimedia/ICONS/transparent.png}

Jenn Ackerman for The New York Times

\begin{itemize}
\tightlist
\item
  Daetney Ewing, Certified Nursing Assistant
\item
  Minneapolis
\item
  June 11
\end{itemize}

The residents have dementia and Alzheimer's. The routine is not the same
so it puts some stress on them, and us. Even if you have goggles and a
mask and all this P.P.E., you have this nervousness inside of you.

I've been at the protests every day since they started. Me and my wife
went to bring snacks, water and milk so that when they got gassed we
could put milk on them to help the burning. We drove around to see if
anybody needed assistance.

Now I go to the 38th and Chicago location where George was killed. I go
there every day. I met the people who created the Say Their Name
memorial cemetery, and I put up a fund-raiser on Facebook so we can make
it a permanent site. What we're hoping for is that people can come at
any time and look at it and pray and do whatever they would like to do,
as if it was an actual cemetery.

There's a lot of us out there. We are having conversations like, ``I
just got off work and I was in the middle of the pandemic and then I
came out here and my dedication is to the protest.'' We can relate. We
are all exhausted. But that isn't keeping us from doing our due
diligence.

We are thanking each other and acknowledging each other's hard work. I
think that's very important. No one is like, ``I'm more important than
you because I'm a nurse and you're a C.N.A. or I'm a doctor.'' It's like
we're all on the same level.

\includegraphics{https://static01.graylady3jvrrxbe.onion/packages/flash/multimedia/ICONS/transparent.png}

\begin{itemize}
\tightlist
\item
  Bryan Adrian Priego Parra, General Practitioner
\item
  Boca del Río, Mexico
\item
  June 25
\end{itemize}

In Mexico, some people believe that the medical staff is infecting
patients and that the government is paying us to kill people. It's
common to hear people say, ``En el seguro los matan'' or ``In the social
security hospitals, they kill them.''

Someone threw bleach at a nurse in Sinaloa. A nurse in Mérida was
attacked with hot coffee. People vandalized a hospital in Las Rosas in
Chiapas, burned an ambulance and attacked staff.

Some taxi drivers or bus drivers won't stop for me or my co-workers.
Last week, while I was walking down the street, a man looked at my
uniform and moved to the other side of the sidewalk.

Because of this fear, many people arrive at the hospital too late.

The world looks different inside the Covid area of the hospital. There
is less sound. We are more expressive than before and use more body
language and eye expressions because the special suit creates a barrier
to oral communication. I use my hands more often when I interview my
patients.

This week was the first that we had few new hospitalizations. The
outlook seems good, but we know that we depend on people's choices and
behaviors.

We don't know if this peace will last, or if we will return to the war.

\includegraphics{https://static01.graylady3jvrrxbe.onion/packages/flash/multimedia/ICONS/transparent.png}

\begin{itemize}
\tightlist
\item
  Ninoska Flores Chávez , Nurse
\item
  Guayaquil, Ecuador
\item
  June 24
\end{itemize}

When our colleagues started getting sick, it became really stressful. We
had to do 24-hour shifts. We had no time to eat, to go to the bathroom
or even sit down for a moment. My colleagues and I crumbled on several
occasions.

I remember wanting to cry and hold on to them in the emergency room, but
we were not allowed to for safety reasons. The only thing we did
together was pray.

Sometimes, the shifts were longer because many colleagues could not make
it to the hospital: There were no buses, taxis, nor anyone else willing
to drive them to work.

At one point, I told a cabdriver where to take me and he said: ``I am
sorry. In that hospital, there are coronavirus patients. I cannot go
there.'' He forced me to get out of his car.

When you're a health worker, your sole purpose is to send patients home
sound. In those days, that rarely happened. Most came to their final
breath at the hospital. We were not prepared to see so many people
dying. We would lay them on a stretcher, they would go into cardiac
arrest and die.

You feel powerless and you are traumatized.

\includegraphics{https://static01.graylady3jvrrxbe.onion/packages/flash/multimedia/ICONS/transparent.png}

\begin{itemize}
\tightlist
\item
  Colleen Hill, Physician Assistant
\item
  Atlanta
\item
  April 16
\end{itemize}

One of the elements of my job is calling family members to tell them
their loved one has died. Right before Covid really hit my city, I got
that call myself: My father died, very unexpectedly, of a heart attack.

I've been making an awful lot of those calls now that I'm back at work
after his funeral.

They're all blending together, as are the death certificates that I
sign. But this juxtaposition of experiences has forced me to face
head-on the problem of inevitability.

People die. All people die. My father, your father, me and you.

So I come to work not to fight death, not to prevent death, but to
support life and hope while they last.

My most intense experiences have been relaying messages from families to
their loved ones.

``Tell Pookie I love her.''

``Tell my husband I lit a candle for him.''

And I put on my protective gear --- the same N95 mask I've been wearing
since the pandemic started --- and I go into the room and I squeeze
their hand and tell them they are loved.

\includegraphics{https://static01.graylady3jvrrxbe.onion/packages/flash/multimedia/ICONS/transparent.png}

Sasha Arutyunova for The New York Times

\begin{itemize}
\tightlist
\item
  Daniel Akinyemi, I.C.U. Nurse
\item
  Montclair, N.J.
\item
  April 23
\end{itemize}

Our I.C.U. is very busy. It's all Covid and there are so many deaths. A
lot of the patients that I've been taking care of, they go within three
days.

The phone rings relentlessly with family and friends seeking updates. A
lot of times you don't want to take off your P.P.E. to answer the phone
because you could put yourself at risk. So usually I will put a glove
over the phone so that I can answer.

About a month ago, I had a patient on a ventilator whose husband kept
calling. When I spoke with him, he said he believed in God and the power
of a miracle. We said a little prayer together over the phone and I
asked, ``Does she have a favorite song?''

He said she's been singing Blue Bayou ever since they met. When she does
laundry, when she does her hair. I knew the song and I said, ``I will
sing it for her.''

I play songs for patients on my phone. I found out that you can have
your iPhone in a sandwich bag and you can actually unlock it with your
thumb. I sang it to her while it was playing. And I read her favorite
verse in the Bible, Psalm 23.

The monitor started going off. Basically she started sucking in the air
that the ventilator was providing. I was on her right side and she was
looking toward the right. So I quickly moved to the left, and she turned
to the left. I called a couple of my co-workers and everyone was happy.

She was weaned off the ventilator the next day.

When a patient is recovering, it's just like a baby was born, you know?

\includegraphics{https://static01.graylady3jvrrxbe.onion/packages/flash/multimedia/ICONS/transparent.png}

Alessandro Falco for The New York Times

\begin{itemize}
\tightlist
\item
  Maitê Silva Martins Gadelha, General Practioner
\item
  Belém do Pará, Brazil
\item
  June 29
\end{itemize}

I graduated on April 16 and started working five days later.

I went to the hospital for a meeting for new doctors. That same day, the
number of patients was so great that I was asked to start attending
immediately.

The moment I saw my first patient die in front of me is a moment I will
never forget. You feel a sense of helplessness. It was one of the worst
days of my life.

Since I started, there have been only two days I have not worked: the
day after my grandfather was hospitalized with Covid-19 and after he
died about a week later.

My front line work against Covid has changed me a lot. I was afraid, but
I didn't have time to feel that fear. I simply went to work, day after
day.

Pará is a huge state, and there are big differences between the various
municipalities. I traveled to work in the suburbs and villages to serve
that population. To reach many inland regions, you have to travel in
small boats.

The situation inland has changed a lot, but in the first cities we
reached, we met a population totally without assistance and without
information about the pandemic.

Many times the problems were not even serious, but being able to inform
patients and help them deal with the situation can make the difference.
That was our main goal: to bring assistance and warmth to the heart of
those patients.

\includegraphics{https://static01.graylady3jvrrxbe.onion/packages/flash/multimedia/ICONS/transparent.png}

\begin{itemize}
\tightlist
\item
  Joyce Lamb, Medical-Surgical Nurse
\item
  Queens, N.Y.
\item
  May 3
\end{itemize}

When I entered the rooms of Covid patients, I could tell they were
scared. They told you that. They were looking to me for comfort.

I was scared, too, but I went in and I held their hand. I let them know
that they were not alone. I tried to fill the room with love.

This pandemic has increased my compassion and strengthened my belief in
humanity.

We are in this together.

\includegraphics{https://static01.graylady3jvrrxbe.onion/packages/flash/multimedia/ICONS/transparent.png}

\begin{itemize}
\tightlist
\item
  Sanjai Sinha, Internist
\item
  Pelham, N.Y.
\item
  April 25
\end{itemize}

I told a woman over the phone that she would indeed be all right. My
words soothed her so much.

She had viewed it as a death sentence until, based on my overall
assessment, I comforted her.

``May Jesus watch over you forever,'' she said. I'm an atheist, but it
touched me beyond words.

\includegraphics{https://static01.graylady3jvrrxbe.onion/packages/flash/multimedia/ICONS/transparent.png}

\begin{itemize}
\tightlist
\item
  Kenji Fujiwara, Surgeon
\item
  Fukuoka, Japan
\item
  April 28
\end{itemize}

I'm a surgeon in Japan, but I volunteered with the team fighting
infectious diseases, including Covid-19.

I have a family. My wife is pregnant. So I asked my mother to support me
by taking care of my family in her house. Fortunately, my family
understood and my son cheered me by saying, ``It is cool to fight
against the virus.''

\includegraphics{https://static01.graylady3jvrrxbe.onion/packages/flash/multimedia/ICONS/transparent.png}

Gianni Cipriano for The New York Times

\begin{itemize}
\tightlist
\item
  Romolo Villani, Anesthesiologist
\item
  Salerno, Italy
\item
  May 3
\end{itemize}

Fortunately, the lockdown of the last two months in our region has
greatly limited the transmission of the virus and therefore the victims
of the epidemic. In Naples, we have had a couple hundred people die from
the virus. Most of them were elderly people coming from nursing homes.

Now, we are also seeing people who had recovered from the virus testing
positive again. Among these are some of my colleagues who had returned
to their jobs as health workers.

This shows that we cannot yet let our guard down in the fight against
the coronavirus.

\includegraphics{https://static01.graylady3jvrrxbe.onion/packages/flash/multimedia/ICONS/transparent.png}

\begin{itemize}
\tightlist
\item
  Shane Smith, E.R. Nurse
\item
  Queens Village, N.Y.
\item
  June 26
\end{itemize}

Night after night, we were inundated with Covid-19 patients. There were
stretchers on top of stretchers with critically ill patients. Some of
these people were dying sandwiched between patients grasping for air.

I would spend the entire night in P.P.E. We didn't have enough to change
between patients. I had a face shield I reused for weeks. Each morning
after my shift, I would clean it and store it in a brown paper bag to be
reused the next shift.

I dreaded going into work every night.

At one point, I thought I was going to explode from sheer sadness and
the feeling of impending doom. We all did.

My sister and I got together during the height of it. We wore our masks
and attempted to socially distance while sharing a bottle or two of
wine. She is the only person outside of work I have come into contact
with since the pandemic began.

We swapped war stories. She too is a nurse. She too was experiencing the
madness of Covid in real time.

Things have dramatically changed. The number of people coming in to the
emergency department with Covid-19 has significantly dropped.

I feel like I'm finally unfettered, at least for now, from the feelings
of helplessness and hopelessness.

\includegraphics{https://static01.graylady3jvrrxbe.onion/packages/flash/multimedia/ICONS/transparent.png}

\begin{itemize}
\tightlist
\item
  Josefina Opazo, E.R. Doctor
\item
  Santiago, Chile
\item
  June 9
\end{itemize}

I've been working five or six 12-hour shifts per week for nearly four
months because many doctors are falling ill. As the weeks have gone by,
I have felt more anxious. I'm sleeping less, eating poorly and losing
weight.

In Chile, emergency doctors are particularly exposed to the virus
because we go into people's homes to treat them. I'm not scared of
falling ill, but I live with my mother, who would be at risk because of
her age, so I take every precaution I can.

While I'm used to dealing with patients whose chances of survival are
low, it's very different to see people die alone. The virus is killing
young people, not just the elderly. A few weeks ago, I had to call the
mother of a 36-year-old patient to explain that her son was in critical
condition.

At times, it can feel like everything has been turned on its head.

\includegraphics{https://static01.graylady3jvrrxbe.onion/packages/flash/multimedia/ICONS/transparent.png}

\begin{itemize}
\tightlist
\item
  Carmen Presti, Acute Care Nurse Practitioner
\item
  Hollywood, Fla.
\item
  April 9
\end{itemize}

I was not answering a calling when I became a nurse. I abandoned my
career working in production in the film industry when my desire to
start a family and have a stable work schedule superseded the excitement
of working on a film set.

As a nurse, I am witness to and part of real-life drama that can only be
mimicked on film.

I volunteered to work in the Covid I.C.U. along with my medical staff
and nurses.

It is difficult to give loved ones updates on the phone because they are
clinging to any shred of hope that their relative will improve, and
often I cannot provide this.

Telling a wife that her husband's oxygen levels on the ventilator had
improved slightly brought her to tears.

She prayed for me to be blessed with wisdom and strength and for my
family's health. In the midst of her suffering, she conveyed her
gratitude.

With both of us in tears, and my heart broken for her, I thought, ``This
is why we do what we do.''

During one shift, my patient's ventilator tubing became disconnected and
I was about a foot away from a blast of air from the patient. I am
fortunate that my hospital has provided us with N95s and appropriate
P.P.E., but we are having to reuse our masks and therefore can't be
ensured full protection from contamination. Like many providers, I
quarantined myself from my 12-year-old son and husband for two weeks.

\includegraphics{https://static01.graylady3jvrrxbe.onion/packages/flash/multimedia/ICONS/transparent.png}

\begin{itemize}
\tightlist
\item
  Phillip Scotti, Cardiology Physician Assistant
\item
  Brooklyn, N.Y.
\item
  April 9
\end{itemize}

I was a musician when I first went to college, but I loved and wanted to
work in the sciences. When one of my classmates was discussing what
physician assistants do, I knew it was the right job for me.

My unit changed overnight when the virus struck. It took only one week
for the entire unit to be filled with Covid-19 patients. They were
gasping for every breath.

I am lucky that I was able to separate from my family to keep them safe.
I didn't see them for about two months. It was very difficult for us,
but medicine is my calling.

I am lucky to work with a talented group of colleagues in this fight. We
will continue to fight the good fight and care for our fellow New
Yorkers. We sing songs together to release some tension, keep our
spirits up and keep pushing forward.

\includegraphics{https://static01.graylady3jvrrxbe.onion/packages/flash/multimedia/ICONS/transparent.png}

Nicola Vigilanti

\begin{itemize}
\tightlist
\item
  Marion Levigne, I.C.U. Nurse
\item
  Lyon, France
\item
  April 19
\end{itemize}

The pandemic came. We didn't plan for it. We were not ready.

We could barely keep up with the pace. Patients arrived one after the
other. There were hurried intubations. Families were in distress, and we
had to deal with a different kind of stress.

But as in every storm, we also experienced little unexpected rays of
sunshine: children's drawings, chocolates, equipment, applause and music
in the windows.

And then our first victories came. An 86-year-old woman who was
extubated caught the breath that she had missed so much.

A 62-year-old man finally returned home to the delight of his daughters.

We may have lost our bearings, but not the meaning of the job we chose
to do.

\includegraphics{https://static01.graylady3jvrrxbe.onion/packages/flash/multimedia/ICONS/transparent.png}

Sasha Arutyunova for The New York Times

\begin{itemize}
\tightlist
\item
  Linda Wang, Internist
\item
  Manhattan, N.Y.
\item
  April 20
\end{itemize}

It has been emotionally draining to navigate this crisis. Not only am I
trying to be present for my patients who are sick and dying alone in the
hospital, but I am also trying to be present for their families, who are
sick with fear and uncertainty.

On normal days, I am a primary care physician. For two weeks in April, I
worked a night shift at the hospital where I did my earliest medical
training, going between the E.R. and the inpatient ''medicine Covid''
units to evaluate new patients and take care of those already
hospitalized with Covid pneumonia.

One night, I got an alert around 2 a.m. that one of my primary care
patients was in the E.R. She was 50, and just got an apartment for
herself and her daughter. Her son was about to make her a new
grandmother, and she proudly displayed the ultrasound photo on her
phone's background.

For more than two years, we had been meeting every three months to try
to improve her diabetes. I had seen her just a month before, and she was
hopeful as ever to strike a balance between taking insulin and living
her life.

Now, she was lying on her stomach before me, a large and cumbersome mask
on her face delivering oxygen. Her eyes were tired but determined.

I fumbled for my phone underneath my gown. I worried that I would
contaminate myself but I also hoped to get her daughter on FaceTime
before it was too late.

``I love you, Mom,'' her daughter cried. ``Stay strong. Don't give up.''

Within moments, she was intubated, and within a week, gone.

My colleagues who have lived through the H.I.V./AIDS epidemic are blown
away by the scale of the suffering. Some feel lost and helpless. The
most I can do is sit for a few moments with patients, call their
families, and offer a few words of comfort.

\includegraphics{https://static01.graylady3jvrrxbe.onion/packages/flash/multimedia/ICONS/transparent.png}

\begin{itemize}
\tightlist
\item
  Gila Zarbiv, Nurse Midwife
\item
  Jerusalem
\item
  July 7
\end{itemize}

Birth in the time of corona is understanding what it means to be there
for another human with all your being.

It is the need to be covered from head to toe with a mask that cuts your
nose and cheeks and makes it hard to breathe, a visor that covers your
entire face making it hard to see, and a gown that envelops you from
head to toe, rendering you unrecognizable.

Birth in the time of corona is the ability to be there, completely, for
your patient despite all the physical, emotional and mental barriers.

We are bracing for and now entering the second wave of Covid-19. We are
improving and changing every single day to protect mothers and their
unborn and born babies. We are no longer going to separate the mothers
and babies. We are preparing the staff for new guidelines that include
breastfeeding and bonding in order to make sure that despite all the
chaos, unknowns and difficulty during these trying times, the
mother-infant dyad is maintained.

Conquering this mountain in the days of corona requires a dance of
sensitivity, care, quiet and thought.

\includegraphics{https://static01.graylady3jvrrxbe.onion/packages/flash/multimedia/ICONS/transparent.png}

\begin{itemize}
\tightlist
\item
  Dolapo Olugbile, Radiographer
\item
  Laurel, Md.
\item
  June 11
\end{itemize}

I reused my first N95 mask for more than a month before the straps
ripped. I used a stapler to mend it. Recently, the hospital altered and
enclosed an entire floor for patients who had Covid-19. Walking those
hallways and into each room knowing that gruesome virus was looming
there took a selfless army of health care workers.

The virus has forced everyone to see exactly what we have been rioting
and speaking up for since Trayvon, Freddie Gray and other protests.
Quarantine forced the world to sit back and tune in.

I simply feel I owe it to all those before me and those coming after me.
I can't complain about how my people are being treated and not do what I
can to make a change.

One of the biggest moments for me at a recent protest in Washington was
seeing the Pride group with their entirely separate protest merge with
the other major march to the White House. It was astonishing how diverse
the crowd was and how loud their voices were. It elevated our voice and
the energy that day.

To see the world join the protests shows this is a major issue. For
once, I truly believe it is being comprehended. We are not asking for
much. We simply want to be treated equally. Our lives matter, too.

\includegraphics{https://static01.graylady3jvrrxbe.onion/packages/flash/multimedia/ICONS/transparent.png}

\begin{itemize}
\tightlist
\item
  Stephanie Benjamin, E.R. Doctor
\item
  San Diego
\item
  April 9
\end{itemize}

Low oxygen level, fever and confusion. I intubate the patient not long
after she arrives at the emergency department. I remind myself, ``She's
healthy, young, and female --- the odds are in her favor.'' (And mine,
if I get sick, too.)

Hours later I hear, ``She's coding!'' followed by the rhythmic thud of
CPR. I lead the code. Four minutes: no pulse. Compressions continue.
Rounds of epinephrine are given. Eight minutes: no pulse. We work
together to save this patient, this woman, this mother, this human
being. Ten minutes: no pulse.

I start tearing up and berate myself. This is my room, my team. Pull it
together! Sixteen minutes: no pulse. I announce, ``At 20 minutes, if
nothing changes, I'll pronounce her time of death.'' Eighteen minutes:
She has a pulse!

I slip into a nearby office. My N95 mask starts choking me; I rip it off
and sob on the floor. Yes, we got her back, but she's already died once.
Her odds of surviving Covid-19 are dismal now. I dry my eyes, don my
mask, and head back to the emergency department.

\includegraphics{https://static01.graylady3jvrrxbe.onion/packages/flash/multimedia/ICONS/transparent.png}

\begin{itemize}
\tightlist
\item
  Marie Campbell, Medical-Surgical Nurse
\item
  Philadelphia
\item
  June 10
\end{itemize}

My floor went from a medical-surgical floor to a Covid floor. We were
doing things we had not seen since nursing school, because they were not
within our specialty. We were intubating patients daily and acting like
a step-down I.C.U. Patients would decline suddenly and with no warning.
The staff was under constant stress. We needed more resources.

Our census had been trending down on the Covid floor I work on, and we
were able to close most of the makeshift units in doctors' offices and
hallways.

However, there was a spike in cases from the Memorial Day weekend
beachgoers. Our unit is back to being full, and we are severely short
staffed.

We still don't have adequate P.P.E. Many of us have resorted to
re-wearing N95 masks. I have been rotating between two N95 masks for the
past three months.

The most difficult aspect of life right now is seeing other people going
back to pre-Covid life --- hugging family and being with friends --- and
knowing I can't, because there is still such a high risk of me being an
asymptomatic carrier.

When people ask, ``Why do you do this?'' my answer is simple: We have
answered a calling to care for the sick and helpless. We save lives on a
daily basis and this pandemic does not change that.

\includegraphics{https://static01.graylady3jvrrxbe.onion/packages/flash/multimedia/ICONS/transparent.png}

\begin{itemize}
\tightlist
\item
  Ariel-Philip Flores, Palliative Care and Oncology Chaplain
\item
  Los Angeles
\item
  July 4
\end{itemize}

There was an elderly woman dying from Covid-19. She also had
Alzheimer's, which made it more distressing for her daughter. We would
try to call the daughter at least every other day to check in and give
her updates.

She said her mother had been a very social person, worked as a hair
stylist and would always ``be on'' when talking with strangers. She was
the life of the party.

Now, the daughter was struggling with losing her mother twice, first to
Alzheimer's and then to Covid. She could not even say goodbye in person.

Although not religious, they were spiritual. I would offer words of
peace. I focused on the daughter's love for her mom. I said that her
mom's spirituality came from her ability to quickly connect with others
and bring a sense of joy in their lives.

The daughter, tearful over the phone, shared how important it was to
hear about her mom's impact on others. She was grateful for our
near-daily calls to check in with her. She appreciated being given a
sacred space where she was able to name her feelings of pending loss and
find comfort in sharing her mom's story with us.

The mother passed away over the weekend when I was not working. I do
hope that her daughter was able to find solace knowing that she was
loved by many.

\includegraphics{https://static01.graylady3jvrrxbe.onion/packages/flash/multimedia/ICONS/transparent.png}

\begin{itemize}
\tightlist
\item
  Anggy Barrera, Doctor
\item
  Caracas, Venezuela
\item
  June 30
\end{itemize}

Following the appearance of the virus and the quarantine orders, I
quickly got involved with Doctors Without Borders to care for patients
with Covid-19. I stopped my previous work with survivors of sexual
violence and other violent situations.

At first, the uptick in cases seemed very subtle. Now, hospitals are
filling up with people with Covid-19. The protocols change very quickly,
and what you would have done yesterday with a particular case is no
longer used today. It feels like you don't have time to adapt to one
situation before you're already facing another. Going from a few dozen
suspected cases to hundreds of probable or confirmed cases is
overwhelming.

We don't know if our health care system will collapse at any moment,
like those in so many other countries around the world. This is one of
my biggest fears about the pandemic: Now that Covid has become the
priority, patients with other illnesses will be left defenseless,
because if they have any complication, they will have a hard time being
admitted to hospitals to receive the care they need.

\includegraphics{https://static01.graylady3jvrrxbe.onion/packages/flash/multimedia/ICONS/transparent.png}

\begin{itemize}
\tightlist
\item
  Sneh Sonaiya, Internist
\item
  Gujarat, India
\item
  June 9
\end{itemize}

Since the lockdown in India was lifted, the number of cases are
skyrocketing. My state, Gujarat, has one of the highest numbers of
positive cases in all of India.

The hospital is running short of P.P.E. kits. One of my colleagues
tested positive. All of this aggravates my deepest fears.

It has been months since I have seen my parents or my loved ones back in
my hometown. Being with them would put their lives at risk, and that is
the last thing I want.

It baffles me at times how psychologically traumatizing this disease
must be for a patient to cope with. They're without any relatives or
loved ones around and they can't move out of bed except to use the
washroom.

Early on, there was a power outage in my intensive care unit, something
unusual. All ventilator machines should have a battery backup, but we
happened to be unlucky that day. In less than a minute, I could hear the
gasps of the patients and I saw the nursing staff and my resident doctor
running to save them. I quickly joined in providing manual breathing
support to those critical patients.

The struggle continued for nearly half an hour, until we finally had the
electricity back. Thankfully everyone was safe.

\includegraphics{https://static01.graylady3jvrrxbe.onion/packages/flash/multimedia/ICONS/transparent.png}

\begin{itemize}
\tightlist
\item
  Katty Renard, Clinical Research Nurse
\item
  Brussels
\item
  July 11
\end{itemize}

The first week, I was so exhausted physically and emotionally. I had
trouble sleeping and I was very sensitive.

One-and-a-half months have passed and I'm doing better, but I'm still
tired. I know I'll never be the same Katty I was before this terrible
experience.

I'm still affected by the solitude of the living and the dying and the
choices we made to save the maximum number of people. I know it was
necessary, but I still have difficulties accepting it in a country such
as ours.

Now Belgians are concerned about economic issues, a potential second
wave of Covid and the risk of contagion by travelers from abroad. There
is Covid trauma among caregivers and people who were severely infected,
and among elders still in quarantine in nursing homes.

\includegraphics{https://static01.graylady3jvrrxbe.onion/packages/flash/multimedia/ICONS/transparent.png}

\begin{itemize}
\tightlist
\item
  Manya Gupta, Hospitalist
\item
  Chicago
\item
  May 5
\end{itemize}

I am nearly 10 years out from the completion of my residency training,
and prior to the Covid pandemic, I was starting to feel the pangs of
burnout. But since the crisis, I have felt more than ever the things
that burnout had taken away: Joy. Purpose. Camaraderie. A collective
sense of accomplishment, validation and satisfaction.

One of the hardest things about caring for Covid-19 patients is the
absence of visitors. Patients, who are already feeling anxious about
their diagnosis, feel even more alone and isolated.

This is especially true for non-English-speaking patients. Some health
care workers are able to speak other languages, but for those who do
not, one of the few ways to safely communicate is through an interpreter
phone. It can be extremely difficult to have an emotional conversation
with a patient while speaking via an invisible person who cannot see me
or the patient.

My heart feels so heavy when patients ask gut-wrenching questions like,
``Am I going to die?'' and I have to rely on a faraway voice to
translate my answer while watching the person in front of me cling
desperately to the phone, as if the phone itself were a lifeline.

\includegraphics{https://static01.graylady3jvrrxbe.onion/packages/flash/multimedia/ICONS/transparent.png}

\begin{itemize}
\tightlist
\item
  Tiago Valim, Head and Neck Surgeon
\item
  São Paulo, Brazil
\item
  June 3
\end{itemize}

When the coronavirus epidemic reached São Paulo, my routine changed
because elective surgeries were no longer being performed. It felt weird
to watch everything from home and not be on the front line. So, as soon
as I found out that Doctors Without Borders had started a project in São
Paulo, I decided to find out how I could help.

I have been working with vulnerable populations, mostly people who live
on the streets. Fake news spreads easily among them. I've heard from
them that Covid-19 is a ``rich people disease'' and that it ``isn't that
serious.''

I frequently see that they are not taking care of themselves. They also
resist being treated and isolated when they become ill.

Controlling Covid-19 is all about collectivity. It's clear that right
now we lack cohesion, not only among the homeless population, but also
all over the country.

It's sad to see. We could be doing a lot better. The consequences will
probably be more deaths.

\includegraphics{https://static01.graylady3jvrrxbe.onion/packages/flash/multimedia/ICONS/transparent.png}

\begin{itemize}
\tightlist
\item
  Amanda Leone, Orthopedic Physical Therapist
\item
  Waltham, Mass.
\item
  June 9
\end{itemize}

I was laid off because of the pandemic. Though physical therapy is
considered essential, our caseloads dwindled very quickly, as most
patients were no longer willing to come into the office. The majority of
the company was laid off with a promise to rehire once the dust settles.

So I decided to volunteer in a long-term care facility as the sole
rehabilitation provider for about 150 patients. Their entire
rehabilitation department --- all the physical therapists, occupational
therapists and speech-language pathologists --- were either sick or
under quarantine, and the residents at the facility hadn't had any rehab
services in over a week.

I tried to keep the residents moving as best as I could. They were
recovering from strokes, pneumonia, falls, broken bones, amputations due
to diabetes.

Once the initial wave of Covid-19 cases passed, the facility began
taking in more and more patients from nearby hospitals that were coming
out of I.C.U.s and in need of rehabilitation. Physical therapy became a
huge part of their treatment, as it's our job to get them up and moving
again.

I saw patients pass away from Covid-19. I saw them transferred to the
hospital, never to return. But I also saw recoveries and joy.

I will never forget my time in that long-term care facility, and I will
never forget the patients I lost.

\includegraphics{https://static01.graylady3jvrrxbe.onion/packages/flash/multimedia/ICONS/transparent.png}

\begin{itemize}
\tightlist
\item
  Lawrence Asprec, Palliative Care Doctor
\item
  Manhattan, N.Y.
\item
  April 11
\end{itemize}

It's one thing to have end-of-life discussions face to face, in a calm
environment and with enough time for everyone to process a prognosis and
assess goals of care. It's another to accelerate that process, via
telephone, with a family member or loved one who can't physically be
there to see the patient, let alone say goodbye. And by the time you
finish that hurried conversation, without having time to reflect on the
heartbreaking decision or the pleas, the E.R. is already paging you to
start again with the next family.

\includegraphics{https://static01.graylady3jvrrxbe.onion/packages/flash/multimedia/ICONS/transparent.png}

\begin{itemize}
\tightlist
\item
  Sarita Nori, Dermatologist
\item
  Somerville, Mass.
\item
  April 21
\end{itemize}

I just finished my first shift volunteering at the Boston Hope field
hospital for Covid that was set up in the convention center.

It's amazing that in five days a convention center was turned into a
500-bed hospital and 500-bed respite center for homeless patients with
Covid. It's very safe, with full P.P.E., dedicated stations for donning
and doffing it, showers, locker rooms and food.

On the floor, there is no question about the disparities that this virus
highlights. I didn't see one Caucasian patient all night. And a large
number of the patients here can't go home simply because there is no
``other room'' for them to quarantine in, to keep their housemates safe.

It is very frightening to be on the front lines. On the other hand, I
feel useful and appreciated. It's probably the most alive I have felt in
recent years while practicing medicine. I can't wait till my next shift.

\includegraphics{https://static01.graylady3jvrrxbe.onion/packages/flash/multimedia/ICONS/transparent.png}

\begin{itemize}
\tightlist
\item
  Alejandra Ramírez, General Practitioner
\item
  Tijuana, Mexico
\item
  June 30
\end{itemize}

I currently work in Caracas, Venezuela, responding to the pandemic as a
member of Doctors Without Borders.

The uncertainty is the hardest part. As soon as a patient comes in with
symptoms, we need to identify what is a potential case and what isn't,
and to immediately protect them to prevent contact with everyone else.

When a patient is hospitalized, they have a sea of emotions:
uncertainty, fear, loneliness, pain, anxiety, on top of all their
physical symptoms.

We do everything possible to guarantee that they are treated humanely,
to provide not only good medical care, but to give them confidence that
we are by their side.

What worries me the most about this pandemic is the significant number
of people who don't believe that it is real. They don't realize that by
not protecting themselves, they put everyone around them at risk.

\includegraphics{https://static01.graylady3jvrrxbe.onion/packages/flash/multimedia/ICONS/transparent.png}

Brittany Greeson for The New York Times

\begin{itemize}
\tightlist
\item
  Geneva Tatem, Pulmonary and Critical Care Doctor
\item
  Detroit
\item
  April 14
\end{itemize}

Being the only Black pulmonary and critical care physician at my
hospital has made my experience very different than everyone around me.

Our hospital is in Midtown Detroit. Detroit is 80 percent
African-American. Looking around the I.C.U. and seeing who's here, at
first you say, OK, what I'm seeing is just reflective of the local
population. But once I started to see numbers coming out of New York,
coming out of Chicago, coming out of New Orleans, I realized maybe it's
not just a local phenomenon.

I've had parents of my friends die. I've had multiple church members
die. I've had people within my community that I know personally die. And
so that's an additional layer of feeling isolated. Anyone who's ever
been through grief and loss understands that feeling of isolation. Even
though people may express sympathy, they just cannot understand what
you're dealing with because they're not in your situation.

A crisis really magnifies the disparities that are already there. Access
to care, bias in care. When you have a population that doesn't have
running water because of water shut-offs, how do people wash their hands
and maintain basic hygiene in a pandemic? When I sit back and think
about those things, I can't help but get upset and frustrated. You
wonder about basic humanity when it's acceptable to society to have
people who don't even have running water.

\includegraphics{https://static01.graylady3jvrrxbe.onion/packages/flash/multimedia/ICONS/transparent.png}

\begin{itemize}
\tightlist
\item
  Natasha Raziuddin, Physician Assistant
\item
  Jessup, Md.
\item
  May 6
\end{itemize}

People think they're invincible until it hits home. I've had multiple
patients in denial until they find out their friend or family member is
positive.

My patient was a hairdresser who bent the rules for her friends. Now two
of them are in the I.C.U. and she's positive.

We recently found out that one of our own respiratory therapists has
died.

No email was sent.

We're all scared. There is no transparency with our administration.

We're risking our lives daily. We're in the unknown and feel isolated.

\includegraphics{https://static01.graylady3jvrrxbe.onion/packages/flash/multimedia/ICONS/transparent.png}

\begin{itemize}
\tightlist
\item
  Danielle S. Wilcock, E.R. Nurse
\item
  Ballston Spa, N.Y.
\item
  July 7
\end{itemize}

I was informed in March that I was exposed to a Covid-positive patient.
On April 3, I had a sore throat and on April 4, I lost my sense of smell
and taste. I isolated at home with my husband and two kids. After just
seven days of isolation I returned to work because I'm essential. I
never had a fever.

My sense of smell is still absent, and my sense of taste is so
diminished that nothing tastes good. At work, my loss of smell seems
like a blessing at times but it also limits my practice. I rely on
co-workers and doctors to relay odors they smell that can often
contribute to a diagnosis. For example, sweet-smelling urine indicates
glucose and possible uncontrolled diabetes.

Our E.R. has always been busy but we are now increasingly busy.

I have a strong feeling that we did flatten the curve, but I know better
than to think that this is over. As restrictions are lifted, I believe
we will see a rise in cases.

\includegraphics{https://static01.graylady3jvrrxbe.onion/packages/flash/multimedia/ICONS/transparent.png}

\begin{itemize}
\tightlist
\item
  Anil Magge, Pulmonary Critical Care Fellow
\item
  West Hartford, Conn.
\item
  May 2
\end{itemize}

Today was a tough day. I FaceTimed a family in the last moments of my
patient's life. My patient was a 70-year-old man who had a daughter and
son, several grandchildren, and a husband. As they said their last words
to their ``pa,'' I realized what a crazy moment in time we are living
in.

Over the course of the month, I have treated several Covid patients.
Initially, we treated with hydroxychloroquine and azithromycin, then
interleukin 6 receptor antagonists and now convalescent plasma. Although
the treatments have changed rapidly, what I have learned the most is how
important it is to remain human and empathetic. My patients' families
did not thank me for using the most up-to-date treatments, but rather
for standing in as family for their loved ones during their dying
moments.

Being a physician during these times is truly a privilege.

\includegraphics{https://static01.graylady3jvrrxbe.onion/packages/flash/multimedia/ICONS/transparent.png}

\begin{itemize}
\tightlist
\item
  Hadia Kohi, Midwife
\item
  Feroz Koh, Afghanistan
\item
  April 25
\end{itemize}

There aren't enough female doctors and specialists in Afghanistan.
Afghans are so conservative, they don't want male doctors to check their
women. Women are losing their lives. That is why I decided to become a
midwife. I save moms' lives.

When the Covid-19 center was established, I applied for a job.

I work 12 hours per day. Many people, including pregnant women, come for
tests. Sometimes it is hard to handle, but it is a crisis and people
must be kind.

A few days ago, a pregnant woman with symptoms came to the center. Her
husband said no men were allowed to check her, even if she died. We put
the woman in isolation. My co-worker and I went to check her. She was
panicking, she started screaming and fainted. For a moment I lost
myself, but very soon I realized I needed to stay strong and help her.
We took her blood and checked her blood pressure. When she gained
consciousness, she started crying and said she wanted to go home. I told
her the virus doesn't kill everyone, and it isn't a thing to be ashamed
of. Anyone can get it. It was hard to convince her, but we made it.

\includegraphics{https://static01.graylady3jvrrxbe.onion/packages/flash/multimedia/ICONS/transparent.png}

\begin{itemize}
\tightlist
\item
  Ferrukh Faruqui, Family Physician
\item
  Ottawa
\item
  April 10
\end{itemize}

For a few years, I've thought of quitting medicine. In my family
practice, I see affluent people who are discontented with their lives,
the medical system and doctors. And they're not afraid to say so.

This week, I went back to acute care medicine to look after Covid-19
patients. The first day, I stood tall. We gowned, gloved, donned our
masks. And we went out, like soldiers do, to fight a common enemy.

I'm reading EKGs again, interpreting chest X-rays. Most of all, I'm
connecting with patients in a way that matters again. It's made me
realize why I went into medicine 30 years ago.

\includegraphics{https://static01.graylady3jvrrxbe.onion/packages/flash/multimedia/ICONS/transparent.png}

\begin{itemize}
\tightlist
\item
  Shane Woods, Clinical Technician
\item
  Arlington, Va.
\item
  May 5
\end{itemize}

The hardest moments are the thousand tiny anxieties absorbed from the
people closest to us, like a roommate who timidly asks if I can shower
somewhere else before I come home or my mom, who cares deeply, but can't
hide her fear and wonders why I went back to work at the hospital.

Then there are the added demands to protect the people we care about by
staying away. It means that you're alone in carrying the burden of
seeing someone die in isolation.

\includegraphics{https://static01.graylady3jvrrxbe.onion/packages/flash/multimedia/ICONS/transparent.png}

\begin{itemize}
\tightlist
\item
  Viquar Mundozie, Family Physician
\item
  Lake in the Hills, Ill.
\item
  June 8
\end{itemize}

It's been more than three months that the world has been dealing with
this pandemic. We have seen it all: fear, deaths, recoveries, lost jobs,
lockdowns, bad leadership, protests in the name of freedom, mask
culture, social distancing, the failing economy and racism.

Since the curve has flattened for now, we are easing the restrictions in
caring for patients and opening businesses.

We had dinner on the patio in a restaurant yesterday. It felt like life
was getting back to normal.

We felt relieved for a bit.

\includegraphics{https://static01.graylady3jvrrxbe.onion/packages/flash/multimedia/ICONS/transparent.png}

\begin{itemize}
\tightlist
\item
  Zaf Qasim, Emergency and Critical Care Doctor
\item
  Philadelphia
\item
  April 18
\end{itemize}

I had a patient who on any other day I would have intubated. The patient
was sick, but not extremely so. I had a feeling they would deteriorate,
however.

I told them they would likely need a ventilator. There was fear in the
patient's eyes.

I opted instead to try something I had heard about from colleagues. I
put them on high flow oxygen and had them turn on their belly. I stayed
in the room, with full P.P.E., and sat down next to them.

They were scared. I was scared.

Was I doing the right thing? Would this work? I couldn't leave. I needed
to be sure they wouldn't get worse.

I talked to them about their family, their last few weeks and how the
world had changed.

They were struggling to breathe. I was holding my breath.

Over the next hour, I watched their oxygen levels improve, their heart
rate come down. I watched them relax. I lowered their oxygen.

``Will I still need a ventilator, Doc?'' they asked.

``Not today,'' I said.

I left the room, drenched in sweat, but smiling.

\includegraphics{https://static01.graylady3jvrrxbe.onion/packages/flash/multimedia/ICONS/transparent.png}

\begin{itemize}
\tightlist
\item
  Patricio Acosta, Virologist
\item
  Buenos Aires
\item
  June 30
\end{itemize}

Several health workers have been discriminated against, attacked or
abused by their neighbors in different cities of Argentina, thinking
they would bring coronavirus to the neighborhood.

My story is the other side of the coin: Some days ago, I found a poster
in the elevator of the building where I live with my wife and 2-year-old
son.

``Thank you for taking care of the littlest ones! Your Neighbors.''

When H1N1 came in 2009, I was still a doctoral student. The mortality of
that flu strain was particularly high in Argentina.

The challenge is to manipulate a novel virus that potentially can kill
you and not to be scared. What I have learned is to be respectful of
pathogens, but not to be afraid.

When you are too scared, you fail.

\includegraphics{https://static01.graylady3jvrrxbe.onion/packages/flash/multimedia/ICONS/transparent.png}

\begin{itemize}
\tightlist
\item
  Lee Kojanis, Oral and Maxillofacial Surgeon
\item
  Manhattan, N.Y.
\item
  July 7
\end{itemize}

Covid cases have dropped dramatically in North Jersey, allowing oral
surgeons and dentists in the area to slowly get back to ``normal.'' Our
patient volume is about 85 percent what it was last year at this time,
and we have managed to bring back all of the employees that were
furloughed. We have created strict Covid guidelines to ensure optimal
health for staff and patients. Unfortunately, the pandemic has led to
several dental office closures in our area. Many dentists in their 60s
and 70s have decided to retire early to avoid potential transmission of
the virus. We have also seen young clinicians struggle to develop and
grow as many patients are still apprehensive about receiving elective
dental care.

In April, I had to take a Covid-positive patient to the operating room
to drain a life-threatening infection that started in his tooth. The
patient's airway was shifted and partially blocked from the abscess in
his neck. An E.N.T. colleague was on standby just in case we had to
perform a tracheostomy in the event anesthesia couldn't intubate.
Fortunately, the intubation and surgery were successful and the patient
has since recovered.

I think it's important to remind our patients that maintaining their
dental health, even amid a pandemic, is essential to avoid potentially
devastating health consequences.

\includegraphics{https://static01.graylady3jvrrxbe.onion/packages/flash/multimedia/ICONS/transparent.png}

\begin{itemize}
\tightlist
\item
  Matheus Lopes, E.R. Doctor
\item
  Recife, Brazil
\item
  May 3
\end{itemize}

I'm a generalist working in a Covid-19 emergency department.

Eight months ago, I was celebrating my med school graduation. Last
night, I was receiving a 69-year-old patient gasping for air --- and
with no ventilators left because we already had two patients using the
ventilators available in my unit.

I live in Recife, in Pernambuco, one of the hardest-hit places in
Brazil. I was already infected and got back to work this week. I'm still
scared of the possibility of getting reinfected, or getting my brothers
reinfected, as they live with me.

No one prepared us for this. We are being asked to do intensive care
work with very little experience. I just turned 25 years old. I have
daily nightmares. I've lost more than eight pounds, and I've lost hope
in the Brazilian government.

Everything we are going through is making me more politically aware and
active. Every political move in this pandemic turned out to be crucial
to save or to lose lives. Medical school didn't teach me that.

\includegraphics{https://static01.graylady3jvrrxbe.onion/packages/flash/multimedia/ICONS/transparent.png}

\begin{itemize}
\tightlist
\item
  Gino Picano, Physician Assistant
\item
  Mount Kisco, N.Y.
\item
  April 8
\end{itemize}

My wife is pregnant with our first child. Initially, I was so scared
that I would put her and the baby at risk.

I felt conflicted for the first time in my life and thought, ``Is this
even worth it to go to work anymore?''

I've realized since then that it is worth it because our patients need
us. And my co-workers need me as much as I need them.

\includegraphics{https://static01.graylady3jvrrxbe.onion/packages/flash/multimedia/ICONS/transparent.png}

Kayana Szymczak for The New York Times

\begin{itemize}
\tightlist
\item
  Heather Geiger, Palliative Care Nurse
\item
  Boston
\item
  April 22
\end{itemize}

I want my patients and their families to feel really cared for. I ask
family if their loved one likes a certain kind of music. I hold their
hands, assure them, ``I'm here.'' I stroke their hair and talk to them.
All of which feels lacking without family present.

Limiting time with my patients feels unnatural to me, but necessary to
protect myself.

My anxiety has skyrocketed. My mind is constantly running and dissecting
every shift, each patient, every family interaction.

All of this causes a visceral heaviness that makes you ache and keeps
you awake at night. There is so much loss of control with Covid.
Self-isolation after work doesn't help.

I do have a wonderful support system, but most can't possibly understand
what we're seeing and doing. Honestly, I'm too emotionally exhausted to
try to explain it.

My co-workers are incredible. We are leaning heavily on one another. We
cry together. We're collectively grieving.

\includegraphics{https://static01.graylady3jvrrxbe.onion/packages/flash/multimedia/ICONS/transparent.png}

Heather Sten for The New York Times

\begin{itemize}
\tightlist
\item
  Rishi Chopra, Dermatologist
\item
  Brooklyn, N.Y.
\item
  June 10
\end{itemize}

It has been demoralizing. My grandfather passed away after a six-week
battle with Covid-19. My father, sister, brother-in-law and I are all
physicians on the front lines in New York City.

My sister and father both fell sick after treating patients. When my
father fell ill, I forced my mom to move out so that I could return home
to take care of him.

This hit our family incredibly hard and we will never be the same. We
isolated ourselves from one another because we are all so high-risk. The
worst part was facing this alone.

But we have moved past my grandfather's death and I am grateful that the
rest of my family has recovered from Covid-19 without long-term
repercussions. Everybody is overall happier and less stressed.

The contrast between now and two months ago is night and day.

\includegraphics{https://static01.graylady3jvrrxbe.onion/packages/flash/multimedia/ICONS/transparent.png}

\begin{itemize}
\tightlist
\item
  Claudine Aguilera, Internist
\item
  St. Augustine, Fla.
\item
  April 9
\end{itemize}

I treated the first known Covid-19 case in my county, on March 6. Only
we didn't know the patient had it until he had been at our hospital for
several days.

His symptoms were nothing like what we had been told to look for. He
mainly had gastrointestinal symptoms. As the days progressed, he
developed a fever and worsening symptoms. On a CT scan of his chest, we
noted a similar pattern to what we had been seeing on imaging from cases
in Wuhan.

So we tested him. We thought it was a long shot. But when the results
came back positive it sent shock waves through the hospital. I had been
exposed and was placed in quarantine. Thankfully I tested negative.

Suddenly our lives changed forever.

\includegraphics{https://static01.graylady3jvrrxbe.onion/packages/flash/multimedia/ICONS/transparent.png}

\begin{itemize}
\tightlist
\item
  Autumn Hicks, Radiation Therapist
\item
  Moncks Corner, S.C.
\item
  May 20
\end{itemize}

In the beginning, we were short on staff and needed to start screening
all patients at the front door. Without enough nurses on site, it was
suggested that the therapists screen patients.

As their clinical coordinator, I wanted them to stay focused on treating
the patients so I volunteered myself.

I moved my desk into the waiting room, transferred my phone line to the
lobby phone and began taking temperatures and doing a short health
screen of everyone that walked in the door.

I get random people that walk in asking to get tested for Covid-19,
searching for the ``vaccine,'' or just patients who are lost and looking
for their clinic.

It's a daunting task.

This pandemic has made me appreciate the innocence of my two beautiful
children. To not be able to hug them immediately when I get home from
work has been very challenging. I get home and go right into our camper,
sanitize my hands, put dirty scrubs and shoes into a bag, then don a
robe and flip-flops. I then walk into the house and head directly to the
shower.

After that I can grab and hug my family. Their support and love means so
much.

\includegraphics{https://static01.graylady3jvrrxbe.onion/packages/flash/multimedia/ICONS/transparent.png}

\begin{itemize}
\tightlist
\item
  Joe Berger, Acute Dialysis Nurse
\item
  Duluth, Minn.
\item
  July 24
\end{itemize}

In April, my company asked for volunteers from around the country to
help out in Chicago. When I received my hospital assignment, I decided
to look through their Facebook page. They were so low on gowns that they
had taken a donation of rain ponchos from the amusement park a town
over.

I brought a stash of gowns and gave away as many as I could.

My mom, both my sisters and I are all nurses. My brother is a pediatric
pathologist in suburban Houston, and my dad is a clinical psychologist
in Mission, Texas.

My sister's hospital in the Rio Grande Valley is overwhelmed.

We all text each other on a nearly daily basis. I've sent care packages
to my siblings and to some close friends in Texas. One of the benefits
of living in a small metro area is that things don't sell out as quickly
here, so I have wish lists from people and try to get Lysol and other
cleaning supplies out to them when they run out.

I've even mailed out P.P.E. that I had squirreled away, so everybody has
an emergency stash if it's hard to find again.

Normally, we don't exchange gifts in my family, but I think that's going
to change from now on.

\emph{Joe's sister,}
\emph{\href{https://www.nytimes3xbfgragh.onion/interactive/2020/world/coronavirus-health-care-workers.html\#item-sarah-berger}{Sarah
Berger}, worked on the front lines in McAllen, Texas. They come from a
family of medical workers.}

\includegraphics{https://static01.graylady3jvrrxbe.onion/packages/flash/multimedia/ICONS/transparent.png}

\begin{itemize}
\tightlist
\item
  Laura Oakes, Nurse Anesthetist
\item
  New Orleans
\item
  July 28
\end{itemize}

I live alone in New Orleans. I saw a drastic rise in my level of
anxiety. I would go to work to fight the good fight and return to an
empty home. It was an emotional roller coaster.

I have had friends and colleagues come in as my patients on mechanical
ventilators in the Covid I.C.U.

We have had a few employees pass away --- in the hospital --- from
Covid-19. It was heartbreaking.

We have also had several success stories, and the employees are now
fully recovered and back at work. In March, I had Covid-19 myself.

I have had people stop me while I'm wearing scrubs in public to question
if the virus is actually real.

I have also had a couple of patients' family members say that all health
care workers are working for the government, that it's all a conspiracy
theory and that we don't want the patients to actually get well.

That is a tough pill to swallow. I went into the medical field because I
am passionate about caring for others and nursing the sick.

\includegraphics{https://static01.graylady3jvrrxbe.onion/packages/flash/multimedia/ICONS/transparent.png}

\begin{itemize}
\tightlist
\item
  Enrique Boloña Gilbert, Critical Care Doctor
\item
  Guayaquil, Ecuador
\item
  April 30
\end{itemize}

When you work in intensive care, seeing people die on a daily basis
becomes part of your life. So very few things scare you. But I had never
felt what I have felt in the past few weeks: Afraid of getting myself
infected and putting my family at risk. I woke up fearing going to work.
I went to bed afraid. It drains you.

As head of the I.C.U., I have to supervise everything, every case.
Prioritizing, organizing, coordinating is very stressful. There was a
time when we had 64 Covid-19 patients, 34 of whom were intubated. We
have not had a single day off in the past six weeks.

Thank God, we never had to choose to remove ventilation from an old
person to give it to a young person. Yet, I had a 44-year-old patient
who died, and that hit me hard. I thought: ``This guy is my age, he was
pretty healthy. If I get infected, it could happen to me.'' At that
moment, I realized how fragile life is.

\includegraphics{https://static01.graylady3jvrrxbe.onion/packages/flash/multimedia/ICONS/transparent.png}

\begin{itemize}
\tightlist
\item
  Justin Sanborn, Physician Assistant
\item
  League City, Texas
\item
  April 9
\end{itemize}

Our pediatric emergency department was completely shut down and we
morphed it into the ``CV-19'' emergency room during this outbreak. We
have eight Covid-19 beds.

The numbers, at times, were staggering. We have to treat any patient
that presents with respiratory symptoms as a potential Covid-positive
person. I could hear our triage nurses yelling over a crowd and
directing patients to come see me. ``Cough? CV-19. Shortness of breath?
CV-19. Went to Louisiana? CV-19.''

You learn what I call ``chaotic'' efficiency.

\includegraphics{https://static01.graylady3jvrrxbe.onion/packages/flash/multimedia/ICONS/transparent.png}

\begin{itemize}
\tightlist
\item
  Italo M. Brown, E.R. Physician
\item
  Palo Alto, Calif.
\item
  June 11
\end{itemize}

At my hospital, I'm the only Black male physician in my department. One
of one. On most work days, I live a unique paradox --- the Black guy who
everyone knows and recognizes, but no one sees. Most of the people I
pass will not extend a warm smile or greeting in my direction.

Instead, they move by me like wavy apparitions, glancing down at their
phones or staring awkwardly at inanimate objects lining the path to
avoid direct eye contact.

It's OK, but it's not OK. Sadly, this is the culture of medicine for
Black people. Then coronavirus came.

Every hospital employee became radioactive. We were labeled essential
and given the title of ``Frontliner.'' Our closest confidants were
distanced from us, and common comforts were stripped from our lives
indefinitely. While community members sought refuge, we affixed masks to
our faces and stood in the gap. We were in this together, and for the
first time I felt comforted by this newfound unity.

Many touted the novel coronavirus as the great equalizer. We quickly
learned the opposite --- Covid-19 disproportionately affects Black
people. The spirit of togetherness may have swept through the walkways
of the hospital, but in reality I was isolated in plain sight. Once
again, I had to confront an age-old axiom: Regardless of the
circumstance, my Blackness will always make me high risk.

Weeks later, Covid cases plateaued and our embattled department started
to breathe fresh air again. But beneath my mask I still wear a melanated
covering. Will our shared front-line experience change the way I am
perceived --- by patients, by my colleagues, by society?

Maybe not enough to erase color lines completely, but perhaps enough to
smudge those lines so my presence is acknowledged.

\includegraphics{https://static01.graylady3jvrrxbe.onion/packages/flash/multimedia/ICONS/transparent.png}

Victor Moriyama for The New York Times

\begin{itemize}
\tightlist
\item
  Alyne Freiberg, General Practitioner
\item
  São Paulo, Brazil
\item
  June 8
\end{itemize}

I work in the center of downtown São Paulo, the largest metropolitan
area in South America. The area includes tenements and a large homeless
population. We care for patients with suspected Covid-19 in addition to
populations at risk, such as the elderly, children and pregnant women.
Many patients are more vulnerable to contagion, as they have whole
families sharing rooms with poor ventilation and little space for
isolation.

When carrying out home visits, I realize how much we cannot stop. More
than once during this pandemic I have visited a patient and when I
returned the next month, that door did not open anymore. I lost patients
who taught me so much in such a short time about resilience and
community spirit.

Next to our health unit, there was an important public community center
for children and teenagers in socially vulnerable situations. It
developed their self-esteem and skills, offering educational gardening,
drama classes, circus, dance, capoeira, sports and athletics.

A few weeks ago, due to a lack of funds, the place was closed
indefinitely. When schools return, this place will no longer exist. This
will mean more children and young people on the streets in a place where
violence and drugs are gaining strength. The pandemic is the current
problem, but its repercussions will last.

\includegraphics{https://static01.graylady3jvrrxbe.onion/packages/flash/multimedia/ICONS/transparent.png}

\begin{itemize}
\tightlist
\item
  Dawn Arroyo, Labor and Delivery Nurse
\item
  Visalia, Calif.
\item
  June 12
\end{itemize}

We have three known Covid-positive patients currently. One woman had
tested positive, then tested negative on her scheduled visit before
testing positive again two days later, when she was in active labor.

My daughter delivered her. She is a nurse on the unit with me.

Another patient had a mild fever when she was admitted. I spent seven
hours in the room wearing a plastic gown, an N95 mask, a face shield and
goggles. I think I sweated off 10 pounds.

I was trying to limit the amount of P.P.E. we burned through. We are
still reusing our N95 masks and our goggles.

The parents chose to isolate themselves from their baby.

It was the saddest thing after a difficult labor to hold the baby up so
she can see him and then take him straight to isolation.

That poor couple aren't going to see that baby for 14 days.

\includegraphics{https://static01.graylady3jvrrxbe.onion/packages/flash/multimedia/ICONS/transparent.png}

\begin{itemize}
\tightlist
\item
  Vanessa Hernandez, Nurse
\item
  Miami
\item
  April 9
\end{itemize}

In a code blue every second counts, right? We can no longer run into the
room. Ten minutes of suits, shoe covers, gowns, and three masks later,
what is left of the patient? Not enough.

We are not enough. We do not have enough. There are no heroes in a
pandemic.

This virus is breaking us.

\includegraphics{https://static01.graylady3jvrrxbe.onion/packages/flash/multimedia/ICONS/transparent.png}

Heather Sten for The New York Times

\begin{itemize}
\tightlist
\item
  Melissa Todice, Nurse
\item
  Shelton, Conn.
\item
  April 12
\end{itemize}

It kills me when I hear, ``Well, that's what they signed up for.'' I
never took a pandemic nursing class.

The biggest change is worrying about myself. Going into the rooms and
flinching when patients cough, even though we're covered head to toe.
Feeling guilty you flinched at all. Guilty that they can't see my face,
hear my voice, muffled under my face shield. Guilty that I can't hear
them. Then leaving them alone in that room.

The guilt has no beginning or end.

\includegraphics{https://static01.graylady3jvrrxbe.onion/packages/flash/multimedia/ICONS/transparent.png}

Samuel Aranda for The New York Times

\begin{itemize}
\tightlist
\item
  Núria Poveda, E.M.T.
\item
  Barcelona, Spain
\item
  April 28
\end{itemize}

In the first weeks, hospitals were on the verge of collapsing. We had to
assess the risk of taking patients to the hospital against the risk of
leaving them at home. That is a lot more responsibility than we have
ever had. Luckily we had doctors and nurses advising us.

On March 22 we went to pick up a woman, let's call her María. She was a
56-year-old with all the symptoms: dry cough, high fever and a feeling
of suffocation.

Her 72-year-old sister had died from Covid-19. Between short breaths,
she told me that taking her sister to the doctor was what exposed her to
the virus. She was afraid, and I was afraid for her. I didn't say
anything, but her vitals were bad.

When we got to the hospital, I said goodbye, wishing I could hug her.
María said, ``Take care of yourself, you are worth a lot.'' That María
used what little breath she had to praise us, broke me. I left the
hospital to cry for a long time.

Every day I remember María. I don't know what happened to her. The next
day, I was afraid to ask.

\includegraphics{https://static01.graylady3jvrrxbe.onion/packages/flash/multimedia/ICONS/transparent.png}

\begin{itemize}
\tightlist
\item
  Lysette Masendu Droh, General Practitioner
\item
  Abidjan, Ivory Coast
\item
  April 30
\end{itemize}

I work with Doctors Without Borders at the emergency project we recently
opened to support the national response to the Covid-19 pandemic.

The pandemic is now catching all the attention. But we should not forget
other diseases such as malaria, respiratory infections, chronic diseases
or malnutrition, as they kill every day in many African countries.
Yesterday, for instance, my neighbor brought his wife for a
consultation, as she had been diagnosed with diabetes three months ago.
But they were not able to meet any doctors because of the current
pandemic. She will have to wait for days before she can consult again.
It's very worrying. This situation also poses further risks to
vulnerable families who were already struggling to eat every day.

\includegraphics{https://static01.graylady3jvrrxbe.onion/packages/flash/multimedia/ICONS/transparent.png}

\begin{itemize}
\tightlist
\item
  Farhan Bashir, Resident Physician
\item
  Abbottabad, Pakistan
\item
  July 12
\end{itemize}

We got exposed to a patient on May 13. We didn't have proper P.P.E.
except for surgical masks and gloves.

I was on duty in my E.R. six days later when I started to have
generalized body aches with a mild sore throat. I had no fever but when
I woke up the next morning, I had severe body aches and throat
congestion.

I immediately isolated myself from the family.

For the first two days, I had a fever of 101 degrees.

Approximately a week later, my mother and younger sister developed
typical symptoms of Covid-19. Fortunately, they recovered and are now
safe and healthy.

I tested negative on June 11 and am back on duty in our Covid-19 I.C.U.

We don't have enough resources, but still our front-line health care
workers are fighting the pandemic with whatever resources are available
to them.

\includegraphics{https://static01.graylady3jvrrxbe.onion/packages/flash/multimedia/ICONS/transparent.png}

\begin{itemize}
\tightlist
\item
  Kim King-Smith, EKG Technician
\item
  Piscataway, N.J.
\item
  Died on March 31
\end{itemize}

``This thing, I'm so afraid of it,'' Kim King-Smith's cousin recalled
her saying when the first coronavirus cases started showing up at her
hospital.

Ms. King-Smith administered electrocardiograms during the overnight
shift, 7 p.m. to 7 a.m., with seven days on and seven off. She had a way
to make even the most agitated patient relax, and had friends in almost
every department of the hospital, said Yolanda Perrin, a technician who
worked alongside Ms. King-Smith for 16 years.

She was such a dedicated caregiver that she would check on the family
members of even casual acquaintances who were in the hospital, and made
daily visits to a cousin with multiple sclerosis when her shift was
over.

In March, she treated her first confirmed coronavirus patient, Ms.
Perrin said. She felt sick a few days later. After her test came back
positive, she isolated at home for 10 days, keeping in touch several
times a day with her mother, Alice Rose King, and her cousins. She had
Easter decorations in her window and was looking forward to recovering
in time for a big family holiday celebration.

``Just got out of shower. My temp is 103,'' Kim King-Smith texted her
cousin, Hassana Salaam-Rivers, on the last Saturday in March.

The next day, her breathing was so labored she had Lenny Smith, her
husband of 21 years, rush her to a hospital. Two days later, on March
31, she died. She was 53. Ms. King-Smith was the first staff member at
her hospital lost to complications of Covid-19, just as the virus was
taking hold in New Jersey.

--- KIM SEVERSON

\includegraphics{https://static01.graylady3jvrrxbe.onion/packages/flash/multimedia/ICONS/transparent.png}

\begin{itemize}
\tightlist
\item
  Charles Huschle, Chaplain
\item
  Tyngsborough, Mass.
\item
  April 9
\end{itemize}

``No, you cannot come in to see your husband right now,'' I have to say
to the crying woman on the other end of the phone. No visitors unless
the patient is imminently close to death. We switch to video chat so she
can see him, but he is so sick that he cannot respond.

Later, on another floor full of Covid patients, the nurse in charge is
literally wringing her hands. ``We can't do anything for them, and when
they die, we can't even give each other a hug,'' she says of her and her
staff.

The job description for chaplains is to support patients, families and
staff. Most of the time our focus is on patients and families. Now it's
more on staff. They need a lot of support. A lot of nurses and doctors
feel like they're invincible. But they are breaking down.

\includegraphics{https://static01.graylady3jvrrxbe.onion/packages/flash/multimedia/ICONS/transparent.png}

\begin{itemize}
\tightlist
\item
  Siya Dayal, Foundation Program Trainee
\item
  London
\item
  April 28
\end{itemize}

I recognized a name when I glanced at my patient list for the day. My
team had treated her extensively for an unrelated condition a few months
earlier.

I remembered everything about this lady, from her complex medical
history to her husband's favorite food.

She was not improving despite aggressive medical treatment.

Every day, I phoned her husband to update him on how his wife was doing.
I didn't allow myself to digest the hope I heard in this man's voice
despite some frank, heart-wrenching conversations.

On the last day, I said with a quivering voice that she didn't have much
time left.

It was already a few hours past the end of my shift and I still had a
dozen urgent jobs to complete, but when he asked me to put the phone to
his wife's ear, I seized protective gear, put on a pair of oversize
gloves and darted to my patient's bedside.

To respect his privacy, I disabled speakerphone, placed it on my
comatose patient's ear and gave them a few minutes.

I heard the last line.

``Just know that we love you.''

\includegraphics{https://static01.graylady3jvrrxbe.onion/packages/flash/multimedia/ICONS/transparent.png}

\begin{itemize}
\tightlist
\item
  Richelle Sipiora, Physical Therapist
\item
  Brunswick, Maine
\item
  July 7
\end{itemize}

We mask up and literally carry a disinfectant wipe in one hand as we
care for our patients. We had a recent scare when one patient came in
and tested positive and was symptomatic the next day. Her physical
therapist had to be tested and was negative. If the physical therapist
had been positive, we would have had to shut down for two weeks, no
rehab.

Our surgeons are backlogged by nearly 100 elective surgeries, so
catching up is a big concern. We are all experiencing Covid fatigue but
are concerned and outraged by the flippant way we see people behaving.

I live with a front-line nurse who does Covid testing. We play by the
rules. Needless to say, we have no visitors and have canceled all travel
plans. I tell each new patient I see that we must play by the rules or
we will be shut down. Each of us has a responsibility to stay healthy.

\includegraphics{https://static01.graylady3jvrrxbe.onion/packages/flash/multimedia/ICONS/transparent.png}

\begin{itemize}
\tightlist
\item
  Jacqueline Stapleton, I.C.U. Nurse
\item
  Houston
\item
  April 8
\end{itemize}

I decided to be a nurse after spending a year in Afghanistan at a
Level-3 trauma center. I was an Air Force medic and I was so impressed
with the nurses and the care they provided. I am currently working in a
dedicated Covid I.C.U.

Every week when I come to work I have to check in, take my temperature
and receive my two masks, which must last all week. It takes me two to
five minutes to put on all my P.P.E. when I must enter a patient's room,
which can seem like an eternity when my patient is quickly deteriorating
or coding.

I'm as diligent as possible about washing and sanitizing my hands, to
the point that they're dry and cracked and sore. I spend hours sweating
in my P.P.E. so I can care for my patients. Sometimes I eat lunch at 3,
or just don't have time to eat.

Finally, at the end of my shift I come home, where my two young children
meet me at the door. ``Mommy's home!'' They run to hug me and I tell
them to stop, not to touch me until I've showered and put my scrubs in
the wash. It breaks my heart to see the looks on their faces.

\includegraphics{https://static01.graylady3jvrrxbe.onion/packages/flash/multimedia/ICONS/transparent.png}

\begin{itemize}
\tightlist
\item
  Laura Bishop, Internist and Pediatrician
\item
  New Albany, Ind.
\item
  June 11
\end{itemize}

I see the evidence in epidemiologic statistics and, even more
shockingly, in flesh-and-blood human beings in front of me: Black men,
women and children are dying at a disproportionate rate from the day
they enter this world. This is not OK.

This is why I felt the urgency to join a group of local female
physicians to create packages for protesters with masks, sanitizer,
sunscreen and other supplies.

Some have questioned why those of us who have been touting the
importance of staying home and socially distancing are now going out of
our homes for what may appear to be a nonessential activity. My reply is
that this is an essential public health emergency. Leaving our Black
community out there protesting alone only further propagates the
discrepancies that Covid-19 has highlighted.

This is more important than going to a restaurant or getting a haircut.

\includegraphics{https://static01.graylady3jvrrxbe.onion/packages/flash/multimedia/ICONS/transparent.png}

\begin{itemize}
\tightlist
\item
  Neo Liu, Nurse
\item
  Wuhan, China
\item
  April 28
\end{itemize}

Before the epidemic, my work was very orderly and was largely the same
every day. At the end of December, we started hearing about an outbreak,
but because the government had still not publicly disclosed anything, we
really didn't know any details. Then, at the beginning of January, I
heard some colleagues say that the respiratory department was suddenly
having a lot of patients die. Dr. Li Wenliang {[}the whistleblower
doctor who died of the coronavirus and became a symbol of many Chinese
people's anger and frustration{]} was my co-worker. Even though I never
spoke to him, we worked in the same part of the hospital, and I
recognized him. When I first found out that he had tried to warn people
about the virus, and then been censured by police, I thought, ``Wow, is
this outbreak really serious?'' But at the time, the government was
saying there was no human-to-human transmission, so I relaxed a little.
When I saw the news that he had died, I was so surprised. He was so
young, how could this happen so suddenly? I couldn't sleep. I cried all
night.

Five doctors at our hospital died during this epidemic.

Because Wuhan was locked down, from the start of the epidemic until now,
I haven't had any chance to visit my hometown. I can only video chat
with my parents and tell them I'm OK, they shouldn't worry. Now that
Wuhan's lockdown has been released, I think in a few days I will go home
to see my parents.

\includegraphics{https://static01.graylady3jvrrxbe.onion/packages/flash/multimedia/ICONS/transparent.png}

Fabio Bucciarelli for The New York Times

\begin{itemize}
\tightlist
\item
  Claudio Del Monte, Chaplain
\item
  Bergamo, Italy
\item
  April 7
\end{itemize}

I go around dressed like the medical staff, with protective eyewear, a
good surgical mask and another wider one on top to protect it. I put a
cross on my disposable gown using a felt pen so that people understand
that I am the priest. I walk around holding a small cross and a small
bottle of sanitizer. There are already bottles in the halls, but there
are moments when a patient reaches out because they want to hold your
hand. Like doctors and nurses, I wear more than one set of gloves.

I've never gone into a room to pray with one patient where the other
patient didn't want me to say a prayer with them as well. If they tell
me that they are not Christian, there's always time for a greeting
because certainly the mystery of everyone's life is intangible.

During the most delicate moments, you are accompanying someone heading
toward the infinite and there is fear. So you tell them, ``Don't be
afraid. You are going toward the light. Have faith that God is with you.
Don't be afraid.''

Those are the things we whisper in their ears.

It's a medical situation, but it is also about the soul.

\emph{The Times}
\emph{\href{https://www.nytimes3xbfgragh.onion/2020/04/11/world/europe/italy-priests-coronavirus.html}{previously
wrote}} \emph{about Father Del Monte's work as a priest on the front
lines of the outbreak in Italy.}

\includegraphics{https://static01.graylady3jvrrxbe.onion/packages/flash/multimedia/ICONS/transparent.png}

\begin{itemize}
\tightlist
\item
  Hannah Hillebrand, I.C.U. Nurse
\item
  Portland, Ore.
\item
  April 21
\end{itemize}

It's a strange way to know someone. It's so intimate, yet you'll never
hear their voice. Their eyes will never meet yours.

You care for them in the most tender of ways. You hope that somehow,
despite the sedatives, they'll know you're there and your energy will
reach them.

When their family asks you to tell them one last time that they love
them, you hope your presence will bring the comfort their family cannot.

And somehow, there you stand, the last person to see them breathe, the
one with them as they transition to the great unknown.

And you hope you're worthy of the trust they had no choice but to put in
you.

\includegraphics{https://static01.graylady3jvrrxbe.onion/packages/flash/multimedia/ICONS/transparent.png}

\begin{itemize}
\tightlist
\item
  Rebecca Mahn, OB-GYN
\item
  Manhattan, N.Y.
\item
  April 23
\end{itemize}

Clinic ended with time to spare before my night shift on labor and
delivery. I missed the Good Friday liturgy broadcast and I was in search
of solace.

A woman mourned loudly, her headscarf falling down to her shoulders. She
had received the worst news.

I walked up a small staircase to the hospital chapel, tucked away above
the emergency department. The chapel door was bolted. ``Mass canceled
indefinitely. Individual prayer encouraged,'' a sign on the door read.

I ambled back. What would this Good Friday shift hold?

A nurse reported a new mom's fever and oxygen desaturation --- telltale
signs.

My heart fell to the floor. Please, not her.

Earlier that day, she delivered a child with terminal illness, who was
given only a few days to live. Immediately after delivery, nurses
whisked him away on life support. She still had not met him. Now, with
the Covid diagnosis, she would be barred from the N.I.C.U.

She begged, she cried.

An OB-GYN co-resident visited us from her shift in the intensive care
unit. She automatically stood at the opposite end of the workroom,
keeping distance.

``How's it up there?'' I asked.

Tonight has been OK, she said, but that's how last night started. Then
just before going home, the healthiest patient coded and slipped away.

Our gazes shifted uneasily around the room.

The night crept on and a bleak dawn came.

When I arrived home, my 4-month-old daughter slept. I paused for a
moment at the crib. Her steady, shallow breathing softened my shoulders.

I scooped her into my arms and breathed in fresh life.

\includegraphics{https://static01.graylady3jvrrxbe.onion/packages/flash/multimedia/ICONS/transparent.png}

\begin{itemize}
\tightlist
\item
  Stephen Berns, Palliative Care Doctor
\item
  Burlington, Vt.
\item
  April 20
\end{itemize}

I am a physician who specializes in palliative care. Right at the
beginning of this whole Covid crisis, I was caring for a patient with
metastatic cancer who didn't have Covid, but because of Covid, he was
robbed of his chance to say goodbye to his wife.

He was actively dying, and his wife desperately wanted to say goodbye.
They had been together for 40 years. She was immunocompromised so her
physician recommended that she not come in, and she was so torn. Do I go
in, do I get sick? Do I not? How do I say goodbye?

In the end, it was me sticking the phone to his ear while she spoke: I
wish I was there, I love you so much, forgive me for the past
heartaches, thank you for being a good husband and good father. I was
hearing this and it was like I was this conduit between them. The nurse
was at the bedside and both of us were brought to tears.

Since then, I have had many of those kinds of goodbyes with
Covid-positive patients.

When his wife was done, she asked, ``Did he hear it?''

\includegraphics{https://static01.graylady3jvrrxbe.onion/packages/flash/multimedia/ICONS/transparent.png}

\begin{itemize}
\tightlist
\item
  Angélica Carolina Olivares Cuellar, Nurse
\item
  Reynosa, Mexico
\item
  June 30
\end{itemize}

Government hospitals are at capacity, and navigating this situation is
complicated. Not knowing what awaits us day after day is a challenge.
There is always a sense of uncertainty about tomorrow.

What worries me the most about this pandemic is that my family is also
exposed to the virus. I wouldn't want to see any of them go through
this.

The situation is out of my control, and that brings me a lot of anxiety.

\includegraphics{https://static01.graylady3jvrrxbe.onion/packages/flash/multimedia/ICONS/transparent.png}

\begin{itemize}
\tightlist
\item
  Becki Rwubusisi, Infectious Disease Nurse Practitioner
\item
  Denver
\item
  April 10
\end{itemize}

I think about my sickest patients at night. I wonder if they'll be alive
when I come to work the next day. I pray for them.

Some nights I dream Covid nightmares all night. I'm scared of getting it
myself. I'm scared of having to treat a friend or colleague. I'm scared
that this will continue to overwhelm our health care system and we will
have to make choices no one should have to make. I'm trying to focus on
the next patient, the next hour and, sometimes, the next minute. I try
to remind myself that this is not the end of the story.

\includegraphics{https://static01.graylady3jvrrxbe.onion/packages/flash/multimedia/ICONS/transparent.png}

\begin{itemize}
\tightlist
\item
  Paul M. Shaniuk, Internist
\item
  Lyndhurst, Ohio
\item
  April 9
\end{itemize}

I typically am a teaching physician who spends much of my clinical time
working with medical students and medical residents. This has almost
totally disappeared, as I have shifted to providing direct care.

The most intense experiences have been internal as I battle my own fear.
Will I bring Covid-19 back home to my wife, my infant son and two older
children? I have learned to decontaminate myself after coming home from
work, learned to focus on hand hygiene like my life depended on it and
learned how to show compassion, empathy and care while wearing P.P.E.
that makes me look like a character from a science fiction novel.

Fear hangs over the medical center like a cloud, but we care for one
another and for our patients. We move forward, rising to the challenge.

\includegraphics{https://static01.graylady3jvrrxbe.onion/packages/flash/multimedia/ICONS/transparent.png}

\begin{itemize}
\tightlist
\item
  Joshua Schwarzbaum, E.R. Doctor
\item
  Jersey City, N.J.
\item
  April 20
\end{itemize}

It looked like a new emergency room. A place I had never been with
people I could not recognize behind all of their protective gear. But I
was there, directing the care of my patients and teaching valuable
lessons to my residents. I took a long exhale, realizing that we can
only inhale so much. At some point we need to let it out.

It was there, at the end of that exhalation, in the stillness of that
moment, that I felt blessed to be able to do what I do. I had an
unbelievable sense of clarity through the doubt and fear that surrounded
me. I was exactly where I needed to be.

\includegraphics{https://static01.graylady3jvrrxbe.onion/packages/flash/multimedia/ICONS/transparent.png}

Alessandro Falco for The New York Times

\begin{itemize}
\tightlist
\item
  Ana Ruth Santana, Nursing Technician
\item
  Belém do Pará, Brazil
\item
  June 29
\end{itemize}

My hair used to be long. I had to cut it very short to be able to wash
it quickly. I didn't want to bring the virus home.

I used to isolate myself in a room after work. I spoke very little with
my sons, mostly via cellphone, in the same house, just to avoid contact
as much as possible.

We worked for 24 hours straight, every other day, in April. We had to
deal with something we had never seen before. It seemed like a
battlefield.

There were desperate moments. I think I suffered from depression. I used
to cry every day after seeing so may people dying --- friends,
colleagues. I think 90 percent of my colleagues have been infected by
the virus. In this hospital, 19 workers died, including doctors, nurses
and technicians.

I got sick on April 20 and was sent home. My condition deteriorated
rapidly.

Other colleagues were hospitalized together with me in the infirmary. I
would never have imagined that I would be hospitalized in the same place
I work.

I have been back to work for almost two weeks.

I still feel pain all over my body. One lung still hurts, and when I
make a lot of physical effort I feel very tired. The same tiredness as
when I got sick, that lack of air.

I am no longer as agile as I used to be before the virus.

\includegraphics{https://static01.graylady3jvrrxbe.onion/packages/flash/multimedia/ICONS/transparent.png}

\begin{itemize}
\tightlist
\item
  Timothy Wong, Infectious Diseases Doctor
\item
  Singapore
\item
  June 12
\end{itemize}

We have had different waves of infections in Singapore: Chinese
tourists, Singaporeans who had returned from overseas and then migrant
workers. Each required a slightly different approach, which meant that
our protocols had to adjust quickly after getting feedback from the
ground.

We now know some phrases from Tamil or Bengali to reassure our patients
better, which is of crucial importance when dealing with the
uncertainties of a new virus.

Everyone has had to make sacrifices, but it has become increasingly
clear that this fight will be a marathon.

When Singapore announced its first confirmed case, there were so many
questions, underpinned by worry and uncertainty especially among the
younger doctors. During a department meeting, a SARS veteran spoke up
with five simple and reassuring words: ``We will get through this.''

\includegraphics{https://static01.graylady3jvrrxbe.onion/packages/flash/multimedia/ICONS/transparent.png}

\begin{itemize}
\tightlist
\item
  Sahrish Ilyas, Internist
\item
  Detroit
\item
  April 8
\end{itemize}

As my team of residents made our way toward the hospital's Covid-19
ward, the floor was littered with yellow gowns. A team of intensivists
had just rushed out after evaluating patients at the brink of
deterioration. A crowd of anesthesiologists huddled inside another
patient's room, surely preparing an intubation. That day, I realized the
magnitude of this disease.

As I entered an elderly gentleman's room, I could tell he was not
breathing well. I told him we would fight this together but felt I was
largely providing false hope. He was terrified. I wiped his tears. He
was intubated later that evening. I had many such conversations with my
patients that day. None of them cried like him.

I got a call from one of my sisters, a physician on the front line in
New York City. We would check in with each other daily and share our
experiences. This time it was the call I dreaded, as she told me that
she had fallen ill and tested positive for coronavirus. I worried for my
entire family --- they are all health care providers --- and primarily
for my 3-year-old son, who I come home to every night.

A few days later, my elderly male patient successfully came off the
ventilator. I went to see him and burst into tears, fogging up my safety
goggles. He looked at me and prayed and reassured me that things would
be OK, even though I should have been the one comforting him. My sister
was also on her way to recovery. Although the last few weeks have been
filled with anguish, these small moments preserve hope.

\includegraphics{https://static01.graylady3jvrrxbe.onion/packages/flash/multimedia/ICONS/transparent.png}

\begin{itemize}
\tightlist
\item
  Jennie Sablayan, Nurse
\item
  London
\item
  Died on May 5
\end{itemize}

``I fear for my life and the life of all my colleagues who are in the
battlefield without the proper gear,'' Jennie Sablayan wrote on Facebook
on April 5. ``We are like an army going to a battle without the complete
armour.'' Ms. Sablayan was proud to work for Britain's National Health
Service, but worried that its protocols for protective equipment were
insufficient for the coronavirus pandemic.

Ms. Sablayan, who was 44, had studied nursing in the Philippines and in
2002, joined the University College London Hospitals. ``We will miss her
terribly, her humor, her compassion, her friendship and her humbleness
in supporting her team and her patients,'' said the hospital's chief
executive, Marcel Levi.

In addition to being a nurse, Ms. Sablayan worked with an agency that
recruited nurses and placed them in hospitals. There, she emphasized to
employees who worked in finance and other nonclinical departments that
they helped care for patients by caring for nurses. If payroll messed
up, for example, nurses went to work frustrated. If paperwork was
mishandled, a nurse had to spend time fixing that rather than healing
patients.

``She found a way to show everyone how their role does directly impact a
patient's life,'' said the company's director, Fahim Modak. ``She wanted
to help wherever she could.''

As the outbreak grew, Ms. Sablayan was frustrated to see so many people
flouting stay-at-home guidelines. ``These people are truly upsetting,''
she wrote in early April. ``This defiance will add on to the number of
people that we are risking our own lives for.''

Weeks later, she tested positive for the virus and started having
trouble breathing. She was eventually placed on a ventilator, and her
condition deteriorated.

Ms. Sablayan is survived by two daughters, 10 and 14, and a husband who
is an I.C.U. nurse.

--- AIDAN GARDINER

\includegraphics{https://static01.graylady3jvrrxbe.onion/packages/flash/multimedia/ICONS/transparent.png}

\begin{itemize}
\tightlist
\item
  Courtney Bell, E.M.T.
\item
  Tampa, Fla.
\item
  June 9
\end{itemize}

It feels like Black people in America are dealing with two public health
crises at the moment.

Not only do Black people have to worry about dying from the virus and
the wide health care disparities within the medical field, but we have
to worry about police brutality and white supremacy as well.

Can you imagine how exhausting it is to fight both battles?

I've had numerous encounters with patients that I suspected of having
coronavirus. In E.M.S., you do what you can in the moment.

I can't help to create a vaccine, but I can dedicate time and effort to
the Black Lives Matter movement.

I haven't gone to a protest yet for multiple reasons, all of which stem
from fear. I'm scared of getting pepper-sprayed, arrested and
potentially risking my acceptance into physician assistant school.

But I have started a group called Black Life Support with medics, nurses
and others working to provide first aid to peaceful protesters in Tampa
Bay.

I've garnered a solid 15 people together. All that's left is setting up
an account for monetary donations, buying medical supplies, other small
details and connecting with protest leaders so that we know where we're
needed.

\includegraphics{https://static01.graylady3jvrrxbe.onion/packages/flash/multimedia/ICONS/transparent.png}

\begin{itemize}
\tightlist
\item
  David Xavier Menéndez Espinoza, Family Doctor
\item
  Portoviejo, Ecuador
\item
  July 1
\end{itemize}

I had a busy shift with five Covid-19 patients. I don't know what
happened, but at some point the safety barrier broke and I was infected.

My oxygen levels were low. I was short of breath. My whole body ached. I
was weak and didn't eat well for five days.

There were some nights where I was scared of dying.

I'm lucky to have an amazing wife who took care of me every night. She
would prepare me hot teas. This isn't scientific but when it's made with
love it's the best medicine.

Now, I'm working in a rural part of my province. The government wanted
to increase the ability to treat mild cases at home, so that's what I'm
doing.

I've seen about 39 patients. Twelve required oxygen and unfortunately
one woman passed away.

The difference is so apparent between the treatment for a patient that
has means and people who don't. In my city, a tank of oxygen costs about
\$600. A lot of patients have not been able to buy it and by the time
they need it, they could require hospitalization.

The situation at the hospitals was under control a few weeks ago, but
now there's been a spike in cases again and all the I.C.U. beds are full
again in Portoviejo.

\includegraphics{https://static01.graylady3jvrrxbe.onion/packages/flash/multimedia/ICONS/transparent.png}

Nadège Mazars for The New York Times

\begin{itemize}
\tightlist
\item
  Gabriel Herrera, Surgical Oncologist
\item
  Bogotá, Colombia
\item
  June 24
\end{itemize}

Our wave is accelerating in Colombia. We are backup for front-line
providers, but we have kept critical, time-sensitive surgeries going ---
for example, gastric cancer surgery --- while we don't exceed critical
capacity or run down in P.P.E.

As a surgical oncologist, a difficult but frequent part of my day-to-day
is to deliver bad news. We now have to deliver life-altering news
through screens or masks, layers and layers of gear that distance us
from our patients and deny one of the most basic and important
behaviors: comfort and company in times of need.

Since the pandemic began, a close family member of mine was found to
have cancer. As a family, we are grateful that our doctors have been
able to assess and guide us via telemedicine. We have also felt the
emptiness firsthand of receiving such a diagnosis through a computer
screen behind face masks, looking for hope when everything feels so
hopeless.

This pandemic has made me especially aware of the importance of the most
human aspects of being a doctor.

\includegraphics{https://static01.graylady3jvrrxbe.onion/packages/flash/multimedia/ICONS/transparent.png}

\begin{itemize}
\tightlist
\item
  Pria Anand, Neurologist
\item
  Boston
\item
  April 9
\end{itemize}

The physical examination has always seemed to me like a remarkable magic
trick, equal parts deduction and ritual. Of all of the insidious ways
Covid-19 has changed how I practice medicine, this is the one I least
expected: The physical exam no longer feels like essential magic;
instead, it often feels like an unmitigated risk. The best I can do for
my fragile patients is often to keep them at arm's length, or further
--- away from the hospital, away from the exam room and away from me.

In the hospital, I pause in the doorway of every patient's room,
wondering whether this particular exam is worth the air we'll be
sharing.

\includegraphics{https://static01.graylady3jvrrxbe.onion/packages/flash/multimedia/ICONS/transparent.png}

\begin{itemize}
\tightlist
\item
  Jo Murphy, Interfaith Chaplain
\item
  Boston
\item
  April 23
\end{itemize}

This evening, I was called for a Jewish patient who was dying of
Covid-19. Her partner was allowed to visit her for an hour and requested
a Jewish end-of-life prayer.

I came up to the floor with a translation of the Viddui, the Jewish
deathbed confessional, in hand. I couldn't enter the room so the
patient's partner was called to the door.

He could hear me if I spoke into the crack where the door meets the
wall. He put his ear up to the crack and I recited the Viddui on the
patient's behalf.

On my amen, he waited and then backed away from the door, looking at me
tearfully.

He then went back to his partner and I placed my hand on the door,
holding on for a brief moment, before I left.

I try to fight the aloneness with everything I've got.

\includegraphics{https://static01.graylady3jvrrxbe.onion/packages/flash/multimedia/ICONS/transparent.png}

\begin{itemize}
\tightlist
\item
  Eyal Kedar, Rheumatologist
\item
  Potsdam, N.Y.
\item
  April 9
\end{itemize}

The Covid-19 pandemic has strengthened my commitment to living and
working in rural America. Rural areas have roughly one-ninth the number
of medical sub-specialists per capita as urban areas. So in the
pandemic, many underfunded rural hospitals not only lack the beds,
ventilators, testing capacity and personal protective equipment to treat
patients --- they also lack the know-how.

I am the sole rheumatologist for St. Lawrence County, the largest county
in New York State, and several surrounding rural counties. With the
pandemic, I've been unexpectedly repurposed as a Covid-19 specialist.

Though having a solo practice can at times be lonely, I have long been
inspired by the challenge of delivering care to an underserved rural
community. The pandemic has only deepened this sense of mission, and if
there is any small silver lining for me in this ongoing nightmare, this
is it.

\includegraphics{https://static01.graylady3jvrrxbe.onion/packages/flash/multimedia/ICONS/transparent.png}

\begin{itemize}
\tightlist
\item
  Valerie Vaughn, Internist
\item
  Ann Arbor, Mich.
\item
  April 17
\end{itemize}

I'll always remember the first patient I lost to Covid-19. He died
alone, away from the many people who loved him. When I examined him that
morning, as a stranger covered in masks and gloves and gowns, I had
sought the words to convey their love on their behalf. I gave him
updates about his kids, his wife, the neighbors. I told him they missed
him and wished they could be at his side. Sedated and dying, he did not
answer me.

Later that afternoon, knowing the end was near, I watched him through
the glass that separated us. I watched his chest rise as the ventilator
pushed air into his lungs. I memorized his face. I wanted to remember
him forever. In that moment, it was all I could do.

\includegraphics{https://static01.graylady3jvrrxbe.onion/packages/flash/multimedia/ICONS/transparent.png}

\begin{itemize}
\tightlist
\item
  Meg Mueller, E.R. Doctor
\item
  South Kingstown, R.I.
\item
  July 11
\end{itemize}

On an overnight shift in the beginning, I was the only doctor on. Most
patients were triaged to the ``Covid-likely'' zone.

I went back and forth between two particularly sick patients who both
had pneumonia that looked like Covid-19. I was loath to intubate but I
feared they would die if I did not.

That's when a third patient came in dying. There was nothing I could do
but tell the family and then rush back to my patients who were
struggling to breathe and to the 10 others who were less sick but now
cranky due to their prolonged waiting times.

I ended up intubating one of my two critically ill Covid patients. They
were both admitted to the hospital and died days later.

Fortunately, cases have gone down significantly where I work. But we are
seeing a much higher volume of patients over all than a few months ago,
now that people aren't as scared to come into the hospital.

We also have had an influx of psychiatric patients. Today, I started my
shift with 18 psychiatric patients signed out to me. Each had been
deemed unsafe for discharge.

I manage these patients, keeping them safe and hopefully treating their
underlying mental health issues for my entire shift, while trying to see
every other patient coming into the emergency department. I am a board
certified emergency physician, not a psychiatrist.

\includegraphics{https://static01.graylady3jvrrxbe.onion/packages/flash/multimedia/ICONS/transparent.png}

Meghan Dhaliwal for The New York Times

\begin{itemize}
\tightlist
\item
  Susana de Anda, E.R. Nurse
\item
  Mexico City
\item
  June 9
\end{itemize}

Our district, Iztapalapa, is considered one of the most dangerous in
Mexico. The hospital where I work is the biggest in the area and we are
used to getting all kinds of trauma patients: gunshot and stabbing
victims, suicides.

We are used to treating critical patients in big numbers, to working
excessively, to being short of personnel and equipment. We have been
used to all of that because that is how you work in Mexico.

The difference with this pandemic is that you lose five patients in a
single shift. That leaves you emotionally shattered.

Unfortunately, here in Iztapalapa, people didn't believe in the virus.
People thought it was some invention by the government and so they
didn't follow any of the social distancing measures. In a blink of an
eye, our E.R. became saturated and we had the highest number of cases in
the country.

Working here means being more worried about getting infected outside the
hospital than inside.

When I finished a shift to go home, I saw people walking on the streets
carelessly --- no face masks, no social distance, not a worry in the
world. In those moments, all I could think of was the patients lying in
their beds who had told me: ``God forgive me because I didn't believe in
any of it. Now I see how wrong I was.''

What I feared so much finally happened. I got infected. I spend nights
without sleeping asking God not to let me end up in a Covid hospital
ward.

\includegraphics{https://static01.graylady3jvrrxbe.onion/packages/flash/multimedia/ICONS/transparent.png}

\begin{itemize}
\tightlist
\item
  Amanda Ramalho, Nurse
\item
  Pelotas, Brazil
\item
  April 13
\end{itemize}

I am afraid! I never thought I'd be living in this warlike situation.
Still, I'm happy to be of help and I take a lot of care not to be
contaminated.

The hospital provides us the P.P.E., but I buy more with my own money,
like some glasses that were more comfortable than the standard ones
provided.

Since we're wearing the P.P.E., we have to be conscious of going to the
bathroom. Sometimes I skip the bathroom altogether.

I first tested a patient for Covid on March 12, and I haven't hugged
anyone since then. I went to see my family once but didn't get out of
the car.

I live by myself. I miss my family and friends a lot.

\includegraphics{https://static01.graylady3jvrrxbe.onion/packages/flash/multimedia/ICONS/transparent.png}

\begin{itemize}
\tightlist
\item
  Abdul Ahad Amiri, Internist
\item
  Kunduz City, Afghanistan
\item
  April 25
\end{itemize}

When I was a child, it was my dream to become a doctor. In our village
in Imam Sahib District, my father had a pharmacy. There was one medical
doctor who attended to everything --- if someone had a headache or a
sore throat or an eye problem or needed surgery, just that one doctor
would have to examine them.

When the coronavirus epidemic started, the public health ministry
announced that they needed volunteers. If I stayed home, it would weigh
on my conscience. I couldn't sleep well while people were in trouble. So
I joined my home province's Covid-19 hospital. It was time to go to
work, although it was a dangerous task. If we make a single mistake, we
may infect many people, including our family.

Other countries have better medical facilities and they still couldn't
control the crisis. Test results take 10 days. We don't have
ventilators, we do not have any oxygen machines. If the situation
continues like this in Afghanistan, it will be very difficult for us to
control it. The number of positive cases is increasing day by day and
the situation in Afghanistan is deteriorating.

I work 24 hours, then I have 48 hours off. But I have to spend those 48
hours in one room at home so that my family doesn't get sick.

\includegraphics{https://static01.graylady3jvrrxbe.onion/packages/flash/multimedia/ICONS/transparent.png}

Gulshan Khan for The New York Times

\begin{itemize}
\tightlist
\item
  Nabeela Arbee-Kalidas, Anesthesiology Resident
\item
  Johannesburg
\item
  April 9
\end{itemize}

As a doctor in South Africa, I'd never been trained to put myself before
my patients. I have taken antiretroviral drugs multiple times after
being pricked with an H.I.V.-positive needle. I have worked in
tuberculosis wards without proper protective equipment. That's just a
normal day at work. Now I have to delay helping my patients to help
myself first.

The hardest experience was rushing to put on P.P.E. while my patient was
dying in front of my eyes. I am living with guilt and sadness that I
couldn't help in time.

I am learning about a completely new virus and constantly reading new
guidelines. I am working stressful shifts covered in P.P.E. resulting in
extreme headaches, facial pain and dehydration. I don't feel like I can
help my patients in the way they deserve to be helped. I'm trying to be
a positive wife, daughter and sister, and to allay my family's
anxieties.

I am learning how to serve my people but protect myself and my loved
ones at the same time.

\includegraphics{https://static01.graylady3jvrrxbe.onion/packages/flash/multimedia/ICONS/transparent.png}

\begin{itemize}
\tightlist
\item
  H. Joyce Morano, Geriatric and Internal Medicine Physician
\item
  Dallas, Pa.
\item
  July 7
\end{itemize}

March and most of April were filled with rampant Covid-19 infections and
deaths among my elderly patients in two local nursing homes.

My patient population in the local nursing homes was decimated.

It was so heartbreaking because there was no way to get these patients
out of harm's way.

It's devastating having to talk to family members when they first find
out that their loved one is positive for Covid-19. They feel guilty for
putting their loved one in the nursing home in the first place.

There are still a few patients remaining who either managed to survive
their Covid-19 infection or, for unknown reasons, did not get infected,
no thanks to anything I was able to do.

\includegraphics{https://static01.graylady3jvrrxbe.onion/packages/flash/multimedia/ICONS/transparent.png}

\begin{itemize}
\tightlist
\item
  Torie Jones, E.R. Nurse
\item
  Jersey City, N.J.
\item
  April 9
\end{itemize}

It was a Thursday evening. I had worked the night before and endured yet
another terrible shift. It seemed that everyone that night was crashing
and needed an I.C.U. bed, and we had none.

I got home, showered the Covid off and went to sleep. It was filled with
nightmares about dying patients. I woke up particularly groggy and
dreading the shift ahead of me.

I was dragging my feet as I walked the three blocks to the hospital from
the Columbus Circle station in Manhattan. As I was about to turn the
corner onto 59th Street, I heard a voice whooping from high above me.

I looked up and saw people on their balconies and sticking their heads
out of their windows, all clapping and cheering. People were ringing
bells and clanging metal bowls. People were waving from their windows.

Back down around me on the street, pedestrians stopped to clap and
cheer. I felt astounded, recognized and appreciated. Most importantly I
felt a rush of confidence to enter another shift.

Now, if I see one of my colleagues a little down, I tell them, ``Go
outside for the cheer. I'll cover you. It will honestly make you feel
better.''

\includegraphics{https://static01.graylady3jvrrxbe.onion/packages/flash/multimedia/ICONS/transparent.png}

\begin{itemize}
\tightlist
\item
  Frank Gabrin, E.R. Doctor
\item
  Manhattan, N.Y.
\item
  Died on March 31
\end{itemize}

``Something bad happened to me,'' Frank Gabrin told his husband in a
late-March phone call. ``A patient with coronavirus just passed away in
my arms. I'm feeling so desperate. I'm feeling so scared.''

Dr. Gabrin, who was 60, had been reusing an N95 mask for several shifts,
he told friends. It wasn't long before he felt aches and other symptoms.
But they were manageable, he texted a friend from quarantine, and he
thought he could beat the virus as he had beat cancer and other medical
issues.

But on March 31, Dr. Gabrin woke his husband, Angel Vargas, to say he
felt like he was dying and needed help. Mr. Vargas called 911 but it was
too late. ``He passed away in my arms,'' Mr. Vargas said.

For decades, Dr. Gabrin had worked the night shift in various emergency
rooms. He began after leaving the Navy as a lieutenant commander. He
relished the chance to help people, though it could be difficult work.

In 2008, Dr. Gabrin was choked badly by a psychiatric patient whose
incoherent request for help he'd initially dismissed. The incident made
Dr. Gabrin rethink his approach to medicine. ``He realized very clearly
that I didn't care about him,''
\href{https://www.youtube.com/watch?v=6IxPRKdFvxk}{Dr. Gabrin later said
of the patient}. ``Twenty years into a career in medicine, I realized I
was not clear about what `care' was.''

His practice needed more compassion, he concluded. Everyone's did. He
believed compassion was the antidote to the burnout he saw in himself
and his peers.

``He stepped into people's pain, not away from it,'' said his longtime
friend Debra Lyons. ``That's what he taught people.''

Dr. Gabrin wrote and self-published three books, led workshops and
cooked large pots of soups and chili to share at work.

Still, as coronavirus cases took off, he began to feel emotionally
exhausted. ``Everyone wants a Covid test that I do not have to give them
so they are angry and disappointed,'' he texted a friend.

He knew his colleagues felt the same, so he tried to rally them. ``This
is an exceptionally challenging time for all of us,'' he emailed them on
what would turn out to be the night before he died. ``With a limited
number of tools to fight this virus, I would like to remind us that we
do have other tools at our disposal.''

He listed them: decency, empathy and compassion.

--- AIDAN GARDINER

\includegraphics{https://static01.graylady3jvrrxbe.onion/packages/flash/multimedia/ICONS/transparent.png}

\begin{itemize}
\tightlist
\item
  Michael Slater, E.R. Doctor
\item
  Evanston, Ill.
\item
  April 9
\end{itemize}

I've been in practice for 21 years. The uncertainty that Covid-19
presents is making me feel like a young resident again. I don't have the
sense yet for which patients are likely to do OK and can be sent home
versus who needs to be admitted to the hospital.

I feel similar to how I did in the weeks after 9/11, but this enemy
feels more nefarious. I am scared and stressed. I worry that I might get
sick, that my wife --- a pediatrician --- might get sick, that we could
get our kids sick.

I have also never felt prouder to be a physician.

I was asked to join the team of clinicians, chaplains, and
administrators to help make decisions on ``resource allocation'' ---
which patient gets the last ventilator or last dose of medication when
we run out. I've taken a crash refresher course in medical ethics. I
hope not to have to be called.

\includegraphics{https://static01.graylady3jvrrxbe.onion/packages/flash/multimedia/ICONS/transparent.png}

\begin{itemize}
\tightlist
\item
  Leonardo Bianchi, Cardiologist
\item
  São Paulo, Brazil
\item
  June 25
\end{itemize}

We have seen a spike in the number of cases and the death toll is
rising, unfortunately.

The amount of work has more than doubled. Many colleagues are now off
duty because they are sick with Covid-19 and I need to do extra shifts.

But we keep fighting.

When a patient is discharged from I.C.U., our strength is refueled and
we have the stamina to move forward.

\includegraphics{https://static01.graylady3jvrrxbe.onion/packages/flash/multimedia/ICONS/transparent.png}

\begin{itemize}
\tightlist
\item
  Eva Gelernt, Nursing Technician
\item
  Manhattan, N.Y.
\item
  May 19
\end{itemize}

When my school said some of the hospitals in New York City needed help,
I and a group of classmates moved back from all over the country where
we had been quarantining with family.

We went in blind. Nurses were telling me stories, but it seemed crazy to
think it was really that bad.

It is.

A 12-year-old died on one of my shifts.

I watched a corpse being wheeled out on another.

I sat with a 25-year-old on a few shifts. He had been intubated for more
than a month and then was transferred to my hospital. He asked if I had
any free time on my shift to sit with him. He hadn't had a visitor in
more than two months.

I come home at 8 a.m., strip at the door, throw my scrubs into a trash
bag, sear myself in the shower, take a sleeping pill to shut off my head
and sleep as long as I can. And then I do it again.

I'm so sad so much of the time. The other times, I am angry. There are
days I admit I don't think this field was the right choice for me
because this is so much harder than I thought it would be.

But then I remember that the 25-year-old was discharged to rehab. I
remember the night nurses ecstatically checking the number of patients
in the emergency department and seeing fewer than 10 admissions. I
remember the patients who got better and God, does that make me happy.

I imagine we'll all carry this with us for a long time.

\includegraphics{https://static01.graylady3jvrrxbe.onion/packages/flash/multimedia/ICONS/transparent.png}

\begin{itemize}
\tightlist
\item
  Lois Olney, Nurse
\item
  Lancaster, Pa.
\item
  April 9
\end{itemize}

FEMA told the nursing home they had to have their first confirmed case
of Covid-19 before sending P.P.E. In the meantime, we are all exposed.

I take the time --- which I rarely have --- to help the residents
connect to their families via my cellphone, using FaceTime or on speaker
mode. The residents are mostly confused and hard of hearing, making it
virtually impossible for them to understand why their families cannot
visit during this crisis.

\includegraphics{https://static01.graylady3jvrrxbe.onion/packages/flash/multimedia/ICONS/transparent.png}

\begin{itemize}
\tightlist
\item
  Keiji Oi, Vascular Surgeon
\item
  Tokyo
\item
  May 4
\end{itemize}

The hospital isn't doing the heart surgeries that were planned. Now I'm
focused on hooking people up to ECMOs for heart and lung support, and
managing the machines.

I have to be more accurate and safer than usual, but using senses that
are now worse. Our face shields fog up in an instant. We've also started
wearing three pairs of gloves. It's like I'm trying to treat patients
through a fogged-up shower curtain.

I can't do things by myself the way I used to. I understand how
important it is to work as a team. I'm trying to avoid disagreements
about the patients' treatment more than before. Even if there's a
difference of opinion in the team, we don't have the time to discuss it
at leisure.

Yesterday there was a request from another hospital to hook an elderly
patient with advanced cancer to the ECMO. It was my boss's decision.
Thinking about the number of ECMOs and the need for treatment going
forward, the chances of recovery and so on, they thought it wasn't
appropriate.

When you have to make that decision yourself, I think that's really
tough. There are patients coming to us who we absolutely must save.
Finding a way to save them is our job.

When I'm off, I'm really careful that I don't bring the virus home with
me. I'm not seeing my family to make sure I don't give it to them. I
just shout, ``I'm home'' when I walk through the door. We have our meals
at different times.

\includegraphics{https://static01.graylady3jvrrxbe.onion/packages/flash/multimedia/ICONS/transparent.png}

\begin{itemize}
\tightlist
\item
  María Alvares, Nurse
\item
  León, Mexico
\item
  July 1
\end{itemize}

For the first few months of the pandemic, we didn't have a lot of access
to P.P.E. We would get some only when we had patients who were confirmed
to have Covid-19. Now it's better, but sometimes they give us masks that
are not N95 masks, so a lot of us buy our own masks and goggles.

Some doctors and nurses got infected with Covid-19 and cannot work
currently, so we are doing double the work.

We've had a really high number of patients. Sometimes we'll go from one
patient to the next to the next with no break. There are days when we
are working so hard we don't even eat. We'll go more than eight hours
without even drinking any liquids.

I have more responsibility now. Not only do I have to care for my
patients, but I have to act cautiously to protect my family and myself.

I have two sons who I love. I dedicate my life to them. The security of
my family is the most important thing to me.

\includegraphics{https://static01.graylady3jvrrxbe.onion/packages/flash/multimedia/ICONS/transparent.png}

\begin{itemize}
\tightlist
\item
  Aparna Parikh, Oncologist
\item
  Boston
\item
  May 8
\end{itemize}

I tested positive at the beginning of March. I spiraled a little bit
about exposing not only family, but colleagues and patients. I had seen
more than 32 patients without knowing that I had it.

I had shortness of breath and a cough. I felt lousy and wondered, What's
going to happen to the baby? I am just coming out of my first trimester
now.

I isolated for two weeks and was back at work after three.

A couple of times, we've had entire families on our floor. These are
people that don't have the option of isolating. They're living in close
quarters.

A lot of the patients are essential workers who had to keep their jobs.

A grocery worker told me there were eight people living in his
apartment. They had two bedrooms, but only one bathroom. Several family
members had already gotten sick and he was developing some symptoms.

``I knew it,'' he said via an interpreter. ``I knew I was getting sick
but I'm the one who brings in the money and if I don't work, I can't
feed my family.''

He was itching to get out.

I had an elderly Latina woman who wasn't that ill, but was just so
scared. She had a daughter in the hospital, too.

``I was scared, too,'' I was able to say. ``But we're here. We're taking
care of you and I understand how scared you are.''

\includegraphics{https://static01.graylady3jvrrxbe.onion/packages/flash/multimedia/ICONS/transparent.png}

\begin{itemize}
\tightlist
\item
  Rohail Asrar, Internist
\item
  North Brunswick, N.J.
\item
  April 9
\end{itemize}

The most shocking thing is how unpredictable this virus is. It has us
baffled at times. I am seeing some 40- and 50-year-olds need
ventilators, and I am seeing some 90-year-olds not getting to that
point. Of course, it's the older people that don't fare well most of the
time.

I am seeing patients die without a family member beside them. It's
heartbreaking.

It's like nothing I've seen or experienced in my career.

\includegraphics{https://static01.graylady3jvrrxbe.onion/packages/flash/multimedia/ICONS/transparent.png}

\begin{itemize}
\tightlist
\item
  Patricia Lafontant, E.R. Nurse Practitioner
\item
  Washington, D.C.
\item
  April 10
\end{itemize}

A 26-year-old patient asked me if he was going to die. He said he felt
like he was.

This was a young, otherwise healthy African-American male who was unable
to speak in complete sentences due to shortness of breath.

I couldn't answer that question --- I really didn't know. I will never
forget how sick he looked. That's when the virus became very real to me.
That's when I realized that we were dealing with something we had never
seen before.

\includegraphics{https://static01.graylady3jvrrxbe.onion/packages/flash/multimedia/ICONS/transparent.png}

\begin{itemize}
\tightlist
\item
  Erica Wendling, Respiratory Therapist
\item
  Minneapolis
\item
  April 9
\end{itemize}

I was studying to be a dentist but I was in a major car accident. I
sustained permanent neck injuries and was told that I would never be
able to practice as a dentist. I had been a nurse's aide for many years,
so I knew what a respiratory therapist did and felt like my broken body
would be able to handle the demands of the job.

I have always enjoyed being a respiratory therapist. We specialize in
the cardiopulmonary systems. Where we shine are the I.C.U.s. Because
there are so few of us, our managers are trying to keep our exposure
down. We are teaching our nurses through glass doors, with full P.P.E.
on, how to turn off this alarm, or how to position the ventilator tubing
so it doesn't cause pressure sores.

When I get a call from a nurse saying they need me bedside, I know I
need to hurry. We also need to be thorough every time we step into the
isolation suites --- otherwise we might end up on the other side of the
doors. At my hospital, we have someone stationed at every Covid-19 suite
to help with donning and doffing our protective equipment. They
double-check us.

I am living in an R.V. in my driveway to protect my family as much as I
possibly can. I'm missing them from so close.

\includegraphics{https://static01.graylady3jvrrxbe.onion/packages/flash/multimedia/ICONS/transparent.png}

\begin{itemize}
\tightlist
\item
  Sridevi Rajeeve, Hematology and Medical Oncology Fellow
\item
  Manhattan, N.Y.
\item
  July 10
\end{itemize}

It's been an eerie experience to go back to a semblance of normalcy. It
feels like we are walking on eggshells, expecting a surge of cases again
in New York and reliving the nightmarish ordeal from March to May.

The practice of medicine has changed for the foreseeable future. We see
immunosuppressed cancer patients, who could not get lifesaving
chemotherapy or procedures because it meant exposing themselves to
Covid, now coming to the hospital with disease progression. Clinics have
transformed into a medley of in-person visits and televisits.

I graduated from my internal medicine residency training in June after
spending the last three months exclusively taking care of Covid
patients. We all dispersed after a virtual graduation to start new jobs
or subspecialty training. It was a grim and incomplete closure to a
significant era of training.

\includegraphics{https://static01.graylady3jvrrxbe.onion/packages/flash/multimedia/ICONS/transparent.png}

\begin{itemize}
\tightlist
\item
  Zhang Wendan, Nurse
\item
  Huanggang, China
\item
  May 2
\end{itemize}

I received the call to report in for work on the eve of Chinese New
Year. I spent four weeks in the quarantine zone of my hospital, stopping
only to sleep at a nearby hotel that officials turned into temporary
dorms for us medical workers.

One of the most difficult moments for me during the outbreak was when my
superiors --- almost all men --- told me and my fellow female nurses,
who have been working endlessly, day and night, that we lacked the
spirit of devotion. These words were unfair; we were asking for help
getting pads and tampons because supplies were limited and the city was
under lockdown.

One of my colleagues continued to work through her night shifts even
though she vomited and felt unwell. Other colleagues hadn't seen their
sons and daughters in nearly a month, but they never complained. They
only cried, secretly.

There have been moments of happiness and sadness all at once. The other
day there was a 21-year-old woman discharged from the hospital, which
made me happy. The same day, a 26-year-old woman passed away. These
events have broken many families, which makes me sad.

\emph{The Times previously}
\emph{\href{https://www.nytimes3xbfgragh.onion/2020/02/26/business/coronavirus-china-nurse-menstruation.html}{wrote
about}} \emph{Ms. Zhang's life as a nurse on the front lines of the
outbreak in China.}

\includegraphics{https://static01.graylady3jvrrxbe.onion/packages/flash/multimedia/ICONS/transparent.png}

\begin{itemize}
\tightlist
\item
  Min Chul Kim, Internist
\item
  Seoul, South Korea
\item
  April 27
\end{itemize}

Even before we had Covid-19 patients, we already had a system to test
and treat infectious-diseases patients in a safe manner. For example,
whenever patients suspected of MERS came in, we took care of them. But
things got very busy after the first Covid-19 patient arrived. At first,
we did not anticipate this novel infectious disease to turn into a
pandemic like this. Policies were changing by the hour.

We treated 13 Covid-19 patients in total and fortunately, all of them
were cured. The last ones left our hospital last week. They thanked me.
And even after patients left the hospital, they were tested again after
two weeks of quarantine. I put on five pieces of protective gear --- N95
respirator, goggles, gown, gloves and hood --- to see them again in a
safe way. Then I traced them as outpatients until they were fully ready
to go back to work and their daily lives.

Now that there are no Covid-19 patients at our hospital, I don't wear
the Level-D suit and have gone back to seeing outpatients. However, all
of us at the hospital are nervous that another outbreak may occur. It's
actually easier when we have inpatients. We might be without protection
when diagnosing our next patient.

\includegraphics{https://static01.graylady3jvrrxbe.onion/packages/flash/multimedia/ICONS/transparent.png}

\begin{itemize}
\tightlist
\item
  Nancy Dines, Anesthesia Nurse
\item
  Kingston, Pa.
\item
  April 25
\end{itemize}

The patients keep coming and they become so ill so quickly.

It's my job to intubate them when they require mechanical ventilation. I
am used to touching, and talking to my patients. But during this
pandemic, I'm dressed in layers of P.P.E.

When it's the scariest time for the patient, I can hardly comfort them.

I'm beginning to hate my P.P.E. I resent it. It keeps me from connecting
with my patient. It's so sad that these patients are dying alone.

I try to talk to them. I'm not sure they can hear me. I feel sad all the
time.

I'm proud to be a nurse. Knowing I'm an ``essential worker'' positively
affects my self-esteem.

But I feel guilty that I can't do more to help my patients.

\includegraphics{https://static01.graylady3jvrrxbe.onion/packages/flash/multimedia/ICONS/transparent.png}

\begin{itemize}
\tightlist
\item
  Chetna Singh, E.R. Doctor
\item
  Upper Freehold, N.J.
\item
  April 10
\end{itemize}

My husband and I are both E.R. doctors in New Jersey, and to say the
past few weeks have been stressful would be an understatement.

We thought about renting out a place, leaving our house and completely
isolating ourselves from our children, but we chose not to. We are lucky
enough to have a basement area that we can access from the outside. This
is our decontamination area where we can shower and change when we come
home. Even after our showers, though, we try not to hug or kiss our
kids.

I have seen young and old dying. In the many years that I have been an
attending physician, this was truly the first time that I was afraid to
go to work.

I have overcome that fear and am looking forward to seeing the end of
this.

\includegraphics{https://static01.graylady3jvrrxbe.onion/packages/flash/multimedia/ICONS/transparent.png}

Hilary Swift for The New York Times

\begin{itemize}
\tightlist
\item
  Willie Grady, E.R. Nurse
\item
  Atlanta
\item
  April 29
\end{itemize}

This is my first travel assignment. I had a trip planned for the month
of April so I already had the time off. They offered a great salary, but
I think more so than that, I tend to follow God.

About a week before I came, I had a family member that passed away from
Covid. She had sickle cell. She had been in and out of hospitals, which
aren't the safest places right now.

I ended up telling my older kids about the assignment the day before I
was leaving. And my daughter --- she's such a sweetheart --- she cried
for like 30 minutes. But I knew I had to be here.

When I first got to the Bronx, it seemed like it was nonstop. Your
entire shift, everybody was being intubated and placed on ventilators.
You just had to walk out for a mini break sometimes just to get your
mind clear and then come back to it.

One guy came in alert, oriented, ambulatory --- we call them ``walkie
talkies.'' His main concern was how long his imaging and CT scan were
going to take so he could go home.

I went into the break room and one of my co-workers came in, she was
like ``Grady, they're coding your patient.''

I'm thinking, OK, this is my other patient who has pancreatic cancer.

No, it's the guy who was alert, oriented, ambulatory. He was a young
guy, like 50. We didn't get him back after doing CPR for like 45
minutes.That really affected me.

\includegraphics{https://static01.graylady3jvrrxbe.onion/packages/flash/multimedia/ICONS/transparent.png}

\begin{itemize}
\tightlist
\item
  Agustina Huilca, Pediatrician
\item
  Loreto, Peru
\item
  June 10
\end{itemize}

In Loreto, the biggest region of Peru, we face eternal threats. We have
fought dengue, tuberculosis and malaria for decades. Doctors here are
familiar with chaos and death.

I'm a pediatrician at the Iquitos Support Hospital, but now I care for
all kinds of patients. I'm at the triage area for suspected Covid-19
cases, determining who may have the virus and who may not. Because we
don't have any rapid test kits or molecular PCR tests, I diagnose the
disease by clinical exam.

At the most critical times, I've been the only physician in the entire
hospital.

During the last few weeks, I've seen patients die because of the lack of
oxygen tanks, people dead on arrival because they didn't survive the
long journey, people languishing in the waiting line.

Sometimes the pain of my people causes me great indignation. I think of
my colleagues who fell ill every day. I treated them when they were
short of breath and struggled for air. Since we couldn't do more here,
we sent them to the regional hospital. I shipped many into an ambulance,
a journey with no return. I never saw them again.

They were young, talented doctors, with a life ahead of them. How do you
forget something like that?

\includegraphics{https://static01.graylady3jvrrxbe.onion/packages/flash/multimedia/ICONS/transparent.png}

\begin{itemize}
\tightlist
\item
  Ashley Luanne Kay Chermak, E.R. Nurse
\item
  Fridley, Minn.
\item
  April 10
\end{itemize}

The week we started getting Covid-19 cases in our county, I had many
patients come in hyperventilating, tears streaming down their faces,
scared for their lives and for those of their families. With a gloved
hand and soft eyes, I sat and held their hands and told them they were
going to get through this and we were going to do everything we could to
help.

\includegraphics{https://static01.graylady3jvrrxbe.onion/packages/flash/multimedia/ICONS/transparent.png}

\begin{itemize}
\tightlist
\item
  Patricia Tiu, Post-Anesthesia Care Nurse
\item
  Queens, N.Y.
\item
  June 6
\end{itemize}

Since the virus came, I have been very angry. I feel like our government
failed us. We watched people die every day. We had to fight for our
gear.

I had a sedated patient, she was doing well but her son died from Covid.
She had no idea. That just hit my soul. I told her nurses that if she
wakes up, let her know that her son passed away. What do you say?

When you deal with stuff like this, you feel the emotion, but you're
still working, so you put it in the back of your head and you keep
going. Then when you get home, that's when you can just let everything
go and start crying.

I'm fine for the most part, but I could just be sitting down and all of
a sudden I'll be very angry or feel really sad and I don't understand
why.

I've gone to protests, obviously wearing an N95 mask. I don't know if
I'd call myself a medic, but I do walk around with a first aid kit. And
I do have a sign that says first aid available.

As nurses, we are advocates, both in and out of uniform. Just like
medical professionals are held accountable for their actions, so too
should be the police. Racism is also a public health crisis.

\includegraphics{https://static01.graylady3jvrrxbe.onion/packages/flash/multimedia/ICONS/transparent.png}

\begin{itemize}
\tightlist
\item
  Elisa Dannemiller, E.R. Doctor
\item
  Denver
\item
  July 7
\end{itemize}

When I worked for eight days in Hackensack, N.J., it felt like residency
again.

Before I and a few other E.R. doctors arrived to help, some of the
doctors there had not had a day off in 30 days.

One had Covid-19 himself and one just lost his father-in-law to the
virus.

I saw more codes and death in those eight days than I had seen in the
past year.

It is a nasty virus.

While the Covid-19 volumes are decreasing in the New York City area, I
am happy that I am continuing to go back.

While many of the patients I cared for did not live, it is awesome to
see the ones who are now recovering, getting their tracheostomies and
feeding tubes removed, and moving on to rehab. They are grateful to be
alive and to have had the care they received.

Seeing the pandemic through and witnessing the positive outcomes has
made my Covid-19 patient nightmares go away. I had them nightly for a
while after my first trip in early April.

\includegraphics{https://static01.graylady3jvrrxbe.onion/packages/flash/multimedia/ICONS/transparent.png}

\begin{itemize}
\tightlist
\item
  Petronella Benjamin, Nurse
\item
  Cape Town, South Africa
\item
  Died on April 29
\end{itemize}

``Let me go back and go and serve,'' Petronella Benjamin told her
daughter. She was days from retirement and the family clinic where she
was known as Sister Nellie was beginning to see a surge in coronavirus
infections. Her five grown children wanted her home.

``I only have a few days left,'' Ms. Benjamin said, switching from
Afrikaans to English as if to emphasize her resolve to her daughter,
Alicia Maart. ``This is what I want to do.''

On April 29, a day before she would have retired, Ms. Benjamin died of
Covid-19. She was 61.

A nurse since completing vocational school as a teenager, she first took
a job in a home for seniors at 16. In the family clinic where she worked
most recently, she was a favorite with patients, many of whom would go
out of their way for an appointment with her. Ms. Benjamin was also a
pastor, leading a church with her husband, Edwin Benjamin. Under
lockdown, she sent sermons to her congregation via WhatsApp. In her last
message, she prayed for those grieving the victims of Covid-19.

Her own funeral was live streamed on Facebook. As the hearse drove by,
neighbors and congregants lined the sidewalk in her honor. Her husband
could not attend: He had tested positive for the coronavirus and lay in
an intensive care unit on a respirator. Mr. Benjamin is now recovering.

Her daughter, Ms. Maart, is also a nurse and has returned to work in a
public hospital. She is afraid, but inspired by her mother's call to
service.

--- LYNSEY CHUTEL

\includegraphics{https://static01.graylady3jvrrxbe.onion/packages/flash/multimedia/ICONS/transparent.png}

\begin{itemize}
\tightlist
\item
  Samantha Irvine, Nurse
\item
  Staten Island, N.Y.
\item
  April 10
\end{itemize}

Nursing homes are on the front lines as well, and usually forgotten. Our
facilities provide care for the most vulnerable patients, and we are
filled with Covid-19. They are experiencing awful, unimaginable deaths,
and there is nothing we can do. Many patients have advance directives
that include do-not-hospitalize orders.

Staffing is a huge issue. I'm a nursing supervisor and half the staff is
out sick, including those who are responsible for staffing.

It's chaos.

\includegraphics{https://static01.graylady3jvrrxbe.onion/packages/flash/multimedia/ICONS/transparent.png}

Giulia Marchi for The New York Times

\begin{itemize}
\tightlist
\item
  Liu Taotao, Surgical I.C.U. Doctor
\item
  Beijing
\item
  April 28
\end{itemize}

The first day of the Chinese New Year, the hospital made it clear that
it wanted us to go to Wuhan. I signed up. In fact, I really wanted to
go. It suited my training and I thought it was actually a good
opportunity to practice.

I arrived in Wuhan on Feb. 7. It was very humid and cold. Turning on the
heat wasn't allowed because it might lead to more infection.

We are used to dealing with severely ill patients. But there were cases
in which the whole family got sick and died.

It feels like a war. You come away with a deeper understanding of social
mobilization and coordination.

I returned to Beijing on April 6.

\includegraphics{https://static01.graylady3jvrrxbe.onion/packages/flash/multimedia/ICONS/transparent.png}

\begin{itemize}
\tightlist
\item
  Laura Janneck, E.R. Doctor
\item
  Boston
\item
  July 7
\end{itemize}

The last several months in Boston have seen us go from the height of the
local epidemic to a gradual easing toward a new normal. It has been slow
going, and felt very much like an exercise in endurance and adaptation.

The virus still very much affects how we go about our work in the
emergency department. We look for any signs that a patient might have
Covid and we screen anyone getting admitted to the hospital.

In many ways, the more cumbersome impacts of the virus have been on
other aspects of life: the smaller circle of friends I see, negotiating
the education and socialization of my daughter, singing with my church
choir via video recordings rather than rehearsals and so on.

I personally waver between fatalism that I will likely get infected and
it will likely be mild, to trepidation about the higher risk of
complications among health care workers. I feel like I've grown in my
ability to accept uncertainty, and am trying to approach new
opportunities and decisions with an embrace of the fact that things will
always be different going forward.

\includegraphics{https://static01.graylady3jvrrxbe.onion/packages/flash/multimedia/ICONS/transparent.png}

\begin{itemize}
\tightlist
\item
  Yaciel Almira, I.C.U. Nurse
\item
  Miami Beach, Fla.
\item
  May 15
\end{itemize}

I knew what my job entailed. I was going to take care of patients with a
disease never seen before. I was determined to do so with valor. I
signed up to take care of people. Quitting now --- when most needed ---
was never an option in my mind, but there was fear, too.

It is scary to see how sick people get with Covid-19 and know that you
can join them or, even worse, that you can bring it home to the people
you love the most.

It is OK to be scared, but it's vital to be determined to beat the
disease.

\includegraphics{https://static01.graylady3jvrrxbe.onion/packages/flash/multimedia/ICONS/transparent.png}

\begin{itemize}
\tightlist
\item
  Marcelle Pignanelli, Internist
\item
  Queens, N.Y.
\item
  April 10
\end{itemize}

My patient's oxygen level was 86 percent with an oxygen mask, not a good
sign.

He looked at me dead in the eye and told me in Spanish: ``Doctor, please
do whatever you have to do to save me. I will take out a loan to pay my
house and my car if I have to, but please don't let me die.''

I looked at him knowing I couldn't guarantee anything but said, ``That's
why I'm here, I'm here to get you out of this. There is a big chance
that you could get intubated. But let's be hopeful.''

He was discharged, thankfully. He didn't wind up needing intubation. He
was one of the few.

To be completely honest, I'm exhausted. I'm burned out physically and
emotionally. I'm trying to stay sane by meditating. I'm trying to sleep,
but I don't get a lot. There's a lot of times when you get home, you're
too tired to talk, you shower and go to bed.

\includegraphics{https://static01.graylady3jvrrxbe.onion/packages/flash/multimedia/ICONS/transparent.png}

\begin{itemize}
\tightlist
\item
  Moussa Alzouma Mahamadou, Nurse
\item
  Madarounfa, Niger
\item
  April 30
\end{itemize}

I came from Niger to Burkina Faso for a measles vaccination campaign run
by Doctors Without Borders. Following the spread of Covid-19, I was
deployed to the city of Bobo Dioulasso as part of our support to local
health authorities.

It was a first experience for me, and for the team in general, to take
care of patients with a highly contagious disease like Covid-19.

I am concerned about the early lifting of some lockdown measures, from
the reopening of markets to the resumption of transportation, without
sufficiently emphasizing the importance of protective measures:
compulsory mask for all, systematic hand washing and, above all,
avoiding mass gatherings. The pandemic is still there.

\includegraphics{https://static01.graylady3jvrrxbe.onion/packages/flash/multimedia/ICONS/transparent.png}

Heather Sten for The New York Times

\begin{itemize}
\tightlist
\item
  Amanda Dasaro, E.M.T.
\item
  Staten Island, N.Y.
\item
  April 28
\end{itemize}

When people are experiencing their worst moments, I feel good knowing
that once I arrive I can make a difference. We did choose this job, but
we are human and people have their limits. Some of us are just getting
by, some of us are doing fine, and some of us are just not OK.

There was a moment in the very beginning of this when I looked my
7-year-old son in his eyes and said that I loved him and would see him
later. I walked out the door with a few personal belongings and a few
clean uniforms. I chose to live in temporary housing so that I could
protect my family. Did I want to quit and stay home? Yes.

The calls skyrocketed. I have pleaded with patients to hold on while
transporting them. I have told them to be strong while they are wheeled
into a room full of nurses and doctors waiting to intubate them. I feel
horrible for having to tell family members that they could not come with
their loved ones.

There have been plenty of times these past few months when I have broken
down and cried to my work partner. Sometimes I just hug her with no
words needed. I often ask her, when will this end?

\includegraphics{https://static01.graylady3jvrrxbe.onion/packages/flash/multimedia/ICONS/transparent.png}

\begin{itemize}
\tightlist
\item
  Beth Oller, Family Physician
\item
  Stockton, Kan.
\item
  April 9
\end{itemize}

My husband and I are the only physicians for miles, and the parents to
four small kiddos, one of whom is still breastfeeding. How do we protect
our family and our community --- which is a family, too?

As a rural family physician, my patients are my neighbors, my children's
teachers, our friends. I am now unable to hug them, to visit them at the
nursing home, or to have more than one family member attend a birth.

Yesterday, my friend, another family physician and father of five, was
intubated with Covid-19. It was the first time I broke down and cried
since this began.

Looking at every patient as a potential carrier changes you, not
approaching your patients when you enter a room changes you, seeing your
young patients look at you with trepidation because you have to wear a
mask changes you.

All I want to do is hold my kids close, gather with my neighbors, be
together. Instead, I'm on Facebook begging them to stay apart.

\includegraphics{https://static01.graylady3jvrrxbe.onion/packages/flash/multimedia/ICONS/transparent.png}

\begin{itemize}
\tightlist
\item
  Karina Hernández Flores, Oncology Nurse
\item
  Hermosillo, Mexico
\item
  July 1
\end{itemize}

I'll never forget an old man named Rafael. He was very tired and in a
lot of pain. His breathing was really labored and it was hard to
understand what he was saying to me. I moved him to make him
comfortable, gave him an analgesic and his pain slowly eased.

But he had a worried look on his face. I remembered that he might never
see his family again.

I took his hand and told him his family would be all right, that he
should rest, that he had done a great job. I said a prayer asking for
strength for his spirit.

I'm not of any particular religion, but I know that this helps the soul.
So I asked the creator of the universe not to let this man suffer more
and that, if he wanted him, to take him by his side.

And like magic, he fell asleep.

As I drove home, ``Let Me Cry'' by Carla Morrison came on the radio. I
cried as I thought about how many families won't see one another again
and how many will suffer because of this illness.

Rafael died that night. Later, his wife sent me a message thanking me. I
told her that he went peacefully and that he knew his family loved him.

\includegraphics{https://static01.graylady3jvrrxbe.onion/packages/flash/multimedia/ICONS/transparent.png}

\begin{itemize}
\tightlist
\item
  James House, Nurse
\item
  Warren, Mich.
\item
  Died on March 31
\end{itemize}

``Think: Life doesn't happen to you. It happens for you,'' read one of
the messages of encouragement that James House liked to text his sister.

Mr. House was a registered nurse who was planning to become a physician
assistant. His fascination with the human body and the power of medicine
started in childhood, when he and his sister would go with their mother
on her regular trips to a dialysis clinic.

He had worked at the same nursing home for almost a decade. His
easygoing personality and corny sense of humor was a good fit for the
work, and he had real affection for the residents, his sister, Catrisha
House-Phelphs, said.

On March 31, Mr. House returned to work feeling like he had the flu. He
had been denied a test for the coronavirus because he didn't have what,
at the time, seemed to be the classic symptoms. ``That was during the
early stages,'' Ms. House-Phelphs said. ``Nobody really knew a lot about
it.''

After a few hours, he collapsed and was taken to the hospital.

``They're about to intubate me,'' he texted his sister. It was his last
message.

``Within a couple hours, they called and they told me they lost him,''
she said. He was 40.

Mr. House was a member of the Serenity Christian Church and the
Freemasons.

He was divorced in January and is survived by five children, who range
in age from 2 to 19.

--- KIM SEVERSON

\includegraphics{https://static01.graylady3jvrrxbe.onion/packages/flash/multimedia/ICONS/transparent.png}

\begin{itemize}
\tightlist
\item
  Jill Cohen, E.R. Nurse
\item
  Truckee, Calif.
\item
  April 8
\end{itemize}

I was never big on pep talks previously, but now I give myself a big one
before every shift.

\includegraphics{https://static01.graylady3jvrrxbe.onion/packages/flash/multimedia/ICONS/transparent.png}

\begin{itemize}
\tightlist
\item
  Imran Nazir, Junior Resident Doctor
\item
  Pulwama, Kashmir
\item
  April 24
\end{itemize}

There is a lot of learning in treating Covid-19 patients. How careful we
have to be with them, not too close or touching the patient
unnecessarily. How precisely we have to check their vitals. How we need
to talk to them. We need to avoid hurting them with our words or our
actions.

There were eight Covid-19 patients in the ward and all were asymptomatic
without any comorbidities. We just needed to check their vitals, take
their history, find out who came in contact with them. No one even had
fever. Their ages were from 22 to 90 years. They wanted to gossip among
themselves.

\includegraphics{https://static01.graylady3jvrrxbe.onion/packages/flash/multimedia/ICONS/transparent.png}

Hilary Swift for The New York Times

\begin{itemize}
\tightlist
\item
  Ashley Crumpler, Critical Care Nurse
\item
  Atlanta
\item
  May 1
\end{itemize}

Travel nursing was something I always wanted to do, especially helping
out with disaster relief. When the chance to go to New York came around,
I jumped all over it. If I ever have children, I want to be able to say,
``I had my hand in that. I did that.''

It's different from everyday nursing, just seeing how sick patients are,
and to see how the simplest things become so hard. Taking steps to the
bathroom for some patients is like running a marathon. Sitting up and
eating. I see these people and I see them struggling to breathe. It
makes me think, ``I'll never take each breath for granted again.'' I'm
an asthmatic, so I've been through the struggle of breathing when I was
a child. I know what it's like to feel like you're suffocating.

Is my family worried? Worried is an understatement. They've been crying
on the phone. At the end of the day, I believe God's got me either way.
Luckily my grandfather is a pastor. When we first started and everything
was crazy, I called him and my co-workers and I put him on speakerphone,
and he prayed. And we were OK.

\includegraphics{https://static01.graylady3jvrrxbe.onion/packages/flash/multimedia/ICONS/transparent.png}

\begin{itemize}
\tightlist
\item
  Sylvia Fallon, Nurse Practitioner
\item
  Brooklyn, N.Y.
\item
  April 10
\end{itemize}

Working in a community clinic, we are at a loss for what to do for our
community. Do we prescribe antibiotics that are now on back order? Do we
give inhalers to everyone? What about our asthmatics and C.O.P.D.
patients? What happens when the Albuterol is gone?

We provide care for many of the underserved and uninsured New Yorkers
and we are struggling to keep our doors open. Our organization had to
temporarily lay off our medical office assistants. Now we are expected
to manage logistics while increasing our patient load.

We tell our patients to stay inside, isolate, monitor your temperature,
breathing, heart rate, and don't go to the emergency room until you feel
like you can't breathe. I go to bed every night with those last words
haunting me.

\includegraphics{https://static01.graylady3jvrrxbe.onion/packages/flash/multimedia/ICONS/transparent.png}

\begin{itemize}
\tightlist
\item
  Shivyon Mitchell, Surgical Nurse
\item
  Jackson, Wyo.
\item
  April 11
\end{itemize}

In our small ski-town community, my co-workers and I felt we were
sheltered. But we started getting Covid-positive patients.

Thankfully our hospital was prepared and has kept us well-informed
throughout this pandemic. That being said, even with the necessary
precautions, I am afraid to go home because of the possibility of
exposing others.

\includegraphics{https://static01.graylady3jvrrxbe.onion/packages/flash/multimedia/ICONS/transparent.png}

\begin{itemize}
\tightlist
\item
  Colette Badjo, Doctor
\item
  Montreal
\item
  April 30
\end{itemize}

I live in Canada but I am working as an emergency coordinator for
Doctors Without Borders' Covid-19 response in Ivory Coast. I am proud of
what I do.

The biggest challenge here is to get all the necessary supplies, such as
drugs, personal protection equipment and other materials. We usually get
supplies from international providers, but with all the restrictions to
travel and closure of borders, it's getting more and more difficult. We
risk facing shortages when it's crucial for frontline health workers to
be protected in order to avoid any new contamination chains.

This situation makes us think out of the box. For instance, here in
Abidjan, we are now producing one million tissue masks in the local
factories, in order to distribute them to the general population.

\includegraphics{https://static01.graylady3jvrrxbe.onion/packages/flash/multimedia/ICONS/transparent.png}

Saul Martinez for The New York Times

\begin{itemize}
\tightlist
\item
  Ari Ciment, Pulmonary and Critical Care Doctor
\item
  Miami Beach
\item
  April 12
\end{itemize}

As Passover approached, my hospital started to fill up with Covid cases.

A 67-year-old man named D.M. was deteriorating. I moved him into the
I.C.U. and called his daughter. She had a simple but potentially
lifesaving device at home: a pulse oximeter, which measures the oxygen
levels in your blood. With this disease, if you don't notice when people
start losing oxygen quickly, it can be bad. But when these patients are
in a Covid room in the hospital, they're isolated. We can't afford to
wait for the nurse to come around to check their levels --- we need
real-time assessments. That's why a personal pulse ox is potentially
lifesaving. I can FaceTime patients from home and ask them to check.

Unfortunately, D.M. had to be intubated before we could get him the
oximeter. So I took it to a 37-year-old patient, J.K., whose oxygen
level was dipping. He texted me his readings every three hours. D.M.'s
pulse oximeter helped J.K. survive. If he had been intubated, he
probably would have died.

Meanwhile, D.M.'s condition deteriorated and he needed plasma from
someone who had survived Covid-19, but the chances of finding a match
were less than 10 percent. I called another of my patients, J.S., whom I
had just discharged. ``How can I help?'' J.S. said. He donated his
plasma. He was a perfect match.

In short, D.M.'s pulse oximeter helped J.K., and J.S.'s plasma helped
D.M.

The end of the Passover Seder has a stanza called ``Chad Gadya,'' which
is a playful song that essentially highlights a circle of life and
``connectivity'' of seemingly disparate objects and beings. The patients
don't know one another yet, but they're all interwoven. Everybody is
connected. ``Chad Gadya.''

\includegraphics{https://static01.graylady3jvrrxbe.onion/packages/flash/multimedia/ICONS/transparent.png}

\begin{itemize}
\tightlist
\item
  Jay W. Lee, Family Physician at a Community Health Center
\item
  Huntington Beach, Calif.
\item
  April 12
\end{itemize}

Well before Covid-19, my patients were already disadvantaged. How do we
ask our patients to stay home when they don't have homes? To wash their
hands when they don't have sinks? To limit their trips to grocery stores
when they rely on food banks? How do we ask them to stay healthy when
they are underinsured?

\includegraphics{https://static01.graylady3jvrrxbe.onion/packages/flash/multimedia/ICONS/transparent.png}

\begin{itemize}
\tightlist
\item
  Astrid Vázquez, General Practitioner
\item
  Puebla, Mexico
\item
  July 2
\end{itemize}

Every day, I see my co-workers with aching hearts, holding in tears with
their jaws clenched.

I watch my patients suffer, their eyes reflecting an enormous amount of
fear and anxiety. I see the family members of my patients living in
uncertainty, confronting death in a different way. It's more unexpected
than ever, and cruel.

We're no longer just taking care of our patients. We're taking care of
our co-workers, our loved ones, ourselves, both in physical and mental
health. We try to be ten people at once.

We confront the pain, the frustration, the anger, the powerlessness and
we don't have time to recover. It's affecting all of us, whether we have
Covid-19 or not. Whether we know someone with Covid or not. Whether
we're a health care worker or not.

We won't be the same after Covid-19.

\includegraphics{https://static01.graylady3jvrrxbe.onion/packages/flash/multimedia/ICONS/transparent.png}

\begin{itemize}
\tightlist
\item
  Myeong Hae-kyung, Nurse
\item
  Daegu, South Korea
\item
  April 27
\end{itemize}

At first we didn't have supplies arriving quickly enough. We lacked the
knee-high protective shoes so we wrapped our feet in layers of plastic
bags. We reused eye shields after wiping them clean with sanitizer. We
reused masks after drying them.

We have contaminated zones where we come in contact with patients, and
safe zones where we can take a break and chat after taking off our
full-body protective gear. It's hard to breathe in the suit. It's bulky,
stifling and tiring. You get drenched in your own sweat.

I am especially careful when taking off my protective suit, because I
fear I might get the virus while doing so. Patients cough and we often
use the suction device. Stuff flies and lands on your suit. Patients in
delirium yank off your eye shield and mask. Your suit can get torn and
when that happens, you stop everything and rush out to get a new suit.
Sometimes you get so worried that you take a virus test to make sure you
were not infected.

Some nurses don't want to go home for fear they might bring the virus
with them. So we have a motel nearby where they can stay. I go home
because I have two sons, sixth and seventh graders. When I get home, one
of them greets me at the door by spraying disinfectant all over me. I
leave all my clothes out on the veranda. I move immediately into the
bathroom for a shower. I keep my mask on at home and avoid touching my
sons. But I feel good when my sons say they are proud of their mom's
work.

Many of the patients who recover send us thank-you notes and gifts like
fruits and cakes, bottled water and sports drinks. They know we drink a
lot of them because of the sweat we lose working in the suits.

\includegraphics{https://static01.graylady3jvrrxbe.onion/packages/flash/multimedia/ICONS/transparent.png}

\begin{itemize}
\tightlist
\item
  Isabel Muñoz, Doctor
\item
  Salamanca, Spain
\item
  Died on March 24
\end{itemize}

When Isabel Muñoz was 6 years old, her father had to undergo two
surgeries to overcome bone tuberculosis. During his lengthy hospital
stay, she decided to become a doctor.

``I remember a nurse who was working in the hospital telling my father
that my sister had the passion for medicine running in her blood,'' said
her brother, Jesús Muñoz.

Dr. Muñoz lived in the city of Salamanca but commuted to work at a small
health care center in the village of La Fuente de San Esteban, about 35
miles away. As a primary care physician, she treated people with a wide
range of health issues. This year, some started coming to her with
breathing problems and other possible symptoms of Covid-19.

On March 14, just as Spain's government was declaring a state of
emergency, Dr. Muñoz felt feverish and started coughing, so she
immediately decided to self-isolate. She asked her husband to move out
of their apartment because ``she was obsessed with not getting anybody
else infected,'' her brother said.

Ten days later, Dr. Muñoz, who was 59, was found dead in her kitchen,
after her brother and husband alerted the police because they could not
reach her on the phone.

In her final days at work, Dr. Muñoz complained repeatedly to her family
about not receiving enough protective gear from the authorities and not
getting tested for the coronavirus, her brother said. ``I unfortunately
consider that she engaged in this battle like a soldier sent to the
front line without a helmet and a shield,'' he added. ``She was a real
fighter, who kept telling us that she would overcome this disease as
quickly as possible so as to get back to the job she loved.''

--- RAPHAEL MINDER

Due to an editing error, an earlier version of this article misstated
the date of Dr. Muñoz's death.

\includegraphics{https://static01.graylady3jvrrxbe.onion/packages/flash/multimedia/ICONS/transparent.png}

\begin{itemize}
\tightlist
\item
  Jennifer Tszeng, Kidney and Liver Transplant Nurse
\item
  Manhattan, N.Y.
\item
  April 15
\end{itemize}

Nurses have always been on the front lines of deadly, highly contagious
diseases. We're trained to run toward crisis and patients in need. The
main thing that has changed is the chaos around lack of P.P.E.

On 9/11, rescue workers rushed to Ground Zero without adequate
protection. Twenty years later, so many have died or are still suffering
from lung disease. I'm afraid history will now repeat itself with us.

We shouldn't be sacrificing our health by providing our own
do-it-yourself P.P.E. I love what I do, but I deserve to be protected at
my job like anyone else. We are not martyrs.

\includegraphics{https://static01.graylady3jvrrxbe.onion/packages/flash/multimedia/ICONS/transparent.png}

\begin{itemize}
\tightlist
\item
  Sophia Corbitt, I.C.U. Nurse
\item
  Maryland
\item
  April 8
\end{itemize}

I wonder if I'll have enough paid time off if I test positive. I feel a
sense of guilt when someone calls me a hero. I feel pride in my job. I
have to compartmentalize what I see.

\includegraphics{https://static01.graylady3jvrrxbe.onion/packages/flash/multimedia/ICONS/transparent.png}

\begin{itemize}
\tightlist
\item
  Richard Wang, Pulmonary and Critical Care Doctor
\item
  San Francisco
\item
  April 15
\end{itemize}

Mostly, I just feel grateful. I'm grateful that the skills I have honed
for the past decade give me a clear role and purpose at the forefront of
the fight. I am grateful for the quiet, loving heroism of everyone
sheltering at home, and as a result, saving the lives of people who
would otherwise be infected and end up in my I.C.U. I am grateful to my
spouse, who, I'm pretty certain, didn't sign up to risk her own health
when she married me 12 years ago; and my kids, who, at ages 3 and 6, are
too young to consent to risk their health, either.

I ruminate daily on the sacrifices of everyone who has lost their job or
business or loved ones or life to this pandemic, and I am overwhelmed
with feelings of grief, love, solidarity and determination. We are in
this together, and we shall overcome.

\includegraphics{https://static01.graylady3jvrrxbe.onion/packages/flash/multimedia/ICONS/transparent.png}

\begin{itemize}
\tightlist
\item
  Sophia Walker Henry, E.R. Nurse
\item
  Pikesville, Md.
\item
  June 9
\end{itemize}

My emotions are all over the place. My dad and I have just about fully
recovered from the coronavirus. At work, we are seeing more patients
come in even when they are afraid. The pandemic has been exhausting and
now the toll of discriminatory behavior has reached a peak. I vacillate
between being proud of the protesters and being afraid for them. I fear
for their safety. I'm concerned about them getting Covid-19 and becoming
very ill. I worry about how many will get sick.

But I applaud their efforts. We seem to be on the verge of very
necessary change. I believe that what is being done will have a great
impact on the lives of my young sons.

I've observed unfair and discriminatory behavior for almost 25 years as
an R.N. I have been mistaken for a clinical technician even while
wearing my R.N. badge. I've been ignored by my colleagues. I've watched
physicians walk away from me and ask the white assistant for details
about my patient's care while I'm laboring at the bedside.

I was involved in a social media discussion about discrimination in the
workplace, and someone found a way to diminish every example I gave. I
smiled when I read his response because this is exactly why we hide our
pain in professional settings. We know that most will not understand.

This is where we are, filled with anger and brimming with pain. I cannot
suppress my feelings anymore.

\includegraphics{https://static01.graylady3jvrrxbe.onion/packages/flash/multimedia/ICONS/transparent.png}

Samuel Aranda for The New York Times

\begin{itemize}
\tightlist
\item
  Anna Duarte Velasco, Nurse
\item
  Barcelona, Spain
\item
  April 28
\end{itemize}

The look of fear of dying in many people's eyes will never be erased
from my memory. I feel rage and helplessness because many families have
not been able to say goodbye. So much lonely mourning. I don't remember
many names anymore, because there have been so many, but it makes me
feel a bit better to know that although their families were not there,
we were there holding their hands in their last moments.

Many of my colleagues have already been infected, some mildly, others
seriously. The deaths include some family members of my colleagues. It
is terrifying to think that we can be a source of contagion without
knowing it. This makes many of us isolate ourselves, even from the
people who live in our houses.

Something that seemed temporary is becoming a way of life, a way of life
without kisses, without hugs and without the warmth that we need so much
at this moment.

Seeing our patients recover is one of the best therapies. I feel like
I'm in the right place.

\includegraphics{https://static01.graylady3jvrrxbe.onion/packages/flash/multimedia/ICONS/transparent.png}

\begin{itemize}
\tightlist
\item
  Li Wenliang, Ophthalmologist
\item
  Wuhan, China
\item
  Died on Feb. 7
\end{itemize}

``I started coughing on Jan. 10. It will take me another 15 days or so
to recover,'' Li Wenliang said days before he died.

Dr. Li, a 34-year-old ophthalmologist in Wuhan, China, tried to warn
about an alarming new illness he was seeing in late December, only to be
silenced by police and other officials. On Jan. 10, Dr. Li was infected
by one of his patients, a woman with glaucoma.

Until his death, Dr. Li, who was expecting his second child, vowed to
return to the front lines.

``I will join medical workers in fighting the epidemic,'' he said.
``That's where my responsibilities lie.''

\emph{The Times}
\emph{\href{https://www.nytimes3xbfgragh.onion/2020/02/07/world/asia/Li-Wenliang-china-coronavirus.html}{interviewed}}
\emph{and}
\emph{\href{https://www.nytimes3xbfgragh.onion/2020/02/01/world/asia/china-coronavirus.html}{wrote
about}} \emph{Dr. Li in January and February. In April,}
\emph{\href{https://www.nytimes3xbfgragh.onion/interactive/2020/04/13/technology/coronavirus-doctor-whistleblower-weibo.html}{an
article}} \emph{explored how he is being memorialized online.}

\includegraphics{https://static01.graylady3jvrrxbe.onion/packages/flash/multimedia/ICONS/transparent.png}

\begin{itemize}
\tightlist
\item
  John Rose, E.R. Doctor
\item
  Davis, Calif.
\item
  May 5
\end{itemize}

After 25 years of emergency medicine, I was rarely nervous at work. It
changed with Covid-19.

The fear was in the air.

The fear was in the patient, in the staff, in myself.

Fear coupled with the isolation. We were isolated in the equipment we
wear. Patients isolated from their loved ones when given a tragic
diagnosis. People isolated from a simple touch.

The hope came in realizing the whole planet is in this together. We can
do this.

\includegraphics{https://static01.graylady3jvrrxbe.onion/packages/flash/multimedia/ICONS/transparent.png}

\begin{itemize}
\tightlist
\item
  Jaques Sztajnbok, I.C.U. Doctor
\item
  São Paulo, Brazil
\item
  April 30
\end{itemize}

I've been in the I.C.U. for more than 25 years now, and this is the
worst situation I've ever come up against. Four months ago, we knew
nothing about this disease. Then, all of a sudden, the entire hospital
is filled with one single diagnosis. We don't know anything about how
the disease functions, so we don't know how to treat it. It compromises
various organs and systems in ways that we've never seen before.

I don't have a day off anymore. I feel compelled to be there. With all
this stress, I end up sleeping worse than before. I've been getting up
at 4:30 or 5 every day, studying in the morning, reading all the new
articles that have come out about the disease.

There's the stress of protecting yourself, the personal risk, and as the
team manager there's the stress of replacing people we've lost because
they were contaminated. At one point last week I had five doctors
quarantined.

It's so intense that there's no time to think of anything else, not even
your personal life. We can't have existential crises in the moment. In
whatever little time you have at home, sometimes it all hits you and you
think, Lord, look what's happened to our life. There are overdue bills,
and my wife's horrified that I haven't paid them. Then I go to the
I.C.U. and there's a thousand problems I have to solve at the same time.

\includegraphics{https://static01.graylady3jvrrxbe.onion/packages/flash/multimedia/ICONS/transparent.png}

\begin{itemize}
\tightlist
\item
  Sarah Comrie, Physician Assistant
\item
  Brooklyn, N.Y.
\item
  April 15
\end{itemize}

Usually I am neurotic in my morning routine --- reviewing vital signs,
making to-do lists, and checking blood work before seeing patients. When
I returned from being out with Covid myself, a close colleague told me I
needed to discard my old system and learn to proceed as if I were
running through a minefield with only one eye open. I would have to
adapt to a new level of discomfort --- moving forward semi-blindly,
putting out fires and prioritizing the sickest patients and most
necessary tasks, hoping not to stumble and make a costly mistake. Gone
are the days when I had time to be thorough, dotting i's and crossing
t's. From the moment I start my shift at 7 a.m., until I hail a cab
around 8 p.m., I feel like I am sprinting in a marathon-length race.
Walking into my apartment, my husband makes dinner as I pace around
recounting the day's events, leftover adrenaline giving me a second wind
and making it impossible to fall asleep.

\includegraphics{https://static01.graylady3jvrrxbe.onion/packages/flash/multimedia/ICONS/transparent.png}

\begin{itemize}
\tightlist
\item
  Agustin Borjon, General Surgery Resident
\item
  East Orange, N.J.
\item
  April 15
\end{itemize}

We are swimming in Covid, and there's no escape until the shift is over.
It feels like a cesspool.

\includegraphics{https://static01.graylady3jvrrxbe.onion/packages/flash/multimedia/ICONS/transparent.png}

\begin{itemize}
\tightlist
\item
  Jennifer Vaisman, Nurse
\item
  San Jose, Calif.
\item
  April 15
\end{itemize}

I oddly feel safer when I'm assigned to the Covid I.C.U. because at
least I know what I'm dealing with and I am able to wear my N95 mask.
When I'm on my home unit, the surgical I.C.U., I am only wearing a
surgical mask and I don't know if my patients have Covid-19.

Every day I get home from work, I reset my 14-day clock of exposure.
Every day I go to work I'm exposed to Covid-19. It's all over the
hospital.

It's shown me I'm stronger and more resilient than I ever imagined I
could be. And it's forcing me to reconcile my own fears of mortality and
illness at an uncomfortable pace.

\includegraphics{https://static01.graylady3jvrrxbe.onion/packages/flash/multimedia/ICONS/transparent.png}

September Dawn Bottoms for The New York Times

\begin{itemize}
\tightlist
\item
  Shawnaree Lee, OB-GYN
\item
  Oklahoma City
\item
  June 9
\end{itemize}

The reopening of Oklahoma City's ``nonessential'' businesses is still
ongoing, and while the process can seem haphazard at times, I feel like
the caution is justified. It has been disheartening to see many in my
community and across the country disregard --- or outright ridicule ---
those of us in health care who are trying our best to keep them and
ourselves healthy and safe.

Unnecessary exposures still need to be minimized. But recently, I
participated in some peaceful protests at the State Capitol to honor the
local Black Lives Matter movement. I saw almost everyone there wearing a
face covering and maintaining a safe distance from each other while
marching, standing and kneeling.

I felt that it was very much worth a small increase in exposure risks to
one another to go and be physically present to try and show solidarity.

It is crucially important that we all look at our history honestly, as
ugly as that history really and truly is.

\includegraphics{https://static01.graylady3jvrrxbe.onion/packages/flash/multimedia/ICONS/transparent.png}

\begin{itemize}
\tightlist
\item
  Lauren Barlog, Obstetrician-Gynecologist
\item
  Brooklyn, N.Y.
\item
  April 16
\end{itemize}

I walked out of the hospital after 24 hours of no sleep, 24 hours of
wearing a surgical mask over an N95, 24 hours of listening to constant
overhead notifications of people coding and in respiratory distress, 24
hours of watching my own pregnant patients hospitalized and isolated and
afraid for their unborn children --- only to find the newly installed
mobile refrigerated morgue parked outside the hospital.

I felt the tears I had been holding back roll down my face. In the car I
took off my mask and even though I was finally able to breathe, it still
felt like the mask was there. As I drove home the roads were empty
except for ambulances, and the streets were filled with a chorus of
sirens.

I thought about my brother, an intern at a nearby hospital in Brooklyn,
himself immunocompromised from a childhood injury that left him without
a spleen, also working on the front lines. I thought about my dad, a
63-year-old physically fit man with an unfortunate history of
cardiovascular disease also working with patients in Buffalo. I thought
about my mom, a 63-year-old healthy woman who had been intubating
patients for elective surgeries well into knowing the pandemic was here
to stay.

I walked inside my small apartment, stripped off all my contaminated
clothes and sobbed on the floor of the bathroom.

\includegraphics{https://static01.graylady3jvrrxbe.onion/packages/flash/multimedia/ICONS/transparent.png}

\begin{itemize}
\tightlist
\item
  Eric M. Thomas, Paramedic and Physician Assistant
\item
  Rochester, N.Y.
\item
  April 16
\end{itemize}

With patients, you have to ask yourself how far behind the eight ball
are they starting? If there was a delay getting to them because they
waited to call 911 or they waited with their symptoms for weeks, we
always ask, how far behind the eight ball did that patient start?

What we're starting to look at and ask each other is, ``Hey is some of
our staff also starting behind the eight ball because of the unique
stressors of Covid-19?''

Some on our staff have had spouses laid off. That's causing them to
defer mortgage payments. And their kids are at home and you're trying to
teach them. And they're also trying to come to work as a paramedic,
which is an extremely stressful job.

On top of that, they're worried about bringing the virus home to the
family. That's everybody's greatest fear.

\includegraphics{https://static01.graylady3jvrrxbe.onion/packages/flash/multimedia/ICONS/transparent.png}

\begin{itemize}
\tightlist
\item
  K. Pradeep Kumar, Orthopedic Surgeon
\item
  Chennai, India
\item
  April 28
\end{itemize}

A few weeks ago, my colleague, Dr. Simon Hercules, contracted the
coronavirus and later died. It was really scary because we don't know
how he got it.

We were seeing people day in and day out. Every time a patient came to
us, or coughed, we knew we could get it. But knowing that someone in our
immediate circle died from it makes it so much more difficult for us. It
shows us just how vulnerable we are.

I went to Dr. Simon's burial to pay my last respects. Somebody started
digging the grave. The local residents peered through their windows.
They saw that people were wearing P.P.E.

They came running from their homes and started yelling at us, saying,
``We live here and you people bring the Covid body here?''

The mob started hitting people and thrashing the ambulance. People were
running with bleeding heads, blood pouring all over their faces.

They managed to put Dr. Simon's body back into the ambulance. I was in
my car. Somehow the police controlled the crowd. I followed the
ambulance from behind. A police barricade stopped the ambulance, and I
stopped my vehicle. I saw that both of the ambulance drivers were
bleeding and not in a position to drive. I had to do something.

Without a windshield, I drove the ambulance back to the burial ground. I
had to put Dr. Simon inside the grave. We had to literally bury him with
our own hands, fearing being attacked again. I was scared for my life
but I knew I had to bury Dr. Simon.

\includegraphics{https://static01.graylady3jvrrxbe.onion/packages/flash/multimedia/ICONS/transparent.png}

\begin{itemize}
\tightlist
\item
  Ansel Oommen, Clinical Microbiology Scientist
\item
  Manhattan, N.Y.
\item
  April 16
\end{itemize}

I have conducted tests on hundreds of patient samples. When the New York
City outbreak first began, my left thumb and left index finger hurt from
uncapping so many specimen tubes. At one point, I worked 20 nights in a
row.

I was the first person to see the percentage of positives in those
results. It was a strange feeling to be in a quiet lab all by myself and
yet feel this immense gravity of a storm looming overhead.

\includegraphics{https://static01.graylady3jvrrxbe.onion/packages/flash/multimedia/ICONS/transparent.png}

Nadia Shira Cohen for The New York Times

\begin{itemize}
\tightlist
\item
   Letizia Rossi, I.C.U. Anesthesiologist
\item
  Rome, Italy
\item
  June 12
\end{itemize}

I had to explain to a patient, a woman about who was about 70, that we
would have to intubate her. She understood she was dying and most likely
would not wake up again. There was this moment of silence and
communication through the eyes --- a still but lucid gaze of the
inability to change things. She died 10 days later.

Life is slowly getting back to normal. However, you live with the
anxiety that we will start to see a new wave of contagion and that we
will find ourselves again working in these very difficult conditions.

\includegraphics{https://static01.graylady3jvrrxbe.onion/packages/flash/multimedia/ICONS/transparent.png}

\begin{itemize}
\tightlist
\item
  Alison MacLeod, Primary Care Nurse Practitioner
\item
  Charleston, S.C.
\item
  April 16
\end{itemize}

I wear a mask every day and wash my hands until they are almost cracked
and bleeding. I am doing at least half my visits by telephone or
telemedicine. I also see people in the parking lot from their cars.

Although I am symptom-free and have not had the same exposure as
intensive care or emergency room nurses, I have a boyfriend who is
fighting cancer. We used to see each other every other weekend, but I
have not been able to talk to him in person, touch him or hug him for
three weeks. He is terrified of catching Covid-19 from me, as I could be
an asymptomatic carrier.

People like him who are undergoing chemotherapy have to go into
treatments alone and have to convalesce by themselves. It makes it a
very lonely and scary experience.

This feels like a nightmare that just won't end. If we get to the other
side of this, I will never again take holding hands for granted.

\includegraphics{https://static01.graylady3jvrrxbe.onion/packages/flash/multimedia/ICONS/transparent.png}

\begin{itemize}
\tightlist
\item
  Viel Catig, Surgical Nurse
\item
  Los Angeles
\item
  July 1
\end{itemize}

It's been a constant change in the past few weeks at my hospital.

At one point, we had 18 patients, most of them ``do not resuscitate''
and on morphine drips. They sounded like they were drowning in a
bottomless pool. I had never seen so much death in my life.

Then we were seeing fewer and fewer Covid patients. Our medical-surgical
Covid unit was able to close and combine with the Covid unit next to us
that monitors patients who are still not able to breathe without
machines like CPAPs or BiPAPs*.*

Then the numbers started rising again, and on the first day of June, we
reopened. It is heartbreaking.

I have found myself settling into a routine --- the ``new normal'' as
they say --- after hectic 12-hour shifts. I try to shake off the feeling
of hypoxia I get from wearing my single N95 mask for the entire shift. I
get myself home and strip out of my work scrubs, take a hot shower, eat
breakfast and then remember to check my temperature and oxygen
saturation.

It is now a daily habit to keep myself in check in case of any Covid
symptoms.

We are tired and burned out.

\includegraphics{https://static01.graylady3jvrrxbe.onion/packages/flash/multimedia/ICONS/transparent.png}

\begin{itemize}
\tightlist
\item
  Dalifer Freites Nuñez, Rheumatologist
\item
  Madrid
\item
  April 8
\end{itemize}

I have learned to work with the uncertainty of what will arise day by
day.

\includegraphics{https://static01.graylady3jvrrxbe.onion/packages/flash/multimedia/ICONS/transparent.png}

\begin{itemize}
\tightlist
\item
  Satoshi Toyama, Anesthesiologist
\item
  Tokyo
\item
  May 4
\end{itemize}

I was sickly when I was small so I decided to go into medicine.

I'm used to putting in ventilator tubes and taking them out, but now, I
have to be really careful to keep the virus from flying all over the
place. This is not something you can handle alone. You have to do it as
a team. There are doctors of all ages, so teaching them all was
difficult.

The first thing patients say to us when they arrive is, ``Yoroshiku
onegaishimasu'' (roughly, ``I'm counting on you'').

Some of them have light symptoms. Some of them are unconscious. Others
ask, ``How did it come to this?''

One recent patient, when I took the tube out, he said, ``I'll always
remember you, doctor.''

Today, there was a patient from India who gave me a thumbs-up when I
took out his tube.

When I see that kind of thing, I feel good. It's a moment of happiness.

\includegraphics{https://static01.graylady3jvrrxbe.onion/packages/flash/multimedia/ICONS/transparent.png}

\begin{itemize}
\tightlist
\item
  Tom Corso, E.R. Physician Assistant
\item
  Sayville, N.Y.
\item
  April 16
\end{itemize}

I've always been able to separate work and home. This is the first time
those two worlds collided. I love my job in the E.R., and I can't
imagine doing anything else with my life. But the thought of potentially
bringing this virus home to my wife and kids scares me like nothing I've
encountered before.

\includegraphics{https://static01.graylady3jvrrxbe.onion/packages/flash/multimedia/ICONS/transparent.png}

Jake Michaels for The New York Times

\begin{itemize}
\tightlist
\item
  Becky Williams, Paramedic
\item
  San Bernardino, Calif.
\item
  April 21
\end{itemize}

I do not know how to adequately express the emotional toll this is
taking. You can't know what it's like to be loading your 88-year-old
patient with a 103.9-degree fever into your ambulance while one family
member is dialing other family members as fast as he can on his
cellphone.

He's telling them all, ``Pop Pop is going to the hospital, tell him you
love him.''

And the family members are telling him on speaker phone, ``We love you
Pop Pop. We're praying for you Pop Pop.'' Over and over as fast as they
can they are trying to tell him goodbye knowing that if he takes a turn
for the worse they might not be able to see him again with the strict
hospital visitor policies in place.

And as a paramedic you swallow your own tears and emotions and remain
professional because you've got a job to do, and that patient needs you
to take care of him in that moment.

I keep telling everyone I feel like I am working in the Twilight Zone
now.

\includegraphics{https://static01.graylady3jvrrxbe.onion/packages/flash/multimedia/ICONS/transparent.png}

\begin{itemize}
\tightlist
\item
  Kayla Sudduth, E.R. Technician
\item
  New Bedford, Mass.
\item
  April 16
\end{itemize}

The most intense experience in fighting Covid-19 for me is bearing
witness to what true bedside manner and health care looks like despite
fear.

When my co-workers get into the rooms and talk with the patients, it's
like the virus doesn't exist. We joke. We laugh. We talk about their
families.

\includegraphics{https://static01.graylady3jvrrxbe.onion/packages/flash/multimedia/ICONS/transparent.png}

\begin{itemize}
\tightlist
\item
  Karla Chamorro, General Practitioner
\item
  San José, Costa Rica
\item
  April 8
\end{itemize}

This virus has made me fear the profession I so dearly love. Sometimes I
feel like I signed up for this and I should be feeling guilty for how
scared I feel. I have become stressed to the point that I keep believing
I caught Covid-19 every time I feel extra tired, even after a 13-hour
shift. I hear people call me a hero but remain incapable of thinking
that of myself.

\includegraphics{https://static01.graylady3jvrrxbe.onion/packages/flash/multimedia/ICONS/transparent.png}

\begin{itemize}
\tightlist
\item
  Hesham Hassaballa, Pulmonary and Critical Care Doctor
\item
  Hinsdale, Ill.
\item
  April 16
\end{itemize}

This crisis has made me a much better doctor. The skills I've gained
taking care of super sick Covid patients will help me take care of all
other patients in the I.C.U. I will be much more vigilant about hand
washing. I will never take P.P.E. for granted.

\includegraphics{https://static01.graylady3jvrrxbe.onion/packages/flash/multimedia/ICONS/transparent.png}

\begin{itemize}
\tightlist
\item
  Doug Entenman, Nurse
\item
  Dover, Del.
\item
  Died on May 10
\end{itemize}

Doug Entenman developed a fever during his shift on April 20, so he got
a coronavirus test and dropped it off at the Delaware health department
along with a batch of others from work.

Mr. Entenman was a supervisor at a nursing home and rehabilitation
center where he had worked for five years. One of his jobs was to
oversee virus testing for workers and residents.

``My mom would text every day because she was so worried about him,''
said his niece, Lindsay Fitzpatrick. ``He said he was being cautious and
just accepted it as part of his job.''

Mr. Entenman, who was 50, came from a family filled with medical
workers. Before becoming a nurse, he was a paramedic. He once helped
save the life of a secret service officer who had a heart attack during
an appearance by President Clinton in Norristown, Pa., the town he grew
up in.

By the time Mr. Entenman's test came back positive, he had gotten worse.
He spent several days at home, where he lived with his son Charles
Entenman, who is 22. Finally, his breathing became so labored they
decided to call 911.

Soon, doctors wanted to put him on a ventilator. He resisted.

``I suspect this may not end well for me,'' he texted family members.

He had just a short time to put his affairs in order. A Catholic priest
heard his last confession and administered the sacrament of the sick
over FaceTime.

He texted his family one last time. He loved them, he said, and asked
that they hold his children close.

--- KIM SEVERSON

\includegraphics{https://static01.graylady3jvrrxbe.onion/packages/flash/multimedia/ICONS/transparent.png}

Sasha Arutyunova for The New York Times

\begin{itemize}
\tightlist
\item
  Chinazo Cunningham, Internist
\item
  Nyack, N.Y.
\item
  June 15
\end{itemize}

I work in the Bronx. Ninety-eight percent of my patients are people of
color, and most are poor. We see exactly how structural racism plays out
with Covid-19. It is so easy for us to tell people to go home and
self-isolate. It is much more difficult for most of my patients who are
Black and brown to do that. Their housing doesn't afford for isolation.

There was a day in March that was the worst day of my career. We had no
protective gear. We had to wait for the next truck that was on its way
to deliver some to the hospital before we could see our patients. Then I
walked into my patient's room and the person next to him had just died
and was in a body bag. The patient, a Hispanic man, looked at me and I
could see the fear in his eyes.

He said, ``I know I'm going to die,'' and I didn't know what to say. I
knew he was going to die too. And he knew that I knew he was going to. I
didn't have much to offer him and he was alone. Within 24 hours, he was
on a ventilator and about 10 days later he died.

What I've really focused on in my career is racism issues in health
care. The same reckoning needs to happen in other areas of society.

The number of my colleagues who are very interested and actively
participating in protests gives me hope. People have reached out to me
and said that they don't want to just send an email. They want to be
very mindful of what we can actually do, and who should be doing the
work.

I'm cautiously hopeful. I'm also tired, frankly, because this is not
new.

\includegraphics{https://static01.graylady3jvrrxbe.onion/packages/flash/multimedia/ICONS/transparent.png}

September Dawn Bottoms for The New York Times

\begin{itemize}
\tightlist
\item
  Leigh Barrow, E.R. Physician
\item
  Tulsa, Okla.
\item
  June 10
\end{itemize}

My family moved back home! They had moved out so I wouldn't bring the
virus home to them. It was hard to see patients all alone in their worst
hour and then come home, alone, in my difficult hour. I was able to hug
my kids on Easter. I put on my Tyvek suit and dressed up as the Easter
bunny.

But it's so nice now to hug, kiss and love on my littles.

What's frustrating now is living in a city that's wide open and pretends
Covid-19 never existed. People have stepped ahead of us in line at the
doughnut shop when we were six feet from the next person. When I kindly
tell them that we are social distancing, they remind me this phase is
over and we are back to normal.

Patient volume in our emergency room is back up, now that businesses are
back to normal. We still assume anyone can have the virus. Having to
gown on and off for every patient is time consuming and difficult,
especially when there are more patients in the waiting room. I still
have to intubate with a plexiglass box with two hand holes that make me
feel like a puppeteer as I place a breathing tube down my next patient's
throat. I am still wearing anti-fog rainbow goggles and a welder's mask.
I do all of this so I can hug my kids without concerns for Covid.

\includegraphics{https://static01.graylady3jvrrxbe.onion/packages/flash/multimedia/ICONS/transparent.png}

\begin{itemize}
\tightlist
\item
  Maria Walls, Nurse Practitioner
\item
  St. Louis
\item
  April 16
\end{itemize}

I witnessed a family sobbing, watching on a phone over FaceTime as the
father passed away in the I.C.U. It was unclear whether he was in the
regular I.C.U. or the Covid I.C.U., but there cannot be any visitors in
either. You never forget the sound of a ventilator or the sound of
flatlining on the monitor. I could hear it while standing on the other
side of the room. My heart aches for them.

\includegraphics{https://static01.graylady3jvrrxbe.onion/packages/flash/multimedia/ICONS/transparent.png}

\begin{itemize}
\tightlist
\item
  JuJu Linssen, Paramedic and Field Training Officer
\item
  Stewartville, Minn.
\item
  April 16
\end{itemize}

My exposure happened on a Wednesday around 6 p.m. I had a few more hours
left on my shift. I wasn't wearing any P.P.E. It was during a simple
chat as my co-worker arrived for work. I took my surgical mask off and
ate a quick bite of chocolate. My co-worker hadn't placed their mask on
yet.

The next day my co-worker was tested for Covid-19 because of a scratchy
throat.

The nurse from occupational health completed a survey with me to rate my
exposure risk. My heart sank like a knot into my stomach as I told her I
wasn't wearing a surgical mask and I was eating.

I had to be quarantined, and moved to the basement.

The hardest part is being separated from my family. My daughter reached
16 months while I was in here and she seems to be growing like crazy!
She doesn't understand and still asks for me.

Besides my family, I miss going to work. I miss my co-workers. Some of
them are just like family, too. I miss not being there to help people. I
am so ready to get back out there.

\includegraphics{https://static01.graylady3jvrrxbe.onion/packages/flash/multimedia/ICONS/transparent.png}

\begin{itemize}
\tightlist
\item
  Rachel Zang, E.R. Physician
\item
  Philadelphia
\item
  May 5
\end{itemize}

A man came in struggling to breathe and clearly needing to be intubated
urgently. Because of the no visitor policy, he was alone in the E.R. and
asked to call his wife before we intubated him.

``I will talk to you soon. I love you,'' he said as he ended the
conversation.

The whole room had tears behind their masks. We knew he might never talk
to her again.

We intubated him successfully, but he passed a few days later.

These experiences have slowly taken a toll on all of us. We are asked to
be both support system and physician for these patients whose families
can reach them only by phone.

It's unlike anything we've ever experienced before, and it will leave a
lasting, painful impact on an entire generation of physicians.

\includegraphics{https://static01.graylady3jvrrxbe.onion/packages/flash/multimedia/ICONS/transparent.png}

\begin{itemize}
\tightlist
\item
  Eleanore Kim, Ophthalmologist
\item
  Rye, N.Y.
\item
  April 16
\end{itemize}

I remember a young pregnant woman we admitted to the hospital. It was
one of my first shifts on a Covid medicine floor and she was probably
one of the first Covid-19 patients I had met.

Our conversation was short. She struggled to finish sentences due to her
labored breathing. She wanted to know if her baby would be OK.

We assured her that we would take care of her.

She was put on oxygen and was saturating well. She was young and without
pre-existing medical problems. I imagined she would stay in the hospital
a short while and return to her husband and first child.

I had a few nights off. When I came back to the hospital, my colleague
told me she had been intubated. Her premature baby had been delivered
and was also on a respirator.

I'm back to taking care of eyes, so I don't know how she is doing. I try
to imagine that she is recovering and will see her baby soon. I know
that may be far from the case.

\includegraphics{https://static01.graylady3jvrrxbe.onion/packages/flash/multimedia/ICONS/transparent.png}

\begin{itemize}
\tightlist
\item
  Awah Kenneth Achua, Doctor
\item
  Buea, Cameroon
\item
  April 30
\end{itemize}

I work with the Doctors Without Borders mission in Buea, Cameroon.

This is a new disease and very little is known about it. With
conflicting guidelines and constant updates, it becomes very difficult
to organize training sessions.

My main worries are the absence of P.P.E. for medical personnel and
limited testing facilities.

Dispelling rumors and fake information about Covid-19 has been another
challenging task. Some people don't even believe the virus exists.

\includegraphics{https://static01.graylady3jvrrxbe.onion/packages/flash/multimedia/ICONS/transparent.png}

\begin{itemize}
\tightlist
\item
  Maria Saavedra Karlsson, Pediatrician
\item
  Hermosa Beach, Calif.
\item
  April 16
\end{itemize}

As horrible as these last few months have been, there has been beauty
among the devastation.

\includegraphics{https://static01.graylady3jvrrxbe.onion/packages/flash/multimedia/ICONS/transparent.png}

\begin{itemize}
\tightlist
\item
  Luke Hashiguchi, Palliative Care Doctor
\item
  Akron, Ohio
\item
  April 17
\end{itemize}

I held my tears in today at work only to come home and break down. I
cried. I ugly cried.

\includegraphics{https://static01.graylady3jvrrxbe.onion/packages/flash/multimedia/ICONS/transparent.png}

Adam Flamer-Caldera

\begin{itemize}
\tightlist
\item
  Christopher Reverte, E.R. Doctor
\item
  Manhattan, N.Y.
\item
  April 17
\end{itemize}

My wife and I had our first child at the end of February. I had planned
to take off most of March for paternity time, only to end up having
Covid-19 shortly after the birth of our daughter. When I returned to
work in the E.R. at the beginning of April, I decided to stay alone in a
hotel to protect my loved ones.

I have realized how quickly a crisis can cripple our health care system.
I try to stay strong and continue to fight this battle.

\includegraphics{https://static01.graylady3jvrrxbe.onion/packages/flash/multimedia/ICONS/transparent.png}

\begin{itemize}
\tightlist
\item
  Giovanna Cezarino, Infectious Disease Doctor
\item
  São Paulo, Brazil
\item
  April 30
\end{itemize}

It's very worrisome to hear our president say that people should not be
respecting social isolation, because it's leading people to relax.
People don't understand the gravity of the situation, leaving us health
care professionals terrified. Brazil had already seen examples of other
countries proving that social isolation helps prevent hospitals from
overcrowding. Now we are facing the worst part of the outbreak, and I
believe it's going to get worse if we don't stick with social isolation.
I see the streets packed with people. That worries me so much. The
I.C.U. units are already overcrowded. Renowned public hospitals are
packed. We won't have enough hospital beds soon. I feel afraid for
Brazil's future.

The hardest moment so far was seeing an 8-year-old come into the
emergency ward with asthmatic symptoms. When she was intubated it
impacted me greatly. You never expect to see a child so young in
intubation. Thankfully she was able to go home, but it could have easily
gone the other way. I was constantly keeping an eye on her, very
worried.

With everything that has happened, I am not the same person. I don't
even recognize myself anymore. Never have those of us on the front lines
thought so much about other people as we do now. Although there is a lot
of fear, many of us are living more in the spirit of community and less
with the spirit of competition.

\includegraphics{https://static01.graylady3jvrrxbe.onion/packages/flash/multimedia/ICONS/transparent.png}

\begin{itemize}
\tightlist
\item
  Rajat Thawani, Internist
\item
  Brooklyn, N.Y.
\item
  April 17
\end{itemize}

There is a lot of guilt, a lot of fear. I do not connect with patients
like I am used to doing. But then again, so many of my patients are
dying that connecting with them only to lose them tomorrow would be
painful.

\includegraphics{https://static01.graylady3jvrrxbe.onion/packages/flash/multimedia/ICONS/transparent.png}

Gianni Cipriano for The New York Times

\begin{itemize}
\tightlist
\item
  Ilaria Sommonte, Nurse
\item
  Naples, Italy
\item
  April 24
\end{itemize}

When the Coronavirus appeared in China, I never thought it would involve
Italy so directly and in such a short time.

I still remember an elderly, gentle-looking man who came to us with
already well-known diseases, including Alzheimer's. Each day, you lived
a different experience with him. Sometimes he did not understand where
he was. Other times he was convinced that I was his mother or his aunt.
Other times, he silently watched me while I was doing my job.

Covid-19 worsened his situation, hindering him from breathing freely,
making it difficult for him to even drink a glass of water. During a
night shift, he passed away silently. That silence echoes in my head
because his two daughters had called us daily to get news of their
father, but that night they could not wail in pain or hug the man who
raised them for the last time.

This Covid crisis has totally changed the way I live my life. I have
always been shy and sensitive. I thought I was a weak person. Now I am
discovering that I have power and courage above all my expectations.
Often, in family life or in the workplace, I am giving strength to
people much older than me who are blinded by anxiety.

\includegraphics{https://static01.graylady3jvrrxbe.onion/packages/flash/multimedia/ICONS/transparent.png}

\begin{itemize}
\tightlist
\item
  Hillary Duenas, Psychiatry Resident and Research Scientist
\item
  New York City
\item
  June 13
\end{itemize}

The social movements that come from this could be the most important
part of the coronavirus virus pandemic. That relates to both addressing
social injustices in the street and also addressing social injustices in
health care.

People of color in marginalized communities are dying at higher rates,
and there's no reason that should be happening. There are still
communities in New York City where the coronavirus rate is increasing. I
think the most responsible thing to do is to speak up.

I have emailed people at the department of health, I've emailed the
mayor, and I've emailed hospital management saying, let's get testing
and health care into the communities that need it. Let's get a mobile
bus and test people in communities where the testing rate is not
optimal.

There's typically a hesitancy among health care workers to talk about
what we experience and what we see, and now is the right time for that
to change. I think everyone who was on the hospital floors during the
height of the pandemic had a moment where they thought, ``We could die
of this.'' And if that's the case, we don't have time not to speak up
about injustice.

\includegraphics{https://static01.graylady3jvrrxbe.onion/packages/flash/multimedia/ICONS/transparent.png}

\begin{itemize}
\tightlist
\item
  Chan Pich, E.R. Nurse
\item
  Greensburg, Pa.
\item
  June 9
\end{itemize}

I left my job and family to go to New York. When 9/11 happened, I wanted
to help, but I was too young to have anything to offer. I felt that this
was my second chance.

I arrived during the peak of the crisis in New York. On my first day, I
was thrown into the ``hot zone,'' wearing what looked like something
astronauts would wear.

The E.R. was overwhelmed, and everyone was doing as much as they could
to keep up.

There were no visitors allowed, so patients were crashing and taking
their last breath with no family members by their side.

That was the hardest part: the thought of dying alone, and if I were to
get sick, that I too would die there in New York without any of my
family by my side.

When I came back home, my wife wanted me to stay in a hotel for two
weeks to quarantine. All I wanted was to stay in my house and sleep in
my own bed.

\includegraphics{https://static01.graylady3jvrrxbe.onion/packages/flash/multimedia/ICONS/transparent.png}

\begin{itemize}
\tightlist
\item
  Patrick Dougherty, Physician Assistant
\item
  Mattawan, Mich.
\item
  April 17
\end{itemize}

Leaving to go home is the most intense moment. Not knowing if there is
something in my nose, on my skin or on my clothes. Not knowing if I'm
bringing illness to my household.

Death has been here all along. Transporting death to my loved ones has
not.

\includegraphics{https://static01.graylady3jvrrxbe.onion/packages/flash/multimedia/ICONS/transparent.png}

\begin{itemize}
\tightlist
\item
  Zagidat Amayeva, Nephrologist
\item
  Kizilyurt, Russia
\item
  Died on May 11
\end{itemize}

As the virus descended on Russia early this spring and hospitals
canceled planned procedures, Zagidat Amayeva saw no choice but to keep
working: Her chronically ill patients needed regular dialysis to stay
alive.

Her clinic opened just last year in a town on the northern edge of the
Caucasus Mountains, in the southern Russian republic of Dagestan. Now
she told her daughter, Tamara Aliyeva, that she was worried. Yes, her
clinic had a supply of masks, gloves and gowns. But her patients, who
spent hours there every week, did not appear to be taking the pandemic
seriously, even as the government ordered people to stay home.

``Even if the doctors do everything according to the rules,'' Dr.
Amayeva's daughter recalls her mother saying, ``it doesn't mean that the
patients themselves will do the same.''

By mid-April, after treating at least one Covid-19 patient, Dr. Amayeva
was feeling sick.

Two weeks later, her condition worsening, she stood in line for nearly
an hour for a CT scan. It showed pneumonia in both lungs, but with the
hospitals full, Dr. Amayeva insisted on going home.

Eventually Ms. Aliyeva prevailed on her mother to go back to the
hospital, where she was soon in intensive care. The two spoke the
morning of May 11, and Dr. Amayeva asked for a fresh change of clothes.
She died less than an hour later. She is survived by her mother; a
brother and a sister; her daughter, Ms. Aliyeva, and her son, Maksim
Chizhikov; and a grandson.

--- ANTON TROIANOVSKI

\includegraphics{https://static01.graylady3jvrrxbe.onion/packages/flash/multimedia/ICONS/transparent.png}

\begin{itemize}
\tightlist
\item
  Sheridan Brown, Nurse Practitioner
\item
  Manhattan, N.Y.
\item
  April 17
\end{itemize}

I'm a nurse practitioner in outpatient vascular surgery, but I
volunteered and was deployed to a sister hospital's Covid I.C.U. My
background didn't necessarily qualify me, but I wanted to help.

We essentially put out fires every day --- it's all we can do. I cared
for one young man whose mother I would FaceTime every shift. He reminded
me of a friend --- his size, his face, his cheeks. You could tell he was
everyone's favorite, the sweetest guy. I wanted him to make it so badly.
I was so hopeful his youth would pull him through, but that was an
impossible task.

\includegraphics{https://static01.graylady3jvrrxbe.onion/packages/flash/multimedia/ICONS/transparent.png}

Samuel Aranda for The New York Times

\begin{itemize}
\tightlist
\item
  José Manuel Siurana, Pediatric Cardiologist
\item
  Barcelona, Spain
\item
  April 18
\end{itemize}

I usually work in a children's hospital. It was the middle of March when
my medical director called and said, I need you to go to another
hospital and see adult Covid patients. It was like, one minute to think
about it, and I go, ``OK.''

My wife said, ``Oh God, it's very dangerous, you will be on the front
line with Covid.'' But I said, ``Don't worry, I am a doctor, and I
studied for this.'' For me, it's like, ``OK, I am the right man for this
work, so I have to go.''

My biggest worry was avoiding contact with my two daughters, 6 and 8
years old.

The first day was a shock. Seeing the numbers on the TV is not the same
as when you see the faces of the patients. My first patient, I think he
was 74, and he died that same morning. And it's like, wow, what is this
virus?

The heart problems of children are very different from the heart
problems of adults. Before I saw babies, and in the Covid hospital I see
patients who are usually 75 to 85 years old, so it's an extreme change.

Us pediatricians, we like to keep near to the patient. The children
usually give me a high-five or hugs. Wearing three layers of clothes, my
face covered by two masks and goggles, it is very difficult. But I try
to take the elderly patients' hands when they are very sick.

Only one month ago, I had another life, you know? But we are seeing the
light at the end of the tunnel now. And I think that no patients will
die tomorrow.

\includegraphics{https://static01.graylady3jvrrxbe.onion/packages/flash/multimedia/ICONS/transparent.png}

\begin{itemize}
\tightlist
\item
  Chas Carlson, Paramedic
\item
  Philadelphia
\item
  April 18
\end{itemize}

We were called out for a female in her 60s with altered mental status. I
entered the townhouse and my partner remained outside, which is the new
protocol to limit exposure.

I walked through the house in full gown, glasses, gloves and N95 mask.
The husband was yelling for help in the back. There was his wife, who
had been experiencing ``flulike symptoms'' for a couple days, sitting in
a chair with a zombielike gaze. She was conscious, but would not respond
to my questions.

Her heart rate was low. Normally, we would treat any critical illness
right there in the home, but I did not bring in much equipment in an
attempt to not contaminate it. I quickly moved her to the stretcher and
ambulance, where I could work more safely.

It all felt wrong. It's against everything we've been taught about
treating critical illness immediately.

As we were leaving, I told the anxious husband he could not come to the
hospital --- also a new policy. He became desperate, trying to reason
with us to let him stay with his wife.

Here she was, being carted off by men in bright white suits with masks
and goggles, and he didn't know what was going to happen to her. We flew
off to the E.R. as he stood in his doorway.

\includegraphics{https://static01.graylady3jvrrxbe.onion/packages/flash/multimedia/ICONS/transparent.png}

\begin{itemize}
\tightlist
\item
  Madhavi Parekh, Pulmonary and Critical Care Doctor
\item
  Manhattan, N.Y.
\item
  April 18
\end{itemize}

Early in the pandemic, I had a young patient with Covid-19 who was in
respiratory failure and needed intubation. The patient looked me in the
eye and said, ``I'm afraid of dying.'' I responded, ``You will be OK,
and we will take care of you,'' as I have said many times before. But I
felt uncertain in my words.

On the same day, my parents told me they were sick, and soon after, they
tested positive. My mother, a neonatologist, is the person who inspired
me to become a doctor when I was very young.

The following two weeks were among the most stressful of my career and
life, FaceTiming my parents several times a day while taking care of
patients in the I.C.U. and juggling a steady stream of consults from
other hospitals. It was hard to focus, sleep or live without constant
anxiety. I was losing confidence in my ability to take care of people,
including my own loved ones. But as the patients kept coming in, I had
to keep going.

My patient and parents were among the lucky ones who recovered, but it
took weeks. This roller coaster has been unlike anything I could have
ever imagined.

\includegraphics{https://static01.graylady3jvrrxbe.onion/packages/flash/multimedia/ICONS/transparent.png}

\begin{itemize}
\tightlist
\item
  Nicanor Baltazar, Nurse
\item
  Queens, N.Y.
\item
  Died on March 31
\end{itemize}

``What happened to me?'' Nicanor Baltazar texted his wife on the phone
she had placed in his hand when he was admitted to the emergency room.
It was the last message he sent her.

Mr. Baltazar, who was 60, worked as a nursing director at a long-term
care facility in Flushing, Queens. He'd come to New York from the
Philippines as a geriatric nurse. After four decades of work, he boasted
about how many of his patients he could still remember.

He was immensely proud of his job, his wife, Grace, said, and often
skipped their morning walk to arrive early.

Mr. Baltazar liked to work with new nurses because he saw every
interaction with them as a teachable moment. ``He'd never get mad,''
said a colleague, Pamela Matias. ``He let them spread their wings.''

In the early months of 2020, colleagues threw Mr. Baltazar a surprise
birthday party and he danced with residents in the facility's dining
hall. But as the virus approached, he began to frantically prepare. ``It
consumed him,'' Ms. Baltazar said. ``At one point, he texted me and
said, `I am so tired of this coronavirus,' but it hadn't even started.''

On Friday March 20, he secured a box of masks for his staff before
heading home to cook his wife a Lenten dinner of fish. He showed his
love through cooking, she said. Afterward, he came down with a slight
fever and then started coughing. His symptoms only got worse. In the
early hours of March 31, he struggled to breathe and was taken to the
hospital.

When Ms. Baltazar turned on his phone after he died, she discovered that
he had been texting colleagues throughout his illness. Some had worried
about protective gear. Others were out sick. And he had given advice on
patients' care.

--- AIDAN GARDINER

\includegraphics{https://static01.graylady3jvrrxbe.onion/packages/flash/multimedia/ICONS/transparent.png}

\begin{itemize}
\tightlist
\item
  Rita MacDonald, I.C.U. Nurse
\item
  Royal Oak, Mich.
\item
  April 18
\end{itemize}

Nursing has changed. While we fight this virus, I no longer know if my
patient is a veteran, or what they do for a living, or anything about
their family. I don't know what channel they enjoy, what music they
like, if they like lots of blankets or only a few.

One of my first Covid patients had a tattoo across his chest that read,
``FAMILY.'' That's all I knew about him. Another had parents who called
daily and cried. ``Do you have sons? He's our only one. Please save
him.'' Another woke up and couldn't remember how to speak English. One
young man had beautiful blue eyes. His girlfriend never called. I
wondered if she was sick, too.

They don't see my smile or my concern for them through my mask, shield,
glasses, hat, gown and gloves. We all look alike to our patients. Not
only are they alone, but now it's difficult to reassure them even with a
soft touch. What more can Covid take?

\includegraphics{https://static01.graylady3jvrrxbe.onion/packages/flash/multimedia/ICONS/transparent.png}

Gianfranco Tripodo for The New York Times

\begin{itemize}
\tightlist
\item
  José Satué, Internist
\item
  Madrid
\item
  June 9
\end{itemize}

We are not heroes. We are not God. We are human. We suffer for our
patients, for their families, for us and for others.

In April, I had much more work because the patients in our hospital
doubled. For more than two months, we have not had time to attend to
anyone other than Covid-19 patients.

Now, the hospital no longer has just Covid patients and we are trying to
resume normal activity.

I am fine and trying to return to normal.

But it is hard to verify that life is still out there. Those of us who
live close to the horror cannot see it the same again.

It's difficult to see friends and family and not be able to hug them,
because a kiss can be an act of risk rather than one of love.

We will not be the same again.

\includegraphics{https://static01.graylady3jvrrxbe.onion/packages/flash/multimedia/ICONS/transparent.png}

\begin{itemize}
\tightlist
\item
  Anna Condino, E.R. Doctor
\item
  Seattle
\item
  April 16
\end{itemize}

In the last week, I worked 84 hours. After a rush of patients one
evening, we finally admitted or discharged all our patients from the
respiratory emergency department. A nurse and I sat down in front of the
heat lamps that warm the outdoor respiratory triage area. It was 4 a.m.
and the temperature was in the 20s. I literally peeled my N95 mask off
my nose and took my first deep breath in 10 hours.

I am more singularly focused than I've ever been in my life.

We've all taken on administrative tasks as we transform our E.R. space
and systems. Last week, I ordered medical equipment and sourced P.P.E.
between my shifts.

When we see Covid patients, we're trying to minimize the amount of
contact so we don't get sick. I'm starting IVs, cleaning rooms, hanging
fluid bags, anything to keep someone else from having to come in. And
the E.R. nurses do the same for me. Emergency medicine has always been a
team sport, but it's absolutely critical now.

\includegraphics{https://static01.graylady3jvrrxbe.onion/packages/flash/multimedia/ICONS/transparent.png}

\begin{itemize}
\tightlist
\item
  Diane Cusick, Nurse
\item
  Keyport, N.J.
\item
  April 18
\end{itemize}

I asked to be reassigned to the E.R. because we're no longer doing
elective procedures at the outpatient department where I work full-time.
With 15 years of previous E.R. experience, I knew my colleagues could
use the help.

In all my years of nursing, I have dealt with an extraordinary number of
people dying. To watch Covid-19 patients die alone, without their loved
ones, is something else entirely.

\includegraphics{https://static01.graylady3jvrrxbe.onion/packages/flash/multimedia/ICONS/transparent.png}

\begin{itemize}
\tightlist
\item
  Jonathan Ilgen, E.R. Physician
\item
  Seattle
\item
  June 16
\end{itemize}

My family and I attended a miles-long car procession. It was our attempt
to protest in a socially distanced fashion. But it didn't feel as
impactful as actually marching in the streets. So we then attended a
march that was organized by and centered around health care workers.

I was primarily struck by seeing the many ages and different health
professionals marching together. There were small children, medical
students, resident trainees and some of the most senior physicians in
our hospital system.

Our hospital system was truly dependent on Seattle citizens to buy in to
ideas of social distancing and masks as ways to flatten our curve. I
have been so thankful for the leadership that gave us a fighting chance.

Yet it is abundantly clear that Black and brown patients suffer many
more health consequences from the pandemic, and perhaps this is why
these issues of police brutality feel so close to home. Why don't we see
this same level of engagement and mobilization around racism as a public
health crisis?

\includegraphics{https://static01.graylady3jvrrxbe.onion/packages/flash/multimedia/ICONS/transparent.png}

\begin{itemize}
\tightlist
\item
  Hussein Al Abbasi, E.R. Doctor
\item
  Wyong, Australia
\item
  April 24
\end{itemize}

When you can see someone is dying and you can bring them back to life,
that is something amazing. The satisfaction you get from that is beyond
description.

I moved to Australia from Iraq in 2014. When you work in war zones, you
work with limited resources and put your life in danger. You question:
``Why am I doing this?'' And you remember, and this is the part that
keeps you feeling human. Every person I bring back to life, I think that
I'm still connected to that person.

We're facing one of the toughest times ever now. Everyone is scared. We
have the extremes: Lots of people are showing gratitude and lots of
people are treating us like infection-spreading machines. At the moment
I'm in one of the motels near the hospital. I was asked to evacuate the
room I was renting because I work in a hospital. The landlord was very
worried about her health. The thing is, I cannot blame her. But at the
same time, I was devastated. For a couple of days, I was really feeling
bad about myself as a person. I've never experienced that. It made me
think that I'm the one who is responsible for spreading that infection,
instead of the one who is trying to fight it.

\includegraphics{https://static01.graylady3jvrrxbe.onion/packages/flash/multimedia/ICONS/transparent.png}

\begin{itemize}
\tightlist
\item
  Taylor Laufer, Surgical Nursing Aide
\item
  New York, N.Y.
\item
  April 18
\end{itemize}

I'm just finishing up nursing school, but graduation has been replaced
with a virtual ceremony and the licensure exam has been rescheduled. I
will still be a nurse, but right now I am happy I am fighting this
pandemic as a nursing aide. I can still make a difference.

This crisis has made us hyperfocused on hygiene and disciplined about
protection. I have been mindful with every move I make during my shift.
A lot of people are on edge and the anxiety is at a new level.

We do everything in our power to protect ourselves, our colleagues and
our patients, but there's always that fear of bringing this home. The
worst thing a nurse can think about is being responsible for someone
else's illness.

\includegraphics{https://static01.graylady3jvrrxbe.onion/packages/flash/multimedia/ICONS/transparent.png}

\begin{itemize}
\tightlist
\item
  Elizabeth Pitre, I.C.U. Nurse
\item
  Chicago
\item
  June 8
\end{itemize}

I was the first nurse to have a death on my unit. No family was allowed
to visit. I put the daughter and husband on speakerphone. I sobbed while
the patient slowly died as her family screamed out for her. This was
their mom and wife.

I have had as many deaths in these past four months, as I have had in my
entire 10-year career as an I.C.U. nurse.

I feel like patients are becoming numbers. I don't know their stories. I
am starting to look at them as statistics.

It's been depressing. It's been hard to process.

I became a godmother in mid-April and have yet to hold my niece. I have
no idea when I'll be able to give any of my family members a hug, let
alone be invited into their homes, or feel comfortable being around my
friends.

\includegraphics{https://static01.graylady3jvrrxbe.onion/packages/flash/multimedia/ICONS/transparent.png}

Christopher Smith for The New York Times

\begin{itemize}
\tightlist
\item
  Sewon Lee, Surgical Oncology Nurse
\item
  Salina, Kan.
\item
  April 17
\end{itemize}

As a medical-surgical nurse working at a small hospital, I didn't think
this virus would affect me much. Starting in March, however, patients
with suspected coronavirus infections were admitted. Other patients at
the hospital were scared of catching the virus from us nurses.

One day, a patient noticed my ethnicity and said, ``Thank you for the
coronavirus.'' Usually, I try not to mind offensive comments from
patients, but I was heartbroken that day. Coming from Korea and living
here without my family, I was so scared of everything going on. I feared
that I might catch the virus at work, end up on a ventilator, and die
without my family by my side.

When my mom called me that day, she asked me to come back home. I held
my tears back and told her that this is my place now, and that they need
me here more than my home country does.

\includegraphics{https://static01.graylady3jvrrxbe.onion/packages/flash/multimedia/ICONS/transparent.png}

\begin{itemize}
\tightlist
\item
  Lauren Charlton, I.C.U. Nurse
\item
  Summit, N.J.
\item
  April 8
\end{itemize}

I picked up a patient from the emergency department who was
deteriorating quickly. It was a mad dash to intubate him. We had to prep
medications, secure a ventilator and get the doctor and respiratory
therapist in the room. The doctor came in swiftly and was ready to go. I
realized that we had barely said a word to this patient since we met
him.

I untangled the TV remote, brought it to the patient and asked him if
there was any specific music he would like to listen to before we
intubated him and while he was sedated.

His face lit up a bit, as much as it could. He smiled and asked, ``Do
you have Pink Floyd?''

I immediately cried, found some Pink Floyd and checked on him every day
after to make sure his music was playing.

Every time I tell that story, or even think of Pink Floyd, I weep.
Fortunately, the patient was weaned off the ventilator and transferred
to a regular floor. He might never remember that moment but I know I
will for the rest of my life.

\includegraphics{https://static01.graylady3jvrrxbe.onion/packages/flash/multimedia/ICONS/transparent.png}

\begin{itemize}
\tightlist
\item
  Dawn Zhao, Internal Medicine Resident
\item
  New York, N.Y.
\item
  April 18
\end{itemize}

I was on the front lines as Covid came to New York City.

After half my patient list turned into Covid patients, I found myself
crying a lot. I would go into the shower, and I would cry. I would sit
on the train, and tears would fill my eyes. I would be on the phone at
work trying to explain to someone with no medical background that I
couldn't find P.P.E. for whatever reason, and I would have to stop
talking for a few seconds so the lump in my throat could fade away. I
was scared for the patients and their families; I was scared that my
co-residents might get sick; I was scared my husband would catch it from
me.

Best-case scenario, most health care workers will go through adjustment
disorder. Worst-case scenario, we will have P.T.S.D.

\includegraphics{https://static01.graylady3jvrrxbe.onion/packages/flash/multimedia/ICONS/transparent.png}

Gianni Cipriano for The New York Times

\begin{itemize}
\tightlist
\item
  Gabriele Somma, Nurse
\item
  Naples, Italy
\item
  April 30
\end{itemize}

It was an average afternoon in the ward, as average as a pandemic
afternoon may be. We were told a patient in semi-critical condition was
coming. He had a high fever, a persistent cough, 93 percent oxygen
saturation.

In a word: coronavirus.

We promptly equipped the room. As soon as we received him, we prescribed
oxygen therapy.

Sadly, it was not enough. The situation was worsening as we started
considering sub-intensive care, but the patient's glance, with a silent
eloquence, begged us not to send him ``downstairs.''

So, we took one last chance before sub-intensive care, and tried
tocilizumab, a new experimental treatment. It lasted more than an hour,
during which we went in and out checking his blood saturation.

Every 1 percent increase was followed by our cheers and celebration.
Within two hours, it rose to 95 percent, which, while not ideal,
symbolized a tiny victory against the invisible monster, an emblem of
hope at a time when too many stories lack a happy ending.

\includegraphics{https://static01.graylady3jvrrxbe.onion/packages/flash/multimedia/ICONS/transparent.png}

\begin{itemize}
\tightlist
\item
  Valerie Romo, Field Nurse
\item
  Palos Hills, Ill.
\item
  June 8
\end{itemize}

We are the first ones there before the ambulance arrives. We are the
ones making the decisions on our own about whether a patient is about to
arrest or go into respiratory distress.

We are also caring for patients at home who had been hospitalized for
Covid-19 and are re-entering the very place with younger family members
who most likely infected them.

This isn't the hospital, where you can isolate the patient in a negative
pressure room. We are working alone in the home with sometimes more than
one family member who has Covid.

Patients are not fully recovered and we are concerned about
re-infection. It's like walking into a ``hot house.''

They need us.

I have asked myself if the risk to my family was worth it. I realized in
my 21 years as a nurse that this was truly my calling, ingrained in my
heart. I felt that deep inside, this is who I am.

\includegraphics{https://static01.graylady3jvrrxbe.onion/packages/flash/multimedia/ICONS/transparent.png}

\begin{itemize}
\tightlist
\item
  Atif Sohail, Cardiologist
\item
  Arlington, Texas
\item
  April 18
\end{itemize}

Early on I was scared by the thought of having to see patients who were
highly contagious. But I realized it was a unique opportunity to support
them when none of their family members could be there. I talk to them
and reassure them that they are not alone in this fight.

\includegraphics{https://static01.graylady3jvrrxbe.onion/packages/flash/multimedia/ICONS/transparent.png}

\begin{itemize}
\tightlist
\item
  Nadina Abdurahmanovic, Labor and Delivery Nurse
\item
  Macomb, Mich.
\item
  April 19
\end{itemize}

Just imagine going into the hospital on what should be the most exciting
day of your life and your labor and delivery nurse looks like this.

Moms are allowed to have one support person as long as they are healthy
and pass our screening process. We have had situations where the
significant other fails the screening, and we have to ask them to leave.
This leaves moms birthing without their main support system.

I can't imagine being in the hospital alone during a pandemic, taking on
the role of motherhood alone. I try to smile at them with my eyes. I
hold their hand in my gloved hands.

\includegraphics{https://static01.graylady3jvrrxbe.onion/packages/flash/multimedia/ICONS/transparent.png}

Hilary Swift for The New York Times

\begin{itemize}
\tightlist
\item
  Rohan Singh, E.R. Nurse
\item
  Atlanta
\item
  April 29
\end{itemize}

I was born in Suriname and came to New York when I was 18. I moved to
Atlanta in 2010. I was watching the news, listening to how it was
getting bad in New York. New York is still part of me, so I felt like I
needed to help. I spoke to traveling agencies and was given a contract
to come to New York City and work. I'm staying in a hotel. It's not an
easy choice because I had to leave my family. But I felt like I needed
to be here.

This is something I've never seen before in my life. There's nothing
more intense than people dying everywhere, and dying alone with no
family, no friends at their bedside. So we're that family.

The No. 1 thing to do is pray and be grateful, and treat everyone the
way you would want to be treated. You could be the one in that bed
without your family. You are no different from the patients that are
coming in there.

I worry about my own health every second. But I'm doing what I was
called to do and this is what I'm going to continue doing. My family
will understand because they know I'm doing this for the right reason.

\includegraphics{https://static01.graylady3jvrrxbe.onion/packages/flash/multimedia/ICONS/transparent.png}

\begin{itemize}
\tightlist
\item
  Annette Osher, Internist and Cardiologist
\item
  Manhattan, N.Y.
\item
  April 22
\end{itemize}

My private practice team and I transitioned to telemedicine when I
contracted the virus myself in mid-March. Once I recovered, I began
volunteering at the I.C.U. at Bellevue, which was the first time in
decades I had worked directly with critical care patients. Seemingly
overnight, I had to learn about ventilation, blood oxygen gases,
contamination protocols and to relearn how to resuscitate patients.

I had a young patient with a pudgy face. His chart noted he had autism.
Although he had been in the I.C.U. for about two weeks, no one knew much
about him. He had been transferred from Rikers Island prison, and
because of that, an armed police officer sat immediately outside his
glass room. He had Covid-19, bilateral pneumonias, renal failure. He was
receiving dialysis while sedated. He was dying.

Maybe it was the patient's autism that drew me to him. After all, he was
close in age to my own 27-year-old son with autism. Maybe it was his
police officer's stoic emotion. I felt sadness for my patient being so
ill, being absolutely alone, and not having the cognitive ability to
understand it all. I called his mother to learn more about him. Her son
panicked, she said, when apart from her. I sobbed after we hung up. I
arranged a compassionate visit for her to see her son, to comfort him
and to possibly say goodbye. ``Thank you, Doctor,'' she said.

\includegraphics{https://static01.graylady3jvrrxbe.onion/packages/flash/multimedia/ICONS/transparent.png}

\begin{itemize}
\tightlist
\item
  Hideo Yamanouchi, I.C.U. Doctor
\item
  Tokyo
\item
  May 4
\end{itemize}

Covid-19 is completely different, even than SARS. No one knows what's
going to happen in the future, what we have to do to make things better,
what might make things worse. Everyone's just fumbling around.

There are so many things to decide, but no time to decide them. You have
no choice but to figure it out while you're working.

The patients are young, from 5 to 30. There are a lot of men. But we've
also learned that pregnant women can get it. We have some people who
have been in the I.C.U. for more than two weeks. People who have been
there for 22, 23 days. Currently, we have 10 people in the I.C.U.

As for the older people, we have a few. Because of their prognosis, we
had to move them to palliative care. The other day, there was an
80-year-old man who left the I.C.U. and passed away.

I sent my family to Kagawa Prefecture, and am here alone. I haven't been
home in two months and they're telling me not to come. It's in the
countryside, and if I infected someone, it could spread in a flash.

\includegraphics{https://static01.graylady3jvrrxbe.onion/packages/flash/multimedia/ICONS/transparent.png}

\begin{itemize}
\tightlist
\item
  Manuel Penton III, Pediatric Infectious Disease Fellow
\item
  Brooklyn, N.Y.
\item
  April 17
\end{itemize}

I was redeployed to treat adult patients with Covid-19 to fill in gaps
of coverage.

To someone like me, who's seen a great deal more disease and death than
most people my age, the striking thing was not the critical nature of
any given case; it was the sheer number of them. In nearly every room I
went into, someone was fighting for their life, drowning above water.
There was an entire ward full of them, and beyond that, an intensive
care unit with more. I was prepared for the severity, but not the
volume.

In a tragedy of unique cruelty, a middle-aged nurse was lying in the
same room where she used to take care of patients. Her friends and
co-workers were now treating her. And she was deteriorating rapidly. She
was moved to another intensive care unit so her co-workers would not
experience the trauma of doing chest compressions on their friend if she
coded.

At the end of that day, the chaplains held a session for us. Some of the
staff openly broke down crying. I sniffled through a respirator during a
prayer. Others admitted the horrifying feeling of vulnerability in the
microscopic game of Russian roulette that we were all playing. It ended
when several of us peeled off to assess a patient in distress.

I realized we weren't losing our humanity. We were finding its depths.

\includegraphics{https://static01.graylady3jvrrxbe.onion/packages/flash/multimedia/ICONS/transparent.png}

\begin{itemize}
\tightlist
\item
  Brian Lima, Cardiac Surgeon
\item
  Roslyn, N.Y.
\item
  April 19
\end{itemize}

As a heart surgeon, the risk of death hangs in the balance with every
single procedure I do. That risk, however, is usually shouldered solely
by the patient, not by me. Covid-19, of course, has drastically altered
that.

If anything, I have greater appreciation for what doctoring is all
about, for the depth of commitment to patient care spelled out in the
Hippocratic oath that I took nearly 20 years ago.

\includegraphics{https://static01.graylady3jvrrxbe.onion/packages/flash/multimedia/ICONS/transparent.png}

Erik Branch for The New York Times

\begin{itemize}
\tightlist
\item
  Tawana Coates, OB-GYN
\item
  Louisville, Ky.
\item
  June 14
\end{itemize}

A lot of the demonstrations took place around the hospital. I went back
to work a couple of days after the death of George Floyd. You feel
heartbroken. I was an African-American going back to work as a minority,
not knowing if the majority is feeling the same way. The last place you
want to be is at work. The first conversations I had with everyone were:
``What can we do as white Americans? How can I help the Black community?
What can I do to help reduce my own implicit bias?'' It was
uncomfortable, but a lot of people were willing and open to have those
conversations.

Talking to a lot of my patients, women who look like Breonna Taylor and
who also look like me, and having conversations throughout clinic, you
could tell how the city was fired up wanting justice for Breonna Taylor.

Knowing that we serve a majority African-American population and knowing
Black social injustices are happening around the country, I felt it was
really important for us physicians to take a stand to show the community
that we support them. That's why I helped organize the White Coats for
Black Lives demonstration.

As I told everyone, this was just a beginning.

\includegraphics{https://static01.graylady3jvrrxbe.onion/packages/flash/multimedia/ICONS/transparent.png}

\begin{itemize}
\tightlist
\item
  Gabriela M. Maradiaga Panayotti, Pediatrician
\item
  Durham, N.C.
\item
  April 19
\end{itemize}

My whole family and I were sick with Covid-19 in March.

The night my husband was the sickest I went down a dark rabbit hole. In
comparison to others, he was not that sick. But the unknowns of Covid-19
pulled stronger on my imagination than the reassuring medical assessment
the doctor part of me was making.

What would I do if he didn't make it? How could I raise these kids on my
own? How could I continue the amazing work that he is doing as a father?

These questions forced me to confront some assumptions I realized were
running in the background of my everyday life: that we were young and
healthy, that we were destined for a long future together as a married
couple with two awesome kids, that we felt secure in our careers. All of
a sudden, nothing was for sure. My new coronavirus life was full of
vulnerability, uncertainty, fear.

It made me think of how others might have been living this way long
before Covid-19 showed up. For some people, this sense of insecurity is
how they live their everyday life.

\includegraphics{https://static01.graylady3jvrrxbe.onion/packages/flash/multimedia/ICONS/transparent.png}

\begin{itemize}
\tightlist
\item
  Judit Amigó, Endocrinologist
\item
  Barcelona, Spain
\item
  April 19
\end{itemize}

``Tomorrow you are not coming to this department, from now on you'll be
in a Covid ward.''

That's what I was told on March 14th. I had mixed feelings of courage
and fear. Attending to these patients was daunting. The moment I went
through the door of their room, my muscles tensed and I could only think
of how contagious the virus was. Lack of P.P.E. contributed to the
creeping feeling of risk. Patients kept being transferred to the I.C.U.
every day, many of them young, healthy people.

We are all struggling together.

\includegraphics{https://static01.graylady3jvrrxbe.onion/packages/flash/multimedia/ICONS/transparent.png}

\begin{itemize}
\tightlist
\item
  Kimi Chan, Rheumatologist
\item
  Manhattan, N.Y.
\item
  April 20
\end{itemize}

I am an outpatient rheumatologist, but I volunteered to care for
hospitalized Covid-19 patients.

I am caring for three female inpatients who lost their husbands to
Covid.

Two of them weren't able to say goodbye. I cannot even begin to imagine
what it must be like to survive this period of illness and isolation and
fear only to be met on the other side by grief.

The one woman who was able to say goodbye is now herself unresponsive,
and we are asking her children to make end-of-life decisions. How unfair
is it that the sons are already mourning one parent and having to decide
to pull life support from a distance?

\includegraphics{https://static01.graylady3jvrrxbe.onion/packages/flash/multimedia/ICONS/transparent.png}

Sasha Arutyunova for The New York Times

\begin{itemize}
\tightlist
\item
  Manish Garg, E.R. Doctor
\item
  Manhattan, N.Y.
\item
  April 13
\end{itemize}

The most difficult night involved managing multiple patients suspected
of having Covid. This was the night of April 3 into April 4.

The first patient was intubated and began to lose her blood pressure. We
tried to find her family but in the end it was us who had to comfort her
so she wasn't alone.

The second patient was in her 30s. We watched her slowly devolve through
the shift. Escalating oxygen did not help and all of our techniques
couldn't prevent her from intubation.

The third patient came in with respiratory distress and in multi-organ
failure. While we were on the phone with family, her heart stopped.

Although we had prepared her family for the worst, they were talking to
us about how healthy and strong she was. It's so hard to put into words
how sick a loved one is when the family had seen them OK just a few days
earlier. They had an all-too-familiar grief reaction with crying, denial
and guilt.

The fourth patient came in with respiratory failure and I had a ``goals
of care'' discussion with family out in the street. She was intubated
and died.

The way I view mental health has changed. I am in charge of our 48
training physicians. After that night, I tried to take a timeout and
share with my trainees how valuable they are to the community; how they
were the loving support for patients when their families couldn't be
there; how it's OK to have strong reactions to what they were
experiencing; how our teachers and leaders struggle with it, too; how we
can be there for each other.

\includegraphics{https://static01.graylady3jvrrxbe.onion/packages/flash/multimedia/ICONS/transparent.png}

Gianni Cipriano for The New York Times

\begin{itemize}
\tightlist
\item
  Dania Sannino, Anesthesiologist and I.C.U Doctor
\item
  Naples, Italy
\item
  June 14
\end{itemize}

I had been living one of the hardest nights of my career as an
anesthesiologist. Gathering all my energies to face a large challenge,
shrouded in sweat, out of breath and with foggy glasses.

I lost my patient. I cried when I finally left the Covid area after
several hours inside. I felt overwhelmed.

I didn't want to be called ``hero.'' I just wanted support and a short
vacation at the end of this. I wanted colorful flowers and to feel
beauty around me.

Now, the situation seems to be better for most people. But my perception
is different because, as doctors, we still feel that it is not over at
all.

\includegraphics{https://static01.graylady3jvrrxbe.onion/packages/flash/multimedia/ICONS/transparent.png}

\begin{itemize}
\tightlist
\item
  Mitchell Elliott, Family Medicine Physician
\item
  Louisville, Ky.
\item
  June 9
\end{itemize}

Our clinic has been slowly picking up. Before, patients had been scared
to step foot inside for any reason. They'd wait at home for days with
chest pain or shortness of breath. But as Americans are getting braver
and venturing out to stores and restaurants again, they're also feeling
more comfortable visiting the doctor.

We're seeing most of the normal summertime medical problems: poison ivy,
bicycle injuries, tick bites, strep throat. However, the symptom list of
Covid-19 is ever expanding. Any complaint that could at all be
Covid-related earns the patient a ``fast pass'' to an isolated room. The
encounter is almost entirely over the phone, except for the lucky doctor
who dons full P.P.E. to check vital signs and performs a physical exam.

My friends and family have been apprehensive to spend time with me, even
masked and six feet apart in the yard.

I am still frequently asked, ``Have you had a Covid patient recently?''

I worry that the answer will be, ``Yes,'' for the foreseeable future.

\includegraphics{https://static01.graylady3jvrrxbe.onion/packages/flash/multimedia/ICONS/transparent.png}

\begin{itemize}
\tightlist
\item
  Steven Miller, Internal Medicine Resident
\item
  Brooklyn, N.Y.
\item
  April 20
\end{itemize}

I'm so tired.

Every day we do our best, but this virus takes our patients one after
another. We are doing everything we can think of and inevitably they
succumb. Then we step outside into the fading light and hear applause.
The cheers ring out and the sirens blare. It is a celebration of the
survivors. May we all continue to celebrate this survival, continue to
cheer for every day we are alive.

\includegraphics{https://static01.graylady3jvrrxbe.onion/packages/flash/multimedia/ICONS/transparent.png}

\begin{itemize}
\tightlist
\item
  Jeffrey Oppenheim, Neurosurgeon
\item
  Piermont, N.Y.
\item
  April 21
\end{itemize}

Yesterday, I treated an intoxicated man who became paralyzed in a
motorcycle accident. The accident occurred because he was despondent
that he could no longer afford to take care of his family. He might not
have Covid, but he is a victim of the pandemic, too.

As a neurosurgeon, I've always found it important to contain my
emotions. These last two months I've cried more than I have in decades.
Allowing myself that vulnerability has been cathartic.

\includegraphics{https://static01.graylady3jvrrxbe.onion/packages/flash/multimedia/ICONS/transparent.png}

\begin{itemize}
\tightlist
\item
  Margit Anderegg, Labor and Delivery Nurse
\item
  Manhattan, N.Y.
\item
  April 8
\end{itemize}

I used to be a managing partner of a Manhattan restaurant, but I wanted
to do something more meaningful, and I was always interested in women's
health. I received a B.S. in nursing when I was 49.

I am a labor and delivery nurse, and of course pregnant women get Covid,
too.

We test every patient and her partner, but the results aren't usually
back while they're in labor. I had a laboring P.U.I. patient ---
``patient under investigation,'' as we call them while their results are
pending --- having her third baby. Unlike most patients, she did not
want an epidural. She pressed her call bell and told me the baby was
coming. I called the doctor, who proceeded to don his P.P.E. I was
wearing only my N95 mask and the surgical mask we use to protect our N95
during our shift, since we must reuse it.

The patient pressed her call bell again, and I knew the baby was coming.
I went into the room without my body P.P.E. I didn't want her to deliver
her baby alone. So during this time of such sadness and loss, I had the
awesome privilege of helping to escort a new life into this crazy, mad
world.

I found out the next day the patient was positive for Covid-19. As a
nurse, I knew it was just a matter of time before I was exposed.

\includegraphics{https://static01.graylady3jvrrxbe.onion/packages/flash/multimedia/ICONS/transparent.png}

\begin{itemize}
\tightlist
\item
  Naomie Jean, Podiatry Resident Physician
\item
  Mount Vernon, N.Y.
\item
  June 15
\end{itemize}

I crossed paths with protesters while walking to my car after work.
There was an intense energy in that crowd. I've never seen so many
different people of all races and backgrounds standing together as one
to fight racial injustice and anti-Blackness, of all things.

It was awesome to see.

I haven't been to any others yet. I've been really busy and working
night shifts this week. But I do plan on going to a protest soon since
they are calmer and more of a silent march.

Covid-19 exposed the systemic disparities in the health care system,
especially as it relates to the Black community. Blacks in this country
are fighting battles on many fronts: mentally, physically, emotionally.

I have seen patients being taken advantage of, blamed, labeled
non-compliant or flat out not believed when they say that they're in
pain. Sometimes their discomfort is the fault of the medical provider.

The pandemic just exposed it all for the world to see.

\includegraphics{https://static01.graylady3jvrrxbe.onion/packages/flash/multimedia/ICONS/transparent.png}

\begin{itemize}
\tightlist
\item
  Keleke Traore, Nurse
\item
  Koutiala, Mali
\item
  April 30
\end{itemize}

I usually work with Doctors Without Borders in a pediatric ward in
Koutiala, one of the largest pediatric wards in West Africa. When Mali
registered its first two coronavirus cases, I moved to work in another
Doctors Without Borders program, assisting the Malian health authorities
in Bamako, the capital. I'm working in a Covid-19 unit where over 100
patients have been treated so far. We now have a capacity of 90 beds and
14 I.C.U. beds.

The lack of public awareness about the pandemic in my country is a big
concern for me. We are talking about a virus that can spread fast, has
crippled some of the most advanced health systems and for which there is
no available vaccine. I also worry about how deadly this virus can be
for the elderly and for people with chronic diseases like cancer.

Having worked in the pediatric hospital, I also think about what would
happen if we are not able to maintain access to health care for severely
malnourished kids.

\includegraphics{https://static01.graylady3jvrrxbe.onion/packages/flash/multimedia/ICONS/transparent.png}

Ruth Fremson/The New York Times

\begin{itemize}
\tightlist
\item
  Lynn Brown, E.R. Nurse
\item
  Snohomish, Wash.
\item
  April 9
\end{itemize}

As the first hospital in the country with cases of community-spread
Covid-19 and fatalities, we were hit hard and ramped up fast. We
converted half of the department to an isolation unit for positive and
potential or suspected cases of Covid-19.

``Intense'' means different things at different times. Every day there
is something that shakes you. After 27 years in nursing, it was very
hard to be surprised. Not anymore.

One evening, a large family came in to the waiting room. The family
patriarch was upstairs dying. I had my tech escort two family members to
the unit, the limit of what we could allow for end-of-life care due to
safety concerns.

I had to keep the rest of the family away from the bedside, grieving far
from their loved one and left only to imagine what was happening.

One of the younger family members came to ask me a question, and started
to break down. I couldn't keep a six-foot distance. I took her in my
arms and did what I've always done as a nurse --- I met her need in the
only way I could. I gave her a safe place to weep, catch her breath,
feel seen and understood.

Sometimes ``intense'' is the critical, clinical care we provide, and
sometimes it is the intensely personal support when a patient or loved
one reaches their breaking point.

\includegraphics{https://static01.graylady3jvrrxbe.onion/packages/flash/multimedia/ICONS/transparent.png}

\begin{itemize}
\tightlist
\item
  Gretel Honis, E.R. Doctor
\item
  Portland, Ore.
\item
  April 21
\end{itemize}

What I'm most worried about is the mental health crisis that is
beginning. The amount of anxiety and depression I'm seeing in my
patients is heartbreaking; they've lost their jobs and their health
insurance, with no access to their support network. Grandparents are not
able to see their children and grandchildren. People are not going to
church or social clubs. I am fearful that this wave of mental health
challenges will be on par with the horrors of the virus itself.

I've gone through all the stages of grief. I'm just numb now. I feel let
down by my government and those that were supposed to protect me with
masks, gowns, P.P.E.

I'm just tired. Sometimes I just want out.

\includegraphics{https://static01.graylady3jvrrxbe.onion/packages/flash/multimedia/ICONS/transparent.png}

\begin{itemize}
\tightlist
\item
  Ilana Horowitz, Social Worker
\item
  Queens, N.Y.
\item
  April 21
\end{itemize}

Prior to Covid-19, I was supervising and working on our psychiatric
inpatient units. Once our hospital started filling up with Covid
patients, the decision was made to close our psych units completely.

Much of my job now entails calling families of patients in our I.C.U.s
who are sick with Covid-19. We're calling to do basic assessments, since
we can't talk to the patients themselves as they are sedated and
intubated, and also to provide support to the families and make sure
they are in touch with the medical teams.

I try to quickly let them know the purpose of my call. I can only
imagine the millions of thoughts that come to their minds as they see
the hospital's phone number pop up.

I spoke recently to the teenage son of a male patient. He told me a
little about him: no prior medical problems, delivery guy, loving
father. I listened as my own sense of mortality became more tattered by
the second. The son dutifully answered all my questions and then asked
me if his father was going to be OK. After I clumsily attempted to
answer and let him know how helpful he'd been, he said, ``God bless
you.'' I got off the phone and cried.

\includegraphics{https://static01.graylady3jvrrxbe.onion/packages/flash/multimedia/ICONS/transparent.png}

\begin{itemize}
\tightlist
\item
  Christelle Nancy Diane Mike, Doctor
\item
  Yaounde, Cameroon
\item
  April 30
\end{itemize}

I have been attracted to the medical profession for as long as I can
remember. It is in my nature to care for others and ensure that they are
healthy, be it mentally or physically. What would have been the best
career, if not as a medical doctor, to achieve that?

I am currently working with the Doctors Without Borders Covid-19 team in
Buea, in the southwest region of Cameroon.

During the first weeks of the outbreak in my country, my sister
developed the symptoms of Covid-19. She called every night to complain
about a severe headache, which I thought was related to her wearing
glasses. She cried on the phone for days and added that she had lost her
sense of smell and taste. Traveling to her was not possible. Her fear of
going to the treatment center was another stress. But I helped in
managing her case from afar, and she recovered.

I am scared this pandemic will not be controlled soon. There is a high
risk of having to bear this for longer than we wish. But my will to help
has not changed a dime.

\includegraphics{https://static01.graylady3jvrrxbe.onion/packages/flash/multimedia/ICONS/transparent.png}

\begin{itemize}
\tightlist
\item
  Tara Ardolino, Oncology Nurse
\item
  Cranford, N.J.
\item
  April 22
\end{itemize}

I've watched patients in their 30s with no other medical history die
from this virus. We can't get into the rooms fast enough when a patient
is in distress. Our colleagues are getting sick and we are critically
short on staff and supplies. On average I'm caring for 10 Covid patients
at the same time.

My patients can't be with their families right now, so I am doing my
best to provide comfort to them because they are scared and sick. I'd
like to think this has made me stronger, but sometimes I just want to
break down.

This is hell.

\includegraphics{https://static01.graylady3jvrrxbe.onion/packages/flash/multimedia/ICONS/transparent.png}

\begin{itemize}
\tightlist
\item
  Eric Tyra, E.R. Nurse Traveler
\item
  Oakland, Calif.
\item
  April 22
\end{itemize}

We travel from state to state, hospital to hospital, to help out E.R.s
that are in need of staff. It means that we have to learn to navigate a
new hospital and work with new co-workers every so many weeks.

We are understaffed and undersupplied for the Covid-19 crisis but we
will still fight for our patients' lives and well-being. Even at our own
risk.

We spend hours in a room lined with plastic sheets, a makeshift
negative-pressure unit in the E.R., where we can fit up to nine beds at
a time. You wouldn't believe how tight the quarters can be sometimes. I
spend all shift in some form of protective gear to the point that my
skin will sometimes be rubbed raw by it.

\includegraphics{https://static01.graylady3jvrrxbe.onion/packages/flash/multimedia/ICONS/transparent.png}

\begin{itemize}
\tightlist
\item
  Yolanda Mozdzen, Medical Assistant and Phlebotomy Technician
\item
  Staten Island, N.Y.
\item
  April 25
\end{itemize}

I became a medical assistant after undergoing cervical spine surgeries
in 2014 and 2017. It made me want to help others the way I was helped
when I needed it the most. I knew when this pandemic hit I could not
stay home. I have the capability to help and needed to do something
since so many medical professionals needed more assistance.

I was asked to help in a Covid-19 testing tent, helping the patients
with their admission and assisting the providers with the Covid-19
specimens. I also gave the patients proper instructions for isolation,
hydration and helping others around them stay protected.

Sometimes all the patients wanted was the security to be heard and to
know they will get the proper care they need. They were so happy we were
there for them.

\includegraphics{https://static01.graylady3jvrrxbe.onion/packages/flash/multimedia/ICONS/transparent.png}

\begin{itemize}
\tightlist
\item
  Shawn Nishi, Pulmonary and Critical Care Doctor
\item
  League City, Texas
\item
  April 17
\end{itemize}

I was charged with coordinating the Covid-19 surge plan at our main
hospital and three other campuses. After weeks of planning, my family
noted I was exhausted before the ``fight'' had actually begun.

I had to make an appointment to Skype with my 11-year-old daughter to
tutor her in math. I email or text her many days because I would not
otherwise be able to see her. One day at 5 a.m. she ran out of the house
as I was pulling out just to give me a hug and a snack so I'd have
something to eat that day. Now she wakes up to make my lunch every day
and always leaves a sweet surprise for me that makes me smile in the
midst of chaos and the unknown.

My family holds me up so my patients can have my best even when they
cannot.

\includegraphics{https://static01.graylady3jvrrxbe.onion/packages/flash/multimedia/ICONS/transparent.png}

\begin{itemize}
\tightlist
\item
   Sonia Compton, Critical Care Doctor
\item
  Louisville, Ky.
\item
  June 11
\end{itemize}

Once we began to flatten the curve and get into a sort of routine of
diagnosis and treatment, the social and racial disparities became
glaringly obvious. Black people make up about 20 percent of our local
population but are nearly a third of all Covid-19 cases and deaths.

The death of Breonna Taylor, who worked in the same hospital system I
do, was another devastating blow. Our entire community, and especially
its Black residents, is enduring this continued trauma and there is no
end in sight. The pain is palpable.

Each Covid patient I have taken care of has a unique and heartbreaking
story. They are allowed no visitors, and we are wearing so many layers
of P.P.E. that even our emotions are masked.

The end-of-life moments have become even more gut wrenching with family
members only being able to share their loved ones' last moments via an
iPad or a smartphone held by one of us.

I am haunted by one of the youngest patients to die in our area, a
35-year-old Black man who died alone despite everything we knew to do. I
heard his mother's voice over the phone in agony because she was not
able to come see him due to travel restrictions.

I thought of her when I watched George Floyd call out to his own mother
in his final moments.

\includegraphics{https://static01.graylady3jvrrxbe.onion/packages/flash/multimedia/ICONS/transparent.png}

\begin{itemize}
\tightlist
\item
  Alex Arreguin, Respiratory Therapist
\item
  San Diego, Calif.
\item
  April 22
\end{itemize}

The hardest thing for me has been seeing people fight for their lives
and dying alone while seeing other people protest that their lives have
been taken away from them.

Now I don't take the little things in life for granted.

\includegraphics{https://static01.graylady3jvrrxbe.onion/packages/flash/multimedia/ICONS/transparent.png}

\begin{itemize}
\tightlist
\item
  Luis Eduardo Cuntó Icaza, Internist
\item
  Guayaquil, Ecuador
\item
  Died on March 28
\end{itemize}

As head of internal medicine at the hospital where he worked, Luis
Eduardo Cuntó Icaza assessed the growing stream of patients arriving
with flulike symptoms. The pandemic was arriving in Ecuador, but Dr.
Cuntó remained calm, his family remembered. He had faced local outbreaks
of influenza and hepatitis C, so he did what he always did: read the
latest research.

``Finding solutions for his patients' maladies made him happy ---
especially when they were unique or rare diseases,'' said his wife,
Jacqueline Moncayo de Cuntó.

The son of a doctor, Dr. Cuntó carefully followed safety protocols, she
noted.

``We knew he would be on the front lines, as he always was, but we never
thought he would be affected by the disease,'' Mrs. Cuntó said.

When he fell sick, he isolated himself within his house. His condition
worsened rapidly and he was rushed to the hospital, where he could not
summon enough breath to speak. ``He held my hand tightly and looked at
me as if he was saying goodbye,'' Mrs. Cuntó recalled.

After a week in the critical care unit, he died. He was 60. As his wife
and three children grieve, they have kept close a motto of sorts that
Dr. Cuntó used to repeat at family dinners or when someone needed
advice: ``Let nothing and nobody prevent you from achieving your goals.
Always rely on family. Enjoy your achievements and learn from your
mistakes, and never be afraid to ask for advice.''

--- JOSÉ MARÍA LEÓN CABRERA

\includegraphics{https://static01.graylady3jvrrxbe.onion/packages/flash/multimedia/ICONS/transparent.png}

\begin{itemize}
\tightlist
\item
  Maman Brah Ibrahim, Infection Prevention and Control Manager
\item
  Maradi, Niger
\item
  April 30
\end{itemize}

I am working at the Covid-19 treatment center in Bobo-Dioulasso, Burkina
Faso, with Doctors Without Borders. I was already involved in infection
prevention and control, but my responsibility has switched to
exclusively focus on Covid-19. I supervise hand washing, wearing a
protective jumpsuit, the decontamination of premises, waste management
and more. It is not easy because we are dealing with a virus that we do
not fully understand.

Some people think Covid-19 is just a hoax. There is also the
stigmatization of patients, which makes some refuse to come forward ---
they prefer to self-medicate.

Moreover, the majority of people in Burkina Faso live from day to day
and cannot afford to stay home in lockdown, hence the risk of a great
spread of the disease.

\includegraphics{https://static01.graylady3jvrrxbe.onion/packages/flash/multimedia/ICONS/transparent.png}

\begin{itemize}
\tightlist
\item
  Maya Bunik, Pediatrician
\item
  Denver
\item
  June 9
\end{itemize}

Every Friday night I have a special dinner with my three children to
celebrate another week we are Covid-19 free. Most of the time we have an
indoor picnic like we used to when they were small.

I believe my family knows why I do this work, and yet I can see the
worry in their eyes when I leave. As I enter my clinic to get screened
with questions and a temperature check, I am grateful for another day
that I am healthy. I make the sign of the cross. I don my mask, face
shield and scrubs.

As soon as our state called for the overdue shelter-in-place, many
families experienced sudden job loss. I saw a mom with her baby for
concerns about slow weight gain and immunizations. She and the baby's
father had recently lost their restaurant jobs. We were able to give her
diapers, baby wipes and a bag of groceries. At the conclusion of the
visit, she looked at me with a combination of disbelief and gratitude
that I have never experienced in my 28 years as a pediatrician. I left
the room crying.

It is still challenging to push away the fear when you learn that
another patient you took care of tested positive.

With the recent tragedy with George Floyd culminating in protests, I
pray we will not see a surge. Continuing to support our families through
all the health inequities continues to be our main focus.

I am exhausted after three months of this pandemic for many reasons.

\includegraphics{https://static01.graylady3jvrrxbe.onion/packages/flash/multimedia/ICONS/transparent.png}

Chona Kasinger for The New York Times

\begin{itemize}
\tightlist
\item
  Francis X. Riedo, Infectious Disease Doctor
\item
  Kirkland, Wash.
\item
  May 1
\end{itemize}

The improbability was striking --- that two patients in one critical
care unit in a small, community hospital in a suburb of Seattle would
both test positive for Covid-19. It was Feb. 28.

With the first result, I suspected a testing error. With the second
result, we understood that our 318-bed hospital had the first known
cases of community spread in the U.S.

In that moment, our lives changed enormously.

One day, it dawns on you that your entire academic and professional life
has prepared you for this moment.

Am I concerned about getting Covid-19? Yes. I turn that concern into
practical measures: I wash my hands carefully, practice social
distancing, check and recheck my P.P.E. Now, more than ever, my
colleagues and I need to be alert, focused and present. Together, with
others like us all over the world, we're doing what we trained to do:
control the spread, care for people, save lives and take care of each
other.

\includegraphics{https://static01.graylady3jvrrxbe.onion/packages/flash/multimedia/ICONS/transparent.png}

Saul Martinez for The New York Times

\begin{itemize}
\tightlist
\item
  Vanessa Gomez, I.C.U. Nurse
\item
  Miami
\item
  April 29
\end{itemize}

The fear of not knowing who had this virus plagued our surgical
intensive care unit. Staff and nurses were suddenly becoming patients.

But we continued to rise to this occasion.

The one thing I have learned is that our perseverance will prevail.

\includegraphics{https://static01.graylady3jvrrxbe.onion/packages/flash/multimedia/ICONS/transparent.png}

\begin{itemize}
\tightlist
\item
  Jennifer Pierre Paul, E.R. Nurse
\item
  Boston
\item
  April 8
\end{itemize}

I've been a nurse for a decade and I wanted a change. I decided to leave
my job and have the adventure of being a travel nurse. I actually got a
contract in California, got an apartment. And then within two weeks, the
whole world shut down.

I'm in the Bronx now. When it was all brand-new, I didn't think it was
as bad as it really is. I feel like the media sometimes makes things a
little bit more than what they really are. But honestly, it's exactly
what they were saying, times 100. People are dying, all the time, like
flies. It's real and so scary.

I had a 44-year-old gentleman drive himself with his wife in the car to
the hospital because he wasn't feeling well. He had no pre-existing
conditions, was triaged in a tent, walked into the emergency room and
collapsed and died on a stretcher. He threw a clot. That was the first
time I saw the whole blood issue that they're seeing now.

Everyone's learning. We're in hospitals that built new units overnight.
Outpatient nurses are now E.R. nurses. It's mayhem. Everyone in the
department is from somewhere else.

We don't even know what our co-workers look like. The whole mask thing
is so impersonal. I and another nurse treated a dying patient the other
day. We worked together for six straight hours, debriefed and all that.
And I had no idea what she looked like. Then I saw her in the locker
room, said hello and walked right past her. She said, ``Hey, I think we
work together.''

\includegraphics{https://static01.graylady3jvrrxbe.onion/packages/flash/multimedia/ICONS/transparent.png}

\begin{itemize}
\tightlist
\item
  Kie Yamamoto, Orthopedic Surgery Resident
\item
  Tokyo
\item
  May 4
\end{itemize}

I haven't seen my parents since getting involved in this. We're just
talking on the phone. My mom tells me, ``Be careful.''

I've been helping move patients, carry bags, clean wards and move
equipment to the I.C.U. At first I was disappointed that I couldn't do
surgery, but as part of a team fighting Covid-19, I feel like even
cleaning is important. The first thought I had when I wanted to become a
doctor was, ``I want to be useful to people.'' I feel like I'm getting
back to that starting point.

This is a pandemic that a doctor might just see once in a lifetime. I
want to do my best.

I went with a patient who had light symptoms when she was discharged and
transferred to a hotel. Her face was so uneasy. It was just five or 10
minutes riding in the car with her, but I was really concerned about how
she was doing.

When I asked, ``Are you in pain?'' she just responded, ``I'm OK.'' But
she seemed so anxious.

I was completely covered in protective gear and carrying her bag. I
think she was in her 20s, about the same age as me. I hope she's getting
better.

\includegraphics{https://static01.graylady3jvrrxbe.onion/packages/flash/multimedia/ICONS/transparent.png}

\begin{itemize}
\tightlist
\item
  Denise Perry, Medical-Surgical Nurse
\item
  Charlton, Mass.
\item
  April 9
\end{itemize}

Working in a Covid-19 unit is very scary and stressful for me, but it is
my job. During this time of uncertainty, we nurses have the power to
make a difference.

\includegraphics{https://static01.graylady3jvrrxbe.onion/packages/flash/multimedia/ICONS/transparent.png}

\begin{itemize}
\tightlist
\item
  Maya Alexandri, Volunteer E.M.T.
\item
  Queens, N.Y.
\item
  May 1
\end{itemize}

During the pandemic, I've been on the ambulance five days a week near
Baltimore. The work is stressful, but what impresses me is how grateful
I feel to be able to serve the community in this unprecedented time of
need.

In my younger patients, the fear is palpable. One 19-year-old security
guard had shortness of breath for two days and then, when driving home
from work, abruptly couldn't breathe. He pulled over to the side of the
highway and called 911.

``Am I going to die?'' he whispered.

Every molecule in my body wanted to promise him that he'll be all right,
but I don't know that. ``We're a team,'' I replied, ``working together
to ensure that you have the best health care possible.''

Daughters old enough to be retirees have sobbed as I've assisted their
mothers onto the stretcher. ``I don't want her to go to the hospital
because she'll get the virus,'' is a common concern.

``I love you, Mom,'' I hear repeatedly, murmured with last embraces
before I take their beloved family member from their home to an
inaccessible emergency department.

One 93-year-old patient kept asking me, en route to the hospital, ``When
will I see my daughter again?''

I rubbed her shoulder and offered, ``When you're out of the hospital.''

Interviews have been edited and condensed. Locations refer to where
people live. Dates reflect either when the entry was submitted or when
subsequent reporting was done.

Editors

Clinton Cargill, Catrin Einhorn, Jonathan Ellis, Aidan Gardiner, Karen
Hanley, Veronica Majerol, Jeffrey Marcus, Sona Patel, Alison Peterson,
Emily S. Rueb

Photo Editor

Becky Lebowitz Hanger

Digital Design

Rebecca Lieberman

Development

Scott Blumenthal, Aaron Krolik, Jaymin Patel, Michael Strickland, James
Thomas

Additional Production and Development

Aliza Aufrichtig, Dave Braun, Karen Cetinkaya, Amanda Cordero, Alex
Garces, Laura Kaltman, Shannon Lin, Chris O'Brien, Matt Ruby, Umi Syam,
Jessica White, Josh Williams

Photography

Jenn Ackerman, Glenn Arcos, Bryan Anselm, Samuel Aranda, Sasha
Arutyunova, September Dawn Bottoms, Erik Branch, Fabio Bucciarelli,
Gianni Cipriano, Nadia Shira Cohen, Chloe Collyer, Veasey Conway, Meghan
Dhaliwal, Eve Edelheit, Alessandro Falco, Ruth Fremson, Brittany
Greeson, Tamir Kalifa, Chona Kasinger, Gulshan Khan, Kevin D. Liles,
Giulia Marchi, Saul Martinez, Nadège Mazars, Jake Michaels, Bethany
Mollenkof, Andrea Morales, Victor Moriyama, Angela Ponce, Erin Schaff,
Allison V. Smith, Christopher Smith, Michael Starghill Jr., Heather
Sten, Hilary Swift, Kayana Szymczak, Brandon Thibodeaux, Gianfranco
Tripodo

Reporting and Research

Samuel Aranda, John Bartlett, Susan Beachy, Erik Branch, **** Emma
Bubola, José María León Cabrera, Elda Cantú, Lynsey Chutel, Gianni
Cipriano, Nadia Shira Cohen, Ben Dooley, Sabrina Duque, Catrin Einhorn,
Jonathan Ellis, Fatima Faizi, Alessandro Falco, Jake Frankenfield,
Claire Fu, Aidan Gardiner, Gabriel Gianordoli, Charo Henríquez, Jason
Horowitz, Makiko Inoue, Hari Kumar, Isabella Kwai, Su-Hyun Lee, Dan
Levin, Raphael Minder, Sona Patel, Najim Rahim, Giulia McDonnell-Nieto
del Rio, Emily S. Rueb, Choe Sang-Hun, Kai Schultz, Kim Severson,
Mariana Simoes, Alexandra Stevenson, Lucy Tompkins, Anton Troianovski,
Paulina Villegas, Vivian Wang, Liu Yi, Rosa Chávez Yacila, Wang Yiwei

Additional reporting by José María, León Cabrera, Gianni Cipriano,
Catrin Einhorn, Jonathan Ellis, Fatima Faizi, Aidan Gardiner, Jason
Horowitz, Makiko Inoue, Su-Hyun Lee, Emily S. Rueb and Vivian Wang.

Write a comment

\begin{itemize}
\item
\item
\item
\item
\end{itemize}

Advertisement

\protect\hyperlink{after-bottom}{Continue reading the main story}

\hypertarget{site-index}{%
\subsection{Site Index}\label{site-index}}

\hypertarget{site-information-navigation}{%
\subsection{Site Information
Navigation}\label{site-information-navigation}}

\begin{itemize}
\tightlist
\item
  \href{https://help.nytimes3xbfgragh.onion/hc/en-us/articles/115014792127-Copyright-notice}{©~2020~The
  New York Times Company}
\end{itemize}

\begin{itemize}
\tightlist
\item
  \href{https://www.nytco.com/}{NYTCo}
\item
  \href{https://help.nytimes3xbfgragh.onion/hc/en-us/articles/115015385887-Contact-Us}{Contact
  Us}
\item
  \href{https://www.nytco.com/careers/}{Work with us}
\item
  \href{https://nytmediakit.com/}{Advertise}
\item
  \href{http://www.tbrandstudio.com/}{T Brand Studio}
\item
  \href{https://www.nytimes3xbfgragh.onion/privacy/cookie-policy\#how-do-i-manage-trackers}{Your
  Ad Choices}
\item
  \href{https://www.nytimes3xbfgragh.onion/privacy}{Privacy}
\item
  \href{https://help.nytimes3xbfgragh.onion/hc/en-us/articles/115014893428-Terms-of-service}{Terms
  of Service}
\item
  \href{https://help.nytimes3xbfgragh.onion/hc/en-us/articles/115014893968-Terms-of-sale}{Terms
  of Sale}
\item
  \href{https://spiderbites.nytimes3xbfgragh.onion}{Site Map}
\item
  \href{https://help.nytimes3xbfgragh.onion/hc/en-us}{Help}
\item
  \href{https://www.nytimes3xbfgragh.onion/subscription?campaignId=37WXW}{Subscriptions}
\end{itemize}
