Sections

SEARCH

\protect\hyperlink{site-content}{Skip to
content}\protect\hyperlink{site-index}{Skip to site index}

\href{https://myaccount.nytimes3xbfgragh.onion/auth/login?response_type=cookie\&client_id=vi}{}

\href{https://www.nytimes3xbfgragh.onion/section/todayspaper}{Today's
Paper}

\hypertarget{the-coronavirus-outbreak}{%
\subsubsection{\texorpdfstring{\href{https://www.nytimes3xbfgragh.onion/news-event/coronavirus?name=styln-coronavirus-markets\&region=TOP_BANNER\&variant=undefined\&block=storyline_menu_recirc\&action=click\&pgtype=LegacyCollection\&impression_id=a8e2cf90-e38a-11ea-9198-f319f806a34f}{The
Coronavirus
Outbreak}}{The Coronavirus Outbreak}}\label{the-coronavirus-outbreak}}

\begin{itemize}
\tightlist
\item
  live\href{https://www.nytimes3xbfgragh.onion/2020/08/20/world/coronavirus-covid.html?name=styln-coronavirus-markets\&region=TOP_BANNER\&variant=undefined\&block=storyline_menu_recirc\&action=click\&pgtype=LegacyCollection\&impression_id=a8e2cf91-e38a-11ea-9198-f319f806a34f}{Latest
  Updates}
\item
  \href{https://www.nytimes3xbfgragh.onion/interactive/2020/us/coronavirus-us-cases.html?name=styln-coronavirus-markets\&region=TOP_BANNER\&variant=undefined\&block=storyline_menu_recirc\&action=click\&pgtype=LegacyCollection\&impression_id=a8e2f6a0-e38a-11ea-9198-f319f806a34f}{Maps
  and Cases}
\item
  \href{https://www.nytimes3xbfgragh.onion/interactive/2020/science/coronavirus-vaccine-tracker.html?name=styln-coronavirus-markets\&region=TOP_BANNER\&variant=undefined\&block=storyline_menu_recirc\&action=click\&pgtype=LegacyCollection\&impression_id=a8e2f6a1-e38a-11ea-9198-f319f806a34f}{Vaccine
  Tracker}
\item
  \href{https://www.nytimes3xbfgragh.onion/2020/08/19/us/colleges-closing-covid.html?name=styln-coronavirus-markets\&region=TOP_BANNER\&variant=undefined\&block=storyline_menu_recirc\&action=click\&pgtype=LegacyCollection\&impression_id=a8e2f6a2-e38a-11ea-9198-f319f806a34f}{Colleges
  Closing}
\item
  \href{https://www.nytimes3xbfgragh.onion/live/2020/08/20/business/stock-market-today-coronavirus?name=styln-coronavirus-markets\&region=TOP_BANNER\&variant=undefined\&block=storyline_menu_recirc\&action=click\&pgtype=LegacyCollection\&impression_id=a8e2f6a3-e38a-11ea-9198-f319f806a34f}{Economy}
\end{itemize}

\hypertarget{ubers-delivery-business-overtakes-rides}{%
\section{Uber's Delivery Business Overtakes
Rides}\label{ubers-delivery-business-overtakes-rides}}

This briefing is no longer being updated. Follow live updates
\href{https://www.nytimes3xbfgragh.onion/2020/08/06/world/coronavirus-covid.html}{here}.

Last Updated

Aug. 6, 2020, 10:08 p.m. ET

Aug. 6, 2020, 10:08 p.m. ET

\hypertarget{heres-what-you-need-to-know}{%
\subsubsection{Here's what you need to
know:}\label{heres-what-you-need-to-know}}

\begin{itemize}
\item
  \protect\hyperlink{uber-reports-steep-revenue-decline-as-delivery-outpaces-ride-hailing}{}

  Uber reports steep revenue decline, as delivery outpaces ride hailing.
\item
  \protect\hyperlink{small-business-owners-are-wary-as-a-relief-programs-nears-its-end}{}

  Small-business owners are wary as a relief programs nears its end.
\item
  \protect\hyperlink{us-may-insist-chinese-companies-share-audits-or-delist-their-shares-from-american-exchanges}{}

  U.S. may insist Chinese companies share audits or delist their shares
  from American exchanges.
\item
  \protect\hyperlink{amcs-quarterly-revenues-dropped-98-7-percent-from-last-year}{}

  AMC's quarterly revenues dropped 98.7 percent from last year.
\item
  \protect\hyperlink{new-state-jobless-claims-decline-but-exceed-one-million-for-the-20th-week}{}

  New state jobless claims decline, but exceed one million for the 20th
  week.
\end{itemize}

\hypertarget{uber-reports-steep-revenue-decline-as-delivery-outpaces-ride-hailing}{%
\subsection{\texorpdfstring{\protect\hyperlink{uber-reports-steep-revenue-decline-as-delivery-outpaces-ride-hailing}{Uber
reports steep revenue decline, as delivery outpaces ride
hailing.}}{Uber reports steep revenue decline, as delivery outpaces ride hailing.}}\label{uber-reports-steep-revenue-decline-as-delivery-outpaces-ride-hailing}}

Copied to clipboard.

\includegraphics{https://static01.graylady3jvrrxbe.onion/images/2020/08/06/business/06markets-brf-uber/merlin_174443172_7c34e9f9-7a43-4d39-bdab-e0c3348aec87-articleLarge.jpg?quality=75\&auto=webp\&disable=upscale}

\textbf{Uber} said on Thursday that its revenue in the second quarter
dropped 29 percent to \$2.2 billion from a year ago and that its net
loss narrowed to \$1.8 billion, as the ride-hailing giant deals with the
fallout from the coronavirus pandemic.

The revenue decline was the steepest since Uber went public in May 2019,
though total revenue was better than what Wall Street analysts had
projected. Uber's losses improved from
\href{https://www.nytimes3xbfgragh.onion/2019/08/08/technology/uber-earnings.html}{\$5.2
billion} a year ago when it had heavy stock-based compensation costs
after its initial public offering.

The pandemic's hit to Uber's core ride-hailing business was
unmistakable. The company said the amount of money it receives from
rides before paying driver wages and other fees declined 73 percent from
a year earlier.

But delivery surged, with revenue from Uber Eats, its food delivery
business, exceeding the core ride-hailing business for the first time.
Revenue from Uber Eats totaled \$1.2 billion, while rides shrank to
\$790 million.

To bolster its delivery business, Uber recently agreed to acquire the
food delivery company Postmates for
\href{https://www.nytimes3xbfgragh.onion/2020/07/05/technology/uber-postmates-deal.html}{\$2.65
billion}.

--- \href{https://www.nytimes3xbfgragh.onion/by/kate-conger}{Kate
Conger}

\hypertarget{small-business-owners-are-wary-as-a-relief-programs-nears-its-end}{%
\subsection{\texorpdfstring{\protect\hyperlink{small-business-owners-are-wary-as-a-relief-programs-nears-its-end}{Small-business
owners are wary as a relief programs nears its
end.}}{Small-business owners are wary as a relief programs nears its end.}}\label{small-business-owners-are-wary-as-a-relief-programs-nears-its-end}}

Copied to clipboard.

\includegraphics{https://static01.graylady3jvrrxbe.onion/images/2020/08/06/business/06virus-pppdone3/merlin_175376493_fd939eae-3db8-4d71-944f-38dc171a2da5-articleLarge.jpg?quality=75\&auto=webp\&disable=upscale}

Since April, federal government's Paycheck Protection Program has
injected \$523 billion into the economy, allowing small-business owners
to stay afloat and keep employees on payrolls.

But with the program set to end Saturday and an economic rebound nowhere
in sight, the looming question is: What happens to the millions of
workers who have no jobs to return to, and the struggling businesses
that employed them?

The numbers are already dire. On Thursday, the government reported that
\href{https://www.nytimes3xbfgragh.onion/live/2020/08/06/business/stock-market-today-coronavirus/new-state-jobless-claims-decline-but-exceed-one-million-for-the-20th-week}{nearly
1.2 million Americans filed for state unemployment benefits} last week
--- the 20th straight week that new jobless claims have topped one
million. Economists estimate that 30 million Americans are unemployed.

Small businesses employ nearly half of America's nongovernment workers,
and the paycheck program preserved at least 1.4 million jobs through
early June, \href{http://economics.mit.edu/files/20094}{a recent
economic analysis} concluded. More than five million companies received
loans, averaging \$102,000 each.

Ken Bodenstein borrowed \$148,000 from the federal government to help
cover payroll expenses at the Westport day care center he runs with his
wife, Kristen. But by June 5, the money ran out, forcing the Bodensteins
to furlough or lay off all but nine employees.

``We were just about to hit break-even, and then everything collapsed,''
Mr. Bodenstein said.

So far, there is
\href{https://www.nytimes3xbfgragh.onion/2020/08/05/us/politics/congress-coronavirus-stimulus.html}{little
agreement in Washington} about how to continue to help millions of
floundering businesses.

``There's absolutely a need for more help in some industries,'' said
Carson Lappetito, president of Sunwest Bank, a regional lender based in
Irvine, Calif., that has made more than 2,000 P.P.P. loans. ``In the
hotel and restaurant and hospitality sectors, those areas have been
completely hammered.''

--- \href{https://www.nytimes3xbfgragh.onion/by/stacy-cowley}{Stacy
Cowley}

\hypertarget{advertisement}{%
\subsubsection{Advertisement}\label{advertisement}}

\protect\hyperlink{after-dfp-ad-mid1}{Continue reading the main story}

\hypertarget{dow-jones-was-news-corps-only-growing-division-this-past-fiscal-year}{%
\subsection{\texorpdfstring{\protect\hyperlink{dow-jones-was-news-corps-only-growing-division-this-past-fiscal-year}{Dow
Jones was News Corp's only growing division this past fiscal
year.}}{Dow Jones was News Corp's only growing division this past fiscal year.}}\label{dow-jones-was-news-corps-only-growing-division-this-past-fiscal-year}}

Copied to clipboard.

\includegraphics{https://static01.graylady3jvrrxbe.onion/images/2020/08/06/business/06markets-brf-newscorp/merlin_163892256_bb9a9a36-e3ca-41a6-9eb4-8e2b51069b19-articleLarge.jpg?quality=75\&auto=webp\&disable=upscale}

Rupert Murdoch's publishing empire took a staggering loss for the three
months ending in June, as \textbf{News Corp} reported a \$401 million
loss for the period, with much of the decline related to impairment
charges for some of its assets in Britain and Australia and
restructuring costs related to the coronavirus pandemic.

\begin{itemize}
\item
  But a surprise highlight was the company revealing for the first time
  financial details of its Dow Jones division, the group that publishes
  \textbf{The Wall Street Journal}. The unit was News Corp's only
  growing business on an annual basis. For the fiscal year ending in
  June, revenue for Dow Jones, which includes Barron's and the news and
  information site MarketWatch, was up 3 percent to \$1.6 billion, and
  profit rose 13 percent to \$236 million. Circulation sales increased 6
  percent to about \$1.2 billion, while ad revenue fell 9 percent to
  \$359 million.
\item
  In addition to The Journal, News Corp publishes several newspapers
  across the United States, Britain and Australia, including The Times
  of London and the tabloids The New York Post and The Sun. Those papers
  are housed under a separate group from The Journal.
\item
  One advantage to laying out a division's finances is it allows a
  \textbf{potential acquirer} to more accurately value that part of the
  business. The company has not said there are any plans to sell The
  Journal.
\item
  \textbf{Robert Thomson}, chief executive of News Corp, called the
  reporting of Dow Jones financials ``a particularly historic moment.''
  He said it highlighted the division's ``superior profit profile and
  prospects compared to those of our nearest competitor.''
\item
  \textbf{Subscriptions} to The Wall Street Journal rose 15 percent to
  just under three million, including 2.2 million digital-only
  subscriptions.
\item
  Quarterly results revealed advertising challenges at Dow Jones, as at
  other news publishers. Revenue fell 4 percent to \$381 million,
  largely because of a 28 percent drop in advertising sales to \$71
  million. Profit, however, was up 13 percent to \$60 million.
\item
  Total sales for News Corp fell 22 percent to \$1.9 billion, with
  advertising dropping across its properties. This is also the first
  quarter that News Corp is reporting its results without James Murdoch,
  a potential heir to the Murdoch family empire who
  \href{https://www.nytimes3xbfgragh.onion/2020/07/31/business/media/james-murdoch-resigns-news-corp.html}{stepped
  down} from the company's board last week.
\end{itemize}

--- \href{https://www.nytimes3xbfgragh.onion/by/edmund-lee}{Edmund Lee}

\hypertarget{us-may-insist-chinese-companies-share-audits-or-delist-their-shares-from-american-exchanges}{%
\subsection{\texorpdfstring{\protect\hyperlink{us-may-insist-chinese-companies-share-audits-or-delist-their-shares-from-american-exchanges}{U.S.
may insist Chinese companies share audits or delist their shares from
American
exchanges.}}{U.S. may insist Chinese companies share audits or delist their shares from American exchanges.}}\label{us-may-insist-chinese-companies-share-audits-or-delist-their-shares-from-american-exchanges}}

Copied to clipboard.

The Trump administration is considering forcing Chinese companies to
delist their shares from stock exchanges in the United States unless
they share their audits with American regulators, a move that would
further ratchet up tension between the world's two largest economies.

China does not allow its companies to provide their audit information to
the Public Company Accounting Oversight Board, which oversees auditors
in the United States. Officials at the Treasury Department and the
Securities and Exchange Commission acknowledged that the Chinese
government would likely have to amend its laws so that companies would
be in compliance with the rules, if they take effect.

The President's Working Group on Financial Markets recommended the move
in a report released on Thursday as a way to protect American investors
from what it described as the
\href{https://www.nytimes3xbfgragh.onion/2012/07/13/business/in-china-inspecting-the-inspectors.html}{risks
posed by Chinese companies}. The recommendations would require the
S.E.C. to undertake a rule-making process and would not take full effect
until 2022.

``The recommendations outlined in the report will increase investor
protection and level the playing field for all companies listed on U.S.
exchanges,'' Treasury Secretary Steven Mnuchin, who chairs the working
group, said in a statement.

If a Chinese company did not comply with the rule and provide access to
work papers of the principal audit firm, it would have until January
2022 to delist. A Chinese company could potentially satisfy the
requirements by undertaking a co-audit with a comparable firm that would
provide access to relevant information.

Treasury and S.E.C. officials acknowledged that the policy could lead to
a wave of delistings.

There is currently bipartisan legislation in the House and the Senate
that would impose restrictions similar to what the working group is
proposing.

--- \href{https://www.nytimes3xbfgragh.onion/by/alan-rappeport}{Alan
Rappeport}

\hypertarget{amcs-quarterly-revenues-dropped-987-percent-from-last-year}{%
\subsection{\texorpdfstring{\protect\hyperlink{amcs-quarterly-revenues-dropped-98-7-percent-from-last-year}{AMC's
quarterly revenues dropped 98.7 percent from last
year.}}{AMC's quarterly revenues dropped 98.7 percent from last year.}}\label{amcs-quarterly-revenues-dropped-987-percent-from-last-year}}

Copied to clipboard.

\includegraphics{https://static01.graylady3jvrrxbe.onion/images/2020/08/06/business/06markets-brf-amc/merlin_175136823_77c3d62b-0d34-41c8-ac69-05718b1a72f1-articleLarge.jpg?quality=75\&auto=webp\&disable=upscale}

With operations ceased for the entirety of the quarter and most of its
employees laid off or furloughed, \textbf{AMC Entertainment}, the
largest theater chain in the U.S., posted a quarterly loss for the
period ended June of \$561.2 million. Revenues totaled \$18.9 million, a
98.7 percent plunge from the same period last year for the Kansas-based
company.

The coronavirus has laid waste to the company's 1,000 theaters scattered
across the globe, calling into question whether it would be able to stay
financially viable. AMC did restructure \$2.6 billion of its debt, which
Adam Aron, the chief executive, said will allow the company to stay
afloat should theaters have to remain shuttered into 2021.

AMC is planning on reopening two-thirds of its U.S. theaters this month.
Internationally, AMC has already restarted business in 130 markets and
expects its remaining theaters to also open in August.

``We remain optimistic about AMC's long term future,'' Mr. Aron said in
a statement. ``Theatrical exhibition has always been resilient, and we
are confident that at AMC we are taking the right steps to emerge from
this crisis and to thrive once again as the leader in our industry.``

Last week, AMC
\href{https://www.nytimes3xbfgragh.onion/2020/07/28/business/media/universal-amc-movies-at-home.html}{shocked
Hollywood} by signing an agreement with Universal Studios that would
dramatically shorten the amount of time movies need to play exclusively
in its theaters before moving to premium video-on-demand. The deal,
which now only requires a film to play for 17 days, will allow AMC to
share in the revenue generated by the on-demand sales.

--- \href{https://www.nytimes3xbfgragh.onion/by/nicole-sperling}{Nicole
Sperling}

\hypertarget{advertisement-1}{%
\subsubsection{Advertisement}\label{advertisement-1}}

\protect\hyperlink{after-dfp-ad-mid2}{Continue reading the main story}

\hypertarget{wall-street-inches-closer-to-a-record-as-big-technology-shares-rise-again}{%
\subsection{\texorpdfstring{\protect\hyperlink{wall-street-inches-closer-to-a-record-as-big-technology-shares-rise-again}{Wall
Street inches closer to a record as big technology shares rise
again.}}{Wall Street inches closer to a record as big technology shares rise again.}}\label{wall-street-inches-closer-to-a-record-as-big-technology-shares-rise-again}}

Copied to clipboard.

Wall Street regained its footing on Thursday afternoon, pushing higher
after an unsteady start to the day, with the S\&P 500 inching ever
closer to its record and climbing for a fifth consecutive day.

The S\&P 500 rose more than half a percent, gains that put the index
about 1 percent below a high reached on February 19, before markets went
into a tailspin as investors panicked about the fast spreading
coronavirus.

Large technology stocks, which have played a large role in the rebound,
rallied again. On Thursday, it was \textbf{Facebook} that ranked among
the best performing stocks in the S\&P 500, with a gain of almost 6.5
percent. Shares of \textbf{Apple}, \textbf{Google} and
\textbf{Microsoft} were also higher, and the Nasdaq composite rose 1
percent.

The gains came despite more indications of the economic morass the
United States finds itself in, and as prospects for a deal on a new
financial
\href{https://www.nytimes3xbfgragh.onion/2020/08/05/us/politics/congress-coronavirus-stimulus.html}{rescue
package} appeared to dim.

It may have helped that President Trump on Thursday said he was willing
to act unilaterally to
\href{https://www.nytimes3xbfgragh.onion/2020/08/06/world/coronavirus-covid-19.html\#link-d79b4fa}{strengthen
the economy}, saying he was looking at executive orders to forestall
evictions, suspend payroll tax collection and provide extra unemployment
aid and student loan relief.

Mr. Trump's comments came after the Labor Department released data
showing that workers filed more than one million new state
\href{https://www.nytimes3xbfgragh.onion/live/2020/08/06/business/stock-market-today-coronavirus\#new-jobless-claims-are-set-to-exceed-one-million-for-the-20th-week}{jobless
claims} for the 20th straight week, as the coronavirus pandemic
continued to lead to layoffs and business closures. The tally for last
week, 1.2 million claims, was the lowest since March.

Shares in Europe fell, however, weighed down by warnings from Britain's
central bank of a slow recovery ahead.

Bank of England policymakers said that they expected the country's
economy to contract by 9 percent this year,
\href{https://www.nytimes3xbfgragh.onion/live/2020/08/06/business/stock-market-today-coronavirus/britains-economy-wont-fully-recover-till-end-of-2021-its-central-bankers-said}{a
less severe downturn} than they indicated a few months ago, but they
also predicted that the economy would not return to its pre-pandemic
levels until the end of 2021. Even in three years, they said, the
economy will still be smaller than it would have been had the growth
rate followed the path it was on at the end of 2019.

The Turkish lira hit a new low against the dollar Thursday, raising
fears of a currency crisis that would undermine the fragile economy. The
Turkish central bank said Thursday it would ``use all available
instruments to reduce the excessive volatility in the markets'' after
the lira fell as low as 7.3 to the dollar, down from 6 to the dollar at
the beginning of the year.

--- \href{https://www.nytimes3xbfgragh.onion/by/kevin-granville}{Kevin
Granville} and Mohammed Hadi

\hypertarget{new-state-jobless-claims-decline-but-exceed-one-million-for-the-20th-week}{%
\subsection{\texorpdfstring{\protect\hyperlink{new-state-jobless-claims-decline-but-exceed-one-million-for-the-20th-week}{New
state jobless claims decline, but exceed one million for the 20th
week.}}{New state jobless claims decline, but exceed one million for the 20th week.}}\label{new-state-jobless-claims-decline-but-exceed-one-million-for-the-20th-week}}

Copied to clipboard.

\includegraphics{https://static01.graylady3jvrrxbe.onion/images/2020/08/06/business/06markets-brf-preview2/merlin_175356201_fd4aca9e-aea3-4ca5-b997-87448183975f-articleLarge.jpg?quality=75\&auto=webp\&disable=upscale}

The \href{https://oui.doleta.gov/press/2020/080620.pdf}{government
reported} on Thursday that nearly 1.2 million workers filed new claims
for state unemployment benefits last week. It was the lowest weekly
total since March, but signaled the continuing damage that the pandemic
is inflicting on the labor market.

An additional 656,000 claims were filed by freelancers, part-time
workers and others who do not qualify for regular state jobless aid but
are eligible for benefits under a separate federal unemployment
insurance program, the Labor Department announced on Thursday. Unlike
the state figures, that number is not seasonally adjusted.

``Over all, the data was modestly better than we expected, a surprising
improvement,'' said Kathy Bostjancic, chief U.S. financial economist at
Oxford Economics. There were declines across nearly all the states, even
those where there is a resurgence of the virus.

But jobless claims ``remain at alarmingly high levels,'' she said, and
the stubbornly high number of people collecting unemployment ---
estimated by economists at 30 million --- suggests that ``temporary
layoffs are becoming permanent.''

Although the number of new claims is down from the stratospheric levels
reached in the early days of the pandemic, the million-plus tallies that
have continued for 20 weeks in a row are still extraordinarily high by
historical standards.

And now that emergency federal supplemental benefits have expired, the
newest entrants to join the ranks of unemployed will not be receiving
the extra \$600 a week that has helped jobless workers pay bills through
the spring and early summer.

--- \href{https://www.nytimes3xbfgragh.onion/by/patricia-cohen}{Patricia
Cohen}

\hypertarget{nevada-lawmakers-mandate-protective-measures-for-hospitality-workers}{%
\subsection{\texorpdfstring{\protect\hyperlink{nevada-lawmakers-mandate-protective-measures-for-hospitality-workers}{Nevada
lawmakers mandate protective measures for hospitality
workers.}}{Nevada lawmakers mandate protective measures for hospitality workers.}}\label{nevada-lawmakers-mandate-protective-measures-for-hospitality-workers}}

Copied to clipboard.

\includegraphics{https://static01.graylady3jvrrxbe.onion/images/2020/08/06/business/06-markets-brf-workerprotection2/merlin_175354557_415e7e65-ee07-4e72-bdea-12035aa4813f-articleLarge.jpg?quality=75\&auto=webp\&disable=upscale}

The Nevada Legislature has approved a bill mandating health protections
for hundreds of thousands of workers in the hospitality industry while
providing most of the state's businesses with a liability shield against
coronavirus-related lawsuits.

The state's powerful culinary union pushed for the worker protections as
part of what it called the Adolfo Fernandez Bill, named for a utility
porter on the Las Vegas Strip who died after contracting Covid-19.

Gov. Steve Sisolak, a Democrat, is expected to sign the measure, which
was approved Wednesday night. Both houses of the Legislature are
controlled by Democrats.

In the counties that include Las Vegas and Reno, the bill requires daily
temperature screenings of hospitality workers, coronavirus tests for
those returning to work and regular cleaning of high-touch surfaces. The
legislation bars employers from forcing workers with symptoms from
reporting to the job while awaiting test results and offers paid time
off to those testing positive.

At the same time, the bill shields businesses in most of the state's
industries from potential lawsuits by customers who become ill with
Covid-19.

State Senator Yvanna Cancela, a co-majority whip, said the measure
``finds a balance'' in offering health protection to workers while
assuring employers that ``there can be certainty around which you can
operate.''

But progressive groups criticized the liability provisions, saying they
offered a shield to employers not bound by the safety mandates that
apply to casino and hotel workers.

Hugh Baran, a staff lawyer with the National Employment Law Project, a
worker advocacy group, said the liability provisions would grant ``legal
immunity to employers who fail to protect their workers.''

--- \href{https://www.nytimes3xbfgragh.onion/by/noam-scheiber}{Noam
Scheiber}

\hypertarget{advertisement-2}{%
\subsubsection{Advertisement}\label{advertisement-2}}

\protect\hyperlink{after-dfp-ad-mid3}{Continue reading the main story}

\hypertarget{job-postings-are-picking-up-but-more-layoffs-are-expected}{%
\subsection{\texorpdfstring{\protect\hyperlink{job-postings-are-picking-up-but-more-layoffs-are-expected}{Job
postings are picking up, but more layoffs are
expected.}}{Job postings are picking up, but more layoffs are expected.}}\label{job-postings-are-picking-up-but-more-layoffs-are-expected}}

Copied to clipboard.

\includegraphics{https://static01.graylady3jvrrxbe.onion/images/2020/08/06/business/06markets-brf-hiring/merlin_175342185_3dd9a16d-0cd6-4d26-a6de-b6c71d4a7b2b-articleLarge.jpg?quality=75\&auto=webp\&disable=upscale}

While the elevated levels of
\href{https://www.nytimes3xbfgragh.onion/live/2020/08/06/business/stock-market-today-coronavirus\#new-state-jobless-claims-decline-but-exceed-one-million-for-the-20th-week}{jobless
claims} show that businesses are still struggling to keep employees on
the payroll, there has been some pickup in hiring. After drooping, job
postings at the online jobs site ZipRecruiter rose by 7.4 percent in
July and are still climbing, said Julia Pollak, the company's labor
economist.

But the latest economic data is mixed, she cautioned. Surveys from the
\href{https://www.ismworld.org/}{Institute for Supply Management}, for
instance, showed that business activity in service industries expanded
last month, but that the
\href{https://www.nytimes3xbfgragh.onion/live/2020/08/06/business/stock-market-today-coronavirus\#as-unemployment-benefits-began-to-run-out-a-freelance-job-came-just-in-time}{employment
index} declined, an indication that many companies are still not
bringing back workers.

There were steep increases in joblessness related to the performing arts
and other live events in July, Ms. Pollak said.

And announcements of impending layoffs continue to pile in. Ms. Pollak
has been tracking plant closings and layoffs that the government
requires to be announced in advance. ``They are showing that new layoffs
are still taking place at an alarming rate,'' she said. ``Plenty of
layoffs are scheduled for August, September and October, as well.''

``Many companies are realizing now that the effects will be much longer
than expected,'' she said.

--- \href{https://www.nytimes3xbfgragh.onion/by/patricia-cohen}{Patricia
Cohen}

\hypertarget{heres-what-else-is-happening-facebook-extends-remote-work}{%
\subsection{\texorpdfstring{\protect\hyperlink{heres-what-else-is-happening-facebook-extends-remote-work}{Here's
what else is happening: Facebook extends remote
work.}}{Here's what else is happening: Facebook extends remote work.}}\label{heres-what-else-is-happening-facebook-extends-remote-work}}

Copied to clipboard.

\begin{itemize}
\item
  \textbf{Facebook} will allow its employees to work from home until at
  least July 2021, the social media giant confirmed on Friday, and will
  give employees an additional \$1,000 to support continued remote work.
  The company believed in the importance of in-person communication and
  \href{https://www.nytimes3xbfgragh.onion/2020/03/24/technology/virus-facebook-usage-traffic.html}{had
  few remote workers} before the pandemic, but Mark Zuckerberg, the
  chief executive, has said as many as
  \href{https://www.nytimes3xbfgragh.onion/2020/05/21/technology/facebook-remote-work-coronavirus.html}{half
  of Facebook's employees will be permanently remote} within the next
  decade.
\item
  \textbf{Nintendo}, the creator of Animal Crossing for its Switch
  consoles, reported on Thursday
  \href{https://www.nytimes3xbfgragh.onion/2020/08/06/business/everyones-lockdown-obsession-with-animal-crossing-lifts-nintendos-profits-to-1-billion.html}{a
  staggering 541 percent increase in quarterly profit} from the previous
  year. Behind that number were 10.6 million sales of Animal Crossing:
  New Horizons, pushing the Japanese gaming company's net income to
  106.5 billion yen (\$1 billion), and the company said ``sales of this
  title continue to be strong with no loss of momentum.'' Since it was
  released, there have been more than 22 million sales of the game.
\item
  \textbf{Snap}, the company behind the popular social media app
  \textbf{Snapchat}, is
  \href{https://www.nytimes3xbfgragh.onion/2020/08/06/business/snap-will-make-a-major-effort-to-register-first-time-voters.html}{planning
  a major push to register first-time voters} within its app and to
  guide them through the ballot process ahead of the election on Nov. 3.
  Beginning in early September, the app will introduce a new tool in
  partnership with \textbf{TurboVote} that allows users to register from
  within their Snap account with a streamlined set of prompts.
\item
  More than half of Americans who flew in the past year
  \href{https://www.nytimes3xbfgragh.onion/2020/08/06/business/a-majority-of-travelers-are-not-willing-to-get-on-planes-a-new-survey-shows.html}{are
  not ready to do so again}, according to a new survey, underscoring the
  difficulty airlines face in convincing people it is safe for them to
  get back on planes. Younger adults are more willing to travel; only a
  third of those between the ages of 18 and 34 expressed discomfort with
  the idea. But older adults, who tend to have more time and money to
  travel, are far more reluctant. Among those 55 or older, 69 percent
  said they would not be comfortable taking a flight.
\end{itemize}

\hypertarget{site-index}{%
\subsection{Site Index}\label{site-index}}

\hypertarget{site-information-navigation}{%
\subsection{Site Information
Navigation}\label{site-information-navigation}}

\begin{itemize}
\tightlist
\item
  \href{https://help.nytimes3xbfgragh.onion/hc/en-us/articles/115014792127-Copyright-notice}{©~2020~The
  New York Times Company}
\end{itemize}

\begin{itemize}
\tightlist
\item
  \href{https://www.nytco.com/}{NYTCo}
\item
  \href{https://help.nytimes3xbfgragh.onion/hc/en-us/articles/115015385887-Contact-Us}{Contact
  Us}
\item
  \href{https://www.nytco.com/careers/}{Work with us}
\item
  \href{https://nytmediakit.com/}{Advertise}
\item
  \href{http://www.tbrandstudio.com/}{T Brand Studio}
\item
  \href{https://www.nytimes3xbfgragh.onion/privacy/cookie-policy\#how-do-i-manage-trackers}{Your
  Ad Choices}
\item
  \href{https://www.nytimes3xbfgragh.onion/privacy}{Privacy}
\item
  \href{https://help.nytimes3xbfgragh.onion/hc/en-us/articles/115014893428-Terms-of-service}{Terms
  of Service}
\item
  \href{https://help.nytimes3xbfgragh.onion/hc/en-us/articles/115014893968-Terms-of-sale}{Terms
  of Sale}
\item
  \href{https://spiderbites.nytimes3xbfgragh.onion}{Site Map}
\item
  \href{https://help.nytimes3xbfgragh.onion/hc/en-us}{Help}
\item
  \href{https://www.nytimes3xbfgragh.onion/subscription?campaignId=37WXW}{Subscriptions}
\end{itemize}
