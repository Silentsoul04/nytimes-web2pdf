Sections

SEARCH

\protect\hyperlink{site-content}{Skip to
content}\protect\hyperlink{site-index}{Skip to site index}

\href{https://myaccount.nytimes3xbfgragh.onion/auth/login?response_type=cookie\&client_id=vi}{}

\href{https://www.nytimes3xbfgragh.onion/section/todayspaper}{Today's
Paper}

\begin{itemize}
\item
  \href{https://www.nytimes3xbfgragh.onion/live/2020/08/20/us/dnc-convention-election?action=click\&pgtype=Article\&state=default\&region=TOP_BANNER\&context=storylines_menu}{D.N.C.
  Updates}
\item
  \href{https://www.nytimes3xbfgragh.onion/2020/08/20/us/politics/biden-presidential-nomination-dnc.html?action=click\&pgtype=Article\&state=default\&region=TOP_BANNER\&context=storylines_menu}{Biden's
  Speech}
\item
  \href{https://www.nytimes3xbfgragh.onion/interactive/2019/us/elections/2020-presidential-election-calendar.html?action=click\&pgtype=Article\&state=default\&region=TOP_BANNER\&context=storylines_menu}{Election
  Calendar}
\item
  \href{https://www.nytimes3xbfgragh.onion/interactive/2020/08/11/us/politics/vote-by-mail-us-states.html?action=click\&pgtype=Article\&state=default\&region=TOP_BANNER\&context=storylines_menu}{Voting
  by Mail}
\item
  \href{https://www.nytimes3xbfgragh.onion/newsletters/politics?action=click\&pgtype=Article\&state=default\&region=TOP_BANNER\&context=storylines_menu}{Politics
  Newsletter}
\end{itemize}

\hypertarget{tackling-coronavirus-anxiety-with-dating-apps-recipes-and-old-sports-videos}{%
\section{Tackling Coronavirus Anxiety With Dating Apps, Recipes and Old
Sports
Videos}\label{tackling-coronavirus-anxiety-with-dating-apps-recipes-and-old-sports-videos}}

\includegraphics{https://static01.graylady3jvrrxbe.onion/images/2020/03/18/multimedia/live-blog-coronavirus-covid-19-03-18-header4/merlin_170676966_96dc5d43-a38a-4a63-8803-91a05d971996-articleLarge.jpg?quality=75\&auto=webp\&disable=upscale}

The coronavirus is changing how we live our daily lives. Taking a look
at how the global pandemic has affected various aspects of life in the
United States reveals the unique nature of this crisis.

\begin{itemize}
\item
  For the latest updates on how the coronavirus is changing lives in the
  U.S.,
  \href{https://www.nytimes3xbfgragh.onion/live/2020/covid-19-03-19}{read
  Thursday's live updates}.
\item
  Here's a happier postponement: Google wants to help its workers focus
  on ``mission-critical activities'' during the coronavirus pandemic, so
  the company is
  \href{https://www.nytimes3xbfgragh.onion/live/2020/coronavirus-covid-19-03-18?action=click\&module=Top\%20Stories\&pgtype=Homepage\#in-a-happier-postponement-google-presses-pause-on-employee-reviews}{delaying
  performance reviews}. Speaking of focus,
  \href{https://www.nytimes3xbfgragh.onion/2020/03/18/arts/viral-content-coronavirus.html}{the
  unending anxiety of coronavirus} was on our critic's mind.
\item
  Our food distribution networks
  \href{https://www.nytimes3xbfgragh.onion/2020/03/18/business/coronavirus-food-supply-kitchens.html?action=click\&module=RelatedLinks\&pgtype=Article}{are
  under siege}. At the same time,
  \href{https://www.nytimes3xbfgragh.onion/2020/03/17/dining/food-shortage-coronavirus.html}{food
  is proving stressful} for people who are not used to cooking for
  themselves. We have solutions. Our daily pantry recipe is
  \href{https://www.nytimes3xbfgragh.onion/live/2020/coronavirus-covid-19-03-18\#cooking-from-your-pantry-canned-tuna-pasta-edition}{canned
  tuna pasta.} Preparing a meal can be so comforting; here are some easy
  ways to
  \href{https://www.nytimes3xbfgragh.onion/article/easy-recipes-coronavirus.html?action=click\&module=RelatedLinks\&pgtype=Article}{make
  breakfast, lunch or dinner}.
\item
  In the sporting world, both the N.F.L. and N.B.A. announced today that
  they would make archival content available
  \href{https://www.nytimes3xbfgragh.onion/live/2020/coronavirus-covid-19-03-18?action=click\&module=Top\%20Stories\&pgtype=Homepage\#sports-fans-can-rewatch-nba-and-nfl-games-for-free-for-a-limited-time}{for
  free --- for a limited time}. People looking to get a fix of live
  sports can get it,
  \href{https://www.nytimes3xbfgragh.onion/live/2020/coronavirus-covid-19-03-18\#people-fishing-for-sports-to-watch-are-in-luck}{provided
  they like fishing}. And college coaches (along with other celebrities)
  are giving us all a
  \href{https://www.nytimes3xbfgragh.onion/live/2020/coronavirus-covid-19-03-18?action=click\&module=Top\%20Stories\&pgtype=Homepage\#top-coaches-in-college-sports-are-helping-spread-the-word-about-the-coronavirus}{pep
  talk}.
\end{itemize}

Last Updated

March 19, 2020, 9:05 a.m. ET

March 19, 2020, 9:05 a.m. ET

\includegraphics{https://static01.graylady3jvrrxbe.onion/images/icons/t_logo_291_black.png}

March 18, 2020, 9:00 p.m. ET

March 18, 2020, 9:00 p.m. ET

By \href{https://www.nytimes3xbfgragh.onion/by/jacey-fortin}{Jacey
Fortin}

\hypertarget{dont-trust-memes-that-promise-coronavirus-cures}{%
\subsection{\texorpdfstring{\protect\hyperlink{dont-trust-memes-that-promise-coronavirus-cures}{Don't
trust memes that promise coronavirus
cures.}}{Don't trust memes that promise coronavirus cures.}}\label{dont-trust-memes-that-promise-coronavirus-cures}}

\includegraphics{https://static01.graylady3jvrrxbe.onion/images/2020/03/19/us/19xp-virus-badcures-print/merlin_76506772_eb4bb49d-6cbf-4457-ab01-5c8a3aa30fab-articleLarge.jpg?quality=75\&auto=webp\&disable=upscale}

On social media, memes --- often featuring urgent instructions or
dystopian graphics --- have become efficient vectors of bad advice about
how to fight the coronavirus, and health care professionals are working
to
\href{https://www.nytimes3xbfgragh.onion/article/coronavirus-myths.html}{stop
the spread of misinformation}.

One meme, misstating the benefits of gargling salty water, shows the
virus as a cluster of green burrs infecting the throat of a glowing blue
man. One series of posts with bad advice --- including claims that
sunshine could kill the virus and that ice cream should be avoided ---
tacked on the name UNICEF.

``This is, of course, not true,'' said Christopher Tidey, a spokesman
for UNICEF, the United Nations Children's Fund.

``Misinformation during times of a health crisis can result in people
being left unprotected or more vulnerable to the virus,'' he added. ``It
can also spread paranoia, fear and stigmatization, and have other
consequences, like offering a false sense of protection.''

Here are
\href{https://www.nytimes3xbfgragh.onion/2020/03/18/health/coronavirus-cure-gargle-water.html}{some
of the false claims} that are spreading via Twitter, Facebook and
WhatsApp.

Read more

\href{https://www.nytimes3xbfgragh.onion/by/john-branch}{\includegraphics{https://static01.graylady3jvrrxbe.onion/images/2019/03/01/multimedia/author-john-branch/author-john-branch-thumbLarge.png}}

March 18, 2020, 8:30 p.m. ET

March 18, 2020, 8:30 p.m. ET

By \href{https://www.nytimes3xbfgragh.onion/by/john-branch}{John Branch}

\hypertarget{facing-a-combination-of-challenges-colorado-ski-resorts-shut-down}{%
\subsection{\texorpdfstring{\protect\hyperlink{facing-a-combination-of-challenges-colorado-ski-resorts-shut-down}{Facing
a combination of challenges, Colorado ski resorts shut
down.}}{Facing a combination of challenges, Colorado ski resorts shut down.}}\label{facing-a-combination-of-challenges-colorado-ski-resorts-shut-down}}

\includegraphics{https://static01.graylady3jvrrxbe.onion/images/2020/03/17/us/00virus-skiing/merlin_170522604_25d063ec-cf34-46cb-9cd5-61edab0b2bd2-articleLarge.jpg?quality=75\&auto=webp\&disable=upscale}

Ten days after Colorado's first coronavirus case was discovered in the
Rocky Mountains, the state's multibillion-dollar ski industry has shut
itself down.

``It's unlike anything we've seen before in our industry's history,''
said Melanie Mills, president of Colorado Ski Country USA, a trade
association representing 23 of the state's ski areas.

By last weekend, Eagle County, home to Vail and Beaver Creek resorts,
had as many coronavirus infections --- 24 --- as Denver County, despite
having one-tenth of the population. Ski resorts, with their seasonal
spikes in global visitors, and their nearby communities, which have
limited medical services, create a unique combination of concerns.

Last week, there was growing worry that, as mountain medical centers
filled with tourists increasingly sickened by the coronavirus as well as
other ski-related injuries and illnesses, there would be little room for
the expected onslaught of coronavirus cases among residents.

The hope of keeping the slopes open came to a screeching halt when Vail
Resorts announced it would temporarily shut its resorts in North
America, including Vail, Beaver Creek, Breckenridge and Keystone in
Colorado.

Among the disappointed was Jeff Kottkamp, a former lieutenant governor
of Florida, who was in Colorado on vacation. ``Thank you for making this
announcement as we are driving in to Vail,'' he wrote on Twitter in a
post he later deleted.

Gov. Jared Polis of Colorado replied: ``Thank you for your deep concerns
regarding the health of our residents in the face of a global
pandemic.''

Read more

\hypertarget{advertisement}{%
\subsubsection{Advertisement}\label{advertisement}}

\protect\hyperlink{after-dfp-ad-mid1}{Continue reading the main story}

\includegraphics{https://static01.graylady3jvrrxbe.onion/images/icons/t_logo_291_black.png}

March 18, 2020, 8:00 p.m. ET

March 18, 2020, 8:00 p.m. ET

By Helen Ouyang, M.D.

\hypertarget{doctors-are-turning-to-social-media-for-coronavirus-advice-too}{%
\subsection{\texorpdfstring{\protect\hyperlink{doctors-are-turning-to-social-media-for-coronavirus-advice-too}{Doctors
are turning to social media for coronavirus advice,
too.}}{Doctors are turning to social media for coronavirus advice, too.}}\label{doctors-are-turning-to-social-media-for-coronavirus-advice-too}}

\includegraphics{https://static01.graylady3jvrrxbe.onion/images/2020/03/18/world/18virus-live-social-media/merlin_170017497_0a4d51ea-67ce-4f96-9ea0-5f94e3a6ec7d-articleLarge.jpg?quality=75\&auto=webp\&disable=upscale}

Last week, many people were astonished to hear that Dr. Kurt Kloss, an
emergency room physician in New York,
\href{https://www.nytimes3xbfgragh.onion/2020/03/13/us/politics/coronavirus-jared-kushner-kloss.html}{reached
out to a Facebook group} for some 20,000 of his colleagues seeking
advice about how to handle the coronavirus outbreak. ``If you were in
charge of Federal response to the Pandemic,'' his post began, ``what
would your recommendation be?''

The question wasn't just hypothetical. Dr. Kloss's daughter is married
to the brother of Jared Kushner, who had been put in charge of the White
House response to the pandemic. But many people commented in alarm:
Crowdsourcing medical advice on social media, is that a reliable way to
get life-or-death health information?

As an emergency room doctor at the front lines of the pandemic, however,
I wasn't surprised. I've been working back-to-back shifts at
NewYork-Presbyterian Hospital, where the state's first patient
hospitalized with Covid-19 is being treated, and have had several
patients test positive for the novel coronavirus. And like many of my
colleagues, I have been gathering information from Facebook, Twitter and
other social media outlets.

\includegraphics{https://static01.graylady3jvrrxbe.onion/images/icons/t_logo_291_black.png}

March 18, 2020, 7:30 p.m. ET

March 18, 2020, 7:30 p.m. ET

By Lauren Messman

\hypertarget{frieze-cancels-its-new-york-art-fair}{%
\subsection{\texorpdfstring{\protect\hyperlink{frieze-cancels-its-new-york-art-fair}{Frieze
cancels its New York art
fair.}}{Frieze cancels its New York art fair.}}\label{frieze-cancels-its-new-york-art-fair}}

\includegraphics{https://static01.graylady3jvrrxbe.onion/images/2020/03/18/arts/18VIRUS-FRIEZE/18VIRUS-FRIEZE-articleLarge.jpg?quality=75\&auto=webp\&disable=upscale}

Frieze New York has been canceled ``in light of global health concerns
regarding COVID-19,'' Frieze said on Wednesday.

The art fair was to take place May 7 to 10 at Randalls Island Park, with
200 participating galleries and a special focus on artists of Latin
descent and the Chicago artist community.

Frieze said it was also developing a
\href{https://www.nytimes3xbfgragh.onion/2020/03/16/arts/design/art-galleries-online-viewing-coronavirus.html}{virtual
viewing room}, and that its first phase would be tied to Frieze New
York.
\href{https://www.nytimes3xbfgragh.onion/2019/05/01/arts/sculpture-frieze-new-york-.html}{Frieze
Sculpture}, which was to open at Rockefeller Center on April 22, will be
postponed until the summer. The decision to cancel Frieze New York
follows the cancellations and postponements of various international art
fairs amid the coronavirus pandemic.

\includegraphics{https://static01.graylady3jvrrxbe.onion/images/icons/t_logo_291_black.png}

March 18, 2020, 7:15 p.m. ET

March 18, 2020, 7:15 p.m. ET

By \href{https://www.nytimes3xbfgragh.onion/by/neal-e-boudette}{Neal E.
Boudette}

\hypertarget{major-automakers-close-their-plants-in-north-america}{%
\subsection{\texorpdfstring{\protect\hyperlink{major-automakers-close-their-plants-in-north-america}{Major
automakers close their plants in North
America.}}{Major automakers close their plants in North America.}}\label{major-automakers-close-their-plants-in-north-america}}

\includegraphics{https://static01.graylady3jvrrxbe.onion/images/2020/03/18/business/18virus-autos1/merlin_169290813_a5143083-4897-45ee-9647-4bd2a6e2b2f5-articleLarge.jpg?quality=75\&auto=webp\&disable=upscale}

With fear of infection rising among factory workers, and few customers
shopping for cars, several automakers on Wednesday decided to idle their
plants in the United States, Canada and Mexico for at least a week.
Their decisions will put tens of thousands of people out of work and add
to the coronavirus outbreak's growing economic toll.

The country's largest automakers --- General Motors, Ford Motor and Fiat
Chrysler --- decided to close plants after the United Auto Workers union
pressured them to do so to protect workers. That pressure intensified
after it was revealed on Wednesday that a worker at a Ford truck plant
in Dearborn, Mich., had tested positive for the virus.

Honda and Toyota also said they would idle their North American
factories. The shutdown of car plants will force hundreds of companies
that produce parts and components to follow suit over the coming days.

Although some autoworkers will be eligible for sick pay, many will get
only a portion of their income and others will have to rely on
unemployment insurance.

More than one million people are employed in automobile and auto parts
manufacturing in the United States, and 1.3 million work for auto
dealerships, according to the Bureau of Labor Statistics.

Automakers went into the week hoping to keep their plants running and
employees safe by altering shift schedules to leave more time for
sanitizing plants and reducing contact between workers. On Monday,
G.M.'s Chevrolet division began offering zero-percent loans to lure
consumers into dealerships. Hyundai offered to let customers return
recently purchased cars if had they lost their jobs.

At the same time the U.A.W. was pressing the three companies based in
and around Detroit to halt production for two weeks. On Tuesday G.M.,
Ford and Fiat Chrysler agreed to take steps short of shutting down
production.

Then on Wednesday morning Honda announced its plans to stop production,
and the news about the Ford worker in Dearborn, Mich., was made public.
In response, Ford halted work at the final assembly section of the plant
while continuing production in the stamping and body shop areas.

Just hours later, all three of the large U.S. automakers reversed course
and said they would idle their factories.

Niraj Chokshi contributed reporting.

Read more

\hypertarget{advertisement-1}{%
\subsubsection{Advertisement}\label{advertisement-1}}

\protect\hyperlink{after-dfp-ad-mid2}{Continue reading the main story}

\includegraphics{https://static01.graylady3jvrrxbe.onion/images/icons/t_logo_291_black.png}

March 18, 2020, 7:00 p.m. ET

March 18, 2020, 7:00 p.m. ET

By Margaux Laskey

\hypertarget{preparing-and-serving-a-meal-even-a-simple-one-can-bring-great-comfort}{%
\subsection{\texorpdfstring{\protect\hyperlink{preparing-and-serving-a-meal-even-a-simple-one-can-bring-great-comfort}{Preparing
and serving a meal, even a simple one, can bring great
comfort.}}{Preparing and serving a meal, even a simple one, can bring great comfort.}}\label{preparing-and-serving-a-meal-even-a-simple-one-can-bring-great-comfort}}

\includegraphics{https://static01.graylady3jvrrxbe.onion/images/2018/10/23/dining/as-white-bean-tomato/as-white-bean-tomato-articleLarge-v2.jpg?quality=75\&auto=webp\&disable=upscale}

There's no doubt about it: It's an uncertain and scary time, but you and
your loved ones still need to eat, and the act of preparing and serving
a meal --- even a simple one --- can bring great comfort to the cook as
well as to the diner. If you have a decently stocked pantry, you can
make a wonderful meal out of a few staples. And if you are able to
safely get to the store or have groceries delivered to you, your choices
are even greater.

Breakfast: There's something about starting the day with a full belly
that makes everything seem like it's going to be OK.
\href{https://cooking.nytimes3xbfgragh.onion/recipes/1893-everyday-pancakes}{Pancakes},
\href{https://cooking.nytimes3xbfgragh.onion/recipes/1017409-waffles}{waffles},
a
\href{https://cooking.nytimes3xbfgragh.onion/recipes/6648-dutch-baby}{Dutch
baby},
\href{https://cooking.nytimes3xbfgragh.onion/recipes/1015210-buttermilk-biscuits}{biscuits}
and
\href{https://cooking.nytimes3xbfgragh.onion/recipes/11826-simple-blueberry-muffins}{muffins}
are cozy options, and leftovers can be frozen for later.
\href{https://cooking.nytimes3xbfgragh.onion/68861692-nyt-cooking/477-our-favorite-granolas}{Granola},
too. (Top your evening ice cream sundae with it later.) And if you
really have some time on your hands, make breakfast an event and whip up
a dish typically meant for a weekend, like
\href{https://cooking.nytimes3xbfgragh.onion/recipes/1018626-eggs-benedict}{eggs
Benedict},
\href{https://cooking.nytimes3xbfgragh.onion/recipes/1020714-english-muffin-breakfast-casserole}{English
muffin casserole} or
\href{https://cooking.nytimes3xbfgragh.onion/recipes/1013741-all-purpose-biscuits}{biscuits}
with
\href{https://cooking.nytimes3xbfgragh.onion/recipes/1013740-sausage-gravy}{sausage
gravy}.

Lunch: If you're working from home, take this opportunity to make
something delicious. (And maybe don't eat it stooped over your laptop or
scrolling through Twitter?) Melissa Clark's
\href{https://cooking.nytimes3xbfgragh.onion/recipes/1020464-sardine-toasts-with-tomato-and-sweet-onion}{sardine
toasts with tomato and sweet onion} or
\href{https://cooking.nytimes3xbfgragh.onion/recipes/1020532-grilled-cheese-with-apples-and-apple-butter}{grilled
cheese with apples and apple butter} come together in a flash. If soup
is what you're craving, Alexa Weibel's
\href{https://cooking.nytimes3xbfgragh.onion/recipes/1020746-easiest-chicken-noodle-soup}{easiest
chicken noodle soup} or Colu Henry's
\href{https://cooking.nytimes3xbfgragh.onion/recipes/1020860-pasta-e-ceci-italian-pasta-and-chickpea-stew}{pasta
e ceci} will soothe your soul. Or transform that pasta in your pantry
into
\href{https://cooking.nytimes3xbfgragh.onion/recipes/8357-spaghetti-with-fried-eggs}{spaghetti
with fried eggs} or
\href{https://cooking.nytimes3xbfgragh.onion/recipes/1020668-midnight-pasta-with-roasted-garlic-olive-oil-and-chile}{midnight
pasta}, a bright and briny dish made with capers, anchovies and garlic.
\ldots{}

\includegraphics{https://static01.graylady3jvrrxbe.onion/images/icons/t_logo_291_black.png}

March 18, 2020, 6:30 p.m. ET

March 18, 2020, 6:30 p.m. ET

By The New York Times

\hypertarget{answers-to-some-common-questions-about-the-outbreak}{%
\subsection{\texorpdfstring{\protect\hyperlink{answers-to-some-common-questions-about-the-outbreak}{Answers
to some common questions about the
outbreak.}}{Answers to some common questions about the outbreak.}}\label{answers-to-some-common-questions-about-the-outbreak}}

\includegraphics{https://static01.graylady3jvrrxbe.onion/images/2020/03/18/world/18virus-lives-faqs/merlin_170701095_350dd1ff-daa4-4a0d-9762-3af9a81093bc-articleLarge.jpg?quality=75\&auto=webp\&disable=upscale}

The coronavirus has dramatically shifted so much about our lives this
year.

We're here to help with answers to common questions on
\href{https://www.nytimes3xbfgragh.onion/interactive/2020/world/coronavirus-tips-advice.html\#health}{health},
\href{https://www.nytimes3xbfgragh.onion/interactive/2020/world/coronavirus-tips-advice.html\#money}{money},
daily
\href{https://www.nytimes3xbfgragh.onion/interactive/2020/world/coronavirus-tips-advice.html\#life}{life},
\href{https://www.nytimes3xbfgragh.onion/interactive/2020/world/coronavirus-tips-advice.html\#politics}{politics},
\href{https://www.nytimes3xbfgragh.onion/interactive/2020/world/coronavirus-tips-advice.html\#science}{science}
and \href{http://travel/}{travel}.

For a full list of questions,
\href{https://www.nytimes3xbfgragh.onion/interactive/2020/world/coronavirus-tips-advice.html}{click
here}.

\hypertarget{health}{%
\subsection{Health}\label{health}}

\begin{itemize}
\tightlist
\item
  What if somebody in my family gets sick?
\end{itemize}

If the family member doesn't need hospitalization and can be cared for
at home, you should help him or her with basic needs and monitor their
symptoms, while also keeping as much distance as possible, according to
\href{https://www.cdc.gov/coronavirus/2019-ncov/hcp/guidance-prevent-spread.html}{guidelines
issued by the C.D.C. in February.} If there's space, the sick family
member should stay in a separate room and use a separate bathroom to
minimize contact with healthy family members. Any shared spaces should
have good airflow, like an open window or an air conditioner, and you
shouldn't allow any visitors except for those who need to be in the
house. Your family member should wear a face mask when he or she is
around others, and if he or she can't because of difficulty breathing,
you should wear a mask.

Make sure not to share any dishes or other household items with your
sick family member and to regularly clean surfaces like counters,
doorknobs, toilets and tables. Don't forget to wash your hands
frequently.

\hypertarget{money}{%
\subsection{Money}\label{money}}

\begin{itemize}
\tightlist
\item
  Help me understand the economic fallout.
\end{itemize}

It's felt a lot like 2008, hasn't it? Peter S. Goodman, an economics
correspondent for The Times, says there's a key difference between that
economic turmoil and today's: the utter unpredictability of the
outbreak's spread. ``The disaster feels eerily familiar, with trillions
of dollars in wealth annihilated near-daily and deepening fears that
businesses will fail,''
\href{https://www.nytimes3xbfgragh.onion/2020/03/13/business/coronavirus-global-economy.html}{he
writes}. ``Yet the traditional policy prescriptions seem no match for
the affliction at hand.''

The very means of controlling the health crisis --- keeping workers
home, limiting travel and disrupting commerce --- risk making the
economic crisis worse.

``A sense of powerlessness is feeding the fear that is prompting
investors to dump anything risky, which is in turn damaging economic
prospects and reinforcing fear --- a feedback loop,'' he writes.

\href{https://www.nytimes3xbfgragh.onion/2020/03/13/business/coronavirus-global-economy.html}{Read
Peter S. Goodman's full story here}.

\hypertarget{politics-and-government}{%
\subsection{Politics and Government}\label{politics-and-government}}

\begin{itemize}
\tightlist
\item
  How is this going to affect the 2020 election?
\end{itemize}

Several states have already postponed their primaries. Officials in
states that are going ahead with primaries have been
\href{https://www.nytimes3xbfgragh.onion/2020/03/09/us/virus-election-voting.html}{trying
to emphasize the new basics of voting hygiene}, like using hand
sanitizer at polling stations.

And though the general election has yet to begin, it is already
\href{https://www.nytimes3xbfgragh.onion/2020/03/12/us/politics/coronavirus-2020-campaign.html}{being
profoundly shaped by the coronavirus}, which has become the central
issue between President Trump and the two major Democratic candidates,
former Vice President Joseph R. Biden Jr. and Senator Bernie Sanders of
Vermont. The pandemic's effects on the economy have challenged Mr.
Trump's core message, that Americans are better off than before he took
office. Mr. Biden has proposed a detailed plan for hospitals and
research, echoing the language of presidents in past crises. He and Mr.
Sanders curtailed their travel and canceled large gatherings like
rallies.

It would be enormously difficult to cancel or postpone the November
general election, though.

\href{https://www.nytimes3xbfgragh.onion/2020/03/14/us/politics/election-postponed-canceled.html}{Here's
more about holding an election in a crisis}.

Reporting was contributed by: Jenny Gross, Laura M. Holson, Julia
Jacobs, Ron Lieber, Neil MacFarquhar, Erin McCann, Aimee Ortiz, Gretchen
Reynolds, Andrea Salcedo and Alan Yuhas.

Read more

\href{https://www.nytimes3xbfgragh.onion/by/alan-blinder}{\includegraphics{https://static01.graylady3jvrrxbe.onion/images/2018/08/24/multimedia/author-alan-blinder/author-alan-blinder-thumbLarge.png}}\href{https://www.nytimes3xbfgragh.onion/by/neil-vigdor}{\includegraphics{https://static01.graylady3jvrrxbe.onion/images/2019/07/25/reader-center/author-neil-vigdor/author-neil-vigdor-thumbLarge.png}}

March 18, 2020, 6:21 p.m. ET

March 18, 2020, 6:21 p.m. ET

By \href{https://www.nytimes3xbfgragh.onion/by/alan-blinder}{Alan
Blinder} and
\href{https://www.nytimes3xbfgragh.onion/by/neil-vigdor}{Neil Vigdor}

\hypertarget{top-coaches-and-other-celebrities-are-helping-spread-the-word-about-the-coronavirus}{%
\subsection{\texorpdfstring{\protect\hyperlink{top-coaches-and-other-celebrities-are-helping-spread-the-word-about-the-coronavirus}{Top
coaches and other celebrities are helping spread the word about the
coronavirus.}}{Top coaches and other celebrities are helping spread the word about the coronavirus.}}\label{top-coaches-and-other-celebrities-are-helping-spread-the-word-about-the-coronavirus}}

Need a public health pep talk? Louisiana did, so it called in a football
coach.

``For every winning team, a key to success is learning the playbook,''
Ed Orgeron, Louisiana State's coach, says in a public service
announcement that has drawn more than three million views in recent
days. ``That's true in football, and it's also true as we take on the
coronavirus.''

In the spot, Orgeron, whose
\href{https://www.nytimes3xbfgragh.onion/2020/01/09/sports/ncaafootball/ed-orgeron.html}{distinctive
voice} became a sensation during the Tigers'
\href{https://www.nytimes3xbfgragh.onion/2020/01/13/sports/ncaafootball/clemson-lsu.html}{national
championship} season, runs through the list of guidance that doctors and
politicians have been making for weeks: Cover your cough with your
elbow, wash your hands for 20 seconds, stay home if you're feeling sick.
And so on.

Orgeron is part of a growing number of public figures who have used
their status as influencers to raise public awareness about social
distancing and best hygiene practices during the coronavirus pandemic.

\begin{quote}
When Coach O speaks, we all listen.

For more information on how to prevent the spread of COVID-19, visit:
\url{https://t.co/89sZCjY9n3}\href{https://twitter.com/Coach_EdOrgeron?ref_src=twsrc\%5Etfw}{@Coach\_EdOrgeron}
\href{https://twitter.com/LADeptHealth?ref_src=twsrc\%5Etfw}{@LADeptHealth}
\href{https://twitter.com/hashtag/lagov?src=hash\&ref_src=twsrc\%5Etfw}{\#lagov}
\href{https://twitter.com/hashtag/lalege?src=hash\&ref_src=twsrc\%5Etfw}{\#lalege}
\href{https://twitter.com/hashtag/COVID19?src=hash\&ref_src=twsrc\%5Etfw}{\#COVID19}
\href{https://t.co/OxJ5u2xBmo}{pic.twitter.com/OxJ5u2xBmo}

--- John Bel Edwards (@LouisianaGov)
\href{https://twitter.com/LouisianaGov/status/1238905211110019073?ref_src=twsrc\%5Etfw}{March
14, 2020}
\end{quote}

``I just wanted to do anything I could do to help the state of
Louisiana,'' Orgeron said in an interview on Wednesday. ``After winning
the national championship, having the Heisman Trophy winner, people are
going to listen, and it carries a lot of weight.''

Orgeron, who is close to Gov. John Bel Edwards, said that the governor's
office had requested his help. The coach, a Louisiana native, said he
hoped that the state's residents would see the video and think, ``If
Coach O is taking it seriously, hey, let's take it seriously.''

Although public officials have long turned to college sports figures to
help them reach audiences on an array of topics --- in 2015, for
instance, Gus Malzahn of Auburn and Nick Saban of Alabama recorded
\href{https://www.ledger-enquirer.com/news/local/news-columns-blogs/article32974485.html}{spots
about voter registration} --- the pandemic has prompted a new wave of
announcements.

Three of North Carolina's most prominent college basketball coaches ---
Kevin Keatts of North Carolina State, Mike Krzyzewski of Duke and Roy
Williams of North Carolina --- appeared this week in a video similar to
the one that starred Orgeron.

\begin{quote}
Even the biggest rivals agree, when it comes to stopping COVID-19 we've
got to be on the same team.
\href{https://t.co/fsoRuYq3eL}{pic.twitter.com/fsoRuYq3eL}

--- Governor Roy Cooper (@NC\_Governor)
\href{https://twitter.com/NC_Governor/status/1239683368868028428?ref_src=twsrc\%5Etfw}{March
16, 2020}
\end{quote}

``Avoid touching your face,'' Krzyzewski said. ``I know that can be
difficult, but if you practice hard, you can do it.''

``We can beat this,'' Williams added later in the recording, ``but we
all need to play together on the same team right now.''

Kevin Bacon, the ``Footloose'' actor who is synonymous with the game Six
Degrees of Kevin Bacon, joined the celebrities urging people to stay
home during the outbreak. On Wednesday, Bacon posted a photograph of
himself holding up a chalkboard with the words ``I Stay Home for Kyra
Sedgwick'' --- a reference to his wife, who is also an actor --- and the
hashtag 6Degrees.

Public health officials have urged people to keep at least six feet away
from one another to prevent the spread of the virus. In the Six Degrees
of Kevin Bacon, players say the name of an actor. Most have appeared
with Bacon in a movie or with one of his co-stars --- a vast majority
can be connected to Bacon by less than six degrees.

\begin{quote}
\end{quote}

The comedic director Mel Brooks teamed up with his son, Max Brooks, in a
public service announcement that they shared Monday on Twitter. The
elder Brooks watched his son, 47, through a window.

``He's 93,'' Max Brooks said. ``If I get the coronavirus, I'll probably
be OK. But if I give it to him, he could give it to Carl Reiner, who
could give it to Dick Van Dyke, and before I know it, I've wiped out a
whole generation of comedic legends.''

\begin{quote}
A message from me and my dad,
\href{https://twitter.com/MelBrooks?ref_src=twsrc\%5Etfw}{@Melbrooks}.
\href{https://twitter.com/hashtag/coronavirus?src=hash\&ref_src=twsrc\%5Etfw}{\#coronavirus}
\href{https://twitter.com/hashtag/DontBeASpreader?src=hash\&ref_src=twsrc\%5Etfw}{\#DontBeASpreader}
\href{https://t.co/Hqhc4fFXbe}{pic.twitter.com/Hqhc4fFXbe}

--- Max Brooks (@maxbrooksauthor)
\href{https://twitter.com/maxbrooksauthor/status/1239624352305303552?ref_src=twsrc\%5Etfw}{March
16, 2020}
\end{quote}

Read more

\hypertarget{advertisement-2}{%
\subsubsection{Advertisement}\label{advertisement-2}}

\protect\hyperlink{after-dfp-ad-mid3}{Continue reading the main story}

\href{https://www.nytimes3xbfgragh.onion/by/amanda-hess}{\includegraphics{https://static01.graylady3jvrrxbe.onion/images/2018/02/16/multimedia/author-amanda-hess/author-amanda-hess-thumbLarge-v2.png}}

March 18, 2020, 6:00 p.m. ET

March 18, 2020, 6:00 p.m. ET

By \href{https://www.nytimes3xbfgragh.onion/by/amanda-hess}{Amanda Hess}

\hypertarget{a-critic-considers-the-unending-anxiety-of-coronavirus-content}{%
\subsection{\texorpdfstring{\protect\hyperlink{a-critic-considers-the-unending-anxiety-of-coronavirus-content}{A
critic considers the unending anxiety of coronavirus
content.}}{A critic considers the unending anxiety of coronavirus content.}}\label{a-critic-considers-the-unending-anxiety-of-coronavirus-content}}

\includegraphics{https://static01.graylady3jvrrxbe.onion/images/2020/03/21/arts/18virus-viralcontent/18virus-viralcontent-articleLarge.jpg?quality=75\&auto=webp\&disable=upscale}

This past weekend, as coronavirus radiated across the country and sent
Americans scurrying into their homes, Rosanne Cash
\href{https://twitter.com/rosannecash/status/1238700345548627969}{tweeted}:
``Just a reminder that when Shakespeare was quarantined because of the
plague, he wrote King Lear.''

I wonder what the ``King Lear'' of Covid-19 will be. Maybe this woman
\href{https://twitter.com/cbouzy/status/1239210356435812354}{licking an
airplane toilet seat on TikTok}?

Shakespeare's
\href{https://slate.com/culture/2020/03/shakespeare-plague-influence-hot-hand-ben-cohen.html}{plague
streak} --- he's believed to have written ``King Lear,'' ``Macbeth'' and
``Antony and Cleopatra'' in the space of a couple of years --- coincided
with London playhouses shuttering, acting troupes leaving town to play
plague-free villages and the Bard hanging back at home, nothing to do
but plot an elaborate series of tragic murders. But Shakespeare was not
online. Four hundred years later, isolation doesn't help to dispel
creative distraction --- it beckons it in. We are sheltering in place
with devices designed to amplify diversions and exploit obsessions.

\href{https://www.nytimes3xbfgragh.onion/by/kevin-draper}{\includegraphics{https://static01.graylady3jvrrxbe.onion/images/2018/07/18/multimedia/author-kevin-draper/author-kevin-draper-thumbLarge.png}}

March 18, 2020, 5:45 p.m. ET

March 18, 2020, 5:45 p.m. ET

By \href{https://www.nytimes3xbfgragh.onion/by/kevin-draper}{Kevin
Draper}

\hypertarget{sports-fans-can-rewatch-nba-and-nfl-games-for-free-for-a-limited-time}{%
\subsection{\texorpdfstring{\protect\hyperlink{sports-fans-can-rewatch-nba-and-nfl-games-for-free-for-a-limited-time}{Sports
fans can rewatch N.B.A. and N.F.L. games for free (for a limited
time).}}{Sports fans can rewatch N.B.A. and N.F.L. games for free (for a limited time).}}\label{sports-fans-can-rewatch-nba-and-nfl-games-for-free-for-a-limited-time}}

\includegraphics{https://static01.graylady3jvrrxbe.onion/images/2020/04/17/world/18virus-live-nbanfl/18virus-live-nbanfl-articleLarge.jpg?quality=75\&auto=webp\&disable=upscale}

With
\href{https://www.nytimes3xbfgragh.onion/2020/03/12/sports/coronavirus-sports-canceled.html}{sporting
events ground to a halt} across the globe, sports channels built around
live games are scrambling to
\href{https://www.nytimes3xbfgragh.onion/2020/03/16/sports/sports-television-coronavirus.html}{completely
reconfigure their programming lineups}. But some leagues have spied an
opportunity to reach the masses of sports fans socially distancing at
home.

Both the N.F.L. and N.B.A. announced Wednesday that they would make
their subscription out-of-market and archival game services available
for free for a limited time.

The N.F.L.'s Game Pass, which offers full replays of every N.F.L. game
going back to the 2009 season,
\href{http://www.nfl.com/news/story/0ap3000001106855/article/nfl-offers-fans-free-access-to-nfl-game-pass}{will
be free} through May 31 for those living in the United States and
Canada, and through July 31 for those living in other countries.

The N.B.A.'s League Pass, which has full replays of every N.B.A. game
going back to the 2018-19 season,
\href{https://www.nba.com/nba-fan-letter-league-pass-free-preview}{will
be free} through April 22.

\href{https://www.nytimes3xbfgragh.onion/by/miriam-jordan/}{\includegraphics{https://static01.graylady3jvrrxbe.onion/images/2018/02/16/multimedia/author-miriam-jordan/author-miriam-jordan-thumbLarge-v2.png}}

March 18, 2020, 5:30 p.m. ET

March 18, 2020, 5:30 p.m. ET

By \href{https://www.nytimes3xbfgragh.onion/by/miriam-jordan/}{Miriam
Jordan}

\hypertarget{some-immigrants-worry-seeking-medical-care-could-be-risky}{%
\subsection{\texorpdfstring{\protect\hyperlink{some-immigrants-worry-seeking-medical-care-could-be-risky}{Some
immigrants worry seeking medical care could be
risky.}}{Some immigrants worry seeking medical care could be risky.}}\label{some-immigrants-worry-seeking-medical-care-could-be-risky}}

\includegraphics{https://static01.graylady3jvrrxbe.onion/images/2020/03/17/us/00VIRUS-IMMIGRANT/merlin_169973235_8ee84757-0d7a-4b6a-b372-538baaf5a057-articleLarge.jpg?quality=75\&auto=webp\&disable=upscale}

LOS ANGELES --- As the coronavirus sweeps across the United States,
immigrants may be among the least able to self-isolate and seek the
medical care that is essential to protecting their health and slowing
the spread of the disease.

Some of those without health insurance fear that going to a public
hospital or clinic will ruin their chances of getting a green card under
the Trump administration's tough new
\href{https://www.nytimes3xbfgragh.onion/2020/01/27/us/supreme-court-trump-green-cards.html}{public
assistance regulations} for immigrants. Other immigrants fear putting
themselves in the cross-hairs of Immigration and Customs Enforcement if
they step forward for help.

ICE agents over the past week have continued to make arrests in some of
the regions hardest hit by the virus, including California and New York.

The coronavirus was not on the agenda when a legal-aid group two months
ago invited farmworkers who toil in the date groves, lemon orchards and
vineyards of California's Coachella Valley to an information session
about immigration issues.

But when Luz Gallegos and her team showed up over the weekend, they were
cornered by people who peppered them with questions about the virus. On
Monday, public health authorities announced the first two deaths from
the virus in this part of Southern
\href{https://www.nytimes3xbfgragh.onion/article/california-coronavirus.html}{California},
both in the Coachella Valley.

Among the questions the farmworkers had: If I go to the hospital, is it
going to hurt my chances of becoming a legal permanent resident? If I'm
undocumented, could seeking treatment make me vulnerable to deportation?
If I miss work as more people are forced to stay home, how will I feed
my family and make the rent?

``There's a new layer of fear in the immigrant community right now
created by Covid-19,'' said Ms. Gallegos, a director of TODEC Legal
Center, who stood with immigrants in the parking lot of the Hemet town
library, which had abruptly closed as a result of the pandemic. ``We
believe that some members will be afraid to seek the care they need,''
she said.

The Trump administration on Wednesday closed the border with Canada to
all but essential traffic and was also considering
\href{https://www.nytimes3xbfgragh.onion/2020/03/17/us/politics/trump-coronavirus-mexican-border.html?searchResultPosition=1}{shutting
the southern border} to those without legal authorization, hoping to
check the spread of the virus. But many of the unauthorized immigrants
already in the United States face the same threat from the virus as
everyone else --- and are less equipped to protect themselves.

Among all immigrants, 23 percent of those who are lawfully in the
country and 45 percent of those who are undocumented lack health
insurance, according to a
\href{https://www.kff.org/disparities-policy/fact-sheet/health-coverage-of-immigrants/}{report
by the Kaiser Family Foundation.}

Caitlin Dickerson and Annie Correal contributed reporting from New York.

Read more

\hypertarget{advertisement-3}{%
\subsubsection{Advertisement}\label{advertisement-3}}

\protect\hyperlink{after-dfp-ad-mid4}{Continue reading the main story}

\href{https://www.nytimes3xbfgragh.onion/by/penelope-green}{\includegraphics{https://static01.graylady3jvrrxbe.onion/images/2018/07/18/multimedia/author-penelope-green/author-penelope-green-thumbLarge-v3.png}}

March 18, 2020, 5:15 p.m. ET

March 18, 2020, 5:15 p.m. ET

By \href{https://www.nytimes3xbfgragh.onion/by/penelope-green}{Penelope
Green}

\hypertarget{join-us-in-queerantine-a-dating-app-builds-a-community}{%
\subsection{\texorpdfstring{\protect\hyperlink{join-us-in-queerantine-a-dating-app-builds-a-community}{`Join
us in queerantine': a dating app builds a
community.}}{`Join us in queerantine': a dating app builds a community.}}\label{join-us-in-queerantine-a-dating-app-builds-a-community}}

``WHAT WILL WE DO ALL DAY?! Have you put together activity lists yet?
How do you force yourself to put on real pants? And why should we?
Inside exercise? Virtual happy hours? LOTS to discuss. Femmes to the
front.''

Such is the conversation this week on Lex, the old-fashioned-style
dating app for lesbian, bisexual, asexual and queer people. Like the
personal ads of yore, Lex is text-only, which means that since its
beginnings
\href{https://www.nytimes3xbfgragh.onion/2018/08/04/style/lesbian-queer-dating-app.html}{as
an Instagram account called Personals}, it has bubbled with epistolary
wit --- ``Anyone want to add falling in love via the written word to
your growing list of new quarantine hobbies?'' --- and also community
spirit.

Both were on vibrant display in the last few days, as many users reached
out for conversation and fellowship while practicing
\href{https://www.nytimes3xbfgragh.onion/2020/03/16/smarter-living/coronavirus-social-distancing.html}{social
distancing}. One in Detroit offered free tarot readings to gig workers
who had lost their jobs. Another was making art prompts and offered to
share them. A public-school teacher offered money for anyone short of
funds who needed food or medicine, and a car to deliver.

When Wren Harrington, a 21-year-old student in Connecticut, posted
optimistically about how social isolation need not ``consume us,'' she
got a handful of replies, including one from a young woman with whom she
shared a sun and rising sign.

``We had a lovely conversation about that,'' Ms. Harrington said. ``I
came back to Lex to see what was happening in the wake of the chaos. I
was really surprised to see how people are using it for things other
than its intended purpose, and in a really wonderful way.''

In Massachusetts, Lex Blair, a 27-year-old Starbucks barista and actor
who uses the pronoun ``they,'' posted that their show had been canceled
--- they are in two theater companies --- and wondered what other
performers were experiencing. Mx. Blair and their girlfriend, Kaiti
Maddox, 22, are in a nonmonogamous relationship, they said, hence the
Lex app, but this week were looking for a different sort of connection.

Ms. Maddox has a connective tissue disorder, and therefore a compromised
immune system. It is imperative that she stay isolated. She has been
using Lex as a chat room, and the community there has sprung into
action, checking on her daily, sending her lists of favorite movies and
television shows and offering to bring food or other necessities.

``It's a rough time right now for all of us,'' said Mx. Blair.

Or, as another user put it in a post that invited members to a WhatsApp
chat: ``JOIN US IN QUEERANTINE. Self isolation need not be so lonely.''

Read more

\href{https://www.nytimes3xbfgragh.onion/by/tariro-mzezewa}{\includegraphics{https://static01.graylady3jvrrxbe.onion/images/2018/08/24/opinion/tariro-headshot/tariro-headshot-thumbLarge-v2.png}}

March 18, 2020, 5:00 p.m. ET

March 18, 2020, 5:00 p.m. ET

By \href{https://www.nytimes3xbfgragh.onion/by/tariro-mzezewa}{Tariro
Mzezewa}

\hypertarget{americans-stranded-abroad-feel-completely-abandoned}{%
\subsection{\texorpdfstring{\protect\hyperlink{americans-stranded-abroad-feel-completely-abandoned}{Americans
stranded abroad feel `completely
abandoned.'}}{Americans stranded abroad feel `completely abandoned.'}}\label{americans-stranded-abroad-feel-completely-abandoned}}

\includegraphics{https://static01.graylady3jvrrxbe.onion/images/2020/04/17/travel/18virus-stranded-promo/18virus-stranded-promo-articleLarge-v2.jpg?quality=75\&auto=webp\&disable=upscale}

Lauren Davenport and Daniel Fernandez of St. Petersburg, Fla., were on a
camping trip in the Sahara when Morocco announced that it was suspending
all flights in and out of the country ``until further notice.'' They
were stunned to return to Marrakesh on Tuesday and realize they couldn't
get home.

Among the stranded in Morocco are American citizens, residents and other
visa holders who say the United States government has been unresponsive
to their pleas for help, even as British and French authorities have
been aiding their citizens.

``France is being very open with the citizens and is moving mountains to
get them home; meanwhile the U.S. embassy says `call the airlines' and
`prepare to be here for a while, but not indefinitely,''' said Cristina
Pratt, who was visiting Morocco from the East Bay in California with a
friend and the friend's parents when the ban went into place.

\includegraphics{https://static01.graylady3jvrrxbe.onion/images/icons/t_logo_291_black.png}

March 18, 2020, 4:15 p.m. ET

March 18, 2020, 4:15 p.m. ET

By \href{https://www.nytimes3xbfgragh.onion/by/davey-alba}{Davey Alba}

\hypertarget{in-a-happier-postponement-google-presses-pause-on-employee-reviews}{%
\subsection{\texorpdfstring{\protect\hyperlink{in-a-happier-postponement-google-presses-pause-on-employee-reviews}{In
a happier postponement, Google presses pause on employee
reviews.}}{In a happier postponement, Google presses pause on employee reviews.}}\label{in-a-happier-postponement-google-presses-pause-on-employee-reviews}}

\includegraphics{https://static01.graylady3jvrrxbe.onion/images/2020/03/18/world/18virus-lives-google/merlin_165432768_6789db43-4027-4d11-8629-f42a69682aba-articleLarge.jpg?quality=75\&auto=webp\&disable=upscale}

The global coronavirus pandemic has caused one unpleasant work-related
process to be delayed at Google: performance reviews.

On Wednesday, a Google spokeswoman said the company was deferring
performance reviews for the current period. The reason, according to a
copy of the email announcement sent by Eileen Naughton, Google's vice
president of people operations, and which was seen by The New York
Times, was to help the internet company's more than 100,000 workers
focus on their ``most important, mission-critical activities'' during
the coronavirus pandemic.

Google will defer its performance reviews covering the period from
mid-October to the end of this month, the spokeswoman said. In November,
employees are expected to receive their regular annual rating covering
the last 12 months of work. And all employees promoted at that point
would receive post-promotion salaries back-dated to August, the
spokeswoman said.

Business Insider
\href{https://www.businessinsider.com/google-delays-employee-performance-reviews-promotions-covid-19-2020-3}{earlier
reported} the change.

\hypertarget{advertisement-4}{%
\subsubsection{Advertisement}\label{advertisement-4}}

\protect\hyperlink{after-dfp-ad-mid5}{Continue reading the main story}

\href{https://www.nytimes3xbfgragh.onion/by/julia-moskin}{\includegraphics{https://static01.graylady3jvrrxbe.onion/images/2018/09/25/multimedia/author-julia-moskin/author-julia-moskin-thumbLarge.png}}

March 18, 2020, 4:00 p.m. ET

March 18, 2020, 4:00 p.m. ET

By \href{https://www.nytimes3xbfgragh.onion/by/julia-moskin}{Julia
Moskin}

\hypertarget{large-restaurant-companies-announce-layoffs}{%
\subsection{\texorpdfstring{\protect\hyperlink{large-restaurant-companies-announce-layoffs}{Large
restaurant companies announce
layoffs.}}{Large restaurant companies announce layoffs.}}\label{large-restaurant-companies-announce-layoffs}}

\includegraphics{https://static01.graylady3jvrrxbe.onion/images/2020/03/18/multimedia/18-live-restaurantlayoffs1/18-live-restaurantlayoffs1-articleLarge.jpg?quality=75\&auto=webp\&disable=upscale}

Mass layoffs at some restaurant companies around the country have begun.

The \href{https://www.ushgnyc.com/}{Union Square Hospitality Group}, one
of the nation's most prestigious restaurant companies, laid off 2,000
employees on Wednesday morning ``due to a near-complete elimination of
revenue,'' the company said in a statement. That number represents 80
percent of the company's total staff, at 18 restaurants in New York
City, two in Washington and its corporate office in Manhattan.

The chef and restaurateur Tom Colicchio also announced 300 layoffs at
his \href{https://www.craftedhospitality.com/}{Crafted Hospitality}
restaurant group in New York and Los Angeles.

Other big employers, like the \href{https://www.majorfood.com/}{Major
Food Group} and the \href{https://www.jean-georges.com/}{Jean-Georges
Vongerichten} restaurants, began layoffs last week, when employers were
making the difficult calculation of whether to close altogether out of
safety concerns, or remain open for takeout and delivery in order to
maintain some cash flow.

Union Square employees were told in phone calls from their managers that
lost revenue since the coronavirus outbreak had rapidly made the
business financially unsustainable, and that the layoffs were intended
to allow them to apply for unemployment. The company, which
\href{https://www.nytimes3xbfgragh.onion/2020/03/13/dining/restaurant-closings-coronavirus.html}{closed
its New York restaurants} on Friday, promised to rehire as many people
as possible when the crisis has passed.

``In the absence of income, restaurants simply cannot pay our
non-working team members for more than a short period of time without
becoming insolvent. In that scenario, no one wins,'' Danny Meyer, the
chief executive of the Union Square Hospitality Group, said in a
statement.

Mr. Meyer said he would donate his own salary, and ``substantial'' pay
cuts for the group's executives, to a relief fund for employees until
further notice. The group also has a \$220 million investment fund,
Enlightened Hospitality Investments, that has backed related businesses
such as \href{https://saltandstraw.com/}{Salt \& Straw} ice cream,
\href{https://joecoffeecompany.com/}{Joe Coffee} and the restaurant
chain \href{https://www.diginn.com/}{Dig}.

Laid-off restaurant workers in New York have already reported widespread
difficulty in accessing the state's unemployment registration system.
And industry trade groups say that those benefits, as currently
structured, will be inadequate.

The National Restaurant Association said that U.S. restaurants could
suffer a \$225 billion loss in the next three months because of the
coronavirus pandemic. The group is asking the federal government for a
relief package including \$145 billion recovery fund, block grants,
small-business loans and tax help.

\emph{Follow} \href{https://twitter.com/nytfood}{\emph{NYT Food on
Twitter}} \emph{and}
\href{https://www.instagram.com/nytcooking/}{\emph{NYT Cooking on
Instagram}}\emph{,}
\href{https://www.facebookcorewwwi.onion/nytcooking/}{\emph{Facebook}}\emph{,}
\href{https://www.youtube.com/nytcooking}{\emph{YouTube}} \emph{and}
\href{https://www.pinterest.com/nytcooking/}{\emph{Pinterest}}\emph{.}
\href{https://www.nytimes3xbfgragh.onion/newsletters/cooking}{\emph{Get
regular updates from NYT Cooking, with recipe suggestions, cooking tips
and shopping advice}}\emph{.}

Read more

\href{https://www.nytimes3xbfgragh.onion/by/john-eligon}{\includegraphics{https://static01.graylady3jvrrxbe.onion/images/2018/06/12/multimedia/author-john-eligon/author-john-eligon-thumbLarge.png}}\href{https://www.nytimes3xbfgragh.onion/by/nellie-bowles}{\includegraphics{https://static01.graylady3jvrrxbe.onion/images/icons/t_logo_291_black.png}}

March 18, 2020, 3:30 p.m. ET

March 18, 2020, 3:30 p.m. ET

By \href{https://www.nytimes3xbfgragh.onion/by/john-eligon}{John Eligon}
and \href{https://www.nytimes3xbfgragh.onion/by/nellie-bowles}{Nellie
Bowles}

\hypertarget{janitors-are-in-higher-demand-and-at-higher-risk}{%
\subsection{\texorpdfstring{\protect\hyperlink{janitors-are-in-higher-demand-and-at-higher-risk}{Janitors
are in higher demand, and at higher
risk.}}{Janitors are in higher demand, and at higher risk.}}\label{janitors-are-in-higher-demand-and-at-higher-risk}}

\includegraphics{https://static01.graylady3jvrrxbe.onion/images/2020/03/16/us/00VIRUS-JANITORS-santamaria/merlin_170414925_0b2ab51a-1830-4c8f-bea6-710b5a51ddfe-articleLarge.jpg?quality=75\&auto=webp\&disable=upscale}

SAN FRANCISCO --- The rumor unsettled Deborah Santamaria.

A fellow janitor at 555 California Street, a 52-story office tower in
San Francisco's financial district, told her he heard that a floor of
the building was being closed because a worker had contracted the novel
coronavirus.

Her supervisor at Able Services, the contractor that employs her,
reassured her that nothing was wrong, she said.

It was not until five days later that
\href{https://www.bloomberg.com/news/articles/2020-03-09/wells-fargo-worker-in-california-tests-positive-for-coronavirus}{a
news article} appeared saying that Wells Fargo had temporarily evacuated
its offices in the building after an employee had tested positive for
the coronavirus.

The bank had notified building management, which alerted the cleaning
contractor. But according to the employees and their union
representatives, no one had told the janitors.

``I felt as if I didn't matter,'' said Ms. Santamaria, who at age 63
counts herself among those most vulnerable to the virus.

While many Americans are fleeing their offices to avoid any contact with
the coronavirus, low-wage janitors are sometimes being asked to do the
opposite. Although millions of Californians have been ordered to shelter
in place, janitors are still being asked to go into offices to battle
the invisible germs that threaten public health, even as those germs,
and the new, powerful cleaning solutions they are being asked to use,
may endanger their own health.

They often operate without specialized protective gear. And the
increasing demand for their services is adding new stress and risks.

Janitors cleaning the Amazon headquarters in Seattle complained that a
new disinfectant they were asked to use made their eyes and skin burn.
In San Francisco, janitors said they have been asked to clean offices
without having been told that people who had or were exposed to the
virus had worked there.

Janitors wonder why they are left in the dark when companies go to great
lengths to ensure that the tech, finance and other workers occupying the
buildings they clean are aware of the most remote possibility of coming
into contact with the virus. It shows, they say, how disparities play
out in a public health crisis --- how their lives sometimes seem to be
valued less than those of people with resources and power.

Read more

\href{https://www.nytimes3xbfgragh.onion/by/jon-pareles}{\includegraphics{https://static01.graylady3jvrrxbe.onion/images/2018/06/14/multimedia/author-jon-pareles/author-jon-pareles-thumbLarge.png}}

March 18, 2020, 3:00 p.m. ET

March 18, 2020, 3:00 p.m. ET

By \href{https://www.nytimes3xbfgragh.onion/by/jon-pareles}{Jon Pareles}

\hypertarget{a-music-critic-reflects-on-empty-nights-with-no-live-shows}{%
\subsection{\texorpdfstring{\protect\hyperlink{a-music-critic-reflects-on-empty-nights-with-no-live-shows}{A
music critic reflects on empty nights with no live
shows.}}{A music critic reflects on empty nights with no live shows.}}\label{a-music-critic-reflects-on-empty-nights-with-no-live-shows}}

\includegraphics{https://static01.graylady3jvrrxbe.onion/images/2020/03/19/arts/18virus-essay-apollo/merlin_170607033_5efffd5f-2798-46cd-9098-eb3525d29b19-articleLarge.jpg?quality=75\&auto=webp\&disable=upscale}

Empty rooms. Empty stages. Empty seats. Empty dance floors. And not far
away, empty lobbies, empty dressing rooms, empty back rooms, empty bar
areas, empty kitchens, empty lounges, empty sound booths, empty loading
zones. These were places animated by live music, with entire backstage
workdays dedicated to presenting just a few hours of intangible sound
--- gigs, shows, concerts --- for the audiences that gathered there,
often with great anticipation and at significant cost. I already miss
them dearly.

The shutdown of live music by the Covid-19 pandemic has redoubled my
appreciation for the mysterious alchemy that happens at so many
concerts. It's really, when you step back, a very peculiar human
activity. Strangers --- with perhaps a few familiar faces dotted among
them --- gather at an appointed time to watch and hear musicians doing
their job. But it's more than a job; it's a ritual, a confluence of
visible and invisible forces, acoustic and social and psychological.
Heads may bob, toes may tap, singalongs may be joined, dancing may break
out. A roomful of individual reactions somehow adds up to a collective
one, which might crest into a spontaneous ovation or a mesmerized gasp.
Even with a seated, decorous audience, music can summon a certain
quality of heightened collective awareness that I've only experienced at
a live performance.

Making and hearing music in public is inherently social. Each concert is
the intersection of a career arc with a single night out for the
audience members, and the immediate pleasure (or impatience) of each
night's crowd adds up to lasting lessons for musicians.

\hypertarget{advertisement-5}{%
\subsubsection{Advertisement}\label{advertisement-5}}

\protect\hyperlink{after-dfp-ad-mid6}{Continue reading the main story}

\href{https://www.nytimes3xbfgragh.onion/by/jill-cowan}{\includegraphics{https://static01.graylady3jvrrxbe.onion/images/2018/12/10/multimedia/author-jill-cowan/author-jill-cowan-thumbLarge.png}}

March 18, 2020, 2:30 p.m. ET

March 18, 2020, 2:30 p.m. ET

By \href{https://www.nytimes3xbfgragh.onion/by/jill-cowan}{Jill Cowan}

\hypertarget{on-your-next-grocery-run-dont-forget-to-sanitize-your-reusable-bags}{%
\subsection{\texorpdfstring{\protect\hyperlink{on-your-next-grocery-run-dont-forget-to-sanitize-your-reusable-bags}{On
your next grocery run, don't forget to sanitize your reusable
bags.}}{On your next grocery run, don't forget to sanitize your reusable bags.}}\label{on-your-next-grocery-run-dont-forget-to-sanitize-your-reusable-bags}}

\includegraphics{https://static01.graylady3jvrrxbe.onion/images/2020/03/20/us/18grocerycatoday/merlin_170515038_30efd4d5-083d-4fc9-b855-8fa24e573265-articleLarge.jpg?quality=75\&auto=webp\&disable=upscale}

I talked with Dr. John Swartzberg, a professor of infectious diseases at
the University of California, Berkeley, School of Public Health, and
Jeff Nelken, a food safety expert based in Woodland Hills, Calif., about
what to know to make your trip to the supermarket as safe as possible:

\textbf{If you have to go to the store, touch as little as possible and
sanitize all the stuff you buy when you get home.}

Mr. Nelken noted that most produce is placed on displays by hand ---
sometimes they're gloved. But it's safe to assume that if a piece of
produce has been out, it's been handled by at least 10 people.

He recommended misting produce with a very diluted bleach solution (a
teaspoon of bleach per gallon of water) and letting it air dry. Or, if
you're nervous about using bleach on something you'll eat, he said, you
could use a disinfecting wipe.

Soap and water is another alternative, Dr. Swartzberg said, adding that
all these approaches are based only on what's known about Covid-19 and
other viruses.

Other goods are less risky, but it's still worth wiping down cardboard
boxes of crackers or other packaged items.

Mr. Nelken said one thing to watch out for is your reusable grocery
bags: ``When was the last time you sanitized your favorite bag?''

\textbf{Also, wash your hands and maybe wash your hair.}

Dr. Swartzberg underlined the advice you've been hearing from health
officials: Hand soap really will protect you.

``One thing about this coronavirus,'' he said, is that ``it's very
susceptible to disinfectants and soap and water.''

That said, if you go to a store, touch a contaminated surface, then
touch your hair and then your face, even once between hand-washing, you
could become infected.

So, he said, if you're really concerned, it might be worth throwing your
clothes in the wash and showering after you get back from the store.
(Your shoes, he said, are probably fine to bring into the house.)

\textbf{Ultimately, though, the best thing to do is stay away from other
people, particularly if you're part of a vulnerable population.}

Although state health officials
\href{https://www.cdph.ca.gov/Programs/CID/DCDC/Pages/Guidance.aspx}{released
guidelines} for grocery stores saying they should limit how many
customers are in a given store and should keep people at least six feet
apart even when they're in line, this is proving to be easier said than
done.

\href{https://www.latimes.com/california/story/2020-03-17/trader-joes-senior-shopping}{The
Los Angeles Times reported} that seniors hoping to get into a Monrovia,
Calif., Trader Joe's during a special early-shopping period were greeted
with confusion.

Both Dr. Swartzberg and Mr. Nelken said grocers would need to come up
with more sustainable ways to update customers in real time.

``I think the first thing we need to avoid is just showing up,'' Mr.
Nelken told me.

For now, he suggested calling the store you'd like to visit before you
leave, both to inquire about the wait and what's in stock.

Many grocery-store chains have said that they're hiring up and cleaning
more frequently, and that they still have curbside pickup and delivery
available,
\href{https://www.sandiegouniontribune.com/business/retail/story/2020-03-16/online-grocery-delivery-grinds-to-a-halt-due-to-high-demand-in-san-diego-county}{even
if their systems are moving more slowly}.

Read more

\includegraphics{https://static01.graylady3jvrrxbe.onion/images/icons/t_logo_291_black.png}

March 18, 2020, 2:00 p.m. ET

March 18, 2020, 2:00 p.m. ET

By Jessica Testa,
\href{https://www.nytimes3xbfgragh.onion/by/sapna-maheshwari}{Sapna
Maheshwari} and
\href{https://www.nytimes3xbfgragh.onion/by/vanessa-friedman}{Vanessa
Friedman}

\hypertarget{clothing-and-beauty-retailers-are-closing-up-shop}{%
\subsection{\texorpdfstring{\protect\hyperlink{clothing-and-beauty-retailers-are-closing-up-shop}{Clothing
and beauty retailers are closing up
shop.}}{Clothing and beauty retailers are closing up shop.}}\label{clothing-and-beauty-retailers-are-closing-up-shop}}

\includegraphics{https://static01.graylady3jvrrxbe.onion/images/2020/03/17/fashion/17VIRUS-STORES-patagonia/merlin_136361172_b0089985-befe-4a7a-8978-82f2318720f0-articleLarge.jpg?quality=75\&auto=webp\&disable=upscale}

When President Trump declared a state of emergency
\href{https://www.nytimes3xbfgragh.onion/2020/03/17/world/coronavirus-news.html}{over
the novel coronavirus} on Friday, many clothing and beauty retailers
began
\href{https://www.nytimes3xbfgragh.onion/2020/03/18/style/coronavirus-clothing-beauty-stores-closed.html}{announcing
temporary closures}. The questions came just as fast: How long would
this last? Would brick-and-mortar workers still get paid? Would they be
able to keep their jobs?

As panic over the pandemic spreads globally, more businesses are facing
difficult choices: Close up shop entirely, adhering strictly to
social-distancing recommendations; stay open; or something in between.

For the most part, the industry's strongest retailers --- the companies
most able to withstand the costs of a temporary shutdown --- announced
their plans quickly. (Leading the charge outside the fashion industry
was
\href{https://www.nytimes3xbfgragh.onion/2020/03/14/technology/apple-stores-coronavirus.html}{Apple},
which was one of the first and most influential companies to announce
closures; most of its stores outside mainland China, Hong Kong and
Taiwan are closed until March 27.)

Patagonia **** was viewed by many as one of the first retailers to act
decisively when, on Friday, it said that it would close its 39 stores in
North America and give employees their regular pay.

Sephora said on Tuesday that it would close its retail stores in the
United States and Canada from March 17 through April 3 and pay employees
for their scheduled shifts. The company said it is waiving standard
shipping fees until its stores reopen.

Many retailers followed suit. As of Wednesday morning, only a few
national apparel retailers remained open: Kohl's and JCPenney, for
example.

In New York, closures include Bergdorf Goodman, Dover Street Market and
\href{https://www.nytimes3xbfgragh.onion/2017/10/03/fashion/190-bowery-jay-maisel-totokaelo-germania-bank.html}{Totokaelo}.
We will be
\href{https://www.buzzfeednews.com/article/stephaniemcneal/arielle-charnas-something-navy-has-coronavirus}{updating
our list of closures} as we learn more.

Read more

\includegraphics{https://static01.graylady3jvrrxbe.onion/images/icons/t_logo_291_black.png}

March 18, 2020, 1:30 p.m. ET

March 18, 2020, 1:30 p.m. ET

By Mark Miller

\hypertarget{most-social-security-services-will-be-handled-online-or-over-the-phone}{%
\subsection{\texorpdfstring{\protect\hyperlink{most-social-security-services-will-be-handled-online-or-over-the-phone}{Most
Social Security services will be handled online or over the
phone.}}{Most Social Security services will be handled online or over the phone.}}\label{most-social-security-services-will-be-handled-online-or-over-the-phone}}

\includegraphics{https://static01.graylady3jvrrxbe.onion/images/2020/03/17/business/17RETIRE-01/merlin_170580171_c8ffb508-07b5-4f24-94ac-f23a85a7a1e3-articleLarge.jpg?quality=75\&auto=webp\&disable=upscale}

The Social Security Administration
\href{https://www.ssa.gov/news/press/releases/2020/\#3-2020-2}{said
earlier this week} that its field office network would be closed to the
public in most situations starting Tuesday until further notice because
of the coronavirus pandemic. Offices that hear disability insurance
appeals also are closed.

Service will continue to be available via the agency's toll-free line,
(800) 772-1213, and its
\href{https://www.ssa.gov/onlineservices/}{website}. Payments to more
than 69 million Social Security beneficiaries are not affected.

Social Security announced late on Monday afternoon that field offices
would be closed, with most employees working from home. That means most
routine services --- helping with benefit claims, checking the status of
an application or appeal or requesting a replacement Social Security
card --- will be handled only through the agency's toll-free line and
website.

Field offices will only offer in-person assistance on a very short list
of crucial services. These include reinstatement of benefits in dire
circumstances; assistance to people with severe disabilities, blindness
or terminal illnesses; and people in dire need of eligibility decisions
for Supplemental Security Income or Medicaid eligibility related to work
status. Those seeking these services must call in advance.

\hypertarget{heres-how-to-get-social-security-help}{%
\subsubsection{\texorpdfstring{\textbf{Here's how to get Social Security
help:}}{Here's how to get Social Security help:}}\label{heres-how-to-get-social-security-help}}

\begin{itemize}
\item
  \textbf{If you need to visit} a local Social Security office for
  in-person services, call the office to request an appointment. You can
  find the closest office using an
  \href{https://secure.ssa.gov/ICON/main.jsp}{office locator tool} on
  the Social Security \href{https://www.ssa.gov/coronavirus/}{website},
  where the agency is also providing updates and information on
  services.
\item
  \textbf{If you already have an office appointment} or disability
  appeal scheduled, Social Security will contact you by phone to
  reschedule or to handle the matter by phone. The agency cautions that
  this call may come from a private phone number, not a government
  phone. That's because not all employees have government-issued phones
  that can be used for business from remote locations, said Richard
  Couture, a spokesman for the American Federation of Government
  Employees councils, which represent Social Security employees.
\item
  Social Security generally only contacts people who have recently
  applied for benefits, or to update the records of those who are
  receiving benefits, a spokeswoman says. The agency also calls people
  who have requested a callback, including those with scheduled
  appointments. The agency will never call to tell you that your Social
  Security number has been suspended or to demand payments or ask for
  credit card information.
\item
  The number of Social Security identity theft phone scams has been
  rising, with robocalls and live callers posing as government
  employees. The agency generally
  \href{https://faq.ssa.gov/en-US/Topic/article/KA-10018}{reaches out by
  mail} and will call only if you've just recently applied for benefits
  or have requested a callback.
\item
  \textbf{The agency is urging the public} to conduct business whenever
  possible online, and many people do just that: more than half of
  retirement and disability benefit claims are filed online. You will
  need to \href{https://www.ssa.gov/onlineservices/}{set up an account}
  on the site, which allows you to apply for benefits, check the status
  of applications and appeals, request replacement Social Security cards
  and download your current statement of benefits. The website also has
  a section with \href{https://faq.ssa.gov/en-US/}{frequently asked
  questions}.
\item
  \textbf{If you need to enroll in Medicare,} free counseling is
  available from the national network of State Health Insurance
  Assistance Programs, known as SHIP. Availability of services ``will
  vary by state and potentially even county,'' according to Alicia
  Jones, who runs SHIP in Nebraska and is chairwoman of a national
  committee of SHIP managers. She suggests calling your local SHIP
  office to check on availability. Find your
  \href{https://www.shiptacenter.org/}{state SHIP here}. The nonprofit
  Medicare Rights Center runs a free national Medicare help line, which
  can be reached at (800) 333-4114.
\end{itemize}

Read more

\hypertarget{our-2020-election-guide}{%
\section{Our 2020 Election Guide}\label{our-2020-election-guide}}

Updated Aug. 20, 2020

\begin{itemize}
\item
  \begin{center}\rule{0.5\linewidth}{\linethickness}\end{center}

  \hypertarget{convention-recap}{%
  \subsection{Convention Recap}\label{convention-recap}}

  \begin{itemize}
  \tightlist
  \item
    Joe Biden accepted the Democratic nomination, urging Americans to
    have faith that they could
    \href{https://www.nytimes3xbfgragh.onion/2020/08/20/us/politics/Joe-Biden-accepts-democratic-nomination.html?action=click\&pgtype=Article\&state=default\&region=BELOW_MAIN_CONTENT\&context=storylines_guide}{``overcome
    this season of darkness.''}
  \end{itemize}
\item
  \begin{center}\rule{0.5\linewidth}{\linethickness}\end{center}

  \hypertarget{news-analysis}{%
  \subsection{News Analysis}\label{news-analysis}}

  \begin{itemize}
  \tightlist
  \item
    Looming over Mr. Biden's nomination was the ever-present shadow of
    another man who's poised to dominate the campaign:
    \href{https://www.nytimes3xbfgragh.onion/2020/08/20/us/politics/biden-dnc-speech-trump.html?action=click\&pgtype=Article\&state=default\&region=BELOW_MAIN_CONTENT\&context=storylines_guide}{Donald
    J. Trump}.
  \end{itemize}
\item
  \begin{center}\rule{0.5\linewidth}{\linethickness}\end{center}

  \hypertarget{keep-up-with-our-coverage}{%
  \subsection{Keep Up With Our
  Coverage}\label{keep-up-with-our-coverage}}

  \begin{itemize}
  \tightlist
  \item
    Get an
    \href{https://www.nytimes3xbfgragh.onion/newsletters/politics?action=click\&pgtype=Article\&state=default\&region=BELOW_MAIN_CONTENT\&context=storylines_guide}{email}
    recapping the day's news
  \end{itemize}

  \begin{itemize}
  \tightlist
  \item
    Download our mobile app on
    \href{https://apps.apple.com/us/app/nytimes/id284862083?ls=1\&mat_click_id=5c79ae7455014fd1bd66b5610c05b8f2-20191112-16948\&referrer=mat_click_id\%3D5c79ae7455014fd1bd66b5610c05b8f2-20191112-16948\%26link_click_id\%3D722930677036718082}{iOS}
    and
    \href{http://a.localytics.com/android?id=com.nytimes.android\&referrer=utm_source\%3Dother_nyt_mobile_web\%26utm_medium\%3DWeb\%2520page\%26utm_term\%3DGeneral\%2520Mobile\%2520Page\%26utm_campaign\%3DNYT\%2520Mobile\%2520General\%2520Page}{Android}
    and turn on Breaking News and Politics alerts
  \end{itemize}
\end{itemize}

\hypertarget{site-index}{%
\subsection{Site Index}\label{site-index}}

\hypertarget{site-information-navigation}{%
\subsection{Site Information
Navigation}\label{site-information-navigation}}

\begin{itemize}
\tightlist
\item
  \href{https://help.nytimes3xbfgragh.onion/hc/en-us/articles/115014792127-Copyright-notice}{©~2020~The
  New York Times Company}
\end{itemize}

\begin{itemize}
\tightlist
\item
  \href{https://www.nytco.com/}{NYTCo}
\item
  \href{https://help.nytimes3xbfgragh.onion/hc/en-us/articles/115015385887-Contact-Us}{Contact
  Us}
\item
  \href{https://www.nytco.com/careers/}{Work with us}
\item
  \href{https://nytmediakit.com/}{Advertise}
\item
  \href{http://www.tbrandstudio.com/}{T Brand Studio}
\item
  \href{https://www.nytimes3xbfgragh.onion/privacy/cookie-policy\#how-do-i-manage-trackers}{Your
  Ad Choices}
\item
  \href{https://www.nytimes3xbfgragh.onion/privacy}{Privacy}
\item
  \href{https://help.nytimes3xbfgragh.onion/hc/en-us/articles/115014893428-Terms-of-service}{Terms
  of Service}
\item
  \href{https://help.nytimes3xbfgragh.onion/hc/en-us/articles/115014893968-Terms-of-sale}{Terms
  of Sale}
\item
  \href{https://spiderbites.nytimes3xbfgragh.onion}{Site Map}
\item
  \href{https://help.nytimes3xbfgragh.onion/hc/en-us}{Help}
\item
  \href{https://www.nytimes3xbfgragh.onion/subscription?campaignId=37WXW}{Subscriptions}
\end{itemize}
