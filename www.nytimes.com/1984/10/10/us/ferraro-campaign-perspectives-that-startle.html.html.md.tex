Sections

SEARCH

\protect\hyperlink{site-content}{Skip to
content}\protect\hyperlink{site-index}{Skip to site index}

\href{https://www.nytimes.com/section/us}{U.S.}

\href{https://myaccount.nytimes.com/auth/login?response_type=cookie\&client_id=vi}{}

\href{https://www.nytimes.com/section/todayspaper}{Today's Paper}

\href{/section/us}{U.S.}\textbar{}FERRARO CAMPAIGN: PERSPECTIVES THAT
STARTLE

\href{https://nyti.ms/29DYqSA}{https://nyti.ms/29DYqSA}

\begin{itemize}
\item
\item
\item
\item
\item
\end{itemize}

Advertisement

\protect\hyperlink{after-top}{Continue reading the main story}

Supported by

\protect\hyperlink{after-sponsor}{Continue reading the main story}

\hypertarget{ferraro-campaign-perspectives-that-startle}{%
\section{FERRARO CAMPAIGN: PERSPECTIVES THAT
STARTLE}\label{ferraro-campaign-perspectives-that-startle}}

By \href{https://www.nytimes.com/by/maureen-dowd}{Maureen Dowd}

\begin{itemize}
\item
  Oct. 10, 1984
\item
  \begin{itemize}
  \item
  \item
  \item
  \item
  \item
  \end{itemize}
\end{itemize}

\includegraphics{https://s1.nyt.com/timesmachine/pages/1/1984/10/10/259967_360W.png?quality=75\&auto=webp\&disable=upscale}

See the article in its original context from\\
October 10, 1984, Section A, Page
1\href{https://store.nytimes.com/collections/new-york-times-page-reprints?utm_source=nytimes\&utm_medium=article-page\&utm_campaign=reprints}{Buy
Reprints}

\href{http://timesmachine.nytimes.com/timesmachine/1984/10/10/259967.html}{View
on timesmachine}

TimesMachine is an exclusive benefit for home delivery and digital
subscribers.

About the Archive

This is a digitized version of an article from The Times's print
archive, before the start of online publication in 1996. To preserve
these articles as they originally appeared, The Times does not alter,
edit or update them.

Occasionally the digitization process introduces transcription errors or
other problems; we are continuing to work to improve these archived
versions.

Mostly, the first woman to run for the second spot for a major political
party lets the awkward moments slide. But when a Democratic official
presented her with a wrist corsage before a fund-raising meeting in New
York, Geraldine A. Ferraro drew the line.

''No, I will not put it on,'' she told the man gently. ''That I will not
do.''

Since Walter F. Mondale anointed her as his running mate, Mrs. Ferraro
has been remaking the role of Vice-Presidential candidate in the image
of a woman.

Although women have run before for the Senate and for governorships, and
encountered difficulties because of their sex, the length of this
campaign and the constant press attention make the Ferraro campaign, in
imagery and perspective, startingly different.

For the first time, a major candidate for national office hands her
pocketbook to an aide as she begins a news conference. And for the first
time the candidate talks about abortion with the phrase, ''If I were
pregnant,'' and about foreign policy with the phrase, ''As the mother of
a draft- age son.''

Even that requisite totem men face on campaign is taking on a new twist.
''People hand me their babies,'' Mrs. Ferraro said. ''As a mother, my
instinctive reaction is how do you give your baby to someone who's a
total stranger to kiss, especially with so many colds going around? And
especially when the woman is wearing lipstick? I mean, I find that
amazing that someone would do that.''

Though she offers the insights of a mother and feminist, Mrs. Ferraro
is, above all, a pragmatist. She kisses the babies.

She bounces to the music, even if the introductory tune is ''Five foot
two, eyes of blue,'' and she is gracious when introduced, as she often
is, as ''most of all, a mother who cares about her children'' or similar
phrases.

''I'm still a woman and there are very basic views of how you deal with
women,'' she said, speaking in the back seat of a staff car speeding to
a community meeting in Harrisburg, Pa. ''I think as long as someone is
not attempting to embarrass me, if I can sense, as I usually do, where
people are coming from, it doesn't cost me anything and you can kind of
make people feel good. I've got to talk to people about what I think are
the important issues.'' Hopes of the Democrats

In July, when she was selected, Democrats hoped she would ignite the
campaign, attracting more votes from women than she might lose from men.
Even with the passage of time, it is hard to measure the impact
statistically, although her popularity has been rising lately. Polls
show that more Americans now think of her favorably than think of Mr.
Mondale favorably, for example, but not so many as think favorably of
Vice President Bush.

Her appeal differs, depending on the voter, however. The Sept. 30-Oct. 4
New York Times/CBS News Poll found that 35 percent of the electorate
viewed her favorably and 30 percent did not. Among men, the division was
roughly even: 32 percent to 30 percent. Among women, the division was 37
percent to 29 percent.

When she was picked in San Francisco, the pressure and scrutiny were as
intense as the excitement. Everything Open to Debate

Everything became a matter of debate, from the trivial (should Mr.
Mondale kiss her cheek) to the serious (should her husband, John
Zaccaro, release his tax returns).

One thing the experts agreed on: She was going to have to walk a fine
line.

''The first woman Vice-Presidential candidate had to be tough,'' said
Robert Squier, a Democratic campaign consultant. ''She had to be able to
upstage Jeane Kirkpatrick.''

On the other hand, she could not be too tough. Gloria Steinem said,
''Nothing makes men more anxious than for a woman to be masculine.''

The debut of the Democratic ticket in the South was jarring. Jim Buck
Ross, Mississippi's 70-year-old Commissioner of Agriculture, called the
48-year-old Mrs. Ferraro ''young lady'' and asked her if she could bake
blueberry muffins. Strategists Were Fearful

Even though Mrs. Ferraro appeared to have acquitted herself well, ''the
bluberry muffin incident'' made the Mondale camp nervous. Were the
coming weeks going to be mined with other Jim Buck Rosses?

The strategists became more anxious when Mrs. Ferraro said she would not
release her husband's income tax returns, as she had promised, by
painting herself as a subservient wife: ''You women married to Italian
men know how it is.''

Once the outcry subsided, however, Mrs. Ferraro seemed to move more
comfortably into her role.

She has been careful never to let male politicians seem protective. At a
rally in Texas, when Gov. Mark White hovered beside Mrs. Ferraro and
twice tried to take the microphone when she was being heckled, she waved
him away. ''He could do it for a male candidate,'' she said later, ''but
not me.''

In foreign policy speeches, she blended a mother's concerns with the
hard questions of a politician, sometimes incorporating lines about her
children and other times pausing to talk about her family. 'I Didn't
Raise Him to Die'

Attacking the Reagan Administration's policies in Central America, she
said of her son, John: ''I didn't raise him to die in an undeclared war,
against an unnamed enemy, for an uncertain cause.''

In the beginning, perhaps wanting to establish other credentials and
perhaps because of aides' fears of overemphasizing the causes of
feminism, Mrs. Ferraro did not stress day care, comparable pay for jobs
of comparable worth, or the proposed Federal equal rights amendment.

Recently there was a shift. Women's meetings, which have attracted
overflow crowds, have been opened to the press, as one Ferraro aide
said, to ''tap that same electricity'' of San Francisco. The candidate
began talking more about the rights amendment and the House pension bill
she co-sponsored to help widows.

And for the first time she began to talk about the historic nature of
her candidacy. 'I'm Standing In for You'

With an unusual spurt of sisterhood, she told the women at a,fund-
raising event in Raleigh, N. C.: ''I guess I'm standing in for every one
of you.'' They responded with a standing ovation.

Mrs. Ferraro has not taken the conservative dress-for-success route. She
wears makeup and softer clothes, usually pearls and silk dresses.

''This should be a real lesson for women in corporations who have been
hounded to wear little ties and learn to play golf,'' Miss Steinem said.

Mrs. Ferraro does not shy away from feminine gestures. In Raleigh, she
told the crowd she had been asked if Southern men seemed to dislike a
candidate who is a woman. She answered no, adding an echo of Scarlett
O'Hara: ''Maybe they're such gentlemen they wouldn't tell me
otherwise.''

Aides who have worked for other candidates describe differences in this
campaign.

She goes at a slightly less frenetic pace, carving out time every
weekend to spend in Queens. Lightning Rod Effect Seen

If her sex has been a boon in some ways, it has been a bane in others.
The candidate's aides and feminist supporters believe that she has
become a lightning rod for what Miss Steinem calls ''free-floating
hostility to women in power that couldn't be overtly stated.''

They think this hostility has been reflected in the vociferousness of
the attack by opponents of abortion and in the zealous pursuit of her
husband's finances by the press.

''Whether or not I'm treated differently financially will be determined
seeing what happens now to Bush,'' she said, referring to the Vice
President's tax problems. ''Are the press going to let him go on this
one?'' Church Attitude Examined

As evidence of what they consider the Roman Catholic Church's attitude
toward a candidate who is a woman, her aides cite the tone of the press
conference of Bishop James Timlin of Scranton, Pa., last month. The
cleric repeatedly referred to the Republican as ''Mr. Bush'' and to the
Democrat as ''Geraldine.''

Although she has never publicly complained about religious reaction
against her, the idea clearly upsets her.

As her car pulled up to the Harrisburg event, Mrs. Ferraro said that Mr.
Sasso, who had worked on Senator Edward M. Kennedy's 1980 campaign, had
told her the bishops did not raise the abortion issue as a political
touchstone in that race.

''I don't know, would Senator Kennedy have been hit by the bishops as
hard as I have if he were running this time?'' she said. ''That's a
question.''

As she got out of the car, her voice lowered, and she said,''I kind of
doubt it.''

Advertisement

\protect\hyperlink{after-bottom}{Continue reading the main story}

\hypertarget{site-index}{%
\subsection{Site Index}\label{site-index}}

\hypertarget{site-information-navigation}{%
\subsection{Site Information
Navigation}\label{site-information-navigation}}

\begin{itemize}
\tightlist
\item
  \href{https://help.nytimes.com/hc/en-us/articles/115014792127-Copyright-notice}{©~2020~The
  New York Times Company}
\end{itemize}

\begin{itemize}
\tightlist
\item
  \href{https://www.nytco.com/}{NYTCo}
\item
  \href{https://help.nytimes.com/hc/en-us/articles/115015385887-Contact-Us}{Contact
  Us}
\item
  \href{https://www.nytco.com/careers/}{Work with us}
\item
  \href{https://nytmediakit.com/}{Advertise}
\item
  \href{http://www.tbrandstudio.com/}{T Brand Studio}
\item
  \href{https://www.nytimes.com/privacy/cookie-policy\#how-do-i-manage-trackers}{Your
  Ad Choices}
\item
  \href{https://www.nytimes.com/privacy}{Privacy}
\item
  \href{https://help.nytimes.com/hc/en-us/articles/115014893428-Terms-of-service}{Terms
  of Service}
\item
  \href{https://help.nytimes.com/hc/en-us/articles/115014893968-Terms-of-sale}{Terms
  of Sale}
\item
  \href{https://spiderbites.nytimes.com}{Site Map}
\item
  \href{https://help.nytimes.com/hc/en-us}{Help}
\item
  \href{https://www.nytimes.com/subscription?campaignId=37WXW}{Subscriptions}
\end{itemize}
