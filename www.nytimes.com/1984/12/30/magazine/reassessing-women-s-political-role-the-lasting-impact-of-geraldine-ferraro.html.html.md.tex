Sections

SEARCH

\protect\hyperlink{site-content}{Skip to
content}\protect\hyperlink{site-index}{Skip to site index}

\href{https://myaccount.nytimes.com/auth/login?response_type=cookie\&client_id=vi}{}

\href{https://www.nytimes.com/section/todayspaper}{Today's Paper}

REASSESSING WOMEN'S POLITICAL ROLE/THE LASTING IMPACT OF GERALDINE
FERRARO

\href{https://nyti.ms/29qfSuA}{https://nyti.ms/29qfSuA}

\begin{itemize}
\item
\item
\item
\item
\item
\end{itemize}

Advertisement

\protect\hyperlink{after-top}{Continue reading the main story}

Supported by

\protect\hyperlink{after-sponsor}{Continue reading the main story}

\hypertarget{reassessing-womens-political-rolethe-lasting-impact-of-geraldine-ferraro}{%
\section{REASSESSING WOMEN'S POLITICAL ROLE/THE LASTING IMPACT OF
GERALDINE
FERRARO}\label{reassessing-womens-political-rolethe-lasting-impact-of-geraldine-ferraro}}

By \href{https://www.nytimes.com/by/maureen-dowd}{Maureen Dowd}

\begin{itemize}
\item
  Dec. 30, 1984
\item
  \begin{itemize}
  \item
  \item
  \item
  \item
  \item
  \end{itemize}
\end{itemize}

\includegraphics{https://s1.nyt.com/timesmachine/pages/1/1984/12/30/177726_360W.png?quality=75\&auto=webp\&disable=upscale}

See the article in its original context from\\
December 30, 1984, Section 6, Page
18\href{https://store.nytimes.com/collections/new-york-times-page-reprints?utm_source=nytimes\&utm_medium=article-page\&utm_campaign=reprints}{Buy
Reprints}

\href{http://timesmachine.nytimes.com/timesmachine/1984/12/30/177726.html}{View
on timesmachine}

TimesMachine is an exclusive benefit for home delivery and digital
subscribers.

About the Archive

This is a digitized version of an article from The Times's print
archive, before the start of online publication in 1996. To preserve
these articles as they originally appeared, The Times does not alter,
edit or update them.

Occasionally the digitization process introduces transcription errors or
other problems; we are continuing to work to improve these archived
versions.

Maureen Dowd is a reporter for The New York Times. WOMEN LOOKED AT HER
and saw themselves, in a glass reflecting pride and prejudice, power and
fear of power. As it turned out, her candidacy forced American women to
grapple with their ambivalence about their own sex.

No one knew what Geraldine A. Ferraro was supposed to be. She did not
remind you of anyone who had aspired to the inner sanctum. So every time
she spoke, you had to make up your mind all over again whether she was
good enough, tough enough, smart enough, honest enough to be the
archetype. The profound question of whether she should be there at all
was expressed in more trivial concerns. Her staccato style and her
appearance became ways to gauge her worthiness. There was no reassuring
masculine voice. There was no navy blue suit so redolent of power it
seemed invisible. There were, instead, skirts flying in the wind,
pocketbooks clutched at news conferences. And you caught yourself
thinking incongruously: A Vice President with a purse? A Vice President
whose favorite expression is ''Gimme a break!''?

Coming at a time of rapidly shifting ideas about the role of women in
society and in the family, Geraldine Ferraro's candidacy evoked groping
and contradictory perceptions among women - tribal pride and darker
feelings of inferiority and jealousy.

The impact of the election will be felt for years. The way these women
thought about Representative Ferraro and the way they voted has, in
turn, forced feminists to reassess their role in politics and their
future relationship with ordinary American women. It is also recasting
the way the parties will deal with women - as voters, as candidates and
as a movement.

Last summer, scores of women across the country said they were excited
about the Ferraro nomination when they were called for a New York
Times/CBS News poll. In new interviews, they have charted the course of
their reactions to the historic race.

Theda Pate recalled the night she saw Geraldine Ferraro debate George
Bush on television. ''She hardly even blinked,'' Mrs. Pate marveled. ''I
thought, 'Wow, does she have some 200 I.Q. or something?' ''

Then, as the elementary-school teacher from Pineville, La., watched the
Representative from New York, she felt a spurt of envy tarring her
approval. ''She's not that much older than me,'' said Mrs. Pate, who is
42. ''I don't think she's brilliant. Well read, maybe. And I thought
maybe I could have been that if I had wanted to be. How do you know if
you could have been, if you limited yourself in the very beginning? And
she's attractive, no Margaret Mead or Eleanor Roosevelt. So I didn't
have to think if I looked like them, I'd have to develop my mind, too.

''There's a type of subconscious envy, or maybe mistrust, of a woman who
has succeeded where many others have not,'' she continued. ''So instead
of aggressively working to help her reach a position of prominence, we
begin at an elementary level to attack. It's a basic flaw in women's
behavior, which is to be bitchy.''

In Bristol, Tenn., Carol Roberts said that she and her friends decided
they were ''maybe just a little afraid to leave ourselves in her hands
if something was to happen.''

''I put myself in her shoes,'' said the 36-year-old homemaker and mother
of three. ''Could I sit down and logically make decisions for everybody
without cracking up? I think women in general are weak. I know that
sounds awful. But we women know we have our faults. We look at ourselves
and think 'I couldn't handle it so I don't know if she could, either.'

''Maybe that's the wrong thing to do,'' she said, in her soft Southern
voice. ''Men don't do that.''

Mrs. Roberts and her husband, a postal clerk, voted for Ronald Reagan.
But she says that Mrs. Ferraro has made her feel different about women
running for high office and about herself.

''She built us up a little. She said, 'Here's what you can do if you
really want to do it. You don't have to sit at home and be a good
housewife and mother - all that used to be required to make your niche.'
She really kept herself together. She never broke down once. I'm proud
she had the nerve to do what she did.'' TRACING THE ARC OF A CAMPAIGN
BRIGHT with history and dark with disclosure, sorting through all the
euphoria and champagne and tears, the analysts attempted after the
election to calculate Geraldine Ferraro's worth to the Democratic
ticket.

They drained all the history and emotion into dry statistics and, at The
New York Times, reckoned that the first woman to run for the Vice
Presidency on a major party ticket represented a net gain of
eight-tenths of 1 percentage point. The fraction mocked the memory of
the Democratic convention in San Francisco last July, when her selection
seemed suddenly to transform the image of the party from one of stale
egalitarianism into one of possibility.

As I go over my notes from the Democratic convention, it is possible to
re-create the aura that led so many Democrats to envision a shining year
for women in politics, a year when Geraldine Ferraro would unleash
''small-f'' feminism - that flickering and apolitical sense of injustice
they felt was common to all American women. The mood was so electric
that just being female felt terrific. ''One man I know told me I even
looked stronger,'' said Jane O'Reilly, a New York writer floating
through the crowd. Democratic women challenging incumbents for
Congressional seats were suddenly the stars of fund-raising rallies, and
they bubbled with talk of how the Ferraro choice would bring money,
volunteers and credibility to their campaigns.

None of these particular women, whose elation filled my notebooks, were
elected. ''It turned into the biggest non-event in history,'' scoffed
Robert Dole, the Senate's new majority leader, discussing ''The Year of
the Woman.''

The feminists who promised Walter F. Mondale an outpouring of votes and
volunteers have been despondent, scrambling to explain why a majority of
American women preferred the President some feminists call ''the
Caveman.''

Some political bosses in districts where women ran and lost have been
grumbling privately about women making bad candidates. Some Democratic
leaders are talking publicly about the party's need to distance itself
from special interest groups, like women. And memos have circulated
among Republican politicians about the ''antifemale backlash'' in the
vote.

In the end, ironically, the election was dominated by the fight for the
white male vote and played out in the rhetoric and imagery of machismo.
After Geraldine Ferraro was chosen, Democratic Senator Lloyd Bentsen of
Texas joked that he would probably be the last white Anglo-Saxon male to
be considered for the Vice Presidency. Now, drumbeats sound in
Democratic circles that, for jobs like national party chairman, a white
Anglo-Saxon male would be best.

''We can't afford to have a party so feminized that it has no appeal to
males,'' said Patrick Caddell, the Democratic consultant. ''I THINK WE
CAN DECLARE THE GENDER gap closed,'' said Elizabeth Hanford Dole,
President Reagan's Secretary of Transportation.

Not quite. While it disappeared nationally in the Reagan landslide, the
gender gap did affect Republican losses in the Senate, where it showed
up as a significant factor in races in Michigan, Illinois and
Massachusetts. It also helped Madeleine M. Kunin to become the first
female Governor of Vermont.

The women's groups have lost some political clout because they did not
deliver their sisters. But most analysts now say that the vote was
simply not there to be delivered.

''The idea that Geraldine Ferraro or women controlled the women's vote
is as crazy as the A.F.L.- C.I.O. delivering labor's vote,'' said
William Schneider, a political analyst. ''Votes are no longer
deliverable in this day and age.''

One thing the year did prove is that the women's vote does not respond
simply to the symbol of a woman's candidacy.

''There has always been an ambiguous relationship of women voting for
women candidates,'' said Ethel Klein, an associate professor at Columbia
University and the author of the book ''Gender Politics.'' ''Unlike
blacks and other minorities, women do not vote on self-interest. They
vote for a better society as a whole. Women see as selfish the argument
of 'Vote for someone because she's a woman like you and you'll
personally gain.' ''

There was also a lingering feeling that Geraldine Ferraro's sex
superseded her qualifications. ''We put a woman on the ticket who would
not have been on the ticket unless she was a woman,'' said Patrick
Caddell. ''And even though Geraldine Ferraro did terrifically, there was
a sense that Mondale was forced into taking a woman by the women's
groups.''

Women voters were most happy with Mrs. Ferraro in the first few days,
before her halo slipped. ''Women are generally more inclined to support
women candidates unless there's a problem,'' said Kathy Wilson,
president of the National Women's Political Caucus. ''Then, women
candidates don't resuscitate themselves as quickly. The financial thing
was a problem for Ferraro with women. It destroyed her momentum. And it
shook the Mondale campaign's confidence in their ability to use her.''

The flap about her family's finances muddied Mrs. Ferraro's fresh image
and robbed her of a getting-to-know-you phase she dearly needed.

''I didn't even know a Geraldine Ferraro existed until I heard her name
announced,'' said Tamara Fish, a freshman at Harvard College. ''I ran to
the library and tried to find information on this Geraldine Ferraro. It
seemed a bit awkward to pick a running mate no one knows anything about
out of the blue,'' said the 18-year- old from Cleveland. ''You should
know who it is before the deadline. Mondale not only introduced her to
the entire country but then expected to win their support. It takes more
than a few months to get a sense for someone.''

Feminists have argued strongly that it would have been possible to
elicit a women's vote if the Mondale- Ferraro campaign had spent more
time wooing it with a revolutionary strategy designed to exploit the
Vice Presidential candidate's historic status. ''In this particular
election, women were ignored,'' said Marlene Johnson, Lieutenant
Governor of Minnesota, who was one of the most persuasive voices last
summer in convincing Mr. Mondale to choose a woman. While her remark
sounds incredible under the circumstances, it is significant because it
shows the extent of the hard feelings among feminists toward the Mondale
campaign strategists.

''There's a new family situation in this country,'' she explained.
''Almost 50 percent of children under 6 live in families where women
work and more families are headed by women. They don't always have
resources for child care, or to see that their children get educated
properly or live in safe neighborhoods with decent housing. We didn't
talk about that in this campaign.''

Lieutenant Governor Johnson and Dotty Lynch, a polltaker who did some
work for Mr. Mondale, argued that he and Mrs. Ferraro should have been
seen together more often in campaign stops and advertisements as an
eloquent promise of a new era in which men and women share power.

But Mr. Mondale's inner circle of ''smart-ass white boys,'' as they were
dubbed by many blacks and feminists in the party, cast Mrs. Ferraro in
the role of a traditional running mate, part hatchet man, part echo
chamber.

''If you want my frank opinion, after they named Ferraro, they got
scared,'' said Betty Friedan. ''They got so worried about the white male
backlash, they didn't want us women doing what we did before.''

Amid increasing tension with the Mondale men, women's groups tried to
mobilize the troops, but, with campaign funds pouring into the media and
polls rather than into grass-roots organizing, the groups could scarcely
find what they now scathingly call ''the phantom campaign.''

''Women called in all over the state and expected to be put to work,''
said Linda Davidoff, the chairman of Women for Mondale-Ferraro in New
York. ''And then they found out to their frustration - we must've gotten
the comment 500 times - nothing is happening out here.''

Referring to the feminists' complaints that Mrs. Ferraro should have
been seen more in commercials and shows on daytime television, Francis
J. O'Brien, the candidate's iconoclastic press aide, replied acidly:
''What did they want us to do, 'As the World Turns'?''

The Next Time

It has become a favorite parlor game among the political cognoscenti to
debate whether Mrs. Ferraro's candidacy will make it more or less likely
that a woman will be on a national ticket in 1988.

''I think we see dynamics at work where it may lead to the Vice
Presidential spot being 'the woman's spot,' '' said Lee Atwater, a
Reagan-Bush strategist.

Taking a page from Jesse Jackson's campaign, a woman may end up running
on her own in the Democratic primaries, but the odds are that the
Democrats will not choose one for the second spot next time, largely
because they need to focus on recouping the blue-collar white males they
have been losing since 1968.

The Republicans, on the other hand, could keep their strong base of
white males and expand into the two areas they need to solidify - women
and ''yuppies'' - by choosing a woman. As William Schneider said wryly,
''The ideal Republican candidate is a woman. The ideal Democratic
candidate is a general.''

Even Republicans who think Mrs. Ferraro hurt the Democrats blame it on
her personal drawbacks - her liberalism, the financial controversy, her
''strident'' style - rather than her sex. They think their pool of
high-ranking women erases any public perception of tokenism and that a
coed ticket could do wonders for the G.O.P. in a year when a bitter
primary race is expected.

''Kemp and Kirkpatrick,'' said George L. Clark Jr., the head of the New
York State Republican Party, his voice hushed with reverence, as he
referred to Representative Jack F. Kemp and Jeane J. Kirkpatrick, United
States chief delegate to the United Nations. ''I've heard that one more
times than the hair on my head. And I have a lot of hair on my head.''

One pundit joked that another Presidential contender, Senator Dole, will
soon have to ask his wife, the Transportation Secretary, whether she
prefers the title of First Lady or Vice President.

''I'm sure Bush and Baker are devising a strategy right now that would
cut me off by having a woman on the ticket,'' said Mr. Dole, referring
to the political plans of Vice President George Bush and Senator Howard
H. Baker Jr.

And doesn't that make life a little awkward since he is living with the
woman they might be thinking of?

''I kid her about it a lot,'' he said. ''I say, 'Boy, it sounds like a
great opportunity.' But it doesn't seem to ring any bells with her.
Unless she's cagier than I think she is.''

Certainly, the Ferraro experience will affect the sort of woman who is
chosen next time, and the method of that choice. There will be no more
blind dates with history. ''We should try to choose someone who already
has a national image,'' said Gloria Steinem.

The Candidate Remembers

Circumscribed by campaign etiquette, they did not begin their historic
relationship with a kiss. But they ended it with one.

''When I spoke to Fritz on the phone on Election Day,'' Mrs. Ferraro
recalls, ''I said, 'When I see you, I'm going to give you a kiss and I
don't care who's standing there, because I think you're terrific!' ''
Their rendezvous took place in a cold airport office in Washington,
where their planes touched down the day after the election.

''Fritz is a good friend and he deserves a lot of credit,'' she said.
''I'm sure there were times when he must have thought, 'What did I do
this for?' But that was never conveyed to me.''

Representative Ferraro is back in her Queens district office on a
darkening December afternoon, alone now except for her longtime aide Pat
Flynn and two secretaries.

She is asked if the First Woman had to be perfect.

''Yeah. Probably. I wasn't.''

She feels that the body blows she experienced in the campaign will make
it easier for the next woman.

''Next time around you're going to find women running in the primaries.
I think that's where you're going to have the opportunity to be tested.
The decision about whether or not she wants to do it is not going to be
something that's done in 48 hours. She's going to look at my candidacy
and, especially if she's married, say to her husband, 'Let's sit down
and go through everything. And I mean everything. Let's pretend you live
in a police state and you're going to have them check you out.' ''

Mrs. Ferraro spends her days mulling offers from Manhattan law firms,
debating whether to run for the Senate and working on a campaign memoir
that will help pay off, among other things, \$51,000 in accountants'
bills. She plans to be frank but not nasty in the book, the rights to
which she has just sold for about \$1 million. ''I'm really not a word
that rhymes with rich,'' she says, with a fey grin.

She doesn't regret losing her Congressional seat. ''The public works
committee is interesting, but if you've heard enough testimony on coal
slurry pipelines, you kind of figure out that there must be something
better in life,'' she says.

Tanned and relaxed from a St. Croix vacation, she wears dangling gold
earrings bearing her monogram and a purple knit dress with Joan Crawford
shoulders. ''These are my favorite earrings and I never wore them during
the whole campaign,'' she says, fingering them reflectively. ''They're
too gypsyish. I had to go to a lot of conservative areas in the South
and elsewhere.''

Now she wears what she likes. And, free of nervous aides who tried to
tone down her tendency to be ''flip lipped,'' she says what she likes.

The day before the election, when it was clear that the Democrats were
going to lose badly, Geraldine Ferraro and Maryland Representative
Barbara A. Mikulski reflected on whether her candidacy had had the right
stuff.

''Barbara said, 'Gerry, it's kind of like breaking the sound barrier for
the first time.' You know, those guys in those planes starting to get to
Mach 1 and Mach 2, or whatever it is they do to break the barrier. We
got shaken up and pushed and pulled in a lot of directions. We didn't do
it, but it's only the first time.''

Campaign aides maintain that Mrs. Ferraro was the best choice. ''If you
lose 49 states, where was she a drag - Delaware?'' says Francis O'Brien.

As Mrs. Ferraro sees it, ''I thought I would do a little bit better with
women.'' It is odd, she adds, how analysts say women were too
sophisticated to vote for a woman, and yet argue that Southerners would
not have been too sophisticated to vote for Lloyd Bentsen.

In a race that often seemed a qualifying test of toughness, with
reporters waiting to pounce on the first trace of a tear, Mrs. Ferraro's
grit never faltered. She prided herself on not getting nervous, but she
concedes she was scared before she debated George Bush.

''If I had said anything dumb, I really felt that would be making a
mistake for every woman sitting out there,'' she said.

Mrs. Ferraro has just returned from lunch with her cousin Nicholas
Ferraro, the former District Attorney of Queens County who launched her
political career by hiring her as a prosecutor in his office. He told
her she had erred during the debate when she told the Vice President not
to patronize her. ''Men don't like to hear that,'' he chastised gently.

''If I weren't on national TV, I probably would have turned around and
said 'Gimme a break!' '' she said.

She plays the ''If she were a he'' game about Mr. Bush. ''Suppose I had
gushed about Gromyko? Suppose I had said, 'Would you rather talk about
the World Series?' If I had acted as jumpy and giddy as he did, they
would have destroyed me.''

Referring to critics who said she had faltered on foreign affairs, she
asked sarcastically, ''If I can't discuss arms control and war and peace
because I've never been in a war, why should these guys be allowed to
discuss abortion?''

Beyond the debate, there were other pressures. Mrs. Ferraro led an odd
double life, campaigning brightly and tirelessly even as the published
reports mounted about investigations into the financial affairs of her
husband, John A. Zaccaro, and about speculation on ties to organized
crime.

''We used to get up every morning and look at the headlines. I'd get up
and say 'What's in it today?' John and I literally did not sleep. But I
sure as heck wasn't going to let anybody know that I was down.''

Now, she does come close to tears when talking about some of the
charges, and there are flashes of bitterness. She believes, based on
information that she has received from ''friends in criminal justice,''
that the Republicans leaked information on her family through Federal
law-enforcement agencies to blacken her candidacy, a charge a spokesman
for the Reagan-Bush campaign dismissed as a ''ridiculous thought.''

''Whatever they did to me is one thing,'' Mrs. Ferraro said, ''but going
after my husband and my father on stuff from 40, 50 years ago? I've
never seen anything done like this to anybody. Take Ronald Reagan. How
many people know that his father was an alcoholic? Do any of us care?''

There was constant speculation about the ravages of the race on Mrs.
Ferraro's marriage.

Clearly, Geraldine Ferraro and John Zaccaro know a lot more about each
other than they did five months ago. A friend recalled the moment, at a
meeting with accountants preparing for her marathon financial disclosure
press conference, when Mrs. Ferraro discovered that her husband had set
up an extra trust fund for their son.

Giving her husband a steely glance, she said, ''My two daughters will
each have an extra trust fund on Monday morning.''

At a party in Los Angeles, Frank Sinatra remarked, according to a top
Reagan aide, that he reckoned ''when this is all over, Zaccaro is going
to divorce that broad.'' In a beauty parlor at a Phoenix resort
recently, the manager told patrons that Mrs. Ferraro had filed for
divorce.

She has grown accustomed to the buzz about ''splitsville,'' as she calls
it. ''People are saying I would walk out on him because of this stuff. I
said if that man doesn't leave me after what he's been subjected to, he
deserves sainthood.'' She talked of a second honeymoon this summer to
the Far East to celebrate their 25th anniversary.

She is asked if the campaign was worth the public dissection and replies
it will depend on whether her husband comes safely out of the
investigations. ''If we get through it, and he's all right, I guess we
will both say yeah, it was worth it.''

The Feminists' Future

The message the election sent to the women's movement varies widely,
depending on who's talking. Most analysts agree with Mr. Schneider that
the message is: ''Don't try and push politicians around. Demanding
concessions from politicians is the way to ruin your own image and the
image of the Democratic Party.''

But many feminists agree with Laura D. Blackburn, an attorney and
president of the Institute for Mediation and Conflict Resolution, who
deduced quite a different message recently after a meeting of the
National Organization for Women: ''I think we can stop asking now and
demand - not just let the smart-ass white boys make the decisions about
campaigns.''

Gloria Steinem suggested a more independent route. ''All signs indicate
we have to stop wasting so much time convincing Democratic Party leaders
and do our own homework and elect our own candidates,'' she said.

Although most women's leaders feel ''chastened,'' as Dotty Lynch put it,
they are already plotting strategies. They want to move away from the
practice of putting women into ''scapegoat'' races against tough
incumbents and look for more vulnerable seats. They plan to recruit
women legislators aggressively, as well as business and cultural
leaders, to get more candidates into the pipeline. ''We have to think
ahead a dozen years instead of one,'' said Gloria Steinem.

A few days after Harriett Woods, a Democrat, was elected Lieutenant
Governor of Missouri last month, a group of women, unbeknownst to her,
got together in a private meeting in Washington to begin planning her
Senate campaign for 1986. Two years ago, she lost a Senate race to John
C. Danforth by less than two points, after squeezing money out of a
national party which was skeptical that a woman could win. Now, she is
sanguine about the election results and talks about improvements in the
political mood. ''The fact that I was the first woman elected statewide
in Missouri did not even make the headlines this time,'' she says.

Many analysts claimed that the elections showed that the feminists are
out of touch with average American women.

''The women's movement needs to make being a housewife and mother very
much more acceptable,'' said Edith P. Mayo, a curator in the division of
political history at the Smithsonian Institution.

It is a sensitive issue, but some feminists admit they must
''repackage'' their rhetoric and issues.

''We need to have patience with our sisters and we need to understand
that our perspectives are determined by the Washington political
environment,'' said Joanne Symons, director of political education for
the American Nurses' Association. ''A great deal of education still
needs to occur.''

Ferraro's Legacy

In the end, Mrs. Ferraro probably made no difference in the outcome of
the 1984 Presidential race.

And, while that leaves a sour taste now for feminists, because it
sometimes seemed as though the political fate of American women rested
on the 1984 election, it will be interpreted differently in the history
books.

''It was all fairly predictable and in the normal mode of American
politics,'' said Mr. Schneider. ''And isn't that what women want? Can
you imagine nominating a black and saying it made no difference?''

Mrs. Ferraro's legacy is best measured in shattered stereotypes. If male
politicians whistle ''Hail to the Chief'' when they shave in the
morning, jokes Michael Barone, a Washington opinion analyst, women can
now whistle it as they brush their hair.

Putting a woman on the ticket did evoke a backlash of sexist sentiment
and crude jokes.

''I talked to men on the road,'' reported Anne Wexler, a Ferraro
adviser, ''who said, 'I'm not voting for her because she belongs in the
home, she belongs back with her kids, what the hell is she doing this
for?' ''

Rich Bond, a Republican consultant and former aide to George Bush, added
that the climate was ripe for machismo. ''There was no Machiavellian
strategy behind Bush's locker room comment after the debate,'' said Mr.
Bond. ''But it worked fine. It did him a lot of good with Johnny
Lunchbucket and Johnny Sixpack out there saying 'Goddamn right, kick a
little ass!' ''

The women are pragmatic, however. They agree that the reaction among men
represents a necessary catharsis and they give the candidate credit for
having taken the brunt of that reaction.

Mrs. Ferraro's contribution will also be measured in the way women think
about themselves and leadership. ''That vision of that woman and the
Vice President of the United States standing toe-to- toe is implanted in
female brains all across this country, and it will germinate in a lot of
different ways,'' said Joanne Symons. ''Whenever a woman goes in to ask
her boss for a raise, she'll have a sense of 'I have a right to stand
here. Don't patronize me, Mr. Bush.' ''

Ann Richards, the State Treasurer of Texas, agreed: ''How do you
translate what happened to that little girl in the third grade who saw a
woman running for Vice President? We'll only know by the speeches she
makes when she's 25 and running for office that it made a difference in
her life.''

And, certainly, Mrs. Ferraro's legacy will last forever in the hearts of
such women as Georgene Goetting, known to her friends as just George.

At her bridge club, at weddings, at the grocery store on double-coupon
day, George listened to women sing Mrs. Ferraro's faults. ''Those women
don't have much faith in themselves,'' sniffed the amiable 63-year-old,
who lives with her husband, Ralph, a retired Culligan soft- water man,
in Beaver Dam, Wis.

She sent Mrs. Ferraro a check for \$100, the first campaign contribution
she had ever made. And she planted a large Mondale-Ferraro sign in her
yard.

''Women have come a long way and we have such a long way to go yet,''
she said. ''I couldn't find a thing to fault her on. She is kind of a
heroine, if I'm looking for one.''

The day after the election, George tied a black velvet ribbon around the
campaign sign in her yard. When she came home that evening, she was
surprised to find both gone. The woman scorned had become a hot
collector's item.

''It made me feel rather good,'' said George, with a twinkle. ''At least
my granddaughters will know I was so much for her, so enthused by her
candidacy. And they'll know I believed we would move forward.''

Advertisement

\protect\hyperlink{after-bottom}{Continue reading the main story}

\hypertarget{site-index}{%
\subsection{Site Index}\label{site-index}}

\hypertarget{site-information-navigation}{%
\subsection{Site Information
Navigation}\label{site-information-navigation}}

\begin{itemize}
\tightlist
\item
  \href{https://help.nytimes.com/hc/en-us/articles/115014792127-Copyright-notice}{©~2020~The
  New York Times Company}
\end{itemize}

\begin{itemize}
\tightlist
\item
  \href{https://www.nytco.com/}{NYTCo}
\item
  \href{https://help.nytimes.com/hc/en-us/articles/115015385887-Contact-Us}{Contact
  Us}
\item
  \href{https://www.nytco.com/careers/}{Work with us}
\item
  \href{https://nytmediakit.com/}{Advertise}
\item
  \href{http://www.tbrandstudio.com/}{T Brand Studio}
\item
  \href{https://www.nytimes.com/privacy/cookie-policy\#how-do-i-manage-trackers}{Your
  Ad Choices}
\item
  \href{https://www.nytimes.com/privacy}{Privacy}
\item
  \href{https://help.nytimes.com/hc/en-us/articles/115014893428-Terms-of-service}{Terms
  of Service}
\item
  \href{https://help.nytimes.com/hc/en-us/articles/115014893968-Terms-of-sale}{Terms
  of Sale}
\item
  \href{https://spiderbites.nytimes.com}{Site Map}
\item
  \href{https://help.nytimes.com/hc/en-us}{Help}
\item
  \href{https://www.nytimes.com/subscription?campaignId=37WXW}{Subscriptions}
\end{itemize}
