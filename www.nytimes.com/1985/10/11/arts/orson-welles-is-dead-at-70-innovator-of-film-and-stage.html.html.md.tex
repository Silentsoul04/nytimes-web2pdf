Sections

SEARCH

\protect\hyperlink{site-content}{Skip to
content}\protect\hyperlink{site-index}{Skip to site index}

\href{https://www.nytimes.com/section/arts}{Arts}

\href{https://myaccount.nytimes.com/auth/login?response_type=cookie\&client_id=vi}{}

\href{https://www.nytimes.com/section/todayspaper}{Today's Paper}

\href{/section/arts}{Arts}\textbar{}ORSON WELLES IS DEAD AT 70;
INNOVATOR OF FILM AND STAGE

\href{https://nyti.ms/29BToUi}{https://nyti.ms/29BToUi}

\begin{itemize}
\item
\item
\item
\item
\item
\end{itemize}

Advertisement

\protect\hyperlink{after-top}{Continue reading the main story}

Supported by

\protect\hyperlink{after-sponsor}{Continue reading the main story}

ORSON WELLES IS DEAD AT 70

\hypertarget{orson-welles-is-dead-at-70-innovator-of-film-and-stage}{%
\section{ORSON WELLES IS DEAD AT 70; INNOVATOR OF FILM AND
STAGE}\label{orson-welles-is-dead-at-70-innovator-of-film-and-stage}}

Special to the New York Times

\begin{itemize}
\item
  Oct. 11, 1985
\item
  \begin{itemize}
  \item
  \item
  \item
  \item
  \item
  \end{itemize}
\end{itemize}

\includegraphics{https://s1.nyt.com/timesmachine/pages/1/1985/10/11/070522_360W.png?quality=75\&auto=webp\&disable=upscale}

See the article in its original context from\\
October 11, 1985, Section A, Page
1\href{https://store.nytimes.com/collections/new-york-times-page-reprints?utm_source=nytimes\&utm_medium=article-page\&utm_campaign=reprints}{Buy
Reprints}

\href{http://timesmachine.nytimes.com/timesmachine/1985/10/11/070522.html}{View
on timesmachine}

TimesMachine is an exclusive benefit for home delivery and digital
subscribers.

About the Archive

This is a digitized version of an article from The Times's print
archive, before the start of online publication in 1996. To preserve
these articles as they originally appeared, The Times does not alter,
edit or update them.

Occasionally the digitization process introduces transcription errors or
other problems; we are continuing to work to improve these archived
versions.

Orson Welles, the Hollywood ''boy wonder'' who created the film classic
''Citizen Kane,'' scared tens of thousands of Americans with a realistic
radio report of a Martian invasion of New Jersey and changed the face of
film and theater with his daring new ideas, died yesterday in Los
Angeles, apparently of a heart attack. He was 70 years old and lived in
Las Vegas, Nev.

An assistant coroner in Los Angeles, Donald Messerle, said Welles's
death ''appears to be natural in origin.'' He had been under treatment
for diabetes as well as a heart ailment, his physician reported.
Welles's body was found by his chauffeur.

Despite the feeling of many that his career - which evoked almost
constant controversy over its 50 years -was one of largely unfulfilled
promise, Welles eventually won the respect of his colleagues. He
received the Lifetime Achievement Award of the American Film Institute
in 1975, and last year the Directors Guild of America gave him its
highest honor, the D. W. Griffith Award.

An Unorthodox Style

His unorthodox casting and staging for the theater gave new meaning to
the classics and to contemporary works. As the ''Wonder Boy'' of
Broadway in the 1930's, he set the stage on its ear with a ''Julius
Caesar'' set in Fascist Italy, an all-black ''Macbeth'' and his
presentation of Marc Blitzstein's ''Cradle Will Rock.'' His Mercury
Theater of the Air set new standards for radio drama, and in one
performance panicked thousands across the nation.

In film, his innovations in deep-focus technology and his use of theater
esthetics - long takes without close-ups, making the viewer's eye search
the screen as if it were a stage - created a new vocabulary for the
cinema.

Frequently Used Cliche

By age 24, he was already being described by the press as a has-been - a
cliche that would dog him all his life. But at that very moment Welles
was creating ''Citizen Kane,'' generally considered one of the best
motion pictures ever made. This scenario was repeated several times. His
second film, ''The Magnificent Ambersons,'' was poorly received, but is
now also regarded as a classic, although the distributors re-edited it
and Welles never liked the result. ''Falstaff'' and ''Touch of Evil,''
two of his later films, were also changed by others before their
release.

For his failure to realize his dreams, Welles blamed his critics and the
financiers of Hollywood. Others blamed what they described as his
erratic, egotistical, self-indulgent and self-destructive temperament.
But in the end, few denied his genius.

He was a Falstaffian figure, 6 feet 2 inches tall, weighing well over
200 pounds, with a huge appetite for good food and drink and large
cigars. Loud, brash, amusing and insufferable by turns, he made friends
and enemies by the score.

His life was a series of adventures whose details are fuzzy, in part
because he was a bit of a fabulist, delighting in pulling the legs of
listeners, in part because the credit for his achievements is the
subject of fierce controversy.

George Orson Welles was born in Kenosha, Wis., on May 6, 1915, the son
of Richard Head Welles, an inventor and manufacturer, and of the former
Beatrice Ives. His mother was dedicated to the theater, and Welles said
he made his debut at 2 as the child of ''Madame Butterfly'' in an opera
performance.

A Genius at 18 Months

According to ''Orson Welles,'' an authorized biography by Barbara
Leaming published a few weeks ago, Welles's genius was discovered when
he was only 18 months old, not by a Broadway producer or agent but by
his doctor, Maurice Bernstein, who, pronouncing the child a prodigy,
began to furnish him with a long series of educational gifts. These
included a violin, painting supplies, a magic kit, theatrical makeup
kits and even a conductor's baton.

His parents were divorced; Mrs. Welles died when he was 6, and he spent
several years traveling around the world with his father, a bon vivant.

At 10, he entered the Todd School in Woodstock, Ill. His five years
there were his only formal education.

Under the guidance of Roger Hill, the headmaster, young Orson steeped
himself in student theater, staging and acting in a series of
Shakespeare productions. Together, he and Mr. Hill edited ''Everybody's
Shakespeare,'' a text for school productions, which sold well for many
years.

On his graduation, he took a brief course in painting at the Chicago Art
Institute, then sailed for Ireland on a sketching tour. There, smoking a
cigar to disguise the fact that he was only 16, he managed to convince
the Gate Theater in Dublin that he was a Theater Guild actor on a
holiday.

He went on as the Duke in ''Jew Suss,'' followed it with other featured
parts and even achieved a featured role at the eminent Abbey Theater,
all in his first professional season. Then, after a spell of travel in
Spain and Morocco, he returned to Chicago.

Toured With Katharine Cornell

Through Thornton Wilder and Alexander Woollcott, Welles was introduced
to Katharine Cornell, who engaged him for supporting roles in a tour
that included ''Candida,'' ''Romeo and Juliet'' and ''The Barretts of
Wimpole Street.'' When Miss Cornell opened ''Romeo and Juliet'' on
Broadway on Dec. 20, 1934, Welles played Tybalt. He was then 19 years
old.

Like everything else he did, Welles's acting was a subject of
controversy. Some critics would always accuse him of hamming, of hogging
the limelight - especially when he was also the director. But many
professionals and a large public found his presence electrifying. ''He
has the manner of a giant with the look of a child,'' said Jean Cocteau,
''a lazy activeness, a mad wisdom, a solitude encompassing the world.''

Early in his Broadway career, Welles picked up supplementary income as a
radio actor. He became familiar to millions as the sepulchral voice of
''The Shadow,'' a wizard who turned virtually invisible to foil
criminals. But he kept up with the theater; in 1935 he was engaged by
the producer-director John Houseman to star in Archibald MacLeish's
poetic drama of the Depression, ''Panic,'' in which he portrayed a
tycoon.

To combat unemployment, the Roosevelt Administration had set up the
Works Progress Administration, one of the many projects of which was the
Federal Theater. With Mr. Houseman as manager and Welles as director, it
mounted several striking productions - the black ''Macbeth,'' a starkly
austere ''Dr. Faustus,'' a comic ''Horse Eats Hat'' - that excited the
theater world.

Even more than some other W.P.A. projects, the Federal Theater also
stirred conservative wrath. The last straw came when a troupe featuring
Howard da Silva and Will Geer prepared to stage ''The Cradle Will
Rock,'' a leftist musical by Marc Blitzstein, in 1937.

The authorities banned the production and locked the company out of the
theater on opening night. Welles joined the cast and an audience of
2,000 in a march up Sixth Avenue to a rented theater. To evade the ban,
the actors sang from seats in the auditorium, with Mr. Blitzstein
conducting from a piano on stage.

Co-founded the Mercury

The Federal Theater soon was liquidated, but Welles and Mr. Houseman
went on to found the Mercury Theater. Its first production in late 1937,
a ''Julius Caesar'' in modern dress with overtones of Fascist Italy, was
a smash hit. The Mercury took in the production of ''The Cradle Will
Rock'' that had been banned by Government authorities; it had success
also with ''Shoemaker's Holiday'' and ''Heartbreak House.''

Chiefly to provide its actors with steady income, the company signed up
with CBS Radio as the Mercury Theater of the Air. Its acting, dramatic
tension and inventive use of sound effects set new highs in radio
theater.

On Oct. 30, 1938, the Mercury Theater of the Air presented a
dramatization of H. G. Wells's ''War of the Worlds,'' in the form of
news bulletins and field reporting from the scene of a supposed Martian
invasion of New Jersey. It was an event unique in broadcast history,
frequently recalled in books, magazine articles and repeat performances.

Many thousands of listeners tuned in after the introduction, heard the
music interrupted by flash bulletins and panicked. Some armed themselves
and prepared to fight the invaders; many more seized a few belongings
and fled for the hills. Police switchboards around the country were
flooded with calls.

Welles was already famous; a few weeks earlier, at age 23, he had
appeared on the cover of Time magazine as the ''Wonder Boy'' of the
theater. Now he was suddenly a household word -the target of some
indignation, but also of amused admiration.

Hollywood Contract

The Mercury Theater on Broadway was nevertheless a financial failure,
and ended its theatrical existence in early 1939. The following season
the company, including such relatively unknown actors as Agnes
Moorehead, Joseph Cotten and Everett Sloane, went to Hollywood under a
contract with R.K.O. that granted Welles total artistic freedom.

On his first visit to a film studio, Welles is said to have marveled,
''This is the biggest electric-train set any boy ever had.'' The movie
community, however, was not entranced by the unconventional young
interloper.

A Saturday Evening Post profile in 1940 reflected this view. ''Orson was
an old war horse in the infant prodigy line by the time he was 10,'' it
said. ''He had already seen eight years' service as a child genius. Some
see the 24-year-old boy of today as a mere shadow of the 2-year-old man
they used to know.''

Welles was then directing ''Citizen Kane,'' based on a scenario by
Herman J. Mankiewicz, with himself in the title role. An impressionistic
biography of a newspaper publisher strongly suggestive of William
Randolph Hearst, it is now fabled for its use of flashback, deep-focus
photography, sets with ceilings, striking camera angles and imaginative
sound and cutting.

Kenneth Tynan has written, ''Nobody who saw 'Citizen Kane' at an
impressionable age will ever forget the experience; overnight, the
American cinema had acquired an adult vocabulary, a dictionary instead
of a phrase book for illiterates.'' Stanley Kauffmann called it ''the
best serious picture ever made in this country.''

Accusations and Rebuttals

The making of ''Kane'' has been the subject of fierce polemics. Pauline
Kael, in a famous New Yorker article in 1971, called it a ''shallow
masterpiece'' and ''comic-strip tragic,'' and accused Welles of trying
to deny credit to Mr. Mankiewicz, Mr. Houseman and the cameraman, Gregg
Toland. This has been rebutted in part by Mr. Houseman - who said he had
been the pupil and Welles the teacher in stage creation - and in great
detail by many Welles admirers, notably the director Peter Bogdanovich.

It turned out that Miss Kael had not sought to question Welles. His
defenders concede that he had thrown violent tantrums, leading to the
departure of Mr. Houseman, but say he was frequently generous in praise
of his collaborators.

More seriously, the Hearst newspaper chain was accused of seeking to
block the showing of ''Kane'' and it long barred mention of Welles and
his film in its publications. ''Citizen Kane'' could neither be reviewed
nor advertised in its newspapers. An offer was made to pay R.K.O. what
it had cost to make the picture plus a modest profit - well below \$1
million in all - to destroy all prints of the film.

This was refused. But ''Kane'' drew a mixed reception when it opened in
1941, and it was years before it turned into a profit maker. Welles won
an Academy Award for writing the film, and was nominated for directing
and acting awards.

Meanwhile, Welles was making Mercury's second movie, ''The Magnificent
Ambersons.'' At the close of shooting, Welles acceded to a request by
Washington that he fly to Rio de Janeiro to make a good-neighborly
documentary on the Mardi Gras. On his return, he found that an impatient
R.K.O. had done the final cutting of ''The Magnificent Ambersons.''

Difficulty With Financing

He was deeply hurt, and he disowned the film. On the movie company's
side, the assertion was made that Welles was impossible to deal with on
content, and unreliable on costs and completion dates. This perception,
encouraged by some journalists, made it forever afterward difficult for
Welles to obtain financing for his projects.

Welles and his supporters retorted that his budgets were always low,
sometimes remarkably so, and that his shooting schedules were sometimes
extaordinarily tight. Some concede that, never satisfied with his work,
he had an almost neurotic reluctance to view it when done, and several
uncompleted works remain in storage.

After ''The Magnificent Ambersons,'' the tireless Welles returned to
Broadway in 1941 to direct a dramatization of Richard Wright's ''Native
Son,'' which was a triumph; did a series of wartime propaganda
broadcasts for the Government; produced and acted in the movie thriller
''Journey Into Fear'' (1942), which was a failure, and starred as Mr.
Rochester in the highly popular ''Jane Eyre'' (1943).

Rejected by the Army because of flat feet, he took part as a magician -
another of his talents - in a tour of the European Theater of
Operations, in which his act was sawing Marlene Dietrich in half. Back
home after the war, he adapted and staged a Cole Porter musical version
of ''Around the World in 80 Days'' in 1946 that was praised by critics
but failed at the box office. He lost \$350,000 of his own money in the
production.

He also directed and acted in a Hollywood spy thriller, ''The
Stranger,'' in 1946, and produced, directed and co-starred with Rita
Hayworth in ''The Lady From Shanghai,'' in 1948.

Three Marriages

He and Miss Hayworth, who were married in 1943, were divorced in 1948.
They had a daughter, Rebecca. Welles had a son, Christopher, from his
first marriage, to Virginia Nicholson, which also ended in divorce. In
1955, he married the Italian actress Paola Mori, who appeared with him
in his ''Mr. Arkadin.'' They have a daughter, Beatrice.

In part because of his losses from ''Around the World,'' which were
ruled nondeductible for tax purposes, Welles moved to Europe, where he
lived most often in Spain, for many years. From time to time, he would
act in a film or television show or in television commercials - he was
always in demand as a performer - and from time to time would use his
earnings and what financing he could raise to make a picture, or part of
one. His acting talents enhanced such films - made by other directors -
as ''Tomorrow Is Forever,'' ''The Third Man,'' ''Compulsion,'' ''A Man
for All Seasons'' and ''Catch-22.''

In Italy and Morocco, at intervals from 1949 to 1952, he put together
and starred in ''Othello'' and ''Macbeth.'' The latter film, shot in
three weeks, has been violently criticized. In Mexico and Paris,
beginning in 1955, he filmed the not yet completed ''Don Quixote.'' In
four European countries in 1954, he made ''Mr. Arkadin,'' based on a
thriller he had written himself.

In Paris and Zagreb, Yugoslavia, in 1962, he wrote, directed and acted
in ''The Trial,'' based on the Kafka novel. Many critics decry it; some
call it a masterpiece. He completed two other films in Europe and, in
1970, began a major project, ''The Other Side of the Wind,'' which
remains unfinished. His last directorial effort to be released was ''The
Immortal Story'' in 1968; he also performed in it.

In 1958, Welles returned briefly to Hollywood to act with Charlton
Heston in ''Touch of Evil.'' At Mr. Heston's suggestion, Welles was
enlisted as director as well. Some admirers consider it one of his best
films, and its opening scene, coming to a climax in a car explosion, is
a model of the genre, although Welles was to complain that it, too, had
been re-edited by the studio without his permission.

He also staged, and appeared in, a successful run of ''Othello'' in
London, and was featured in dozens of television shows.

Boycotted New York Stage

He refused to appear on Broadway, however, after an unfortunate
appearance in ''King Lear'' during which, having broken an ankle, he
acted in a wheelchair. He vowed that he would never return to the New
York stage while Walter Kerr was still a critic there. Writing for The
New York Herald Tribune, Mr. Kerr had described Welles as ''a buffoon,''
''an actor without talent'' and ''an international joke, possibly the
world's youngest has-been.''

Mr. Kerr was not the only hostile critic. In 1963 Stanley Kauffmann,
although more admiring of Welles's virtuosity, also accused him of
overacting and concluded, ''After 'Kane,' his film directing consists of
sometimes glittering, often wild attempts to recapture that first fine
careful rapture.''

That was the common reception given in this country to Welles's film
''Falstaff,'' which had been hailed in Europe under the title ''Chimes
at Midnight.'' When it appeared here in 1967, a number of critics panned
it, one calling Welles ''inarticulate'' and saying he made Falstaff ''a
sort of Jackie Gleason.''

More recently, however, The Times's Vincent Canby wrote that the picture
''may be the greatest Shakespearean film ever made.''

The film and television writer Stephen Farber commented: ''Looking back
over American movie history - a history of wrecked careers - you begin
to see that the critics have a lot to answer for. The classic victim is
Orson Welles.''

This was, of course, also Welles's view. He complained, ''They don't
review my work - they review me.'' It cannot be doubted that his
flamboyant personality, his enormous early success, his pride and his
lofty aspirations caused critics to measure him against standards they
might not have applied to a more modest film maker.

He was the legendary sort of figure upon whom old anecdotes are rehung.
Mr. Mankiewicz, for example, was reported by Miss Kael to have said of
Welles, ''There, but for the grace of God, goes God.''

Welles inspired harsh criticism, yet most people felt that even his most
unsuccessful, most self-indulgent works all had some feature, some turn
that was memorable. There were no dissenters when, at the dedication of
a Theater Hall of Fame in New York 1n 1972, his name was among the first
to be chosen.

He is survived by his wife and three children.

Advertisement

\protect\hyperlink{after-bottom}{Continue reading the main story}

\hypertarget{site-index}{%
\subsection{Site Index}\label{site-index}}

\hypertarget{site-information-navigation}{%
\subsection{Site Information
Navigation}\label{site-information-navigation}}

\begin{itemize}
\tightlist
\item
  \href{https://help.nytimes.com/hc/en-us/articles/115014792127-Copyright-notice}{©~2020~The
  New York Times Company}
\end{itemize}

\begin{itemize}
\tightlist
\item
  \href{https://www.nytco.com/}{NYTCo}
\item
  \href{https://help.nytimes.com/hc/en-us/articles/115015385887-Contact-Us}{Contact
  Us}
\item
  \href{https://www.nytco.com/careers/}{Work with us}
\item
  \href{https://nytmediakit.com/}{Advertise}
\item
  \href{http://www.tbrandstudio.com/}{T Brand Studio}
\item
  \href{https://www.nytimes.com/privacy/cookie-policy\#how-do-i-manage-trackers}{Your
  Ad Choices}
\item
  \href{https://www.nytimes.com/privacy}{Privacy}
\item
  \href{https://help.nytimes.com/hc/en-us/articles/115014893428-Terms-of-service}{Terms
  of Service}
\item
  \href{https://help.nytimes.com/hc/en-us/articles/115014893968-Terms-of-sale}{Terms
  of Sale}
\item
  \href{https://spiderbites.nytimes.com}{Site Map}
\item
  \href{https://help.nytimes.com/hc/en-us}{Help}
\item
  \href{https://www.nytimes.com/subscription?campaignId=37WXW}{Subscriptions}
\end{itemize}
