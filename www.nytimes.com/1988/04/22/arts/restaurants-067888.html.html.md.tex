Sections

SEARCH

\protect\hyperlink{site-content}{Skip to
content}\protect\hyperlink{site-index}{Skip to site index}

\href{https://www.nytimes.com/section/arts}{Arts}

\href{https://myaccount.nytimes.com/auth/login?response_type=cookie\&client_id=vi}{}

\href{https://www.nytimes.com/section/todayspaper}{Today's Paper}

\href{/section/arts}{Arts}\textbar{}Restaurants

\url{https://nyti.ms/29z7MiC}

\begin{itemize}
\item
\item
\item
\item
\item
\end{itemize}

Advertisement

\protect\hyperlink{after-top}{Continue reading the main story}

Supported by

\protect\hyperlink{after-sponsor}{Continue reading the main story}

\hypertarget{restaurants}{%
\section{Restaurants}\label{restaurants}}

Bryan Miller

\begin{itemize}
\item
  April 22, 1988
\item
  \begin{itemize}
  \item
  \item
  \item
  \item
  \item
  \end{itemize}
\end{itemize}

\includegraphics{https://s1.nyt.com/timesmachine/pages/1/1988/04/22/067888_360W.png?quality=75\&auto=webp\&disable=upscale}

See the article in its original context from\\
April 22, 1988, Section C, Page
26\href{https://store.nytimes.com/collections/new-york-times-page-reprints?utm_source=nytimes\&utm_medium=article-page\&utm_campaign=reprints}{Buy
Reprints}

\href{http://timesmachine.nytimes.com/timesmachine/1988/04/22/067888.html}{View
on timesmachine}

TimesMachine is an exclusive benefit for home delivery and digital
subscribers.

About the Archive

This is a digitized version of an article from The Times's print
archive, before the start of online publication in 1996. To preserve
these articles as they originally appeared, The Times does not alter,
edit or update them.

Occasionally the digitization process introduces transcription errors or
other problems; we are continuing to work to improve these archived
versions.

The Drake Hotel opened its sumptuous restaurant, Lafayette, in July 1986
with the reserve and discretion of a Swiss banker - actually, the hotel
is owned by two Swiss companies, Swissair and Nestle S.A. In less than
two years, Lafayette has become one of the city's most exhilarating
dining rooms under the direction of two French chefs: Louis Outhier, the
consultant, who had a Michelin three-star restaurant on the French
Riviera, and his startlingly gifted protege, Jean-Georges Vongerichten,
who is the full-time chef. Curiously, both remain relatively unknown
here outside of professional circles. That should not last long, for the
kitchen's performance, paired with the highly civilized service, now
merits four stars.

Mr. Outhier, who recently closed his own restaurant, L'Oasis in the town
of La Napoule, now consults at a half-dozen establishments around the
world. He certainly need not lose sleep over his New York operation when
on the road. Mr. Vongerichten, a 31-year-old native of Alsace,
brilliantly executes the sunny, herb-infused cuisine of Mediterranean
France for which his mentor is renowned. The young chef's repertory is
exquisitely refined yet never fussy, and the galaxy of flavors found on
the menu reflects Mr. Vongerichten's cosmopolitan background.

After training for three and a half years at Auberge de l'Ill in Alsace,
Mr. Vongerichten joined Mr. Outhier's flying chef squadron, where he has
been ever since, with the exception of stints with Paul Bocuse and
another apprenticeship in West Germany. Mr. Vongerichten's assignments
have taken him to restaurants in Bangkok, Singapore, London and Boston
before coming to New York.

Lafayette's dining room is plush and spacious, done in soothing shades
of beige and cream. Tables are widely spaced, and extraneous sound is
well-muffled. The crepuscular lighting sometimes makes it difficult to
read the menu. When a waiter saw us straining, he came by and gave us
tiny flashlights, the kind children twirl at the circus.

In addition to the \$55 prix-fixe dinner menu (\$37.50 at lunch), one
can order from two tasting menus, at \$65 and \$75.

The spring menu begins with such lighter offerings as slices of turnip
wrapped around goat cheese to form little raviolis, in a richly flavored
duck broth along with strips of rare duck breast. Another is a carpaccio
of tuna and black bass pounded together to create a marbled effect,
ringed with minced sweet pepper, eggplant, zucchini and olives glossed
with a saffron and basil vinaigrette. Sea urchin is given a new
dimension here. The saline roe is the base for a terrific cold souffle
ringed by fresh periwinkles, Belon oysters and clams in a shellfish
stock enhanced with lemongrass.

A provocative starter that borrows from Asian cooking is called crab and
Swiss chard cannelloni. Fresh crab meat and Swiss chard are enveloped in
sheets of fresh pasta and set over a pale yellow sauce made with reduced
vegetable bouillon and carrot juice, then seasoned with cardamom and
bound with a little butter - an extraordinary fusion, just faintly sweet
from the carrots. Another Eastern-influenced creation is perfectly
cooked shrimp with pearl onions, fresh peas and bacon in a complex yet
incredibly subtle broth blending carrot juice, lime juice, nutmeg and
cinnamon. Typical of the Provencal style is fresh crayfish arranged over
a fan of zucchini and ripe tomatoes in an aromatic basil and olive oil
dressing.

The service staff at Lafayette is as refined as the food. They never
hover, never fawn, yet when you need something, they magically appear.
The international wine list has fine selections in every price range.
Temperature-controlled rooms assure that the reds are always cellar
temperature.

Mr. Vongerichten's light touch with seafood makes for some dynamic
entrees. Fillet of salmon is hidden under a brittle lid of potatoes
Maxim and ringed with a sweetly perfumed sauce of sherry vinegar, leeks
and caviar; halibut was equally arresting as a special at a recent
lunch, in a vegetable-enriched fish-stock sauce bound with beurre blanc,
accompanied by a ''spaghetti' of blanched cucumbers. For showmanship,
nothing matches the whole sea bass in a meticulously sculptured pastry
crust with a tarragon tomato sauce.

Roasted guinea hen, more meaty and flavorful than chicken, came in an
assertive olive oil sauce bolstered with herbs and lemon rind along with
an airy timbale of parsley. Roasted rabbit with crisply fried leeks was
flavorful but dry. Spit-roasted baby goat, its meat ruddy and mild, is
compelling in its reduced marinade with fresh peas and carrots. The list
of superlatives goes on.

The pastry chef, Jean-Marc Burillier, has vastly improved the desserts
in the last year. Among those to watch for are the exceptionally light
meringue and bavarian cream pie, the two-layered cake of passion fruit
mousse and chocolate over a bright citric sauce, gratin of wild
raspberries and hazelnut anisette layer cake. Only a chocolate truffle
and Grand Marnier cake came up short by chocoholic standards. A puff
pastry and rhubarb tart ringed with strawberry puree is a serendipitous
way to celebrate the season of renewal - and New York's newest four-star
chef. Lafayette

* *** 65 East 56th Street, 832-1565.

Atmosphere: Plush and civilized. Well-spaced tables, good noise control.

Service: Low key and vigilant.

Recommended dishes: Sea urchin souffle, crab and Swiss chard cannelloni,
squab and green lentils, turnip ravioli with goat cheese, salmon with
potato crust, guinea hen with lemon rind and olive oil sauce, halibut in
vegetable and fish stock sauce, sea bass in pastry crust, meringue and
bavarian cream pie, passion fruit mousse and chocolate layer cake,
gratin of wild raspberries, rhubarb tart with strawberry puree.

Price range: Dinner tasting menus at \$65 and \$75; dinner prix fixe
\$55; lunch prix fixe \$37.50; lighter lunch (entree of the day and
salad) prix fixe \$23.50.

Credit cards: All major cards.

Hours: Lunch, noon to 2:30 P.M. Monday to Friday; dinner, 6:30 to 10:30
P.M. Monday to Friday, 6 to 10:30 P.M. Saturday. Closed Sunday.

Reservations: Required about a week in advance.

Wheelchair accessibility: All facilities on ground level. What the stars
mean:

(None) ... Poor to satisfactory

* ... Good

** ... Very good

*** ... Excellent

**** .... Extraordinary

These ratings reflect the reviewer's reaction primarily to food, with
ambiance and service taken into consideration. Menu listings and prices
are subject to change.

Advertisement

\protect\hyperlink{after-bottom}{Continue reading the main story}

\hypertarget{site-index}{%
\subsection{Site Index}\label{site-index}}

\hypertarget{site-information-navigation}{%
\subsection{Site Information
Navigation}\label{site-information-navigation}}

\begin{itemize}
\tightlist
\item
  \href{https://help.nytimes.com/hc/en-us/articles/115014792127-Copyright-notice}{©~2020~The
  New York Times Company}
\end{itemize}

\begin{itemize}
\tightlist
\item
  \href{https://www.nytco.com/}{NYTCo}
\item
  \href{https://help.nytimes.com/hc/en-us/articles/115015385887-Contact-Us}{Contact
  Us}
\item
  \href{https://www.nytco.com/careers/}{Work with us}
\item
  \href{https://nytmediakit.com/}{Advertise}
\item
  \href{http://www.tbrandstudio.com/}{T Brand Studio}
\item
  \href{https://www.nytimes.com/privacy/cookie-policy\#how-do-i-manage-trackers}{Your
  Ad Choices}
\item
  \href{https://www.nytimes.com/privacy}{Privacy}
\item
  \href{https://help.nytimes.com/hc/en-us/articles/115014893428-Terms-of-service}{Terms
  of Service}
\item
  \href{https://help.nytimes.com/hc/en-us/articles/115014893968-Terms-of-sale}{Terms
  of Sale}
\item
  \href{https://spiderbites.nytimes.com}{Site Map}
\item
  \href{https://help.nytimes.com/hc/en-us}{Help}
\item
  \href{https://www.nytimes.com/subscription?campaignId=37WXW}{Subscriptions}
\end{itemize}
