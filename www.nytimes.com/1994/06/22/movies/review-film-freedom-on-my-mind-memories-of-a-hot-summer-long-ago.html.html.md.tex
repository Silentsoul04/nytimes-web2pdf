Sections

SEARCH

\protect\hyperlink{site-content}{Skip to
content}\protect\hyperlink{site-index}{Skip to site index}

\href{https://www.nytimes.com/section/movies}{Movies}

\href{https://myaccount.nytimes.com/auth/login?response_type=cookie\&client_id=vi}{}

\href{https://www.nytimes.com/section/todayspaper}{Today's Paper}

\href{/section/movies}{Movies}\textbar{}Review/Film: Freedom on My Mind;
Memories Of a Hot Summer Long Ago

\href{https://nyti.ms/298zFP8}{https://nyti.ms/298zFP8}

\begin{itemize}
\item
\item
\item
\item
\item
\end{itemize}

Advertisement

\protect\hyperlink{after-top}{Continue reading the main story}

Supported by

\protect\hyperlink{after-sponsor}{Continue reading the main story}

\hypertarget{reviewfilm-freedom-on-my-mind-memories-of-a-hot-summer-long-ago}{%
\section{Review/Film: Freedom on My Mind; Memories Of a Hot Summer Long
Ago}\label{reviewfilm-freedom-on-my-mind-memories-of-a-hot-summer-long-ago}}

\begin{itemize}
\tightlist
\item
  Freedom on My Mind\\
  Directed by Connie Field, Marilyn Mulford Documentary 1h 45m
\end{itemize}

By \href{https://www.nytimes.com/by/caryn-james}{Caryn James}

\begin{itemize}
\item
  June 22, 1994
\item
  \begin{itemize}
  \item
  \item
  \item
  \item
  \item
  \end{itemize}
\end{itemize}

\includegraphics{https://s1.nyt.com/timesmachine/pages/1/1994/06/22/467677_360W.png?quality=75\&auto=webp\&disable=upscale}

See the article in its original context from\\
June 22, 1994, Section C, Page
13\href{https://store.nytimes.com/collections/new-york-times-page-reprints?utm_source=nytimes\&utm_medium=article-page\&utm_campaign=reprints}{Buy
Reprints}

\href{http://timesmachine.nytimes.com/timesmachine/1994/06/22/467677.html}{View
on timesmachine}

TimesMachine is an exclusive benefit for home delivery and digital
subscribers.

About the Archive

This is a digitized version of an article from The Times's print
archive, before the start of online publication in 1996. To preserve
these articles as they originally appeared, The Times does not alter,
edit or update them.

Occasionally the digitization process introduces transcription errors or
other problems; we are continuing to work to improve these archived
versions.

The idealism of the 1960's can seem like a cultural artifact now, as
distant from the 90's as those quaint black-and-white television reports
are from today's high-tech, second-to-second coverage. "Freedom on My
Mind," the story of the volatile battle to register black voters in
Mississippi during the summer of 1964, makes provocative use of that old
film to situate viewers in a blatantly racist time and place. "I am a
Mississippi segregationist and proud of it," says Ross Barnett, the
Governor of the state. A well-dressed white man sitting in a little cafe
says, "The colored people are very happy in Mississippi." Among the
clips are scenes of the idealistic civil rights volunteers, black and
white, who gave the lie to statements like that.

Interwoven with the archival material are recent interviews with many
who were active in the civil rights movement: L. C. Dorsey, a
sharecropper's daughter from Mississippi; Bob Moses, a black graduate
student from Harvard; Marshall Ganz, one of many white, middle-class
college students bused in to register black voters and to attract the
kind of news-media attention that Southern blacks would have been
unlikely to draw on their own. As they look back 30 years to what was
called Freedom Summer, their testimony adds a complex layer to the film.
An absorbing work of historical preservation and strong ideas, "Freedom
on My Mind" won the grand jury prize for documentary at this year's
Sundance Film Festival. It opens today at Film Forum 1.

Among those the film follows through Freedom Summer, Endesha Ida Mae
Holland offers the most dramatic personal story. Known today as a
playwright, she recalls being raped at the age of 11 by a white man. She
soon dropped out of school and became a prostitute. When the white
volunteers arrived in Mississippi in 1964, she responded by looking for
customers, but stayed on as a volunteer. "The movement said to me I was
somebody," she says buoyantly.

The film doesn't bypass harsh facts about the movement. Black people
lost their jobs and risked their lives for daring to register to vote.
There were cultural tensions between the Northern white students and the
Southern black families with whom they lodged. Many of the black
volunteers had never sat at a table with white people before. The
students were aware (though perhaps not aware enough) that they were in
the touchy paternalistic position of self-appointed saviors.

Mr. Moses was at the center of the political strategy. He led the
Mississippi Freedom Democratic Party (a group that wanted to unseat the
official, all-white Dixiecrat delegates) to the 1964 Democratic
Convention in Atlantic City. For true culture shock, "Freedom on My
Mind" offers scenes of a boardwalk packed with middle-American
delegates, and the sound of that convention's theme song, "Hello, Dolly"
(recast as "Hello, Lyndon"). While Fannie Lou Hamer spoke on live
television in favor of seating the civil rights delegates, President
Johnson called a news conference that strategically bumped her off the
air. Mr. Moses calls the rejection of the Freedom Democratic Party "a
betrayal" by the Democrats and sees it as a turning point in the civil
rights movmment. "It led directly to armed struggle," he says, "one of
the great tragedies of this country."

Mr. Moses alone expresses such a sense of betrayal. Only when the final
credits roll does the audience discover that he eventually left the
United States to spend several years working in Africa before returning
to create a public education program in Boston. He and the other Freedom
Summer volunteers interviewed here remain idealistic, in ways that are
inexplicable but convincing. Mr. Ganz, who went on to work with Cesar
Chavez, says simply that the movement "gave us hope."

Ms. Dorsey, the sharecropper's daughter, who went on to earn a Ph.D. in
public health, is eloquent on the subject of the deep cultural shift the
civil rights movement set in motion. Black children in her generation
were taught to stay in their place in regard to white people, she
recalls; the next generation was not.

Connie Field and Marilyn Mulford, who together produced and directed
"Freedom on My Mind," have created the best kind of historical record,
one that resonates today.

Freedom on My Mind Directed and produced by Connie Field and Marilyn
Mulford; written and edited by Michael Chandler; a Tara Release. At Film
Forum 1, 209 West Houston Street, South Village. Running time: 105
minutes. This film has no rating.

Advertisement

\protect\hyperlink{after-bottom}{Continue reading the main story}

\hypertarget{site-index}{%
\subsection{Site Index}\label{site-index}}

\hypertarget{site-information-navigation}{%
\subsection{Site Information
Navigation}\label{site-information-navigation}}

\begin{itemize}
\tightlist
\item
  \href{https://help.nytimes.com/hc/en-us/articles/115014792127-Copyright-notice}{©~2020~The
  New York Times Company}
\end{itemize}

\begin{itemize}
\tightlist
\item
  \href{https://www.nytco.com/}{NYTCo}
\item
  \href{https://help.nytimes.com/hc/en-us/articles/115015385887-Contact-Us}{Contact
  Us}
\item
  \href{https://www.nytco.com/careers/}{Work with us}
\item
  \href{https://nytmediakit.com/}{Advertise}
\item
  \href{http://www.tbrandstudio.com/}{T Brand Studio}
\item
  \href{https://www.nytimes.com/privacy/cookie-policy\#how-do-i-manage-trackers}{Your
  Ad Choices}
\item
  \href{https://www.nytimes.com/privacy}{Privacy}
\item
  \href{https://help.nytimes.com/hc/en-us/articles/115014893428-Terms-of-service}{Terms
  of Service}
\item
  \href{https://help.nytimes.com/hc/en-us/articles/115014893968-Terms-of-sale}{Terms
  of Sale}
\item
  \href{https://spiderbites.nytimes.com}{Site Map}
\item
  \href{https://help.nytimes.com/hc/en-us}{Help}
\item
  \href{https://www.nytimes.com/subscription?campaignId=37WXW}{Subscriptions}
\end{itemize}
