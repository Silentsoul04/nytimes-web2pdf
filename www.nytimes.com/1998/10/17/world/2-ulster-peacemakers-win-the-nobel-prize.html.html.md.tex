Sections

SEARCH

\protect\hyperlink{site-content}{Skip to
content}\protect\hyperlink{site-index}{Skip to site index}

\href{https://www.nytimes.com/section/world}{World}

\href{https://myaccount.nytimes.com/auth/login?response_type=cookie\&client_id=vi}{}

\href{https://www.nytimes.com/section/todayspaper}{Today's Paper}

\href{/section/world}{World}\textbar{}2 Ulster Peacemakers Win the Nobel
Prize

\begin{itemize}
\item
\item
\item
\item
\item
\end{itemize}

Advertisement

\protect\hyperlink{after-top}{Continue reading the main story}

Supported by

\protect\hyperlink{after-sponsor}{Continue reading the main story}

\hypertarget{2-ulster-peacemakers-win-the-nobel-prize}{%
\section{2 Ulster Peacemakers Win the Nobel
Prize}\label{2-ulster-peacemakers-win-the-nobel-prize}}

By \href{https://www.nytimes.com/by/warren-hoge}{Warren Hoge}

\begin{itemize}
\item
  Oct. 17, 1998
\item
  \begin{itemize}
  \item
  \item
  \item
  \item
  \item
  \end{itemize}
\end{itemize}

See the article in its original context from\\
October 17, 1998, Section A, Page
7\href{https://store.nytimes.com/collections/new-york-times-page-reprints?utm_source=nytimes\&utm_medium=article-page\&utm_campaign=reprints}{Buy
Reprints}

\href{http://timesmachine.nytimes.com/timesmachine/1998/10/17/385980.html}{View
on timesmachine}

TimesMachine is an exclusive benefit for home delivery and digital
subscribers.

The 1998 Nobel Peace Prize was awarded today to John Hume and David
Trimble, leaders of the largest Roman Catholic and Protestant political
parties in Northern Ireland, for their efforts to bring peace to the
long-polarized British province.

Mr. Hume, 61, the Catholic head of the Social Democratic and Labor
Party, was cited by the Nobel Committee in Oslo for having been ''the
clearest and most consistent of Northern Ireland's political leaders in
his work for a peaceful solution.''

Mr. Trimble, 54, the Protestant leader of the Ulster Unionists, was
honored for having demonstrated ''great political courage when, at a
critical stage in the process, he advocated solutions which led to the
peace agreement.''

A leader of a third prominent party, Gerry Adams of Sinn Fein, the
political wing of the Irish Republican Army, was not named as a prize
winner. Nor were several other figures mentioned as possibilities,
including former Senator George Mitchell, who led the talks; Prime
Ministers Tony Blair of Britain and Bertie Ahern of Ireland; President
Clinton, and Mo Mowlam, the British Government's Secretary for Northern
Ireland.

The accord, signed on April 10 and known as the Good Friday Agreement,
has given the 1.7 million residents of Northern Ireland a respite from
the sectarian violence that has claimed more than 3,200 lives in the
last 30 years. It has also opened the possibility of lasting stability
for the first time since the establishment of Northern Ireland with
partition from Ireland in 1921.

Forging concessions from fiercely antagonistic populations, the accord
seeks to balance the Protestant majority's wish to remain part of
Britain with Catholic desires to strengthen ties to the Irish Republic
to the south.

The committee, seeing in Northern Ireland's two warring groups a dispute
with notable similarities to violent tribal confrontations elsewhere,
expressed the hope that the accord would serve ''to inspire peaceful
solutions to other religious, ethnic and national conflicts around the
world.''

While it did not honor Mr. Adams, the committee said it wished to
''emphasize the importance of the positive contributions to the peace
process made by other Northern Irish leaders.''

Asked if the prize givers had ''shied away'' from Mr. Adams because of
his past association with violence, the chairman of the secretive
five-man committee, Francis Sejersted, said: ''We don't shy away from
anybody. We just try to pick the most worthy people.'' Mr. Adams, he
said, was ''clearly one of those who has contributed significantly to
the process so far.''

Mr. Adams was almost certainly on the list of 139 nominees, the most
ever, because the chairman of the Norwegian Nobel Committee today
implied strongly that it had considered him for the prize but decided to
limit it to Mr. Hume and Mr. Trimble.

The two men will share the prize money of \$960,000.

Nominations can be made by university professors, members of national
legislatures, former Nobel laureates, Nobel Committee members and a
number of others considered qualified. The committee does not identify
the nominees.

Explaining why the committee selected the two men from the number of
people associated with the Good Friday Agreement, Mr. Sejersted said,
''There is one from the Catholic side and one from the Protestant side,
so I do not fear that it is going to alienate any of the other parties
in the process.''

Mr. Adams, in New York on a fund-raising trip for Sinn Fein, welcomed
the Oslo announcement and particularly praised Mr. Hume, who is widely
seen as having helped persuade the I.R.A. to adopt a cease-fire and
having eased Sinn Fein's entry into the talks.

''Indeed, there would be no peace process but for his courage and
vision,'' Mr. Adams said, adding, ''No one deserves this accolade
more.'' He also wished Mr. Trimble well and said the prize imposed on
everyone the responsibility to ''push ahead through the speedy
implementation of the agreement.''

The immediate next steps are the creation in the new power-sharing
Northern Ireland Assembly of a 10-member executive that will in effect
be the new government of the province, and a cross-border body linking
the North with the Irish Republic, one of the key attractions in the
pact to the Catholic minority.

Mr. Trimble, who is the First Minister of the new legislature, faces
pressure from dissidents in his quarrelsome party who insist that the
I.R.A. must start disarming before Sinn Fein can take up the positions
in the Assembly that its vote total in June elections entitles it to.

In the unforgiving politics of Northern Ireland, the Unionist dissidents
and members of other Protestant parties who did not join in the peace
talks attacked both Mr. Trimble and Mr. Hume today.

Ian Paisley Jr., son of the head of the Democratic Unionist Party,
called the Nobel Committee's decision a ''farce'' and said of the
winners, ''These people have not delivered peace, and they are not
peacemakers.''

Mr. Trimble said today that he was ''slightly uncomfortable'' with the
award because so many other people had been involved beside him in
reaching the settlement and much remained to be done to put it in place.

''We know that while we have the makings of peace, it is not wholly
secure yet,'' he told the BBC from Denver, where he was on an 11-city
North American tour to spur foreign investment in Ulster. ''I hope it
does not turn out to be premature.''

Hume Cites 'Work Of Very Many People'

Mr. Hume received word of the prize at his home in Londonderry and
termed it ''an expression of the total endorsement of the work of very
many people.'' He added: ''This isn't just an award to David Trimble and
myself. It is an award to all the people in Northern Ireland.''

Today was the second time the Nobel Peace Prize has gone to figures in
the Northern Ireland conflict. Betty Williams and Mairead Corrigan, two
Belfast women, won it in 1976 for mobilizing Catholics and Protestants
to stage rallies calling for an end to violence.

In Washington today, Mr. Clinton said ''how very pleased'' he was,
''personally and as President, that the Nobel Prize Committee has
rewarded the courage and the people of Northern Ireland by giving the
Nobel Peace Prize to John Hume and to David Trimble.'' He added ''a
special word of thanks'' to Mr. Mitchell, who issued a statement
praising Mr. Hume and Mr. Trimble as ''fully deserving of this honor.''

Mr. Hume, nominated for the Peace Prize twice before, is widely credited
with being the single most important influence for peace in Northern
Ireland. For 30 years he has spoken out against violence as a member of
a short-lived Northern Ireland assembly in the early 1970's and then as
a member of the British Parliament and the European Parliament. He has
made frequent trips to the United States to encourage investment in
Ulster and to dissuade Irish-Americans from furnishing money for the
I.R.A.

He was a founder of his party in 1970 to represent ''nationalist''
sentiment. In Northern Ireland, a nationalist is someone wanting closer
ties and even union with Ireland but, unlike republicans, unwilling to
countenance bloodshed to reach the goal.

Mr. Hume began the process that led to April's settlement five years
ago, with a decision to meet secretly with Mr. Adams, an outcast among
nationalists because of his identification with the I.R.A. Those talks
led to a cease-fire in 1993 that prompted hopes for an end to the
violence. Those hopes were dashed 17 months later when the I.R.A. set
off a bomb in London.

Though he said he was devastated by the act, Mr. Hume pressed on, and
his efforts produced a second cease-fire and brought Mr. Adams to the
negotiating table.

Mr. Trimble, by contrast, is a late-comer to accommodation with his
opponents. A former law professor, he began his political life as an
enemy of the party he now leads and an agitator against the last peace
settlement, the Sunningdale Agreement, which died in 1974.

As a member of the hard-line ''vanguard'' movement at that time, he
said, ''I would personally draw the line at violence and terrorism, but
if we are talking about a campaign that involves demonstrations and so
on, then a certain amount of violence may be inescapable.''

In 1995 Mr. Trimble, already a member of the British Parliament,
attracted much attention by leading Orange Order Protestant marchers at
Drumcree Church into a police formation set up to block their path down
an avenue in a Catholic neighborhood. When the parade successfully
reached the Orange Lodge hall in downtown Portadown, he danced a jig of
triumph with the Rev. Ian Paisley, the fulminating preacher-politician
with a history of fire-breathing speeches condemning Catholic advances.

This fall he met with Mr. Adams, an unthinkable act for a unionist
politician until now and one for which he was pilloried by dissidents in
his party. He now travels the province with Seamus Mallon, the deputy
speaker and a Catholic, by his side in an expression of cooperation.

Trimble Took 'A Political Chance'

''He has taken a political chance in identifying himself with the
process,'' Mr. Sejersted, the Nobel committee chairman, said today.

The peace talks began in the summer of 1996. They eventually drew the
participation of 8 of the 10 Northern Irish parties, with many of the
men around the table convicted murderers and bombers who had emerged
from prison with a commitment to peaceful resolution to what for nearly
a century have been referred to wearily as ''the Troubles.''

The paramilitary groups had also made the tactical decision that
violence would not secure their goals, a shared conviction that gave
these talks a chance for success that past fitful attempts at settlement
lacked.

The peace talks moved in a desultory manner until Mr. Blair took office
in May 1997 and highlighted the cause of peace in Northern Ireland as an
early commitment. At his and Mr. Ahern's urging, the I.R.A. declared a
cease-fire in July, and by September Sinn Fein was permitted to join the
talks.

Mr. Blair also gave Mr. Trimble and Mr. Adams unprecedented access to 10
Downing Street, and the Ulster Protestants reported that they obtained
from Mr. Clinton the most sympathetic hearing they ever had from an
American President, allaying their longtime suspicions of Washington's
bias in favor of the Catholic minority.

Advertisement

\protect\hyperlink{after-bottom}{Continue reading the main story}

\hypertarget{site-index}{%
\subsection{Site Index}\label{site-index}}

\hypertarget{site-information-navigation}{%
\subsection{Site Information
Navigation}\label{site-information-navigation}}

\begin{itemize}
\tightlist
\item
  \href{https://help.nytimes.com/hc/en-us/articles/115014792127-Copyright-notice}{©~2020~The
  New York Times Company}
\end{itemize}

\begin{itemize}
\tightlist
\item
  \href{https://www.nytco.com/}{NYTCo}
\item
  \href{https://help.nytimes.com/hc/en-us/articles/115015385887-Contact-Us}{Contact
  Us}
\item
  \href{https://www.nytco.com/careers/}{Work with us}
\item
  \href{https://nytmediakit.com/}{Advertise}
\item
  \href{http://www.tbrandstudio.com/}{T Brand Studio}
\item
  \href{https://www.nytimes.com/privacy/cookie-policy\#how-do-i-manage-trackers}{Your
  Ad Choices}
\item
  \href{https://www.nytimes.com/privacy}{Privacy}
\item
  \href{https://help.nytimes.com/hc/en-us/articles/115014893428-Terms-of-service}{Terms
  of Service}
\item
  \href{https://help.nytimes.com/hc/en-us/articles/115014893968-Terms-of-sale}{Terms
  of Sale}
\item
  \href{https://spiderbites.nytimes.com}{Site Map}
\item
  \href{https://help.nytimes.com/hc/en-us}{Help}
\item
  \href{https://www.nytimes.com/subscription?campaignId=37WXW}{Subscriptions}
\end{itemize}
