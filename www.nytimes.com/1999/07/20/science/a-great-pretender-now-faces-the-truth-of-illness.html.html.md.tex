Sections

SEARCH

\protect\hyperlink{site-content}{Skip to
content}\protect\hyperlink{site-index}{Skip to site index}

\href{https://www.nytimes.com/section/science}{Science}

\href{https://myaccount.nytimes.com/auth/login?response_type=cookie\&client_id=vi}{}

\href{https://www.nytimes.com/section/todayspaper}{Today's Paper}

\href{/section/science}{Science}\textbar{}A Great Pretender Now Faces
the Truth of Illness

\begin{itemize}
\item
\item
\item
\item
\item
\end{itemize}

Advertisement

\protect\hyperlink{after-top}{Continue reading the main story}

Supported by

\protect\hyperlink{after-sponsor}{Continue reading the main story}

\hypertarget{a-great-pretender-now-faces-the-truth-of-illness}{%
\section{A Great Pretender Now Faces the Truth of
Illness}\label{a-great-pretender-now-faces-the-truth-of-illness}}

By \href{https://www.nytimes.com/by/denise-grady}{Denise Grady}

\begin{itemize}
\item
  July 20, 1999
\item
  \begin{itemize}
  \item
  \item
  \item
  \item
  \item
  \end{itemize}
\end{itemize}

See the article in its original context from\\
July 20, 1999, Section F, Page
5\href{https://store.nytimes.com/collections/new-york-times-page-reprints?utm_source=nytimes\&utm_medium=article-page\&utm_campaign=reprints}{Buy
Reprints}

\href{http://timesmachine.nytimes.com/timesmachine/1999/07/20/677973.html}{View
on timesmachine}

TimesMachine is an exclusive benefit for home delivery and digital
subscribers.

Seated in her wheelchair on a terrace at the university hospital, Wendy
Scott gazed out at a thunderstorm that had blackened the noon skies and
tried to explain why she had spent 12 years of her life traveling from
one hospital to another in Britain and Europe, pretending to be ill so
that she would be admitted as a patient.

''To begin with, it was just something I did when I needed someone to
care about me,'' she said, speaking with the burr of her native
Scotland. ''Then it became something I had to do. It was as if something
took me over. I just had to be in hospital. I had to.''

Miss Scott, who is visiting the United States from her home in London,
says she has not faked her way into a hospital in 20 years. She is 50,
with streaks of gray in her hair and deep creases in her face that make
her look older than she is. This time, she belongs in the hospital,
because she is genuinely, gravely ill.

But her past still qualifies as one of the most severe examples ever
known of Munchausen syndrome, a rare psychiatric condition in which
people feign illness or make themselves sick because they crave medical
attention. By her own count, she became a patient at more than 600
hospitals, sometimes being released from one in the morning and getting
herself admitted to another by nightfall. Her portrayals of agonizing
stomach problems were so convincing that many doctors resorted to
surgery to help her, and she underwent 42 operations, nearly all of them
unnecessary. Her abdomen is criss-crossed with scars, leading a doctor
who examined her recently to comment that she looked as if she had lost
a duel with Zorro.

''She is the most extreme case I've ever heard of,'' said Dr. Marc
Feldman, vice chairman of the department of psychiatry at the University
of Alabama at Birmingham, and coauthor of ''Patient or Pretender: Inside
the Strange World of Factitious Disorders'' (John Wiley \& Sons, 1994),
about people who feign illness or induce it in themselves. ''When it
comes to number of hospitalizations and number of countries, I've never
heard of someone even close.'' Considering all the surgery and
complications she has had, he added, it is remarkable that she survived.

Miss Scott also stands out, he said, because she recovered from
Munchausen syndrome, which many doctors consider untreatable, and
because she is willing to speak openly about behavior that most patients
are too ashamed even to admit. Few doctors have had the chance to talk
to Munchausen patients about the condition, because they generally flee
when they are found out. Dr. Feldman said that among patients who seek
treatment for the syndrome, psychotherapy can sometimes help.

Given Miss Scott's phenomenal career as a liar, Dr. Feldman acknowledged
that she might be pulling his leg. But, he said, he did not really think
so. Her psychiatrist in England had confirmed the broad outlines of her
history, though not every hospital admission, and Dr. Feldman said he
had come to like and trust Miss Scott, and to believe that her goal in
revealing herself was to help others.

There have always been people who faked illness for one reason or
another, and for centuries doctors have known about such phony ailments,
called factitious disorders. In 1951 a group of factitious disorders was
named for Baron Karl Friedrich Hieronymus von Munchausen, a war hero who
traveled around Germany in the 18th century, telling tall stories about
his exploits. Medically, the name refers to people who wander from one
hospital to the next, telling tall stories about their illnesses and
seeking unnecessary treatment.

The syndrome differs from hypochondria, which is not a factitious
disorder: hypochondriacs think they are sick, whereas Munchausen
patients know they are not. And it is not the same as the factitious
disorder malingering, because malingerers play sick to get out of work,
obtain drugs or collect insurance, while Munchausen patients are in it
strictly for attention.

Munchausen syndrome is rare. Of all patients in the hospital, only 1
percent of those on whom psychiatrists are asked to consult have
factitious disorders, Dr. Feldman said, and 10 percent of those have
Munchausen syndrome. Some people with the syndrome, like Miss Scott,
simply make up stories and act out symptoms. Several have caused
commercial flights to make unscheduled landings so they could be rushed
to hospitals for emergency treatment.

Others go to extraordinary lengths to make themselves ill. They bleed
themselves, take laxatives, inject saliva or feces to cause infections
or secretly take powerful drugs like insulin that they do not need. They
have hidden syringes under their mattresses, dangled them outside
hospital windows on string or slipped them into the hems of their
clothing. Dr. Feldman said he encountered a patient who had introduced
drain cleaner into her bladder, where it did so much damage that the
bladder had to be removed. Another, described in his book, injected
yeast and cornstarch into her veins to make herself sick, but the
substances formed lumps that lodged in her lungs and killed her.

Psychiatrists say that Munchausen patients long for attention,
nurturing, care and concern, and do not know how to find them in healthy
ways. Some patients may also be a bit sadistic, Dr. Feldman said, and
take ''duping delight'' in tricking others, especially doctors.

''Doctors hate being played for fools,'' he said, adding that many
despise Munchausen patients, for manipulating them, wasting their time
and making them appear incompetent. Dealing with such patients, many
doctors feel they cannot win: if they withhold treatment, the patient
may sue, but if they offer treatment and it turns out to have been
unnecessary, the patient may still sue.

''Probably no other diagnosis is viewed as so contemptible,'' Dr.
Feldman said.

But Miss Scott said: ''I never intended to make doctors look stupid. I
just wanted to be in hospital.''

Her childhood, like that of many Munchausen patients, was rough, to say
the least. She was sexually abused, she said, and her mother was distant
and unaffectionate. She repeatedly ran away from home. One of her few
pleasant experiences was having her appendix out, when she was about 16.

''A nurse would come in the morning and plump up the pillows and say,
'How are you today, Wendy?' '' Miss Scott recalled. ''It was just little
things like that, asking how was your pain, how was your night.''

No one had ever shown her so much kindness or concern.

Soon after, she left home and took various jobs at a dairy, a bakery, a
spinning mill and a potato farm. Then she became a hotel maid,
''cleaning up everybody else's muck and being expected to act cheerful
24 hours a day,'' she said. The work was hard, depressing and lonely. It
seemed to her that no one in the world cared about her.

One day, she made believe she had a stomachache and went to the nearest
hospital. ''I thought, somebody will care,'' she said.

She spent several days there being tended to. ''It recharged my
batteries,'' she said.

Over the next year or so, she tried the tactic a few more times, at
different hospitals. It worked, and soon she was spending all her time
hitchhiking from town to town, trying to get into the hospital.

She was not close to anyone during those years. ''I didn't have
friends,'' she said. ''I didn't want anybody. If I had friends, they
might find out what I was doing.''

Eventually, doctors began suggesting exploratory operations to find out
what was wrong with her. She did not want surgery, she said, but went
along with it because it meant that she would be allowed to spend more
time in the hospital. She knew that what she was doing was wrong, she
said, but she could not make herself stop.

But she found that the people who had treated her so kindly could turn
very nasty indeed when they figured out what she was up to. One surgeon
stormed up to her bed, scolded her in front of the rest of the ward and
ordered her to get out. At a few hospitals, employees snapped her
picture to warn others about her.

Like most Munchausen patients, she simply ran away whenever she was
caught. A few times she fled with stitches still in place, and removed
them herself later. Twice, she was jailed, charged with stealing lodging
and food by checking into the hospital needlessly.

Looking back, she said, she might have asked for psychiatric help, ''if
somebody had said, 'I know what you're doing but I don't know why.' ''
But, she said, ''It was always approached in a confrontational way. The
barriers go up.''

After 12 years, two things finally helped her recover. The first was an
operation that went wrong, and left her ill and suffering from severe
infections and complications that took her months to recover from.

''I realized that if I kept doing this, I might die,'' she said.

The second step in her recovery was a kitten, left to her by someone in
a shelter she was living in. She realized that if she went into the
hospital, no one would take care of it.

Two and a half years ago, she contacted Dr. Feldman, who has a Web site
about Munchausen syndrome
(\href{http://www.munchausen.com}{www.munchausen.com}), and offered to
participate in any studies he might be conducting. But Dr. Feldman
thought she might be able to help other patients, and he put her in
touch with them by E-mail.

''A number said she made a decisive difference,'' he said. ''Not so much
what she said, but that nobody's case is as severe as hers and yet she
recovered.''

But now she has become a painful example of what can happen to a person
who is labeled as having cried wolf. She lives in London, and, even
though she has not lied about her health for 20 years, her notorious
record has made it difficult for her to get doctors there to take her
seriously when she is truly sick. Although she had suffered abdominal
pain and other symptoms for a year and a half, few tests were ordered
and no diagnosis was made in London. On a visit to the United States,
she called Dr. Feldman, who urged her to come to Birmingham, where
doctors discovered a large mass in her intestine that required immediate
surgery.

Last Friday, Miss Scott learned what the surgeons had found: a malignant
tumor, too large to remove. The news came as a shock, and she wept. She
wondered bitterly what her outlook might have been had British doctors
paid attention to her when she first described her symptoms.

''Once you've been branded,'' she said, ''it's like you've got it
written across your forehead: 'Not to be trusted. Munchausen.' ''

She was frightened. ''I'm not really ready to die yet,'' she said.
''There are too many things I want to do.'' She had hoped to fly first
class sometime, to see New Zealand, to drive her car down the open road
once again.

Now, she said: ''I suppose I wonder what the future holds. I wonder if
there is any future.''

Advertisement

\protect\hyperlink{after-bottom}{Continue reading the main story}

\hypertarget{site-index}{%
\subsection{Site Index}\label{site-index}}

\hypertarget{site-information-navigation}{%
\subsection{Site Information
Navigation}\label{site-information-navigation}}

\begin{itemize}
\tightlist
\item
  \href{https://help.nytimes.com/hc/en-us/articles/115014792127-Copyright-notice}{©~2020~The
  New York Times Company}
\end{itemize}

\begin{itemize}
\tightlist
\item
  \href{https://www.nytco.com/}{NYTCo}
\item
  \href{https://help.nytimes.com/hc/en-us/articles/115015385887-Contact-Us}{Contact
  Us}
\item
  \href{https://www.nytco.com/careers/}{Work with us}
\item
  \href{https://nytmediakit.com/}{Advertise}
\item
  \href{http://www.tbrandstudio.com/}{T Brand Studio}
\item
  \href{https://www.nytimes.com/privacy/cookie-policy\#how-do-i-manage-trackers}{Your
  Ad Choices}
\item
  \href{https://www.nytimes.com/privacy}{Privacy}
\item
  \href{https://help.nytimes.com/hc/en-us/articles/115014893428-Terms-of-service}{Terms
  of Service}
\item
  \href{https://help.nytimes.com/hc/en-us/articles/115014893968-Terms-of-sale}{Terms
  of Sale}
\item
  \href{https://spiderbites.nytimes.com}{Site Map}
\item
  \href{https://help.nytimes.com/hc/en-us}{Help}
\item
  \href{https://www.nytimes.com/subscription?campaignId=37WXW}{Subscriptions}
\end{itemize}
