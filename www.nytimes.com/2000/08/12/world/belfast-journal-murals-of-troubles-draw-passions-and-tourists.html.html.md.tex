Sections

SEARCH

\protect\hyperlink{site-content}{Skip to
content}\protect\hyperlink{site-index}{Skip to site index}

\href{https://www.nytimes.com/section/world}{World}

\href{https://myaccount.nytimes.com/auth/login?response_type=cookie\&client_id=vi}{}

\href{https://www.nytimes.com/section/todayspaper}{Today's Paper}

\href{/section/world}{World}\textbar{}Belfast Journal; Murals of
'Troubles' Draw Passions, and Tourists

\begin{itemize}
\item
\item
\item
\item
\item
\end{itemize}

Advertisement

\protect\hyperlink{after-top}{Continue reading the main story}

Supported by

\protect\hyperlink{after-sponsor}{Continue reading the main story}

\hypertarget{belfast-journal-murals-of-troubles-draw-passions-and-tourists}{%
\section{Belfast Journal; Murals of 'Troubles' Draw Passions, and
Tourists}\label{belfast-journal-murals-of-troubles-draw-passions-and-tourists}}

By \href{https://www.nytimes.com/by/dan-barry}{Dan Barry}

\begin{itemize}
\item
  Aug. 12, 2000
\item
  \begin{itemize}
  \item
  \item
  \item
  \item
  \item
  \end{itemize}
\end{itemize}

See the article in its original context from\\
August 12, 2000, Section A, Page
4\href{https://store.nytimes.com/collections/new-york-times-page-reprints?utm_source=nytimes\&utm_medium=article-page\&utm_campaign=reprints}{Buy
Reprints}

\href{http://timesmachine.nytimes.com/timesmachine/2000/08/12/640697.html}{View
on timesmachine}

TimesMachine is an exclusive benefit for home delivery and digital
subscribers.

Along the way, the tour guide pointed out some of the more benign
landmarks in this city of Old World charm and fairly recent terror. To
the right was the ancient Crown Liquor Saloon; farther on, the
gargantuan City Hall. And there, listing a bit, the Albert Clocktower.

Then Norman Reilly aimed his puttering black taxi west toward the
central destinations of his unusual tour: the Shankill Road and Falls
Road neighborhoods that were the center of the sectarian violence that
held this city hostage for the last 30 years. The bloodshed has largely
subsided, but a whiff of ''the troubles'' remains -\/- and tourists
excitedly breathe it in. Especially the murals.

''We give them options, and most times they want to see the murals,''
said Mr. Reilly, a driver for a popular tourism operation called Black
Taxi Tours.

As peace becomes something more than a breath-holding pause here, the
demand for city tours is surging, thanks in part to an explosion of
provocative murals in the Protestant neighborhood called Shankill Estate
and, to a lesser extent, in the Catholic community along the Falls Road.

A muse for some of this artistic revival is Johnny Adair, a Protestant
guerrilla also known as ''Mad Dog,'' who was among the dozens of former
paramilitaries released from jail in recent months under the terms of
the 1998 peace agreement and who now presents himself as the guardian of
Shankill spirit.

''We might even see Johnny,'' Mr. Reilly said as the buildings of
Belfast blurred past his window.

Northern Ireland is gradually adjusting to a peace agreement that
struggled to establish a new government and has helped to perpetuate
cease-fires declared by gunmen on both sides of the province's divide:
the mostly Catholic Republicans, who want Ulster united with the rest of
Ireland, and the mostly Protestant loyalists, who want it to remain part
of the United Kingdom.

Part of the story of the peace effort is unfolding on the walls of
Belfast. The sides of the city's buildings have long served as canvases
for those wanting to honor the dead or threaten the living. And for
several years, tour guides have been carrying visitors to see the curbs
and street lamps painted in the colors of division: the Union Jack
colors of red, white and blue for the Protestant neighborhoods, the
Irish colors of green, white and gold for the Catholic.

But now the artwork seems particularly intended to draw the gapes of
tourists as much as to stir the emotions of residents. ''They're painted
on the walls now for others to see as well,'' Mr. Reilly said.

And with the number of visitors to the city gradually increasing to 1.6
million last year, the demand for tours has gone up, said Michael
Johnston, 27, who started Black Taxi Tours five years ago.

''The first couple of years it was on and off because of the troubles
and the bomb attacks,'' he said. ''At the time, I was the only one. But
because there's a cease-fire on both sides, there's a bunch of black
taxi crews jumping on the bandwagon.''

The interest makes sense, said Orla Farren, a spokeswoman for the
Northern Ireland Tourist Board. ''The political graffiti is a very
symbolic reminder of what's been going on in recent history.''

Mr. Reilly turned his taxi into Shankill Estate, its curbstones and
street lamps freshly painted, the light blue flags of the Ulster Freedom
Fighters -\/- the outlawed loyalist group for which Mr. Adair once
served as a commando -\/- fluttering from every pole and post.

''They're a terrorist group, a paramilitary group,'' Mr. Reilly said
matter-of-factly. ''It's a way of letting people know it's a U.F.F.
neighborhood.''

Even more eye-catching were the new murals. One glorified the Ulster
Freedom Fighters by depicting three masked gunmen. Another honored Billy
Wright, a loyalist commando killed by the Irish Republican Army in
prison in 1997. A third showed a gun-toting Grim Reaper, standing over
the marked graves of three I.R.A. members still living.

One bore the name of Sean Kelly, another guerrilla recently released
from prison, who bombed a fish shop on the Shankill Road in 1993 with
the intention of killing Mr. Adair and other loyalist leaders. It killed
nine civilians instead, including a friend of Mr. Reilly's.

Mr. Reilly, 29, is a Protestant from Shankill. Like many who grew up in
the strongest Protestant or Catholic areas of Ulster's cities, he can
remember hearing the cry ''Riot!'' ring through the Shankill streets,
and sprinting to watch the brawls between Protestants and Catholics. But
he said he never did anything more than a bit of rock-throwing, avoiding
the paramilitary groups whose ranks swelled with his peers.

After years of doing ''this and that,'' Mr. Reilly said, he became a
taxi driver who found that his close-up experiences had a value in the
marketplace. He eventually joined Black Taxi Tours, whose owner, Mr.
Johnston, said he has purposely employed five Protestant and five
Catholic drivers.

''We all get along together,'' he said. ''And none of us have a
political criminal record.''

Suddenly, Mr. Reilly pointed to a short man with biceps the size of ham
hocks and a shaved head that glistened in the late-afternoon sun.

''There's Johnny.''

Mr. Adair, 36, was most recently in the public eye when he, his dog,
Rebel, and dozens of associates all showed up in Ulster Freedom Fighters
T-shirts last month in Portadown, the town 30 miles from Belfast that
has become the site of an annual showdown over the government's refusal
to allow Protestants to hold their traditional march through a Catholic
neighborhood.

The government is now investigating whether Mr. Adair provoked any of
this year's mayhem, which lasted several days.

Mr. Adair rattled some keys and said he had no time to chat, then
chatted anyway, explaining that several groups, including the Ulster
Freedom Fighters, were working to brighten up the working-class
neighborhood in a way that would ''identify our Britishness and identify
who we are.''

Nineteen murals are planned for a formal unveiling on Aug. 19, he said.
The festivities will include a huge bonfire, which explained why there
were stacks of wood on the estate's grassy center.

''It'll be a fun day for kids,'' he said. ''And there's a beer tent.''

He agreed that the Grim Reaper mural conveyed a threat -\/- ''It says it
all; comin' to get you'' -\/- but added that he had just supported the
removal of another, more offensive mural along the Shankhill Road. He
also pointed out that there was graffiti in Catholic neighborhoods that
threatened his life.

''Sticks and stones can break my bones, but words will never harm me,''
he said. ''It's the bullets I'm worried about.''

Mr. Adair has survived several assassination attempts and is always on
guard -\/- not just against his traditional enemies in the I.R.A. but
fellow Protestant paramilitaries, too. He displayed a bullet scar on his
back and one on his skull, the latter inflicted by a bullet fired at him
in a Belfast park last year, shortly after his release from jail.

''Just there,'' he said, rubbing the area with his finger. ''Point-blank
range.''

Mr. Reilly, who said he takes tourists along this route about four times
a day, drove out of the Shankill Estate and into the working-class
Catholic neighborhood called Falls Road. One mural depicted a British
soldier under the Gaelic phrase for ''Safe Home''; another showed a dove
carrying a British soldier away, along with the caption, ''Time for
Peace, Time to Go.''

As he drove up the Falls Road, Mr. Reilly pointed out the black flags
hanging from lampposts to honor Bobby Sands, an I.R.A. bomber who died
in 1981 after holding a hunger strike in the Maze prison. He then pulled
into Milltown Cemetery, where, in 1988, a Protestant guerrilla named
Michael Stone attacked an I.R.A. funeral, killing three mourners. Mr.
Stone was freed last month as part of the prisoner-release program, but
his attack -\/- filmed by television cameras -\/- remains a singular
moment.

''It's something that people cannot forget on both sides,'' Mr. Reilly
said.

A little farther on, he pointed out a fresh mural commemorating 17
people killed by plastic bullets used by British forces.

''It's all Catholics on it, no Protestants,'' Mr. Reilly said. ''It's
just them getting at the police.''

Another tour had ended; it was time for Mr. Reilly to go home, where the
only murals on display are the tattoos that festoon his arms -\/-
including one of a little red devil that bears the name of his young
son. And when the bonfire roars later this month in the Shankill Estate,
Mr. Reilly said, he and his family will be on vacation, out of the
country.

Advertisement

\protect\hyperlink{after-bottom}{Continue reading the main story}

\hypertarget{site-index}{%
\subsection{Site Index}\label{site-index}}

\hypertarget{site-information-navigation}{%
\subsection{Site Information
Navigation}\label{site-information-navigation}}

\begin{itemize}
\tightlist
\item
  \href{https://help.nytimes.com/hc/en-us/articles/115014792127-Copyright-notice}{©~2020~The
  New York Times Company}
\end{itemize}

\begin{itemize}
\tightlist
\item
  \href{https://www.nytco.com/}{NYTCo}
\item
  \href{https://help.nytimes.com/hc/en-us/articles/115015385887-Contact-Us}{Contact
  Us}
\item
  \href{https://www.nytco.com/careers/}{Work with us}
\item
  \href{https://nytmediakit.com/}{Advertise}
\item
  \href{http://www.tbrandstudio.com/}{T Brand Studio}
\item
  \href{https://www.nytimes.com/privacy/cookie-policy\#how-do-i-manage-trackers}{Your
  Ad Choices}
\item
  \href{https://www.nytimes.com/privacy}{Privacy}
\item
  \href{https://help.nytimes.com/hc/en-us/articles/115014893428-Terms-of-service}{Terms
  of Service}
\item
  \href{https://help.nytimes.com/hc/en-us/articles/115014893968-Terms-of-sale}{Terms
  of Sale}
\item
  \href{https://spiderbites.nytimes.com}{Site Map}
\item
  \href{https://help.nytimes.com/hc/en-us}{Help}
\item
  \href{https://www.nytimes.com/subscription?campaignId=37WXW}{Subscriptions}
\end{itemize}
