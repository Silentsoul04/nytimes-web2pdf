Sections

SEARCH

\protect\hyperlink{site-content}{Skip to
content}\protect\hyperlink{site-index}{Skip to site index}

\href{https://www.nytimes.com/section/us}{U.S.}

\href{https://myaccount.nytimes.com/auth/login?response_type=cookie\&client_id=vi}{}

\href{https://www.nytimes.com/section/todayspaper}{Today's Paper}

\href{/section/us}{U.S.}\textbar{}BUSH PREVAILS; BY SINGLE VOTE,
JUSTICES END RECOUNT, BLOCKING GORE AFTER 5-WEEK STRUGGLE

\begin{itemize}
\item
\item
\item
\item
\item
\end{itemize}

Advertisement

\protect\hyperlink{after-top}{Continue reading the main story}

Supported by

\protect\hyperlink{after-sponsor}{Continue reading the main story}

BUSH PREVAILS

\hypertarget{bush-prevails-by-single-vote-justices-end-recount-blocking-gore-after-5-week-struggle}{%
\section{BUSH PREVAILS; BY SINGLE VOTE, JUSTICES END RECOUNT, BLOCKING
GORE AFTER 5-WEEK
STRUGGLE}\label{bush-prevails-by-single-vote-justices-end-recount-blocking-gore-after-5-week-struggle}}

By \href{https://www.nytimes.com/by/linda-greenhouse}{Linda Greenhouse}

\begin{itemize}
\item
  Dec. 13, 2000
\item
  \begin{itemize}
  \item
  \item
  \item
  \item
  \item
  \end{itemize}
\end{itemize}

See the article in its original context from\\
December 13, 2000, Section A, Page
1\href{https://store.nytimes.com/collections/new-york-times-page-reprints?utm_source=nytimes\&utm_medium=article-page\&utm_campaign=reprints}{Buy
Reprints}

\href{http://timesmachine.nytimes.com/timesmachine/2000/12/13/667188.html}{View
on timesmachine}

TimesMachine is an exclusive benefit for home delivery and digital
subscribers.

The Supreme Court effectively handed the presidential election to George
W. Bush tonight, overturning the Florida Supreme Court and ruling by a
vote of 5 to 4 that there could be no further counting of Florida's
disputed presidential votes.

The ruling came after a long and tense day of waiting at 10 p.m., just
two hours before the Dec. 12 ''safe harbor'' for immunizing a state's
electors from challenge in Congress was to come to an end. The unsigned
majority opinion said it was the immediacy of this deadline that made it
impossible to come up with a way of counting the votes that could both
meet ''minimal constitutional standards'' and be accomplished within the
deadline.

The five members of the majority were Chief Justice William H. Rehnquist
and Justices Sandra Day O'Connor, Antonin Scalia, Anthony M. Kennedy and
Clarence Thomas.

Among the four dissenters, two justices, Stephen G. Breyer and David H.
Souter, agreed with the majority that the varying standards in different
Florida counties for counting the punch-card ballots presented problems
of both due process and equal protection. But unlike the majority, these
justices said the answer should be not to shut the recount down, but to
extend it until the Dec. 18 date for the meeting of the Electoral
College.

Justice Souter said that such a recount would be a ''tall order'' but
that ''there is no justification for denying the state the opportunity
to try to count all the disputed ballots now.'' {[}Text, Page A27.{]}

The six separate opinions, totaling 65 pages, were filled with evidence
that the justices were acutely aware of the controversy the court had
entered by accepting Governor Bush's appeal of last Friday's Florida
Supreme Court ruling and by granting him a stay of the recount on
Saturday afternoon, just hours after the vote counting had begun.

''None are more conscious of the vital limits on judicial authority than
are the members of this court,'' the majority opinion said, referring to
''our unsought responsibility to resolve the federal and constitutional
issues the judicial system has been forced to confront.''

The dissenters said nearly all the objections raised by Mr. Bush were
insubstantial. The court should not have reviewed either this case or
the one it decided last week, they said.

Justice John Paul Stevens said the court's action ''can only lend
credence to the most cynical appraisal of the work of judges throughout
the land.''

His dissenting opinion, also signed by Justices Breyer and Ruth Bader
Ginsburg, added: ''It is confidence in the men and women who administer
the judicial system that is the true backbone of the rule of law. Time
will one day heal the wound to that confidence that will be inflicted by
today's decision. One thing, however, is certain. Although we may never
know with complete certainty the identity of the winner of this year's
Presidential election, the identity of the loser is perfectly clear. It
is the nation's confidence in the judge as an impartial guardian of the
rule of law.''

What the court's day and a half of deliberations yielded tonight was a
messy product that bore the earmarks of a failed attempt at a compromise
solution that would have permitted the vote counting to continue.

It appeared that Justices Souter and Breyer, by taking seriously the
equal protection concerns that Justices Kennedy and O'Connor had raised
at the argument, had tried to persuade them that those concerns could be
addressed in a remedy that would permit the disputed votes to be
counted.

Justices O'Connor and Kennedy were the only justices whose names did not
appear separately on any opinion, indicating that one or both of them
wrote the court's unsigned majority opinion, labelled only ''per
curiam,'' or ''by the court.'' Its focus was narrow, limited to the
ballot counting process itself. The opinion objected not only to the
varying standards used by different counties for determining voter
intent, but to aspects of the Florida Supreme Court's order determining
which ballots should be counted.

''We are presented with a situation where a state court with the power
to assure uniformity has ordered a statewide recount with minimal
procedural safeguards,'' the opinion said. ''When a court orders a
statewide remedy, there must be at least some assurance that the
rudimentary requirements of equal treatment and fundamental fairness are
satisfied.''

Three members of the majority -\/- the Chief Justice, and Justices
Scalia and Thomas -\/- raised further, more basic objections to the
recount and said the Florida Supreme Court had violated state law in
ordering it.

The fact that Justices O'Connor and Kennedy evidently did not share
these deeper concerns had offered a potential basis for a coalition
between them and the dissenters. That effort apparently foundered on the
two justices' conviction that the midnight deadling of Dec. 12 had to be
met.

The majority said that ''substantial additional work'' was needed to
undertake a constitutional recount, including not only uniform statewide
standards for determining a legal vote, but also ''practical procedures
to implement them'' and ''orderly judicial review of any disputed
matters that might arise.'' There was no way all this could be done, the
majority said.

The dissenters said the concern with Dec. 12 was misplaced. Justices
Souter and Breyer offered to send the case back to the Florida courts
''with instructions to establish uniform standards for evaluating the
several types of ballots that have prompted differing treatments,'' as
Justice Souter described his proposed remand order. He added: ''unlike
the majority, I see no warrant for this court to assume that Florida
could not possibly bomply with this requirement before the date set for
the meeting of electors, Dec. 18.''

Justices Stevens and Ginsburg said they did not share the view that the
lack of a uniform vote-counting standard presented an equal protection
problem.

In addition to joining Justice Souter's dissenting opinion, Justice
Breyer wrote one of his own, signed by the three other dissenters, in
which he recounted the history of the deadlocked presidential election
of 1876 and of the partisan role that one Supreme Court justice, Joseph
P. Bradley, played in awarding the presidency to Rutherford B. Hayes.

''This history may help to explain why I think it not only legally
wrong, but also most unfortunate, for the Court simply to have
terminated the Florida recount,'' Justice Breyer said. He said the time
problem that Florida faced was ''in significant part, a problem of the
Court's own making.'' The recount was moving ahead in an ''orderly
fashion,'' Justice Breyer said, when ''this court improvidently entered
a stay.'' He said: ''As a result, we will never know whether the recount
could have been completed.''

There was no need for the court to have involved itself in the election
dispute this time, he said, adding: ''Above all, in this highly
politicized matter, the appearance of a split decision runs the risk of
undermining the public's confidence in the court itself. That confidence
is a public treasure. It has been built slowly over many years, some of
which were marked by a Civil War and the tragedy of segregation. It is a
vitally necessary ingredient of any successful effort to protect basic
liberty and, indeed, the rule of law itself.''

''We do risk a self-inflicted wound,'' Justice Breyer said, ''a wound
that may harm not just the court, but the nation.''

Justice Ginsburg also wrote a dissenting opinion, joined by the other
dissenters. Her focus was on the implications for federalism of the
majority's action. ''I might join the chief justice were it my
commission to interpret Florida law,'' she said, adding: ''The
extraordinary setting of this case has obscured the ordinary principle
that dictates its proper resolution: federal courts defer to state high
courts' interpretations of their state's own law. This principle
reflects the core of federalism, on which all agree.''

''Were the other members of this court as mindful as they generally are
of our system of dual sovereignty,'' Justice Ginsburg concluded, ''they
would affirm the judgment of the Florida Supreme Court.''

Unlike the other dissenters, who said they dissented ''respectfully,''
Justice Ginsburg said only: ''I dissent.''

Nothing about this case, Bush v. Gore, No. 00-949, was ordinary: not its
context, not its acceptance over the weekend, not the enormously
accelerated schedule with argument on Monday, and not the way the
decision was released to the public tonight.

When the court issues an opinion, the justices ordinarily take the bench
and the justice who has written for the majority gives a brief oral
description of the case and the holding.

Today, after darkness fell and their work was done, the justices left
the Supreme Court building individually from the underground garage,
with no word to dozens of journalists from around the world who were
waiting in the crowded pressroom for word as to when, or whether, a
decision might come. By the time the pressroom staff passed out copies
of the decision, the justices were gone.

Advertisement

\protect\hyperlink{after-bottom}{Continue reading the main story}

\hypertarget{site-index}{%
\subsection{Site Index}\label{site-index}}

\hypertarget{site-information-navigation}{%
\subsection{Site Information
Navigation}\label{site-information-navigation}}

\begin{itemize}
\tightlist
\item
  \href{https://help.nytimes.com/hc/en-us/articles/115014792127-Copyright-notice}{©~2020~The
  New York Times Company}
\end{itemize}

\begin{itemize}
\tightlist
\item
  \href{https://www.nytco.com/}{NYTCo}
\item
  \href{https://help.nytimes.com/hc/en-us/articles/115015385887-Contact-Us}{Contact
  Us}
\item
  \href{https://www.nytco.com/careers/}{Work with us}
\item
  \href{https://nytmediakit.com/}{Advertise}
\item
  \href{http://www.tbrandstudio.com/}{T Brand Studio}
\item
  \href{https://www.nytimes.com/privacy/cookie-policy\#how-do-i-manage-trackers}{Your
  Ad Choices}
\item
  \href{https://www.nytimes.com/privacy}{Privacy}
\item
  \href{https://help.nytimes.com/hc/en-us/articles/115014893428-Terms-of-service}{Terms
  of Service}
\item
  \href{https://help.nytimes.com/hc/en-us/articles/115014893968-Terms-of-sale}{Terms
  of Sale}
\item
  \href{https://spiderbites.nytimes.com}{Site Map}
\item
  \href{https://help.nytimes.com/hc/en-us}{Help}
\item
  \href{https://www.nytimes.com/subscription?campaignId=37WXW}{Subscriptions}
\end{itemize}
