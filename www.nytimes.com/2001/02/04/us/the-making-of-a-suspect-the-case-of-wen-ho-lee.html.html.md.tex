Sections

SEARCH

\protect\hyperlink{site-content}{Skip to
content}\protect\hyperlink{site-index}{Skip to site index}

\href{https://www.nytimes.com/section/us}{U.S.}

\href{https://myaccount.nytimes.com/auth/login?response_type=cookie\&client_id=vi}{}

\href{https://www.nytimes.com/section/todayspaper}{Today's Paper}

\href{/section/us}{U.S.}\textbar{}The Making of a Suspect: The Case of
Wen Ho Lee

\begin{itemize}
\item
\item
\item
\item
\item
\end{itemize}

Advertisement

\protect\hyperlink{after-top}{Continue reading the main story}

Supported by

\protect\hyperlink{after-sponsor}{Continue reading the main story}

\hypertarget{the-making-of-a-suspect-the-case-of-wen-ho-lee}{%
\section{The Making of a Suspect: The Case of Wen Ho
Lee}\label{the-making-of-a-suspect-the-case-of-wen-ho-lee}}

By Matthew Purdy

\begin{itemize}
\item
  Feb. 4, 2001
\item
  \begin{itemize}
  \item
  \item
  \item
  \item
  \item
  \end{itemize}
\end{itemize}

See the article in its original context from\\
February 4, 2001, Section 1, Page
1\href{https://store.nytimes.com/collections/new-york-times-page-reprints?utm_source=nytimes\&utm_medium=article-page\&utm_campaign=reprints}{Buy
Reprints}

\href{http://timesmachine.nytimes.com/timesmachine/2001/02/04/760501.html}{View
on timesmachine}

TimesMachine is an exclusive benefit for home delivery and digital
subscribers.

The crime sounded alarming: China had stolen the design of America's
most advanced nuclear weapon. The suspect seemed suspicious enough: Wen
Ho Lee, a Taiwanese-born scientist at Los Alamos nuclear laboratory, had
a history of contact with Chinese scientists and a record of deceiving
the authorities on security matters.

After a meandering five-year investigation, Dr. Lee was incarcerated and
interrogated, shackled and polygraphed, and all but threatened with
execution by a federal agent for not admitting spying. But prosecutors
were never able to connect him to espionage. They discovered that he had
downloaded a mountain of classified weapons information, but he was
freed last September after pleading guilty to one felony count of
mishandling secrets. Ultimately, the case of Wen Ho Lee was a spy story
in which the most tantalizing mystery was whether the central character
ever was a spy.

In the aftermath, the government was roundly criticized for its handling
of the case; so was the press, especially The New York Times. In an
effort to untangle this convoluted episode, The Times undertook an
extensive re-examination of the case, interviewing participants and
examining scientific and government documents, many containing secrets
never before disclosed.

This review showed how, in constructing a narrative to fit their
unnerving suspicions, investigators took fragmentary, often ambiguous
evidence about Dr. Lee's behavior and Chinese atomic espionage and wove
it into a grander case that eventually collapsed of its own light
weight.

Before the criminal investigation began, weapons experts consulted by
the government concluded that stolen American secrets had helped China
improve its nuclear weapons, according to inside accounts of the
experts' meetings. They also said the Chinese wanted to replicate key
elements of America's most sophisticated warhead, the W-88, and had
obtained some secrets about it. However, most of the experts agreed that
those secrets were rudimentary, and that there was no evidence China had
built anything like the W-88.

But in the echo chamber of Washington, that measured scientific finding
was distorted and amplified as it bounced from intelligence analysts to
criminal investigators to elected officials, most of them ill equipped
to deal with the atomic complexities at the heart of the matter.
Eventually, the notion that the Chinese had swiped the W-88 design
became the accepted wisdom.

Investigators made Dr. Lee their prime suspect in the W-88 case even
though they had no evidence he had leaked weapons secrets. Unanswered
questions about his contacts with foreign scientists had made him
suspect, but as it searched for a spy, the Federal Bureau of
Investigation ignored the urging of a senior agent on the case to look
beyond Dr. Lee. As a result, it failed to examine hundreds, if not
thousands, of people outside Los Alamos who had access to the stolen
information about the W-88.

When the government's case fizzled, Wen Ho Lee went from public enemy
No. 1 to public victim No. 1. But the new label seemed no more
appropriate than the first. Off and on for two decades, Dr. Lee's
behavior was curious, if not criminal.

He had a knack for wandering into circumstances that aroused suspicion.
In 1982, he had a walk-on role in a major espionage investigation, when
he inexplicably offered to help the suspect, whom he apparently did not
even know. In 1994, Dr. Lee surprised laboratory officials when he
appeared uninvited at a Los Alamos briefing for visiting Chinese
scientists and warmly greeted China's leading bomb designer.

As the investigation unfolded, Dr. Lee, 61, began revealing details of
his contacts with Chinese scientists, including one encounter he had
improperly hidden from laboratory officials. Dr. Lee, it turned out, had
met the bomb designer in a Beijing hotel room years before.

Eventually, Dr. Lee fit perfectly into agents' portrait of a scientist
being recruited as a spy by China.

The government's pursuit was as erratic as its quarry. The investigation
was so low-key at times that Dr. Lee was allowed to travel overseas
unmonitored at least twice. But after the download was discovered, the
government imprisoned him for nine months by arguing that his freedom
could threaten the global nuclear balance. Prosecutors charged him with
crimes that carry potential life sentences, even though they had only
circumstantial evidence to support the charges.

The case, like so much in the world of espionage, was a haze of
ambiguity, in which everything from intelligence data to Dr. Lee's
activities was subject to interpretation. Often what mattered was who
the interpreters were, and what perspective they brought.

The case was first framed by Notra Trulock III, a Soviet analyst during
the cold war who had become director of intelligence at the Energy
Department, which maintains the nation's nuclear arsenal and runs its
weapons laboratories. His influence was magnified because much of the
government infrastructure that provided nuclear intelligence at the
height of the arms race had fallen into disrepair.

As the case passed to the F.B.I., it acquired a classic cold war plot:
spy for competing superpower steals blueprints for America's premier
bomb. But this was a different, more complex story.

The other country was not Russia but China. And while Washington and
Beijing had hardly become allies, their nuclear scientists were meeting
regularly and sharing research. That gave China the opportunity to spy
the way experts say it prefers to, mining nuggets from countless
foreigners bearing secret knowledge rather than relying on a few master
spies.

The case of Wen Ho Lee was propelled by the divisive politics of
Clinton-era Washington. It languished for several years, only to be
revived in 1998 by a confluence of forces -\/- a White House under siege
of impeachment, festering accusations of Chinese money funneled to
Democratic campaigns and a House panel that saw the W-88 case as only
the newest evidence of China's voracious appetite for American
technology secrets.

The spying charges gained wide public attention on March 6, 1999, after
The Times reported that China possessed ''nuclear secrets stolen from an
American government laboratory,'' and that American experts believed
Beijing had tested a weapon ''configured remarkably like the W-88.''
Descriptions of the espionage escalated rapidly. Two months later, the
chairman of the House panel, Christopher Cox, Republican of California,
wrote publicly that the Chinese had a ''knockoff version of the world's
most sophisticated nuclear design.''

Today, the crime, whatever its extent, remains unsolved, the spy, or
spies, unidentified. In its long pursuit of Wen Ho Lee, the government
was driven by fear that he had given up the nation's deepest atomic
secrets. The one secret he most certainly never gave up was himself.

STARTING OUT

Wen Ho Lee arrived at Los Alamos in 1978 and joined the bomb-design unit
two years later. It was a time of growing scientific cooperation between
China and the United States.

In a tale laced with cross-cultural subtleties, the arcana of atomic
science and the feints of the intelligence world, the most
indecipherable character is the man at the center.

In part, Wen Ho Lee is an immigrant striver, one of 10 children of poor,
uneducated farmers whose roots traced to Fujian province in China,
across the strait from Taiwan.

Dr. Lee's childhood was an adventure of swimming and fishing and
catching monkeys for pets in bamboo forests. But it was also hard,
according to relatives and information Dr. Lee provided through his
lawyers. (Dr. Lee declined several requests for interviews.) While Dr.
Lee was in high school, his mother committed suicide after years of
declining health; his father died after a stroke a few years later.

The Lees lived through the Japanese colonization of Taiwan and the
martial law of the Nationalists, who detained intellectuals suspected of
subversive activity. Lee Tse-ling, Dr. Lee's nephew and a doctor in
Taiwan, said the lesson the family took from these experiences was,
''Don't get involved in politics.''

Mathematics was Wen Ho Lee's ticket out. He studied mechanical
engineering at Cheng Kung University and then came to the United States
in 1964, earning a doctorate in mechanical engineering from Texas A \& M
in 1970. His English was heavily accented, but he embraced things
American, from Aggie football to his blue Mustang. In 1974, he became a
United States citizen.

Dr. Lee, his wife and two children got to Los Alamos, N.M., in 1978, and
two years later he joined X Division, the bomb-design unit. As a
specialist in hydrodynamics, he wrote computer codes that model the
fluidlike movement of explosions. The codes help scientists design bombs
and simulate weapons tests.

Los Alamos is typically suburban, with sizable homes, good schools, low
crime. But it is also a place apart, a spectacular mountaintop village
anointed as science headquarters of the Manhattan Project in the 1940's.
Streets named Trinity Drive and Bikini Road commemorate bomb tests, and
a gift shop sells \$13.50 pewter key chains of Fat Man and Little Boy,
the bombs dropped on Japan.

In those early days, Dr. Lee stood out. The local Chinese community was
so tiny ''everybody knew everybody,'' recalled Cecilia Chang, a friend
who became a vocal supporter of Dr. Lee. The Lees' house in White Rock,
just outside Los Alamos, was an ethnic oasis where Dr. Lee offered
Chinese meals made from homegrown vegetables and fish he caught.

When Dr. Lee arrived, the laboratory was assuming an important role in
the changing relationship between the United States and China. Exchanges
between the two countries' nuclear scientists had begun soon after
President Jimmy Carter officially recognized China in 1978. They were
extraordinary at first, given the secrecy shrouding America's weapons
laboratories. But eventually, with the Reagan administration eager to
isolate the Soviet Union, hundreds of scientists traveled between the
United States and China, and the cooperation expanded to the development
of torpedoes, artillery shells and jet fighters.

The exchanges were spying opportunities as well.

''In 1979, we knew virtually nothing'' about China's nuclear program,
said George A. Keyworth II, who was Ronald Reagan's science adviser.
''By 1981, we knew a large fraction of the strategic intelligence, the
big questions.''

China was spying, too. Shortly after the exchanges started, the F.B.I.
began an espionage investigation code-named Tiger Trap, which focused on
a Taiwanese-American nuclear scientist at the government's Lawrence
Livermore laboratory in California. Agents were wiretapping the
scientist's phone, and on Dec. 3, 1982, the tap picked up Wen Ho Lee
offering to help find out who had ''squealed on'' him.

Dr. Lee's first encounter with investigators set a pattern for the
future. When confronted, he said he had not known the scientist and had
not tried to contact him; he confessed only when presented with evidence
of his call, according to government records and Congressional
testimony. Then he told investigators that he thought the suspect was in
trouble for passing unclassified information. Dr. Lee said he was
concerned because he himself had been giving Taiwanese officials
unclassified documents that American officials say dealt with
nuclear-reactor safety.

According to a secret F.B.I. report recently obtained by The Times, Dr.
Lee told agents that he had not informed American government officials,
''even though the documents he passed specifically stated they were not
for foreign dissemination.''

The report continued, ''Wen Ho Lee stated that his motive for sending
the publications was brought on out of a desire to help in scientific
exchange.'' Dr. Lee also said ''he helps other scientists routinely and
had no desire to receive any monetary or other type of reward from
Taiwan.''

Dr. Lee's call could be viewed as a simple overture to a fellow
immigrant scientist in trouble. It could also be seen through the eyes
of a seasoned spy catcher. ''This says this guy wants to be a player,''
said Paul D. Moore, then the F.B.I.'s chief analyst for Chinese
counterintelligence.

But Dr. Lee passed a polygraph test on whether he had divulged
classified data and cooperated with F.B.I. agents trying to get
incriminating information on the Tiger Trap suspect. The incident was
apparently never reported to the Energy Department, and the F.B.I.
closed its investigation of Dr. Lee in 1984.

Had the department known, ''it would have been enough to remove his
security clearance,'' an agency official said. ''The lights should have
gone off with somebody.''

MAKING FRIENDS

Dr. Lee traveled to Beijing twice in the 1980's. What worried his bosses
was what he did not tell them when he got home.

Throughout his career at Los Alamos, Dr. Lee traveled widely, attending
scientific meetings and giving papers in places like Venice and
Budapest, Britain and Hawaii. In March of 1985, he and other government
scientists attended a conference in Hilton Head, S.C. Two scientists
from China were also there.

''They sat in the back wearing their Mao jackets and stuck out like a
sore thumb,'' said Robert A. Clark, a scientist who attended the
conference. ''Wen Ho chatted with them quite a bit.'' The scientists
suggested that Dr. Lee and Dr. Clark attend a conference in Beijing the
next year, and, with approval from Los Alamos, they went with their
wives.

Dr. Clark, a defender of Dr. Lee, said it was clear in Beijing that his
colleague had befriended some Chinese scientists.

''It's obvious they would chat him up with the idea that maybe one day
they would get information from him,'' he said. ''You might say he was
friendlier than he should have been with these guys.'' But if it looked
suspicious, he said, it was only because of fears of China.

Dr. Lee's wife, Sylvia, a secretary and data-entry clerk at Los Alamos,
was making friends, too. She had become an unofficial hostess for
visiting Chinese. Correspondence obtained by The Times shows that she
served as both tour guide and research contact.

''I am very sorry to hear that Wen Ho is ill and hope he will get better
soon,'' a Chinese scientist wrote her in a telex about a coming trip
with a colleague. ''Both Chen and I will be very happy if we can learn
something in computational hydrodynamics and get some papers.''

Mrs. Lee also gave the F.B.I. and C.I.A. information about scientists
she met. She had repeated contacts with the F.B.I. in the mid-1980's,
government officials and others knowledgeable about the case said. In
about a dozen instances, they said, a C.I.A. agent was present and paid
for the hotel room where the meetings took place.

In 1988, the Lees attended another conference in Beijing. In post-trip
debriefings, American scientists often reported being approached by
Chinese scientists seeking classified information, but Dr. Lee reported
nothing of the sort. That worried Robert Vrooman, then the chief of
counterintelligence at Los Alamos.

Mr. Vrooman says he considered Dr. Lee naïve, not nefarious. Even so, in
1990, he urged laboratory officials to deny Dr. Lee's request to visit
China again. Officials decided to end Mrs. Lee's role as a hostess at
about the same time.

''I have been concerned for some time that Dr. Lee did not understand
the ruthlessness of intelligence agencies in trying to collect
information being vital to national survival,'' Mr. Vrooman said last
year in court documents.

BLAST IN THE DESERT

At first, the Chinese bomb test didn't alarm American officials. But how
did Dr. Lee know the designer of China's new bomb?

On Sept. 25, 1992, a nuclear blast shook China's western desert near the
Silk Road once traveled by Marco Polo.

From spies and electronic surveillance, American intelligence officials
determined that the test was a breakthrough in China's long quest to
match American technology for smaller, more sophisticated hydrogen
bombs.

China had entered the nuclear arena after other big powers and feared
its large, stationary missiles were becoming vulnerable to disarming
first strikes. Smaller bombs that fit on trucks and submarines would be
easier to hide, have greater range and aid China's transformation from a
regional to a global nuclear power.

Miniaturization was difficult science, involving complicated physics,
computer work and machining. Older bombs use a ball of atomic fuel
surrounded by a cumbersome array of conventional explosives that
compress the fuel until it reaches critical mass. The secret to the
smaller American design was an oval-shaped mass of atomic fuel detonated
by just two charges -\/- one at each end of the oval. That step helped
cut the width of bomb casings from feet to mere inches.

Shrinking weapons by using ''two-point'' detonation became China's holy
grail. The first American nuclear scientists who went to China in the
late 1970's were peppered with questions about miniaturization. When the
Tiger Trap suspect was stopped at an airport en route to China in 1981,
officials said, he was carrying detailed answers to five weapons
questions, including one about two-point detonation. Though officials
believed that secrets leaked in the Tiger Trap case, they felt the
evidence was too weak to bring criminal charges. The suspect maintained
his innocence; he now refuses to discuss the case.

The 1992 test was a leap forward, but it did not initially alarm
American nuclear intelligence experts, since countries like Russia and
Britain had mastered two-point technology years before. Besides, the
diplomatic wind was blowing in a different direction.

With the cold war over, the United States and other countries were
trying to defuse the arms race with global cooperation. As a sign of the
new openness, the Energy Department began declassifying millions of
ideas and documents about nuclear arms, and even encouraged weapons
scientists to share unclassified computer codes with their foreign
counterparts.

Washington began working with Moscow to secure its plutonium stockpiles.
And Beijing agreed to a partnership on arms control and methods of
verifying a test-ban treaty -\/- an agreement destined to bring the two
nations' nuclear scientists even closer together.

On Feb. 23, 1994, Los Alamos was host to the highest-level group of
Chinese weapons officials ever to visit the United States. Leading the
delegation was Hu Side (pronounced se-DUH), the new head of the Chinese
Academy of Engineering Physics, the nation's bombmakers. American
intelligence officials had learned that he was the designer of China's
two-point bomb.

One person not on the guest list was Wen Ho Lee. ''We had very tight
controls on access,'' a laboratory official said. ''The door was closed.
The session was not advertised.'' But that afternoon, Dr. Lee appeared
at a briefing and was warmly greeted by Dr. Hu.

''There is a lot of bowing and exchanging cards,'' another official
recalled. He was startled that a midlevel hydrodynamics expert at Los
Alamos knew China's top nuclear scientist. And Wen Ho Lee was not simply
relatively obscure; just months before, he had learned he might be laid
off because of budget cuts.

Then a translator told the official that Dr. Hu was thanking Dr. Lee in
Mandarin. ''They're thanking him because the computer software and
calculations on hydrodynamics that he provided them have helped China a
great deal,'' the translator said.

Laboratory officials informed the F.B.I., which had suspicions of Dr.
Lee from Tiger Trap and opened an investigation. Officials did not know
what to think. Dr. Lee had never reported meeting Dr. Hu in China. If
the two had an improper relationship, why expose it at Los Alamos?

A GREAT LEAP FORWARD

China's new bomb, one expert said, was 'like they were driving a Model
T' and 'suddenly had a Corvette.' Was it espionage?

Tension between security officers and scientists who see their work as
apolitical and dependent on open discourse has existed at Los Alamos
since the laboratory's founder, J. Robert Oppenheimer, clashed with
Leslie Groves, the Manhattan Project's top military man, who so
mistrusted the scientists that he wanted them to enlist and wear
uniforms.

Little surprise, then, that scientific diplomacy was not universally
applauded. As the Energy Department's new intelligence director, Notra
Trulock, saw it, scientists might ''think they're too smart to be
bamboozled by some foreign intelligence officer.'' Periodic leaks and
other security breaches, he believed, indicated otherwise.

Mr. Trulock entered the fray not as an expert on China or spy hunting or
even bomb building. He had a political science degree from Indiana
University and in the Army during the cold war had monitored Warsaw Pact
radio transmissions on the German-Czech border. Later, he led a Los
Alamos research project on the dangers of post-Communist Russia losing
control of its nuclear weapons, a study that won two government awards.

In his new job in Washington, Mr. Trulock said, he figured warnings
about Russia would go unheeded given President Bill Clinton's policy of
engaging the former enemy. But the risks posed by China might be heard.
''We focused on China because we could,'' he said recently.

Siegfried S. Hecker, the director of Los Alamos from 1986 to 1997, said
that, in several discussions, Mr. Trulock had implied that Los Alamos
''was infiltrated by Chinese agents.'' Once, Dr. Hecker added, Mr.
Trulock told him that ''just the fact that there are five Chinese
restaurants here meant that the Chinese government had an interest.''
Mr. Trulock denies that remark.

Mr. Trulock's focus on China began when Robert M. Henson, a Los Alamos
scientist and intelligence analyst, went to him in early 1995 and said
his analysis showed that the Chinese had so dramatically shrunk their
weapons that they had to have used stolen American secrets. ''It's like
they were driving a Model T and went around the corner and suddenly had
a Corvette,'' Dr. Henson said.

Now Mr. Trulock turned to John L. Richter, a legendary bomb designer
whose specialty was the main bomb component the Chinese had improved
-\/- the atomic trigger for a hydrogen bomb, known as a primary. Dr.
Richter said the sketchy evidence suggested that China might have
significant information about the primary of the W-88.

Dr. Richter, who had overseen the design team for the W-88, calls it ''a
darling.'' The W-88 warhead is 30 times more powerful than the bomb that
leveled Hiroshima, but the compact design of its primary allows for
unusual accuracy. Beginning in 1990, hundreds were affixed to Trident
missiles and deployed on submarines.

The question was how much the Chinese had reduced the size of their bomb
primaries. Making a smaller weapon was a natural evolution for China,
but making one as small and sophisticated as the W-88, and doing so
quickly, was a monumental leap of physics and engineering that
presumably would have required knowing American bomb secrets. After all,
it had taken the United States three decades to go from its first
miniaturized hydrogen bomb -\/- a warhead with a primary casing about 20
inches across -\/- to the W-88, with its 9-inch casing.

Mr. Trulock sensed espionage. He likened China's 1992 test to the first
clue in other great spy cases, like the unexplained deaths of Russians
working for the United States in the Aldrich Ames affair. ''In this
case,'' he said, ''you had something go boom in the desert.''

'THOUSAND-PIECE PUZZLE'

Officials knew the Chinese had stolen some secrets about the W-88. But
how much did they know, and what had they done with it?

To probe deeper, Mr. Trulock assembled nearly 20 weapons and
intelligence experts who met in the summer of 1995 in a spy-proof room
at Energy Department headquarters in Washington, sifting through
intercepted signals, purloined Chinese documents, accounts of spies.

But determining the physical size of China's test bombs was nearly
impossible. ''You get three pieces of a thousand-piece puzzle and try to
figure out what it is,'' one participant said. ''People read in their
own prejudices.''

The pieces they had were hardly clear, intelligence officials said. A
spy's vague report spoke of Chinese interest in a primary whose outer
casing was the size of a soccer ball -\/- about nine inches, the width
of the W-88 casing. And a Chinese scientist visiting Los Alamos had
recently bragged about the size of China's new bombs by holding his
hands close together.

Still, while there was no question China had built smaller bombs with
two-point detonation, most of the experts agreed there was no proof the
Chinese had figured out anything about the W-88.

Then, in midsummer, the experts got from the C.I.A. a seven-year-old
Chinese document showing that Beijing knew distinctive characteristics
of the W-88, including almost the precise width of the primary casing.
In spy-speak, it was a ''walk-in document'' because someone had offered
it out of the blue.

The document, which compared China's weapons with those of various
countries, was far from a blueprint for the W-88. It contained secret
but rudimentary information of value mainly in making missiles that
carry bombs. To Dr. Richter, the walk-in confirmed that China knew ''the
periphery'' of the W-88, but not its design. ''If you get a map of New
York, is that New York?'' he said. ''No, it's an image.''

Michael G. Henderson, a bomb designer who headed the panel of experts,
said, ''We all agreed there had been some hanky-panky.''

But in wrestling with the implications of the espionage, the experts
clashed, with their debate breaking into three positions.

The most benign was that China had effectively made all its advances on
its own, even if it had done some spying.

The second, that China had benefited from a slow drip of secrets about
two-point detonation, was supported by reports of many scientists asked
to give up secrets while visiting China, by the files of Tiger Trap and
by the walk-in document itself.

The last view was that a cold-war-style superspy had betrayed much more
in a single delivery of bomb blueprints than the slow drip ever could.
Dr. Henson, who had first sounded alarms about Chinese spying to Mr.
Trulock, was virtually alone in arguing angrily that the magnitude of
China's advancement implied the existence of a major spy. One
participant recalled him ''literally cursing, swearing at us,'' and
added, ''His face was red.''

Having reached an impasse near the end of the summer, the group stopped
its formal meetings. Months later, the few remaining experts agreed on a
compromise that was spelled out in secret briefing documents, which were
recently described by participants and federal officials.

On the one hand, they said, Tiger Trap had likely given the Chinese the
two-point concept, and over all, espionage had ''been of material
assistance'' to Beijing's nuclear advances. Further, they believed that
China had plans to try to build a ''W-88-like aspheric primary.''

Even so, the experts said they had no way of knowing how small China's
bombs had actually gotten and saw no evidence that Beijing had copied
America's premier weapon.

Mr. Trulock remembers it differently. The panel, he said, generally
agreed that the 1992 test involved something akin to the W-88 primary.
''Words like 'resembled' and 'similar to,' were words that were used,''
he said. He accused the scientists of rewriting history to play down
their role in the Lee ordeal.

Dr. Henderson, the panel's chairman, said Mr. Trulock took his own view
''and ran with it.'' He added: ''I'm sure he believes in the veracity of
what he had. But, unfortunately, that doesn't mean it's true.''

SEARCHING FOR SUSPECTS

Though the exact crime was unclear, an espionage investigation settled
on Los Alamos, the birthplace of the W-88. Soon, the focus narrowed to
Wen Ho Lee.

If Notra Trulock ran with it, he hardly ran alone. He informed his
bosses at the Energy Department. Alarmed, they asked the C.I.A. for its
assessment. Initially skeptical, the C.I.A. reviewed the evidence and
agreed that espionage had probably aided China. The Energy Department
gave Mr. Trulock a green light to expand his inquiry and to brief top
officials, from the White House, in April 1996, to the Strategic Command
in Omaha.

Mr. Trulock called the investigation Kindred Spirit, and from the start,
it reflected his belief that the Chinese had come close to replicating
the W-88, and that one spy might have given them the blueprints.

In his briefings, he was typically careful not to overstate how much was
known about Chinese spying. But he also took the stance of a military
analyst in stating the worst-case scenario, people who heard his
briefings said. Sometimes, he included images of China's newest missile
and the W-88, implying that was where China was headed.

''We thought it best to focus on the W-88 because it was the newest
system in our inventory and it was the system within the 'walk-in
document' for which the most detailed information was provided,'' Mr.
Trulock wrote in an unpublished article. And he said he feared that the
secrets in the walk-in document might represent just a sampling of what
the Chinese had stolen about the W-88.

The idea of a theft, without the scientists' caveats, was alarming. ''I
said, 'Holy cow, this is the last thing we need,' '' said Daniel J.
Bruno, Mr. Trulock's chief investigator on the case. ''It's a very
serious thing that affects your children, our children, our
grandchildren.''

In searching for suspects, Energy Department investigators, aided by an
F.B.I. agent experienced in Chinese espionage, looked at other weapons
laboratories but concentrated on Los Alamos, where the W-88 had been
developed.

Since the laboratory had no records showing all contacts between
American and Chinese scientists, the investigators gleaned a list of 70
potential suspects from records of laboratory employees who had traveled
to China in the mid-1980's, before the walk-in document was written. The
Energy Department's final report shows that more than a third were on
the list for travel that had nothing to do with the scientific work of
the laboratory: ''chaperone with Santa Fe High School band's trip to
Beijing,'' ''personal vacation cruise to Whangpo.''

Investigators also looked at people who had access to W-88 information
or had security problems. The list was narrowed to a dozen suspects,
half with Chinese surnames. Wen Ho Lee and Sylvia Lee were on top. The
Lees had visited China twice. Dr. Lee, whose access to weapons secrets
was listed as ''moderate,'' had worked on the W-88 computer code. His
appearance in Tiger Trap remained suspicious. And investigators found
Mrs. Lee suspect because laboratory supervisors said she had been so
eager to play host to Chinese visitors that it conflicted with her job.
(The investigators were never told that Mrs. Lee had also been a source
for the F.B.I. and the C.I.A.)

''Quite frankly, Wen Ho Lee being a suspect at that point is only
natural, since at that time they had been looking at him for 13 years,''
said Dr. Hecker, then the Los Alamos director. ''They would have been
derelict not to look at him.''

But it may also have been derelict to look only at Dr. Lee, especially
since the most concrete evidence of spying was the walk-in document, and
its secrets had been distributed to hundreds, if not thousands, of
people at military installations and missile contractors.

It is true that Energy Department investigators were legally prohibited
from looking for suspects outside their agency. But Mr. Trulock and Mr.
Bruno said they told F.B.I. officials that the leak might have come from
the other sources. In addition, T. Van Majors, the F.B.I. agent
assisting the Energy Department, wrote a memorandum warning against
focusing just on Dr. Lee, a law enforcement official said. However, the
memorandum was not reflected in the Energy Department's report on the
case, and in the subsequent F.B.I. investigation.

''This guy stands out higher than the rest, based on circumstantial
issues,'' Mr. Bruno said.

Defenders of Dr. Lee have said that investigators focused solely on him
because of ethnic profiling, a charge government officials deny. Still,
ethnicity did play some role in their thinking. Mr. Moore, the F.B.I.'s
former China espionage analyst, said that while the Chinese routinely
seek information from visiting scientists of all nationalities, they
concentrate on ethnic Chinese, including Taiwanese, by appealing to a
''perceived obligation to help China.''

When Mr. Trulock's office issued its secret report, it said Dr. Lee
''appears to have the opportunity, means and motivation'' to compromise
the W-88. A secret Justice Department review of the case, completed last
year, called Mr. Trulock's report ''a virtual indictment'' of Dr. Lee, a
law enforcement official said.

The crime, though, was unclear. The report's damage assessment, never
before disclosed, contained a hodgepodge of formulations, from the
tentative (the W-88 ''may have been compromised'') to the certain (the
Chinese had ''almost a total duplicate of the W-88 warhead'').

AN ERRATIC PURSUIT

The F.B.I.'s investigation

of Dr. Lee started and

stalled as it passed from agent to agent and was overshadowed by
higher-profile cases.

Two days after receiving the Energy Department's report in late May of
1996, and still three years before the case became public, the F.B.I.
opened an investigation of Wen Ho Lee. The old inquiry, begun after Dr.
Lee's encounter with Dr. Hu, was folded in.

Usually, the F.B.I. looks askance at the investigative work of other
agencies. But in this case, F.B.I. officials neither interviewed the
panel of weapons experts nor searched beyond the Energy Department for
suspects. They accepted the Energy Department's finding as confirming
their own suspicions about Dr. Lee and shipped it out to the field.

The case fell to David Lieberman, a veteran agent who worked Los Alamos
counterintelligence cases part time from an F.B.I. satellite office in
Santa Fe. The Lee investigation was added to his lineup of drug cases,
bank robberies and crimes on nearby Indian reservations.

Promised help never came. Headquarters sent two agents to assist, but
Albuquerque F.B.I. officials assigned them to general crime cases, law
enforcement officials said. ''It's not the way to handle anything that's
a big investigation,'' a former official involved in the case said.
''You don't send it out to the backwater of America and assign it to
someone part time.''

Neil J. Gallagher, head of the F.B.I.'s national security division,
acknowledged that more resources should have been devoted to the case.
But he said the investigation was hamstrung because it involved
espionage suspected to have occurred a decade earlier.

There were more current national security cases at the time, including
the Oklahoma City bombing and the Unabomber. Besides, Chinese espionage
had always been a stepchild to Eastern Bloc cases, and in the aftermath
of the cold war, F.B.I. resources had shifted to things like terrorism
and urban drug gangs.

Still, as the case passed from one agent to another, the F.B.I. seemed
to miss one opportunity after another.

For years, F.B.I. agents did not search Dr. Lee's computer because they
believed they lacked legal authority. They never looked far enough to
find a waiver Dr. Lee had signed in April 1995 stating, ''Activities on
these systems are monitored and recorded and subject to audit.'' Agents
never used standard investigative tools, like trash searches and
stakeouts. F.B.I. officials said it was difficult to operate
surreptitiously in the closed society of Los Alamos. But a veteran
F.B.I. espionage investigator said agents have worked in more
challenging circumstances. ''We've run cases inside C.I.A.
headquarters,'' he said.

In 1997, a new agent on the case requested a permit to eavesdrop
electronically on the Lees. A secret F.B.I. report prepared to support
that application flatly stated that China ''seemed to have had a copy of
the design'' of the W-88.

Allan Kornblum, a Justice Department lawyer who reviewed the permit
application, later told a Senate committee, ''I was also shocked by the
facts, the idea that this guy is making official trips to the P.R.C. to
meet with his counterparts in nuclear weapons design.''

Still, weaknesses in the Lee case were obvious. Agents had not examined
any other suspects on the Energy Department's list. They had not
sufficiently demonstrated a link between Dr. Lee and the compromised
W-88 information, Mr. Kornblum said. Intriguing elements of the case
were old. In short, ''we had little to show that they were presently
engaged in clandestine intelligence activities,'' he said, according to
a report by Senator Arlen Specter, Republican of Pennsylvania.

Justice Department officials declined to act on the F.B.I.'s
application. That rejection stalled the investigation again. Mr.
Kornblum said he told agents in August 1997 how to ''flesh out'' their
application, but they did not respond for nearly 18 months. F.B.I.
supervisors in Washington sent Albuquerque a list of 15 investigative
tasks, but only 2 were done, a Senate investigation later determined.

With the investigation flagging, the F.B.I. director, Louis J. Freeh,
told Energy Department officials that concerns about exposing the
investigation were no longer a reason to keep Dr. Lee in his job.

But the laboratory's top officials were never told. According to
internal Energy Department correspondence, Mr. Vrooman, the Los Alamos
security chief, decided after consulting with a local F.B.I. agent that
it would be better for the investigation if Dr. Lee remained in the
laboratory's inner sanctum, X Division.

IN THE ECHO CHAMBER

In Washington, anger at the Clinton administration and concern over
China brought the W-88 case to a boil.

In Washington, Notra Trulock was pressing his case. By his own
estimation, he gave his standard briefing about China, the W-88 and
leaks at the national laboratories 60 times from 1995 to 1998.

He was relentless. Unable to get an appointment with a new top official
at the Energy Department, Mr. Trulock recalled, he lingered outside her
office until he could slip in and hit her with his pitch. Mr. Moore, the
former F.B.I. analyst, said Mr. Trulock had figured out that to get
heard in Washington: ''He had to hype it. He wanted people to get
interested in the problem.''

Mr. Trulock denies any exaggeration. In fact, there was new evidence to
support his anxiety about Chinese espionage. A September 1997
Congressional report found that foreign visitors were streaming into
government laboratories without background checks. Los Alamos, for
example, had 2,714 visitors in two years from ''sensitive'' countries,
but only 139 were checked. Also in 1997, a scientist named Peter Lee
pleaded guilty to charges related to passing American nuclear secrets to
the Chinese.

Early the next year, President Clinton issued a directive to improve
security at the laboratories. But Mr. Trulock felt that changes were
coming too slowly, and that laboratory officials' view of espionage was
that ''it couldn't happen here.''

If Mr. Trulock's warnings about lax security rang true for many
officials, his central point -\/- the theft of the W-88 -\/- met with
some skepticism.

A 1997 report, prepared for the White House by the C.I.A., found that
while spying had aided China's ''remarkable progress in advanced nuclear
weapons design,'' it had saved Beijing a mere two years of development.
The report went on to judge that China had no W-88 duplicate.

Some experts, hearing Mr. Trulock's classified briefing, questioned
whether China would even want to expend the vast resources needed to
produce the W-88. Richard L. Garwin, a top federal science adviser, said
he dismissed the notion as whimsical. While the highly accurate W-88 was
designed for a specific cold war objective -\/- knocking out missile
silos -\/- China's nuclear program focuses on the ability to destroy
cities.

But suddenly, in 1998, Mr. Trulock found a larger and more receptive
audience.

With impeachment as a backdrop, allegations that the Clinton
administration was allowing China easy access to American secrets
collided with charges that China's military had funneled money into
Democratic coffers. The New York Times reported that the daughter of a
senior Chinese military officer was giving money to Democrats while also
working to acquire sensitive American technology.

Republicans, opening a new front against a beleaguered president,
created a House select committee, headed by Representative Cox, to
investigate whether the government was compromising technology secrets
by letting American companies work too closely with China's rocket
industry. With its deadline approaching, the committee stumbled on the
W-88 case.

Mr. Trulock became a star witness, and committee members were riveted by
his testimony. C.I.A. analysts who testified before the committee agreed
there was espionage, people who heard the secret proceedings said, but
were more equivocal about its value to China.

As it was completing its work, the panel received a secret report from
the National Counterintelligence Center, a federal group that seeks to
outwit spies. In a brief reference, the report echoed Mr. Trulock's view
that China had stolen ''the design information on a current U.S.
warhead,'' the W-88, but offered no evidence to back that finding.

The Cox committee wrote its report in late 1998, but it was not
declassified and released until May 1999, after the case had broken into
public view. The unanimous report accused China of stealing nuclear
secrets -\/- possibly even entire blueprints -\/- for the warheads of
''every currently deployed'' long-range American missile. While
acknowledging that ''much is unknown'' about the impact of the thefts,
it judged that future Chinese designs would ''exploit elements'' of the
W-88, and that the stolen secrets put China's bomb-design information
''on a par with our own.''

But John M. Spratt Jr., a Democratic representative on the committee,
said the panel lacked the time and witnesses with sufficient technical
background to fully examine the issues. In retrospect, he said, Mr.
Trulock's testimony was more alarming than warranted.

He pointed to a 1999 report by the nation's top intelligence experts,
done in response to the Cox panel, that concluded that China's theft of
American secrets had ''probably accelerated'' its weapons development,
though more ''to inform their own program than to replicate U.S. weapons
design.''

The Chinese government issued its own response to the Cox committee. Its
report, ''Facts Speak Louder Than Words and Lies Will Collapse by
Themselves,'' denied any espionage.

And in a recent e-mail response to questions from The Times, Hu Side,
China's top bomb designer, said his nation's scientists ''can create
every advanced technology and glory which they need by their own
efforts.''

CLOSING IN

Bit by bit, new details of Dr. Lee's activities came tumbling out.

The Cox committee's deliberations built pressure within the government
to revive the languishing W-88 investigation.

David V. Kitchen, who became head of the F.B.I.'s Albuquerque office in
August 1998, said he first learned details of the case that October,
when his assistant brought him the Energy Department's 1996
administrative report.

''We couldn't understand how they came to the conclusion they came to,
specifically about how Lee was the main suspect,'' said Mr. Kitchen, who
is now retired from the F.B.I.

Mr. Kitchen wanted to close the investigation. ''We worked the case for
quite a while, and what did we have to show for it?'' he asked. The
answer was very little.

But Edward J. Curran, an F.B.I. official working at the Energy
Department, had heard a secret Cox committee briefing and was aghast at
what he saw as a lack of rigor in the F.B.I. investigation.

In August, the F.B.I. had run a sting operation, with an agent posing as
a Chinese intelligence officer trying to lure Dr. Lee to a meeting. Even
though Dr. Lee did not take the bait, Mr. Curran was concerned that if
Dr. Lee was a spy, that call could have alerted him that the authorities
were onto him. In December, investigators knew Dr. Lee was going to
Taiwan for three weeks, but did not monitor him. Laboratory officials
had not even informed the F.B.I. when Dr. Lee went to Taiwan for six
weeks earlier that year to consult at a military institute.

The new energy secretary, Bill Richardson, said he decided that leaving
Dr. Lee in X Division ''was an unacceptable risk.'' On Dec. 23, after
Dr. Lee returned from Taiwan, the department gave him a lie detector
test. Dr. Lee was initially found to have passed the test, which
included questions about divulging secrets. But he made one startling
revelation.

One night during his 1988 trip to Beijing, a Chinese scientist he knew
had called his hotel room and asked to meet alone. Dr. Lee agreed, and
the scientist, an official in China's nuclear program, showed up with Hu
Side. Dr. Hu, law enforcement officials said, asked Dr. Lee questions
about how to make smaller hydrogen bombs using oval-shaped fuel.

China's top bomb designer, then, was pressuring Dr. Lee for information
about two-point detonation four years before China achieved that goal.
Perhaps that explained why Dr. Hu greeted Dr. Lee so warmly during the
briefing at Los Alamos in 1994.

Dr. Lee told investigators that he had not answered Dr. Hu, since the
information was secret, but he had never before reported the meeting to
security officers, as required. It was precisely the kind of approach
Mr. Vrooman, the laboratory security official, was surprised Dr. Lee had
not reported in the 1980's.

That day, Los Alamos officials suspended Dr. Lee's access to X Division.
F.B.I. agents had heard Dr. Lee's admission about Dr. Hu, but they did
not interview him for three weeks, and even then did not grill him about
it, a laboratory official who was present said. ''They didn't press him
to go into details,'' he said. ''It will bother me for years.''

Believing that Dr. Lee had passed the polygraph test, Mr. Kitchen asked
an agent on the case to write a memorandum proposing ending the
investigation, which he forwarded to Washington. But on Feb. 2, the case
turned again, this time on the analysis of a polygraph test. F.B.I.
analysts reviewed tapes of the December test and decided that Dr. Lee's
answers were inconclusive, after all.

Polygraph tests record factors like pulse rate and sweat gland activity
to determine if a subject is being truthful. Although results are not
admissible in court, law enforcement agencies, particularly the F.B.I.,
place great stock in their investigative value.

On Feb. 10, bureau officials administered their own test in a hotel room
in Los Alamos. Dr. Lee was wired to a machine, and for the first time
since he was singled out in 1996, was asked, ''Have you ever provided
W-88 information to any unauthorized person?''

''No,'' he answered.

He also said he had never given nuclear-weapons codes to an unauthorized
person.

The polygraph examiner determined that Dr. Lee was deceptive, a
Congressional report said.

He also told the examiner that he had helped a Chinese scientist with a
mathematical problem that ''could easily be used in developing nuclear
weapons,'' Mr. Freeh later told Congress.

That evening, Dr. Lee told one of his bosses, Richard A. Krajcik, that
he had failed the test, and acknowledged that ''he may have accidentally
passed'' secrets to a foreign country, Dr. Krajcik testified in court.
Dr. Lee's lawyers say he never made such a statement.

The investigation that was nearly closed weeks before was reaching a
boil. After having gone on in secret for years, it was also leaking.

Back in January, The Wall Street Journal had run a news article under
the headline ''China Got Secret Data on U.S. Warhead -\/- Chief Suspect
Is a Scientist at Weapons Laboratory of Energy Department.'' The article
said the Chinese had obtained information on the W-88 from Los Alamos,
but investigators said they had no sign the article had alerted Dr. Lee.

Two months later, when the authorities were informed that The New York
Times was preparing a major article on the W-88 case, they realized time
was running out to get a confession from Dr. Lee.

Federal officials asked The Times to delay publication for several
weeks, saying they were preparing to confront their suspect. Although
The Times did not know the identity of the chief suspect, F.B.I.
officials said they feared he would recognize himself from details in
the article. The Times withheld publication for one day and said it
would consider a further delay if asked personally by Mr. Freeh, the
F.B.I. director. He never called.

The F.B.I. interviewed Dr. Lee on March 5, and he consented to a search
of his office. The next day, a Saturday, The Times published its
article, ''China Stole Nuclear Secrets for Bombs, U.S. Aides Say.'' The
article said American officials believed ''Beijing was testing a smaller
and more lethal nuclear device configured remarkably like the W-88.''
And it reflected criticism of the White House and the F.B.I. for not
dealing swiftly with the Los Alamos case. It included Paul Redmond, the
C.I.A.'s former chief spy hunter, saying that ''this is going to be just
as bad as the Rosenbergs.''

The Times article prompted a flood of press attention and upended the
F.B.I.'s strategy, forcing agents to rush into a confrontation interview
with Dr. Lee before they were ready, Mr. Freeh told Congress.

The F.B.I lured Dr. Lee to Santa Fe that Sunday and subjected him to a
harsh interrogation. An F.B.I. agent thrust a copy of The Times at him.
''Basically that is indicating that there is a person at the laboratory
that's committed espionage, and that points to you,'' she said,
according to a transcript.

''But do they have any proof, evidence?'' Dr. Lee asked.

The F.B.I. had only suspicion, and the agent, who has been identified by
several government officials and in court testimony as Carol Covert,
laid it out in the interrogation. The Lees went to China in 1986 and
''they were good to you,'' she said. ''They took care of your family.
They took you to the Great Wall. They had dinners for you. Everything.
And then in 1988 you go back and they do the same thing and, you know,
you feel some sort of obligation to people to, to talk to them and
answer their questions.''

She focused on Dr. Lee's 1988 hotel room encounter with Dr. Hu.
''Something had to have happened when they came to your room,'' Ms.
Covert said. ''We know how the Chinese operate.''

Dr. Lee said he had ''a rule in my mind'' about what was secret and what
he could reveal. ''You may think,'' he told the agents, ''when people,
when the Chinese people do me a favor, and I will end up with tell them
some secret, but that's not the case, O.K.?''

They threatened him with losing his job, with being handcuffed, with
being thrown in jail. In preparing for the interview, Mr. Kitchen said
he had suggested to Ms. Covert that she bring up the Rosenbergs because
of the reference in the Times article.

''Do you know who the Rosenbergs are?'' Ms. Covert asked.

''I heard them, yeah, I heard them mention,'' Dr. Lee said.

''The Rosenbergs are the only people that never cooperated with the
federal government in an espionage case,'' she said. ''You know what
happened to them? They electrocuted them, Wen Ho.''

When the transcript was made public, F.B.I. officials denounced the
Rosenberg reference. ''She carried that a bit further than we expected
her to,'' Mr. Kitchen said.

But Dr. Lee did not crack. Always polite, he thanked the F.B.I. agents
as he left. ''I hope you have good health,'' he said. He added: ''If
they want to put me in jail, whatever. I will, I will take it.''

Driving up the mountain to Los Alamos from Santa Fe that afternoon with
his friend Bob Clark, Dr. Lee was distraught. ''They kept saying I had
to say that I did this thing I didn't do,'' Dr. Clark recalls him
saying.

Mr. Richardson announced Dr. Lee's dismissal the next day, based on a
failure to report contacts with people from a ''sensitive country'' and
mishandled classified documents found on Dr. Lee's desk.

But the F.B.I. was no closer to knowing if Dr. Lee was the suspected
W-88 thief. They just had a more detailed, if more frustrating, picture
of him.

''It seemed like the more times you hit him upside the head, the more
truth comes out,'' Mr. Kitchen said. ''It's like a little kid.''

Tomorrow: The prosecution unravels.

Under Suspicion

After Wen Ho Lee was freed from jail last September, a furor erupted
over how the government had handled the case and how the press,
especially The New York Times, had covered it. Several weeks later, The
Times published an unusual statement assessing its coverage. It found
many strengths, but also some weaknesses. In the statement, the paper
promised a thorough re-examination of the case. After more than four
months of reporting, the results appear today and tomorrow.

Advertisement

\protect\hyperlink{after-bottom}{Continue reading the main story}

\hypertarget{site-index}{%
\subsection{Site Index}\label{site-index}}

\hypertarget{site-information-navigation}{%
\subsection{Site Information
Navigation}\label{site-information-navigation}}

\begin{itemize}
\tightlist
\item
  \href{https://help.nytimes.com/hc/en-us/articles/115014792127-Copyright-notice}{©~2020~The
  New York Times Company}
\end{itemize}

\begin{itemize}
\tightlist
\item
  \href{https://www.nytco.com/}{NYTCo}
\item
  \href{https://help.nytimes.com/hc/en-us/articles/115015385887-Contact-Us}{Contact
  Us}
\item
  \href{https://www.nytco.com/careers/}{Work with us}
\item
  \href{https://nytmediakit.com/}{Advertise}
\item
  \href{http://www.tbrandstudio.com/}{T Brand Studio}
\item
  \href{https://www.nytimes.com/privacy/cookie-policy\#how-do-i-manage-trackers}{Your
  Ad Choices}
\item
  \href{https://www.nytimes.com/privacy}{Privacy}
\item
  \href{https://help.nytimes.com/hc/en-us/articles/115014893428-Terms-of-service}{Terms
  of Service}
\item
  \href{https://help.nytimes.com/hc/en-us/articles/115014893968-Terms-of-sale}{Terms
  of Sale}
\item
  \href{https://spiderbites.nytimes.com}{Site Map}
\item
  \href{https://help.nytimes.com/hc/en-us}{Help}
\item
  \href{https://www.nytimes.com/subscription?campaignId=37WXW}{Subscriptions}
\end{itemize}
