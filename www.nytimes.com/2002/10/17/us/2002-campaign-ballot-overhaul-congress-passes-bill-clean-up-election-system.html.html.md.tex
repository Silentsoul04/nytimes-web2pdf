Sections

SEARCH

\protect\hyperlink{site-content}{Skip to
content}\protect\hyperlink{site-index}{Skip to site index}

\href{https://www.nytimes.com/section/us}{U.S.}

\href{https://myaccount.nytimes.com/auth/login?response_type=cookie\&client_id=vi}{}

\href{https://www.nytimes.com/section/todayspaper}{Today's Paper}

\href{/section/us}{U.S.}\textbar{}THE 2002 CAMPAIGN: BALLOT OVERHAUL;
CONGRESS PASSES BILL TO CLEAN UP ELECTION SYSTEM

\begin{itemize}
\item
\item
\item
\item
\item
\end{itemize}

Advertisement

\protect\hyperlink{after-top}{Continue reading the main story}

Supported by

\protect\hyperlink{after-sponsor}{Continue reading the main story}

THE 2002 CAMPAIGN: BALLOT OVERHAUL

\hypertarget{the-2002-campaign-ballot-overhaul-congress-passes-bill-to-clean-up-election-system}{%
\section{THE 2002 CAMPAIGN: BALLOT OVERHAUL; CONGRESS PASSES BILL TO
CLEAN UP ELECTION
SYSTEM}\label{the-2002-campaign-ballot-overhaul-congress-passes-bill-to-clean-up-election-system}}

By \href{https://www.nytimes.com/by/robert-pear}{Robert Pear}

\begin{itemize}
\item
  Oct. 17, 2002
\item
  \begin{itemize}
  \item
  \item
  \item
  \item
  \item
  \end{itemize}
\end{itemize}

See the article in its original context from\\
October 17, 2002, Section A, Page
1\href{https://store.nytimes.com/collections/new-york-times-page-reprints?utm_source=nytimes\&utm_medium=article-page\&utm_campaign=reprints}{Buy
Reprints}

\href{http://timesmachine.nytimes.com/timesmachine/2002/10/17/427322.html}{View
on timesmachine}

TimesMachine is an exclusive benefit for home delivery and digital
subscribers.

Congress gave final approval today to a bill to clean up the nation's
election procedures, nearly two years after a breakdown in Florida
balloting muddled the results of the 2000 presidential election.

The bill calls for a major expansion of the federal role in regulating
how voters are registered and elections conducted. States would retain
the primary responsibility, but Congress is setting minimum standards
and will provide money to help the states comply.

Today's Senate vote, 92 to 2, clears the bill for President Bush, who
said he would sign it. The House approved the measure last week, 357 to
48.

The bill authorizes \$3.9 billion of federal aid in the next three
years, but it is unclear how much will actually be provided.
Congressional leaders said they hoped to provide the full amount, but
there is no guarantee. Spending bills are caught in a logjam as Congress
struggles to finish its work for the year.

States can use the federal money to upgrade their election systems in
many ways: to replace punch-card and lever voting machines, to train
poll workers, to establish accurate statewide lists of registered voters
and to make polling places more accessible to people with disabilities.

State officials warned that the bill's promise would go unfulfilled
without an infusion of federal money. Pending appropriations bills do
not provide enough, the National Conference of State Legislatures said.

The election bill was born of a bipartisan determination to avoid
another debacle like the one in Florida, which put the outcome of the
presidential election in doubt for more than a month and left painful
memories of butterfly ballots and hanging chads.

Senator Christopher J. Dodd, the Connecticut Democrat who is the bill's
chief Senate sponsor, described it as ''the first civil rights act of
the 21st century.'' In a joint statement, former Presidents Jimmy Carter
and Gerald R. Ford said the bill promised ''the most meaningful
improvements in voting safeguards since the civil rights laws of the
1960's.''

Representative Bob Ney, Republican of Ohio, the chief sponsor of the
House bill, said that many states had problems like those in Florida:
old and unreliable voting equipment, inadequate training and education
of voters and poll workers, duplicate registrations for some people,
improper purges of others from voting records.

The senators from New York, Charles E. Schumer and Hillary Rodham
Clinton, both Democrats, cast the only no votes today. They said they
feared that some provisions of the bill, including new identification
requirements, would raise hurdles to registration and voting by poor
people and members of minority groups, especially Hispanics.

''On balance,'' Mr. Schumer said, ''I thought this was a bad bill for
New York, although probably a good bill for the country.'' But with
growing budget deficits and the possibility of a war with Iraq, he said,
''it is highly unlikely'' that Congress will provide all the money
authorized.

Mrs. Clinton said the identification requirements in the bill would
probably ''repress voter participation'' by recently naturalized
American citizens, homeless people and millions of New Yorkers who have
no driver's license.

Civil rights groups were divided. The Congressional Black Caucus and the
National Association for the Advancement of Colored People supported the
legislation. But major Latino groups and the American Civil Liberties
Union strenuously opposed it. Among Hispanic members of the House, six
voted against the final bill and seven voted for it, but expressed
reservations.

Republicans enthusiastically supported the legislation, especially after
the Senate added provisions to crack down on vote fraud. Republicans
said that such fraud, though widespread, had often been overlooked or
minimized.

''This legislation recognizes that illegal votes dilute the value of
legally cast votes,'' said Senator Christopher S. Bond, Republican of
Missouri. ''If your vote is canceled by the vote of a dog or a dead
person, it's as if you did not have a right to vote.''

In Florida in 2000, many people were turned away from the polls. The
bill that was passed today, the Help America Vote Act, says that a
person not listed on the registration rolls must be allowed to cast a
provisional ballot. The ballot would be counted if state or local
officials could confirm that the person was eligible to vote under state
law.

In addition, the bill says, voters must be allowed to check their
ballots and correct any errors before the ballots are cast. Lawmakers
said this would reduce the number of ballots thrown out because a voter
erroneously marked or punched the ballot for more than one candidate.

Under the bill, new voters who registered by mail would have to provide
proof of identity at some point in the process, when they register or
when they vote, in person or by mail. Voters could present a photo
identification card, a utility bill, a bank statement, a paycheck or a
government document showing name and address.

Anyone who registers to vote would have to provide the state with a
driver's license number or, if the person does not have a license, the
last four digits of his or her Social Security number. Election
officials must verify the information with state motor vehicle agencies
or the Social Security Administration.

An application for voter registration ''may not be accepted or processed
by a state'' if a person with a driver's license fails to write the
license number on the application form, the bill says.

Another provision says that all state mail-in registration forms must
include the question, ''Are you a citizen of the United States of
America?'' with boxes to answer yes or no. Only citizens are eligible to
vote in federal elections.

The bill would create a federal agency, the Election Assistance
Commission, to test voting equipment and to serve as a clearinghouse for
information on election technology. The agency would provide ''voluntary
guidance'' to states but could not issue rules with the force of law.

In Florida two years ago, counties had different standards for deciding
whether to count punch-card ballots with so-called hanging chads. The
bill passed today says that each state must ''adopt uniform and
nondiscriminatory standards that define what constitutes a vote and what
will be counted as a vote.''

Representative Eddie Bernice Johnson, the Texas Democrat who is
chairwoman of the black caucus, said, ''This bill goes a long way toward
correcting the disenfranchisement that we witnessed in 2000.''

Advertisement

\protect\hyperlink{after-bottom}{Continue reading the main story}

\hypertarget{site-index}{%
\subsection{Site Index}\label{site-index}}

\hypertarget{site-information-navigation}{%
\subsection{Site Information
Navigation}\label{site-information-navigation}}

\begin{itemize}
\tightlist
\item
  \href{https://help.nytimes.com/hc/en-us/articles/115014792127-Copyright-notice}{©~2020~The
  New York Times Company}
\end{itemize}

\begin{itemize}
\tightlist
\item
  \href{https://www.nytco.com/}{NYTCo}
\item
  \href{https://help.nytimes.com/hc/en-us/articles/115015385887-Contact-Us}{Contact
  Us}
\item
  \href{https://www.nytco.com/careers/}{Work with us}
\item
  \href{https://nytmediakit.com/}{Advertise}
\item
  \href{http://www.tbrandstudio.com/}{T Brand Studio}
\item
  \href{https://www.nytimes.com/privacy/cookie-policy\#how-do-i-manage-trackers}{Your
  Ad Choices}
\item
  \href{https://www.nytimes.com/privacy}{Privacy}
\item
  \href{https://help.nytimes.com/hc/en-us/articles/115014893428-Terms-of-service}{Terms
  of Service}
\item
  \href{https://help.nytimes.com/hc/en-us/articles/115014893968-Terms-of-sale}{Terms
  of Sale}
\item
  \href{https://spiderbites.nytimes.com}{Site Map}
\item
  \href{https://help.nytimes.com/hc/en-us}{Help}
\item
  \href{https://www.nytimes.com/subscription?campaignId=37WXW}{Subscriptions}
\end{itemize}
