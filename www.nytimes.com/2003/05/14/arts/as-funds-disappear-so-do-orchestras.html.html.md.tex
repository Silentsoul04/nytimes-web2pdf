Sections

SEARCH

\protect\hyperlink{site-content}{Skip to
content}\protect\hyperlink{site-index}{Skip to site index}

\href{https://www.nytimes.com/section/arts}{Arts}

\href{https://myaccount.nytimes.com/auth/login?response_type=cookie\&client_id=vi}{}

\href{https://www.nytimes.com/section/todayspaper}{Today's Paper}

\href{/section/arts}{Arts}\textbar{}As Funds Disappear, So Do Orchestras

\begin{itemize}
\item
\item
\item
\item
\item
\end{itemize}

Advertisement

\protect\hyperlink{after-top}{Continue reading the main story}

Supported by

\protect\hyperlink{after-sponsor}{Continue reading the main story}

\hypertarget{as-funds-disappear-so-do-orchestras}{%
\section{As Funds Disappear, So Do
Orchestras}\label{as-funds-disappear-so-do-orchestras}}

By \href{https://www.nytimes.com/by/stephen-kinzer}{Stephen Kinzer}

\begin{itemize}
\item
  May 14, 2003
\item
  \begin{itemize}
  \item
  \item
  \item
  \item
  \item
  \end{itemize}
\end{itemize}

After the Florida Philharmonic Orchestra finished playing Wagner and
Tchaikovsky to an audience of 1,300 in Boca Raton on Friday, the
orchestra's executive director, Trey Devey, took to the stage to
announce that this might be its last concert.

''If you have the potential to help us and be a hero, then call us,''
Mr. Devey pleaded. ''We need a hero.''

No one called, at least no one with the necessary resources. Later Mr.
Devey issued a statement saying the orchestra was ''temporarily
suspending operations and terminating the employment of musicians.''

The apparent collapse of the Florida Philharmonic, the only major
orchestra in South Florida, is the latest in a series of tremors that
have shaken the symphonic world this season. Nearly a dozen orchestras
across the country have either closed or are in danger of doing so.

This season's first orchestral casualty was the San Jose Symphony, which
shut down in November. The Tulsa Philharmonic, the Colorado Springs
Symphony and the San Antonio Symphony followed.

In February the 49-year-old Savannah Symphony Orchestra canceled the
rest of its season. It was \$1.3 million in debt, had gone through five
executive directors in seven years and was unable to meet its payroll.

The musicians of the Houston Symphony went on strike for three weeks in
March and April and in the end were forced to settle for a contract that
imposed sharp curbs on wages and benefits. Their counterparts at the
Baltimore Symphony accepted a similarly harsh contract after what their
union leader called ''difficult and painful consideration.''

The Pittsburgh Symphony, one of the country's major ensembles, is facing
a \$2 million deficit, and its board has proposed selling its concert
hall. It will also begin what are likely to be difficult labor
negotiations this summer.

Earlier this month dozens of musicians from the 66-year-old Louisville
Orchestra appeared in formal attire at the city's unemployment office to
file for benefits. They had not been paid for three weeks, and their
orchestra faces an \$800,000 deficit.

Several orchestras, including the New York Philharmonic, have recently
issued emergency appeals to donors. Although the Philharmonic is in no
danger of collapse, some others say they may not reopen next season if
their appeals are not successful.

Orchestra administrators blame their woes on the weak economy, but
critics say many of them have failed to adapt to changing times.

Ed Wulfe, a Houston real estate developer who helped mediate the dispute
in his city, said ''a combination of lethargy and 'that's the way it's
always been done' thinking'' had shaped the culture of the Houston
Symphony ''and probably a lot of orchestras across the country.''

''This is a competitive world, and we've got to find ways to tell the
story better, to get out into the community, to reach out to new
audiences,'' Mr. Wulfe said. ''It takes people with some imagination.''

The plight of the Houston Symphony reflects the challenges that
orchestras are confronting across the country. Houston, the nation's
fourth-largest city, has had a full-size symphony orchestra since 1972.
Mr. Wulfe said the orchestra was vital. ''We realize that to attract
business, to attract broad-minded, innovative, creative people, we have
to offer more than just a job,'' he said.

Critics of orchestra management are not so sure. They suggest that if a
city cannot come up with the money to support a symphony orchestra,
perhaps it does not need one.

Some cultural figures say it is hard to sell classical music in places
where much of the population has no direct connection to the northern
European cultures that produced most of it. Others lament that many
universities today emphasize career training over the humanities,
allowing students to reach adulthood without any exposure to fine arts.
Still others share Mr. Wulfe's view that orchestra administrators are
too slow moving and unimaginative.

''I don't think there's a deep systemic problem that's unique to
symphony orchestras, since airlines and hotel chains and hockey teams
are also suffering from this economic downturn,'' said Henry Fogel, who
is about to leave the top administrative post at the Chicago Symphony to
become president of the American Symphony Orchestra League. ''During the
great economy of the 1990's orchestras perhaps expanded faster than they
should have. I believe that in really good economic times orchestras
should not spend up to their revenues and should instead go for a
surplus and keep a cushion for the future. I wish I'd believed that 10
or 15 years ago.''

Mr. Fogel said some orchestras were run by insular groups that did not
inspire donor confidence. ''Are there enough people in a city who want
an orchestra and are willing to support it?'' he asked. ''That becomes a
difficult and tricky question when you ask whether an administration or
board has done all that can be done in garnering that support, or if it
has not done enough.''

Those questions have been raised repeatedly in Houston. Ann Kennedy, who
was hired in 2001 as the Houston Symphony's executive director, had
never before run an arts organization, and some active in the city's
cultural life said she had not managed to energize either the orchestra
or donors. Her board is also considered weaker than those guiding the
Museum of Fine Arts and other more successful Houston arts institutions.

''Somehow this board didn't become the board to go on in Houston, and
that's very difficult to fix,'' said an arts administrator who spoke on
the condition of anonymity because he deals regularly with the Houston
Symphony. ''When you have the wrong mix, the people of real power don't
become interested in joining. It's going to take somebody of importance
in the community stepping forward and taking responsibility for
reshaping the board, and the board allowing itself to be reshaped.''

The musicians' strike in Houston ended with the signing of a contract
that calls for the cancellation of 10 concerts in the season, cuts in
staff size and salaries, higher health care premiums and mandatory
unpaid furloughs for musicians.

Partly because of tensions stemming from the labor dispute, nearly a
dozen musicians have left the orchestra in recent months, either
permanently or temporarily. A union survey suggested that 80 percent
would leave if they had a good offer elsewhere. ''The mood has
changed,'' said Houston's music director, Hans Graf.

This season has forced many orchestra administrators to recognize the
need for a new approach, but they are uncertain what it should be.

''There has to be a sea change in the way these organizations are run,''
said Jeffrey B. Early, a banker who is completing a two-year term as
president of the Houston Symphony's board. ''We have to find ways of
putting more people in the seats.''

Mr. Early said he hoped the Houston orchestra's new associate conductor,
Carlos Miguel Prieto, who was formerly music director of the Xalapa
Symphony Orchestra and associate conductor of the Mexico City
Philharmonic, would find ways to attract more Hispanic patrons.

''And we've got Beethoven's 'Ode to Joy' coming up,'' he added. ''We
ought to be able to fill the hall with that.''

Advertisement

\protect\hyperlink{after-bottom}{Continue reading the main story}

\hypertarget{site-index}{%
\subsection{Site Index}\label{site-index}}

\hypertarget{site-information-navigation}{%
\subsection{Site Information
Navigation}\label{site-information-navigation}}

\begin{itemize}
\tightlist
\item
  \href{https://help.nytimes.com/hc/en-us/articles/115014792127-Copyright-notice}{©~2020~The
  New York Times Company}
\end{itemize}

\begin{itemize}
\tightlist
\item
  \href{https://www.nytco.com/}{NYTCo}
\item
  \href{https://help.nytimes.com/hc/en-us/articles/115015385887-Contact-Us}{Contact
  Us}
\item
  \href{https://www.nytco.com/careers/}{Work with us}
\item
  \href{https://nytmediakit.com/}{Advertise}
\item
  \href{http://www.tbrandstudio.com/}{T Brand Studio}
\item
  \href{https://www.nytimes.com/privacy/cookie-policy\#how-do-i-manage-trackers}{Your
  Ad Choices}
\item
  \href{https://www.nytimes.com/privacy}{Privacy}
\item
  \href{https://help.nytimes.com/hc/en-us/articles/115014893428-Terms-of-service}{Terms
  of Service}
\item
  \href{https://help.nytimes.com/hc/en-us/articles/115014893968-Terms-of-sale}{Terms
  of Sale}
\item
  \href{https://spiderbites.nytimes.com}{Site Map}
\item
  \href{https://help.nytimes.com/hc/en-us}{Help}
\item
  \href{https://www.nytimes.com/subscription?campaignId=37WXW}{Subscriptions}
\end{itemize}
