Sections

SEARCH

\protect\hyperlink{site-content}{Skip to
content}\protect\hyperlink{site-index}{Skip to site index}

\href{https://www.nytimes.com/section/business}{Business}

\href{https://myaccount.nytimes.com/auth/login?response_type=cookie\&client_id=vi}{}

\href{https://www.nytimes.com/section/todayspaper}{Today's Paper}

\href{/section/business}{Business}\textbar{}Talks Implode During a Day
of Chaos; Fate of Bailout Plan Remains Unresolved

\begin{itemize}
\item
\item
\item
\item
\item
\item
\end{itemize}

Advertisement

\protect\hyperlink{after-top}{Continue reading the main story}

Supported by

\protect\hyperlink{after-sponsor}{Continue reading the main story}

\hypertarget{talks-implode-during-a-day-of-chaos-fate-of-bailout-plan-remains-unresolved}{%
\section{Talks Implode During a Day of Chaos; Fate of Bailout Plan
Remains
Unresolved}\label{talks-implode-during-a-day-of-chaos-fate-of-bailout-plan-remains-unresolved}}

\includegraphics{https://static01.nyt.com/images/2008/09/25/business/26abush4_600.jpg?quality=75\&auto=webp\&disable=upscale}

By \href{https://www.nytimes.com/by/david-m-herszenhorn}{David M.
Herszenhorn}, \href{https://www.nytimes.com/by/carl-hulse}{Carl Hulse}
and \href{https://www.nytimes.com/by/sheryl-gay-stolberg}{Sheryl Gay
Stolberg}

\begin{itemize}
\item
  Sept. 25, 2008
\item
  \begin{itemize}
  \item
  \item
  \item
  \item
  \item
  \item
  \end{itemize}
\end{itemize}

\emph{This article was reported by} \emph{\emph{\textbf{David M.
Herszenhorn}},} \textbf{\textbf{Carl Hulse}} \emph{and}
\textbf{\textbf{Sheryl Gay Stolberg}} \emph{and written by Ms.
Stolberg.}

WASHINGTON --- The day began with an agreement that Washington hoped
would end the financial crisis that has gripped the nation. It dissolved
into a verbal brawl in the Cabinet Room of the White House, urgent
warnings from the president and pleas from a Treasury secretary who
knelt before the House speaker and appealed for her support.

``If money isn't loosened up, this sucker could go down,'' President
Bush declared Thursday as he watched the \$700 billion bailout package
fall apart before his eyes, according to one person in the room.

It was an implosion that spilled out from behind closed doors into
public view in a way rarely seen in Washington.

By 10:30 p.m., after another round of talks, Congressional negotiators
gave up for the night and said they would try again on Friday. Left
uncertain was the fate of the bailout, which the White House says is
urgently needed to fix broken financial and credit markets, as well as
whether the first presidential debate would go forward as planned Friday
night in Mississippi.

When Congressional leaders and Senators John McCain and Barack Obama,
the two major party presidential candidates, trooped to the White House
on Thursday afternoon, most signs pointed toward a bipartisan agreement
on a grand compromise that could be accepted by all sides and signed
into law by the weekend. It was intended to pump billions of dollars
into the financial system, restoring liquidity and keeping credit
flowing to businesses and consumers.

``We're in a serious economic crisis,'' Mr. Bush told reporters as the
meeting began shortly before 4 p.m. in the Cabinet Room, adding, ``My
hope is we can reach an agreement very shortly.''

But once the doors closed, the smooth-talking House Republican leader,
John A. Boehner of Ohio, surprised many in the room by declaring that
his caucus could not support the plan to allow the government to buy
distressed mortgage assets from ailing financial companies.

Mr. Boehner pressed an alternative that involved a smaller role for the
government, and Mr. McCain, whose support of the deal is critical if
fellow Republicans are to sign on, declined to take a stand.

The talks broke up in angry recriminations, according to accounts
provided by a participant and others who were briefed on the session,
and were followed by dueling news conferences and interviews rife with
partisan finger-pointing.

Friday morning, on CBS's ``The Early Show,'' Representative Barney Frank
of Massachusetts, the lead Democratic negotiator, said the bailout had
been derailed by internal Republican politics.

``I didn't know I was going to be the referee for an internal G.O.P.
ideological civil war,'' Mr. Frank said, according to The A.P.Thursday,
in the Roosevelt Room after the session, the Treasury secretary, Henry
M. Paulson Jr., literally bent down on one knee as he pleaded with Nancy
Pelosi, the House Speaker, not to ``blow it up'' by withdrawing her
party's support for the package over what Ms. Pelosi derided as a
Republican betrayal.

``I didn't know you were Catholic,'' Ms. Pelosi said, a wry reference to
Mr. Paulson's kneeling, according to someone who observed the exchange.
She went on: ``It's not me blowing this up, it's the Republicans.''

Mr. Paulson sighed. ``I know. I know.''

It was the very outcome the White House had said it intended to avoid,
with partisan presidential politics appearing to trample what had been
exceedingly delicate Congressional negotiations.

Senator Christopher J. Dodd, Democrat of Connecticut and chairman of the
Senate banking committee, denounced the session as ``a rescue plan for
John McCain,'' and proclaimed it a waste of precious hours that could
have been spent negotiating.

But a top aide to Mr. Boehner said it was Democrats who had done the
political posturing. The aide, Kevin Smith, said Republicans revolted,
in part, because they were chafing at what they saw as an attempt by
Democrats to jam through an agreement on the bailout early Thursday and
deny Mr. McCain an opportunity to participate in the agreement.

The day seemed to hold promise as it began. On Wednesday night, Mr. Bush
had delivered a prime-time televised address to the nation, warning that
''our country could experience a long and painful recession'' if
lawmakers did not act quickly to pass a huge Wall Street bailout plan.

After spending Thursday morning behind closed doors, senior lawmakers
from both parties emerged shortly before 1 p.m. in the ornate painted
corridors on the first floor of the Capitol to herald their agreement on
the broad outlines of a deal.

They said the legislation, which would authorize unprecedented
government intervention to buy distressed debt from private firms, would
include limits on pay packages for executives of some firms that seek
assistance and a mechanism for the government to take an equity stake in
some of the firms, so taxpayers have a chance to profit if the bailout
plan works.

``I now expect we will indeed have a plan that can pass the House, pass
the Senate, be signed by the president, and bring a sense of certainty
to this crisis that is still roiling in the markets,'' said Robert F.
Bennett, Republican of Utah, a member of the banking committee.

He made a point of describing that meeting as free of political
maneuvering. ``It was one of the most productive sessions in that regard
that I have participated in since I have been in the Senate,'' Mr.
Bennett said.

But a few blocks away, a senior House Republican lawmaker was at a
luncheon with reporters, saying his caucus would never go along with the
deal. This Republican said Representative Eric Cantor of Virginia, the
chief deputy whip, was circulating an alternative course that would rely
on government-backed insurance, not taxpayer-financed purchase of
mortgage assets.

He said the recalcitrant Republicans were calculating that Ms. Pelosi,
Democrat of California, would not want to leave her caucus politically
exposed in an election season by passing a bailout bill without
rank-and-file Republican support.

``You can have all the meetings you want,'' this Republican said,
referring to the White House session with Mr. Bush, the presidential
candidates and Congressional leaders, still hours away. ``It comes to
the floor and the votes aren't there. It won't pass.''

House Republicans have spent days expressing their unease about a huge
government intervention, which they regard as a step down the path to
socialism.

Mr. Smith, the aide to Mr. Boehner, said the leader had directed a group
of Republicans a few days ago to see whether they could come up with
alternatives that relied less on tax funds in providing the rescue
package; that led to Mr. Cantor's mortgage-insurance approach. He said
Mr. Boehner thought Mr. Cantor's idea should be taken into consideration
in the talks.

At 4 p.m., Mr. Bush convened his meeting at the White House; Mr. McCain
had already met with House Republicans to hear their concerns. He later
said on ABC that he had known going into the White House that ``there
never was a deal,'' but he kept that sentiment to himself.

The meeting opened with Mr. Paulson, the chief architect of the bailout
plan, ``giving a status report on the condition of the market,'' Tony
Fratto, Mr. Bush's deputy press secretary, said. Mr. Fratto said Mr.
Paulson warned in particular of the tightening of credit markets
overnight, adding, ``that is something very much on his mind.''

Mr. McCain was at one end of the long conference table, Mr. Obama at the
other, with the president and senior Congressional leaders between them.
Participants said Mr. Obama peppered Mr. Paulson with questions, while
Mr. McCain said little. Outside the West Wing, a huge crowd of reporters
gathered in the driveway, anxiously awaiting an appearance by either
presidential candidate, with expectations running high.

Instead, the first politician to emerge was Senator Richard C. Shelby of
Alabama, the senior Republican on the banking committee, waving a sheet
of paper that he said detailed his own concerns. ``The agreement,'' Mr.
Shelby declared, ''is obviously no agreement.''

Friday morning, on the CBS morning program, Mr. Shelby said,
``Basically, I believe the Paulson proposal is badly structured.''

``It does nothing basically for the stressed mortgage payer,'' he said,
according to The A.P.

The House Republicans' revolt shocked Democrats; the Senate majority
leader, Harry Reid of Nevada, said later that he was under the
impression that Mr. Boehner had been a strong advocate for moving
forward with the Paulson plan.

Mr. Frank, who attended the White House meeting, was shocked as well.
``We were ready to make a deal,'' Mr. Frank said later.

At 8 p.m., an exasperated Mr. Frank walked back to the Rules Committee
room on the second floor of the Senate side of the Capitol, with a pack
of reporters on his heels. He was headed for another late-night meeting
with Mr. Paulson and many other lawmakers to see whether they could
restart the negotiations --- and ward off a Friday morning bloodbath in
the markets.

Ms. Pelosi told reporters that she was open to considering ideas
proposed by the House Republicans. And Mr. McCain and Mr. Obama both
said they held out hope that a deal could be reached soon.

At the White House, Mr. Bush was holding fast to the approach that Mr.
Paulson has championed.

``In case there's any confusion,'' Mr. Fratto, the deputy press
secretary, wrote in an e-mail message. ``The president supports the core
of Secretary Paulson's plan.''

Advertisement

\protect\hyperlink{after-bottom}{Continue reading the main story}

\hypertarget{site-index}{%
\subsection{Site Index}\label{site-index}}

\hypertarget{site-information-navigation}{%
\subsection{Site Information
Navigation}\label{site-information-navigation}}

\begin{itemize}
\tightlist
\item
  \href{https://help.nytimes.com/hc/en-us/articles/115014792127-Copyright-notice}{©~2020~The
  New York Times Company}
\end{itemize}

\begin{itemize}
\tightlist
\item
  \href{https://www.nytco.com/}{NYTCo}
\item
  \href{https://help.nytimes.com/hc/en-us/articles/115015385887-Contact-Us}{Contact
  Us}
\item
  \href{https://www.nytco.com/careers/}{Work with us}
\item
  \href{https://nytmediakit.com/}{Advertise}
\item
  \href{http://www.tbrandstudio.com/}{T Brand Studio}
\item
  \href{https://www.nytimes.com/privacy/cookie-policy\#how-do-i-manage-trackers}{Your
  Ad Choices}
\item
  \href{https://www.nytimes.com/privacy}{Privacy}
\item
  \href{https://help.nytimes.com/hc/en-us/articles/115014893428-Terms-of-service}{Terms
  of Service}
\item
  \href{https://help.nytimes.com/hc/en-us/articles/115014893968-Terms-of-sale}{Terms
  of Sale}
\item
  \href{https://spiderbites.nytimes.com}{Site Map}
\item
  \href{https://help.nytimes.com/hc/en-us}{Help}
\item
  \href{https://www.nytimes.com/subscription?campaignId=37WXW}{Subscriptions}
\end{itemize}
