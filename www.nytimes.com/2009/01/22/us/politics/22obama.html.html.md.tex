Sections

SEARCH

\protect\hyperlink{site-content}{Skip to
content}\protect\hyperlink{site-index}{Skip to site index}

\href{https://www.nytimes.com/section/politics}{Politics}

\href{https://myaccount.nytimes.com/auth/login?response_type=cookie\&client_id=vi}{}

\href{https://www.nytimes.com/section/todayspaper}{Today's Paper}

\href{/section/politics}{Politics}\textbar{}On First Day, Obama Quickly
Sets a New Tone

\begin{itemize}
\item
\item
\item
\item
\item
\end{itemize}

Advertisement

\protect\hyperlink{after-top}{Continue reading the main story}

Supported by

\protect\hyperlink{after-sponsor}{Continue reading the main story}

\hypertarget{on-first-day-obama-quickly-sets-a-new-tone}{%
\section{On First Day, Obama Quickly Sets a New
Tone}\label{on-first-day-obama-quickly-sets-a-new-tone}}

\includegraphics{https://static01.nyt.com/images/2009/01/21/us/21obama5-600.jpg?quality=75\&auto=webp\&disable=upscale}

By \href{https://www.nytimes.com/by/sheryl-gay-stolberg}{Sheryl Gay
Stolberg}

\begin{itemize}
\item
  Jan. 21, 2009
\item
  \begin{itemize}
  \item
  \item
  \item
  \item
  \item
  \end{itemize}
\end{itemize}

WASHINGTON --- President Obama moved swiftly on Wednesday to impose new
rules on government transparency and ethics, using his first full day in
office to freeze the salaries of his senior aides, mandate new limits on
lobbyists and demand that the government disclose more information.

Mr. Obama called the moves, which overturned two policies of his
predecessor, ``a clean break from business as usual.'' Coupled with
Tuesday's Inaugural Address, which repudiated the Bush administration's
decisions on everything from science policy to fighting terrorism, the
actions were another sign of the new president's effort to emphasize an
across-the-board shift in priorities, values and tone.

``For a long time now there's been too much secrecy in this city,'' Mr.
Obama said at a swearing-in ceremony for senior officials at the
Eisenhower Executive Office Building, adjacent to the White House. He
added, ``Transparency and rule of law will be the touchstones of this
presidency.''

With the pageantry of Tuesday's inaugural festivities behind them, Mr.
Obama and his team spent Wednesday grappling with matters as mundane as
e-mail access and getting to work (some aides arrived at the gates of
1600 Pennsylvania Avenue on Tuesday morning to discover they lacked
clearance to enter) and as weighty as Senate confirmation of cabinet
secretaries.

On Capitol Hill, Hillary Rodham Clinton was confirmed by the Senate as
Mr. Obama's secretary of state --- and later sworn in --- and it
appeared that Timothy F. Geithner, the Treasury secretary nominee, was
headed for confirmation. But Republicans forced a one-week delay in the
vote on Mr. Obama's nominee for attorney general, Eric H. Holder Jr.,
and there are other jobs yet to fill, including that of commerce
secretary.

Image

Smile? Even before President Obama signed his first executive orders, a
pen was immortalized.Credit...Doug Mills/The New York Times

The transparency and ethics moves were set forth in two executive orders
and three presidential memorandums; Mr. Obama signed them at the
swearing-in ceremony with a left-handed flourish.

The new president effectively reversed a post-9/11 Bush administration
policy making it easier for government agencies to deny requests for
records under the Freedom of Information Act, and effectively repealed a
Bush executive order that allowed former presidents or their heirs to
claim executive privilege in an effort to keep records secret.

``Starting today,'' Mr. Obama said, ``every agency and department should
know that this administration stands on the side not of those who seek
to withhold information, but those who seek to make it known.''

Advocates for openness in government, who had been pressing for the
moves, said they were pleased. They said the new president had traded a
presumption of secrecy for a presumption of disclosure.

``You couldn't ask for anything better,'' said Melanie Sloan, the
executive director of Citizens for Responsibility and Ethics in
Washington, an advocacy group that tangled frequently with the Bush
administration over records. ``For the president to say this on Day 1
says: `We mean it. Turn your records over.'~''

A president's first act in office carries great symbolism. Aides to Mr.
Obama spent weeks debating a variety of options including an executive
order to shut down the prison at Guantánamo Bay, Cuba --- a decision
that is now expected to come on Thursday.

In the end, Mr. Obama used his first day to send two messages that
echoed themes from his campaign: first, that he is intent on keeping his
promises to run a clean and open government; and, second, that he
understands the pain Americans are feeling as a result of the economic
crisis.

``These executive orders are traditional for presidents --- we did them
the first day as have others,'' said Dan Bartlett, who was counselor to
President George W. Bush. ``But he has decided to put a finer point on
it by elevating a clear theme from his campaign, which was, `We're not
going to do business as usual.' I think it's a smart move, and the type
of thing that the public wants to hear right now.''

It may not be the type of thing that Mr. Bush wants to hear, however.
Experts said Mr. Obama's moves would have the practical effect of
allowing reporters and historians to obtain access to records from the
Bush administration that might otherwise have been kept under wraps.

``Historians are overjoyed by this,'' said Lee White, executive director
of the National Coalition for History.

In announcing the salary freeze, Mr. Obama effectively gave pay cuts to
roughly 100 top executive branch officials, like the national security
adviser, the press secretary and the White House counsel, who earn more
than \$100,000 a year. ``Families are tightening their belts,'' Mr.
Obama said, ``and so should Washington.''

The new president also moved to fulfill his campaign pledge to end the
so-called revolving door, the longstanding Washington practice whereby
White House officials depart for the private sector and cash in on their
connections by lobbying former colleagues.

Video

President Barack Obama discussed new rules on government transparency as
well as strict regulations governing the activities of lobbyists in the
executive branch.

In what ethics-in-government advocates described as a particularly
far-reaching move, Mr. Obama barred officials of his administration from
lobbying their former colleagues ``for as long as I am president.'' He
barred former lobbyists from working for agencies they had lobbied
within the past two years and required them to recuse themselves from
issues they had handled during that time.

The Republican National Committee criticized the Obama administration
for violating this new standard in some of its appointments. Mr. Obama's
nominee for deputy secretary of defense, William Lynn, has been a
lobbyist for the defense contractor Raytheon, and his nominee for deputy
secretary of health and human services, William V. Corr, lobbied for
stricter tobacco regulations as an official with the Campaign for
Tobacco-Free Kids.

A senior White House official, speaking on the condition of anonymity,
conceded the two nominees did not adhere to the new rules. But he said
that Mr. Lynn had the support of Republicans and Democrats, and would
receive a waiver under the policy, and that Mr. Corr did not need a
waiver because he had agreed to recuse himself from tobacco issues.

``When you set very tough rules, you need to have a mechanism for the
occasional exception,'' this official said, adding, ``We wanted to be
really tough, but at the same time we didn't want to hamstring the new
administration or turn the town upside down.''

Mr. Obama's pledge for openness and transparency also ran smack into the
stark reality that setting up a new administration takes time. During
his campaign, Candidate Obama and his team of technically savvy young
aides promised to harness the power of the Internet to allow the public
easy access to government documents and presidential decisions.

It took six hours on Tuesday for the ordinarily fast-moving aides to Mr.
Obama to post his executive orders on the White House Web site. Until
then, the site declared, ``The president has not issued any executive
orders.''

Advertisement

\protect\hyperlink{after-bottom}{Continue reading the main story}

\hypertarget{site-index}{%
\subsection{Site Index}\label{site-index}}

\hypertarget{site-information-navigation}{%
\subsection{Site Information
Navigation}\label{site-information-navigation}}

\begin{itemize}
\tightlist
\item
  \href{https://help.nytimes.com/hc/en-us/articles/115014792127-Copyright-notice}{©~2020~The
  New York Times Company}
\end{itemize}

\begin{itemize}
\tightlist
\item
  \href{https://www.nytco.com/}{NYTCo}
\item
  \href{https://help.nytimes.com/hc/en-us/articles/115015385887-Contact-Us}{Contact
  Us}
\item
  \href{https://www.nytco.com/careers/}{Work with us}
\item
  \href{https://nytmediakit.com/}{Advertise}
\item
  \href{http://www.tbrandstudio.com/}{T Brand Studio}
\item
  \href{https://www.nytimes.com/privacy/cookie-policy\#how-do-i-manage-trackers}{Your
  Ad Choices}
\item
  \href{https://www.nytimes.com/privacy}{Privacy}
\item
  \href{https://help.nytimes.com/hc/en-us/articles/115014893428-Terms-of-service}{Terms
  of Service}
\item
  \href{https://help.nytimes.com/hc/en-us/articles/115014893968-Terms-of-sale}{Terms
  of Sale}
\item
  \href{https://spiderbites.nytimes.com}{Site Map}
\item
  \href{https://help.nytimes.com/hc/en-us}{Help}
\item
  \href{https://www.nytimes.com/subscription?campaignId=37WXW}{Subscriptions}
\end{itemize}
