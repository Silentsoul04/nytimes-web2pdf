Sections

SEARCH

\protect\hyperlink{site-content}{Skip to
content}\protect\hyperlink{site-index}{Skip to site index}

\href{https://www.nytimes.com/section/world/europe}{Europe}

\href{https://myaccount.nytimes.com/auth/login?response_type=cookie\&client_id=vi}{}

\href{https://www.nytimes.com/section/todayspaper}{Today's Paper}

\href{/section/world/europe}{Europe}\textbar{}Scandal in House of Lords
rocks Labour Party

\begin{itemize}
\item
\item
\item
\item
\item
\end{itemize}

Advertisement

\protect\hyperlink{after-top}{Continue reading the main story}

Supported by

\protect\hyperlink{after-sponsor}{Continue reading the main story}

\hypertarget{scandal-in-house-of-lords-rocks-labour-party}{%
\section{Scandal in House of Lords rocks Labour
Party}\label{scandal-in-house-of-lords-rocks-labour-party}}

By \href{https://www.nytimes.com/by/john-f-burns}{John F. Burns}

\begin{itemize}
\item
  Feb. 9, 2009
\item
  \begin{itemize}
  \item
  \item
  \item
  \item
  \item
  \end{itemize}
\end{itemize}

\textbf{LONDON ---} At a meeting late last year at the House of Lords,
Thomas Taylor, a peer and stalwart of the governing Labour Party, told
visitors who introduced themselves as representatives of a Hong Kong
businessman seeking tax relief on an investment in Britain that "you've
got to whet my appetite to get me on board." He added that some
companies he had worked with paid him the equivalent of about \$140,000.

"That's cheap for what I do for them," he said.

In reality, the visitors to the Lords' guest room overlooking the Thames
were reporters for The Sunday Times, one of Britain's most widely
circulated newspapers. Their front-page account last weekend of their
meetings with Taylor and three other Labour Party peers who were said to
have agreed to accept payments for lobbying on behalf of the fictitious
businessman have sent shock waves through British politics.

All four of the men named by the newspaper have denied any wrongdoing,
and senior Labour officials have accused the paper of entrapment. The
House of Lords has begun an inquiry, including a review of creaky,
hard-to-carry-out procedures that require an act of Parliament to oust
miscreants in the chamber.

The newspaper's revelations appeared to stun Prime Minister Gordon
Brown, already facing a fresh plunge in Labour's opinion poll ratings as
Britain's financial crisis deepens; Brown's assertive handling of the
crisis' early stages had helped Labour pull nearly equal with the
opposition Conservatives.

"These are very serious allegations," Brown said. "Whatever action needs
to be taken will be taken."

Rumors of corruption, encouraged by loosely worded rules governing
lords' ability to engage in consulting work, have enveloped the chamber
for years. Members of the chamber are allowed to accept paid
consultancies so long as they concentrate their efforts outside
Parliament and do not lobby parliamentary colleagues or government
officials for legislative and regulatory changes that benefit their
employers - a distinction that critics say is impractical and difficult
to police.

Taylor, who has not disputed the accuracy of the transcription of the
reporters' meeting that the newspaper published, has insisted that the
arrangement he discussed did not breach the Lords' ethics rules and said
he would provide details during the Lords' inquiry.

For Labour, there is an irony in the political damage inflicted by the
affair, since the party has been in the forefront of decades of efforts
to overhaul or abolish the chamber. Established nearly 700 years ago,
the Lords was for centuries a bastion of Britain's ruling class of
land-owning aristocrats. That changed in the late 1990s when a Labour
plan ousted all but 90 handpicked hereditary lords and vested control of
the chamber in so-called life peers, appointed by the government from
careers in politics, public service, the arts and other professions.

But the Lords has remained an anomaly, with its 740 members, all
appointed, wearing ermine-trimmed robes on ceremonial occasions, and,
harsher critics maintain, using the chamber as a sort of retirement
home, with its plush red leather benches ideal for after-dinner
snoozing.

Fresh attempts at change, including a 2007 Labour bill that would have
established a new body to be called the Senate, with 80 percent to 100
percent of its members elected, have failed because of competing agendas
among top parties.

The Sunday Times said it had assigned reporters posing as lobbyists to
approach 10 peers. It said that four of five Labour members it
approached showed a willingness to take payments to help amend a pending
bill in ways that would lower taxes for the fictitious businessman in a
plan to open 30 shops.

The paper named the four peers: Taylor, 79; Lord Peter Truscott, 49, an
Oxford-educated former energy minister in Tony Blair's Labour
government; Lord Peter Snape, 66, a former railroad man and Labour whip
in the Commons; and Lord Lewis Moonie, 61, a psychiatrist and former
junior defense minister. It said three Conservative peers had not
answered the reporters' calls, and two members of smaller opposition
parties had rejected offers of payment, one of them saying angrily that
the offer was contrary to basic concepts of integrity.

As a furor over the allegations erupted this week, the Labour leader in
the Lords, Baroness Janet Royall, a former Blair government minister,
promised a "swift and vigorous" inquiry.

Some Lords veterans said the scandal was lending new urgency to reform
efforts. Although peers get no legislative salary, they are paid an
"attendance allowance" of about \$460 a day. But they are not provided
offices or support staff to assist in the chamber's work, which involves
detailed review of bills passed in the Commons. Lord Ivor Richard, a
barrister who is a former Labour leader in the Lords, told a BBC
interviewer that existing arrangements were ludicrous because they
rested on "conventions that go back to the 19th century."

"I mean, I do get quite hot under the collar about this," he said. "You
can never actually stave off the possibility of corruption; you can't do
it in the House of Commons, you can't do it in any legislature. But the
question to start off with is what sort of House of Lords, what sort of
second chamber, do you want? I want a properly constituted, functioning
legislative house, and we haven't got that at the moment."

Advertisement

\protect\hyperlink{after-bottom}{Continue reading the main story}

\hypertarget{site-index}{%
\subsection{Site Index}\label{site-index}}

\hypertarget{site-information-navigation}{%
\subsection{Site Information
Navigation}\label{site-information-navigation}}

\begin{itemize}
\tightlist
\item
  \href{https://help.nytimes.com/hc/en-us/articles/115014792127-Copyright-notice}{©~2020~The
  New York Times Company}
\end{itemize}

\begin{itemize}
\tightlist
\item
  \href{https://www.nytco.com/}{NYTCo}
\item
  \href{https://help.nytimes.com/hc/en-us/articles/115015385887-Contact-Us}{Contact
  Us}
\item
  \href{https://www.nytco.com/careers/}{Work with us}
\item
  \href{https://nytmediakit.com/}{Advertise}
\item
  \href{http://www.tbrandstudio.com/}{T Brand Studio}
\item
  \href{https://www.nytimes.com/privacy/cookie-policy\#how-do-i-manage-trackers}{Your
  Ad Choices}
\item
  \href{https://www.nytimes.com/privacy}{Privacy}
\item
  \href{https://help.nytimes.com/hc/en-us/articles/115014893428-Terms-of-service}{Terms
  of Service}
\item
  \href{https://help.nytimes.com/hc/en-us/articles/115014893968-Terms-of-sale}{Terms
  of Sale}
\item
  \href{https://spiderbites.nytimes.com}{Site Map}
\item
  \href{https://help.nytimes.com/hc/en-us}{Help}
\item
  \href{https://www.nytimes.com/subscription?campaignId=37WXW}{Subscriptions}
\end{itemize}
