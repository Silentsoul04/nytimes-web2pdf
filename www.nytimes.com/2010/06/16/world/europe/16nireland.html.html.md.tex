Sections

SEARCH

\protect\hyperlink{site-content}{Skip to
content}\protect\hyperlink{site-index}{Skip to site index}

\href{https://www.nytimes.com/section/world/europe}{Europe}

\href{https://myaccount.nytimes.com/auth/login?response_type=cookie\&client_id=vi}{}

\href{https://www.nytimes.com/section/todayspaper}{Today's Paper}

\href{/section/world/europe}{Europe}\textbar{}Cameron Calls N. Ireland
Killings `Unjustified'

\begin{itemize}
\item
\item
\item
\item
\item
\item
\end{itemize}

Advertisement

\protect\hyperlink{after-top}{Continue reading the main story}

Supported by

\protect\hyperlink{after-sponsor}{Continue reading the main story}

\hypertarget{cameron-calls-n-ireland-killings-unjustified}{%
\section{Cameron Calls N. Ireland Killings
`Unjustified'}\label{cameron-calls-n-ireland-killings-unjustified}}

\includegraphics{https://static01.nyt.com/images/2010/06/16/world/europe/NIRELAND/NIRELAND-articleLarge.jpg?quality=75\&auto=webp\&disable=upscale}

By \href{https://www.nytimes.com/by/john-f-burns}{John F. Burns}

\begin{itemize}
\item
  June 15, 2010
\item
  \begin{itemize}
  \item
  \item
  \item
  \item
  \item
  \item
  \end{itemize}
\end{itemize}

LONDON --- Prime Minister David Cameron offered an extraordinary apology
on Tuesday for the 1972 killings of 14 unarmed demonstrators by British
soldiers in Northern Ireland, saying that a long-awaited judicial
inquiry had left no doubt that the ``Bloody Sunday'' shootings were
``both unjustified and unjustifiable.''

``What happened should never, ever have happened,'' Mr. Cameron said in
a House of Commons statement. ``The families of those who died should
not have had to live with the pain and hurt of that day, and a lifetime
of loss. Some members of our armed forces acted wrongly. The government
is ultimately responsible for the conduct of the armed forces. And for
that, on behalf of the government --- and indeed our country --- I am
deeply sorry.''

While the inquiry seemed to settle the issue of responsibility for the
killings, the government in London will still have to tackle the
difficult question of whether any of the soldiers involved, or their
commanders, should be exposed to the possibility of criminal
prosecution, or be granted an indemnity, as the opposition Labour
Party's acting leader, Harriet Harman, urged in the Commons in her
response to Mr. Cameron's remarks.

The publication of the
\href{http://report.bloody-sunday-inquiry.org/}{5,000-page report}
plunged Mr. Cameron, in office barely a month, into the heart of
Northern Ireland's still volatile sectarian politics. Like previous
prime ministers going back decades, Mr. Cameron had to tread a wary path
for fear of reigniting tensions among Catholics and Protestants in
Northern Ireland, who have endured nearly 40 years of bitter dispute
over events in the city of Londonderry on Jan. 30, 1972.

On that Sunday, in events that were to generate an extensive archive of
feature films, documentaries, investigative books, popular songs and
poetry that helped build worldwide sympathy for Northern Ireland's
Catholics, a crowd of about 10,000 gathered to protest the practice of
detention without trial, used frequently by the British authorities to
curb those suspected of paramilitary extremism.

Image

A crowd in Londonderry, Northern Ireland, responded Tuesday to a report
on the 1972 shootings of protesters by British soldiers.Credit...Pool
photo by Paul Faith

The outburst of violence that followed effectively ended a nonviolent
campaign for civil rights and led to three decades of sectarian strife
that claimed more than 3,600 lives. Within weeks of the shootings,
another Conservative prime minister, Edward Heath, suspended the
Parliament in Belfast and imposed direct British rule, which lasted
until the 1998 Good Friday peace pact ushered in the new era of
power-sharing in Belfast.

The previous British government of Prime Minister Gordon Brown had
delayed publication of the report until after the May 6 general
election, fearing that the findings might stir political passions during
the campaign and undermine the power-sharing government established in
Belfast under the Good Friday agreement. One of the co-leaders of that
government, Martin McGuinness, was present at the site of the 1972
killings as an Irish Republican Army commander.

Mr. Cameron praised the overall performance of the 250,000 British
troops who served in Northern Ireland during the 30 years of violence.
He called their mission there ``the longest continuous military
operation in British military history,'' and noted that about 1,000
soldiers and policemen had been killed.

But he chose not to equivocate on the central issue of whether the
troops of Britain's crack Parachute Regiment had any justification for
opening fire with high-powered combat rifles on the Londonderry
demonstrators.

In effect, the prime minister endorsed almost every contention that the
victims' families had made over the decades: that the British commander
should not have ordered the troops to open fire; that the army fired the
first shots; that no warning was given before the army fusillade began;
that ``none of the casualties'' were carrying a firearm; and that some
soldiers had ``knowingly put forward false accounts'' of their actions.

In addition, Mr. Cameron quoted approvingly from sections in which the
high-ranking judge who led the inquiry, Lord Saville, 74, concluded that
although there was ``some firing'' by republican paramilitaries mixed in
with the protesters, ``none of this firing provided any justification
for the shooting of civilian casualties,'' and that none of the soldiers
fired ``in response to attack or threatened attacks by nail or petrol
bombers,'' as the soldiers and their lawyers had maintained.

Image

Londonderry, Jan. 30, 1972. On that day, some British troops shot
protesters. The report called the shootings
unjustified.Credit...Associated Press

Rather, the soldiers reacted to perceived threats from the protesters by
``losing their self-control,'' ``forgetting or ignoring their
instructions and training'' and with a ``serious and widespread loss of
fire discipline,'' the report said. The document described one of the
victims as having been shot while ``crawling away'' from the soldiers,
with another, ``in all probability,'' taking fire ``while he was lying
mortally wounded on the ground.''

Mr. Cameron, calling sections of the report ``shocking,'' said: ``You do
not defend the British Army by defending the indefensible. We do not
honor all those who have served with distinction in keeping the peace
and upholding the rule of law in Northern Ireland by hiding from the
truth.''

On the role of Mr. McGuinness, a point of particular political
sensitivity, the inquiry concluded that although he was present and
probably armed with a ``submachine gun,'' he did not ``engage in any
activity that provided any of the soldiers with any justification for
opening fire.'' Allegations to the contrary have fed years of
vilification of Mr. McGuinness by Protestant politicians.

The shootings have been angrily contested since the day they happened,
and especially after a report completed within weeks by a top British
judge, Lord Widgery, saying they had been provoked by demonstrators
using nail bombs and other weapons.

The Saville inquiry was commissioned in 1998 by Tony Blair, then prime
minister, as part of the negotiations that brought about the Good Friday
pact, and broke records for the 12 years it took to complete, the nearly
1,400 witnesses who gave evidence and the cost: \$280 million. More than
half of the money --- government funds --- went to the lawyers involved,
two of whom earned nearly \$6 million each.

In Londonderry, thousands of people gathered at the shooting site and
cheered as Mr. Cameron's speech was broadcast live on giant screens.
Repeated bursts of applause heralded sections of the report he read that
declared that all of those killed were innocents, and copies of the
Widgery report were shredded in front of the crowd by relatives of
victims.

Image

On what became known as Bloody Sunday, a British soldier grabbed a
Catholic protester; 14 other protesters were killed.Credit...Agence
France-Presse

One after another, relatives emphasized the youth of many of those
killed --- 7 of the 14 were teenagers --- and their innocence of any
wrongdoing. They used words like ``murdered'' and ``assassinated.''

``Thirty-eight years, four months, 15 days, almost to the minute ---
Kevin is innocent,'' said the sister of one of the victims, Kevin
McElhinney.

Mr. McGuinness, who joined the crowd, denied that he was carrying a gun
at the time of the shootings and said that the allegation originated
with ``British agents or with people who were very close to British
agents.''

He added, ``I think that the key message out of this was the courage and
heroism of the families who were prepared to stand for justice for their
loved ones and for the citizens of this city, who for almost 40 years
had been waiting for those who had been shot on that day to be
vindicated.''

Relatives of the victims left little doubt in their statements to the
crowd that they would press for murder trials, or at least for
prosecutions under an ``unlawful killing'' provision in British law ---
a step certain to provoke an angry reaction among the province's
Protestant politicians.

Mr. Cameron said it was an issue for the independent prosecution service
and made no mention of any government move to grant the soldiers
indemnity. ``These judgments are not matters for a tribunal --- or for
us as politicians --- to decide,'' he said.

Advertisement

\protect\hyperlink{after-bottom}{Continue reading the main story}

\hypertarget{site-index}{%
\subsection{Site Index}\label{site-index}}

\hypertarget{site-information-navigation}{%
\subsection{Site Information
Navigation}\label{site-information-navigation}}

\begin{itemize}
\tightlist
\item
  \href{https://help.nytimes.com/hc/en-us/articles/115014792127-Copyright-notice}{©~2020~The
  New York Times Company}
\end{itemize}

\begin{itemize}
\tightlist
\item
  \href{https://www.nytco.com/}{NYTCo}
\item
  \href{https://help.nytimes.com/hc/en-us/articles/115015385887-Contact-Us}{Contact
  Us}
\item
  \href{https://www.nytco.com/careers/}{Work with us}
\item
  \href{https://nytmediakit.com/}{Advertise}
\item
  \href{http://www.tbrandstudio.com/}{T Brand Studio}
\item
  \href{https://www.nytimes.com/privacy/cookie-policy\#how-do-i-manage-trackers}{Your
  Ad Choices}
\item
  \href{https://www.nytimes.com/privacy}{Privacy}
\item
  \href{https://help.nytimes.com/hc/en-us/articles/115014893428-Terms-of-service}{Terms
  of Service}
\item
  \href{https://help.nytimes.com/hc/en-us/articles/115014893968-Terms-of-sale}{Terms
  of Sale}
\item
  \href{https://spiderbites.nytimes.com}{Site Map}
\item
  \href{https://help.nytimes.com/hc/en-us}{Help}
\item
  \href{https://www.nytimes.com/subscription?campaignId=37WXW}{Subscriptions}
\end{itemize}
