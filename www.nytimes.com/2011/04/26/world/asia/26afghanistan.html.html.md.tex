Sections

SEARCH

\protect\hyperlink{site-content}{Skip to
content}\protect\hyperlink{site-index}{Skip to site index}

\href{https://www.nytimes.com/section/world/asia}{Asia Pacific}

\href{https://myaccount.nytimes.com/auth/login?response_type=cookie\&client_id=vi}{}

\href{https://www.nytimes.com/section/todayspaper}{Today's Paper}

\href{/section/world/asia}{Asia Pacific}\textbar{}Taliban Breach Afghan
Prison; Hundreds Free

\begin{itemize}
\item
\item
\item
\item
\item
\item
\end{itemize}

Advertisement

\protect\hyperlink{after-top}{Continue reading the main story}

Supported by

\protect\hyperlink{after-sponsor}{Continue reading the main story}

\hypertarget{taliban-breach-afghan-prison-hundreds-free}{%
\section{Taliban Breach Afghan Prison; Hundreds
Free}\label{taliban-breach-afghan-prison-hundreds-free}}

\includegraphics{https://static01.nyt.com/images/2011/04/26/world/26afghanistan1_cnd/26afghanistan1_cnd-articleLarge.jpg?quality=75\&auto=webp\&disable=upscale}

By Taimoor Shah and
\href{https://www.nytimes.com/by/alissa-j-rubin}{Alissa J. Rubin}

\begin{itemize}
\item
  April 25, 2011
\item
  \begin{itemize}
  \item
  \item
  \item
  \item
  \item
  \item
  \end{itemize}
\end{itemize}

KANDAHAR, Afghanistan --- Taliban leaders carried out an audacious plot
on Monday to free nearly 500 fighters from southern Afghanistan's
largest prison, leading them through a tunnel dug over more than five
months and equipped with electricity and air pipes, which suggested that
the insurgents remained formidable and wily opponents despite recent
setbacks.

The plan was so closely held that one young Taliban fighter who got out
said he knew nothing of it until a fellow inmate tugged his sleeve to
wake him in the night and led him to the three-foot-wide tunnel, which
ran more than half a mile from a hole in a cell's floor, under security
posts, tall concrete walls and a highway, and came up in a nearby house.
From there, a waiting car took the fighter a few miles away, where he
hailed a taxi to safety.

``I was just praying to God that he would free me,'' said the fighter,
Allah Mohammed Agha, 22, recounting his escape from Sarposa Prison,
where he had been held for 28 days. ``Last night was the night that my
dream was made true.'' He spoke by phone from Spinbaldak, near the
Pakistani border.

The Afghan government called the breach a disaster. The prison break
called into question the extent of the gains made against the Taliban in
18 months of hard fighting in Kandahar Province, and whether any
progress would be sustainable once NATO troops began to reduce their
numbers as planned this summer, members of Parliament, tribal leaders
and Western officials said in interviews.

Some worried that the jailbreak might strengthen the Taliban in the
coming weeks as the spring fighting season began. Having so many
fighters back in circulation --- possibly including hard-core commanders
--- also threatened to undermine efforts to bring Taliban fighters over
to the government side, Afghan officials and former Taliban said.

There is no doubt that the incident demonstrated the ability of the
Taliban to organize such an elaborate operation, even after they were
driven largely underground in Kandahar and Helmand Provinces, and
despite police and prison guards, prison visits by NATO mentors, and
sophisticated NATO surveillance in Kandahar.

The prison break comes after four recent attacks by the Taliban, in
which they used suicide bombers, often disguised as police officers or
soldiers, to penetrate secure buildings, including an
\href{http://www.nytimes.com/2011/04/17/world/asia/17afghanistan.html}{Afghan
army corps' headquarters} in Laghman Province and the
\href{http://www.nytimes.com/2011/04/19/world/asia/19afghanistan.html}{Ministry
of Defense headquarters} in the capital, Kabul.

\includegraphics{https://static01.nyt.com/images/2011/04/26/world/26afghanistan_cnd/26afghanistan_cnd-articleLarge.jpg?quality=75\&auto=webp\&disable=upscale}

Members of Parliament and others were scathing about the lapses. Some
questioned whether the prison guards or police officers were bribed not
to notice the tunnel's construction.

``It's a big achievement for the Taliban and shows a big failure and
weakness in the government,'' said Muhammad Naiem Lalay Hamidzai, a
Parliament member from Kandahar and chairman of the internal security
committee.

``The Taliban gain two things from this jailbreak,'' he said. ``First,
coming after the incidents in Kunduz, Laghman, Kandahar and at the
Ministry of Defense headquarters, it sends a message that they can do
whatever they want, even at the heart of the most secure and important
jail, and it allows them to strengthen their ranks with more manpower.''

The Afghan government was reeling Monday as details of the escape
emerged. ``This is bad news for the government and the people of
Afghanistan,'' the spokesman for President Hamid Karzai, Waheed Omar,
said at a news conference. ``This shows a vulnerability on the part of
the government.'' He called the prison break a disaster.

One unexplained question was why the cells where prisoners were supposed
to sleep were left open so that they could make their way to the cell
with the tunnel. It also seemed that none of the guards checked on the
prisoners during the night, even though Afghan intelligence officials
and Western military officials said that there had been intelligence
about the possibility of a security breach.

``This is absolutely the fault of the ignorance of the security
forces,'' said the Kandahar provincial governor, Tooryalai Wesa. ``This
was not the work of a day, a week or a month of activities. This was
actually months of work they spent to dig and free their men.''

Clearly embarrassed, Afghan officials had little else to say, other than
to acknowledge that the prison break showed unexpected weaknesses in
security. Since the Taliban engineered
\href{http://www.nytimes.com/2008/06/14/world/asia/14kandahar.html}{a
major break at the same prison} in 2008 --- freeing 1,200 prisoners ---
Canadian forces have mentored the Afghans who run the prison and NATO
countries have spent several million dollars upgrading and training the
prison administration, according to a Western official in Kabul.

``There are a lot of people asking questions today,'' said a NATO
officer at the coalition's headquarters in Kabul.

\includegraphics{https://static01.nyt.com/images/2011/04/25/multimedia/video-tc-042511-afghan/video-tc-042511-afghan-videoSmall.jpg}

There was no official comment from the NATO command. Two Western
officials described the break as ``at least partially an inside job,''
but both said they could not be named because of the delicacy of the
situation.

Of the 488 men who escaped, fewer than 20 were from the criminal section
of the prison; the rest were security detainees believed to be Taliban
fighters and commanders.

An escapee, who asked not to be identified, said that among those freed
were two shadow governors and 14 shadow district governors. The Taliban
have a shadow government that has varying influence in different
provinces.

However, Muhammad Qasim Hashimzai, a deputy justice minister, said that
the government did not yet know who had escaped. ``The detainees
included all kinds of people,'' he said, and he promised to have more
information on Tuesday.

Mr. Wesa, the Kandahar provincial governor, said a manhunt was on and
that 26 escapees had been captured by late afternoon.

The security section of the prison was eerily empty on Monday when
reporters were shown around. Prisoners' belongings were strewn about,
but appeared heaped in the cell with the tunnel in an effort to obscure
the entrance. A second tunnel branched off to the criminal side of the
prison, according to the warden, Gen. Ghulam Dastagir Mayar, and Mr.
Wesa.

Now, with so many Taliban back in the fight, it will be even harder to
convince Taliban fighters that they will be safe if they defect to the
government, a former Taliban commander said.

``The prison break will slow down the peace process,'' said Mullah
Noorul Aziz Agha, a Taliban member who recently decided to lay down his
arms and work with the government. ``I was talking to Taliban on the
phone to try to persuade them to come over, but now with this, how can
we promise them that we can offer them security and protection?''

Advertisement

\protect\hyperlink{after-bottom}{Continue reading the main story}

\hypertarget{site-index}{%
\subsection{Site Index}\label{site-index}}

\hypertarget{site-information-navigation}{%
\subsection{Site Information
Navigation}\label{site-information-navigation}}

\begin{itemize}
\tightlist
\item
  \href{https://help.nytimes.com/hc/en-us/articles/115014792127-Copyright-notice}{©~2020~The
  New York Times Company}
\end{itemize}

\begin{itemize}
\tightlist
\item
  \href{https://www.nytco.com/}{NYTCo}
\item
  \href{https://help.nytimes.com/hc/en-us/articles/115015385887-Contact-Us}{Contact
  Us}
\item
  \href{https://www.nytco.com/careers/}{Work with us}
\item
  \href{https://nytmediakit.com/}{Advertise}
\item
  \href{http://www.tbrandstudio.com/}{T Brand Studio}
\item
  \href{https://www.nytimes.com/privacy/cookie-policy\#how-do-i-manage-trackers}{Your
  Ad Choices}
\item
  \href{https://www.nytimes.com/privacy}{Privacy}
\item
  \href{https://help.nytimes.com/hc/en-us/articles/115014893428-Terms-of-service}{Terms
  of Service}
\item
  \href{https://help.nytimes.com/hc/en-us/articles/115014893968-Terms-of-sale}{Terms
  of Sale}
\item
  \href{https://spiderbites.nytimes.com}{Site Map}
\item
  \href{https://help.nytimes.com/hc/en-us}{Help}
\item
  \href{https://www.nytimes.com/subscription?campaignId=37WXW}{Subscriptions}
\end{itemize}
