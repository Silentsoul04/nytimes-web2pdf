Sections

SEARCH

\protect\hyperlink{site-content}{Skip to
content}\protect\hyperlink{site-index}{Skip to site index}

\href{https://www.nytimes.com/section/politics}{Politics}

\href{https://myaccount.nytimes.com/auth/login?response_type=cookie\&client_id=vi}{}

\href{https://www.nytimes.com/section/todayspaper}{Today's Paper}

\href{/section/politics}{Politics}\textbar{}A Defiant Herman Cain
Suspends His Bid for Presidency

\begin{itemize}
\item
\item
\item
\item
\item
\end{itemize}

Advertisement

\protect\hyperlink{after-top}{Continue reading the main story}

Supported by

\protect\hyperlink{after-sponsor}{Continue reading the main story}

\hypertarget{a-defiant-herman-cain-suspends-his-bid-for-presidency}{%
\section{A Defiant Herman Cain Suspends His Bid for
Presidency}\label{a-defiant-herman-cain-suspends-his-bid-for-presidency}}

\href{https://www.nytimes.com/slideshow/2011/12/03/us/politics/20111204cain-embed.html}{}

\hypertarget{outside-cain-headquarters}{%
\subsection{Outside Cain Headquarters}\label{outside-cain-headquarters}}

7 Photos

View Slide Show ›

\includegraphics{https://static01.nyt.com/images/2011/12/03/us/politics/20111204cain-embed-slide-ZJEV/20111204cain-embed-slide-ZJEV-articleLarge.jpg?quality=75\&auto=webp\&disable=upscale}

Scott Olson/Getty Images

By \href{https://www.nytimes.com/by/susan-saulny}{Susan Saulny}

\begin{itemize}
\item
  Dec. 3, 2011
\item
  \begin{itemize}
  \item
  \item
  \item
  \item
  \item
  \end{itemize}
\end{itemize}

An unapologetic and defiant Herman Cain suspended his presidential
campaign on Saturday, pledging that he ``would not go away'' even as he
abandoned the Republican presidential race in the face of escalating
accusations of sexual misconduct.

``As of today, with a lot of prayer and soul-searching, I am suspending
my presidential campaign,'' Mr. Cain said at a rally in Atlanta,
surrounded by supporters chanting his name. ``Because of the continued
distractions, the continued hurt caused on me and my family, not because
we are not fighters. Not because I'm not a fighter.''

In suspending his candidacy, as opposed to saying that he was ending his
bid, Mr. Cain, according to campaign finance lawyers, maintains an
ability to accept money to pay for his campaign so far and potentially
to finance the new venture that he called his Plan B: to travel the
country promoting his tax and foreign policy plans.

The collapse of Mr. Cain's campaign came as a
\href{http://caucuses.desmoinesregister.com/2011/12/03/iowa-poll-gingrich-leading-the-pack/}{new
Des Moines Register poll showed that his supporters appeared to be
gravitating toward Newt Gingrich}, the former House speaker. According
to the poll, Mr. Gingrich is backed by 25 percent of likely Republican
caucusgoers, followed by Representative Ron Paul of Texas with 18
percent and Mitt Romney with 16 percent. The poll was conducted before
Mr. Cain suspended his campaign, and it showed him with the support of
just 8 percent of respondents,
\href{http://www.nytimes.com/interactive/2011/11/30/us/politics/ups-and-downs-of-the-cain-candidacy.html}{a
sharp drop from previous polls}.

The other Republican candidates are also in single digits. In the
previous Iowa poll, conducted in late October, 7 percent supported Mr.
Gingrich, while Mr. Cain was the choice of 23 percent.

Mr. Romney said on Saturday that the race remained wide open. ``I don't
think people have really settled down in a final way to decide who
they're going to support in the nomination process,'' he said, adding,
``I hope they give us a good careful look.''

Mr. Cain said he would issue an endorsement soon. With his wife, Gloria,
at his side at the Atlanta rally, Mr. Cain said the accusations of
sexual harassment and of a 13-year affair were untrue. ``I'm at peace
with my God,'' he said. ``I'm at peace with my wife, and she is at peace
with me.''

Mr. Cain exited much the way he entered. The circuslike atmosphere ---
complete with numerous delays, barbecue, a blues band and supporters in
colonial-era dress --- was in keeping with the campaign's irreverence
and disarray since its inception.

For days now, the campaign had fueled a ``will he or won't he?'' storm
of speculation, at once thriving on the news media's attention while
denouncing it as the cause of Mr. Cain's plummeting popularity. Mr.
Cain's critics have long posited that he has been more interested in
creating celebrity for himself --- as a means to sell books and increase
speaking fees --- than in making a serious bid for the presidency.

Indeed, in his remarks on Saturday, Mr. Cain boasted about rising from
near obscurity, saying, ``Right now, my name ID is probably 99.9,'' a
reference to his ``9-9-9'' plan, which mixes a flat tax with a national
sales tax.

\includegraphics{https://static01.nyt.com/images/2011/12/03/multimedia/video-cain-suspends-campaign/video-cain-suspends-campaign-videoSmall.jpg}

Still, Mr. Cain took what may be his last moment in the national
spotlight to denounce the political culture in Washington, calling
politics ``a dirty game.''

Mr. Cain's admirers in Atlanta were surprised and disappointed. They
blamed the news media, some screaming insults at the press corps.

Lisa Chambers, 48, a volunteer from Snellville, Ga., said: ``This is not
what I wanted. Not at all. I'm not sure what to do now. I'm so
disappointed.''

But other supporters were more pragmatic. Dean Kleckner, a former
president of the Iowa Farm Bureau who gave Mr. Cain an early
endorsement, said: ``I hate to say this, because he was a remarkable man
in many ways, but I honestly think he did the right thing. I'm
disappointed in a way, relieved in a way.''

The other Republican candidates quickly praised Mr. Cain and his agenda,
in an effort to attract his supporters.

``It's very import to remember,'' Mr. Gingrich said on Saturday at a
campaign event, ``he was the person who had the courage to launch the
9-9-9 plan. Whether you liked it or disliked it, it raised the general
level of discussion.''

Mr. Cain's political downfall was as swift as his ascent. It began just
one month after an unlikely surge in the polls, fueled by the strength
of his debate performances, the novelty of his tax plan and his
\href{http://thecaucus.blogs.nytimes.com/2011/09/24/herman-cain-wins-florida-straw-poll/}{surprise
victory} in the Florida straw poll in September.

With his golden voice and folksy manner, Mr. Cain appealed to voters who
sought an anti-establishment candidate. Mr. Cain, 65, grew up in poverty
in the segregated South, the son of a janitor and a maid. But beyond his
personal charm and rags-to-riches biography, he had an eclectic résumé:
chief executive of Godfather's Pizza, conservative radio host and
chairman of the Federal Reserve Bank of Kansas City in Missouri.

Toward the end of October, more than one survey found Mr. Cain, who has
never held elected office, essentially tied with Mr. Romney, the former
Massachusetts governor who has consistently been near the top in most
polls.

But accusations of sexual misconduct rocked the campaign of a candidate
who professed to be a devout Christian and family man. And some of the
details were graphic.

A Chicago woman, Sharon Bialek, was
the\href{http://www.nytimes.com/2011/11/08/us/politics/woman-accuses-cain-of-groping-he-denies-charge.html}{first
to come forward publicly}. Ms. Bialek said that Mr. Cain made an
unwanted and rough physical advance on her 14 years ago when he was the
chief of the National Restaurant Association, a lobbying group. After
taking her out for a night on the town in Washington, she said, he
suggested she engage with him sexually in return for his assistance in
finding a job.

\includegraphics{https://static01.nyt.com/images/2011/12/03/multimedia/video-newt-cain/video-newt-cain-videoSmall.jpg}

Within days, a
\href{http://www.nytimes.com/2011/11/09/us/politics/cain-to-respond-to-allegation-after-vowing-to-move-on.html}{second
woman came forward}. That woman, Karen Kraushaar, 55, worked in the
government affairs office of the restaurant association for a relatively
short time from 1998 to 1999, her tenure being cut short, she said, by
her run-ins with Mr. Cain and the discomfort it created for her.

Two other women who complained of harassment by Mr. Cain remained
anonymous. But one of those women and Ms. Kraushaar both received the
equivalent of a year's salary in settlements from the restaurant group.

From the moment the harassment accusations were revealed, Mr. Cain
proclaimed his innocence and sought to cast blame for what he called a
smear campaign in a number of different directions. He first accused the
news media, then the rival campaign of Gov. Rick Perry of Texas.
Ultimately, the Cain campaign acknowledged that it had no evidence of a
conspiracy. But still, Mr. Cain, inexperienced on the national stage,
issued an avalanche of confusing and often contradictory statements.

Polls conducted at the time, however, suggested that the crisis was not
eroding Mr. Cain's standing as a top-tier candidate. He continued to
campaign as if he was not at the center of a swirling controversy,
ignoring the accusations in speeches and not taking questions on the
subject from reporters.

``We're getting back on message, end of story,'' Mr. Cain said after a
debate in early November.

The accusations of sexual misconduct were not Mr. Cain's only stumbling
block. The very qualities that endeared Mr. Cain to so many
conservatives appeared to undercut his chances, as questions were raised
about his management style and foreign policy expertise.

In a videotaped interview with the editorial board of The Milwaukee
Journal Sentinel that went viral on the Web, Mr. Cain
\href{http://www.nytimes.com/2011/11/15/us/politics/herman-cain-libya-comments-draw-criticism.html}{became
flustered} when asked to assess President Obama's policy toward Libya,
lurching over five minutes from awkward pauses to halting efforts to
answer.

Compared with his rivals, Mr. Cain hardly campaigned in New Hampshire
and Iowa.

Former staff members complained that he spent the bulk of his
time\href{http://www.nytimes.com/2011/10/06/us/politics/gop-hopeful-herman-cain-on-book-tour-not-campaign-trail.html}{on
a book tour} through the South when he should have been organizing a
grass-roots operation. He occasionally mishandled potential big donors
or ignored real voters, said former staff members and supporters.

On the Monday after Thanksgiving, a fifth woman, Ginger White,
\href{http://thecaucus.blogs.nytimes.com/2011/11/28/cain-says-new-accuser-comes-forward/}{came
forward}, telling a local television reporter in Atlanta that she and
Mr. Cain had only recently ended a 13-year extramarital affair.

Ms. White produced phone records to prove that they had called or texted
each other frequently, and Mr. Cain acknowledged giving her financial
support --- and also that his wife of 43 years had been unaware of what
he insisted was only a friendship.

The day after Ms. White's revelation, Mr. Cain said he was considering
dropping his bid as some of his supporters and defenders began backing
away.

On Friday night, Mr. Cain returned home to suburban Atlanta to meet and
consult with his wife for the first time since Ms. White came forward
with her claim. Mr. Cain said the ultimate decision would rest with his
wife.

On Saturday, Mr. Cain directed supporters to a Web site,
\href{http://thecainsolutions.com/}{TheCainSolutions.com}.

The site was registered on Friday by Bell Research Companies of Tifton,
Ga., which manufactures low-fat powdered peanut butter and alternative
fuels. The company also owns the group Americans for Jobs and Energy
Security, which promotes natural gas. In documents filed last year with
the Securities and Exchange Commission, Mr. Cain is listed on Bell's
board of directors.

Advertisement

\protect\hyperlink{after-bottom}{Continue reading the main story}

\hypertarget{site-index}{%
\subsection{Site Index}\label{site-index}}

\hypertarget{site-information-navigation}{%
\subsection{Site Information
Navigation}\label{site-information-navigation}}

\begin{itemize}
\tightlist
\item
  \href{https://help.nytimes.com/hc/en-us/articles/115014792127-Copyright-notice}{©~2020~The
  New York Times Company}
\end{itemize}

\begin{itemize}
\tightlist
\item
  \href{https://www.nytco.com/}{NYTCo}
\item
  \href{https://help.nytimes.com/hc/en-us/articles/115015385887-Contact-Us}{Contact
  Us}
\item
  \href{https://www.nytco.com/careers/}{Work with us}
\item
  \href{https://nytmediakit.com/}{Advertise}
\item
  \href{http://www.tbrandstudio.com/}{T Brand Studio}
\item
  \href{https://www.nytimes.com/privacy/cookie-policy\#how-do-i-manage-trackers}{Your
  Ad Choices}
\item
  \href{https://www.nytimes.com/privacy}{Privacy}
\item
  \href{https://help.nytimes.com/hc/en-us/articles/115014893428-Terms-of-service}{Terms
  of Service}
\item
  \href{https://help.nytimes.com/hc/en-us/articles/115014893968-Terms-of-sale}{Terms
  of Sale}
\item
  \href{https://spiderbites.nytimes.com}{Site Map}
\item
  \href{https://help.nytimes.com/hc/en-us}{Help}
\item
  \href{https://www.nytimes.com/subscription?campaignId=37WXW}{Subscriptions}
\end{itemize}
