Sections

SEARCH

\protect\hyperlink{site-content}{Skip to
content}\protect\hyperlink{site-index}{Skip to site index}

\href{https://www.nytimes.com/section/politics}{Politics}

\href{https://myaccount.nytimes.com/auth/login?response_type=cookie\&client_id=vi}{}

\href{https://www.nytimes.com/section/todayspaper}{Today's Paper}

\href{/section/politics}{Politics}\textbar{}A Scramble as Biden Backs
Same-Sex Marriage

\begin{itemize}
\item
\item
\item
\item
\item
\end{itemize}

Advertisement

\protect\hyperlink{after-top}{Continue reading the main story}

Supported by

\protect\hyperlink{after-sponsor}{Continue reading the main story}

\hypertarget{a-scramble-as-biden-backs-same-sex-marriage}{%
\section{A Scramble as Biden Backs Same-Sex
Marriage}\label{a-scramble-as-biden-backs-same-sex-marriage}}

By \href{https://www.nytimes.com/by/michael-barbaro}{Michael Barbaro}

\begin{itemize}
\item
  May 6, 2012
\item
  \begin{itemize}
  \item
  \item
  \item
  \item
  \item
  \end{itemize}
\end{itemize}

Vice President Joseph R. Biden Jr. said on Sunday that he was
``absolutely comfortable'' with same-sex marriages and was heartened by
their growing acceptance across the country, a position that moves well
beyond the ``evolving'' views that President Obama has said he holds on
the issue.

The comments, which aides described as the off-the-cuff views of a vice
president not known for fidelity to a script, sent the White House
scrambling to clarify that Mr. Biden was not articulating an official
change in policy, a reaction that highlighted the administration's
unease over the subject.

In an interview on NBC's ``Meet the Press,'' Mr. Biden invoked some of
the same language that advocates of same-sex marriage use, speaking of
family, equality and love.

``I am absolutely comfortable with the fact that men marrying men, women
marrying women and heterosexual men and women marrying one another are
entitled to the same exact rights, all the civil rights, all the civil
liberties,'' Mr. Biden said, while noting that the president, not he,
sets policy on such matters.

Mr. Biden's unexpectedly expansive remarks made him by far the
highest-ranking White House official to move closer to a formal embrace
of same-sex marriage, which is now legal in six states and the District
of Columbia but is unrecognized by the federal government. The Obama
administration has endorsed civil unions but not marriage for gay
couples.

The vice president's comments are likely to intensify pressure on Mr.
Obama, who says he is still wrestling with his feelings about same-sex
marriage, to take a clearer stance on it before the presidential
election this fall, something the White House has shown reluctance to
do.

Mr. Biden's aides, in insisting that he was not deviating from White
House policy, pointed to a 2010 statement by the vice president that the
country was moving toward a ``national consensus'' on same-sex marriage.
And in Sunday's interview, Mr. Biden did not say explicitly that the
federal government should recognize it.

But gay rights advocates, who spent Sunday morning parsing Mr. Biden's
words, said the president's running mate had, in their analysis,
conveyed new and unmistakable support for their biggest cause.

Mr. Biden called the debate surrounding the issue a simple question of
``who do you love?'' and ``and will you be loyal to the person you
love?''

``That's what people are finding out is what, what all marriages, at
their root, are about,'' he said, ``Whether they're marriages of
lesbians or gay men or heterosexuals.''

Mr. Obama faces growing calls from gay and lesbian voters and a
formidable array of wealthy gay donors to support same-sex marriage and
make it a part of the Democratic Party's platform at its convention.
Many of his supporters believe that he privately backs it but is
unwilling to say so before a general election that may be decided in
states like Ohio, Pennsylvania and Virginia, where such a position could
provoke a backlash.

In 1996, as a candidate for the Illinois Senate, Mr. Obama answered on a
candidate's questionnaire, ``I favor legalizing same-sex marriages.''
But after he became president, White House officials said Mr. Obama had
been referring to civil unions.

Since then, Mr. Obama has repeatedly said he believes that gay couples
should have the same rights as heterosexual couples, but he has not used
the symbolically freighted term marriage, as Mr. Biden did. In 2010,
when asked about same-sex marriage, Mr. Obama said, ``My baseline is a
strong civil union that provides them the protections and the legal
rights that married couples have.''

\includegraphics{https://static01.nyt.com/images/2012/05/07/us/JP-MARRIAGE/JP-MARRIAGE-jumbo.jpg?quality=75\&auto=webp\&disable=upscale}

Gay rights leaders expressed frustration and dismay on Sunday over
attempts by the White House to play down the vice president's words and
said that the president's own endorsement of same-sex marriage was long
overdue.

``Trying now to walk this back will only hurt them,'' said
\href{http://richardsocarides.com/}{Richard Socarides}, a former White
House aide who advised President Bill Clinton on gay rights. ``You can't
clarify an answer as direct and candid and expansive as the one he
gave.''

But some Democrats, particularly those representing black and Hispanic
constituencies, have opposed same-sex marriage, arguing that it
conflicts with Christian doctrine. The issue has become a hot topic in
two potential swing states, with North Carolina voting this week on a
measure to ban same-sex marriages and Maine voting on a measure in
November to legalize them.

The difference of opinion between Mr. Biden and Mr. Obama, however wide
it is, is somewhat reminiscent of that between President George W. Bush
and his vice president, Dick Cheney, on the same subject. Mr. Bush
opposed same sex-marriage and in 2004 even endorsed a constitutional
amendment banning it, while Mr. Cheney disagreed publicly and said
``freedom means freedom for everyone.''

Mr. Biden, a 69-year-old Roman Catholic, seemed to acknowledge in his
interview that he was diverging from the president's stated position. He
prefaced his remarks by saying: ``I am vice president of the United
States of America. The president sets the policy.''

His remarks about same-sex marriage and
\href{https://www.nytimes.com/2020/06/21/us/politics/biden-gay-rights-lgbt.html}{gay
rights} were wide-ranging, even touching on the influence of the
television show ``Will and Grace.'' (He said the show ``probably did
more to educate the American public than almost anything anybody's ever
done so far.'')

At times Mr. Biden's remarks were strikingly personal, as well. He
recalled meeting the two children of a gay couple at a Los Angeles
fund-raiser two weeks ago, an experience that aides said had left a deep
impression on him. During a question-and-answer session at the
fund-raiser, Mr. Biden recalled, a gay man asked, ``How do you feel
about us?''

Mr. Biden recounted his reply: ``I turned to the man who owned the
house. I said, `What did I do when I walked in?' He said: `You walked
right to my children. They were 7 and 5, giving you flowers.' ''

``And I said, `I wish every American could see the look of love those
kids had in their eyes for you guys. And they wouldn't have any doubt
about what this is about.' ''

This is not the first time that White House officials have staked out
stronger positions on gay rights than the president has, raising the
possibility that Mr. Obama is relying on aides to telegraph his
intentions to avoid the political consequences of articulating them
himself.

Last year, for example, Secretary of State Hillary Rodham Clinton said
in Geneva that ``gay rights are human rights, and human rights are gay
rights,'' a message Mr. Obama later enthusiastically endorsed.

And when Shaun Donovan, the secretary of housing and urban development,
said in an interview that he supported the right of gay couples to
marry, a senior administration official said Mr. Donovan enjoyed ``the
trust and respect of the president.''

The White House denied that Mr. Biden was acting as a surrogate for the
president on Sunday, saying that Mr. Biden's views, influenced by gay
friends and fund-raisers, had changed and that, with characteristic
candor, he was willing to volunteer them at length.

When pressed on whether Mr. Obama would support same-sex marriage in a
second term, though, Mr. Biden kept his response short: ``I don't know
the answer to that.''

Advertisement

\protect\hyperlink{after-bottom}{Continue reading the main story}

\hypertarget{site-index}{%
\subsection{Site Index}\label{site-index}}

\hypertarget{site-information-navigation}{%
\subsection{Site Information
Navigation}\label{site-information-navigation}}

\begin{itemize}
\tightlist
\item
  \href{https://help.nytimes.com/hc/en-us/articles/115014792127-Copyright-notice}{©~2020~The
  New York Times Company}
\end{itemize}

\begin{itemize}
\tightlist
\item
  \href{https://www.nytco.com/}{NYTCo}
\item
  \href{https://help.nytimes.com/hc/en-us/articles/115015385887-Contact-Us}{Contact
  Us}
\item
  \href{https://www.nytco.com/careers/}{Work with us}
\item
  \href{https://nytmediakit.com/}{Advertise}
\item
  \href{http://www.tbrandstudio.com/}{T Brand Studio}
\item
  \href{https://www.nytimes.com/privacy/cookie-policy\#how-do-i-manage-trackers}{Your
  Ad Choices}
\item
  \href{https://www.nytimes.com/privacy}{Privacy}
\item
  \href{https://help.nytimes.com/hc/en-us/articles/115014893428-Terms-of-service}{Terms
  of Service}
\item
  \href{https://help.nytimes.com/hc/en-us/articles/115014893968-Terms-of-sale}{Terms
  of Sale}
\item
  \href{https://spiderbites.nytimes.com}{Site Map}
\item
  \href{https://help.nytimes.com/hc/en-us}{Help}
\item
  \href{https://www.nytimes.com/subscription?campaignId=37WXW}{Subscriptions}
\end{itemize}
