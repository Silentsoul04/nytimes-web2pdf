Sections

SEARCH

\protect\hyperlink{site-content}{Skip to
content}\protect\hyperlink{site-index}{Skip to site index}

\href{https://www.nytimes.com/section/food}{Food}

\href{https://myaccount.nytimes.com/auth/login?response_type=cookie\&client_id=vi}{}

\href{https://www.nytimes.com/section/todayspaper}{Today's Paper}

\href{/section/food}{Food}\textbar{}The Once and Future Spago

\url{https://nyti.ms/Q4u7V5}

\begin{itemize}
\item
\item
\item
\item
\item
\end{itemize}

Advertisement

\protect\hyperlink{after-top}{Continue reading the main story}

Supported by

\protect\hyperlink{after-sponsor}{Continue reading the main story}

\hypertarget{the-once-and-future-spago}{%
\section{The Once and Future Spago}\label{the-once-and-future-spago}}

\includegraphics{https://static01.nyt.com/images/2012/10/31/dining/31PUCK_SPAN/31PUCK-articleLarge.jpg?quality=75\&auto=webp\&disable=upscale}

By \href{https://www.nytimes.com/by/adam-nagourney}{Adam Nagourney}

\begin{itemize}
\item
  Oct. 30, 2012
\item
  \begin{itemize}
  \item
  \item
  \item
  \item
  \item
  \end{itemize}
\end{itemize}

BEVERLY HILLS, Calif.

THIRTY years ago,
\href{http://topics.nytimes.com/top/reference/timestopics/people/p/wolfgang_puck/index.html?8qa}{Wolfgang
Puck} opened Spago in West Hollywood, a restaurant that is still
celebrated across Los Angeles, pioneering a lively new style of
California cooking and offering a festive rebuke to all the white
tablecloths in town.

But these days, Mr. Puck's name resonates with most Americans not for
what he did in his kitchen but for the one-man culinary behemoth he
built.

\href{http://www.wolfgangpuck.com/}{His smiling face} peers from
organic-soup cans and boxes of frozen pizza in supermarkets. Wolfgang
Puck concessions sell grab-and-go meals in airports. There are Wolfgang
Puck cookware, knives, coffee makers and rice steamers. At 63, Mr. Puck,
with his tart Austrian accent, has been a mainstay on cooking shows for
25 years, a celebrity chef long before the phrase was invented. Today
there are 101 Puck restaurants, from Singapore to London.

With his relentless merchandising, self-branding and kitchen
globe-trotting, with his ease in moving from the highbrow to the
lower-brow, Mr. Puck has often seemed in danger of sliding into culinary
parody, more Barry Becher pitching Ginsu knives on late-night television
(``Wait, there's more!'') than the avatar of California cuisine.

``When he started showing up at airports, people were shocked,'' said
Evan Kleiman, the restaurateur who just closed the acclaimed Angeli
Caffe in Los Angeles and
\href{http://www.kcrw.com/people/kleiman_evan?role=host}{the host} of
``Good Food'' on the public radio station KCRW.

Yet three weeks ago, Mr. Puck unveiled an entirely reinvented
\href{http://www.wolfgangpuck.com/restaurants/fine-dining/3635}{Spago
Beverly Hills}, with a décor he describes as ``edgy,'' a soundtrack of
Arcade Fire and Spoon, and a menu sprinkled with adventurous small-plate
offerings that have some of his bluer-hair customers wincing. This year
he received the
\href{http://www.jamesbeard.org/awards/lifetime-achievement}{James Beard
Foundation lifetime achievement award}. And even as he presents an
overhauled Spago, he is marking the anniversary of the luxurious
restaurant he opened as part of the refurbishment of that Hollywood
classic, the \href{http://www.hotelbelair.com/}{Hotel Bel-Air}, named
(but of course) Wolfgang Puck at Hotel Bel-Air.

By almost every measure, he has not only survived but thrived, earning
lasting respect from younger chefs, many of whom trained in his
kitchens. Other peers wandered from their stoves as they became boldface
fixtures on television or Page Six. Mr. Puck, ever restless, has
continued to refashion his restaurants, his menus, his dishes and
ultimately himself.

``Why stop?'' he said, standing in the Spago dining room the other
morning, the smell of paint fresh in the air, as workers rolled in
tables and hammered down final touches. ``What would you do at home?''

Even winning the James Beard award seemed more an annoyance than an
honor. ``He was like, `What do they think, that I retired?'~'' said Ruth
Reichl, a former restaurant critic for The Los Angeles Times and The New
York Times, who has been close to Mr. Puck for much of his career.

Mr. Puck's ambition does not appear to be dulled by success or age. The
other night, he moved from kitchen to table with the genial spark that
recalled his presence on the same floor a decade ago. He showed off the
artwork on the walls, curated by his wife, Gelila Assefa; the receding
canopy over the outdoor dining room framed by two outdoor fireplaces;
and the soft lines and muted colors of the \$4 million renovation by the
designer Waldo Fernandez.

``Before, it was more ungapatchka, as we say in Austria,'' Mr. Puck
said. ``With little niches and this and that all over the place. It
somehow had this dark feeling.''

Nancy Silverton, an owner of
\href{http://www.osteriamozza.com/LA/home.cfm}{Osteria Mozza} and
Pizzeria Mozza (and a graduate of the Spago kitchen), said she picked up
the telephone the other day to hear Mr. Puck's unmistakable voice. ``He
must be under an incredible amount of pressure, with the transition, the
expectations,'' she said. ``But he called me up and said, `Mama' --- he
always calls me Mama --- `Mama, how come you haven't made a reservation
to come to my restaurant?'~''

The very fact that Spago Beverly Hills continues to exist, much less
prosper, is striking at a time when the Los Angeles restaurant scene is
so dynamic and punishing. Over the last year, some of this city's most
popular spots have announced they were closing: Angeli Caffe, Campanile,
Sushi Nozawa and Lou among them. One place on Los Angeles Magazine's
list of this year's 10 best new restaurants that was all but impossible
to get into eight months ago had empty tables on a recent Friday night.
The extravagance Mr. Puck championed at Spago has taken a back seat to
restaurants that are quieter, smaller, more adventurous and less pricey.

\includegraphics{https://static01.nyt.com/images/2012/10/29/dining/video-spago/video-spago-videoSmall.jpg}

``It's a very different world now,'' Ms. Kleiman said. ``It's not like
where it was 10 years ago, when a lot of people could go out and eat at
fine dining places on expense accounts. I think people in their 30s or
40s don't think about going to Bouchon and Spago.''

Mr. Puck has tried to accommodate them. At this latest of Spagos, he
jettisoned two staples, the smoked salmon pizza and the Wiener schnitzel
(though he said he would be glad to make either for old-time customers
who ask) as he dappled his menu with dishes like a veal filet mignon
tartare with smoked mascarpone, and a soba pasta studded with pieces of
Dungeness crab. His challenge, Mr. Puck said, is rolling out innovative
dishes that would bring in new diners without frightening the horses ---
the patrons who have been eating at Puck restaurants from the beginning.

That said, he does have the buffer of a reputation built over years
spent cooking here.

``I have to tell you, the best meal I had last year was at Spago,'' Ms.
Reichl said. ``When he wants to do it, he is an amazing chef. He's still
enormously important.''

And that reputation has carried through to a new generation of Los
Angeles chefs who have picked up the ladle Mr. Puck held 30 years ago.
``I'm pretty amazed by him,'' said
\href{http://www.nytimes.com/2010/08/04/dining/04notebook.html?pagewanted=all}{Ludo
Lefebvre}, the French cook who has orchestrated a series of popular
pop-up restaurants. ``There is nothing left to prove, but he just can't
take a break. It's a good example for every chef: never think that
you're good enough.''

``He's like the Steven Spielberg or James Cameron of restaurants,'' Mr.
Lefebvre said. ``It's true. It's true. He is very smart. We are all
watching him.''

At his age, with two young children with his second wife, Mr. Puck could
surely turn down the gas on one of his burners, perhaps close a
restaurant or take one trip fewer to London. When he was a younger man,
he said, he worried that he would one day grow bored with the monotony
of doing the same thing six nights a week.

``I thought, well, I will have to find something to do,'' he said. ``I
tried to play golf, and I was like, uh, O.K., that's really boring.
Playing tennis is fun, but I can't play six hours a day. I would get
bored. With food, I could never get really bored.''

Even 30 years later, people still talk about the original Spago: that
smoked salmon pizza, the Oscars parties hosted by Swifty Lazar, the
tables bustling with celebrities, and Mr. Puck at the center of it all,
gliding from his kitchen and across the shimmering dining room
overlooking Sunset Boulevard. It opened in 1982 and closed 19 years
later, as Spago Beverly Hills took over its glamorous perch.

Perhaps inevitably, Mr. Puck's restaurants don't have quite the same
celebrity cachet they once had. The post-Oscars party has moved to the
Sunset Tower Bar, and you are just as likely to spot a movie star at a
more intimate place like Ammo (Jake Gyllenhaal) or at the always-buzzing
Mozza Pizzeria (Natalie Portman). ``At the old restaurant, I still have
people tell me all the time: `Oh, my God, I can remember sitting next to
Gene Kelly. I can't believe he was there. He was my idol,'~'' Mr. Puck
said. ``But that was 30 years ago.''

Still, it was Mr. Puck who catered the Governors Ball dinner at the
Oscars this year for the 18th time, not to mention the West Coast
campaign fund-raiser of the year: the one George Clooney hosted at his
home for President Obama. The Puck name was on the napkins at the
premiere party for the movie ``Brave'' in June. ``And we had Woody Allen
here last night,'' Mr. Puck said in an interview before Spago closed for
the renovation.

If Mr. Puck was once the David Chang of his day, today he is not quite a
cutting-edge figure in the Los Angeles food world. He may be the only
local chef who doesn't sport any visible tattoos. He doesn't seem
particularly curious about the restaurants that have transformed the
dining scene. He hasn't joined the caravan of diners who move from
places like Animal to A-Frame to Superba Snack Bar.

``People say to me: `Have you ever gone to this restaurant or that
restaurant?'~'' he said. ``I say `no.' Most of the young ones, they've
worked with us at one point or another.''

Ms. Reichl said that over the years, she wondered why Mr. Puck would
risk his success or name with all his new enterprises, instead of
resting on his considerable laurels.

And why did he? Mr. Puck said he always saw himself first as a cook. He
seemed mystified that anyone would think less of him for going into the
canned-soup business.

``If anyone had said anything, I would have abandoned it,'' he said.
``My passion was always fine-dining restaurants.''

Advertisement

\protect\hyperlink{after-bottom}{Continue reading the main story}

\hypertarget{site-index}{%
\subsection{Site Index}\label{site-index}}

\hypertarget{site-information-navigation}{%
\subsection{Site Information
Navigation}\label{site-information-navigation}}

\begin{itemize}
\tightlist
\item
  \href{https://help.nytimes.com/hc/en-us/articles/115014792127-Copyright-notice}{©~2020~The
  New York Times Company}
\end{itemize}

\begin{itemize}
\tightlist
\item
  \href{https://www.nytco.com/}{NYTCo}
\item
  \href{https://help.nytimes.com/hc/en-us/articles/115015385887-Contact-Us}{Contact
  Us}
\item
  \href{https://www.nytco.com/careers/}{Work with us}
\item
  \href{https://nytmediakit.com/}{Advertise}
\item
  \href{http://www.tbrandstudio.com/}{T Brand Studio}
\item
  \href{https://www.nytimes.com/privacy/cookie-policy\#how-do-i-manage-trackers}{Your
  Ad Choices}
\item
  \href{https://www.nytimes.com/privacy}{Privacy}
\item
  \href{https://help.nytimes.com/hc/en-us/articles/115014893428-Terms-of-service}{Terms
  of Service}
\item
  \href{https://help.nytimes.com/hc/en-us/articles/115014893968-Terms-of-sale}{Terms
  of Sale}
\item
  \href{https://spiderbites.nytimes.com}{Site Map}
\item
  \href{https://help.nytimes.com/hc/en-us}{Help}
\item
  \href{https://www.nytimes.com/subscription?campaignId=37WXW}{Subscriptions}
\end{itemize}
