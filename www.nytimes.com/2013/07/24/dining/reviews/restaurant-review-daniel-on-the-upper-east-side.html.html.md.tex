Sections

SEARCH

\protect\hyperlink{site-content}{Skip to
content}\protect\hyperlink{site-index}{Skip to site index}

\href{https://www.nytimes.com/section/food}{Food}

\href{https://myaccount.nytimes.com/auth/login?response_type=cookie\&client_id=vi}{}

\href{https://www.nytimes.com/section/todayspaper}{Today's Paper}

\href{/section/food}{Food}\textbar{}Serving the Stuff of Privilege

\url{https://nyti.ms/12gpBsD}

\begin{itemize}
\item
\item
\item
\item
\item
\item
\end{itemize}

Advertisement

\protect\hyperlink{after-top}{Continue reading the main story}

Supported by

\protect\hyperlink{after-sponsor}{Continue reading the main story}

\hypertarget{serving-the-stuff-of-privilege}{%
\section{Serving the Stuff of
Privilege}\label{serving-the-stuff-of-privilege}}

\includegraphics{https://static01.nyt.com/images/2013/07/24/dining/24REST_SPAN/24REST_SPAN-articleLarge.jpg?quality=75\&auto=webp\&disable=upscale}

\begin{itemize}
\tightlist
\item
  Daniel\\
  **NYT Critic's Pick ★★★ French \$\$\$\$ 60 East 65th Street
  212-288-0033
\end{itemize}

\href{http://www.opentable.com/single.aspx?ref=4201\&rid=337}{Reserve a
Table}

When you make a reservation at an independently reviewed restaurant
through our site, we earn an affiliate commission.

By \href{https://www.nytimes.com/by/pete-wells}{Pete Wells}

\begin{itemize}
\item
  July 23, 2013
\item
  \begin{itemize}
  \item
  \item
  \item
  \item
  \item
  \item
  \end{itemize}
\end{itemize}

Your job may be worrying you, or your father's health, or your own. You
may have been up at 2 that morning drafting a better ending for a
long-ago memory. But certain restaurants, if you can afford them, can
knock down the barriers between you and happiness for a few hours. Every
taste seems to transport you to another world, while every gesture of
the staff convinces you that you live in the privileged center of this
one.

\href{http://danielnyc.com/}{Daniel}, which turned 20 this year, can
make you feel that way. Does chilled pea soup sound like the stuff of
privilege? It is when it comes from this kitchen, where
\href{http://danielnyc.com/jean-francois-bruel-executive-chef}{Jean
François Bruel} has been the executive chef since 2003, and which Daniel
Boulud, the proprietor, watches over from a windowed perch above the
saucepans and sieves.

Those sieves got a workout on this soup, straining it to a gliding
smoothness. It had the clear, refreshing sweetness of the smallest peas
eaten straight from the pod. Salty diamonds of smoked sable and a white
ring of rosemary-infused cream helped the soup's purity shine more
clearly. This kind of exquisitely sensitive, profoundly seasonal,
fundamentally French cooking helped lift Daniel to several four-star
reviews in The New York Times,
\href{http://www.nytimes.com/2009/01/21/dining/reviews/21rest.html?pagewanted=all}{the
most recent one} by Frank Bruni in 2009.

Again and again, I have been startled by the excellence of Mr. Bruel's
ingredients and his talent for unlocking all they had to offer. I have
never tasted more calmly flavorful veal tenderloin, or fresher and more
gently handled swordfish, or a more skillfully roasted breast of guinea
hen.

But some of these star ingredients were embedded in elaborate, multipart
compositions that didn't fully reward the attention they demanded. At
times, the restaurant gave the impression that it was trying to garnish
its way to greatness.

And while the service can be among the best in the city, with a supreme
attentiveness softened by a surprising warmth and even chattiness, it is
not always that way for everyone. When people who are known at the
restaurant tell me about their meals, they look blissful. Others look
disappointed or resentful as they tell me about cramped tables in the
neoclassical arcades around the grand sunken dining room and hasty,
perfunctory service.

\href{https://www.nytimes.com/slideshow/2013/07/24/dining/20130724-REST.html}{}

\hypertarget{daniel}{%
\subsection{Daniel}\label{daniel}}

15 Photos

View Slide Show ›

\includegraphics{https://static01.nyt.com/images/2013/07/24/dining/20130724-REST-slide-37Q2/20130724-REST-slide-37Q2-articleLarge.jpg?quality=75\&auto=webp\&disable=upscale}

Evan Sung for The New York Times

One night I had a reservation 15 minutes apart from a colleague who
wasn't likely to be recognized, as I repeatedly was. We both ordered the
six-course \$195 tasting menu. (A three-course prix fixe dinner is
\$116.) Our meals were virtually identical. Our experiences were not.

The kitchen sent two amuse courses to my table. His got one. A few
remaining sips of my wine, ordered by the glass, were topped off. His
glass sat empty at times while he waited to be offered another.

We both ate extraordinary fried lollipops of filleted frogs' legs on a
long stick of bone, but only I was then brought a napkin-covered bowl of
rosemary- and lemon-scented water for rinsing my fingers.

My servers were solicitous: Was this course, or that one, or that one,
prepared to your liking? Was the pacing of the meal satisfactory? Could
we interest you in a cheese course? Would you like your espresso with
dessert, or after? Finally, as I neared the revolving door on East 65th
Street: Can we help you find a cab tonight?

My colleague wasn't asked any of those questions. Still, the next
morning, he reported feeling very well taken care of. And a restaurant
can't be blamed for trying to impress a critic.

It can be faulted, though, for turning its best face away from the
unknowns, the first-timers, the birthday splurgers, the tourists. They
are precisely the people who would remember a little coddling at a place
like Daniel for years.

And while a missing finger bowl won't seriously mar anyone's evening,
missing Daniel's cheese cart might. It is one of the finest four-wheeled
vehicles in New York. Whenever I wondered if I really wanted cheese, a
server would lay his knife on a soft wheel, pressing gently. The mounded
top would fall for a moment then rise up again, gracefully and almost
willingly. After that, the question was not if I should have some, but
how many kinds could fit on one plate.

It was just as pointless to try to wave away the basket of Mark
Fiorentino's gorgeous breads, like a garlic focaccia, round and dimpled
in the center. Rajeev Vaidya, the head sommelier, shepherded me past the
many bottles that could land a weak wine lover in debt to more
affordable ones. He has a 2007 halbtrocken from the German riesling
maker Georg Breuer. Some buyers scoffed at the vintage, pushing prices
down, but not Mr. Vaidya. A bottle can be yours for the princely sum of
\$25.

\includegraphics{https://static01.nyt.com/images/2013/07/24/dining/24SUBREST2/24SUBREST2-articleLarge.jpg?quality=75\&auto=webp\&disable=upscale}

Recently, the title of executive pastry chef passed from Sandro Micheli
to Ghaya Oliveira, and the dessert course, already exciting, has a
little more energy. Ms. Oliveira's approach is more modern and
wide-ranging, embracing unusual spices and exuberant swipes of color.
Her mint-scented strawberries are a giddy, flagrant essay in pink, with
triangles of watermelon, columns of half-frozen strawberry mousse and
ladyfingers tinted with powdered strawberry skin. It was a soft, lilting
summer tune I won't get out of my head before Labor Day.

The courses before dessert could be just as wonderfully haunting. I'd
give a lot to recapture the happiness I got from slow-baked abalone,
rich with creamed avocado and slightly tart with heart-shaped wood
sorrel. I'm still transfixed by a peekytoe crab salad's bravura
variations on apple and celery, carried through to the juices in a
walnut-oil vinaigrette.

And nothing quite prepared me for the untamed whoosh of intense green
herbs in a bowl of olive-oil-poached cod teased into big, glistening
flakes, then seasoned with za'atar and a bright cilantro sauce.

But the kitchen's compulsion toward complexity could also result in a
proliferation of dollhouse garnishes. Grilled sweet shrimp were
outfitted with a heart of palm purée, microcubes of mango and cucumber,
bok choy, tiny tapioca crackers, curls of shaved hearts of palm, among
other things. The parts never quite gathered into a rush of flavor.

A variation on Mr. Boulud's classic roasted sea bass with syrah sauce
came with radicchio so bitter I wanted to slap it. A drum of sweet
potato purée with a candylike crust of marrow on top only made the next
bite of radicchio harder to take.

The kitchen loves to put two or three treatments of an ingredient side
by side, when it might do better to focus on the one that works best. In
a triptych of striped jack, a poached piece on a salad of mustard seeds
with cubes of riesling gelée tasted as if the components were destined
to be together. But there wasn't the same inevitability about the
lettuce-wrapped dumpling of striped jack tartare topped with caviar, or
the smoky rillettes surrounded by crunchy carrot and asparagus.

Daniel built its fame on Mr. Boulud's exquisite refinements on French
peasant food. Over the years, the refinements have multiplied while the
peasant food has been sent away to his many spinoff bistros.

Traces of it are still around, as in the short rib braised in red wine,
half of a beef duo. The last time I tasted it, I was sure it was the
finest French beef stew in existence. I knew my servers were trying to
make my night one I'd recall with a smile. And I wished everyone could
be so lucky.

Advertisement

\protect\hyperlink{after-bottom}{Continue reading the main story}

\hypertarget{site-index}{%
\subsection{Site Index}\label{site-index}}

\hypertarget{site-information-navigation}{%
\subsection{Site Information
Navigation}\label{site-information-navigation}}

\begin{itemize}
\tightlist
\item
  \href{https://help.nytimes.com/hc/en-us/articles/115014792127-Copyright-notice}{©~2020~The
  New York Times Company}
\end{itemize}

\begin{itemize}
\tightlist
\item
  \href{https://www.nytco.com/}{NYTCo}
\item
  \href{https://help.nytimes.com/hc/en-us/articles/115015385887-Contact-Us}{Contact
  Us}
\item
  \href{https://www.nytco.com/careers/}{Work with us}
\item
  \href{https://nytmediakit.com/}{Advertise}
\item
  \href{http://www.tbrandstudio.com/}{T Brand Studio}
\item
  \href{https://www.nytimes.com/privacy/cookie-policy\#how-do-i-manage-trackers}{Your
  Ad Choices}
\item
  \href{https://www.nytimes.com/privacy}{Privacy}
\item
  \href{https://help.nytimes.com/hc/en-us/articles/115014893428-Terms-of-service}{Terms
  of Service}
\item
  \href{https://help.nytimes.com/hc/en-us/articles/115014893968-Terms-of-sale}{Terms
  of Sale}
\item
  \href{https://spiderbites.nytimes.com}{Site Map}
\item
  \href{https://help.nytimes.com/hc/en-us}{Help}
\item
  \href{https://www.nytimes.com/subscription?campaignId=37WXW}{Subscriptions}
\end{itemize}
