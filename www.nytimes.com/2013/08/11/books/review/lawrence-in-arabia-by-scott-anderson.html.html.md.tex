Sections

SEARCH

\protect\hyperlink{site-content}{Skip to
content}\protect\hyperlink{site-index}{Skip to site index}

\href{https://www.nytimes.com/section/books/review}{Book Review}

\href{https://myaccount.nytimes.com/auth/login?response_type=cookie\&client_id=vi}{}

\href{https://www.nytimes.com/section/todayspaper}{Today's Paper}

\href{/section/books/review}{Book Review}\textbar{}Who Was T. E.
Lawrence?

\url{https://nyti.ms/1bcJQrC}

\begin{itemize}
\item
\item
\item
\item
\item
\end{itemize}

Advertisement

\protect\hyperlink{after-top}{Continue reading the main story}

Supported by

\protect\hyperlink{after-sponsor}{Continue reading the main story}

\hypertarget{who-was-t-e-lawrence}{%
\section{Who Was T. E. Lawrence?}\label{who-was-t-e-lawrence}}

By Alex von Tunzelmann

\begin{itemize}
\item
  Aug. 8, 2013
\item
  \begin{itemize}
  \item
  \item
  \item
  \item
  \item
  \end{itemize}
\end{itemize}

Among the many individual stories of World War I that will doubtless be
told and retold for the centenary years between 2014 and 2018, that of
T. E. Lawrence stands out from all the rest. As Scott Anderson says at
the beginning of ``Lawrence in Arabia,'' ``historians have never quite
decided what to make of the young, bashful Oxford scholar who rode into
battle at the head of an Arab army and changed history.''

There have, of course, been shelf-loads of books on Lawrence and his
sphere, and an extremely famous film. But the existence of previous
works may trouble critics more than readers. After all, somebody keeps
buying the stuff. Anderson, a veteran war correspondent and an author of
both fiction and nonfiction, gives Lawrence's story a new spin by
contextualizing him in a group biography. He weaves in the lives of
three contemporary Middle Eastern spies: Curt Prüfer, a German
conspiring with the Ottomans to bring down the British Empire; Aaron
Aaronsohn, a Zionist agronomist of Romanian origin, settled in
Palestine; and William Yale, an East Coast aristocrat and an agent of
Standard Oil who ended up in the service of the American State
Department. This allows him to bring in such rousingly modern themes as
oil, jihad and Arab-Jewish conflict --- though each of these was a
markedly different prospect a century ago.

There are also plenty of Middle Easterners in this book --- including
Prince Faisal, later king of Iraq, and the Turks Djemal Pasha, governor
of Syria, and Enver Pasha, minister of war, who The New York Times said
in 1915 had a reputation as ``the handsomest man in the Turkish Army''
(though it is hard to believe the correspondent inspected them all).
Even so, Anderson focuses on Westerners and their meddling --- the
Sykes-Picot Agreement, the Balfour Declaration, the Arab revolt --- and
his book could not be better timed. As global attention is drawn to
Syria and Egypt, it is arresting to look back 100 years and see a
familiar picture: Britain, France, Russia and the United States gingerly
stirring the pot of the Middle East from as far away as possible. The
result was familiar, too. ``We have appropriated too many Moslem
countries for them to have any real trust in our disinterestedness,''
Lawrence wrote in 1916.

Anderson's setting of Lawrence among other foreign agents is an
interesting and creative idea, and opens the way for some clever
connections, though Prüfer, Aaronsohn and Yale are not historical
figures of Lawrence's stature. Of the three, Aaronsohn provides the best
story, thanks not only to his own exploits but also to those of his
remarkable sister. Sarah Aaronsohn ditched a husband in Constantinople
and went to Palestine to build her own spy ring, drawn in part from her
``ardent coterie of male suitors.'' When she was interrogated by the
Turks, which involved being strapped to a gatepost and beaten, she
taunted them ``until she fell into unconsciousness.'' Aaron Aaronsohn,
no less daring, also had a talent for snappy retorts. When Djemal Pasha
threatened to hang him, Aaronsohn replied, ``Your Excellency, the weight
of my body would break the gallows with a noise loud enough to be heard
in America''; as Anderson helpfully points out, he was ``alluding to
both his considerable girth and to his network of influential friends
abroad.''

Yale's inclusion is more puzzling. Despite being ``literally the only
American field intelligence officer for the entire region,'' he appears
to have done little, not least because the region's oil industry was
then in its infancy, and --- for reasons not unrelated --- so was the
interest of the United States government. Eventually, he got himself
attached to the British Army, though after a month had ``learned
virtually nothing.'' A moment of potential excitement arrived when he
ended up watching British artillery shell Turkish positions in September
1918, but he didn't like it much: ``It was nowhere near as thrilling as
the sham battles I had watched as a boy at Van Cortlandt Park,'' he
wrote. Meanwhile, Lawrence was ``careening through the desert around
Deraa in a Rolls-Royce armored car, blowing up bridges and tearing up
railway tracks, dodging ineffective enemy air attacks, skirmishing with
the occasional unlucky Turkish foot patrol.''

\includegraphics{https://static01.nyt.com/images/2013/08/11/books/review/11VONTUNZELMAN/11VONTUNZELMAN-articleLarge.jpg?quality=75\&auto=webp\&disable=upscale}

A fine storyteller, Anderson does his best to drum up a narrative for
his American character but is ultimately defeated by the modesty of the
man's achievements. The hapless Yale might have been better suited to a
supporting role in a Graham Greene novel, where he could have had scorn
poured upon him by a jaded Englishman. Although Anderson is an American,
he takes up that duty. About a particularly poorly informed dispatch
Yale sent to the State Department, Anderson says: ``He was establishing
a tradition of fundamentally misreading the situation in the Middle East
that his successors in the American intelligence community would
rigorously maintain for the next 95 years.'' Ouch.

Regardless of the relative historical value of these individuals,
however, the multi­character approach has the great virtue of opening up
the story's complexity. Through his large cast, Anderson is able to
explore the muddles of the early-20th-century Middle East from several
distinct and enlightening perspectives. Furthermore, while he maintains
an invigorating pace, his fabulous details are given room to illuminate.
And the book is thick with them, whether it is Lawrence attempting to
collar a live leopard; Prüfer arranging 10 days of ``boozing, dancing
and flirting'' with a wayward German princess for Abbas Hilmi, the
deposed khedive of Egypt; or Aaronsohn fending off a strikingly biblical
plague of locusts.

Anderson's insight into Lawrence's character is at its sharpest when it
comes to one of the most discussed incidents in his autobiography,
``Seven Pillars of Wisdom'': the torture and rape he claimed to have
suffered while a prisoner in Deraa. Dealing with this episode, too many
biographers tend either toward amateur psychoanalysis and sensationalism
or bluster, obfuscation, even denial. For some, the question is
complicated by suggestions that Lawrence may have been homosexual and
clearly was to some degree a masochist. Was the Deraa ­torture-rape a
fantasy?

Confronted by an unknowable, historians demand evidence, witnesses,
corroboration. The fact that Lawrence's case offers none of these does
not mean he was making it up. As Anderson allows, reports of Lawrence's
swift physical recovery may indicate that he exaggerated the severity of
his torture --- or they may not. Either way, he writes,
``\emph{something} happened in Deraa,'' and it is not surprising ``that
someone enduring such a trauma might wish to adorn its memory with
staggering violence, the kind of violence that offers an absolution of
guilt by making all questions of will or resistance moot.'' This is not
a conclusion, but something more nuanced and perhaps more appropriate:
considered inconclusiveness.

Anderson is right that historians have never decided what to make of
Lawrence. (Indeed, en masse, they never really decide what to make of
anything.) Had things gone differently, he writes, ``it's hard to
imagine that any of this could possibly have produced a sadder history
than what has actually transpired over the past century, a catalog of
war, religious strife and brutal dictatorships that has haunted not just
the Middle East but the entire world.''

Despite his best efforts, Lawrence was obliged to leave the Middle East
in a state of considered inconclusiveness, too. This engrossing,
thoughtful and intricate account raises the question of whether that
might be the most outsiders can ever achieve.

Advertisement

\protect\hyperlink{after-bottom}{Continue reading the main story}

\hypertarget{site-index}{%
\subsection{Site Index}\label{site-index}}

\hypertarget{site-information-navigation}{%
\subsection{Site Information
Navigation}\label{site-information-navigation}}

\begin{itemize}
\tightlist
\item
  \href{https://help.nytimes.com/hc/en-us/articles/115014792127-Copyright-notice}{©~2020~The
  New York Times Company}
\end{itemize}

\begin{itemize}
\tightlist
\item
  \href{https://www.nytco.com/}{NYTCo}
\item
  \href{https://help.nytimes.com/hc/en-us/articles/115015385887-Contact-Us}{Contact
  Us}
\item
  \href{https://www.nytco.com/careers/}{Work with us}
\item
  \href{https://nytmediakit.com/}{Advertise}
\item
  \href{http://www.tbrandstudio.com/}{T Brand Studio}
\item
  \href{https://www.nytimes.com/privacy/cookie-policy\#how-do-i-manage-trackers}{Your
  Ad Choices}
\item
  \href{https://www.nytimes.com/privacy}{Privacy}
\item
  \href{https://help.nytimes.com/hc/en-us/articles/115014893428-Terms-of-service}{Terms
  of Service}
\item
  \href{https://help.nytimes.com/hc/en-us/articles/115014893968-Terms-of-sale}{Terms
  of Sale}
\item
  \href{https://spiderbites.nytimes.com}{Site Map}
\item
  \href{https://help.nytimes.com/hc/en-us}{Help}
\item
  \href{https://www.nytimes.com/subscription?campaignId=37WXW}{Subscriptions}
\end{itemize}
