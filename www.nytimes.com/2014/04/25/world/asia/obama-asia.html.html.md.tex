Sections

SEARCH

\protect\hyperlink{site-content}{Skip to
content}\protect\hyperlink{site-index}{Skip to site index}

\href{https://www.nytimes.com/section/world/asia}{Asia Pacific}

\href{https://myaccount.nytimes.com/auth/login?response_type=cookie\&client_id=vi}{}

\href{https://www.nytimes.com/section/todayspaper}{Today's Paper}

\href{/section/world/asia}{Asia Pacific}\textbar{}Obama Suffers Setbacks
in Japan and the Mideast

\url{https://nyti.ms/PvCLfK}

\begin{itemize}
\item
\item
\item
\item
\item
\item
\end{itemize}

Advertisement

\protect\hyperlink{after-top}{Continue reading the main story}

Supported by

\protect\hyperlink{after-sponsor}{Continue reading the main story}

\hypertarget{obama-suffers-setbacks-in-japan-and-the-mideast}{%
\section{Obama Suffers Setbacks in Japan and the
Mideast}\label{obama-suffers-setbacks-in-japan-and-the-mideast}}

\includegraphics{https://static01.nyt.com/images/2014/04/25/world/asia/subPREXY/subPREXY-articleLarge.jpg?quality=75\&auto=webp\&disable=upscale}

By \href{http://www.nytimes.com/by/mark-landler}{Mark Landler} and
\href{http://www.nytimes.com/by/jodi-rudoren}{Jodi Rudoren}

\begin{itemize}
\item
  April 24, 2014
\item
  \begin{itemize}
  \item
  \item
  \item
  \item
  \item
  \item
  \end{itemize}
\end{itemize}

TOKYO --- President Obama encountered setbacks to two of his most
cherished foreign-policy projects on Thursday, as he failed to achieve a
trade deal that undergirds his strategic pivot to Asia and the Middle
East peace process suffered a potentially irreparable breakdown.

Mr. Obama had hoped to use his visit here to announce an agreement under
which Japan would open its markets in rice, beef, poultry and pork, a
critical step toward the Trans-Pacific Partnership, the proposed
regional trade pact. But Prime Minister Shinzo Abe was not able to
overcome entrenched resistance from Japan's farmers in time for the
president's visit.

In Jerusalem,
\href{http://www.nytimes.com/2014/04/25/world/middleeast/israel.html}{Israel's
announcement} that it was suspending stalemated peace negotiations with
the Palestinians, after
\href{http://www.nytimes.com/2014/04/24/world/middleeast/palestinian-factions-announce-deal-on-unity-government.html}{a
reconciliation} between the Palestine Liberation Organization and the
militant group Hamas, posed yet another obstacle to restarting a
troubled peace process in which Secretary of State John Kerry has been
greatly invested.

The setbacks, though worlds apart in geography and history, speak to the
common challenge Mr. Obama has had in translating his ideas and
ambitions into enduring policies. He has watched outside forces unravel
his best-laid plans, from resetting relations with Russia to managing
the epochal political change in the Arab world. On Thursday, as Russia
\href{http://www.nytimes.com/2014/04/25/world/europe/ukraine-crisis.html}{staged
military exercises} on the border with Ukraine, Mr. Kerry denounced
broken promises from the Kremlin but took no specific action.

Mr. Obama has not given up. Indeed, his advisers insisted that they had
achieved a ``pathway'' to resolving the sticking points in a trade deal
during marathon talks that continued until just before the president
left Tokyo for Seoul.

On Friday in South Korea, Mr. Obama continued his weeklong quest to
breathe life into his shift to Asia. Mr. Kerry, the tireless campaigner
for Middle East peace, was still working the phones, trying to maneuver
the Israelis and Palestinians back to the negotiating room.

One of the president's most grandiose foreign-policy projects, a nuclear
agreement with Iran, remains very much on the table, with diplomats from
Iran and the West beginning to draft language that would limit Iran's
nuclear program and inhibit its ability to produce a weapon.

In one sense, the latest news from the Middle East offers a rationale
for Mr. Obama to keep his gaze fixed on the fast-growing economies of
Asia. While the troubles with the peace negotiations have surprised
almost no one, the trade talks with Japan still hold some hope of
yielding a landmark deal, since it is in the interests of both Mr. Abe
and Mr. Obama --- a bet on the future rather than an effort to clear the
enmities of the past.

First, though, Mr. Obama has to overcome the stubborn hurdles to any
trade agreement. Back home, he has been unable to win support from
Congress for the deal. ``Prime Minister Abe has got to deal with his
politics; I've got to deal with mine,'' he said on Thursday. ``It means
that we sometimes have to push our constituencies beyond their current
comfort levels.''

Mr. Obama also declared that the United States was obligated by a
security treaty to protect Japan in its confrontation with China over a
clump of islands in the East China Sea. But he stopped short of siding
with Japan in the dispute regarding who has sovereignty over the
islands, and carefully calibrated his statement to avoid antagonizing
China.

The net result, seen in a news conference in which the leaders referred
to each other a bit stiffly as Barack and Shinzo, was an alliance
clearly on firmer footing than it was earlier, but still vulnerable to
political frailties on each side.

Similar frailties were on display in Jerusalem, where Prime Minister
Benjamin Netanyahu of Israel acted swiftly to suspend talks after his
Palestinian counterpart, President Mahmoud Abbas, signed a deal seeking
to reconcile his Fatah faction, which dominates the Palestine Liberation
Organization and leads the West Bank government, with Hamas, the
Islamist group that controls the Gaza Strip.

Mr. Abbas ``had a choice: peace with Israel or pact with the terrorist
Hamas,'' Mr. Netanyahu said in an interview with NBC News. ``So that's
the blow for peace, and I hope he changes his mind.''

Mr. Kerry told Mr. Abbas on Thursday that he was disappointed by the
reconciliation announcement, and he planned to speak later with Mr.
Netanyahu. The administration's Middle East peace envoy, Martin S.
Indyk, remained in the region, refusing to give up.

\includegraphics{https://static01.nyt.com/images/2014/04/25/world/PREXY/PREXY-videoSixteenByNine1050.jpg}

``Choices need to be made by both parties, and we'll see what happens in
the days ahead,'' Jen Psaki, a State Department spokeswoman, told
reporters in Washington. Palestinian leaders sought to shift the blame
for the breakdown to Israel.

Saeb Erekat, the chief Palestinian negotiator, said Israel had
``deliberately sabotaged the peace process by stopping the
negotiations'' and by refusing to freeze settlement construction. Mr.
Erekat said in a telephone interview that reconciliation was ``an
internal Palestinian affair that Israel has no right to interfere in''
and that it ``should not be used as a pretext to evade negotiations.''

But after a six-hour meeting on Thursday, Israel's top ministers voted
unanimously to halt the talks and to impose an unspecified series of
punitive measures against the Palestinians for their promise to form a
new government within five weeks that would prepare for long-overdue
elections. Negotiations could only resume, Israeli officials said, if
Mr. Abbas abandoned or failed to carry out the deal with Hamas ---
something Palestinian analysts say is a possibility.

``What will happen now is the usual routine: Israel will try to punish
the Palestinians; the Palestinians will complain; Israel will not really
punish the Palestinians, because we cannot afford the Palestinian
Authority to collapse,'' said Shlomo Brom, a researcher at the Institute
for National Security Studies in Tel Aviv. ``So it will be another
episode in this never-ending story.''

Events in Asia tend to move at a slower pace. But with trade a key
pillar of Mr. Obama's strategy in Asia and each side looking for
something from the other, the negotiations assumed a Middle East-like
intensity.

In round-the-clock negotiations this week, American and Japanese
officials said, the United States pressed Japan to make major
concessions, including cutting protective tariffs close to zero on
agricultural products like cheese and pork.

``There are a lot of people in Japan who question whether Japan should
make big concessions just because Obama is here,'' said Nobuhiro Suzuki,
a professor of agriculture at the University of Tokyo. ``Abe has to heed
them, too, to avoid appearing like an American patsy.''

Other analysts faulted Mr. Obama, saying his decision not to fight for
the legislative authority at home to pass major trade deals had robbed
him of leverage with the Japanese, who are reluctant to make concessions
for a deal that may not survive Congress.

``Their strategy was to get the Japanese to do the deal, then go to
Congress and say, `Look what a great deal we got, now give us the
authority,' '' said Michael J. Green, an Asia adviser to President
George W. Bush. ``He made a decision to go into this with one hand tied
behind his back.''

Speaking to reporters on Air Force One en route to Seoul, a senior
administration official said that after their meeting, Mr. Obama and Mr.
Abe instructed their staffs to find a way to close the remaining gaps.
``Nothing is agreed to until everything is agreed to,'' the official
said. But he added, ``We're at a moment where we see where we're going
to achieve resolution.''

In Tokyo, the Japanese trade minister, Akira Amari, told reporters that
no agreement had been reached, but noted: ``We are moving toward a
conclusion.''

The president's statement about the United States' obligations toward
Japan was important because it was the first time he had explicitly put
the disputed islands under American protection, though Defense Secretary
Chuck Hagel recently made the same statement and the policy has been
held by successive administrations.

``This is a very important turning point for the United States-Japan
alliance because it means the period of drift under President Obama has
finally come to an end,'' said Yuichi Hosoya, an expert on
American-Japanese relations at Keio University in Tokyo. ``The fact that
this was said by the president will have a huge psychological impact on
Japanese officials and people.''

The Chinese government reacted swiftly, saying it was ``firmly opposed''
to Mr. Obama's position. More than anything, Mr. Obama appeared eager to
defuse tensions over the islands, referring to them as a ``rock'' and
saying they should not be allowed to derail a relationship that could
otherwise be productive.

``It would be a profound mistake to continue to see escalation around
this issue rather than dialogue and confidence-building measures between
Japan and China,'' Mr. Obama said.

Mr. Abe said he was encouraged by Mr. Obama's pledge to protect the
islands. ``On this point,'' he said, ``I fully trust President Obama.''

Advertisement

\protect\hyperlink{after-bottom}{Continue reading the main story}

\hypertarget{site-index}{%
\subsection{Site Index}\label{site-index}}

\hypertarget{site-information-navigation}{%
\subsection{Site Information
Navigation}\label{site-information-navigation}}

\begin{itemize}
\tightlist
\item
  \href{https://help.nytimes.com/hc/en-us/articles/115014792127-Copyright-notice}{©~2020~The
  New York Times Company}
\end{itemize}

\begin{itemize}
\tightlist
\item
  \href{https://www.nytco.com/}{NYTCo}
\item
  \href{https://help.nytimes.com/hc/en-us/articles/115015385887-Contact-Us}{Contact
  Us}
\item
  \href{https://www.nytco.com/careers/}{Work with us}
\item
  \href{https://nytmediakit.com/}{Advertise}
\item
  \href{http://www.tbrandstudio.com/}{T Brand Studio}
\item
  \href{https://www.nytimes.com/privacy/cookie-policy\#how-do-i-manage-trackers}{Your
  Ad Choices}
\item
  \href{https://www.nytimes.com/privacy}{Privacy}
\item
  \href{https://help.nytimes.com/hc/en-us/articles/115014893428-Terms-of-service}{Terms
  of Service}
\item
  \href{https://help.nytimes.com/hc/en-us/articles/115014893968-Terms-of-sale}{Terms
  of Sale}
\item
  \href{https://spiderbites.nytimes.com}{Site Map}
\item
  \href{https://help.nytimes.com/hc/en-us}{Help}
\item
  \href{https://www.nytimes.com/subscription?campaignId=37WXW}{Subscriptions}
\end{itemize}
