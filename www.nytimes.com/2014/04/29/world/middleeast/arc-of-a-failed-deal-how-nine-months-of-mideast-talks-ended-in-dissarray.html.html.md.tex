Sections

SEARCH

\protect\hyperlink{site-content}{Skip to
content}\protect\hyperlink{site-index}{Skip to site index}

\href{https://www.nytimes.com/section/world/middleeast}{Middle East}

\href{https://myaccount.nytimes.com/auth/login?response_type=cookie\&client_id=vi}{}

\href{https://www.nytimes.com/section/todayspaper}{Today's Paper}

\href{/section/world/middleeast}{Middle East}\textbar{}Arc of a Failed
Deal: How Nine Months of Mideast Talks Ended in Disarray

\url{https://nyti.ms/1nAGzeq}

\begin{itemize}
\item
\item
\item
\item
\item
\end{itemize}

Advertisement

\protect\hyperlink{after-top}{Continue reading the main story}

Supported by

\protect\hyperlink{after-sponsor}{Continue reading the main story}

\hypertarget{arc-of-a-failed-deal-how-nine-months-of-mideast-talks-ended-in-disarray}{%
\section{Arc of a Failed Deal: How Nine Months of Mideast Talks Ended in
Disarray}\label{arc-of-a-failed-deal-how-nine-months-of-mideast-talks-ended-in-disarray}}

\includegraphics{https://static01.nyt.com/images/2014/04/29/world/RECONSTRUCT/RECONSTRUCT-articleLarge.jpg?quality=75\&auto=webp\&disable=upscale}

By \href{http://www.nytimes.com/by/jodi-rudoren}{Jodi Rudoren} and
\href{https://www.nytimes.com/by/isabel-kershner}{Isabel Kershner}

\begin{itemize}
\item
  April 28, 2014
\item
  \begin{itemize}
  \item
  \item
  \item
  \item
  \item
  \end{itemize}
\end{itemize}

JERUSALEM --- There were late-night video conferences with Secretary of
State John Kerry, including one from beneath mosquito netting in an
Indonesian hotel. Mr. Kerry met a total of 34 times with President
Mahmoud Abbas of the Palestinian Authority, and about twice that with
Prime Minister Benjamin Netanyahu of Israel.

Israeli and Palestinian representatives were summoned for talks in
Amman, Jordan; Davos, Switzerland; London; Munich; Paris; Rome; and
Washington. And Mr. Kerry's peace envoy, Martin S. Indyk, trekked with a
Palestinian leader to ancient ruins in Jericho.

In the last few weeks, even as both sides took steps that undermined the
process, Mr. Kerry and his team produced a new package of incentives,
including Palestinian autonomy for planning and zoning in
Israeli-controlled parts of the West Bank. All sides left a meeting last
Tuesday optimistic.

The talks nonetheless
\href{http://www.nytimes.com/2014/04/26/world/middleeast/collapse-of-peace-talks-leaves-israel-in-precarious-position.html?hpw\&rref=world}{collapsed}
two days later. Mr. Kerry has his share of the blame, at times leaving
Israeli and Palestinian leaders with disparate understandings that would
lead to later blowups and, toward the end, pushing beyond the White
House's comfort zone to create a new layer of internal negotiations that
slowed events down.

But Mr. Netanyahu refused to risk alienating Israel's right wing by
restraining construction in West Bank and East Jerusalem settlements;
about 13,000 new units moved forward during the talks. Mr. Abbas,
looking for a dignified exit from the public stage and furious over the
settlement building, never responded to the ideas Mr. Kerry's team had
formulated for a framework to guide further negotiations.

Ultimately, the latest round proved the perennial truth with Middle East
peacemaking: Washington cannot force an agreement if the parties are
unwilling.

``It's part of the pathology of the Israeli-Palestinian relationship
that what one side demands the other side has a predisposition to
reject,'' said an American official knowledgeable about the
negotiations, speaking on the condition of anonymity under White House
dictate. ``It's one of the reasons that it's so difficult to sustain
negotiations, never mind get an agreement.''

Mr. Kerry set the lofty goal last July ``to achieve a final-status
agreement over the course of the next nine months.'' Instead, as that
deadline passes Tuesday, Israeli and Palestinian leaders are preparing a
battery of punitive measures and unilateral steps that could spiral into
the dissolution of the Palestinian Authority, bringing one of the
world's most intractable conflicts to a new low.

After
\href{http://www.nytimes.com/2014/04/24/world/middleeast/palestinian-factions-announce-deal-on-unity-government.html?hpw\&rref=world}{the
pact signed last week} by the Palestine Liberation Organization and the
militant Islamic faction Hamas led Israel to halt the talks, President
Obama said Friday it may be time for a pause in American intervention.

``I've been going through some soul-searching: Why, if we both say
two-state solution, what's gone wrong?'' Saeb Erekat, the chief
Palestinian negotiator, asked in an interview Monday. ``My only answer
is we have failed to sit and agree on a map on borders of the two
states. Everything else will be a domino effect.''

Mr. Erekat's Israeli counterpart, Tzipi Livni, acknowledged that
continued settlement construction was a problem, but said the
Palestinians knew it was coming. Twice in April, she pointed out, even
as details of new deals were being completed, the Palestinians surprised
Israel and Washington,
\href{http://www.nytimes.com/2014/04/02/world/middleeast/jonathan-pollard.html}{first
by joining 15 international conventions} to protest Israel's failure to
release a promised fourth batch of prisoners, and last week by
reconciling with Hamas.

``When you make an agreement with somebody, if he says to you, `Listen,
I'm going to pay, but it's going to take some time,'~'' Ms. Livni said,
``if you want the deal and if you really want to continue negotiations,
you wait.''

The nine-month talks, likened from the beginning to a pregnancy, broke
roughly into three trimesters. First were about 20 bilateral meetings in
which both Israeli and Palestinian negotiators failed to budge from
their opening, maximalist positions. Then, after settlement
announcements prompted the resignation of one Palestinian negotiator,
those unproductive sessions were replaced in November with so-called
proximity talks between each side and the Americans,
\href{http://www.nytimes.com/2014/01/01/world/middleeast/kerry-to-press-for-framework-accord-to-keep-mideast-peace-effort-moving.html}{focused
on the framework}. Finally, starting in March, the goal was truncated to
simply
\href{http://www.nytimes.com/2014/04/05/world/middleeast/mideast.html}{extending
the talks}.

Joining Mr. Erekat at first was Mohammed Shtayyeh, an economist (he was
later replaced by Majid Faraj, the intelligence chief). The Israeli side
was led by Ms. Livni, Israel's justice minister, and Isaac Molho, Mr.
Netanyahu's discreet lawyer. Tal Becker, on loan from Israel's foreign
ministry, and Waseem Khazmo, Mr. Erekat's lawyer, sat in on all
sessions, and Michael Herzog, a retired Israeli general, joined
midstream.

They talked for hours, in English, with Mr. Erekat repeating certain
mantras: ``I'm willing to limit my sovereignty but not my dignity,'' he
would say. ``I don't walk around with a neon sign saying `stupid' on my
head.''

There was always food. In a first for Israeli peacemakers, there was
Chinese takeout in December at Mr. Kerry's townhouse. At the King David
hotel recently, Mr. Faraj gobbled up matzo, having developed a taste for
it in Israeli prisons. When a snowstorm shuttered Washington offices ---
and supermarkets --- Mr. Indyk's wife, Gahl Burt, scrambled to assemble
supper two nights running, only to discover too late that Ms. Livni was
a vegetarian and Mr. Erekat did not eat fish.

Sometimes they sat in Mr. Erekat's office in Jericho, where he has a
blown-up photograph from his December trip to Hisham's Palace with Mr.
Indyk.

``I meant to take Martin to ruins to show him nothing lasts and life
goes on,'' Mr. Erekat explained. ``These were great empires --- they're
gone. I know that the Israeli occupation will go. I know.''

The first turning point came Nov. 5. After the second of Israel's four
promised batches of prisoners were released, amid anguished protests in
Jerusalem, various plans for nearly 20,000 settlement units were pushed
forward over five days (some were later withdrawn). The Palestinians
were outraged not only at the scale, but that Israelis were suggesting
they had agreed to trade construction for prisoners, when in fact the
``price'' was a pledge not to join international agencies and
conventions for the duration of the talks.

At a negotiating session in a Jerusalem suburb, Mr. Erekat pulled from
his bag the computer disk he always carried containing accession papers
and threatened to join 15 conventions ``tomorrow.'' He and Mr. Shtayyeh
drove directly to Bethlehem to submit their resignations (only Mr.
Shtayyeh's was accepted).

Mr. Kerry condemned the construction, asking in a television interview,
``How can you say we're planning to build in the place that will
eventually be Palestine?''

Prime Minister Netanyahu suggested shifting to separate talks with the
Americans, with the idea that the Israeli and Palestinian publics might
more easily swallow a third round of prisoner releases and settlement
announcements if they came with substantive progress. In a sign of
progress, Mr. Abbas suggested that the Israeli military could remain in
the West Bank for five years and then be replaced by either NATO or
United States troops. Israel did delay settlement after December's
prisoner release, but only for the few days of a Kerry visit.

It was Feb. 19, at the five-star Hotel Le Meurice in Paris, that Mr.
Kerry's team began to believe their mission was doomed. President Abbas,
who is 79, told Mr. Kerry he had been battling a cold for two weeks and
was ``very stressed.'' The two men talked for two hours but got nowhere.
A month later at the White House, President Obama offered the framework
outlines --- though no document was ever shared --- and Mr. Abbas simply
did not respond.

``He had shut down,'' said one of several American officials
interviewed. ``As he comes to the end of his life and certainly the end
of his term in office, he's fed up.''

``His experience in the last nine months, of settlements gone wild,''
this official added, ``has just, I think, convinced him that he doesn't
have a partner.''

As the March 29 deadline
\href{http://www.nytimes.com/2014/03/24/world/middleeast/standoff-over-prisoner-release-threatens-mideast-talks.html}{approached
for releasing} the final prisoners, a lingering problem re-emerged. Mr.
Kerry had allowed the Palestinians to believe Arab-Israeli citizens
would be among those freed without securing such a commitment from Mr.
Netanyahu. The Israelis said no one would be let go unless talks were
extended.

Mr. Kerry dangled the prospect of freeing
\href{http://www.nytimes.com/2014/04/01/world/middleeast/pollard.html}{Jonathan
J. Pollard}, the American convicted of spying for Israel, despite White
House reservations. But on April 1, even as Mr. Netanyahu was gathering
votes for the new deal, an old tender for 708 apartments in East
Jerusalem's Gilo, was republished. Soon Mr. Abbas was on television
signing documents to join the international conventions.

Still they kept talking. Negotiators barely slept for three weeks. Last
Tuesday, the latest package was presented at a meeting variously
described as ``serious,'' ``positive'' and ``excellent.''

``We had ups and downs: One day if you would ask me I would tell you
it's going to happen, the next day I would tell you it's not going to
happen,'' one Israeli said. That Tuesday, he added, ``we felt maybe we
are going to make it. They asked, `Let's meet the next day.'~''

But the next day, Palestine Liberation Organization leaders held hands
aloft with those from Hamas. Israel immediately canceled the scheduled
negotiating session, and 24 hours later, froze talks indefinitely.

Advertisement

\protect\hyperlink{after-bottom}{Continue reading the main story}

\hypertarget{site-index}{%
\subsection{Site Index}\label{site-index}}

\hypertarget{site-information-navigation}{%
\subsection{Site Information
Navigation}\label{site-information-navigation}}

\begin{itemize}
\tightlist
\item
  \href{https://help.nytimes.com/hc/en-us/articles/115014792127-Copyright-notice}{©~2020~The
  New York Times Company}
\end{itemize}

\begin{itemize}
\tightlist
\item
  \href{https://www.nytco.com/}{NYTCo}
\item
  \href{https://help.nytimes.com/hc/en-us/articles/115015385887-Contact-Us}{Contact
  Us}
\item
  \href{https://www.nytco.com/careers/}{Work with us}
\item
  \href{https://nytmediakit.com/}{Advertise}
\item
  \href{http://www.tbrandstudio.com/}{T Brand Studio}
\item
  \href{https://www.nytimes.com/privacy/cookie-policy\#how-do-i-manage-trackers}{Your
  Ad Choices}
\item
  \href{https://www.nytimes.com/privacy}{Privacy}
\item
  \href{https://help.nytimes.com/hc/en-us/articles/115014893428-Terms-of-service}{Terms
  of Service}
\item
  \href{https://help.nytimes.com/hc/en-us/articles/115014893968-Terms-of-sale}{Terms
  of Sale}
\item
  \href{https://spiderbites.nytimes.com}{Site Map}
\item
  \href{https://help.nytimes.com/hc/en-us}{Help}
\item
  \href{https://www.nytimes.com/subscription?campaignId=37WXW}{Subscriptions}
\end{itemize}
