Sections

SEARCH

\protect\hyperlink{site-content}{Skip to
content}\protect\hyperlink{site-index}{Skip to site index}

\href{https://www.nytimes.com/section/nyregion}{New York}

\href{https://myaccount.nytimes.com/auth/login?response_type=cookie\&client_id=vi}{}

\href{https://www.nytimes.com/section/todayspaper}{Today's Paper}

\href{/section/nyregion}{New York}\textbar{}Arthur G. Cohen, Real Estate
Developer, Is Dead at 84

\url{https://nyti.ms/1qdOhI1}

\begin{itemize}
\item
\item
\item
\item
\item
\end{itemize}

Advertisement

\protect\hyperlink{after-top}{Continue reading the main story}

Supported by

\protect\hyperlink{after-sponsor}{Continue reading the main story}

\hypertarget{arthur-g-cohen-real-estate-developer-is-dead-at-84}{%
\section{Arthur G. Cohen, Real Estate Developer, Is Dead at
84}\label{arthur-g-cohen-real-estate-developer-is-dead-at-84}}

\includegraphics{https://static01.nyt.com/images/2014/08/16/nyregion/cohen-obit/cohen-obit-articleLarge.jpg?quality=75\&auto=webp\&disable=upscale}

By \href{http://www.nytimes.com/by/douglas-martin}{Douglas Martin}

\begin{itemize}
\item
  Aug. 15, 2014
\item
  \begin{itemize}
  \item
  \item
  \item
  \item
  \item
  \end{itemize}
\end{itemize}

Arthur G. Cohen, who began a roller-coaster real estate career with a
\$25,000 investment in tract housing on Long Island before creating the
nation's largest publicly held real estate company, teaming up with
tycoons like Aristotle Onassis to build trophy Manhattan skyscrapers,
died on Aug. 9 at his home in Kings Point, N.Y. He was 84.

His family announced the death.

``There was a time when every deal had Arthur Cohen,'' Andrew Albstein,
a lawyer who worked with Mr. Cohen,
\href{http://observer.com/2011/06/the-property-pretenders/}{told The New
York Observer} in 2011. ``If there was a deal to be done in New York,
you had to go through him.''

Mr. Cohen helped rehabilitate Times Square by investing in an office
tower and the Crowne Plaza hotel; developed the Westyard Distribution
Center in 1970 on what was then a far frontier, mid-Manhattan's West
Side; and helped Ian Schrager turn dowager hotels like the Royalton on
West 44th Street into havens for the hip. He helped build One Worldwide
Plaza on the Eighth Avenue site of the old Madison Square Garden and a
gleaming glass skyscraper across from Carnegie Hall.

In 1971, Mr. Cohen joined Mr. Onassis to build an audacious 51-story
tower on 51st Street across from St. Patrick's Cathedral. Named Olympic
Tower in reference to Mr. Onassis' homeland, Greece, its combination of
residential, office and shopping uses made it the prototype for new
zoning rules mandating that new Fifth Avenue buildings include retail
stores on the first two floors.

Mr. Cohen, who helped write the new regulations, joined Mayor John V.
Lindsay at the news conference to announce the new building and zoning,
but he usually preferred to be in the background. His many partners in
his many projects often got top billing. His behind-the-scenes ability
to structure elaborate deals nonetheless dazzled the industry.

``Associates call him `a laser beam' for his power to slice immediately
to the heart of intricate problems,'' Fortune magazine said in 1972.

In 1991, Newsday suggested that Mr. Cohen had his hand in one in seven
real estate deals in New York City. Fortune said Mr. Cohen operated ``on
a larger scale'' than the powerful developer William Zeckendorf ``at his
zenith.''

His reach extended well beyond New York City. Fortune called the company
he took public in 1971, the Arlen Realty and Development Corporation,
the largest publicly held real estate enterprise in the nation.

It operated in 39 states and bought, sold, developed or managed a vast
array of shopping centers, office buildings, apartment houses and a
large planned community in Florida, Aventura. In its heyday, in 1975, it
controlled over \$1.7 billion (\$7.5 billion in today's dollars) of real
estate assets.

In the New York region, the Brooklyn-born Mr. Cohen ventured outside
Manhattan. Most of the 15,000 apartments in the area that he bought and
refurbished to sell as condominiums or co-ops were in other boroughs or
suburbs. He teamed up with David Walentas in acquiring two million
square feet of industrial buildings in Dumbo, Brooklyn, to convert to
residential and commercial use.

He also ventured beyond real estate. He owned parts of Braniff Airlines
and the restaurant chains Houlihan's and Darryl's as a result of
leveraged buyouts.

Image

The Olympic Tower, across from St. Patrick's Cathedral.Credit...Jack
Manning/The New York Times

Through Arlen Realty, he owned much of E. J. Korvette (also known as
Korvette's), the discount-store chain. He was unsuccessful in attempts
to buy The New York Post and the Plaza Hotel, across from Central Park.

Other rough spots were rougher. Real estate prices crashed in the
mid-1970s, and Korvette's hemorrhaged cash as it vainly scrambled for
new retailing strategies. Arlen's market value plummeted from \$500
million in 1972 to \$60 million five years later. In 1979, Arlen sold
Korvette's to a French company, which soon closed all of its stores.

Realizing that in real estate one's reputation with creditors is as
important as location, Mr. Cohen dug into his personal savings for \$30
million to pay back Arlen's debts.

Another setback came in 1998 when federal authorities charged that a
savings bank Mr. Cohen had bought in Virginia was illegally funneling
money to dubious construction projects in New York in which he was
accused of having undisclosed interests. He paid \$4.5 million to
resolve the case without admitting guilt.

Arthur George Cohen was born in Brooklyn on April 23, 1930. His father,
Louis, was a lawyer who invested in numerous gasoline stations and
apartments. Arthur was exposed to the real estate business through
shoptalk at the dinner table and summer work assignments.

He earned a degree in business from the University of Miami and a law
degree from New York Law School, taking classes at night. His golf game
was so good that he considered turning pro. In June 1954, he married
Karen Bassine from Great Neck, on Long Island.

She survives him, as do their daughters, Lauren Reddington, Susan
Siegel, Debra Duran, Rochelle Rosenberg and Kathy Horowitz; his sister,
Marilyn Davimos; and nine grandchildren.

Looking for a way to break into real estate in 1954, Mr. Cohen joined
two partners to invest \$75,000 in Long Island tract housing, in
Huntington. He netted \$100,000 from his \$25,000 investment and soon
began building one-family houses in Yonkers and in Florida.

Mr. Cohen's formula for success was set: Select property with a good
chance to increase in value, recruit partners and then borrow the full
amount of the development cost --- and sometimes more. He often
contributed only his efforts and creativity --- or ``sweat equity,'' as
it is known in the trade. With the right ideas, not having much money
was not an insurmountable obstacle, he learned.

``I'd look at every project that came along,'' he told Fortune. ``It
never made any difference whether the thing had an equity requirement of
a million dollars or \$20 million, because I didn't have either. Having
no money at all, there was no limit to what I could do, because to do
anything I had to create value and so generate the capital I needed.''

Mr. Cohen had another major advantage: his father-in-law, Charles
Bassine, who owned Spartan Industries, which owned Korvette's. In 1971,
Mr. Cohen and Mr. Bassine merged their companies to form Arlen and
issued shares on the New York Stock Exchange to raise capital.

At the stockholders meeting in which the merger was approved, some
shareholders scoffed that Mr. Cohen's nomination as chairman was rank
nepotism. Mr. Bassine responded that Mr. Cohen did not fit the
television caricature of ``the dimwitted son-in-law who races around in
an auto that his father-in-law bought.''

``I cordially wish you all such a son-in-law,'' he said. ``He's a better
man than I am.''

Advertisement

\protect\hyperlink{after-bottom}{Continue reading the main story}

\hypertarget{site-index}{%
\subsection{Site Index}\label{site-index}}

\hypertarget{site-information-navigation}{%
\subsection{Site Information
Navigation}\label{site-information-navigation}}

\begin{itemize}
\tightlist
\item
  \href{https://help.nytimes.com/hc/en-us/articles/115014792127-Copyright-notice}{©~2020~The
  New York Times Company}
\end{itemize}

\begin{itemize}
\tightlist
\item
  \href{https://www.nytco.com/}{NYTCo}
\item
  \href{https://help.nytimes.com/hc/en-us/articles/115015385887-Contact-Us}{Contact
  Us}
\item
  \href{https://www.nytco.com/careers/}{Work with us}
\item
  \href{https://nytmediakit.com/}{Advertise}
\item
  \href{http://www.tbrandstudio.com/}{T Brand Studio}
\item
  \href{https://www.nytimes.com/privacy/cookie-policy\#how-do-i-manage-trackers}{Your
  Ad Choices}
\item
  \href{https://www.nytimes.com/privacy}{Privacy}
\item
  \href{https://help.nytimes.com/hc/en-us/articles/115014893428-Terms-of-service}{Terms
  of Service}
\item
  \href{https://help.nytimes.com/hc/en-us/articles/115014893968-Terms-of-sale}{Terms
  of Sale}
\item
  \href{https://spiderbites.nytimes.com}{Site Map}
\item
  \href{https://help.nytimes.com/hc/en-us}{Help}
\item
  \href{https://www.nytimes.com/subscription?campaignId=37WXW}{Subscriptions}
\end{itemize}
