Sections

SEARCH

\protect\hyperlink{site-content}{Skip to
content}\protect\hyperlink{site-index}{Skip to site index}

\href{https://www.nytimes.com/pages/business/media/index.html}{Media}

\href{https://myaccount.nytimes.com/auth/login?response_type=cookie\&client_id=vi}{}

\href{https://www.nytimes.com/section/todayspaper}{Today's Paper}

\href{/pages/business/media/index.html}{Media}\textbar{}Sony
Cyberattack, First a Nuisance, Swiftly Grew Into a Firestorm

\url{https://nyti.ms/1wzuG7O}

\begin{itemize}
\item
\item
\item
\item
\item
\item
\end{itemize}

Advertisement

\protect\hyperlink{after-top}{Continue reading the main story}

Supported by

\protect\hyperlink{after-sponsor}{Continue reading the main story}

\hypertarget{sony-cyberattack-first-a-nuisance-swiftly-grew-into-a-firestorm}{%
\section{Sony Cyberattack, First a Nuisance, Swiftly Grew Into a
Firestorm}\label{sony-cyberattack-first-a-nuisance-swiftly-grew-into-a-firestorm}}

\includegraphics{https://static01.nyt.com/images/2014/12/31/business/jpSONY1/jpSONY1-articleLarge.jpg?quality=75\&auto=webp\&disable=upscale}

By \href{http://www.nytimes.com/by/michael-cieply}{Michael Cieply} and
\href{http://www.nytimes.com/by/brooks-barnes}{Brooks Barnes}

\begin{itemize}
\item
  Dec. 30, 2014
\item
  \begin{itemize}
  \item
  \item
  \item
  \item
  \item
  \item
  \end{itemize}
\end{itemize}

LOS ANGELES --- It was three days before Thanksgiving, the beginning of
a quiet week for Sony Pictures. But Michael Lynton, the studio's chief
executive, was nonetheless driving his Volkswagen GTI toward Sony's lot
at 6 a.m. Final planning for corporate meetings in Tokyo was on his
agenda --- at least until his cellphone rang.

The studio's chief financial officer, David C. Hendler, was calling to
tell his boss that Sony's computer system had been compromised in a
hacking of unknown proportions. To prevent further damage, technicians
were debating whether to take Sony Pictures entirely offline.

Shortly after Mr. Lynton reached his office in the stately Thalberg
building at Sony headquarters in Culver City, Calif., it became clear
that the situation was much more dire. Some of the studio's 7,000
employees, arriving at work, turned on their computers to find macabre
images of Mr. Lynton's severed head. Sony shut down all computer systems
shortly thereafter, including those in overseas offices, leaving the
company in the digital dark ages: no voice mail, no corporate email, no
production systems.

A handful of old BlackBerrys, located in a storage room in the Thalberg
basement, were given to executives. Staff members began to trade text
messages using hastily arranged phone trees. Sony's already lean
technical staff began working around the clock, with some people
sleeping in company offices that became littered with stale pizza.
Administrators hauled out old machines that allowed them to cut physical
payroll checks in lieu of electronic direct deposit.

Still, for days the episode was viewed inside Sony as little more than a
colossal annoyance. Though Sony executives were quickly in touch with
federal law enforcement officials, the company's initial focus was on
setting up jury-rigged systems to let it limp through what was expected
to be a few days or weeks of inconvenience. The company's first
statement on the breach, made on Nov. 24, seems almost absurdly bland in
retrospect: ``We are investigating an I.T. matter.''

In fact, less than three weeks later Sony would be the focal point of a
global firestorm over a growing digital attack on its corporate identity
and data; its movie ``The Interview,'' about the fictional assassination
of the North Korean leader Kim Jong-un; and its own handling of the
ensuing crisis.

Interviews with over two dozen people involved in the episode suggest
that Sony --- slow to realize the depths of its peril --- let its
troubles deepen by mounting a public defense only after enormous damage
had been done. The initial decision to treat the attack as largely an
internal matter reflected Hollywood habit and the executive sang-froid
of Mr. Lynton, who can be cool almost to a fault. As Mr. Lynton
discovered, however, at a midpoint in the episode, this predicament
required a wholly different approach.

In truth, ``There is no playbook for us to turn to,'' Mr. Lynton told
his staff at one point. Mr. Lynton and his colleagues underestimated the
ferocity of the interaction between the news media and the hackers as
the drama unfolded in December. Hackers released the information to
traffic-hungry websites, which published the most embarrassing details,
while Sony mostly stayed publicly silent.

Hurt by a misstep when it announced the cancellation of a Christmas Day
release for ``The Interview,'' Sony was knocked about by criticism by
the White House, Hollywood stars and others who accused it of
capitulating to extortionist threats. The studio's ultimate success in
showing its film in face of a terror threat came after Mr. Lynton's
natural reserve fell more in line with the passion and grit of the
studio's co-chairwoman, Amy Pascal, who was undermined early in the
attack by the disclosure of embarrassing personal emails.

\includegraphics{https://static01.nyt.com/images/2014/12/31/business/SUBjpSONY3/SUBjpSONY3-articleLarge.jpg?quality=75\&auto=webp\&disable=upscale}

The son of a German Jew who served in British intelligence during World
War II, Mr. Lynton, 54, had weathered past corporate crises, including
an inherited accounting scandal when he ran the Penguin publishing house
and a recent attempt by the activist investor Daniel S. Loeb to force
change at Sony. But neither of those episodes matched the complexity and
surreal twists of the hacking, which ultimately became a test of
national will, a referendum on media behavior and a defense of free
expression, even of the crudest sort.

``What it amounted to was criminal extortion,'' Mr. Lynton said in an
interview.

\hypertarget{rising-sense-of-urgency}{%
\subsection{Rising Sense of Urgency}\label{rising-sense-of-urgency}}

By Dec. 1, a week after Sony discovered the breach, a sense of urgency
and horror had penetrated the studio. More than a dozen F.B.I.
investigators were setting up shop on the Culver City lot and in a
separate Sony facility near the Los Angeles airport called Corporate
Pointe, helping Sony deal with one of the worst cyberattacks ever on an
American company.

Mountains of documents had been stolen, internal data centers had been
wiped clean, and 75 percent of the servers had been destroyed.

Everything and anything had been taken. Contracts. Salary lists. Film
budgets. Medical records. Social Security numbers. Personal emails. Five
entire movies, including the yet-to-be-released ``Annie.''

Later, it would become apparent through files stolen by the hackers and
published online that Mr. Lynton and Ms. Pascal had
\href{http://www.wired.com/2014/12/sony-hack-part-deux/}{been given an
oblique warning}. On Nov. 21, in an email signed by ``God's Apstls,''
the studio was told to pay money for an unspecified reason by Nov. 24.
If the studio did not comply, the bizarre missive said, ``Sony Pictures
will be bombarded as a whole.''

But the warning either did not find its way to Mr. Lynton or he missed
its importance in the daily flood of messages to his inbox. In the first
days of the attack, responsibility for which was claimed by a group
calling itself ``Guardians of Peace,'' the notion of North Korean
involvement was little more than a paranoid whisper.

In June, a spokesman for North Korea's Ministry of Foreign Affairs said
in a statement said the country would take ``a decisive and merciless
countermeasure'' if the United States government permitted Sony to make
its planned Christmas release of the comedy ``The Interview.''

At the time, the threat seemed to many almost as absurd as the film,
which was not mentioned in early communications from the hackers.

In the gossipy nexus that quickly connected Hollywood's trade news media
with studio insiders and a growing circuit of information technology
experts, talk circulated of a ``mole'' --- a Sony employee who was
presumed by many to have been instrumental in penetrating the computer
systems and spotting the most sensitive data.

The theory of violation by an ex-employee or disgruntled insider
persists among computer security experts who remain unpersuaded by
\href{http://www.politico.com/story/2014/12/fbi-briefed-on-alternate-sony-hack-theory-113866.html}{the
F.B.I.'s focus} on evidence pointing toward North Korea, which the
agency made public
\href{http://www.nytimes.com/aponline/2014/12/19/arts/ap-us-sony-hack.html}{in
a news release} on Dec. 19.

Image

Amy Pascal, co-chairwoman of Sony, offered apologies and outrage as
executive emails were dumped online.Credit...Kevork Djansezian/Reuters

But senior Sony executives, speaking on the condition of anonymity
because the investigation is incomplete, now say the talk of a rogue
insider reflects a misunderstanding of the F.B.I.'s initial conclusions
about the hacking. Federal investigators, they said, did not strongly
suspect an inside job.

Rather, these executives said, the F.B.I. found that the hackers had
used digital techniques to steal the credentials and passwords from a
systems administrator who had maximum access to Sony's computer systems.
Once in control of the gateways those items opened, theft of information
was relatively easy.

Government investigators and Sony's private security experts traced the
hacking through a network of foreign servers and identified malicious
software bearing the familiar hallmarks of a hacking gang known as Dark
Seoul. Prodded for inside information at a social gathering --- long
before the F.B.I. announced any conclusions --- Doug Belgrad, president
of Sony's motion picture group, responded, ``It's the Koreans.''

\hypertarget{hackers-release-information}{%
\subsection{Hackers Release
Information}\label{hackers-release-information}}

As the F.B.I. stepped up its inquiry, the hackers --- who still had made
no explicit mention of ``The Interview'' --- dropped the first in a
series of data bundles that were to prove a feast for websites like
Gawker and mainstream services like Bloomberg News for weeks.

And so was set a pattern. Every few days, hackers would dump a vast new
group of documents onto anonymous posting sites. Reporters and other
parties who had shown an interest in searching the Sony files were then
sent email alerts --- essentially digital treasure maps from the
hackers.

The files seemed to fulfill every Hollywood gossip's fantasy of what is
said behind studio walls. Ms. Pascal was caught swapping racially
insensitive jokes about President Obama's presumed taste in
African-American films. A top Sony producer, Scott Rudin, was discovered
harshly criticizing Angelina Jolie. Mr. Lynton was revealed to be
angling for a job at New York University.

Sony technicians privately started fighting back by
\href{http://recode.net/2014/12/10/sony-pictures-tries-to-disrupt-downloads-of-its-stolen-files/}{moving
to disrupt} access to the data dumps. But the studio --- apart from
public apologies by Ms. Pascal --- was largely silent on the
disclosures.

In this, Mr. Lynton was perhaps betrayed by his own cool. While Ms.
Pascal alternately wept and raged about the violation, Mr. Lynton
assumed the more detached manner that had served him well in the
publishing world. Mr. Lynton engaged in debates with lawyers who
rendered conflicting opinions as to whether media outlets could in fact
be stopped from trading in goods that were, after all, stolen.

As a tough and seasoned executive in her own right, Ms. Pascal brought
badly needed expression to emotions that many, perhaps most, Sony
employees were feeling. Hoarse and humbled, she would eventually bring
colleagues to her side with an address at an all-hands gathering on the
Sony lot in which she said: ``I'm so terribly sorry. All I can really do
now is apologize and ask for your forgiveness.''

Until shortly before that, Mr. Lynton was hesitant about confronting
media outlets with legal action. But the lawyer David Boies persuaded
him there was a case to be made against free trade in information that
was essentially stolen property. Mr. Boies on Dec. 14 began sending
legal warnings to about 40 media outlets using the stolen data.

On Dec. 15, while rallying the troops at that gathering on the Sony lot,
Mr. Lynton displayed flashes of anger and words of resolve --- fighting
spirit he had not shown publicly. ``Some of the reporting on this
situation has been truly outrageous, and is, quite frankly,
disgusting,'' he said. He urged employees not to read the anticipated
next waves of emails, lest they turn on one another.

``I'm concerned, very concerned, that if people continue to read these
emails, relationships will be damaged and hurt here at the studio,'' he
said.

\hypertarget{a-crucial-threat}{%
\subsection{A Crucial Threat}\label{a-crucial-threat}}

Shortly before 10 a.m. the next day, Dec. 16, the hackers made good on
their promise of a ``Christmas gift,'' delivering thousands of Mr.
Lynton's emails to the posting sites. With the emails came a message
that within minutes converted the hacking from corporate annoyance to
national threat and fully jolted Sony from defense to offense.

``Soon all the world will see what an awful movie Sony Pictures
Entertainment has made,'' it said. ``The world will be full of fear.
Remember the 11th of September 2001.'' The message specifically cited
``The Interview'' and its planned opening.

Unfazed until then by Sony's problems, exhibitors were instantly
galvanized. ``When you invoke 9/11, it's a game changer,'' said one
theater executive.

Within hours, the National Association of Theater Owners convened a
board meeting. Through the day, the exhibitors were briefed by Sony
executives (though not by Mr. Lynton), who took a position that
infuriated some owners: The studio would not cancel the film, but it
would not quarrel with any theater that withdrew it because of security
concerns.

``Sony basically punted,'' said one theater executive, speaking on the
condition of anonymity because of confidentiality strictures.
``Frankly,'' the executive added, ``it's their movie, and their mess.''

Carmike Cinemas, one of the country's four largest chains, was the first
to withdraw. By the morning of Dec. 17, owners of about 80 percent of
the country's movie theaters --- including Regal Entertainment, AMC
Entertainment, and Cinemark, already mired in legal fights over a 2012
theater shooting in Colorado --- had pulled out.

At the same time, Mr. Lynton was advised by George Rose, who is in
charge of human resources, that employees, for the first time since the
initial attack, were showing signs of being deeply shaken by the
possibility of violence to themselves and to the audience.

That afternoon, Sony dropped ``The Interview'' from its schedule. In
theory, the studio had gotten its way by putting the onus for
cancellation on apprehensive theater owners.

Image

**"**The Interview" opened online and in 331 theaters, like this one in
Atlanta, on Christmas Day after its withdrawal by Sony drew wide
criticism.Credit...Marcus Ingram/Getty Images

But Sony at that moment made a critical error. In a hasty statement, in
some cases delivered orally to reporters, the studio said it had ``no
further release plan'' for ``The Interview.'' In fact, Mr. Lynton had
been talking with Google's chairman, Eric E. Schmidt, and others about
an alternative online release --- discussions that Google would later
confirm publicly. But Sony's statement was widely interpreted to mean
Sony would shelve the movie for good, leaving an impression that it had
caved to the hackers and a terrorist threat.

The reaction was swift and furious. Hollywood stars and free speech
advocates sharply criticized the decision. On Friday, Dec. 19, President
Obama used his final news briefing of the year to rebuke Sony for its
handling of the North Korean threat: ``We cannot have a dictator
imposing censorship in the U.S.'' For Mr. Lynton, the president's
remarks became a personal low point in the entire affair. He had
expected support from Mr. Obama --- of whom Mr. Lynton and his wife,
Jamie, were early and ardent backers in 2007. ``I would be fibbing to
say I wasn't disappointed,'' Mr. Lynton told a CNN interviewer shortly
afterward, understating his reaction. (Mr. Lynton had already agreed to
the CNN interview and, in fact, watched the president's news conference
from a TV in a CNN lounge.)

``You know, the president and I haven't spoken,'' Mr. Lynton added. ``I
don't know exactly whether he understands the sequence of events that
led up to the movies' not being shown in the movie theaters.''

The president's decision to specifically --- and harshly --- criticize
Sony was not mapped before the news conference, according to two senior
American officials. But it was clear to Mr. Obama's aides and national
security staff that the president felt passionately about the issue and
was eager to push for the film's release, the officials said.

Shortly after the president spoke, shocked Sony executives spoke with
senior members of the White House staff, asking whether they had known
that the president was going to criticize them. The staff members told
the executives that nothing had been planned.

In the end, the exchanges were constructive, as administration officials
persuaded Sony that an expanded electronic attack was unlikely; that
gave the studio cover to tell the distributors and theaters they were
very likely safe to show the film. But Mr. Obama played no direct role
in pushing deals that, in less than a week, would put ``The Interview''
online and in 331 smaller theaters.

Sony's Christmas Eve triumph in announcing an immediate online release
of ``The Interview'' was more fragile than it looked. While Google had
been committed for a week, Microsoft and its Xbox service came aboard
only late the night before.

In the end, the film may be seen by more viewers than if it had
experienced an unimpeded, conventional release, particularly if, as
studio executives suspect, those who paid for the film online were
joined by friends and family. Sony said ``The Interview'' generated
roughly \$15 million in online sales and rentals during its first four
days of availability.

Now, five weeks into the episode, Sony's internal technology is still
impaired. Executives estimate that a return to normal is at least five
to seven weeks away.

But the studio's spirit apparently remains intact. Showing up in the
Sony cafeteria for lunch last week, as a theatrical release and the
Google and Microsoft deals were announced, Mr. Lynton was surrounded by
30 to 40 employees who told him they were proud to be at Sony and to get
the movie out.

``If we put our heads down and focus on our work, I honestly think we
can recover from this in short order,'' Mr. Lynton said on Sunday.

Advertisement

\protect\hyperlink{after-bottom}{Continue reading the main story}

\hypertarget{site-index}{%
\subsection{Site Index}\label{site-index}}

\hypertarget{site-information-navigation}{%
\subsection{Site Information
Navigation}\label{site-information-navigation}}

\begin{itemize}
\tightlist
\item
  \href{https://help.nytimes.com/hc/en-us/articles/115014792127-Copyright-notice}{©~2020~The
  New York Times Company}
\end{itemize}

\begin{itemize}
\tightlist
\item
  \href{https://www.nytco.com/}{NYTCo}
\item
  \href{https://help.nytimes.com/hc/en-us/articles/115015385887-Contact-Us}{Contact
  Us}
\item
  \href{https://www.nytco.com/careers/}{Work with us}
\item
  \href{https://nytmediakit.com/}{Advertise}
\item
  \href{http://www.tbrandstudio.com/}{T Brand Studio}
\item
  \href{https://www.nytimes.com/privacy/cookie-policy\#how-do-i-manage-trackers}{Your
  Ad Choices}
\item
  \href{https://www.nytimes.com/privacy}{Privacy}
\item
  \href{https://help.nytimes.com/hc/en-us/articles/115014893428-Terms-of-service}{Terms
  of Service}
\item
  \href{https://help.nytimes.com/hc/en-us/articles/115014893968-Terms-of-sale}{Terms
  of Sale}
\item
  \href{https://spiderbites.nytimes.com}{Site Map}
\item
  \href{https://help.nytimes.com/hc/en-us}{Help}
\item
  \href{https://www.nytimes.com/subscription?campaignId=37WXW}{Subscriptions}
\end{itemize}
