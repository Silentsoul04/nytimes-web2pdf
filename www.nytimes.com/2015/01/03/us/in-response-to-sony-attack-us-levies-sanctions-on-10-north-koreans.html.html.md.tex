Sections

SEARCH

\protect\hyperlink{site-content}{Skip to
content}\protect\hyperlink{site-index}{Skip to site index}

\href{https://www.nytimes.com/section/politics}{Politics}

\href{https://myaccount.nytimes.com/auth/login?response_type=cookie\&client_id=vi}{}

\href{https://www.nytimes.com/section/todayspaper}{Today's Paper}

\href{/section/politics}{Politics}\textbar{}More Sanctions on North
Korea After Sony Case

\url{https://nyti.ms/1zJnnfV}

\begin{itemize}
\item
\item
\item
\item
\item
\item
\end{itemize}

Advertisement

\protect\hyperlink{after-top}{Continue reading the main story}

Supported by

\protect\hyperlink{after-sponsor}{Continue reading the main story}

\hypertarget{more-sanctions-on-north-korea-after-sony-case}{%
\section{More Sanctions on North Korea After Sony
Case}\label{more-sanctions-on-north-korea-after-sony-case}}

By \href{http://www.nytimes.com/by/david-e-sanger}{David E. Sanger} and
\href{http://www.nytimes.com/by/michael-s-schmidt}{Michael S. Schmidt}

\begin{itemize}
\item
  Jan. 2, 2015
\item
  \begin{itemize}
  \item
  \item
  \item
  \item
  \item
  \item
  \end{itemize}
\end{itemize}

The Obama administration doubled down on Friday on its allegation that
\href{http://topics.nytimes.com/top/news/international/countriesandterritories/northkorea/index.html?inline=nyt-geo}{North
Korea}'s leadership was behind the hacking of Sony Pictures, announcing
new, if largely symbolic, economic sanctions against 10 senior North
Korean officials and the intelligence agency it said was the source of
``many of North Korea's major cyberoperations.''

The actions were based on an executive order President Obama signed on
vacation in Hawaii, as part of what he had promised would be a
``proportional response'' against the country. But in briefings for
reporters, officials said they could not establish that any of the 10
officials had been directly involved in the destruction of much of the
studio's computing infrastructure.

In fact, most seemed linked to the North's missile and weapons sales.
Two are senior North Korean representatives in Iran, a major buyer of
North Korean military technology, and five others are representatives in
Syria, Russia, China and Namibia.

The sanctions were a public part of the response to the cyberattack on
Sony, which was targeted as it prepared to release ``The Interview,'' a
crude comedy about a C.I.A. plot to kill Kim Jong-un, North Korea's
leader.

The administration has said there would be a covert element of its
response as well. Officials sidestepped questions about whether the
United States was involved in
\href{https://www.nytimes.com/2014/12/28/world/asia/north-korea-sony-hacking-the-interview.html}{bringing
down North Korea's Internet connectivity} to the outside world over the
past two weeks.

Perhaps the most noticeable element of the announcement was the
administration's effort to push back on the growing chorus of doubters
about the evidence that the attack on Sony was North Korean in origin.
Several cybersecurity firms have argued that when Mr. Obama took the
unusual step of
\href{https://www.nytimes.com/2014/12/20/world/fbi-accuses-north-korean-government-in-cyberattack-on-sony-pictures.html}{naming
the North's leadership} --- on Dec. 19 the president declared that
``North Korea engaged in this attack'' --- he had been misled by
American intelligence agencies that were too eager to blame a longtime
adversary and allowed themselves to be duped by ingenious hackers
skilled at hiding their tracks.

But Mr. Obama's critics do not have a consistent explanation of who
might have been culpable. Some blame corporate insiders or an angry
former employee, a theory Sony Pictures' top executive, Michael Lynton,
has denied. Others say it was the work of outside hacking groups that
were simply using the release of ``The Interview'' as cover for their
actions.

Both the F.B.I. and Mr. Obama's aides used the sanctions announcement to
argue that the critics of the administration's decision to attribute the
attack to North Korea have no access to the classified evidence that led
the intelligence agencies, and Mr. Obama, to their conclusion.

``We remain very confident in the attribution,'' a senior administration
official who has been at the center of the Sony case told reporters in a
briefing that, under guidelines set by the White House, barred the use
of the briefer's name.

Still, the administration is clearly stung by the comparisons to the
George W. Bush administration's reliance on faulty intelligence
assessments about Iraq's weapons of mass destruction before the 2003
American-led invasion of the country. They note how rare it is for Mr.
Obama, usually cautious on intelligence issues, to blame a specific
country so directly. But they continue to insist that they cannot
explain the basis of the president's declaration without revealing some
of the most sensitive sources and technologies at their disposal.

By naming 10 individuals at the center of the North's effort to sell or
obtain weapons technology, the administration seemed to be trying to
echo sanctions that the Bush administration imposed eight years ago
against a Macao bank that the North Korean leadership used to buy goods
illicitly and to reward loyalists. President Bush, speaking to reporters
one evening in the White House, argued that those sanctions were the
only ones that got the attention of Kim Jong-il, whose son has ruled the
country since his death in 2011.

In another sign of how Mr. Obama was seeking to punish individual
leaders, the executive order he signed gives the Treasury Department
broad authority to name anyone in the country's leadership believed to
be involved in illicit activity, and to take action against the Workers'
Party, which has complete control of North Korea's politics.

In a statement, Treasury Secretary Jacob J. Lew suggested that the
sanctions were intended not only to punish North Korea for the hacking
of Sony --- which resulted in the destruction of about three-quarters of
the computers and servers at the studio's main operations --- but also
to warn the country not to try anything like it again.

``Today's actions are driven by our commitment to hold North Korea
accountable for its destructive and destabilizing conduct,'' Mr. Lew
said. ``Even as the F.B.I. continues its investigation into the
cyberattack against Sony Pictures Entertainment, these steps underscore
that we will employ a broad set of tools to defend U.S. businesses and
citizens, and to respond to attempts to undermine our values or threaten
the national security of the United States.''

Beyond the initial sanctions, the power of the president's order might
come from its breadth and its use in the future. One senior official
said the order would allow the Treasury to impose sanctions on any
person who is an official of the North Korean government or of the
Worker's Party or anyone judged ``controlled by the North Korean
government'' or acting on its behalf.

Yet it is easy to overestimate the impact of sanctions. Six decades of
efforts to isolate North Korea have not stopped it from building and
testing a nuclear arsenal, launching terrorist attacks on the South,
testing missiles or maintaining large prison camps.

In addition, the Reconnaissance General Bureau, the country's main
intelligence organization, has long been under heavy sanctions for
directing the country's arms trade, including the Proliferation Security
Initiative, an effort started by the Bush administration to intercept
the sales of missiles and other arms.

Still, the Treasury's statement on Friday that ``many of North Korea's
major cyberoperations run through R.G.B.'' was more than has been said
publicly by the United States about how the North Koreans structure
their cyberoperations. And administration officials insisted again that
the Sony attack ``clearly crossed a threshold,'' in the words of one
senior official, from ``website defacement and digital graffiti'' to an
attack on computer infrastructure.

Advertisement

\protect\hyperlink{after-bottom}{Continue reading the main story}

\hypertarget{site-index}{%
\subsection{Site Index}\label{site-index}}

\hypertarget{site-information-navigation}{%
\subsection{Site Information
Navigation}\label{site-information-navigation}}

\begin{itemize}
\tightlist
\item
  \href{https://help.nytimes.com/hc/en-us/articles/115014792127-Copyright-notice}{©~2020~The
  New York Times Company}
\end{itemize}

\begin{itemize}
\tightlist
\item
  \href{https://www.nytco.com/}{NYTCo}
\item
  \href{https://help.nytimes.com/hc/en-us/articles/115015385887-Contact-Us}{Contact
  Us}
\item
  \href{https://www.nytco.com/careers/}{Work with us}
\item
  \href{https://nytmediakit.com/}{Advertise}
\item
  \href{http://www.tbrandstudio.com/}{T Brand Studio}
\item
  \href{https://www.nytimes.com/privacy/cookie-policy\#how-do-i-manage-trackers}{Your
  Ad Choices}
\item
  \href{https://www.nytimes.com/privacy}{Privacy}
\item
  \href{https://help.nytimes.com/hc/en-us/articles/115014893428-Terms-of-service}{Terms
  of Service}
\item
  \href{https://help.nytimes.com/hc/en-us/articles/115014893968-Terms-of-sale}{Terms
  of Sale}
\item
  \href{https://spiderbites.nytimes.com}{Site Map}
\item
  \href{https://help.nytimes.com/hc/en-us}{Help}
\item
  \href{https://www.nytimes.com/subscription?campaignId=37WXW}{Subscriptions}
\end{itemize}
