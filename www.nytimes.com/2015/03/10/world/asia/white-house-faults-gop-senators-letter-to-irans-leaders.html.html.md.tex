Sections

SEARCH

\protect\hyperlink{site-content}{Skip to
content}\protect\hyperlink{site-index}{Skip to site index}

\href{https://www.nytimes.com/section/world/middleeast}{Middle East}

\href{https://myaccount.nytimes.com/auth/login?response_type=cookie\&client_id=vi}{}

\href{https://www.nytimes.com/section/todayspaper}{Today's Paper}

\href{/section/world/middleeast}{Middle East}\textbar{}G.O.P. Senators'
Letter to Iran About Nuclear Deal Angers White House

\url{https://nyti.ms/1HowAQw}

\begin{itemize}
\item
\item
\item
\item
\item
\item
\end{itemize}

Advertisement

\protect\hyperlink{after-top}{Continue reading the main story}

Supported by

\protect\hyperlink{after-sponsor}{Continue reading the main story}

\hypertarget{gop-senators-letter-to-iran-about-nuclear-deal-angers-white-house}{%
\section{G.O.P. Senators' Letter to Iran About Nuclear Deal Angers White
House}\label{gop-senators-letter-to-iran-about-nuclear-deal-angers-white-house}}

By \href{http://www.nytimes.com/by/peter-baker}{Peter Baker}

\begin{itemize}
\item
  March 9, 2015
\item
  \begin{itemize}
  \item
  \item
  \item
  \item
  \item
  \item
  \end{itemize}
\end{itemize}

WASHINGTON --- The fractious debate over a possible nuclear deal with
Iran escalated on Monday as 47 Republican senators warned Iran about
making an agreement with President Obama, and the White House accused
them of undercutting foreign policy.

In a rare direct congressional intervention into diplomatic
negotiations, the Republicans signed an
\href{http://www.nytimes.com/interactive/2015/03/09/world/middleeast/document-the-letter-senate-republicans-addressed-to-the-leaders-of-iran.html?_r=1}{open
letter} addressed to ``leaders of the Islamic Republic of Iran''
declaring that any agreement without legislative approval could be
reversed by the next president ``with the stroke of a pen.''

The letter appeared aimed at unraveling a framework agreement even as
negotiators grew close to reaching it. Mr. Obama, working with leaders
of five other world powers, argues that the pact would be the best way
to keep Iran from obtaining a nuclear bomb. But critics from both
parties say that such a deal would be a dangerous charade that would
leave Iran with the opportunity to eventually build weapons that could
be used against Israel or other foes.

While the possible agreement has drawn bipartisan criticism, the letter,
signed only by Republicans, underscored the increasingly party-line
flavor of the clash. Just last week, the Republican House speaker, John
A. Boehner, gave Prime Minister Benjamin Netanyahu of Israel the
platform of a joint meeting of Congress to denounce the developing deal,
and Senate Republicans briefly tried to advance legislation aimed at
forcing Mr. Obama to submit it to Congress, alienating Democratic
allies.

The letter came as Secretary of State John Kerry's office announced that
he would return to Switzerland on Sunday in hopes of completing the
framework agreement before an end-of-March deadline. Under the terms
being discussed, Iran would pare back its nuclear program enough so that
it would be unable to produce enough fuel for a bomb in less than a year
if it tried to break out of the agreement. The pact would last at least
10 years; in exchange the world powers would lift sanctions.

\href{https://www.nytimes.com/interactive/2015/03/09/world/middleeast/document-the-letter-senate-republicans-addressed-to-the-leaders-of-iran.html}{}

\includegraphics{https://static01.nyt.com/images/2015/03/09/world/middleeast/document-the-letter-senate-republicans-addressed-to-the-leaders-of-iran-1425939590078/document-the-letter-senate-republicans-addressed-to-the-leaders-of-iran-1425939590078-articleLarge.jpg}

\hypertarget{letter-from-senate-republicans-to-the-leaders-of-iran}{%
\subsection{Letter From Senate Republicans to the Leaders of
Iran}\label{letter-from-senate-republicans-to-the-leaders-of-iran}}

The letter warned that any nuclear agreement could be reversed by the
next president ``with the stroke of a pen.''

Whether the Republican letter might undercut Iran's willingness to
strike a deal was not clear. Iran reacted with scorn. ``In our view,
this letter has no legal value and is mostly a propaganda ploy,''
Mohammad Javad Zarif, Iran's foreign minister, said in a statement. ``It
is very interesting that while negotiations are still in progress and
while no agreement has been reached, some political pressure groups are
so afraid even of the prospect of an agreement that they resort to
unconventional methods, unprecedented in diplomatic history.''

A senior American official said the letter probably would not stop an
agreement from being reached, but could make it harder to blame Iran if
the talks fail. ``The problem is if there is not an agreement, the
perception of who is at fault is critically important to our ability to
maintain pressure, and this type of thing would likely be used by the
Iranians in that scenario,'' said the official, who spoke anonymously to
discuss the negotiations.

The White House and congressional Democrats expressed outrage, calling
the letter an unprecedented violation of the tradition of leaving
politics at the water's edge. Republicans said that by styling it as an
``open letter,'' it was akin to a statement, not an overt intervention
in the talks.

``It's somewhat ironic to see some members of Congress wanting to make
common cause with the hard-liners in Iran,'' Mr. Obama told reporters.
``It's an unusual coalition.''

Other Democrats were sharper. Josh Earnest, the White House press
secretary, called it ``just the latest in an ongoing strategy, a
partisan strategy, to undermine the president's ability to conduct
foreign policy.'' Senator Harry M. Reid of Nevada, the Democratic
minority leader, said the ``Republicans are undermining our commander in
chief while empowering the ayatollahs.''

The letter, drafted by Senator Tom Cotton, a freshman from Arkansas, and
signed by all but seven members of the Senate Republican majority,
warned Iran that a deal with Mr. Obama might not stick. ``The next
president could revoke such an executive agreement with the stroke of a
pen, and future Congresses could modify the terms of the agreement at
any time,'' said the letter, whose existence was reported earlier by
Bloomberg News.

\href{https://www.nytimes.com/interactive/2015/03/02/world/middleeast/2015-03-02-iran.html}{}

\includegraphics{https://static01.nyt.com/images/2015/03/02/world/middleeast/2015-03-02-iran-1425322863633/2015-03-02-iran-1425322863633-articleLarge-v3.jpg}

\hypertarget{the-nuclear-talks-with-iran-explained}{%
\subsection{The Nuclear Talks With Iran,
Explained}\label{the-nuclear-talks-with-iran-explained}}

What the United States and Iran want out of discussions over Iran's
nuclear development.

Mr. Cotton said he drafted the letter because Iran's leaders might not
understand America's constitutional system. He also said the terms of
the emerging deal were dangerous because they would not be permanent and
would leave Iran with nuclear infrastructure. He noted that four
Republican senators who may run for president signed his letter and
added that he tried without success to get Democrats to sign.

``The only thing unprecedented is an American president negotiating a
nuclear deal with the world's leading state sponsor of terrorism without
submitting it to Congress,'' he said on CNN.

The letter revived an old debate about what role Congress should have in
diplomacy.

Jim Wright, the Democratic House speaker during Ronald Reagan's
presidency, was accused of interfering when he met with opposing leaders
in Nicaragua's contra war. Three House Democrats went to Iraq in 2002
before President George W. Bush's invasion to try to head off war. And
Nancy Pelosi, the House Democratic leader, went to Syria in 2007 to meet
with President Bashar al-Assad against the wishes of the Bush
administration, which was trying to isolate him.

An agreement with Iran would not require immediate congressional action
because Mr. Obama has the power to lift sanctions he imposed under his
executive authority and to suspend others imposed by Congress. But
permanently lifting those imposed by Congress, as Iran has sought, would
eventually require a vote.

Rather than wait, Republicans, joined by several Democrats, drafted
legislation aimed at forcing Mr. Obama to submit the agreement to
Congress. But when Senator Mitch McConnell of Kentucky, the Republican
majority leader, moved to advance that legislation for a vote, Democrats
who support it balked at taking action before the talks with Iran
concluded. Mr. McConnell backed off, but the bill may be revived if a
deal is reached.

Among the Republicans who declined to sign Mr. Cotton's letter was
Senator Bob Corker of Tennessee, the Foreign Relations Committee
chairman, who has been working with Democrats on Iran legislation.
``We've got a bipartisan effort that's underway that has a chance of
being successful, and while I understand all kinds of people want to
weigh in,'' he said, he concluded that it would not ``be helpful in that
effort for me to be involved in it.''

Some Democrats, like Representative Brad Sherman of California, said the
letter and other moves risked making it a party-line issue, in which
case it would be impossible to muster a two-thirds vote to override a
presidential veto. ``The number of Democrats not willing to follow the
president's lead is reduced when it becomes a personal or political
issue,'' he said.

Advertisement

\protect\hyperlink{after-bottom}{Continue reading the main story}

\hypertarget{site-index}{%
\subsection{Site Index}\label{site-index}}

\hypertarget{site-information-navigation}{%
\subsection{Site Information
Navigation}\label{site-information-navigation}}

\begin{itemize}
\tightlist
\item
  \href{https://help.nytimes.com/hc/en-us/articles/115014792127-Copyright-notice}{©~2020~The
  New York Times Company}
\end{itemize}

\begin{itemize}
\tightlist
\item
  \href{https://www.nytco.com/}{NYTCo}
\item
  \href{https://help.nytimes.com/hc/en-us/articles/115015385887-Contact-Us}{Contact
  Us}
\item
  \href{https://www.nytco.com/careers/}{Work with us}
\item
  \href{https://nytmediakit.com/}{Advertise}
\item
  \href{http://www.tbrandstudio.com/}{T Brand Studio}
\item
  \href{https://www.nytimes.com/privacy/cookie-policy\#how-do-i-manage-trackers}{Your
  Ad Choices}
\item
  \href{https://www.nytimes.com/privacy}{Privacy}
\item
  \href{https://help.nytimes.com/hc/en-us/articles/115014893428-Terms-of-service}{Terms
  of Service}
\item
  \href{https://help.nytimes.com/hc/en-us/articles/115014893968-Terms-of-sale}{Terms
  of Sale}
\item
  \href{https://spiderbites.nytimes.com}{Site Map}
\item
  \href{https://help.nytimes.com/hc/en-us}{Help}
\item
  \href{https://www.nytimes.com/subscription?campaignId=37WXW}{Subscriptions}
\end{itemize}
