Sections

SEARCH

\protect\hyperlink{site-content}{Skip to
content}\protect\hyperlink{site-index}{Skip to site index}

\href{https://www.nytimes.com/section/food}{Food}

\href{https://myaccount.nytimes.com/auth/login?response_type=cookie\&client_id=vi}{}

\href{https://www.nytimes.com/section/todayspaper}{Today's Paper}

\href{/section/food}{Food}\textbar{}Marco Pierre White's `White Heat': A
Game Changer, Revisited

\url{https://nyti.ms/1ELMy9m}

\begin{itemize}
\item
\item
\item
\item
\item
\item
\end{itemize}

Advertisement

\protect\hyperlink{after-top}{Continue reading the main story}

Supported by

\protect\hyperlink{after-sponsor}{Continue reading the main story}

\hypertarget{marco-pierre-whites-white-heat-a-game-changer-revisited}{%
\section{Marco Pierre White's `White Heat': A Game Changer,
Revisited}\label{marco-pierre-whites-white-heat-a-game-changer-revisited}}

Slide 1 of 7

1/7

In Marco Pierre White, young chefs saw themselves --- or at least they
saw the person they hoped to be. He looked like an artist, not like a
hospitality director.

Credit...Bob Carlos Clarke

\begin{itemize}
\item
  \includegraphics{https://static01.nyt.com/images/2015/04/08/dining/20150408WHITE-slide-91PP/20150408WHITE-slide-91PP-superJumbo.jpg}
\item
  \includegraphics{https://static01.nyt.com/images/2015/04/08/dining/20150408WHITE-slide-QWKX/20150408WHITE-slide-QWKX-superJumbo.jpg}
\item
  \includegraphics{https://static01.nyt.com/images/2015/04/08/dining/20150408WHITE-slide-L0ZB/20150408WHITE-slide-L0ZB-superJumbo.jpg}
\item
  \includegraphics{https://static01.nyt.com/images/2015/04/08/dining/20150408WHITE-slide-LTYB/20150408WHITE-slide-LTYB-superJumbo.jpg}
\item
  \includegraphics{https://static01.nyt.com/images/2015/04/08/dining/20150408WHITE-slide-V4EU/20150408WHITE-slide-V4EU-superJumbo.jpg}
\item
  \includegraphics{https://static01.nyt.com/images/2015/04/08/dining/20150408WHITE-slide-N9N7/20150408WHITE-slide-N9N7-superJumbo.jpg}
\item
  \includegraphics{https://static01.nyt.com/images/2015/04/08/dining/20150408WHITE-slide-QMDC/20150408WHITE-slide-QMDC-superJumbo.jpg}
\end{itemize}

By \href{http://www.nytimes.com/by/dwight-garner}{Dwight Garner}

\begin{itemize}
\item
  April 2, 2015
\item
  \begin{itemize}
  \item
  \item
  \item
  \item
  \item
  \item
  \end{itemize}
\end{itemize}

In his heyday in the late 1980s and early 1990s, when he was the
tantrum-throwing enfant terrible of the London food world, Marco Pierre
White had a method for dealing with rude or obnoxious customers in his
restaurants. It was called the Whoosh.

Here's how it worked. A phalanx of waiters would swoop in on the
offending table and clear away everything, including half-filled wine
bottles, in a moment or two. At this point, the bemused diners might
still be expecting fresh silver, port and chocolate cake. Instead they
got Mr. White himself. He'd stride into the dining room and seize their
tablecloth (whoosh!) with a bullfighter's flourish. The humiliated
guests didn't have to pay, but their evening was over.

These details are related in ``White Heat 25,'' a new and updated
edition of Mr. White's vastly influential first cookbook, ``White
Heat,'' first published in 1990. It is relatively little known in the
United States, among civilians at any rate. But prominent chefs in the
waves that followed, including Mario Batali and David Chang, considered
it to be perhaps the most important cookbook of the modern food era.

``White Heat'' changed the rules of the game. It altered how chefs saw
themselves. Its republication now is a chance to gaze back at the last
25 years in the restaurant world, all that chaos and preening and
tattooed magic, and to observe a world that, for better and occasionally
worse, Mr. White had spawned.

It wasn't Mr. White's recipes that won a generation over, not that he
wasn't more than adept in the kitchen. He was, in the 1990s, the
youngest chef ever to have three Michelin stars, at Restaurant Marco
Pierre White. His food was French and often fussy (his feuilleté of
roast rabbit sounds hard to make but easy to eat), but it was cooked
with a surgeon's care. His carefully composed plates also had a visual
impact that seemed new in England. Everything appeared to come with a
side order of pheromones.

It wasn't Mr. White's antics, related most fully in ``White Heat 25,''
that caught the eyes of aspiring chefs, either, though he was funny in a
mean way. If an assistant chef erred at a big moment, Mr. White might
toss him into a garbage bin, hang him by his apron from hooks on the
wall or make him go stand in the corner for a while. If these sound like
crude mortifications Gordon Ramsay might inflict on underlings to jack
up his reality television ratings, well, Mr. Ramsay trained under Mr.
White.

It was the photographs in ``White Heat'' that put the book across. No
cookbook had looked like this one. The book's photographer, Bob Carlos
Clarke, shot Mr. White's kitchen at Harvey's, which opened in 1987, as
if it were a war zone. The black-and-white photos were filled with
blood, with cigarettes dangling from lips and with rattled, unshaven
young men who appeared to be on a mission up the Congo.

At the center was Mr. White himself: thin, 28 when the book was
published, with unruly dark hair, penetrating eyes and veins running
down his forearms that made them resemble hydraulic pork shanks.

Image

Credit...Patricia Wall/The New York Times

Before then, well-known chefs and food writers tended to be plump, jolly
figures, like Russian nesting dolls: James Beard, Julia Child, A. J.
Liebling. The French master Fernand Point wasn't so jolly, but he had a
belly that toddled in front of him like a kettle grill. These men and
women were not especially sexy beasts.

Mr. White, on the other hand, looked as if he had been raised in the
woods. He resembled Jim Morrison, Sweeney Todd and Lord Byron. He
wielded a cleaver the way Bruce Lee wielded nunchucks. He seemed as if
he popped supermodels into his mouth like ortolans. (If the British
tabloids are correct, he more or less did.)

In him, young chefs saw themselves --- or at least they saw the person
they hoped to be. He looked like an artist, not like a hospitality
director. When he opened his mouth, it got even better.

His book began this way: ``You're buying `White Heat' because you want
to cook well? Because you want to cook Michelin stars? Forget it. Save
your money. Go and buy a saucepan.''

He spoke blasphemies other chefs recognized as hard-won truths. ``Any
chef who says he does it for love is a liar,'' Mr. White said. ``At the
end of the day it's all about money. I never thought I would ever think
like that, but I do now. I don't enjoy it. I don't enjoy having to kill
myself six days a week to pay the bank.'' Can you blame him, or any
other chef, for wanting to live like his customers?

He had never been to France, he admitted. This was a big deal. It had
never occurred to serious aspirant chefs that you could become a real
cook without paying your dues in a French three-star. Mr. White's
heresies arrived a decade before those scattered throughout Anthony
Bourdain's ``Kitchen Confidential: Adventures in the Culinary
Underbelly,'' his landmark tell-all published in 2000.

One of the best reasons to pick up ``White Heat 25'' is for the
encomiums from chefs like Mr. Chang, Mr. Ramsay and many others across
America and England. Mr. Bourdain's is especially alive.

``Marco, unlike any chef we'd ever seen, in any cookbook ever, looked
stressed,'' he writes. ``It was carved into his face. Look! He's
smoking, leaning up against the kitchen wall, pulling on that cigarette
as if he's trying to suck that whole thing down in one go. We knew that
feeling. We knew how that cigarette tasted. We were grateful to finally
see a chef who admitted to stress and exhaustion like us.''

``White Heat 25'' arrives essentially unchanged. It's the original book
with new material tacked on at the end. Reading it, you recall how
forceful Mr. White was, also how annoying. Every once in a while, you
want to pull the Whoosh and eject him from his own book.

\includegraphics{https://static01.nyt.com/images/2015/04/08/dining/08WHITE1/08WHITE1-articleLarge.jpg?quality=75\&auto=webp\&disable=upscale}

He spoke of himself in the third person. (``You want ideas, inspiration,
a bit of Marco?'' --- his spin on Robert De Niro's ``You talkin' to
me?'') In his kitchen, men were men and women were beside the point, at
least during work hours. He wrote things like, ``I like women because
they aren't competitive, because their sensuality is untroubled.''

About a dish called Red Mullet With Citrus Fruits, he declared: ``It's a
dish for a lady; clean, not robust. I don't think women should eat
robust dishes. Women are much cleaner creatures than men, so they need a
cleaner diet.''

The best place to learn about Mr. White is probably in his 2007 memoir,
\href{http://www.nytimes.com/2007/05/27/books/review/Kamp-t.html?pagewanted=1}{``The
Devil in the Kitchen,''} published in Britain under the title ``White
Slave.'' He was born in a council house in Leeds in 1961. His brothers
had more conventionally British names: Graham, Clive and Craig. His
father was a chef. His mother, who was Italian, died young while in
childbirth.

But ``The Devil in the Kitchen'' will take you only so close. Mr. White
has admitted it was written basically by his co-writer, James Steen.
Thus, like so many things in his post-``White Heat'' career, it was a
bit of a letdown. He looked, in ``White Heat,'' as if he was close to
burning out. He essentially did.

He retired from cooking in 1999, and has been involved in running a
series of sometimes short-lived restaurants since. He's been married
three times. He's appeared on several reality TV shows in England,
though his attempt at an American cooking show, ``The Chopping Block,''
was canceled by NBC after only a handful of episodes. He's had public
falling-outs with friends and business partners.

He's a regular tabloid presence in England, rarely in an upbeat way. An
\href{http://www.telegraph.co.uk/news/uknews/law-and-order/9589784/Marco-Pierre-Whites-wife-scrawled-insults-in-blood-on-wall-of-chefs-home.html}{article
in The Telegraph in 2012} began this way: ``The estranged wife of
celebrity chef Marco Pierre White today admitted smashing up his Range
Rover and daubing insulting graffiti in blood and red paint on his west
London home.''

In
\href{http://www.theguardian.com/lifeandstyle/2007/oct/21/foodanddrink.features9}{The
Guardian in 2007}, the gifted interviewer Lynn Barber wrote that while
Mr. White once ``bestrode the British restaurant industry like a
colossus,'' he is now best known, at least among young people, as ``the
man who trained Gordon Ramsay.'' Ms. Barber wrote, ``Oh Marco, I sobbed,
has it come to this?''

``White Heat 25'' reminds us, in a necessary way, of Mr. White's
achievements. He almost single-handedly made becoming a professional
cook, once an anonymous and lowly profession, something the best and
brightest (and frequently the most malcontent) now aspire to. Clive
James once told me in an interview that the world has lost a lot of
poets to cooking, for both good and ill.

Great movements require charismatic figures. The Beats needed Kerouac,
rock music needed Mick Jagger. The sexy will lead us, at least as long
as they bring the goods. Like Kerouac and Mr. Jagger, Mr. White had the
talent and the ambition to back up his swagger.

Though he had his share of mentors in the kitchen, Mr. White didn't wait
for anyone to anoint him. As Saul Bellow remarked about the best
self-made artists everywhere, he got the oil and anointed himself.

Advertisement

\protect\hyperlink{after-bottom}{Continue reading the main story}

\hypertarget{site-index}{%
\subsection{Site Index}\label{site-index}}

\hypertarget{site-information-navigation}{%
\subsection{Site Information
Navigation}\label{site-information-navigation}}

\begin{itemize}
\tightlist
\item
  \href{https://help.nytimes.com/hc/en-us/articles/115014792127-Copyright-notice}{©~2020~The
  New York Times Company}
\end{itemize}

\begin{itemize}
\tightlist
\item
  \href{https://www.nytco.com/}{NYTCo}
\item
  \href{https://help.nytimes.com/hc/en-us/articles/115015385887-Contact-Us}{Contact
  Us}
\item
  \href{https://www.nytco.com/careers/}{Work with us}
\item
  \href{https://nytmediakit.com/}{Advertise}
\item
  \href{http://www.tbrandstudio.com/}{T Brand Studio}
\item
  \href{https://www.nytimes.com/privacy/cookie-policy\#how-do-i-manage-trackers}{Your
  Ad Choices}
\item
  \href{https://www.nytimes.com/privacy}{Privacy}
\item
  \href{https://help.nytimes.com/hc/en-us/articles/115014893428-Terms-of-service}{Terms
  of Service}
\item
  \href{https://help.nytimes.com/hc/en-us/articles/115014893968-Terms-of-sale}{Terms
  of Sale}
\item
  \href{https://spiderbites.nytimes.com}{Site Map}
\item
  \href{https://help.nytimes.com/hc/en-us}{Help}
\item
  \href{https://www.nytimes.com/subscription?campaignId=37WXW}{Subscriptions}
\end{itemize}
