Sections

SEARCH

\protect\hyperlink{site-content}{Skip to
content}\protect\hyperlink{site-index}{Skip to site index}

\href{https://www.nytimes.com/section/business}{Business}

\href{https://myaccount.nytimes.com/auth/login?response_type=cookie\&client_id=vi}{}

\href{https://www.nytimes.com/section/todayspaper}{Today's Paper}

\href{/section/business}{Business}\textbar{}Accused of Spying for China,
Until She Wasn't

\url{https://nyti.ms/1P7ACUU}

\begin{itemize}
\item
\item
\item
\item
\item
\end{itemize}

Advertisement

\protect\hyperlink{after-top}{Continue reading the main story}

Supported by

\protect\hyperlink{after-sponsor}{Continue reading the main story}

\hypertarget{accused-of-spying-for-china-until-she-wasnt}{%
\section{Accused of Spying for China, Until She
Wasn't}\label{accused-of-spying-for-china-until-she-wasnt}}

\includegraphics{https://static01.nyt.com/images/2015/05/10/business/ESPIONAGE/ESPIONAGE-articleLarge.jpg?quality=75\&auto=webp\&disable=upscale}

By \href{http://www.nytimes.com/by/nicole-perlroth}{Nicole Perlroth}

\begin{itemize}
\item
  May 9, 2015
\item
  \begin{itemize}
  \item
  \item
  \item
  \item
  \item
  \end{itemize}
\end{itemize}

On Monday, Oct. 20, 2014, Sherry Chen drove, as usual, to her office at
the National Weather Service in Wilmington, Ohio, where she forecast
flood threats along the Ohio River. She was a bit jet-lagged, having
returned a few days earlier from a visit to China. But as she headed to
her desk, she says, she had no reason to think it was anything other
than an ordinary day. Then her boss summoned her.

Once inside his office, a back door opened and in walked six agents from
the Federal Bureau of Investigation.

The agents accused Mrs. Chen, a hydrologist born in China and now a
naturalized American citizen, of using a stolen password to download
information about the nation's dams and of lying about meeting with a
high-ranking Chinese official.

Mrs. Chen, 59, an adoptive Midwesterner who had received awards for her
government service, was now suspected of being a Chinese spy. She was
arrested and led in handcuffs past her co-workers to a federal
courthouse 40 miles away in Dayton, where she was told she faced 25
years in prison and \$1 million in fines.

Her life went into a tailspin. She was suspended without pay from her
job, and her family in China had to scramble for money to pay for her
legal defense. Friends and co-workers said they were afraid to visit.
Television news trucks parked outside her house, waiting to spot a
foreign spy hiding in plain sight in suburban Wilmington, population
12,500.

``I could not sleep,'' Mrs. Chen said in a recent interview. ``I could
not eat. I did nothing but cry for days.''

Then, five months later, the ordeal abruptly ended. In March, just a
week before she was scheduled to go on trial, prosecutors dropped all
charges against Mrs. Chen without explanation.

``We are exercising our prosecutorial discretion,'' said Jennifer
Thornton, the spokeswoman for the United States attorney for the
Southern District of Ohio. She added that last year the Justice
Department filed 400 indictments and ``criminal informations'' ---
charges filed in connection with plea agreements --- and dismissed 13 of
them, including Mrs. Chen's. The United States attorney would not
comment on the investigation, but there is little question that law
enforcement is facing new pressure to pursue any lead that could be
related to trade-secret theft.

For the last few years, government officials have noted with growing
alarm that Chinese hackers and paid insiders were spiriting trade
secrets and other confidential information out of the United States. The
mantra, these days, is that there are only two types of companies left
in this country: those that have been hacked by China, and those that do
not know they have been hacked by China.

In 2013, President Obama
\href{http://www.nytimes.com/2013/02/25/world/asia/us-confronts-cyber-cold-war-with-china.html}{announced
a new strategy} to fight back. The cornerstone was more aggressive
investigations and prosecutions, and Justice Department prosecutions
under the Economic Espionage Act jumped more than 30 percent from the
year before. During the first nine months of 2014, the total increased
an additional 33 percent. Notably, more than half of the economic
espionage indictments since 2013 have had a China connection, public
documents show.

It was in this climate that prosecutors zeroed in on Mrs. Chen.

``They came across a person of Chinese descent and a little bit of
evidence that they may have been trying to benefit the Chinese
government, but it's clear there was a little bit of Red Scare and
racism involved,'' said Peter J. Toren, a former federal prosecutor who
specialized in computer crimes and industrial espionage. He is now a
partner at Weisbrod Matteis \& Copley in Washington, and the author of
``Intellectual Property and Computer Crimes.''

Interviews with Mrs. Chen and her former colleagues and a review of
court filings, which include a year's worth of Mrs. Chen's work and
personal emails, suggest that prosecutors hunted for evidence of
espionage, failed and settled on lesser charges that they eventually
dropped.

``The government thought they had struck gold with this case,'' said
Mark D. Rasch, a former Justice Department espionage and computer-crimes
prosecutor who reviewed the case. ``The problem was the facts didn't
quite meet the law here.''

\textbf{A Favor Gone Wrong}

Mrs. Chen, whose given name is Xiafen, was born in Beijing. From an
early age, she was an engineering type. An uncle encouraged her to
pursue a career in building design, but she says she was more interested
in the abstract nature of water and air. ``You can't see them with your
own two eyes,'' she said, growing animated. ``It's so much more complex
than that. I found it fascinating.''

Image

Peter R. Zeidenberg, Ms. Chen's lawyer.Credit...Greg Kahn for The New
York Times

She earned advanced degrees in hydrology in Beijing, married and moved
to the United States to pursue a degree in water resources and
climatology at the University of Nebraska. She became an American
citizen in 1997. After 11 years working for the state of Missouri, she
took the job at the weather service in Ohio in 2007.

In Wilmington, she and her husband, an electronics specialist, moved
into a ranch-style house a short drive from her office, settling into a
life of comfortable routine.

Ask Mrs. Chen about her home or hobbies and you may get a word or two.
Ask her about water flow or the Ohio River and she will talk for hours.
Some 25 million people live along the Ohio River basin, which runs more
than 900 miles from Pittsburgh to Cairo, Ill., where it joins the
Mississippi River. Along the way, it flows through locks and dams
operated by the United States Army Corps of Engineers. Mrs. Chen
developed a forecasting model for predicting floods along the Ohio and
its tributaries. The model involves constant data-gathering about water
levels and rainfall, as well as how dam and lock operators respond to
water flow.

Mrs. Chen was known to be tenacious in her pursuit of data for her
predictions. She developed carpal-tunnel syndrome in her right hand from
eight years of repetitive mouse clicks. Thomas Adams, who hired Mrs.
Chen at the National Weather Service in 2007, said her fascination with
data made her perfect for the job.

``Sherry is and was dedicated to getting the details right --- and that
matters significantly,'' Mr. Adams said, noting that one inch of water
could make the difference between a levee holding or failing.

Mrs. Chen would return to Beijing every year to visit her parents, which
is how her troubles began. During her 2012 trip, a nephew said that his
future father-in-law was in a payment dispute with provincial officials
over a water pipeline.

The nephew knew that one of Mrs. Chen's former hydrology classmates,
Jiao Yong, had become vice minister of China's Ministry of Water
Resources, which oversees much of China's water infrastructure. As Mrs.
Chen tells it, her nephew asked her to reach out to Mr. Jiao, hoping he
might be able to help his future wife's father. Mrs. Chen said she was
reluctant to do so since she had not seen Mr. Jiao in many years, but
ultimately contacted him.

Mr. Jiao's secretary set up a 15-minute chat in his office in downtown
Beijing, and Mr. Jiao said he would try to intercede. As their
conversation wound down, he also mentioned that he was in the process of
funding repairs for China's aging reservoir systems and was curious how
such projects were funded in the United States.

It was a casual question, Mrs. Chen said, but she was embarrassed not to
know the answer. As a young hydrology student in China, she had been
well versed in water project finance. It was not until that moment, she
said, that she realized how little she knew about financing of such
projects in her new home country.

Always a master of details, she said her ignorance in this case gnawed
at her.

When she returned to Ohio, she set out to find an answer. She eventually
sent Mr. Jiao an email with links to websites, but nothing directly
relevant to his question.

She also asked for help from Deborah H. Lee, then the chief of the water
management division at the Army Corps of Engineers, with whom Mrs. Chen
had worked on projects over the years.

Copies of emails included in court documents show that Ms. Lee directed
Mrs. Chen to her agency's website and told her that if her former
classmate had further questions, he could contact her directly. Mrs.
Chen then sent a second, final email to Mr. Jiao instructing him to call
Ms. Lee directly with any additional questions.

Shortly after her email exchange with Mrs. Chen, Ms. Lee reported their
correspondence to security staff at the Department of Commerce, the
agency over the National Weather Service. ``I'm concerned that an effort
is being made to collect a comprehensive collection of U.S. Army Corps
of Engineers water control manuals on behalf of a foreign interest,''
Ms. Lee wrote.

Ms. Lee would not comment on her motivations for sending the email. Last
September, she left the Army Corps of Engineers for a job at the
National Oceanic Atmospheric Administration. A spokeswoman for N.O.A.A.
said neither Ms. Lee nor the agency would comment on what they deemed a
``personnel matter.''

\includegraphics{https://static01.nyt.com/images/2015/05/08/business/10espionage-web4/10espionage-web4-articleLarge.jpg?quality=75\&auto=webp\&disable=upscale}

If Mr. Jiao was trying to recruit Mrs. Chen, he was awfully
lackadaisical about it. It took a week to respond to her first email.
``Hi Xiafen: Your email received,'' he wrote, in English. ``Thanks for
information you forward me. I will go through it.'' His second email was
briefer: ``Thanks a lot.''

That was the extent of their correspondence, according to findings of a
search warrant for Mrs. Chen's work and personal email records.

Mrs. Chen said she never did find out whether Mr. Jiao had helped her
nephew's father-in-law, and has not heard from him since. Mr. Jiao did
not respond to requests for comment.

But in her search for an answer to Mr. Jiao's question, Mrs. Chen had
gone through the National Inventory of Dams database. That database,
which is maintained by the Army Corps of Engineers, is available to
government workers and members of the public who request login
credentials. A small subset of the data on the site --- six of 70 data
fields --- is available only to government workers.

As a government employee, Mrs. Chen would have had full access to the
database. But she didn't have a password; the government began requiring
passwords in 2009, after the last time Mrs. Chen had used it. So she
asked a colleague, Ray Davis, in the adjacent cubicle, for help. Mr.
Davis, who had already provided the password and login instructions to
the whole office, emailed the password to her.

Mrs. Chen didn't find much useful information for Mr. Jiao, but did
download data about Ohio dams that she thought could be relevant to her
forecasting model. For Mr. Jiao, she included a link to the database in
her second email and noted that ``this database is only for government
users, and nongovernment users are not able to download any data from
this site.'' If he had any questions, or needed information, she told
him, he should contact Ms. Lee --- who had just reported Mrs. Chen as a
possible spy.

When Mr. Davis was later questioned by Commerce officials, he said he
did not remember giving Mrs. Chen a password. Mrs. Chen said she did not
remember receiving one. And neither believed they had done anything
wrong, according to reports of their interviews.

The password, however, would come to haunt her. Nearly a year after Ms.
Lee's tip, Mrs. Chen was visited at her office by two special agents
from the Commerce Department. They interrogated her for seven hours
about the password, and her 15-minute meeting with the Chinese official.

Asked when she last met with Mr. Jiao, she responded, ``It was the last
time I visited my parents, I think 2011, May 2011.''

That was June 2013. Mrs. Chen did not hear from the government for
another three months, when she and her husband returned from a four-week
trip to see her parents. Her father, who had been ill, died during the
visit. The day after she returned, the Commerce Department agents showed
up at her office.

A slow-motion investigation was gathering momentum. An F.B.I. memo
regarding Mrs. Chen, dated July 11, 2014, listed the Army Corps of
Engineers as ``victim --- economic espionage --- PRC,'' short for
People's Republic of China.

In September 2014, Mrs. Chen and her husband were stopped while boarding
a United Airlines flight to Beijing from Newark. Their baggage was
pulled and searched. Ms. Chen said they were ultimately allowed to
return to the plane, which had been held for over an hour.

It was after returning from that trip, in October, that she was
arrested. At the Dayton courthouse, she was charged with four felonies,
including that she had illegally downloaded data about ``critical
national infrastructure'' from a restricted government database --- the
National Inventory of Dams --- and made false statements.

The false statement referred to telling the agents that she had last
seen Mr. Jiao in 2011, not 2012. Four other charges were added later.

Image

The upriver entrance to the McAlpine Locks along the Ohio River. Ms.
Chen was accused of using a stolen password to download information
about the nation's dams and of lying about meeting with a high-ranking
Chinese official.Credit...Luke Sharrett for The New York Times

She was released the same day and placed on unpaid administrative leave.

\textbf{`Why Would You Do That?'}

After the arrest, Mrs. Chen's name was all over the Internet. The case
was picked up by local media and The Washington Free Beacon, a
conservative news website, which
\href{http://freebeacon.com/national-security/noaa-employee-charged-with-computer-breach-met-senior-chinese-official-in-beijing/}{played
up Mrs. Chen's meeting} with a ``senior Chinese official.''

Mrs. Chen says she was living a nightmare. Peter R. Zeidenberg, a
partner at Arent Fox in Washington who represented Mrs. Chen, said he
believed it was telling that the government went after Mrs. Chen for
using a colleague's password, but not after the colleague who gave it to
her --- and to the entire office. (Neither Mr. Davis nor anyone else
currently employed at the National Weather Service would comment for
this article.)

Mr. Adams, her former colleague, said he thought that Mrs. Chen's
Chinese background played a role. ``If this had been you or me or
someone of European descent who borrowed someone else's password,'' he
said, ``they would have said, `Don't do this again.'~'' He added: ``This
is the gratitude the government has shown for her hard work and
dedication as a federal public servant. It's shameful.''

A week before trial was to begin, Mr. Zeidenberg requested a meeting
with Carter M. Stewart and Mark T. D'Alessandro, two United States
attorneys for the Southern District of Ohio.

``Why,'' Mr. Zeidenberg said he asked, ``if she's a spy, is she coming
back from China and telling her colleagues that `I met this guy in China
and this is what he wants to know'? Why is she telling the guy in China,
`Here's my boss's phone number'? Why is she asking for a password over
email? Why would you do that?''

Mr. Zeidenberg says the prosecutors listened. On March 10, the day after
their meeting, they dismissed the charges.

``Thank God,'' Mr. Zeidenberg added.

\textbf{Looking Everywhere for Spies}

Mrs. Chen was caught in a much broader dragnet aimed at combating
Chinese industrial espionage. Law enforcement investigations into
trade-secret theft are now at record levels, jumping 60 percent between
2009 and 2013, according to an F.B.I. report last year.

In 2013, Eric H. Holder Jr., then the attorney general, said the Justice
Department would bring more economic espionage cases, and in 2014 it
secured the first ever indictment of foreign actors when it
\href{http://www.nytimes.com/2014/05/20/us/us-to-charge-chinese-workers-with-cyberspying.html}{charged
five Chinese military officers} with trade-secret theft. (The chances of
arrests, however, are slim.)

Inside the United States, prosecutors recently invoked the Foreign
Intelligence Surveillance Act in the case of a Chinese citizen living in
this country accused of stealing hybrid seeds from an Iowa cornfield. In
addition to physical surveillance, the government used
\href{http://justsecurity.org/wp-content/uploads/2015/03/Hailong-Order-Denying-Defendants-Motion-to-Compel.pdf}{a
secret FISA warrant} to intercept the defendant's mail, email and phone
calls and install location-tracking and listening devices in his car.

In another case, in Philadelphia, an American resident with Chinese
citizenship stands accused of damaging a server to cover up trade-secret
theft. He's been held in a federal detention center for over two years;
his trial is set for November.

``If you're looking everywhere for spies, you will find spies
everywhere, even where they don't exist,'' said Mr. Rasch, the former
computer-crimes prosecutor.

The case against Mrs. Chen has made her wary. After I had interviewed
her several times but did not contact her for some days, she said she
had convinced herself that I was not a reporter at all, but an
undercover agent.

Mrs. Chen says she recalls becoming an American citizen as her proudest
moment. She told me about all the positive performance reviews she
received and began to cry when she remembered the way she was handcuffed
in front of co-workers and put into the back of an F.B.I. car.

Still, she says, she wants her job back. ``I know they treated me
unfairly, but I'm proud of my service,'' she said. ``The forecasting
model is very important. I miss my colleagues. I miss my work. It's my
life.''

Mrs. Chen's benefits and pay have been restored, but she is waiting to
hear whether the Commerce Department will allow her to return to work.
Sara Ryan, the department lawyer handling Mrs. Chen's case, said she
would not discuss it. Representatives for the department did not return
requests for comment.

Asked whether he thought Mrs. Chen should get her job back, Mr. Adams,
her former colleague, said he was torn. ``I want her to get her job back
as soon as possible,'' he said. ``But on the other hand, I also hope she
never goes back there again. After the way she was treated, she should
be concerned that the government hasn't given up the ghost.''

Advertisement

\protect\hyperlink{after-bottom}{Continue reading the main story}

\hypertarget{site-index}{%
\subsection{Site Index}\label{site-index}}

\hypertarget{site-information-navigation}{%
\subsection{Site Information
Navigation}\label{site-information-navigation}}

\begin{itemize}
\tightlist
\item
  \href{https://help.nytimes.com/hc/en-us/articles/115014792127-Copyright-notice}{©~2020~The
  New York Times Company}
\end{itemize}

\begin{itemize}
\tightlist
\item
  \href{https://www.nytco.com/}{NYTCo}
\item
  \href{https://help.nytimes.com/hc/en-us/articles/115015385887-Contact-Us}{Contact
  Us}
\item
  \href{https://www.nytco.com/careers/}{Work with us}
\item
  \href{https://nytmediakit.com/}{Advertise}
\item
  \href{http://www.tbrandstudio.com/}{T Brand Studio}
\item
  \href{https://www.nytimes.com/privacy/cookie-policy\#how-do-i-manage-trackers}{Your
  Ad Choices}
\item
  \href{https://www.nytimes.com/privacy}{Privacy}
\item
  \href{https://help.nytimes.com/hc/en-us/articles/115014893428-Terms-of-service}{Terms
  of Service}
\item
  \href{https://help.nytimes.com/hc/en-us/articles/115014893968-Terms-of-sale}{Terms
  of Sale}
\item
  \href{https://spiderbites.nytimes.com}{Site Map}
\item
  \href{https://help.nytimes.com/hc/en-us}{Help}
\item
  \href{https://www.nytimes.com/subscription?campaignId=37WXW}{Subscriptions}
\end{itemize}
