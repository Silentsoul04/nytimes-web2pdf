Sections

SEARCH

\protect\hyperlink{site-content}{Skip to
content}\protect\hyperlink{site-index}{Skip to site index}

\href{https://www.nytimes.com/section/world/asia}{Asia Pacific}

\href{https://myaccount.nytimes.com/auth/login?response_type=cookie\&client_id=vi}{}

\href{https://www.nytimes.com/section/todayspaper}{Today's Paper}

\href{/section/world/asia}{Asia Pacific}\textbar{}North Korea Said to
Execute a Top Official, With an Antiaircraft Gun

\url{https://nyti.ms/1G4VGpl}

\begin{itemize}
\item
\item
\item
\item
\item
\end{itemize}

Advertisement

\protect\hyperlink{after-top}{Continue reading the main story}

Supported by

\protect\hyperlink{after-sponsor}{Continue reading the main story}

\hypertarget{north-korea-said-to-execute-a-top-official-with-an-antiaircraft-gun}{%
\section{North Korea Said to Execute a Top Official, With an
Antiaircraft
Gun}\label{north-korea-said-to-execute-a-top-official-with-an-antiaircraft-gun}}

\includegraphics{https://static01.nyt.com/images/2015/05/14/world/14NKOREA-WEB/14NKOREA-WEB-articleLarge.jpg?quality=75\&auto=webp\&disable=upscale}

By \href{http://www.nytimes.com/by/choe-sang-hun}{Choe Sang-Hun}

\begin{itemize}
\item
  May 12, 2015
\item
  \begin{itemize}
  \item
  \item
  \item
  \item
  \item
  \end{itemize}
\end{itemize}

SEOUL, South Korea --- The second-highest officer in North Korea's
military was recently executed as a traitor for showing disrespect for
the nation's leader, Kim Jong-un, South Korean intelligence officials
told lawmakers here Wednesday.

Gen. Hyon Yong-chol, the minister of the People's Armed Forces, is
believed to have been executed with an antiaircraft gun in Pyongyang,
the North's capital, around April 30, National Intelligence Service
officials told South Korean lawmakers during a closed parliamentary
session.

Mr. Kim deemed General Hyon disloyal after he dozed off during military
events and second-guessed Mr. Kim's orders, the intelligence officials
were quoted as saying by two lawmakers, who attended the session. With
hundreds of North Korea's elite watching, General Hyon was executed on
charges of being a traitor, the officials said. General Hyon, who is
considered second in the military hierarchy only to Vice Marshal Hwang
Pyong-so, has disappeared from North Korea's state-run news media
starting in late April.

The National Intelligence Service referred any queries from the news
media to the two lawmakers, Lee Cheol-woo and Shin Kyoung-min.

Mr. Kim is believed to have been terrorizing North Korea's elites with
executions and purges as he has struggled to establish his authority
since the death of his father,
\href{http://www.nytimes.com/2011/12/19/world/asia/kim-jong-il-is-dead.html?pagewanted=all}{Kim
Jong-il}, in 2011.

It is not clear how the South Korean spy agency acquired information on
General Hyon's supposed execution. Last month, the agency told the
parliamentary intelligence committee that North Korea had
\href{http://www.nytimes.com/2015/04/30/world/asia/north-korea-executed-15-top-officials-in-2015-south-korean-agency-says.html}{executed
15 high-ranking government officials} this year.

Information the spy agency has provided during closed parliamentary
hearings has been considered reliable. But analysts caution that
gathering verifiable data on the inner workings of the North's
government is difficult.

When Mr. Kim's father died, South Korean intelligence officials were not
aware of it until Pyongyang announced the news two days later.

Cheong Seong-chang, a senior analyst at the Sejong Institute in South
Korea, warned that the spy agency was publicizing ``unverified
intelligence'' on the supposed execution of General Hyon and said that
prudent analysts should wait for more solid evidence.

``If he was really executed before other officials in late April, North
Korea by now would have erased all his images from old documentary
footage being broadcast on the North Korean TV, but that apparently has
not happened yet,'' Mr. Cheong said.

The spy agency has in the past been accused of leaking shocking news
about North Korea to unsettle its government or divert attention from
domestic scandals. In recent weeks, the South Korean government has been
rocked by the North's
\href{http://www.nytimes.com/2015/05/09/world/asia/north-korea-says-it-test-fired-missile-from-submarine.html}{test
of a submarine-launched missile} and a domestic bribery scandal that led
to the
\href{http://www.nytimes.com/2015/04/21/world/asia/south-korean-premier-offers-his-resignation.html}{resignation
of the prime minister}.

South Korean officials said North Korea's leader, believed to be in his
early 30s, was resorting to a mix of terror and rewards to thwart any
challenge to his leadership. He is believed to have ordered the
execution of 68 senior officials from 2012 to last year, according to
the South Korean spy agency. The reasons given included failure to
follow through with Mr. Kim's orders or raising questions about his
decisions.

In 2013, an uncle of Mr. Kim's, Jang Song-thaek, long considered the
second most powerful man in North Korea,
\href{http://www.nytimes.com/2013/12/13/world/asia/north-korea-says-uncle-of-executed.html}{was
executed}, accused of stealing state funds and plotting to overthrow Mr.
Kim.

General Hyon has been one of many generals whose fortunes appear to
fluctuate according to Mr. Kim's whim. The general's status seemed to
soar in 2012, when he became vice marshal as chief of the general staff
of the North Korean People's Army.

He did not last long in that post, however, as he was soon demoted to
general. He resurfaced as the head of the Ministry of People's Armed
Forces in June.

Advertisement

\protect\hyperlink{after-bottom}{Continue reading the main story}

\hypertarget{site-index}{%
\subsection{Site Index}\label{site-index}}

\hypertarget{site-information-navigation}{%
\subsection{Site Information
Navigation}\label{site-information-navigation}}

\begin{itemize}
\tightlist
\item
  \href{https://help.nytimes.com/hc/en-us/articles/115014792127-Copyright-notice}{©~2020~The
  New York Times Company}
\end{itemize}

\begin{itemize}
\tightlist
\item
  \href{https://www.nytco.com/}{NYTCo}
\item
  \href{https://help.nytimes.com/hc/en-us/articles/115015385887-Contact-Us}{Contact
  Us}
\item
  \href{https://www.nytco.com/careers/}{Work with us}
\item
  \href{https://nytmediakit.com/}{Advertise}
\item
  \href{http://www.tbrandstudio.com/}{T Brand Studio}
\item
  \href{https://www.nytimes.com/privacy/cookie-policy\#how-do-i-manage-trackers}{Your
  Ad Choices}
\item
  \href{https://www.nytimes.com/privacy}{Privacy}
\item
  \href{https://help.nytimes.com/hc/en-us/articles/115014893428-Terms-of-service}{Terms
  of Service}
\item
  \href{https://help.nytimes.com/hc/en-us/articles/115014893968-Terms-of-sale}{Terms
  of Sale}
\item
  \href{https://spiderbites.nytimes.com}{Site Map}
\item
  \href{https://help.nytimes.com/hc/en-us}{Help}
\item
  \href{https://www.nytimes.com/subscription?campaignId=37WXW}{Subscriptions}
\end{itemize}
