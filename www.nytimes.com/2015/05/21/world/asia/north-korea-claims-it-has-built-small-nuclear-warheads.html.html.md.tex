Sections

SEARCH

\protect\hyperlink{site-content}{Skip to
content}\protect\hyperlink{site-index}{Skip to site index}

\href{https://www.nytimes.com/section/world/asia}{Asia Pacific}

\href{https://myaccount.nytimes.com/auth/login?response_type=cookie\&client_id=vi}{}

\href{https://www.nytimes.com/section/todayspaper}{Today's Paper}

\href{/section/world/asia}{Asia Pacific}\textbar{}North Korea Claims It
Has Built Small Nuclear Warheads

\url{https://nyti.ms/1KjPsEs}

\begin{itemize}
\item
\item
\item
\item
\item
\end{itemize}

Advertisement

\protect\hyperlink{after-top}{Continue reading the main story}

Supported by

\protect\hyperlink{after-sponsor}{Continue reading the main story}

\hypertarget{north-korea-claims-it-has-built-small-nuclear-warheads}{%
\section{North Korea Claims It Has Built Small Nuclear
Warheads}\label{north-korea-claims-it-has-built-small-nuclear-warheads}}

By \href{http://www.nytimes.com/by/choe-sang-hun}{Choe Sang-Hun}

\begin{itemize}
\item
  May 20, 2015
\item
  \begin{itemize}
  \item
  \item
  \item
  \item
  \item
  \end{itemize}
\end{itemize}

SEOUL, South Korea --- North Korea said on Wednesday that it had already
built nuclear weapons small enough to be carried by missiles, even as a
senior American general questioned the country's recent claim that it
had successfully tested a submarine-launched ballistic missile.

``It is long since the D.P.R.K.'s nuclear striking means have entered
the stage of producing smaller nukes and diversifying them,'' the
National Defense Commission said in a statement, using the initials of
North Korea's formal name, the Democratic People's Republic of Korea.

The statement was carried by the official news agency, K.C.N.A.

``The D.P.R.K. has reached the stage of ensuring the highest precision
and intelligence and best accuracy of not only medium- and short-range
rockets, but long-range ones,'' the agency said.

Officials and analysts in Washington and Seoul remain uncertain and even
divided over how close North Korea has come to acquiring a nuclear
weapon small enough to be put on a missile, or its ability to deliver a
nuclear warhead on an intercontinental ballistic missile. But their
concern has grown since the North placed a satellite into orbit in
December 2012, successfully demonstrating a rocket technology needed for
a long-range missile.

In February 2013, North Korea also claimed that it had conducted its
third underground nuclear test with ``a smaller and lighter A-bomb.''

A month later, the North's main government newspaper, the Rodong Sinmun,
quoted a North Korean general as saying that the North's
``intercontinental ballistic missiles and other missiles are on a
standby, loaded with lighter, smaller and diversified nuclear
warheads.''

Adm. William E. Gortney, the commander of the North American Aerospace
Defense Command, told reporters last month that American intelligence
officials believed that North Korea had the ability to put a nuclear
weapon on its KN-08 intercontinental ballistic missile ``and shoot it at
the homeland,'' although he said the North had yet to run a flight test
of the missile.

North Korea's statement on Wednesday came in response to international
criticism of
\href{http://www.nytimes.com/2015/05/09/world/asia/north-korea-says-it-test-fired-missile-from-submarine.html}{a
ballistic missile test} Pyongyang said it conducted on May 8. United
Nations resolutions prohibit North Korea from testing such a missile.

North Korea said the May 8 test involved successfully launching a
strategic missile from a submarine. But some analysts have since
questioned the claim, saying that some of the photographs of the episode
that North Korea released may have been altered and that the test launch
may have been conducted from a
\href{http://38north.org/2015/05/jbermudez051315/}{submerged barge},
rather than a submarine.

Speaking at the Center for Strategic and International Studies in
Washington on Tuesday, Adm. James A. Winnefeld Jr., vice chairman of the
Joint Chiefs of Staff of the United States, voiced similar misgivings.

``They have not gotten as far as their clever video editors and
spinmeisters would have us believe,'' Admiral Winnefeld said. ``They are
many years away from developing this capability. But if they are
eventually able to do so, it will present a hard-to-detect danger for
Japan and South Korea, as well as our service members stationed in the
region.''

The South Korean Defense Ministry, however, says that the North
successfully tested the ``ejection'' of a submarine-launched missile on
May 9, although it did not involve a full flight test.

Advertisement

\protect\hyperlink{after-bottom}{Continue reading the main story}

\hypertarget{site-index}{%
\subsection{Site Index}\label{site-index}}

\hypertarget{site-information-navigation}{%
\subsection{Site Information
Navigation}\label{site-information-navigation}}

\begin{itemize}
\tightlist
\item
  \href{https://help.nytimes.com/hc/en-us/articles/115014792127-Copyright-notice}{©~2020~The
  New York Times Company}
\end{itemize}

\begin{itemize}
\tightlist
\item
  \href{https://www.nytco.com/}{NYTCo}
\item
  \href{https://help.nytimes.com/hc/en-us/articles/115015385887-Contact-Us}{Contact
  Us}
\item
  \href{https://www.nytco.com/careers/}{Work with us}
\item
  \href{https://nytmediakit.com/}{Advertise}
\item
  \href{http://www.tbrandstudio.com/}{T Brand Studio}
\item
  \href{https://www.nytimes.com/privacy/cookie-policy\#how-do-i-manage-trackers}{Your
  Ad Choices}
\item
  \href{https://www.nytimes.com/privacy}{Privacy}
\item
  \href{https://help.nytimes.com/hc/en-us/articles/115014893428-Terms-of-service}{Terms
  of Service}
\item
  \href{https://help.nytimes.com/hc/en-us/articles/115014893968-Terms-of-sale}{Terms
  of Sale}
\item
  \href{https://spiderbites.nytimes.com}{Site Map}
\item
  \href{https://help.nytimes.com/hc/en-us}{Help}
\item
  \href{https://www.nytimes.com/subscription?campaignId=37WXW}{Subscriptions}
\end{itemize}
