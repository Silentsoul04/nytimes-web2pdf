Sections

SEARCH

\protect\hyperlink{site-content}{Skip to
content}\protect\hyperlink{site-index}{Skip to site index}

\href{https://www.nytimes.com/section/us}{U.S.}

\href{https://myaccount.nytimes.com/auth/login?response_type=cookie\&client_id=vi}{}

\href{https://www.nytimes.com/section/todayspaper}{Today's Paper}

\href{/section/us}{U.S.}\textbar{}Watts,~50 Years On, Stands in Contrast
to Today's Conflicts

\href{https://nyti.ms/1MgrCtc}{https://nyti.ms/1MgrCtc}

\begin{itemize}
\item
\item
\item
\item
\item
\end{itemize}

Advertisement

\protect\hyperlink{after-top}{Continue reading the main story}

Supported by

\protect\hyperlink{after-sponsor}{Continue reading the main story}

\hypertarget{watts-50-years-on-stands-in-contrast-to-todays-conflicts}{%
\section{Watts,~50 Years On, Stands in Contrast to Today's
Conflicts}\label{watts-50-years-on-stands-in-contrast-to-todays-conflicts}}

\includegraphics{https://static01.nyt.com/images/2015/08/11/us/WATTS/WATTS-articleLarge.jpg?quality=75\&auto=webp\&disable=upscale}

By \href{http://www.nytimes.com/by/jennifer-medina}{Jennifer Medina}

\begin{itemize}
\item
  Aug. 10, 2015
\item
  \begin{itemize}
  \item
  \item
  \item
  \item
  \item
  \end{itemize}
\end{itemize}

LOS ANGELES --- Donny Joubert watched the boy round the corner of the
housing project holding what looked like a handgun. The barrel was
pointed at him and the two officers from the Los Angeles Police
Department. The boy was 10 or 11 years old, Mr. Joubert figured, and had
more gleam than anger in his eyes. Mr. Joubert, a community activist who
grew up in the project, shouted and lunged for the gun.

It was plastic. The police officers did not even reach for their
holsters.

``Somewhere else, that kid would be dead,'' Mr. Joubert said.

That interaction, Mr. Joubert said, is the best illustration of the way
the community has changed significantly in the 50 years since
\href{https://www.youtube.com/watch?v=Au9oohI1MuM}{the Watts riots}
broke out on the streets here for six days starting on Aug. 11, 1965.

Confrontations between African-Americans and the police are once again
convulsing the country; in Ferguson, Mo., where protesters gathered over
the weekend to commemorate the anniversary of the death of Michael Brown
and the riots that ensued, a gunman fired at the police on Sunday night
and was shot, and other gunfire and skirmishes broke out. But Watts ---
once a symbol of urban strife and racial tensions --- stands as a stark
contrast. There were fewer than a dozen homicides in the neighborhood
last year, compared with hundreds in 1965. Community leaders like Mr.
Joubert, a former gang leader turned peacemaker and respected mentor,
say relations with the police have never been better.

``They don't think the kid is out to kill them; they're not out to kill
the kid,'' Mr. Joubert said. ``They walk and they know who they are
talking to. We've been through this before, we're still kind of
recovering and saying there's another way.''

Still, this is no utopia.

Each summer, Mr. Joubert, 54, helps run a jobs program for teenagers at
Nickerson Gardens, the low-slung public housing complex where he was
raised. Twice last week, the teenagers were summoned inside because of
shootings, as administrators worried that a stray bullet would endanger
them. And little of the trust Mr. Joubert has for the police has
filtered to these teenagers.

``They harass us all the time,'' said Raydon Boyce, 19.

\href{https://www.nytimes.com/interactive/2015/08/05/us/watts-riots-archives.html}{}

\hypertarget{the-watts-riots}{%
\subsection{The Watts Riots}\label{the-watts-riots}}

A look back at how The New York Times covered the Watts riots over the
years.

``Don't matter what you do,'' Nigel Ewers added, echoing the sentiment
expressed by all of the teenage boys taking a break one morning last
week.

This is not the same Watts their parents grew up in. While the area
remains persistently poor, demographics have transformed it from an
African-American enclave to a neighborhood that is more than 70 percent
Latino. Many blacks have moved to the suburbs in the Inland Empire and
the desert north of Los Angeles. Those changes have brought their own
tensions; many black residents talk of feeling pushed out while Latinos
have struggled to rise to political leadership.

``Sometimes we're all against everyone,'' said Steve Torres, 17, whose
sister left Watts for a small town in Virginia last year. Sitting across
the table, an African-American teenager spoke of police officers
``killing us off.''

``Don't really matter who you are; we're just labeled as bad people,''
Mr. Torres said.

\includegraphics{https://static01.nyt.com/images/2015/08/11/us/11WATTSJP1/11WATTSJP1-articleLarge.jpg?quality=75\&auto=webp\&disable=upscale}

Big questions hang in the air, sometimes asked aloud: Could what
happened in Ferguson happen here? Could Watts explode as it did five
decades ago? Alternatively, could the improvements in Watts happen in
Ferguson? There is a deep generational divide in the answers.

Last summer, when Los Angeles police officers shot and killed Ezell
Ford, an unarmed mentally ill black man, less than a mile away,
protesters wanted to march from downtown to Watts. Mr. Joubert and other
leaders urged them to stay away, he said.

``There's always a sense things can boil over,'' said Nina Revoyr, the
chief operating officer for the Children's Institute here, which runs
dozens of programs and offers free mental health services in the
neighborhood. ``But there's a sense of maturity here; the neighborhood
has been through all this before and the transformation has happened.
There's a true relationship --- you see a problem, and you talk about
it.''

Image

Watts residents sell clothing outside an apartment complex; demographics
have transformed the area from a black enclave to a neighborhood more
than 70 percent Latino.Credit...Monica Almeida/The New York Times

Image

The Watts Towers. It has been 50 years since riots broke out on the
streets in Watts for six days starting on Aug. 11, 1965.Credit...Monica
Almeida/The New York Times

After the police caught a group of youngsters who had been stealing from
the offices of the
\href{http://www.childrensinstitute.org/about/staff}{Children's
Institute}, Ms. Revoyr worked with officers to avoid pressing charges
and instead sent them to a diversion program where they completed hours
of community service.

Every week for the better part of a decade, Mr. Joubert and other local
leaders have met as part of the Watts Gang Task Force, exchanging
information with the police and trying to find ways to quell tensions in
the community, whether they stem from a gang fight or a police
interaction.

In some sense, the changes in the area are evidence of the uniqueness of
the neighborhood, which covers just more than two square miles. It is,
as some residents put it, the smallest neighborhood with the biggest
reputation.

The city's Housing Authority has poured more than \$10 million into
special projects there in the last several years. The Police Department
has dedicated 10 officers and a sergeant to each of the housing
complexes, with officers generally signing on for a five-year commitment
to patrol the area by foot each day. The police officers have begun a
football league for 9- to 11-year-olds and work as coaches on their days
off.

Image

Donny Joubert, a community activist who grew up in a housing project in
Watts, said that relationships with the police have never been
better.Credit...Monica Almeida/The New York Times

There are signs, too, of enormous challenges. The perimeters of the
sports fields at one middle school are fortified with mounds of dirt,
meant to protect students from bullets. Residents celebrated a park when
it opened this year on what had been a weed-infested lot. Now, the gate
to the park is locked, and the slides and skateboard ramps were empty on
recent summer afternoons.

The area remains physically isolated, crisscrossed by freeways and
railroads. There is still no sit-down restaurant, but the father of the
city's food trucks, Roy Choi, has announced plans to open one. The
Children's Institute will soon unveil plans for its new building
designed by Frank Gehry, which it hopes will function as a community
center.

The persistent doubts remain.

``We look around at what other places have, and we just don't see the
opportunities here,'' said Tim Watkins, who runs the Watts Labor
CommunityAction Committee, which his father created after the 1965
uprising, as locals refer to it. ``There's still a lot of desperation
around here, and that can lead to desperate acts at any time.''

Sgt. Emada Tingirides, who grew up in the neighborhood and now serves as
the coordinator of the Community Safety Partnership program in the
housing projects, said hardly a day goes by without talking with
residents here about police shootings in other parts of the country.

Image

Capt. Alfred Pasos of the Los Angeles Police Department posed for a
photo with residents at a National Night Out street fair in Watts last
week.Credit...Monica Almeida/The New York Times

``If it happened here, we would know what to do after the fact,'' she
said. But she acknowledged change does not come easily and officers
still face mistrust from the young men in the neighborhood. ``This is a
cultural shift that is going to take time, not just years but decades
and generations.''

She thought back to the boy who approached her with the toy gun about
six months ago. Mr. Joubert snapped the object in half and then
persuaded the ice cream trucks and liquor stores to stop selling them.

That boy, she said, might be someone who now believes the police are out
there to protect him, or at least not out to get him.

Advertisement

\protect\hyperlink{after-bottom}{Continue reading the main story}

\hypertarget{site-index}{%
\subsection{Site Index}\label{site-index}}

\hypertarget{site-information-navigation}{%
\subsection{Site Information
Navigation}\label{site-information-navigation}}

\begin{itemize}
\tightlist
\item
  \href{https://help.nytimes.com/hc/en-us/articles/115014792127-Copyright-notice}{©~2020~The
  New York Times Company}
\end{itemize}

\begin{itemize}
\tightlist
\item
  \href{https://www.nytco.com/}{NYTCo}
\item
  \href{https://help.nytimes.com/hc/en-us/articles/115015385887-Contact-Us}{Contact
  Us}
\item
  \href{https://www.nytco.com/careers/}{Work with us}
\item
  \href{https://nytmediakit.com/}{Advertise}
\item
  \href{http://www.tbrandstudio.com/}{T Brand Studio}
\item
  \href{https://www.nytimes.com/privacy/cookie-policy\#how-do-i-manage-trackers}{Your
  Ad Choices}
\item
  \href{https://www.nytimes.com/privacy}{Privacy}
\item
  \href{https://help.nytimes.com/hc/en-us/articles/115014893428-Terms-of-service}{Terms
  of Service}
\item
  \href{https://help.nytimes.com/hc/en-us/articles/115014893968-Terms-of-sale}{Terms
  of Sale}
\item
  \href{https://spiderbites.nytimes.com}{Site Map}
\item
  \href{https://help.nytimes.com/hc/en-us}{Help}
\item
  \href{https://www.nytimes.com/subscription?campaignId=37WXW}{Subscriptions}
\end{itemize}
