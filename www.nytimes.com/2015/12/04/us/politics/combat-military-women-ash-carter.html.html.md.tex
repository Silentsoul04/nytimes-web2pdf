Sections

SEARCH

\protect\hyperlink{site-content}{Skip to
content}\protect\hyperlink{site-index}{Skip to site index}

\href{https://www.nytimes.com/section/politics}{Politics}

\href{https://myaccount.nytimes.com/auth/login?response_type=cookie\&client_id=vi}{}

\href{https://www.nytimes.com/section/todayspaper}{Today's Paper}

\href{/section/politics}{Politics}\textbar{}All Combat Roles Now Open to
Women, Defense Secretary Says

\url{https://nyti.ms/1Ns1Lv7}

\begin{itemize}
\item
\item
\item
\item
\item
\item
\end{itemize}

Advertisement

\protect\hyperlink{after-top}{Continue reading the main story}

Supported by

\protect\hyperlink{after-sponsor}{Continue reading the main story}

\hypertarget{all-combat-roles-now-open-to-women-defense-secretary-says}{%
\section{All Combat Roles Now Open to Women, Defense Secretary
Says}\label{all-combat-roles-now-open-to-women-defense-secretary-says}}

\includegraphics{https://static01.nyt.com/images/2015/12/03/multimedia/ash-carter-women/ash-carter-women-videoSixteenByNine3000.jpg}

By \href{http://www.nytimes.com/by/matthew-rosenberg}{Matthew Rosenberg}
and \href{http://www.nytimes.com/by/dave-philipps}{Dave Philipps}

\begin{itemize}
\item
  Dec. 3, 2015
\item
  \begin{itemize}
  \item
  \item
  \item
  \item
  \item
  \item
  \end{itemize}
\end{itemize}

In a historic transformation of the American military, Defense Secretary
Ashton B. Carter said on Thursday that the Pentagon would open all
combat jobs to women.

``There will be no exceptions,'' Mr. Carter said at a news conference.
He added, ``They'll be allowed to drive tanks, fire mortars and lead
infantry soldiers into combat. They'll be able to serve as Army Rangers
and Green Berets, Navy SEALs, Marine Corps infantry, Air Force
parajumpers and everything else that was previously open only to men.''

The groundbreaking decision overturns a longstanding rule that had
restricted women from combat roles, even though women have often found
themselves in combat in Iraq and Afghanistan over the past 14 years.

It is the latest in a long march of inclusive steps by the military,
including racial integration in 1948 and the lifting of the ban on gay
men and lesbians serving openly in the military in 2011. The decision
this week will open about 220,000 military jobs to women.

The military faced a deadline set by the Obama administration three
years ago to integrate women into all combat jobs by January or ask for
specific exemptions. The Navy and Air Force have already opened almost
all combat positions to women, and the Army has increasingly integrated
its forces.

The announcement Thursday was a rebuke to the Marine Corps, which has a
93 percent male force dominated by infantry and a culture that still
segregates recruits by gender for basic training. In September, the
Marines requested an exemption for infantry and armor positions, citing
a yearlong study that showed integration could hurt its fighting
ability. But Mr. Carter said he overruled the Marines because the
military should operate under a common set of standards.

\includegraphics{https://static01.nyt.com/images/2015/12/04/us/04COMBAT/04COMBAT-articleLarge.jpg?quality=75\&auto=webp\&disable=upscale}

Gen. Joseph E. Dunford Jr., the former commandant of the Marine Corps
who recently became chairman of the Joint Chiefs of Staff, did not
attend the announcement, and in a statement Thursday appeared to give
only tepid support, saying, ``I have had the opportunity to provide my
advice on the issue of full integration of women into the armed forces.
In the wake of the secretary's decision, my responsibility is to ensure
his decision is properly implemented.''

Women have long chafed under the combat restrictions, which allowed them
to serve in combat zones, often under fire, but prevented them from
officially holding combat positions, including in the infantry, which
remain crucial to career advancement. Women have long said that by not
recognizing their real service, the military has unfairly held them
back.

A major barrier fell this year when women were permitted to go through
the grueling training that would allow them to qualify as Army Rangers,
the service's elite infantry.

Mr. Carter said that women would be allowed to serve in all military
combat roles by early next year. He characterized the change as
necessary to ensure that the United States military remained the world's
most powerful.

``When I became secretary of defense, I made a commitment to building
America's force of the future,'' Mr. Carter told reporters. ``In the
21st century that requires drawing strength from the broadest possible
pool of talent. This includes women.''

Many women hailed the decision. ``I'm overjoyed,'' said Katelyn van Dam,
an attack helicopter pilot in the Marine Corps who has deployed to
Afghanistan. ``Now if there is some little girl who wants to be a
tanker, no one can tell her she can't.''

But the Republican chairmen of the Senate and House Armed Services
Committees expressed caution and noted that by law Congress had 30 days
to review the decision.

Image

Cpl. Christina Oliver, 25, a United States Marine with the Female
Engagement Team, patrolled near an Afghan village to clear the area of
Taliban in 2010.Credit...Lynsey Addario for The New York Times

``Secretary Carter's decision to open all combat positions to women will
have a consequential impact on our service members and our military's
warfighting capabilities,'' Senator John McCain of Arizona and
Representative Mac Thornberry of Texas said in a statement. ``The Senate
and House Armed Services Committees intend to carefully and thoroughly
review all relevant documentation related to today's decision.''

Some in the military have privately voiced concern that integration will
prove impractical, especially in the infantry, where heavy loads and
long periods of deprivation are part of the job.

``Humping a hundred pounds, man, that ain't easy, and it remains the
defining physical requirement of the infantry,'' said Paul Davis, an
exercise scientist who did a multiyear study of the Marine infantry.
``The practical reality is that even though we want to knock down this
last bastion of exclusion, the preponderance of women will not be able
to do the job.''

Mr. Carter acknowledged at the news conference that simply opening up
combat roles to women was not going to lead to a fully integrated
military. Senior defense officials and military officers would have to
overcome the perception among many service members, men and women alike,
that the change would reduce the effectiveness of the armed services.

The defense secretary sought to assuage those concerns on Thursday by
saying that every service member would have to meet the standards of the
jobs they wished to fill, and ``there must be no quotas or perception
thereof.''

He also acknowledged that many units were likely to remain largely male,
especially elite infantry troops and Special Operations forces, where
``only small numbers of women could'' likely meet the standards.

``Studies say there are physical differences,'' Mr. Carter said, though
he added that some women could meet the most demanding physical
requirements, just as some men could not.

At the same time, he said, military leaders are going to be required to
assign jobs and tasks and determine who is promoted based on ``ability,
not gender.''

Lt. Col. Kate Germano, who oversaw the training of female recruits for
the Marines until she was removed this summer from duty during a dispute
over what she said were lower standards for women in basic training,
said by creating standards, the military would improve across both
genders.

She said while Marines have long resisted the idea of women in combat
units, she did not expect a backlash.

``One thing about the Marine Corps, once you tell us what we have to do,
we'll do it,'' she said. ``There was resistance to lifting the ban on
gays, too, and when it was lifted there were no issues. We are a
stronger force for it.''

Mr. Carter's announcement came less than a month from the three-year
deadline set by the Obama administration to integrate the force.

Some veterans of recent wars say the unexpectedly long period of combat
with no clear enemy lines may have been a driver for the change.

``I honestly didn't think about women in combat much until Iraq,'' said
Jonathan Silk, a retired Army major who served in Afghanistan and Iraq
as a cavalry scout.

In the fray of the insurgency, he said, integrated military police units
near him often faced ferocious attacks. ``That is where I encountered
female soldiers that were in the same firefights as us, facing the same
horrible stuff, even if they weren't technically in combat units. They
could fight just as well as I could, and some of those women were
tremendous leaders. It gave me such respect.''

Advertisement

\protect\hyperlink{after-bottom}{Continue reading the main story}

\hypertarget{site-index}{%
\subsection{Site Index}\label{site-index}}

\hypertarget{site-information-navigation}{%
\subsection{Site Information
Navigation}\label{site-information-navigation}}

\begin{itemize}
\tightlist
\item
  \href{https://help.nytimes.com/hc/en-us/articles/115014792127-Copyright-notice}{©~2020~The
  New York Times Company}
\end{itemize}

\begin{itemize}
\tightlist
\item
  \href{https://www.nytco.com/}{NYTCo}
\item
  \href{https://help.nytimes.com/hc/en-us/articles/115015385887-Contact-Us}{Contact
  Us}
\item
  \href{https://www.nytco.com/careers/}{Work with us}
\item
  \href{https://nytmediakit.com/}{Advertise}
\item
  \href{http://www.tbrandstudio.com/}{T Brand Studio}
\item
  \href{https://www.nytimes.com/privacy/cookie-policy\#how-do-i-manage-trackers}{Your
  Ad Choices}
\item
  \href{https://www.nytimes.com/privacy}{Privacy}
\item
  \href{https://help.nytimes.com/hc/en-us/articles/115014893428-Terms-of-service}{Terms
  of Service}
\item
  \href{https://help.nytimes.com/hc/en-us/articles/115014893968-Terms-of-sale}{Terms
  of Sale}
\item
  \href{https://spiderbites.nytimes.com}{Site Map}
\item
  \href{https://help.nytimes.com/hc/en-us}{Help}
\item
  \href{https://www.nytimes.com/subscription?campaignId=37WXW}{Subscriptions}
\end{itemize}
