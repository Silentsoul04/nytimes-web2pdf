Sections

SEARCH

\protect\hyperlink{site-content}{Skip to
content}\protect\hyperlink{site-index}{Skip to site index}

\href{https://www.nytimes.com/section/world/europe}{Europe}

\href{https://myaccount.nytimes.com/auth/login?response_type=cookie\&client_id=vi}{}

\href{https://www.nytimes.com/section/todayspaper}{Today's Paper}

\href{/section/world/europe}{Europe}\textbar{}Nations Approve Landmark
Climate Accord in Paris

\url{https://nyti.ms/21X15IW}

\begin{itemize}
\item
\item
\item
\item
\item
\item
\end{itemize}

Advertisement

\protect\hyperlink{after-top}{Continue reading the main story}

Supported by

\protect\hyperlink{after-sponsor}{Continue reading the main story}

\hypertarget{nations-approve-landmark-climate-accord-in-paris}{%
\section{Nations Approve Landmark Climate Accord in
Paris}\label{nations-approve-landmark-climate-accord-in-paris}}

\includegraphics{https://static01.nyt.com/images/2015/12/13/science/13climate-obama-hp-sub/13climate-obama-hp-sub-videoSixteenByNine3000.jpg}

By \href{https://www.nytimes.com/by/coral-davenport}{Coral Davenport}

\begin{itemize}
\item
  Dec. 12, 2015
\item
  \begin{itemize}
  \item
  \item
  \item
  \item
  \item
  \item
  \end{itemize}
\end{itemize}

LE BOURGET, France --- With the sudden bang of a gavel Saturday night,
representatives of 195 nations reached a landmark accord that will, for
the first time, commit nearly every country to lowering planet-warming
greenhouse gas emissions to help stave off the most drastic effects of
climate change.

The deal, which was met with an eruption of cheers and ovations from
thousands of delegates gathered from around the world, represents a
historic breakthrough on an issue that has foiled decades of
international efforts to address climate change.

Traditionally, such pacts have required developed economies like the
United States to take action to lower greenhouse gas emissions, but they
have exempted developing countries like China and India from such
obligations.

The accord, which United Nations diplomats have been working toward for
nine years, changes that dynamic by requiring action in some form from
every country, rich or poor.

``This is truly a historic moment,'' the United Nations secretary
general, Ban Ki-moon, said in an interview. ``For the first time, we
have a truly universal agreement on climate change, one of the most
crucial problems on earth.''

\href{https://www.nytimes.com/interactive/2015/12/12/science/What-Does-the-Climate-Deal-Mean.html}{}

\includegraphics{https://static01.nyt.com/images/2015/12/13/science/Climate-Listy-01/Climate-Listy-01-videoLarge.jpg}

\hypertarget{what-does-a-climate-deal-mean-for-the-world}{%
\subsection{What Does a Climate Deal Mean for the
World?}\label{what-does-a-climate-deal-mean-for-the-world}}

A group of 195 nations reached a landmark climate agreement on Saturday.
Here is what it means for the planet, business, politics and other
areas.

President Obama, who regards tackling climate change as a central
element of his legacy, spoke of the deal in a televised address from the
White House. ``This agreement sends a powerful signal that the world is
fully committed to a low-carbon future,'' he said. ``We've shown that
the world has both the will and the ability to take on this challenge.''

Scientists and leaders said the talks here represented the world's last,
best hope of striking a deal that would begin to avert the most
devastating effects of a warming planet.

Mr. Ban said there was ``no Plan B'' if the deal fell apart. The Eiffel
Tower was illuminated with that phrase Friday night.

The new deal will not, on its own, solve global warming. At best,
scientists who have analyzed it say, it will cut global greenhouse gas
emissions by about half enough as is necessary to stave off an increase
in atmospheric temperatures of 2 degrees Celsius or 3.6 degrees
Fahrenheit. That is the point at which, scientific studies have
concluded, the world will be locked into a future of devastating
consequences, including rising sea levels, severe droughts and flooding,
widespread food and water shortages and more destructive storms.

But the Paris deal could represent the moment at which, because of a
shift in global economic policy, the inexorable rise in planet-warming
carbon emissions that started during the Industrial Revolution began to
level out and eventually decline.

At the same time, the deal could be viewed as a signal to global
financial and energy markets, triggering a fundamental shift away from
investment in coal, oil and gas as primary energy sources toward
zero-carbon energy sources like wind, solar and nuclear power.

\href{https://www.nytimes.com/interactive/2015/12/12/world/paris-climate-change-deal-explainer.html}{}

\includegraphics{https://static01.nyt.com/images/2015/12/12/world/paris-climate-change-deal-explainer-1449937527569/paris-climate-change-deal-explainer-1449937527569-videoLarge-v5.jpg}

\hypertarget{inside-the-paris-climate-deal}{%
\subsection{Inside the Paris Climate
Deal}\label{inside-the-paris-climate-deal}}

Highlights from the final draft text of a climate agreement submitted to
the delegates in Paris.

``The world finally has a framework for cooperating on climate change
that's suited to the task,'' said Michael Levi, an expert on energy and
climate change policy at the Council on Foreign Relations. ``Whether or
not this becomes a true turning point for the world, though, depends
critically on how seriously countries follow through.''

Just five years ago, such a deal seemed politically impossible. A
similar 2009 climate change summit meeting in Copenhagen collapsed in
acrimonious failure after countries could not unite around a deal.

Unlike in Copenhagen, Foreign Minister Laurent Fabius of France said on
Saturday, the stars for this assembly were aligned.

The changes that led to the Paris accord came about through a mix of
factors, particularly major shifts in the domestic politics and
bilateral relationships of China and the United States, the world's two
largest greenhouse gas polluters.

Since the Copenhagen deal collapsed, scientific studies have confirmed
that the earliest impacts of climate change have started to sweep across
the planet. While scientists once warned that climate change was a
problem for future generations, recent scientific reports have concluded
that it has started to wreak havoc now, from flooding in Miami to
droughts and water shortages in China.

In a remarkable shift from their previous standoffs over the issue,
senior officials from both the United States and China praised the Paris
accord on Saturday night.

\includegraphics{https://static01.nyt.com/images/2015/12/13/science/13climate-web5/13climate-web5-articleLarge.jpg?quality=75\&auto=webp\&disable=upscale}

Secretary of State John Kerry, who has spent the past year negotiating
behind the scenes with his Chinese and Indian counterparts in order to
help broker the deal, said, ``The world has come together around an
agreement that will empower us to chart a new path for our planet.''

Xie Zhenhua, the senior Chinese climate change negotiator, said, ``The
agreement is not perfect, and there are some areas in need of
improvement.'' But he added, ``This does not prevent us from marching
forward with this historic step.'' Mr. Xie called the deal ``fair and
just, comprehensive and balanced, highly ambitious, enduring and
effective.''

Negotiators from many countries have said that a crucial moment in the
path to the Paris accord came last year in the United States, when Mr.
Obama enacted the nation's first climate change policy --- a set of
stringent new Environmental Protection Agency regulations designed to
slash greenhouse gas pollution from the nation's coal-fired power
plants. Meanwhile, in China, the growing internal criticism over air
pollution from coal-fired power plants led President Xi Jinping to
pursue domestic policies to cut coal use.

In November 2014 in Beijing, Mr. Obama and Mr. Xi announced that they
would jointly pursue plans to cut domestic greenhouse gas emissions.
That breakthrough announcement was seen as paving the way to the Paris
deal, in which nearly all the world's nations have jointly announced
similar plans.

The final language did not fully satisfy everyone. Representatives of
some developing nations expressed consternation. Poorer countries had
pushed for a legally binding provision requiring that rich countries
appropriate a minimum of at least \$100 billion a year to help them
mitigate and adapt to the ravages of climate change. In the final deal,
that \$100 billion figure appears only in a preamble, not in the legally
binding portion of the agreement.

``We've always said that it was important that the \$100 billion was
anchored in the agreement,'' said Tosi Mpanu-Mpanu, a negotiator for the
Democratic Republic of Congo and the incoming leader of a coalition
known as the Least Developed Countries coalition. In the end, though,
they let it go.

Despite the historic nature of the Paris climate accord, its success
still depends heavily on two factors outside the parameter of the deal:
global peer pressure and the actions of future governments.

The core of the Paris deal is a requirement that every nation take part.
Ahead of the Paris talks, governments of 186 nations put forth public
plans detailing how they would cut carbon emissions through 2025 or
2030.

Those plans alone, once enacted, will cut emissions by half the levels
required to stave off the worst effects of global warming. The national
plans vary vastly in scope and ambition --- while every country is
required to put forward a plan, there is no legal requirement dictating
how, or how much, countries should cut emissions.

Thus, the Paris pact has built in a series of legally binding
requirements that countries ratchet up the stringency of their climate
change policies in the future. Countries will be required to reconvene
every five years, starting in 2020, with updated plans that would
tighten their emissions cuts.

Countries will also be legally required to reconvene every five years
starting in 2023 to publicly report on how they are doing in cutting
emissions compared to their plans. They will be legally required to
monitor and report on their emissions levels and reductions, using a
universal accounting system.

That hybrid legal structure was explicitly designed in response to the
political reality in the United States. A deal that would have assigned
legal requirements for countries to cut emissions at specific levels
would need to go before the United States Senate for ratification. That
language would have been dead on arrival in the Republican-controlled
Senate, where many members question the established science of
human-caused climate change, and still more wish to thwart Mr. Obama's
climate change agenda.

\href{https://www.nytimes.com/interactive/2015/12/03/upshot/what-you-can-do-about-climate-change.html}{}

\includegraphics{https://static01.nyt.com/images/2015/12/03/science/-what-you-can-do-about-climate-change-promo-image/-what-you-can-do-about-climate-change-promo-image-videoLarge.png}

\hypertarget{what-you-can-do-about-climate-change}{%
\subsection{What You Can Do About Climate
Change}\label{what-you-can-do-about-climate-change}}

Seven simple guidelines on how your choices today affect the climate
tomorrow.

So the individual countries' plans are voluntary, but the legal
requirements that they publicly monitor, verify and report what they are
doing, as well as publicly put forth updated plans, are designed to
create a ``name-and-shame'' system of global peer pressure, in hopes
that countries will not want to be seen as international laggards.

That system depends heavily on the views of the future world leaders who
will carry out those policies. In the United States, every Republican
candidate running for president in 2016 has publicly questioned or
denied the science of climate change, and has voiced opposition to Mr.
Obama's climate change policies.

In the Senate, Mitch McConnell, the Republican leader, who has led the
charge against Mr. Obama's climate change agenda, said, ``Before his
international partners pop the champagne, they should remember that this
is an unattainable deal based on a domestic energy plan that is likely
illegal, that half the states have sued to halt, and that Congress has
already voted to reject.''

There were few of those concerns at the makeshift negotiations center
here in this suburb north of Paris. The delegates rose to their feet in
applause to thank the French delegation, which drew on the finest
elements of the country's longstanding traditions of diplomacy to broker
a deal that was acceptable to all sides.

France's European partners recalled the coordinated Nov. 13 terrorist
attacks in Paris, which killed 130 people and threatened to cast a
shadow over the negotiations. But, bound by a collective good will
toward France, countries redoubled their efforts.

``This demonstrates the strength of the French nation and makes us
Europeans all proud of the French nation,'' said Miguel Arias Cañete,
the European Union's commissioner for energy and climate action.

Yet amid the spirit of success that dominated the final hours of the
negotiations, Mr. Arias Cañete reminded delegates that the accord was
the beginning of the real work. ``Today, we celebrate,'' he said.
``Tomorrow, we have to act. This is what the world expects of us.''

Advertisement

\protect\hyperlink{after-bottom}{Continue reading the main story}

\hypertarget{site-index}{%
\subsection{Site Index}\label{site-index}}

\hypertarget{site-information-navigation}{%
\subsection{Site Information
Navigation}\label{site-information-navigation}}

\begin{itemize}
\tightlist
\item
  \href{https://help.nytimes.com/hc/en-us/articles/115014792127-Copyright-notice}{©~2020~The
  New York Times Company}
\end{itemize}

\begin{itemize}
\tightlist
\item
  \href{https://www.nytco.com/}{NYTCo}
\item
  \href{https://help.nytimes.com/hc/en-us/articles/115015385887-Contact-Us}{Contact
  Us}
\item
  \href{https://www.nytco.com/careers/}{Work with us}
\item
  \href{https://nytmediakit.com/}{Advertise}
\item
  \href{http://www.tbrandstudio.com/}{T Brand Studio}
\item
  \href{https://www.nytimes.com/privacy/cookie-policy\#how-do-i-manage-trackers}{Your
  Ad Choices}
\item
  \href{https://www.nytimes.com/privacy}{Privacy}
\item
  \href{https://help.nytimes.com/hc/en-us/articles/115014893428-Terms-of-service}{Terms
  of Service}
\item
  \href{https://help.nytimes.com/hc/en-us/articles/115014893968-Terms-of-sale}{Terms
  of Sale}
\item
  \href{https://spiderbites.nytimes.com}{Site Map}
\item
  \href{https://help.nytimes.com/hc/en-us}{Help}
\item
  \href{https://www.nytimes.com/subscription?campaignId=37WXW}{Subscriptions}
\end{itemize}
