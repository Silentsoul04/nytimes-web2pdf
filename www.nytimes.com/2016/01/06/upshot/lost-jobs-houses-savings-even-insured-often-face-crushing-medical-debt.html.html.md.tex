Sections

SEARCH

\protect\hyperlink{site-content}{Skip to
content}\protect\hyperlink{site-index}{Skip to site index}

\href{https://myaccount.nytimes.com/auth/login?response_type=cookie\&client_id=vi}{}

\href{https://www.nytimes.com/section/todayspaper}{Today's Paper}

\href{/section/upshot}{The Upshot}\textbar{}Even Insured Can Face
Crushing Medical Debt, Study Finds

\url{https://nyti.ms/1O82fb2}

\begin{itemize}
\item
\item
\item
\item
\item
\item
\end{itemize}

Advertisement

\protect\hyperlink{after-top}{Continue reading the main story}

Supported by

\protect\hyperlink{after-sponsor}{Continue reading the main story}

Upshot

Public Health

\hypertarget{even-insured-can-face-crushing-medical-debt-study-finds}{%
\section{Even Insured Can Face Crushing Medical Debt, Study
Finds}\label{even-insured-can-face-crushing-medical-debt-study-finds}}

By \href{http://www.nytimes.com/by/margot-sanger-katz}{Margot
Sanger-Katz}

\begin{itemize}
\item
  Jan. 5, 2016
\item
  \begin{itemize}
  \item
  \item
  \item
  \item
  \item
  \item
  \end{itemize}
\end{itemize}

Here is the surest way to enjoy the peace of mind that comes with having
health insurance: Don't get sick.

The number of uninsured Americans has fallen by
\href{https://aspe.hhs.gov/basic-report/health-insurance-coverage-and-affordable-care-act-september-2015}{an
estimated 15 million} since 2013, thanks largely to the Affordable Care
Act. But a new survey, the first detailed study of Americans struggling
with medical bills, shows that insurance often fails as a safety net.
Health plans often require hundreds or thousands of dollars in
out-of-pocket payments --- sums that can create a cascade of financial
troubles for the many households living paycheck to paycheck.

Carrie Cota learned the hard way that health insurance does not
guarantee financial security. Ms. Cota, a 56-year-old travel agent from
Rosamond, Calif., learned she had the autoimmune disease lupus in 2007.
She ran up thousands of dollars in medical and dental bills and ended up
losing her job, and eventually her house.

``I had to move in temporarily with my ex-husband,'' she said in a
recent interview. ``I'm staying with him until I can figure out what to
do.''

In the new poll, conducted by The New York Times and the Kaiser Family
Foundation, roughly 20 percent of people under age 65 with health
insurance nonetheless reported having problems paying their medical
bills over the last year. By comparison, 53 percent of people without
insurance said the same.

These financial vulnerabilities reflect the high costs of health care in
the United States, the most expensive place in the world to get sick.
They also highlight a substantial shift in the nature of health
insurance.
\href{http://kff.org/interactive/premiums-and-worker-contributions/}{Since
the late 1990s}, insurance plans have begun asking their customers to
pay
\href{http://www.nytimes.com/2015/09/23/business/health-insurance-deductibles-outpacing-wage-increases-study-finds.html?_r=0}{an
increasingly greater share} of their bills out of pocket though rising
deductibles and co-payments. The Affordable Care Act, signed by
President Obama in 2010, protected many Americans from very high health
costs by
\href{http://www.nytimes.com/2012/11/21/us/politics/administration-defines-benefits-under-health-law.html}{requiring
insurance plans to be more comprehensive}, but at the same time it
allowed
\href{http://www.nytimes.com/2013/05/28/business/cadillac-tax-health-insurance.html}{or
even encouraged increases} in deductibles.

``We're at a point where there's been slow growth in health care costs
and huge improvements in the numbers of people who have health
insurance,'' said Sara Collins, a vice president at the Commonwealth
Fund, a health research group. ``But there is this underlying trend
towards higher cost sharing that could put increasing numbers of people
at risk for being underinsured.''

Among those who reported having problems paying their bills despite
having insurance, 63 percent said they used up all or most of their
savings; 42 percent took on an extra job or more work hours; 14 percent
moved or took in roommates; and 11 percent turned to charity.

Randy Farris, 58, a factory worker from Conger, Minn., needed a knee
replacement three years ago. His insurance covered 80 percent of the
bill, but he needed to cash in an I.R.A. to pay his \$4,000 share. ``I
haven't been to the doctor since because I don't want any more doctor
bills,'' he said. His wife's retirement savings had been wiped out years
before, he said, when he used them to pay her hospital bills after she
died of cancer.

The health law has led to a
\href{http://www.cdc.gov/nchs/data/nhis/earlyrelease/probs_paying_medical_bills_jan_2011_jun_2014.pdf}{decline
in the number of Americans suffering financial stress} from health
problems, thanks to the new options for receiving coverage, especially
for the poor. But the problem is still widespread, touching roughly a
quarter of Americans under 65, when the insured and uninsured are looked
at together. Americans older than 65 are covered by Medicare, which more
frequently protects people from major financial trouble.

Unlike other polls, which have focused on the ways that insurance
affects health care, the new Times-Kaiser survey explored the effects of
medical bills on people's daily lives well beyond the medical system. We
found that medical bills don't just keep people from
\href{http://www.gallup.com/poll/187190/cost-delays-healthcare-one-three.aspx?g_source=CATEGORY_WELLBEING\&g_medium=topic\&g_campaign=tiles}{filling
prescriptions and scheduling doctors' visits}. They can also prompt deep
financial and personal sacrifices, affecting their housing, employment,
credit and daily lives.
\href{http://kff.org/health-costs/report/the-burden-of-medical-debt-results-from-the-kaiser-family-foundationnew-york-times-medical-bills-survey}{Kaiser
has released a report today}, detailing the survey's main findings about
this population.

``The major impact is actually a pocketbook or economic impact: their
ability to pay the rent or the mortgage or buy food,'' said Drew Altman,
president of the Kaiser Family Foundation.

People without health insurance, of course, are more vulnerable to
medical bills than those with health coverage. The study found that the
people most likely to report bill problems were uninsured, poor or
disabled. But the majority of people struggling with bills are insured.
Of the people in the survey reporting difficulty with their medical
bills, 34 percent lacked health insurance, 39 percent had insurance
through work, 14 percent were covered through public programs and 7
percent had purchased their own health plans.

One reason, many experts said, is a gradual shift in the norms about the
generosity of health insurance. In recent years, health plans have come
with
\href{http://www.nytimes.com/2015/09/23/business/health-insurance-deductibles-outpacing-wage-increases-study-finds.html}{growing
deductibles} and
\href{http://www.nytimes.com/2014/07/30/upshot/why-health-insurance-plans-with-narrow-networks-are-here-to-stay.html}{narrowing
networks of providers}, provisions devised to lower the cost of
premiums. Those features have made health insurance accessible to a
larger share of the population, but may also be leaving more insured
Americans vulnerable.

Ten years ago, David Dranove, a professor of health management at
Northwestern's Kellogg School of Management, conducted research on
people experiencing
\href{http://content.healthaffairs.org/content/25/2/w74.full}{medical
bankruptcies}. The study he co-authored found that bankruptcy was
largely a problem of the uninsured. ``But with more people buying less
generous health insurance, I think the old evidence might no longer be
relevant,'' he said.

Insured people with financial problems often have plans with higher
deductibles. But many said that the smaller co-payments piled up to make
their care unaffordable. Many also received big bills that were not
covered by their insurance. Among the 32 percent of insured patients
stuck with an out-of-network bill, more than than two-thirds of patients
said they didn't know the provider wasn't covered. More than 25 percent
of the insured respondents said a medical claim had been denied.

Medical bill problems rarely occur in a vacuum, the survey found. Most
of the people surveyed said their finances were tight even before there
was an illness in their family. This pattern held true even for families
higher on the income scale. The rates at which people with medical bill
problems sought charity or borrowed money from friends was similar among
people earning less than \$25,000 and those earning more than \$100,000.

Research on medical bankruptcies
\href{http://content.healthaffairs.org/content/25/2/w89.full}{has been
controversial} because it can be hard to untangle how medical bills fit
into a family's overall pattern of financial troubles. Twenty-nine
percent of the people with medical bill problems said a family member
had been forced to stop working or cut back on hours. (On the other
side, about 41 percent of people said they'd taken on extra work to help
pay bills.)

``Is that a job problem or a medical bill problem?'' said David
Himmelstein, a professor of public health at the City University of New
York's Hunter College School of Public Health who has
\href{http://content.healthaffairs.org/content/suppl/2005/01/28/hlthaff.w5.63.DC1}{studied
medical bankruptcies}. ``It's both of those things.''

The survey included a random sample of 1,204 adults under 65 who
reported problems paying household medical bills in the past 12 months.
Interviews were conducted online and by telephone between Aug. 28 and
Sept. 28, and some respondents gave follow-up interviews in December.
The margin of sampling error is plus or minus 4 percentage points.
Information about the poll methodology is available
\href{http://www.nytimes.com/2016/01/06/upshot/how-the-poll-on-medical-bills-was-conducted.html}{here}.

The survey asked people to describe the ways that bills had changed
their lives. The chart above shows some of the most common answers. The
quotations that are displayed throughout this article were entered by
survey respondents when they were asked to describe ``what other
significant changes'' they made in their lives. We'd like to hear how
readers would answer a similar question. If you're struggling with
medical bills, please tell us about how your life has changed in the box
at the top of this article. We may contact you about featuring your
story in the future.

\href{https://www.nytimes.com/slideshow/2015/12/21/upshot/the-weight-of-medical-bills.html}{}

\hypertarget{the-weight-of-medical-bills}{%
\subsection{The Weight of Medical
Bills}\label{the-weight-of-medical-bills}}

7 Photos

View Slide Show ›

\includegraphics{https://static01.nyt.com/images/2015/12/21/upshot/up-healthpoll-slide-3NN1/up-healthpoll-slide-3NN1-articleLarge.jpg?quality=75\&auto=webp\&disable=upscale}

Meggan Haller for The New York Times

Advertisement

\protect\hyperlink{after-bottom}{Continue reading the main story}

\hypertarget{site-index}{%
\subsection{Site Index}\label{site-index}}

\hypertarget{site-information-navigation}{%
\subsection{Site Information
Navigation}\label{site-information-navigation}}

\begin{itemize}
\tightlist
\item
  \href{https://help.nytimes.com/hc/en-us/articles/115014792127-Copyright-notice}{©~2020~The
  New York Times Company}
\end{itemize}

\begin{itemize}
\tightlist
\item
  \href{https://www.nytco.com/}{NYTCo}
\item
  \href{https://help.nytimes.com/hc/en-us/articles/115015385887-Contact-Us}{Contact
  Us}
\item
  \href{https://www.nytco.com/careers/}{Work with us}
\item
  \href{https://nytmediakit.com/}{Advertise}
\item
  \href{http://www.tbrandstudio.com/}{T Brand Studio}
\item
  \href{https://www.nytimes.com/privacy/cookie-policy\#how-do-i-manage-trackers}{Your
  Ad Choices}
\item
  \href{https://www.nytimes.com/privacy}{Privacy}
\item
  \href{https://help.nytimes.com/hc/en-us/articles/115014893428-Terms-of-service}{Terms
  of Service}
\item
  \href{https://help.nytimes.com/hc/en-us/articles/115014893968-Terms-of-sale}{Terms
  of Sale}
\item
  \href{https://spiderbites.nytimes.com}{Site Map}
\item
  \href{https://help.nytimes.com/hc/en-us}{Help}
\item
  \href{https://www.nytimes.com/subscription?campaignId=37WXW}{Subscriptions}
\end{itemize}
