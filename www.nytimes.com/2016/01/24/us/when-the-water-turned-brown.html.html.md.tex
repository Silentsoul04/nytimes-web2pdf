Sections

SEARCH

\protect\hyperlink{site-content}{Skip to
content}\protect\hyperlink{site-index}{Skip to site index}

\href{https://www.nytimes.com/section/us}{U.S.}

\href{https://myaccount.nytimes.com/auth/login?response_type=cookie\&client_id=vi}{}

\href{https://www.nytimes.com/section/todayspaper}{Today's Paper}

\href{/section/us}{U.S.}\textbar{}When the Water Turned Brown

\url{https://nyti.ms/1VhdZMA}

\begin{itemize}
\item
\item
\item
\item
\item
\item
\end{itemize}

Advertisement

\protect\hyperlink{after-top}{Continue reading the main story}

Supported by

\protect\hyperlink{after-sponsor}{Continue reading the main story}

\hypertarget{when-the-water-turned-brown}{%
\section{When the Water Turned
Brown}\label{when-the-water-turned-brown}}

\includegraphics{https://static01.nyt.com/images/2016/01/24/us/24flint-web01/24flint-web01-articleLarge.jpg?quality=75\&auto=webp\&disable=upscale}

By \href{http://www.nytimes.com/by/abby-goodnough}{Abby Goodnough},
\href{http://www.nytimes.com/by/monica-davey}{Monica Davey} and
\href{http://www.nytimes.com/by/mitch-smith}{Mitch Smith}

\begin{itemize}
\item
  Jan. 23, 2016
\item
  \begin{itemize}
  \item
  \item
  \item
  \item
  \item
  \item
  \end{itemize}
\end{itemize}

FLINT, Mich. --- Standing at a microphone in September holding up a baby
bottle, Dr. Mona Hanna-Attisha, a local pediatrician, said she was
deeply worried about the water. The number of Flint children with
elevated levels of lead in their blood had risen alarmingly since the
city changed its water supply the previous year, her analysis showed.

Within hours of Dr. Hanna-Attisha's news conference, Michigan state
officials pushed back --- hard. A Department of Health and Human
Services official
\href{http://www.mlive.com/news/flint/index.ssf/2015/09/state_says_its_data_shows_no_c.html}{said
that the state had not seen similar results} and that it was working
with a much larger set of data. A Department of Environmental Quality
official was quoted as saying the pediatrician's remarks were
``\href{http://www.crainsdetroit.com/article/20150928/NEWS01/150929872/doctors-urge-flint-to-stop-using-water-from-flint-river}{unfortunate},''
described the mood over Flint's water as ``near-hysteria'' and said, as
the authorities had insisted for months, that the water met state and
federal standards.

Dr. Hanna-Attisha said she went home that night feeling shaky and sick,
her heart racing. ``When a state with a team of 50 epidemiologists tells
you you're wrong,'' she said, ``how can you not second-guess yourself?''

No one now argues with Dr. Hanna-Attisha's
\href{http://ajph.aphapublications.org/doi/full/10.2105/AJPH.2015.303003}{findings}.
Not only has she been proved right, but Gov. Rick Snyder publicly
thanked her on Tuesday ``for bringing these issues to light.''

Nearly a year and a half after the city started using water from the
long-polluted Flint River and soon after Dr. Hanna-Attisha's news
conference, the
\href{http://www.nytimes.com/2015/10/08/us/reassurances-end-in-flint-after-months-of-concern.html?_r=0}{authorities
reversed course,} acknowledging that the number of children with high
lead levels in this struggling, industrial city had jumped, and no one
should be drinking unfiltered tap water. Residents had been complaining
about the strange smells and colors pouring from their taps ever since
the switch.

Already this month, federal and state investigations have been
announced, National Guard troops were distributing thousands of bottles
of water and filters, and Mr. Snyder was calling for millions in state
dollars to fix a situation he acknowledged was a ``catastrophe.''

Yet interviews, documents and emails show that as every major decision
was made over more than a year, officials at all levels of government
acted in ways that contributed to the public health emergency and
allowed it to persist for months. The government continued on its
harmful course even after lead levels were found to be rising, and after
pointed, detailed warnings came from a federal water expert, a Virginia
Tech researcher and others.

For more than a year after an emergency manager --- appointed by Mr.
Snyder to oversee the city --- approved a switch from the Detroit system
to water from the Flint River to save money, workers assigned to manage
the city's water system failed to lower lead risks with a simple
solution: adding chemicals to prevent old pipes from corroding and
leaching metals like lead. Disagreements and miscommunication between
state and local officials about what federal law requires of so-called
corrosion control measures further delayed fixing the problem, the
documents show.

\includegraphics{https://static01.nyt.com/images/2016/01/24/us/24flint-web02/24flint-web02-articleLarge.jpg?quality=75\&auto=webp\&disable=upscale}

Image

The softening clarifier room at the Flint water plant in
March.Credit...Joshua Lott for The New York Times

``This could have been nipped in the bud before last summer,'' said
Daniel Giammar, an environmental engineer at Washington University in
St. Louis.

The testing of homes in Flint for lead, too, was insufficient and
flawed, some experts say. Officials failed to focus on the many homes
with lead service lines that were most likely to be tainted, instead
looking at wider problems that would have muted the calls of alarm.

The city authorities also urged, and state regulators allowed, methods
of sampling that experts say had been shown to underestimate lead
levels. Residents were advised, for example, to run their water before
taking samples, a move that tends to flush out concentrations of lead
particles that might have accumulated.

And through it all, officials persisted in playing down and dismissing
the concerns of Flint residents --- one referred to concerned residents
groups as ``anti-everything'' --- and authoritatively vouching for the
water's purity, even as they themselves were debating whether it was
pure.

Three months before Dr. Hanna-Attisha voiced her fears and findings, a
regulations manager for the federal Environmental Protection Agency had
sent a detailed interim report to the state and federal authorities that
included unambiguous warnings like this: ``Recent drinking water sample
results indicate the presence of high lead results in the drinking
water, which is to be expected in a public water system that is not
providing corrosion control treatment.''

It is unclear how many people have had elevated lead levels in their
blood over the last year and a half. The state has identified 233 since
April 2014, but Dr. Hanna-Attisha said its numbers likely ``grossly
underestimate'' exposure, partly because testing was generally limited
to 1- and 2-year-olds until recently. Lead remains traceable in the
blood for only about a month after exposure.

As criticisms have mounted, high-ranking officials have resigned,
including Howard Croft, Flint's director of public works; Dan Wyant, the
state's Environmental Quality director; and
\href{http://www.nytimes.com/2016/01/22/us/flint-fallout-a-resignation-a-hearing-and-us-aid.html}{Susan
Hedman, the E.P.A. regional director}.

Dave Murray, a spokesman for Mr. Snyder, issued a statement on Friday
calling the crisis ``a failure of government --- at the local, state and
federal levels.'' He added that the governor was ``committed to fixing
the problem and addressing the immediate and long-term needs of the
people of Flint.''

Dr. Hanna-Attisha also cited the wholesale failure of government. ``They
had the information,'' she said. ``They just weren't looking closely or
believing it.''

\textbf{Repeated Assurances}

On April 25, 2014, Flint, whose population had dwindled from more than
195,000 in 1960
\href{http://quickfacts.census.gov/qfd/states/26/2629000.html}{to fewer
than 100,000 people}, switched to using the Flint River as its water
supply. The city had drawn water from Detroit's system for decades, but
it was expensive, and so Flint joined efforts to create a new, regional
system that would draw from Lake Huron.

Costs had become a central concern in a city that has lost thousands of
auto industry jobs. Fiscal troubles were so significant that the state
sent an emergency manager --- with ultimate decision-making power --- to
oversee a recovery. Until the new pipeline to Lake Huron was
constructed, the city would take its water from the Flint River, which
it had used as a backup.

City
leaders\href{http://www.nytimes.com/2014/05/26/business/detroit-plan-to-profit-on-water-looks-half-empty.html?_r=0}{toasted
the switch with cups of water}. Residents were less sure. For years the
Flint River had been a dumping ground --- for cars and even bodies.
Aware of the doubts, the city's first news release on the switch
trumpeted state and local officials' assurances.

Then came the odd colors from the tap --- greens and browns --- and the
offensive smells and tastes. Soon there were reports of rashes and
clumps of hair falling out. Parts from a General Motors engine plant
here were corroding, so the company stopped using Flint's water.

Image

Tammy Loren pointed out dry patches on the face of her son Elijah that
she believed were caused by showering in the water.Credit...Brittany
Greeson for The New York Times

Tammy Loren, a mother of four who rents a home, was having a hard time
believing the answers she got about why her sons' skin had itchy rashes.
At various times over the last year and a half, she said, their doctors
diagnosed scabies, ringworm and other fungal infections, but prescribed
medicines never worked. The family even had the home treated by an
exterminator, thinking the problem might be fleas.

``The water was brown, and it had a disgusting smell,'' said Ms. Loren,
whose sons are now 14, 12, 11 and 10. ``It was like dirt coming out.''

For months, Ms. Loren said, she conducted her own research on the
Internet and asked plaintive questions on community Facebook pages. Her
family started drinking bottled water when it could, but Ms. Loren, who
receives federal disability payments for her back and other problems and
relies on food stamps, said it was not that often.

``There was times when we couldn't afford it,'' she said. ``We just kept
drinking out of the tap.''

Through it all, the government reassurances were constant, insistent and
unequivocal. ``It's a quality, safe product,'' Mayor Dayne
Walling\href{http://www.mlive.com/news/flint/index.ssf/2014/06/treated_flint_river_water_meet.html}{told
The Flint Journal} in June 2014.

At points, the city's water tested positive for E. coli bacteria, which
can cause intestinal illness, and residents were advised to boil their
water. City officials pumped extra chlorine into the system to address
the bacteria issue, which led to elevated levels of total
trihalomethanes, or TTHMs, chemical compounds that may cause health
problems after long-term exposure.

A state briefing in February last year acknowledged the TTHM level was
``not `nothing' '' but also not an imminent ``threat to public health.''

In July, Flint
\href{https://www.cityofflint.com/wp-content/uploads/July-2015-Letter-to-Water-Customers.pdf}{sent
residents a letter} saying it was ``pleased to report'' the ``water is
safe.''

But officials' efforts to soothe residents about other contaminants
seemed to overshadow the growing signs of trouble about lead.

By March 2015, with residents
\href{http://www.nytimes.com/2015/03/25/us/a-water-dilemma-in-michigan-cheaper-or-clearer.html}{turning
up at public events bearing bottles of murky water}, the City Council
voted to ``do all things necessary'' to reconnect to Detroit's water
system. But the state-appointed emergency manager, Gerald Ambrose, said
no. He repeated the official mantra: The water meets state and federal
standards. And he noted, once more, that Detroit water was among the
most costly in the state.

``Water from Detroit is no safer than water from Flint,'' Mr. Ambrose
said.

\textbf{Corrosion Control Failure}

Behind the scenes, though, officials seemed far less sure.

By the end of February, Miguel Del Toral, the E.P.A. regulations manager
who had learned of high lead content in one Flint resident's water, was
raising a fundamental question with his state and federal colleagues:
What was Flint using to treat the river water to avoid corrosion?

``They are required to have O.C.C.T. in place which is why I was asking
what they were using,'' he wrote in an email on Feb. 27, using the
initials for ``optimal corrosion control treatment.''

Surely, the assumption was, the city was adding a chemical to the water
to coat its aging pipes and prevent corrosion, since controlling
corrosion is required by a federal rule governing lead and copper. The
water that Flint had drawn for years via Detroit from Lake Huron had
been treated with orthophosphate, a common anti-corrosion additive. And
Flint River water is naturally even harder and more corrosive, experts
say, than the water the city was buying from Detroit.

An official from Michigan's Department of Environmental Quality answered
Mr. Del Toral's inquiry the same day: Flint has ``an optimized corrosion
control program.'' But less than two months later, the state said it had
been wrong. There actually was no treatment in place in Flint to stop
corrosion, a timeline of events provided by the state now shows.

Image

A filter that shows signs of corrosion, lead, and copper particles in
Flint.Credit...Laura McDermott for The New York Times

Image

The long-polluted Flint River, from which the city began drawing water
in 2014 to save money.Credit...Laura McDermott for The New York Times

The authorities themselves did not agree on what the federal rules
meant. Some state officials believed that testing needed to be done over
a year before a new plan could be put in place to block corrosion,
documents suggest, while other officials thought the treatment with
chemicals needed to start the moment Flint began receiving water from
the river.

``We made a mistake,'' Mr. Wyant, then the state's environmental quality
director, said in October. Corrosion controls, he said, ``should have
been required from the beginning.''

The lead issues should have been anticipated long before the city
switched water supplies, experts said. ``I think that's pretty obvious,
in going from having a corrosion inhibitor to not having one, you might
have expected to have increased corrosion,'' said Professor Giammar.

By June, Mr. Del Toral wrote in a memo to state and federal colleagues
that Flint had essentially stopped providing treatment used to mitigate
lead and copper levels in drinking water, which he called a ``major
concern from a public health standpoint.''

E.P.A. officials contend that they pressed Michigan regulators to take
more decisive action after Mr. Del Toral's report, but for months
federal officials did little to inform the public of those findings or
take decisive action. It was not until Thursday that the federal agency
issued an emergency order and assumed oversight of lead testing in
Flint.

\textbf{Flaws in Testing}

All along, Flint's water was being tested for lead.

Yet when health officials studied tests showing higher levels of lead in
children's blood in the summer of 2014, they suggested that the
increases were
\href{http://www.mlive.com/news/flint/index.ssf/2015/09/state_says_its_data_shows_no_c.html}{a
result of ordinary seasonal fluctuations}. Water samples, too, showed
rising levels of lead in the first half of 2015 compared with late 2014,
and a Flint Journal data analysis concluded that
\href{http://www.mlive.com/news/flint/index.ssf/2015/09/flint_water_lead_levels_spiked.html}{they
were at their highest in 20 years}.

There was so much lead found in water at the home of LeeAnne Walters
that officials shut her water off in April and temporarily installed a
garden hose to carry water from a neighbor's house. Still, state
officials noted that the city's levels remained within federal and state
standards.

But the water tests themselves were flawed, experts say.

According to the American Civil Liberties Union of Michigan, which
conducted its own investigation, \href{http://flintwaterstudy.org/}{as
did researchers at Virginia Tech}, the city was not only advising
residents to run their water before collecting a sample, but doing other
things to ``skew the outcome of its tests to produce favorable
results.'' For example, the A.C.L.U. reported in September, the city
retested water from homes found to have low lead levels, but not from
homes whose initial levels were high.

Image

Specialist John Rhodes of the Michigan National Guard helped a resident
carry bottled water to her car in Flint.Credit...Brittany Greeson for
The New York Times

The city also appeared to be unsure which houses had lead service lines
connecting them to its water distribution system, the report said.
Federal law requires cities testing for lead in drinking water to focus
on homes with the highest risk for contamination, but the report found
no evidence Flint had done so.

Dr. Hanna-Attisha said that after she shared her methodology with the
state, it replicated her findings. Mr. Snyder then announced that the
state would provide filters and test tap water.

Marc Edwards, the Virginia Tech professor who helped identify and expose
Flint's lead problem, said the state ``had no sense of urgency at all,
nor did E.P.A.''

Ms. Loren, the mother of four, said her sons' skin remained irritated,
and she is worrying obsessively about their lead levels, particularly
that of her 11-year-old, who has learning disabilities.

``My trust in everybody is completely gone, out the door,'' she said.
``We've been lied to so much, and these aren't little white lies. These
lies are affecting our kids for the rest of their lives, and it breaks
my heart.''

Advertisement

\protect\hyperlink{after-bottom}{Continue reading the main story}

\hypertarget{site-index}{%
\subsection{Site Index}\label{site-index}}

\hypertarget{site-information-navigation}{%
\subsection{Site Information
Navigation}\label{site-information-navigation}}

\begin{itemize}
\tightlist
\item
  \href{https://help.nytimes.com/hc/en-us/articles/115014792127-Copyright-notice}{©~2020~The
  New York Times Company}
\end{itemize}

\begin{itemize}
\tightlist
\item
  \href{https://www.nytco.com/}{NYTCo}
\item
  \href{https://help.nytimes.com/hc/en-us/articles/115015385887-Contact-Us}{Contact
  Us}
\item
  \href{https://www.nytco.com/careers/}{Work with us}
\item
  \href{https://nytmediakit.com/}{Advertise}
\item
  \href{http://www.tbrandstudio.com/}{T Brand Studio}
\item
  \href{https://www.nytimes.com/privacy/cookie-policy\#how-do-i-manage-trackers}{Your
  Ad Choices}
\item
  \href{https://www.nytimes.com/privacy}{Privacy}
\item
  \href{https://help.nytimes.com/hc/en-us/articles/115014893428-Terms-of-service}{Terms
  of Service}
\item
  \href{https://help.nytimes.com/hc/en-us/articles/115014893968-Terms-of-sale}{Terms
  of Sale}
\item
  \href{https://spiderbites.nytimes.com}{Site Map}
\item
  \href{https://help.nytimes.com/hc/en-us}{Help}
\item
  \href{https://www.nytimes.com/subscription?campaignId=37WXW}{Subscriptions}
\end{itemize}
