Sections

SEARCH

\protect\hyperlink{site-content}{Skip to
content}\protect\hyperlink{site-index}{Skip to site index}

\href{https://www.nytimes.com/section/world/asia}{Asia Pacific}

\href{https://myaccount.nytimes.com/auth/login?response_type=cookie\&client_id=vi}{}

\href{https://www.nytimes.com/section/todayspaper}{Today's Paper}

\href{/section/world/asia}{Asia Pacific}\textbar{}South Korea to Shut
Joint Factory Park, Kaesong, Over Nuclear Test and Rocket

\url{https://nyti.ms/1RoFDbK}

\begin{itemize}
\item
\item
\item
\item
\item
\item
\end{itemize}

Advertisement

\protect\hyperlink{after-top}{Continue reading the main story}

Supported by

\protect\hyperlink{after-sponsor}{Continue reading the main story}

\hypertarget{south-korea-to-shut-joint-factory-park-kaesong-over-nuclear-test-and-rocket}{%
\section{South Korea to Shut Joint Factory Park, Kaesong, Over Nuclear
Test and
Rocket}\label{south-korea-to-shut-joint-factory-park-kaesong-over-nuclear-test-and-rocket}}

\includegraphics{https://static01.nyt.com/images/2016/02/11/world/11Korea-web1/11Korea-web1-articleLarge.jpg?quality=75\&auto=webp\&disable=upscale}

By \href{http://www.nytimes.com/by/choe-sang-hun}{Choe Sang-Hun}

\begin{itemize}
\item
  Feb. 10, 2016
\item
  \begin{itemize}
  \item
  \item
  \item
  \item
  \item
  \item
  \end{itemize}
\end{itemize}

SEOUL, South Korea --- South Korea said on Wednesday that it would shut
down an industrial complex that it runs jointly with North Korea, its
strongest retaliation yet for
\href{http://www.nytimes.com/2016/01/07/world/asia/north-korea-hydrogen-bomb-q-a.html}{the
North's recent nuclear test} and its
\href{http://www.nytimes.com/2016/02/07/world/asia/north-korea-moves-up-rocket-launching-plan.html}{launching
of a long-range rocket} over the weekend.

In announcing the decision, Unification Minister Hong Yong-pyo said the
industrial complex in the North Korean border town of Kaesong, which
went into operation in 2004, had wound up providing funds for the
North's weapons programs.

Mr. Hong, the South's point man for negotiations on the North, said
South Korea had informed Pyongyang of its decision and asked it to help
the 184 South Korean factory managers at the complex to cross the border
safely and return home.

Although the Kaesong complex was
\href{http://www.nytimes.com/2013/04/09/world/asia/north-korea.html}{temporarily
shut down in 2013}, it was the North that initiated the closing, by
pulling out its workers to protest joint South Korean-American military
drills. The South responded by withdrawing its managers.

Wednesday's action was the first time the South unilaterally closed the
complex. And by officially identifying the factory park as a source of
cash for the North's nuclear program, the South seemed to indicate that
the shutdown was permanent.

There was no immediate public response from North Korea.

The South's announcement came as the United States and its allies were
trying to persuade the United Nations Security Council to impose
stronger sanctions against the North. South Korea and the United States
have also said they will impose unilateral sanctions.

Image

Hong Yong-pyo, the South Korean unification minister,~on
Wednesday.Credit...Kim Jin-A/NEWSIS, via Associated Press

``We cannot stop North Korea's nuclear and missile programs with the
existing methods of response,'' Mr. Hong said in a nationally televised
statement. ``We need to act strongly together with the international
community to ensure that North Korea pay a price, and we need to take
special actions to leave the North with no option but to give up its
nuclear program and change.''

Mr. Hong said the Kaesong factory park had been an important source of
cash for the North Korean government. He said North Korea had earned
more than \$560 million in wages for its workers there, including \$120
million last year. Businesses and the government from the South have
also invested \$852 million in factories, roads and other facilities
there.

``In the end, it appears that the money was used not for the peace the
international community wanted but to advance the North's nuclear
weapons and long-range missiles,'' he said.

\includegraphics{https://static01.nyt.com/images/2016/02/11/world/11Korea-video/11Korea-video-videoSixteenByNine1050.jpg}

On Jan. 6,
\href{http://www.nytimes.com/2016/01/06/world/asia/north-korea-hydrogen-bomb-test.html}{North
Korea conducted its fourth nuclear test}, which it claimed was of a
hydrogen bomb, and on Sunday, it launched a satellite into orbit using a
long-range rocket. Both actions were in defiance of Security Council
resolutions banning the North from pursuing nuclear arms or ballistic
missile technology.

The conservative South Korean government of President Park Geun-hye has
responded by resuming propaganda broadcasts along the border and,
despite China's opposition, agreeing to negotiate the deployment of an
American missile defense system on its territory.

But Ms. Park's conservative supporters have been urging her to go
further by shutting down Kaesong, the only industrial complex in North
Korea that is jointly operated with investors from outside the country.

The Kaesong factory park began more than a decade ago as an experiment:
combining South Korean manufacturing skills with cheap North Korea
labor. It was the most important of a number of cooperative projects
begun during a period of reconciliation between the Koreas, when the
South hoped that increased economic exchanges would help ease mutual
mistrust and, eventually, lead to the reunification of the divided
Korean Peninsula.

But most of those projects, like a joint tourism venture at Diamond
Mountain in southeast North Korea, have been abandoned during the past
eight years, as the North's continuing nuclear arms development and
occasional armed provocations, like the
\href{http://www.nytimes.com/2010/11/24/world/asia/24korea.html}{shelling
of a South Korean island} in 2010, have soured many South Koreans on the
prospect of better relations.

Kaesong was the last of those joint projects still functioning and the
most important symbol of inter-Korean good will. Streams of cars and
trucks going to and from the complex crossed the otherwise tightly
sealed border daily, carrying South Korean factory managers into the
North and manufactured goods into the South. More than 45,000 North
Koreans worked for 123 South Korean-owned factories at Kaesong last
year, producing more than \$515 million worth of textiles, electronic
parts and other labor-intensive goods, according to the South Korean
government.

But the wages, paid in American dollars, did not go directly to the
North Korean workers. Instead, the Pyongyang government took the bulk of
the cash, with the workers getting just a small fraction of their wages
in the local currency, according to South Korean officials here.
Conservative South Koreans and some American policy makers have long
feared that proceeds from Kaesong have benefited the North's nuclear
arms program.

Until now, Kaesong had continued to operate despite such misgivings. Its
operations
\href{http://www.nytimes.com/2013/05/04/world/asia/last-south-koreans-leave-industrial-park-in-north.html}{were
suspended for five months} after the North's third nuclear test in 2013,
but they eventually resumed, with the Koreas agreeing to ensure that the
complex would not be affected by ``political situations under any
circumstance.''

Mr. Hong, the unification minister, said on Wednesday that the South had
needed to take drastic action. He said the North's nuclear ambitions, if
left unchecked, could set off a ``nuclear domino effect'' in the region,
with other countries pursuing their own arms programs in response to the
North's.

South Koreans who have argued for keeping Kaesong open said that cutting
off trade with the North would only weaken Seoul's economic leverage and
push Pyongyang closer to China. South Korea was once a major trading
partner of the North, but almost all of the isolated country's trade now
goes through China, which has resisted appeals from Seoul and Washington
to use its influence to curb the North's nuclear ambitions.

Cheong Seong-chang, a senior analyst at the Sejong Institute in South
Korea, said the shutdown of Kaesong was ``the worst possible option''
for the South. Economically, he said, it would do the North less harm
than Seoul hoped, because the North could earn more cash by sending the
same skilled North Korean workers to China. ``When you look at the South
Korean government's policies since the North's nuclear test, you cannot
help thinking that it is reacting emotionally,'' Mr. Cheong said.

The Kaesong complex had been closed since Sunday for the Lunar New Year
holiday; the South's announcement means it will not reopen on Thursday
as planned. Most of the roughly 500 South Korean managers based at the
complex are home for the holiday, but 184 are still in Kaesong, South
Korean officials said.

Advertisement

\protect\hyperlink{after-bottom}{Continue reading the main story}

\hypertarget{site-index}{%
\subsection{Site Index}\label{site-index}}

\hypertarget{site-information-navigation}{%
\subsection{Site Information
Navigation}\label{site-information-navigation}}

\begin{itemize}
\tightlist
\item
  \href{https://help.nytimes.com/hc/en-us/articles/115014792127-Copyright-notice}{©~2020~The
  New York Times Company}
\end{itemize}

\begin{itemize}
\tightlist
\item
  \href{https://www.nytco.com/}{NYTCo}
\item
  \href{https://help.nytimes.com/hc/en-us/articles/115015385887-Contact-Us}{Contact
  Us}
\item
  \href{https://www.nytco.com/careers/}{Work with us}
\item
  \href{https://nytmediakit.com/}{Advertise}
\item
  \href{http://www.tbrandstudio.com/}{T Brand Studio}
\item
  \href{https://www.nytimes.com/privacy/cookie-policy\#how-do-i-manage-trackers}{Your
  Ad Choices}
\item
  \href{https://www.nytimes.com/privacy}{Privacy}
\item
  \href{https://help.nytimes.com/hc/en-us/articles/115014893428-Terms-of-service}{Terms
  of Service}
\item
  \href{https://help.nytimes.com/hc/en-us/articles/115014893968-Terms-of-sale}{Terms
  of Sale}
\item
  \href{https://spiderbites.nytimes.com}{Site Map}
\item
  \href{https://help.nytimes.com/hc/en-us}{Help}
\item
  \href{https://www.nytimes.com/subscription?campaignId=37WXW}{Subscriptions}
\end{itemize}
