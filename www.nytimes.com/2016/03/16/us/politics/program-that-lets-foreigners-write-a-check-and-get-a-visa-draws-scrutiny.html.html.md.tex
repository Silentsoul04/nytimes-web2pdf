Sections

SEARCH

\protect\hyperlink{site-content}{Skip to
content}\protect\hyperlink{site-index}{Skip to site index}

\href{https://www.nytimes.com/section/politics}{Politics}

\href{https://myaccount.nytimes.com/auth/login?response_type=cookie\&client_id=vi}{}

\href{https://www.nytimes.com/section/todayspaper}{Today's Paper}

\href{/section/politics}{Politics}\textbar{}Program That Lets Foreigners
Write a Check, and Get a Visa, Draws Scrutiny

\url{https://nyti.ms/1UvlYrb}

\begin{itemize}
\item
\item
\item
\item
\item
\end{itemize}

Advertisement

\protect\hyperlink{after-top}{Continue reading the main story}

Supported by

\protect\hyperlink{after-sponsor}{Continue reading the main story}

\hypertarget{program-that-lets-foreigners-write-a-check-and-get-a-visa-draws-scrutiny}{%
\section{Program That Lets Foreigners Write a Check, and Get a Visa,
Draws
Scrutiny}\label{program-that-lets-foreigners-write-a-check-and-get-a-visa-draws-scrutiny}}

\includegraphics{https://static01.nyt.com/images/2016/03/16/us/16visa/16visa-articleLarge.jpg?quality=75\&auto=webp\&disable=upscale}

By \href{http://www.nytimes.com/by/ron-nixon}{Ron Nixon}

\begin{itemize}
\item
  March 15, 2016
\item
  \begin{itemize}
  \item
  \item
  \item
  \item
  \item
  \end{itemize}
\end{itemize}

WASHINGTON --- The easiest way to gain entry into the United States is
not to walk across the border in the dead of night. It is to write a
check.

A visa process enacted by Congress in 1990 to create jobs and pump
billions of dollars into the economy has evolved into a program that
federal investigators and some prominent lawmakers say has become a risk
to national security and an easy mark for abuse, particularly from
China.

The program, called
\href{http://www.nytimes.com/2015/05/17/realestate/want-a-green-card-invest-in-real-estate.html}{EB-5},
allows wealthy foreign investors, for a price ranging from \$500,000 to
more than \$1 million, to put themselves on a path to United States
citizenship. The money must be used to finance a business in this
country and eventually employ --- directly or indirectly --- at least 10
American workers in economically depressed areas.

But EB-5 has been the subject of increasing scrutiny since investigators
uncovered numerous cases of fraud, discovered individuals with possible
ties to Chinese and Iranian intelligence using fake documents and
learned that international fugitives who have laundered money had
infiltrated the program.

``It's no secret that the program has long been riddled with corruption
and national security vulnerabilities,'' said Senator Charles E.
Grassley, Republican of Iowa and a frequent critic of the program.

A number of Democrats echo his criticism, in large part because while
most visa applicants must meet education or work requirements, the
primary requirement for the EB-5 program is a ``lawful source of
investment income,'' one Department of Homeland Security memo said.

``I don't believe that America should be selling visas and eventually
citizenship,'' said Senator Dianne Feinstein, Democrat of California,
who wants to terminate a part of the program that allows foreign
applicants to invest through regional development centers that pool
investor money. ``The right to immigrate should not be for sale.''

The Department of Homeland Security, led by Secretary Jeh Johnson, said
it was taking steps to address the issues raised by senators like Mr.
Grassley and Ms. Feinstein. ``The secretary intends to do all he can
within his legal authority to do so,'' said Marsha Catron, a spokeswoman
for the department.

The foreign investor visa has defenders. One is Senator Chuck Schumer,
Democrat of New York, along with some in the Obama administration who
say it has delivered billions of dollars into the American economy:
\$8.7 billion and 35,140 jobs since Oct. 12. But federal auditors have
found that in many cases, those numbers are ``not valid and reliable.''

Supporters of the EB-5 program, including real estate developers, spent
as much as \$3 million to help defeat legislation sponsored by Mr.
Grassley and Senator Patrick J. Leahy, Democrat of Vermont, intended to
address fraud and national security concerns.

Mr. Schumer, who opposed the Grassley-Leahy legislation, said he
supported national security and fraud reforms to the EB-5 program.

New criticism of the program surfaced recently when a former federal
investigator, Taylor Johnson, a special agent with Immigration and
Customs Enforcement, said she was fired after raising questions about
the vetting of individuals involved in a development project in Las
Vegas. She filed a complaint with the Merit Systems Protection Board, a
quasi-judicial agency that protects whistle-blowers, saying she was
fired because she exposed national security concerns. Immigration and
Customs Enforcement said Ms. Johnson's termination was unrelated to her
EB-5 investigation.

But her accusations have prompted investigations by the Office of
Special Counsel, an independent agency that protects federal employees
from reprisal, and the Department of Homeland Security's Office of the
Inspector General.

Foreign investors can gain green cards by investing \$1 million in a new
business or \$500,000 through one of nearly 800 regional development
centers across the country that pool EB-5 money and are certified by the
federal government. Most EB-5 visa seekers --- about 95 percent ---
invest through these regional centers, which are largely unregulated. In
some cases, the investors can also gain citizenship.

The program has grown rapidly, to nearly 9,000 conditional visas last
year, of which 80 percent were issued to Chinese investors, from 64 EB-5
visas in 2003. Investigators have found that security risks have risen
rapidly with the growth.

A Government Accountability Office report released in August found that
the agency could not be sure that money used for the visas was not
coming from ``the drug trade, human trafficking or other criminal
activities.''

Officials at Homeland Security Investigations, a division of Immigration
and Customs Enforcement, said they were concerned that those who prepare
overseas documents ``may try to use increasingly sophisticated methods
to circumvent'' the program. In a 2013 memo, the agency suggested that
the EB-5 regional center program end because ``there are no safeguards
that can be put in place that will ensure the integrity'' of the
regional center model.

An internal review by the fraud detection office at
\href{http://topics.nytimes.com/top/reference/timestopics/organizations/c/citizenship_and_immigration_services_us/index.html?inline=nyt-org}{United
States Citizenship and Immigration Services} found numerous fraudulent
documents when it conducted a random sampling of pending visa
applications. Officials at the agency said they did not have the
authority to shut down a regional center that has received money from
foreign investors solely because of possible criminal or national
security concerns.

Court records and law enforcement documents show that several
individuals with questionable backgrounds have used the program to
launder money and gain entry to the United States.

Last year, the law enforcement authorities
\href{http://www.nytimes.com/2015/05/16/world/asia/china-hunts-fugitives-accused-of-corruption-many-in-us.html}{arrested
a Chinese national}, Zhao Shilan, who they say obtained a visa using
money stolen from a grain storage house in China, where her husband,
Qiao Jianjun, was director. According to court records, the couple, who
had divorced in China, said they were still married. Over a period of
months, they sent money stolen from the storage facilities to banks in
Canada and Hong Kong. Mr. Qiao then submitted false immigration and
financial documents to immigration officials. The couple later used the
money they stole to buy property in Washington State, including a
four-bedroom home worth nearly \$700,000, according to court records.

Ms. Zhao pleaded not guilty last year. Mr. Qiao remains a fugitive and
is listed on
\href{http://www.interpol.int/notice/search/wanted}{Interpol's most
wanted list}.

A growing concern among United States intelligence and law enforcement
officials is that foreign government agents might be trying to
infiltrate the program to conduct economic espionage or gain access to
technology that is banned from export.

A 2013 Homeland Security investigation found that an individual involved
in the EB-5 program who was later arrested in connection with exporting
electronics to Iran had ties to Iranian intelligence operatives who
might try to abuse the programs to enter the United States.

Officials from Citizenship and Immigration Services acknowledged that
the EB-5 program has had problems, but they said the agency had made a
number of changes to address them, including shutting down 35 troubled
centers since 2014.

And Homeland Security officials said they were being more proactive in
tracing the sources of foreign investor income, including establishing
working relationships with Chinese government officials and conducting
overseas visits to verify applicants' sources of income.

Mr. Grassley, the chairman of the Senate Judiciary Committee, said he
was not convinced the changes were enough.

Advertisement

\protect\hyperlink{after-bottom}{Continue reading the main story}

\hypertarget{site-index}{%
\subsection{Site Index}\label{site-index}}

\hypertarget{site-information-navigation}{%
\subsection{Site Information
Navigation}\label{site-information-navigation}}

\begin{itemize}
\tightlist
\item
  \href{https://help.nytimes.com/hc/en-us/articles/115014792127-Copyright-notice}{©~2020~The
  New York Times Company}
\end{itemize}

\begin{itemize}
\tightlist
\item
  \href{https://www.nytco.com/}{NYTCo}
\item
  \href{https://help.nytimes.com/hc/en-us/articles/115015385887-Contact-Us}{Contact
  Us}
\item
  \href{https://www.nytco.com/careers/}{Work with us}
\item
  \href{https://nytmediakit.com/}{Advertise}
\item
  \href{http://www.tbrandstudio.com/}{T Brand Studio}
\item
  \href{https://www.nytimes.com/privacy/cookie-policy\#how-do-i-manage-trackers}{Your
  Ad Choices}
\item
  \href{https://www.nytimes.com/privacy}{Privacy}
\item
  \href{https://help.nytimes.com/hc/en-us/articles/115014893428-Terms-of-service}{Terms
  of Service}
\item
  \href{https://help.nytimes.com/hc/en-us/articles/115014893968-Terms-of-sale}{Terms
  of Sale}
\item
  \href{https://spiderbites.nytimes.com}{Site Map}
\item
  \href{https://help.nytimes.com/hc/en-us}{Help}
\item
  \href{https://www.nytimes.com/subscription?campaignId=37WXW}{Subscriptions}
\end{itemize}
