Sections

SEARCH

\protect\hyperlink{site-content}{Skip to
content}\protect\hyperlink{site-index}{Skip to site index}

\href{https://www.nytimes.com/section/politics}{Politics}

\href{https://myaccount.nytimes.com/auth/login?response_type=cookie\&client_id=vi}{}

\href{https://www.nytimes.com/section/todayspaper}{Today's Paper}

\href{/section/politics}{Politics}\textbar{}Transcript: Donald Trump
Expounds on His Foreign Policy Views

\url{https://nyti.ms/1VO7a89}

\begin{itemize}
\item
\item
\item
\item
\item
\end{itemize}

Advertisement

\protect\hyperlink{after-top}{Continue reading the main story}

Supported by

\protect\hyperlink{after-sponsor}{Continue reading the main story}

\hypertarget{transcript-donald-trump-expounds-on-his-foreign-policy-views}{%
\section{Transcript: Donald Trump Expounds on His Foreign Policy
Views}\label{transcript-donald-trump-expounds-on-his-foreign-policy-views}}

\includegraphics{https://static01.nyt.com/images/2016/03/27/us/27transcript-web/27transcript-web-articleLarge.jpg?quality=75\&auto=webp\&disable=upscale}

March 26, 2016

\begin{itemize}
\item
\item
\item
\item
\item
\end{itemize}

\emph{Over two telephone conversations on Friday,}
\href{http://www.nytimes.com/2016/03/27/us/politics/donald-trump-foreign-policy.html}{\emph{Donald
J. Trump}}\emph{, the Republican presidential candidate, discussed}
\href{http://www.nytimes.com/2016/03/27/us/politics/donald-trump-foreign-policy.html}{\emph{his
views on foreign policy}} \emph{with Maggie Haberman and David E. Sanger
of The New York Times. Here is an edited transcript of their interview
(or}
\href{http://www.nytimes.com/2016/03/27/us/politics/donald-trump-interview-highlights.html}{\emph{just
the highlights}}\emph{).}

\textbf{HABERMAN:} I wanted to ask you about some things that you said
in Washington on Monday, more recently. But you've talked about them a
bunch. So, you have said on several occasions that you want Japan and
South Korea to pay more for their own defense. You've been saying
versions of that about Japan for 30 years. Would you object if they got
their own nuclear arsenal, given the threat that they face from North
Korea and China?

\textbf{TRUMP:} Well, you know, at some point, there is going to be a
point at which we just can't do this anymore. And, I know the upsides
and the downsides. But right now we're protecting, we're basically
protecting Japan, and we are, every time North Korea raises its head,
you know, we get calls from Japan and we get calls from everybody else,
and ``Do something.'' And there'll be a point at which we're just not
going to be able to do it anymore. Now, does that mean nuclear? It could
mean nuclear. It's a very scary nuclear world. Biggest problem, to me,
in the world, is nuclear, and proliferation. At the same time, you know,
we're a country that doesn't have money. You know, when we did these
deals, we were a rich country. We're not a rich country. We were a rich
country with a very strong military and tremendous capability in so many
ways. We're not anymore. We have a military that's severely depleted. We
have nuclear arsenals which are in very terrible shape. They don't even
know if they work. We're not the same country, Maggie and David, I mean,
I think you would both agree.

\textbf{SANGER:} So, just to follow Maggie's thought there, though, the
Japanese view has always been, if the United States, at any point, felt
as if it was uncomfortable defending them, there has always been a
segment of Japanese society, and of Korean society that said, ``Well,
maybe we should have our own nuclear deterrent, because if the U.S.
isn't certain, we need to make sure the North Koreans know that.'' Is
that a reasonable position. Do you think at some point they should have
their own arsenal?

\textbf{TRUMP:} Well, it's a position that we have to talk about, and
it's a position that at some point is something that \emph{we} have to
talk about, and if the United States keeps on its path, its current path
of weakness, they're going to want to have that anyway with or without
me discussing it, because I don't think they feel very secure in what's
going on with our country, David. You know, if you look at how we backed
our enemies, it hasn't -- how we backed our allies -- it hasn't exactly
been strong. When you look at various places throughout the world, it
hasn't been very strong. And I just don't think we're viewed the same
way that we were 20 or 25 years ago, or 30 years ago. And, you know, I
think it's a problem. You know, something like that, unless we get very
strong, very powerful and very rich, quickly, I'm sure those things are
being discussed over there anyway without our discussion.

\textbf{HABERMAN:} Will you --

\textbf{SANGER:} And would you have an objection to it?

\textbf{TRUMP:} Um, at some point, we cannot be the policeman of the
world. And unfortunately, we have a nuclear world now. And you have,
Pakistan has them. You have, probably, North Korea has them. I mean,
they don't have delivery yet, but you know, probably, I mean to me,
that's a big problem. And, would I rather have North Korea have them
with Japan sitting there having them also? You may very well be better
off if that's the case. In other words, where Japan is defending itself
against North Korea, which is a real problem. You very well may have a
better case right there. We certainly haven't been able to do much with
him and with North Korea. But you may very well have a better case. You
know, one of the things with the, with our Japanese relationship, and
I'm a big fan of Japan, by the way. I have many, many friends there. I
do business with Japan. But, that, if we are attacked, they don't have
to do anything. If they're attacked, we have to go out with full force.
You understand. That's a pretty one-sided agreement, right there. In
other words, if we're attacked, they do not have to come to our defense,
if they're attacked, we have to come totally to their defense. And that
is a, that's a real problem.

\hypertarget{nuclear-weapons-cyberwarfare-and-spying-on-allies}{%
\subsection{Nuclear Weapons, Cyberwarfare and Spying on
Allies}\label{nuclear-weapons-cyberwarfare-and-spying-on-allies}}

\textbf{HABERMAN:} Would you, you were just talking about the nuclear
world we live in, and you've said many times, and I've heard you say it
throughout the campaign, that you want the U.S. to be more
unpredictable. Would you be willing to have the U.S. be the first to use
nuclear weapons in a confrontation with adversaries?

\textbf{TRUMP:} An absolute last step. I think it's the biggest, I
personally think it's the biggest problem the world has, nuclear
capability. I think it's the single biggest problem. When people talk
global warming, I say the global warming that we have to be careful of
is the nuclear global warming. Single biggest problem that the world
has. Power of weaponry today is beyond anything ever thought of, or
even, you know, it's unthinkable, the power. You look at Hiroshima and
you can multiply that times many, many times, is what you have today.
And to me, it's the single biggest, it's the single biggest problem.

\textbf{SANGER:} You know, we have an alternative these days in a
growing cyberarsenal. You've seen the growing cybercommand and so forth.
Could you give us a vision of whether or not you think that the United
States should regularly be using cyberweapons, perhaps, as an
alternative to nuclear? And if so, how would you either threaten or
employ those?

\textbf{TRUMP:} I don't see it as an alternative to nuclear in terms of,
in terms of ultimate power. Look, in the perfect world everybody would
agree that nuclear would, you know, be so destructive, and this was
always the theory, or was certainly the theory of many. That the power
is so enormous that nobody would ever use them. But, as you know, we're
dealing with people in the world today that would use them, O.K.?
Possibly numerous people that use them, and use them without hesitation
if they had them. And there's nothing, there's nothing as, there's
nothing as meaningful or as powerful as that, and you know the problem
is, and it used to be, and you would hear this, David, and I would hear
it, and everybody would hear it, and --- I'm not sure I believed it,
ever. I talk sometimes about my uncle from M.I.T., and he would tell me
many years ago when he was up at M.I.T. as a, he was a professor, he was
a great guy in many respects, but a very brilliant guy, and he would
tell me many years ago about the power of weapons someday, that the
destructive force of these weapons would be so massive, that it's going
to be a scary world. And, you know, we have been under the impression
that, well we've been, I think it's misguided somewhat, I've always felt
this but that nobody would ever use them because of the power. And the
first one to use them, I think that would be a very bad thing. And I
will tell you, I would very much not want to be the first one to use
them, that I can say.

\textbf{HABERMAN:} O.K.

\textbf{SANGER:} The question was about cyber, how would you envision
using cyberweapons? Cyberweapons in an attack to take out a power grid
in a city, so forth.

\textbf{TRUMP:} First off, we're so obsolete in cyber. We're the ones
that sort of were very much involved with the creation, but we're so
obsolete, we just seem to be toyed with by so many different countries,
already. And we don't know who's doing what. We don't know who's got the
power, who's got that capability, some people say it's China, some
people say it's Russia. But certainly cyber has to be a, you know,
certainly cyber has to be in our thought process, very strongly in our
thought process. Inconceivable that, inconceivable the power of cyber.
But as you say, you can take out, you can take out, you can make
countries nonfunctioning with a strong use of cyber. I don't think we're
there. I don't think we're as advanced as other countries are, and I
think you probably would agree with that. I don't think we're advanced,
I think we're going backwards in so many different ways. I think we're
going backwards with our military. I certainly don't think we are, we
move forward with cyber, but other countries are moving forward at a
much more rapid pace. We are frankly not being led very well in terms of
the protection of this country.

\textbf{HABERMAN:} Mr. Trump, just a quick follow-up on that question.
As you know, we discovered in recent years that the U.S. spies
extensively against its allies. That's what came up with Edward Snowden
and his data trove including Israel and Germany.

\textbf{TRUMP:} Edward Snowden has caused us tremendous problems.

\textbf{HABERMAN:} But would you continue the programs that are in place
now, or would you halt them, in terms of spying against our allies?

\textbf{SANGER:} Like Israel and Germany.

\textbf{TRUMP:} Right. They're spying against us. Edward Snowden has
caused us tremendous problems. Edward Snowden has been, you know, you
have the two views on Snowden, obviously: You have, he's wonderful, and
you have he's horrible. I'm in the horrible category. He's caused us
tremendous problems with trust, with everything about, you know, when
they're showing, Merkel's cellphone has been spied on, and are -- Now,
they're doing it to us, and other countries certainly are doing it to
us, and but what I think what he did, I think it was a tremendous, a
tremendous disservice to the United States. I think and I think it's
amazing that we can't get him back.

\textbf{SANGER:} President Obama ordered an end to the spying, to the
listening in on Angela Merkel's cellphone, if that's in fact what we
were doing. Was that the right decision?

\textbf{TRUMP:} Well you see, I don't know that, you know, when I talk
about unpredictability, I'm not sure that we should be talking about me
-- On the assumption that I'm doing well, which I am, and that I may be
in that position, I'm not sure that I would want to be talking about
that. You understand what I mean by that, David. We're so open, we're
so, ``Oh I wouldn't do this, I wouldn't do that, I would do this, I
would do that.'' And it's not so much with Merkel, but it's certainly
with other countries. You know, that really, where there's, where
there's a different kind of relationship, and a much worse relationship
than with Germany. So, you know there's so, there's such predictability
with our country. We go and we send 50 soldiers over to the Middle East
and President Obama gets up and announces that we're sending 50 soldiers
to the Middle East. Fifty very special soldiers. And they now have a
target on their back, and everything we do, we announce, instead of
winning, and announcing when it's all over. There's such, total
predictability of this country, and it's one of the reasons we do so
poorly. You know, I'd rather not say that. I would like to see what
they're doing. Because you know, many countries, I can't say Germany,
but many countries are spying on us. I think that was a great disservice
done by Edward Snowden. That I can tell you.

\hypertarget{how-to-defeat-isis}{%
\subsection{How to Defeat ISIS}\label{how-to-defeat-isis}}

\textbf{HABERMAN:} Mr. Trump, you have talked about your plans to defeat
ISIS, and how you would approach it. Would you be willing to stop buying
oil from the Saudis if they're unwilling to go in and help?

\textbf{SANGER:} On the ground?

\textbf{TRUMP:} Oh yeah, sure. I would do that. The beautiful thing
about oil is that, you know, we're really getting close, because of
fracking, and because of new technology, we're really in a position that
we weren't in, you know, years ago, and the reason we're in the Middle
East is for oil. And all of a sudden we're finding out that there's less
reason to be. Now, now, we're in the Middle East for really defense,
because we can't allow them, I mean, look, I was against the war in
Iraq. I thought it would destabilize the Middle East, and it has
destabilized it, it's totally destabilized the Middle East. The way
Obama got out of the war was, you know, disgraceful, and idiotic. When
he announced the date certain, they pulled back, and they said, ``Oh,
well.'' As much as they don't mind dying, they do mind dying. And they
pulled back, and then, you know, it's a, it was a terrible thing the way
he announced that, and then he didn't leave troops behind so that, you
know, whatever there was of Iraq, which in my opinion wasn't very much,
because I think that, you know, the government was totally corrupt, and
they put the wrong people in charge, and you know, that in its own way
led to the formation of ISIS, because they weren't given their due. But,
I think that President Obama, the way he got out of that war was
unbelievable. I think Hillary Clinton was catastrophic in those
decisions, having to do with Libya and just about everything else. Every
bad decision that you could make in the Middle East was made. And now if
you look at it, if you would go back 15 years ago, and I'm not saying it
was only Obama, It was Obama's getting out, it was other people's
getting in, but you go back 15 years ago, and I say this, if our
presidents would have just gone to the beach and enjoyed the ocean and
the sun, we would've been much better off in the Middle East, than all
of this tremendous death, destruction, and you know, monetary loss, it's
just incredible. 'Cause we're further, we're far worse off today than we
were 15 years ago or 10 years ago in the Middle East. Far worse.

\textbf{SANGER:} But I just want to make sure I understand your answer
to Maggie's question. So you said earlier this week that we should use
air power but not send in ground forces. That had to be done by the
regional Arab partners. We assume by that, you mean the Saudis, the
U.A.E. and others from whom we might purchase oil or have alliances. I
think Maggie's question, if I understood it right, was if these
countries are unwilling to send in ground troops against ISIS, and so
far they have been, despite President Obama's efforts to get them in,
would you be willing to say, ``We will stop buying oil from you, until
you send ground troops?''

\textbf{TRUMP:} There's two answers to that. The answer is, probably
yes, but I would also say this: We are not being reimbursed for our
protection of many of the countries that you'll be talking about, that,
including Saudi Arabia. You know, Saudi Arabia, for a period of time,
now the oil has gone down, but still the numbers are phenomenal, and the
amount of money they have is phenomenal. But we protect countries, and
take tremendous monetary hits on protecting countries. That would
include Saudi Arabia, but it would include many other countries, as you
know. We have, there's a whole big list of them. We lose, everywhere. We
lose monetarily, everywhere. And yet, without us, Saudi Arabia wouldn't
exist for very long. It would be, you know, a catastrophic failure
without our protection. And I'm trying to figure out, why is it that we
aren't going in and saying, at a minimum, at a minimum it's a two-part
question, with respect to Maggie's question. But why aren't we going in
and saying, ``At a minimum, I'm sorry folks, but you have to, under no
circumstances can we continue to do this.'' You know, we needed, we
needed oil desperately years ago. Today, because -- again, because of
the new technologies, and because of places that we never thought had
oil, and they do have oil, and there's a glut on the market, there's a
tremendous glut on the market, I mean you have ships out at sea that are
loaded up and they don't even know where to go dump it. But we don't
have that same pressure anymore, at all. And we shouldn't have that for
a long period of time, because there's so many places. I mean, they're
closing wells all over the place. So, I would say this, I would say at a
minimum, we have to be reimbursed, substantially reimbursed, I mean, to
a point that's far greater than what we're being paid right now. Because
we're not being reimbursed for the kind of tremendous service that we're
performing by protecting various countries. Now Saudi Arabia's one of
them. I think if Saudi Arabia was without the cloak of American
protection of our country's, of U.S. protection, think of Saudi Arabia.
I don't think it would be around. It would be, whether it was internal
or external, it wouldn't be around for very long. And they're a money
machine, they're a monetary machine, and yet they don't reimburse us the
way we should be reimbursed. So that's a real problem. And frankly, I
think it's a real, in terms of bringing our country back, because our
country's a poor country. Our country is a debtor nation, we're a debtor
nation. I mean, we owe trillions of dollars to people that are buying
our bonds, in the form of other countries. You look at China, where we
owe them \$1.7 trillion, you have Japan, \$1.5 trillion. We're a debtor
nation. We can't be a debtor nation. I don't want to be a debtor nation.
I want it to be the other way. One of the reasons we're a debtor nation,
we spend so much on the military, but the military isn't for us. The
military is to be policeman for other countries. And to watch over other
countries. And there comes a point that, and many of these countries are
tremendously rich countries. Not powerful countries, but -- in some
cases they are powerful -- but rich countries.

\textbf{SANGER:} One more along the lines of your ISIS strategy. You've
seen the current strategy, which is, you've seen Secretary Kerry trying
to seek a political accord between President Assad and the rebel forces,
with Assad eventually leaving. And then the hope is to turn all those
forces, including Russia and Iran, against ISIS. Is that the right way
to do it? Do you have an alternative approach?

\textbf{TRUMP:} Well, I thought the approach of fighting Assad and ISIS
simultaneously was madness, and idiocy. They're fighting each other and
yet we're fighting both of them. You know, we were fighting both of
them. I think that our far bigger problem than Assad is ISIS, I've
always felt that. Assad is, you know I'm not saying Assad is a good man,
'cause he's not, but our far greater problem is not Assad, it's ISIS.

\textbf{SANGER:} I think President Obama would agree with that.

\textbf{TRUMP:} O.K., well, that's good. But at the same time -- yeah,
he would agree with that, I think to an extent. But I think, you can't
be fighting two people that are fighting each other, and fighting them
together. You have to pick one or the other. And you have to go at --

\textbf{SANGER:} So how would your strategy differ from what he's doing
right now?

\textbf{TRUMP:} Well I can only tell you -- I can't tell you, because
his strategy, it's open and it would seem to be fighting ISIS but he's
fighting it in such a limited capacity. I've been saying, take the oil.
I've been saying it for years. Take the oil. They still haven't taken
the oil. They still haven't taken it. And they hardly hit the oil. They
hardly make a dent in the oil.

\textbf{SANGER:} The oil that ISIS is pumping.

\textbf{TRUMP:} Yes, the oil that ISIS is pumping, where they're getting
tremendous amounts of revenue. I've said, hit the banking channels. You
know, they have very sophisticated banking channels, which I understand,
but I don't think a lot of people do understand. You know, they're
taking in tremendous amounts of money from banking channels. That, you
know, many people in countries that you think are our allies, are giving
ISIS tremendous amounts of money and it's going through very dark
banking channels. And we should have stopped those banking channels long
ago and I think we've done nothing to stop them, and that money is
massive. Massive. It's a massive amount of money. So it's not only from
oil, David, it's from also the bank, the bank. It's through banks. And
very sophisticated channels. They call them the dark channels. Very
sophisticated channels. And money is coming in from people that we think
are our allies.

\hypertarget{nato-is-obsolete}{%
\subsection{`NATO Is Obsolete'}\label{nato-is-obsolete}}

\textbf{HABERMAN:} Mr. Trump, I also want to go back to something you
said earlier this week about NATO being ineffective. Do you think it's
the right institution for countering terror or do we need a new one and
what might that new one look like?

\textbf{TRUMP:} Well I said something a few days ago and I was vastly
criticized and I notice now this morning, people are saying Donald Trump
is a genius. Because what I said -- which of course is always nice to
hear, David. But I was asked a question about NATO, and I've thought
this but I have never expressed my opinion because until recently I've
been an entrepreneur, I've been a very successful entrepreneur as
opposed to a politician. And -- I'd love to ask David, Maggie, if he's a
little surprised at how well I've done. You know, we've knocked out a
lot. We're down to the leftovers now, from the way I look at it. I call
them the leftovers.

\emph{(Laughter.)}

So anyway, but the question was asked of me a few days ago about NATO,
and I said, well, I have two problems with NATO. No. 1, it's obsolete.
When NATO was formed many decades ago we were a different country. There
was a different threat. Soviet Union was, the Soviet Union, not Russia,
which was much bigger than Russia, as you know. And, it was certainly
much more powerful than even today's Russia, although again you go back
into the weaponry. But, but -- I said, I think NATO is obsolete, and I
think that -- because I don't think -- right now we don't have somebody
looking at terror, and we should be looking at terror. And you may want
to add and subtract from NATO in terms of countries. But we have to be
looking at terror, because terror today is the big threat. Terror from
all different parts. You know in the old days you'd have uniforms and
you'd go to war and you'd see who your enemy was, and today we have no
idea who the enemy is.

\textbf{SANGER:} If you just think about Maggie's question about whether
it's the right institution for this, when you go to NATO these days, in
Brussels, not far from where we've seen --- just miles from where we saw
the attacks the other day ---

\textbf{TRUMP:} Which is amazing, right? Which is amazing in itself.
Yes?

\textbf{SANGER:} What they'll say to you is that Russia is resurgent
right now. They are rebuilding their nuclear arsenal. They're
{[}unintelligible{]} Baltics. We've got submarine runs, air runs. Things
that have at least echoes of the old Cold War. The view is that their
mission is coming back. Do you agree with that?

\textbf{TRUMP:} I'll tell you the problems I have with NATO. No. 1, we
pay far too much. We are spending --- you know, in fact, they're even
making it so the percentages are greater. NATO is unfair, economically,
to us, to the United States. Because it really helps them more so than
the United States, and we pay a disproportionate share. Now, I'm a
person that --- you notice I talk about economics quite a bit, in these
military situations, because it is about economics, because we don't
have money anymore because we've been taking care of so many people in
so many different forms that we don't have money --- and countries, and
countries. So NATO is something that at the time was excellent. Today,
it has to be changed. It has to be changed to include terror. It has to
be changed from the standpoint of cost because the United States bears
far too much of the cost of NATO. And one of the things that I hated
seeing is Ukraine. Now I'm all for Ukraine, I have friends that live in
Ukraine, but it didn't seem to me, when the Ukrainian problem arose, you
know, not so long ago, and we were, and Russia was getting very
confrontational, it didn't seem to me like anyone else cared other than
us. And we are the least affected by what happens with Ukraine because
we're the farthest away. But even their neighbors didn't seem to be
talking about it. And, you know, you look at Germany, you look at other
countries, and they didn't seem to be very much involved. It was all
about us and Russia. And I wondered, why is it that countries that are
bordering the Ukraine and near the Ukraine -- why is it that they're not
more involved? Why is it that they are not more involved? Why is it
always the United States that gets right in the middle of things, with
something that -- you know, it affects us, but not nearly as much as it
affects other countries. And then I say, and on top of everything else
-- and I think you understand that, David -- because, if you look back,
and if you study your reports and everybody else's reports, how often do
you see other countries saying `We must stop, we must stop.'' They don't
do it! And, in fact, with the gas, you know, they wanted the oil, they
wanted other things from Russia, and they were just keeping their mouths
shut. And here the United States was going out and, you know, being
fairly tough on the Ukraine. And I said to myself, isn't that
interesting? We're fighting for the Ukraine, but nobody else is fighting
for the Ukraine other than the Ukraine itself, of course, and I said, it
doesn't seem fair and it doesn't seem logical.

\textbf{HABERMAN:} Mr. Trump, speaking of --

\textbf{TRUMP:} David, does that make sense to you, by the way?

\textbf{SANGER:} Well, President Obama said the other day in an
interview he had that he thought that Russia, over time, was always
going to have more influence over Ukraine than we would or anyone else
would just given both the history and the geography.

\textbf{TRUMP:} And the location, right. The geography. I would agree
with him.

\textbf{SANGER:} And so in the end do you agree that Russia is going to
end up dominating the Ukraine?

\textbf{TRUMP:} Well, unless, unless there is, you know, somewhat of a
resurgence frankly from people that are around it. Or they would ask us
for help. But they don't ask us for help. They're not even asking us for
help. They're literally not even talking about it, and these are the
countries that border the Ukraine.

\textbf{HABERMAN:} Mr. Trump --

\textbf{TRUMP:} There doesn't seem to be any great anxiety over the
Ukraine by everybody that should be affected and that's bordering the
Ukraine.

\textbf{SANGER:} There are several countries that have joined NATO in
recent times -- Estonia, among them, and so forth -- that we are now
bound by treaty to defend if Russia moved in. Would you observe that
part of the treaty?

\textbf{TRUMP:} Yeah, I would. It's a treaty, it's there. I mean, we
defend everybody. \emph{(Laughs.)} We defend everybody. No matter who it
is, we defend everybody. We're defending the world. But we owe, soon,
it's soon to be \$21 trillion. You know, it's 19 now but it's soon to be
21 trillion. But we defend everybody. When in doubt, come to the United
States. We'll defend you. In some cases free of charge. And in all cases
for a substantially, you know, greater amount. We spend a substantially
greater amount than what the people are paying. We, we have to think
also in terms -- we have to think about the world, but we also have -- I
mean look at what China's doing in the South China Sea. I mean they are
totally disregarding our country and yet we have made China a rich
country because of our bad trade deals. Our trade deals are so bad. And
we have made them -- we have rebuilt China and yet they will go in the
South China Sea and build a military fortress the likes of which perhaps
the world has not seen. Amazing, actually. They do that, and they do
that at will because they have no respect for our president and they
have no respect for our country. Hey folks, I'm going to have to get off
here now. Did you --

\hypertarget{tensions-in-the-south-china-sea}{%
\subsection{Tensions in the South China
Sea}\label{tensions-in-the-south-china-sea}}

\textbf{HABERMAN:} I just had one quick follow-up on what you were
saying about the South China Sea. How would you counter that
assertiveness over those islands? Among other things, it's increasingly
valuable real estate strategically. Would you be willing to build our
own islands there?

\textbf{TRUMP:} Well what you have to do -- and you have to speak to
Japan and other countries, because they're affected far greater than we
are -- you understand that -- I mean, they're affected far -- I just
think the act is so brazen, and it's so terrible that they would do that
without any consultation, without anything, and yet they'll sell their
products to the United States and rebuild China, and frankly, even the
islands, I mean, you know, they've made so much economic progress
because of the United States. And in the meantime we're becoming a
third-world nation. You look at our airports, you look at our roadways,
you look at our bridges are falling down. They're building bridges all
over the place, ours are falling down. You know, we've rebuilt China.
The money they've drained out of the United States has rebuilt China.
And they've done it through monetary manipulation, by devaluations. And
very sophisticated. I mean, they're grand chess players at devaluation.
But they've done it --

\textbf{SANGER:} I think what Maggie was asking was how would you deter
their activity. Right now \emph{(Crosstalk)} -- But would you claim some
of those reef scenarios to try to build our own military --

\textbf{TRUMP:} Perhaps, but we have great economic -- and people don't
understand this -- but we have tremendous economic power over China. We
have tremendous power. And that's the power of trade. Because they use
us as their bank, as their piggy bank, they take -- but they don't have
to pay us back. It's better than a bank because they take money out but
then they don't have to pay us back.

\textbf{SANGER:} So you would cut into trade in return --

\textbf{TRUMP:} No, I would use trade to negotiate.

\textbf{HABERMAN:} Oh, O.K. My last question. Sir, my last --

\textbf{TRUMP:} I would use trade to negotiate. Would I go to war? Look,
let me just tell you. There's a question I wouldn't want to answer.
Because I don't want to say I won't or I will or -- do you understand
that, David? That's the problem with our country. A politician would
say, `Oh I would never go to war,' or they'd say, `Oh I would go to
war.' I don't want to say what I'd do because, again, we need
unpredictability. You know, if I win, I don't want to be in a position
where I've said I would or I wouldn't. I don't want them to know what
I'm thinking. The problem we have is that, maybe because it's a
democracy and maybe because we have to be so open -- maybe because you
have to say what you have to say in order to get elected -- who knows?
But I wouldn't want to say. I wouldn't want them to know what my real
thinking is. But I will tell you this. This is the one aspect I can tell
you. I would use trade, absolutely, as a bargaining chip.

\hypertarget{his-foreign-policy-team}{%
\subsection{His Foreign Policy Team}\label{his-foreign-policy-team}}

\textbf{HABERMAN:} Mr. Trump, how did you come to settle on your foreign
policy team? I know that it's still in formation and you've said --

\textbf{TRUMP:} Recommended by people. And we're going to have new
people put in. In fact, we have additional people too. You've got the
one list, I think, but we have -- we actually have -- I only gave
certain names.

\textbf{HABERMAN:} But did you meet with them?

\textbf{TRUMP:} We have some others that I really like a lot and we're
going to put them in. Maj. Gen. Gary Harrell. Maj. Gen. Bert Mizuwawa.
\emph{(Ed. note: It's Mizusawa.)}

\textbf{HABERMAN:} These are the additional ones?

\textbf{TRUMP:} Rear Adm. Chuck Kubic. Yeah. He's Navy, retired. Very
good, nice, supposedly.

\textbf{HABERMAN:} Interesting. Interesting.

\textbf{TRUMP:} These are people recommended -- people that I respect
recommended them. People -- I've heard very good things about them. In
addition, we're going to be adding some additional names that I've liked
over the years.

\textbf{HABERMAN:} Ah, O.K.

\textbf{TRUMP:} I have very strong -- as you've probably noticed -- I've
had very strong feelings on foreign policy and I've had very strong
feelings on defense and offense. And I've been right about a lot of the
things I've been saying. I've been right about a lot. And The New York
Times criticized me very badly with a very major article
\href{http://www.nytimes.com/2016/01/28/world/europe/trump-finds-new-city-to-insult-brussels.html}{when
I said Brussels is a hellhole}, and I talked about Brussels in a very
negative way because of what they're doing over there. And yesterday all
over Twitter, as you probably saw, everybody said that Trump is right,
The New York Times -- You know, The New York Times really hit me hard on
Brussels when I said recently that it's a hellhole, and waiting to
explode. And I didn't even realize it, and then yesterday all over the
place, Twitter was crazy that Trump was right, again, this time about
Brussels.

\textbf{HABERMAN:} You mean after the attacks?

\textbf{TRUMP:} I've been right -- Yeah, after the attack. I've been
right about a lot of different things. So. Anyway. You know, in my book
I mention Osama bin Laden, and I wrote the book in 2000, prior to the
World Trade Center coming down and the reason I did is that I saw this
guy and I read about this terrorist who was a very aggressive, bad dude.
And I wrote about it in ``The America We Deserve.'' I wrote about Osama
bin Laden. You know, not a lot, but a couple paragraphs -- about Osama
bin Laden.
\href{http://www.nytimes.com/interactive/2016/us/elections/fact-check.html\#/factcheck-15?smid=tw-share}{Look
at him. You better take a look at him}. And a year and a half later the
World Trade Center came down. And your friend Joe Scarborough,
interestingly, in one of his -- you know, somebody had mentioned that,
and Joe said, `No way. There's no way he wrote about it before the
fact.' And they said no, no, and they sent out for the book, and they
put it before him and he said, `Wow, you're right. Trump wrote about
Osama bin Laden before the World Trade Center came down. That's
amazing.' So look, I've said a lot. I don't get a lot of credit. I do
from the people. I don't from a lot of the media. But that's O.K. I'd
rather keep it that way. Hey David, I'd rather have it that way, I
guess, right?

\emph{(Laughter.)}

\hypertarget{the-iran-deal}{%
\subsection{The Iran Deal}\label{the-iran-deal}}

\textbf{SANGER:} You have told us a lot about what your leverage would
be over China in trade. Tell us on Iran: I know that you've said that
you think that the Iran deal was an extremely bad deal. I'd be
interested to know what your goals would be in renegotiating it. What
your leverage would be and what you would renegotiate, what parts of the
agreement.

\textbf{TRUMP:} Sure. It's not just that it's a bad deal, David. It's a
deal that could've been so much better just if they'd walked a couple of
times. They negotiated so badly. They were being mocked, they were being
scorned, they were being harassed, our negotiators, including Kerry,
back in Iran, by the various representatives and the leaders of Iran at
the highest level. And they never walked. They should've walked, doubled
up the sanctions, and made a good deal. Gotten the prisoners out long
before, not just after they gave the \$150 billion. They should've never
given the money back. There were so many things that were done, they
were so, the negotiation was, and I think deals are fine, I think
they're good, not bad. But, you gotta make good deals, not bad deals.
This deal was a disaster.

\textbf{SANGER:} So, it's a deal you would inherit if you were elected,
so what I'm trying to get at is, what would you insist on. Are the
restrictions on nuclear not long enough, are the missile restrictions
not strong enough?

\textbf{TRUMP:} Certainly the deal is not long enough. Because at the
end of the deal they're going to have great nuclear capability. So
certainly the deal isn't long enough. I would never have given them back
the \$150 billion under any circumstances. I would've never allowed that
to happen. They are, they are now rich, and did you notice they're
buying from everybody but the United States? They're buying planes,
they're buying everything, they're buying from everybody but the United
States. I would never have made the deal.

\textbf{SANGER:} Our law prevents us from selling to them, sir.

\textbf{TRUMP:} Uh, excuse me?

\textbf{SANGER:} Our law prevents us from selling any planes or, we
still have sanctions in the U.S. that would prevent the U.S. from being
able to sell that equipment.

\textbf{TRUMP:} So, how stupid is that? We give them the money, and we
now say, ``Go buy Airbus instead of Boeing,'' right? So how stupid is
that? In itself, what you just said, which is correct by the way, but
would they now go and buy, you know, they bought 118 approximately, 118
Airbus planes. They didn't buy Boeing planes, O.K.? We give them the
money, and we say you can't spend it in the United States, and create
wealth and jobs in the United States. And on top of it, they didn't,
they in theory, I guess, cannot do that, you know, based on what I've
understood. They can't do that. It's hard to believe. We gave them \$150
billion and they can't spend it in our country.

\textbf{SANGER:} So you would lift the domestic sanctions so they could
buy American goods?

\textbf{TRUMP:} Well, I wouldn't have given them back the money. So I
wouldn't be in that position. I would never have given them back the --
that would never be a part of the negotiation. I would have never, ever
given it to them, and I would've made a better deal than they made,
without the money, and I would've made a better deal.

\textbf{SANGER:} And to stop the missile launches they've been doing?

\textbf{TRUMP:} Well, it's ridiculous, I mean, now they're doing missile
launches, and they're buying missiles from Russia, and they're doing
things that nobody thought were, you know, even permissible or in the
deal, and they're doing them.

\textbf{HABERMAN:} Mr. Trump, one thing you didn't talk about --

\textbf{TRUMP:} That deal was one of the most incompetent deals of any
kind I've ever seen.

\textbf{HABERMAN:} One thing you talked about at Aipac --

\textbf{TRUMP:} Right, David, so I wouldn't talk in terms of not buying
because I would've never, ever given them the money. Go ahead.

\textbf{HABERMAN:} Sorry, sir, one thing that didn't come up at Aipac, I
think in actually anyone's speeches, but in yours also, I'm curious, in
terms of Israel, and in terms of the peace process, do you think it
should result in a two-state solution, or in a single state?

\textbf{TRUMP:} Well, I think a lot of people are saying it's going to
result in a two-state solution. What I would love to do is to, a lot of
people are saying that. I'm not saying anything. What I'm going to do
is, you know, I specifically don't want to address the issue because I
would love to see if a deal could be made. If a deal could be made. Now,
I'm not sure it can be made, there's such unbelievable hatred, there's
such, it's ingrained, it's in the blood, the hatred and the distrust,
and the horror. But I would love to see if a real deal could be made.
Not a deal that you know, lasts for three months, and then everybody
starts shooting again. And a big part of that deal, you know, has to be
to end terror, we have to end terror. But I would say this, in order to
negotiate a deal, I'd want to go in there as evenly as possible and
we'll see if we can negotiate a deal. But I would absolutely give that a
very hard try to do. You know, a lot of people think that's the hardest
of all deals to negotiate. A lot of people think that. So, but I would
say that I would have a better chance than anybody of making a deal.
I'll tell you one thing, people that I know from Israel, many people,
many, many people, and almost everybody would love to see a deal on the
side of Israel. Everybody would, now with that being said, most people
don't think a deal can be made. But from the Israeli side, they would
love to see a deal. And I've been a little bit surprised here. Now that
I'm really into it, I've been a little bit surprised to hear that. I
would've said, I would've said that maybe, maybe you know, maybe Israel
never really wanted to make a deal or doesn't really want to make a
deal. They really want to make a deal, they want to make a good deal,
they want to make a fair deal, but they do want to make a deal. And,
almost everybody, and I'm talking to people off the record, and off the
record, they really would like to see a deal. I'm not so sure that the
other side can mentally, you know, get their heads around the deal,
because the hatred is so incredible. Folks, I have to go.

\hypertarget{developing-views-on-foreign-affairs}{%
\subsection{Developing Views on Foreign
Affairs}\label{developing-views-on-foreign-affairs}}

\emph{Second interview begins:}

\textbf{TRUMP:} So go ahead, start off wherever you want.

\textbf{SANGER:} One place that might be a good place to start is where
we ended up on the foreign policy advisers. Because we're trying to
figure out how much time you're cutting out now for foreign policy as
you said --- it's not an area you focused on in your business career as
much.

\textbf{TRUMP:} Well I enjoyed it, I enjoyed reading about it. But it
wasn't something that came into play as a business person. But I had an
aptitude for it I think, and I enjoyed reading about and I would read
about it.

\textbf{SANGER:} One question we had for you is, first of all, since you
enjoyed reading about it, is there any particular book or set of
articles that you found influential in developing your own foreign
policy views?

\textbf{TRUMP:} More than anything else would be various newspapers
including your own, you really get a vast array and, you know a big menu
of different people and different ideas. You know you get a very big
array of things from reading the media, from seeing the media, the
papers, including yours. And it's something that I've always found
interesting and I think I've adapted to it pretty well. I will tell you
my whole stance on NATO, David, has been --- I just got back and I'm
watching television and that's all they're talking about. And you know
when I first said it, they sort of were scoffing. And now they're really
saying, well wait, do you know it's really right? And maybe NATO --- you
know, it doesn't talk about terror. Terror is a big thing right now.
That wasn't the big thing when it originated and people are starting to
talk about the cost.

\textbf{SANGER:} Well it's geared toward state actors and you're
discussing gearing something toward nonstate actors. Is it possible that
we need a new institution that is not burdened by the military structure
of NATO in order to deal with nonstate actors and terrorists?

\textbf{TRUMP:} I actually think in terms of terror you may be better
off with a new institution, an institution that would be more fairly
based, an institution that would be more fairly taken care of from an
economic standpoint. You have many wealthy states over there that are
not going to be there if it's not for us, and they're not going to be
there if it is for terror. Whether it's Saudi Arabia or others. I
actually do think, while I'd like to adapt it, I think you have a
different set of players, frankly. You have more of a Middle Eastern
player and others but you would have in addition, Middle Eastern
players.

\textbf{SANGER:} Who are not currently members of NATO. You think the
membership of NATO is not set up right for combating terror.

\textbf{TRUMP:} No, it was set up to talk about the Soviet Union. Now of
course the Soviet Union doesn't exist now it's Russia, which is not the
same size, in theory not the same power, but who knows about that
because of weaponry, but it's not the same size and this was set up for
numerous things but for the Soviet Union. The point is the world is a
much different place right now. And today all you have to do is read and
see the world is, the big threat would seem to be based on terror and
based on what's going on in 90 percent, 95 percent of the horror
stories. I think, probably a new institution maybe would be better for
that than using NATO which was not meant for that. And it's become very
bureaucratic, extremely expensive and maybe is not flexible enough to go
after terror. Terror is very much different than what NATO was set up
for.

\textbf{SANGER:} And requires a different kind of force.

\textbf{TRUMP:} I think it requires a different flexibility, it requires
a different speed maybe, watching nations or a nation or nations. I
think it requires flexibility and speed.

\textbf{SANGER:} So Maggie and I were at the end of our conversation
this morning we were talking with you a little bit about your foreign
policy advisers. There's been a little bit of a sense that you've had a
hard time attracting some of the bigger names of your party. There were
a lot of former deputy secretaries of state, of defense, others were out
there. And the list of advisers you've released so far has been very
strong on having military backgrounds but not many with diplomatic
backgrounds. We were wondering whether or not you are looking for a
different mix or whether you're having trouble attracting some of the
big names.

\textbf{TRUMP:} It's interesting, it's not trouble attracting. Many of
them that I actually like a lot and that like me a lot and that want to
do 100 percent, many of them are tied up with contracts working for
various networks, you understand? I mean, I've had some that are --- I
currently have some that are thinking about getting out of their
contract `cause they're so excited about it. I've had a lot of
excitement but there are some that are tied up where they have a
contract with, as an example, they might have a contract with Fox, they
may have a contract with CNN and they can't do it. They have contracts
with the various networks and maybe the media too. I don't know about
The Times but it's possible --- I think less likely, I'm not sure how
that structure works with the actual newspapers. But there are some that
I've spoken to that want to do it but they're tied up with contracts
that are with somebody else. There are some that were with campaigns
that have now imploded, and I think they're going to be free agents very
shortly. Hey, a lot of campaigns have imploded in the last couple of
months, which you people perhaps have seen just as vividly as I have.
Right? Not as happily as I have, but nevertheless just as vividly. So
you know there are actually, there are a lot of people available, there
are a lot of good people available. But some of the good people are
currently under contract. Does that make sense to you, David?

\textbf{SANGER:} Yup, Maggie, did you having anything more on that
before we wanted to turn back to Israel?

\textbf{HABERMAN:} Yeah, Mr. Trump, if you could just say how much time
are you devoting a week at this point either to briefings to studying,
you know, and if there's no major change now what it might look like in
the future?

\textbf{TRUMP:} I think that you know, what I've really had to do is get
through 17, cause it was really 18 total when we started. So I had to
get through 17 people. I've gotten through almost all of the 17 people.
But I'm down to two, from 17 to two. And you know many of them were
front-runners, and they weren't front-runners for very long. You can go
through the list, you know the list as well as I do. And my primary
focus was that.

But during the period I've been, I think very well versed on matters as
we're discussing and many more than just what we're just discussing. Now
as it gets --- as we get you know closer to the end of the process it`ll
take place more and more. I'm setting up a council, I'm setting up ---
and I have other people coming in, I gave you the other few names I
think that we added, we have a few more coming in. But I have a few more
that are going to come in. I just don't want to I just don't want to
mention them unless they give me approval, meaning they're on board.

And we're going to have a very substantial council of very good people.
And some of them are military. Look, the military is going to be very
important because we have to do something with ISIS, David, and you know
we do want the military. And I think that over the next few weeks I'll
be able to give you some more names. People that are going to be coming
in.

\textbf{SANGER:} Do you fear that if you have too many military on your
council, they tend to search for the military solution first instead of
the diplomatic or economic sanction solution first?

\textbf{TRUMP:} Yeah but I'd like to know the military solution and I'm
working on the military solution. Because there's not huge negotiation
involved with ISS, because there's an irrationality that is pretty ---
this is not something, `Oh let's make a deal.' I don't see deals being
made with ISIS. Nobody knows what ISIS is, nobody knows who is leading
it, who is alive, who is not alive, I mean we're really not talking
about too many diplomatic solutions. We're not talking about diplomatic
solutions with ISIS, let me put it that way.

\hypertarget{us-influence-in-east-asia}{%
\subsection{U.S. Influence in East
Asia}\label{us-influence-in-east-asia}}

\textbf{SANGER:} I wasn't referring to that in the ISIS context, I was
referring more in the realm of dealing with our allies, dealing with
China, dealing with Japan, the other places that we've discussed.

\textbf{TRUMP:} So ISIS I think you'd agree with me on that and the rest
will come. I have really strong feelings on China. I like China very
much I like Chinese people. I respect the Chinese leaders, but you know
China's been taking advantage of us for many, many years and we can't
allow it to go on. And at the same time we'll be able to keep a good
relationship with China. And same with Japan and same with --- you have
to see the trade imbalance between Japan and the United States, it's
unbelievable. They sell to us and we practically give them back nothing
by comparison. It's a very unfair situation.

\textbf{SANGER:} They also pay more for troop support than any other
country in the world.

\textbf{TRUMP:} They do but still far less than it costs us.

\textbf{HABERMAN:} Would you be willing ---

\textbf{TRUMP:} You're right about that David, but it's --- and they do
pay somewhat more, but they pay more because of the tremendous amount of
business that they do with us, uneconomic business from our standpoint.

\textbf{HABERMAN:} Would you be willing to withdraw U.S. forces from
places like Japan and South Korea if they don't increase their
contribution significantly?

\textbf{TRUMP:} Yes, I would. I would not do so happily, but I would be
willing to do it. Not happily. David actually asked me that question
before, this morning before we sort of finalized out. The answer is not
happily but the answer is yes. We cannot afford to be losing vast
amounts of billions of dollars on all of this. We just can't do it
anymore. Now there was a time when we could have done it. When we
started doing it. But we can't do it anymore. And I have a feeling that
they'd up the ante very much. I think they would, and if they wouldn't I
would really have to say yes.

\textbf{SANGER:} So we talked a little this morning about Japan and
South Korea, whether or not they would move to an independent nuclear
capability. Just last week the United States removed from Japan, after a
long negotiation, many bombs worth, probably 40 or more bombs worth of
plutonium or highly enriched uranium that we provided them over the
years. And that's part of a very bipartisan effort to keep them from
going nuclear. So I was a little surprised this morning when you said
you would be open to them having their own nuclear deterrent. Certainly
if you pull back one of the risks is that they would go nuclear.

\textbf{TRUMP:} You know you're more right except for the fact that you
have North Korea which is acting extremely aggressively, very close to
Japan. And had you not had that, I would have felt much, I would have
felt differently. You have North Korea, and we are very far away and we
are protecting a lot of different people and I don't know that we are
necessarily equipped to protect them. And if we didn't have the North
Korea threat, I think I'd feel a lot differently, David.

\textbf{SANGER:} But with the North Korea threat you think maybe Japan
does need its own nuclear\ldots{}

\textbf{TRUMP:} Well I think maybe it's not so bad to have Japan --- if
Japan had that nuclear threat, I'm not sure that would be a bad thing
for us.

\textbf{SANGER:} You mean if Japan had a nuclear weapon it wouldn't be
so bad for us?

\textbf{TRUMP:} Well, because of North Korea. Because of North Korea.
Because we don't know what he's going to do. We don't know if he's all
bluster or is he a serious maniac that would be willing to use it. I was
talking about before, the deterrent in some people's minds was that the
consequence is so great that nobody would ever use it. Well that may
have been true at one point but you have many people that would use it
right now in this world.

\textbf{SANGER:} For that reason, they may well need their own and not
be able to just depend on us\ldots{}

\textbf{TRUMP:} I really believe that's true. Especially because of the
threat of North Korea. And they are very aggressive toward Japan. Well I
mean look, he's aggressive toward everybody. Except for China and Iran.

See we should use our economic power to have them disarm --- now then it
becomes different, then it becomes purely economic, but then it becomes
different. China has great power over North Korea even though they don't
necessarily say that. Now, Iran, we had a great opportunity during this
negotiation when we gave them the 150 billion and many other things.
Iran is the No. 1 trading partner of North Korea. Now we could have put
something in our agreement that they would have led the charge if we had
people with substance and with brainpower and with some negotiating
ability. But the No. 1 trading partner with North Korea is Iran. And we
did a deal with them, and we just did a deal with them, and we don't
even mention North Korea in the deal. That was a great opportunity to
put another five pages in the deal, or less, and they do have a great
influence over North Korea. Same thing with China, China has great
influence over North Korea but they don't say they do because they're
tweaking us. I have this from Chinese. I have many Chinese friends, I
have people of vast wealth, some of the most important people in China
have purchased apartments from me for tens of millions of dollars and
frankly I know them very well. And I ask them about their relationship
to North Korea, these are top people. And they say we have tremendous
power over North Korea. I know they do. I think you know they do.

\textbf{SANGER:} They signed on to the most recent sanctions, more
aggressive sanctions than we thought the Chinese would agree to.

\textbf{TRUMP:} Well that's good, but, I mean I know they did, but I
think that they have power beyond the sanctions.

\textbf{SANGER:} So you would advocate that they have to turn off the
oil to North Korea basically.

\textbf{TRUMP:} So much of their lifeblood comes through China, that's
the way it comes through. They have tremendous power over North Korea,
but China doesn't say that. China says well we'll try. I can see them
saying, ``We'll try, we'll try.'' And I can see them laughing in the
room next door when they're together. So China should be talking to
North Korea. But China's tweaking us. China's toying with us. They are
when they're building in the South China Sea. They should not be doing
that but they have no respect for our country and they have no respect
for our president. So, and the other one, and this is an opportunity
passed because why would Iran go back and renegotiate it having to do
with North Korea?But Iran is the No. 1 trading partner, but we should
have had something in that document that was signed having to do with
North Korea as the No. 1 trading partner and as somebody with a certain
power because of that. A very substantial power over North Korea.

\textbf{SANGER:} Mr. Trump with all due respect, I think it's China
that's the No. 1 trading partner with North Korea.

\textbf{TRUMP:} I've heard that certainly, but I've also heard from
other sources that it's Iran.

\textbf{SANGER:} Iran is a major arms exchanger with...

\textbf{TRUMP:} Well that is true but I've heard it both ways. They are
certainly major arms exchangers, which in itself is terrible that we
would make a deal with somebody that's a major arms exchanger with North
Korea. But had that deal not been done and they were desperate to do it,
and they wanted to do it much more so than we know in my opinion,
meaning Iran wanted to make the deal much more than we know. We should
have backed off that deal, doubled the sanctions and made a real deal.
And part of that deal should have been that Iran would help us with
North Korea. So, the bottom line is, I think that frankly, as long as
North Korea's there, I think that Japan having a capability is something
that maybe is going to happen whether we like it or not.

\hypertarget{boots-on-the-ground}{%
\subsection{Boots on the Ground}\label{boots-on-the-ground}}

\textbf{SANGER:} O.K.. We wanted to ask you a little bit, and Maggie
maybe you may have something on this as well, about what standards you
would use for using American troops abroad. You've said you wouldn't
want to send them in against ISIS, that that should be the neighbors.
But you did say this morning that if we have a treaty obligation under
NATO to protect the Baltics, you would do that. When you think of your
standards under which you would put American lives\ldots{}

\textbf{TRUMP:} Well I think, I do think I'd want to renegotiate some of
those treaties. I think those treaties are very unfair, and they're very
one-sided and I do think that some of those treaties, just like the Iran
deal. But I think that some of those treaties would --- will be ---
renegotiated.

\textbf{SANGER:} Such as the U.S.-Japan defense treaty?

\textbf{TRUMP:} Well, like Japan as an example. I mean that's not a fair
deal.

\textbf{SANGER:} Do you have general standards in mind? And, we're
trying to understand your hierarchy of threats.

\textbf{TRUMP:} Are you talking about for\ldots{}

\textbf{SANGER:} For when you would commit American troops abroad?

\textbf{TRUMP:} O.K. You absol --- I know you'll criticize me for this,
but you cannot just have a standard. You cannot just say that we have a
blanket standard all over the world because each instance is totally
different, David. I mean, each instance is so different that you can't
have a blanket standard. You may say\ldots{} it sounds nice to say, ``I
have a blanket standard; here's what it is.'' Number one is the
protection of our country, O.K.? That's always going to be number one,
by far. That's by a factor of a hundred. But you know, then there will
be standards for other places but it won't be a blanket standard.

\textbf{SANGER:} Humanitarian intervention: Are you in favor of that or
not?

\textbf{TRUMP:} Humanitarian? Yes, I would be. You know, to help I would
be, depending on where and who and what. And, you know, again ---
generally speaking --- I'd have to see the country; I'd have to see
what's going on in the region and you just cannot have a blanket. The
one blanket you could say is, ``protection of our country.'' That's the
one blanket. After that it depends on the country, the region, how
friendly they've been toward us. You have countries that haven't been
friendly to us that we're protecting. So it's how good they've been
toward us, et cetera, et cetera. So you can't say a blanket. You could
say standards for different areas, different regions, and different
countries.

\hypertarget{israel-and-the-palestinians}{%
\subsection{Israel and the
Palestinians}\label{israel-and-the-palestinians}}

\textbf{HABERMAN:} You had said earlier, I think, when you called David
that you had wanted to elaborate on your answer about Israel and a
two-state solution. I just wanted to...

\textbf{TRUMP:} Well, not elaborate. I just put it off because I was
running out of time and I didn't want to get into it too much because
it's actually not that. So should we talk about Israel for a little
while?

\textbf{SANGER:} Sure.

\textbf{TRUMP:} I have gotten some of the reviews of my speech at Aipac
and, really, they've been very nice. They were very nice. Were you
there? Were either of you at that speech?

\textbf{HABERMAN:} I was.

\textbf{SANGER:} I saw it on TV.

\textbf{TRUMP:} You saw the response Maggie, then, from the crowd?

\textbf{HABERMAN:} I did. I did.

\textbf{TRUMP:} Many, many standing ovations and they agreed with what I
said. Basically I support a two-state solution on Israel. But the
Palestinian Authority has to recognize Israel's right to exist as a
Jewish state. Have to do that. And they have to stop the terror, stop
the attacks, stop the teaching of hatred, you know? The children, I sort
of talked about it pretty much in the speech, but the children are
aspiring to grow up to be terrorists. They are taught to grow up to be
terrorists. And they have to stop. They have to stop the terror. They
have to stop the stabbings and all of the things going on. And they have
to recognize that Israel's right to exist as a Jewish state. And they
have to be able to do that. And if they can't, you're never going to
make a deal. One state, two states, it doesn't matter: you're never
going to be able to make a deal. Because Israel would have to have that.
They have to stop the terror. They have to stop the teaching of children
to aspire to grow up as terrorists, which is a real problem. So with
that you'd go two states, but in order to go there, before you, you
know, prior to getting there, you have to get those basic things done.

Now whether or not the Palestinians can live with that? You would think
they could. It shouldn't be hard except that the ingrained hatred is
tremendous.

\hypertarget{countering-extremism}{%
\subsection{Countering Extremism}\label{countering-extremism}}

\textbf{HABERMAN:} You had talked, and you've talked a lot recently,
about wanting to expand laws regarding torture.

\textbf{TRUMP:} Yes.

\textbf{HABERMAN:} Much of that is governed by international law.

\textbf{TRUMP:} Yes.

\textbf{HABERMAN:} How would you go about bringing changes to\ldots{}?

\textbf{TRUMP:} O.K., when you see a thing like an attack in Brussels,
when you see as an example they have somebody that they've wanted very
much, and they got him three, four days before Brussels, right? Before
the bombing. Had they immediately subjected him to very serious
interrogation --- very, very serious --- you might have stopped the
bombing. He knew about the bombing. Just like the people, just like all
of that people in the area where he grew up --- where he was housed a
couple of houses down the road --- they all knew he was there. And they
never turned him in. This is what I'm saying: there's something going on
and it's not good. He was the No. 1 wanted fugitive in the world and
he's living in his neighborhood, and I believe I saw a picture of him
shopping in his neighborhood, right? In a grocery store? You know:
shopping! Buying food! I mean, it's ridiculous they don't turn him in.
Just like in California, the two people, where she probably radicalized
him but they don't know, but the two --- the married couple --- that
killed the 14 people: they had bombs all over the floor of their
apartment and nobody said anything. And many people saw that apartment
and many people saw bombs. You know, if you walk into an apartment,
Maggie or David, you're going to say, ``Oh, this is a little strange.''

\textbf{SANGER:} So would you invest in programs, or help the Europeans
invest in programs, for counter-radicalization? For finding jobs and so
forth for the refugees who come in so that their temptation to go to
become radicalized in Europe would be lower? In other words do you have
a program in mind to stop the radicalization?

\textbf{TRUMP:} The one thing I'd do, David, is build safe zones in
Syria. You know this whole concept of us accepting, you know, tens of
thousands of people, and you see I was originally right when I said,
many more people, you know he was talking about 10,000, you know it's
many more people than 10,000 are coming in. And will come in.

\textbf{SANGER:} And who would protect those safe zones, you know as
soon as you build one\ldots{}?

\textbf{TRUMP:} O.K., what I would do is this: We could lead it, but I
would get the Gulf states and others to put up the money. I mean Germany
should put up money. Look what's happened to Germany. Germany's being
destroyed and I have friends, I just left people from Germany and they
don't even want to go back. Germany's being destroyed by Merkel's
naiveté or worse. But Germany is a whole different place and you're
going to have a problem in Germany. The German people are not going to
take it. The German people are not going to take what's going on there.
You have people leaving the country, permanently leaving the country.
You have tremendous crime, you have tremendous, you know, you read the
same stories that I do. You write them, actually, it's even better. So
you have tremendous problems over there but I do believe in building a
safe zone, a number of safe zones, in sections of Syria and that when
this war, this horrible war, is over people can go back and rebuild if
they want to and I would have the Gulf states finance it because they
have the money and they should finance it. So far, they've put up very
little money and they taken nobody in, essentially nobody in. I would be
very strong with them because they have tremendous, they have unlimited
amounts of money, and I would ask them to finance it. We can lead it but
I don't want to spend the money on it, because we don't have any money.
Our country doesn't have money.

\hypertarget{a-strong-china}{%
\subsection{A Strong China}\label{a-strong-china}}

\textbf{SANGER:} I wanted to take you back to something you said on
China earlier because your arguments about China so far have really
been, over the years, very much about how to deal with a strong and
rising China. But what we've seen in the past six months to a year has
been a China that is economically weakening. I'm sure you see it in your
own businesses there. So do you have a sense...

\textbf{TRUMP:} Well, they're down to G.D.P. of 7 percent.

\textbf{SANGER:} If you believe their numbers.

\textbf{TRUMP:} Yeah, if we ever hit 7 percent we'd have the most
successful country. We'd be in a boom, the likes of which we've rarely
seen before, right?

\textbf{SANGER:} What I'm getting at is a weakening China may have
different effects on the world and on the United States than a
strengthening China. Do you fear a weaker China or a weakening China
more than a strong China?

\textbf{TRUMP:} No. I want a strong United States and I hope China does
well, but before I worry about China I have to worry about the United
States and we're not doing well.

\textbf{SANGER:} You've given us a lot of your impressions of Vladimir
Putin. We haven't heard you very much on Xi Jinping.

\textbf{TRUMP:} Well I haven't said anything. By the way, I've been
really misquoted. Vladimir Putin said, ``Donald Trump is brilliant and
Donald Trump is a real leader. And Donald Trump will be the real
leader.'' O.K.? I didn't say anything about him other than to
say\ldots{} I said, we were on ``60 Minutes'' the same night, remember?
That was six months ago. But I never said good, bad, or indifferent. I
said he is a strong leader, he is a strong leader. But I didn't say
that, and I'm not saying that positively or negatively, I'm just saying
he's a strong leader. That's pretty obvious that he's a strong leader.

\textbf{SANGER:} What's your impression of Xi Jinping?

\textbf{TRUMP:} I think they are in a very interesting position. The
economy is going to be, I think actually very strong but the economy, I
think they're doing better than people understand. Nobody has
manipulated economic conditions better than they have. And I think
they're doing just fine and I think they will continue to do just fine.
But a lot of it's being taken out of the hide of our country and we
can't allow that to happen. You know if you look at the number of jobs
that we've lost, it's millions of jobs. It's not a little bit, it's
millions. And if you look at our phony numbers of 5 percent
unemployment, even opponents would say that, and would agree to that
fact that the jobs that we have are bad jobs. They're not good jobs,
they're bad jobs. We're losing, you know, when you see a Carrier move
into Mexico, those are good jobs. We're losing the good jobs. We now
have a lot of bad jobs, we have a lot of part-time jobs. It's not the
same country. We're losing our companies. I mean when we lose Pfizer to
Ireland, when we lose Ford and Carrier and many others to, Nabisco as an
example from Chicago to Mexico, when we lose all of these companies
going to Mexico and to many other places, we're going to end up having
no comp**---** we're going to have nothing left. And it has to be
stopped, and it has to be stopped fast and I know how to stop it. Nobody
else, the politicians don't know how to stop it. And besides that the
politicians are all taken care of by the special interests and the
lobbyists. Lobbyists for hire. And somebody will get to them and they
will pay them a lot of money and the politicians will not do what they
have to do, which is keep companies in this country. Those companies
that want to leave will get to the lobbyists and the special interests
and those politicians will do what they want them to do, which is not in
the interest of our country. O.K.

\hypertarget{lessons-learned-from-iraq}{%
\subsection{Lessons Learned From Iraq}\label{lessons-learned-from-iraq}}

\textbf{HABERMAN:} Mr. Trump, I have heard you say for years now,
including at your CPAC speech back in 2011, ``Take the oil.'' That
America should have taken the oil from Iraq.

\textbf{TRUMP:} I've said it for years.

\textbf{HABERMAN:} Why should the American\ldots{}?

\textbf{TRUMP:} Originally I didn't say it. Originally I said, ``Don't
go into Iraq.''

\textbf{HABERMAN:} Right.

\textbf{TRUMP:} Now, we went in, we destroyed a military base that was
equal to, if not greater than, Iran. And we've destroyed that military,
and they were holding each other off for many, you know for decades,
decades, and we destroyed one of those military powers. And I said don't
go in because if you dest --- now, I didn't know that they didn't have
weapons of mass destruction. But on top of everything else they had no
weapons of mass destruction.

\textbf{HABERMAN:} Well, but sir, why should the American approach to
rebuilding Iraq, or other countries where we have shed blood, why should
that differ from how we rebuilt postwar Japan and Germany in the
Marshall Plan?

\textbf{TRUMP:} Well it was much different. We rebuild Iraq and it gets
blown up. We build a school? Gets blown up. Build it again? Gets blown
up. You know, it's a mess. I mean you have government that's totally
corrupt. The country is totally, totally corrupt and corruptible. The
leader, I mean one of the big decisions that was made putting the people
in charge of Iraq that were in charge of Iraq, and they were
exclusionary. They excluded people that ultimately, you know large
groups of people, that ultimately became ISIS. Became stronger than
them. And the sad thing is, I always talk about the bad deal that we
made with Iran as being one of the worst deals, actually the worst deal
is what we've done again involving Iran, we've destroyed the military
capability of Iraq and destroyed Iraq, period, and Iran is now going to
take over Iraq, they've essentially already done that in my opinion, but
they're going to officially take over Iraq in the very near future.And I
mean Iraqis were already reporting to Iran, but Iran is going to take
over Iraq, they've wanted to do it for decades. They're going to take
over Iraq, they're going to take the oil reserves which are the second
biggest in the world, extremely high quality oil under the ground,
extremely high quality, they're going to take all of that over because
of us. Because we destroyed ---

\textbf{SANGER:} But Mr. Trump you've argued many times that you don't
want to have ground troops, but''We take the oil'' implies you're going
to have to go in there and take it by force, defend it ---

\textbf{TRUMP:} Well what I said is, I said when we left that we should
have taken the oil.

\textbf{SANGER:} If you want to take the oil today you're going to have
to go into a country that is now an ally, Iraq, even if it's a
dysfunctional one, put your troops on the ground.

\textbf{TRUMP:} Yeah, yeah, O.K.. Ready? I said take the oil. I've been
saying that for years. And many very smart scholars and military
scholars said that'd be a great thing to do, but people didn't do it.
So, but I have been saying that for years, I'm glad you know that. At
least four or five years. When we left I said take the oil. We shouldn't
have been there, we shouldn't have destroyed the country, and Saddam
Hussein was a bad guy but he was good at one thing: Killing terrorists.
He killed terrorists like nobody, all right? Now it's Harvard of
terrorism. You want to be a terrorist you go to Iraq. But he killed
terrorists. O.K., so we destroyed that. By the way, bad guy, just so you
know, officially, I want to say that, bad guy, but it was a lot better
of situation than we have right now. And he did not knock down the World
Trade Center, O.K.? So officially speaking, he did not, Iraq did not
knock down the World Trade center. We went in there after the World
Trade Center, well he didn't knock down the World Trade Center, so you
could say why are we doing this, all right, that was another thing. I
never felt that he did it, and it turned out that he didn't. And it'll
be very interesting when those documents are opened up and released in
the future, I think maybe they should be opened up and released sooner
rather than later.

\textbf{HABERMAN:} You mean the House, the House and Senate report?

\textbf{TRUMP:} Yes, yes, exactly. It'd be very interesting to see
because they must know. They must know, if they're anything, they must
know what happened in terms of who were the people. But it wasn't Iraq,
O.K.? You're not going to find that it was Iraq. So it was very faulty,
but I was, I was talking about, I was talking about taking the oil, now
we have a different situation because now we have to go in again and
start fighting, you know, at that time we had it and we should've kept
it. Now I would say knock the hell out of the oil and do it because it's
a primary source of money for ISIS.

\textbf{SANGER:} So in other words you don't want to take the oil right
now, you want to just destroy the oil fields.

\textbf{TRUMP:} Well now, we have to destroy the oil. We should've taken
it and we would've have it. Now we have to destroy the oil. We don't do
it, I just can't believe we don't do it.

\textbf{SANGER:} So you know Mr. Trump, from listening and enjoying
these two conversations we've had today which have been extremely
interesting, I've been trying to sort of fit where your worldview and
your philosophy here, your doctrine fits in with sort of the previous
Republican mainlines of inquiry. And so if you think back to George H.
W. Bush, the most recent President Bush's father, he was an
internationalist who was in the realist school, he wanted to sort of
change the foreign policy of other nations but you didn't see him
messing inside those countries and then you had a group of people around
---

\textbf{TRUMP:} Well he did the right thing, David, he did the right
thing. He went in, he knocked the hell out of Iraq and then he let it
go, O.K.? He didn't go in. Now I don't know was that Schwarzkopf, was
that, was that ---

\textbf{SANGER:} It was George W. Bush himself.

\textbf{TRUMP:} Or maybe it was him, but he didn't go in, he didn't get
into the quicksand, right? He didn't get into the quicksand and I mean,
history will show that he was right. And with that Saddam Hussein
overplayed his card more than any human being I think I've ever seen.
Instead of saying ``Wow, I got lucky'' that they didn't come in and take
this all away from me. He should've just relaxed a little bit, O.K.? And
instead he taunted Bush Sr. He taunted him. And Bush Jr. loves his
father and didn't like what was happening, but I remember very vividly
how Saddam Hussein was taunting, absolutely taunting, saying we have
beaten the Americans, you know, meaning they didn't come in so he would
tell everybody he beat them. Do you remember that, right?

\textbf{SANGER:} I do indeed.

\textbf{TRUMP:} And he was taunting to them, he was saying, and even I
used to say ``Wow'' because I knew that we could've gone further. We
went in for a short period of time and just knocked the hell out of them
and then went back, sort of gave them a lesson, but we didn't destroy
the country, we didn't destroy the grid, we didn't, you know, there was
something left. There was a lot left. And instead of just sort of saying
he got lucky and to himself, just going about, he was taunting the
Bushes. And Junior said, ``Well I'm not going to take it'' and he went
in. And you know, look that was ---

\hypertarget{america-first}{%
\subsection{`America First'}\label{america-first}}

\textbf{SANGER:} There was something else to George W. Bush, Bush 43's
philosophy. If we believed that his father was an internationalist, I
think it's fair to say, at least a lot of the people around George W.
Bush were transformational, they actually wanted to change the nature of
regime. You heard this in George W. Bush's second inaugural address.

\textbf{TRUMP:} Yeah.

\textbf{SANGER:} What you are describing to us, I think is something of
a third category, but tell me if I have this right, which is much more
of a, if not isolationist, then at least something of ``America First''
kind of approach, a mistrust of many foreigners, both our adversaries
and some of our allies, a sense that they've been freeloading off of us
for many years.

\textbf{TRUMP:} Correct. O.K.? That's fine.

\textbf{SANGER:} O.K.? Am I describing this correctly here?

\textbf{TRUMP:} I'll tell you --- you're getting close. Not
isolationist, I'm not isolationist, but I am ``America First.'' So I
like the expression. I'm ``America First.'' We have been disrespected,
mocked, and ripped off for many many years by people that were smarter,
shrewder, tougher. We were the big bully, but we were not smartly led.
And we were the big bully who was --- the big stupid bully and we were
systematically ripped off by everybody. From China to Japan to South
Korea to the Middle East, many states in the Middle East, for instance,
protecting Saudi Arabia and not being properly reimbursed for every
penny that we spend, when they're sitting with trillions of dollars, I
mean they were making a billion dollars a day before the oil went down,
now they're still making a fortune, you know, their oil is very high and
very easy to get it, very inexpensive, but they're still making a lot of
money, but they were making a billion dollars a day and we were paying
leases for bases? We're paying leases, we're paying rent? O.K.? To have
bases over there? The whole thing is preposterous. So we had, so America
first, yes, we will not be ripped off anymore. We're going to be
friendly with everybody, but we're not going to be taken advantage of by
anybody. We won't be isolationists --- I don't want to go there because
I don't believe in that. I think we'll be very worldview, but we're not
going to be ripped off anymore by all of these countries. I mean think
of it.We have \$21 trillion, essentially, very shortly, we'll be up to
\$21 trillion in debt. O.K.? A lot of that is just all of these
horrible, horrible decisions. You know, I'll give you another one, I
talked about NATO and we fund disproportionately, the United Nations, we
get nothing out of the United Nations other than good real estate
prices. We get nothing out of the United Nations. They don't respect us,
they don't do what we want, and yet we fund them disproportionately
again. Why are we always the ones that funds everybody
disproportionately, you know? So everything is like that. There's
nothing that's not like that. That's why if I win and if I go in, it's
always never sounds --- I have a woman who came up to me, I tell this
story, she said ``Mr. Trump, I think you're great, I think you're going
to be a great president, but I don't like what you say I got to make
America rich again.'' But you can't make America great again unless you
make it rich again, in other words, we're a poor nation, we're a debtor
nation, we don't have the money to do, we don't have the money to fix
our military and the reason we don't is because of the fact that because
of all of the things we've been talking about for the last 25 min and
other things.

\hypertarget{when-america-was-great}{%
\subsection{When America Was `Great'}\label{when-america-was-great}}

\textbf{HABERMAN:} Mr. Trump, you --- I was looking back at your speech
in New Hampshire back in 1987 when you were releasing ``The Art of the
Deal'' and a lot of your concerns are very similar to the ones you're
voicing now.

\textbf{TRUMP:} Right, even similar countries.

\textbf{HABERMAN:} Right, and I'm just wondering what is the era when
you think the United States last had the right balance, either in terms
of defense footprint or in terms of trade?

\textbf{TRUMP:} Well sometime long before that. Because one of the
presidents that I really liked was Ronald Reagan but I never felt on
trade we did great. O.K.? So it was actually, it would be long before
that.

\textbf{SANGER:} So was it Eisenhower, was it Truman, was it F.D.R.?

\textbf{TRUMP:} No if you really look at it, it was the turn of the
century, that's when we were a great, when we were really starting to go
robust. But if you look back, it really was, there was a period of time
when we were developing at the turn of the century which was a pretty
wild time for this country and pretty wild in terms of building that
machine, that machine was really based on entrepreneurship etc, etc. And
then I would say, yeah, prior to, I would say during the 1940s and the
late `40s and `50s we started getting, we were not pushed around, we
were respected by everybody, we had just won a war, we were pretty much
doing what we had to do, yeah around that period.

\textbf{SANGER:} So basically Truman, Eisenhower, the beginning of the
1947 national security reviews, that's the period?

\textbf{TRUMP:} Yes, yes. Because as much as I liked Ronald Reagan, he
started Nafta, now Clinton really was the one that --- Nafta has been a
disaster for our country, O.K., and Clinton is the one as you know that
got it done, but it was conceived even before Clinton, but you could say
that maybe those people didn't want done what was ultimately signed
because it was changed a lot by the time it got finalized. But Nafta has
been a disaster for our country.

\textbf{SANGER:} But you think of that period time that you most admire:
late `40s, early `50s, it was also the most terrifying time with the
build up of the Cold War, it's when the Russians got nuclear weapons, we
got into an arms race, we were ---

\textbf{TRUMP:} But David, a lot of that was just pure technology. The
technology was really coming in at that time. And so a lot of that was
just timing of technology.

\textbf{SANGER:} It was also a period of time when we were threatening
to use nuclear weapons against the North Koreans and the Chinese in the
war. Was that approach you saw of Douglas MacArthur's approach at that
time, so forth, is that what you're admiring?

\textbf{TRUMP:} Well I was a fan as you probably know, I was a fan of
Douglas MacArthur. I was a fan of George Patton. If we had Douglas
MacArthur today or if we had George Patton today and if we had a
president that would let them do their thing you wouldn't have ISIS,
O.K.? You wouldn't be talking about ISIS right now, we'd be talking
about something else, but you wouldn't be talking about ISIS right now.
So I was a fan of Douglas MacArthur, I was a fan of --- as generals ---
I was a fan of George Patton. We don't have, we don't have seemingly
those people today, now I know they exist, I know we have some very, I
know the Air Force Academy and West Point and Annapolis, I know that
great people come out of those schools. A lot of times the people that
get to the top aren't necessarily those people anymore because they're
politically correct. George Patton was not a politically correct person.

\textbf{SANGER:} Yeah I think we can all agree on that.

\textbf{TRUMP:} He was a great general and his soldiers would do
anything for him.

\textbf{SANGER:} But the other day, I'm sorry, this morning, you
suggested to us you would only use nuclear weapons as a last resort.

\textbf{TRUMP:} Totally last resort.

\textbf{SANGER:} And what did Douglas MacArthur advocate?

\textbf{TRUMP:} I would hate, I would hate ---

\textbf{SANGER:} General MacArthur wanted to go use them against the
Chinese and the North Koreans, not as a last resort.

\textbf{TRUMP:} That's right. He did. Yes, well you don't know if he
wanted to use them but he certainly said that at least.

\textbf{SANGER:} He certainly asked Harry Truman if he could.

\textbf{TRUMP:} Yeah, well, O.K.. He certainly talked it and was he
doing that to negotiate, was he doing that to win? Perhaps. Perhaps. Was
he doing that for what reason? I mean, I think he played, he did play
the nuclear card but he didn't use it, he played the nuclear card. He
talked the nuclear card, did he do that to win? Maybe, maybe, you know,
maybe that's what got him victory. But in the meantime he didn't use
them. So, you know. So, we need a different mind set. So you talked
about torture before, well what did it say --- well I guess you had
enough and I hope you're going to treat me fairly and if you're not
it'll be forgotten in three or four days and that'll be the story. It is
a crazy world out there, I've never seen anything like it, the volume of
press that I'm getting is just crazy. It's just absolutely crazy, but
hopefully you'll treat me fairly, I do know my subject and I do know
that our country cannot continue to do what it's doing. See, I know many
people from China, I know many people from other countries, I deal at a
very high level with people from various countries because I've become
very international. I'm all over the world with deals and people and
they can't believe what their countries get away with. I can tell you
people from China cannot believe what their country's, what their
country's getting away with. At let's say free trade, where, you know,
it's free there but it's not free here. In other words, we try sell ---
it's very hard for us to do business in China, it's very easy for China
to do business with us. Plus with us there's a tremendous tax that we
pay when we go into China, where's when China sells to us there's no
tax. I mean, it's a whole double standard, it's so crazy, and they
cannot believe they get away with it, David. They cannot believe they
get away with it. They are shocked, and I'm talking about people at the
highest level, people at --- the richest people, people with great
influence over, you know, together with the leaders and they cannot
believe it. Mexico can't believe what they get away with. When I talked
about Mexico and I talked about they will build a wall, when you look at
the trade deficit we have with Mexico it's very easy, it's a tiny
fraction of what the cost of the wall is. The wall is a tiny fraction of
what the cost of the deficit is. When people hear that they say ``Oh now
I get it.'' They don't get it. But Mexico will pay for the wall. But
they can't believe what they get away with. There's such a double
standard. With many countries. It's almost, we do well with almost
nobody anymore and a lot of that is because of politics as we know it,
political hacks get appointed to negotiate with the smartest people in
China, when we negotiate deals with China, China is putting the smartest
people in all of China on that negotiation, we're not doing that. So
anyway, I hope you guys are happy.

\textbf{SANGER:} Thank you, you've been very generous with your time.

Advertisement

\protect\hyperlink{after-bottom}{Continue reading the main story}

\hypertarget{site-index}{%
\subsection{Site Index}\label{site-index}}

\hypertarget{site-information-navigation}{%
\subsection{Site Information
Navigation}\label{site-information-navigation}}

\begin{itemize}
\tightlist
\item
  \href{https://help.nytimes.com/hc/en-us/articles/115014792127-Copyright-notice}{©~2020~The
  New York Times Company}
\end{itemize}

\begin{itemize}
\tightlist
\item
  \href{https://www.nytco.com/}{NYTCo}
\item
  \href{https://help.nytimes.com/hc/en-us/articles/115015385887-Contact-Us}{Contact
  Us}
\item
  \href{https://www.nytco.com/careers/}{Work with us}
\item
  \href{https://nytmediakit.com/}{Advertise}
\item
  \href{http://www.tbrandstudio.com/}{T Brand Studio}
\item
  \href{https://www.nytimes.com/privacy/cookie-policy\#how-do-i-manage-trackers}{Your
  Ad Choices}
\item
  \href{https://www.nytimes.com/privacy}{Privacy}
\item
  \href{https://help.nytimes.com/hc/en-us/articles/115014893428-Terms-of-service}{Terms
  of Service}
\item
  \href{https://help.nytimes.com/hc/en-us/articles/115014893968-Terms-of-sale}{Terms
  of Sale}
\item
  \href{https://spiderbites.nytimes.com}{Site Map}
\item
  \href{https://help.nytimes.com/hc/en-us}{Help}
\item
  \href{https://www.nytimes.com/subscription?campaignId=37WXW}{Subscriptions}
\end{itemize}
