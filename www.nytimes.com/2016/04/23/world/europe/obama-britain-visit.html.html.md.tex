Sections

SEARCH

\protect\hyperlink{site-content}{Skip to
content}\protect\hyperlink{site-index}{Skip to site index}

\href{https://www.nytimes.com/section/world/europe}{Europe}

\href{https://myaccount.nytimes.com/auth/login?response_type=cookie\&client_id=vi}{}

\href{https://www.nytimes.com/section/todayspaper}{Today's Paper}

\href{/section/world/europe}{Europe}\textbar{}Obama Warns Britain on
Trade if It Leaves European Union

\url{https://nyti.ms/1WKfc2l}

\begin{itemize}
\item
\item
\item
\item
\item
\end{itemize}

Advertisement

\protect\hyperlink{after-top}{Continue reading the main story}

Supported by

\protect\hyperlink{after-sponsor}{Continue reading the main story}

\hypertarget{obama-warns-britain-on-trade-if-it-leaves-european-union}{%
\section{Obama Warns Britain on Trade if It Leaves European
Union}\label{obama-warns-britain-on-trade-if-it-leaves-european-union}}

\includegraphics{https://static01.nyt.com/images/2016/04/23/world/video-obama/video-obama-videoSixteenByNineJumbo1600.jpg}

By \href{http://www.nytimes.com/by/michael-d-shear}{Michael D. Shear}
and \href{http://www.nytimes.com/by/steven-erlanger}{Steven Erlanger}

\begin{itemize}
\item
  April 22, 2016
\item
  \begin{itemize}
  \item
  \item
  \item
  \item
  \item
  \end{itemize}
\end{itemize}

LONDON --- President Obama on Friday bluntly urged Britain to vote to
remain inside the European Union in a referendum scheduled for June 23,
and warned that a Britain outside the bloc could not count on
maintaining its current economic relationship with the United States.

Taking an unusually direct position on another country's internal
politics, Mr. Obama asserted that Britain's membership in the bloc did
not limit British influence but ``magnifies it.''

Speaking alongside Prime Minister David Cameron at a news conference, he
also directly addressed the potential consequences of a vote by Britain
to leave. The president said that to do so would send Britain to the
``back of the queue'' for a trade deal with the United States,
challenging those who have argued that Britain could quickly replicate
the same favorable terms it enjoys as a European Union member.

Mr. Cameron is leading the campaign to remain part of Europe, but the
issue has deeply divided his Conservative Party and polls suggest that
the outcome could be close, making the forcefulness of Mr. Obama's
statements especially striking.

Mr. Obama was stating his view of American national interests, while
also clearly trying to support Mr. Cameron. But the arguments here are
fierce and increasingly bitter, and Mr. Obama was attacked as a
hypocrite and worse by those who favor a British exit, or Brexit.

Mr. Obama defended the right of a close friend to give an opinion on a
matter of mutual interest. ``Part of being friends is being honest, and,
speaking honestly, the outcome of that referendum is a matter of deep
interest to the United States, because it affects our interests as
well,'' he said.

Mr. Obama sidestepped a question on whether the ``special relationship''
between Washington and London would be damaged if Britain voted to leave
the European Union.

Nor did Mr. Obama comment
on\href{http://www.nytimes.com/2016/04/23/world/europe/boris-johnson-the-sun-brexit.html}{a
suggestion by Mayor Boris Johnson} of London, a leader of the campaign
to leave the bloc, that the president was unfriendly to Britain because
of his ancestry. Mr. Johnson, a Conservative, suggested on Friday that
Mr. Obama removed a bust of Winston Churchill from the Oval Office
because it ``was a symbol of the part-Kenyan president's ancestral
dislike of the British Empire, of which Churchill had been such a
fervent defender.''

Mr. Obama said he saw another bust of Churchill every day in the White
House residence.

``I love the guy,'' he said. But as the first African-American
president, he said, he ``thought it appropriate'' to have a bust of the
Rev. Dr. Martin Luther King Jr. in the Oval Office.

Mr. Cameron, for his part, smiled thinly and said that ``questions for
Boris are questions for Boris and not questions for me.''

The prime minister praised Mr. Obama for his friendship and ``sage
advice,'' and said that Britain was made stronger through its continued
membership in the European Union and that ``the stronger we are, the
stronger that special relationship'' with the United States will be. Mr.
Cameron noted that ``it was hard to find any country that wishes Britain
well that wants us to leave the E.U.,'' and that Britons should listen
to their friends and then vote as they choose.

Strong historical, emotional, cultural and security ties with Britain
would continue no matter the vote, Mr. Obama said. But the United
States, he said, was convinced that Britain made a shaky Europe stronger
and more stable by its membership and that the bloc ``enhances''
Britain's ``influence and power and economy.''

\includegraphics{https://static01.nyt.com/images/2016/04/23/world/video-obama-cameron/video-obama-cameron-videoSixteenByNineJumbo1600.jpg}

He argued that the United States accepted constraints on its
sovereignty, too, in multilateral institutions like NATO, the United
Nations Security Council, the Group of 7 and Group of 20, and did so for
the common good, which also was in America's interests.

Earlier on Friday, Mr. Obama and his wife, Michelle, traveled by
helicopter to Windsor Castle for lunch with Queen Elizabeth II, who
turned 90 on Thursday, and her husband, Prince Philip, 94.

The Obamas also had a private dinner Friday night with Prince William
and his wife, Kate, the Duke and Duchess of Cambridge, and Prince Harry
at Kensington Palace in London.

Mr. Obama's comments at the news conference underlined his
argument,\href{http://www.telegraph.co.uk/news/2016/04/21/as-your-friend-let-me-tell-you-that-the-eu-makes-britain-even-gr/}{made
in an op-ed piece} published on Friday in The Daily Telegraph, that
Britain is stronger and more influential inside the European Union.

The president's decision to wade into the issue during his two-day visit
prompted objections from many supporters of the campaign to leave the
European Union. Nigel Farage, leader of the U.K. Independence Party,
said that Mr. Obama should ``butt out.''

One of the most polarizing responses came from Mr. Johnson, who was born
in Manhattan and retains his American passport as a dual citizen. Mr.
Johnson, who is also a member of Parliament and has ambitions to replace
Mr. Cameron as prime minister, has argued that the United States is a
traditional nation-state that would never transfer some of its
sovereignty to any European Union-like organization.

``For the United States to tell us in the U.K. that we must surrender
control of so much of our democracy --- it is a breathtaking example of
the principle of do-as-I-say-but-not-as-I-do,'' Mr. Johnson
\href{http://www.thesun.co.uk/sol/homepage/news/politics/7095695/UK-and-America-can-better-friends-than-ever-Mr-Obama-if-we-LEAVE-the-EU-says-Boris-Johnson.html}{wrote
in The Sun, a newspaper} that is influential among working-class voters.

``It is incoherent. It is inconsistent, and yes it is downright
hypocritical,'' Mr. Johnson said.

On Twitter, Nicholas Soames, a Conservative member of Parliament and a
grandson of Churchill, condemned Mr. Johnson's essay, calling it ``an
appalling article'' that is ``totally wrong on almost everything.'' It
was ``inconceivable,'' Mr. Soames said, that his grandfather would ``not
have welcomed'' the president's views on Britain's role in Europe.

A former leader of the Liberal Democrats, Menzies Campbell, said that
``many people will find Boris Johnson's loaded attack on President
Obama's sincerity deeply offensive,'' and Diane Abbott, a Labour Party
lawmaker, said that ``Boris dismissing President Obama as `half-Kenyan'
reflects the worst Tea Party rhetoric.''

American officials suggested that there was an internal debate about the
wisdom of the straightforward opinion editorial Mr. Obama wrote in The
Telegraph, but decided that it was better to be upfront about the
president's views as he arrived in Britain and not pretend to be coy.

Mr. Cameron clearly favored a strong message from the American
president, whose position is shared by the leaders of Britain's main
European allies, France and Germany.

Britain and France are the two strongest military powers in Europe, and
Britain is the second-largest economy in the European Union and fifth
largest in the world.

The economic risks of Britain's exit from the union are important to
British voters, but so are immigration and the inability of Britain to
limit the number of European Union citizens who want to live and work
here. While studies show that the immigrants contribute considerably
more to the British budget in taxes than they receive in benefits, there
are worries that the immigrants are taking jobs away from Britons.

These arguments are likely to be more important than those about
Britain's standing in the world or the country's relationship with the
United States.

Advertisement

\protect\hyperlink{after-bottom}{Continue reading the main story}

\hypertarget{site-index}{%
\subsection{Site Index}\label{site-index}}

\hypertarget{site-information-navigation}{%
\subsection{Site Information
Navigation}\label{site-information-navigation}}

\begin{itemize}
\tightlist
\item
  \href{https://help.nytimes.com/hc/en-us/articles/115014792127-Copyright-notice}{©~2020~The
  New York Times Company}
\end{itemize}

\begin{itemize}
\tightlist
\item
  \href{https://www.nytco.com/}{NYTCo}
\item
  \href{https://help.nytimes.com/hc/en-us/articles/115015385887-Contact-Us}{Contact
  Us}
\item
  \href{https://www.nytco.com/careers/}{Work with us}
\item
  \href{https://nytmediakit.com/}{Advertise}
\item
  \href{http://www.tbrandstudio.com/}{T Brand Studio}
\item
  \href{https://www.nytimes.com/privacy/cookie-policy\#how-do-i-manage-trackers}{Your
  Ad Choices}
\item
  \href{https://www.nytimes.com/privacy}{Privacy}
\item
  \href{https://help.nytimes.com/hc/en-us/articles/115014893428-Terms-of-service}{Terms
  of Service}
\item
  \href{https://help.nytimes.com/hc/en-us/articles/115014893968-Terms-of-sale}{Terms
  of Sale}
\item
  \href{https://spiderbites.nytimes.com}{Site Map}
\item
  \href{https://help.nytimes.com/hc/en-us}{Help}
\item
  \href{https://www.nytimes.com/subscription?campaignId=37WXW}{Subscriptions}
\end{itemize}
