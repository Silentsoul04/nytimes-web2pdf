Sections

SEARCH

\protect\hyperlink{site-content}{Skip to
content}\protect\hyperlink{site-index}{Skip to site index}

\href{https://www.nytimes.com/section/world/asia}{Asia Pacific}

\href{https://myaccount.nytimes.com/auth/login?response_type=cookie\&client_id=vi}{}

\href{https://www.nytimes.com/section/todayspaper}{Today's Paper}

\href{/section/world/asia}{Asia Pacific}\textbar{}As North Korea's
Nuclear Program Advances, U.S. Strategy Is Tested

\url{https://nyti.ms/1q4cFml}

\begin{itemize}
\item
\item
\item
\item
\item
\end{itemize}

Advertisement

\protect\hyperlink{after-top}{Continue reading the main story}

Supported by

\protect\hyperlink{after-sponsor}{Continue reading the main story}

\hypertarget{as-north-koreas-nuclear-program-advances-us-strategy-is-tested}{%
\section{As North Korea's Nuclear Program Advances, U.S. Strategy Is
Tested}\label{as-north-koreas-nuclear-program-advances-us-strategy-is-tested}}

\includegraphics{https://static01.nyt.com/images/2016/05/07/world/07korea-web/07korea-web-articleLarge.jpg?quality=75\&auto=webp\&disable=upscale}

By \href{http://www.nytimes.com/by/david-e-sanger}{David E. Sanger} and
\href{http://www.nytimes.com/by/choe-sang-hun}{Choe Sang-Hun}

\begin{itemize}
\item
  May 6, 2016
\item
  \begin{itemize}
  \item
  \item
  \item
  \item
  \item
  \end{itemize}
\end{itemize}

SEOUL, South Korea --- After years of trying to separate fact from
propaganda about North Korea's nuclear program, American and South
Korean intelligence officials say they have concluded that the country
can now mount a small nuclear warhead on short- and medium-range
missiles capable of hitting much of Japan and South Korea.

The United States and its allies have sought for nearly a decade to
prevent the North from gaining such capabilities, ever since it
detonated its first atomic device a decade ago. Their failure is likely
to raise new questions about the effectiveness of the policy toward
North Korea, while ushering the long-simmering nuclear standoff with the
North into a more perilous phase under its combative young leader, Kim
Jong-un.

The assessment of the North's new capabilities is not based on direct
evidence from inside its nuclear program, senior officials said, but
draws on intelligence gleaned from high-level defectors, analysis of
propaganda images and data collected from North
Korean\href{http://www.nytimes.com/topic/subject/north-koreas-nuclear-program?8qa}{missile
and nuclear tests}, which have accelerated over the past six months.

While some intelligence agencies suggested as early as 2013 that the
North had learned enough about rocket engineering and the
miniaturization of nuclear warheads to mount one on a shorter-range
missile, there is a new consensus and greater confidence in that view in
both Washington and Seoul, the officials said.

Given the years of research North Korea has devoted to the program,
experts do not consider the conclusion particularly surprising. But the
politics of the assessment, which means the North can target American
bases in South Korea and Japan, are delicate, both in the region and in
the midst of a presidential election in the United States.

\includegraphics{https://static01.nyt.com/images/2016/05/07/world/07korea-web2/07korea-web2-articleLarge.jpg?quality=75\&auto=webp\&disable=upscale}

The Obama administration and the South Korean government are reluctant
to discuss the North's new capabilities publicly. Stung by the fiasco
over whether unconventional weapons existed in Iraq 13 years ago,
American intelligence officials say they no longer advertise conclusions
about other nations' capabilities, and a senior South Korean government
official who described the assessment to foreign reporters insisted on
anonymity.

The officials said the public silence reflected an effort to avoid
strengthening and encouraging Mr. Kim, who has doubled down on the
nuclear program begun by his grandfather and father and has used it to
tighten his grip on power. Publicly acknowledging the North's advances
would play into Mr. Kim's narrative that only he can protect his nation,
by defying its enemies and building a nuclear arsenal, the officials
said.

Victor Cha, who was a senior official on President George W. Bush's
National Security Council, said American policy had been ``concerned
about not overreacting to every North Korean provocation, and that made
sense when their capabilities were not all that formidable.''

``But now they have been in a spiral of escalation,'' he said, ``and we
are underreacting when their capabilities are accelerating.''

Park Ji-young, a nuclear policy analyst at the Asan Institute for Policy
Studies in Seoul, said officials did not want to discuss the North's new
capabilities ``because they don't know exactly how they can stop them,
and because they don't want to scare the people.''

A master of bombast, Mr. Kim appears increasingly volatile during his
fifth year in power.
\href{http://www.nytimes.com/2016/05/07/world/asia/north-korea-congress.html?_r=0}{In
a speech on Friday} to the first congress of his Workers' Party in 36
years, he boasted that his nuclear weapons and missile programs brought
his country ``dignity and national power.''

Image

A laboratory where North Korea separates weapons-grade plutonium from
waste from a nuclear reactor.Credit...38 North, via Associated Press

A few weeks ago, he posed with what appeared to be a mock-up of a small
nuclear warhead, and his government released a video depicting a nuclear
strike on the Lincoln Memorial.

But experts say North Korea is years away from deploying an
intercontinental ballistic missile capable of striking the mainland
United States with a nuclear payload, and even then no one sees the
backward nation taking the enormous strides needed to build a much more
destructive hydrogen warhead, capable of leveling cities.

Still, the North's new capabilities have prompted a rethinking of
American military strategy in Asia. ``We know they have nuclear weapons
and the means to deliver them,'' Gen. Robert B. Neller, commandant of
the Marine Corps, said on Tuesday at the Council on Foreign Relations in
New York. ``If that's where they are going, that changes the calculus.''

For President Obama, who has completed a nuclear deal with Iran and
renewed diplomatic relations with Cuba and Myanmar, the advances in
North Korea highlight its status as the rogue state that got away.

Mr. Obama has pursued a policy of ``strategic patience'' --- not
overreacting to the North's missile and nuclear tests, while using
sanctions to press it to negotiate. But North Korea has refused to
accept his demand that it commit to denuclearization as a goal before
talks begin.

Instead, it has accelerated its nuclear effort, conducting tests in
2006, 2009 and 2013, and in January. The two most recent tests took
place under Mr. Kim, and South Korean officials say the North may
attempt a fifth nuclear test soon, perhaps to mark the party congress.

In March, Mr. Kim specified that the next test should involve a
``nuclear warhead explosion.'' Analysts said that suggested that the
North might be on the verge of demonstrating progress in making a
smaller device, building on previous tests that were perhaps more
focused on the basics of detonation.

Shrinking a nuclear weapon is important because the smaller it is, the
easier it will be for a missile to lift it and the farther the missile
can fly. American and South Korean officials said they believed that
North Korea could make a nuclear warhead small enough to mount on its
midrange Nodong missile, which usually carries a 1,500-pound payload but
can carry as much as 2,200 pounds over shorter distances.

``Given the time that has elapsed since its first nuclear test, we
believe that North Korea has achieved a significant level of
miniaturization,'' Han Min-koo, South Korea's defense minister, said in
March. He also noted that North Korea had conducted more missile tests
under Mr. Kim than during his father's entire 17-year rule.

But, Mr. Han said, the North has not mastered the complex technology
needed to protect a nuclear warhead from destruction as an
intercontinental ballistic missile re-enters the atmosphere.

Soon after Mr. Kim took power, American satellites began picking up
pictures of
\href{http://www.nytimes.com/2013/01/18/world/asia/north-koreas-missile-movements-worry-us.html}{mobile
missile launchers, which are harder to find and to target}.

The missile launchers were of Chinese design, and the missiles resembled
Russian weaponry. The Nodong, sometimes spelled Rodong, is a modified
version of the Scud missile and can reach American bases in Japan.
Another missile, the Musudan, can target American bases as far as Guam.

In April, North Korea
\href{http://www.nytimes.com/2016/04/15/world/asia/north-korea-ballistic-missile-launch-a-failure-pentagon-says.html}{tested
the Musudan}
\href{http://www.nytimes.com/2016/04/29/world/asia/north-korea-missile-test.html}{three
times}, but it crashed into the sea or exploded seconds after takeoff
each time. North Korea also has a long record of problems in its effort
to develop intercontinental missiles, including an
\href{http://www.nytimes.com/2012/04/13/world/asia/north-korea-launches-rocket-defying-world-warnings.html}{embarrassing
failure in April 2012}, just months after Mr. Kim took power.

But there have been enough successes to worry American commanders. In
February, North Korea
\href{http://www.nytimes.com/2016/02/07/world/asia/north-korea-moves-up-rocket-launching-plan.html}{put
a satellite in orbit} with a three-stage rocket that, if successfully
reconfigured as a missile, some analysts believe could reach the West
Coast of the United States.

A new concern is recent tests of a submarine-launched missile.
Deployment is likely to be years away. But submarines could stealthily
move missiles within range of additional targets and give the North a
``second strike'' capability --- to launch after its land-based arsenal
has been destroyed.

North Korea first claimed to have launched
\href{http://www.nytimes.com/2015/05/09/world/asia/north-korea-says-it-test-fired-missile-from-submarine.html}{a
ballistic missile from a submarine} a year ago, but photos released of
Mr. Kim observing the test appeared to have been doctored.
\href{http://www.nytimes.com/2016/01/13/world/asia/north-korea-faked-test-video-group-says.html}{Video}
released from another test in December suggested that the missile was
launched from a sunken barge, not a submarine.

On April 23, North Korea conducted
\href{http://www.nytimes.com/2016/04/24/world/asia/north-korea-fires-ballistic-missile-from-submarine-south-says.html}{another
launch}, apparently from a 2,000-ton Sinpo-class submarine. But the
missile did not travel far, officials said.

\href{http://38north.org/author/john-schilling/}{John Schilling}, an
expert on North Korea's missile program, has estimated that the North
may have an operational system by 2020. But its current submarines are
old and noisy, must surface frequently and cannot make it across the
Pacific to North America.

The question now, for both President Obama and his successor, is whether
to set new red lines beyond which the North Korean nuclear program
cannot go --- or whether drawing those lines will only encourage the
North to step over them, as it has done before.

Gary Samore, Mr. Obama's top nuclear adviser in his first term, said the
policy of ``strategic patience'' had failed to change the North's
calculations. ``But that doesn't mean you just build more missile
defenses and walk away,'' he said. ``We need some kind of process to
begin to freeze what they are doing.''

The more progress North Korea makes, though, the less willing it may be
to stop.

\href{http://www.brookings.edu/experts/einhornr}{Robert J. Einhorn}, a
leading expert on proliferation, said a crucial question was whether Mr.
Kim would dig in and ``refuse to cap their capability before they are
able to deliver an ICBM with a warhead to the homeland.''

There are concerns aside from missile capabilities. ``Should developing
a long-range nuclear missile be the next red line? Or does that make
less sense when the North could sell a bomb to a terror group, or put
one in a basement in a big city?'' asked
\href{http://csis.org/expert/sam-nunn}{Sam Nunn}, co-chairman of the
Nuclear Threat Initiative.

For now, the new American response looks a lot like the old American
response, with the same weakness: China's fear of destabilizing its
neighbor with sanctions that hurt too much.

New sanctions enacted with China's support after the January nuclear
test have the potential to bottle up North Korean ships as they visit
ports around the world. But the sanctions are difficult to enforce, and
there is no requirement to cut off fuel shipments, which come almost
entirely from China.

``So far we have Kim Jung-un to thank for driving the Chinese in our
direction,'' Mr. Samore said. ``But they are still primarily worried
about a collapse in the North'' that leaves South Korean and American
forces on the Chinese border.

Advertisement

\protect\hyperlink{after-bottom}{Continue reading the main story}

\hypertarget{site-index}{%
\subsection{Site Index}\label{site-index}}

\hypertarget{site-information-navigation}{%
\subsection{Site Information
Navigation}\label{site-information-navigation}}

\begin{itemize}
\tightlist
\item
  \href{https://help.nytimes.com/hc/en-us/articles/115014792127-Copyright-notice}{©~2020~The
  New York Times Company}
\end{itemize}

\begin{itemize}
\tightlist
\item
  \href{https://www.nytco.com/}{NYTCo}
\item
  \href{https://help.nytimes.com/hc/en-us/articles/115015385887-Contact-Us}{Contact
  Us}
\item
  \href{https://www.nytco.com/careers/}{Work with us}
\item
  \href{https://nytmediakit.com/}{Advertise}
\item
  \href{http://www.tbrandstudio.com/}{T Brand Studio}
\item
  \href{https://www.nytimes.com/privacy/cookie-policy\#how-do-i-manage-trackers}{Your
  Ad Choices}
\item
  \href{https://www.nytimes.com/privacy}{Privacy}
\item
  \href{https://help.nytimes.com/hc/en-us/articles/115014893428-Terms-of-service}{Terms
  of Service}
\item
  \href{https://help.nytimes.com/hc/en-us/articles/115014893968-Terms-of-sale}{Terms
  of Sale}
\item
  \href{https://spiderbites.nytimes.com}{Site Map}
\item
  \href{https://help.nytimes.com/hc/en-us}{Help}
\item
  \href{https://www.nytimes.com/subscription?campaignId=37WXW}{Subscriptions}
\end{itemize}
