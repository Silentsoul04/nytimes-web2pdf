Sections

SEARCH

\protect\hyperlink{site-content}{Skip to
content}\protect\hyperlink{site-index}{Skip to site index}

\href{https://www.nytimes.com/section/business}{Business}

\href{https://myaccount.nytimes.com/auth/login?response_type=cookie\&client_id=vi}{}

\href{https://www.nytimes.com/section/todayspaper}{Today's Paper}

\href{/section/business}{Business}\textbar{}Health Insurer Hoped to
Disrupt the Industry, but Struggles in State Marketplaces

\url{https://nyti.ms/200Nrly}

\begin{itemize}
\item
\item
\item
\item
\item
\end{itemize}

Advertisement

\protect\hyperlink{after-top}{Continue reading the main story}

Supported by

\protect\hyperlink{after-sponsor}{Continue reading the main story}

\hypertarget{health-insurer-hoped-to-disrupt-the-industry-but-struggles-in-state-marketplaces}{%
\section{Health Insurer Hoped to Disrupt the Industry, but Struggles in
State
Marketplaces}\label{health-insurer-hoped-to-disrupt-the-industry-but-struggles-in-state-marketplaces}}

\includegraphics{https://static01.nyt.com/images/2016/06/08/business/09OSCAR-1/09OSCAR-1-articleLarge.jpg?quality=75\&auto=webp\&disable=upscale}

By \href{http://www.nytimes.com/by/reed-abelson}{Reed Abelson}

\begin{itemize}
\item
  June 19, 2016
\item
  \begin{itemize}
  \item
  \item
  \item
  \item
  \item
  \end{itemize}
\end{itemize}

\href{https://www.hioscar.com/}{Oscar Health} was going to be a new kind
of insurance company. Started in 2012, just in time to offer plans to
people buying insurance under the new federal health care law, the
business promised to use technology to push less costly care and more
consumer-friendly coverage.

``We're trying to build something that's going to turn the industry on
its head,'' Joshua Kushner, one of the company's founders, said in 2014,
as Oscar began to enroll its first customers.

These days, though, Oscar is more of a case study in how brutally tough
it is to keep a business above water in the state marketplaces created
under the Affordable Care Act. And its struggles highlight a critical
question about the act: Can insurance companies run a viable business in
the individual market?

Oscar has attracted 135,000 customers, about half of them in New York
State. And some of its efforts with technology have been successful. But
for every dollar of premium Oscar collects in New York, the company is
losing 15 cents. It lost \$92 million in the state last year and another
\$39 million in the first three months of 2016.

``That's not a sustainable position,'' said Mario Schlosser, chief
executive at Oscar.

Companies like Oscar were initially attracted by the potential of
millions of new customers added to the individual market by the health
law. But the reality has been far messier.

In an effort to attract customers, insurers put prices on their plans
that have turned out to be too low to make a profit. The companies also
assumed they could offer the same sort of plans as they do through
employer-based coverage, including broad networks of doctors and
hospitals.

But the market has turned out to be smaller than they hoped, with 12
million signed up for coverage in 2016.
\href{https://www.nytimes.com/2016/04/05/business/employers-keep-health-insurance-despite-affordable-care-act.html?_r=0}{Fewer
employers have dropped health insurance} than expected, for example,
keeping many healthy adults out of the individual market.

And among the remaining population, the insurers cannot pick and choose
their customers. The law forces them to insure people with pre-existing
conditions, no matter how expensive those conditions may be.

As a result, most insurers are still trying to develop a successful
business model. Last year, only a quarter of the insurers appear to have
made money selling individual policies, according to a preliminary
analysis from
\href{http://healthcare.mckinsey.com/sites/default/files/Intel\%20Brief\%20-\%20Individual\%20Market\%20Performance\%20and\%20Outlook\%20\%28public\%29_vF.pdf}{McKinsey},
the consulting firm. Giant insurers like UnitedHealth Group have stopped
offering individual coverage through the public exchanges in some
states. And most of the new insurance co-ops, which were founded to
create more competition, have failed.

\includegraphics{https://static01.nyt.com/images/2016/06/08/business/09OSCAR-2/09OSCAR-2-articleLarge.jpg?quality=75\&auto=webp\&disable=upscale}

The heavy losses do not necessarily mean that the individual market is
ready to implode. Some insurers, including large companies like Anthem,
say they remain committed to the market, and some insurers have made
money.

But the turbulence is certainly greater than expected. And it may well
lead many insurers to seek double-digit percentage rate increases and
tighten their networks.

``There was tremendous uncertainty that even the very established
companies were flummoxed by,'' said Larry Levitt, an executive with the
\href{http://kff.org/}{Kaiser Family Foundation}, which has been closely
following the insurers' progress.

Over all, insurance companies continue to make profits. The dearth of
profits from the individual markets, though, show how challenging it is
to make insurance affordable when it is not subsidized by the government
or an employer.

The troubles in the individual market also underscore how some of the
law's provisions meant to protect the insurers have not worked as well
as desired. Insurers did not receive all the payments they were due
under one of the law's provisions, and another provision, meant to even
out the risk among companies to protect those that enroll sicker
individuals, has been described as flawed by many health care experts.
Federal officials have said they would tweak those formulas.

The companies that have fared best so far are those that have kept the
tightest control over their costs, by working closely with low-cost
providers or a limited group of hospitals and doctors. Many have
abandoned the idea of offering the kind of access available through many
employer plans. The successful companies have also avoided the very low
prices found in some of the co-ops.

For most of the insurers, though, the math has just not added up, which
is the case with Oscar.

In New York State, where Oscar is based, the company recently filed
eye-catching requests to raise rates by a weighted average of nearly 20
percent for 2017. Regulators will make a decision in August.

``The market is over all too low in price,'' Mr. Schlosser said. ``We,
like everybody else, have priced in a very aggressive way.''

Many of the big insurers, like Anthem, can rely on their other
businesses to generate profits while they wait for this market to
stabilize. Oscar does not have that luxury; it is focused on individual
marketplaces. (In addition to New York, Oscar operates in California,
New Jersey and Texas.)

Other new insurers that sell plans to employers or under government
programs like Medicare have been a little more insulated. When Northwell
Health, the system in New York previously known as North Shore-LIJ
Health System, entered the insurance market, it created a new company.
That company, \href{https://www.careconnect.com/}{CareConnect}, has
100,000 customers, most of them individuals insured through both large
and small employers.

Image

Mihir Patel is the pharmacy benefits manager for Oscar Health. While
Oscar has loyal customers, others say they are disappointed to find the
insurer behaving like everyone else.Credit...Richard Perry/The New York
Times

``If we only had the individual market, we would have taken undue risk
because we would not have understood that market,'' said Alan J. Murray,
CareConnect's chief executive. He said the company is close to turning a
profit.

Oscar says it plans to begin offering coverage to small businesses, but
Mr. Schlosser was adamant that individuals will eventually be buying
their own coverage, rather than relying on employers. The company is
also racing to incorporate plans with smaller networks.

\href{http://brighthealthplan.com/\#welcome-to-bright-health}{Bright
Health}, another start-up, also plans to work closely with health
systems to offer consumer-friendly plans.

While Oscar has had to use another insurer's network in New York, the
company's goal is to form partnerships with systems to create networks
that specialize in managing care. The company began experimenting with
these networks this year in Texas and California.

``Oscar talks about narrow networks like no one has seen one before,''
said Dr. Sanjay B. Saxena, who works with insurers and health systems at
the Boston Consulting Group.

Oscar has received \$750 million from its investors, and Mr. Schlosser
insists that the company understood how long it would take for the new
insurance marketplaces to develop, calling these ``very, very early
days.''

Oscar points to its technological edge as a way to manage patients'
health better than the established insurers. It has created teams,
including nurses, who are assigned to groups of patients and can
intervene when its data flags a potentially worrisome condition like a
high blood sugar level.

Promoting itself as a consumer-friendly alternative to the other
insurers also has its risks. While Oscar has loyal customers, others say
they are disappointed to find the insurer behaving like everyone else.
Cosmin Bita, a real estate broker in New York, switched to Oscar from an
insurer that had given him the runaround about whether it would pay for
blood tests as part of his annual physical. Although Oscar said when he
enrolled that the tests would be covered, he said, he found himself
fighting with the company over whether everything was covered.

``The exact same thing happened,'' Mr. Bita said.

Oscar executives said the company works hard to keep customers
satisfied.

But so far, it has not proved that it has created a better model than
the rest of the industry.

As Darren Walsh, a principal at Power \& Walsh Insurance Advisors, said:
``They haven't invented a new mousetrap.''

Advertisement

\protect\hyperlink{after-bottom}{Continue reading the main story}

\hypertarget{site-index}{%
\subsection{Site Index}\label{site-index}}

\hypertarget{site-information-navigation}{%
\subsection{Site Information
Navigation}\label{site-information-navigation}}

\begin{itemize}
\tightlist
\item
  \href{https://help.nytimes.com/hc/en-us/articles/115014792127-Copyright-notice}{©~2020~The
  New York Times Company}
\end{itemize}

\begin{itemize}
\tightlist
\item
  \href{https://www.nytco.com/}{NYTCo}
\item
  \href{https://help.nytimes.com/hc/en-us/articles/115015385887-Contact-Us}{Contact
  Us}
\item
  \href{https://www.nytco.com/careers/}{Work with us}
\item
  \href{https://nytmediakit.com/}{Advertise}
\item
  \href{http://www.tbrandstudio.com/}{T Brand Studio}
\item
  \href{https://www.nytimes.com/privacy/cookie-policy\#how-do-i-manage-trackers}{Your
  Ad Choices}
\item
  \href{https://www.nytimes.com/privacy}{Privacy}
\item
  \href{https://help.nytimes.com/hc/en-us/articles/115014893428-Terms-of-service}{Terms
  of Service}
\item
  \href{https://help.nytimes.com/hc/en-us/articles/115014893968-Terms-of-sale}{Terms
  of Sale}
\item
  \href{https://spiderbites.nytimes.com}{Site Map}
\item
  \href{https://help.nytimes.com/hc/en-us}{Help}
\item
  \href{https://www.nytimes.com/subscription?campaignId=37WXW}{Subscriptions}
\end{itemize}
