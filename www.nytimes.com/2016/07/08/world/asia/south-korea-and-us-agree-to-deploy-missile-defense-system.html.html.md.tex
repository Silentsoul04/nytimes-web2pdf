Sections

SEARCH

\protect\hyperlink{site-content}{Skip to
content}\protect\hyperlink{site-index}{Skip to site index}

\href{https://www.nytimes.com/section/world/asia}{Asia Pacific}

\href{https://myaccount.nytimes.com/auth/login?response_type=cookie\&client_id=vi}{}

\href{https://www.nytimes.com/section/todayspaper}{Today's Paper}

\href{/section/world/asia}{Asia Pacific}\textbar{}South Korea and U.S.
Agree to Deploy Missile Defense System

\url{https://nyti.ms/29EQctA}

\begin{itemize}
\item
\item
\item
\item
\item
\end{itemize}

Advertisement

\protect\hyperlink{after-top}{Continue reading the main story}

Supported by

\protect\hyperlink{after-sponsor}{Continue reading the main story}

\hypertarget{south-korea-and-us-agree-to-deploy-missile-defense-system}{%
\section{South Korea and U.S. Agree to Deploy Missile Defense
System}\label{south-korea-and-us-agree-to-deploy-missile-defense-system}}

\includegraphics{https://static01.nyt.com/images/2016/07/09/world/08KOREA-web1/08KOREA-web1-articleLarge.jpg?quality=75\&auto=webp\&disable=upscale}

By \href{http://www.nytimes.com/by/choe-sang-hun}{Choe Sang-Hun}

\begin{itemize}
\item
  July 7, 2016
\item
  \begin{itemize}
  \item
  \item
  \item
  \item
  \item
  \end{itemize}
\end{itemize}

SEOUL, South Korea --- South Korea and the United States announced on
Friday that they have decided to deploy an advanced American missile
defense system in the South, despite strong protests from China, which
sees it as a threat to its own security.

The two allies agreed to the deployment of the so-called Terminal
High-Altitude Area Defense system, or Thaad, to better protect South
Korea and the United States military in the region from North Korea's
growing nuclear and ballistic missile capabilities, a senior Defense
Ministry official, Ryu Jae-seung, said at a news conference.

Seoul and Washington have been in talks for months about implementing
the new system. Mr. Ryu said that officials from both nations were in
the final stage of recommending a site for a Thaad base to their defense
chiefs.

In a swift and sharp reaction against the deployment, China's Foreign
Ministry said in a statement that the decision would change the
strategic balance in the region and undermine China's security
interests.

``The Chinese side hereby expresses strong dissatisfaction and firm
opposition,'' the statement said. China's leader, Xi Jinping, had spent
considerable political capital trying to convince President Park
Geun-hye of South Korea to reject the push by the Obama administration
for the missile system.

The new system was also likely to face resistance from residents in
whatever part of South Korea is selected for the base. Villagers and
politicians from towns that have been mentioned as possible sites have
said they will oppose it, fearing that strong electronic signals from
the radar might be harmful to residents' health, and that their towns
would become an early target for North Korean missiles should war break
out.

``This is an important R.O.K-U.S. decision,'' Gen. Vincent K. Brooks,
the top commander of the American military in South Korea, said in a
statement, using the acronym for the South's formal name, the Republic
of Korea. ``North Korea's continued development of ballistic missiles
and weapons of mass destruction require the alliance to take this
prudent, protective measure to bolster our layered and effective missile
defense.''

The United States military emphasized that Thaad would ``be focused
solely on North Korea'' and would contribute to a layered system
enhancing the alliance's existing missile defense capabilities against
North Korean missile threats.

South Korea agreed to consider the Thaad deployment after the North's
launching of a long-range rocket on Feb. 7, an event widely seen as a
cover for developing a long-range ballistic nuclear missile. In March,
Washington and Seoul established a task force to discuss a possible
deployment.

South Korea's military has said that Thaad will bolster its defense
against North Korean missiles, but its political leaders had been
reluctant to commit to it because of China's objections.

They have expressed fear that the deployment might prompt China to move
closer to North Korea as a buffer against the United States and South
Korea, and that China might retaliate economically. China is South
Korea's No. 1 trade partner and sends more tourists than any other
country.

China is particularly concerned about Thaad in South Korea because its
powerful radar could give the United States military the ability to
quickly detect and track missiles launched in China, analysts said. The
United States military already has a Thaad battery deployed on Guam and
operates powerful radar in the region, as well as satellites over China.

While United States policy makers have increasingly worried about North
Korea's efforts to develop a capability to deliver a nuclear warhead on
an intercontinental ballistic missile, South Koreans have tended to be
less concerned. For decades, they have lived in the shadow of North
Korea's ability to deliver catastrophic destruction: Seoul, the capital,
is within the range of thousands of conventional North Korean rockets
and artillery pieces.

North Korea had no immediate reaction to the announcement.

Advertisement

\protect\hyperlink{after-bottom}{Continue reading the main story}

\hypertarget{site-index}{%
\subsection{Site Index}\label{site-index}}

\hypertarget{site-information-navigation}{%
\subsection{Site Information
Navigation}\label{site-information-navigation}}

\begin{itemize}
\tightlist
\item
  \href{https://help.nytimes.com/hc/en-us/articles/115014792127-Copyright-notice}{©~2020~The
  New York Times Company}
\end{itemize}

\begin{itemize}
\tightlist
\item
  \href{https://www.nytco.com/}{NYTCo}
\item
  \href{https://help.nytimes.com/hc/en-us/articles/115015385887-Contact-Us}{Contact
  Us}
\item
  \href{https://www.nytco.com/careers/}{Work with us}
\item
  \href{https://nytmediakit.com/}{Advertise}
\item
  \href{http://www.tbrandstudio.com/}{T Brand Studio}
\item
  \href{https://www.nytimes.com/privacy/cookie-policy\#how-do-i-manage-trackers}{Your
  Ad Choices}
\item
  \href{https://www.nytimes.com/privacy}{Privacy}
\item
  \href{https://help.nytimes.com/hc/en-us/articles/115014893428-Terms-of-service}{Terms
  of Service}
\item
  \href{https://help.nytimes.com/hc/en-us/articles/115014893968-Terms-of-sale}{Terms
  of Sale}
\item
  \href{https://spiderbites.nytimes.com}{Site Map}
\item
  \href{https://help.nytimes.com/hc/en-us}{Help}
\item
  \href{https://www.nytimes.com/subscription?campaignId=37WXW}{Subscriptions}
\end{itemize}
