Sections

SEARCH

\protect\hyperlink{site-content}{Skip to
content}\protect\hyperlink{site-index}{Skip to site index}

\href{https://www.nytimes.com/section/world/middleeast}{Middle East}

\href{https://myaccount.nytimes.com/auth/login?response_type=cookie\&client_id=vi}{}

\href{https://www.nytimes.com/section/todayspaper}{Today's Paper}

\href{/section/world/middleeast}{Middle East}\textbar{}A Saudi Morals
Enforcer Called for a More Liberal Islam. Then the Death Threats Began.

\url{https://nyti.ms/29NMIES}

\begin{itemize}
\item
\item
\item
\item
\item
\item
\end{itemize}

Advertisement

\protect\hyperlink{after-top}{Continue reading the main story}

Supported by

\protect\hyperlink{after-sponsor}{Continue reading the main story}

\hypertarget{a-saudi-morals-enforcer-called-for-a-more-liberal-islam-then-the-death-threats-began}{%
\section{A Saudi Morals Enforcer Called for a More Liberal Islam. Then
the Death Threats
Began.}\label{a-saudi-morals-enforcer-called-for-a-more-liberal-islam-then-the-death-threats-began}}

\includegraphics{https://static01.nyt.com/images/2016/07/11/world/11saudi1/11saudi1-articleLarge.jpg?quality=75\&auto=webp\&disable=upscale}

By \href{http://www.nytimes.com/by/ben-hubbard}{Ben Hubbard}

\begin{itemize}
\item
  July 10, 2016
\item
  \begin{itemize}
  \item
  \item
  \item
  \item
  \item
  \item
  \end{itemize}
\end{itemize}

JIDDA, Saudi Arabia --- For most of his adult life, Ahmed Qassim
al-Ghamdi worked among the bearded enforcers of Saudi Arabia. He was a
dedicated employee of the Commission for the Promotion of Virtue and the
Prevention of Vice --- known abroad as the religious police --- serving
with the front-line troops protecting the Islamic kingdom from
Westernization, secularism and anything but the most conservative
Islamic practices.

Some of that resembled ordinary police work: busting drug dealers and
bootleggers in a country that bans alcohol. But the men of ``the
Commission,'' as Saudis call it, spent most of their time maintaining
the puritanical public norms that set Saudi Arabia apart not only from
the West, but from most of the Muslim world.

A key offense was ikhtilat, or unauthorized mixing between men and
women. The kingdom's clerics warn that it could lead to fornication,
adultery, broken homes, children born of unmarried couples and
full-blown societal collapse.

For years, Mr. Ghamdi stuck with the program and was eventually put in
charge of the Commission for the region of Mecca, Islam's holiest city.
Then he had a reckoning and began to question the rules. So he turned to
the Quran and the stories of the Prophet Muhammad and his companions,
considered the exemplars of Islamic conduct. What he found was striking
and life altering: There had been plenty of mixing among the first
generation of Muslims, and no one had seemed to mind.

So he spoke out. In articles and television appearances, he argued that
much of what Saudis practiced as religion was in fact Arabian cultural
practices that had been mixed up with their faith.

There was no need to close shops for prayer, he said, nor to bar women
from driving, as Saudi Arabia does. At the time of the Prophet, women
rode around on camels, which he said was far more provocative than
veiled women piloting S.U.V.s.

He even said that while women should conceal their bodies, they needed
to cover their faces only if they chose to do so. And to demonstrate the
depth of his own conviction, Mr. Ghamdi went on television with his
wife, Jawahir, who smiled to the camera, her face bare and adorned with
a dusting of makeup.

It was like a bomb inside the kingdom's religious establishment,
threatening the social order that granted prominence to the sheikhs and
made them the arbiters of right and wrong in all aspects of life. He
threatened their control.

Mr. Ghamdi's colleagues at work refused to speak to him. Angry calls
poured into his cellphone and anonymous death threats hit him on
Twitter. Prominent sheikhs took to the airwaves to denounce him as an
ignorant upstart who should be punished, tried --- and even tortured.

\hypertarget{challenge-of-understanding}{%
\subsection{Challenge of
Understanding}\label{challenge-of-understanding}}

I had come to Saudi Arabia to explore Wahhabism, the hyper-conservative
Saudi strain of Sunni Islam that is often blamed for fueling intolerance
around the world --- and nurturing terrorism. I spent weeks in Riyadh,
Jidda and other cities speaking with sheikhs, imams, religious
professors and many others as I tried to peel back the layers of a
closed and private society.

For the Western visitor, Saudi Arabia is a baffling mix of modern
urbanism, desert culture and the never-ending effort to adhere to a
rigid interpretation of scriptures that are more than 1,000 years old.
It is a kingdom flooded with oil wealth, skyscrapers, S.U.V.s and
shopping malls, where questions about how to invest money, interact with
non-Muslims or even treat cats are answered with quotes from the Quran
or stories about the Prophet Muhammad.

Religion is woven into daily life. Banks employ clerics to ensure they
follow Shariah law. Mannequins lack heads because of religious
sensitivities to showing the human form. And schoolbooks detail how boys
should cut their hair, how girls should cover their bodies and how often
a person should trim his or her pubic hair.

While Islam is meant to be a complete program for human life,
interpretation is key when it comes to practices. The Saudi
interpretation is steeped in the conservatism of central Arabia,
especially regarding relations between women and men.

In public, most women wear baggy black gowns called abayas, designed to
hide their forms, as well as veils that cover their hair and faces, with
only thin slits for their eyes. Restaurants have separate sections for
``families,'' meaning groups that include women, and for ``singles,''
which means men.

Many Saudis mix in private, and men and women can usually meet in hotel
lobbies with little problem. Others do not want to mix and see gender
segregation as part of their cultural identity. In some conservative
circles, men go their whole lives without seeing the faces of women
other than their immediate family --- even their brothers' wives.

Inside the kingdom, all other religions are suppressed. Not only are
there no public churches, there is no Church's Chicken. (It is called
\href{http://www.texaschickenme.com/ksa/company/about_texas_chicken_ksa}{Texas
Chicken} in the kingdom.) When asked about this, Saudis deny that this
reflects intolerance. They compare their country to the Vatican, saying
it is a unique place for Muslims, with its own rules.

Officials I spoke with were upset by the kingdom's increasingly troubled
reputation abroad and said over and over that they supported ``moderate
Islam.''

But what exactly did they mean by ``moderate Islam''? Unpacking that
term made it clear how wide the values gap is between Saudi Arabia and
its American ally. The kingdom's ``moderate Islam'' publicly beheads
criminals, punishes apostates and prevents women from traveling abroad
without the permission of a male ``guardian.''

Don't even ask about gay rights.

Instead of calls for jihad, what I heard were religious leaders
insisting that the faithful obey the state. The Saudi royal family is
terrified that the jihadist fervor inflaming the region will catch fire
at home and threaten its control. So it has marshaled the state's
religious apparatus to condemn the jihadists and proclaim the religious
duty of obedience to the rulers.

And while it was once common, I heard little disparaging talk about
Christians and Jews, although it was open season on Shiites, whose faith
is frequently bashed as part of the rivalry with Iran.

The only Saudis who suggested I was an infidel were children.

Once, a Saudi journalist proudly introduced me to his 9-year-old
daughter, whom he had put in private school so she could study English.

``What is your name?'' I asked.

``My name is Dana,'' she said.

``How old are you?''

``I am 9.''

``When is your birthday?''

Confused, she switched to Arabic.

``We don't have that in Saudi Arabia,'' she said. ``That's an infidel
holiday.''

Shocked, her father asked where she had learned that, and she fetched
one of her government-issued textbooks, flipping to a lesson that listed
``forbidden holidays'': Christmas and Thanksgiving. Birthdays had been
part of the same lesson.

Another time, I met a religious friend for coffee, and he brought his
two young sons. When the call to prayer sounded, my friend went to pray.
His sons, confused that I did not follow, looked at me wide-eyed and
asked, ``Are you an infidel?''

\hypertarget{what-is-a-wahhabi}{%
\subsection{What Is a Wahhabi?}\label{what-is-a-wahhabi}}

The first thing many Saudis will tell you about Wahhabism is that it
does not exist.

``There is no such thing as Wahhabism,'' Hisham al-Sheikh told me the
first time we met. ``There is only true Islam.''

The irony is that fewer people have a purer Wahhabi pedigree than Mr.
Sheikh, a direct descendant of the cleric who started it all.

In the early 18th century, Sheikh Mohammed ibn Abdul-Wahhab called for a
religious reformation in central Arabia. Feeling that Islam had been
corrupted by practices like the veneration of saints and tombs, he
called for the stripping away of ``innovations'' and the return to what
he considered the pure religion.

He formed an alliance with a chieftain named Mohammed ibn Saud that has
underpinned the area's history ever since. Then the Saud family assumed
political leadership while Sheikh Abdul-Wahhab and his descendants gave
legitimacy to their rule and managed religious affairs.

That mix proved potent among the warring Arabian tribes, as Wahhabi
clerics provided justification for military conquest in some cases:
Those who resisted the House of Saud were not just enemies, but infidels
who deserved the sword.

The first Saudi state was destroyed by the Ottomans in 1818, and
attempts to build another failed until the early 20th century, when King
Abdulaziz al-Saud undertook a campaign that put him in control of most
of the Arabian Peninsula.

But the king faced a choice: to continue expansionary jihad, which would
have invited conflict with the British, or to build a modern state. He
chose the latter, even crushing a group of his own warriors who refused
to stop fighting.

Since then, the alliance between the royal family and the clerics has
endured, although the tensions between the quest for ideological purity
and the exigencies of modern statehood remain throughout Saudi society.

Fast forward to 2016, and the main players have transformed because of
time and oil wealth. The royal family has grown from a group of scrappy
desert dwellers into a sprawling clan awash in palaces and private jets.
The Wahhabi establishment has evolved from a puritan reform movement
into a bloated state bureaucracy.

\includegraphics{https://static01.nyt.com/images/2016/07/11/world/11saudi2/11saudi2-articleLarge.jpg?quality=75\&auto=webp\&disable=upscale}

It consists of universities that churn out graduates trained in
religious disciplines; a legal system in which judges apply Shariah law;
a council of top clerics who advise the king; a network of offices that
dispense fatwas, or religious opinions; a force of religious police who
monitor public behavior; and tens of thousands of mosque imams who can
be tapped to deliver the government's message from the pulpit.

The call to prayer sounds five times a day from mosques and inside of
malls so clearly that many Saudis use it to organize their days.

``Let's meet after the sunset prayer,'' they would tell me, sometimes
unsure what time that was. So I installed an app on my phone that let me
look up prayer times and buzzed when the call sounded.

And so it was, after the sunset prayer, that I met Mr. Sheikh, a proud
sixth-generation descendant of Mohammed ibn Abdul-Wahhab.

He was a portly man of 42 who wore a long white robe and covered his
head with a schmag, or checkered cloth. His beard was long and he had no
mustache, in imitation of the Prophet Muhammad, and he squinted through
reading glasses perched on his nose while peering at his iPhone.

We sat on purple couches in the music-free lobby of a Riyadh hotel and
shared dates and coffee while he answered my questions about Islam in
Saudi Arabia.

``I am an open-minded person,'' he told me early on.

It was clear that he hoped I would become a Muslim.

His life had been defined by the religious establishment, but he proved
to be a case study in the complexity of terms like ``modern'' and
``traditional'' in Saudi Arabia. He had memorized the Quran at a young
age and studied with prominent clerics before completing his doctorate
in Shariah, with his thesis on how technology changed the application of
Shariah.

Now he had a successful career and a host of religious jobs. He trained
judges for the Shariah courts, advised the minister of Islamic affairs,
wrote studies for the clerics who advise the king and served on the
Shariah board of the Medgulf insurance company. On Fridays, he preached
at a mosque near his mother's house and welcomed visitors who came to
see his uncle, the grand mufti.

He had traveled extensively abroad, and when he found out I was American
he told me that he loved the United States. He had visited Oregon, New
York, Massachusetts and Los Angeles. On one trip, he visited a
synagogue. On another, a black church. He had also visited an Amish
community, which he found fascinating.

A relative of his lived in Montgomery, Ala., and he had spent happy
months there, often visiting the local Islamic center. The hardest part,
he said, was Ramadan, because there were few eateries open late that did
not have bars.

``All I had was IHOP,'' he said.

He said Islam did not forbid doing business or having friendships with
Christians or Jews. He opposed Shiite beliefs and practices, but said it
was wrong to do as the extremists of the Islamic State and declare
takfir, or infidelity, on entire groups.

When it came to birthdays, which many Saudi clerics condemn, he said he
did not oppose them, although his wife did, so their children did not go
to birthday parties. But they had celebrations of their own, he said,
showing me a video on his phone of his family gathered around a cake
bearing the face of his son Abdullah, 15, who had just memorized the
Quran. They lit sparklers and cheered, but did not sing.

He was on the fence about music, which many Wahhabis also forbid. He
said he had no problem with background music in restaurants, but opposed
music that put listeners in a state similar to drunkenness, causing them
to jump around and bang their heads.

``We have something better,'' he said. ``You can listen to the Quran.''

Since much of what differentiates Saudi Arabia is the place of women, I
wanted to talk to a conservative Saudi woman, which was tricky because
most would refuse to meet with any unrelated male --- let alone a
non-Muslim correspondent from the United States. So I had a female Saudi
colleague, Sheikha al-Dosary, contact Mr. Sheikh's wife, Meshael, who
said she would meet me.

But I asked Mr. Sheikh's permission.

``She is very busy,'' he said, and changed the subject.

So Ms. Sheikh met Ms. Dosary at a women's coffee shop in Riyadh, where
women can uncover their faces and hair.

Her marriage to Mr. Sheikh had been arranged, she said. They met once
for less than an hour before they were married, and he had seen her
face.

Image

A Friday Prayer in April at the Fahed bin Saidaan mosque in
Riyadh.Credit...Sergey Ponomarev for The New York Times

``It was hard for me to look at him or to check him out as I was so
shy,'' she said.

They were cousins. He was 21; she was 16. He agreed to her condition for
marriage that she continue her studies, and she was now working on a
doctorate in education while raising their four children.

She disputed the Western idea that Saudi women lack rights.

``They believe we are oppressed because we don't drive, but that is
incorrect,'' Ms. Sheikh said, adding that driving would be a hassle in
Riyadh's snarled traffic.

``Here women are respected and honored in many ways you don't find in
the West,'' she continued.

She, too, is a descendant of Sheikh Abdul-Wahhab and said proudly that
her grandfather had founded the kingdom's religious police. ``Praise God
that we have the Commission to protect our country,'' she said.

\hypertarget{a-flurry-of-fatwas}{%
\subsection{A Flurry of Fatwas}\label{a-flurry-of-fatwas}}

The primacy of Islam in Saudi life has led to a huge religious sphere
that extends beyond the state's official clerics. Public life is filled
with celebrity sheikhs whose moves, comments and conflicts Saudis track
just as Americans follow Hollywood actors. There are old sheikhs and
young sheikhs, sheikhs who used to be extremists and now preach
tolerance, sheikhs whom women find sexy, and a
\href{http://www.nytimes.com/2009/04/11/world/middleeast/11saudi.html}{black
sheikh} who has compared himself to Barack Obama.

In the kingdom's
\href{http://www.nytimes.com/2015/05/23/world/middleeast/saudi-arabia-youths-cellphone-apps-freedom.html?_r=0}{hyper-wired
society}, they compete for followers on Twitter, Facebook and Snapchat.
The grand mufti, the state's highest religious official, has a regular
television show, too.

Their embrace of technology runs counter to the history of Wahhabi
clerics rejecting nearly everything new as a threat to the religion.
Formerly banned items include the telegraph, the radio, the camera,
soccer, girls' education and televisions, whose introduction in the
1960s caused outrage.

For Saudis, trying to navigate what is permitted, halal, and what is
not, haram, can be challenging. So they turn to clerics for fatwas, or
nonbinding religious rulings. While some may get a lot of attention ---
as when Ayatollah Ruhollah Khomeini of Iran called for killing the
author Salman Rushdie --- most concern the details of religious
practice. Others can reveal the sometimes comical contortions that
clerics go through to reconcile modernity with their understanding of
religion.

There was, for example, the cleric who appeared to call for the
\href{https://www.youtube.com/watch?v=j7IpMIhR6Yg}{death of Mickey
Mouse}, then tried to backtrack. Another prominent cleric
\href{http://english.alarabiya.net/en/variety/2014/03/18/Saudi-cleric-claims-he-didn-t-issue-a-fatwa-against-all-you-can-eat-buffets.html}{issued
a clarification} that he had not in fact forbidden all-you-can-eat
buffets. That same sheikh was recently asked about people taking photos
with cats. He
\href{https://www.youtube.com/watch?v=aCMLdH1O6YU}{responded} that the
feline presence was irrelevant; the photos were the problem.

``Photography is not permitted unless necessary,'' he said. ``Not with
cats, not with dogs, not with wolves, not with anything.''

The government has sought to control the flow of religious opinions with
official fatwa institutions. But state-sanctioned fatwas have provoked
laughter, too, like
\href{http://alifta.com/Fatawa/fatawacoeval.aspx?languagename=en\&View=Page\&HajjEntryID=0\&HajjEntryName=\&RamadanEntryID=0\&RamadanEntryName=\&NodeID=4655\&PageID=14440\&SectionID=7\&SubjectPageTitlesID=15174\&MarkIndex=3\&0\#Intensiveintellectualinvasion}{the
fatwa} calling spending money on Pokemon products ``cooperation in sin
and transgression.''

While the government seeks to get more women into the work force, the
state fatwa organization preaches on the
``\href{http://alifta.com/Fatawa/fatawaChapters.aspx?languagename=en\&View=Page\&PageID=75\&PageNo=1\&BookID=14\&TopFatawa=true}{danger
of women joining men in the workplace},'' which it calls ``the reason
behind the destruction of societies.''

And there are fatwas that arm extremists with religious justification.
There is one fatwa,
\href{http://alifta.com/Fatawa/fatawacoeval.aspx?languagename=en\&View=Page\&HajjEntryID=0\&HajjEntryName=\&RamadanEntryID=0\&RamadanEntryName=\&NodeID=4661\&PageID=6286\&SectionID=7\&SubjectPageTitlesID=6338\&MarkIndex=0\&0\#WhydoesIslamnotprohibitslavery}{still
available in English on a government website} and signed by the previous
grand mufti, that states, ``Whoever refuses to follow the straight path
deserves to be killed or enslaved in order to establish justice,
maintain security and peace and safeguard lives, honor and property.''

It goes on: ``Slavery in Islam is like a purifying machine or sauna in
which those who are captured enter to wash off their dirt and then they
come out clean, pure and safe, from another door.''

Once while we were having coffee, Mr. Sheikh answered his cellphone,
listened seriously and issued a fatwa on the spot. He got such calls
frequently.

The query had been about where a pilgrim headed to Mecca had to don the
white cloths of ritual purity --- an easy one. The answer, in this case,
was Jidda. Others were harder, and he demurred if he was not sure. Once,
a woman asked about fake eyelashes. He told her that he did not know,
but thought about it later and decided they were fine, on one condition:
``that there is no cheating involved.''

A woman, for example, could put them on before a man came to propose.

``And then after they get married, they're gone!'' he said. ``That is
not permitted.''

One Friday, Mr. Sheikh took me to see his uncle, Grand Mufti Abdulaziz
al-Sheikh.

We entered a vast reception hall near the mufti's house in Riyadh, with
padded benches along the walls where a dozen bearded students sat. In
the center, on a raised armchair, sat the mufti, his feet in brown socks
and perched on a pillow. The students read religious texts, and the
mufti interjected with commentary. He was 75, Mr. Sheikh said, and had
been blind since age 14, when a German doctor carried out a failed
operation on his eyes.

Mr. Sheikh said I could ask him a question, so I asked how he responded
to those who compared Wahhabism to the Islamic State.

``That is all lies and slander. Daesh is an aggressive, tyrannous group
that has no relation,'' he said, using another term for the Islamic
State.

After a pause, he asked, ``Why don't you become a Muslim?''

I responded that I was from a Christian family.

``The religion you follow has no source,'' he said, adding that I should
accept the Prophet Muhammad's revelation.

``Your religion is not a religion,'' he said. ``In the end, you will
have to face God.''

\hypertarget{the-unexpected-reformer}{%
\subsection{The Unexpected Reformer}\label{the-unexpected-reformer}}

The first time I met Mr. Ghamdi, 51, formerly of the religious police,
was this year in a sitting room in his apartment in Jidda, the port city
on the Red Sea. The room had been outfitted to look like a Bedouin tent.
Burgundy fabric adorned the walls, gold tassels hung from the ceiling,
and carpets covered the floor, to which Mr. Ghamdi pressed his forehead
in prayer during breaks in our conversation.

He spoke of how the world of sheikhs, fatwas and the meticulous
application of religion to everything had defined his life.

But that world --- his world --- had frozen him out.

Little in his background suggested that he would become a religious
reformer. While at a university, he quit a job at the customs office in
the Jidda port because a sheikh told him that collecting duties was
haram.

After graduation, he studied religion in his spare time and handled
international accounts for a government office --- a job requiring
travel to non-Muslim countries.

``The clerics at that time were releasing fatwas that it was not right
to travel to the countries of the infidels unless it was necessary,''
Mr. Ghamdi said.

So he quit.

Then he taught economics at a technical school in Saudi Arabia, but
didn't like that it taught only capitalism and socialism. So he said he
had added material on Islamic finance, but the students complained about
the extra work, and he left.

He finally landed a job that he felt was consistent with his religious
convictions, as a member of the Commission in Jidda.

Over the next few years, he transferred to Mecca and cycled through
different positions. There were occasional prostitution cases, and the
force sometimes caught sorcerers --- who can be beheaded if convicted in
court.

But he developed reservations about how the force worked. His
colleagues' religious zeal sometimes led them to overreact, breaking
into people's homes or humiliating detainees.

``Let's say someone drank alcohol,'' he said. ``That does not represent
an attack on the religion, but they exaggerated in how they treated
people.''

At one point, Mr. Ghamdi was assigned to review cases and tried to use
his position to report abuses and force agents to return items they had
wrongfully confiscated, he said.

He recalled the case of an older, single man who was reported to receive
two young women in his home on the weekends. Since the man did not pray
at the mosque, his neighbors suspected he was up to no good, so the
Commission raided the house and caught the man red-handed --- visiting
with his daughters.

``Often, people were humiliated in inhuman ways, and that humiliation
could cause hatred of religion,'' Mr. Ghamdi said.

Image

Saudi women at the Amex Luxury Expo in Riyadh.Credit...Sergey Ponomarev
for The New York Times

In 2005, the head of the Commission for the Mecca region died, and Mr.
Ghamdi was promoted. It was a big job, with some 90 stations throughout
a large, diverse area containing Islam's holiest sites. He did his best
to keep up, while worrying that the Commission's focus was misguided.

In private, he looked to the scriptures and the sayings of the Prophet
Muhammad for guidance on what was halal and what was haram, and he
documented his findings.

``I was surprised because we used to hear from the scholars, `Haram,
haram, haram,' but they never talked about the evidence,'' he said.

Realizing the gravity of such a conclusion for someone in his position,
he stayed silent and filed the document away.

But his conclusions would, soon, emerge.

Around the time he was rethinking his worldview, King Abdullah, then the
monarch, announced plans to open a world-class university, the
\href{https://www.kaust.edu.sa/en}{King Abdullah University of Science
and Technology}, or Kaust. What shocked the kingdom's religious
establishment was his decision to not segregate students by gender, nor
impose a dress code on women.

Kaust followed the precedent of Saudi Aramco, the state oil company,
which had also been shielded from clerical interference, highlighting
one of the great contradictions of Saudi Arabia: Regardless of how much
the royal family lauds its Islamic values, when it wants to earn money
or innovate, it does not turn to the clerics for advice. It puts up a
wall and locks them out.

Most clerics kept quiet out of deference to the king. But one member of
the top clerical body addressed the issue on a call-in show, warning of
the dangers of mixed universities: sexual harassment; men and women
flirting and getting distracted from their studies; husbands growing
jealous of their wives; rape.

``Mixing has many corrupting factors, and its evil is great,'' said the
cleric, Sheikh Saad al-Shathri, adding that if the king had known this
was the plan, he would have stopped it.

But mixing was in fact the king's idea, and he was not amused. He
dismissed the sheikh with a royal decree.

From his office in Mecca, Mr. Ghamdi watched, frustrated that the
clerics were not backing a project he felt was good for the kingdom.

So after praying about it, he retrieved his report and boiled it down to
two long articles that were published in the newspaper Okaz in 2009.

They were the first strikes in a yearslong battle between Mr. Ghamdi and
the religious establishment. He followed with other articles, went on TV
and faced off against other clerics who insulted him and marshaled their
own evidence from the scriptures. His colleagues at the Commission
shunned him, so he requested --- and was swiftly granted --- early
retirement.

Once off the force, he questioned other practices: forcing shops to
close during prayer times and urging people to go to the mosque,
requiring face veils, barring women from driving.

Each comment lit a new inferno. A woman once asked him on Twitter if she
could not only show her face, but also wear makeup. Sure, Mr. Ghamdi
said, setting off new attacks.

Then in 2014, he was to appear on a popular talk show, and the producers
\href{http://www.mbc.net/ar/programs/badriya/articles/-\%D8\%A8\%D8\%AF\%D8\%B1\%D9\%8A\%D8\%A9--\%D8\%AA\%D8\%B3\%D8\%AA\%D8\%B6\%D9\%8A\%D9\%81--\%D8\%A7\%D9\%84\%D8\%BA\%D8\%A7\%D9\%85\%D8\%AF\%D9\%8A--\%D9\%85\%D8\%B9-\%D8\%B2\%D9\%88\%D8\%AC\%D8\%AA\%D9\%87-\%D9\%81\%D9\%8A-\%D8\%A3\%D9\%88\%D9\%84-\%D8\%B8\%D9\%87\%D9\%88\%D8\%B1-\%D9\%84\%D9\%87\%D8\%A7.html}{filmed
a segment} about him and his wife, who appeared with her face showing
and said she supported him.

Harsh responses came from the top of the religious establishment.

Many attacked his religious credentials, saying he was not really a
sheikh --- a dubious accusation since there is no standard qualification
to be one. They targeted his résumé, too, saying he had no degree in
religion and pointing out, correctly, that his doctorate was from
\href{http://www.ambassador-university.com/index.php?option=com_content\&task=view\&id=21\&Itemid=43}{Ambassador
University Corporation}, a diploma mill that gives degrees based on work
experience ``in the Middle East.''

``There is no doubt that this man is bad,''
\href{https://www.youtube.com/watch?v=mY35rIEFCO8\&app=desktop}{said}
Sheikh Saleh al-Luheidan, a member of the top clerical body. ``It is
necessary for the state to assign someone to summon and torture him.''

Image

A Saudi street market vendor conducting an auction of antiques in
downtown Riyadh.Credit...Sergey Ponomarev for The New York Times

The grand mufti addressed the issue
\href{https://www.youtube.com/watch?v=BxXin1ITQvM}{on his call-in show},
saying that the veil was ``a necessary order and an Islamic creation''
and calling on the kingdom's television channels to ban content that
``corrupts the religion and the morals and values of society.''

If the clerical attacks on Mr. Ghamdi were loud, the blowback from
society was more painful. His tribe issued a statement, disowning him
and calling him ``troubled and confused.'' His cellphone rang day and
night with callers shouting at him. He came home to find graffiti on the
wall of his house. And a group of men showed up at his door, demanding
to ``mix'' with the family's women. His sons --- he has nine children
--- called the police.

Before the dust-up, Mr. Ghamdi had also delivered Friday sermons at a
mosque in Mecca, earning a government stipend. But the congregation
complained after he spoke out, and he was asked to stay home, later
losing his pay.

Mr. Ghamdi had not broken any laws and never faced legal action. But in
Saudi Arabia's close-knit society, the attacks echoed through his
family. The relatives of his eldest son's fiancée called off their
wedding, not wanting to associate their family with his.

``Are you with your brother or with me?'' Mr. Ghamdi said his sister's
husband had asked her. ``She said, `I am with my brother.''' They soon
divorced.

Mr. Ghamdi's son Ammar, 15, was taunted at school. Ammar said another
boy had once asked him: ``How did your mom go on TV? That's not right.
You have no manners.''

So Ammar punched him.

\hypertarget{not-a-place-to-speak-up}{%
\subsection{Not a Place to Speak Up}\label{not-a-place-to-speak-up}}

One evening in Jidda, a university professor invited me to his home for
dinner. His wife, a doctor, joined us at the table, her hair covered
with a stylish veil.

They had recently been married and he joked that they were meant for
each other because she was good at cooking and he was good at eating.
His wife chuckled and gave him more soup.

I asked about Mr. Ghamdi.

``From what I read and what I saw, I think he's right and he stood up
for what he believes in,'' the professor said. ``I admire that.''

The problem, he said, is that tolerance for opposing views is not taught
in Saudi society.

``Either follow what I say or I will classify you, I will hurt you, I
will push you out of the discussion,'' he said. ``This is anti-Islam. We
have many people thinking in different ways. You can fight, but you have
to live under the same roof.''

His wife had no problem with mixing or with women working, but did not
like that Mr. Ghamdi had caused a scandal by making his views public.
The royal family sets the rules, and it was inappropriate for subjects
to publicly campaign for changes, she said.

``He has to follow the ruler,'' she said. ``If everyone just comes out
with his own opinion, we'll be in chaos.''

After dinner, a young cleric who works for the security services dropped
by. He, too, agreed with Mr. Ghamdi, but would not talk about it openly.
The response, he said, is part of the deep conservatism in the clerical
establishment that is impeding development.

He often gave lectures to security officers, followed by discussions, he
said, and a common question he heard was, ``Isn't the military uniform
haram?'' Many Wahhabi clerics preach against resembling the infidels,
leading to confusion.

He believed that wearing uniforms was fine, and worried that such narrow
thinking made people susceptible to extremism.

``It's like in those American movies when they invent a robot and then
they lose control and it attacks them and the remote control stops
working,'' he said.

The next day, the professor thanked me for my visit in a text message.

``I'd like to remind u that any story that would uncover the source may
hurt us. I trust your discretion,'' he wrote, followed by three flowers.

Image

A fountain in Riyadh.Credit...Sergey Ponomarev for The New York Times

All that was left, really, was to to speak with the Commission. What did
its leaders and rank and file think about all of this? But for a force
portrayed as ever-present and all powerful, it proved surprisingly shy.

I could not visit Mr. Ghamdi's former office because non-Muslims are
barred from entering Mecca. So I had multiple contacts ask for
interviews with relatives who worked for the Commission, but they all
declined to speak. I called the Commission's spokesman, who told me that
he was traveling and then stopped answering my calls.

I even dropped by the Commission's headquarters, a boxy, steel-and-glass
building on a Riyadh highway between a gas station and a car dealership.
Its website advertised open hours with the director, so I went to his
office, through halls filled with bearded men milling about and slick
banners proclaiming ``A Policy of Excellence'' and ``Together Against
Corruption.''

``He didn't come today,'' the director's secretary told me. ``Maybe next
week.''

On my way out, two men invited me into an office and served me coffee.

``How do you like working for the Commission?'' I asked.

``Everyone who chooses this job loves it,'' one said. It was the work of
``the entire Islamic nation,'' and it felt good ``to bring people from
the darkness into the light.''

The other man had been on the force for 15 years and said he preferred
working in the office.

``You rest more in the administration,'' he said. ``Out there we have
problems with people. They call us the religious police. Criminals!
Thieves! You never get to rest out in the field.''

A scowling man appeared in the doorway and told me that I was not
allowed to talk to anyone. The first man soon left. The second offered
me more coffee, then tea, then forced me to take a bottle of water when
I left.

\hypertarget{reform-the-hard-way}{%
\subsection{Reform, the Hard Way}\label{reform-the-hard-way}}

The first irony of Mr. Ghamdi's situation is that many Saudis, including
members of the royal family and even important clerics, agree with him,
although mostly in private. And public mixing of the sexes in some
places --- hospitals, conferences and in Mecca during the pilgrimage ---
is common. In some Saudi cities it is not uncommon to see women's faces,
or even their hair.

But there is a split in society between the conservatives who want to
maintain what they consider the kingdom's pure Islamic identity and the
liberals (in the Saudi context) who want more personal freedoms.
Liberals make cases like Mr. Ghamdi's all the time. But sheikhs don't,
which is why he was branded a traitor.

The second irony is that this year, Saudi Arabia instituted some of the
reform Mr. Ghamdi had called for.

It had been a rough year for the Commission. A
\href{https://www.youtube.com/watch?v=A6DlHxe7D-I\&oref=https\%3A\%2F\%2Fwww.youtube.com\%2Fwatch\%3Fv\%3DA6DlHxe7D-I\&has_verified=1}{video}
went viral of a girl yelping as she was thrown to the ground outside a
Riyadh mall during a confrontation with the Commission, her abaya flying
over her head and exposing her legs and torso. For many Saudis, ``the
Nakheel Mall girl'' symbolized the Commission's overreach.

Then the Commission arrested Ali al-Oleyani, a popular talk show host
who often criticized religious figures.
\href{http://hawl-alkhaleej.sa/8180.html}{Photos} appeared online of Mr.
Oleyani in handcuffs with bottles of liquor. The photos were clearly
staged and apparently had been leaked as a form of character
assassination. Many people were outraged.

In April, the government responded with a surprise decree defanging the
religious police. It denied them the power to arrest, question or pursue
subjects, forced them to work with the police and advised them to be
``gentle and kind'' in their interactions with citizens.

Mr. Ghamdi applauded the decision, although he remains an outcast, a
sheikh whose positions rendered him unemployable in the Islamic kingdom.

These days, he keeps a low profile because he still gets insults when he
appears in public. He has no job, but publishes regular newspaper
columns, mostly abroad.

Near the end of our last conversation, his wife, Jawahir, entered the
room, dressed in a black abaya, with her face showing. She shook my
hand, exuding a cloud of fragrance, and sat next to her husband.

The experience had changed her life in unexpected ways, she said. And
like her husband, she had no regrets.

``We sent our message, and the goal was not for us to keep appearing and
to get famous,'' she said. ``It was to send a message to society that
religion is not customs and traditions. Religion is something else.''

Advertisement

\protect\hyperlink{after-bottom}{Continue reading the main story}

\hypertarget{site-index}{%
\subsection{Site Index}\label{site-index}}

\hypertarget{site-information-navigation}{%
\subsection{Site Information
Navigation}\label{site-information-navigation}}

\begin{itemize}
\tightlist
\item
  \href{https://help.nytimes.com/hc/en-us/articles/115014792127-Copyright-notice}{©~2020~The
  New York Times Company}
\end{itemize}

\begin{itemize}
\tightlist
\item
  \href{https://www.nytco.com/}{NYTCo}
\item
  \href{https://help.nytimes.com/hc/en-us/articles/115015385887-Contact-Us}{Contact
  Us}
\item
  \href{https://www.nytco.com/careers/}{Work with us}
\item
  \href{https://nytmediakit.com/}{Advertise}
\item
  \href{http://www.tbrandstudio.com/}{T Brand Studio}
\item
  \href{https://www.nytimes.com/privacy/cookie-policy\#how-do-i-manage-trackers}{Your
  Ad Choices}
\item
  \href{https://www.nytimes.com/privacy}{Privacy}
\item
  \href{https://help.nytimes.com/hc/en-us/articles/115014893428-Terms-of-service}{Terms
  of Service}
\item
  \href{https://help.nytimes.com/hc/en-us/articles/115014893968-Terms-of-sale}{Terms
  of Sale}
\item
  \href{https://spiderbites.nytimes.com}{Site Map}
\item
  \href{https://help.nytimes.com/hc/en-us}{Help}
\item
  \href{https://www.nytimes.com/subscription?campaignId=37WXW}{Subscriptions}
\end{itemize}
