Sections

SEARCH

\protect\hyperlink{site-content}{Skip to
content}\protect\hyperlink{site-index}{Skip to site index}

\href{https://www.nytimes.com/section/politics}{Politics}

\href{https://myaccount.nytimes.com/auth/login?response_type=cookie\&client_id=vi}{}

\href{https://www.nytimes.com/section/todayspaper}{Today's Paper}

\href{/section/politics}{Politics}\textbar{}Melania Trump: From
Small-Town Slovenia to Doorstep of White House

\url{https://nyti.ms/2a42JUZ}

\begin{itemize}
\item
\item
\item
\item
\item
\end{itemize}

Advertisement

\protect\hyperlink{after-top}{Continue reading the main story}

Supported by

\protect\hyperlink{after-sponsor}{Continue reading the main story}

\hypertarget{melania-trump-from-small-town-slovenia-to-doorstep-of-white-house}{%
\section{Melania Trump: From Small-Town Slovenia to Doorstep of White
House}\label{melania-trump-from-small-town-slovenia-to-doorstep-of-white-house}}

\href{https://www.nytimes.com/slideshow/2016/07/09/us/the-slovenian-roots-of-melania-trump.html}{}

\hypertarget{the-slovenian-roots-of-melania-trump}{%
\subsection{The Slovenian Roots of Melania
Trump}\label{the-slovenian-roots-of-melania-trump}}

8 Photos

View Slide Show ›

\includegraphics{https://static01.nyt.com/images/2016/07/09/us/00slovenia-slide-XGC9/00slovenia-slide-XGC9-articleLarge.jpg?quality=75\&auto=webp\&disable=upscale}

Sergey Ponomarev for The New York Times

By \href{https://www.nytimes.com/by/jason-horowitz}{Jason Horowitz}

\begin{itemize}
\item
  July 18, 2016
\item
  \begin{itemize}
  \item
  \item
  \item
  \item
  \item
  \end{itemize}
\end{itemize}

\href{https://www.nytimes.com/es/2016/07/19/melania-trump-de-un-pueblo-en-eslovenia-a-las-puertas-de-la-casa-blanca/}{Leer
en español}

SEVNICA, Slovenia --- On days when Melanija Knavs could not play outside
or grew tired of knitting her navy blue sweaters, she and her friends
would exchange notes along the lines of yarn they strung between their
apartment block balconies.

In clear handwriting, Melanija mused about the boys of her dreams.

She could not have seen what was coming. Melanija Knavs is now
\href{http://www.nytimes.com/2016/07/20/us/politics/melania-trump-speech.html}{Melania
Trump}, and she is one election away from being the first foreign-born
first lady since
\href{https://www.whitehouse.gov/1600/first-ladies/louisaadams}{Louisa
Adams}. She addressed millions of Americans on Monday night in a
televised speech at the Republican National Convention in Cleveland.

But interviews with her former classmates, friends of her family and
others who knew them during her youth in Slovenia suggest that her
transformation owes less to chance than to the Knavs family's
determination to seize openings and avoid getting stuck.

Her father, a larger-than-life personality who reminds her childhood
friends of Mr. Trump, belonged to the Communist Party, an exclusive club
whose members sometimes joined because of career ambitions as much as
ideology. Her mother, an industrious and striking woman, went from
harvesting red onions on her family's farm to a career in the town's
textile factory. She always found time to make sure her two daughters
dressed to impress, sewing clothes for them after her work shift ended.

\includegraphics{https://static01.nyt.com/images/2016/07/08/us/00sloveniaweb1/00sloveniaweb1-articleLarge.jpg?quality=75\&auto=webp\&disable=upscale}

Ms. Trump herself trained her bright eyes on the next thing. Once she
left Sevnica for high school in Ljubljana, now Slovenia's capital, she
rarely came back to see her old friends. Once she left Ljubljana for a
modeling career in Milan and then elsewhere in Europe, Slovenia receded
from view. And once she moved to New York, where she caught the eye of
Donald J. Trump, 24 years her senior, during a Fashion Week party at the
Kit Kat Club, she never looked back.

``She tried to find opportunities,'' said Damijan Kracina, 46, a high
school classmate. ``And took them.''

Ms. Trump, born in 1970, grew up in this hilly town of 4,500 best known
around Slovenia, at least until Mr. Trump entered the presidential race,
for its medieval castle and annual salami festival. Then, Slovenia was
the northern region of Yugoslavia, ruled by
\href{http://nyti.ms/29IAGb3}{Josip Broz Tito}, a Communist dictator who
kept his distance from the Soviet Union and allowed more freedoms than
did other Eastern bloc leaders.

But under Tito, there were clear benefits to being a member of the
Communist Party, to which only a tiny percentage of Slovenians belonged.
Some inherited membership through parents, particularly if they had
resisted the Nazis, as Tito had; others by exhibiting unusual talent.

While it is not clear how Ms. Trump's father, Viktor, joined ---
available records in Ljubljana simply list him as a member --- others
from the Sevnica Communist Party mentioned his work as a driver for a
neighboring mayor and then for the director of the government-owned
textile factory, Jutranjka, across the river, as possible entry points.

While the Knavses, along with Ms. Trump, declined to be interviewed
about their years in Slovenia, a spokeswoman for the Trump campaign,
Hope Hicks, said that Mr. Knavs had never been an ``active member'' of
the party.

\href{https://www.nytimes.com/interactive/2016/07/18/us/elections/gop-conventions-speakers.html}{}

\includegraphics{https://static01.nyt.com/images/2016/07/19/us/19livepromo/19livepromo-videoLarge.jpg}

\hypertarget{republican-convention-day-1-analysis}{%
\subsection{Republican Convention Day 1:
Analysis}\label{republican-convention-day-1-analysis}}

Times journalists provided live analysis of the first night of the
Republican National Convention as Donald J. Trump aims to unify the
party.

Mr. Trump, in an interview last month, said he had never discussed the
topic with his father-in-law. ``But he was pretty successful over
there,'' he said. ``It's a different kind of success than you have here.
But he was successful.''

After being introduced by her husband on Monday night as ``the next
first lady of the United States,'' Ms. Trump spoke about her love of the
United States and her home country.

``I was born in Slovenia, a small and then-Communist country,'' she
said, adding that her parents instilled in her a love of fashion, beauty
and business, and --- in a part of her speech that appeared to be copied
\href{https://twitter.com/JarrettHill/status/755242423991709697}{nearly
word-for-word} from a 2008 speech by Michelle Obama --- the value of
working hard ``for what you want in life.''

In 1972, the Knavses moved into a larger apartment in a new housing
block for workers of the government-owned textile factory, including
Melania's mother, Amalija, nicknamed Malci. She drew patterns for
children's clothes and later designed them, crossing the bridge to the
factory every day in heels.

Mr. Knavs, a traveling car salesman, spent a lot of time on the road.
But when he was home, he was noticed. Friends say he had a jocular
personality and a fondness for his Mercedes sedans and his coveted
Maserati. Ms. Trump's childhood friends recalled him incessantly washing
the cars, but also carrying himself in a self-assured way that now
reminded them of Mr. Trump.

``Donald and Melania are similar to Viktor and Amalija,'' said Nena
Bedek, who was close to Ms. Trump in childhood, and who added that she
was ``not surprised'' that her friend had married someone similar to her
father. ``Melania was closer to her mother than her father. Viktor was
often away, and Malci and the girls were often alone.''

Image

Viktor and Amalija Knavs, Ms. Trump's parents, in Aberdeen, Scotland, in
2011. The Knavs went there to spend time with Ms. Trump, her son and her
husband.Credit...Pacific Coast News

Social life centered on the school down the block. Melania wrapped her
notebooks in magazine perfume ads and kept her knitted sweaters in
purple lockers. Friends say that she enjoyed geography lessons in a room
adorned with maps of the world, and that she adored art class. The
future creator of the QVC collection
``\href{https://www.youtube.com/watch?v=ID0KKalefqw}{Melania Timepieces
\& Jewelry}'' made bracelets there. When
\href{http://nyti.ms/29OUbBu}{Tito died in 1980}, her weeping classmates
threw flowers as a train carrying his body rolled past on the way to
Yugoslavia's capital, Belgrade.

In 1985, Melania left Sevnica, traveling on the narrow roads along the
slow-moving Sava River, green from the reflection of the wooded hills,
and through coal mining towns on the way to Ljubljana. There she
attended the Secondary School of Design and Photography, housed in an
arcaded Renaissance monastery.

She lived in an apartment that her father, who had opened a bicycle and
car parts shop in Ljubljana, had bought a few years earlier on the
outskirts of the city. The building superintendent, Joze Vuk, lived on
their floor, and he recalled that Mr. Knavs was displeased that after he
had paid for his unit, the government decided to set aside some of the
apartments as rentals for construction workers.

``We were all angry because most of the residents were not prepared to
invest in the block,'' said Mr. Vuk, who also owned an apartment. ``They
were renters of a public property and did not care.''

Mr. Knavs sought to distinguish himself from his neighbors. ``He always
wore a tie, smart clothes and carried a briefcase,'' Mr. Vuk said. ``You
could not avoid noticing him.''

Melania and her older sister, Ines, also stood out, for their looks,
their wardrobe and the makeup they put on whenever they left the
apartment. At school, Melania kept her distance from peers listening to
the Cure or Metallica, Mr. Kracina said, and gravitated toward a clique
of pop music fans who hung out at the Horse's Tail bar by the Triple
Bridge in Ljubljana.

It was there that Peter Butoln, who prided himself on having Ljubljana's
only metallic blue Vespa, noticed Melania one night among the regulars
dressed in bleached jeans and Benetton shirts, drinking Mish Mash (Fanta
and wine) and chatting each other up. Now 17, Melania was abstemious and
more wholesome than the other girls, he said, and they started dating.
He would pick her up on weekends and drive her around on his Vespa, and
they would dance badly to Wham in ``a nice discothèque'' by the
cathedral.

Image

Peter Butoln in Ljubljana, Slovenia, this spring. Mr. Butoln dated Ms.
Trump when she was 17.Credit...Sergey Ponomarev for The New York Times

Mr. Butoln soon went into the army, and, after sending him a friendly
postcard in her exact, all-capitals handwriting, Melania started dating
one of his friends. ``He had a red Vespa,'' Mr. Butoln said, shrugging.

Melania had also begun a process that would carry her away from
Slovenia. In January 1987, the photographer Stane Jerko spotted her and
asked if she would be interested in modeling.

She proved somewhat wooden, but ``pridna --- diligent, obedient,'' Mr.
Jerko said. She told him she wanted to get better. Mr. Jerko passed the
photographs he snapped of Melania --- hair up, hair down, gym clothes,
flowing dress --- to a Slovenian cultural center, which admitted her to
a fashion course for models in the fall of 1987.

Melania's entire family sensed potential in her modeling. After high
school, she concentrated on her career, dropping out of architecture
school. (She still claims
\href{http://www.melaniatrump.com/my-world/}{on her website} to have
graduated.) On one occasion, Mr. Kravs drove his Mercedes to the shop of
the seamstress Silva Njegac, hours from Ljubljana, to order leather
dresses for Melania that his wife had designed.

In 1992, a year after Slovenia's independence, Mr. Jerko saw Melania on
the catwalk at the Grand Hotel Toplice on Lake Bled. Twenty years later,
she and Mr. Trump dined there with her parents. That day trip amounted
to Mr. Trump's only visit to Slovenia.

``At least I can say that I went,'' Mr. Trump said. When asked if his
wife, who he said spoke warmly about her Slovenian youth, hoped for him
to see her hometown, he added: ``I went to Slovenia. The fact that I
even went there was very much appreciated.''

A second-place finish in Jana magazine's Slovenian Face of the Year
contest in 1992 expanded Melania's ambitions. In a fashion video for a
Slovenian label, she wore a skirt suit, exited a plane shadowed by
bodyguards and signed papers at the national library. ``She was acting
like the president of the United States,'' said Andrej Kosak, the
director.

Image

The director Andrej Kosak, far left, with Ms. Trump on the set of a
fashion video in 1992.

She would soon Germanize her name to Melania Knauss and become an
international model.

These days in Sevnica, where Ms. Trump made a \$25,000 contribution to a
hospital after her 2005 wedding, residents are fascinated by tales of
their local girl made great.

The Slovenian news media brings the latest word of the Trump campaign,
especially details from a recent article in GQ that revealed that Mr.
Knavs had, before his marriage to Melania's mother, fathered a child out
of wedlock and then fought attempts to claim child support all the way
to the country's highest court, where he lost. The GQ reporter then
began receiving anti-Semitic messages.

``Because of story about half brother Denis, journalist is targeted by
anonymous Trump supporters,'' read a headline, accompanied by a
photograph of a Melania in a plunging V-neck dress, on the cover of the
tabloid Svet 24.

\includegraphics{https://static01.nyt.com/images/2016/07/18/multimedia/rnc-little-slovenia/rnc-little-slovenia-videoSixteenByNineJumbo1600.jpg}

Ms. Trump's parents spend much of the year with their daughter and her
10-year-old son, Barron, at Trump Tower in Manhattan or at Mr. Trump's
\href{http://www.nytimes.com/2016/02/26/us/politics/donald-trump-taps-foreign-work-force-for-his-florida-club.html}{Mar-a-Lago
Club} in Palm Beach, Fla., where they enjoy the pool. But they have also
brought a whiff of the campaign back to Sevnica, where they now own a
handsome house. Alongside the sloping lawn and the beige Mercedes, one
finds security guards to turn away unwanted visitors.

In Sevnica, Mr. Knavs has confided in Matej Novsak, his longtime
mechanic, and complained recently about Mr. Trump's whiplash-inducing
inconsistency.

`` `One time it is this, the other time that,' '' Mr. Novsak said Mr.
Knavs had told him. The mechanic said that Mr. Knavs had also said that
Mr. Trump was unwanted by Republicans and that he did not understand his
wealthy son-in-law's need to pursue the presidency. `` `Why does he have
to do it?' '' the mechanic said Mr. Knavs had told him.

When told of his in-laws' bewilderment, Mr. Trump said, ``They are not
the only ones.''

Mr. Knavs is close enough to his son-in-law, five years his junior, to
accept his hand-me-downs. A few years ago, Mr. Knavs took two of Mr.
Trump's leather jackets --- one black, one dark brown --- to Ms.
Njegac's shop in Slovenia for alterations. The sleeves were too long.

Meanwhile, his daughter, who is now an American citizen, has fit well
into life with Mr. Trump. She has echoed his doubts about President
Obama's place of birth, given his campaign a touch of glamour and
domesticity, and fully embraced his extravagant lifestyle.

Mirjana Jelancic, a classmate of Ms. Trump's who is now the principal of
their old school, recalled a conversation she had over coffee last
August with Ms. Trump's mother. Ms. Knavs told Ms. Jelancic that she had
asked her daughter what to do with all the sweaters she had knitted as a
child. `` `Throw them away,' '' Ms. Trump told her mother, who said she
replied, testily, ``Come home, pick some out and throw them away
yourself.''

Ms. Jelancic suggested a compromise. Ms. Knavs now intends to donate
those old clothes to a planned exhibit at the school dedicated to
Melania Trump, the town's most famous brand name.

Advertisement

\protect\hyperlink{after-bottom}{Continue reading the main story}

\hypertarget{site-index}{%
\subsection{Site Index}\label{site-index}}

\hypertarget{site-information-navigation}{%
\subsection{Site Information
Navigation}\label{site-information-navigation}}

\begin{itemize}
\tightlist
\item
  \href{https://help.nytimes.com/hc/en-us/articles/115014792127-Copyright-notice}{©~2020~The
  New York Times Company}
\end{itemize}

\begin{itemize}
\tightlist
\item
  \href{https://www.nytco.com/}{NYTCo}
\item
  \href{https://help.nytimes.com/hc/en-us/articles/115015385887-Contact-Us}{Contact
  Us}
\item
  \href{https://www.nytco.com/careers/}{Work with us}
\item
  \href{https://nytmediakit.com/}{Advertise}
\item
  \href{http://www.tbrandstudio.com/}{T Brand Studio}
\item
  \href{https://www.nytimes.com/privacy/cookie-policy\#how-do-i-manage-trackers}{Your
  Ad Choices}
\item
  \href{https://www.nytimes.com/privacy}{Privacy}
\item
  \href{https://help.nytimes.com/hc/en-us/articles/115014893428-Terms-of-service}{Terms
  of Service}
\item
  \href{https://help.nytimes.com/hc/en-us/articles/115014893968-Terms-of-sale}{Terms
  of Sale}
\item
  \href{https://spiderbites.nytimes.com}{Site Map}
\item
  \href{https://help.nytimes.com/hc/en-us}{Help}
\item
  \href{https://www.nytimes.com/subscription?campaignId=37WXW}{Subscriptions}
\end{itemize}
