Sections

SEARCH

\protect\hyperlink{site-content}{Skip to
content}\protect\hyperlink{site-index}{Skip to site index}

\href{https://www.nytimes.com/section/world/europe}{Europe}

\href{https://myaccount.nytimes.com/auth/login?response_type=cookie\&client_id=vi}{}

\href{https://www.nytimes.com/section/todayspaper}{Today's Paper}

\href{/section/world/europe}{Europe}\textbar{}Failed Turkish Coup
Accelerated a Purge Years in the Making

\url{https://nyti.ms/2a3v8IP}

\begin{itemize}
\item
\item
\item
\item
\item
\item
\end{itemize}

Advertisement

\protect\hyperlink{after-top}{Continue reading the main story}

Supported by

\protect\hyperlink{after-sponsor}{Continue reading the main story}

\hypertarget{failed-turkish-coup-accelerated-a-purge-years-in-the-making}{%
\section{Failed Turkish Coup Accelerated a Purge Years in the
Making}\label{failed-turkish-coup-accelerated-a-purge-years-in-the-making}}

\includegraphics{https://static01.nyt.com/images/2016/07/23/world/23TURKEY-PLOT-web1/23TURKEY-PLOT-web1-articleLarge.jpg?quality=75\&auto=webp\&disable=upscale}

By \href{http://www.nytimes.com/by/ben-hubbard}{Ben Hubbard},
\href{http://www.nytimes.com/by/tim-arango}{Tim Arango} and
\href{http://www.nytimes.com/by/ceylan-yeginsu}{Ceylan Yeginsu}

\begin{itemize}
\item
  July 22, 2016
\item
  \begin{itemize}
  \item
  \item
  \item
  \item
  \item
  \item
  \end{itemize}
\end{itemize}

ANKARA, Turkey --- Night had fallen and the weekend had begun, but the
head of Turkey's spy agency remained at work in the main security
compound in Ankara, struggling to track reports of strange military
activities across the country.

Suddenly, a roar of gunfire erupted as a fleet of choppers blasted the
gates of the compound. As guards fired in the air, a helicopter tried to
land beside the agency while others dropped ropes to send down
commandos, according to a security official who was inside at the time,
and spoke on condition of anonymity to discuss intelligence matters.

More than a sudden attack on the government, the attempted coup this
month has emerged as a turning point in a yearslong struggle for control
of the Turkish state. The battle lines were clear: allies of President
Recep Tayyip Erdogan against a collection of adversaries, including
members of the military and followers of
\href{https://www.nytimes.com/2016/07/20/world/europe/fethullah-gulen-erdogan-extradition.html}{Fethullah
Gulen}, a Muslim cleric who leads a secretive religious movement from
his self-exile in Pennsylvania.

Agents inside the intelligence service had long feared a fifth column
was taking shape inside the Turkish state, and they spent years
compiling dossiers on tens of thousands of citizens, scrutinizing their
pasts for any hints of rebellion or links to Mr. Gulen.

As dramatic and violent as the night was, the aftermath has been equally
stunning. Mr. Erdogan has imposed a sweeping purge, labeling tens of
thousands of civil servants and others as potential enemies of the
state.

While Turkish officials say that followers of Mr. Gulen spearheaded the
plot, it remains unclear whether the attempt was ordered by Mr. Gulen
himself and how much support the plotters received from other parts of
Turkish society. Mr. Gulen has denied any involvement.

\includegraphics{https://static01.nyt.com/images/2016/07/23/world/23TURKEY-PLOT-web3/23TURKEY-PLOT-web3-articleLarge.jpg?quality=75\&auto=webp\&disable=upscale}

In the short term, the purge has threatened to hobble a state already
deeply divided and infected with widespread distrust and uncertainty.

But in the long term, it could mean a complete --- almost revolutionary
--- reordering of the state. There is already talk of devising a
committee system to judge the guilt or innocence of those charged, the
kind of process that might further stoke divisions, turning citizens
against one another.

Mehmet Simsek, Turkey's deputy prime minister, told reporters on
Thursday that he expected that senior figures would be empowered to form
some sort of judgment committees.

Trying to grasp for historical parallels to the upheaval, historians and
analysts have compared this purge to
\href{http://www.nytimes.com/2016/05/15/world/asia/china-cultural-revolution-explainer.html}{Mao
Zedong's Cultural Revolution in China} in the 1960s and 1970s, and the
Islamic Revolution in Iran in 1979.

``Mao and the Iranian Revolution are the ones that come to mind,'' said
Henri J. Barkey, a longtime expert on Turkey who is the director of the
Middle East Program at the Woodrow Wilson International Center for
Scholars. ``But these were revolutions. You expect this.''

He added: ``So the interesting question is, Is Erdogan having his own
revolution? He is going to completely restructure the Turkish state.''

Image

Detained Turkish soldiers thought to have taken part in a failed coup
arrived by bus at the courthouse in Istanbul on
Wednesday.Credit...Bulent Kilic/Agence France-Presse --- Getty Images

This has raised fears of a prolonged witch hunt reminiscent of the
McCarthy era in the United States in the 1950s.

Already, the country's education system is stretched, with tens of
thousands of teachers fired and every university dean, more than 1,500
in total, forced to resign.

``We were given no information as to what will happen next,'' said Ugur
Tanyeli, dean of the faculty of architecture at Bilgi University in
Istanbul. ``We were just asked to resign, and we resigned.''

Many have wondered how the government could so quickly identify so many
thought to be traitors. The answer, Turkish officials say, is that they
had been preparing for this for years.

Government officials have been candid on that point, saying that before
the attempted coup, they were already compiling lists of military
officers and other officials who were suspected of loyalty to Mr. Gulen.

But since the government officials did not have sufficient evidence to
convict them in court, they planned to sideline them over time, Mr.
Simsek, the deputy prime minister, told reporters on Thursday.

Image

Outside the Kocatepe Mosque in Ankara, Turkey, a police officer on
Monday saluted a passing ambulance with bodies of those killed last
Friday during a failed coup.Credit...Nicole Tung for The New York Times

``We knew a lot, but either we didn't have enough legal basis or the
time'' to remove the Gulenists from government, he said.

``We are not making up these stories; this is not some Jason Bourne
trilogy,'' Mr. Simsek added. ``We have these massive cells, networks,
and they have a bank. They have massive financial resources.''

Those efforts got a boost this year, when Turkey's intelligence service
captured a secret communications channel used by Mr. Gulen's followers,
revealing tens of thousands of names and identification numbers,
according to the Turkish security official who was inside the government
compound in Ankara when it was attacked.

The revelation included 600 military officers whose names were shared
with the armed forces so that they could be sidelined when the military
announced new promotions in August. But Mr. Gulen's followers learned
that they had been uncovered and planned the coup to pre-empt their
sidelining, officials said.

The coup was scheduled to begin at 3 a.m. on July 16, a date chosen
partly because the head of the Air Force, Gen. Abidin Unal, and other
commanders were attending a wedding in Istanbul the night before,
according to officials.

But on the afternoon of July 15, the spy agency received reports of
strange activity at a military facility in Guvercinlik, in southwest
Turkey. That evening, the intelligence chief, Hakan Fidan, shared what
he knew with the chief of general staff, Gen. Hulusi Akar, his deputy,
and the head of the army in a meeting at their headquarters, the
security official said.

Image

Supporters of Mr. Erdogan waving Turkish national flags during a
pro-government demonstration at Taksim Square in Istanbul on
Monday.Credit...Alkis Konstantinidis/Reuters

He apparently did not notify Mr. Erdogan, who said in a television
interview this week that he learned about it hours later from his
brother-in-law. Whenever he found out, it was just in time: He managed
to elude a team of commandos dispatched to capture him while he was
vacationing in the resort town of Marmaris.

The putschists apparently realized that their plot had been uncovered,
so they hastily launched the coup on Friday night, coordinating moves
across the country through the messaging platform WhatsApp, according to
security officials.

As the army closed off bridges in Istanbul and fighter jets set off
sonic booms over the city, renegade soldiers attacked government sites
across Ankara, including the military headquarters, where they captured
the three top military officers who had been briefed that afternoon.
Guards repelled the attack on the intelligence headquarters.

``The sense at the Prime Ministry was that if this was in the chain of
command, nothing could be done,'' said Cemalettin Hasimi, a senior
adviser to Prime Minister Binali Yildirim.

Once it became clear that the army leadership had not endorsed the
military action, Mr. Hasimi tried to head to Parliament, only to learn
that it had been bombed, leaving a hole in the roof and rubble in the
corridors, he said.

Mr. Erdogan called his supporters to the streets in an address over
FaceTime that was aired on CNN Turk, and civilians, the police and loyal
army units mobilized and put down the coup.

Mr. Erdogan returned. And the purges began not just of those in the
Gulen movement, but of government critics of all stripes. Some academics
who had signed a petition this year protesting the government's war
against Kurdish militants were suspended from their jobs this week.

Steven A. Cook, a Turkey expert at the Council on Foreign Relations,
asked a simple question: ``Who is going to run the universities? They
will open in six or seven weeks.''

No one seems to know.

``The purge of the education system --- that's the most remote from
grabbing a tank or a plane and doing a coup,'' said Emma Sinclair-Webb,
the Turkey director for Human Rights Watch. ``I don't know how the
country will be governable at this point.''

Government officials insist that the purge is a necessary security
measure to prevent future violence. But its speed and scale have worried
many.

Turkey's political opposition --- secularists, nationalists and Kurds
--- publicly sided with the government against the coup, but it is now
watching warily. Many opponents of Mr. Erdogan have little sympathy for
the Gulen movement, which for years made common cause with the
president's Islamist political movement.

``We know these people, this religious sect,'' said Metin Feyzioglu, the
head of Turkey's bar association and a member of Turkey's main secular
party, who explained that he had long believed the Gulenists had stocked
the country's judiciary.

The worry, he said, is that the purge will broaden to include secular
Turks and others. ``We are afraid it is becoming a real madness, a witch
hunt,'' he said.

Since the coup was put down, Mr. Erdogan's followers have held large
rallies in a number of Turkish cities, calling the failure of the coup a
victory for Turkish democracy.

Out of Turkey's chaos has come more power and more popularity for Mr.
Erdogan, who has used the moment to galvanize his constituency of the
country's religious masses. Seemingly speaking for all who gathered at
the rally, Faruk Akkaya, who works in the tourism industry, said,
``We'll stay here until Recep Tayyip Erdogan tells us to go home.''

Advertisement

\protect\hyperlink{after-bottom}{Continue reading the main story}

\hypertarget{site-index}{%
\subsection{Site Index}\label{site-index}}

\hypertarget{site-information-navigation}{%
\subsection{Site Information
Navigation}\label{site-information-navigation}}

\begin{itemize}
\tightlist
\item
  \href{https://help.nytimes.com/hc/en-us/articles/115014792127-Copyright-notice}{©~2020~The
  New York Times Company}
\end{itemize}

\begin{itemize}
\tightlist
\item
  \href{https://www.nytco.com/}{NYTCo}
\item
  \href{https://help.nytimes.com/hc/en-us/articles/115015385887-Contact-Us}{Contact
  Us}
\item
  \href{https://www.nytco.com/careers/}{Work with us}
\item
  \href{https://nytmediakit.com/}{Advertise}
\item
  \href{http://www.tbrandstudio.com/}{T Brand Studio}
\item
  \href{https://www.nytimes.com/privacy/cookie-policy\#how-do-i-manage-trackers}{Your
  Ad Choices}
\item
  \href{https://www.nytimes.com/privacy}{Privacy}
\item
  \href{https://help.nytimes.com/hc/en-us/articles/115014893428-Terms-of-service}{Terms
  of Service}
\item
  \href{https://help.nytimes.com/hc/en-us/articles/115014893968-Terms-of-sale}{Terms
  of Sale}
\item
  \href{https://spiderbites.nytimes.com}{Site Map}
\item
  \href{https://help.nytimes.com/hc/en-us}{Help}
\item
  \href{https://www.nytimes.com/subscription?campaignId=37WXW}{Subscriptions}
\end{itemize}
