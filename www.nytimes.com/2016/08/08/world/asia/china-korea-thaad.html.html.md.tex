Sections

SEARCH

\protect\hyperlink{site-content}{Skip to
content}\protect\hyperlink{site-index}{Skip to site index}

\href{https://www.nytimes.com/section/world/asia}{Asia Pacific}

\href{https://myaccount.nytimes.com/auth/login?response_type=cookie\&client_id=vi}{}

\href{https://www.nytimes.com/section/todayspaper}{Today's Paper}

\href{/section/world/asia}{Asia Pacific}\textbar{}South Korean Missile
Defense Deal Appears to Sour China's Taste for K-Pop

\url{https://nyti.ms/2aMXw2m}

\begin{itemize}
\item
\item
\item
\item
\item
\end{itemize}

Advertisement

\protect\hyperlink{after-top}{Continue reading the main story}

Supported by

\protect\hyperlink{after-sponsor}{Continue reading the main story}

\hypertarget{south-korean-missile-defense-deal-appears-to-sour-chinas-taste-for-k-pop}{%
\section{South Korean Missile Defense Deal Appears to Sour China's Taste
for
K-Pop}\label{south-korean-missile-defense-deal-appears-to-sour-chinas-taste-for-k-pop}}

\includegraphics{https://static01.nyt.com/images/2016/08/08/world/CHINAKOREA1/CHINAKOREA1-articleLarge.jpg?quality=75\&auto=webp\&disable=upscale}

By Amy Qin and \href{http://www.nytimes.com/by/choe-sang-hun}{Choe
Sang-Hun}

\begin{itemize}
\item
  Aug. 7, 2016
\item
  \begin{itemize}
  \item
  \item
  \item
  \item
  \item
  \end{itemize}
\end{itemize}

BEIJING --- Has K-pop become one of the first victims of a recent
fraying of relations between China and South Korea?

When South Korea angered China last month by
\href{http://www.nytimes.com/2016/07/08/world/asia/south-korea-and-us-agree-to-deploy-missile-defense-system.html?_r=0}{agreeing}
to allow a United States missile-defense system on its soil, there was
much speculation that Beijing would retaliate economically. Now signs
have emerged that China is targeting some of the South's most colorful
exports: its brand of popular music known as K-pop and its widely
popular television shows.

South Korean popular culture is huge in China, which in recent years has
become the largest export market for the South Korean entertainment
industry. But in the past week, several events in China featuring music
and television stars have been called off, and those cancellations have
caused jitters in both countries and sent the stock prices of some of
South Korea's top entertainment companies tumbling.

The cancellations may have been coincidental, but several employees at
South Korean and Chinese media companies, who requested anonymity
because they feared jeopardizing business in the future, expressed
worries that there had been official pressure to put some South Korean
projects on hold.

A fan event in Beijing with the South Korean stars Kim Woo-bin and Suzy
Bae of the television series ``Uncontrollably Fond'' was postponed last
week after the Chinese host, the online streaming company Youku,
received a notice from the police bureau in Beijing suggesting that it
delay the event, according to a Youku employee.

In a
\href{http://weibo.com/1642904381/E1QOp4Dg4?from=page_1002061642904381_profile\&wvr=6\&mod=weibotime\&type=comment\#_rnd1470372650398}{statement}
posted to its official microblog account on Wednesday, the company
announced the postponement of the event, which was scheduled for
Saturday, citing ``forces beyond our control.''

\includegraphics{https://static01.nyt.com/images/2016/08/08/world/CHINAKOREA2/CHINAKOREA2-articleInline.jpg?quality=75\&auto=webp\&disable=upscale}

Other cancellations have included two concerts by the popular South
Korean boy band EXO that were to be held in Shanghai this month. Those
cancellations were confirmed by an employee at the Shanghai-based
ticketing website \href{http://fcpiao.com/}{fcpiao.com}. Two sources in
the Chinese entertainment industry also said some Chinese-Korean
television projects had been put on hold.

Kim Hyong-woo, a spokesman for JYP Entertainment, one of the South
Korean companies whose share prices have suffered, said it had ``nothing
to say except that we are watching the situation.'' Other agencies
offered similar comments mixing caution and anxiety.

Other figures in the South Korean entertainment industry noted that
cancellations of events were not unusual in China. And big K-pop stars
like Psy and Rain have reported no disruptions in their schedules in the
country.

In China, some people in the entertainment industry said companies might
be acting pre-emptively to avoid stepping onto what has become a
political minefield.

It would not be the first time that Korean entertainers have been the
victims of regional tensions.

In 2012, the South Korean president at the time, Lee Myung-bak,
\href{http://www.nytimes.com/2012/08/11/world/asia/south-koreans-visit-to-disputed-islets-angers-japan.html}{visited}
islets at the center of a territorial dispute with Japan. As anti-South
Korea sentiment rose, once wildly popular South Korean TV dramas and boy
and girl bands were abruptly banished from Japanese broadcast channels.

South Korean government officials are watching the developments in China
closely.

''We are analyzing the situation in various angles over whether this has
anything to do with the Thaad deployment, and we will respond
accordingly,'' said Cho June-hyuck, a spokesman for the Foreign
Ministry. Thaad is the acronym for Terminal High Altitude Area Defense,
the United States antimissile system.

The anxiety in South Korea highlighted the increasingly awkward position
that Seoul often finds itself in between China, its largest trade
partner, and the United States, its traditional military ally.

Over the decades, South Korea's export-driven economy has come to depend
on trade with China. While South Koreans and their policy makers have
welcomed the trade relations as an economic boon, they have also
expressed concerns that China might exploit their country's increasing
dependence as political leverage to drive a wedge between Seoul and
Washington.

China's Foreign Ministry did not respond to faxed requests for comment
on the K-pop cancellations. But it did respond to news last week that
the license of a visa agency serving South Korean businesspeople
applying for multi-entry visas to China had been revoked.

In a faxed statement, the ministry wrote that the visa policy had not
changed.

``China attaches great importance to facilitating the personnel
exchanges between China and South Korea, and will continue to provide
convenience for South Korean nationals visiting China,'' it said.

The Chinese state news media has stepped up its criticism of South Korea
in the last week, publishing a number of commentaries attacking the
plans to deploy Thaad.

South Korea insists that the Thaad deployment, in Seongju in southern
South Korea, was intended to protect South Korea and American forces in
the region from North Korea's growing missile threats.

But China says the deployment is part of Washington's plan to bring
South Korea into a missile-defense system aimed at undermining Chinese
and Russian security.

Some Chinese media outlets have even said a government ban on South
Korean entertainers would enjoy widespread support in China.

Image

Members of EXO in Hong Kong in 2015.Credit...Philippe Lopez/Agence
France-Presse --- Getty Images

``A recent survey showed that more than four-fifths of Chinese people
would support the ban on the appearance of South Korean entertainers in
Chinese TV programs if the government does so,''
\href{http://news.xinhuanet.com/english/2016-08/04/c_135563560.htm}{wrote}
Xinhua, the state news agency. ``It reflects Chinese placing love for
their home country before popularity of entertainment stars.''

If South Korea persists in its decision to deploy the antimissile
system, the ``failure of the Korean Wave in China will be inevitable,''
the state-backed Global Times wrote in an
\href{http://opinion.huanqiu.com/editorial/2016-08/9262614.html}{editorial}
published on Thursday.

``Even without official government orders, those embattled South Korean
television stations will drown in the spit of online commentators,'' the
editorial added.

Also last week, the American footwear company K-Swiss became the object
of criticism for a video advertisement that showed the South Korean
actor Park Bo-gum beating a person named ``Great Wall of China'' in the
board game Go.

After taking the video down, the company said on the microblogging
service Weibo, ``With regard to matters relating to a recent K-Swiss
video, top managers at K-Swiss China are taking urgent steps to review
and address the matter.''

``This is pure provocation,'' one online user wrote, responding to the
statement. ``From now on, I will admire only stars respecting China.''

``Get out of China and go eat your kimchi,'' wrote another.

Advertisement

\protect\hyperlink{after-bottom}{Continue reading the main story}

\hypertarget{site-index}{%
\subsection{Site Index}\label{site-index}}

\hypertarget{site-information-navigation}{%
\subsection{Site Information
Navigation}\label{site-information-navigation}}

\begin{itemize}
\tightlist
\item
  \href{https://help.nytimes.com/hc/en-us/articles/115014792127-Copyright-notice}{©~2020~The
  New York Times Company}
\end{itemize}

\begin{itemize}
\tightlist
\item
  \href{https://www.nytco.com/}{NYTCo}
\item
  \href{https://help.nytimes.com/hc/en-us/articles/115015385887-Contact-Us}{Contact
  Us}
\item
  \href{https://www.nytco.com/careers/}{Work with us}
\item
  \href{https://nytmediakit.com/}{Advertise}
\item
  \href{http://www.tbrandstudio.com/}{T Brand Studio}
\item
  \href{https://www.nytimes.com/privacy/cookie-policy\#how-do-i-manage-trackers}{Your
  Ad Choices}
\item
  \href{https://www.nytimes.com/privacy}{Privacy}
\item
  \href{https://help.nytimes.com/hc/en-us/articles/115014893428-Terms-of-service}{Terms
  of Service}
\item
  \href{https://help.nytimes.com/hc/en-us/articles/115014893968-Terms-of-sale}{Terms
  of Sale}
\item
  \href{https://spiderbites.nytimes.com}{Site Map}
\item
  \href{https://help.nytimes.com/hc/en-us}{Help}
\item
  \href{https://www.nytimes.com/subscription?campaignId=37WXW}{Subscriptions}
\end{itemize}
