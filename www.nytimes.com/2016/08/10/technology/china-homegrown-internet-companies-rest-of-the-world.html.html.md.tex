Sections

SEARCH

\protect\hyperlink{site-content}{Skip to
content}\protect\hyperlink{site-index}{Skip to site index}

\href{https://www.nytimes.com/section/technology}{Technology}

\href{https://myaccount.nytimes.com/auth/login?response_type=cookie\&client_id=vi}{}

\href{https://www.nytimes.com/section/todayspaper}{Today's Paper}

\href{/section/technology}{Technology}\textbar{}Chinese Tech Firms
Forced to Choose Market: Home or Everywhere Else

\url{https://nyti.ms/2b4mMEE}

\begin{itemize}
\item
\item
\item
\item
\item
\end{itemize}

Advertisement

\protect\hyperlink{after-top}{Continue reading the main story}

Supported by

\protect\hyperlink{after-sponsor}{Continue reading the main story}

\hypertarget{chinese-tech-firms-forced-to-choose-market-home-or-everywhere-else}{%
\section{Chinese Tech Firms Forced to Choose Market: Home or Everywhere
Else}\label{chinese-tech-firms-forced-to-choose-market-home-or-everywhere-else}}

\includegraphics{https://static01.nyt.com/images/2016/08/08/business/international/wechat-explainer-still/wechat-explainer-still-videoSixteenByNine3000.jpg}

By \href{https://www.nytimes.com/by/paul-mozur}{Paul Mozur}

\begin{itemize}
\item
  Aug. 9, 2016
\item
  \begin{itemize}
  \item
  \item
  \item
  \item
  \item
  \end{itemize}
\end{itemize}

HONG KONG --- For teenagers who like to sing along with Ariana Grande
and Flo Rida, Musical.ly is a must-have. The app that lets users
lip-sync and dance in their own music videos boasts 100 million users
and partners with pop stars like Ms. Grande and Meghan Trainor.

It's not easy to tell Musical.ly is Chinese --- and that's deliberate.
To find success in America, its parent company has ignored China, its
home market and a country with 700 million internet users.

The reason is simple, says Alex Zhu, co-founder of Shanghai-based
Musical.ly: China's internet is fundamentally different from the one
used in much of the rest of the world.

``It's still very difficult to get into China,'' said Mr. Zhu, who
studied civil engineering at Zhejiang University in the eastern city of
Hangzhou. ``It's a closed environment, and you have to be quite
different to compete in that market.''

Two decades after Beijing began walling off its homegrown internet from
the rest of the planet, the digital world
\href{http://www.nytimes.com/2016/08/02/technology/uber-china-internet.html}{has
split between China and everybody else}. That has prevented American
technology companies like Facebook and Uber, which recently
\href{http://www.nytimes.com/2016/08/02/business/dealbook/china-uber-didi-chuxing.html?hpw\&rref=technology\&action=click\&pgtype=Homepage\&module=well-region\&region=bottom-well\&WT.nav=bottom-well}{agreed
to sell its China operations}, from independently being able to tap the
Chinese market.

For China's web companies, the divide may have even more significant
implications.

It has penned in the country's biggest and most innovative internet
companies. Alibaba, Baidu and Tencent have grown to be some of the
world's largest internet companies, but they rely almost entirely on
domestic businesses. Their ventures abroad have been mostly desultory,
and prognostications that they will challenge American giants
internationally have not materialized.

\includegraphics{https://static01.nyt.com/images/2016/08/09/business/00CHINAABROAD4/00CHINAABROAD4-articleLarge.jpg?quality=75\&auto=webp\&disable=upscale}

For Chinese web start-ups like Musical.ly, the internet split has also
forced them to choose --- either create something that caters to China's
digital population or focus on the rest of the globe.

In many ways, the split is like 19th century railroads in the United
States, when rails of different sizes hindered a train's ability to go
from one place to another.

``The barrier to entering the U.S. or China market is becoming higher
and higher,'' said Kai-fu Lee, a venture investor from Taiwan and former
head of Google China.

The difficulties that China's internet companies face in expanding their
success abroad are epitomized by WeChat, the messaging app owned by
Tencent. In China, WeChat combines e-commerce and real-world services in
ways that have
\href{http://www.nytimes.com/2016/08/03/technology/china-mobile-tech-innovation-silicon-valley.html}{Western
companies playing catch-up}. It has about 700 million users, most of
whom are Chinese or use it to connect with people in China.

In 2012, armed with a cash stockpile of several hundred million dollars,
the world soccer star Lionel Messi as a spokesman and local ads like
Bollywood-inspired commercials in India, Tencent began a push that
executives said would be its best chance of breaking out of China. The
effort flopped.

Critics pointed to Tencent's lack of distinctive marketing, a record of
censorship and surveillance in China and its late arrival to foreign
markets. Yet the biggest problem was that outside of China, WeChat was
just not the same. Within China, WeChat can be used to do almost
everything, like pay bills, hail a taxi, book a doctor's appointment,
share photos and chat. Yet its ability to do that is dependent on other
Chinese internet services that are limited outside the country.

That leaves WeChat outside China as an app that people mostly use to
chat and share photos --- not that different from WhatsApp and
Messenger, which are both owned by Facebook. Baidu and Alibaba have apps
that similarly offer a range of capabilities, yet are less useful
outside China.

Image

A screenshot of a video by a popular user of Musical.ly, an app that
combines social media and music.Credit...Gilles Sabrie for The New York
Times

The same problem hurts start-ups in China. Those companies start out
accustomed to using Chinese internet sites and apps to market and
enhance their business. But going abroad means a different world of
services to master, such as a solid understanding of Facebook and
Google's platforms and ads, not Baidu's and Tencent's.

By contrast, Musical.ly chose the opposite approach and linked itself to
the most popular social networks in the United States. If someone
records an impressively coordinated dance or flawlessly lip-synced song,
the person can put it up not just on the app, but also add it to
Instagram, send it on WhatsApp or post it to Facebook. That has helped
Musical.ly grow naturally to Europe, South America and Southeast Asia,
Mr. Zhu said.

``The thing about this young generation in the U.S. is, they're
creative,'' said Mr. Zhu. ``They'll say, `Please follow me on Instagram
or Snapchat.' If your app can attract some people in an age group and
make them super excited to share, you will probably grow.''

For Cheetah Mobile, a maker of smartphone utility apps based in Beijing
whose users are mostly outside China, the solution was finding a
steppingstone to the rest of the world. In early 2014, it opened an
office in Taiwan, where use of Google and Facebook dominates. That
helped it gain employees who intimately understood Facebook, YouTube and
other major Western platforms that could be used for advertising.

``Taiwan served as a bridge for us across the Pacific to the United
States,'' said Charles Fan, Cheetah's chief technology officer.

Tencent, Alibaba and Baidu have all opened American offices, but they
have mostly turned to investments and acquisitions to gain footholds
overseas. Over the last two years, Alibaba has invested in emerging
markets, including two online commerce companies, Paytm and Snapdeal, in
India. It also spent \$1 billion to acquire Lazada, an e-commerce site
popular in Southeast Asia.

Tencent has been more aggressive in Western markets. In June, it made
its
\href{http://www.nytimes.com/2016/06/22/business/dealbook/tencent-softcell-softbank-deal.html}{largest
overseas deal}, paying \$8.6 billion for Supercell, the Finnish company
that created the popular mobile game Clash of Clans. Tencent also has a
stake in the games company Activision Blizzard and bought one of the
most played games in the world, League of Legends.

Image

The Shanghai offices of Musical.ly, a Chinese start-up that chose to
build its business abroad.Credit...Gilles Sabrie for The New York Times

Perhaps the greatest indication of Tencent's overseas ambitions was a
deal that never happened. In 2014, with its global WeChat campaign
faltering, it was preparing to start negotiations to bid for WhatsApp
when Facebook swooped in, according to a senior Tencent executive who
asked for anonymity in discussing corporate strategy.

Tencent and Baidu declined to comment. An Alibaba spokeswoman referred
to recent remarks by Alibaba's president, J. Michael Evans, in which he
pointed to acquisitions as a way the company was attracting new
consumers in developing markets. He also said Alibaba was focused on
attracting more foreign businesses to sell on its markets in China.

Mr. Lee said it might take a new technological jump for Chinese
companies to get a chance at building a platform inside China and
internationally. He said Chinese companies could prove competitive in
emerging sectors like virtual reality, artificial intelligence and
robotics.

``I think what might be surprising is, China will catch up rapidly,'' he
said. ``Partly because of Chinese universities, partly because of
returnees to China who form a portion of the top engineers in the
world.''

Musical.ly is in many ways a product of the cultural exchange between
the United States and China that Mr. Lee described. Mr. Zhu, 37,
graduated from a Chinese university, but moved to the United States with
the German software company SAP in 2010. He had the idea for the music
app while riding the train from San Francisco to Mountain View, Calif.,
in a car full of high school students.

``Half were listening to music and the other half were using their phone
to take photos and add emojis, and they were passing them around,'' Mr.
Zhu said. Then it hit him: combine the selfie and social media part with
the music part and turn it into one product. In 2015, Mr. Zhu moved to
Shanghai, where his co-founder has been based since Musical.ly's 2013
inception.

Yet Musical.ly is unlikely to be the social network to link both sides
of the Pacific. For the demographic the app is focusing on, it's far
better to be outside its home market, Mr. Zhu said.

``Teenagers in the U.S. are a golden audience,'' he said. ``If you look
at China, the teenage culture doesn't exist --- the teens are super busy
in school studying for tests, so they don't have the time and luxury to
play social media apps.''

Advertisement

\protect\hyperlink{after-bottom}{Continue reading the main story}

\hypertarget{site-index}{%
\subsection{Site Index}\label{site-index}}

\hypertarget{site-information-navigation}{%
\subsection{Site Information
Navigation}\label{site-information-navigation}}

\begin{itemize}
\tightlist
\item
  \href{https://help.nytimes.com/hc/en-us/articles/115014792127-Copyright-notice}{©~2020~The
  New York Times Company}
\end{itemize}

\begin{itemize}
\tightlist
\item
  \href{https://www.nytco.com/}{NYTCo}
\item
  \href{https://help.nytimes.com/hc/en-us/articles/115015385887-Contact-Us}{Contact
  Us}
\item
  \href{https://www.nytco.com/careers/}{Work with us}
\item
  \href{https://nytmediakit.com/}{Advertise}
\item
  \href{http://www.tbrandstudio.com/}{T Brand Studio}
\item
  \href{https://www.nytimes.com/privacy/cookie-policy\#how-do-i-manage-trackers}{Your
  Ad Choices}
\item
  \href{https://www.nytimes.com/privacy}{Privacy}
\item
  \href{https://help.nytimes.com/hc/en-us/articles/115014893428-Terms-of-service}{Terms
  of Service}
\item
  \href{https://help.nytimes.com/hc/en-us/articles/115014893968-Terms-of-sale}{Terms
  of Sale}
\item
  \href{https://spiderbites.nytimes.com}{Site Map}
\item
  \href{https://help.nytimes.com/hc/en-us}{Help}
\item
  \href{https://www.nytimes.com/subscription?campaignId=37WXW}{Subscriptions}
\end{itemize}
