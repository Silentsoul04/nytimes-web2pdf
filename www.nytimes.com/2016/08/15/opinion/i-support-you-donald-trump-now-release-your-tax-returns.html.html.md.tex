Sections

SEARCH

\protect\hyperlink{site-content}{Skip to
content}\protect\hyperlink{site-index}{Skip to site index}

\href{https://myaccount.nytimes.com/auth/login?response_type=cookie\&client_id=vi}{}

\href{https://www.nytimes.com/section/todayspaper}{Today's Paper}

\href{/section/opinion}{Opinion}\textbar{}Mark Sanford: I Support You,
Donald Trump. Now Release Your Tax Returns.

\url{https://nyti.ms/2b9ADpO}

\begin{itemize}
\item
\item
\item
\item
\item
\item
\end{itemize}

Advertisement

\protect\hyperlink{after-top}{Continue reading the main story}

Supported by

\protect\hyperlink{after-sponsor}{Continue reading the main story}

\href{/section/opinion}{Opinion}

Op-Ed Contributor

\hypertarget{mark-sanford-i-support-you-donald-trump-now-release-your-tax-returns}{%
\section{Mark Sanford: I Support You, Donald Trump. Now Release Your Tax
Returns.}\label{mark-sanford-i-support-you-donald-trump-now-release-your-tax-returns}}

By Mark Sanford

\begin{itemize}
\item
  Aug. 14, 2016
\item
  \begin{itemize}
  \item
  \item
  \item
  \item
  \item
  \item
  \end{itemize}
\end{itemize}

\includegraphics{https://static01.nyt.com/images/2016/08/15/opinion/15Sanford-web/15Sanford-web-articleInline.jpg?quality=75\&auto=webp\&disable=upscale}

Among Donald J. Trump's traits is his penchant for internalizing and
personalizing things --- insults, rejections and even policy
disagreements. Trading slights seems essential to his personality, or at
least something he feels is necessary for his presidential campaign.

Although it's not something I feel comfortable with, the bombast of this
campaign will not be remembered with the passage of time. Words come and
go. The problem is what happens when his words lead him to do things
that will reverberate long after the campaign is over.

To him, demands that he release his tax returns are just a ploy by his
opponents and enemies to undermine his campaign. But that obstinacy will
have consequences. Not releasing his tax returns would hurt transparency
in our democratic process, and particularly in how voters evaluate the
men and women vying to be our leaders. Whether he wins or loses, that is
something our country cannot afford.

I suggest this not as a partisan against Mr. Trump. I am a conservative
Republican who, though I have no stomach for his personal style and his
penchant for regularly demeaning others, intends to support my party's
nominee because of the importance of filling the existing vacancy on the
Supreme Court, and others that might open in the next four years.
However, my ability to continue to do so will in part be driven by
whether Mr. Trump keeps his word that he will release his tax records.

Let me explain why this issue is so important to me --- and, indeed,
much bigger than Mr. Trump and the current campaign for the presidency.

For one, it's not really about his tax records per se. It's about the
American public's ability to see other candidates' returns. We have a
long precedent in which every major-party presidential candidate since I
was a child has released his returns. Break it now, and it stays broken.

The presidency is the most powerful political position on earth, and the
idea of enabling the voter the chance to see how a candidate has handled
his or her finances is a central part of making sure the right person
gets the job. There is a reason a banker wants to see tax returns in
determining whether you are eligible for a mortgage. You may talk a good
game; tax returns don't. Mr. Trump knows all this, which is why his team
had his running mate, Gov. Mike Pence of Indiana, disclose his tax
returns --- again, an accepted and expected practice in vetting
potential vice-presidential candidates.

In fact, the real issue is not even about presidential tax returns.
Rather, it's about the hundreds of down-ballot races, in states and
localities, and the transparency voters deserve here, too. I ran twice
for governor of South Carolina, and I released my tax returns both
times. To be frank, it felt a bit like a colonoscopy: I didn't like it,
but it was our tradition in South Carolina. The power of staying true to
the precedent that had been set prevailed. If presidential candidates
won't release their tax returns, you can expect the same in the states.
If a presidential nominee doesn't do it, why should a candidate for
governor?

And it matters in ways that aren't immediately obvious. I once
participated in a debate in Congress on raising the president's salary.
As it turned out, the debate wasn't really about the president's
earnings in office --- modern presidents wind up anything but poor ---
but instead about the pay scale for federal judges: Their pay was
ultimately capped at a fixed range below the president's, so for judges'
pay to rise, the president's had to also. It was an important reminder:
As with so many things in our country, the standards we set for the
president determine what political standards we set for the rest of the
country.

Finally, this is about taking Mr. Trump at his own word. He has
certainly dodged and hedged on the subject recently, but many other
times he has been
\href{http://www.nationalmemo.com/showusyourtaxes-heres-every-time-donald-trump-has-said-he-would-release-his-tax-returns/}{remarkably
clear} that he would make his tax returns public. He consistently chided
Mitt Romney, the 2012 Republican presidential nominee, for not releasing
his returns sooner in his campaign. In May 2014, as Mr. Trump
entertained the idea of running for president, he said, on television:
``If I decide to run for office, I'll produce my tax returns,
absolutely, and I would love to do that.'' Nothing has changed that
should justify Mr. Trump's changing his mind.

A maxim often attributed to Thomas Jefferson holds that ``an educated
citizenry is a vital requisite for our survival as a free people.''
Equipping voters with more, not less, information as they pick those who
run for the highest offices in our land seems, to me, a reasonable
requirement for anyone aspiring to those positions.

Advertisement

\protect\hyperlink{after-bottom}{Continue reading the main story}

\hypertarget{site-index}{%
\subsection{Site Index}\label{site-index}}

\hypertarget{site-information-navigation}{%
\subsection{Site Information
Navigation}\label{site-information-navigation}}

\begin{itemize}
\tightlist
\item
  \href{https://help.nytimes.com/hc/en-us/articles/115014792127-Copyright-notice}{©~2020~The
  New York Times Company}
\end{itemize}

\begin{itemize}
\tightlist
\item
  \href{https://www.nytco.com/}{NYTCo}
\item
  \href{https://help.nytimes.com/hc/en-us/articles/115015385887-Contact-Us}{Contact
  Us}
\item
  \href{https://www.nytco.com/careers/}{Work with us}
\item
  \href{https://nytmediakit.com/}{Advertise}
\item
  \href{http://www.tbrandstudio.com/}{T Brand Studio}
\item
  \href{https://www.nytimes.com/privacy/cookie-policy\#how-do-i-manage-trackers}{Your
  Ad Choices}
\item
  \href{https://www.nytimes.com/privacy}{Privacy}
\item
  \href{https://help.nytimes.com/hc/en-us/articles/115014893428-Terms-of-service}{Terms
  of Service}
\item
  \href{https://help.nytimes.com/hc/en-us/articles/115014893968-Terms-of-sale}{Terms
  of Sale}
\item
  \href{https://spiderbites.nytimes.com}{Site Map}
\item
  \href{https://help.nytimes.com/hc/en-us}{Help}
\item
  \href{https://www.nytimes.com/subscription?campaignId=37WXW}{Subscriptions}
\end{itemize}
