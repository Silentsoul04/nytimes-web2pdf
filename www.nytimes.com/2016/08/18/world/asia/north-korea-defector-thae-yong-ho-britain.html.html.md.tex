Sections

SEARCH

\protect\hyperlink{site-content}{Skip to
content}\protect\hyperlink{site-index}{Skip to site index}

\href{https://www.nytimes.com/section/world/asia}{Asia Pacific}

\href{https://myaccount.nytimes.com/auth/login?response_type=cookie\&client_id=vi}{}

\href{https://www.nytimes.com/section/todayspaper}{Today's Paper}

\href{/section/world/asia}{Asia Pacific}\textbar{}North Korea's No. 2
Diplomat in London Defects to the South

\url{https://nyti.ms/2bmNFUx}

\begin{itemize}
\item
\item
\item
\item
\item
\end{itemize}

Advertisement

\protect\hyperlink{after-top}{Continue reading the main story}

Supported by

\protect\hyperlink{after-sponsor}{Continue reading the main story}

\hypertarget{north-koreas-no-2-diplomat-in-london-defects-to-the-south}{%
\section{North Korea's No. 2 Diplomat in London Defects to the
South}\label{north-koreas-no-2-diplomat-in-london-defects-to-the-south}}

\includegraphics{https://static01.nyt.com/images/2016/08/19/world/18KOREA-web5/18KOREA-web5-videoSixteenByNineJumbo1600.jpg}

By \href{http://www.nytimes.com/by/choe-sang-hun}{Choe Sang-Hun} and
\href{https://www.nytimes.com/by/rick-gladstone}{Rick Gladstone}

\begin{itemize}
\item
  Aug. 17, 2016
\item
  \begin{itemize}
  \item
  \item
  \item
  \item
  \item
  \end{itemize}
\end{itemize}

SEOUL, South Korea --- He enjoyed a bit of tennis at the local club. He
indulged in curry at an Indian restaurant in the west London
neighborhood where he lived. As the No. 2 North Korean diplomat in
Britain, he chaperoned a brother of his country's reclusive leader to an
Eric Clapton concert last year.

The diplomat, Thae Yong-ho, 55, seemed to embrace the trappings of a
comfortable life in a capitalist capital thousands of miles from dreary
\href{http://topics.nytimes.com/top/news/international/countriesandterritories/northkorea/index.html?inline=nyt-geo}{North
Korea}, never hinting at disloyalty. He had lived in London for a
decade, trusted because of his family's impeccable legacy in North
Korean history.

So it was a shock on Wednesday when South Korea announced that Mr. Thae
had betrayed his hermetic homeland by becoming the most senior North
Korean official to defect in nearly two decades.

How and when the diplomat had eluded his colleagues at the North Korean
Embassy, who are required to monitor one another to thwart treason, was
not clear. But a South Korean government spokesman, Jeong Joon-hee, said
at a news conference that the diplomat had arrived recently in South
Korea with his wife and family, proclaiming disillusionment with the
increasingly isolated government of the North Korean leader, Kim
Jong-un.

Mr. Jeong did not specify how many family members had accompanied Mr.
Thae or whether any remained in North Korea, where they could be at risk
of reprisal. Nor did he explain the route taken by Mr. Thae, second in
rank to Ambassador Hyon Hak-bong in London.

``We see his defection as a sign that some of the core elite in the
North are losing hope in the Kim Jong-un regime,'' Mr. Jeong said, ``and
that the internal unity of the ruling class in the North is weakening.''

South Korean officials expressed similar conclusions in April when 13
people working at a
\href{http://www.nytimes.com/2016/04/09/world/asia/north-korean-defectors-restaurant-south-korea.html}{restaurant
run by the North Korean government} in China fled to the South.
Officials said that unusual group defection reflected growing
dissatisfaction in the North.

But analysts have cautioned against drawing such conclusions.

Cheong Seong-chang, an expert on North Korea at the Sejong Institute in
Seoul, said that isolated defections like Mr. Thae's should not be taken
as an indication of instability in the North, and that there was no sign
of an organized challenge to Mr. Kim's rule.

Others were cautious because some North Korean defectors to South Korea
have not always found happiness, a message that may have found its way
back to Pyongyang, the North's capital, and elsewhere.

``I think we will continue to see senior defections but at a trickling
pace,'' said Jae H. Ku, director of the U.S. Korea Institute at the
Johns Hopkins University's School of Advanced International Studies in
Washington. ``I don't think we are at a point where we will see massive
defections by elites because these elites have not yet found ways to
live comfortably in South Korea.''

Still, Mr. Thae's defection could yield a trove of intelligence
information. It came as relations between the Koreas had worsened over
the North's nuclear weapons and missile programs, tested in defiance of
international sanctions.

Image

Jeong Joon-hee, above, a South Korean government spokesman, said that
the defection of the diplomat Thae Yong-ho was ``a sign that some of the
core elite in the North are losing hope in the Kim Jong-un
regime.''Credit...Kim Hyun-Tae/Yonhap, via Associated Press

While the North had no immediate reaction to the defection announcement,
it was seen in the South and elsewhere as a major embarrassment for Mr.
Kim, who has disciplined subordinates by demoting them or in some cases
executing them.

The last time a North Korean diplomat of such high rank defected was in
1997, when
\href{http://www.nytimes.com/1997/08/25/world/north-korean-envoy-said-to-defect-in-cairo.html}{Jang
Seung-gil}, the ambassador to Egypt, sought refuge in the United States
with his younger brother, a North Korean diplomat in Paris.

Inklings of a betrayal in the North Korean Embassy in London surfaced a
few days ago when a South Korean mass-circulation newspaper, JoongAng
Ilbo, quoted an anonymous source saying a diplomat there had defected in
early August after ``painstaking preparation.'' By the time other
embassy officials realized this, the newspaper said, the diplomat had
fled.

Mr. Thae has been well known in the British news media, acting as the
embassy's main point of contact for British correspondents traveling to
Pyongyang. Reuters reported that Mr. Thae spoke regularly at far-left
events in London, including meetings of a British Communist Party where
he would make impassioned speeches in defense of North Korea.

Steve Evans, a BBC Korea correspondent who had met Mr. Thae in London,
\href{http://www.bbc.com/news/magazine-37098904}{remembered the North
Korean as a middle-aged man who appeared to enjoy life in the suburbs of
west London}. He frequented a curry restaurant and liked to talk about
family and health, including worries about the onset of diabetes, Mr.
Evans said. He switched to tennis after his wife complained about his
obsession with golf.

Mr. Thae was one of the North Korean escorts seen accompanying Mr. Kim's
elder brother, Kim Jong-chol, to
\href{http://www.bbc.com/news/world-asia-32843186}{a Clapton concert} in
London in 2015.

The diplomat had been scheduled to return to Pyongyang this summer with
his wife and son, Mr. Evans reported.

``But he seemed so British,'' he wrote. ``He seemed so at home. He
seemed so middle class, so conservative, so dapper. He had never given
any hint of disloyalty to the regime, not a flicker of doubt.''

According to South Korean news media, Mr. Thae and his wife, Oh Hae-son,
50, had elite family backgrounds. Mr. Thae was a son of Thae Byong-ryol,
a comrade-in-arms of Mr. Kim's grandfather, the North's founding
president, Kim Il-sung, when the elder Mr. Kim was a leader of Korean
guerrillas fighting against Japanese colonialists in the early 20th
century, the South Korean news agency Yonhap reported. Ms. Oh was a
relative of another former Korean partisan guerrilla, Oh Baek-ryong, the
South Korean newspaper Chosun Ilbo reported.

Offspring of the former guerrillas occupy key posts of the Pyongyang
government, constituting a core elite buttressing Mr. Kim's rule and
living with luxuries ordinary North Koreans can only dream about. They
are allowed to study abroad, as Mr. Thae and his sons did, in China and
in Europe. Mr. Thae served in London for 10 years, an unusually long
stint in a prime outpost for a North Korean diplomat.

South Korean officials have cited recent defectors as proof that some
North Korean elites abroad were defecting rather than facing
persecution, as it became increasingly difficult to perform their
missions under tightened international sanctions.

The number of North Korean defectors arriving in South Korea dropped
from a high of 2,706 in 2011 to 1,275 last year, as Mr. Kim ordered his
country to tighten border control with China, the first stop for almost
all asylum seekers. The number of defectors began picking up again this
year, with 749 arriving in the first six months.

Advertisement

\protect\hyperlink{after-bottom}{Continue reading the main story}

\hypertarget{site-index}{%
\subsection{Site Index}\label{site-index}}

\hypertarget{site-information-navigation}{%
\subsection{Site Information
Navigation}\label{site-information-navigation}}

\begin{itemize}
\tightlist
\item
  \href{https://help.nytimes.com/hc/en-us/articles/115014792127-Copyright-notice}{©~2020~The
  New York Times Company}
\end{itemize}

\begin{itemize}
\tightlist
\item
  \href{https://www.nytco.com/}{NYTCo}
\item
  \href{https://help.nytimes.com/hc/en-us/articles/115015385887-Contact-Us}{Contact
  Us}
\item
  \href{https://www.nytco.com/careers/}{Work with us}
\item
  \href{https://nytmediakit.com/}{Advertise}
\item
  \href{http://www.tbrandstudio.com/}{T Brand Studio}
\item
  \href{https://www.nytimes.com/privacy/cookie-policy\#how-do-i-manage-trackers}{Your
  Ad Choices}
\item
  \href{https://www.nytimes.com/privacy}{Privacy}
\item
  \href{https://help.nytimes.com/hc/en-us/articles/115014893428-Terms-of-service}{Terms
  of Service}
\item
  \href{https://help.nytimes.com/hc/en-us/articles/115014893968-Terms-of-sale}{Terms
  of Sale}
\item
  \href{https://spiderbites.nytimes.com}{Site Map}
\item
  \href{https://help.nytimes.com/hc/en-us}{Help}
\item
  \href{https://www.nytimes.com/subscription?campaignId=37WXW}{Subscriptions}
\end{itemize}
