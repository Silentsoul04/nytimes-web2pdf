Sections

SEARCH

\protect\hyperlink{site-content}{Skip to
content}\protect\hyperlink{site-index}{Skip to site index}

\href{https://www.nytimes.com/section/politics}{Politics}

\href{https://myaccount.nytimes.com/auth/login?response_type=cookie\&client_id=vi}{}

\href{https://www.nytimes.com/section/todayspaper}{Today's Paper}

\href{/section/politics}{Politics}\textbar{}`No Vacancies' for Blacks:
How Donald Trump Got His Start, and Was First Accused of Bias

\url{https://nyti.ms/2bOYUVI}

\begin{itemize}
\item
\item
\item
\item
\item
\item
\end{itemize}

Advertisement

\protect\hyperlink{after-top}{Continue reading the main story}

Supported by

\protect\hyperlink{after-sponsor}{Continue reading the main story}

\hypertarget{no-vacancies-for-blacks-how-donald-trump-got-his-start-and-was-first-accused-of-bias}{%
\section{`No Vacancies' for Blacks: How Donald Trump Got His Start, and
Was First Accused of
Bias}\label{no-vacancies-for-blacks-how-donald-trump-got-his-start-and-was-first-accused-of-bias}}

\includegraphics{https://static01.nyt.com/images/2016/08/27/us/28trumphousing1/28trumphousing1-articleInline.jpg?quality=75\&auto=webp\&disable=upscale}

By \href{https://www.nytimes.com/by/jonathan-mahler}{Jonathan Mahler}
and \href{http://www.nytimes.com/by/steve-eder}{Steve Eder}

\begin{itemize}
\item
  Aug. 27, 2016
\item
  \begin{itemize}
  \item
  \item
  \item
  \item
  \item
  \item
  \end{itemize}
\end{itemize}

She seemed like the model tenant. A 33-year-old nurse who was living at
the Y.W.C.A. in Harlem, she had come to rent a one-bedroom at the
still-unfinished Wilshire Apartments in the Jamaica Estates neighborhood
of Queens. She filled out what the rental agent remembers as a
``beautiful application.'' She did not even want to look at the unit.

There was just one hitch: Maxine Brown was black.

Stanley Leibowitz, the rental agent, talked to his boss, Fred C. Trump.

``I asked him what to do and he says, `Take the application and put it
in a drawer and leave it there,''' Mr. Leibowitz, now 88, recalled in an
interview.

It was late 1963 --- just months before President Lyndon B. Johnson
signed the landmark Civil Rights Act --- and the tall, mustachioed Fred
Trump was approaching the apex of his building career. He was about to
complete the jewel in the crown of his middle-class housing empire:
seven 23-story towers, called Trump Village, spread across nearly 40
acres in Coney Island.

He was also grooming his heir. His son Donald, 17, would soon enroll at
Fordham University in the Bronx, living at his parents' home in Queens
and spending much of his free time touring construction sites in his
father's Cadillac, driven by a black chauffeur.

``His father was his idol,'' Mr. Leibowitz recalled. ``Anytime he would
come into the building, Donald would be by his side.''

Over the next decade, as
\href{http://www.nytimes.com/2016/08/31/us/politics/donald-trump-mexico.html}{Donald
J. Trump} assumed an increasingly prominent role in the business, the
company's practice of turning away potential black tenants was
painstakingly documented by activists and organizations that viewed
equal housing as the next frontier in the civil rights struggle.

The Justice Department undertook its own investigation and, in 1973,
sued Trump Management for discriminating against blacks. Both Fred
Trump, the company's chairman, and Donald Trump, its president, were
named as defendants. It
\href{http://www.nytimes.com/times-insider/2015/07/30/1973-meet-donald-trump/?_r=0}{was
front-page news}, and for Donald, amounted to his debut in the public
eye.

``Absolutely ridiculous,''
\href{http://www.nytimes.com/interactive/2016/08/27/us/politics/times-1973.html}{he
was quoted as saying} of the government's allegations.

\href{https://www.nytimes.com/interactive/2016/08/27/us/politics/trump-affidavit.html}{}

\includegraphics{https://static01.nyt.com/images/2016/08/27/us/politics/trumpaffidavitcrop/trumpaffidavitcrop-largeHorizontalJumbo.png}

\hypertarget{affidavit-of-donald-trump-in-1973-about-the-housing-bias-case}{%
\subsection{Affidavit of Donald Trump in 1973 About the Housing Bias
Case}\label{affidavit-of-donald-trump-in-1973-about-the-housing-bias-case}}

In a 1973 court filing, Donald Trump denied allegations by the
government that Trump Management had engaged in racial bias in its
renting of apartments. The document was contained in the National
Archives.

Looking back, Mr. Trump's response to the lawsuit can be seen as
presaging his handling of subsequent challenges, in business and in
politics. Rather than quietly trying to settle --- as another New York
developer had done a couple of years earlier --- he turned the lawsuit
into a protracted battle, complete with angry denials, character
assassination, charges that the government was trying to force him to
rent to ``welfare recipients'' and a \$100 million countersuit accusing
the Justice Department of defamation.

When it was over, Mr. Trump declared victory, emphasizing that the
consent decree he ultimately signed did not include an admission of
guilt.

But an investigation by The New York Times --- drawing on decades-old
files from the New York City Commission on Human Rights, internal
Justice Department records, court documents and interviews with tenants,
civil rights activists and prosecutors --- uncovered a long history of
racial bias at his family's properties, in New York and beyond.

That history has taken on fresh relevance with Mr. Trump arguing that
black voters
\href{http://www.nytimes.com/2016/08/25/us/politics/donald-trump-black-voters.html}{should
support him} over Hillary Clinton, whom he has called a bigot.

While there is no evidence that Mr. Trump personally set the rental
policies at his father's properties, he was on hand while they were in
place, working out of a cubicle in Trump Management's Brooklyn offices
as early as the summer of 1968.

Then and now, Mr. Trump has steadfastly denied any awareness of any
discrimination at Trump properties. While Mr. Trump declined to be
interviewed for this article, his general counsel, Alan Garten, said in
a statement that there was ``no merit to the allegations.'' And there
has been no suggestion of racial bias toward prospective residents in
the luxury housing that Mr. Trump focused on as his career took off in
Manhattan in the 1980s.

In the past, Mr. Trump has treated the case as a footnote in the
narrative of his career. In his memoir ``The Art of the Deal,'' he
dispensed with it in five paragraphs. And while stumping in Ohio, he
even singled out his work at one of his father's properties in
Cincinnati, omitting that, at the time, the development was the subject
of a separate discrimination lawsuit --- one that included claims of
racial slurs uttered by a manager whom Mr. Trump had personally praised.

As eager as he was to leave behind the working-class precincts of New
York City where Fred Trump had made his fortune, Donald Trump often
speaks admiringly of him, recalling what he learned at his father's side
when the Trump name was synonymous with utilitarian housing, not yet
with luxury, celebrity, or a polarizing brand of politics.

``My legacy has its roots in my father's legacy,''
\href{https://www.washingtonpost.com/news/wonk/wp/2015/08/10/the-middle-class-housing-empire-donald-trump-abandoned-for-luxury-building/}{he
said last year}.

\hypertarget{coming-under-scrutiny}{%
\subsection{Coming Under Scrutiny}\label{coming-under-scrutiny}}

Fred Trump got into the housing business when he was in his early 20s,
building a single-family home for a neighbor in Queens. During World War
II, he constructed housing for shipyard workers and Navy personnel in
Norfolk, Va. After the war, he returned to New York, setting his sights
on bigger, more ambitious projects, realized with the help of federal
government loans.

His establishment as one of the city's biggest developers was hardly
free of controversy: The Senate Banking Committee subpoenaed him in 1954
during an investigation into profiteering off federal housing loans.
Under oath, he acknowledged that he had wildly overstated the costs of a
development to obtain a larger mortgage from the government.

In 1966, as the investigative journalist Wayne Barrett detailed in
``Trump: The Greatest Show on Earth,'' a New York legislative committee
accused Fred Trump of using state money earmarked for middle-income
housing to build a shopping center instead. One lawmaker called Mr.
Trump ``greedy and grasping.''

By this point, the Trump organization's business practices were
beginning to come under scrutiny from civil rights groups that had
received complaints from prospective African-American tenants.

People like Maxine Brown.

Mr. Leibowitz, the rental agent at the Wilshire, remembered Ms. Brown
\href{http://www.nytimes.com/interactive/2016/08/27/us/politics/maxine-brown.html}{repeatedly
inquiring about the apartment}. ``Finally, she realized what it was all
about,'' he said.

Ms. Brown's first instinct was to let the matter go; she was happy
enough at the Y.W.C.A. ``I had a big room and two meals a day for five
dollars a week,'' she said in an interview.

But a friend, Mae Wiggins, who had also been denied an apartment at the
Wilshire, told her that she ought to have her own place, with a private
bathroom and a kitchen. She encouraged Ms. Brown to file a complaint
with the New York City Commission on Human Rights, as she was doing.

``We knew there was prejudice in renting,'' Ms. Wiggins recalled. ``It
was rampant in New York. It made me feel really bad, and I wanted to do
something to right the wrong.''

\includegraphics{https://static01.nyt.com/images/2016/08/26/multimedia/trump-housing/trump-housing-videoSixteenByNine3000-v2.jpg}

Mr. Leibowitz was called to testify at the commission's hearing on Ms.
Brown's case. Asked to estimate how many blacks lived in Mr. Trump's
various properties, he remembered replying: ``To the best of my
knowledge, none.''

After the hearing, Ms. Brown was offered an apartment in the Wilshire,
and in the spring of 1964, she moved in. For 10 years, she said, she was
the only African-American in the building.

Complaints about the Trump organization's rental policies continued to
mount: By 1967, state investigators found that out of some 3,700
apartments in Trump Village, seven were occupied by African-American
families.

Like Ms. Brown, the few minorities who did live in Trump-owned buildings
often had to force their way in.

A black woman named Agnes Bunn recalled hearing in early 1970 about a
vacant Trump apartment in another part of Queens, from a white friend
who lived in the building. But when she went by, she was told there were
no vacancies.

``The super came out and stood there until I left the property,'' Ms.
Bunn said.

Ms. Bunn testified about
\href{http://www.nytimes.com/interactive/2016/08/27/us/politics/agnes-bunn-file.html}{the
experience} at a meeting with the New York City Commission on Human
Rights in 1970. According to a summary, recovered from the New York City
Municipal Archives, she told a Trump lawyer that it was known that no
``colored'' people were wanted as tenants in the building.

The lawyer concluded that the episode was ``all a misunderstanding.''
Ms. Bunn and her husband, a Manhattan accountant, soon became the
building's first black tenants.

Unlike the public schools, the housing market could not be desegregated
simply by court order. Even after passage of the Fair Housing Act of
1968, which prohibited racial discrimination in housing, developments in
white neighborhoods continued to rebuff blacks.

For years, it fell largely to local civil rights groups to highlight the
problem by sending white ``testers'' into apartment complexes after
blacks had been turned away.

``Everything was sort of whispers and innuendo and you wanted to try to
bring it out into the open,'' recalled Phyllis Kirschenbaum, who
volunteered for Operation Open City, a housing rights advocacy
organization. ``I'd walk in with my freckles and red hair and Jewish
name and get an apartment immediately.''

The complaints of discrimination were not limited to New York.

In 1969, a young black couple, Haywood and Rennell Cash, sued after
being denied a home in Cincinnati at one of the first projects in which
Donald Trump, fresh out of college, played an active role.

Mr. Cash was repeatedly rejected by the Trumps' rental agent, according
to court records and notes kept by Housing Opportunities Made Equal of
Cincinnati, which sent in white testers posing as a young couple while
Mr. Cash waited in the car.

After the agent, Irving Wolper, offered the testers an apartment, they
brought in Mr. Cash. Mr. Wolper grew furious, shoving them out of the
office and calling the young female tester, Maggie Durham, a
``nigger-lover,'' according to court records.

``To this day I have not forgotten the fury in his voice and in his
face,'' Ms. Durham recalled recently, adding that she also remembered
him calling her a ``traitor to the race.''

The Cashes were ultimately offered an apartment.

At a campaign stop in Ohio recently, Mr. Trump shared warm memories of
his time in Cincinnati, calling it one of the early successes of his
career. And in ``The Art of the Deal,'' he praised Mr. Wolper, without
using his surname, calling him a ``fabulous man'' and ``an amazing
manager.''

``Irving was a classic,'' Mr. Trump wrote.

The young Mr. Trump also spent time in Norfolk, helping manage the
housing complexes his father built there in the 1940s. Similar
complaints of discrimination surfaced at those properties beginning in
the mid-1960s, and were documented by Ellis James, an equal housing
activist.

``The managers on site were usually not very sophisticated,'' Mr. James,
now 78, recalled. ``Some were dedicated segregationists, but most of
them were more concerned with following the policies they were directed
to keep.''

\hypertarget{battling-the-government}{%
\subsection{Battling the Government}\label{battling-the-government}}

Donald Trump said he had first heard about the lawsuit, which was filed
in the fall of 1973, on his car radio.

The government had charged him, his father and their company, Trump
Management Inc., with violating the Fair Housing Act.

Another major New York developer, the LeFrak Organization, had been hit
with a similar suit a few years earlier. Its founder, Samuel LeFrak, had
appeared at a news conference alongside the United States attorney,
trumpeting a
\href{http://www.nytimes.com/1976/02/01/archives/lefrak-city-crucible-of-racial-change-lefrak-city-crucible-of.html}{consent
agreement to prohibit discrimination} in his buildings by saying it
would ``make open housing in our cities a reality.'' The LeFrak company
even offered the equivalent of one month's rent to help 50 black
families move into predominantly white buildings.

Donald Trump took a different approach. He retained Senator Joseph
McCarthy's red-baiting counsel,
\href{http://www.nytimes.com/2016/06/21/us/politics/donald-trump-roy-cohn.html}{Roy
Cohn, to defend him}. Mr. Trump soon called his own news conference ---
to announce
his\href{http://www.nytimes.com/1973/12/13/archives/realty-company-asks-100million-bias-damages-no-names-or-addresses.html}{countersuit}against
the government.

The government's lawyers took as their starting point the years of
research conducted by civil rights groups at Trump properties.

\includegraphics{https://static01.nyt.com/images/2016/08/27/us/28trumphousing2/28trumphousing2-articleInline.jpg?quality=75\&auto=webp\&disable=upscale}

``We did our own investigation and enlarged the case,'' said Elyse
Goldweber, who as a young assistant United States attorney worked on the
lawsuit, U.S.A. v. Trump.

A former Trump superintendent named Thomas Miranda testified that
multiple Trump Management employees had instructed him to attach a
separate piece of paper with a big letter ``C'' on it --- for
``colored'' --- to any application filed by a black apartment-seeker.

The Trumps went on the offensive, filing a contempt-of-court charge
against one of the prosecutors, accusing her of turning the
investigation into a ``Gestapo-like interrogation.'' The Trumps derided
the lawsuit as a pressure tactic to get them to sign a consent decree
like the one agreed to by Mr. LeFrak.

The judge dismissed both the countersuit and the contempt-of-court
charge. After nearly two years of legal wrangling, the Trumps gave up
and signed a consent decree.

As is customary, it did not include an admission of guilt. But it did
include pages of stipulations intended to ensure the desegregation of
Trump properties.

Equal housing activists celebrated the agreement as more robust than the
one signed by Mr. LeFrak. It required that Trump Management provide the
New York Urban League with a weekly list of all its vacancies.

This did not stop Mr. Trump from declaring victory. ``In the end the
government couldn't prove its case, and we ended up making a minor
settlement without admitting any guilt,'' he wrote in ``The Art of the
Deal.''

Only this was not quite the end.

A few years later, the government accused the Trumps of violating the
consent decree. ``We believe that an underlying pattern of
discrimination continues to exist in the Trump Management
organization,'' a Justice Department lawyer wrote to Mr. Cohn in 1978.

Once again, the government marshaled numerous examples of blacks being
denied Trump apartments. But this time, it also identified a pattern of
racial steering.

While more black families were now renting in Trump-owned buildings, the
government said, many had been confined to a small number of complexes.
And tenants in some of these buildings had complained about the
conditions, from falling plaster to rusty light fixtures to bloodstained
floors.

The Trumps effectively wore the government down. The original consent
decree expired before the Justice Department had accumulated enough
evidence to press its new case.

The issue was becoming academic, anyway. New York's white working-class
population was shrinking. Shifting demographics would soon make it
impractical to turn away black tenants.

By the spring of 1982, when the case was officially closed, Donald
Trump's prized project, Trump Tower, was just months from completion.
The rebranding of the Trump name was well underway.

As for Ms. Brown, she still lives in the same apartment in the Wilshire.

Over the years, she has watched the building's complexion begin to
change --- along with some of her neighbors' attitudes toward her.
During the 1990s, one man who used to step off the elevator whenever she
stepped on suddenly started greeting her warmly.

On a recent afternoon, she reminisced about the unlikely role she played
in breaking the color barrier of the Trump real estate empire.

``I just wanted a decent place to live,'' she said.

Advertisement

\protect\hyperlink{after-bottom}{Continue reading the main story}

\hypertarget{site-index}{%
\subsection{Site Index}\label{site-index}}

\hypertarget{site-information-navigation}{%
\subsection{Site Information
Navigation}\label{site-information-navigation}}

\begin{itemize}
\tightlist
\item
  \href{https://help.nytimes.com/hc/en-us/articles/115014792127-Copyright-notice}{©~2020~The
  New York Times Company}
\end{itemize}

\begin{itemize}
\tightlist
\item
  \href{https://www.nytco.com/}{NYTCo}
\item
  \href{https://help.nytimes.com/hc/en-us/articles/115015385887-Contact-Us}{Contact
  Us}
\item
  \href{https://www.nytco.com/careers/}{Work with us}
\item
  \href{https://nytmediakit.com/}{Advertise}
\item
  \href{http://www.tbrandstudio.com/}{T Brand Studio}
\item
  \href{https://www.nytimes.com/privacy/cookie-policy\#how-do-i-manage-trackers}{Your
  Ad Choices}
\item
  \href{https://www.nytimes.com/privacy}{Privacy}
\item
  \href{https://help.nytimes.com/hc/en-us/articles/115014893428-Terms-of-service}{Terms
  of Service}
\item
  \href{https://help.nytimes.com/hc/en-us/articles/115014893968-Terms-of-sale}{Terms
  of Sale}
\item
  \href{https://spiderbites.nytimes.com}{Site Map}
\item
  \href{https://help.nytimes.com/hc/en-us}{Help}
\item
  \href{https://www.nytimes.com/subscription?campaignId=37WXW}{Subscriptions}
\end{itemize}
