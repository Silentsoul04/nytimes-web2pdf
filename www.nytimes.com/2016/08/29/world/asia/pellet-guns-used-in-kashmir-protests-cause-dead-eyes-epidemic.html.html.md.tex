Sections

SEARCH

\protect\hyperlink{site-content}{Skip to
content}\protect\hyperlink{site-index}{Skip to site index}

\href{https://www.nytimes.com/section/world/asia}{Asia Pacific}

\href{https://myaccount.nytimes.com/auth/login?response_type=cookie\&client_id=vi}{}

\href{https://www.nytimes.com/section/todayspaper}{Today's Paper}

\href{/section/world/asia}{Asia Pacific}\textbar{}An Epidemic of `Dead
Eyes' in Kashmir as India Uses Pellet Guns on Protesters

\url{https://nyti.ms/2c1iiMn}

\begin{itemize}
\item
\item
\item
\item
\item
\item
\end{itemize}

Advertisement

\protect\hyperlink{after-top}{Continue reading the main story}

Supported by

\protect\hyperlink{after-sponsor}{Continue reading the main story}

\hypertarget{an-epidemic-of-dead-eyes-in-kashmir-as-india-uses-pellet-guns-on-protesters}{%
\section{An Epidemic of `Dead Eyes' in Kashmir as India Uses Pellet Guns
on
Protesters}\label{an-epidemic-of-dead-eyes-in-kashmir-as-india-uses-pellet-guns-on-protesters}}

\includegraphics{https://static01.nyt.com/images/2016/08/29/world/29KASHMIR/29KASHMIR-videoSixteenByNine3000-v3.jpg}

By \href{https://www.nytimes.com/by/ellen-barry}{Ellen Barry}

\begin{itemize}
\item
  Aug. 28, 2016
\item
  \begin{itemize}
  \item
  \item
  \item
  \item
  \item
  \item
  \end{itemize}
\end{itemize}

SRINAGAR, Kashmir --- The street outside is patrolled by riot police
officers in camouflage, bracing for the nightly spasm of violence, but
it is quiet here inside the operating room. The surgeon's knife slides
into an eyeball as if it were a soft fruit.

The patient's eyelids have been stretched back with a metal clamp, so
his eyeball bulges out of glistening pink tissue. The surgeon sits with
his back very straight, cutting with tiny movements of his fingers.
Every now and then, a thread of blood appears in the patient's eye
socket. The patient is 8 years old.

``Very bad,'' murmurs the surgeon, Dr. S. Natarajan. But then, all 13
cases he will see today will be very bad.

Since mid-July, when the current wave of protests against the Indian
military presence started, more than 570 patients have reported to
Srinagar's main government hospital with eyes ruptured by lead pellets,
sometimes known as birdshot, fired by security forces armed with
pump-action shotguns to disperse crowds.

The patients have mutilated retinas, severed optic nerves, irises
seeping out like puddles of ink. ``Dead eyes,'' the ophthalmology
department's chief calls them.

Every season of popular revolt in Kashmir has its marker.

This summer's protests in the part of Kashmir controlled by India, the
most sustained and violent since 2010, caught the authorities in New
Delhi unaware. The stone-throwing crowds have no political leaders, put
forward no specific demands and metastasized with alarming speed. Around
60 civilians and two members of the security forces have been killed; on
each side, thousands have been wounded.

\includegraphics{https://static01.nyt.com/images/2016/08/29/world/29KASHMIR1/29KASHMIR1-articleInline.jpg?quality=75\&auto=webp\&disable=upscale}

But 2016 will almost certainly be remembered as the year of dead eyes.
The eye injuries have become such a focus of public anger that last
week, in a conciliatory gesture, India's home minister, Rajnath Singh,
promised that the pellet guns, as they are known here, would be replaced
by another type of nonlethal weapon in the coming days.

On the ophthalmology ward at the main
\href{http://www.gmcsrinagar.net/asstd_hosp/smhs.htm}{Shri Maharaja Hari
Singh Hospital}, however, new patients arrive every day. Walking the
hospital hallway, you first notice a handful of young men in blackout
goggles. Then you see them everywhere. A weary ophthalmologist looks on
from the break room as Dr. Natarajan's young patient, waking from
anesthesia, stirs and begins to moan.

``That 8-year-old boy, he will live for 70 or 80 years,'' says the
doctor, Afroz Khan. ``The history remains there, even if it is not in
the books.''

\hypertarget{retinal-repair}{%
\subsection{Retinal Repair}\label{retinal-repair}}

On July 9, Tariq Qureshi, the head of the ophthalmology department, was
at a seminar on pediatric retinal repair.

The previous day, Indian security forces raided a village and
\href{http://www.bbc.com/news/world-asia-india-36762043}{killed Burhan
Muzzafar Wani}, a 22-year-old militant leader whose videos posted on
WhatsApp and Facebook attracted a vast following. But major violence was
not expected. Dr. Qureshi was in the seminar when his phone rang.

It was the hospital emergency room, calling to let him know that two
patients had come in with pellets in their eyes. Dr. Qureshi sent a
doctor over, and the seminar resumed. Ten minutes later, the phone rang
again. It was the same doctor in the emergency room, telling Dr. Qureshi
to come immediately, that the number of patients had risen to 15.

Image

An Indian paramilitary trooper held a pellet gun as he stood guard on a
road during a curfew.Credit...Sajjad Hussain/Agence France-Presse ---
Getty Images

The four ophthalmologists, who were across the hospital campus from the
emergency room, ran.

For the next 72 hours, they operated in shifts around the clock,
suturing the eyes to keep the matter inside from leaking out. In most
cases, it became clear, the pellets had burst into through the cornea
and out through the retina, leaving little hope of fully restoring
vision. Twenty-seven patients were hit in both eyes. The pellets, when
they could be removed, were preserved on the heads of cotton swabs.

``Once it goes in the eye, it rotates like this, and destroys everything
there inside,'' Dr. Qureshi said. ``It's physics. This is a
high-velocity body. It releases a high amount of energy inside. The
lens, the iris, the retina get matted up.''

The doctors were told to take all possible measures to save their
patients' vision, including complex surgery, at a cost to the government
of 70,000 rupees, or around \$1,040, per operation, Dr. Qureshi said.

The worst cases go to Dr. Natarajan, the director of
\href{http://adityajyoteyehospital.org/}{Aditya Jyot Eye Hospital} in
Mumbai, whose visits are facilitated by the
\href{http://borderlessworldfoundation.org/about-us/}{Borderless World
Foundation}, a nonprofit group. Dr. Natarajan specializes in patients
whose eyes have been punctured by projectiles --- typically, children
standing too near fireworks, or industrial workers who did not wear
protective goggles, or boxers whose eyes have been punctured by thumbs.

He works in a bubble of calm, eyes pressed to a microscope, using his
hands to work a cutter and a light, and using his bare feet to control
the machines that surround him. On a screen opposite him is an image
captured by a microscopic camera inside the boy's eye. At times the
image is cloudy, a flashlight searching in the fog; at one point there
are swimming glints of colored light, like those cast by a chandelier in
the sun.

In cases of catastrophic injuries, Dr. Natarajan's goal is to save a
small portion of the eye's function, enough to sense light, or movement
of a hand.

Image

A father, who said his son had been injured by pellets shot by security
forces, comforted his son earlier this month.Credit...Cathal
McNaughton/Reuters

``Even that minor change from zero matters a lot, for a man with no
light,'' Dr. Natarajan said. ``It is like, if you have no money in your
pocket, 10 rupees seems like big money.''

Slowly, as residents stood around him in hushed silence, the surgeon
flattened out the boy's retina, as thin and delicate as a lace doily,
and used a laser to reattach it to the back of his eye.

\hypertarget{boys-hurling-stones}{%
\subsection{Boys Hurling Stones}\label{boys-hurling-stones}}

For an Indian security official, to be engulfed by a hostile crowd in
Kashmir is, without a doubt, a life-threatening situation.

At sunset on Friday, Bhavesh Chaudhary, the second-in-command of the
161st Battalion of the Central Reserve Police Force, was drinking tea in
the camp garden when an officer called with the news that 20 or 30 young
men had begun to gather, chanting slogans. He continued drinking tea.
The crowd outside kept growing.

Then, all of a sudden, Commandant Chaudhary and his troops strapped on
helmets and leapt into a column of armored vehicles. As they raced
through the neighborhood, masked boys appeared from the left and the
right, darting out of alleyways, hurling stones. The troops sent stones
rocketing back with small slingshots. The convoy halted at an
intersection. Chanting could be heard, coming closer: ``What do we want?
Freedom!''

Commandant Chaudhary would spend the next hour and a half trying to push
the crowd back. His troops may be heavily armed, but especially at
sunset, when they withdraw to their encampments for the night, it is
clear to everyone that they are outnumbered.

On the streets of Srinagar, which have a ghostly emptiness after 50 days
of curfew, people have scrawled, ``Indian dogs,'' ``Go India, go back,''
``We love Pakistan'' and ``Burhan is alive in our hearts.''

Commandant Chaudhary has dedicated much of his career to battling
stone-throwing crowds. He knows the current of excitement that will
surge through them if they see his forces retreat even a few feet ---
or, more powerfully, if they see an officer fall. If the stone-throwers
managed to reach the camp, he said, they would set it on fire.

``They are not afraid, that is the thing,'' he said of the protesters.
``Once somebody has put on a uniform and picked up a weapon, the law
should be maintained, just because the person is there. That is not
happening these days. We lost that in 2010.''

Indian troops use pellet guns for crowd control only in Kashmir. They
were introduced in 2010, halfway through a particularly bloody season of
protest. Pellet guns have been used to break up protests in Egypt,
Bahrain and Tunisia, but most countries do not use them on unarmed
civilians, as the pellets spray widely and cannot be aimed. For
Commandant Chaudhary, who sometimes faces crowds of more than 1,000
hostile young men with a contingent of 20 or 30, it is by far the most
effective weapon at his disposal.

``It causes bodily injury, so you will be feared,'' he said.

His battalion commander, Rajesh Yadav, nodded at this assessment. ``If
you pinch them,'' he said, ``only then people will understand.''

This year, the use of pellets on Kashmiri protesters increased sharply,
with the police firing more than 3,000 canisters, or upward of 1.2
million pellets, in the first 32 days of the protests, the Central
Reserve Police Force has said.

Though troops are instructed to aim them below the waist, ``sometimes it
is difficult to go in for precise aimed fire at a moving, bending and
running target,'' the police explained in response to a lawsuit seeking
to ban their use. If they are withdrawn from the arsenal, Commander
Yadav said matter-of-factly, troops will have to use their firearms.

As for the government hospital, now jammed with injured protesters and
sympathetic volunteers, Commander Yadav said it was no longer a safe
place for his officers to go. Not long ago, one of his men sought
medical help for chest pain but fled in fear of being lynched.

\hypertarget{8-year-olds-prognosis}{%
\subsection{8-Year-Old's Prognosis}\label{8-year-olds-prognosis}}

In a recovery ward at Shri Maharaja Hari Singh Hospital, a nurse pushes
a trolley down a row of beds, distributing cups of tea and slices of
white bread to a row of young men in sunglasses.

To converse with them is to see new energy coursing into Kashmir's old
cycle of violence. It is difficult to find a patient here who admits to
mourning the loss of his eye. They say it is an acceptable price to pay
for azadi, or freedom from Indian rule. Quite a few offer to sacrifice
their second eye for the cause.

Wazira Banwo, 40, is watching her 8-year-old son, Asif Sheikh, recover
from surgery. The boy is curled on his side under a blanket, his head
swathed in surgical gauze, woozy and sick. It was his third operation;
now, with his retina reattached, he may be able to see for a distance of
three to five feet, according to Dr. Natarajan.

Asked whether she was grateful to the government for providing the child
medical care, Ms. Banwo grimaces.

``Not a single person from the government has come to help,'' she says.
``If any one of them come to me, I will tell them, `You give me your
eyes, I will put them in my child.'''

Ms. Banwo says she often participated in anti-Indian protests herself
but discouraged Asif from taking part this summer because of his youth.

On the day he was injured, she says, he just happened to be standing in
the market when security forces arrived in a van and fired pellet guns.

``This time he is very young,'' she says. ``But he will grow. He will
understand what happened to him. And he will go out to the street and
throw stones.''

Advertisement

\protect\hyperlink{after-bottom}{Continue reading the main story}

\hypertarget{site-index}{%
\subsection{Site Index}\label{site-index}}

\hypertarget{site-information-navigation}{%
\subsection{Site Information
Navigation}\label{site-information-navigation}}

\begin{itemize}
\tightlist
\item
  \href{https://help.nytimes.com/hc/en-us/articles/115014792127-Copyright-notice}{©~2020~The
  New York Times Company}
\end{itemize}

\begin{itemize}
\tightlist
\item
  \href{https://www.nytco.com/}{NYTCo}
\item
  \href{https://help.nytimes.com/hc/en-us/articles/115015385887-Contact-Us}{Contact
  Us}
\item
  \href{https://www.nytco.com/careers/}{Work with us}
\item
  \href{https://nytmediakit.com/}{Advertise}
\item
  \href{http://www.tbrandstudio.com/}{T Brand Studio}
\item
  \href{https://www.nytimes.com/privacy/cookie-policy\#how-do-i-manage-trackers}{Your
  Ad Choices}
\item
  \href{https://www.nytimes.com/privacy}{Privacy}
\item
  \href{https://help.nytimes.com/hc/en-us/articles/115014893428-Terms-of-service}{Terms
  of Service}
\item
  \href{https://help.nytimes.com/hc/en-us/articles/115014893968-Terms-of-sale}{Terms
  of Sale}
\item
  \href{https://spiderbites.nytimes.com}{Site Map}
\item
  \href{https://help.nytimes.com/hc/en-us}{Help}
\item
  \href{https://www.nytimes.com/subscription?campaignId=37WXW}{Subscriptions}
\end{itemize}
