Sections

SEARCH

\protect\hyperlink{site-content}{Skip to
content}\protect\hyperlink{site-index}{Skip to site index}

\href{https://www.nytimes.com/section/world/asia}{Asia Pacific}

\href{https://myaccount.nytimes.com/auth/login?response_type=cookie\&client_id=vi}{}

\href{https://www.nytimes.com/section/todayspaper}{Today's Paper}

\href{/section/world/asia}{Asia Pacific}\textbar{}North Korea Has
Executed a Deputy Premier, Seoul Reports

\url{https://nyti.ms/2c7m4E3}

\begin{itemize}
\item
\item
\item
\item
\item
\end{itemize}

Advertisement

\protect\hyperlink{after-top}{Continue reading the main story}

Supported by

\protect\hyperlink{after-sponsor}{Continue reading the main story}

\hypertarget{north-korea-has-executed-a-deputy-premier-seoul-reports}{%
\section{North Korea Has Executed a Deputy Premier, Seoul
Reports}\label{north-korea-has-executed-a-deputy-premier-seoul-reports}}

\includegraphics{https://static01.nyt.com/images/2016/09/01/world/01nkorea-web2/01nkorea-web2-articleInline.jpg?quality=75\&auto=webp\&disable=upscale}

By \href{http://www.nytimes.com/by/choe-sang-hun}{Choe Sang-Hun}

\begin{itemize}
\item
  Aug. 31, 2016
\item
  \begin{itemize}
  \item
  \item
  \item
  \item
  \item
  \end{itemize}
\end{itemize}

SEOUL, South Korea --- The North Korean leader,
\href{http://topics.nytimes.com/top/reference/timestopics/people/k/kim_jongun/index.html?inline=nyt-per}{Kim
Jong-un}, has executed his deputy premier for education and purged two
other senior officials, sending them to re-education camps, the South
Korean government said on Wednesday.

Jeong Joon-hee, a spokesman for the South's Unification Ministry, said
at a news briefing that the South Korean government had used various
means to confirm the execution of Kim Yong-jin, the deputy premier, and
the purge of Kim Yong-chol, the head of the United Front Department of
the ruling Workers' Party, which handles relations with, as well as
spying operations against, South Korea. Choe Hui, a deputy chief of the
party's Propaganda and Agitation Department, was also banished for
re-education, Mr. Jeong said.

Mr. Jeong provided no further details, including when the reported
punishments were believed to have taken place or how South Korea had
learned of them. But in a later briefing, a senior Unification Ministry
official, who spoke on the condition of anonymity to discuss
intelligence matters, said that Kim Jong-un had found fault with the
63-year-old deputy premier's ``disrespectful posture'' during a meeting
that Mr. Kim oversaw in late June.

A subsequent investigation found the deputy premier to be an
``anti-party reactionary'' guilty of ``modern-day factionalism,'' and he
was executed by firing squad in July, the official said.

\includegraphics{https://static01.nyt.com/images/2016/09/01/world/01nkorea-web1/01nkorea-web1-videoSixteenByNine3000.jpg}

Kim Yong-jin would be the highest-ranking official known to have been
executed since 2013, when North Korea confirmed in a rare announcement
that Kim Jong-un
\href{http://www.nytimes.com/2013/12/13/world/asia/north-korea-says-uncle-of-executed.html}{had
executed} his own uncle and No. 2 official, Jang Song-thaek, on charges
of factionalism, corruption and plotting to overthrow his government.

The ministry official who spoke on the condition of anonymity said that
Kim Yong-chol, the leader of the United Front Department, had spent a
month at a re-education camp on suspicion of abuse of power and that he
had been released in mid-August.

Kim Yong-chol is seen as a hard-liner by South Korean officials. He was
accused of helping orchestrate recent armed provocations by the North
along the inter-Korean border, including an artillery barrage against a
South Korean island in 2010, when he was the army's intelligence chief.
The Unification Ministry official said Mr. Kim would now need to prove
his loyalty, which the official said raised the possibility that the
North could take more aggressive actions toward South Korea.

Since taking power in 2011, Kim Jong-un has frequently reshuffled the
party and military elites as he has consolidated his authority in North
Korea, which his family has ruled for seven decades. Mr. Kim has also
executed dozens of top officials in what President Park Geun-hye of
South Korea has called a ``reign of terror,'' according to South Korean
intelligence officials.

It remains difficult to independently verify
\href{http://thelede.blogs.nytimes.com/2014/01/03/inside-the-tale-of-north-korea-execution-by-ravenous-dog/}{reports
of executions and purges} in the secretive North. North Korea rarely
announces them.

It was unusual for a South Korean government spokesman to make them
public in an open news briefing, though intelligence officials have
often briefed lawmakers in closed-door parliamentary sessions. In one
such briefing last year, lawmakers were told that Gen. Hyon Yong-chol,
the defense minister, had been executed with an antiaircraft gun in
Pyongyang, the North's capital, after dozing off during military events
and second-guessing Mr. Kim's orders.

Mr. Jeong, the government spokesman, said that he was responding to
recent reports in the South Korean news media. On Tuesday, the
mass-circulation daily JoongAng Ilbo, citing an anonymous source,
reported that Hwang Min, a former North Korean agriculture minister, and
Ri Yong-jin, a senior Education Ministry official,
\href{http://www.bloomberg.com/news/articles/2016-08-30/kim-has-two-officials-killed-by-anti-aircraft-gun-joongang-says}{had
been executed with antiaircraft guns} in early August. The newspaper
reported that Mr. Ri had been arrested after dozing off during a meeting
supervised by Mr. Kim and that Mr. Hwang had proposed a policy that was
deemed to represent a challenge to Mr. Kim's leadership.

Mr. Jeong did not comment on the fates of those two officials in his
briefing on Wednesday.

JoongAng Ilbo reported that the officials' reported executions might
have been aimed at tightening Mr. Kim's control after a senior North
Korean diplomat's
\href{http://www.nytimes.com/2016/08/18/world/asia/north-korea-defector-thae-yong-ho-britain.html}{recent
defection} to the South.

South Korean officials often cite such high-level defections, and purges
like those announced Wednesday, as potential sources of instability in
Mr. Kim's totalitarian regime. But some analysts dispute such
conclusions.

Purges and executions remain a key feature of political life in the
North, said Cheong Seong-chang, a senior analyst at the Sejong
Institute, a South Korean research organization. But he said such
persecutions, while barbaric, had become less frequent and ``relatively
restrained'' under Kim Jong-un. Mr. Kim's father, Kim Jong-il, was
estimated to have purged more than 2,000 officials from 1994 to 2000, he
said.

Advertisement

\protect\hyperlink{after-bottom}{Continue reading the main story}

\hypertarget{site-index}{%
\subsection{Site Index}\label{site-index}}

\hypertarget{site-information-navigation}{%
\subsection{Site Information
Navigation}\label{site-information-navigation}}

\begin{itemize}
\tightlist
\item
  \href{https://help.nytimes.com/hc/en-us/articles/115014792127-Copyright-notice}{©~2020~The
  New York Times Company}
\end{itemize}

\begin{itemize}
\tightlist
\item
  \href{https://www.nytco.com/}{NYTCo}
\item
  \href{https://help.nytimes.com/hc/en-us/articles/115015385887-Contact-Us}{Contact
  Us}
\item
  \href{https://www.nytco.com/careers/}{Work with us}
\item
  \href{https://nytmediakit.com/}{Advertise}
\item
  \href{http://www.tbrandstudio.com/}{T Brand Studio}
\item
  \href{https://www.nytimes.com/privacy/cookie-policy\#how-do-i-manage-trackers}{Your
  Ad Choices}
\item
  \href{https://www.nytimes.com/privacy}{Privacy}
\item
  \href{https://help.nytimes.com/hc/en-us/articles/115014893428-Terms-of-service}{Terms
  of Service}
\item
  \href{https://help.nytimes.com/hc/en-us/articles/115014893968-Terms-of-sale}{Terms
  of Sale}
\item
  \href{https://spiderbites.nytimes.com}{Site Map}
\item
  \href{https://help.nytimes.com/hc/en-us}{Help}
\item
  \href{https://www.nytimes.com/subscription?campaignId=37WXW}{Subscriptions}
\end{itemize}
