Sections

SEARCH

\protect\hyperlink{site-content}{Skip to
content}\protect\hyperlink{site-index}{Skip to site index}

\href{https://www.nytimes.com/section/world/asia}{Asia Pacific}

\href{https://myaccount.nytimes.com/auth/login?response_type=cookie\&client_id=vi}{}

\href{https://www.nytimes.com/section/todayspaper}{Today's Paper}

\href{/section/world/asia}{Asia Pacific}\textbar{}Behind China's Anbang:
Empty Offices and Obscure Names

\url{https://nyti.ms/2cdak2Q}

\begin{itemize}
\item
\item
\item
\item
\item
\item
\end{itemize}

Advertisement

\protect\hyperlink{after-top}{Continue reading the main story}

Supported by

\protect\hyperlink{after-sponsor}{Continue reading the main story}

\href{/column/sinosphere}{Sinosphere}

\hypertarget{behind-chinas-anbang-empty-offices-and-obscure-names}{%
\section{Behind China's Anbang: Empty Offices and Obscure
Names}\label{behind-chinas-anbang-empty-offices-and-obscure-names}}

\includegraphics{https://static01.nyt.com/images/2016/07/14/world/15ANBANG-SINOSPHERE-web1/15ANBANG-SINOSPHERE-web1-articleInline.jpg?quality=75\&auto=webp\&disable=upscale}

By \href{http://www.nytimes.com/by/michael-forsythe}{Michael Forsythe}

\begin{itemize}
\item
  Sept. 1, 2016
\item
  \begin{itemize}
  \item
  \item
  \item
  \item
  \item
  \item
  \end{itemize}
\end{itemize}

In the center of Beijing's booming commercial district, with soaring
office towers, gleaming shopping malls and luxury apartment complexes,
sits a shabby, four-story building with an office that houses stupendous
wealth.

To get there, visitors must first pass through an unmarked entrance next
to a gray-tiled post office, then pass through a maze of carelessly
strung phone and power lines. On the fourth floor, down a hallway where
black scuff marks on the walls indicate years of neglect, is the
Hujialou Concentrated Office Zone, an officially sanctioned domicile for
shell companies.

According to government records, that office is home to two companies
with a total stake that accounts for more than \$15 billion in assets of
one of China's biggest financial conglomerates: the Anbang Insurance
Group.

Prying visitors are not welcome. A woman at the office, who did not give
her name, grew angry when a journalist told her that two companies there
controlled a large share of Anbang.

``We're just a private company, helping others register with the State
Administration for Industry and Commerce,'' she said, referring to the
keepers of China's corporate database. ``It is none of our business
whether they are shell companies.''

Another Anbang shareholding firm --- one that controls \$5.6 billion in
assets --- lists its address a few blocks away from the Hujialou
building, at an office tower's empty 27th floor.

Near the Temple of Heaven, the former imperial religious complex in
Beijing, the listed address of yet another shareholding company turns
out to be another empty office. The directory downstairs lists the
former tenant: a shoe seller.

A group of 39 companies control Anbang, which was once a sleepy
insurance company and now has \$295 billion in assets and a reputation
as an ambitious global deal maker. Many of those companies are in turn
owned by a welter of shell companies, many with similar names and
addresses or common owners.

Ultimately, as
\href{http://www.nytimes.com/2016/09/02/business/dealbook/anbang-global-shopping-spree-china-mystery-ownership.html}{The
New York Times reports}, they are controlled by about 100 people, many
of whom hail from a county called Pingyang on China's east coast.
Pingyang County is the home county of Anbang's chairman, Wu Xiaohui.

In any major country, the shareholders of marquee companies are often
household names themselves. General Electric counts major institutional
investors such as the Vanguard Group and BlackRock as top owners.
Berkshire Hathaway's biggest shareholder is its chairman, Warren E.
Buffett.

China is no exception. Dalian Wanda Commercial Properties, for example,
led by China's richest man, Wang Jianlin, may match Anbang in the
breadth of its
\href{http://www.nytimes.com/2015/04/29/world/asia/wang-jianlin-abillionaire-at-the-intersection-of-business-and-power-in-china.html}{political
connections}, but many of its biggest shareholders are easily
recognizable Chinese companies like China Life Insurance.

Anbang is different. The companies that own it, and the people who back
them, are almost all obscure. Two state-owned companies that
collectively own less than 2 percent of Anbang are the only exceptions.
I first started writing about Chinese companies 16 years ago and have
never seen a similar ownership structure at a major company.

That is remarkable, given Anbang's increasing size and prominence.
Earlier this year, Anbang
\href{http://www.nytimes.com/2016/04/01/business/dealbook/starwood-hotels-chinese-suitor-backs-out-of-bidding.html}{came
close} to executing what would have been the biggest takeover ever of an
American company by a Chinese company, offering more than \$14 billion
for Starwood Hotels \& Resorts.

In China, the company not only sells auto and life insurance, but also
controls a major bank in southwestern China, is the
\href{http://dealbook.nytimes.com/2015/01/31/president-of-china-minsheng-bank-steps-down/?_r=0}{largest
shareholder} of one of the country's biggest financial conglomerates,
China Minsheng Banking Corporation, and is the
\href{http://www.hkexnews.hk/listedco/listconews/SEHK/2016/0428/LTN201604281519.pdf}{second-largest}
shareholder in another, China Merchants Bank.

But China is not an offshore haven like the Cayman Islands or the
British Virgin Islands. The country's online corporate records system
allows those with patience to find the names behind the holding
companies, even if --- as with Anbang --- the corporate shareholders
frequently change names, addresses and owners.

After more than three months of combing through thousands of pages of
records, The Times was able to piece together a corporate history for
those 39 shareholders. One clear pattern emerged. At least 35 of the
companies, collectively owning more than 92 percent of Anbang, can trace
all or part of their ownership to relatives of Mr. Wu or to his wife,
Zhuo Ran, who is the granddaughter of the former Chinese leader Deng
Xiaoping; or to Chen Xiaolu, the son of one of China's most famous
marshals, who helped Mao's Communists to victory in 1949. Those
relatives are either current or former owners or directors of those
companies, or current or former owners of predecessor firms.

One example is Lin Cong, who owns a small share of the company. Mr. Lin
is Mr. Wu's cousin, the nephew of his mother, Lin Xiangmei, relatives
say.

Back at the Hujialou Concentrated Office Zone, one of the Anbang
shareholders there, a company called Beijing Bibo Investment Management
Company, controls \$10.9 billion in Anbang assets. Until Dec. 1, 2014,
that stake was ultimately owned by a woman named Wu Xiaoxia, corporate
records show. Relatives in Pingyang County say she is very close to Mr.
Wu.

She is his sister.

Advertisement

\protect\hyperlink{after-bottom}{Continue reading the main story}

\hypertarget{site-index}{%
\subsection{Site Index}\label{site-index}}

\hypertarget{site-information-navigation}{%
\subsection{Site Information
Navigation}\label{site-information-navigation}}

\begin{itemize}
\tightlist
\item
  \href{https://help.nytimes.com/hc/en-us/articles/115014792127-Copyright-notice}{©~2020~The
  New York Times Company}
\end{itemize}

\begin{itemize}
\tightlist
\item
  \href{https://www.nytco.com/}{NYTCo}
\item
  \href{https://help.nytimes.com/hc/en-us/articles/115015385887-Contact-Us}{Contact
  Us}
\item
  \href{https://www.nytco.com/careers/}{Work with us}
\item
  \href{https://nytmediakit.com/}{Advertise}
\item
  \href{http://www.tbrandstudio.com/}{T Brand Studio}
\item
  \href{https://www.nytimes.com/privacy/cookie-policy\#how-do-i-manage-trackers}{Your
  Ad Choices}
\item
  \href{https://www.nytimes.com/privacy}{Privacy}
\item
  \href{https://help.nytimes.com/hc/en-us/articles/115014893428-Terms-of-service}{Terms
  of Service}
\item
  \href{https://help.nytimes.com/hc/en-us/articles/115014893968-Terms-of-sale}{Terms
  of Sale}
\item
  \href{https://spiderbites.nytimes.com}{Site Map}
\item
  \href{https://help.nytimes.com/hc/en-us}{Help}
\item
  \href{https://www.nytimes.com/subscription?campaignId=37WXW}{Subscriptions}
\end{itemize}
