Sections

SEARCH

\protect\hyperlink{site-content}{Skip to
content}\protect\hyperlink{site-index}{Skip to site index}

\href{https://www.nytimes.com/section/politics}{Politics}

\href{https://myaccount.nytimes.com/auth/login?response_type=cookie\&client_id=vi}{}

\href{https://www.nytimes.com/section/todayspaper}{Today's Paper}

\href{/section/politics}{Politics}\textbar{}Donald Trump Vows to Bolster
Nation's Military Capacities

\url{https://nyti.ms/2ckxkjA}

\begin{itemize}
\item
\item
\item
\item
\item
\end{itemize}

Advertisement

\protect\hyperlink{after-top}{Continue reading the main story}

Supported by

\protect\hyperlink{after-sponsor}{Continue reading the main story}

\hypertarget{donald-trump-vows-to-bolster-nations-military-capacities}{%
\section{Donald Trump Vows to Bolster Nation's Military
Capacities}\label{donald-trump-vows-to-bolster-nations-military-capacities}}

\includegraphics{https://static01.nyt.com/images/2016/09/08/us/08trump-speak/08trump-speak-videoSixteenByNineJumbo1600.jpg}

By \href{http://www.nytimes.com/by/ashley-parker}{Ashley Parker} and
\href{http://www.nytimes.com/by/matthew-rosenberg}{Matthew Rosenberg}

\begin{itemize}
\item
  Sept. 7, 2016
\item
  \begin{itemize}
  \item
  \item
  \item
  \item
  \item
  \end{itemize}
\end{itemize}

PHILADELPHIA --- Donald J. Trump on Wednesday called for a vast
expansion of the military, including 90,000 new soldiers for the Army
and nearly 75 new ships for the Navy, requiring up to \$90 billion a
year in additional spending.

But Mr. Trump did not match those numbers with details about how the
country would raise the money, other than a promise to take steps like
reducing wasteful spending, which military budget analysts said would be
insufficient.

Mr. Trump, in a speech at the Union League of Philadelphia, also vowed
to order the military to devise a new plan to defeat the Islamic State
``immediately upon taking office.'' The plan would come within 30 days
from ``my generals,'' he added, without mentioning that those generals
are the same ones who came up with the current strategy, which they
believe is working.

The speech was the latest effort by Mr. Trump's campaign to demonstrate
to voters that he can lay out detailed policy prescriptions to problems
confronting the nation. It also seemed to be directed at the
conservative foreign policy establishment, coming a day after Mr. Trump
\href{http://www.nytimes.com/2016/09/07/us/politics/donald-trump-earns-backing-of-nearly-90-military-figures.html}{released
a letter} from about 90 retired military officials endorsing his
campaign.

He lamented the shrinking of the military and warned that enemies were
preparing to capitalize on perceptions of American weakness around the
world. ``Our adversaries are chomping at the bit,'' Mr. Trump said. ``We
want to deter, avoid and prevent conflict through our unquestioned
military strength.''

With the speech, Mr. Trump also moved to refocus his campaign on
critiques of Hillary Clinton, his Democratic rival. Espousing a foreign
policy ``tempered by realism,'' Mr. Trump portrayed Mrs. Clinton, the
former secretary of state, as unsuited to lead the nation's armed
forces.

\href{https://www.nytimes.com/interactive/2016/09/07/us/politics/trump-clinton-forum.html}{}

\includegraphics{https://static01.nyt.com/images/2016/09/08/us/politics/08live-livechattop/08live-livechattop-articleLarge.jpg}

\hypertarget{hillary-clinton-and-donald-trump-at-the-presidential-forum-analysis}{%
\subsection{Hillary Clinton and Donald Trump at the Presidential Forum:
Analysis}\label{hillary-clinton-and-donald-trump-at-the-presidential-forum-analysis}}

It wasn't the first debate, but at least Ms. Clinton and Mr. Trump were
on the same stage on Wednesday, though not at the same time. Here's how
we analyzed tonight's forum live.

``Unlike my opponent, my foreign policy will emphasize diplomacy, not
destruction,'' Mr. Trump said. ``Hillary Clinton's legacy in Iraq,
Libya, Syria has produced only turmoil and suffering and death.''

But it was the size of the military, and the amount that the United
States spends on its defense, that lay at the heart of Mr. Trump's
speech. In addition to increasing the Army to 540,000 soldiers and
adding the Navy ships, Mr. Trump proposed buying dozens of new fighter
aircraft for the Air Force.

To pay for the expansion, Mr. Trump said he would call on Congress to
reverse the cuts to military spending made as part of the
\href{http://www.nytimes.com/2013/02/22/us/politics/questions-and-answers-about-the-sequester.html}{budget
sequester} in 2013, which was the result of a compromise reached between
Democrats and Republicans.

The new spending, Mr. Trump said, would not cost taxpayers an additional
penny. He said he would eliminate wasteful government spending, increase
energy production and trim the federal work force, including the
military bureaucracy. He also suggested that he would collect unpaid
taxes, which he said amounted to \$385 billion.

Asked about the plan, Todd Harrison, a military budget expert with the
Center for Strategic and International Studies, said, ``Good luck.''

``Everyone comes in saying they want to reduce wasteful spending,'' Mr.
Harrison said. ``Folks have tried that again and again, and they have
largely not been successful.''

Although Mr. Trump said little about how much his plan would cost, the
new military spending would probably amount to \$80 billion to \$90
billion a year, experts said. The additional soldiers alone would cost
around \$9 billion.

\href{https://www.nytimes.com/interactive/2016/09/07/us/politics/trump-products-reaction.html}{}

\includegraphics{https://static01.nyt.com/images/2016/09/07/us/07interactive-trump2/07interactive-trump2-articleLarge.jpg}

\hypertarget{have-you-done-something-to-support-or-oppose-trumps-brands}{%
\subsection{Have You Done Something to Support or Oppose Trump's
Brands?}\label{have-you-done-something-to-support-or-oppose-trumps-brands}}

Have you registered your disapproval or approval for Donald J. Trump's
candidacy by, say, discarding or returning Trump-brand clothes --- or by
booking a night at a Trump hotel? We want to hear your stories.

Mr. Trump's call for ending the sequester on military spending is
unlikely to gain traction in Washington. Republicans have long pushed
for lifting these limits --- a proposal Democrats will consider only
with comparable relief on domestic spending.

And on Tuesday night, in a sign of the likely stalemate, Senate
Democrats filibustered a military appropriations bill because it would
have allowed for bursting through the caps on military spending without
also doing the same for domestic spending.

Mr. Trump's remarks came before he was to appear Wednesday night at a
\href{http://www.nytimes.com/2016/09/08/us/politics/hillary-clinton-donald-trump-national-security.html?ref=politics}{Commander
in Chief Forum} televised from the Intrepid Sea, Air and Space Museum in
New York and focused on security and veterans issues. Mr. Trump was
questioned after Mrs. Clinton.

Earlier on Wednesday, he described Mrs. Clinton as ``reckless'' and
``totally unfit to be our commander in chief.''

Mr. Trump was turning the tables on a frequent line of attack from Mrs.
Clinton and trying to lay the blame for the bloodshed in the Middle East
at her feet, a critique that most experts have said is grossly
oversimplified and misleading.

``Sometimes it seemed like there wasn't a country in the Middle East
that Hillary Clinton didn't want to invade, intervene in or topple,'' he
said. ``She's trigger-happy and very unstable.''

The Clinton campaign fought back after the speech by highlighting the
endorsement of more retired generals and admirals, saying Mrs. Clinton
had gotten greater support ``than any non-incumbent Democrat due to her
proven record of diplomacy and steady leadership on the world stage.''

In
\href{http://www.nytimes.com/2016/04/28/us/politics/donald-trump-foreign-policy-speech.html?_r=0}{a
foreign policy speech} in April, Mr. Trump offered much the same thrust
as in Wednesday's address --- presenting an at times paradoxical
approach of using fiery oratory to promise a military buildup and the
immediate destruction of the Islamic State, while also rejecting the
nation-building and interventionist instincts of George W. Bush's
administration.

Mr. Trump also echoed other themes that he has used during his campaign,
calling on allies to pay more for American military protection.

``Early in my term, I will also be requesting that all NATO nations
promptly pay their bills,'' he said. ``Only five NATO countries,
including the United States, are currently meeting their minimum
requirements to spend 2 percent of G.D.P. on defense.''

He also accused the Obama administration of agreeing to bad deals with
Iran.

``Our president lied to us,'' Mr. Trump said of President Obama, saying
the nuclear deal with Iran put the country ``on a path to nuclear
weapons.''

But Mr. Trump's fiercest criticism was saved for Mrs. Clinton. He
accused her of being complicit in an array of foreign policy stumbles,
and of deleting her emails as secretary of state to hide her
participation in a ``pay for play'' scandal in which Clinton Foundation
donors were granted special access.

Damning his rival with false praise, Mr. Trump --- in one of his
speech's biggest applause lines --- also said that maybe Mrs. Clinton
did have some wisdom to impart.

``Hillary Clinton has taught us really how vulnerable we are in
cyberhacking,'' he said. ``It's probably the only thing that we've
learned from Hillary Clinton.''

Advertisement

\protect\hyperlink{after-bottom}{Continue reading the main story}

\hypertarget{site-index}{%
\subsection{Site Index}\label{site-index}}

\hypertarget{site-information-navigation}{%
\subsection{Site Information
Navigation}\label{site-information-navigation}}

\begin{itemize}
\tightlist
\item
  \href{https://help.nytimes.com/hc/en-us/articles/115014792127-Copyright-notice}{©~2020~The
  New York Times Company}
\end{itemize}

\begin{itemize}
\tightlist
\item
  \href{https://www.nytco.com/}{NYTCo}
\item
  \href{https://help.nytimes.com/hc/en-us/articles/115015385887-Contact-Us}{Contact
  Us}
\item
  \href{https://www.nytco.com/careers/}{Work with us}
\item
  \href{https://nytmediakit.com/}{Advertise}
\item
  \href{http://www.tbrandstudio.com/}{T Brand Studio}
\item
  \href{https://www.nytimes.com/privacy/cookie-policy\#how-do-i-manage-trackers}{Your
  Ad Choices}
\item
  \href{https://www.nytimes.com/privacy}{Privacy}
\item
  \href{https://help.nytimes.com/hc/en-us/articles/115014893428-Terms-of-service}{Terms
  of Service}
\item
  \href{https://help.nytimes.com/hc/en-us/articles/115014893968-Terms-of-sale}{Terms
  of Sale}
\item
  \href{https://spiderbites.nytimes.com}{Site Map}
\item
  \href{https://help.nytimes.com/hc/en-us}{Help}
\item
  \href{https://www.nytimes.com/subscription?campaignId=37WXW}{Subscriptions}
\end{itemize}
