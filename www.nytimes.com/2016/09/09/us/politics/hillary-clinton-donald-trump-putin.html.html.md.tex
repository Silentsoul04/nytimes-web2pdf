Sections

SEARCH

\protect\hyperlink{site-content}{Skip to
content}\protect\hyperlink{site-index}{Skip to site index}

\href{https://www.nytimes.com/section/politics}{Politics}

\href{https://myaccount.nytimes.com/auth/login?response_type=cookie\&client_id=vi}{}

\href{https://www.nytimes.com/section/todayspaper}{Today's Paper}

\href{/section/politics}{Politics}\textbar{}Donald Trump's Campaign
Stands By Embrace of Putin

\url{https://nyti.ms/2cwtbpK}

\begin{itemize}
\item
\item
\item
\item
\item
\item
\end{itemize}

Advertisement

\protect\hyperlink{after-top}{Continue reading the main story}

Supported by

\protect\hyperlink{after-sponsor}{Continue reading the main story}

\hypertarget{donald-trumps-campaign-stands-by-embrace-of-putin}{%
\section{Donald Trump's Campaign Stands By Embrace of
Putin}\label{donald-trumps-campaign-stands-by-embrace-of-putin}}

\includegraphics{https://static01.nyt.com/images/2016/09/08/multimedia/09clinton-charlotte/09clinton-charlotte-videoSixteenByNine3000.jpg}

By \href{http://www.nytimes.com/by/jonathan-martin}{Jonathan Martin} and
\href{http://www.nytimes.com/by/amy-chozick}{Amy Chozick}

\begin{itemize}
\item
  Sept. 8, 2016
\item
  \begin{itemize}
  \item
  \item
  \item
  \item
  \item
  \item
  \end{itemize}
\end{itemize}

WASHINGTON --- Donald J. Trump's campaign on Thursday reaffirmed its
extraordinary embrace of Russia's president, Vladimir V. Putin,
signaling a preference for the leadership of an authoritarian adversary
over that of America's own president, despite a cascade of criticism
from Democrats and expressions of discomfort among Republicans.

``I think it's inarguable that Vladimir Putin has been a stronger leader
in his country than Barack Obama has been in this country,'' Gov. Mike
Pence of Indiana, Mr. Trump's running mate, said on CNN, defending Mr.
Trump by echoing his latest praise for the Russian leader, offered
Wednesday night in a televised candidate forum.

Hillary Clinton excoriated Mr. Trump for asserting that Mr. Putin is a
better leader than President Obama, saying it was ``not just unpatriotic
and insulting to the people of our country, as well as to our commander
in chief --- it is scary.''

She seized on Mr. Trump's assertion in the televised forum that Mr.
Putin's incursions into neighboring countries, crackdown on Russia's
independent news media and support for America's enemies were no more
troublesome than Mr. Obama's transgressions. She said it showed that, if
elected, Mr. Trump would be little more than a tool of Mr. Putin.

``It suggests he will let Putin do whatever Putin wants to do and then
make excuses for him,'' Mrs. Clinton told reporters Thursday morning at
Westchester County Airport in New York, stepping up her criticism as
polls indicate the race has tightened, and as Mr. Trump continues to say
things rarely heard before from a major party's presidential nominee.

In Wednesday's forum, which was moderated by Matt Lauer of NBC and was
devoted to national security issues, Mr. Trump twice denigrated
America's generals; suggested he would fire the country's current
military leadership; and insinuated --- vaguely, unverifiably and
without evidence --- that the intelligence officials who recently gave
him a classified briefing about threats to the United States had said
that the president had flouted their advice.

Mrs. Clinton delighted at the chance to change the subject from her
uneven performance at the forum, under treatment by Mr. Lauer that many
observers believed was harsher than his handling of Mr. Trump. Her
campaign could barely contain its wonder that her opponents were now
allowing her to chain Mr. Trump to a Russian leader widely seen as
hostile to the United States.

In the forum, Mr. Trump said of Mr. Putin that he had been a leader
``far more than our president,'' and he praised Mr. Putin's firm grip on
Russia.

And after Mr. Lauer highlighted Mr. Putin's record, Mr. Trump shot back,
``But do you want me to start naming some of the things that President
Obama does at the same time?''

\includegraphics{https://static01.nyt.com/images/2016/09/08/multimedia/09ryan/09ryan-videoSixteenByNine3000.jpg}

Such talk is a remarkable break from the traditional boundaries of
American political speech. And, as with his past provocations, Mr. Trump
once again left his fellow Republicans scrambling to defend what many
effectively conceded was indefensible.

``Vladimir Putin is an aggressor who does not share our interests,''
Speaker Paul D. Ryan told reporters on Thursday in Washington, accusing
the Russian leader of ``conducting state-sponsored cyberattacks'' on
``our political system.''

Mr. Ryan was referring to the hack of the servers of the Democratic
National Committee, which American officials believe was conducted by
Russian intelligence services. At the NBC forum, Mr. Trump disputed
Russia's guilt, telling Mr. Lauer the culprits were not definitively
known.

Mr. Trump went even further on Thursday, saying in an interview on the
Kremlin-backed Russia Today network that it was ``probably unlikely''
Russia was trying to interfere in the election and that Democrats ``are
putting that out.''

In a fashion that would have been unheard-of for a Republican during or
immediately after the Cold War, Mr. Trump has made improved relations
with the Kremlin a centerpiece of his candidacy. And Russia has been a
subplot of the campaign that Tom Clancy and John le Carre together may
have been unable to conjure, complete with the apparent Russian hack of
one of America's political parties, a threat that Russian hackers may
try to tamper with electronic voting machines, and Mr. Putin's unsubtle
preference for Mr. Trump over Mrs. Clinton.

While railing against Asian, Latin American and Middle Eastern
countries, Mr. Trump has continually praised Mr. Putin's government: He
has hailed Mr. Putin's tight control over Russian society, hinted that
he may not defend the NATO-aligned Baltic nations formerly in Moscow's
sphere of influence, and for a time employed a campaign chief with close
ties to Ukraine's pro-Russian forces.

Most extraordinarily, he used a news conference over the summer to urge
the Russians to hack into Mrs. Clinton's emails to find messages the
F.B.I. might have missed.

It is all rather confounding --- unless Mr. Trump is simply eyeing
postelection business interests --- for congressional Republicans, who
evince little doubt that Moscow was behind the hack of the Democratic
National Committee. On Thursday, they volunteered the sort of hard-edged
criticism of Mr. Putin more typical of conservatives discussing an
adversary of the United States.

``He's a thug,'' said Senator Marco Rubio of Florida. ``He's a dangerous
and bad guy.''

But Mr. Rubio, who is running for re-election, has gotten behind Mr.
Trump since withdrawing from the presidential primary, and he declined
to say whether Mr. Trump's comments were out of bounds because, he said,
he did not want to ``be a commentator.''

Even Senator Jeff Sessions of Alabama, perhaps Mr. Trump's closest ally
on Capitol Hill, appeared ill at ease when pressed about Mr. Trump's
statements.

\includegraphics{https://static01.nyt.com/images/2016/09/09/us/09CLINTONPUTINweb2sub/09CLINTONPUTINweb2-articleLarge.jpg?quality=75\&auto=webp\&disable=upscale}

Asked whether political combat should stop at the water's edge, Mr.
Sessions paused for nearly 10 seconds before saying, ``I've tried to
adhere to that line pretty assiduously, but less and less does that get
adhered to in the modern world.''

Democrats were at once dumbfounded over Mr. Trump's latest verbal
excess, gleeful over a fresh opportunity to portray him as
unpresidential and irritated that he had not been pressed more
aggressively by Mr. Lauer.

Mingling outside the Capitol on a broiling day, the Senate Democratic
leader, Harry Reid, and Representative Charles B. Rangel of New York,
two of the longest-serving and bluntest-speaking members of Congress,
found themselves uncharacteristically at a loss for words.

``If Rangel or Reid had said that, 15 years ago or five years ago, we
would be through,'' Mr. Reid said of Mr. Trump's Putin praise. ``Can you
imagine somebody running for president who has acknowledged publicly
that he likes Putin better than Obama? How about that one?''

Mr. Rangel interjected: ``A communist leader that's a potential enemy!''

Other Democrats, though, saw Mr. Trump's comments about Mr. Putin as a
bonanza, given the scrutiny of Mrs. Clinton's use of a private email
server as secretary of state. Representative Joseph Crowley of New York
called Mr. Trump's suggestion that Russians should hack into Mrs.
Clinton's emails ``verbal treason'' and said Mr. Trump's ``diarrhea of
the mouth'' would be his undoing.

Democrats and even some Republicans said the fury would have been
unceasing on the right had a Democratic presidential candidate held up
the leader of a hostile power to deride a Republican president.

Scholars could recall few parallels in modern American history. Only the
campaign of Henry Wallace, the Progressive Party nominee in 1948, was so
willing to align itself with Russia, the historian Richard Norton Smith
said. ``We've become to some degree numbed to this, saying, `That's just
Trump,''' he said. ``And that's dangerous.''

In her news conference Thursday, Mrs. Clinton invoked the right's most
venerated president, from whose library Mr. Pence appeared on CNN.
``What would Ronald Reagan say about a Republican nominee who attacks
American generals and heaps praise on Russia's president?'' she asked.

After the news conference, Mrs. Clinton flew to North Carolina to rally
African-American voters, and seized the chance to assail Mr. Trump's
comments again. ``He prefers the Russian president to our president,''
Mrs. Clinton said in Charlotte.

But Mr. Trump showed no sign of regret. His aides did not reply to an
email asking if the campaign wanted to clarify his comments about Mr.
Putin, and deemed Mrs. Clinton's assault ``the desperate attacks of a
flailing campaign sinking in the polls.''

Mr. Trump himself appeared mostly focused on news coverage of the NBC
forum. ``Wow, reviews are in --- THANK YOU!'' he wrote on Twitter.

Advertisement

\protect\hyperlink{after-bottom}{Continue reading the main story}

\hypertarget{site-index}{%
\subsection{Site Index}\label{site-index}}

\hypertarget{site-information-navigation}{%
\subsection{Site Information
Navigation}\label{site-information-navigation}}

\begin{itemize}
\tightlist
\item
  \href{https://help.nytimes.com/hc/en-us/articles/115014792127-Copyright-notice}{©~2020~The
  New York Times Company}
\end{itemize}

\begin{itemize}
\tightlist
\item
  \href{https://www.nytco.com/}{NYTCo}
\item
  \href{https://help.nytimes.com/hc/en-us/articles/115015385887-Contact-Us}{Contact
  Us}
\item
  \href{https://www.nytco.com/careers/}{Work with us}
\item
  \href{https://nytmediakit.com/}{Advertise}
\item
  \href{http://www.tbrandstudio.com/}{T Brand Studio}
\item
  \href{https://www.nytimes.com/privacy/cookie-policy\#how-do-i-manage-trackers}{Your
  Ad Choices}
\item
  \href{https://www.nytimes.com/privacy}{Privacy}
\item
  \href{https://help.nytimes.com/hc/en-us/articles/115014893428-Terms-of-service}{Terms
  of Service}
\item
  \href{https://help.nytimes.com/hc/en-us/articles/115014893968-Terms-of-sale}{Terms
  of Sale}
\item
  \href{https://spiderbites.nytimes.com}{Site Map}
\item
  \href{https://help.nytimes.com/hc/en-us}{Help}
\item
  \href{https://www.nytimes.com/subscription?campaignId=37WXW}{Subscriptions}
\end{itemize}
