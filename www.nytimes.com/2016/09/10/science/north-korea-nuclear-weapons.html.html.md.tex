Sections

SEARCH

\protect\hyperlink{site-content}{Skip to
content}\protect\hyperlink{site-index}{Skip to site index}

\href{https://www.nytimes.com/section/science}{Science}

\href{https://myaccount.nytimes.com/auth/login?response_type=cookie\&client_id=vi}{}

\href{https://www.nytimes.com/section/todayspaper}{Today's Paper}

\href{/section/science}{Science}\textbar{}North Korea Will Have the
Skills to Make a Nuclear Warhead by 2020, Experts Say

\url{https://nyti.ms/2csf2wB}

\begin{itemize}
\item
\item
\item
\item
\item
\end{itemize}

Advertisement

\protect\hyperlink{after-top}{Continue reading the main story}

Supported by

\protect\hyperlink{after-sponsor}{Continue reading the main story}

\hypertarget{north-korea-will-have-the-skills-to-make-a-nuclear-warhead-by-2020-experts-say}{%
\section{North Korea Will Have the Skills to Make a Nuclear Warhead by
2020, Experts
Say}\label{north-korea-will-have-the-skills-to-make-a-nuclear-warhead-by-2020-experts-say}}

\includegraphics{https://static01.nyt.com/images/2016/09/10/science/10NUKES/10NUKES-articleLarge.jpg?quality=75\&auto=webp\&disable=upscale}

By \href{http://www.nytimes.com/by/william-j-broad}{William J. Broad}

\begin{itemize}
\item
  Sept. 9, 2016
\item
  \begin{itemize}
  \item
  \item
  \item
  \item
  \item
  \end{itemize}
\end{itemize}

North Korea's fifth nuclear test is ominous not only because the country
is slowly mastering atomic weaponry, but because it is making headway in
developing missiles that could hurl nuclear warheads halfway around the
globe, threatening Washington and New York City.

The reclusive, hostile nation has been rushing to perfect missiles that
are small, fast, light and surprisingly advanced, according to analysts
and military officials. This spring and summer, Pyongyang successfully
tested some of these missiles, while earlier efforts had fizzled or
failed.

``They've greatly increased the tempo of their testing --- in a way,
showing off their capabilities, showing us images of ground tests they
could have kept hidden,''
\href{http://38north.org/author/john-schilling/}{John Schilling}, an
aerospace engineer and expert on North Korea's missile program, said in
an interview on Friday. ``This isn't something that can be ignored
anymore. It's going to be a high priority for the next president.''

Military experts say that by 2020, Pyongyang will most likely have the
skills to make a reliable intercontinental ballistic missile topped by a
nuclear warhead. They also expect that by then North Korea may have
accumulated enough nuclear material to build up to 100 warheads.

Siegfried S. Hecker, a Stanford professor who has traveled to North
Korea and who formerly directed the Los Alamos weapons lab in New
Mexico, the birthplace of the atomic bomb, said North Korea's progress
in missile and nuclear development signals that it has gone from seeing
unconventional weapons as bargaining chips to ``deciding they need a
nuclear weapons fighting force.''

The Pentagon warned Congress
\href{http://www.defense.gov/Portals/1/Documents/pubs/Military_and_Security_Developments_Involving_the_Democratic_Peoples_Republic_of_Korea_2015.PDF}{in
a report} earlier this year that one of Pyongyang's latest missiles, if
perfected, ``would be capable of reaching much of the continental United
States.''

In congressional testimony, American officials have provided more
details. Intelligence analysts, they say, now judge that North Korea can
miniaturize a nuclear weapon, place it atop a missile and fire it at the
United States --- though the odds of a successful nuclear strike are
seen as low.

Adm. Samuel J. Locklear III, head of the Pacific Command, last year
summed up the deep concern. ``All the indications are that we have to be
prepared to defend the homeland,''
\href{http://www.pacom.mil/Media/Speeches-Testimony/Article/585392/transcript-senate-armed-services-committee-hearing-on-the-us-pacific-command-us/}{he
told} the Senate Armed Services Committee.

North Korea's own claims about its nuclear capacities are generally
viewed with extreme skepticism. The state, led by an erratic, young
leader, Kim Jong-un, is notorious for blustering propaganda, fake photos
and outright lies.

So private analysts and United States intelligence officials have in
recent years tracked the country's progress by studying carefully vetted
imagery from satellites, and from North Korea itself, of the growing
number of missile firings and engine tests. The experts track how far
and fast the missiles travel, and the color of their plumes. Recently,
one set of plumes became much cleaner, indicating the successful use of
advanced propellants, analysts reported.

North Korea is an impoverished nation whose sophisticated missile
program has been built with Cold War-era Russian technology as well as
the expertise of Russian engineers who moved there in the early 1990s
looking for lucrative work after the Soviet Union fell apart, rocket
experts and intelligence analysts say.

The Soviet Union, if poor in consumer goods, inaugurated the space age
in dazzling firsts. Eventually, the United States caught up and won the
race, landing astronauts on the moon. As it turns out, Russia's rocket
engines were far more innovative than those the Americans used.

Jeffrey Lewis, a North Korea specialist at the
\href{http://www.miis.edu/}{Middlebury Institute for International
Studies at Monterey}, in California, recently noted the grim
implications of a test-firing on land that featured the debut of a
powerful new engine.

``That means that, rather than simply hitting the West Coast, an
operational North Korean ICBM could probably reach targets throughout
the United States, including Washington, D.C.,''
\href{http://www.armscontrolwonk.com/archive/1201278/north-korea-tests-a-fancy-new-rocket-engine/}{he
wrote}in a blog.

Pyongyang obtained its first wave of Russian rocket technology in the
1980s, giving it an ability to make Scuds, short-range missiles with
engines that burn kerosene and emit smoky exhaust. Soon, the collapse of
the Soviet rocket industry brought North Korea a second wave of far more
potent technology.

The collapse began late in the Cold War as arms agreements led to deep
cuts in both Soviet and American nuclear forces. It accelerated when
Russia was unable to create a private industry for putting commercial
satellites into orbit. Soon, impoverished rocket designers were fleeing
Russia.

In one incident in late 1992, officials at a Moscow airport blocked a
group of nearly two dozen missile experts, along with their wives and
children, from traveling to Pyongyang, the North Korean capital. ``I
have always believed that our work is the most important,'' Yuri
Bessarabov, one of the rocket scientists, told Moscow News. ``But it has
turned out that we are unnecessary.''

By the time President Obama took office, in January 2009, Pyongyang had
deployed hundreds of short- and medium-range missiles that used motors
of Russian design, and had exported hundreds of the weapons armed with
conventional warheads to countries including Egypt, Iran and Syria.
Typically, the countries bought Scuds.

At this time, North Korea was also developing the new generation of
missiles powered by a much more advanced engine. Western intelligence
analysts were alarmed to discover that the new engine derived from the
R-27, a compact missile made for Soviet submarines that had carried a
nuclear warhead. Its creator was the Makeyev Design Bureau, an
industrial complex in the Ural Mountains whose rogue experts had been
detained at the Moscow airport.

The engine jacked up heat, thrust and range, outpacing the Scud motor.
And its propellants were more energetic than the old kerosene fuels.
They were hypergolic. That meant the ingredients, when mixed, ignited
spontaneously in powerful blasts. They made the smoky kerosene look
archaic.

The engine was being developed to power a new missile known as the
Musudan, named after Pyongyang's main launching site. The greater thrust
of its single engine translated into greater range. Analysts warned that
the missile's warhead might fly for up to 2,400 miles --- far enough to
hit the American base at Guam but shy of the minimum intercontinental
range of 3,400 miles.

At a military parade in late 2010, Pyongyang unveiled its R-27 spinoff,
giving substance to years of American intelligence warnings. The Musudan
turned out to be 5 feet wide and 40 feet long --- remarkably small
compared to North Korea's large missiles, which military analysis saw as
sitting ducks.

The smaller missiles displayed that day were transported on trucks and
could be hauled on country roads through forested regions or kept in
tunnels, making them easy to hide and, as a target, difficult to find
and destroy.

Pyongyang also used the R-27 engine design as a building block to make
compact missiles that could fire warheads between continents.

The KN-08 missile (Korea North military type 8) was powered by two of
the advanced motors. Analysts said its range was intercontinental and
might send a warhead plummeting down on the West Coast. The KN-14, a
longer version of the KN-08, appeared able, in theory, to send one of
Pyongyang's nuclear warheads crashing down on Washington, D.C.

Today, the KN-08 and the KN-14 are widely seen as the most threatening
missiles in North Korea's developing arsenal, especially given the land
test in April of the potent engine that apparently powers them.

Still, experts note that North Korea is years away from deploying a
reliable long-range missile. For instance, it has yet to master the
complex technology needed to protect a nuclear warhead from the searing
heat generated as it plunges from outer space to a fiery re-entry.

Experts also do not see North Korea as being capable anytime soon of
building a much more destructive hydrogen warhead, capable of destroying
large cities.

Still, military officials worry about a day of reckoning.

``The intel community assesses North Korea's ability to successfully
shoot an ICBM with a nuclear weapon and reach the homeland as low,''
William E. Gortney, commander of North American Aerospace Defense
Command, told a subcommittee of the Senate Armed Services Committee in
April.

Eventually, he added, ``we assess that this low probability will
increase,'' and the United States will need to invest in better
defenses.

Making sure Pyongyang has serious doubts about whether a nuclear strike
would ever succeed, Commander Gortney added, ``is absolutely critical.''

Advertisement

\protect\hyperlink{after-bottom}{Continue reading the main story}

\hypertarget{site-index}{%
\subsection{Site Index}\label{site-index}}

\hypertarget{site-information-navigation}{%
\subsection{Site Information
Navigation}\label{site-information-navigation}}

\begin{itemize}
\tightlist
\item
  \href{https://help.nytimes.com/hc/en-us/articles/115014792127-Copyright-notice}{©~2020~The
  New York Times Company}
\end{itemize}

\begin{itemize}
\tightlist
\item
  \href{https://www.nytco.com/}{NYTCo}
\item
  \href{https://help.nytimes.com/hc/en-us/articles/115015385887-Contact-Us}{Contact
  Us}
\item
  \href{https://www.nytco.com/careers/}{Work with us}
\item
  \href{https://nytmediakit.com/}{Advertise}
\item
  \href{http://www.tbrandstudio.com/}{T Brand Studio}
\item
  \href{https://www.nytimes.com/privacy/cookie-policy\#how-do-i-manage-trackers}{Your
  Ad Choices}
\item
  \href{https://www.nytimes.com/privacy}{Privacy}
\item
  \href{https://help.nytimes.com/hc/en-us/articles/115014893428-Terms-of-service}{Terms
  of Service}
\item
  \href{https://help.nytimes.com/hc/en-us/articles/115014893968-Terms-of-sale}{Terms
  of Sale}
\item
  \href{https://spiderbites.nytimes.com}{Site Map}
\item
  \href{https://help.nytimes.com/hc/en-us}{Help}
\item
  \href{https://www.nytimes.com/subscription?campaignId=37WXW}{Subscriptions}
\end{itemize}
