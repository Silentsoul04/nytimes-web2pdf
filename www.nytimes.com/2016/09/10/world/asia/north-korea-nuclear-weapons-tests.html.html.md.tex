Sections

SEARCH

\protect\hyperlink{site-content}{Skip to
content}\protect\hyperlink{site-index}{Skip to site index}

\href{https://www.nytimes.com/section/world/asia}{Asia Pacific}

\href{https://myaccount.nytimes.com/auth/login?response_type=cookie\&client_id=vi}{}

\href{https://www.nytimes.com/section/todayspaper}{Today's Paper}

\href{/section/world/asia}{Asia Pacific}\textbar{}North Korea's Nuclear
Blasts Keep Getting Stronger

\url{https://nyti.ms/2cyMLlh}

\begin{itemize}
\item
\item
\item
\item
\item
\end{itemize}

Advertisement

\protect\hyperlink{after-top}{Continue reading the main story}

Supported by

\protect\hyperlink{after-sponsor}{Continue reading the main story}

\hypertarget{north-koreas-nuclear-blasts-keep-getting-stronger}{%
\section{North Korea's Nuclear Blasts Keep Getting
Stronger}\label{north-koreas-nuclear-blasts-keep-getting-stronger}}

\includegraphics{https://static01.nyt.com/images/2016/09/10/world/10korea-side-web1/10korea-side-web1-articleLarge.jpg?quality=75\&auto=webp\&disable=upscale}

By \href{http://www.nytimes.com/by/michael-forsythe}{Michael Forsythe}

\begin{itemize}
\item
  Sept. 9, 2016
\item
  \begin{itemize}
  \item
  \item
  \item
  \item
  \item
  \end{itemize}
\end{itemize}

North Korea said it conducted its
\href{http://www.nytimes.com/2016/09/09/world/asia/north-korea-nuclear-test.html}{fifth
underground nuclear test} on Friday. Since
\href{http://www.nytimes.com/2006/10/09/world/asia/09korea.html}{the
first test}, almost a decade ago, the size of the resulting earthquakes
from the country's test site have increased, indicating that the devices
are becoming increasingly powerful.

The device detonated on Friday looks to have had a force equivalent of
10 kilotons of TNT, according to the South's Defense Ministry. In
contrast, the last device tested by the North, in January, had a force
equivalent of six kilotons of TNT, the South's intelligence agency said.
The aboveground Trinity Test in New Mexico in July 1945, which ushered
in the nuclear age, had a yield of 20 kilotons.

But power is not the only measure of a device's lethality. The weapon
must also have a way to be delivered. South Korean, American and
Japanese officials want to determine whether the North Koreans are
capable of building a miniaturized nuclear device that can be mounted on
a ballistic missile and successfully detonated at a target hundreds, if
not thousands, of miles from the launch site. In the past decade, South
Korean and American experts have said that the North appears to be
closer to achieving that goal.

Here is a timeline of how North Korea
\href{http://www.nytimes.com/2016/09/10/world/asia/north-korea-nuclear-test-statement.html}{built
up} the capability of its nuclear weapons. The earthquake magnitudes are
from the United States Geological Survey, which differ from those
measured by the South Korean authorities. They may also be slightly
revised from numbers reported immediately after the events.

\hypertarget{oct-8-2006-935-pm-et}{%
\subsection{Oct. 8, 2006: 9:35 p.m. E.T.}\label{oct-8-2006-935-pm-et}}

\emph{Magnitude of Earthquake}: 4.3

\emph{Device}: United States officials said at the time that the weapon
used
\href{http://www.nytimes.com/2006/10/17/world/asia/17diplo.html}{plutonium}
and had a yield of less than one kiloton.

\emph{Missiles}: Three months before the nuclear test, North Korea fired
a
\href{http://www.nytimes.com/2006/07/05/world/asia/05missile.html}{barrage
of missiles} into the Sea of Japan, including a Taepodong 2
intercontinental missile designed to be capable of reaching Alaska. The
Taepodong 2 test was a failure, with the missile falling into the sea
before its first stage burned out.

Image

A screen showing the seismic waves on Friday. The device detonated on
Friday appears to have been the equivalent of 10 kilotons of TNT,
according to South Korea's Defense Ministry.Credit...Ahn
Young-Joon/Associated Press

\hypertarget{may-24-2009-854-pm-et}{%
\subsection{May 24, 2009: 8:54 p.m. E.T.}\label{may-24-2009-854-pm-et}}

\emph{Magnitude of Earthquake}: 4.7

\emph{Device:} Chinese scientists
\href{http://www.bssaonline.org/content/102/2/467.abstract?sid=7c769220-2dfc-45b2-96d7-73fef9aa8d48}{estimated}
that this bomb had a yield of 2.35 kilotons.

\emph{Missiles}: A
\href{http://www.nytimes.com/2009/04/06/world/asia/06korea.html}{failed
satellite launch} using a Taepodong 2 missile in April 2009 sent its
payload into the Pacific Ocean. On July 4, 2009, North Korea launched
\href{http://www.nytimes.com/2009/07/04/world/asia/04korea.html}{three
missiles} into the sea, with none apparently flying more than 300 miles.

\hypertarget{feb-12-2013-957-pm-et}{%
\subsection{Feb 12, 2013: 9:57 p.m. E.T.}\label{feb-12-2013-957-pm-et}}

\emph{Magnitude of Earthquake}: 5.1

\emph{Device:} North Korea said this bomb,
\href{http://www.ctbto.org/press-centre/press-releases/2013/ctbto-detects-radioactivity-consistent-with-12-february-announced-north-korean-nuclear-test/}{stronger}
than the first two tests, was miniaturized. After the launch, the
Pentagon's Defense Intelligence Agency
\href{http://www.nytimes.com/2013/04/12/world/asia/north-korea-may-have-nuclear-missile-capability-us-agency-says.html}{estimated}
with ``moderate confidence'' that North Korea had learned how to make a
miniaturized nuclear weapon capable of being delivered by a ballistic
missile. But the report said the weapon's ``reliability will be low.''
Military officials in the United States and South Korea publicly
expressed doubt that North Korea had actually developed such a warhead.

\emph{Missiles}: In May 2013, North Korea launched
\href{http://www.nytimes.com/2013/05/19/world/asia/north-korea-missiles.html}{three
short-range missiles} into the Sea of Japan.

\hypertarget{jan-5-2016-830-pm-et}{%
\subsection{Jan. 5, 2016: 8:30 p.m. E.T.}\label{jan-5-2016-830-pm-et}}

\emph{Magnitude of Earthquake}: 5.1

\emph{Device}: North Korea claimed this device was a
\href{http://www.nytimes.com/2016/01/06/world/asia/north-korea-hydrogen-bomb-test.html}{hydrogen
bomb}. In May, American and South Korean intelligence officials
\href{http://www.nytimes.com/2016/05/07/world/asia/north-korea-nuclear-us-strategy.html}{concluded}
that North Korea was now able to mount nuclear warheads on short- and
medium-range missiles that would be capable of hitting Japan and South
Korea.

\emph{Missiles}: In April, North Korea launched a
\href{http://www.nytimes.com/2016/08/24/world/asia/north-korea-submarine-missile.html}{missile
from a submarine}.

\hypertarget{sept-8-2016-830-pm-et}{%
\subsection{Sept. 8, 2016: 8:30 p.m. E.T.}\label{sept-8-2016-830-pm-et}}

\emph{Magnitude of Earthquake}: 5.3

\emph{Device}: South Korean officials said this was North Korea's most
powerful device to date.

\emph{Missiles}: In June, North Korea
\href{http://www.nytimes.com/2016/06/23/world/asia/north-korea-missile-test.html}{successfully
launched} an intermediate-range ballistic missile into high altitude
after five consecutive failures. The missile may be capable of reaching
American forces based on Guam, in the Pacific Ocean.

Advertisement

\protect\hyperlink{after-bottom}{Continue reading the main story}

\hypertarget{site-index}{%
\subsection{Site Index}\label{site-index}}

\hypertarget{site-information-navigation}{%
\subsection{Site Information
Navigation}\label{site-information-navigation}}

\begin{itemize}
\tightlist
\item
  \href{https://help.nytimes.com/hc/en-us/articles/115014792127-Copyright-notice}{©~2020~The
  New York Times Company}
\end{itemize}

\begin{itemize}
\tightlist
\item
  \href{https://www.nytco.com/}{NYTCo}
\item
  \href{https://help.nytimes.com/hc/en-us/articles/115015385887-Contact-Us}{Contact
  Us}
\item
  \href{https://www.nytco.com/careers/}{Work with us}
\item
  \href{https://nytmediakit.com/}{Advertise}
\item
  \href{http://www.tbrandstudio.com/}{T Brand Studio}
\item
  \href{https://www.nytimes.com/privacy/cookie-policy\#how-do-i-manage-trackers}{Your
  Ad Choices}
\item
  \href{https://www.nytimes.com/privacy}{Privacy}
\item
  \href{https://help.nytimes.com/hc/en-us/articles/115014893428-Terms-of-service}{Terms
  of Service}
\item
  \href{https://help.nytimes.com/hc/en-us/articles/115014893968-Terms-of-sale}{Terms
  of Sale}
\item
  \href{https://spiderbites.nytimes.com}{Site Map}
\item
  \href{https://help.nytimes.com/hc/en-us}{Help}
\item
  \href{https://www.nytimes.com/subscription?campaignId=37WXW}{Subscriptions}
\end{itemize}
