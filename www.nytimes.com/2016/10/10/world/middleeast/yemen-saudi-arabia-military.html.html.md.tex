Sections

SEARCH

\protect\hyperlink{site-content}{Skip to
content}\protect\hyperlink{site-index}{Skip to site index}

\href{https://www.nytimes.com/section/world/middleeast}{Middle East}

\href{https://myaccount.nytimes.com/auth/login?response_type=cookie\&client_id=vi}{}

\href{https://www.nytimes.com/section/todayspaper}{Today's Paper}

\href{/section/world/middleeast}{Middle East}\textbar{}A Roar at a
Funeral, and Yemen's War Is Altered

\url{https://nyti.ms/2dQ5RqC}

\begin{itemize}
\item
\item
\item
\item
\item
\end{itemize}

Advertisement

\protect\hyperlink{after-top}{Continue reading the main story}

Supported by

\protect\hyperlink{after-sponsor}{Continue reading the main story}

\hypertarget{a-roar-at-a-funeral-and-yemens-war-is-altered}{%
\section{A Roar at a Funeral, and Yemen's War Is
Altered}\label{a-roar-at-a-funeral-and-yemens-war-is-altered}}

\includegraphics{https://static01.nyt.com/images/2016/10/10/world/10YEMEN1/10YEMEN1-articleInline.jpg?quality=75\&auto=webp\&disable=upscale}

By Shuaib Almosawa and \href{http://www.nytimes.com/by/ben-hubbard}{Ben
Hubbard}

\begin{itemize}
\item
  Oct. 9, 2016
\item
  \begin{itemize}
  \item
  \item
  \item
  \item
  \item
  \end{itemize}
\end{itemize}

SANA, Yemen --- Large speakers played verses from the Quran as hundreds
of mourners filed through the fanciest reception hall in Sana, the
capital, to pay their respects to a prominent family after the death of
its patriarch.

Then there was a roar, the hall shook, and the guests were knocked to
the floor and enveloped in fire and smoke. Some rushed for the exits as
parts of the ceiling collapsed, trapping others under the rubble.

``We did not think they would attack a funeral,'' said Abdulla al-Shamy,
27, a clothing salesman who was in the hall at the time. ``We did not
think they would be so vile.''

The attack on Saturday, which Yemeni officials and witnesses said was a
series of airstrikes by the military coalition led by Saudi Arabia,
killed more than 100 people and unleashed political forces that could
drastically change the course of Yemen's war.

``This really might be the watershed,'' said April Longley Alley, an
analyst with the International Crisis Group who follows events in Yemen.

The carnage in the heart of the capital could hamper any return to talks
aimed at ending the conflict, while galvanizing support in northern
Yemen for military escalation against Saudi Arabia, Ms. Alley said.

The United States will conduct ``an immediate review'' of its support
for the Saudi-led coalition, with possible adjustments ``to better align
with U.S. principles, values and interests,'' according to
\href{https://www.whitehouse.gov/the-press-office/2016/10/08/statement-nsc-spokesperson-ned-price-yemen}{a
statement} from Ned Price, the National Security Council spokesman.

``U.S. security cooperation with Saudi Arabia is not a blank check,''
Mr. Price's statement said. ``Even as we assist Saudi Arabia regarding
the defense of their territorial integrity, we have and will continue to
express our serious concerns about the conflict in Yemen and how it has
been waged.''

Secretary of State John Kerry also spoke by phone on Sunday with top
Saudi officials and called for an immediate ``cessation of
hostilities,'' the State Department said in a statement. Mohammed bin
Salman, the Saudi deputy crown prince, said his government was prepared
to ``institute a renewable 72-hour cessation as soon as possible,
provided the Houthis will agree,'' the State Department added.

Initially, Saudi Arabia denied that jets from its coalition had been
involved in the attack. But in a statement on Sunday, the Saudis
announced an investigation into ``reports about the regrettable and
painful bombing.''

The conflict in Yemen broke out in 2014 when rebels known as
\href{http://www.nytimes.com/2015/01/21/world/middleeast/who-are-the-houthis-of-yemen.html}{the
Houthis} seized the capital and sent the government into exile. The
Houthis are allied with army units loyal to a former president, Ali
Abdullah Saleh; they have been fighting for control of the country
against groups at least nominally loyal to the current president, Abdu
Rabbu Mansour Hadi, who is backed by Saudi Arabia and its Persian Gulf
allies.

In March 2015, the Saudi-led coalition began a campaign of airstrikes
aimed at turning the tide against the Houthi-Saleh alliance. The
campaign has largely failed, while reports of civilian deaths have grown
common, and much of the country is on the brink of famine.

The airstrikes on Saturday followed a period of escalation since August,
when the last round of internationally backed peace talks broke down.
Both sides have sought to bolster their positions since then.

Over the summer, the Houthis announced the creation of a political
council to govern their areas. Mr. Hadi, the exiled president, decreed
that he was relocating the country's central bank from Sana to the
southern port city of Aden, where his government has a presence.

It is not clear where the bank will obtain funds or how it will function
in Aden, and the move could worsen the country's economic crisis, said
Peter Salisbury, who studies Yemen as an associate fellow at Chatham
House, a think tank based in London.

``The decree has really created a limbo state, where we don't know what
is going to happen, and what you really don't want in a time of civil
war is instability in the banking sector,'' Mr. Salisbury said.

Diplomats, including Mr. Kerry, have struggled to restart peace talks, a
possibility that appeared remote after Saturday's strikes. Yemeni
leaders who supported peace talks were among those killed in the
airstrike, along with ordinary civilians.

Yousif al-Emad, who sells insurance, was in the reception hall in Sana
when the first strike hit, filling the hall with smoke and causing a
stampede.

\includegraphics{https://static01.nyt.com/images/2016/10/10/world/10YEMEN3/10YEMEN3-articleInline.jpg?quality=75\&auto=webp\&disable=upscale}

``It was like a movie, when all of a sudden the roof started falling on
the gathering,'' Mr. Emad, 27, said from his bed in a Sana hospital.

When he heard a second strike, he jumped from a window to escape,
breaking his leg. Then he hid in a bathroom as a third strike hit.

He lost six friends and one cousin in the attack, and he now feels
nothing but anger at Saudi Arabia, saying that Yemenis should stage
counterattacks along the Saudi-Yemeni frontier.

``There is nothing for us to do but to go to the fronts at the border,''
he said. ``That is the only weapon at our disposal.''

Tamim al-Shami, a spokesman for the Yemeni Health Ministry, said that
hospitals had received at least 114 bodies from the airstrikes and that
more than 600 people had been wounded.

In a statement on Saturday, the United Nations said more than 140 had
been killed in all. Mr. Shami said the higher figure probably included
victims who had not been taken to medical facilities.

``Some bodies were shredded to pieces, an ear here, a head there,'' Mr.
Shami said.

The dead included many members of prominent tribes from northern Yemen.
Ms. Alley, the analyst with the International Crisis Group, said those
tribes might now ally with the rebels in new attacks on Saudi Arabia.
Also killed were Abdulqader Hilal, the mayor of Sana, and a number of
other political and military leaders who not only supported peace talks
with the exiled government, but also had the credibility to put an
accord into effect.

``They killed and injured several important moderate leaders who were
working with them, who wanted a deal,'' Ms. Alley said of the Saudi-led
coalition. ``Now the desire for revenge is high, and militants will be
empowered, which puts us in a situation where a compromise might not be
possible.''

The attack occurred at a time of growing tension between the United
States and Saudi Arabia. Their decades-old alliance has been strained by
the United States' push for a nuclear agreement with Iran, a bitter
Saudi enemy, as well as by American policy in Syria.

American officials have stepped up public and private criticism of the
Saudi-led air campaign. Several expressed frustration that the campaign
continued to inflict grievous harm on civilians despite American
warnings and growing international condemnation, and some suggested that
the attack on Saturday could be something of a last straw between
Washington and Riyadh.

One senior American official, who spoke about internal administration
deliberations on the condition of anonymity, said that after several
private warnings about airstrikes that killed civilians, the latest
attack was the most serious so far, and the administration needed to
review the situation.

The official said there was no evidence that the coalition had
deliberately tried to hit civilians; rather, the official said,
shortcomings in intelligence and targeting procedures were the most
likely explanation.

The United States has sold billions of dollars' worth of military
hardware and munitions to Saudi Arabia over the years, and a new arms
deal worth \$1.15 billion was approved this year, despite efforts by
dozens of members of Congress to block it.

The United States does not provide the Saudi-led coalition with
targeting information for strikes within Yemen, but it does help Saudi
Arabia guard its borders and provides training and refueling for the
Saudi Air Force. It is this support that could be curtailed after a
policy review.

Some analysts argue that the United States should use this leverage to
press Saudi Arabia and its allies for changes in how they are fighting
in Yemen.

``I'm sick at the current situation in Yemen --- we have to take more
responsibility for it,'' Thomas C. Krajeski, a former United States
ambassador to Yemen, said in an email. ``We can put a lot of pressure on
the Saudis. It's time to do it. They have to know that this war is going
badly for them, if not on the ground then in the arena of world
opinion.''

In Iran, the Islamic Revolutionary Guards Corps strongly condemned the
airstrikes, calling them a ``U.S-Saudi-Zionist joint plot,'' the
semiofficial Tasnim News Agency reported.

In a statement, the Revolutionary Guards predicted that the Houthis
would seek revenge and said that Saudi leaders would suffer the same
fate as that of ``dictators'' like the former Iraqi president Saddam
Hussein and the former Libyan leader Col. Muammar el-Qaddafi.

For Ali el-Shabani, a Yemeni journalist who fled the reception hall
after the first strike and watched it unfold from nearby, the toll on
his community continues to mount.

``Every hour that goes by, I learn that someone I knew was either killed
or wounded,'' he said. ``We are getting worse by the hour. That was like
our little Hiroshima.''

Advertisement

\protect\hyperlink{after-bottom}{Continue reading the main story}

\hypertarget{site-index}{%
\subsection{Site Index}\label{site-index}}

\hypertarget{site-information-navigation}{%
\subsection{Site Information
Navigation}\label{site-information-navigation}}

\begin{itemize}
\tightlist
\item
  \href{https://help.nytimes.com/hc/en-us/articles/115014792127-Copyright-notice}{©~2020~The
  New York Times Company}
\end{itemize}

\begin{itemize}
\tightlist
\item
  \href{https://www.nytco.com/}{NYTCo}
\item
  \href{https://help.nytimes.com/hc/en-us/articles/115015385887-Contact-Us}{Contact
  Us}
\item
  \href{https://www.nytco.com/careers/}{Work with us}
\item
  \href{https://nytmediakit.com/}{Advertise}
\item
  \href{http://www.tbrandstudio.com/}{T Brand Studio}
\item
  \href{https://www.nytimes.com/privacy/cookie-policy\#how-do-i-manage-trackers}{Your
  Ad Choices}
\item
  \href{https://www.nytimes.com/privacy}{Privacy}
\item
  \href{https://help.nytimes.com/hc/en-us/articles/115014893428-Terms-of-service}{Terms
  of Service}
\item
  \href{https://help.nytimes.com/hc/en-us/articles/115014893968-Terms-of-sale}{Terms
  of Sale}
\item
  \href{https://spiderbites.nytimes.com}{Site Map}
\item
  \href{https://help.nytimes.com/hc/en-us}{Help}
\item
  \href{https://www.nytimes.com/subscription?campaignId=37WXW}{Subscriptions}
\end{itemize}
