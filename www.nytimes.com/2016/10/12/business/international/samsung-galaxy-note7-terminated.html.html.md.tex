Sections

SEARCH

\protect\hyperlink{site-content}{Skip to
content}\protect\hyperlink{site-index}{Skip to site index}

\href{https://www.nytimes.com/section/business}{International Business}

\href{https://myaccount.nytimes.com/auth/login?response_type=cookie\&client_id=vi}{}

\href{https://www.nytimes.com/section/todayspaper}{Today's Paper}

\href{/section/business}{International Business}\textbar{}Why Samsung
Abandoned Its Galaxy Note 7 Flagship Phone

\url{https://nyti.ms/2dTV6Uf}

\begin{itemize}
\item
\item
\item
\item
\item
\item
\end{itemize}

Advertisement

\protect\hyperlink{after-top}{Continue reading the main story}

Supported by

\protect\hyperlink{after-sponsor}{Continue reading the main story}

\hypertarget{why-samsung-abandoned-its-galaxy-note-7-flagship-phone}{%
\section{Why Samsung Abandoned Its Galaxy Note 7 Flagship
Phone}\label{why-samsung-abandoned-its-galaxy-note-7-flagship-phone}}

\includegraphics{https://static01.nyt.com/images/2016/10/12/world/12Samsung-web/12Samsung-web-videoSixteenByNine3000.jpg}

By \href{http://www.nytimes.com/by/brian-x-chen}{Brian X. Chen} and
\href{http://www.nytimes.com/by/choe-sang-hun}{Choe Sang-Hun}

\begin{itemize}
\item
  Oct. 11, 2016
\item
  \begin{itemize}
  \item
  \item
  \item
  \item
  \item
  \item
  \end{itemize}
\end{itemize}

When several Samsung Galaxy Note 7 smartphones spontaneously exploded in
August, the South Korean company went into overdrive. It urged hundreds
of employees to quickly diagnose the problem.

None were able to get a phone to explode. Samsung's engineers, on a
tight deadline, initially concluded the defect was caused
\href{http://www.nytimes.com/2016/09/03/technology/samsungs-recall-the-problem-with-lithium-ion-batteries.html}{by
faulty batteries} from one of the company's suppliers. Samsung, which
\href{http://www.nytimes.com/2016/09/03/business/samsung-galaxy-note-battery.html}{announced
a recall}of the Note 7 devices in September, decided to continue
shipping new Galaxy Note 7s containing batteries from a different
supplier.

The solution failed. Reports soon surfaced that some of the replacement
devices were blowing up too. Company engineers went back to the drawing
board, according to a person briefed on the test process who spoke on
the condition of anonymity because the internal workings were
confidential. As of this week, Samsung's testers were still unable to
reproduce the explosions.

By then, it was too late. On Tuesday, Samsung said it was killing the
Galaxy Note 7 entirely. The drastic move is highly unusual in the
technology industry, where companies tend to keep trying to improve a
product rather than pull it altogether. And it caps a nearly two-month
fall for Samsung, which has taken a beating from investors, safety
regulators and consumers over its trustworthiness --- especially with a
marquee product that was supposed to rival Apple's iPhone.

The damage has been severe. Even before Samsung announced it was ceasing
production of the Galaxy Note 7, its South Korea-traded shares fell more
than 8 percent, its biggest daily drop since 2008, knocking \$17 billion
off the company's market value. Strategy Analytics, a research firm, had
estimated earlier that Samsung could lose more than \$10 billion because
of the phone's problems. Samsung's smartphone business has helped its
other divisions by buying their computer chips and panel screens.

Scotching the Note 7 does not end the questions facing Samsung. It still
has not disclosed what specifically caused the Note 7s to smoke and
catch fire --- or even whether it knows what the problem was. And the
company may face questions about the safety of its other products, such
as kitchen appliances and washing machines.

Samsung has received at least
\href{https://www.cpsc.gov/Recalls/2016/Samsung-Recalls-Galaxy-Note7-Smartphones/}{92
reports of Note 7 batteries overheating} in the United States, with 26
reports of burns and 55 reports of property damage, according to
information posted by the United States Consumer Product Safety
Commission. The agency is now working on a potential second recall of
the Note 7s, this time focused on the devices that Samsung had shipped
to replace the original smartphones.

``The fact that we are dealing with potentially a second recall on top
of a first recall is not your normal situation and indicative of a
less-than-ideal process that should have involved earlier coordination
with the government,'' Elliot F. Kaye, chairman of the safety
commission, said in an interview.

A Samsung spokeswoman referred to an earlier statement from the company:
``For the benefit of consumers' safety, we stopped sales and exchanges
of the Galaxy Note 7 and have consequently decided to stop production.''

In killing the Note 7, Samsung made a move reminiscent of
\href{http://mobile.nytimes.com/2002/03/23/your-money/tylenol-made-a-hero-of-johnson-johnson-the-recall-that-started.html}{Tylenol's
1980s recall}, which is held up as a case study in business schools
today. In 1982, seven people died after taking cyanide-laced capsules of
Extra-Strength Tylenol, the company's best-selling product. Tylenol
yanked 31 million bottles of capsules from stores. Two months later, its
painkiller was back on the market with tamper-proof packaging and an
extensive media campaign.

How quickly Samsung will emerge from the Note 7 fiasco is less clear.
The company is facing an immediate, and substantial, financial blow.
Perhaps more worrisome is how
\href{http://www.nytimes.com/2016/10/12/technology/personaltech/do-you-have-a-samsung-galaxy-note-7-heres-what-to-do.html}{people
may lose trust} in the Samsung brand. An editorial in South Korea's
largest newspaper, the Chosun Ilbo, said: ``You cannot really calculate
the loss of consumer trust in money.'' It said that Samsung must realize
that it ``didn't take many years for Nokia to tumble from its position
as the world's top cellphone maker.''

Eric Schiffer, chairman of Reputation Management Consultants, which
helps celebrities and companies manage brand crises, said Samsung's
decision to kill the Note 7 might help it in the long run. ``They made a
really intelligent, hard choice that saved their brand and prevented
what could have been a complete melting down of all the good will they
had built over the last five years,'' he said.

The Galaxy Note 7 was one of the most ambitious products Samsung had
begun marketing under the leadership of its vice chairman, Lee Jae-yong,
who took the helm of the country's largest family-controlled
conglomerate, or chaebol, after his father, Lee Kun-hee, the chairman,
became ill in 2014. The senior Mr. Lee, who has not been seen in public
since, famously burned a pile of 150,000 defective Samsung phones 21
years ago to demonstrate the company's commitment to quality.

The Galaxy Note 7 was released in August, largely to acclaim from
reviewers. In the month before the rollout, Samsung had hundreds of
``beta testers'' using early versions of the units, including
third-party testers like its carrier partners AT\&T and Verizon. None
identified a problem that might cause phones to explode, according to
the person briefed on the testing process.

Samsung's chief smartphone rival, Apple, announced
\href{http://www.nytimes.com/2016/09/08/technology/iphone-7-apple-headphone-jack.html}{new
iPhones last month}. Samsung's fight to compete with Apple by cramming
increasingly sophisticated features into the device may not have helped.
Industry experts are scrutinizing Samsung's supply chain to see whether
the rush to market caused technical problems or led to corners being
cut.

Internally, Samsung's corporate culture may also have compounded any
issues. Two former Samsung employees, who asked not to be named for fear
of retaliation from the company, described the workplace as
militaristic, with a top-down approach where orders came from people
high above who did not necessarily understand how product technologies
actually worked.

``Maybe they should look harder and closer at what is happening at the
management level,'' said Roberta Cozza, a research director with Gartner
Research, who cited the damage to Samsung's credibility with customers
as well as telecommunications carriers.

After the original Note 7s began running into exploding problems in
August, Samsung initially concluded that the problem was batteries
supplied by its subsidiary, Samsung SDI, according to documents from the
Korean Agency for Technology and Standards, a government regulator,
which were leaked to South Korea's SBS TV. The plates inside the SDI
battery were too close to each other near its rounded corners, making it
vulnerable to a short circuit, according to the documents, and the
battery also had defects in its insulating tape and the coating of its
negative electrode.

On Sept. 2, Samsung decided to recall 2.5 million Note 7s with SDI
batteries. But the company was working on an alternative. Both Samsung
and the regulatory agency decided that batteries from another supplier,
ATL, did not have the same defects.

And so Samsung continued to ship Note 7s with ATL batteries, offering
them as replacement phones. That decision backfired.

``It was too quick to blame the batteries; I think there was nothing
wrong with them or that they were not the main problem,'' said Park
Chul-wan, former director of the Center for Advanced Batteries at the
Korea Electronics Technology Institute, who said he reviewed the
regulatory agency's documents.

It did not help that the hundreds of Samsung testers trying to pinpoint
the problem could not easily communicate with one another: Fearing
lawsuits and subpoenas, Samsung told employees involved in the testing
to keep communications about the tests offline --- meaning no emails
were allowed, according to the person briefed on the process.

Mr. Park said he had talked with some Samsung engineers but none seemed
to know what happened, nor were they able to replicate the problem.
Replication would have been quick and easy if the problem was with the
chip board and designs, he said.

``The problem seems to be far more complex,'' Mr. Park said in a phone
interview. ``The Note 7 had more features and was more complex than any
other phone manufactured. In a race to surpass iPhone, Samsung seems to
have packed it with so much innovation it became uncontrollable.''

Advertisement

\protect\hyperlink{after-bottom}{Continue reading the main story}

\hypertarget{site-index}{%
\subsection{Site Index}\label{site-index}}

\hypertarget{site-information-navigation}{%
\subsection{Site Information
Navigation}\label{site-information-navigation}}

\begin{itemize}
\tightlist
\item
  \href{https://help.nytimes.com/hc/en-us/articles/115014792127-Copyright-notice}{©~2020~The
  New York Times Company}
\end{itemize}

\begin{itemize}
\tightlist
\item
  \href{https://www.nytco.com/}{NYTCo}
\item
  \href{https://help.nytimes.com/hc/en-us/articles/115015385887-Contact-Us}{Contact
  Us}
\item
  \href{https://www.nytco.com/careers/}{Work with us}
\item
  \href{https://nytmediakit.com/}{Advertise}
\item
  \href{http://www.tbrandstudio.com/}{T Brand Studio}
\item
  \href{https://www.nytimes.com/privacy/cookie-policy\#how-do-i-manage-trackers}{Your
  Ad Choices}
\item
  \href{https://www.nytimes.com/privacy}{Privacy}
\item
  \href{https://help.nytimes.com/hc/en-us/articles/115014893428-Terms-of-service}{Terms
  of Service}
\item
  \href{https://help.nytimes.com/hc/en-us/articles/115014893968-Terms-of-sale}{Terms
  of Sale}
\item
  \href{https://spiderbites.nytimes.com}{Site Map}
\item
  \href{https://help.nytimes.com/hc/en-us}{Help}
\item
  \href{https://www.nytimes.com/subscription?campaignId=37WXW}{Subscriptions}
\end{itemize}
