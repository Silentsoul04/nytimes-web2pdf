Sections

SEARCH

\protect\hyperlink{site-content}{Skip to
content}\protect\hyperlink{site-index}{Skip to site index}

\href{https://www.nytimes.com/section/world/europe}{Europe}

\href{https://myaccount.nytimes.com/auth/login?response_type=cookie\&client_id=vi}{}

\href{https://www.nytimes.com/section/todayspaper}{Today's Paper}

\href{/section/world/europe}{Europe}\textbar{}António Guterres, Known
for Nerve and Deal Making, Will Need Both at U.N.

\url{https://nyti.ms/2dYHeGf}

\begin{itemize}
\item
\item
\item
\item
\item
\end{itemize}

Advertisement

\protect\hyperlink{after-top}{Continue reading the main story}

Supported by

\protect\hyperlink{after-sponsor}{Continue reading the main story}

\hypertarget{antuxf3nio-guterres-known-for-nerve-and-deal-making-will-need-both-at-un}{%
\section{António Guterres, Known for Nerve and Deal Making, Will Need
Both at
U.N.}\label{antuxf3nio-guterres-known-for-nerve-and-deal-making-will-need-both-at-un}}

\includegraphics{https://static01.nyt.com/images/2016/10/13/world/14guterres-web/14guterres-web-videoSixteenByNineJumbo1600.jpg}

By \href{http://www.nytimes.com/by/somini-sengupta}{Somini Sengupta}

\begin{itemize}
\item
  Oct. 13, 2016
\item
  \begin{itemize}
  \item
  \item
  \item
  \item
  \item
  \end{itemize}
\end{itemize}

UNITED NATIONS --- Kenya was seething. In the spring of 2015, its
leaders complained that a wave of terrorist attacks had been planned in
a refugee camp for Somalis. Shut it down, Kenya demanded of the United
Nations refugee agency --- or we will shut it down for you and send the
Somalis packing.

The man in charge of protecting the world's refugees at the time,
António Guterres, shuttled from his headquarters in Geneva to Mogadishu
to meet with the Somali president, to Nairobi to meet with the Kenyan
president, and on to the
\href{https://www.unhcr-regional.or.ke/news/kenya-somalia-ant\%C3\%B3nio-guterres-ends-visit-discuss-future-dadaab}{refugee
camp}, Dadaab.

His diplomacy led to a deal to keep the camp open, send home only those
Somalis who wanted to return, and rally more international aid for
Somalia and Kenya.

It was not a perfect deal, said Bill Frelick, the refugees expert at
Human Rights Watch, but Mr. Guterres contained a potentially explosive
situation --- at least for a while.

``He was managing a very politicized situation with a lot of raw
nerves,'' he said.

Many raw-nerve reckonings are sure to confront Mr. Guterres when he
takes over as the United Nations secretary general in January for a
five-year term. The 15-member Security Council picked him last week, and
the
\href{https://twitter.com/UNwebcast/status/786577450276839424}{General
Assembly unanimously approved the choice on Thursday}.

Mr. Guterres's predecessor, Ban Ki-moon, who spoke to the General
Assembly after the vote, called him ``perhaps best known where it counts
most: on the front lines of armed conflict and humanitarian suffering.''

Speaking to the General Assembly, Mr. Guterres acknowledged the
challenges that he will face in bringing world powers together on the
most pressing war and peace issues, starting with Syria.

``Whatever divisions might exist, now it's more important to unite,'' he
said.

Mr. Guterres will take over at a time when the credibility of the United
Nations is under intense scrutiny, and when the chasm between Russia and
the West raises the specter of what Mr. Ban
\href{http://www.unric.org/en/latest-un-buzz/30340-ban-ki-moon-let-us-summon-the-spirit-of-reykjavik}{calls}
``Cold War ghosts.''

An engineer by training and a Catholic by conviction, Mr. Guterres, 67,
of Portugal, has described himself as ``an honest broker.'' He has said
that as secretary general, he will embody ``those truly universal values
that are enshrined in the U.N. charter.'' But he also repeatedly cites
the need for what he calls ``discreet diplomacy.''

Those who have worked closely with him often cite his political savvy.
As prime minister of Portugal in the late 1990s, he pushed through
spending cuts necessary for the country to adopt the common European
currency. He negotiated the transfer of Macau, which had been a
Portuguese colony, to Chinese control. As the chief of the perennially
cash-short refugee agency from 2005 to 2015, he traveled constantly,
cultivating the trust of leaders in both countries that host refugees
and those that pay for them.

Critics say Mr. Guterres's penchant for deal-making has constrained him,
especially when trying to persuade powerful countries he depended upon
for financial support.

Shortly after Mr. Guterres returned from Dadaab in 2015, Doctors Without
Borders
\href{https://www.doctorswithoutborders.org/sites/usa/files/msf_obstacle_course_to_europe_report2.pdf}{accused}
the refugee agency and European governments of an ``overwhelming
failure'' to aid and protect the hundreds of thousands of refugees who
were pouring into Europe.

At the time, Mr. Guterres's agency neither stepped up its operations to
manage the dirty, chaotic tent cities that had mushroomed in Greece, nor
could it protect refugees as they made their way across the Continent,
braving razor fences and water cannons.

\includegraphics{https://static01.nyt.com/images/2016/10/14/world/GUTERRES/GUTERRES-articleInline.jpg?quality=75\&auto=webp\&disable=upscale}

Arjan Hehenkamp, the head of the Dutch branch of Doctors Without
Borders, said Mr. Guterres could have pushed the world's richest and
most powerful countries, including the United States, to respond more
robustly and take in many more people fleeing the world's deadliest
battlefields.

``I felt he was doing mainly what was feasible, trying hard to strike
bilateral deals with specific countries to keep their borders open,''
Mr. Hehenkamp said. ``He should have, in my opinion, demanded the world
to do what was necessary instead.''

The refugee agency spokeswoman, Melissa Fleming, who worked with Mr.
Guterres for seven years, said he was reluctant to spend the agency's
limited resources to help wealthy nations manage refugees on their
territory. ``This is Europe,'' she recalled him saying. ``We work in
countries that don't have the means.''

Born in 1949, Mr. Guterres studied engineering and taught briefly while
he was in graduate school.

He found his calling, though, when he began volunteering in a Lisbon
slum. He joined the protests that led to the overthrow of the
authoritarian government in 1974. He helped found his country's
Socialist Party and became its leader. He added a red rose to the
party's clenched-fist logo, in a bid to recast it as less militant.

Mr. Guterres extols the gender quotas his party adopted in the early
1990s to promote women in the party. He does this to underscore what he
calls his commitment to women's rights, and he has promised gender
parity in senior United Nations appointments.

As prime minister between 1995 and 2001, he earned a reputation as ``a
skilled negotiator,'' according to a former minister in his government,
Joao Cravinho, who credited him for reaching agreements with the right
and left at a time when his own party did not have a parliamentary
majority.

One of his signature measures was to decriminalize drug use, in response
to a surge in heroin addiction in Portugal. Less successful was his
party's effort to loosen curbs on abortion. A majority of the Socialist
Party favored the move, though Mr. Guterres said that, as a Catholic, he
opposed it. The law was abandoned after an unsuccessful referendum.

In 2005, he was named the United Nations high commissioner for refugees.

His mathematical thinking never quite left him. Once, during a visit
with refugee children in Lebanon's Bekaa Valley, he started teaching a
math class, his former spokeswoman, Ms. Fleming, recalled.

T. Alex Aleinikoff, an American law professor who served as his deputy,
only half-jokingly referred to Mr. Guterres as the guy with the
accountant's green eye shade, poring over agency budgets. Usually, he
found a mistake.

During Mr. Guterres's tenure, the refugee agency's budget grew sharply,
though still short of what it needed to assist the record numbers of
displaced people worldwide. In a nod to donors, Mr. Guterres moved
agency staff members around, slashing the head count at its headquarters
in Geneva and adding more personnel in field offices.

``He understands things politically as much as operationally,'' Mr.
Aleinikoff said.

Raw nerves are often hard to overcome in diplomacy. In recent months,
Kenya has sought to send refugees back to Somalia, and again revived its
demands to close the Dadaab camp. And the European Union earlier this
year entered into a widely criticized deal with Turkey, promising
billions in aid in exchange for keeping refugees from crossing the
Mediterranean.

Mr. Guterres inherits challenges that will test his ability to balance
the demands of the world's most powerful countries with the needs of the
world's most vulnerable people --- starting, no doubt, with the wars in
Syria and Yemen.

He once referred to a lesson learned from his late first wife, a
psychoanalyst. When two people meet, she told him, there are at least
six perceptions to manage: how they perceive themselves, how they think
the other perceives them, and how the two perceive each other.

Mr. Guterres said that the lesson applied to countries, too, and that
his role was helping them see through the thicket.

``I don't see myself as a threat,'' he said.

Advertisement

\protect\hyperlink{after-bottom}{Continue reading the main story}

\hypertarget{site-index}{%
\subsection{Site Index}\label{site-index}}

\hypertarget{site-information-navigation}{%
\subsection{Site Information
Navigation}\label{site-information-navigation}}

\begin{itemize}
\tightlist
\item
  \href{https://help.nytimes.com/hc/en-us/articles/115014792127-Copyright-notice}{©~2020~The
  New York Times Company}
\end{itemize}

\begin{itemize}
\tightlist
\item
  \href{https://www.nytco.com/}{NYTCo}
\item
  \href{https://help.nytimes.com/hc/en-us/articles/115015385887-Contact-Us}{Contact
  Us}
\item
  \href{https://www.nytco.com/careers/}{Work with us}
\item
  \href{https://nytmediakit.com/}{Advertise}
\item
  \href{http://www.tbrandstudio.com/}{T Brand Studio}
\item
  \href{https://www.nytimes.com/privacy/cookie-policy\#how-do-i-manage-trackers}{Your
  Ad Choices}
\item
  \href{https://www.nytimes.com/privacy}{Privacy}
\item
  \href{https://help.nytimes.com/hc/en-us/articles/115014893428-Terms-of-service}{Terms
  of Service}
\item
  \href{https://help.nytimes.com/hc/en-us/articles/115014893968-Terms-of-sale}{Terms
  of Sale}
\item
  \href{https://spiderbites.nytimes.com}{Site Map}
\item
  \href{https://help.nytimes.com/hc/en-us}{Help}
\item
  \href{https://www.nytimes.com/subscription?campaignId=37WXW}{Subscriptions}
\end{itemize}
