Sections

SEARCH

\protect\hyperlink{site-content}{Skip to
content}\protect\hyperlink{site-index}{Skip to site index}

\href{https://www.nytimes.com/section/us}{U.S.}

\href{https://myaccount.nytimes.com/auth/login?response_type=cookie\&client_id=vi}{}

\href{https://www.nytimes.com/section/todayspaper}{Today's Paper}

\href{/section/us}{U.S.}\textbar{}James Cartwright, Ex-General, Pleads
Guilty in Leak Case

\url{https://nyti.ms/2ec28Uu}

\begin{itemize}
\item
\item
\item
\item
\item
\end{itemize}

Advertisement

\protect\hyperlink{after-top}{Continue reading the main story}

Supported by

\protect\hyperlink{after-sponsor}{Continue reading the main story}

\hypertarget{james-cartwright-ex-general-pleads-guilty-in-leak-case}{%
\section{James Cartwright, Ex-General, Pleads Guilty in Leak
Case}\label{james-cartwright-ex-general-pleads-guilty-in-leak-case}}

\includegraphics{https://static01.nyt.com/images/2016/10/18/us/18court/18court-articleInline.jpg?quality=75\&auto=webp\&disable=upscale}

By \href{http://www.nytimes.com/by/charlie-savage}{Charlie Savage}

\begin{itemize}
\item
  Oct. 17, 2016
\item
  \begin{itemize}
  \item
  \item
  \item
  \item
  \item
  \end{itemize}
\end{itemize}

WASHINGTON --- James E. Cartwright, a retired Marine Corps general who
as vice chairman of the Joint Chiefs of Staff served as a key member of
President Obama's national security team, agreed to plead guilty on
Monday to lying to the F.B.I. about his discussions with reporters about
Iran's nuclear program.

General Cartwright entered the guilty plea before Judge Richard J. Leon
of Federal District Court for the District of Columbia. As part of the
deal, prosecutors and defense lawyers agreed that under sentencing
guidelines, the punishment could range from a \$500 fine to six months
in prison. However, the government reserved the right to argue for a
higher sentence, and the judge is not bound by the guidelines. Judge
Leon set a sentencing hearing for Jan. 17.

During the half-hour hearing, General Cartwright spoke stoically and in
a calm voice, answering ``Yes, sir'' to a series of questions posed by
the judge to make sure he understood what he was doing.

He did not speak to reporters afterward, but in a statement said that he
was not the original source of the information.

``It was wrong for me to mislead the F.B.I. on Nov. 2, 2012, and I
accept full responsibility for this,'' General Cartwright said. ``I knew
I was not the source of the story and I didn't want to be blamed for the
leak. My only goal in talking to the reporters was to protect American
interests and lives; I love my country and continue to this day to do
everything I can to defend it.''

His lawyer, Gregory B. Craig, said in a statement that his client had
spoken to journalists after they had already reported their stories and
that his motive was to prevent publication of information that might
have harmed national security. The investigation focused on leaks to
reporters for The New York Times and Newsweek.

For General Cartwright, who was known as ``Obama's favorite general''
before his retirement in 2011, the plea amounts to a stunning fall. It
also adds a new twist to a surge of leak-related criminal cases in the
Obama era.

The case grew out of a period of political furor over leaks in the
summer of 2012, when numerous books and articles appeared about Mr.
Obama's national security record during his first term.

Republicans in Congress accused the White House of deliberately leaking
government secrets, endangering national security to make Mr. Obama look
tough in an election year.

The administration denied that charge, and the attorney general at the
time, Eric H. Holder Jr.,
\href{http://www.nytimes.com/2012/06/09/us/politics/holder-directs-us-attorneys-to-investigate-leaks.html}{appointed}
two United States attorneys to look into two specific disclosures, one
of which was the cyberattack on Iran's nuclear program.

NBC News
\href{http://investigations.nbcnews.com/_news/2013/06/27/19174350-ex-pentagon-general-target-of-leak-investigation-sources-say}{reported
in 2013} that the Iran cyberattack investigation was focused on General
Cartwright.

The cyberattack was Operation Olympic Games, a joint United
States-Israeli effort to sabotage Iranian nuclear centrifuges with a
computer virus sometimes called Stuxnet. A description of it was
contained in ``Confront and Conceal,'' a book by David E. Sanger, a New
York Times reporter, that was also adapted as an
\href{http://www.nytimes.com/2012/06/01/world/middleeast/obama-ordered-wave-of-cyberattacks-against-iran.html}{article}
published by The Times.

``In researching his book `Confront and Conceal' and his stories for The
New York Times, David E. Sanger relied on multiple sources in
Washington, Europe, the Middle East and elsewhere. Most of them spoke on
the condition of anonymity,'' The Times said in a statement on Monday.

``As in the past, neither The Times nor Mr. Sanger will discuss whether
a particular person was a source or the sourcing of particular
information that was published, beyond what has been disclosed in our
stories and in the book,'' the statement continued. ``Reporting like
this serves a vital public interest: explaining how the United States is
using a powerful new technology against its adversaries and the concern
that it raises about how similar weapons can be used against the U.S. We
will continue to pursue that reporting vigorously.

``We are disappointed that the Justice Department has gone forward with
the leak investigation that led to today's guilty plea by General
Cartwright,'' it added. ``These investigations send a chilling message
to all government employees that they should not speak to reporters. The
inevitable result is that the American public is deprived of information
that it needs to know.''

Prosecutors also accused General Cartwright of lying about his
conversations with another reporter, Daniel Klaidman, then of Newsweek.
They said that the general had falsely told investigators that he had
never discussed an unnamed country with Mr. Klaidman, but that he had
sent an email to that reporter that ``confirmed certain classified
information relating'' to that country in February 2012.

Mr. Klaidman wrote an
\href{http://www.newsweek.com/obamas-dangerous-game-iran-65711}{article}
in February 2012 about the Obama administration's policy toward
disrupting the Iranian nuclear program, including a section about a
conversation between General Cartwright and Mr. Obama in early 2009
about various covert sabotage efforts. The list included cyberwarfare
programs to damage centrifuges. He declined to comment on Monday.

It was reported by
\href{http://foreignpolicy.com/2013/09/24/obamas-favorite-general-stripped-of-his-security-clearance/}{Foreign
Policy} in the fall of 2013 that General Cartwright had been stripped of
his security clearance. But with no official word from the Justice
Department since then, it had seemed that the case was being handled
administratively rather than criminally.

On Monday morning, however, federal prosecutors filed criminal
information against General Cartwright stating that on Nov. 2, 2012,
investigators showed him ``a list of quotes and quotations from David
Sanger's book, a number of which contained classified information,'' but
that he falsely told investigators that ``he was not the source of any
of the quotes and statements'' and that ``he did not provide or confirm
classified information to David Sanger.''

General Cartwright's guilty plea --- not for leaking, but for lying to
agents pursuing an investigation into an apparent leak --- was
reminiscent of the case prosecutors brought during the George W. Bush
administration against I. Lewis Libby, a former chief of staff to Vice
President Dick Cheney. Mr. Libby was charged with lying about his
conversations with journalists to investigators looking into the
disclosure of a C.I.A. official's identity, but was not charged over the
leak itself.

Before General Cartwright's plea, the Obama administration had already
brought criminal charges in more than twice as many cases involving
leaks of government secrets to the news media as were brought under all
previous presidents combined. They included eight officials it charged
under the Espionage Act, although in some cases that charge was dropped.

In a ninth leak-related case in the Obama era, the government
\href{http://www.nytimes.com/2015/03/04/us/petraeus-plea-deal-over-giving-classified-data-to-lover.html}{struck
a deal} with David H. Petraeus, a prominent retired general who served
as director of the C.I.A. He pleaded guilty to a misdemeanor charge of
mishandling classified information related to accusations that he let
his biographer read notebooks containing national security secrets,
although she did not publish any of them.

Mr. Petraeus also admitted to lying to the F.B.I., but he was not
charged with that offense under his plea deal. He paid a fine and was
sentenced to two years of probation.

Advertisement

\protect\hyperlink{after-bottom}{Continue reading the main story}

\hypertarget{site-index}{%
\subsection{Site Index}\label{site-index}}

\hypertarget{site-information-navigation}{%
\subsection{Site Information
Navigation}\label{site-information-navigation}}

\begin{itemize}
\tightlist
\item
  \href{https://help.nytimes.com/hc/en-us/articles/115014792127-Copyright-notice}{©~2020~The
  New York Times Company}
\end{itemize}

\begin{itemize}
\tightlist
\item
  \href{https://www.nytco.com/}{NYTCo}
\item
  \href{https://help.nytimes.com/hc/en-us/articles/115015385887-Contact-Us}{Contact
  Us}
\item
  \href{https://www.nytco.com/careers/}{Work with us}
\item
  \href{https://nytmediakit.com/}{Advertise}
\item
  \href{http://www.tbrandstudio.com/}{T Brand Studio}
\item
  \href{https://www.nytimes.com/privacy/cookie-policy\#how-do-i-manage-trackers}{Your
  Ad Choices}
\item
  \href{https://www.nytimes.com/privacy}{Privacy}
\item
  \href{https://help.nytimes.com/hc/en-us/articles/115014893428-Terms-of-service}{Terms
  of Service}
\item
  \href{https://help.nytimes.com/hc/en-us/articles/115014893968-Terms-of-sale}{Terms
  of Sale}
\item
  \href{https://spiderbites.nytimes.com}{Site Map}
\item
  \href{https://help.nytimes.com/hc/en-us}{Help}
\item
  \href{https://www.nytimes.com/subscription?campaignId=37WXW}{Subscriptions}
\end{itemize}
