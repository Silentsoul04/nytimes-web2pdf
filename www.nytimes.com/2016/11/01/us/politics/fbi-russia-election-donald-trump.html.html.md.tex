Sections

SEARCH

\protect\hyperlink{site-content}{Skip to
content}\protect\hyperlink{site-index}{Skip to site index}

\href{https://www.nytimes.com/section/politics}{Politics}

\href{https://myaccount.nytimes.com/auth/login?response_type=cookie\&client_id=vi}{}

\href{https://www.nytimes.com/section/todayspaper}{Today's Paper}

\href{/section/politics}{Politics}\textbar{}Investigating Donald Trump,
F.B.I. Sees No Clear Link to Russia

\url{https://nyti.ms/2dWIfBL}

\begin{itemize}
\item
\item
\item
\item
\item
\end{itemize}

Advertisement

\protect\hyperlink{after-top}{Continue reading the main story}

Supported by

\protect\hyperlink{after-sponsor}{Continue reading the main story}

\hypertarget{investigating-donald-trump-fbi-sees-no-clear-link-to-russia}{%
\section{Investigating Donald Trump, F.B.I. Sees No Clear Link to
Russia}\label{investigating-donald-trump-fbi-sees-no-clear-link-to-russia}}

\includegraphics{https://static01.nyt.com/images/2016/11/01/us/01russiaelect/01russiaelect-articleInline.jpg?quality=75\&auto=webp\&disable=upscale}

By \href{https://www.nytimes.com/by/eric-lichtblau}{Eric Lichtblau} and
\href{https://www.nytimes.com/by/steven-lee-myers}{Steven Lee Myers}

\begin{itemize}
\item
  Oct. 31, 2016
\item
  \begin{itemize}
  \item
  \item
  \item
  \item
  \item
  \end{itemize}
\end{itemize}

WASHINGTON --- For much of the summer, the F.B.I. pursued a widening
investigation into a Russian role in the American presidential campaign.
Agents scrutinized advisers close to Donald J. Trump, looked for
financial connections with Russian financial figures, searched for those
involved in hacking the computers of Democrats, and even chased a lead
--- which they ultimately came to doubt --- about a possible secret
channel of email communication from the Trump Organization to a Russian
bank.

Law enforcement officials say that none of the investigations so far
have found any conclusive or direct link between
\href{https://www.nytimes.com/2020/07/01/us/politics/trump-putin-russia-taliban-bounty.html}{Mr.
Trump} and the Russian government. And even the hacking into Democratic
emails, F.B.I. and intelligence officials now believe, was aimed at
disrupting the presidential election rather than electing Mr. Trump.

Hillary Clinton's supporters, angry over what they regard as a lack of
scrutiny of Mr. Trump by law enforcement officials, pushed for these
investigations. In recent days they have also demanded that James B.
Comey, the director of the F.B.I., discuss them publicly, as he did last
week when he announced that a
\href{http://www.nytimes.com/2016/10/29/us/politics/fbi-hillary-clinton-email.html}{new
batch of emails} possibly connected to Mrs. Clinton had been discovered.

Supporters of Mrs. Clinton have argued that Mr. Trump's evident affinity
for Russia's president, Vladimir V. Putin --- Mr. Trump has called him a
great leader and echoed his policies toward NATO, Ukraine and the war in
Syria --- and the hacks of leading Democrats like John D. Podesta, the
chairman of the Clinton campaign, are clear indications that Russia has
taken sides in the presidential race and that voters should know what
the F.B.I. has found.

The F.B.I.'s inquiries into Russia's possible role continue, as does the
investigation into the emails involving Mrs. Clinton's top aide, Huma
Abedin, on a computer she shared with her estranged husband, Anthony D.
Weiner. Mrs. Clinton's supporters argue that voters have as much right
to know what the F.B.I. has found in Mr. Trump's case, even if the
findings are not yet conclusive.

``You do not hear the director talking about any other investigation he
is involved in,'' Representative Gregory W. Meeks, Democrat of New York,
said after Mr. Comey's letter to Congress was made public. ``Is he
investigating the Trump Foundation? Is he looking into the Russians
hacking into all of our emails? Is he looking into and deciding what is
going on with regards to other allegations of the Trump Organization?''

Mr. Comey would not even confirm the existence of any investigation of
Mr. Trump's aides when asked during an appearance in September before
Congress. In the Obama administration's internal deliberations over
identifying the Russians as the source of the hacks, Mr. Comey also
argued against doing so and succeeded in keeping the F.B.I.'s imprimatur
off the formal findings, a law enforcement official said. His stance was
first reported by CNBC.

Senator Harry Reid of Nevada, the minority leader, responded angrily on
Sunday with a letter accusing the F.B.I. of not being forthcoming about
Mr. Trump's alleged ties with Moscow.

``It has become clear that you possess explosive information about close
ties and coordination between Donald Trump, his top advisers, and the
Russian government --- a foreign interest openly hostile to the United
States, which Trump praises at every opportunity,'' Mr. Reid wrote.
``The public has a right to know this information.''

F.B.I. officials declined to comment on Monday. Intelligence officials
have said in interviews over the last six weeks that apparent
connections between some of Mr. Trump's aides and Moscow originally
compelled them to open a broad investigation into possible links between
the Russian government and the Republican presidential candidate. Still,
they have said that Mr. Trump himself has not become a target. And no
evidence has emerged that would link him or anyone else in his business
or political circle directly to Russia's election operations.

At least one part of the investigation has involved Paul Manafort, Mr.
Trump's campaign chairman for much of the year. Mr. Manafort, a veteran
Republican political strategist, has had extensive business ties in
Russia and other former Soviet states, especially Ukraine, where he
served as an adviser to that country's ousted president, Viktor F.
Yanukovych.

But the focus in that case was on Mr. Manafort's ties with a
kleptocratic government in Ukraine --- and whether he had declared the
income in the United States --- and not necessarily on any Russian
influence over Mr. Trump's campaign, one official said.

In classified sessions in August and September, intelligence officials
also briefed congressional leaders on the possibility of financial ties
between Russians and people connected to Mr. Trump. They focused
particular attention on what cyberexperts said appeared to be a
mysterious computer back channel between the Trump Organization and the
Alfa Bank, which is one of Russia's biggest banks and whose owners have
longstanding ties to Mr. Putin.

F.B.I. officials spent weeks examining computer data showing an odd
stream of activity to a Trump Organization server and Alfa Bank.
Computer logs obtained by The New York Times show that two servers at
Alfa Bank sent more than 2,700 ``look-up'' messages --- a first step for
one system's computers to talk to another --- to a Trump-connected
server beginning in the spring. But the F.B.I. ultimately concluded that
there could be an innocuous explanation, like a marketing email or spam,
for the computer contacts.

The most serious part of the F.B.I.'s investigation has focused on the
computer hacks that the Obama administration now formally blames on
Russia. That investigation also involves numerous officials from the
intelligence agencies. Investigators, the officials said, have become
increasingly confident, based on the evidence they have uncovered, that
Russia's direct goal is not to support the election of Mr. Trump, as
many Democrats have asserted, but rather to disrupt the integrity of the
political system and undermine America's standing in the world more
broadly.

The hacking, they said, reflected an intensification of spy-versus-spy
operations that never entirely abated after the Cold War but that have
become more aggressive in recent years as relations with Mr. Putin's
Russia have soured.

A senior intelligence official, who like the others spoke on the
condition of anonymity to discuss a continuing national security
investigation, said the Russians had become adept at exploiting computer
vulnerabilities created by the relative openness of and reliance on the
internet. Election officials in several states have reported what
appeared to be cyberintrusions from Russia, and while many doubt that
\href{http://www.nytimes.com/2016/09/15/us/politics/sowing-doubt-is-seen-as-prime-danger-in-hacking-voting-system.html}{an
Election Day hack} could alter the outcome of the election, the F.B.I.
agencies across the government are on alert for potential disruptions
that could wreak havoc with the voting process itself.

``It isn't about the election,'' a second senior official said,
referring to the aims of Russia's interference. ``It's about a threat to
democracy.''

The investigation has treated it as a counterintelligence operation as
much as a criminal one, though agents are also focusing on whether
anyone in the United States was involved. The officials declined to
discuss any individual targets of the investigation, even when assured
of anonymity.

As has been the case with the investigation into Mrs. Clinton, the
F.B.I. has come under intense partisan political pressure --- something
the bureau's leaders have long sought to avoid. Supporters of both Mrs.
Clinton and Mr. Trump have been equally impassioned in calling for
investigations --- and even in providing leads for investigators to
follow.

Mr. Reid, in a letter to Mr. Comey in August, asserted that Mr. Trump's
campaign ``has employed a number of individuals with significant and
disturbing ties to the Russia and the Kremlin.'' Although Mr. Reid cited
no evidence and offered no names explicitly, he clearly referred to one
of Mr. Trump's earlier campaign advisers, Carter Page.

Mr. Page, a former Merrill Lynch banker who founded an investment
company in New York, Global Energy Capital, drew attention during the
summer for a speech in which he criticized the United States and other
Western nations for a ``hypocritical focus on ideas such as
democratization, inequality, corruption and regime change'' in Russia
and other parts of the former Soviet Union.

Mr. Page responded with his own letter to Mr. Comey, denying wrongdoing
and calling Mr. Reid's accusations ``a witch hunt.'' In an interview, he
said that he had never been contacted by the F.B.I. and that the
accusations were baseless and purely partisan because of his policy
views on Russia. ``These people really seem to be grasping at straws,''
he said.

Democrats have also accused another Republican strategist and Trump
confidant, Roger Stone, of being a conduit between the Russian hackers
and WikiLeaks, which has published the emails of the Democratic National
Committee and Mr. Podesta, the Clinton campaign manager. Mr. Stone
boasted of having contacts with the WikiLeaks founder, Julian Assange,
and appeared to predict the hacking of Mr. Podesta's account, though he
later denied having any prior knowledge.

Mr. Stone derided the accusations and those raised by Michael J. Morell,
a former C.I.A. director and a Clinton supporter, who has called Mr.
Trump ``an unwitting agent of the Russian Federation.'' In an
\href{http://www.breitbart.com/hillary-clinton/2016/10/19/stone-wikileaks-mike-morell-russia/}{article}
on the conservative news site Breitbart, Mr. Stone denied having links
to Russians and called the accusations ``the new McCarthyism.''

Advertisement

\protect\hyperlink{after-bottom}{Continue reading the main story}

\hypertarget{site-index}{%
\subsection{Site Index}\label{site-index}}

\hypertarget{site-information-navigation}{%
\subsection{Site Information
Navigation}\label{site-information-navigation}}

\begin{itemize}
\tightlist
\item
  \href{https://help.nytimes.com/hc/en-us/articles/115014792127-Copyright-notice}{©~2020~The
  New York Times Company}
\end{itemize}

\begin{itemize}
\tightlist
\item
  \href{https://www.nytco.com/}{NYTCo}
\item
  \href{https://help.nytimes.com/hc/en-us/articles/115015385887-Contact-Us}{Contact
  Us}
\item
  \href{https://www.nytco.com/careers/}{Work with us}
\item
  \href{https://nytmediakit.com/}{Advertise}
\item
  \href{http://www.tbrandstudio.com/}{T Brand Studio}
\item
  \href{https://www.nytimes.com/privacy/cookie-policy\#how-do-i-manage-trackers}{Your
  Ad Choices}
\item
  \href{https://www.nytimes.com/privacy}{Privacy}
\item
  \href{https://help.nytimes.com/hc/en-us/articles/115014893428-Terms-of-service}{Terms
  of Service}
\item
  \href{https://help.nytimes.com/hc/en-us/articles/115014893968-Terms-of-sale}{Terms
  of Sale}
\item
  \href{https://spiderbites.nytimes.com}{Site Map}
\item
  \href{https://help.nytimes.com/hc/en-us}{Help}
\item
  \href{https://www.nytimes.com/subscription?campaignId=37WXW}{Subscriptions}
\end{itemize}
