Sections

SEARCH

\protect\hyperlink{site-content}{Skip to
content}\protect\hyperlink{site-index}{Skip to site index}

\href{https://www.nytimes.com/section/world/europe}{Europe}

\href{https://myaccount.nytimes.com/auth/login?response_type=cookie\&client_id=vi}{}

\href{https://www.nytimes.com/section/todayspaper}{Today's Paper}

\href{/section/world/europe}{Europe}\textbar{}Russian Officials Were in
Contact With Trump Allies, Diplomat Says

\url{https://nyti.ms/2eFpwGe}

\begin{itemize}
\item
\item
\item
\item
\item
\end{itemize}

Advertisement

\protect\hyperlink{after-top}{Continue reading the main story}

Supported by

\protect\hyperlink{after-sponsor}{Continue reading the main story}

\hypertarget{russian-officials-were-in-contact-with-trump-allies-diplomat-says}{%
\section{Russian Officials Were in Contact With Trump Allies, Diplomat
Says}\label{russian-officials-were-in-contact-with-trump-allies-diplomat-says}}

\includegraphics{https://static01.nyt.com/images/2016/11/11/world/11Russiatrump-web/11Russiatrump-web-articleInline.jpg?quality=75\&auto=webp\&disable=upscale}

By \href{https://www.nytimes.com/by/ivan-nechepurenko}{Ivan
Nechepurenko}

\begin{itemize}
\item
  Nov. 10, 2016
\item
  \begin{itemize}
  \item
  \item
  \item
  \item
  \item
  \end{itemize}
\end{itemize}

MOSCOW --- The Russian government maintained contacts with members of
Donald J. Trump's ``immediate entourage'' during the American
presidential campaign, one of Russia's top diplomats said Thursday.

``There were contacts,'' Sergei A. Ryabkov, the deputy foreign minister,
was quoted as saying by the Interfax news agency. ``We continue to do
this and have been doing this work during the election campaign,'' he
said.

Mr. Ryabkov said officials in the Russian Foreign Ministry were familiar
with many of the people he described as Mr. Trump's entourage. ``I
cannot say that all, but a number of them maintained contacts with
Russian representatives,'' Mr. Ryabkov said.

Later, the Foreign Ministry in Moscow said Mr. Ryabkov had been
referring to American politicians and supporters of Mr. Trump, not
members of his campaign staff. The contacts were carried out through the
Russian ambassador in Washington, who reached out to the senators and
other political allies to get a better sense of Mr. Trump's positions on
various issues involving Russia.

A Trump spokeswoman, Hope Hicks, said Thursday that there had been no
meeting between campaign staff members and Russian government officials
during the campaign.

``We are not aware of any campaign representatives that were in touch
with any foreign entities before yesterday, when Mr. Trump spoke with
many world leaders,'' Ms. Hicks, said. ``Those discussions were
congratulatory and forward looking.''

It is not uncommon for the presidential nominees of major parties to
have contact with foreign leaders, or to meet with foreign government
officials. During the campaign, Mr. Trump traveled to
\href{http://www.nytimes.com/video/us/elections/100000004621483/trump-addresses-mexico.html}{Mexico
to meet with President Enrique Peña Nieto}, and Mr. Trump and Hillary
Clinton
\href{http://www.cnn.com/2016/09/25/politics/netanyahu-trump-clinton-meetings/}{met
separately}with Prime Minister Benjamin Netanyahu of Israel in
September. Mrs. Clinton also met Prime Minister Shinzo Abe of Japan and
President Abdel Fattah el-Sisi of Egypt during the United Nations
General Assembly session in October; a spokesman for her campaign said
there were no communications with Russia.

But the possibility that Russian officials would be in touch with a
candidate in the United States was a particularly sensitive one, because
Russia has been accused of trying to interfere with the election.

After embarrassing emails stolen from the Democratic National Committee
and other institutions and prominent individuals were released by
WikiLeaks, the Obama administration said in October that Russia had
\href{http://www.nytimes.com/2016/10/08/us/politics/us-formally-accuses-russia-of-stealing-dnc-emails.html}{ordered
the hacking} --- an assertion the Russians denied.

\includegraphics{https://static01.nyt.com/images/2017/07/08/us/27TRUMP-PUTIN-COMBO/27TRUMP-PUTIN-COMBO-videoSixteenByNine3000.jpg}

Beyond that, the Senate minority leader, Harry Reid of Nevada, asked the
Federal Bureau of Investigation in August to investigate whether
\href{http://www.nytimes.com/2016/11/01/us/politics/fbi-russia-election-donald-trump.html}{Russia
might be trying to manipulate the vote}.

But law enforcement officials said that
\href{http://www.nytimes.com/2016/11/01/us/politics/fbi-russia-election-donald-trump.html}{their
investigations} found no direct link between Mr. Trump and the Russian
government in the hacking of the Democrats' computers. They also found
no conclusive evidence of financial connections between Mr. Trump's
associates and Russian financial institutions.

Still, some advisers to Mr. Trump have had contact with the Russian
government.
\href{http://www.nytimes.com/2016/10/19/us/politics/michael-flynn-donald-trump.html}{Lt.
Gen. Michael T. Flynn}, a retired intelligence officer and an adviser to
Trump on security issues, was seated next to Mr. Putin during an
anniversary dinner in Moscow for the English-language satellite
television network, RT, in December 2015. And Paul Manafort, Mr. Trump's
former campaign chairman, had previously been
\href{http://www.nytimes.com/2016/08/15/us/politics/paul-manafort-ukraine-donald-trump.html}{a
paid consultant to former President Viktor F. Yanukovych} of Ukraine, a
Kremlin ally before he was ousted in a civic uprising.

On Thursday, Mr. Rybakov sought to play down the perception that Moscow
was thrilled by Mr. Trump's victory --- though members of the state
Duma, or Parliament, did burst into applause at the news.

``We feel no euphoria,'' Mr. Rybakov was quoted as saying in an
interview in Moscow.

``There is diverse experience in dealing with U.S. administrations,
representing both Republican and Democratic periods,'' he said. ``There
were periods when we started on a good note, but then rolled down to
crisis. There were other periods in our complicated history.''

Russia is looking forward to a potentially less complicated relationship
with the United States going forward, Mr. Rybakov added. ``We are not
rejecting a single opportunity for dialogue and for cooperation, and
will immediately become involved in such work at the moment when our
American colleagues will be ready for this,'' he said.

Dmitri S. Peskov, spokesman for Mr. Putin, compared Mr. Trump's victory
speech after the election to a recent speech by Mr. Putin. ``It is
phenomenal, to what extent, it appears they are close in their
conceptual approaches to foreign policy,'' Mr. Peskov told journalists
in New York, where he was attending a chess championship. The two
leaders' similarities, he said, could help improve relations between
Moscow and Washington.

The Russian president had been among the first world leaders
\href{http://www.nytimes.com/2016/11/10/world/europe/russia-putin-donald-trump.html}{to
congratulate Mr. Trump} on his stunning victory in a bitter presidential
campaign in which Mr. Trump made improved relations with Russia a
centerpiece of his bid for office.

Mr.
Trump\href{http://www.nytimes.com/2016/07/26/us/politics/kremlin-donald-trump-vladimir-putin.html}{repeatedly
praised Mr. Putin} during the campaign, saying he is
\href{http://www.nytimes.com/2016/09/09/us/politics/donald-trump-vladimir-putin.html}{a
stronger leader than President Obama}. In October, Mr. Trump said that
should he win, he would consider
\href{http://www.nytimes.com/2016/10/18/us/politics/russia-putin-trump.html}{meeting
with the Russian president} ahead of the inauguration.

Mr. Peskov, the Kremlin spokesman, said Wednesday that there were no
plans for a meeting.

Mr. Putin was careful not to appear to publicly endorse either candidate
in the American presidential race, but Russian state-run media made no
secret of its preference for Mr. Trump, prompting Mrs. Clinton to accuse
her rival of being Moscow's ``puppet.''

Advertisement

\protect\hyperlink{after-bottom}{Continue reading the main story}

\hypertarget{site-index}{%
\subsection{Site Index}\label{site-index}}

\hypertarget{site-information-navigation}{%
\subsection{Site Information
Navigation}\label{site-information-navigation}}

\begin{itemize}
\tightlist
\item
  \href{https://help.nytimes.com/hc/en-us/articles/115014792127-Copyright-notice}{©~2020~The
  New York Times Company}
\end{itemize}

\begin{itemize}
\tightlist
\item
  \href{https://www.nytco.com/}{NYTCo}
\item
  \href{https://help.nytimes.com/hc/en-us/articles/115015385887-Contact-Us}{Contact
  Us}
\item
  \href{https://www.nytco.com/careers/}{Work with us}
\item
  \href{https://nytmediakit.com/}{Advertise}
\item
  \href{http://www.tbrandstudio.com/}{T Brand Studio}
\item
  \href{https://www.nytimes.com/privacy/cookie-policy\#how-do-i-manage-trackers}{Your
  Ad Choices}
\item
  \href{https://www.nytimes.com/privacy}{Privacy}
\item
  \href{https://help.nytimes.com/hc/en-us/articles/115014893428-Terms-of-service}{Terms
  of Service}
\item
  \href{https://help.nytimes.com/hc/en-us/articles/115014893968-Terms-of-sale}{Terms
  of Sale}
\item
  \href{https://spiderbites.nytimes.com}{Site Map}
\item
  \href{https://help.nytimes.com/hc/en-us}{Help}
\item
  \href{https://www.nytimes.com/subscription?campaignId=37WXW}{Subscriptions}
\end{itemize}
