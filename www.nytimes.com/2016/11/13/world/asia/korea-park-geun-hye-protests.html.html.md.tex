Sections

SEARCH

\protect\hyperlink{site-content}{Skip to
content}\protect\hyperlink{site-index}{Skip to site index}

\href{https://www.nytimes.com/section/world/asia}{Asia Pacific}

\href{https://myaccount.nytimes.com/auth/login?response_type=cookie\&client_id=vi}{}

\href{https://www.nytimes.com/section/todayspaper}{Today's Paper}

\href{/section/world/asia}{Asia Pacific}\textbar{}South Koreans Rally in
Largest Protest in Decades to Demand President's Ouster

\url{https://nyti.ms/2escXTd}

\begin{itemize}
\item
\item
\item
\item
\item
\end{itemize}

Advertisement

\protect\hyperlink{after-top}{Continue reading the main story}

Supported by

\protect\hyperlink{after-sponsor}{Continue reading the main story}

\hypertarget{south-koreans-rally-in-largest-protest-in-decades-to-demand-presidents-ouster}{%
\section{South Koreans Rally in Largest Protest in Decades to Demand
President's
Ouster}\label{south-koreans-rally-in-largest-protest-in-decades-to-demand-presidents-ouster}}

\includegraphics{https://static01.nyt.com/images/2016/11/13/world/KOREA/KOREA-videoSixteenByNineJumbo1600.jpg}

By \href{http://www.nytimes.com/by/choe-sang-hun}{Choe Sang-Hun}

\begin{itemize}
\item
  Nov. 12, 2016
\item
  \begin{itemize}
  \item
  \item
  \item
  \item
  \item
  \end{itemize}
\end{itemize}

SEOUL, South Korea --- In one of the largest anti-government protests in
recent decades, hundreds of thousands of South Koreans filled central
Seoul on Saturday to demand the resignation of President Park Geun-hye,
whose administration has been paralyzed by a scandal involving
\href{http://www.nytimes.com/2016/11/12/world/asia/south-korea-park-geun-hye.html}{an
unofficial presidential adviser}.

``You are surrounded! Park Geun-hye, come out and surrender!''
protesters chanted, their voices reverberating through the center of the
capital.

The main boulevard that faces the presidential offices and residence,
known as the Blue House, shimmered with light from candles held by the
protesters. Police buses formed a barricade to block protesters from
getting too close to the presidential compound.

Outrage at Ms. Park has grown in recent weeks over allegations that she
let a
\href{http://www.nytimes.com/2016/11/06/world/asia/south-koreans-ashamed-over-les-secretive-adviser.html}{private
adviser} manipulate her and extort large sums from Korean companies.

Protesters have rallied in downtown Seoul over the past three weekends,
and their numbers have continued to grow. Ms. Park has delivered
\href{http://www.nytimes.com/2016/11/04/world/asia/south-korea-park-geun-hye-investigation.html}{repeated
apologies} for the scandal and offered to share power with a prime
minister to be
\href{http://www.nytimes.com/2016/11/08/world/asia/south-korea-park-choi-scandal-parliament.html}{appointed
by the opposition-dominated Parliament}, but her efforts to stem the
crisis have failed.

The police estimated the crowd on Saturday at 260,000, while organizers
said as many as one million people had turned out.

By either estimate, the rally rivaled the huge demonstrations in 1987
that forced the government, then controlled by the military, to hold a
free presidential election. Those protests were pivotal in a long
struggle to end the military dictatorship started by Ms. Park's father,
Park Chung-hee, in the 1960s.

Ms. Park has become the least popular South Korean leader since the late
1980s, according to recent polls.

Her secretive adviser, Choi Soon-sil, has been arrested on charges of
leveraging her ties to the president to bully businesses into donating
\$69 million to two foundations she controlled. Ms. Choi is
\href{http://www.nytimes.com/2016/11/06/world/asia/south-koreans-ashamed-over-les-secretive-adviser.html}{a
daughter of a cult leader} who became Ms. Park's mentor in the 1970s,
when Ms. Park's father was still in power.

Two former aides to Ms. Park have been charged with helping Ms. Choi
meddle in state affairs from behind the scenes. Ms. Park has admitted
only that she let Ms. Choi, who has no background in government or
policy matters, edit her speeches. But she has apologized for the
scandal, promised to sever ties with Ms. Choi and agreed to be
questioned by prosecutors.

On Saturday, train and bus stations in provincial cities reported that
tickets to Seoul had been sold out, and people struggled to find
transportation to the capital.

Many protesters brought their children, and some mothers were pushing
baby carriages. Some marched alongside a mock hearse meant to symbolize
the death of Ms. Park's government. Teenagers in school uniforms marched
holding signs that said, ``Park Geun-hye, step down!''

Protesters also used Saturday's rally to voice their anger at Ms. Park's
unpopular policies, including her decision to
\href{http://www.nytimes.com/2015/10/13/world/asia/south-korea-to-issue-state-history-textbooks-rejecting-private-publishers.html}{replace
privately published history textbooks} with uniform government-issued
texts by next year.

Many also criticized the agreement that Ms. Park's government struck
with Japan on the issue of the so-called
\href{http://www.nytimes.com/2015/12/29/world/asia/comfort-women-south-korea-japan.html}{comfort
women}, Korean sex slaves who were forced to work in brothels for
Japanese soldiers during World War II.

The main opposition party has yet to call for Ms. Park's resignation,
although its leaders joined the rally on Saturday. The party wants to
reduce her role to that of a figurehead, demanding that she distance
herself from key policy decisions.

Ms. Park's five-year term ends in early 2018. Since 1948, South Koreans
have seen three governments ousted before the end of their terms. The
country's first president, Syngman Rhee, fled into exile in Hawaii amid
a popular uprising in 1960. The succeeding government was overthrown by
Mr. Park's father, who seized power in a military coup in 1961. His rule
ended when he was assassinated in 1979.

Advertisement

\protect\hyperlink{after-bottom}{Continue reading the main story}

\hypertarget{site-index}{%
\subsection{Site Index}\label{site-index}}

\hypertarget{site-information-navigation}{%
\subsection{Site Information
Navigation}\label{site-information-navigation}}

\begin{itemize}
\tightlist
\item
  \href{https://help.nytimes.com/hc/en-us/articles/115014792127-Copyright-notice}{©~2020~The
  New York Times Company}
\end{itemize}

\begin{itemize}
\tightlist
\item
  \href{https://www.nytco.com/}{NYTCo}
\item
  \href{https://help.nytimes.com/hc/en-us/articles/115015385887-Contact-Us}{Contact
  Us}
\item
  \href{https://www.nytco.com/careers/}{Work with us}
\item
  \href{https://nytmediakit.com/}{Advertise}
\item
  \href{http://www.tbrandstudio.com/}{T Brand Studio}
\item
  \href{https://www.nytimes.com/privacy/cookie-policy\#how-do-i-manage-trackers}{Your
  Ad Choices}
\item
  \href{https://www.nytimes.com/privacy}{Privacy}
\item
  \href{https://help.nytimes.com/hc/en-us/articles/115014893428-Terms-of-service}{Terms
  of Service}
\item
  \href{https://help.nytimes.com/hc/en-us/articles/115014893968-Terms-of-sale}{Terms
  of Sale}
\item
  \href{https://spiderbites.nytimes.com}{Site Map}
\item
  \href{https://help.nytimes.com/hc/en-us}{Help}
\item
  \href{https://www.nytimes.com/subscription?campaignId=37WXW}{Subscriptions}
\end{itemize}
