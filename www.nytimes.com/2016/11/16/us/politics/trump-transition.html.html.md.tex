Sections

SEARCH

\protect\hyperlink{site-content}{Skip to
content}\protect\hyperlink{site-index}{Skip to site index}

\href{https://www.nytimes.com/section/politics}{Politics}

\href{https://myaccount.nytimes.com/auth/login?response_type=cookie\&client_id=vi}{}

\href{https://www.nytimes.com/section/todayspaper}{Today's Paper}

\href{/section/politics}{Politics}\textbar{}Firings and Discord Put
Trump Transition Team in a State of Disarray

\url{https://nyti.ms/2eBEWQp}

\begin{itemize}
\item
\item
\item
\item
\item
\item
\end{itemize}

Advertisement

\protect\hyperlink{after-top}{Continue reading the main story}

Supported by

\protect\hyperlink{after-sponsor}{Continue reading the main story}

\hypertarget{firings-and-discord-put-trump-transition-team-in-a-state-of-disarray}{%
\section{Firings and Discord Put Trump Transition Team in a State of
Disarray}\label{firings-and-discord-put-trump-transition-team-in-a-state-of-disarray}}

\includegraphics{https://static01.nyt.com/images/2016/11/16/us/16transition1/16transition1-articleInline.jpg?quality=75\&auto=webp\&disable=upscale}

By \href{https://www.nytimes.com/by/julie-hirschfeld-davis}{Julie
Hirschfeld Davis}, \href{http://www.nytimes.com/by/mark-mazzetti}{Mark
Mazzetti} and \href{http://www.nytimes.com/by/maggie-haberman}{Maggie
Haberman}

\begin{itemize}
\item
  Nov. 15, 2016
\item
  \begin{itemize}
  \item
  \item
  \item
  \item
  \item
  \item
  \end{itemize}
\end{itemize}

WASHINGTON --- President-elect Donald J. Trump's transition was in
disarray on Tuesday, marked by firings, infighting and revelations that
American allies were blindly dialing in to Trump Tower to try to reach
the soon-to-be-leader of the free world.

One week after
\href{http://www.nytimes.com/2016/11/09/us/politics/hillary-clinton-donald-trump-president.html}{Mr.
Trump scored an upset victory} that took him by surprise, his team was
improvising the most basic traditions of assuming power. That included
working without official State Department briefing materials in his
first conversations with foreign leaders.

Two officials who had been handling national security for the
transition, former Representative Mike Rogers of Michigan and Matthew
Freedman, a lobbyist who consults with corporations and foreign
governments, were fired. Both were part of what officials described as a
purge orchestrated by Jared Kushner, Mr. Trump's son-in-law and close
adviser.

The dismissals followed the abrupt
\href{http://www.nytimes.com/2016/11/12/us/politics/trump-cabinet.html}{firing
on Friday of Gov. Chris Christie} of New Jersey, who was replaced as
chief of the transition by Vice President-elect Mike Pence. Mr. Kushner,
a transition official said, was systematically dismissing people like
Mr. Rogers who had ties with Mr. Christie. As a federal prosecutor, Mr.
Christie had sent Mr. Kushner's father to jail.

\href{https://www.nytimes.com/interactive/2016/us/politics/donald-trump-administration.html}{}

\includegraphics{https://static01.nyt.com/images/2016/11/11/us/politics/donald-trump-administration-1478905372015/donald-trump-administration-1478905372015-square640.jpg}

\hypertarget{donald-trumps-cabinet-is-complete-heres-the-full-list}{%
\subsection{Donald Trump's Cabinet Is Complete. Here's the Full
List.}\label{donald-trumps-cabinet-is-complete-heres-the-full-list}}

A list of appointees and nominees for top posts in the new
administration.

Prominent American allies were in the meantime scrambling to figure out
how and when to contact Mr. Trump. At times, they have been patched
through to him in his luxury office tower with little warning, according
to a Western diplomat who spoke on the condition of anonymity to detail
private conversations.

President Abdel Fattah el-Sisi of Egypt was the first to reach Mr. Trump
for such a call last Wednesday, followed by Prime Minister Benjamin
Netanyahu of Israel not long afterward. But that was about 24 hours
before Prime Minister Theresa May of Britain got through --- a striking
break from diplomatic practice given the close alliance between the
United States and Britain.

Despite the haphazard nature of Mr. Trump's early calls with world
leaders, his advisers said the transition team was not suffering unusual
setbacks. They argued that they were hard at work behind the scenes
dealing with the same troubles that incoming presidents have faced for
decades.

And Mr. Trump himself fired back at critics with a
\href{https://mobile.twitter.com/realDonaldTrump/status/798721142525665280}{Twitter
message} he sent about 10 p.m. ``Very organized process taking place as
I decide on Cabinet and many other positions,'' he wrote. ``I am the
only one who knows who the finalists are!''

The process is ``completely normal,'' said
\href{http://www.nytimes.com/2016/11/16/us/politics/donald-trump-cabinet-rudy-giuliani.html}{Rudolph
W. Giuliani}, the former New York mayor, who emerged on Tuesday as the
leading contender to be Mr. Trump's secretary of state. ``It happened in
the Reagan transition. Clinton had delays in hiring people.''

Image

Former Representative Mike Rogers, who had been advising the new
administration on national security issues, has been fired from the
transition team.

Credit...Manuel Balce Ceneta/Associated Press

Mr. Giuliani, who made his comments in a telephone interview, added:
``This is a hard thing to do. Transitions always have glitches. This is
an enormously complex process.''

There were some reports within the transition of score-settling.

One member of the transition team said that at least one reason Mr.
Rogers had fallen out of favor among Mr. Trump's advisers was that, as
chairman of the House Intelligence Committee, he had overseen a report
about the 2012 attacks on the American diplomatic compound in Benghazi,
Libya, which concluded that the Obama administration had not
intentionally misled the public about the events there. That report
echoed the findings of numerous other government investigations into the
episode.

The report's conclusions were at odds with the campaign position of Mr.
Trump, who repeatedly blamed Hillary Clinton, his Democratic opponent
and the secretary of state during the attacks, for the resulting deaths
of four Americans.

Eliot A. Cohen, a former State Department official who had criticized
Mr. Trump during the campaign but said after his election that he would
keep an open mind about advising him,
\href{https://twitter.com/EliotACohen/status/798512852931788800}{said
Tuesday on Twitter} that he had changed his opinion. After speaking to
the transition team, he wrote, he had ``changed my recommendation: stay
away.''

He added: ``They're angry, arrogant, screaming `you LOST!' Will be
ugly.''

Mr. Cohen, a conservative Republican who served under President George
W. Bush, said Trump transition officials had excoriated him after he
offered some names of people who might serve in the new administration,
but only if they felt departments were led by credible people.

``They think of these jobs as lollipops,'' Mr. Cohen said in an
interview.

Senator John McCain, Republican of Arizona and the chairman of the
Senate Armed Services Committee, weighed in as well. On Tuesday, he
issued a blunt warning to Mr. Trump and his emerging foreign policy team
not to be taken in by President Vladimir V. Putin of Russia, whom Mr.
Trump praised during the campaign.

``The Obama administration's last attempt at resetting relations with
Russia culminated in Putin's invasion of Ukraine and military
intervention in the Middle East,'' Mr. McCain said.

Some of the early transition difficulties may reflect the fact that Mr.
Trump, who has no governing experience or Washington network and
campaigned as an agent of change, does not have a long list of
establishment figures from the Bush era to tap. His allies suggested
that might ultimately prove positive for Mr. Trump if he was able to
assemble a functioning team that would bring new perspectives to his
administration.

For advice on building Mr. Trump's national security team, his inner
circle has been relying on three hawkish current and former American
officials: Representative Devin Nunes, Republican of California, who is
chairman of the House Intelligence Committee; Peter Hoekstra, a former
Republican congressman and former chairman of the Intelligence
Committee; and Frank Gaffney, a Pentagon official during the Reagan
administration and a founder of the Center for Security Policy.

Mr. Gaffney has long advanced baseless conspiracy theories, including
that President Obama might be a closet Muslim. The Southern Poverty Law
Center described him as ``one of America's most notorious
Islamophobes.''

\href{https://www.nytimes.com/interactive/2016/11/11/us/politics/what-trump-wants-to-change.html}{}

\includegraphics{https://static01.nyt.com/images/2016/11/11/us/politics/what-trump-wants-to-change-1479009739985/what-trump-wants-to-change-1479009739985-largeHorizontalJumbo.png}

\hypertarget{20-things-donald-trump-said-he-wanted-to-get-rid-of-as-president}{%
\subsection{20 Things Donald Trump Said He Wanted to Get Rid of as
President}\label{20-things-donald-trump-said-he-wanted-to-get-rid-of-as-president}}

Some of the parts of the government that Mr. Trump promised to dismantle
if he was elected.

Prominent donors to Mr. Trump were also having little success in
recruiting people for rank-and-file posts in his administration.

Rebekah Mercer, the scion of a powerful family of conservative donors
and a member of Mr. Trump's executive transition committee, has said in
conversations with Republican operatives and previous administration
officials that she was having trouble finding takers for posts at the
under secretary level and below, according to a person familiar with her
outreach efforts. She told them that the transition team was more than a
month behind schedule and on a tight timeline.

In another delay, Mr. Pence did not sign legally required paperwork to
allow his team to begin collaborating with Mr. Obama's aides until
Tuesday evening, a transition spokesman said. Mr. Christie on Election
Day signed a memorandum of understanding to put the process into motion
as soon as the outcome was determined, but once he was ousted from the
job, Mr. Pence had to sign a new agreement.

The paperwork serves as a nondisclosure agreement for both sides,
ensuring that members of the president-elect's team do not divulge
information about the inner workings of the government.

Teams throughout the federal government that have prepared briefing
materials and reports for the incoming president's team are on standby,
waiting to begin passing the information to counterparts on Mr. Trump's
staff.

As of Tuesday afternoon, officials at key agencies including the Justice
and Defense Departments said they had received no contact from the
president-elect's team.

Advertisement

\protect\hyperlink{after-bottom}{Continue reading the main story}

\hypertarget{site-index}{%
\subsection{Site Index}\label{site-index}}

\hypertarget{site-information-navigation}{%
\subsection{Site Information
Navigation}\label{site-information-navigation}}

\begin{itemize}
\tightlist
\item
  \href{https://help.nytimes.com/hc/en-us/articles/115014792127-Copyright-notice}{©~2020~The
  New York Times Company}
\end{itemize}

\begin{itemize}
\tightlist
\item
  \href{https://www.nytco.com/}{NYTCo}
\item
  \href{https://help.nytimes.com/hc/en-us/articles/115015385887-Contact-Us}{Contact
  Us}
\item
  \href{https://www.nytco.com/careers/}{Work with us}
\item
  \href{https://nytmediakit.com/}{Advertise}
\item
  \href{http://www.tbrandstudio.com/}{T Brand Studio}
\item
  \href{https://www.nytimes.com/privacy/cookie-policy\#how-do-i-manage-trackers}{Your
  Ad Choices}
\item
  \href{https://www.nytimes.com/privacy}{Privacy}
\item
  \href{https://help.nytimes.com/hc/en-us/articles/115014893428-Terms-of-service}{Terms
  of Service}
\item
  \href{https://help.nytimes.com/hc/en-us/articles/115014893968-Terms-of-sale}{Terms
  of Sale}
\item
  \href{https://spiderbites.nytimes.com}{Site Map}
\item
  \href{https://help.nytimes.com/hc/en-us}{Help}
\item
  \href{https://www.nytimes.com/subscription?campaignId=37WXW}{Subscriptions}
\end{itemize}
