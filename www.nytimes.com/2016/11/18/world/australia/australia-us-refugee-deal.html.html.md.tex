Sections

SEARCH

\protect\hyperlink{site-content}{Skip to
content}\protect\hyperlink{site-index}{Skip to site index}

\href{https://www.nytimes.com/section/world/australia}{Australia}

\href{https://myaccount.nytimes.com/auth/login?response_type=cookie\&client_id=vi}{}

\href{https://www.nytimes.com/section/todayspaper}{Today's Paper}

\href{/section/world/australia}{Australia}\textbar{}As Trump Nears
Office, Australian Deal to Move Refugees to U.S. Is in Doubt

\url{https://nyti.ms/2eM1NZO}

\begin{itemize}
\item
\item
\item
\item
\item
\end{itemize}

Advertisement

\protect\hyperlink{after-top}{Continue reading the main story}

Supported by

\protect\hyperlink{after-sponsor}{Continue reading the main story}

\hypertarget{as-trump-nears-office-australian-deal-to-move-refugees-to-us-is-in-doubt}{%
\section{As Trump Nears Office, Australian Deal to Move Refugees to U.S.
Is in
Doubt}\label{as-trump-nears-office-australian-deal-to-move-refugees-to-us-is-in-doubt}}

\includegraphics{https://static01.nyt.com/images/2016/11/19/world/18AUSTRALIA-3/18AUSTRALIA-3-articleInline.jpg?quality=75\&auto=webp\&disable=upscale}

By Michelle Innis

\begin{itemize}
\item
  Nov. 17, 2016
\item
  \begin{itemize}
  \item
  \item
  \item
  \item
  \item
  \end{itemize}
\end{itemize}

CANBERRA, Australia --- Barring some unexpected development, none of the
refugees held in detention camps on the Pacific islands of Nauru and
Manus are likely to be resettled in the United States before
President-elect Donald J. Trump takes office, Australian and American
officials said this week.

The election of Mr. Trump, whose harsh talk about immigration and
Muslims was central to his campaign, leaves in doubt the deal the two
countries recently struck to move some of the hundreds of people who
were banished to the camps by Australia, after being intercepted at sea
trying to reach its shores. Many of the detainees are Muslims.

The issue is a contentious and emotional one in Australia, whose
government has pledged never to accept a migrant who tries to come to
the country by boat. The stated purpose of the so-called Pacific
solution, in which such people are housed indefinitely on offshore
islands, is to discourage human traffickers, who often pack migrants
into rickety boats for the journeys, some of which have ended in mass
drownings.

But conditions on those islands are dire. On Friday, a United Nations
envoy visiting Australia said that daily life for detainees on Nauru was
cruel, inhumane and degrading. Human rights groups
\href{http://www.nytimes.com/2016/08/04/world/australia/nauru-refugees-abuse-conditions.html}{have
reached similar conclusions} about conditions on both islands.

The Australian government said Sunday that it had
\href{http://www.nytimes.com/2016/11/13/world/australia/australia-refugees-united-states.html}{reached
a one-time agreement} in which the United States would take in hundreds
of the detainees.

But just days after the announcement of the plan, an Australian
official, Michael Pezzullo, secretary of the Immigration Department,
told an Australian Senate panel that it seemed unlikely that any
detainees would be resettled by Jan. 20, Inauguration Day in the United
States.

On Wednesday, in a written response to questions, the United States
State Department said that the screening process to determine whether
refugees were eligible to enter the United States was lengthy, taking as
long as 18 to 24 months to complete.

``The safety and security of the American people is our top priority,''
the State Department said. It said refugees were subject to the highest
level of security checks of any category of traveler to the United
States, involving the Department of Homeland Security, the National
Counterterrorism Center and the F.B.I.

Image

François Crépeau, the United Nations special rapporteur on the rights of
migrants.Credit...Karim Jaafar/Agence France-Presse --- Getty Images

Mr. Trump has not commented on the deal. As a presidential candidate, he
called for a temporary ban on all Muslim immigration, though his
campaign later said that the ban would apply only to migrants from
``terror-prone regions.'' One of his supporters, Mark Krikorian,
executive director of the \href{http://cis.org/}{Center for Immigration
Studies}, which favors more restrictive immigration policies, was quoted
in Australian news reports as saying that the deal was likely to be
``dead on arrival'' once Mr. Trump took office.

The United Nations envoy, François Crépeau, said on Friday that
conditions for detainees on Nauru were grim.

``Mental health issues are rife, with post-traumatic stress disorder,
anxiety and depression being among the most common ailments,'' Mr.
Crépeau said at a news conference in Canberra, the Australian capital,
describing what he found on a recent visit to Nauru. ``Many refugees and
asylum seekers are on a constant diet of sleeping tablets and
antidepressants.'' Children suffered from insomnia, nightmares and
bed-wetting, he said.

Mr. Crépeau said Australia's policy was punitive. ``It sends a message
to the people smugglers,'' he said. ``But to me this is not a
justification. To me, mandatory detention is a violation of human rights
law.''

The United Nations has endorsed the plan to send about 1,600 detainees
to the United States. The camps were opened in 2012.

Manus Island is holding 823 men. About 410 men, women and children,
mostly in family groups, are held on Nauru and would be given the first
option to resettle in the United States under the new agreement. About
400 others in Australia seeking some medical treatment must return to
Nauru or Manus Island before applying to go to the United States.

Only those who have been granted refugee status by the United Nations
will be eligible to go to the United States, the Australian government
has said. Prime Minister Malcolm Turnbull of Australia and the State
Department have said the deal would not affect the total number of
refugees accepted by the United States each year.

Mr. Turnbull said the deal was the result of months of talks, but he
offered little reassurance that it would be honored under Mr. Trump.
``We deal with one administration at a time,'' Mr. Turnbull said Sunday.

Similar concerns were raised during the hearing on Tuesday. One senator,
Murray Watt, asked Mr. Pezzullo, the immigration secretary, whether the
government was sure that the deal would be honored, given Mr. Trump's
negative statements about refugees and Muslims. About 10 percent of the
detainees are from Iran, while others are from Afghanistan, Vietnam,
Malaysia and Sri Lanka.

``We have an agreement struck between the two governments and, should
other contingencies or other eventualities arise, that is a matter that
the Australian government will have to deal with at that time,'' Mr.
Pezzullo said.

Advertisement

\protect\hyperlink{after-bottom}{Continue reading the main story}

\hypertarget{site-index}{%
\subsection{Site Index}\label{site-index}}

\hypertarget{site-information-navigation}{%
\subsection{Site Information
Navigation}\label{site-information-navigation}}

\begin{itemize}
\tightlist
\item
  \href{https://help.nytimes.com/hc/en-us/articles/115014792127-Copyright-notice}{©~2020~The
  New York Times Company}
\end{itemize}

\begin{itemize}
\tightlist
\item
  \href{https://www.nytco.com/}{NYTCo}
\item
  \href{https://help.nytimes.com/hc/en-us/articles/115015385887-Contact-Us}{Contact
  Us}
\item
  \href{https://www.nytco.com/careers/}{Work with us}
\item
  \href{https://nytmediakit.com/}{Advertise}
\item
  \href{http://www.tbrandstudio.com/}{T Brand Studio}
\item
  \href{https://www.nytimes.com/privacy/cookie-policy\#how-do-i-manage-trackers}{Your
  Ad Choices}
\item
  \href{https://www.nytimes.com/privacy}{Privacy}
\item
  \href{https://help.nytimes.com/hc/en-us/articles/115014893428-Terms-of-service}{Terms
  of Service}
\item
  \href{https://help.nytimes.com/hc/en-us/articles/115014893968-Terms-of-sale}{Terms
  of Sale}
\item
  \href{https://spiderbites.nytimes.com}{Site Map}
\item
  \href{https://help.nytimes.com/hc/en-us}{Help}
\item
  \href{https://www.nytimes.com/subscription?campaignId=37WXW}{Subscriptions}
\end{itemize}
