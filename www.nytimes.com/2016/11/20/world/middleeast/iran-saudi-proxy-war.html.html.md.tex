Sections

SEARCH

\protect\hyperlink{site-content}{Skip to
content}\protect\hyperlink{site-index}{Skip to site index}

\href{https://www.nytimes.com/section/world/middleeast}{Middle East}

\href{https://myaccount.nytimes.com/auth/login?response_type=cookie\&client_id=vi}{}

\href{https://www.nytimes.com/section/todayspaper}{Today's Paper}

\href{/section/world/middleeast}{Middle East}\textbar{}How the
Iranian-Saudi Proxy Struggle Tore Apart the Middle East

\url{https://nyti.ms/2eRf5o2}

\begin{itemize}
\item
\item
\item
\item
\item
\end{itemize}

Advertisement

\protect\hyperlink{after-top}{Continue reading the main story}

Supported by

\protect\hyperlink{after-sponsor}{Continue reading the main story}

\href{/column/the-interpreter}{The Interpreter}

\hypertarget{how-the-iranian-saudi-proxy-struggle-tore-apart-the-middle-east}{%
\section{How the Iranian-Saudi Proxy Struggle Tore Apart the Middle
East}\label{how-the-iranian-saudi-proxy-struggle-tore-apart-the-middle-east}}

\includegraphics{https://static01.nyt.com/images/2016/11/20/world/INT-IRANSAUDI/INT-IRANSAUDI-articleLarge.jpg?quality=75\&auto=webp\&disable=upscale}

By \href{https://www.nytimes.com/by/max-fisher}{Max Fisher}

\begin{itemize}
\item
  Nov. 19, 2016
\item
  \begin{itemize}
  \item
  \item
  \item
  \item
  \item
  \end{itemize}
\end{itemize}

Behind much of the Middle East's chaos --- the wars in Syria and Yemen,
the political upheaval in Iraq and Lebanon and Bahrain --- there is
another conflict.

Saudi Arabia and Iran are waging a struggle for dominance that has
turned much of the Middle East into their battlefield. Rather than
fighting directly, they wield and in that way worsen the region's direst
problems: dictatorship, militia violence and religious extremism.

The history of their rivalry tracks --- and helps to explain --- the
Middle East's disintegration, particularly the Sunni-Shiite sectarianism
both powers have found useful to cultivate. It is a story in which the
United States has been a supporting but constant player, most recently
by backing the Saudi war in Yemen, which kills hundreds of civilians.
These dynamics, scholars warn, point toward a future of civil wars,
divided societies and unstable governments.

F. Gregory Gause III, an international relations scholar at Texas A \& M
University, struggled to name another region that had been torn apart in
this way. Central Africa could be similar, he suggested, referring to
the two decades of interrelated wars and genocides that, driven by
meddling regional powers,
\href{http://www.nytimes.com/2012/12/16/sunday-review/congos-never-ending-war.html}{killed
five million}. But in the Middle East, it is just getting started.

\hypertarget{1979-a-threatening-revolution}{%
\subsection{1979: A threatening
revolution}\label{1979-a-threatening-revolution}}

Saudi Arabia, a young country pieced together only in the 1930s, has
built its legitimacy on religion. By promoting its stewardship of the
holy sites at Mecca and Medina, it could justify its royal family's grip
on power.

Iran's revolution, in 1979, threatened that legitimacy. Iranians toppled
their authoritarian government, installing Islamists who claimed to
represent ``a revolution for the entire Islamic world,'' said Kenneth M.
Pollack, a senior fellow at the Brookings Institution.

The revolutionaries encouraged all Muslims, especially Saudis, to
overthrow their rulers as well.

But because Iran is mostly Shiite, they ``had the greatest influence
with, and tended to reach out to, Shia groups,'' Dr. Pollack said.

Some Saudi Shiites, who make up about
\href{http://www.nytimes.com/2014/03/15/world/middleeast/saudis-lonely-costly-bid-for-sunni-shiite-equality.html}{10
percent} of the population, protested in solidarity or even set up
offices in Tehran --- stoking Saudi fears of internal unrest and
separatism.

This was the opening shot in the sectarianization of their rivalry,
which would encompass the whole region.

``The Saudis have looked at Iran as a domestic threat from the get-go,
from 1979,'' Dr. Gause said. Seeing the threat as intolerable, they
began looking for a way to strike back.

\hypertarget{1980-88-the-first-proxy-war}{%
\subsection{1980-88: The first proxy
war}\label{1980-88-the-first-proxy-war}}

They found that way the next year, when Saddam Hussein's Iraq invaded
Iran, hoping to seize oil-rich territory.

Saudi Arabia, Dr. Pollack said, ``backed the Iraqis to the hilt because
they want the Iranian revolution stopped.''

The war, over eight years of trench warfare and chemical weapons
attacks, killed perhaps a million people. It set a pattern of
Iranian-Saudi struggle through proxies, and of sucking in the United
States, whose policy is to maintain access to the vast oil and gas
reserves that lie between the rivals.

The conflict's toll exhausted Iran's zeal for sowing revolution abroad,
but gave it a new mission: to overturn the Saudi-led, American-backed
regional order that Tehran saw as an existential threat.

That sense of insecurity would later drive Iran's meddling abroad, said
Marc Lynch, a political scientist at George Washington University, and
perhaps its missile and nuclear programs.

\hypertarget{1989-2002-setting-up-a-powder-keg}{%
\subsection{1989-2002: Setting up a powder
keg}\label{1989-2002-setting-up-a-powder-keg}}

The 1990s provided a pause in the regional rivalry, but also set up the
conditions that would allow it to later explode in such force.

Saudi Arabia, wishing to contain Iran's reach to the region's minority
Shiite populations, sought to harden Sunni-Shiite rifts. Government
programs promoted ``anti-Shia incitement in schools, Islamic
universities, and the media,'' Toby Matthiesen, an Oxford University
scholar, wrote in \href{http://carnegieendowment.org/sada/?fa=60799}{a
brief for the Carnegie Endowment}.

These policies, Dr. Matthiesen warned, cultivated sectarian fears and
sometimes violence that would later feed into the ideology of the
Islamic State.

In 1990, Iraq invaded Kuwait, a Saudi ally. The United States, after
expelling the Iraqis, established military bases in the region to defend
its allies from Iraq. This further tilted the regional power balance
against Iran, which saw the American forces as a threat.

Iraq's humiliating defeat also spurred many of its citizens to rise up,
particularly in poorer communities that happened to be Shiite Arab.

In response, Dr. Gause said, ``Saddam's regime became explicitly
sectarian,'' widening Sunni-Shiite divides to deter future uprisings.
That allowed Iran, still worried about Iraq, to cultivate allies among
Iraq's increasingly disenfranchised Shiites, including militias that had
risen up.

Though it was not obvious at the time, Iraq had become a powder keg, one
that would ignite when its government was toppled a decade later.

\hypertarget{2003-04-the-iraqi-vacuum-opens}{%
\subsection{2003-04: The Iraqi vacuum
opens}\label{2003-04-the-iraqi-vacuum-opens}}

The 2003 American-led invasion, by toppling an Iraqi government that had
been hostile to both Saudi Arabia and Iran, upended the region's power
balance.

Iran, convinced that the United States and Saudi Arabia would install a
pliant Iraqi government --- and remembering the horrors they had
inflicted on Iran in the 1980s --- raced to fill the postwar vacuum. Its
leverage with Shiite groups, which are Iraq's largest demographic group,
allowed it to influence Baghdad politics.

Iran also wielded Shiite militias to control Iraqi streets and undermine
the American-led occupation. But sectarian violence took on its own
inevitable momentum, hastening the country's slide into civil war.

Saudi Arabia sought to match Iran's reach but, after years of oppressing
its own Shiite population,
\href{https://books.google.com/books?id=2uERDAAAQBAJ\&pg=PT144\&lpg=PT144\&dq=saudi+arabia+allawi\&source=bl\&ots=HXp1i5A9H2\&sig=X5vklAl_0daKB_-1PeuHw-W9F_4\&hl=en\&sa=X\&ved=0ahUKEwjy3NO9_vbPAhVILSYKHeANDv8Q6AEIQjAI\#v=onepage\&q=saudi\%20arabia\%20allawi\&f=false}{struggled}
to make inroads with those in Iraq.

``The problem for the Saudis is that their natural allies in Iraq,'' Dr.
Gause said, referring to Sunni groups that were increasingly turning to
jihadism, ``wanted to kill them.''

This was the first sign that Saudi Arabia's strategy for containing
Iran, by fostering sectarianism and aligning itself with the region's
Sunni majority, had backfired. As Sunni governments collapsed and Sunni
militias turned to jihadism, Riyadh would be left with few reliable
proxies.

As their competition in Iraq heated up, Saudi Arabia and Iran sought to
counterbalance each other through another weak state: Lebanon.

\hypertarget{2005-10-a-new-kind-of-proxy-war}{%
\subsection{2005-10: A new kind of proxy
war}\label{2005-10-a-new-kind-of-proxy-war}}

Lebanon provided the perfect opening: a frail democracy recovering from
civil war, with parties and lingering militias primarily organized by
religion.

Iran and Saudi Arabia exploited those dynamics, waging a new kind of
proxy struggle ``not on conventional military battlefields,'' Dr. Gause
said, but ``within the domestic politics of weakened institutional
structures.''

Iran, for instance, supported Hezbollah, the Shiite militia and
political movement, which it had earlier cultivated to use against
Israel. Riyadh, in turn, funneled money to
\href{http://www.reuters.com/article/us-saudi-lebanon-idUSKCN0X20DG}{political
allies} such as the Sunni prime minister, Rafik Hariri.

By competing along Lebanon's religious lines, they helped drive the
Lebanese government's
\href{https://books.google.com/books?id=TMjmDAAAQBAJ\&pg=PT177\&lpg=PT177\&dq=1990s+saudi+lebanon\&source=bl\&ots=xuVUuq5llQ\&sig=Gh9WwgWd9as9O30u-vVSq4Sgd54\&hl=en\&sa=X\&ved=0ahUKEwi9raSw6_bPAhXL4yYKHfNeBI0Q6AEINjAD\#v=onepage\&q=1990s\%20saudi\%20lebanon\&f=false}{frequent
breakdowns}, as parties relied on foreign backers who wanted to oppose
one another more than build a functioning state.

With Iran promoting Hezbollah as the nation's defender and Saudi Arabia
backing the Lebanese military, neither had a full mandate, and Lebanon
struggled to maintain order.

As the foreign powers escalated their antagonism, Lebanon's dysfunction
spiraled into violence. In 2005, after Mr. Hariri called for the
withdrawal of Iranian-backed Syrian troops, he was assassinated.
(Hezbollah has
\href{http://www.nytimes.com/2015/02/15/magazine/the-hezbollah-connection.html}{long
been suspected}.)

Another political crisis, in 2008,
\href{http://www.nytimes.com/2008/05/10/world/middleeast/10lebanon.html}{culminated}
with Hezbollah overpowering Sunni militias to seize much of Beirut.
Saudi Arabia
\href{https://www.theguardian.com/world/2010/dec/07/wikileaks-saudi-arab-invasion-lebanon}{requested}
United States air cover, according to a WikiLeaks cable, for a Pan-Arab
force to retake the city. Though the intervention never materialized,
the episode was a dress rehearsal for the turmoil that would soon come
to the wider region.

\hypertarget{2011-14-the-implosion}{%
\subsection{2011-14: The implosion}\label{2011-14-the-implosion}}

When the Arab Spring toppled governments across the Middle East, many of
them Saudi allies, Riyadh feared that Iran would again fill the vacuums.
So it rushed to close them, at times with force. It promised
\href{https://english.alarabiya.net/articles/2012/09/19/239028.html}{billions
in aid} to Jordan,
\href{http://www.ft.com/cms/s/0/85c1ba8c-a4df-11e1-9908-00144feabdc0.html\#axzz3buob0jrp}{Yemen},
Egypt and others, often urging those governments to crack down.

After pro-democracy protesters rose up in Bahrain, a Saudi ally whose
Sunni king rules over a majority Shiite population, Saudi Arabia
\href{http://www.nytimes.com/2011/03/15/world/middleeast/15bahrain.html}{sent
1,200 troops}.

In Egypt, Saudi Arabia tacitly supported a 2013 military takeover,
seeing the military as a more reliable ally than the elected Islamist
government it replaced. As Libya fell into civil war, it
\href{http://www.nytimes.com/2014/10/16/world/middleeast/general-escalates-libya-attack-.html}{backed}
a hard-line general who was driving to consolidate control.

Though Iran has little influence in either country, Saudi Arabia's fear
of losing ground to Iran made it fight harder to retain influence
wherever it could, analysts believe.

Syria, an Iranian ally, reversed the usual dynamic. Saudi Arabia and
other oil-rich Sunni states steered
\href{https://www.ft.com/content/f2d9bbc8-bdbc-11e2-890a-00144feab7de}{money
and arms} to rebels, including Sunni Islamists. Iran intervened in turn,
sending officers and later Hezbollah to fight on behalf of Syria's
government, whose leaders mostly follow a sect of Shiism.

Their
interventions,\href{http://www.nytimes.com/2016/08/27/world/middleeast/syria-civil-war-why-get-worse.html?rref=collection\%2Fcolumn\%2Fthe-interpreter\&action=click\&contentCollection=world\&region=stream\&module=stream_unit\&version=latest\&contentPlacement=8\&pgtype=collection}{civil
war scholars say}, helped lock Syria in the ever-worsening stalemate
that has killed over 400,000.

\hypertarget{2015-16-what-is-wrong-with-you-people}{%
\subsection{2015-16: `What is wrong with you
people?'}\label{2015-16-what-is-wrong-with-you-people}}

The United States has struggled to restore the region's balance.

President Obama has urged Iran and Saudi Arabia ``to find an effective
way to share the neighborhood and institute some sort of cold peace,''
he
\href{http://www.theatlantic.com/magazine/archive/2016/04/the-obama-doctrine/471525/}{told
The Atlantic}.

But Dr. Lynch called this plan for ``a self-regulating equilibrium''
between the Mideast powers ``far-fetched.''

The nuclear agreement with Iran, instead of calming Saudi nerves, hit on
fears that ``the United States wants to abandon them in order to ally
with Iran,'' Dr. Lynch said, calling the belief ``crazy'' but
widespread.

Mr. Pollack said he often heard Sunni Arab leaders express this as a
metaphor.

``They would say, `What is wrong with you people? You have this good,
loving, loyal wife in us, and this crazy mistress in Iran. You don't
understand how bad she is for you, and yet you endlessly run off to her
the moment that she winks at you,''' he recounted.

The White House looked for other ways to reassure Saudi leaders,
\href{http://www.reuters.com/article/us-airshow-britain-usa-arms-idUSKCN0ZT0ZH}{facilitating}
arms sales and overlooking Saudi actions in Egypt and Bahrain.

Then came Yemen. A rebel group with loose ties to Iran ousted the
Saudi-backed president, deepening Riyadh's fears. Saudi Arabia launched
a bombing campaign that inflicted horror on civilians but accomplished
little else.

The assault receives
\href{http://www.wsj.com/articles/u-s-widens-role-in-saudi-led-campaign-against-yemen-rebels-1428882967}{heavy
American support}, though the United States has few interests in Yemen
other than counterterrorism and sometimes
\href{http://www.al-monitor.com/pulse/originals/2016/10/yemen-war-us-attacks-houthis-saudi-arabia.html}{criticizes}
the campaign. In exchange, Riyadh acquiesced to the Iran deal and began
to follow Washington's lead on Syria. But the underlying proxy war
remained.

\hypertarget{a-future-of-failed-and-failing-states}{%
\subsection{A future of `failed and failing'
states}\label{a-future-of-failed-and-failing-states}}

Asked when the Iran-Saudi struggle might cool, Mr. Pollack said he
doubted that it would: ``Where we're headed with the Middle East is the
current trend extrapolated, with more failed and failing governments.''

In Yemen, this is already ``reorganizing Yemeni society along sectarian
lines and rearranging people's relationships to one another on a
non-nationalist basis,''
\href{http://carnegie-mec.org/diwan/62375?lang=en}{Farea al-Muslimi}, an
analyst, wrote in a Carnegie Endowment paper, which cited similar trends
across the region.

Continued crises will risk sucking in the United States again, Mr. Lynch
said, adding that no American president was likely to persuade Saudi
Arabia or Iran to stay out of regional conflicts that it saw as
potentially existential threats.

President-elect Donald J. Trump will enter office having echoed Saudi
Arabia's view of the region.

Iran ``took over Iraq,'' he said at a rally in January. ``They're going
to have Yemen. They're going to have Syria. They're going to have
everything.''

Mentioning both the president-elect and Hillary Clinton, Dr. Gause said
he doubted that any administration could reset the Middle East's power
struggles.

``I do not think that the fundamental problem of the region,'' he said,
``is something that either Mr. Trump or Mrs. Clinton could do that much
about.''

Advertisement

\protect\hyperlink{after-bottom}{Continue reading the main story}

\hypertarget{site-index}{%
\subsection{Site Index}\label{site-index}}

\hypertarget{site-information-navigation}{%
\subsection{Site Information
Navigation}\label{site-information-navigation}}

\begin{itemize}
\tightlist
\item
  \href{https://help.nytimes.com/hc/en-us/articles/115014792127-Copyright-notice}{©~2020~The
  New York Times Company}
\end{itemize}

\begin{itemize}
\tightlist
\item
  \href{https://www.nytco.com/}{NYTCo}
\item
  \href{https://help.nytimes.com/hc/en-us/articles/115015385887-Contact-Us}{Contact
  Us}
\item
  \href{https://www.nytco.com/careers/}{Work with us}
\item
  \href{https://nytmediakit.com/}{Advertise}
\item
  \href{http://www.tbrandstudio.com/}{T Brand Studio}
\item
  \href{https://www.nytimes.com/privacy/cookie-policy\#how-do-i-manage-trackers}{Your
  Ad Choices}
\item
  \href{https://www.nytimes.com/privacy}{Privacy}
\item
  \href{https://help.nytimes.com/hc/en-us/articles/115014893428-Terms-of-service}{Terms
  of Service}
\item
  \href{https://help.nytimes.com/hc/en-us/articles/115014893968-Terms-of-sale}{Terms
  of Sale}
\item
  \href{https://spiderbites.nytimes.com}{Site Map}
\item
  \href{https://help.nytimes.com/hc/en-us}{Help}
\item
  \href{https://www.nytimes.com/subscription?campaignId=37WXW}{Subscriptions}
\end{itemize}
