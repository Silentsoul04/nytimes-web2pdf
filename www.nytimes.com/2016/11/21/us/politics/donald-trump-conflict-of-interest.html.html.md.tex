Sections

SEARCH

\protect\hyperlink{site-content}{Skip to
content}\protect\hyperlink{site-index}{Skip to site index}

\href{https://www.nytimes.com/section/politics}{Politics}

\href{https://myaccount.nytimes.com/auth/login?response_type=cookie\&client_id=vi}{}

\href{https://www.nytimes.com/section/todayspaper}{Today's Paper}

\href{/section/politics}{Politics}\textbar{}Donald Trump's Business
Dealings Test a Constitutional Limit

\url{https://nyti.ms/2fjCidS}

\begin{itemize}
\item
\item
\item
\item
\item
\end{itemize}

Advertisement

\protect\hyperlink{after-top}{Continue reading the main story}

Supported by

\protect\hyperlink{after-sponsor}{Continue reading the main story}

\hypertarget{donald-trumps-business-dealings-test-a-constitutional-limit}{%
\section{Donald Trump's Business Dealings Test a Constitutional
Limit}\label{donald-trumps-business-dealings-test-a-constitutional-limit}}

\includegraphics{https://static01.nyt.com/images/2016/11/22/us/22legal-2/22legal-2-articleLarge.jpg?quality=75\&auto=webp\&disable=upscale}

By \href{http://www.nytimes.com/by/adam-liptak}{Adam Liptak}

\begin{itemize}
\item
  Nov. 21, 2016
\item
  \begin{itemize}
  \item
  \item
  \item
  \item
  \item
  \end{itemize}
\end{itemize}

WASHINGTON --- Not long after he took office, President Obama sought
advice from the Justice Department about a potential conflict of
interest involving a foreign government. He wanted to know whether he
could accept the Nobel Peace Prize.

The answer turned on the
\href{http://www.heritage.org/constitution/\#!/articles/1/essays/68/emoluments-clause}{Emoluments
Clause}, an obscure provision of the Constitution that now poses risks
for President-elect Donald J. Trump should he continue to reap benefits
from transactions with companies controlled by foreign governments.

``Emolument'' means compensation for labor or services. And the clause
says that ``no person holding any office of profit or trust'' shall
``accept of any present, emolument, office or title, of any kind
whatever, from any king, prince or foreign state'' unless Congress
consents.

It took \href{http://www.ca1.uscourts.gov/david-j-barron}{David J.
Barron}, a Justice Department official who is now a federal appeals
court judge in Boston,
\href{https://www.justice.gov/sites/default/files/olc/opinions/2009/12/31/emoluments-nobel-peace.pdf}{13
single-spaced pages} to answer Mr. Obama's question.

Two things were clear, he wrote. The Emoluments Clause ``surely''
applied to the president, and the prize, which included a check for
about \$1.4 million, was the sort of thing that would be barred if it
came from a foreign state. In the end, however, Mr. Barron concluded
that Mr. Obama could accept the prize because the committee that chose
him was independent of the Norwegian government and the prize itself was
privately financed.

But he said that the answer would be different if a foreign government
sought to make a payment to a sitting president. In a footnote, Mr.
Barron added, ``Corporations owned or controlled by a foreign government
are presumptively foreign states under the Emoluments Clause.''

Mr. Trump's companies do business with entities controlled by foreign
governments and people with ties to them. The ventures include
multimillion-dollar real estate arrangements --- with Mr. Trump's
companies either as a full owner or a ``branding'' partner --- in
Ireland and Uruguay. The Bank of China is a
\href{http://www.nytimes.com/2016/11/15/us/politics/donald-trump-holdings-conflict-of-interest.html}{tenant
in Trump Tower} and a
\href{http://www.nytimes.com/2016/08/21/us/politics/donald-trump-debt.html}{lender
for another building} in Midtown Manhattan where Mr. Trump has a
significant partnership interest.

\includegraphics{https://static01.nyt.com/images/2016/11/22/us/22legal/22legal-articleLarge.jpg?quality=75\&auto=webp\&disable=upscale}

Experts in legal ethics say those kinds of arrangements could easily run
afoul of the Emoluments Clause if they continue after Mr. Trump takes
office. ``The founders very clearly intended that officers of the United
States, including the president, not accept presents from foreign
sovereigns,'' said
\href{https://www.brookings.edu/experts/norman-eisen/}{Norman Eisen},
who was the chief White House ethics lawyer for Mr. Obama from 2009 to
2011.

``Whenever Mr. Trump receives anything from a foreign sovereign, to the
extent that it's not an arm's-length transaction,'' Mr. Eisen said,
``every dollar in excess that they pay over the fair market price will
be a dollar paid in violation of the Emoluments Clause and will be a
present to Mr. Trump.''

The Supreme Court has never squarely considered the scope of the clause,
and there are no historical analogies to help understand how it should
apply to a president who owns a sprawling international business empire.
Earlier presidents worked hard to avoid even the appearance of a
conflict of interest involving a foreign power, said
\href{https://www.fordham.edu/info/23186/zephyr_teachout}{Zephyr
Teachout}, a law professor at Fordham who ran for Congress in New York
this year as a Democrat and lost.

``The reason we don't really have a lot of precedent here is that
presidents in the past have gone out of their way to avoid getting even
close to the Emoluments Clause,'' she said.

But if Mr. Trump takes a different approach, it is not clear that anyone
would have standing to challenge him in court.

``There are a lot of very smart lawyers turning that question over in
their minds today,'' Mr. Eisen said, adding that a business competitor
injured by foreign favoritism toward a Trump company might have
standing.

But \href{https://www.law.umn.edu/profiles/richard-w-painter}{Richard W.
Painter}, who was the chief White House ethics lawyer for President
George W. Bush from 2005 to 2007, said such a business most likely would
not have standing to sue.

``It's not there to protect a competitor,'' he said of the clause.
``It's there to protect the United States government.''

The way to address violations of the clause, Mr. Painter said, is not a
lawsuit but impeachment.

Lawmakers could take steps short of impeachment, particularly because
the clause itself describes a role for Congress, which can give its
consent to payments that would otherwise be barred. Mr. Painter said
Congress should embrace that role by passing a resolution directed at
Mr. Trump.

\href{https://www.nytimes.com/interactive/2016/11/21/us/politics/what-trump-wants-to-do-in-his-first-100-days-and-how-difficult-each-will-be.html}{}

\includegraphics{https://static01.nyt.com/images/2016/11/21/us/politics/what-trump-wants-to-do-in-his-first-100-days-and-how-difficult-each-will-be-1479737914360/what-trump-wants-to-do-in-his-first-100-days-and-how-difficult-each-will-be-1479737914360-thumbLarge-v2.png}

\hypertarget{how-hard-or-easy-it-will-be-for-trump-to-fulfill-his-100-day-plan}{%
\subsection{How Hard (or Easy) It Will Be for Trump to Fulfill His
100-Day
Plan}\label{how-hard-or-easy-it-will-be-for-trump-to-fulfill-his-100-day-plan}}

He can accomplish some of his promises entirely on his own, but others
require Congress or pose other significant obstacles.

``It should send a clear message to him that he should divest his
assets, and that they will regard dealings with his companies that he
owns abroad and any entities owned by foreign governments as a potential
violation of the Emoluments Clause unless he can prove it was an
arm's-length transaction,'' he said.

Professor Teachout agreed that Congress had ``an institutional,
constitutional obligation to make sure that Trump isn't violating this
clause.''

``You would think the responsible action --- Republican, Democrat,
whatever,'' she said, ``would be for Congress to say, `We want to make
sure that there isn't a violation of this clause, and in order to do so,
we need to look at the transactions to make sure they're fair market
transactions instead of gifts.' ''

Not everyone agrees that the clause covers the president.
\href{https://www.maynoothuniversity.ie/people/seth-barrett-tillman}{Seth
Barrett Tillman}, a lecturer at the Maynooth University Department of
Law in Ireland, noted that
\href{http://press-pubs.uchicago.edu/founders/tocs/a1_2_5.html}{a
different clause of the Constitution}, which makes bribery an
impeachable offense, specifically mentions the ``president, vice
president and all civil officers of the United States.'' The different
language in the Emoluments Clause, along with historical evidence, he
said, indicates that it does not apply to the president.

``That isn't to say that we shouldn't be concerned as a policy matter
with Trump,'' Mr. Tillman said. ``I just want to see the conversation
moved away from constitutionalizing what should be an argument about
good governance.''

\href{http://hls.harvard.edu/faculty/directory/10899/Tribe}{Laurence H.
Tribe}, a law professor at Harvard, said that he found Mr. Tillman's
argument ``singularly unpersuasive'' and that it ``would pose grave
danger to the republic, especially in the case of a president with
extensive global holdings that he seems bent on having his own children
manage even after he assumes office.''

In 2011, \href{https://www.bu.edu/law/profile/jay-d-wexler/}{Jay D.
Wexler}, a law professor at Boston University, published ``The Odd
Clauses,'' a book about the Constitution's more obscure provisions. He
said such obscurity could be impermanent, as the recent attention to to
Emoluments Clause demonstrates.

``I've seen over and over how parts of the Constitution that were
considered vestigial or irrelevant for decades or more can suddenly
resurface and take on enormous importance with a quick change of
events,'' Professor Wexler said.

``The framers were prescient men who created a government that could
withstand the worst of human foibles --- corruption, vindictiveness, the
thirst for tyranny --- and wrote a Constitution to combat those foibles
in many of their forms, not all of which will always be present, but
which emerge in different guises in different eras,'' he said.

Advertisement

\protect\hyperlink{after-bottom}{Continue reading the main story}

\hypertarget{site-index}{%
\subsection{Site Index}\label{site-index}}

\hypertarget{site-information-navigation}{%
\subsection{Site Information
Navigation}\label{site-information-navigation}}

\begin{itemize}
\tightlist
\item
  \href{https://help.nytimes.com/hc/en-us/articles/115014792127-Copyright-notice}{©~2020~The
  New York Times Company}
\end{itemize}

\begin{itemize}
\tightlist
\item
  \href{https://www.nytco.com/}{NYTCo}
\item
  \href{https://help.nytimes.com/hc/en-us/articles/115015385887-Contact-Us}{Contact
  Us}
\item
  \href{https://www.nytco.com/careers/}{Work with us}
\item
  \href{https://nytmediakit.com/}{Advertise}
\item
  \href{http://www.tbrandstudio.com/}{T Brand Studio}
\item
  \href{https://www.nytimes.com/privacy/cookie-policy\#how-do-i-manage-trackers}{Your
  Ad Choices}
\item
  \href{https://www.nytimes.com/privacy}{Privacy}
\item
  \href{https://help.nytimes.com/hc/en-us/articles/115014893428-Terms-of-service}{Terms
  of Service}
\item
  \href{https://help.nytimes.com/hc/en-us/articles/115014893968-Terms-of-sale}{Terms
  of Sale}
\item
  \href{https://spiderbites.nytimes.com}{Site Map}
\item
  \href{https://help.nytimes.com/hc/en-us}{Help}
\item
  \href{https://www.nytimes.com/subscription?campaignId=37WXW}{Subscriptions}
\end{itemize}
