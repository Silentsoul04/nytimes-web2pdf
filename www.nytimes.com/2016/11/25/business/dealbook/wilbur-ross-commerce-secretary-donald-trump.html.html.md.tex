Sections

SEARCH

\protect\hyperlink{site-content}{Skip to
content}\protect\hyperlink{site-index}{Skip to site index}

\href{https://myaccount.nytimes.com/auth/login?response_type=cookie\&client_id=vi}{}

\href{https://www.nytimes.com/section/todayspaper}{Today's Paper}

\href{/section/business/dealbook}{DealBook}\textbar{}`Vulture' or
`Phoenix'? Wilbur Ross, Risk-Taker, Is Eyed for Commerce Post

\url{https://nyti.ms/2gpUIKG}

\begin{itemize}
\item
\item
\item
\item
\item
\end{itemize}

Advertisement

\protect\hyperlink{after-top}{Continue reading the main story}

Supported by

\protect\hyperlink{after-sponsor}{Continue reading the main story}

DealBook Business and Policy

\hypertarget{vulture-or-phoenix-wilbur-ross-risk-taker-is-eyed-for-commerce-post}{%
\section{`Vulture' or `Phoenix'? Wilbur Ross, Risk-Taker, Is Eyed for
Commerce
Post}\label{vulture-or-phoenix-wilbur-ross-risk-taker-is-eyed-for-commerce-post}}

\includegraphics{https://static01.nyt.com/images/2016/11/26/business/26ROSS/26ROSS-articleLarge.jpg?quality=75\&auto=webp\&disable=upscale}

By \href{http://www.nytimes.com/by/matthew-goldstein}{Matthew Goldstein}

\begin{itemize}
\item
  Nov. 25, 2016
\item
  \begin{itemize}
  \item
  \item
  \item
  \item
  \item
  \end{itemize}
\end{itemize}

Wilbur L. Ross, the billionaire investor
\href{http://www.nytimes.com/2016/11/24/us/politics/wilbur-ross-commerce-trump.html}{expected
to be nominated} as the next commerce secretary, has made his fortune
through the tricky business of buying deeply troubled companies.

With wealth estimated at \$2.9 billion, Mr. Ross, who turns 79 on
Monday, would join a cabinet that is already expected to include one of
the superwealthy in Betsy DeVos,
\href{http://www.nytimes.com/2016/11/23/us/politics/betsy-devos-trumps-education-pick-has-steered-money-from-public-schools.html}{the
nominee for secretary of education}, and that may soon have others.

In choosing Mr. Ross to be the face of American business for the rest of
the world, President-elect Donald J. Trump is turning not to a cautious
corporate chieftain, but to a risk-taking speculator. Like his
presumptive boss, Mr. Ross has been considered either a hero or a
villain during his career. There is not a lot in between.

In 2002, he won praise from workers when he bought the shuttered steel
mills of LTV, a bankrupt company in Cleveland. Four years later, Mr.
Ross was pilloried after an explosion at the Sago Mine in West Virginia,
which his company had bought a few weeks earlier,
\href{http://www.nytimes.com/2006/05/03/us/03mine.html}{killed 12
miners}.

The stark contrasts reflect the nature of the world in which Mr. Ross
operates: distressed investing. He made his name scouring the landscape
for businesses left for dead that he could sink money into and then
profit from when they were resurrected.

It is a business that requires nerves of steel and a strong stomach. The
chances of failure --- along with headlines about collapsed businesses
and lost jobs --- are balanced against the opportunity for a big reward
if a turnaround strategy works.

Some businesses where he has invested, like textile mills, have
struggled. But other investments have salvaged industry and jobs while
providing a lucrative payday.

His best-known bet was in the steel industry more than a decade ago, a
time when few wanted anything to do with it. Mr. Ross cobbled together
the ailing assets of LTV and Bethlehem Steel into a new company called
International Steel Group, which was
\href{http://www.nytimes.com/2004/10/26/business/worldbusiness/mergers-show-steel-industry-is-still-worthy-of-big.html}{sold
in 2004 to Mittal Steel for \$4.5 billion}.

``There were no outsiders who were willing to step forward and make an
investment,'' said Ron Bloom, an investment banker who negotiated with
Mr. Ross for the United Steelworkers. ``He went where angels feared to
tread.''

Like his investing, the politics of Mr. Ross, a former Democrat, do not
always stick to orthodox points of view.

Mr. Ross has expressed strong conservative beliefs on some issues ---
favoring big tax cuts for businesses, for example, and a repeal of
President Obama's health law. Yet Mr. Ross has also suggested that he is
receptive to some of the anti-trade views favored by American labor
unions and by Mr. Trump.

``The president has a huge amount of fire in terms of abrogating
treaties, and he can do a lot without reference to Congress,'' Mr. Ross
said in an interview the day after the election.

Mr. Ross, a member of Mr. Trump's economic team during the campaign,
said he expected the new president to do a lot on trade and regulation
through executive action.

``He is serious about suspending any new regulations,'' said Mr. Ross,
who held one of Mr. Trump's first fund-raisers.

A spokesman for Mr. Trump, Jason Miller, said: ``Though President-elect
Trump has not yet announced his pick for this position, it goes without
saying that Mr. Ross has been a fantastic advocate for the
president-elect's plan to bring back jobs, eliminate the trade deficit
and make good deals for America's workers.''

The nomination of Mr. Ross would be the capstone to a career on Wall
Street that has spanned decades and has made him one of the most visible
and successful of a breed of investor known as ``vultures'' because of
their penchant for going after nearly dead businesses.

Mr. Ross, by contrast, has often preferred to see himself as another
kind of bird --- the mythical phoenix, helping businesses rise from the
ashes.

To some degree, Mr. Ross helped Mr. Trump do that when some of his
casinos in Atlantic City fell on hard times. Mr. Ross and Carl C. Icahn,
another billionaire investor and supporter of Mr. Trump, were both
bondholders in the Trump Taj Mahal casino when it was teetering on
financial collapse in 1990. Instead of pushing the casino into an
immediate bankruptcy, Mr. Ross and Mr. Icahn worked with Mr. Trump and
others to structure a more orderly bankruptcy filing in 1991.

The negotiated restructuring helped Mr. Trump salvage his name and brand
at a time when he arguably did not have many friends on Wall Street.

The low point of Mr. Ross's career was the deadly mine disaster in West
Virginia. Although he helped set up a charitable fund for the families
of the victims, and his company contributed more than \$1 million to
them as well, some in organized labor remain bitter about Mr. Ross and
his firm.

Still, that episode has not deterred some unions from doing business
with him when it was in their interest. In 2012, Mr. Ross was one of the
wealthy investors who gave a
\href{http://dealbook.nytimes.com/2014/08/01/at-union-owned-amalgamated-bank-new-chief-charts-a-progressive-course/}{\$50
million cash infusion to Amalgamated Bank}, one of the nation's largest
union-owned lenders, which was struggling to stay in business after the
financial crisis.

An affiliate of the Service Employees International Union, which
controlled the ailing Amalgamated Bank, did not hesitate to take Mr.
Ross's money.

``Wilbur Ross's investment firm has been an investor in the bank and
held a seat on our board for nearly five years,'' said Loren
Riegelhaupt, a spokesman for Amalgamated Bank. ``While we appreciate his
financial acumen, his relationship to the bank has never impacted our
core progressive principles: providing quality financial services to our
clients while advancing the values we believe in.''

Mr. Ross sold his firm in 2006 to Invesco, an Atlanta-based investment
company, for about \$375 million. Since then, he has pulled back on the
daily operations of the business.

While remaining as chairman of the firm, Mr. Ross has spent more time in
recent years at his Palm Beach, Fla., home, not far from Mr. Trump's
Mar-a-Largo estate.

Mr. Ross was born in Weehawken, N.J., and has been married three times.
His second wife, Betsy McCaughey, a Republican, served as New York's
lieutenant governor from 1995 to 1998. The experience, Mr. Ross would
later tell New York magazine, ``gave one a very close-up view of
politics.'' Ms. McCaughey, too, was named to Mr. Trump's economic team
during the campaign.

Some suggest Mr. Ross's business ties may pose potential conflicts of
interest. But the sale of his company, W.L. Ross, may make it easier for
Mr. Ross to separate himself from its far-flung interests, which include
businesses in Europe, China and India. His remaining financial interests
in the firm's funds could be put in a blind trust, and he could easily
resign from the five corporate boards on which he sits.

Still, those overseas deals could raise questions about his
relationships with foreign leaders and businesspeople from China and
Russia.

He is vice chairman of the Bank of Cyprus, the biggest bank in that
European island nation, and he is credited with helping the bank to
recover from a severe crisis in 2013. But Mr. Ross's investment in the
bank also makes him a de facto business partner with Viktor F.
Vekselberg, one of Russia's most prominent businesspeople and a man with
ties to the Kremlin.

Mr. Ross has complained about China's having taken jobs from Americans
--- a message similar to the one Mr. Trump repeated throughout his
campaign. Yet for all the anti-China commentary, Mr. Ross has been a
frequent visitor in the past two decades and has made inroads in that
country's energy industry.

Mr. Ross linked up with the most powerful player in the country's power
generation business, China Huaneng Group, in 2008. The state-owned
company had been run for years by the eldest son of Li Peng, the former
prime minister who was the godfather of the country's electricity
industry.

Advertisement

\protect\hyperlink{after-bottom}{Continue reading the main story}

\hypertarget{site-index}{%
\subsection{Site Index}\label{site-index}}

\hypertarget{site-information-navigation}{%
\subsection{Site Information
Navigation}\label{site-information-navigation}}

\begin{itemize}
\tightlist
\item
  \href{https://help.nytimes.com/hc/en-us/articles/115014792127-Copyright-notice}{©~2020~The
  New York Times Company}
\end{itemize}

\begin{itemize}
\tightlist
\item
  \href{https://www.nytco.com/}{NYTCo}
\item
  \href{https://help.nytimes.com/hc/en-us/articles/115015385887-Contact-Us}{Contact
  Us}
\item
  \href{https://www.nytco.com/careers/}{Work with us}
\item
  \href{https://nytmediakit.com/}{Advertise}
\item
  \href{http://www.tbrandstudio.com/}{T Brand Studio}
\item
  \href{https://www.nytimes.com/privacy/cookie-policy\#how-do-i-manage-trackers}{Your
  Ad Choices}
\item
  \href{https://www.nytimes.com/privacy}{Privacy}
\item
  \href{https://help.nytimes.com/hc/en-us/articles/115014893428-Terms-of-service}{Terms
  of Service}
\item
  \href{https://help.nytimes.com/hc/en-us/articles/115014893968-Terms-of-sale}{Terms
  of Sale}
\item
  \href{https://spiderbites.nytimes.com}{Site Map}
\item
  \href{https://help.nytimes.com/hc/en-us}{Help}
\item
  \href{https://www.nytimes.com/subscription?campaignId=37WXW}{Subscriptions}
\end{itemize}
