Sections

SEARCH

\protect\hyperlink{site-content}{Skip to
content}\protect\hyperlink{site-index}{Skip to site index}

\href{https://myaccount.nytimes.com/auth/login?response_type=cookie\&client_id=vi}{}

\href{https://www.nytimes.com/section/todayspaper}{Today's Paper}

\href{/section/business/dealbook}{DealBook}\textbar{}Trump's Economic
Cabinet Picks Signal Embrace of Wall St. Elite

\url{https://nyti.ms/2gMB0sN}

\begin{itemize}
\item
\item
\item
\item
\item
\item
\end{itemize}

Advertisement

\protect\hyperlink{after-top}{Continue reading the main story}

Supported by

\protect\hyperlink{after-sponsor}{Continue reading the main story}

DealBook Business and Policy

\hypertarget{trumps-economic-cabinet-picks-signal-embrace-of-wall-st-elite}{%
\section{Trump's Economic Cabinet Picks Signal Embrace of Wall St.
Elite}\label{trumps-economic-cabinet-picks-signal-embrace-of-wall-st-elite}}

\includegraphics{https://static01.nyt.com/images/2016/11/30/business/cnbc-mnuchin/cnbc-mnuchin-videoSixteenByNineJumbo1600.png}

By \href{http://www.nytimes.com/by/landon-thomas-jr}{Landon Thomas Jr.}
and \href{http://www.nytimes.com/by/alexandra-stevenson}{Alexandra
Stevenson}

\begin{itemize}
\item
  Nov. 30, 2016
\item
  \begin{itemize}
  \item
  \item
  \item
  \item
  \item
  \item
  \end{itemize}
\end{itemize}

In a \href{https://www.youtube.com/watch?v=vST61W4bGm8}{campaign
commercial} that ran just before the election, Donald J. Trump's voice
boomed over a series of Wall Street images. He described ``a global
power structure that is responsible for the economic decisions that have
robbed our working class, stripped our country of its wealth, and put
that money into the pockets of a handful of large corporations.''

The New York Stock Exchange, the hedge fund billionaire George Soros and
the chief executive of the investment bank Goldman Sachs flashed across
the screen.

Now Mr. Trump has named a former Goldman executive and co-investor with
Mr. Soros to spearhead his economic policy.

With Wednesday's nomination of Steven Mnuchin, a Goldman trader turned
hedge fund manager and Hollywood financier, to be Treasury secretary, a
new economic leadership is taking shape in Washington.

Mr. Mnuchin will join
\href{http://www.nytimes.com/2016/11/25/business/dealbook/wilbur-ross-commerce-secretary-donald-trump.html}{Wilbur
L. Ross Jr., a billionaire investor in distressed assets}, who has been
chosen to run the Commerce Department, and Todd Ricketts, owner of the
Chicago Cubs, who has been picked to be deputy commerce secretary. All
are superwealthy and to be overseen by the first billionaire president
in United States history.

That two investors --- Mr. Mnuchin and Mr. Ross --- will occupy two
major economic positions in the new administration is the most powerful
signal yet that Mr. Trump plans to emphasize policies friendly to Wall
Street, like tax cuts and a relaxation of regulation, in the early days
of his administration.

While that approach has been cheered by investors (the stocks of Bank of
America, Goldman Sachs and Morgan Stanley have been on a tear since the
election), it stands in stark contrast to the populist campaign that Mr.
Trump ran and the support he received from working-class voters across
the country.

Anthony Scaramucci, a hedge fund executive and member of the Trump
transition team, insisted on Wednesday that appointing wealthy investors
did not contradict the campaign's populist message.

\includegraphics{https://static01.nyt.com/images/2016/12/01/us/01MNUCHEN1/01MNUCHEN1-articleLarge.jpg?quality=75\&auto=webp\&disable=upscale}

``The working-class people of the United States, they need a break,''
Mr. Scaramucci said. ``And we need to switch them from going from the
working class into the working poor into what I call the aspirational
working class, which my dad was a member of.''

Still, Democrats were quick to attack the latest nomination.

``Steve Mnuchin is just another Wall Street insider,'' Senator Bernie
Sanders of Vermont and Senator Elizabeth Warren of Massachusetts said in
a joint statement. ``That is not the type of change that Donald Trump
promised to bring to Washington --- that is hypocrisy at its worst.''

So far, none of the nominees who will be shaping economic policy have
any significant experience in government.

Mr. Mnuchin, 53, and Mr. Ross, 79, are both familiar with buying
distressed properties and selling for a profit. But they are political
neophytes with scant experience in managing large organizations. They
will oversee two government agencies that together employ about 130,000
people around the world.

In the case of Mr. Mnuchin at Treasury, his experience as a principal
investor who made large sums of money through high-risk, high-return
wagers suggests that he will look critically at the thicket of
regulations that now constrain the risk-taking activities of investment
banks.

That could mean a reassessment of what has come to be known as the
Volcker Rule, part of the Dodd-Frank financial overhaul that followed
the 2008 financial crisis. The rule forbids banks to make certain
speculative investments with their own capital.

``I would say the No. 1 problem with the Volcker Rule is it's too
complicated and people don't know how to interpret it,'' Mr. Mnuchin
said in an interview with CNBC on Wednesday. ``So we're going to look at
what to do with it as we are with all of Dodd-Frank. The No. 1 priority
is going to be to make sure that banks lend.''

In the interview, Mr. Mnuchin also said he would look to cutting
corporate tax rates as a way to increase economic growth. And he said
the wealthy would not see a big tax cut.

Image

Wilbur Ross was tapped for the Commerce Department.Credit...Sam Hodgson
for The New York Times

``Any reductions we have in upper-income taxes will be offset by less
deductions so that there will be no absolute tax cut for the upper
class,'' Mr. Mnuchin said in the
\href{http://www.cnbc.com/2016/11/30/exclusive-trumps-treasury-pick-says-i-want-to-slash-taxes-across-the-board.html}{interview}.
``There will be a big tax cut for the middle class, but any tax cuts we
have for the upper class will be offset by less deductions that pay for
it.''

There is a Washington tradition of presidents calling on a Goldman Sachs
luminary to take the reins of the economy, including the Democrat Robert
E. Rubin in 1995 and Henry M. Paulson Jr., a Republican, in 2006.

Mr. Mnuchin's Goldman pedigree is as good as it gets, given that his
father, Robert, was a pioneer in stock trading who spent 35 years at the
firm.

While the Goldman brand may have initially attracted Mr. Trump, for the
broader financial community it is Mr. Mnuchin's track record at hedge
and private equity funds, which is where the real money is made on Wall
Street these days, that makes him appealing.

``Mnuchin as Treasury secretary is somebody who can speak to bankers ---
Jamie Dimon, Lloyd Blankfein, James Gorman and Brian Moynihan. He can
speak their language,'' said Gary Kaminsky, a former vice chairman at
Morgan Stanley, referring to the chief executives of JPMorgan Chase,
Goldman, Morgan Stanley and Bank of America. ``He comes from a trading
desk, and that's something that is very strong,'' Mr. Kaminsky, who has
attended fund-raisers for Mr. Trump, added.

While there is little doubt that Mr. Mnuchin can speak the language of
Wall Street, he has had little experience running large, complex
bureaucracies. Mr. Rubin and Mr. Paulson had ascended to the top at
Goldman, and had many years of experience managing people and
organizations under their belt.

Mr. Mnuchin did assume a leading role in the restructuring and
reinventing of IndyMac, now known as OneWest, a California mortgage
giant that collapsed in 2008. He and partners acquired the firm and
later made billions.

After he moved from New York to the West Coast, Mr. Mnuchin was targeted
by protesters who claimed that the bank was too quick to foreclose on
struggling homeowners. Last year the bank was sold to the CIT Group, a
small-business lender run at the time by another Goldman Sachs alumnus,
John A. Thain.

Image

The president-elect with Todd Ricketts, another investor, who will be a
deputy at Commerce.Credit...Hilary Swift for The New York Times

In a statement announcing his economic appointments, Mr. Trump
highlighted the deal. ``He purchased IndyMac Bank for \$1.6 billion and
ran it very professionally, selling it for \$3.4 billion plus a return
of capital,'' he said of Mr. Mnuchin. ``That's the kind of people I want
in my administration representing our country.''

Mr. Mnuchin has faced other controversies. In 2010, he and his brother,
Alan, were sued over their mother's early investment with Bernard L.
Madoff, an investor who was convicted of running a Ponzi scheme. The
lawsuit, filed by a trustee for Madoff victims, alleged that \$3.2
million of the money Mr. Mnuchin withdrew from his mother's account
shortly after she died belonged to other victims. The lawsuit was
dropped last year because of a time limit.

Hollywood has been another reinvention for Mr. Mnuchin.

In 2006, he and a partner, Chip Seelig, struck a deal through their
company Dune Entertainment to invest \$325 million in 28 movies produced
by 20th Century Fox. It was a successful partnership; Fox delivered hits
(made in part with Dune's money) like ``Avatar,'' which took in \$2.8
billion worldwide in 2009.

After breaking with Mr. Seelig in 2012, Mr. Mnuchin teamed up with a
company called RatPac, owned by the rowdy filmmaker Brett Ratner and the
Australian billionaire James Packer. Mr. Ratner was then notorious in
Hollywood; he
\href{http://www.nytimes.com/2011/11/09/business/media/movie-figure-brett-ratner-resigns-as-oscar-co-producer.html}{resigned}
as a producer of the Academy Awards in 2011 after using an anti-gay slur
at a public event and making frank remarks about his sex life on Howard
Stern's radio show. (He
\href{http://www.hollywoodreporter.com/news/brett-ratner-gay-slur-apology-tower-heist-258316}{apologized}.)
But together the three men formed a vehicle to invest \$450 million in
an extensive array of Warner Bros. movies.

Some have been major hits, like ``Gravity,'' which took in \$723.2
million. But there have also been money losers, including ``Pan'' and
``In the Heart of the Sea.''

Still, Mr. Mnuchin has clearly enjoyed a Hollywood lifestyle, whether
attending celebrity-filled parties at the Hôtel du Cap-Eden-Roc during
the Cannes Film Festival or going in with a movie industry friend to buy
a Dassault Falcon 50 jet (since sold). He can currently be seen in a
cameo --- playing a Merrill Lynch executive --- in Warren Beatty's new
movie ``Rules Don't Apply.''

Beyond the entertainment industry, there other similarities between Mr.
Trump and Mr. Mnuchin. They are both twice divorced, and their third
partners are decidedly younger. (Mr. Mnuchin's fiancée is Louise Linton,
a 34-year-old actress from Scotland.) They also both have a taste for
landmark Manhattan real estate: Trump Tower for the president-elect and
740 Park Avenue for Mr. Mnuchin.

But there are differences, too. Despite his Hollywood appetites, Mr.
Mnuchin is described by people who know him as slightly awkward and not
one to command a room. Friends of the two men describe them more as
social and professional acquaintances than close friends.

The names of flashier prospects had been floated as possible candidates
for Treasury, according to a fund manager close to the president-elect's
economic brain trust. Among them: Henry Kravis of Kohlberg Kravis
Roberts \& Company, Jonathan Gray of Blackstone, Jamie Dimon, Mitt
Romney, and Thomas J. Barrack Jr. of Colony Capital, a Los Angeles-based
real estate investor who has been close to Mr. Trump for decades.

Advertisement

\protect\hyperlink{after-bottom}{Continue reading the main story}

\hypertarget{site-index}{%
\subsection{Site Index}\label{site-index}}

\hypertarget{site-information-navigation}{%
\subsection{Site Information
Navigation}\label{site-information-navigation}}

\begin{itemize}
\tightlist
\item
  \href{https://help.nytimes.com/hc/en-us/articles/115014792127-Copyright-notice}{©~2020~The
  New York Times Company}
\end{itemize}

\begin{itemize}
\tightlist
\item
  \href{https://www.nytco.com/}{NYTCo}
\item
  \href{https://help.nytimes.com/hc/en-us/articles/115015385887-Contact-Us}{Contact
  Us}
\item
  \href{https://www.nytco.com/careers/}{Work with us}
\item
  \href{https://nytmediakit.com/}{Advertise}
\item
  \href{http://www.tbrandstudio.com/}{T Brand Studio}
\item
  \href{https://www.nytimes.com/privacy/cookie-policy\#how-do-i-manage-trackers}{Your
  Ad Choices}
\item
  \href{https://www.nytimes.com/privacy}{Privacy}
\item
  \href{https://help.nytimes.com/hc/en-us/articles/115014893428-Terms-of-service}{Terms
  of Service}
\item
  \href{https://help.nytimes.com/hc/en-us/articles/115014893968-Terms-of-sale}{Terms
  of Sale}
\item
  \href{https://spiderbites.nytimes.com}{Site Map}
\item
  \href{https://help.nytimes.com/hc/en-us}{Help}
\item
  \href{https://www.nytimes.com/subscription?campaignId=37WXW}{Subscriptions}
\end{itemize}
