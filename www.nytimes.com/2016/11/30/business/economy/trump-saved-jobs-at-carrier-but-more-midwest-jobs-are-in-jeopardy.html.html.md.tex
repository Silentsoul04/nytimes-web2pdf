Sections

SEARCH

\protect\hyperlink{site-content}{Skip to
content}\protect\hyperlink{site-index}{Skip to site index}

\href{https://www.nytimes.com/section/business/economy}{Economy}

\href{https://myaccount.nytimes.com/auth/login?response_type=cookie\&client_id=vi}{}

\href{https://www.nytimes.com/section/todayspaper}{Today's Paper}

\href{/section/business/economy}{Economy}\textbar{}Trump Saved Jobs at
Carrier, but More Midwest Jobs Are in Jeopardy

\url{https://nyti.ms/2gmH52h}

\begin{itemize}
\item
\item
\item
\item
\item
\end{itemize}

Advertisement

\protect\hyperlink{after-top}{Continue reading the main story}

Supported by

\protect\hyperlink{after-sponsor}{Continue reading the main story}

\hypertarget{trump-saved-jobs-at-carrier-but-more-midwest-jobs-are-in-jeopardy}{%
\section{Trump Saved Jobs at Carrier, but More Midwest Jobs Are in
Jeopardy}\label{trump-saved-jobs-at-carrier-but-more-midwest-jobs-are-in-jeopardy}}

\includegraphics{https://static01.nyt.com/images/2016/12/01/us/01BLUECOLLAR/01BLUECOLLAR-articleLarge.jpg?quality=75\&auto=webp\&disable=upscale}

By \href{http://www.nytimes.com/by/nelson-d-schwartz}{Nelson D.
Schwartz}

\begin{itemize}
\item
  Nov. 30, 2016
\item
  \begin{itemize}
  \item
  \item
  \item
  \item
  \item
  \end{itemize}
\end{itemize}

In tiny Sellersburg, Ind., just across the border from Kentucky,
Manitowoc Foodservice is in the final stages of closing a factory that
makes beverage dispensers and ice machines and is laying off 84 workers.

The company is moving production to Mexico.

Just 100 miles away, President-elect Donald J. Trump will appear on
Thursday with workers at Carrier's Indianapolis plant to boast of his
success
\href{http://www.nytimes.com/2016/11/29/business/trump-to-announce-carrier-plant-will-keep-jobs-in-us.html}{in
saving at least 1,000 jobs} from moving to Mexico.

The truth across the Rust Belt is that there are more Manitowoc
Foodservices than Carriers. The layoffs and closing in Sellersburg
follow similar shutdowns by Manitowoc in Ohio and Wisconsin.

``I'll give Trump his due, but I hope he and the American people and
Congress don't forget about all these other jobs going to Mexico,'' said
Chuck Jones, the president of Local 1999 of the United Steelworkers in
Indianapolis, which represents Carrier. ``Down the pike, a lot more are
going to be moving out.''

Indeed, Rexnord, the ball bearing factory in Indianapolis where Mr.
Jones went to work straight out of high school nearly 40 years ago, said
in October it would be moving to Mexico. It is just a mile from the
Carrier plant.

The mayor of Indianapolis, Joe Hogsett, and Senator Joe Donnelly, both
Democrats, tried to exert Trumplike pressure to force Rexnord to rethink
its plans, but so far the company has not shown any sign it will change
course.

``On a personal level at Carrier, it is huge,'' said Jerry N. Conover,
director of the Indiana Business Research Center at the Kelley School of
Business at Indiana University. ``But by itself, the disappearance or
retention of 1,000 jobs is a small slice of the total economy in
Indiana.''

``I think there will be continued downward pressure on employment in
factories because of trends toward automation especially and moving to
lower-cost areas for production,'' he added.

\href{https://www.nytimes.com/interactive/2016/11/30/us/politics/trump-manufacturing-jobs-indiana-carrier.html}{}

\includegraphics{https://static01.nyt.com/images/2016/11/30/us/politics/trump-manufacturing-jobs-indiana-carrier-1480538094077/trump-manufacturing-jobs-indiana-carrier-1480538094077-articleLarge-v3.png}

\hypertarget{what-it-means-for-trump-to-save-1000-jobs-in-indiana}{%
\subsection{What It Means for Trump to Save 1,000 Jobs in
Indiana}\label{what-it-means-for-trump-to-save-1000-jobs-in-indiana}}

How the president-elect's deal measures up to U.S. manufacturing job
losses.

Carrier, in its official statement on the deal on Wednesday, said that
it thought the agreement it negotiated with Mr. Trump and Vice
President-elect Mike Pence ``benefits our workers, the state of Indiana
and our company.'' But it said that incentives provided by Indiana,
where Mr. Pence is governor, ``were an important consideration.'' It
added that ``the forces of globalization will continue to require
solutions for the long-term competitiveness of the U.S. and American
workers.''

Those 1,000 Carrier jobs saved represent just
\href{http://www.nytimes.com/interactive/2016/11/30/us/politics/trump-manufacturing-jobs-indiana-carrier.html}{0.2
percent of total manufacturing employment} in the state. And despite a
rebound since the aftermath of the Great Recession, at just over half a
million positions, factory employment in Indiana this year is still down
by more than 20 percent since 2000.

The good news is that Indiana has been doing well economically, with an
unemployment rate below the national average and steady gains in
employment like food service, retail and logistics.

But those service jobs pay well below the \$20 to \$25 an hour that
veteran Carrier employees --- with only a high school diploma --- can
earn building furnaces and fan coils in Indianapolis. The typical
manufacturing worker in the state earns \$59,000 a year, about \$20,000
a year more than the typical service job.

And for less credentialed workers, that margin is the difference between
having a shot at a middle-class life, including owning a home and
sending children to college, and having to struggle to make ends meet.

``These are truly irreplaceable jobs,'' said Scott Paul, president of
the Alliance for American Manufacturing, an advocacy group, and a native
of Rensselaer, Ind. ``A manufacturing job is one of the only ladders to
fulfilling the American dream for a worker without a college degree.''

``A manufacturing worker who loses their job at Carrier will be resigned
to facing a lower standard of living and leaner retirement years,'' Mr.
Paul added. ``Carrier is special because it happened at the right time
and the right place and it gained a high profile. But obviously, Donald
Trump and Mike Pence can't intervene every time a plant closes.''

The economic fortunes for this group have been shrinking for years,
which is a major reason the story of Mr. Trump and Carrier has resonated
so deeply.

In Indiana, in particular, as in other so-called Rust Belt states, there
are a lot of people who are
\href{https://www.in.gov/che/files/IN_-_A_Stronger_Nation.pdf}{less
educated}: Just 16.5 percent of the state's residents ages 25 to 64 have
a bachelor's degree, half the rate for the country over all. And while
about 30 percent have an associate degree or some college, the bulk of
Indiana residents, 44 percent, have only a high school diploma --- or
less.

Nor has manufacturing remained the sole domain of whites. It provides a
crucial source of higher-paying jobs for minorities.

In the popular imagination, the Indianapolis factory where 1,400 Carrier
workers build furnaces and fan coils looks like a scene out of ``The
Deer Hunter'' or ``Norma Rae.'' Blue-collar guys walking through the
plant gate, lunch pail in hand, or white women barely getting by after
years on the line.

But the reality at the Carrier plant that Mr. Trump will visit on
Thursday is very different. About half the workers are African-American,
making it a much more diverse workplace than many white-collar settings.

Women account for a substantial portion of the work force as well, but
the wages are anything but subsistence: over \$20 an hour plus benefits
for workers with just a high school diploma. That is an almost
unheard-of level of pay for Indiana workers with that level of education
in other sectors like food service and retail or even many health care
jobs.

Carol Bigbee, 59, who has worked at Carrier for over 13 years, earns
\$22 an hour. Her daughter has a bachelor's degree and works in a
medical lab, but earns one-third less.

``You have to be really blessed to find a job that pays that kind of
money,'' she said.

In southern Indiana, where the Manitowoc Foodservice factory will close
next year, good-paying blue-collar jobs are just as rare.

But Rich Sheffer, vice president for investor relations and treasurer at
the company, said it had little choice but to relocate to Mexico.

``This company has 20 percent excess manufacturing capacity,'' he said.
A few of the jobs are being transferred to Covington, Tenn., he said,
but the Sellersburg plant ``would have required a massive investment in
automation. And we have to deal with profit margins that are trailing
the industry.''

Mr. Sheffer said his company's situation was different from that of
Carrier, which has profitable operations in Indiana but could make more
money in Mexico.

``Our motivation is completely different, but,'' he added, ``we haven't
been contacted by anybody in the Trump administration.''

Advertisement

\protect\hyperlink{after-bottom}{Continue reading the main story}

\hypertarget{site-index}{%
\subsection{Site Index}\label{site-index}}

\hypertarget{site-information-navigation}{%
\subsection{Site Information
Navigation}\label{site-information-navigation}}

\begin{itemize}
\tightlist
\item
  \href{https://help.nytimes.com/hc/en-us/articles/115014792127-Copyright-notice}{©~2020~The
  New York Times Company}
\end{itemize}

\begin{itemize}
\tightlist
\item
  \href{https://www.nytco.com/}{NYTCo}
\item
  \href{https://help.nytimes.com/hc/en-us/articles/115015385887-Contact-Us}{Contact
  Us}
\item
  \href{https://www.nytco.com/careers/}{Work with us}
\item
  \href{https://nytmediakit.com/}{Advertise}
\item
  \href{http://www.tbrandstudio.com/}{T Brand Studio}
\item
  \href{https://www.nytimes.com/privacy/cookie-policy\#how-do-i-manage-trackers}{Your
  Ad Choices}
\item
  \href{https://www.nytimes.com/privacy}{Privacy}
\item
  \href{https://help.nytimes.com/hc/en-us/articles/115014893428-Terms-of-service}{Terms
  of Service}
\item
  \href{https://help.nytimes.com/hc/en-us/articles/115014893968-Terms-of-sale}{Terms
  of Sale}
\item
  \href{https://spiderbites.nytimes.com}{Site Map}
\item
  \href{https://help.nytimes.com/hc/en-us}{Help}
\item
  \href{https://www.nytimes.com/subscription?campaignId=37WXW}{Subscriptions}
\end{itemize}
