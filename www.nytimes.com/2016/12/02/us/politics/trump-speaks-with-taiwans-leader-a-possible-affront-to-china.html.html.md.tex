Sections

SEARCH

\protect\hyperlink{site-content}{Skip to
content}\protect\hyperlink{site-index}{Skip to site index}

\href{https://www.nytimes.com/section/politics}{Politics}

\href{https://myaccount.nytimes.com/auth/login?response_type=cookie\&client_id=vi}{}

\href{https://www.nytimes.com/section/todayspaper}{Today's Paper}

\href{/section/politics}{Politics}\textbar{}Trump Speaks With Taiwan's
Leader, an Affront to China

\url{https://nyti.ms/2gwshOM}

\begin{itemize}
\item
\item
\item
\item
\item
\item
\end{itemize}

Advertisement

\protect\hyperlink{after-top}{Continue reading the main story}

Supported by

\protect\hyperlink{after-sponsor}{Continue reading the main story}

\hypertarget{trump-speaks-with-taiwans-leader-an-affront-to-china}{%
\section{Trump Speaks With Taiwan's Leader, an Affront to
China}\label{trump-speaks-with-taiwans-leader-an-affront-to-china}}

\includegraphics{https://static01.nyt.com/images/2016/12/03/us/03trump/03trump-articleLarge.jpg?quality=75\&auto=webp\&disable=upscale}

By \href{http://www.nytimes.com/by/mark-landler}{Mark Landler} and
\href{http://www.nytimes.com/by/david-e-sanger}{David E. Sanger}

\begin{itemize}
\item
  Dec. 2, 2016
\item
  \begin{itemize}
  \item
  \item
  \item
  \item
  \item
  \item
  \end{itemize}
\end{itemize}

\href{http://cn.nytimes.com/usa/20161203/trump-speaks-with-taiwans-leader-a-possible-affront-to-china/}{阅读简体中文版}

WASHINGTON --- President-elect Donald J. Trump spoke by telephone with
Taiwan's president on Friday, a striking break with nearly four decades
of diplomatic practice that could precipitate a major rift with China
even before Mr. Trump takes office.

Mr. Trump's office said he had spoken with the Taiwanese president, Tsai
Ing-wen, ``who offered her congratulations.'' He is believed to be the
first president or president-elect who has spoken to a Taiwanese leader
since at least 1979, when the United States severed diplomatic ties with
Taiwan as part of its recognition of the People's Republic of China.

In the statement, Mr. Trump's office said the two leaders had noted
``the close economic, political, and security ties'' between Taiwan and
the United States. Mr. Trump, it said, ``also congratulated President
Tsai on becoming President of Taiwan earlier this year.''

Mr. Trump's motives in taking the call, which lasted more than 10
minutes, were not clear. In a Twitter message late Friday,
\href{https://twitter.com/realDonaldTrump/status/804848711599882240}{he
said} Ms. Tsai ``CALLED ME.''

But diplomats with ties to Taiwan said it was highly unlikely that the
Taiwanese leader would have made the call without arranging it in
advance. Ms. Tsai's office
\href{http://www.president.gov.tw/Default.aspx?tabid=131\&itemid=38402\&rmid=514}{confirmed}
that it had taken place, saying the two had discussed promoting economic
development and ``strengthening defense.'' Taiwan's Central News Agency
hailed the call as ``historic.''

On Saturday, China's foreign minister, Wang Yi, in his government's
first official reaction, played down the call.

Stressing the good relationship between the United States and China, he
said, ``I also believe this will not change the One China policy upheld
by the American government for many years.''

Mr. Wang, speaking to reporters in Beijing, characterized the call as
initiated by the Taiwanese government. ``We believe it's a petty action
by the Taiwan side.''

The president-elect has shown little heed for the nuances of
international diplomacy, holding a series of unscripted phone calls to
foreign leaders that have
\href{http://www.nytimes.com/interactive/2016/12/02/world/trump-calls-to-world-leaders.html}{roiled
sensitive relationships} with Britain, India and Pakistan. On Thursday,
the White House urged Mr. Trump to use experts from the State Department
to prepare him for these exchanges.

The White House was not told about Mr. Trump's call until after it
happened, according to a senior administration official. But afterward,
the Chinese government contacted the White House to discuss the matter.

Image

President Tsai Ing-wen of Taiwan.Credit...Ritchie B. Tongo/European
Pressphoto Agency

The longer-term fallout from the Trump-Tsai conversation could be
significant, the administration official said, noting that the Chinese
government issued a bitter protest after the United States sold weapons
to Taiwan as part of a well-established arms agreement grudgingly
accepted by Beijing.

Mr. Trump's call with President Tsai is a bigger provocation. Beijing
views Taiwan as a breakaway province and has adamantly opposed the
attempts of any country to carry on official relations with it.

On Nov. 14, Mr. Trump spoke with Xi Jinping, China's president, and
\href{https://www.nytimes.com/2016/11/14/us/politics/donald-trump-transition.html}{a
statement from the transition team said} the two men had a ``clear sense
of mutual respect.''

Initial reaction from China about Friday's telephone call was surprise
verging on disbelief. ``This is a big event, the first challenge the
president-elect has made to China,'' said Shi Yinhong, professor of
international relations at Renmin University in Beijing. ``This must be
bad news for the Chinese leadership.''

Official state-run media have portrayed Mr. Trump in a positive light,
casting him as a businessman China could get along with. He was favored
among Chinese commentators during the election over Hillary Clinton, who
was perceived as being too hard on China.

Mr. Trump's exchange touched ``the most sensitive spot'' for China's
foreign policy, Mr. Shi said. The government, he said, would most likely
interpret it as encouraging Ms. Tsai, the leader of the party that
favors independence from the mainland, to continue to resist pressure
from Beijing.

Among diplomats in the United States, there was similar shock. ``This is
a change of historic proportions,'' said Evan S. Medeiros, a former
senior director of Asian affairs in the Obama administration. ``The real
question is, what are the Chinese going to do?''

In a second Twitter message about the call Friday night,
\href{https://twitter.com/realDonaldTrump/status/804863098138005504}{Mr.
Trump said}, ``Interesting how the U.S. sells Taiwan billions of dollars
of military equipment but I should not accept a congratulatory call.''

Ties between the United States and Taiwan are currently managed through
quasi-official institutions. The American Institute in Taiwan issues
visas and provides other basic consular services, and Taiwan has an
equivalent institution with offices in several cities in the United
States.

These arrangements are the outgrowth of the One China policy that has
governed relations between the United States and China since President
Richard M. Nixon's historic meeting with Mao Zedong in Beijing in 1972.
In 1978, President Jimmy Carter formally recognized Beijing as the sole
government of China, abrogating its ties with Taipei a year later.

The call also raised questions of conflicts of interest.

Newspapers in \href{http://www.taiwannews.com.tw/en/news/3031091}{Taiwan
reported last month}that a Trump Organization representative had visited
the country, expressing interest in perhaps developing a hotel project
adjacent to Taiwan Taoyuan International Airport, which is undergoing a
major expansion. The mayor of Taoyuan, Cheng Wen-tsan, was quoted as
confirming that visit.

A spokeswoman for the Trump Organization, Amanda Miller, said that the
company had ``no plans for expansion into Taiwan,'' and that there had
been no ``authorized visits'' to the country to push a Trump development
project. But Ms. Miller did not dispute that Anne-Marie Donoghue, a
sales manager overseeing Asia for Trump Hotels, had visited Taiwan in
October, a trip that Ms. Donoghue
\href{https://www.documentcloud.org/documents/3230995-10-Anne-Marie-Donoghue-Trump.html}{recorded
on her Facebook page}.

Ms. Donoghue did not respond to requests for comment.

Mr. Trump's call with the Taiwanese president came just as President
Obama delivered a more subtle, but also aggressive, rebuff of China: He
\href{http://www.nytimes.com/2016/12/02/business/dealbook/china-aixtron-obama-cfius.html}{blocked,
by executive order}, an effort by Chinese investors to buy a
semiconductor production firm called Aixtron.

Mr. Obama took the action on national security grounds, after an
intelligence review concluded that the technology could be used for
``military applications'' and help provide an ``overall technical body
of knowledge and experience'' to the Chinese.

The decision is likely to accelerate tension with Beijing, as Chinese
authorities make it extraordinarily difficult for American technology
companies, including Google and Facebook, to gain access to the Chinese
market, and Washington seeks to slow China's acquisition of critical
technology.

Mr. Trump has made little effort to avoid antagonizing China. He
\href{https://twitter.com/realdonaldtrump/status/265895292191248385?lang=en}{has
characterized} climate change as a ``Chinese hoax,'' designed to
undermine the American economy. He has said China's manipulation of its
currency deepened a trade deficit with the United States. And he has
threatened to impose a 45 percent tariff on Chinese goods, a proposal
that critics said would set off a trade war.

By happenstance, just hours before Mr. Trump's conversation with Ms.
Tsai, Henry A. Kissinger, the former secretary of state who designed the
``One China'' policy, was in Beijing meeting with Mr. Xi. It was unclear
if Mr. Kissinger, 93, was carrying any message from Mr. Trump, with whom
he met again recently in his role as the Republican Party's foreign
policy sage.

``The presidential election has taken place in the United States and we
are now in the key moment. We, on the Chinese side, are watching the
situation very closely. Now it is in the transition period,'' Mr. Xi
told Mr. Kissinger in front of reporters.

A small, hard-line faction of Republicans has periodically urged a more
confrontational approach to Beijing, and many of President George W.
Bush's advisers were pressing such an approach in the first months of
his presidency in 2001. But the attacks of Sept. 11 defused that move,
and Iraq became the No. 1 enemy. After that, Mr. Bush needed China ---
for North Korea diplomacy, counterterrorism and as an economic partner
--- and any movement toward confrontation was quashed.

For his part, Mr. Trump has shown little concern about ruffling feathers
in his exchanges with leaders. He also spoke on Friday with the
Philippine president, Rodrigo Duterte, who has called Mr. Obama a ``son
of a whore'' and been accused of ordering the extrajudicial killings of
thousands of suspected drug dealers. On Saturday, Mr. Duterte said that
Mr. Trump had
\href{https://www.nytimes.com/2016/12/03/world/asia/philippines-rodrigo-duterte-donald-trump.html}{wished
him well in his antidrug campaign}, though his account could not
immediately be verified.

This week, Mr. Trump appeared to accept an invitation from Prime
Minister Nawaz Sharif to visit Pakistan, a country that Mr. Obama has
steered clear of, largely over tensions between Washington and Islamabad
over counterterrorism policy and nuclear proliferation.

Lawmakers expressed alarm at the implications of Mr. Trump's
freewheeling
approach.``\href{https://twitter.com/ChrisMurphyCT/status/804809228401672192}{What
has happened in the last 48 hours is not a shift}. These are major
pivots in foreign policy w/out any plan. That's how wars start,''
Senator Christopher S. Murphy, Democrat of Connecticut, wrote on
Twitter.
``\href{https://twitter.com/ChrisMurphyCT/status/804811098205650944}{It's
probably time we get a Secretary of State nominee on board}. Preferably
w experience. Like, really really soon.''

Advertisement

\protect\hyperlink{after-bottom}{Continue reading the main story}

\hypertarget{site-index}{%
\subsection{Site Index}\label{site-index}}

\hypertarget{site-information-navigation}{%
\subsection{Site Information
Navigation}\label{site-information-navigation}}

\begin{itemize}
\tightlist
\item
  \href{https://help.nytimes.com/hc/en-us/articles/115014792127-Copyright-notice}{©~2020~The
  New York Times Company}
\end{itemize}

\begin{itemize}
\tightlist
\item
  \href{https://www.nytco.com/}{NYTCo}
\item
  \href{https://help.nytimes.com/hc/en-us/articles/115015385887-Contact-Us}{Contact
  Us}
\item
  \href{https://www.nytco.com/careers/}{Work with us}
\item
  \href{https://nytmediakit.com/}{Advertise}
\item
  \href{http://www.tbrandstudio.com/}{T Brand Studio}
\item
  \href{https://www.nytimes.com/privacy/cookie-policy\#how-do-i-manage-trackers}{Your
  Ad Choices}
\item
  \href{https://www.nytimes.com/privacy}{Privacy}
\item
  \href{https://help.nytimes.com/hc/en-us/articles/115014893428-Terms-of-service}{Terms
  of Service}
\item
  \href{https://help.nytimes.com/hc/en-us/articles/115014893968-Terms-of-sale}{Terms
  of Sale}
\item
  \href{https://spiderbites.nytimes.com}{Site Map}
\item
  \href{https://help.nytimes.com/hc/en-us}{Help}
\item
  \href{https://www.nytimes.com/subscription?campaignId=37WXW}{Subscriptions}
\end{itemize}
