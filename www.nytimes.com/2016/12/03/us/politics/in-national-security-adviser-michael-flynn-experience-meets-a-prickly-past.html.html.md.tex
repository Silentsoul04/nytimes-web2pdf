Sections

SEARCH

\protect\hyperlink{site-content}{Skip to
content}\protect\hyperlink{site-index}{Skip to site index}

\href{https://www.nytimes.com/section/politics}{Politics}

\href{https://myaccount.nytimes.com/auth/login?response_type=cookie\&client_id=vi}{}

\href{https://www.nytimes.com/section/todayspaper}{Today's Paper}

\href{/section/politics}{Politics}\textbar{}In Trump's Security Pick,
Michael Flynn, `Sharp Elbows' and No Dissent

\url{https://nyti.ms/2gZsW8k}

\begin{itemize}
\item
\item
\item
\item
\item
\end{itemize}

Advertisement

\protect\hyperlink{after-top}{Continue reading the main story}

Supported by

\protect\hyperlink{after-sponsor}{Continue reading the main story}

\hypertarget{in-trumps-security-pick-michael-flynn-sharp-elbows-and-no-dissent}{%
\section{In Trump's Security Pick, Michael Flynn, `Sharp Elbows' and No
Dissent}\label{in-trumps-security-pick-michael-flynn-sharp-elbows-and-no-dissent}}

\includegraphics{https://static01.nyt.com/images/2016/12/04/us/jp-FLYNN/jp-FLYNN-articleLarge.jpg?quality=75\&auto=webp\&disable=upscale}

By \href{http://www.nytimes.com/by/matthew-rosenberg}{Matthew
Rosenberg}, \href{http://www.nytimes.com/by/mark-mazzetti}{Mark
Mazzetti} and \href{http://www.nytimes.com/by/eric-schmitt}{Eric
Schmitt}

\begin{itemize}
\item
  Dec. 3, 2016
\item
  \begin{itemize}
  \item
  \item
  \item
  \item
  \item
  \end{itemize}
\end{itemize}

WASHINGTON --- Days after Islamist militants stormed the American
diplomatic compound in Benghazi, Libya, in 2012, Lt. Gen. Michael T.
Flynn reached a conclusion that stunned some of his subordinates at the
Defense Intelligence Agency: Iran had a role in the attack, he told
them.

Now, he added, it was their job to prove it --- and, by implication, to
show that the White House was wrong about what had led to the attack.

Mr. Flynn, whom President-elect Donald J. Trump has chosen to be his
national security adviser, soon took to pushing analysts to find Iran's
hidden hand in the disaster, according to current and former officials
familiar with the episode. But like many other investigations into
Benghazi, theirs found no evidence of any links, and the general's
stubborn insistence reminded some officials at the agency of how the
Bush administration had once relentlessly sought to connect Saddam
Hussein and Iraq to the Sept. 11, 2001, attacks.

Years before Mr. Flynn met Mr. Trump, his brief tenure running the
Defense Intelligence Agency foreshadowed some of the same qualities he
has exhibited more recently as he has plunged into politics and
controversy as a key campaign adviser to Mr. Trump, who shared his
desire to usurp what he viewed as Washington's incompetent and corrupt
elite.

Many of those who observed the general's time at the agency described
him as someone who alienated both superiors and subordinates with his
sharp temperament, his refusal to brook dissent, and what his critics
considered a conspiratorial worldview.

Those qualities could prove problematic for a national security adviser,
especially one who will have to mediate the conflicting views of cabinet
secretaries and agencies for a president with no experience in defense
or foreign policy issues. Traditionally, the job has gone to a
Washington veteran: Condoleezza Rice, for instance, or Thomas E.
Donilon.

\hypertarget{the-last-word}{%
\subsection{The Last Word}\label{the-last-word}}

The new job will give Mr. Flynn, 57, nearly unfettered access to the
Oval Office. Whether it is renewed bloodletting in Ukraine, a North
Korean nuclear test or a hurricane swamping Haiti, he will often have
the last word with Mr. Trump about how the United States should react.

For Mr. Flynn, serving as the president's chief adviser on defense and
foreign policy matters, represents a triumphal return to government
after being dismissed as agency director in 2014 after two years there.

Heading the agency, the Pentagon's intelligence arm, was supposed to be
the capstone of a storied career. Through tours in Iraq and Afghanistan,
Mr. Flynn had built a reputation as a brash and outspoken officer with
an unusual talent for unraveling terrorist networks, and both his
fiercest critics and his outspoken supporters praise his work from those
wars.

In numerous interviews and speeches over the past year, Mr. Flynn, who
did not respond to requests for comment for this article, has maintained
that he was forced out as director for refusing to toe the Obama
administration's line that Al Qaeda was in retreat. The claim has made
the general something of a cult figure among many Republicans.

``D.I.A. has always been a problem child and it remains that way,'' said
Representative Devin Nunes, the California Republican who is chairman of
the House Intelligence Committee and a member of Mr. Trump's transition
team. ``Flynn tried to get in there and fix things and he was only given
two years until they ran him out because they didn't like his
assessment.''

The congressman added: ``They didn't have an excuse to fire him, so they
made it up. Nobody has been able to fix that place.''

But others say he was forced out for a relatively simple reason: He
failed to effectively manage a sprawling, largely civilian bureaucracy.

At the agency, ``Flynn surrounded himself with loyalists. In
implementing his vision, he moved at light speed, but he didn't
communicate effectively,'' said
\href{http://www.dia.mil/About/Leadership/Article-View/Article/567071/douglas-h-wise/}{Douglas
H. Wise,} deputy director from 2014 until he retired in August. ``He
didn't tolerate it well when subordinates didn't move fast enough,'' he
said. ``As a senior military officer, he expected compliance and didn't
want any pushback.''

\hypertarget{the-boss-is-always-right}{%
\subsection{The Boss Is Always Right}\label{the-boss-is-always-right}}

Founded in 1961, the Defense Intelligence Agency has long been in the
shadow of the Central Intelligence Agency, and with the end of the Cold
War it lost its primary mission of collecting and analyzing information
about the Soviet military. Strained by a decade of conflict in
Afghanistan and Iraq, it was performing an uncertain role within the
constellation of American spy agencies when Mr. Flynn arrived at
headquarters in mid-2012.

The agency's system of human intelligence collection was perceived as
largely broken. The effort to rebuild it was underway when Mr. Flynn
took control in 2012, but he made it immediately known that he had a dim
view of the agency's recent performance.

During a tense gathering of senior officials at an off-site retreat, he
gave the assembled group a taste of his leadership philosophy, according
to one person who attended the meeting and insisted on anonymity to
discuss classified matters. Mr. Flynn said that the first thing everyone
needed to know was that he was always right. His staff would know they
were right, he said, when their views melded to his. The room fell
silent, as employees processed the lecture from their new boss.

Current and former employees said Mr. Flynn had trouble adjusting his
style for an organization with a 16,500-person work force that was 80
percent civilian. He was used to a strict military chain of command, and
was at times uncomfortable with the often-messy give-and-take that is
common among intelligence analysts.

Some also described him as a Captain Queeg-like character, paranoid that
his staff members were undercutting him and credulous of conspiracy
theories.

At times, the general also exhibited what a number of officials
described as tone-deafness on the larger strategic challenges
confronting the nation.

The most glaring example came in early March 2014, just after Russia had
seized Crimea. American officials were weighing whether to impose
sanctions in response, but Mr. Flynn was pushing ahead with plans to
travel to Moscow to build on an existing intelligence-sharing initiative
with his Russian counterparts. He also wanted to invite Russian military
intelligence officials to Washington to discuss the threat of Islamist
militants. His superiors ordered both canceled.

By the end of his tenure, he had largely cut out senior staff members
from significant decision-making, relying instead on a small circle of
trusted advisers he had come to know during his overseas military
deployments.

His bosses --- Michael G. Vickers, the under secretary of defense for
intelligence, and James R. Clapper, the director of national
intelligence --- came to think that the agency was adrift, and that Mr.
Flynn refused to address its biggest problems.

``Regrettably, he got engaged in an increasingly bitter and
organizationally paralyzing feud with his senior staff when he should
have been focused on building the intelligence capabilities'' of the
agency, said Mr. Vickers, who was Mr. Flynn's immediate boss at the
Pentagon.

During his tour in Iraq, he served under Gen. Stanley A. McChrystal,
running intelligence for the military's Joint Special Operations
Command, whose relentless campaign of raids and airstrikes hollowed out
Al Qaeda in Iraq. When General McChrystal went to run the war in
Afghanistan in 2009, Mr. Flynn signed on as his intelligence chief.

``He wasn't a staid intelligence officer. He was aggressive. He was
about the mission,'' said Richard M. Frankel, a former senior F.B.I.
official who worked with Mr. Flynn at the Office of the Director of
National Intelligence. ``He can have sharp elbows because he is about
the mission.''

He burnished his reputation as an intelligence officer --- but also for
controversy. He co-wrote a paper, ``Fixing Intel,'' that offered an
early hint of his disdain for the civilian intelligence analysts he
would later clash with at the Defense Intelligence Agency. Published by
a Washington think tank, it bluntly stated that ``the U.S. intelligence
community is only marginally relevant to the overall strategy,''
infuriating officials at the D.I.A. and the C.I.A.

More problematic from the military's perspective was Mr. Flynn's
willingness to share intelligence with other countries. He returned to
Washington at the end of 2010, and found himself under investigation for
sharing sensitive data with Pakistan about the Haqqani network, arguably
the most capable faction of the Taliban, and for providing highly
classified intelligence to British and Australian forces fighting in
Afghanistan.

His superiors eventually concluded that he was trying to prod Pakistan
to crack down on the Haqqanis (they have yet to do so), and the general
remains unapologetic about sharing intelligence with British and
Australian forces. ``They're our closest allies! I mean, really, we're
fighting together and I can't share a single piece of paper?'' he said
in an interview last year.

Around the same time, he was also getting to know Michael A. Ledeen, a
controversial writer and former Reagan administration official. The two
men connected immediately, sharing a similar worldview and a belief that
America was in a world war against Islamist militants allied with
Russia, Cuba and North Korea. That worldview is what Mr. Flynn came to
be best known for during the presidential campaign, when he argued that
the United States faced a singular, overarching threat, and that there
was just one accurate way to describe it: ``radical Islamic terrorism.''

He has posted on Twitter that fear of Muslims is rational, written that
Islamic law is spreading in the United States, and said that Islam
itself is more like a political ideology than a religion. The United
States, he wrote in ``Field of Fight,'' a book about radical Islam he
co-wrote with Mr. Ledeen, is ``in a world war, but very few people
recognize it.''

\hypertarget{implicating-iran}{%
\subsection{Implicating Iran}\label{implicating-iran}}

Mr. Flynn saw the Benghazi attack in September 2012 as just one skirmish
in this global war. But it was his initial reaction to the event,
immediately seeking evidence of an Iranian role, that many saw as
emblematic of a conspiratorial bent. Iran, a Shiite nation, has
generally eschewed any alliance with Sunni militants like the ones who
attacked the American diplomatic compound.

For weeks, he pushed analysts for evidence that the attack might have
had a state sponsor --- sometimes shouting at them when they didn't come
to the conclusions he wanted. The attack, he told his analysts, was a
``black swan'' event that required more creative intelligence analysis
to decipher.

``To ask employees to look for the .0001 percent chance of something
when you have an actual emergency and dead Americans is beyond the
pale,'' said Joshua Manning, an agency analyst from 2009 to 2013.

Beyond Benghazi, American officials said that in time, the general grew
angrier at what he saw as the Obama administration's passivity in
dealing with worldwide threats --- from Sunni extremist terrorism to
Iran. He also saw the C.I.A., an organization he had long disdained, as
overly political and too willing to advance the White House's agenda.

In particular, he became convinced that the C.I.A. was refusing to
declassify many of the documents found at Osama bin Laden's compound in
Abbottabad, Pakistan, because they seemed to undercut the
administration's narrative about Qaeda strength at the time Bin Laden
was killed.

``If they put out what we knew, then the president could've not said, in
a national election, Al Qaeda's on the run and we've killed Bin Laden,''
Mr. Flynn said before the latest election, referring to Mr. Obama's 2012
re-election bid. ``Even today, he talks about Bin Laden as though that
was a stroke of genius. I mean, c'mon!''

Advertisement

\protect\hyperlink{after-bottom}{Continue reading the main story}

\hypertarget{site-index}{%
\subsection{Site Index}\label{site-index}}

\hypertarget{site-information-navigation}{%
\subsection{Site Information
Navigation}\label{site-information-navigation}}

\begin{itemize}
\tightlist
\item
  \href{https://help.nytimes.com/hc/en-us/articles/115014792127-Copyright-notice}{©~2020~The
  New York Times Company}
\end{itemize}

\begin{itemize}
\tightlist
\item
  \href{https://www.nytco.com/}{NYTCo}
\item
  \href{https://help.nytimes.com/hc/en-us/articles/115015385887-Contact-Us}{Contact
  Us}
\item
  \href{https://www.nytco.com/careers/}{Work with us}
\item
  \href{https://nytmediakit.com/}{Advertise}
\item
  \href{http://www.tbrandstudio.com/}{T Brand Studio}
\item
  \href{https://www.nytimes.com/privacy/cookie-policy\#how-do-i-manage-trackers}{Your
  Ad Choices}
\item
  \href{https://www.nytimes.com/privacy}{Privacy}
\item
  \href{https://help.nytimes.com/hc/en-us/articles/115014893428-Terms-of-service}{Terms
  of Service}
\item
  \href{https://help.nytimes.com/hc/en-us/articles/115014893968-Terms-of-sale}{Terms
  of Sale}
\item
  \href{https://spiderbites.nytimes.com}{Site Map}
\item
  \href{https://help.nytimes.com/hc/en-us}{Help}
\item
  \href{https://www.nytimes.com/subscription?campaignId=37WXW}{Subscriptions}
\end{itemize}
