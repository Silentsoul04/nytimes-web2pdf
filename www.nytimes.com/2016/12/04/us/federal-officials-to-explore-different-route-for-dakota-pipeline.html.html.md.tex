Sections

SEARCH

\protect\hyperlink{site-content}{Skip to
content}\protect\hyperlink{site-index}{Skip to site index}

\href{https://www.nytimes.com/section/us}{U.S.}

\href{https://myaccount.nytimes.com/auth/login?response_type=cookie\&client_id=vi}{}

\href{https://www.nytimes.com/section/todayspaper}{Today's Paper}

\href{/section/us}{U.S.}\textbar{}Protesters Gain Victory in Fight Over
Dakota Access Oil Pipeline

\url{https://nyti.ms/2h36z1y}

\begin{itemize}
\item
\item
\item
\item
\item
\item
\end{itemize}

Advertisement

\protect\hyperlink{after-top}{Continue reading the main story}

Supported by

\protect\hyperlink{after-sponsor}{Continue reading the main story}

\hypertarget{protesters-gain-victory-in-fight-over-dakota-access-oil-pipeline}{%
\section{Protesters Gain Victory in Fight Over Dakota Access Oil
Pipeline}\label{protesters-gain-victory-in-fight-over-dakota-access-oil-pipeline}}

\includegraphics{https://static01.nyt.com/images/2016/12/05/us/05dakota-2/05dakota-2-videoSixteenByNineJumbo1600.jpg}

By \href{https://www.nytimes.com/by/jack-healy}{Jack Healy} and
\href{https://www.nytimes.com/by/nicholas-fandos}{Nicholas Fandos}

\begin{itemize}
\item
  Dec. 4, 2016
\item
  \begin{itemize}
  \item
  \item
  \item
  \item
  \item
  \item
  \end{itemize}
\end{itemize}

CANNON BALL, N.D. --- The Standing Rock Sioux Tribe won a major victory
on Sunday in its battle to block an oil pipeline being built near its
reservation when the Department of the Army announced that it would not
allow the pipeline to be drilled under a dammed section of the Missouri
River.

The Army said it would look for alternative routes for the \$3.7 billion
\href{https://www.nytimes.com/2020/07/06/us/dakota-access-pipeline.html}{Dakota
Access pipeline}. Construction of the route a half-mile from the
Standing Rock Sioux reservation has become a global flash point for
environmental and indigenous activism, drawing thousands of people out
here to a sprawling prairie camp of tents, tepees and yurts.

``The best way to complete that work responsibly and expeditiously is to
explore alternate routes for the pipeline crossing,'' Jo-Ellen Darcy,
the Army's assistant secretary for civil works, said in a statement. The
move could presage a lengthy environmental review that has the potential
to block the pipeline's construction for months or years.

But it was unclear how durable the government's decision would be.
Sunday's announcement came in the dwindling days of the Obama
administration, which revealed in November that the
\href{http://topics.nytimes.com/top/reference/timestopics/organizations/a/army_corps_of_engineers/index.html?inline1=nyt\%2Dorg}{Army
Corps of Engineers} was
\href{http://www.nytimes.com/2016/11/03/us/president-obama-says-engineers-considering-alternate-route-for-dakota-pipeline.html}{considering
an alternative route}. The Corps of Engineers is part of the Department
of the Army.

\href{https://www.nytimes.com/interactive/2016/11/23/us/dakota-access-pipeline-protest-map.html}{}

\includegraphics{https://static01.nyt.com/images/2016/11/23/us/dakota-access-pipeline-1479877294784/dakota-access-pipeline-1479877294784-thumbLarge.jpg}

\hypertarget{the-conflicts-along-1172-miles-of-the-dakota-access-pipeline}{%
\subsection{The Conflicts Along 1,172 Miles of the Dakota Access
Pipeline}\label{the-conflicts-along-1172-miles-of-the-dakota-access-pipeline}}

A detailed map showing the Dakota Access Pipeline, the site of months of
clashes near the Standing Rock Sioux Reservation in North Dakota.

President-elect Donald J. Trump, however, has taken a different view of
the project and said as recently as last week that he supported
finishing the 1,170-mile pipeline, which crosses four states and is
almost complete.

Though the Army's decision calls for an environmental study of
alternative routes, the Trump administration could ultimately decide to
allow the original, contested route. Representatives for Mr. Trump's
transition team did not immediately respond to requests for comment.

Mr. Trump owns stock in the company building the pipeline, Energy
Transfer Partners, but he has said that his support has nothing to do
with his investment.

There was no immediate response from Energy Transfer Partners, but its
chief executive, Kelcy Warren, has said that the company was unwilling
to reroute the pipeline, which is intended to transport as much as
550,000 barrels of oil a day from the oil fields of western North Dakota
to a terminal in Illinois.

Reaction was swift on both sides, with environmental groups like
Greenpeace praising the decision. ``The water protectors have done it,''
a Greenpeace spokeswoman, Lilian Molina, said. ``This is a monumental
victory in the fight to protect indigenous rights and sovereignty.''

\includegraphics{https://static01.nyt.com/images/2016/12/05/us/05dakota1/05dakota1-articleInline.jpg?quality=75\&auto=webp\&disable=upscale}

But Craig Stevens, a spokesman for the MAIN Coalition, a
pro-infrastructure group, condemned the move as ``a purely political
decision that flies in the face of common sense and the rule of law.''

``Unfortunately, it's not surprising that the president would, again,
use executive fiat in an attempt to enhance his legacy among the extreme
left,'' Mr. Stevens said in a statement. ``With President-elect Trump
set to take office in 47 days, we are hopeful that this is not the final
word on the Dakota Access Pipeline.''

Representative Kevin Cramer, Republican of North Dakota and a Trump
supporter, called Sunday's decision a ``chilling signal to others who
want to build infrastructure in this country.''

``I can't wait for the adults to be in charge on Jan. 20,'' Mr. Cramer
said, referring to Mr. Trump's inauguration.

Still, the announcement set off whoops of joy inside the Oceti Sakowin
camp. Tribal members paraded through the camp on horseback, jubilantly
beating drums and gathering around a fire at the center of the camp.
Tribal elders celebrated what they said was the validation of months of
prayer and protest.

\includegraphics{https://static01.nyt.com/images/2016/12/05/us/360-pipelineblock/360-pipelineblock-videoSixteenByNineJumbo1600.jpg}

``It's wonderful,'' Dave Archambault II, the Standing Rock tribal
chairman,
\href{http://www.nytimes.com/2016/12/03/us/standing-rock-pipeline-protest-north-dakota.html}{told
cheering supporters} who stood in the melting snow on a mild North
Dakota afternoon. ``You all did that. Your presence has brought the
attention of the world.''

The decision, he said, meant that people no longer had to stay at the
camp during North Dakota's brutal winter. The Corps of Engineers, which
manages the land, had
\href{http://www.nytimes.com/2016/11/26/us/dakota-pipeline-protest.html}{ordered
it to be closed}, but the thousands of protesters had built yurts,
tepees and bunkhouses and vowed to hunker down.

``It's time now that we move forward,'' Mr. Archambault said. ``We don't
have to stand and endure this hard winter. We can spend the winter with
our families.''

Law enforcement officials and non-Native ranchers in this conservative,
heavily white part of North Dakota would like little more than to see
the
\href{http://www.nytimes.com/2016/09/14/us/north-dakota-pipeline-protests.html}{thousands
of protesters return home}. The sheriff has called the demonstrations an
unlawful protest, and officials have characterized the demonstrators as
rioters who have intimidated ranchers and threatened and attacked law
enforcement --- charges that protest leaders deny.

But on Sunday, several campers said
\href{http://www.nytimes.com/2016/10/11/us/tribes-protest-oil-pipeline-north-dakota.html}{they
were not going anywhere}. They said that there were too many
uncertainties surrounding the Army's decision, and that they had
dedicated too much time and emotion to this fight to leave now.

Federal and state regulators had issued the pipeline the necessary
permits to proceed, but the Corps of Engineers had not yet granted it a
final easement to drill under a stretch of the Missouri River called
Lake Oahe.

The Standing Rock Sioux had objected to the pipeline's path so close to
the source of their drinking water, and said any spill could poison
water supplies for them and other reservations and cities downstream.
They also said the pipeline's route through what are now privately owned
ranches bordering the river crossed through sacred ancestral lands.

News of the government's denial came after the size of the camp had
swelled with hundreds, perhaps thousands, of Native and non-Native
veterans who had arrived to support the tribe. As word spread, people
who had camped out here for months, sometimes in bitterly cold
temperatures, and who had clashed violently with local law enforcement,
linked arms and cheered and cried.

They screamed, ``Mni wiconi!'' --- the movement's rallying cry --- which
means ``Water is life.''

Jon Eagle Sr., a member of the Standing Rock Tribe, said the
announcement was a vindication for the thousands who had traveled here,
and for the multitudes who had rallied to the tribe's fight on social
media or donated. Millions of dollars in donations and goods have flowed
into the camps for months as the tribe's fight and the scenes of
protesters being tear-gassed and
\href{http://www.nytimes.com/2016/11/21/us/dakota-access-pipeline-protesters-police.html}{sprayed
with freezing water} stirred outrage on social media. (Law enforcement
officials have insisted the entire time that they have acted responsibly
and with restraint.)

``I don't know quite how to put into words how proud I am of our
people,'' Mr. Eagle said. ``And I mean our people. I don't just mean the
indigenous people of this continent. I mean all the people who came to
stand with us. And it's a beautiful day. It's a powerful day.''

Ken Many Wounds, who has served as a tribal liaison to express concerns
and questions to law enforcement, said he had been standing by the
camp's main fire --- one that is tended constantly --- when he heard the
news from the tribal chairman's wife. He said he did not believe it at
first.

``I hugged her, I cried,'' he said. ``Our prayers have been answered. A
lot of people didn't believe that prayer was going to be the answer. But
our people stayed together. In our hearts, we knew.''

Advertisement

\protect\hyperlink{after-bottom}{Continue reading the main story}

\hypertarget{site-index}{%
\subsection{Site Index}\label{site-index}}

\hypertarget{site-information-navigation}{%
\subsection{Site Information
Navigation}\label{site-information-navigation}}

\begin{itemize}
\tightlist
\item
  \href{https://help.nytimes.com/hc/en-us/articles/115014792127-Copyright-notice}{©~2020~The
  New York Times Company}
\end{itemize}

\begin{itemize}
\tightlist
\item
  \href{https://www.nytco.com/}{NYTCo}
\item
  \href{https://help.nytimes.com/hc/en-us/articles/115015385887-Contact-Us}{Contact
  Us}
\item
  \href{https://www.nytco.com/careers/}{Work with us}
\item
  \href{https://nytmediakit.com/}{Advertise}
\item
  \href{http://www.tbrandstudio.com/}{T Brand Studio}
\item
  \href{https://www.nytimes.com/privacy/cookie-policy\#how-do-i-manage-trackers}{Your
  Ad Choices}
\item
  \href{https://www.nytimes.com/privacy}{Privacy}
\item
  \href{https://help.nytimes.com/hc/en-us/articles/115014893428-Terms-of-service}{Terms
  of Service}
\item
  \href{https://help.nytimes.com/hc/en-us/articles/115014893968-Terms-of-sale}{Terms
  of Sale}
\item
  \href{https://spiderbites.nytimes.com}{Site Map}
\item
  \href{https://help.nytimes.com/hc/en-us}{Help}
\item
  \href{https://www.nytimes.com/subscription?campaignId=37WXW}{Subscriptions}
\end{itemize}
