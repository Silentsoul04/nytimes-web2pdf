Sections

SEARCH

\protect\hyperlink{site-content}{Skip to
content}\protect\hyperlink{site-index}{Skip to site index}

\href{https://myaccount.nytimes.com/auth/login?response_type=cookie\&client_id=vi}{}

\href{https://www.nytimes.com/section/todayspaper}{Today's Paper}

For the Missoni Matriarch, a Dizzying Collection of Hands

\href{https://nyti.ms/2gGwrnl}{https://nyti.ms/2gGwrnl}

\begin{itemize}
\item
\item
\item
\item
\item
\end{itemize}

Advertisement

\protect\hyperlink{after-top}{Continue reading the main story}

Supported by

\protect\hyperlink{after-sponsor}{Continue reading the main story}

Shrines

\hypertarget{for-the-missoni-matriarch-a-dizzying-collection-of-hands}{%
\section{For the Missoni Matriarch, a Dizzying Collection of
Hands}\label{for-the-missoni-matriarch-a-dizzying-collection-of-hands}}

\href{https://www.nytimes.com/slideshow/2016/12/05/t-magazine/art/at-home-with-angela-missoni-and-her-hundreds-of-hands.html}{}

\hypertarget{at-home-with-angela-missoni--and-her-hundreds-of-hands}{%
\subsection{At Home With Angela Missoni --- and Her Hundreds of
Hands}\label{at-home-with-angela-missoni--and-her-hundreds-of-hands}}

8 Photos

View Slide Show ›

\includegraphics{https://static01.nyt.com/images/2016/12/02/t-magazine/art/02tmag-missoni-slide-I97E/02tmag-missoni-slide-I97E-articleLarge.jpg?quality=75\&auto=webp\&disable=upscale}

Andrea Wyner

By Alexander Fury

\begin{itemize}
\item
  Dec. 5, 2016
\item
  \begin{itemize}
  \item
  \item
  \item
  \item
  \item
  \end{itemize}
\end{itemize}

Fashion designers base their lives around collections: spring/summer,
fall/winter. But Angela Missoni, 58, scion of the Missoni fashion house
that was founded in 1953 --- and whose languid, luridly colored work
came to epitomize the '70s and the constant revivals of that decade
forever after --- is responsible for more collections than most.

At home, however, her collections are decidedly non-fashion: they're
comprised of \emph{Provençal} French ceramics, Italian glassware, and
various tchotchkes unearthed from flea markets and secondhand stores
around the world. Her latest affection is for those pastel porcelain
poodles from the '50s and '60s. ``It's not an official collection,
yet,'' she says of them, by phone from her home in Milan. ``But very
soon, it might be.''

In 2013, Missoni moved into her villa in Brunello, in the hills to the
northwest of the city, but hasn't yet had the chance to unpack all of
her collections of things. Another home, in Sardinia, houses yet more
\emph{objets trouvés}. Many stay permanently packed, and come out only
for special occasions. At Christmastime, Missoni makes holiday
centerpieces out of the objects: This year, it will be a cluster of
patently unfashionable fancy crystal glassware.

But of all the collections of objects in her homes, Missoni's favorite
is a cluster of ornaments in the shape of hands --- of which she has
amassed hundreds over more than 30 years. In a multitude of materials,
sizes and colors, the pieces overflow across many surfaces. There are
hand-shaped trinket bowls and hand-shaped ashtrays. Clasped hands ---
originally given as wedding souvenirs --- sit alongside palmistry
figurines and photo-printed plates from the Italian brand Fornasetti.
``It's mostly a very kitsch collection,'' Missoni reasons. ``Some of
them are really only decorative,'' while others can be used to display
jewelry, or as paperweights or bookends.

In person, Missoni gesticulates grandly as she converses --- a trait
often perceived as a national Italian characteristic. A theory goes that
hand gestures were used as forms of hidden communication between natives
while different regions of pre-unification Italy spent five centuries
under foreign rule. In fashion, the hand is, of course, connected with
handicraft --- something synonymous with Missoni's multihued,
multitextured multitudes of knits, which often involve handmade
techniques like crochet.

I wonder if those are the reasons Missoni is drawn to hand-shaped
objects again and again. ``I don't know. Very strange. I never thought I
had to talk about it,'' she laughs. ``I do collect many other things,
it's not just the hands.'' She reels off the aforementioned pots, and
poodles. Nevertheless, though her numerous other collections slide in
and out of favor (and in and out of packing boxes), her hands are always
on view.

Missoni is also always on the lookout for them. Hunting for her next big
hand is, she states, her biggest hobby. ``I'm a searcher,'' she says.
``I've been to flea markets since an early age. I was just a kid, but I
never stopped!'' It's something that continues today, with her mother
Rosita or her daughter Margherita. ``It's a feminine passion, very
strong,'' Missoni says. ``Of course, you find objects you didn't expect
to find. Sometime you get surprised by something.'' A trip to any city,
like Paris or New York (Missoni was there in October), automatically
means scouring local flea markets, namely the Brooklyn Flea or the famed
Les Puces at Porte de Clignancourt. But she never has a specific
purchase in mind. ``I want to be surprised at a flea market,'' she says.
``I'm very open. I'm open to everything!''

``Recently I realized,'' she continues, ``that I tend to recreate
families, families of forgotten objects. As soon as I see objects around
that no one cares for, I think they need a home and I start putting them
together.'' Family is an important idea for Missoni. Her company is a
family concern in a uniquely Italian way --- like the Medicis, but their
intrigues have been replaced by intarsia knits. Angela is the eldest
daughter of the founders Tai and Rosita; her own daughter, Margherita,
designed the label's accessories between 2009 and 2015. The rest of the
extended clan is often present at the brand's shows during Milan Fashion
Week. (They are
\href{https://www.nytimes.com/2014/10/17/t-magazine/missoni-cutest-family-ever.html}{a
beautiful bunch of Italians}, often appearing with a couple of Bruce
Weber-worthy dogs in tow, for true picturesque overkill.)

Missoni is, it seems, an inveterate mother, even to her objects. But
still, the hands are something else --- something specific. They are
certainly different (and less obvious) than her other major passion:
amassing an array of costume jewelry. ``It's normal, right, that a woman
collects jewels?'' she laughs. As for the hands, she admits that there
probably is a reason she collects them. ``Probably I love hands because
they mean care,'' she says. ``I love hands because they mean work. I
love hands because they are one of the most important ways of
expression, besides mouth and eyes.''

Advertisement

\protect\hyperlink{after-bottom}{Continue reading the main story}

\hypertarget{site-index}{%
\subsection{Site Index}\label{site-index}}

\hypertarget{site-information-navigation}{%
\subsection{Site Information
Navigation}\label{site-information-navigation}}

\begin{itemize}
\tightlist
\item
  \href{https://help.nytimes.com/hc/en-us/articles/115014792127-Copyright-notice}{©~2020~The
  New York Times Company}
\end{itemize}

\begin{itemize}
\tightlist
\item
  \href{https://www.nytco.com/}{NYTCo}
\item
  \href{https://help.nytimes.com/hc/en-us/articles/115015385887-Contact-Us}{Contact
  Us}
\item
  \href{https://www.nytco.com/careers/}{Work with us}
\item
  \href{https://nytmediakit.com/}{Advertise}
\item
  \href{http://www.tbrandstudio.com/}{T Brand Studio}
\item
  \href{https://www.nytimes.com/privacy/cookie-policy\#how-do-i-manage-trackers}{Your
  Ad Choices}
\item
  \href{https://www.nytimes.com/privacy}{Privacy}
\item
  \href{https://help.nytimes.com/hc/en-us/articles/115014893428-Terms-of-service}{Terms
  of Service}
\item
  \href{https://help.nytimes.com/hc/en-us/articles/115014893968-Terms-of-sale}{Terms
  of Sale}
\item
  \href{https://spiderbites.nytimes.com}{Site Map}
\item
  \href{https://help.nytimes.com/hc/en-us}{Help}
\item
  \href{https://www.nytimes.com/subscription?campaignId=37WXW}{Subscriptions}
\end{itemize}
