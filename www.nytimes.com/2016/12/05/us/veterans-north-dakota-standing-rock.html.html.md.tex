Sections

SEARCH

\protect\hyperlink{site-content}{Skip to
content}\protect\hyperlink{site-index}{Skip to site index}

\href{https://www.nytimes.com/section/us}{U.S.}

\href{https://myaccount.nytimes.com/auth/login?response_type=cookie\&client_id=vi}{}

\href{https://www.nytimes.com/section/todayspaper}{Today's Paper}

\href{/section/us}{U.S.}\textbar{}As North Dakota Pipeline Is Blocked,
Veterans at Standing Rock Cheer

\url{https://nyti.ms/2gIIK2a}

\begin{itemize}
\item
\item
\item
\item
\item
\end{itemize}

Advertisement

\protect\hyperlink{after-top}{Continue reading the main story}

Supported by

\protect\hyperlink{after-sponsor}{Continue reading the main story}

\hypertarget{as-north-dakota-pipeline-is-blocked-veterans-at-standing-rock-cheer}{%
\section{As North Dakota Pipeline Is Blocked, Veterans at Standing Rock
Cheer}\label{as-north-dakota-pipeline-is-blocked-veterans-at-standing-rock-cheer}}

\includegraphics{https://static01.nyt.com/images/2016/12/06/us/06dakota-timeline/06dakota-timeline-videoSixteenByNine3000.jpg}

By \href{http://www.nytimes.com/by/jack-healy}{Jack Healy}

\begin{itemize}
\item
  Dec. 5, 2016
\item
  \begin{itemize}
  \item
  \item
  \item
  \item
  \item
  \end{itemize}
\end{itemize}

FORT YATES, N.D. --- After four deployments to Iraq and Afghanistan,
after being hit by a roadside bomb and losing two friends to explosions,
Jason Brocar floated from job to job, earning enough to pay for long
solo hikes where his only worries were what he would eat and where he
would sleep. He was deep into a rainy trek through Scotland when he
noticed friends back home talking about a place called Standing Rock.

He decided to join them, which is why he was lined up inside a huge shed
this weekend with hundreds of other veterans, some of them Native
Americans, who have come to North Dakota to join the Standing Rock Sioux
Tribe's fight to block an oil pipeline.

On Sunday, they cheered the
\href{https://www.nytimes.com/2016/12/04/us/federal-officials-to-explore-different-route-for-dakota-pipeline.html}{Department
of the Army's announcement} that it would seek other routes for the
pipeline and would not allow a crucial section to be drilled under the
Missouri River just upstream from the tribe's reservation, where there
were worries it could pollute their drinking water and cross near sacred
burial sites.

But President-elect Donald J. Trump's support for finishing the pipeline
means the saga is far from over. His administration could undo the
Sunday decision and order the pipeline through, though the tribe and
environmental activists would almost surely sue to stop him. Reflecting
the continued uncertainty, the veterans were out singing and marching on
Monday in gale-force winds and driving snow.

The presence of many hundreds of veterans --- organizers were
anticipating 2,000 or more --- adds another potent layer to a fight that
is already steeped in sharp contrasts, between a tribe and an oil
company, between environmentalists and pro-energy advocates, between
tan-shirted sheriff's deputies
\href{http://www.nytimes.com/2016/11/21/us/dakota-access-pipeline-protesters-police.html}{armed
with rubber bullets} and water cannons and protesters wearing
traditional dress and feathers in their hair.

``Fall in!'' came a cry one night this weekend. Hundreds of men and
women packed into the building to get their orders from Brenda White
Bull and Loreal Black Shawl, who are leading the
\href{http://www.nytimes.com/2016/11/29/us/veterans-to-serve-as-human-shields-for-pipeline-protesters.html}{veterans'
groups at the protest camps}.

\includegraphics{https://static01.nyt.com/images/2016/12/05/us/05STANDING/05STANDING-articleInline.jpg?quality=75\&auto=webp\&disable=upscale}

The orders, they said, were ``peace and prayer.'' No confrontations
between veterans and law enforcement officers who are guarding a
still-closed highway at what protesters call the front lines. On Monday,
many protesters
\href{http://www.nytimes.com/2016/11/26/us/dakota-pipeline-protest.html}{defied
an order} by the Army Corps of Engineers to leave a campsite north of
the Cannonball River.

``You guys are very symbolic,'' Dave Archambault II, the Standing Rock
Sioux tribal chairman, told the lines of veterans at a meeting at
Sitting Bull College here on the tribe's reservation. ``What you're
doing is sacred.''

Law enforcement officials leading the response to the
\href{http://www.nytimes.com/2016/10/29/us/dakota-access-pipeline-protest.html}{monthslong
protest} in Morton County say they have only used force when threatened
or attacked by protesters.

Sheriff Paul Laney of Cass County said that officers wanted to calm
things down after weeks of rising tensions and violent flare-ups, and
that they were willing to pull back from a blockaded bridge where
several confrontations had occurred. He said protesters first needed to
meet conditions like agreeing not to cross the bridge and not to tear
down barriers or wires that law enforcement had put up.

``We all want this to de-escalate and end peacefully,'' Sheriff Laney
said.

Veterans' views are hardly monolithic, and as the veterans began to
arrive, the Morton County Sheriff's office --- whose ranks include
veterans --- sought to show it had the support of local veterans. The
county released a
\href{https://www.youtube.com/watch?v=C1WzkHhSDPY}{video} featuring
Raymond Morrell, a Marine veteran. He criticized the protests and
questioned why veterans arriving from outside North Dakota would join
what the sheriff has called an unlawful protest.

At a news conference, Sheriff Laney said he had received information
that an ``element'' within the protest camps wanted to exploit veterans
with post-traumatic stress and goad them into acts of violence. Tribal
leaders and protesters say they are nonviolent and have no weapons.

Several of the veterans who lined up wore caps saying, ``Native
Veteran.'' Some were old men, veterans of Korea and postwar Europe, who
said they had grown up in Indian boarding schools where they were beaten
for speaking their language. Some drove in from reservations across the
Plains.

Image

Ben Wright, a Native American and Army veteran who served in Vietnam, at
left, walked with Rob McHaney, a retired Navy diver, through the camp in
Cannon Ball.Credit...Alyssa Schukar for The New York Times

Some of the arriving veterans have spent years in the antiwar movement
after returning from Vietnam or Iraq. They said they saw the pipeline
protests as a new chapter in that activism. They came with open letters
and leaflets, and they raised flags in the camp that fluttered alongside
the names of Native American nations.

Many said they came ready to form a barrier between protesters and law
enforcement.

``A lot of people here are willing to sacrifice their body, willing to
give their life,'' said Vincent Emanuele, 32, a former Marine who served
in Iraq and has spoken out extensively against what he called a futile
war. ``You might as well die for something that means something.''

Others said they did not care much about politics and had never joined a
protest. But they said they had been moved by the tribe's fight to block
a crucial section of the 1,170-mile pipeline. Or they said they were
angry at seeing images of violent clashes between lines of law
enforcement and Native Americans.

``I just couldn't believe what was happening in the United States,'' Mr.
Brocar, 44, said. ``Even in Iraq, there was some rule of engagement. If
these guys don't have weapons, it just doesn't make sense to me that
it's a shooting gallery.''

Like other veterans of Iraq and Afghanistan who came here --- any many
who did not --- he said he had grown disillusioned with the grinding
wars and their human toll. On his wrist were two metal bracelets with
the names of his two dead friends --- ``hometown guys who joined to save
the world.''

After the meeting ended, the veterans dispersed across the dark plains
to sleep, some heading to tents and yurts at the camp, others to
borrowed beds. Robert Abbey, 37, a former soldier who joined the
military at 17 and deployed once to Iraq, ended up sleeping at the
community college.

He said he came because he wanted to help, and to see what was unfolding
five hours north of his home in Hermosa, S.D. Some of the veterans here
said they might stay for weeks, but Mr. Abbey had to get back home for
an appointment at the local Veterans Affairs agency.

Advertisement

\protect\hyperlink{after-bottom}{Continue reading the main story}

\hypertarget{site-index}{%
\subsection{Site Index}\label{site-index}}

\hypertarget{site-information-navigation}{%
\subsection{Site Information
Navigation}\label{site-information-navigation}}

\begin{itemize}
\tightlist
\item
  \href{https://help.nytimes.com/hc/en-us/articles/115014792127-Copyright-notice}{©~2020~The
  New York Times Company}
\end{itemize}

\begin{itemize}
\tightlist
\item
  \href{https://www.nytco.com/}{NYTCo}
\item
  \href{https://help.nytimes.com/hc/en-us/articles/115015385887-Contact-Us}{Contact
  Us}
\item
  \href{https://www.nytco.com/careers/}{Work with us}
\item
  \href{https://nytmediakit.com/}{Advertise}
\item
  \href{http://www.tbrandstudio.com/}{T Brand Studio}
\item
  \href{https://www.nytimes.com/privacy/cookie-policy\#how-do-i-manage-trackers}{Your
  Ad Choices}
\item
  \href{https://www.nytimes.com/privacy}{Privacy}
\item
  \href{https://help.nytimes.com/hc/en-us/articles/115014893428-Terms-of-service}{Terms
  of Service}
\item
  \href{https://help.nytimes.com/hc/en-us/articles/115014893968-Terms-of-sale}{Terms
  of Sale}
\item
  \href{https://spiderbites.nytimes.com}{Site Map}
\item
  \href{https://help.nytimes.com/hc/en-us}{Help}
\item
  \href{https://www.nytimes.com/subscription?campaignId=37WXW}{Subscriptions}
\end{itemize}
