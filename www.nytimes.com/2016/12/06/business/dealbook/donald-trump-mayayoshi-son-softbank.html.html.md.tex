Sections

SEARCH

\protect\hyperlink{site-content}{Skip to
content}\protect\hyperlink{site-index}{Skip to site index}

\href{https://myaccount.nytimes.com/auth/login?response_type=cookie\&client_id=vi}{}

\href{https://www.nytimes.com/section/todayspaper}{Today's Paper}

\href{/section/business/dealbook}{DealBook}\textbar{}After Meeting
Trump, Japanese Mogul Pledges \$50 Billion Investment in the U.S.

\url{https://nyti.ms/2gQdvm4}

\begin{itemize}
\item
\item
\item
\item
\item
\end{itemize}

Advertisement

\protect\hyperlink{after-top}{Continue reading the main story}

Supported by

\protect\hyperlink{after-sponsor}{Continue reading the main story}

DealBook Business and Policy

\hypertarget{after-meeting-trump-japanese-mogul-pledges-50-billion-investment-in-the-us}{%
\section{After Meeting Trump, Japanese Mogul Pledges \$50 Billion
Investment in the
U.S.}\label{after-meeting-trump-japanese-mogul-pledges-50-billion-investment-in-the-us}}

\includegraphics{https://static01.nyt.com/images/2016/12/07/business/07DB-SOFTBANK/07DB-SOFTBANK-articleInline.jpg?quality=75\&auto=webp\&disable=upscale}

By \href{http://www.nytimes.com/by/michael-j-de-la-merced}{Michael J. de
la Merced}

\begin{itemize}
\item
  Dec. 6, 2016
\item
  \begin{itemize}
  \item
  \item
  \item
  \item
  \item
  \end{itemize}
\end{itemize}

The brash and outspoken founder of SoftBank, Masayoshi Son, has long
been known for
\href{https://www.nytimes.com/2016/12/04/business/dealbook/masayoshi-son-softbank-mobile.html}{his
outsize ambitions}, from taking on the United States' biggest telephone
companies to creating a \$100 billion fund for technology deals.

On Tuesday, Mr. Son, a Japanese mogul, struck another big pledge after
meeting with President-elect Donald J. Trump: to invest \$50 billion in
the United States, a move that he said would create some 50,000 jobs.

But the \$50 billion investment pledge is not an entirely new initiative
that SoftBank is undertaking. Instead, the money is projected to come
from the Japanese company's previously announced Vision fund, a \$100
billion vehicle for investing in technology companies worldwide.

The fund --- which includes Saudi Arabia, a target of Mr. Trump's ire
during the presidential campaign, as a key partner --- was always
expected to strike a significant portion of its deals in the United
States.

Speaking to reporters on Tuesday, Mr. Son said the new jobs would come
from investing in American start-ups. Mr. Trump
\href{https://twitter.com/realDonaldTrump/status/806214236053667842}{later
declared} on Twitter, ``Masa said he would never do this had we (Trump)
not won the election!''

Mr. Son's visit to Trump Tower in Manhattan on Tuesday was the latest
outreach by global business leaders to an incoming president seen as
friendlier to corporate interests. On Friday, the Trump transition team
announced
\href{https://www.nytimes.com/2016/12/02/business/dealbook/silicon-valley-chiefs-absent-trump-cabinet-of-business-advisers.html}{the
formation of a business advisory group}, led by Stephen A. Schwarzman,
the chief executive of the private equity behemoth the Blackstone Group,
that would consult monthly with the Trump White House.

The advisory group was noticeably shy of technology names; Mr. Son is
perhaps the most prominent tech executive so far to have met with the
president-elect.

Although it has its headquarters in Japan, SoftBank has deep roots in
the United States. It is the majority owner of the wireless operator
Sprint, which two years ago unsuccessfully pursued a takeover bid for
T-Mobile. That effort was effectively blocked by the Obama
administration on antitrust grounds.

Mr. Son was not expected to discuss specific issues with Mr. Trump
during their meeting, including the prospects of renewing a Sprint bid
for its rival, according to a person briefed on the matter. Shares of
Sprint and T-Mobile briefly rose Tuesday afternoon, before giving up
their gains.

Beyond what Mr. Trump and Mr. Son said, ``we aren't able to say more,''
a SoftBank spokesman in Tokyo, Matthew Nicholson, said.

``I just came to celebrate his new job,'' Mr. Son, 59, told reporters at
Trump Tower. ``I said: `This is great. The U.S. will become great
again.' ''

Such bold talk is customary from the American-educated Mr. Son, who has
built one of Japan's biggest personal fortunes through sometimes brash
deal-making. SoftBank, which began life as a software distributor,
became one of Japan's biggest phone service companies through shrewd
negotiations that gave the company early exclusive rights to the iPhone.
It has since become a global empire with stakes in the likes of Sprint
and Alibaba Group of China and in a welter of start-ups in the United
States and abroad.

This year, Mr. Son struck
\href{https://www.nytimes.com/2016/07/19/business/dealbook/softbank-buys-chip-designer-arm.html}{a
\$32 billion takeover of ARM Holdings}, a British chip designer whose
products sit at the heart of devices like the iPhone.

Like the president-elect, Mr. Son has been known for sometimes impolitic
remarks. As Mr. Son sought to compete against Verizon and AT\&T with his
investment in Sprint, he compared
\href{http://dealbook.nytimes.com/2014/06/05/the-biggest-champion-of-a-sprint-t-mobile-deal-softbanks-chief/}{the
quality of American wireless service to the air quality of Beijing}. And
he threatened to set himself on fire in the offices of Japan's
telecommunications regulator on at least one occasion.

In speaking with reporters alongside Mr. Trump on Tuesday, Mr. Son
clutched what appeared to be a presentation from the meeting. One page
featured the logos of both SoftBank and Foxconn, the Taiwanese
manufacturing giant that assembles Apple's iPhone. On the page --- and
circled --- was the text committing to investing \$50 billion in the
United States and generating 50,000 jobs over the next four years.

Below it appeared to be Mr. Son's signature.

Foxconn said in a statement that it was in preliminary discussions about
a potential American expansion, adding that the scale had not yet been
decided. Foxconn has said in the past that it plans to expand its
limited operations in the United States.

Any Foxconn investment is not likely to mean a major flow of jobs back
to America. The company has been plowing funds into automation and has
said in the past that it would invest in a plant to build robots in
Pennsylvania, indicating that new American jobs would probably be
higher-end and limited in number.

Foxconn employs about one million workers in China. But many of those
jobs are low skilled, and as overall wages have risen in China, their
pay levels have become less appealing to workers there. Maintaining
enough staffing to meet production needs can be a challenge.

Advertisement

\protect\hyperlink{after-bottom}{Continue reading the main story}

\hypertarget{site-index}{%
\subsection{Site Index}\label{site-index}}

\hypertarget{site-information-navigation}{%
\subsection{Site Information
Navigation}\label{site-information-navigation}}

\begin{itemize}
\tightlist
\item
  \href{https://help.nytimes.com/hc/en-us/articles/115014792127-Copyright-notice}{©~2020~The
  New York Times Company}
\end{itemize}

\begin{itemize}
\tightlist
\item
  \href{https://www.nytco.com/}{NYTCo}
\item
  \href{https://help.nytimes.com/hc/en-us/articles/115015385887-Contact-Us}{Contact
  Us}
\item
  \href{https://www.nytco.com/careers/}{Work with us}
\item
  \href{https://nytmediakit.com/}{Advertise}
\item
  \href{http://www.tbrandstudio.com/}{T Brand Studio}
\item
  \href{https://www.nytimes.com/privacy/cookie-policy\#how-do-i-manage-trackers}{Your
  Ad Choices}
\item
  \href{https://www.nytimes.com/privacy}{Privacy}
\item
  \href{https://help.nytimes.com/hc/en-us/articles/115014893428-Terms-of-service}{Terms
  of Service}
\item
  \href{https://help.nytimes.com/hc/en-us/articles/115014893968-Terms-of-sale}{Terms
  of Sale}
\item
  \href{https://spiderbites.nytimes.com}{Site Map}
\item
  \href{https://help.nytimes.com/hc/en-us}{Help}
\item
  \href{https://www.nytimes.com/subscription?campaignId=37WXW}{Subscriptions}
\end{itemize}
