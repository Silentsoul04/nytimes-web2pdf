Sections

SEARCH

\protect\hyperlink{site-content}{Skip to
content}\protect\hyperlink{site-index}{Skip to site index}

\href{https://www.nytimes.com/section/technology}{Technology}

\href{https://myaccount.nytimes.com/auth/login?response_type=cookie\&client_id=vi}{}

\href{https://www.nytimes.com/section/todayspaper}{Today's Paper}

\href{/section/technology}{Technology}\textbar{}With LinkedIn, Microsoft
Looks to Avoid Past Acquisition Busts

\url{https://nyti.ms/2gZIWdv}

\begin{itemize}
\item
\item
\item
\item
\item
\end{itemize}

Advertisement

\protect\hyperlink{after-top}{Continue reading the main story}

Supported by

\protect\hyperlink{after-sponsor}{Continue reading the main story}

\hypertarget{with-linkedin-microsoft-looks-to-avoid-past-acquisition-busts}{%
\section{With LinkedIn, Microsoft Looks to Avoid Past Acquisition
Busts}\label{with-linkedin-microsoft-looks-to-avoid-past-acquisition-busts}}

\includegraphics{https://static01.nyt.com/images/2016/12/08/business/09MICROSOFT/09MICROSOFT-articleLarge.jpg?quality=75\&auto=webp\&disable=upscale}

By \href{http://www.nytimes.com/by/nick-wingfield}{Nick Wingfield}

\begin{itemize}
\item
  Dec. 8, 2016
\item
  \begin{itemize}
  \item
  \item
  \item
  \item
  \item
  \end{itemize}
\end{itemize}

SEATTLE --- Microsoft announced on Thursday that it had completed its
\$26.2 billion
\href{https://www.linkedin.com/pulse/linkedin-microsoft-our-next-play-begins-jeff-weiner?trk=prof-post}{acquisition
of LinkedIn}, the social network for professionals.

There are ample reasons to be skeptical that the deal, the biggest by
far in Microsoft's history, will pay off.

First, the company has not had a great track record with this sort of
thing. Two of Microsoft's largest acquisitions --- the digital
advertising firm aQuantive and the mobile unit of Nokia --- were
disappointments that eventually led to the company writing off nearly
the entire value of the deals, more than \$13 billion in all.

And Microsoft is not the only big company that has ended up wasting
money on acquisitions. In fact,
\href{https://hbr.org/2011/03/the-big-idea-the-new-ma-playbook}{decades}
of
\href{http://www.lek.com/press-releases/ceos-pop-question-they-need-understand-why-most-mergers-fail-and-secret-behind-those}{research}
by academics and consulting firms have shown that from 60 to 80 percent
of mergers and acquisitions end up destroying, rather than creating,
shareholder value.

``Mergers go on anyway, even though there's not much evidence they work
out,'' said Jeffrey Pfeffer, a professor of organizational behavior at
the Stanford Graduate School of Business. ``Everybody believes they are
going to be different.''

Still, the Microsoft of 2016 is different from the unfocused giant of
the past that lurched from deal to deal with wild-eyed ambitions of
catching rivals like Google and Apple. It has a new chief executive who
has made a series of smaller deals that have shown positive results. The
company's stock is trading at record highs.

In a joint interview shortly before their deal closed, Satya Nadella and
Jeff Weiner, the chiefs of Microsoft and LinkedIn, described how they
intended to make the acquisition work, where many before had failed.
``In this case, this is the most substantial, big M\&A that Microsoft
has done in its history,'' Mr. Nadella said. ``So the stakes are
absolutely high.''

A key difference in the way Microsoft has approached the deal is the
degree of independence it plans to give LinkedIn. It will not weave
LinkedIn, which is based in Silicon Valley, into one of its existing
product lines, nor will it treat it like a disconnected business. Mr.
Weiner will remain LinkedIn's chief executive.

``Neither one of us is a Pollyanna,'' said Mr. Weiner. ``We both know
that acquisition integrations are challenging.''

A good model inside Microsoft is the company's \$2.5 billion
\href{http://dealbook.nytimes.com/2014/09/15/microsoft-to-buy-creator-of-minecraft-for-2-5-billion/}{purchase
in 2014 of Mojang}, the developer behind Minecraft, which has continued
to grow under Microsoft's ownership, retaining key employees along the
way. Mr. Nadella and Mr. Weiner said they had also looked to Facebook's
success in acquiring companies like the photo-sharing service Instagram,
while granting them autonomy.

``I absolutely think of LinkedIn as our Instagram,'' Mr. Nadella said.

Both men said that expanding the business of LinkedIn, which has more
470 million members, was what they cared most about. Microsoft will use
its sales and distribution muscle to do just that.

The company's executives on Thursday will outline plans to integrate the
professional identity people have on LinkedIn with Microsoft Outlook and
the rest of the Office suite. LinkedIn members will be able to draft
résumés in Word to update their LinkedIn profiles.

``Satya said starting literally on Day 1 the first priority is growing
LinkedIn,'' Mr. Weiner said. ``That if LinkedIn continues to grow its
membership, if it continues to realize its mission, its vision, if it
continues to grow the business, that's going to create value for
Microsoft.''

So determined was Mr. Nadella to get off on the right foot that he
emailed an unusual request to Mr. Weiner a few days after the
\href{http://www.nytimes.com/2016/06/14/business/dealbook/microsoft-to-buy-linkedin-for-26-2-billion.html}{announcement
of their deal} in June. He asked Mr. Weiner to take the lead on an
integration team responsible for merging their two companies, a
responsibility that normally falls to an executive at the acquiring
company.

``I had to read it at least twice,'' Mr. Weiner said. ``I did a bit of a
double take.''

Microsoft and LinkedIn have spent much of the last six months --- as the
proposed deal went through regulatory reviews --- getting to know each
other.

In a series of meetings the companies called ``learning days,'' teams of
Microsoft employees from the Seattle area flew to LinkedIn's
headquarters in Silicon Valley to talk about what's going on with
Microsoft research, engineers, sales and other areas. LinkedIn employees
flew to Microsoft to do the same.

A Microsoft rival, Salesforce.com, raised objections with European
antitrust regulators, saying the deal would hurt competition. In the
end, the Europeans
\href{https://blogs.microsoft.com/blog/2016/12/06/microsoft-linkedin-deal-cleared-regulators-opening-doors-people-around-world/\#sm.00000d2vaycgfmeqvrs5xuf7ztpe9}{greenlighted}
the deal after Microsoft made a series of minor concessions.

About 10,000 LinkedIn employees will join Microsoft. David B. Yoffie, a
professor at the Harvard Business School who has done studies on
LinkedIn and Microsoft, said one of the greatest challenges in making
the deal work would be retaining talent.

``Critical in any merger of this type is, it's made a lot of people very
rich,'' he said. ``Will they stay?''

Mr. Weiner, for one, said he was committed to LinkedIn.

``I'm in my dream job, and the combination with Microsoft provides that
much more opportunity to realize what it is we set out to do here,'' he
said. ``So I'm not going anywhere.''

Advertisement

\protect\hyperlink{after-bottom}{Continue reading the main story}

\hypertarget{site-index}{%
\subsection{Site Index}\label{site-index}}

\hypertarget{site-information-navigation}{%
\subsection{Site Information
Navigation}\label{site-information-navigation}}

\begin{itemize}
\tightlist
\item
  \href{https://help.nytimes.com/hc/en-us/articles/115014792127-Copyright-notice}{©~2020~The
  New York Times Company}
\end{itemize}

\begin{itemize}
\tightlist
\item
  \href{https://www.nytco.com/}{NYTCo}
\item
  \href{https://help.nytimes.com/hc/en-us/articles/115015385887-Contact-Us}{Contact
  Us}
\item
  \href{https://www.nytco.com/careers/}{Work with us}
\item
  \href{https://nytmediakit.com/}{Advertise}
\item
  \href{http://www.tbrandstudio.com/}{T Brand Studio}
\item
  \href{https://www.nytimes.com/privacy/cookie-policy\#how-do-i-manage-trackers}{Your
  Ad Choices}
\item
  \href{https://www.nytimes.com/privacy}{Privacy}
\item
  \href{https://help.nytimes.com/hc/en-us/articles/115014893428-Terms-of-service}{Terms
  of Service}
\item
  \href{https://help.nytimes.com/hc/en-us/articles/115014893968-Terms-of-sale}{Terms
  of Sale}
\item
  \href{https://spiderbites.nytimes.com}{Site Map}
\item
  \href{https://help.nytimes.com/hc/en-us}{Help}
\item
  \href{https://www.nytimes.com/subscription?campaignId=37WXW}{Subscriptions}
\end{itemize}
