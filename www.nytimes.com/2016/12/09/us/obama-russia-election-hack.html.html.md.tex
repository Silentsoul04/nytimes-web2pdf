Sections

SEARCH

\protect\hyperlink{site-content}{Skip to
content}\protect\hyperlink{site-index}{Skip to site index}

\href{https://www.nytimes.com/section/us}{U.S.}

\href{https://myaccount.nytimes.com/auth/login?response_type=cookie\&client_id=vi}{}

\href{https://www.nytimes.com/section/todayspaper}{Today's Paper}

\href{/section/us}{U.S.}\textbar{}Russian Hackers Acted to Aid Trump in
Election, U.S. Says

\url{https://nyti.ms/2h5Xpoi}

\begin{itemize}
\item
\item
\item
\item
\item
\item
\end{itemize}

Advertisement

\protect\hyperlink{after-top}{Continue reading the main story}

Supported by

\protect\hyperlink{after-sponsor}{Continue reading the main story}

\hypertarget{russian-hackers-acted-to-aid-trump-in-election-us-says}{%
\section{Russian Hackers Acted to Aid Trump in Election, U.S.
Says}\label{russian-hackers-acted-to-aid-trump-in-election-us-says}}

\includegraphics{https://static01.nyt.com/images/2016/12/10/us/10hacks/10hacks-articleInline.jpg?quality=75\&auto=webp\&disable=upscale}

By \href{http://www.nytimes.com/by/david-e-sanger}{David E. Sanger} and
\href{http://www.nytimes.com/by/scott-shane}{Scott Shane}

\begin{itemize}
\item
  Dec. 9, 2016
\item
  \begin{itemize}
  \item
  \item
  \item
  \item
  \item
  \item
  \end{itemize}
\end{itemize}

WASHINGTON --- American intelligence agencies have concluded with ``high
confidence'' that Russia acted covertly in the latter stages of the
presidential campaign to harm Hillary Clinton's chances and promote
Donald J. Trump, according to senior administration officials.

They based that conclusion, in part, on another finding --- which they
say was also reached with high confidence --- that the Russians hacked
the Republican National Committee's computer systems in addition to
their attacks on Democratic organizations, but did not release whatever
information they gleaned from the Republican networks.

In the months before the election, it was largely documents from
Democratic Party systems that were leaked to the public. Intelligence
agencies have concluded that the Russians gave the Democrats' documents
to WikiLeaks.

Republicans have a different explanation for why no documents from their
networks were ever released. Over the past several months, officials
from the Republican committee have consistently said that their networks
were not compromised, asserting that only the accounts of individual
Republicans were attacked. On Friday, a senior committee official said
he had no comment.

Mr. Trump's transition office issued a statement Friday evening
reflecting the deep divisions that emerged between his campaign and the
intelligence agencies over Russian meddling in the election. ``These are
the same people that said Saddam Hussein had weapons of mass
destruction,'' the statement said. ``The election ended a long time ago
in one of the
\href{http://www.politifact.com/wisconsin/statements/2016/nov/21/reince-priebus/despite-losing-popular-vote-donald-trump-won-elect/}{biggest
Electoral College victories} in history. It's now time to move on and
`Make America Great Again.'''

One senior government official, who had been briefed on an F.B.I.
investigation into the matter, said that while there were attempts to
penetrate the Republican committee's systems, they were not successful.

But the intelligence agencies' conclusions that the hacking efforts were
successful, which have been presented to President Obama and other
senior officials, add a complex wrinkle to the question of what the
Kremlin's evolving objectives were in intervening in the American
presidential election.

``We now have high confidence that they hacked the D.N.C. and the
R.N.C., and conspicuously released no documents'' from the Republican
organization, one senior administration official said, referring to the
Russians.

It is unclear how many files were stolen from the Republican committee;
in some cases, investigators never get a clear picture. It is also far
from clear that Russia's original intent was to support Mr. Trump, and
many intelligence officials --- and former officials in Mrs. Clinton's
campaign --- believe that the primary motive of the Russians was to
simply disrupt the campaign and undercut confidence in the integrity of
the vote.

The Russians were as surprised as everyone else at Mr. Trump's victory,
intelligence officials said. Had Mrs. Clinton won, they believe, emails
stolen from the Democratic committee and from senior members of her
campaign could have been used to undercut her legitimacy. The
intelligence agencies' conclusion that Russia tried to help Mr. Trump
was first reported by The Washington Post.

In briefings to the White House and Congress, intelligence officials,
including those from the C.I.A. and the National Security Agency, have
identified individual Russian officials they believe were responsible.
But none have been publicly penalized.

It is possible that in hacking into the Republican committee, Russian
agents were simply hedging their bets. The attack took place in the
spring, the senior officials said, about the same time that a group of
hackers believed to be linked to the G.R.U., Russia's military
intelligence agency, stole the emails of senior officials of the
Democratic National Committee. Intelligence agencies believe that the
Republican committee hack was carried out by the same Russians who
penetrated the Democratic committee and other Democratic groups.

The finding about the Republican committee is expected to be included in
a detailed report of ``lessons learned'' that Mr. Obama has ordered
intelligence agencies to assemble before he leaves office on Jan. 20.
That report is intended, in part, to create a comprehensive history of
the Russian effort to influence the election, and to solidify the
intelligence findings before Mr. Trump is sworn in.

Mr. Trump has repeatedly cast doubt about any intelligence suggesting a
Russian effort to influence the election. ``I don't believe they
interfered,'' he told Time magazine in an interview published this week.
He suggested that hackers could come from China, or that ``it could be
some guy in his home in New Jersey.''

Intelligence officials and private cybersecurity companies believe that
the Democratic National Committee was hacked by two different Russian
cyberunits. One, called ``Cozy Bear'' or ``A.P.T. 29'' by some Western
security experts, is believed to have spent months inside the D.N.C.
computer network, as well as other government and political
institutions, but never made public any of the documents it took.
(A.P.T. stands for ``Advanced Persistent Threat,'' which usually
describes a sophisticated state-sponsored cyberintruder.)

The other, the G.R.U.-controlled unit known as ``Fancy Bear,'' or
``A.P.T. 28,'' is believed to have created two outlets on the internet,
Guccifer 2.0 and DCLeaks, to make Democratic documents public. Many of
the documents were also provided to WikiLeaks, which released them over
many weeks before the Nov. 8 election.

Representative Michael McCaul, the Texas Republican who is the chairman
of the House Homeland Security Committee, said on CNN in September that
the R.N.C. had been hacked by Russia, but then quickly withdrew the
claim.

Mr. McCaul, who was considered by Mr. Trump for secretary of Homeland
Security, initially told CNN's Wolf Blitzer, ``It's important to note,
Wolf, that they have not only hacked into the D.N.C. but also into the
R.N.C.'' He added that ``the Russians have basically hacked into both
parties at the national level, and that gives us all concern about what
their motivations are.''

Minutes later, the R.N.C. issued a statement denying that it had been
hacked. Mr. McCaul subsequently said that he had misspoken, but that it
was true that ``Republican political operatives'' had been the target of
Russian hacking. So were establishment Republicans with no ties to the
campaign, including former Secretary of State Colin L. Powell.

Mr. McCaul may have had in mind a
\href{http://dcleaks.com/index.php/the-united-states-republican-party/}{collection
of more than 200 emails} of Republican officials and activists that
appeared this year on the website
\href{http://dcleaks.com/}{DCLeaks.com}. That website got far more
attention for the many Democratic Party documents it posted.

The messages stolen from Republicans have drawn little attention because
most are routine business emails from local Republican Party officials
in several states, congressional staff members and party activists.

Among those whose emails were posted was Peter W. Smith, who runs a
venture capital firm in Chicago and has long been active in ``opposition
research'' for the Republican Party. He said he was unaware that his
emails had been hacked until he was called by a reporter on Thursday.

He said he believes that his material came from a hack of the Illinois
Republican Party.

``I'm not upset at all,'' he said. ``I try in my communications, quite
frankly, not to say anything that would be embarrassing if made
public.''

Advertisement

\protect\hyperlink{after-bottom}{Continue reading the main story}

\hypertarget{site-index}{%
\subsection{Site Index}\label{site-index}}

\hypertarget{site-information-navigation}{%
\subsection{Site Information
Navigation}\label{site-information-navigation}}

\begin{itemize}
\tightlist
\item
  \href{https://help.nytimes.com/hc/en-us/articles/115014792127-Copyright-notice}{©~2020~The
  New York Times Company}
\end{itemize}

\begin{itemize}
\tightlist
\item
  \href{https://www.nytco.com/}{NYTCo}
\item
  \href{https://help.nytimes.com/hc/en-us/articles/115015385887-Contact-Us}{Contact
  Us}
\item
  \href{https://www.nytco.com/careers/}{Work with us}
\item
  \href{https://nytmediakit.com/}{Advertise}
\item
  \href{http://www.tbrandstudio.com/}{T Brand Studio}
\item
  \href{https://www.nytimes.com/privacy/cookie-policy\#how-do-i-manage-trackers}{Your
  Ad Choices}
\item
  \href{https://www.nytimes.com/privacy}{Privacy}
\item
  \href{https://help.nytimes.com/hc/en-us/articles/115014893428-Terms-of-service}{Terms
  of Service}
\item
  \href{https://help.nytimes.com/hc/en-us/articles/115014893968-Terms-of-sale}{Terms
  of Sale}
\item
  \href{https://spiderbites.nytimes.com}{Site Map}
\item
  \href{https://help.nytimes.com/hc/en-us}{Help}
\item
  \href{https://www.nytimes.com/subscription?campaignId=37WXW}{Subscriptions}
\end{itemize}
