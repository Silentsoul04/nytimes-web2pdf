Sections

SEARCH

\protect\hyperlink{site-content}{Skip to
content}\protect\hyperlink{site-index}{Skip to site index}

\href{https://www.nytimes.com/section/politics}{Politics}

\href{https://myaccount.nytimes.com/auth/login?response_type=cookie\&client_id=vi}{}

\href{https://www.nytimes.com/section/todayspaper}{Today's Paper}

\href{/section/politics}{Politics}\textbar{}Trump Links C.I.A. Reports
on Russia to Democrats' Shame Over Election

\url{https://nyti.ms/2hcO6TE}

\begin{itemize}
\item
\item
\item
\item
\item
\end{itemize}

Advertisement

\protect\hyperlink{after-top}{Continue reading the main story}

Supported by

\protect\hyperlink{after-sponsor}{Continue reading the main story}

\hypertarget{trump-links-cia-reports-on-russia-to-democrats-shame-over-election}{%
\section{Trump Links C.I.A. Reports on Russia to Democrats' Shame Over
Election}\label{trump-links-cia-reports-on-russia-to-democrats-shame-over-election}}

\includegraphics{https://static01.nyt.com/images/2016/12/12/us/12TRUMP/12TRUMP-articleLarge.jpg?quality=75\&auto=webp\&disable=upscale}

By \href{https://www.nytimes.com/by/nicholas-fandos}{Nicholas Fandos}

\begin{itemize}
\item
  Dec. 11, 2016
\item
  \begin{itemize}
  \item
  \item
  \item
  \item
  \item
  \end{itemize}
\end{itemize}

WASHINGTON --- President-elect Donald J. Trump said in an interview
broadcast on Sunday that he did not believe American intelligence
assessments that Russia had intervened to help his candidacy, casting
blame for the reports on Democrats, who he said were embarrassed about
losing to him.

``I think it's ridiculous. I think it's just another excuse,'' Mr. Trump
said in the interview, on ``Fox News Sunday.'' ``I don't believe it.''

He also indicated that as president, he would not take the daily
intelligence briefing that President Obama and his predecessors have
received. Mr. Trump, who has received the briefing sparingly as
president-elect, said that it was often repetitive and that he would
take it ``when I need it.'' He said his vice president, Mike Pence,
would receive the daily briefing.

``You know, I'm, like, a smart person,'' he said. ``I don't have to be
told the same thing in the same words every single day for the next
eight years.''

He added that he had instructed the officials who give the briefing:
```If something should change from this point, immediately call me. I'm
available on a one-minute's notice.'''

Mr. Trump's seeming dismissal of the importance of that daily
interaction with intelligence agencies, as well as his claims of
politically tainted intelligence reports on Russia,
\href{https://www.nytimes.com/2016/12/10/us/politics/trump-mocking-claim-that-russia-hacked-election-at-odds-with-gop.html?hp\&action=click\&pgtype=Homepage\&clickSource=story-heading\&module=first-column-region\&region=top-news\&WT.nav=top-news}{widened
a breach} between a president-elect and the agencies he will have to
rely on to carry out priorities like fighting terrorism and deterring
cyberattacks.

His stance on the issue is also putting him increasingly at odds with
senior lawmakers on Capitol Hill, including members of his own party,
who say that the evidence of Russian interference is clear and warrants
a congressional investigation.

The Obama administration reached a consensus months ago that Russia was
trying to meddle in the election. After initially believing that
Russia's goal was to undermine American democratic processes, the
intelligence agencies concluded a week after the vote that the Russian
efforts had been intended, at least in their latter stages, to help Mr.
Trump.

The president-elect said those new reports were politically motivated.
``I think the Democrats are putting it out because they suffered one of
the greatest defeats in the history of politics in this country,'' he
said in the interview, recorded on Saturday. During the campaign, he
also dismissed suggestions of Russian meddling.

Pressed about why he did not believe the intelligence agencies'
conclusions, Mr. Trump said there was
\href{https://www.washingtonpost.com/world/national-security/fbi-and-cia-give-differing-accounts-to-lawmakers-on-russias-motives-in-2016-hacks/2016/12/10/c6dfadfa-bef0-11e6-94ac-3d324840106c_story.html?hpid=hp_hp-top-table-main_russiahack816pm\%3Ahomepage\%2Fstory\&utm_term=.a659d909067f}{disagreement
among intelligence agencies} about the extent and the origin of the
hacking.

``They're fighting among themselves,'' he said. ``They're not sure.''

The Washington Post and The New York Times reported on Friday that
American intelligence agencies
\href{http://www.nytimes.com/2016/12/09/us/obama-russia-election-hack.html?action=click\&contentCollection=Politics\&module=RelatedCoverage\&region=Marginalia\&pgtype=article}{had
concluded} that Russia took covert action during the campaign to harm
the candidacy of Hillary Clinton. The new conclusion, The Times
reported, was based in part on evidence found by the C.I.A. that Russian
hackers had penetrated the Republican National Committee's computer
system, as well as that of the Democrats and several of Mrs. Clinton's
senior aides, but had leaked only Democratic correspondence.

Mr. Trump's transition office responded to those reports with a
statement on Friday night dismissing the intelligence agencies as ``the
same people that said Saddam Hussein had weapons of mass destruction.''
The office said it was time to ``move on'' from the election.

The Iraq case has been the subject of a long-running debate over whether
the intelligence was tainted or whether the Bush White House read it
selectively to support its decision to go to war.

On the subject of Russian interference, Senator John McCain, Republican
of Arizona and the chairman of the Armed Services Committee, said Sunday
that it would be dangerous to dismiss the issue as a matter of partisan
politics. He urged Mr. Trump to accept the agencies' conclusions, and
called on his colleagues to move forward with an investigation.

``You can't make this issue partisan; it's too important,'' Mr. McCain
said on the CBS program ``Face the Nation.'' ``A fundamental of
democracy is a free and fair election.''

Referring to the hacking, Mr. McCain added, ``The Russians have been
using it as a tool as part of Vladimir Putin's ambition to regain
Russian prominence and dominance in some parts of the world.''

Mr. McCain was among a bipartisan group of four senior lawmakers,
including the coming minority leader, Chuck Schumer of New York, who
issued
\href{http://www.armed-services.senate.gov/press-releases/mccain-graham-schumer-reed-joint-statement-on-reports-that-russia-interfered-with-the-2016-election}{a
statement} Sunday pledging to work to respond to the incursions. The
statement adds pressure to Republicans, who control Congress, to
investigate the hacking.

``Democrats and Republicans must work together, and across the
jurisdictional lines of the Congress, to examine these recent incidents
thoroughly and devise comprehensive solutions to deter and defend
against further cyberattacks,'' the statement said.

Several senators, including Rand Paul of Kentucky and James Lankford of
Oklahoma, both Republicans, expressed support for such an investigation
on Sunday.

The Senate majority leader, Mitch McConnell of Kentucky, did not comment
on the issue over the weekend, but he was expected to address it in a
news conference scheduled for Monday morning.

Mr. McCain said on ``Face the Nation'' that he would like to see a
select committee formed to look into the C.I.A.'s conclusions, but that
in the meantime, an armed services subcommittee under his control would
``go to work on it.''

Advertisement

\protect\hyperlink{after-bottom}{Continue reading the main story}

\hypertarget{site-index}{%
\subsection{Site Index}\label{site-index}}

\hypertarget{site-information-navigation}{%
\subsection{Site Information
Navigation}\label{site-information-navigation}}

\begin{itemize}
\tightlist
\item
  \href{https://help.nytimes.com/hc/en-us/articles/115014792127-Copyright-notice}{©~2020~The
  New York Times Company}
\end{itemize}

\begin{itemize}
\tightlist
\item
  \href{https://www.nytco.com/}{NYTCo}
\item
  \href{https://help.nytimes.com/hc/en-us/articles/115015385887-Contact-Us}{Contact
  Us}
\item
  \href{https://www.nytco.com/careers/}{Work with us}
\item
  \href{https://nytmediakit.com/}{Advertise}
\item
  \href{http://www.tbrandstudio.com/}{T Brand Studio}
\item
  \href{https://www.nytimes.com/privacy/cookie-policy\#how-do-i-manage-trackers}{Your
  Ad Choices}
\item
  \href{https://www.nytimes.com/privacy}{Privacy}
\item
  \href{https://help.nytimes.com/hc/en-us/articles/115014893428-Terms-of-service}{Terms
  of Service}
\item
  \href{https://help.nytimes.com/hc/en-us/articles/115014893968-Terms-of-sale}{Terms
  of Sale}
\item
  \href{https://spiderbites.nytimes.com}{Site Map}
\item
  \href{https://help.nytimes.com/hc/en-us}{Help}
\item
  \href{https://www.nytimes.com/subscription?campaignId=37WXW}{Subscriptions}
\end{itemize}
