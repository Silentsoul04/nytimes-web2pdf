Sections

SEARCH

\protect\hyperlink{site-content}{Skip to
content}\protect\hyperlink{site-index}{Skip to site index}

\href{https://www.nytimes.com/section/politics}{Politics}

\href{https://myaccount.nytimes.com/auth/login?response_type=cookie\&client_id=vi}{}

\href{https://www.nytimes.com/section/todayspaper}{Today's Paper}

\href{/section/politics}{Politics}\textbar{}Trump Suggests Using Bedrock
China Policy as Bargaining Chip

\url{https://nyti.ms/2hdI05H}

\begin{itemize}
\item
\item
\item
\item
\item
\end{itemize}

Advertisement

\protect\hyperlink{after-top}{Continue reading the main story}

Supported by

\protect\hyperlink{after-sponsor}{Continue reading the main story}

\hypertarget{trump-suggests-using-bedrock-china-policy-as-bargaining-chip}{%
\section{Trump Suggests Using Bedrock China Policy as Bargaining
Chip}\label{trump-suggests-using-bedrock-china-policy-as-bargaining-chip}}

\includegraphics{https://static01.nyt.com/images/2016/12/12/us/12diplo1/12diplo1-videoSixteenByNineJumbo1600.jpg}

By \href{http://www.nytimes.com/by/mark-landler}{Mark Landler}

\begin{itemize}
\item
  Dec. 11, 2016
\item
  \begin{itemize}
  \item
  \item
  \item
  \item
  \item
  \end{itemize}
\end{itemize}

\href{http://cn.nytimes.com/world/20161212/trump-taiwan-one-china/}{阅读简体中文版}

WASHINGTON --- President-elect Donald J. Trump, defending
\href{https://www.nytimes.com/2016/12/02/us/politics/trump-speaks-with-taiwans-leader-a-possible-affront-to-china.html}{his
recent phone call with Taiwan's president}, asserted in an interview
broadcast on Sunday that the United States was not bound by the One
China policy, the 44-year diplomatic understanding that underpins
America's relationship with its biggest rival.

Mr. Trump, speaking on Fox News, said he understood the principle of a
single China that includes Taiwan, but declared, ``I don't know why we
have to be bound by a One China policy unless we make a deal with China
having to do with other things, including trade.''

``I mean, look,'' he continued, ``we're being hurt very badly by China
with devaluation; with taxing us heavy at the borders when we don't tax
them; with building a massive fortress in the middle of the South China
Sea, which they shouldn't be doing; and, frankly, with not helping us at
all with North Korea.''

Mr. Trump is not the first incoming Republican president to question the
One China policy, but his suggestion that it could be used as a chip to
correct Chinese behavior sets him apart, several Asia experts said.
While Mr. Trump has been praised by some Republicans for taking a new
look at China policy, his stance could risk a backlash by Beijing, the
analysts said.

Not since 1972, when President Richard M. Nixon and Mao Zedong enshrined
the One China principle in the Shanghai Communiqué, has an American
president or president-elect so publicly and explicitly questioned the
agreement, which resulted in the United States' ending its diplomatic
recognition of Taiwan in 1979.

A spokesman for the Chinese Foreign Ministry said on Monday that the
government had ``serious concern'' about Mr. Trump's remarks, renewing a
debate that erupted nine days ago when he took a congratulatory phone
call from President Tsai Ing-wen of Taiwan.

At first, Mr. Trump played down the implications of the call, saying he
was just being polite. Later, his aides said he was well aware of the
diplomatic repercussions of speaking to Taiwan's leader. Lobbyists for
Taiwan,
\href{https://www.nytimes.com/2016/12/06/us/politics/bob-dole-taiwan-lobby-trump.html}{including
the law firm of former Senator Bob Dole of Kansas}, spent months laying
the groundwork for the call.

On Friday, China's senior foreign policy official, Yang Jiechi, met with
Lt. Gen. Michael T. Flynn, whom Mr. Trump has designated as his national
security adviser, according to a person told about the meeting. It was
not clear what the men had discussed.

Some Republican foreign policy experts --- including John R. Bolton, who
is believed to be a front-runner for the post of deputy secretary of
state --- have praised Mr. Trump for shaking up a decades-old diplomatic
agreement.

As a candidate, Ronald Reagan criticized the decision to abrogate
recognition of Taiwan; after his election, he invited a delegation from
Taiwan to attend his inauguration, antagonizing Beijing.

In 1982, as president, Reagan pushed for the so-called Six Assurances,
one of which was a reaffirmation that the United States did not formally
recognize Chinese sovereignty over Taiwan. Still, he abided by the terms
of the 1979 joint communiqué that established relations between the
United States and China.

\includegraphics{https://static01.nyt.com/images/2016/12/12/us/12diplo2/12diplo2-articleInline.jpg?quality=75\&auto=webp\&disable=upscale}

But Mr. Trump's suggestion that the policy could be wielded as a chip in
a broader negotiation with China has implications not just for
Washington's relationship with Beijing, several experts on Asia said,
but also for America's support for Taiwan.

``By putting One China up for grabs, Trump will suck all the oxygen out
of the U.S.-China relationship, and it risks eventually trading away
U.S. support for Taiwan for another U.S. interest,'' said Evan Medeiros,
a former senior director for Asia at the National Security Council.

``There are good reasons why eight presidents since 1972 have relied on
the One China policy,'' he added. ``This is one area where the Trump
team would do well to heed the lessons of history instead of bucking
them in the uncertain hope of getting something.''

Jeffrey A. Bader, Mr. Medeiros's predecessor in the Obama
administration, said the One China policy had ``always been seen as a
foundation of the relationship.''

``Now Trump apparently sees it as part of a broader set of new
transactions,'' he said. ``Mixing trade with an issue seen by Beijing as
involving sovereignty is likely to produce an angry Chinese backlash and
worsen both issues.''

\href{http://opinion.huanqiu.com/editorial/2016-12/9797239.html}{An
editorial} on Monday in The Global Times, a Chinese state-run tabloid,
said that Mr. Trump was ``like a child in his ignorance of foreign
policy.''

``The One China policy cannot be bought and sold,'' the editorial said.
``Trump, it seems, only understands business and believes that
everything has a price.''

Mr. Trump, however, did not appear worried about inflaming Beijing. He
repeated in the Fox News interview many of the criticisms he has made
about China, emphasizing what he said was its unwillingness to help curb
the nuclear ambitions of its neighbor North Korea --- an issue that
foreign policy experts believe could confront Mr. Trump as the first
geopolitical crisis of his presidency.

The president-elect said he would not tolerate having the Chinese
government dictate whether he could take a call from the president of
Taiwan. He reiterated that he had not placed the call, and described it
as ``a very short call saying, `Congratulations, sir, on the victory.'''

The Chinese government, which once viewed Mr. Trump favorably as an
alternative to the hawkish Hillary Clinton, has struggled to respond to
Mr. Trump's unorthodox approach. China's foreign minister, Wang Yi,
initially played down the significance of the phone call, calling it a
``petty action by the Taiwan side'' that he said would not upset the
longstanding policy of One China.

But as Mr. Trump has repeated his campaign criticisms of China --- and
as his statements about Taiwan have rippled throughout the region ---
Beijing has noticeably hardened its tone. It warned him last week, in a
front-page editorial in the overseas edition of People's Daily, that
``creating troubles for the China-U.S. relationship is creating troubles
for the U.S. itself.''

In a pointed rejoinder to Mr. Trump, the editorial said that pushing
China on Taiwan ``would greatly reduce the chance to achieve the goal of
making America great again.''

Advertisement

\protect\hyperlink{after-bottom}{Continue reading the main story}

\hypertarget{site-index}{%
\subsection{Site Index}\label{site-index}}

\hypertarget{site-information-navigation}{%
\subsection{Site Information
Navigation}\label{site-information-navigation}}

\begin{itemize}
\tightlist
\item
  \href{https://help.nytimes.com/hc/en-us/articles/115014792127-Copyright-notice}{©~2020~The
  New York Times Company}
\end{itemize}

\begin{itemize}
\tightlist
\item
  \href{https://www.nytco.com/}{NYTCo}
\item
  \href{https://help.nytimes.com/hc/en-us/articles/115015385887-Contact-Us}{Contact
  Us}
\item
  \href{https://www.nytco.com/careers/}{Work with us}
\item
  \href{https://nytmediakit.com/}{Advertise}
\item
  \href{http://www.tbrandstudio.com/}{T Brand Studio}
\item
  \href{https://www.nytimes.com/privacy/cookie-policy\#how-do-i-manage-trackers}{Your
  Ad Choices}
\item
  \href{https://www.nytimes.com/privacy}{Privacy}
\item
  \href{https://help.nytimes.com/hc/en-us/articles/115014893428-Terms-of-service}{Terms
  of Service}
\item
  \href{https://help.nytimes.com/hc/en-us/articles/115014893968-Terms-of-sale}{Terms
  of Sale}
\item
  \href{https://spiderbites.nytimes.com}{Site Map}
\item
  \href{https://help.nytimes.com/hc/en-us}{Help}
\item
  \href{https://www.nytimes.com/subscription?campaignId=37WXW}{Subscriptions}
\end{itemize}
