Sections

SEARCH

\protect\hyperlink{site-content}{Skip to
content}\protect\hyperlink{site-index}{Skip to site index}

\href{https://myaccount.nytimes.com/auth/login?response_type=cookie\&client_id=vi}{}

\href{https://www.nytimes.com/section/todayspaper}{Today's Paper}

\href{/section/business/dealbook}{DealBook}\textbar{}Trump's Treasury
Pick Moves in Secretive Hedge Fund Circles

\url{https://nyti.ms/2i3iP6M}

\begin{itemize}
\item
\item
\item
\item
\item
\end{itemize}

Advertisement

\protect\hyperlink{after-top}{Continue reading the main story}

Supported by

\protect\hyperlink{after-sponsor}{Continue reading the main story}

DealBook Business and Policy

\hypertarget{trumps-treasury-pick-moves-in-secretive-hedge-fund-circles}{%
\section{Trump's Treasury Pick Moves in Secretive Hedge Fund
Circles}\label{trumps-treasury-pick-moves-in-secretive-hedge-fund-circles}}

\includegraphics{https://static01.nyt.com/images/2016/12/20/business/20DB-MNUCHIN1/17DB-MNUCHIN1-articleInline.jpg?quality=75\&auto=webp\&disable=upscale}

By \href{http://www.nytimes.com/by/matthew-goldstein}{Matthew Goldstein}
and \href{http://www.nytimes.com/by/alexandra-stevenson}{Alexandra
Stevenson}

\begin{itemize}
\item
  Dec. 19, 2016
\item
  \begin{itemize}
  \item
  \item
  \item
  \item
  \item
  \end{itemize}
\end{itemize}

As a hedge fund manager, Goldman Sachs trader and bank chief executive,
Steven T. Mnuchin has long been a member of the financial elite.

Yet even on Wall Street he was not widely known before Donald J. Trump
chose him to be his campaign fund-raiser last spring.

Now, Mr. Mnuchin is on a path to become the first hedge fund manager to
head the Treasury. As befitting that closed-door world of finance, Mr.
Mnuchin's record shows a willingness to take on risks and a penchant for
secrecy that members of both parties expect will be a focus of his
Senate confirmation hearing.

A case in point is a Delaware company that he owns, Steven T. Mnuchin
Inc., whose existence has not been reported outside of official records.

Mr. Mnuchin set up the company months before Goldman went public in May
1999. It had a Goldman mailing address and at least one Goldman
director. It stayed that way years after he left the Wall Street bank in
2002 to start his hedge fund career.

Its purpose is a mystery. The only clue is in a description on corporate
filings in Delaware calling the entity an ``investment in
partnerships.''

Barney Keller, a representative for Mr. Mnuchin, said the company ``held
small legacy Goldman Sachs investments'' and ``hasn't made any new
investments since 2002.''

Over the course of his career, Mr. Mnuchin, 53, has kept a low profile
and has rarely granted interviews. His one previous spell in the
limelight was when he ran OneWest Bank.

He and other deep-pocketed investors cobbled together the California
bank out of the wreckage of IndyMac, a lender that the federal
government seized in 2008 as the financial crisis grabbed hold of the
nation. His tenure at OneWest spurred controversy because tens of
thousands of borrowers --- many of them older people --- were foreclosed
on.

When a group of protesters gathered outside his Los Angeles house in
October 2011 to voice their anger at OneWest's foreclosure practices,
police officers were called to disperse the crowd. He responded later by
periodically scrubbing the internet of any reference to his home address
to protect the privacy of his three young children, according to a
filing Mr. Mnuchin made in a 2014 divorce proceeding.

He is more comfortable in his own element --- wealthy hedge fund
managers like himself.

Days after being tapped by Mr. Trump to be his finance chairman in May,
Mr. Mnuchin flew out to a hedge fund conference at the Bellagio hotel in
Las Vegas. There, in a swirl of rock concerts, pool parties and casino
nights, he schmoozed with billionaires.

Mr. Mnuchin dined with Kenneth C. Griffin, the Chicago hedge fund titan,
and government veterans like David H. Petraeus, the former C.I.A.
director and retired general, and John A. Boehner, the former House
speaker. He shook hands with Leon Cooperman, a 73-year-old pioneer in
the industry and another Goldman Sachs alumnus.

The courting paid off: Not long after the hedge fund conference, Mr.
Mnuchin helped bring the billionaire investor John A. Paulson on as a
donor, as well as Wilbur L. Ross Jr., now Mr. Trump's
\href{https://www.nytimes.com/2016/11/30/business/dealbook/trumps-economic-cabinet-picks-signal-embrace-of-wall-st-elite.html}{commerce
secretary nominee}.

It is these ties, however, that are expected to be fertile ground of
investigation for senators who must confirm his selection as Treasury
secretary.

``I think he would be controversial --- former Treasury secretaries have
been top bankers and businessmen or lawyers and they were
well-recognized top leaders in their professions,'' said Prof. Richard
Sylla at the Stern School of Business of New York University, whose
research focuses on financial history and economics. ``But Mnuchin
wasn't that famous as a Goldman Sachs guy, and probably not even as a
hedge fund guy, so why is he there? Well, he took the job of raising
money for Trump.''

Mr. Mnuchin shares a network with some of the biggest names in the \$3
trillion hedge fund industry, but his work with Mr. Paulson is of
particular interest.

An adviser to Mr. Trump, Mr. Paulson, who is best known for making a
\$15 billion wager that the housing market would collapse, has managed
some money for the president-elect.

The day after Mr. Trump formally selected him, Mr. Mnuchin said that the
government should get out of the business of
\href{http://www.foxbusiness.com/politics/2016/11/30/steve-mnuchin-time-to-jettison-fannie-mae-freddie-mac.html}{running
Fannie Mae and Freddie Mac}, the two giant mortgage finance firms that
the federal government
\href{http://www.nytimes.com/2008/09/08/business/08takeover.html}{bailed
out} in 2008. The shares of the two companies soared in response to his
words --- some of the sharpest one-day gains for the stocks since the
government conservatorship began.

Image

Steve Mnuchin and other investors cobbled together OneWest Bank from the
wreckage of IndyMac, a lender that the federal government seized in
2008.Credit...Rex Features, via Associated Press

The seemingly off-the-cuff declaration was cheered by hedge fund
managers and others who have bet big on the privatization of Fannie and
Freddie. One of Mr. Paulson's bigger bets is that the federal government
will return Fannie and Freddie to private investors largely unfettered.

And Mr. Mnuchin's remark may have benefited someone else: Mr. Trump has
invested \$3 million to \$15 million with Mr. Paulson's hedge fund,
according to the financial disclosure statement filed during his run for
president.

To date, the Trump transition team has not commented on whether Mr.
Trump continues to own stakes in Mr. Paulson's firm and other hedge
funds listed on the disclosure statement. ``We're not sharing any
additional information at this time,'' said Jason Miller, a Trump
representative.

The president-elect has said that he will announce a plan for separating
himself from his many business interests before his inauguration on Jan.
20. Similarly, Mr. Mnuchin will have to file financial disclosures with
the Senate Finance Committee before his confirmation hearing. On Monday,
he began the process by filing three years of tax returns.

Mr. Mnuchin's time at his hedge fund, Dune Capital Management, which he
formed with two former Goldman colleagues in 2004, is one area that the
Senate Finance Committee is expected to examine more closely.

At its peak, the firm had roughly \$2 billion and was backed by the
billionaire investor George Soros. It had a taste for real estate, movie
financing deals and exotic investments including life insurance
policies, which Dune bought through a third party at discounted prices
from cash-poor older Americans.

Dune had plans to package the insurance policies --- called life
settlements --- into bonds that could be sold to investors. Life
settlements represent one of the most macabre actuarial bets that Wall
Street has dreamed up. It's a wager that the elderly person selling the
policy will die sooner rather than later, meaning the hedge fund does
not have to make many premium payments to keep the insurance policy in
force and collect the payout upon that person's death.

But the market for life settlements largely collapsed during the
financial crisis.

Eventually, Dune, like many hedge funds during the worst of the crisis,
faced investor withdrawals. Mr. Mnuchin and one of his co-founding
partners, Chip Seelig, decided to wind down the operation. The real
estate arm of Dune was spun off into a firm led by Dune's third
co-founder, Daniel M. Neidich.

After the split, Mr. Mnuchin and Mr. Seelig set up shop in Southern
California. Both became big players in movie financing, lending money to
hits like ``Avatar'' and ``Gravity'' along with flops like ``In the
Heart of the Sea.''

But it was the deal for a failed California bank, IndyMac, that
contributed mightily to Mr. Mnuchin's personal fortune and the one deal
that has created the most political heat. On Friday, Senate Democrats,
in a sign that they intend to go hard after Mr. Mnuchin,
\href{https://democrats.senate.gov/foreclosureking/\#.WFfkiVMrLIX}{set
up a website} that calls him the ``foreclosure king,'' and asked
customers of OneWest to submit complaints in advance of the confirmation
hearing.

In late 2008, Mr. Mnuchin joined with Mr. Paulson to put together the
OneWest deal. They were part of the initial round of bidding for the
assets of IndyMac, which the Federal Deposit Insurance Corporation had
seized that summer. The group, which also included Mr. Soros and other
investors, won the deal with a \$1.55 billion bid.

One of the competing bids was put together by J. Tomilson Hill, head of
the Blackstone Group's hedge fund and asset management group, and
Goldman.

``As a deal person, I had nothing but admiration for how entrepreneurial
he was and how quickly he adapted to the circumstances,'' Mr. Hill said
in a recent interview. He was one of the first to call Mr. Mnuchin to
congratulate him on his new job as campaign finance chairman.

The remnants of IndyMac reopened under the name OneWest with Mr. Mnuchin
as chairman and chief executive. But controversy followed, including
complaints over foreclosures on soured mortgages that IndyMac had
written before the crisis. In particular, it drew scrutiny for its
business in reverse mortgages --- products pitched to older people who
had paid off their initial mortgages but needed cash.

The Department of Housing and Urban Development is investigating
complaints about the foreclosure practices of OneWest's reverse mortgage
business, which resulted in 16,000 foreclosures alone.

CIT Group, which
\href{https://www.nytimes.com/2015/07/22/business/dealbook/regulators-approve-merger-of-cit-and-onewest-in-3-4-billion-deal.html}{merged
with OneWest} last year in a \$3.4 billion deal that provided a hefty
payout for Mr. Mnuchin and his co-investors, has said it is trying to
resolve the HUD inquiry. It is also dealing with a \$230 million charge
the company said in July it had to take in connection with accounting
issues from the reverse mortgage business.

An analysis in 2015 presented by the California Reinvestment Coalition,
which lobbied against the merger, found that 68 percent of the 36,000
foreclosures in California by OneWest, including those on reverse
mortgages, occurred in communities that were primarily nonwhite. A
government filing shows OneWest offered to modify about 101,000
mortgages for its customers.

Mr. Mnuchin stepped down from the helm of OneWest in March, not long
before he became among the first in finance to come out for Mr. Trump
and against Hillary Clinton. Mr. Keller said Mr. Mnuchin was ``proud of
his record at OneWest.''

Still, the worlds of finance and politics can be small ones, and Mr.
Mnuchin even has a connection to Mrs. Clinton. The in-house accountant
at Goldman who prepared a tax filing for Steven T. Mnuchin Inc. many
years ago has among her clients, at her current accounting firm, the
former Democratic presidential candidate.

Advertisement

\protect\hyperlink{after-bottom}{Continue reading the main story}

\hypertarget{site-index}{%
\subsection{Site Index}\label{site-index}}

\hypertarget{site-information-navigation}{%
\subsection{Site Information
Navigation}\label{site-information-navigation}}

\begin{itemize}
\tightlist
\item
  \href{https://help.nytimes.com/hc/en-us/articles/115014792127-Copyright-notice}{©~2020~The
  New York Times Company}
\end{itemize}

\begin{itemize}
\tightlist
\item
  \href{https://www.nytco.com/}{NYTCo}
\item
  \href{https://help.nytimes.com/hc/en-us/articles/115015385887-Contact-Us}{Contact
  Us}
\item
  \href{https://www.nytco.com/careers/}{Work with us}
\item
  \href{https://nytmediakit.com/}{Advertise}
\item
  \href{http://www.tbrandstudio.com/}{T Brand Studio}
\item
  \href{https://www.nytimes.com/privacy/cookie-policy\#how-do-i-manage-trackers}{Your
  Ad Choices}
\item
  \href{https://www.nytimes.com/privacy}{Privacy}
\item
  \href{https://help.nytimes.com/hc/en-us/articles/115014893428-Terms-of-service}{Terms
  of Service}
\item
  \href{https://help.nytimes.com/hc/en-us/articles/115014893968-Terms-of-sale}{Terms
  of Sale}
\item
  \href{https://spiderbites.nytimes.com}{Site Map}
\item
  \href{https://help.nytimes.com/hc/en-us}{Help}
\item
  \href{https://www.nytimes.com/subscription?campaignId=37WXW}{Subscriptions}
\end{itemize}
