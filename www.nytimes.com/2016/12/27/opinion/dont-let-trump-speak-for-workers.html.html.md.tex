Sections

SEARCH

\protect\hyperlink{site-content}{Skip to
content}\protect\hyperlink{site-index}{Skip to site index}

\href{https://myaccount.nytimes.com/auth/login?response_type=cookie\&client_id=vi}{}

\href{https://www.nytimes.com/section/todayspaper}{Today's Paper}

\href{/section/opinion}{Opinion}\textbar{}Don't Let Trump Speak for
Workers

\url{https://nyti.ms/2iBo3D1}

\begin{itemize}
\item
\item
\item
\item
\item
\end{itemize}

Advertisement

\protect\hyperlink{after-top}{Continue reading the main story}

Supported by

\protect\hyperlink{after-sponsor}{Continue reading the main story}

\href{/section/opinion}{Opinion}

Op-Ed Contributor

\hypertarget{dont-let-trump-speak-for-workers}{%
\section{Don't Let Trump Speak for
Workers}\label{dont-let-trump-speak-for-workers}}

By Richard Trumka

\begin{itemize}
\item
  Dec. 27, 2016
\item
  \begin{itemize}
  \item
  \item
  \item
  \item
  \item
  \end{itemize}
\end{itemize}

\includegraphics{https://static01.nyt.com/images/2016/12/27/opinion/26trumkaWeb/27trumkaWeb-articleLarge.jpg?quality=75\&auto=webp\&disable=upscale}

``I am your voice,'' President-elect Donald J. Trump declared at the
Republican National Convention. In another campaign speech, he told his
supporters, ``I alone can fix it.''

Before he has even taken office, Mr. Trump has tried that go-it-alone
strategy on behalf of American workers. He has browbeaten Carrier into
reversing a decision to move some jobs from Indiana to Mexico, and he
attempted --- unsuccessfully --- to do the same with Rexnord, which owns
a neighboring manufacturing plant. But publicity stunts and Twitter
rants are no substitute for a comprehensive, coherent economic strategy
that invests in America and lifts up the voices and the power of working
people.

Working people do not want a savior to speak for us. We want to raise
our own voices through our unions --- and those voices are more
essential than ever. The share of income going to the middle class has
fallen in almost perfect correlation with the declining percentage of
people working in jobs where they enjoy a union. Collective, democratic
representation in the workplace is essential to shared and durable
economic prosperity.

Yet Mr. Trump's emerging cabinet and policy pronouncements seem to treat
actual working people as bottom lines rather than human beings, our
unions as a threat rather than a partner, and rising wages as a problem
rather than the foundation of our prosperity.

If Mr. Trump's strategy to keep jobs in America relies on busting
unions, keeping wages down, deregulating everything in sight and cutting
taxes for the wealthy, he'll certainly fail, and in the process he'll
undermine the foundations of American democracy.

The Carrier jobs are good because those workers have a voice through
their union and have bargained to share in the profit derived from
producing relatively high-value consumer goods for American markets.
When manufacturing workers have a union to make their voices heard, the
jobs pay better and typically provide decent health and retirement
benefits. These careers breathe life into local economies: Suppliers,
service providers and public sector jobs around Carrier and Rexnord
would not exist but for those plants.

But union jobs are about more than economics. Through their union, the
United Steelworkers, Carrier and Rexnord workers have a say in what
happens at their workplace and in the nation. They are connected to one
another and to the larger community where they live. Jobs where working
people have a say are critical to the fabric of our democracy.

The United States can't win a global race to the bottom, and we
shouldn't be trying to. Instead, the path to success for a wealthy,
industrialized country like the United States lies in strategic
investments in infrastructure and skills, and trade and tax policies
that nurture and reward domestic production --- not outsourcing.

When democratic capitalism is managed in ways that fail to provide good
jobs, working people will turn in desperation toward authoritarian
solutions. This is the great lesson of the 20th century, and we face the
threat once again today. In industrialized countries all over the world,
working people have come to believe that the institutions of liberal
democracy have failed to protect them against the ravages of
globalization. The leaders who exploit those very real anxieties are
interested in power, not helping working people.

To avoid that same turn, the United States needs a coordinated suite of
public policies that preserve, nurture and create good jobs. And we need
a president who sees workers as partners to be engaged, not as subjects
to be manipulated. The workers at Carrier and Rexnord have elected
representatives and are part of a union with real experience partnering
with employers to save jobs. But Mr. Trump refused to engage with the
steelworkers' union; instead, he insulted the local president, who
started working at the Rexnord plant at age 17.

If President-elect Trump is serious about building a high-productivity,
high-wage economy, he needs to put a moratorium on flawed trade
agreements and crack down on unfair trade practices, and he must work to
end all tax subsidies for offshoring and put those and other revenues
toward funding quality education, skills and infrastructure investment.
The American labor movement has long advocated for these policies.

The next time Mr. Trump wants to save good jobs, he should listen to the
people whose jobs he is trying to save. Without unions to amplify those
voices, we cannot create good jobs on the scale that is needed --- or
preserve our democracy.

Advertisement

\protect\hyperlink{after-bottom}{Continue reading the main story}

\hypertarget{site-index}{%
\subsection{Site Index}\label{site-index}}

\hypertarget{site-information-navigation}{%
\subsection{Site Information
Navigation}\label{site-information-navigation}}

\begin{itemize}
\tightlist
\item
  \href{https://help.nytimes.com/hc/en-us/articles/115014792127-Copyright-notice}{©~2020~The
  New York Times Company}
\end{itemize}

\begin{itemize}
\tightlist
\item
  \href{https://www.nytco.com/}{NYTCo}
\item
  \href{https://help.nytimes.com/hc/en-us/articles/115015385887-Contact-Us}{Contact
  Us}
\item
  \href{https://www.nytco.com/careers/}{Work with us}
\item
  \href{https://nytmediakit.com/}{Advertise}
\item
  \href{http://www.tbrandstudio.com/}{T Brand Studio}
\item
  \href{https://www.nytimes.com/privacy/cookie-policy\#how-do-i-manage-trackers}{Your
  Ad Choices}
\item
  \href{https://www.nytimes.com/privacy}{Privacy}
\item
  \href{https://help.nytimes.com/hc/en-us/articles/115014893428-Terms-of-service}{Terms
  of Service}
\item
  \href{https://help.nytimes.com/hc/en-us/articles/115014893968-Terms-of-sale}{Terms
  of Sale}
\item
  \href{https://spiderbites.nytimes.com}{Site Map}
\item
  \href{https://help.nytimes.com/hc/en-us}{Help}
\item
  \href{https://www.nytimes.com/subscription?campaignId=37WXW}{Subscriptions}
\end{itemize}
