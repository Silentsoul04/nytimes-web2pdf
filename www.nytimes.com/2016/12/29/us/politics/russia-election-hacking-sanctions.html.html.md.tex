Sections

SEARCH

\protect\hyperlink{site-content}{Skip to
content}\protect\hyperlink{site-index}{Skip to site index}

\href{https://www.nytimes.com/section/politics}{Politics}

\href{https://myaccount.nytimes.com/auth/login?response_type=cookie\&client_id=vi}{}

\href{https://www.nytimes.com/section/todayspaper}{Today's Paper}

\href{/section/politics}{Politics}\textbar{}Obama Strikes Back at Russia
for Election Hacking

\url{https://nyti.ms/2hwS8pA}

\begin{itemize}
\item
\item
\item
\item
\item
\item
\end{itemize}

Advertisement

\protect\hyperlink{after-top}{Continue reading the main story}

Supported by

\protect\hyperlink{after-sponsor}{Continue reading the main story}

\hypertarget{obama-strikes-back-at-russia-for-election-hacking}{%
\section{Obama Strikes Back at Russia for Election
Hacking}\label{obama-strikes-back-at-russia-for-election-hacking}}

\includegraphics{https://static01.nyt.com/images/2016/12/30/international-home/30HACKING-5/30HACKING-5-articleInline.jpg?quality=75\&auto=webp\&disable=upscale}

By \href{http://www.nytimes.com/by/david-e-sanger}{David E. Sanger}

\begin{itemize}
\item
  Dec. 29, 2016
\item
  \begin{itemize}
  \item
  \item
  \item
  \item
  \item
  \item
  \end{itemize}
\end{itemize}

WASHINGTON --- President Obama struck back at Russia on Thursday for its
efforts to influence the 2016 election, ejecting 35 suspected Russian
intelligence operatives from the United States and imposing sanctions on
Russia's two leading intelligence services.

The administration also penalized four top officers of one of those
services, the powerful military intelligence unit known as the G.R.U.

Image

President Vladimir V. Putin of Russia at the Kremlin on
Tuesday.Credit...Pool photo by Alexei Druzhinin

Intelligence agencies have concluded that the G.R.U. ordered the attacks
on the Democratic National Committee and other political organizations,
with the approval of the Kremlin, and ultimately enabled the publication
of the emails it harvested to benefit Donald J. Trump's campaign.

The expulsion of the 35 Russians, who the administration said were spies
posing as diplomats and other officials, and their families was in
response to the harassment of American diplomats in Russia, State
Department officials said. It was unclear if they were involved in the
hacking.

\includegraphics{https://static01.nyt.com/images/2016/12/30/world/30HACKING-1/30HACKING-1-articleInline.jpg?quality=75\&auto=webp\&disable=upscale}

In addition, the State Department announced
\href{https://www.nytimes.com/2016/12/29/us/politics/russia-spy-compounds-maryland-long-island.html}{the
closing of two waterfront estates} --- one in Upper Brookville, N.Y.,
and another on Maryland's Eastern Shore --- that it said were used for
Russian intelligence activities, although officials declined to say
whether they were specifically used in the election-related hacks.

Taken together, the sweeping actions announced by the White House, the
Treasury, the State Department and intelligence agencies on Thursday
amount to the strongest American response yet to a state-sponsored
cyberattack. They also appeared intended to box in President-elect
Trump, who will now have to decide whether to lift the sanctions on
Russian intelligence agencies when he takes office next month.

\href{https://www.nytimes.com/interactive/2016/12/29/us/politics/russian-hack-in-200-words.html}{}

\includegraphics{https://static01.nyt.com/images/2016/12/29/us/politics/russian-hack-in-200-words-1483060431834/russian-hack-in-200-words-1483060431834-square640-v2.png}

\hypertarget{the-russian-hacking-in-200-words}{%
\subsection{The Russian Hacking in 200
Words}\label{the-russian-hacking-in-200-words}}

President Obama announced sanctions against Russia for trying to
influence the 2016 election through cyberattacks. Here's what led to the
sanctions.

Mr. Trump responded to the Russian sanctions late Thursday by
reiterating a call to ``move on.'' But he pledged to meet with
intelligence officials, who have concluded that the Russian hacking was
an attempt to tip the election to Mr. Trump.

In an earlier statement from Hawaii, Mr. Obama took a subtle dig at Mr.
Trump, who has consistently cast doubt on the intelligence showing that
the Russian government was deeply involved in the hacking. ``All
Americans should be alarmed by Russia's actions,'' Mr. Obama said, and
added that the United States acted after ``repeated private and public
warnings that we have issued to the Russian government, and are a
necessary and appropriate response to efforts to harm U.S. interests in
violation of established international norms of behavior.''

Image

President Vladimir V. Putin of Russia in 2006. A spokesman for him
expressed ``regret'' about President Obama's decision.Credit...Dmitri
Astakhov/Agence France-Presse --- Getty Images

He issued a new executive order that allows him, and his successors, to
retaliate for efforts to influence elections in the United States or
those of ``allies and partners,'' a clear reference to concern that
Russia's next target may be Germany and France. Already there are
reports of influence operations in both.

Mr. Trump's position is at odds with most members of his party, who
after classified briefings have called for investigations into the
combination of cyberattacks and old-style information warfare used in
the 2016 campaign. Mr. Trump has largely stuck to the theory he set
forth in a debate with Hillary Clinton in September, when he said the
hacks could have been organized by ``somebody sitting on their bed that
weighs 400 pounds.''

\href{https://www.nytimes.com/interactive/2016/12/29/us/politics/document-Report-on-Russian-Hacking.html}{}

\includegraphics{https://static01.nyt.com/images/2016/12/29/us/hacking-report-promo-image/hacking-report-promo-image-largeHorizontalJumbo.png}

\hypertarget{report-on-russian-hacking}{%
\subsection{Report on Russian Hacking}\label{report-on-russian-hacking}}

The F.B.I. and Department of Homeland Security released a report on
Thursday detailing the ways that Russia acted to influence the American
election through cyberespionage.

Russia criticized the sanctions and vowed retaliation.

``Such steps of the U.S. administration that has three weeks left to
work are aimed at two things: to further harm Russian-American ties,
which are at a low point as it is, as well as, obviously, deal a blow on
the foreign policy plans of the incoming administration of the
president-elect,'' Dmitri S. Peskov, the spokesman for President
Vladimir V. Putin, told reporters.

Konstantin Kosachyov, the head of the foreign affairs committee in the
upper house of the Russian Parliament, told Interfax that ``this is the
agony not even of `lame ducks,' but of `political corpses.'''

Image

Igor Valentinovich KorobovCredit...Ministry of Defence of the Russian
Federation

Despite the international fallout and political repercussions
surrounding the announcement, it is not clear how much effect the
sanctions will have, except on the ousted diplomats, who have until
midday Sunday to leave the country. G.R.U. officials rarely travel to
the United States, or keep assets here.

The four Russian intelligence officials are Igor Valentinovich Korobov,
the chief of the G.R.U., and three deputies: Sergey Aleksandrovich
Gizunov, Igor Olegovich Kostyukov and Vladimir Stepanovich Alexseyev.

Image

President-elect Donald J. Trump responded to the Russian sanctions late
Thursday by reiterating a call to ``move on.''Credit...Kevin Hagen for
The New York Times

The administration also put sanctions on three companies and
organizations that it said supported the hacking operations: the Special
Technology Center, a signals intelligence operation in St. Petersburg,
Russia; a firm called Zorsecurity that is also known as Esage Lab; and
the Autonomous Noncommercial Organization Professional Association of
Designers of Data Processing Systems, whose lengthy name, American
officials said, was cover for a group that provided special training for
the hacking.

Still, the sanctions go well beyond the modest sanctions
\href{http://www.nytimes.com/2015/01/03/us/in-response-to-sony-attack-us-levies-sanctions-on-10-north-koreans.html}{imposed
against North Korea} for its attack on Sony Pictures Entertainment two
years ago, which Mr. Obama said at the time was an effort to repress
free speech --- a somewhat crude comedy, called ``The Interview,''
imagining a C.I.A. plot to assassinate Kim Jung-un, the country's
leader.

The sanctions are not as biting as previous ones in which the United
States and its Western allies took aim at broad sectors of the Russian
economy and blacklisted dozens of people, some of them close friends of
Mr. Putin's. Those sanctions were in response to the Russian annexation
of Crimea and its activities to destabilize Ukraine. Mr. Trump suggested
in an interview with The New York Times this year that he believed those
sanctions were useless, and left open the possibility he might lift
them.

The F.B.I. and the Department of Homeland Security on Thursday also
released samples of malware and other indicators of Russian
cyberactivity, including network addresses of computers commonly used by
the Russians to start attacks. But the evidence in a report, in which
the administration referred to the Russian cyberactivity as Grizzly
Steppe, fell short of anything that would directly tie senior officers
of the G.R.U. or the F.S.B., the other intelligence service, to a plan
to influence the election.

\href{https://www.nytimes.com/interactive/2016/07/27/us/politics/trail-of-dnc-emails-russia-hacking.html}{}

\includegraphics{https://static01.nyt.com/images/2016/07/27/us/politics/trail-of-dnc-emails-russia-hacking-1469656463301/trail-of-dnc-emails-russia-hacking-1469656463301-thumbLarge-v6.png}

\hypertarget{following-the-links-from-russian-hackers-to-the-us-election}{%
\subsection{Following the Links From Russian Hackers to the U.S.
Election}\label{following-the-links-from-russian-hackers-to-the-us-election}}

How U.S. intelligence officials have connected the Russian government to
an attempt to disrupt the 2016 presidential election.

A more detailed report on the intelligence, ordered by Mr. Obama, will
be published in the next three weeks, though much of the information ---
especially evidence collected from ``implants'' in Russian computer
systems, tapped conversations and spies --- is expected to remain
classified.

Several Obama administration officials, including Vice President Joseph
R. Biden Jr., have suggested that there may also be a covert response,
one that would be obvious to Mr. Putin but not to the public.

While that may prove satisfying, many outside experts have said that
unless the public response is strong enough to impose a real cost on Mr.
Putin, his government and his vast intelligence apparatus, it might not
deter further activity.

``They are concerned about controlling retaliation,'' said James A.
Lewis, a cyberexpert at the Center for Strategic and International
Studies in Washington.

But John P. Carlin, who recently left the administration as the chief of
the Justice Department's national security division, where he assembled
cases against North Korean, Chinese and Iranian hackers, called the
administration's actions a ``significant step that is consistent with a
new model: When you violate norms of behavior in this space, we can
figure out who did it and we can impose consequences.''

The Obama administration was riven for months by an internal debate
about how much of its evidence to make public. In interviews
\href{https://www.nytimes.com/2016/12/13/us/politics/russia-hack-election-dnc.html?hp\&action=click\&pgtype=Homepage\&clickSource=story-heading\&module=b-lede-package-region\&region=top-news\&WT.nav=top-news\&_r=0,}{for
a New York Times investigation into the hack}, several of Mr. Obama's
top aides expressed regret that they had not made evidence public
earlier, or reacted more strongly. None said they believed it would have
affected the outcome of the election, however.

In recent weeks, Mr. Obama decided that the authorities he created in
April 2015 to retaliate against states or individuals that conduct
hacking after the Sony attack did not go far enough. They made no
provision issuing sanctions in response to an incursion on the electoral
system --- an attack few saw coming.

So he ordered his lawyers to amend the executive order, specifically
giving himself and his successor the authority to issue travel bans and
asset freezes on those who ``tamper with, alter, or cause a
misappropriation of information, with a purpose or effect of interfering
with or undermining election processes or institutions.''

The administration has not publicly criticized how its own officials
handled the case. But the Times investigation revealed that the F.B.I.
first informed the Democratic National Committee that it saw evidence
that the committee's email systems had been hacked in the fall of 2015.
Months of fumbling and slow responses followed.

Mr. Obama said at a news conference that he was first notified early
this summer. But one of his top aides met Russian officials in Geneva to
complain about activity in April.

By the time the leadership of the committee woke up to what was
happening, the G.R.U. had not only obtained emails through a hacking
group that has been closely associated with it for years, but,
investigators say, also allowed them to be published on a number of
websites, including a newly created one called DC Leaks and the far more
established WikiLeaks. Meanwhile, several states reported the
``scanning'' of their voter databases --- which American intelligence
agencies also attributed to Russian hackers. But there is no evidence,
American officials said, that Russia sought to manipulate votes or voter
rolls on Nov. 8.

Mr. Obama decided not to issue sanctions earlier for fear of Russian
retaliation ahead of Election Day. Some of his aides now believe that
was a mistake. But the president made clear before leaving for Hawaii
that he planned to respond.

Advertisement

\protect\hyperlink{after-bottom}{Continue reading the main story}

\hypertarget{site-index}{%
\subsection{Site Index}\label{site-index}}

\hypertarget{site-information-navigation}{%
\subsection{Site Information
Navigation}\label{site-information-navigation}}

\begin{itemize}
\tightlist
\item
  \href{https://help.nytimes.com/hc/en-us/articles/115014792127-Copyright-notice}{©~2020~The
  New York Times Company}
\end{itemize}

\begin{itemize}
\tightlist
\item
  \href{https://www.nytco.com/}{NYTCo}
\item
  \href{https://help.nytimes.com/hc/en-us/articles/115015385887-Contact-Us}{Contact
  Us}
\item
  \href{https://www.nytco.com/careers/}{Work with us}
\item
  \href{https://nytmediakit.com/}{Advertise}
\item
  \href{http://www.tbrandstudio.com/}{T Brand Studio}
\item
  \href{https://www.nytimes.com/privacy/cookie-policy\#how-do-i-manage-trackers}{Your
  Ad Choices}
\item
  \href{https://www.nytimes.com/privacy}{Privacy}
\item
  \href{https://help.nytimes.com/hc/en-us/articles/115014893428-Terms-of-service}{Terms
  of Service}
\item
  \href{https://help.nytimes.com/hc/en-us/articles/115014893968-Terms-of-sale}{Terms
  of Sale}
\item
  \href{https://spiderbites.nytimes.com}{Site Map}
\item
  \href{https://help.nytimes.com/hc/en-us}{Help}
\item
  \href{https://www.nytimes.com/subscription?campaignId=37WXW}{Subscriptions}
\end{itemize}
