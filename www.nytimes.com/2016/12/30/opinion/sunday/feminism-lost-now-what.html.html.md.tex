\href{/section/opinion/sunday}{Sunday Review}\textbar{}Feminism Lost.
Now What?

\url{https://nyti.ms/2iLe2TR}

\begin{itemize}
\item
\item
\item
\item
\item
\item
\end{itemize}

\includegraphics{https://static01.nyt.com/images/2017/01/01/opinion/sunday/01chiraWEB/01chiraWEB-articleLarge-v2.jpg?quality=75\&auto=webp\&disable=upscale}

Sections

\protect\hyperlink{site-content}{Skip to
content}\protect\hyperlink{site-index}{Skip to site index}

\href{/section/opinion}{Opinion}

\hypertarget{feminism-lost-now-what}{%
\section{Feminism Lost. Now What?}\label{feminism-lost-now-what}}

``It's amazing to me the lightning speed at which these issues have
receded. The story is the total omission of women.''

Credit...Hanna Barczyk

Supported by

\protect\hyperlink{after-sponsor}{Continue reading the main story}

By Susan Chira

\begin{itemize}
\item
  Dec. 30, 2016
\item
  \begin{itemize}
  \item
  \item
  \item
  \item
  \item
  \item
  \end{itemize}
\end{itemize}

This was supposed to be the year of triumph for American women.

A year that would cap an arc of progress: Seneca Falls, 1848. The 19th
Amendment, 1920. The first female American president, 2017. An
inauguration that would usher in a triumvirate of women running major
Western democracies. Little girls getting to see a woman in the White
House.

Instead, for those at the forefront of the women's movement, there is
despair, division and defiance. Hillary Clinton's loss was feminism's,
too.

A man whose behavior toward women is a throwback to pre-feminist days is
now setting the tone for the country. The cabinet that Donald J. Trump
has nominated includes men --- and a few women --- with public records
hostile to a range of issues at the heart of the women's movement. A
majority of white women voted for him, shattering myths of female
solidarity and the belief that demeaning women would make a politician
unelectable.

More broadly, there is a fear that women's issues as the movement has
defined them --- reproductive rights, women's health, workplace
advancement and the fight against sexual harassment, among others ---
could be trampled or ignored.

The
\href{https://www.theguardian.com/us-news/2016/dec/27/womens-march-on-washington-dc-guide}{Women's
March on Washington} on Jan. 21 is an apt metaphor for the moment:
movement as primal scream. It grew out of a post on Facebook, was
unconnected to any established women's organization, and has no set list
of demands. Hundreds of thousands of women say they are going, but will
their anger turn into a broader movement?

``We need a `come to Jesus' moment,'' said C. Nicole Mason of the Center
for Research and Policy in the Public Interest at the New York Women's
Foundation. ``I feel like the denial is very severe.''

In the weeks after the election, in conversations with nearly two dozen
advocates for women, I heard the fractures of a movement still
regrouping after an unexpected defeat. They know that Mrs. Clinton
didn't stand for the feminist movement directly, and that you could vote
against her without saying you were voting against feminism. But one of
the movement's goals was shattering that ultimate glass ceiling. Some
say the failure to do so was so devastating that now is the time to
rebuild from the ground up. Others insist it's time to stay the course.

The challenges are a proxy for the questions the Democratic Party must
face over class, race, identity politics and tactics. The women's
movement must balance how to broaden its message without losing its
base. Courting the white working class could alienate black women still
smarting over white women voting for a man whom many saw as racist --- a
choice that seemed to put racial identity over gender solidarity. Some
younger women shun the feminist label altogether. It's not clear how far
the tent can stretch without leaving some outside.

The overall struggle is to stay relevant in the age of Trump. ``Before
the election, even I was stunned by the sheer number of people I knew
who came forward saying they'd been survivors of sexual assault,'' said
Vivien Labaton, co-executive director of Make It Work, which promotes
working families' economic security. ``It's amazing to me the lightning
speed at which these issues have receded. The story is the total
omission of women. Overnight.''

Many veterans of the women's movement bristle at the thought that the
election was a rejection of feminism. Hillary Clinton won the popular
vote by the largest recorded margin for a defeated candidate and won the
majority of all women's votes. Eleanor Smeal, president of the Feminist
Majority Foundation,
\href{http://msmagazine.com/blog/2016/12/22/numbers-gender-feminism-2016-presidential-election/}{cites
a poll} commissioned from Lake Research Partners conducted on the eve of
the election. It found that 59 percent of women voters over all, and 59
percent of younger women, identify as feminists, up from 2012.

Heather Booth, long active in the movement, notes that polls
consistently show majority support for child care, equal pay,
prohibitions against sexual discrimination and the right to abortion.
Ms. Smeal's and many other groups have reported intensified
mobilization, donations and volunteerism after the election. For many,
the defeat may well be an awakening, a visible sign of barriers they
thought had been swept away.

But this consensus masks real struggles.

Although exit polls suggest that a majority of young women voted for
Mrs. Clinton, their enthusiasm for Bernie Sanders during the primary
seemed to say that for some, feminism's traditional preoccupations seem
out of date.

In late October, when the polls indicated that Mrs. Clinton would win, I
sought out young women to talk about their perceptions of her. Jessica
Salans, 27, who is running for local office in Los Angeles in 2017, said
she found Mrs. Clinton's feminism outdated, failing to prioritize
climate change, income inequality and the toll of American intervention
overseas.

\includegraphics{https://static01.nyt.com/images/2017/01/01/opinion/sunday/01chiraJUMP/01chiraJUMP-articleLarge.jpg?quality=75\&auto=webp\&disable=upscale}

``I saw a great documentary about the second-wave feminist movement, and
it made me realize why people like Gloria Steinem were coming out in
support of Hillary Clinton,'' she said. The brand of feminism that spoke
to her, though, wasn't about breaking historic barriers. It was more
specific: ``progressive feminism, eco-feminism.''

To many inside and outside the feminist movement, the Clinton campaign
message missed the mark.

``White working-class women saw Hillary Clinton as another privileged
white woman wanting to break the glass ceiling,'' said Joan C. Williams,
professor at University of California Hastings College of the Law.
``That metaphor makes sense if your central goal is to gain access to
jobs that privileged men have. Hillary's feminism was not about them.''

Feminism, which at its heart should mean opportunities for women in
every sphere, has also come to be seen as a proxy for liberalism,
alienating conservatives.

S. E. Cupp, a columnist for The Daily News in New York and a
conservative who did not vote for Mr. Trump, said: ``There's a
condescension that comes across from some in the women's movement.
There's this idea that if you're not liberal, you're a traitor to your
gender. Is our message alienating entire groups of people, including
women?''

She raised the provocative possibility that many women believed that Mr.
Trump would keep the country safe in part because of his paternalistic,
alpha male persona --- and that was an implicit rejection of feminism's
attempts to redefine gender roles.

Others worry that the women's movement has spent too much time policing
language and behavior, blaming and shaming at the expense of dialogue.
That, Professor Williams argues, can make misogyny attractive to the
white working class, a way to rebel against condescending elites.

The answer, some argue, is rebranding feminism --- recasting issues in
economic terms relevant to the working class, men as well as women.

``While the working families agenda is very strong, it's not big enough
to get the country back on its feet,'' said Celinda Lake, a longtime
Democratic pollster. ``It needs to be embedded in a bigger economic
message. Sometimes we talk about it in ways that make it sound like it's
just for women, to the exclusion of men.''

By contrast, she said, Mr. Trump's economic platform was clear and
compelling. Mrs. Clinton's calls for equal pay, child care, paid family
leave and health insurance that covered birth control and mammograms,
paled before the appeal of someone who promised to bring back
better-paying manufacturing jobs and restore a lost standard of living.

The key is to link the two messages, to take issues that benefit women
and show how they help families as a whole.

Ms. Lake described a focus group on equal pay she conducted with white
union members in Michigan. She found that men were enthusiastic if they
connected it with their own economic security. ``One guy said, `If the
little lady doesn't get paid the same as I do, I need to get overtime
and there's no OT anymore.' '' The other men in the room, she said,
agreed with him.

Men may also be more receptive when the message is applied to their
daughters. Reshma Saujani runs the group Girls Who Code, aimed at
preparing girls for careers in technology. She noted that when her
organization tried to persuade parents to enroll their girls, abstract
appeals to gender equity fell flat. Evoking fathers' dreams for their
daughters had more resonance.

``You've lost your home but your daughter has a shot at going back up to
the middle class,'' she said, explaining why that kind of pitch
succeeded. ``We have to talk to different parts of the country
differently. We can't make the same gender arguments --- it doesn't
work.''

In these postelection conversations, the rawest wounds were expressed by
black women who felt betrayed by white women's support for Mr. Trump.
These women worry that the national chest-beating about identity
politics and the resolve to win back the white working class will come
at their expense, subordinating issues of racial justice.

``You blame the people who voted for him, not the ones who didn't,''
said Salamishah Tillet, an associate professor of English and Africana
studies at the University of Pennsylvania.

Early organizers of the women's march faced scorn for initially failing
to include minority women in leadership positions, then drew fire for
the original name of the event --- Million Woman March --- which
appropriated the name of a march by black women in 1997.

``Ashes to ashes, dust to white liberal feminism,'' wrote LeRhonda
Manigault-Bryant, associate professor of Africana studies at Williams
College, in an impassioned open letter noting that white feminists now
shared the kind of fears long known to black women.

Rather than playing down race, these women argue it's essential to
recognize its interconnection with feminism. Allowing racism to fester,
they say, threatens not only black women but also white women, because
it encourages white nationalism, which is also hostile to women's
rights.

But building bridges across racial and ethnic lines requires white
feminists to understand that their experience is not universal,
Professor Manigault-Bryant said. And it means defining women's issues as
broadly as possible.

One of the paradoxes of 2016 was that some referendums on issues dear to
the women's movement passed on the local level, from tax increases to
expand child care programs in Ohio to raising the minimum wage in four
states. Advocates see opportunities in localities, a key battleground.
The Michigan People's Campaign focused on a statehouse district called
Downriver Detroit, dispatching campaigners door to door to talk across
party lines about issues like caring for the elderly, disabled relatives
and children. Their progressive candidate won the local election,
although Mr. Trump carried the same district.

Others see ballot initiatives as a potent weapon. ``Maybe this moment
was tailor-made for ballot measures as a critical form of policy making
and protest,'' said Justine Sarver, executive director of the Ballot
Initiative Strategy Center. The center conducted polling in 11 states
and found wide support for issues like equal pay, child care, paid
family leave and higher wages, and is gearing up for the 2018 midterm
elections.

In the end, it's hard to argue that this election over all was a vote
for the subordination of women. But it's a warning that feminism, as it
has been defined, did not inspire enough people in enough places around
the country. You didn't hear explicit calls for women to stay at home or
be subservient to men, although it's an open question how many Americans
are receptive to questioning traditional gender roles. Many who care
about the place of women in American society are gripped by fears that
men will now feel they have a free pass to demean women at home or in
the workplace, that women's health, economic security and reproductive
rights will be dealt severe blows.

Talking to women who voted for Mr. Trump, I found many who were working,
divorced or single, opinionated and outspoken. They saw Hillary Clinton
as a menace and Donald Trump as an agent of change, if a flawed one.
Many were living what might be called liberated lives.

The challenge for the women's movement is to persuade more of the
electorate that feminism is not merely a luxury for the privileged or
the province only of liberals, but rather that it is essential to the
freedom of every woman --- and to her choices.

Advertisement

\protect\hyperlink{after-bottom}{Continue reading the main story}

\hypertarget{site-index}{%
\subsection{Site Index}\label{site-index}}

\hypertarget{site-information-navigation}{%
\subsection{Site Information
Navigation}\label{site-information-navigation}}

\begin{itemize}
\tightlist
\item
  \href{https://help.nytimes.com/hc/en-us/articles/115014792127-Copyright-notice}{©~2020~The
  New York Times Company}
\end{itemize}

\begin{itemize}
\tightlist
\item
  \href{https://www.nytco.com/}{NYTCo}
\item
  \href{https://help.nytimes.com/hc/en-us/articles/115015385887-Contact-Us}{Contact
  Us}
\item
  \href{https://www.nytco.com/careers/}{Work with us}
\item
  \href{https://nytmediakit.com/}{Advertise}
\item
  \href{http://www.tbrandstudio.com/}{T Brand Studio}
\item
  \href{https://www.nytimes.com/privacy/cookie-policy\#how-do-i-manage-trackers}{Your
  Ad Choices}
\item
  \href{https://www.nytimes.com/privacy}{Privacy}
\item
  \href{https://help.nytimes.com/hc/en-us/articles/115014893428-Terms-of-service}{Terms
  of Service}
\item
  \href{https://help.nytimes.com/hc/en-us/articles/115014893968-Terms-of-sale}{Terms
  of Sale}
\item
  \href{https://spiderbites.nytimes.com}{Site Map}
\item
  \href{https://help.nytimes.com/hc/en-us}{Help}
\item
  \href{https://www.nytimes.com/subscription?campaignId=37WXW}{Subscriptions}
\end{itemize}
