Sections

SEARCH

\protect\hyperlink{site-content}{Skip to
content}\protect\hyperlink{site-index}{Skip to site index}

\href{https://www.nytimes.com/section/world/asia}{Asia Pacific}

\href{https://myaccount.nytimes.com/auth/login?response_type=cookie\&client_id=vi}{}

\href{https://www.nytimes.com/section/todayspaper}{Today's Paper}

\href{/section/world/asia}{Asia Pacific}\textbar{}Korean Official
Charged With Illegally Swaying Samsung Merger Vote

\url{https://nyti.ms/2hCfElm}

\begin{itemize}
\item
\item
\item
\item
\item
\end{itemize}

Advertisement

\protect\hyperlink{after-top}{Continue reading the main story}

Supported by

\protect\hyperlink{after-sponsor}{Continue reading the main story}

\hypertarget{korean-official-charged-with-illegally-swaying-samsung-merger-vote}{%
\section{Korean Official Charged With Illegally Swaying Samsung Merger
Vote}\label{korean-official-charged-with-illegally-swaying-samsung-merger-vote}}

By \href{http://www.nytimes.com/by/choe-sang-hun}{Choe Sang-Hun}

\begin{itemize}
\item
  Dec. 31, 2016
\item
  \begin{itemize}
  \item
  \item
  \item
  \item
  \item
  \end{itemize}
\end{itemize}

\includegraphics{https://static01.nyt.com/images/2017/01/01/world/asia/01korea-photo1/01korea-photo1-articleInline.jpg?quality=75\&auto=webp\&disable=upscale}

SEOUL, South Korea --- A former cabinet minister was arrested Saturday
on charges that he illegally pressured the national pension fund to
approve a merger between two Samsung subsidiaries, a deal that helped
ensure that control of South Korea's most powerful conglomerate passed
from its chairman to his son.

Prosecutors have accused the former official, Moon Hyung-pyo, of
ordering South Korea's National Pension Service to cast a crucial vote
in favor of
\href{http://www.nytimes.com/2015/07/18/business/dealbook/samsung-ct-shareholders-back-cheil-industries-merger-defying-activist-hedge-fund-elliott-associates.html}{the
merger between Samsung C\&T and Cheil Industries} in July of 2015, when
Mr. Moon was the health and welfare minister. Mr. Moon is now chairman
of the pension fund, the world's third largest, which is overseen by the
Health Ministry.

Mr. Moon denied the accusation at a parliamentary hearing on Nov. 30.
But on Saturday, a court granted a special prosecutor a warrant to
arrest him on charges of abusing power and interfering with what was
supposed to be the pension fund's independent investment decision.

A conviction of Mr. Moon on the charges would raise serious questions
about the legitimacy of the \$8 billion merger, which helped Samsung's
vice chairman, Jay Y. Lee, tighten his control over the business that
his father and grandfather built into South Korea's biggest, most
lucrative chaebol, or family-controlled conglomerate. Mr. Lee's father,
Lee Kun-hee, is Samsung's chairman but has been incapacitated by health
problems.

Mr. Moon was arrested by the special prosecutor investigating President
Park Geun-hye, whose powers have been suspended since the
\href{https://www.nytimes.com/2016/12/09/world/asia/south-korea-president-park-geun-hye-impeached.html}{National
Assembly voted to impeach her} on Dec. 9. The special prosecutor is
seeking to determine whether Mr. Moon acted on Ms. Park's behalf.

The National Assembly's impeachment bill alleged that Ms. Park solicited
bribes from Samsung and other conglomerates, which she has denied. South
Korea's Constitutional Court is to decide in the coming months whether
Ms. Park should be formally removed from office or reinstated.

In the months after the Samsung merger was approved, the conglomerate
contributed \$17 million to two foundations controlled by Choi Soon-sil,
a longtime friend and confidante of Ms. Park's who is a central figure
in the corruption scandal. Samsung also signed an \$18 million contract
with a sports management company that Ms. Choi ran in Germany, to fund a
program for training Korean equestrians that mainly benefited Ms. Choi's
daughter. Samsung also contributed \$1.3 million to a winter sports
program for young athletes that was run by Ms. Choi and her nephew.

State prosecutors, who have indicted Ms. Choi on extortion charges and
\href{http://www.nytimes.com/2016/11/20/world/asia/park-geun-hye-south-korea-extortion-accomplice-prosecutors.html}{named
Ms. Park as an accomplice}, have alleged that the two conspired to force
Samsung and other businesses to donate to Ms. Choi's foundations. But
the special prosecutor, separately, is considering a much more serious
charge of bribery, investigating whether Ms. Park, through Mr. Moon,
ordered the pension fund to back the merger in exchange for Samsung's
support for Ms. Choi.

During parliamentary hearings into the corruption scandal, top
executives at Samsung and other conglomerates admitted contributing to
Ms. Choi's foundations and companies. But they all denied seeking favors
in return; they said they felt pressured to give the money, suggesting
that they were victims of extortion, not participants in bribery.

The merger last year was widely seen as an important step in the
transfer of Samsung's leadership from Lee Kun-hee to Jay Y. Lee.

The Lee family held a controlling stake in Cheil and wanted to use it as
a de facto holding company for the entire group. The merger enabled
Cheil to absorb Samsung C\&T's shares in other Samsung subsidiaries,
including the flagship company, Samsung Electronics.

Elliott Management, an American activist hedge fund, had campaigned to
block the merger, saying that it wronged minority shareholders by
grossly undervaluing Samsung C\&T shares.

But the pension fund, Samsung C\&T's largest shareholder, supported the
deal, which was approved at a shareholders' meeting by a very thin
margin. The merger cut the value of the pension fund's 11.9 percent
stake in C\&T by \$300 million, according to some estimates.

Both Samsung and the pension fund have said that the merger would
benefit all shareholders eventually.

Advertisement

\protect\hyperlink{after-bottom}{Continue reading the main story}

\hypertarget{site-index}{%
\subsection{Site Index}\label{site-index}}

\hypertarget{site-information-navigation}{%
\subsection{Site Information
Navigation}\label{site-information-navigation}}

\begin{itemize}
\tightlist
\item
  \href{https://help.nytimes.com/hc/en-us/articles/115014792127-Copyright-notice}{©~2020~The
  New York Times Company}
\end{itemize}

\begin{itemize}
\tightlist
\item
  \href{https://www.nytco.com/}{NYTCo}
\item
  \href{https://help.nytimes.com/hc/en-us/articles/115015385887-Contact-Us}{Contact
  Us}
\item
  \href{https://www.nytco.com/careers/}{Work with us}
\item
  \href{https://nytmediakit.com/}{Advertise}
\item
  \href{http://www.tbrandstudio.com/}{T Brand Studio}
\item
  \href{https://www.nytimes.com/privacy/cookie-policy\#how-do-i-manage-trackers}{Your
  Ad Choices}
\item
  \href{https://www.nytimes.com/privacy}{Privacy}
\item
  \href{https://help.nytimes.com/hc/en-us/articles/115014893428-Terms-of-service}{Terms
  of Service}
\item
  \href{https://help.nytimes.com/hc/en-us/articles/115014893968-Terms-of-sale}{Terms
  of Sale}
\item
  \href{https://spiderbites.nytimes.com}{Site Map}
\item
  \href{https://help.nytimes.com/hc/en-us}{Help}
\item
  \href{https://www.nytimes.com/subscription?campaignId=37WXW}{Subscriptions}
\end{itemize}
