Sections

SEARCH

\protect\hyperlink{site-content}{Skip to
content}\protect\hyperlink{site-index}{Skip to site index}

\href{https://www.nytimes.com/section/politics}{Politics}

\href{https://myaccount.nytimes.com/auth/login?response_type=cookie\&client_id=vi}{}

\href{https://www.nytimes.com/section/todayspaper}{Today's Paper}

\href{/section/politics}{Politics}\textbar{}Nixon Tried to Spoil
Johnson's Vietnam Peace Talks in '68, Notes Show

\href{https://nyti.ms/2hLWhq9}{https://nyti.ms/2hLWhq9}

\begin{itemize}
\item
\item
\item
\item
\item
\end{itemize}

Advertisement

\protect\hyperlink{after-top}{Continue reading the main story}

Supported by

\protect\hyperlink{after-sponsor}{Continue reading the main story}

\hypertarget{nixon-tried-to-spoil-johnsons-vietnam-peace-talks-in-68-notes-show}{%
\section{Nixon Tried to Spoil Johnson's Vietnam Peace Talks in '68,
Notes
Show}\label{nixon-tried-to-spoil-johnsons-vietnam-peace-talks-in-68-notes-show}}

\includegraphics{https://static01.nyt.com/images/2017/01/03/us/03NIXON/03NIXON-articleInline.jpg?quality=75\&auto=webp\&disable=upscale}

By \href{http://www.nytimes.com/by/peter-baker}{Peter Baker}

\begin{itemize}
\item
  Jan. 2, 2017
\item
  \begin{itemize}
  \item
  \item
  \item
  \item
  \item
  \end{itemize}
\end{itemize}

\href{http://www.nytimes.com/topic/person/richard-milhous-nixon}{Richard
M. Nixon} told an aide that they should find a way to secretly ``monkey
wrench'' peace talks in Vietnam in the waning days of the 1968 campaign
for fear that progress toward ending the war would hurt his chances for
the presidency, according to newly discovered notes.

In a telephone conversation with
\href{http://www.nytimes.com/1993/11/13/obituaries/h-r-haldeman-nixon-aide-who-had-central-role-in-watergate-is-dead-at-67.html}{H.
R. Haldeman}, who would go on to become White House chief of staff,
Nixon gave instructions that a friendly intermediary should keep
``working on'' South Vietnamese leaders to persuade them not to agree to
a deal before the election, according to the notes, taken by Mr.
Haldeman.

The Nixon campaign's clandestine effort to thwart President
\href{http://www.nytimes.com/topic/person/lyndon-b-johnson}{Lyndon B.
Johnson}'s peace initiative that fall has long been a source of
controversy and scholarship. Ample evidence has emerged documenting the
involvement of Nixon's campaign. But
\href{http://www.nytimes.com/interactive/2016/12/31/opinion/sunday/haldeman-notes.html}{Mr.
Haldeman's notes} appear to confirm longstanding suspicions that Nixon
himself was directly involved, despite his later denials.

``There's really no doubt this was a step beyond the normal political
jockeying, to interfere in an active peace negotiation given the stakes
with all the lives,'' said John A. Farrell, who discovered the notes at
the Richard Nixon Presidential Library for his forthcoming biography,
``Richard Nixon: The Life,'' to be published in March by Doubleday.
``Potentially, this is worse than anything he did in Watergate.''

Mr. Farrell,
\href{https://www.nytimes.com/2016/12/31/opinion/sunday/nixons-vietnam-treachery.html?_r=0}{in
an article in The New York Times Sunday Review} over the weekend,
highlighted the notes by Mr. Haldeman, along with many of Nixon's
fulsome denials of any efforts to thwart the peace process before the
election.

His discovery, according to numerous historians who have written books
about Nixon and conducted extensive research of his papers, finally
provides validation of what had largely been surmise.

While overshadowed by Watergate, the Nixon campaign's intervention in
the peace talks has captivated historians for years. At times resembling
a Hollywood thriller, the story involves colorful characters, secret
liaisons, bitter rivalries and plenty of lying and spying. Whether it
changed the course of history remains open to debate, but at the very
least it encapsulated an almost-anything-goes approach that
characterized the nation's politics in that era.

As the Republican candidate in 1968, Nixon was convinced that Johnson, a
Democrat who decided not to seek re-election, was deliberately trying to
sabotage his campaign with a politically motivated peace effort meant
mainly to boost the candidacy of his vice president,
\href{http://topics.nytimes.com/top/reference/timestopics/people/h/hubert_h_jr_humphrey/index.html}{Hubert
H. Humphrey}. His suspicions were understandable, and at least
\href{http://adst.org/oral-history/fascinating-figures/philip-habib-cursed-is-the-peacemaker/}{one
of Johnson's aides later acknowledged} that they were anxious to make
progress before the election to help Mr. Humphrey.

Through much of the campaign, the Nixon team maintained a secret channel
to the South Vietnamese through Anna Chennault, widow of Claire Lee
Chennault, leader of the Flying Tigers in China during World War II.
Mrs. Chennault had become a prominent Republican fund-raiser and
Washington hostess.

Nixon met with Mrs. Chennault and the South Vietnamese ambassador
earlier in the year to make clear that she was the campaign's ``sole
representative'' to the Saigon government. But whether he knew what came
later has always been uncertain. She was the conduit for urging the
South Vietnamese to resist Johnson's entreaties to join the Paris talks
and wait for a better deal under Nixon. At one point, she told the
ambassador she had a message from ``her boss'': ``Hold on, we are gonna
win.''

Learning of this through wiretaps and surveillance, Johnson was livid.
He ordered more bugs and privately groused that Nixon's behavior
amounted to ``treason.'' But lacking hard evidence that Nixon was
directly involved, Johnson opted not to go public.

The notes Mr. Farrell found come from a phone call on Oct. 22, 1968, as
Johnson prepared to order a pause in the bombing to encourage peace
talks in Paris. Scribbling down what Nixon was telling him, Mr. Haldeman
wrote, ``Keep Anna Chennault working on SVN,'' or South Vietnam.

A little later, he wrote that Nixon wanted Senator Everett Dirksen, a
Republican from Illinois, to call the president and denounce the planned
bombing pause. ``Any other way to monkey wrench it?'' Mr. Haldeman
wrote. ``Anything RN can do.''

\href{https://www.nytimes.com/interactive/2016/12/31/opinion/sunday/haldeman-notes.html}{}

\includegraphics{https://static01.nyt.com/images/2016/12/30/opinion/sunday/haldeman-notes/haldeman-notes-square640.gif}

\hypertarget{hr-haldemans-notes-from-oct-22-1968}{%
\subsection{H.R. Haldeman's Notes from Oct. 22,
1968}\label{hr-haldemans-notes-from-oct-22-1968}}

During a phone call on the night of Oct. 22, 1968, Richard M. Nixon told
his closest aide (and future chief of staff) H.R. Haldeman to "monkey
wrench" President Lyndon B. Johnson's efforts to begin peace
negotiations over the Vietnam War.

Nixon added later that
\href{http://topics.nytimes.com/top/reference/timestopics/people/a/spiro_t_agnew/index.html}{Spiro
T. Agnew}, his vice-presidential running mate, should contact Richard
Helms, the director of the Central Intelligence Agency, and threaten not
to keep him on in a new administration if he did not provide more inside
information. ``Go see Helms,'' Mr. Haldeman wrote. ``Tell him we want
the truth --- or he hasn't got the job.''

After leaving office, Nixon denied knowing about Mrs. Chennault's
messages to the South Vietnamese late in the 1968 campaign, despite
proof that she had been in touch with John N. Mitchell, Mr. Nixon's
campaign manager and later attorney general.

Other Nixon scholars called Mr. Farrell's discovery a breakthrough.
Robert Dallek, an author of books on Nixon and Johnson, said the notes
``seem to confirm suspicions'' of Nixon's involvement in violation of
federal law. Evan Thomas, the author of ``Being Nixon,'' said Mr.
Farrell had ``nailed down what has been talked about for a long time.''

Ken Hughes, a researcher at the Miller Center of the University of
Virginia, who in 2014 published ``Chasing Shadows,'' a book about the
episode, said Mr. Farrell had found a smoking gun. ``This appears to be
the missing piece of the puzzle in the Chennault affair,'' Mr. Hughes
said. The notes ``show that Nixon committed a crime to win the
presidential election.''

Still, as tantalizing as they are, the notes do not reveal what, if
anything, Mr. Haldeman actually did with the instruction, and it is
unclear that the South Vietnamese needed to be told to resist joining
peace talks that they considered disadvantageous already.

Moreover, it cannot be said definitively whether a peace deal could have
been reached without Nixon's intervention or that it would have helped
Mr. Humphrey. William P. Bundy, a foreign affairs adviser to Johnson and
John F. Kennedy who was highly critical of Nixon, nonetheless concluded
that prospects for the peace deal were slim anyway, so ``probably no
great chance was lost.''

Luke A. Nichter, a scholar at Texas A\&M University and one of the
foremost students of the Nixon White House secret tape recordings, said
he liked more of Mr. Farrell's book than not, but disagreed with the
conclusions about Mr. Haldeman's notes. In his view, they do not prove
anything new and are too thin to draw larger conclusions.

``Because sabotaging the '68 peace efforts seems like a Nixon-like thing
to do, we are willing to accept a very low bar of evidence on this,''
Mr. Nichter said.

Tom Charles Huston, a Nixon aide who investigated the affair years ago,
found no definitive proof that the future president was involved but
concluded that it was reasonable to infer he was because of Mr.
Mitchell's role. Responding to Mr. Farrell's findings, Mr. Huston
\href{https://www.facebook.com/RichardNixonBiography/posts/705700802930178}{wrote
on Facebook} that the latest notes still do not fully answer the
question.

The notes, he wrote, ``reinforce the inference but don't push us over
the line into a necessary verdict.'' Critics, he added, ignore that
there was little chance of a peace deal, believing that ``it is
irrelevant that Saigon would have walked away without intervention by
the Nixon campaign.'' In effect, he said, ``they wish to try RN for
thought crimes.''

An open question is whether Johnson, if he had had proof of Nixon's
personal involvement, would have publicized it before the election.

Tom Johnson, the note taker in White House meetings about this episode,
said that the president considered the Nixon campaign's actions to be
treasonous but that no direct link to Nixon was established until Mr.
Farrell's discovery.

``It is my personal view that disclosure of the Nixon-sanctioned actions
by Mrs. Chennault would have been so explosive and damaging to the Nixon
1968 campaign that Hubert Humphrey would have been elected president,''
said Mr. Johnson, who went on to become the publisher of The Los Angeles
Times and later chief executive of CNN.

Mr. Farrell found the notes amid papers that were made public by the
Nixon library in July 2007 after the Nixon estate gave them back.

Timothy Naftali, a former director of the Nixon library, said the notes
``remove the fig leaf of plausible deniability'' of the former
president's involvement. The episode would set the tone for the
administration that would follow. ``This covert action by the Nixon
campaign,'' he said, ``laid the ground for the skulduggery of his
presidency.''

Advertisement

\protect\hyperlink{after-bottom}{Continue reading the main story}

\hypertarget{site-index}{%
\subsection{Site Index}\label{site-index}}

\hypertarget{site-information-navigation}{%
\subsection{Site Information
Navigation}\label{site-information-navigation}}

\begin{itemize}
\tightlist
\item
  \href{https://help.nytimes.com/hc/en-us/articles/115014792127-Copyright-notice}{©~2020~The
  New York Times Company}
\end{itemize}

\begin{itemize}
\tightlist
\item
  \href{https://www.nytco.com/}{NYTCo}
\item
  \href{https://help.nytimes.com/hc/en-us/articles/115015385887-Contact-Us}{Contact
  Us}
\item
  \href{https://www.nytco.com/careers/}{Work with us}
\item
  \href{https://nytmediakit.com/}{Advertise}
\item
  \href{http://www.tbrandstudio.com/}{T Brand Studio}
\item
  \href{https://www.nytimes.com/privacy/cookie-policy\#how-do-i-manage-trackers}{Your
  Ad Choices}
\item
  \href{https://www.nytimes.com/privacy}{Privacy}
\item
  \href{https://help.nytimes.com/hc/en-us/articles/115014893428-Terms-of-service}{Terms
  of Service}
\item
  \href{https://help.nytimes.com/hc/en-us/articles/115014893968-Terms-of-sale}{Terms
  of Sale}
\item
  \href{https://spiderbites.nytimes.com}{Site Map}
\item
  \href{https://help.nytimes.com/hc/en-us}{Help}
\item
  \href{https://www.nytimes.com/subscription?campaignId=37WXW}{Subscriptions}
\end{itemize}
