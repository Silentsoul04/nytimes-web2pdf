Sections

SEARCH

\protect\hyperlink{site-content}{Skip to
content}\protect\hyperlink{site-index}{Skip to site index}

\href{https://www.nytimes.com/section/world/asia}{Asia Pacific}

\href{https://myaccount.nytimes.com/auth/login?response_type=cookie\&client_id=vi}{}

\href{https://www.nytimes.com/section/todayspaper}{Today's Paper}

\href{/section/world/asia}{Asia Pacific}\textbar{}Park Geun-hye, South
Korean President, Is a No-Show at Impeachment Trial

\url{https://nyti.ms/2iZIny9}

\begin{itemize}
\item
\item
\item
\item
\item
\end{itemize}

Advertisement

\protect\hyperlink{after-top}{Continue reading the main story}

Supported by

\protect\hyperlink{after-sponsor}{Continue reading the main story}

\hypertarget{park-geun-hye-south-korean-president-is-a-no-show-at-impeachment-trial}{%
\section{Park Geun-hye, South Korean President, Is a No-Show at
Impeachment
Trial}\label{park-geun-hye-south-korean-president-is-a-no-show-at-impeachment-trial}}

\includegraphics{https://static01.nyt.com/images/2017/01/04/world/04KOREA-1/04KOREA-1-articleInline.jpg?quality=75\&auto=webp\&disable=upscale}

By \href{http://www.nytimes.com/by/choe-sang-hun}{Choe Sang-Hun}

\begin{itemize}
\item
  Jan. 3, 2017
\item
  \begin{itemize}
  \item
  \item
  \item
  \item
  \item
  \end{itemize}
\end{itemize}

SEOUL, South Korea --- South Korea's Constitutional Court formally
opened President Park Geun-hye's impeachment trial on Tuesday, despite
the absence of Ms. Park, whose lawyers said she was unlikely to attend
any of the proceedings.

The nine-member court has until June to decide whether Ms. Park, whose
powers have been suspended since the National Assembly
\href{http://www.nytimes.com/2016/12/09/world/asia/south-korea-president-park-geun-hye-impeached.html}{voted
on Dec. 9 to impeach her} over a corruption scandal, will be reinstated
or removed from office.

The court, which had held three
\href{http://www.nytimes.com/2016/12/22/world/asia/south-korea-president-park-impeachment.html}{preliminary
hearings} on Ms. Park's impeachment, convened in full for the first time
on Tuesday, with the intent of inviting her to respond to the National
Assembly's charges and answer questions. But she did not appear, and the
hearing was adjourned after nine minutes.

A lawyer for the president, Lee Joong-hwan, said after the hearing that
Ms. Park would make her case through her attorneys. ``She won't appear
in court unless there is an exceptionally special reason to do so,'' Mr.
Lee said.

By law, Ms. Park cannot be compelled to testify. If she declines to
appear for a second time, the court can proceed without her. Chief
Justice Park Han-chul, who is not related to the president, said that
the next hearing would be held on Thursday and that oral arguments would
begin regardless of whether Ms. Park attends. Four former or current
presidential aides were also asked to testify on Thursday.

``We recognize the weighty significance this case has in our nation's
constitutional order,'' Chief Justice Park said. ``We will do our best
to ensure an utterly fair and appropriate trial.''

Ms. Park has been accused of conspiring with a longtime friend and
confidante, Choi Soon-sil, to
\href{http://www.nytimes.com/2017/01/02/world/asia/south-korea-park-geun-hye-samsung.html}{extort
\$69 million from South Korean businesses}. In its impeachment motion,
the National Assembly characterized the money as bribes. The legislature
also accused Ms. Park of undermining freedom of the press by cracking
down on her critics and of shirking her duty to protect citizens' lives
by neglecting to respond efficiently to a
\href{http://www.nytimes.com/interactive/2015/04/12/world/asia/12ferry-timeline.html}{ferry
disaster in 2014} that killed more than 300 people.

No South Korean president has been forced out of office through
impeachment. The National Assembly voted in 2004 to
\href{http://www.nytimes.com/2004/03/13/world/president-s-impeachment-stirs-angry-protests-in-south-korea.html}{impeach
President Roh Moo-hyun}, but the Constitutional Court
\href{http://www.nytimes.com/2004/05/14/world/constitutional-court-reinstates-south-korea-s-impeached-president.html}{reinstated
him}, ruling that his violations of election law were too minor to
justify ending his presidency. Mr. Roh did not attend the court's
hearings on his impeachment.

The charges against Ms. Park are much more serious than those Mr. Roh
faced, and they have infuriated the public.
\href{http://www.nytimes.com/2016/11/26/world/asia/korea-park-geun-hye-protests.html}{Large
crowds have gathered} in central Seoul for the past 10 consecutive
Saturdays demanding an end to her presidency. Small groups of protesters
gathered Tuesday outside the Constitutional Court, some calling for Ms.
Park's ouster and others supporting her.

After Ms. Choi was arrested on extortion charges in November, Ms. Park
promised to cooperate with prosecutors investigating the scandal. But
she later refused to be questioned by them, calling them politically
biased. Ms. Park cannot be indicted while in office, but the
prosecutors' indictment of Ms. Choi
\href{http://www.nytimes.com/2016/11/20/world/asia/park-geun-hye-south-korea-extortion-accomplice-prosecutors.html}{names
her as an accomplice}.

While refusing to testify or be questioned, Ms. Park has vehemently
asserted her innocence in other forums. She did so again on Sunday, in a
meeting with South Korean reporters at the presidential Blue House,
arguing that the allegations against her had been fabricated.

Kweon Seong-dong, who leads the legal team arguing for impeachment
before the Constitutional Court, chided Ms. Park for speaking to
reporters but not appearing in court. ``If she had anything to say, she
should have appeared at court, out of courtesy for the judges,'' Mr.
Kweon said after Tuesday's hearing.

Advertisement

\protect\hyperlink{after-bottom}{Continue reading the main story}

\hypertarget{site-index}{%
\subsection{Site Index}\label{site-index}}

\hypertarget{site-information-navigation}{%
\subsection{Site Information
Navigation}\label{site-information-navigation}}

\begin{itemize}
\tightlist
\item
  \href{https://help.nytimes.com/hc/en-us/articles/115014792127-Copyright-notice}{©~2020~The
  New York Times Company}
\end{itemize}

\begin{itemize}
\tightlist
\item
  \href{https://www.nytco.com/}{NYTCo}
\item
  \href{https://help.nytimes.com/hc/en-us/articles/115015385887-Contact-Us}{Contact
  Us}
\item
  \href{https://www.nytco.com/careers/}{Work with us}
\item
  \href{https://nytmediakit.com/}{Advertise}
\item
  \href{http://www.tbrandstudio.com/}{T Brand Studio}
\item
  \href{https://www.nytimes.com/privacy/cookie-policy\#how-do-i-manage-trackers}{Your
  Ad Choices}
\item
  \href{https://www.nytimes.com/privacy}{Privacy}
\item
  \href{https://help.nytimes.com/hc/en-us/articles/115014893428-Terms-of-service}{Terms
  of Service}
\item
  \href{https://help.nytimes.com/hc/en-us/articles/115014893968-Terms-of-sale}{Terms
  of Sale}
\item
  \href{https://spiderbites.nytimes.com}{Site Map}
\item
  \href{https://help.nytimes.com/hc/en-us}{Help}
\item
  \href{https://www.nytimes.com/subscription?campaignId=37WXW}{Subscriptions}
\end{itemize}
