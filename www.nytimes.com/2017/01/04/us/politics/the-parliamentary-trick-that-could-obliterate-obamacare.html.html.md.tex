Sections

SEARCH

\protect\hyperlink{site-content}{Skip to
content}\protect\hyperlink{site-index}{Skip to site index}

\href{https://www.nytimes.com/section/politics}{Politics}

\href{https://myaccount.nytimes.com/auth/login?response_type=cookie\&client_id=vi}{}

\href{https://www.nytimes.com/section/todayspaper}{Today's Paper}

\href{/section/politics}{Politics}\textbar{}The Parliamentary Tactic
That Could Obliterate Obamacare

\url{https://nyti.ms/2j3Jhd7}

\begin{itemize}
\item
\item
\item
\item
\item
\end{itemize}

Advertisement

\protect\hyperlink{after-top}{Continue reading the main story}

Supported by

\protect\hyperlink{after-sponsor}{Continue reading the main story}

\hypertarget{the-parliamentary-tactic-that-could-obliterate-obamacare}{%
\section{The Parliamentary Tactic That Could Obliterate
Obamacare}\label{the-parliamentary-tactic-that-could-obliterate-obamacare}}

\includegraphics{https://static01.nyt.com/images/2017/01/05/us/05qanda/05qanda-articleLarge.jpg?quality=75\&auto=webp\&disable=upscale}

By \href{https://www.nytimes.com/by/robert-pear}{Robert Pear}

\begin{itemize}
\item
  Jan. 4, 2017
\item
  \begin{itemize}
  \item
  \item
  \item
  \item
  \item
  \end{itemize}
\end{itemize}

WASHINGTON --- Republicans hope to repeal major parts of the Affordable
Care Act using an expedited procedure known as budget reconciliation.

The process is sometimes called arcane, but it has been used often in
the past 35 years to write some of the nation's most important laws.
``Reconciliation is probably the most potent budget enforcement tool
available to Congress for a large portion of the budget,'' the
Congressional Research Service, a nonpartisan arm of Congress, has said.

Here is a primer.

\textbf{Q.} What is the budget reconciliation process?

\textbf{A.} It is a way for Congress to speed action on legislation that
changes taxes or spending, especially spending for entitlement programs
like Medicare and Medicaid. Although conceived primarily as a way to
reduce federal budget deficits, it has also been used to cut taxes and
to create programs that increase spending --- changes that can raise
deficits.

In the Senate, a reconciliation bill can ordinarily be passed with a
simple majority. For other bills, a 60-vote majority is often needed to
limit debate and move to a final vote.

\textbf{Q.} Why is it called reconciliation?

\textbf{A.} The term originated in the Congressional Budget and
Impoundment Control Act of 1974, which was intended to give Congress
more control over the budget process by allowing lawmakers to set
overall levels of spending and revenue.

The process begins with a budget blueprint, a resolution that guides
Congress but is not presented to the president for a signature or veto.
It recommends federal revenue, deficit, debt and spending levels in
areas like defense, energy, education and health care.

The resolution may direct one or more committees to develop legislation
to achieve specified budgetary results. By adopting these proposals,
Congress can change existing laws so that actual revenue and spending
are brought into line with --- reconciled with --- policies in the
budget resolution.

\textbf{Q.} How has reconciliation been used?

\textbf{A.} Since 1980, Congress has completed action on 24 budget
reconciliation bills. Twenty became law. Four were vetoed.

The Omnibus Budget Reconciliation Act of 1981 was a vehicle for much of
the ``Reagan revolution.'' It squeezed savings out of Social Security,
Medicare, Medicaid, food stamps, the school lunch program, farm
subsidies, student loans, welfare and jobless benefits, among many other
programs.

In 1996, Congress reversed six decades of social welfare policy,
eliminating the individual entitlement to cash assistance for the
nation's poorest children and giving each state a lump sum of federal
money with vast discretion over its use. Those changes were made in a
reconciliation bill, pushed by Republicans but signed by President Bill
Clinton.

Congress reduced deficits with another reconciliation bill, the Balanced
Budget Act of 1997. That law also created the Children's Health
Insurance Program, primarily for uninsured children in low-income
families. On the same day in 1997, Mr. Clinton signed a separate
reconciliation bill that cut taxes.

\href{https://www.nytimes.com/interactive/2016/12/03/us/politics/why-it-will-be-hard-to-repeal-obamacare.html}{}

\includegraphics{https://static01.nyt.com/images/2016/12/03/us/politics/why-it-will-be-hard-to-repeal-obamacare-1480740532639/why-it-will-be-hard-to-repeal-obamacare-1480740532639-thumbLarge.jpg}

\hypertarget{how-republicans-can-repeal-obamacare-piece-by-piece}{%
\subsection{How Republicans Can Repeal Obamacare Piece by
Piece}\label{how-republicans-can-repeal-obamacare-piece-by-piece}}

Peeling away pieces of the law could lead to market chaos.

The Bush tax cuts were adopted in reconciliation bills signed by
President George W. Bush in 2001 and 2003.

On several occasions, Congress has increased assistance to low-income
working families by increasing the earned-income tax credit in
reconciliation bills.

Congress also made changes to the Affordable Care Act in a
reconciliation bill passed immediately after President Obama signed the
health care overhaul in 2010. Later, when Republicans controlled both
houses of Congress, they passed a reconciliation bill to eviscerate the
Affordable Care Act, but Mr. Obama vetoed the bill in January 2016.

Republicans say that measure will provide a template or starting point
for their efforts to undo the health care law this year, with support
from President-elect Donald J. Trump, who calls the law ``an absolute
disaster.''

\textbf{Q.} How does the reconciliation process work in the Senate?

\textbf{A.} In the House, leaders of the majority party can usually
control what happens if their members stick together. In the Senate, by
contrast, one member or a handful of senators can often derail the
leaders' plans. The reconciliation process enhances the power of the
majority party and its leaders. Senate debate on a reconciliation bill
is normally limited to 20 hours, so it cannot be filibustered on the
Senate floor.

The Senate has a special rule to prevent abuse of the budget
reconciliation process. The rule, named for former Senator Robert C.
Byrd, Democrat of West Virginia, generally bars use of the procedure to
consider legislation that has no effect on spending, taxes and deficits.
The Senate parliamentarian normally decides whether particular
provisions violate the Byrd rule, but the Senate can waive the rule with
a 60-vote majority.

\textbf{Q.} What does this mean for the Affordable Care Act?

\textbf{A.} Republicans hope to use the fast-track procedure of budget
reconciliation to repeal or nullify provisions of the law that affect
spending and taxes. They could, for example, eliminate penalties imposed
on people who go without insurance and on larger employers who do not
offer coverage to employees.

They could use a reconciliation bill to eliminate tens of billions of
dollars provided each year to states that have expanded eligibility for
Medicaid. And they could use it to repeal subsidies for private health
insurance coverage obtained through the public marketplaces known as
exchanges.

Republicans could also repeal a number of taxes and fees imposed on
certain high-income people and on health insurers and manufacturers of
brand-name prescription drugs and medical devices: tax increases that
help offset the cost of the insurance coverage expansions.

Those provisions were all rolled back in the reconciliation bill Mr.
Obama vetoed last January. That bill did not touch insurance market
standards established in the Affordable Care Act, which do not directly
cost the government money or raise taxes. The standards stipulate, for
example, that insurers cannot deny coverage or charge higher premiums
because of a person's pre-existing conditions. Insurers must allow
parents to keep children on their policies until the age of 26, and they
cannot charge women higher rates than men, as they often did in the
past.

Such provisions are politically popular, but it is not clear how they
could remain in force without the coverage expansions that help insurers
afford such regulations. Without an effective requirement for people to
carry insurance, and without subsidies, supporters of the health law say
many healthy people would go without coverage, knowing they could obtain
it if they became ill and needed it.

Democrats say they will fight to preserve the law after Mr. Obama leaves
office. Recent history shows that lobbying and public pressure can
sometimes make a difference, altering the votes of individual lawmakers
and changing the contents of a reconciliation bill.

Advertisement

\protect\hyperlink{after-bottom}{Continue reading the main story}

\hypertarget{site-index}{%
\subsection{Site Index}\label{site-index}}

\hypertarget{site-information-navigation}{%
\subsection{Site Information
Navigation}\label{site-information-navigation}}

\begin{itemize}
\tightlist
\item
  \href{https://help.nytimes.com/hc/en-us/articles/115014792127-Copyright-notice}{©~2020~The
  New York Times Company}
\end{itemize}

\begin{itemize}
\tightlist
\item
  \href{https://www.nytco.com/}{NYTCo}
\item
  \href{https://help.nytimes.com/hc/en-us/articles/115015385887-Contact-Us}{Contact
  Us}
\item
  \href{https://www.nytco.com/careers/}{Work with us}
\item
  \href{https://nytmediakit.com/}{Advertise}
\item
  \href{http://www.tbrandstudio.com/}{T Brand Studio}
\item
  \href{https://www.nytimes.com/privacy/cookie-policy\#how-do-i-manage-trackers}{Your
  Ad Choices}
\item
  \href{https://www.nytimes.com/privacy}{Privacy}
\item
  \href{https://help.nytimes.com/hc/en-us/articles/115014893428-Terms-of-service}{Terms
  of Service}
\item
  \href{https://help.nytimes.com/hc/en-us/articles/115014893968-Terms-of-sale}{Terms
  of Sale}
\item
  \href{https://spiderbites.nytimes.com}{Site Map}
\item
  \href{https://help.nytimes.com/hc/en-us}{Help}
\item
  \href{https://www.nytimes.com/subscription?campaignId=37WXW}{Subscriptions}
\end{itemize}
