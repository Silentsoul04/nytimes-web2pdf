Sections

SEARCH

\protect\hyperlink{site-content}{Skip to
content}\protect\hyperlink{site-index}{Skip to site index}

\href{https://www.nytimes.com/section/world/americas}{Americas}

\href{https://myaccount.nytimes.com/auth/login?response_type=cookie\&client_id=vi}{}

\href{https://www.nytimes.com/section/todayspaper}{Today's Paper}

\href{/section/world/americas}{Americas}\textbar{}Mexicans Are the Nafta
Winners? It's News to Them

\url{https://nyti.ms/2hTqvqV}

\begin{itemize}
\item
\item
\item
\item
\item
\end{itemize}

Advertisement

\protect\hyperlink{after-top}{Continue reading the main story}

Supported by

\protect\hyperlink{after-sponsor}{Continue reading the main story}

\hypertarget{mexicans-are-the-nafta-winners-its-news-to-them}{%
\section{Mexicans Are the Nafta Winners? It's News to
Them}\label{mexicans-are-the-nafta-winners-its-news-to-them}}

\includegraphics{https://static01.nyt.com/images/2017/01/05/world/05nafta/05nafta-articleLarge.jpg?quality=75\&auto=webp\&disable=upscale}

By \href{http://www.nytimes.com/by/azam-ahmed}{Azam Ahmed} and
\href{https://www.nytimes.com/by/elisabeth-malkin}{Elisabeth Malkin}

\begin{itemize}
\item
  Jan. 4, 2017
\item
  \begin{itemize}
  \item
  \item
  \item
  \item
  \item
  \end{itemize}
\end{itemize}

APODACA, Mexico --- In 30 years at Whirlpool, working at the company's
manufacturing plant in this industrial Mexican town, José Luis Rico has
witnessed some pretty major changes.

The work force has grown, churning out refrigerators that look more like
robots than the simple models of his early career. Fueling the changes
was a free-trade agreement among Mexico, Canada and the United States
that promised to lift Mexico into the future.

What did not seem to go up, however, was Mr. Rico's salary. After a
handful of raises, he still earns well under \$10,000 a year --- a sum,
he argues, that hardly makes Mexico the big winner of the North American
Free Trade Agreement that President-elect Donald J. Trump says it is.

In fact, to Mr. Rico and many other Mexican workers, politicians and
economists, Nafta does not feel much like a win at all.

``It's more like survival,'' Mr. Rico said. ``I thought it would make my
life better, that this agreement would create opportunities for
everyone.''

``Maybe it has,'' he added, nodding toward the Whirlpool logo on the
entrance to the complex. ``Just not for us.''

Mr. Trump made questioning the virtues of Nafta a centerpiece of his
campaign, at one point calling it ``the worst trade deal maybe ever
signed anywhere,'' and he has not slowed down since his election. On
Tuesday alone,
\href{https://twitter.com/realDonaldTrump/status/816260343391514624}{he
criticized General Motors} for shipping cars made in Mexico to the
United States, claimed credit for
\href{https://www.nytimes.com/2017/01/03/business/ford-general-motors-trump.html?hp\&action=click\&pgtype=Homepage\&clickSource=story-heading\&module=first-column-region\&region=top-news\&WT.nav=top-news}{a
decision by Ford} to cancel plans for a new factory in Mexico, and named
a well-known advocate of protectionist policies, Robert Lighthizer, his
\href{https://www.nytimes.com/2017/01/03/us/politics/trump-robert-lighthizer-trade-mexico.html}{chief
trade negotiator}.

His argument has driven the narrative that where the American worker
lost, the Mexican economy gained.

But here in Mexico, there is an increasing belief that Nafta, despite
drawing an enormous amount of investment to the country, has been a big
disappointment.

``At the end of the day, as a development strategy, it should have led
to higher sustained growth, generated well-paid salaries and reduced the
gap between Mexico and the United States,'' said Gerardo Esquivel, an
economist at the Colegio de México. ``It has remained well below what
was hoped for.''

Mexico's economy has grown an average of just 2.5 percent a year under
Nafta, a fraction of what was needed to provide the jobs and prosperity
its supporters promised. More than half of Mexicans still live below the
poverty line, a proportion that remains unchanged from 1993, before the
deal went into effect.

Wages in Mexico have stagnated for more than a decade, and the stubborn
gap between the nation's rich and poor persists. A majority of workers
in Mexico toil in the obscurity of under-the-table jobs at workshops,
markets and farms for their survival.

New technologies, meanwhile, have cut many jobs while increasing
productivity, which is good news for businesses but a blow to the work
force.

``Mexico is seeing exactly the same phenomenon as in the United
States,'' said Timothy A. Wise, a research fellow at Tufts University.
``Workers have declining bargaining power on both sides of the border.''

In part, Nafta's failure to achieve its potential falls on the Mexican
government's shoulders, experts say. Rather than use the agreement as a
launching point to grow and invest in many sectors of the Mexican
economy, successive governments viewed the trade deal as a silver bullet
for the country's economic woes.

All of this is not lost on Mexicans, despite their government's defense
of Nafta. A recent poll by Parametría, a respected Mexican pollster,
found that more than two-thirds of respondents believed that Nafta had
benefited American consumers and businesses, while just 20 percent
believed it had been good for them. The poll, consisting of 800
interviews in people's homes, had a margin of sampling error of plus or
minus 3.5 percentage points.

``There is a grand narrative in the United States that Mexico was the
great winner of Nafta,'' said Fernando Turner Dávila, the secretary of
the economy and labor in the industrial state of Nuevo León.
``Meanwhile, here in Mexico, they only see the benefits, which are
glorified. They never see the downsides, much less talk about them.''

Mr. Turner cited the loss of nearly two million jobs in the agricultural
industry because of the treaty, which benefited highly subsidized
industries in the United States like corn to the detriment of Mexican
farmers. And while the federal government lauds the increase in
manufacturing exports, Mexico still relies on a tremendous number of
imports from the United States.

``The Mexican government has not established policies to protect Mexican
businesses,'' said Mr. Turner, himself a businessman, with factories in
a half-dozen countries.

That said, even critics like Mr. Turner do not want to see Nafta gutted.
It is an imperfect deal, one that has failed to deliver on its promise,
he said. But to terminate the treaty would be a disaster, he said,
hurting both Mexico and the United States and creating even more job
losses.

It would also not happen easily, critics contend.

After two decades, the two economies are tightly braided together. Goods
manufactured by companies operating in both countries --- whether
speakers, cars or airplanes --- cross the border multiple times during
production, a shared manufacturing process that, if destroyed, would
mean shared job losses.

``A lot of people are taking solace in the reality that it's very
difficult for the U.S. to impose tariffs on Mexico without damaging the
U.S. economy as well,'' said Christopher Wilson, a scholar at the
Woodrow Wilson Institute. ``You need something to replace Nafta.
Otherwise you're going to leave a lot of American workers out in the
cold.''

The agreement has certainly brought positive changes to Mexico,
economists note. Since it went into effect at the beginning of 1994,
billions of dollars in investment has been pouring into Mexico every
year.

Sleepy provincial towns have become manufacturing hubs. Workers assemble
Ford Fusion Hybrid cars in the city of Hermosillo and Whirlpool
refrigerators outside Monterrey. Tijuana sends flat-screen televisions
across the border and the state of Querétaro hammers out parts for
helicopters and corporate jets.

For two decades, those exports have been the main driver of growth in
Mexico, which is why Mexico's government is so eager to defend the
country's trade relationship with the United States.

Without the agreement, the foreign investment that creates new jobs will
slow, or even vanish, some fear. Mexicans got a forewarning of the
possible effect this week. After poor sales and criticism from Mr.
Trump, Ford announced that it would cancel a planned car plant in San
Luis Potosí, a state that Nafta has transformed into a hub for auto
manufacturing.

``Mexico has done a lot right,'' said Gordon H. Hanson, a trade expert
at the University of California, San Diego. ``It has a lot to be proud
of. It has developed a middle class that lives in cities, that educates
their children. It's not the Mexico of 1993.''

The image of these bustling factories feeds the idea that Mexico is
responsible for the hollowing out of America's industrial heartland. But
the reality has turned out to be much more complicated.

While American companies moved jobs to lower-wage Mexico to remain
competitive, some new jobs emerged in the United States, in design or
engineering, or in plants to make parts for the Mexican factories. In
the end, ``Nafta did not cause the huge job losses feared by the critics
or the large economic gains predicted by supporters,'' the
\href{https://fas.org/sgp/crs/row/R42965.pdf}{Congressional Research
Service} concluded in 2015.

In Mexico, the hope was to mimic the success of East Asia's so-called
tigers, using free trade as the catalyst to modernize and overhaul the
economy through exports. Instead, Mexico produced the exports, but not
the growth. It even fell behind most other countries in Latin America
during the 2000s.

But Nafta was not necessarily the problem. Much of the misstep, experts
say, was the Mexican government's belief that the agreement would be
enough to transform the economy all by itself. Thinking of the trade
deal as a panacea, the government failed to come up with a broader
policy or make the investments needed to use the trade agreement as a
lever to transform the whole economy.

Investments in research and development, for instance, have failed to
materialize in both the public and private sectors. Government spending
on infrastructure has dropped to its lowest level in seven decades,
experts say, leaving an unreliable network of ports, highways and even
internet connections across the country. Burdensome regulation and
corruption stifled investment, while the nation's banks lent far less
than their Latin American peers, leaving small companies to scramble for
credit.

Even where Nafta is succeeding, it is not pushing wages up, or creating
enough needed jobs.

Rodolfo de la Torre, an economist with the Espinosa Yglesias Center for
Studies in Mexico City, said officials initially hoped Nafta would bring
jobs to the mass of poorly educated workers in Mexico. But by the early
2000s, much of that low-skilled work had left for China, where labor was
cheaper.

Jobs for better-educated workers in Mexico remained, in part because of
the technological advances in the industrial plants.

Now, in many of the manufacturing hubs of Mexico, wages, and hopes, have
been frozen.

For 10 years, Jorge Augustín Martínez has driven a forklift for Prolec,
a joint venture with General Electric that makes transformers. A father
of two, he earns about \$100 for a six-day workweek.

Though he has received modest cost-of-living increases, his last raise
was five years ago, when gas, food and household items were far cheaper,
he said. It was also before his second son was born. Between housing,
insurance, savings and other requisites, he is left with about \$40 a
week to buy food and other necessities for his family, he said.

Some of the engineers in the plant make more, he said, but no one is
thriving.

``We're all the same, fighting to make ends meet,'' he said. ``I don't
know anyone who is very comfortable.''

Advertisement

\protect\hyperlink{after-bottom}{Continue reading the main story}

\hypertarget{site-index}{%
\subsection{Site Index}\label{site-index}}

\hypertarget{site-information-navigation}{%
\subsection{Site Information
Navigation}\label{site-information-navigation}}

\begin{itemize}
\tightlist
\item
  \href{https://help.nytimes.com/hc/en-us/articles/115014792127-Copyright-notice}{©~2020~The
  New York Times Company}
\end{itemize}

\begin{itemize}
\tightlist
\item
  \href{https://www.nytco.com/}{NYTCo}
\item
  \href{https://help.nytimes.com/hc/en-us/articles/115015385887-Contact-Us}{Contact
  Us}
\item
  \href{https://www.nytco.com/careers/}{Work with us}
\item
  \href{https://nytmediakit.com/}{Advertise}
\item
  \href{http://www.tbrandstudio.com/}{T Brand Studio}
\item
  \href{https://www.nytimes.com/privacy/cookie-policy\#how-do-i-manage-trackers}{Your
  Ad Choices}
\item
  \href{https://www.nytimes.com/privacy}{Privacy}
\item
  \href{https://help.nytimes.com/hc/en-us/articles/115014893428-Terms-of-service}{Terms
  of Service}
\item
  \href{https://help.nytimes.com/hc/en-us/articles/115014893968-Terms-of-sale}{Terms
  of Sale}
\item
  \href{https://spiderbites.nytimes.com}{Site Map}
\item
  \href{https://help.nytimes.com/hc/en-us}{Help}
\item
  \href{https://www.nytimes.com/subscription?campaignId=37WXW}{Subscriptions}
\end{itemize}
