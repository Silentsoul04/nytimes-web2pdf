Sections

SEARCH

\protect\hyperlink{site-content}{Skip to
content}\protect\hyperlink{site-index}{Skip to site index}

\href{https://www.nytimes.com/section/business}{Business}

\href{https://myaccount.nytimes.com/auth/login?response_type=cookie\&client_id=vi}{}

\href{https://www.nytimes.com/section/todayspaper}{Today's Paper}

\href{/section/business}{Business}\textbar{}Trump's Twitter Warning to
Toyota Unsettles Japanese Carmakers

\url{https://nyti.ms/2hZMtZH}

\begin{itemize}
\item
\item
\item
\item
\item
\end{itemize}

Advertisement

\protect\hyperlink{after-top}{Continue reading the main story}

Supported by

\protect\hyperlink{after-sponsor}{Continue reading the main story}

\hypertarget{trumps-twitter-warning-to-toyota-unsettles-japanese-carmakers}{%
\section{Trump's Twitter Warning to Toyota Unsettles Japanese
Carmakers}\label{trumps-twitter-warning-to-toyota-unsettles-japanese-carmakers}}

\includegraphics{https://static01.nyt.com/images/2017/01/07/us/07TOYOTA-1/07TOYOTA-1-articleInline.jpg?quality=75\&auto=webp\&disable=upscale}

By \href{http://www.nytimes.com/by/motoko-rich}{Motoko Rich}

\begin{itemize}
\item
  Jan. 6, 2017
\item
  \begin{itemize}
  \item
  \item
  \item
  \item
  \item
  \end{itemize}
\end{itemize}

TOKYO --- Donald J. Trump sent shivers across the Japanese auto industry
on Friday after warning Toyota on Twitter that he would impose a
``\href{https://twitter.com/realDonaldTrump/status/817071792711942145}{big
border tax}'' on the company if it built a new plant in Mexico.

It appeared to be the first time he had taken on a foreign company for
plans that did not directly involve the United States. The effects were
immediate: Shares in Toyota and other carmakers fell in trading on the
Tokyo Stock Exchange on Friday. And Japanese government officials
hustled to respond to the rhetoric with soothing reminders of the jobs
that the country's auto manufacturers had created in the United States.

``Toyota itself has tried to be a good corporate citizen in the U.S. to
date,'' said Yoshihide Suga, chief cabinet secretary to Prime Minister
Shinzo Abe.

Daily Gendai, one of Japan's leading tabloid newspapers, declared in a
front-page headline that ``Trump tries to smash Toyota,'' and another
tabloid, Evening Fuji, hinted at a coming battle with its headline,
``Trump vs. Toyota.''

Throughout his presidential campaign, Mr. Trump said he would punish
American companies that moved manufacturing plants offshore. Soon after
his election, he claimed credit for persuading Carrier, the
air-conditioner company,
\href{https://www.nytimes.com/2016/11/29/business/trump-to-announce-carrier-plant-will-keep-jobs-in-us.html?_r=0}{to
keep 1,000 jobs in Indiana} that it had previously planned to move to
Mexico. And he
\href{https://twitter.com/realDonaldTrump/status/816635078067490816}{thanked
Ford Motor} on Twitter this week for abandoning plans to build a
small-car assembly plant in Mexico that Mr. Trump had repeatedly
criticized.

Mr. Trump's Twitter post was not entirely accurate. He said Toyota would
build a new Corolla factory in Baja, but the company is actually
planning to build a new plant in Guanajuato, Mexico. (It already has a
factory in Baja.) More significant, Toyota's new plant in Mexico will
not replace any of its 10 factories in the United States, where the
company employs 136,000 people. The company said it had invested about
\$21.9 billion in the United States.

``I think being fair is not really in the playbook of the
president-elect,'' said Takuji Okubo, managing director and chief
economist at Japan Macro Advisors.

Toyota builds Corollas in Cambridge, Ontario, as well as in Blue
Springs, Miss. No workers in either of those plants will lose jobs, and
when Toyota opens the new facility in Mexico, the company plans to shift
the Canadian workers to making small RAV4 sport utility vehicles.

According to Hiroshige Seko, Japan's minister of economy, trade and
industry, Japanese carmakers manufactured about 3.86 million cars in the
United States in 2015, up from 1.5 million in the 1990s, and employ
about 1.5 million people.

In response to a question about Mr. Trump's post at a meeting of the
Japan Automobile Manufacturers Association, Akio Toyoda, president of
Toyota, said the company would not change its manufacturing plans in
Mexico.

``I don't know yet exactly how, but, regardless of who becomes
president, our business is about being good corporate citizens,'' Mr.
Toyoda said, according to a report by The Associated Press. ``And by
becoming good corporate citizens, we are facing the same goal of making
America strong. And so we will continue to do our best.''

Even though there are only two weeks left to the inauguration, some
analysts suggested Mr. Trump might still be in campaign-promise mode.

``At the moment, it's quite unclear whether Mr. Trump actually will push
this policy after assuming office,'' said Yoshio Tsukuda, founder of
Tsukuda Mobility Research Institute, an auto industry research firm.
``Right now we don't know if he is just bluffing or serious.''

Image

Akio Toyoda, president of Toyota.Credit...Kiyoshi Ota/European
Pressphoto Agency

But others said some of Mr. Trump's words could spur changes on their
own.

``I think the threat alone can actually force companies to behave in a
more Trump way,'' Mr. Okubo said. ``So I think regardless of whether the
border tax would actually be implemented, I think he will continue to
use the threat to pressure companies.''

While Mr. Trump singled out Toyota on Thursday, analysts said that his
views on trade clearly extended beyond a single company.

``The U.S. is no longer a champion of free trade as a nation,'' wrote
Genki Fujii, visiting professor at Takushoku University in Tokyo, in a
column in Evening Fuji.

In comments to the Japanese broadcaster NHK, Takao Shindo, president of
\href{http://www.nssmc.com/}{Nippon Steel \& Sumitomo Metal
Corporation}, expressed concerns that Mr. Trump would repeal the North
American Free Trade Agreement, which affects Japanese manufacturers with
plants in Mexico shipping to the United States. These companies ``will
face a tough situation,'' Mr. Shindo said.

Some analysts said that once Mr. Trump understood what Toyota was doing
with its plants, he might back down.

``Toyota still is expanding local U.S. production in the long term,''
said Takaki Nakanishi, an independent auto industry analyst in Tokyo.
``Not at this point in time, because the U.S. market is near saturation,
but over the long term I think there is more capacity to expand the
Mississippi plant.''

``Job growth is coming in the long term'' in the United States, he
added. ``That's why I think Donald is misunderstanding Toyota's
intention.''

But Mr. Nakanishi, who was on his way to the annual North American
International Auto Show in Detroit, said he was not surprised by Mr.
Trump's rhetoric.

``He wants to protect jobs for the United States,'' he said. ``So
regardless of the origin of the company, he is just trying to do his
job.''

Japan has already been rattled by Mr. Trump's election given his
campaign criticism of Japanese trade barriers and the cost of United
States military support. Prime Minister Abe was the
\href{https://www.nytimes.com/2016/11/17/world/asia/shinzo-abe-donald-trump.html}{first
world leader to land a meeting} with the president-elect, and he is
seeking a follow-up meeting shortly after the inauguration. His advisers
have also been
\href{https://www3.nhk.or.jp/nhkworld/en/news/20170106_13/}{meeting with
members of Congress and Mr. Trump's transition team} .

Some economists fear that Mr. Trump could return to the trade wars of
the 1980s with his policies. Jun Saito, a senior research fellow at the
Japan Center for Economic Research, said economic conditions were
starting to resemble those of the United States in the 1980s, when the
dollar was stronger and the country had a large trade deficit with
Japan.

``That was the background for the trade friction between Japan and the
United States,'' Mr. Saito said. Japan's trade surplus with the United
States --- about 7.2 trillion yen, or about \$62 billion, in 2015 --- is
smaller than China's, but it could nevertheless draw Mr. Trump's ire.
``I think we have to be very careful. We can't be too optimistic.''

Mr. Okubo of Japan Macro Advisors said Mr. Trump might also accuse the
Bank of Japan of currency manipulation, given that in inflation-adjusted
terms,
\href{https://www.japanmacroadvisors.com/page/category/economic-indicators/financial-markets/exchange-rates/?gids=190-5\&graph_index=0\&graph_type=line\&graph_data_from=11-2-1970\&graph_data_to=1-11-2016}{the
yen is at a 40-year low}.

``I think this could be just the beginning,'' Mr. Okubo said. ``I think
the Japanese government has to be really careful not to act in a way
that could be interpreted as manipulating the exchange rate.''

Advertisement

\protect\hyperlink{after-bottom}{Continue reading the main story}

\hypertarget{site-index}{%
\subsection{Site Index}\label{site-index}}

\hypertarget{site-information-navigation}{%
\subsection{Site Information
Navigation}\label{site-information-navigation}}

\begin{itemize}
\tightlist
\item
  \href{https://help.nytimes.com/hc/en-us/articles/115014792127-Copyright-notice}{©~2020~The
  New York Times Company}
\end{itemize}

\begin{itemize}
\tightlist
\item
  \href{https://www.nytco.com/}{NYTCo}
\item
  \href{https://help.nytimes.com/hc/en-us/articles/115015385887-Contact-Us}{Contact
  Us}
\item
  \href{https://www.nytco.com/careers/}{Work with us}
\item
  \href{https://nytmediakit.com/}{Advertise}
\item
  \href{http://www.tbrandstudio.com/}{T Brand Studio}
\item
  \href{https://www.nytimes.com/privacy/cookie-policy\#how-do-i-manage-trackers}{Your
  Ad Choices}
\item
  \href{https://www.nytimes.com/privacy}{Privacy}
\item
  \href{https://help.nytimes.com/hc/en-us/articles/115014893428-Terms-of-service}{Terms
  of Service}
\item
  \href{https://help.nytimes.com/hc/en-us/articles/115014893968-Terms-of-sale}{Terms
  of Sale}
\item
  \href{https://spiderbites.nytimes.com}{Site Map}
\item
  \href{https://help.nytimes.com/hc/en-us}{Help}
\item
  \href{https://www.nytimes.com/subscription?campaignId=37WXW}{Subscriptions}
\end{itemize}
