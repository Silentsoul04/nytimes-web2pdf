Sections

SEARCH

\protect\hyperlink{site-content}{Skip to
content}\protect\hyperlink{site-index}{Skip to site index}

\href{https://www.nytimes.com/section/politics}{Politics}

\href{https://myaccount.nytimes.com/auth/login?response_type=cookie\&client_id=vi}{}

\href{https://www.nytimes.com/section/todayspaper}{Today's Paper}

\href{/section/politics}{Politics}\textbar{}Putin Led a Complex
Cyberattack Scheme to Aid Trump, Report Finds

\url{https://nyti.ms/2jbRig6}

\begin{itemize}
\item
\item
\item
\item
\item
\item
\end{itemize}

Advertisement

\protect\hyperlink{after-top}{Continue reading the main story}

Supported by

\protect\hyperlink{after-sponsor}{Continue reading the main story}

\hypertarget{putin-led-a-complex-cyberattack-scheme-to-aid-trump-report-finds}{%
\section{Putin Led a Complex Cyberattack Scheme to Aid Trump, Report
Finds}\label{putin-led-a-complex-cyberattack-scheme-to-aid-trump-report-finds}}

\includegraphics{https://static01.nyt.com/images/2017/01/07/us/07trump3_hp/07trump3_hp-articleInline.jpg?quality=75\&auto=webp\&disable=upscale}

By \href{http://www.nytimes.com/by/michael-d-shear}{Michael D. Shear}
and \href{http://www.nytimes.com/by/david-e-sanger}{David E. Sanger}

\begin{itemize}
\item
  Jan. 6, 2017
\item
  \begin{itemize}
  \item
  \item
  \item
  \item
  \item
  \item
  \end{itemize}
\end{itemize}

WASHINGTON --- President Vladimir V. Putin of Russia directed a vast
cyberattack aimed at denying Hillary Clinton the presidency and
installing Donald J. Trump in the Oval Office, the nation's top
intelligence agencies said in an extraordinary report they delivered on
Friday to Mr. Trump.

The officials presented their unanimous conclusions to Mr. Trump in a
two-hour briefing at Trump Tower in New York that brought the leaders of
America's intelligence agencies face to face with their most vocal
skeptic, the president-elect, who has repeatedly cast doubt on Russia's
role. The meeting came just two weeks before Mr. Trump's inauguration
and was underway even as the electoral votes from his victory were being
formally counted in a joint session of Congress.

Soon after leaving the meeting, intelligence officials
\href{http://www.nytimes.com/interactive/2017/01/06/us/politics/document-russia-hacking-report-intelligence-agencies.html?_r=0}{released
the declassified, damning report} that described the sophisticated
cybercampaign as part of a continuing Russian effort to weaken the
United States government and its democratic institutions. The report ---
a virtually unheard-of, real-time revelation by the American
intelligence agencies that undermined the legitimacy of the president
who is about to direct them --- made the case that Mr. Trump was the
favored candidate of Mr. Putin.

The Russian leader, the report said, sought to denigrate Mrs. Clinton,
and the report detailed what the officials had revealed to President
Obama a day earlier: Mr. Trump's victory followed a complicated,
multipart cyberinformation attack whose goal had evolved to help the
Republican win.

The 25-page report did not conclude that Russian involvement tipped the
election to Mr. Trump.

The public report lacked the evidence that intelligence officials said
was included in a classified version, which they described as
information on the sources and methods used to collect the information
about Mr. Putin and his associates. Those would include intercepts of
conversations and the harvesting of computer data from ``implants'' that
the United States and its allies have put in Russian computer networks.

\href{https://www.nytimes.com/interactive/2017/01/06/us/russian-hack-evidence.html}{}

\includegraphics{https://static01.nyt.com/images/2017/01/06/us/russian-hack-evidence-1483678537382/russian-hack-evidence-1483678537382-largeHorizontalJumbo.jpg}

\hypertarget{was-it-a-400-pound-14-year-old-hacker-or-russia-heres-some-of-the-evidence}{%
\subsection{Was It a 400-Pound, 14-Year-Old Hacker, or Russia? Here's
Some of the
Evidence}\label{was-it-a-400-pound-14-year-old-hacker-or-russia-heres-some-of-the-evidence}}

Reports released by information security companies provide evidence
about the hacking of United States political officials and
organizations.

Much of the unclassified report focused instead on an overt Kremlin
propaganda campaign that would be unlikely to convince skeptics of the
report's more serious conclusions.

The report may be a political blow to Mr. Trump. But it is also a risky
moment for the intelligence agencies that have become more powerful
since the Sept. 11, 2001, attacks, but have had to fend off allegations
that they exaggerated intelligence during the buildup to the Iraq war.

The declassified report did describe in detail the efforts of Mr. Putin
and his security services, including the creation of the online Guccifer
2.0 persona and DCLeaks.com to release information gained from the hacks
to the public.

``Putin and the Russian Government aspired to help President-elect
Trump's election chances when possible by discrediting Secretary Clinton
and publicly contrasting her unfavorably to him,'' the report by the
nation's intelligence agencies concluded.

Mr. Trump, whose resistance to that very conclusion has led him to
repeatedly mock the country's intelligence services on Twitter since
Election Day, issued a written statement that appeared to concede some
Russian involvement. But Mr. Trump said nothing about the conclusion
that Mr. Putin had sought to aid his candidacy, other than insisting
that he still believes the Russian attacks had no effect on the outcome.

The president-elect's written statement came just hours after Mr. Trump
\href{https://www.nytimes.com/2017/01/06/us/politics/donald-trump-wall-hack-russia.html}{told
The New York Times} in an interview that the storm surrounding Russian
hacking was nothing more than a ``political witch hunt'' carried out by
his adversaries, who he said were embarrassed by their loss to him in
the 2016 election. Speaking by telephone three hours before the
intelligence briefing, Mr. Trump repeatedly criticized the intense focus
on Russia.

``China, relatively recently, hacked 20 million government names,'' he
said, referring to the breach of computers at the Office of Personnel
Management in late 2014 and early 2015. ``How come nobody even talks
about that? This is a political witch hunt.''

Later, Mr. Trump sought to blame the Democrats for any cyberattacks that
might have occurred. ``Gross negligence by the Democratic National
Committee allowed hacking to take place,''
\href{https://twitter.com/realDonaldTrump/status/817579925771341825}{he
said in a Twitter} message posted about 11 p.m. ``The Republican
National Committee had strong defense!''

Vice President-elect Mike Pence told reporters that he and Mr. Trump had
``appreciated the presentation'' by the intelligence officials and
described the conversation as ``respectful.'' Mr. Pence said the new
administration would take aggressive action ``to combat cyberattacks and
protect the security of the American people from this type of intrusion
in the future.''

Mr. Trump, who has consistently questioned the evidence of Russian
hacking during the election, did so again Friday before he met with the
intelligence officials. Asked why he thought there was so much attention
on the Russian cyberattacks, the president-elect said the motivation was
political.

He also repeated his criticism of the American intelligence agencies,
saying that ``a lot of mistakes were made'' in the past, noting in
particular the attacks on the World Trade Center and saying, as he has
repeatedly, that ``weapons of mass destruction was one of the great
mistakes of all time.''

But after meeting with the intelligence officials, Mr. Trump appeared to
moderate his position, conceding that ``Russia, China, other countries,
outside groups and people are consistently trying to break through the
cyberinfrastructure of our governmental institutions, businesses and
organizations, including the Democrat National Committee.''

\href{https://www.nytimes.com/interactive/2016/12/29/us/politics/russian-hack-in-200-words.html}{}

\includegraphics{https://static01.nyt.com/images/2016/12/29/us/politics/russian-hack-in-200-words-1483060431834/russian-hack-in-200-words-1483060431834-square640-v2.png}

\hypertarget{the-russian-hacking-in-200-words}{%
\subsection{The Russian Hacking in 200
Words}\label{the-russian-hacking-in-200-words}}

President Obama announced sanctions against Russia for trying to
influence the 2016 election through cyberattacks. Here's what led to the
sanctions.

The report described a broad campaign of covert operations, including
the ``trolling'' on the internet of people who were viewed as opponents
of Russia's effort. While it accused Russian intelligence agencies of
obtaining and maintaining ``access to elements of multiple U.S. state or
local electoral boards,'' it concluded --- as officials have publicly
--- that there was no evidence of tampering with the tallying of the
vote on Nov. 8.

The report, reflecting the assessments of the C.I.A., the F.B.I. and the
National Security Agency, stopped short of backing up Mr. Trump on his
declaration that the hacking activity had no effect on the election.

``We did not make an assessment of the impact that Russian activities
had on the outcome of the 2016 election,'' the report concluded, saying
it was beyond its responsibility to analyze American ``political
processes'' or public opinion.

The intelligence agencies also concluded ``with high confidence'' that
Russia's main military intelligence unit, the G.R.U., created a
``persona'' called Guccifer 2.0 and a website, DCLeaks.com, to release
the emails of the Democratic National Committee and of the chairman of
the Clinton campaign, John D. Podesta.

When those disclosures received what was seen as insufficient attention,
the report said, the G.R.U. ``relayed material it acquired from the
D.N.C. and senior Democratic officials to WikiLeaks.'' The founder of
WikiLeaks, Julian Assange, has denied that Russia was the source of the
emails it published.

The role of RT --- the Russian English-language news organization that
American intelligence says is a Kremlin propaganda operation --- in the
Kremlin's effort to influence the election is covered in far more detail
by the report than any other aspect of the Russian campaign. An annex in
the report on RT, which was first written in 2012 but not previously
made public, takes up eight pages of the report's 14-page main section.

The report's unequivocal assessment of RT presents an awkward
development for Lt. Gen. Michael T. Flynn, who is Mr. Trump's choice to
serve as national security adviser. Mr. Flynn has appeared repeatedly on
RT's news programs and in December 2015 was paid by the network to give
a speech in Russia and attend its lavish anniversary party, where he sat
at the elbow of Mr. Putin. Mr. Flynn has since defended his speech,
insisting that RT is no different from CNN or MSNBC.

The report also stated that Russia collected data ``on some
Republican-affiliated targets,'' but did not disclose the contents of
whatever it harvested.

Intelligence officials who prepared the classified report have concluded
that British intelligence was among the first to raise an alarm that
Moscow hacked into the Democratic National Committee's computer servers,
and alerted their American counterparts, according to two people
familiar with the conclusions.

The British role, which has been closely held, is a critical part of the
timeline because it suggests that some of the first tipoffs, in fall
2015, came from voice intercepts, computer traffic or informants outside
the United States, as emails and other data from the Democratic National
Committee flowed out of the country.

The conclusions in the report were described on Thursday to President
Obama and on Friday to Mr. Trump by James R. Clapper Jr., the director
of national intelligence; John O. Brennan, the director of the C.I.A.;
Adm. Michael S. Rogers, the director of the National Security Agency;
and James B. Comey, the director of the F.B.I.

The key to the public report's assessment is that Russia's motives
``evolved over the course of the campaign.'' When it appeared that Mrs.
Clinton was more likely to win, it concluded, the Russian effort focused
``on undermining her future presidency,'' with pro-Kremlin bloggers
preparing a Twitter campaign with the hashtag \#DemocracyRIP. It noted
that Mr. Putin had a particular animus for Mrs. Clinton because he
believed she had incited protests against him in 2011.

Yet the attacks, the report said, began long before anyone could have
known that Mr. Trump, considered a dark horse, would win the Republican
nomination. It said the attacks began as early as July 2015, when
Russian intelligence operatives first
\href{https://www.nytimes.com/2016/12/13/us/politics/russia-hack-election-dnc.html}{gained
access to the Democratic National Committee's networks}. Russia
maintained that access for 11 months, until ``at least June 2016,'' the
report concludes, leaving open the possibility that Russian
cyberattackers may have had access even after the firm CrowdStrike
believed that it had kicked them off the networks.

Advertisement

\protect\hyperlink{after-bottom}{Continue reading the main story}

\hypertarget{site-index}{%
\subsection{Site Index}\label{site-index}}

\hypertarget{site-information-navigation}{%
\subsection{Site Information
Navigation}\label{site-information-navigation}}

\begin{itemize}
\tightlist
\item
  \href{https://help.nytimes.com/hc/en-us/articles/115014792127-Copyright-notice}{©~2020~The
  New York Times Company}
\end{itemize}

\begin{itemize}
\tightlist
\item
  \href{https://www.nytco.com/}{NYTCo}
\item
  \href{https://help.nytimes.com/hc/en-us/articles/115015385887-Contact-Us}{Contact
  Us}
\item
  \href{https://www.nytco.com/careers/}{Work with us}
\item
  \href{https://nytmediakit.com/}{Advertise}
\item
  \href{http://www.tbrandstudio.com/}{T Brand Studio}
\item
  \href{https://www.nytimes.com/privacy/cookie-policy\#how-do-i-manage-trackers}{Your
  Ad Choices}
\item
  \href{https://www.nytimes.com/privacy}{Privacy}
\item
  \href{https://help.nytimes.com/hc/en-us/articles/115014893428-Terms-of-service}{Terms
  of Service}
\item
  \href{https://help.nytimes.com/hc/en-us/articles/115014893968-Terms-of-sale}{Terms
  of Sale}
\item
  \href{https://spiderbites.nytimes.com}{Site Map}
\item
  \href{https://help.nytimes.com/hc/en-us}{Help}
\item
  \href{https://www.nytimes.com/subscription?campaignId=37WXW}{Subscriptions}
\end{itemize}
