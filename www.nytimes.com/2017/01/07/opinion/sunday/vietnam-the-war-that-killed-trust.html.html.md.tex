Sections

SEARCH

\protect\hyperlink{site-content}{Skip to
content}\protect\hyperlink{site-index}{Skip to site index}

\href{https://www.nytimes.com/section/opinion/sunday}{Sunday Review}

\href{https://myaccount.nytimes.com/auth/login?response_type=cookie\&client_id=vi}{}

\href{https://www.nytimes.com/section/todayspaper}{Today's Paper}

\href{/section/opinion/sunday}{Sunday Review}\textbar{}Vietnam: The War
That Killed Trust

\href{https://nyti.ms/2jfrjEJ}{https://nyti.ms/2jfrjEJ}

\begin{itemize}
\item
\item
\item
\item
\item
\item
\end{itemize}

Advertisement

\protect\hyperlink{after-top}{Continue reading the main story}

Supported by

\protect\hyperlink{after-sponsor}{Continue reading the main story}

\href{/section/opinion}{Opinion}

\href{/column/vietnam-67}{Vietnam '67}

\hypertarget{vietnam-the-war-that-killed-trust}{%
\section{Vietnam: The War That Killed
Trust}\label{vietnam-the-war-that-killed-trust}}

By Karl Marlantes

\begin{itemize}
\item
  Jan. 7, 2017
\item
  \begin{itemize}
  \item
  \item
  \item
  \item
  \item
  \item
  \end{itemize}
\end{itemize}

\includegraphics{https://static01.nyt.com/images/2017/01/08/opinion/sunday/08marlantes2/08marlantes2-articleInline-v3.jpg?quality=75\&auto=webp\&disable=upscale}

In the early spring of 1967, I was in the middle of a heated 2 a.m.
hallway discussion with fellow students at Yale about the Vietnam War. I
was from a small town in Oregon, and I had already joined the Marine
Corps Reserve. My friends were mostly from East Coast prep schools. One
said that Lyndon B. Johnson was lying to us about the war. I blurted
out, ``But \ldots{} but an American president wouldn't lie to
Americans!'' They all burst out laughing.

When I told that story to my children, they all burst out laughing, too.
Of course presidents lie. All politicians lie. God, Dad, what planet are
you from?

Before the Vietnam War, most Americans were like me. After the Vietnam
War, most Americans are like my children.

America didn't just lose the war, and the lives of 58,000 young men and
women; Vietnam changed us as a country. In many ways, for the worse: It
made us cynical and distrustful of our institutions, especially of
government. For many people, it eroded the notion, once nearly
universal, that part of being an American was serving your country.

But not everything about the war was negative. As a Marine lieutenant in
Vietnam, I saw how it threw together young men from diverse racial and
ethnic backgrounds and forced them to trust one another with their
lives. It was a racial crucible that played an enormous, if often
unappreciated, role in moving America toward real integration.

And yet even as Vietnam continues to shape our country, its place in our
national consciousness is slipping. Some 65 percent of Americans are
under 45 and so unable to even remember the war. Meanwhile, our wars in
Iraq and Afghanistan, our involvement in Syria, our struggle with
terrorism --- these conflicts are pushing Vietnam further into the
background.

All the more reason, then, for us to revisit the war and its
consequences for today. This essay inaugurates a new series by The
Times, Vietnam '67, that will examine how the events of 1967 and early
1968 shaped Vietnam, America and the world. Hopefully, it will generate
renewed conversation around that history, now half a century past.

What readers take away from that conversation is another matter. If all
we do is debate why we lost, or why we were there at all, we will miss
the truly important question: What did the war do to us as Americans?

\hypertarget{cynicism}{%
\subsection{\texorpdfstring{\textbf{CYNICISM}}{CYNICISM}}\label{cynicism}}

Vietnam changed the way we looked at politics. We became inured to our
leaders lying in the war: the fabricated Gulf of Tonkin incident, the
number of ``pacified provinces'' (and what did ``pacified'' mean,
anyway?), the inflated body counts.

People talked about Johnson's ``credibility gap.'' This was a genteel
way of saying that the president was lying. Then, however, a credibility
gap was considered unusual and bad. By the end of the war, it was still
considered bad, but it was no longer unusual. When politicians lie
today, fact checkers might point out what is true, but then everyone
moves on.

We have switched from naïveté to cynicism. One could argue that they are
opposites, but I think not. With naïveté you risk disillusionment, which
is what happened to me and many of my generation. Cynicism, however,
stops you before you start. It alienates us from ``the government,'' a
phrase that today connotes bureaucratic quagmire. It threatens
democracy, because it destroys the power of the people to even want to
make change.

You don't finish the world's largest highway system, build huge numbers
of public schools and universities, institute the Great Society, fight a
major war, and go to the moon, which we did in the 1960s ---
simultaneously --- if you're cynical about government and politicians.

I live near Seattle, hardly Donald J. Trump territory. Most of my
friends cynically deride Mr. Trump's slogan, Make America Great Again,
citing all that was wrong in the olden days. Indeed, it wasn't paradise,
particularly for minorities. But there's some truth to it. We
\emph{were} greater then. It was the war --- not liberalism, not
immigration, not globalization --- that changed us.

\hypertarget{race}{%
\subsection{\texorpdfstring{\textbf{RACE}}{RACE}}\label{race}}

In December 1968, I was on a blasted and remote jungle hilltop about a
kilometer from the demilitarized zone. A chopper dropped off about three
weeks of sodden mail and crumpled care packages. In that pile was a
package for Ray Delgado, an 18-year-old Hispanic kid from Texas. I
watched Ray tear into the aluminum foil wrapping and, smiling broadly,
hold something up for me to see.

``What's that?'' I asked.

``It's tamales. From my mother.''

``What are tamales?''

``You want to try one?'' he asked.

``Sure.'' I looked at it, turned it over, then stuck it in my mouth and
started chewing. Ray and his other Hispanic friends were barely
containing themselves as I was gamely chewing away and thinking, ``No
wonder these Mexicans have such great teeth.''

\includegraphics{https://static01.nyt.com/images/2017/01/08/opinion/sunday/08marlantes1/08marlantes1-articleInline.jpg?quality=75\&auto=webp\&disable=upscale}

``Lieutenant,'' Ray finally said. ``You take the corn husk off.''

I was from a logging town on the Oregon coast. I'd heard of tamales, but
I'd never seen one. Until I joined my company of Marines in Vietnam, I'd
never even talked to a Mexican. Yes, people like me called people like
Ray ``Mexicans,'' even though they were as American as apple pie --- and
tamales. Racial tension where I grew up was the Swedes and Norwegians
squaring off against the Finns every Saturday night in the parking lot
outside the dance at the Labor Temple.

President Harry Truman ordered the integration of the military in 1948.
By the time of the Vietnam War, the races were serving together. But
putting everyone \emph{in} the same units is very different than having
them work together \emph{as} a unit.

Our national memory of integration is mostly about the brave people of
the civil rights movement. Imagine arming all those high school students
from Birmingham, Ala. --- white and black --- with automatic weapons in
an environment where using these weapons was as common as having lunch
and they are all jacked up on testosterone. \emph{That's} racial
tension.

During the war there were over 200 fraggings in the American military
--- murders carried out by fragmentation grenades, which made it
impossible to identify the killer. Almost all fraggings, at least when
the perpetrator was caught, were found to be racially motivated.

And yet the more common experience in combat was cooperation and
respect. If I was pinned down by enemy fire and I needed an M-79 man,
I'd scream for Thompson, because he was the best. I didn't even think
about what color Thompson was.

White guys had to listen to soul music and black guys had to listen to
country music. We didn't fear one another. And the experience stuck with
us. Hundreds of thousands of young men came home from Vietnam with
different ideas about race --- some for the worse, but most for the
better. Racism wasn't solved in Vietnam, but I believe it was where our
country finally learned that it just might be possible for us all to get
along.

\hypertarget{service}{%
\subsection{\texorpdfstring{\textbf{SERVICE}}{SERVICE}}\label{service}}

I was at a reading recently in Fayetteville, N.C., when a young couple
appeared at the signing table. He was standing straight and tall in Army
fatigues. She was holding a baby in one arm and hauling a toddler with
the other. They both looked to be about two years out of high school.
The woman started to cry. I asked her what was wrong, and she said, ``My
husband is shipping out again, tomorrow.'' I turned to him and said,
``Wow, your second tour?''

``No, sir,'' he replied. ``My seventh.''

My heart sank. Is this a republic?

The Vietnam War ushered in the end of the draft, and the creation of
what the Pentagon calls the ``all-volunteer military.'' But I don't. I
call it the all-recruited military. Volunteers are people who rush down
to the post office to sign up after Pearl Harbor or the World Trade
Center gets bombed. Recruits, well, it's more complicated.

When I was growing up, almost every friend's father or uncle had served
in World War II. All the women in town knew that a destroyer was smaller
than a cruiser and a platoon was smaller than a company, because their
husbands had all been on destroyers or in platoons. Back then it was
called ``the service.'' Today, we call it ``the military.''

That shift in language indicates a profound shift in the attitudes of
the republic toward its armed forces. The draft was unfair. Only males
got drafted. And men who could afford to go to college did not get
drafted until late in the war, when the fighting had fallen off.

But getting rid of the draft did not solve unfairness.

America's elites have mostly dropped out of military service. Engraved
on the walls of Woolsey Hall at Yale are the names of hundreds of Yalies
who died in World Wars I and II. Thirty-five died in Vietnam, and none
since.

Instead, the American working class has increasingly borne the burden of
death and casualties since World War II. In
\href{http://www.memphis.edu/law/documents/kriner-shen46.pdf}{a study}
in The University of Memphis Law Review, Douglas Kriner and Francis Shen
looked at the income casualty gap, the difference between the median
household incomes (in constant 2000 dollars) of communities with the
highest casualties (the top 25 percent) and all the other communities.
Starting from almost dead-even in World War II, the casualty gap was
\$5,000 in the Korean War, \$8,200 in the Vietnam War, and is now more
than \$11,000 in Iraq and Afghanistan. Put another way, the lowest three
income deciles have suffered 50 percent more casualties than the highest
three.

If these inequities continue to grow, resentment will grow with it. With
growing resentment, the already wide divide between the military and
civilians will also widen. This is how republics fall, with armies and
parts of the country more loyal to their commander than their country.

We need to return to the \emph{spirit} of the military draft, and how
people felt about service to their country. The military draft was
viewed by most of us the same way we view income tax. I wouldn't pay my
taxes if there wasn't the threat of jail. But as a responsible citizen,
I also see that paying taxes is necessary to fund the government --- my
government.

People would still grumble. We grumble about taxes. People would still
try to pull strings to get more pleasant assignments. But everyone would
serve. They'd work for ``the government,'' and maybe start to see it as
``our government.'' It's a lot harder to be cynical about your country
if you devoted two years of your life making it a better place.

Let the armed services be just one of many ways young people can serve
their country. With universal service, some boy from Seattle could find
himself sharing a tamale with some Hispanic girl from El Paso.
Conservatives and liberals would learn to work together for a common
cause. We could return to the \emph{spirit} of people of different races
learning to work together in combat during the Vietnam War.

The Vietnam War continues to define us, even if we have forgotten how.
But it's not too late to remember, and to do something about it.

Advertisement

\protect\hyperlink{after-bottom}{Continue reading the main story}

\hypertarget{site-index}{%
\subsection{Site Index}\label{site-index}}

\hypertarget{site-information-navigation}{%
\subsection{Site Information
Navigation}\label{site-information-navigation}}

\begin{itemize}
\tightlist
\item
  \href{https://help.nytimes.com/hc/en-us/articles/115014792127-Copyright-notice}{©~2020~The
  New York Times Company}
\end{itemize}

\begin{itemize}
\tightlist
\item
  \href{https://www.nytco.com/}{NYTCo}
\item
  \href{https://help.nytimes.com/hc/en-us/articles/115015385887-Contact-Us}{Contact
  Us}
\item
  \href{https://www.nytco.com/careers/}{Work with us}
\item
  \href{https://nytmediakit.com/}{Advertise}
\item
  \href{http://www.tbrandstudio.com/}{T Brand Studio}
\item
  \href{https://www.nytimes.com/privacy/cookie-policy\#how-do-i-manage-trackers}{Your
  Ad Choices}
\item
  \href{https://www.nytimes.com/privacy}{Privacy}
\item
  \href{https://help.nytimes.com/hc/en-us/articles/115014893428-Terms-of-service}{Terms
  of Service}
\item
  \href{https://help.nytimes.com/hc/en-us/articles/115014893968-Terms-of-sale}{Terms
  of Sale}
\item
  \href{https://spiderbites.nytimes.com}{Site Map}
\item
  \href{https://help.nytimes.com/hc/en-us}{Help}
\item
  \href{https://www.nytimes.com/subscription?campaignId=37WXW}{Subscriptions}
\end{itemize}
