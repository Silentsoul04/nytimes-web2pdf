Sections

SEARCH

\protect\hyperlink{site-content}{Skip to
content}\protect\hyperlink{site-index}{Skip to site index}

\href{https://myaccount.nytimes.com/auth/login?response_type=cookie\&client_id=vi}{}

\href{https://www.nytimes.com/section/todayspaper}{Today's Paper}

\href{/section/upshot}{The Upshot}\textbar{}The Major Potential Impact
of a Corporate Tax Overhaul

\url{https://nyti.ms/2jeVcVl}

\begin{itemize}
\item
\item
\item
\item
\item
\item
\end{itemize}

Advertisement

\protect\hyperlink{after-top}{Continue reading the main story}

Supported by

\protect\hyperlink{after-sponsor}{Continue reading the main story}

Upshot

\href{/column/economic-view}{Economic View}

\hypertarget{the-major-potential-impact-of-a-corporate-tax-overhaul}{%
\section{The Major Potential Impact of a Corporate Tax
Overhaul}\label{the-major-potential-impact-of-a-corporate-tax-overhaul}}

By \href{http://www.nytimes.com/by/neil-irwin}{Neil Irwin}

\begin{itemize}
\item
  Jan. 7, 2017
\item
  \begin{itemize}
  \item
  \item
  \item
  \item
  \item
  \item
  \end{itemize}
\end{itemize}

The United States system for taxing businesses is a mess. If there's one
thing nearly everyone can agree upon, it is that.

The current corporate income tax manages the weird trick of both taxing
companies at
\href{http://taxfoundation.org/article/corporate-income-tax-rates-around-world-2016}{a
higher statutory rate} than other advanced countries while collecting
\href{http://fivethirtyeight.com/datalab/u-s-tax-rates-the-big-picture/}{less
money}, as a percentage of the overall economy, than most of them. It is
infinitely complicated and it gives companies incentives to borrow too
much money and move operations to countries with lower tax rates.

Now, the moment for trying to fix all of that appears to have arrived.
With the House, Senate and presidency all soon to be in Republican hands
and with all agreeing that a major tax bill is a top priority, some kind
of change appears likely to happen. And it may turn out to be a very big
deal, particularly if a tax plan that House Republicans proposed last
summer becomes the core of new legislation.

Among Washington's lobbying shops and policy analysis crowd, it's known
as a ``destination-based cash flow tax with border adjustment.'' It's
easier to think of it as the most substantial reworking of how
businesses are taxed since the corporate income tax was introduced a
century ago. And it could, if enacted, have big effects not just in the
tax departments of major corporations but in global financial markets
and the aisles of your local Walmart.

This possible revamping of the corporate tax code is less politically
polarizing than the debates sure to unfold in the months ahead over
health care, or even over individual income taxes. But the consequences
for business --- and for the long-term trajectory of the economy --- are
huge.

The basic idea behind a D.B.C.F.T. (to use the abbreviation that has
taken hold in a particularly nerdy corner of Twitter) is this: Right now
companies are taxed based on their income generated in the United
States. But there are countless tricks that corporate accountants can
play to reduce the income companies report and to reduce their tax
burden, and those tricks distort the economy.

Two prime examples are transferring intellectual property to overseas
holding companies and engaging in corporate inversions that move a
company's legal headquarters to a country with lower taxes. Moreover,
because interest payments on debt are tax-deductible, the current system
makes it appealing to take on as much debt as possible, even though that
can increase the risk of bankruptcy when a downturn comes along.

The House Republicans' approach, instead of taxing the
easy-to-manipulate corporate income, goes after a firm's domestic cash
flow: money that comes in from sales within the United States borders
minus money that goes out to pay employees and buy supplies and so
forth. There's no incentive to play games with overseas companies that
exist only to exploit tax differences or to relocate production to
countries with lower taxes because you'll be taxed on things you sell in
the United States, regardless.

``With an income tax, one of the key issues is `how do you measure
income,' '' said Alan Auerbach, an economist at the University of
California, Berkeley, who is a leading advocate of the idea. ``But with
cash flow you just follow the money.''

And the tax, Mr. Auerbach argues, could spur business investment while
not encouraging companies to rely on debt. It allows companies to enjoy
the tax savings of capital investments immediately rather than
depreciating them over time. And it doesn't give favorable treatment to
debt, as opposed to equity.

\includegraphics{https://static01.nyt.com/images/2017/01/08/business/08UP-VIEW/08UP-VIEW-articleInline.jpg?quality=75\&auto=webp\&disable=upscale}

That alone would amount to a major shift in the tax system.
Congressional staff members, the incoming administration and armies of
lobbyists will spend countless hours hammering out the details of any
such proposal: how it might be phased in, and how to treat financial
services, and much more.

Some of the most complex, and politically problematic, elements of the
plan revolve around its treatment of international trade, which creates
winners and losers. And some of those potential losers are powerful.

Consider what border adjustment means: When an American company exports
goods under this new tax system, it would not pay any taxes on its
international sales, while its imports would be taxed. So a company that
spent \$80 making something that it sold overseas for \$100 would pay no
tax on its earnings. A company that imported goods worth \$80 from
abroad and them sold them domestically for \$100 would pay tax on the
full \$100.

At first glance this looks as if it would boost exports and reduce the
trade deficit. Indeed, it might prove politically promising for
advocates of the strategy to pitch the plan as one that would do this.

Many economists think it won't work that way, however. That's because as
soon as a cash-flow-based tax with border adjustment looks likely to
become law, the value of the dollar should rise in currency markets. And
that stronger dollar could eliminate the apparent pro-export,
anti-import effects of the tax. The dollar could rise by, say, 20 to 25
percent, and the trade balance could remain about where it started.

Essentially, moving to this system means betting on a ``textbook
economic theory,'' as analysts at Evercore ISI put it, becoming a
reality even though the effect hasn't been tested in practice.

If the dollar doesn't strengthen as expected, for example,
import-dependent industries, especially those with lean profit margins,
could face disaster. That helps explain why some of the stiffest
opposition to this tax overhaul is coming
\href{http://www.politico.com/story/2016/11/retailers-fear-massive-tax-increases-under-house-republicans-tax-plan-231817}{from
the retail industry}. Essentially, economists are telling them ``trust
us, our models say the currency will adjust and it will all come out in
the wash,'' but if the models are wrong, for companies like Walmart,
Target and many others that sell large volumes of imported goods, their
viability could be threatened.

If the models turn out to be right, there is a different set of risks.
The United States dollar is the linchpin of the global financial system,
and a large move in its value triggered by changes in domestic tax
policy could have unforeseen effects.

Many companies worldwide, especially banks and especially in emerging
markets, have debt denominated in dollars, which would become more of a
burden after a new dollar appreciation. A big dollar rise would also
effectively shift trillions in wealth from American investments overseas
toward global investors with assets in the United States.

As Jared Bernstein of the Center on Budget and Policy Priorities
\href{https://www.washingtonpost.com/posteverything/wp/2016/12/30/my-take-on-the-republicans-new-interesting-corporate-tax-plan/?utm_term=.d7e54ba12abe}{has
noted,} we don't really know what the distributional consequences of
this tax overhaul would be. It could increase the costs of imported
goods that the poor spend a disproportionate portion of their income on,
like clothing and gasoline. That would be bad news for poorer Americans
even as it makes the overall economy more efficient.

There's still a lot of work to be done to understand the far-reaching
consequences of the D.B.C.F.T. (also, work to be done to find a catchier
name). But there's a broader point about the nature of any major policy
reform. The benefits of a reworked corporate tax code would emerge
slowly; these disruptions and costs could arrive almost instantly.

No matter the outcome, 2017 will be a fascinating year in which core
components of the tax system --- with long-lasting economic consequences
--- will be up for grabs.

Advertisement

\protect\hyperlink{after-bottom}{Continue reading the main story}

\hypertarget{site-index}{%
\subsection{Site Index}\label{site-index}}

\hypertarget{site-information-navigation}{%
\subsection{Site Information
Navigation}\label{site-information-navigation}}

\begin{itemize}
\tightlist
\item
  \href{https://help.nytimes.com/hc/en-us/articles/115014792127-Copyright-notice}{©~2020~The
  New York Times Company}
\end{itemize}

\begin{itemize}
\tightlist
\item
  \href{https://www.nytco.com/}{NYTCo}
\item
  \href{https://help.nytimes.com/hc/en-us/articles/115015385887-Contact-Us}{Contact
  Us}
\item
  \href{https://www.nytco.com/careers/}{Work with us}
\item
  \href{https://nytmediakit.com/}{Advertise}
\item
  \href{http://www.tbrandstudio.com/}{T Brand Studio}
\item
  \href{https://www.nytimes.com/privacy/cookie-policy\#how-do-i-manage-trackers}{Your
  Ad Choices}
\item
  \href{https://www.nytimes.com/privacy}{Privacy}
\item
  \href{https://help.nytimes.com/hc/en-us/articles/115014893428-Terms-of-service}{Terms
  of Service}
\item
  \href{https://help.nytimes.com/hc/en-us/articles/115014893968-Terms-of-sale}{Terms
  of Sale}
\item
  \href{https://spiderbites.nytimes.com}{Site Map}
\item
  \href{https://help.nytimes.com/hc/en-us}{Help}
\item
  \href{https://www.nytimes.com/subscription?campaignId=37WXW}{Subscriptions}
\end{itemize}
